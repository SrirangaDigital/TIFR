\setcounter{chapter}{-1}
\chapter{Preliminaries}\label{chap0:chap0}\pageoriginale

\section{}\label{chap0:sec1}
In this chapter we formulate certain notions connected with the notion
of a manifold in a form we need later on and fix some notations and
conventions.

By means of a basis we identify a $d$-dimensional real vector space
$\mathcal{V}$ with set $\mathbb{R}^{d}$ of all $d$-tuples of real
numbers with the standard vector space structure and {\em denote} the
element with 1 in the $i^{\text{th}}$ place and zeros elsewhere {\em
  by} $e_{i}$. Then $\{e_{i}\}$ form a basis of $\mathbb{R}^{d}$, we
call it the {\em canonical basis} of $\mathbb{R}^{d}$, and its {\em
  dual basis} $(u^{1},\ldots,u^{d})$ the canonical coordinate system
in $\mathbb{R}^{d}$. (For $\mathbb{R}(=\mathbb{R}^{1})$ we set
$u^{1}=t$). With this identification any real vector space becomes a
manifold and this structure is independent of the basis chosen.

A {\em manifold} will always be a $C^{\infty}$-manifold which is
Hausdorff, para compact and of constant dimension $d$. Generally we
denote it by $M$, a typical {\em chart of it} by $(U,r)$ and {\em
  local coordinates} with respect to $(U,r)$ by $x^{i}=u^{i}\circ
r$. Let us note that because of para compactness partitions of unity
for $M$ exist.

We denote the {\em tangent space} of $M$ at $a$ by $T_{a}(M)$ and
define the {\em tangent} bundle $T(M)$ of $M$ to be
\begin{equation*}\label{chap0:0.1.1}
\bigcup_{a\in M} \bigcup_{x\in T_{a}(M)}\{x\}=T(M)\tag{0.1.1}
\end{equation*}
and call elements of $T(M)$ {\em vectors} of $M$. Then we have the
{\em natural projection map} $p_{M}$ (or simply $p$) from $T(M)$ onto
$M$ which takes every vector of $M$ at $a$ onto $a$.

We \pageoriginale {\em denote} the set of all {\em maps} ($C^{\infty}$-maps,
differentiable maps) of a manifold $M$ into a manifold $N$ {\em by}
$D(M,N)$ and when $N=\mathbb{R}$ we write $F(M)$ for
$D(M,\mathbb{R})$. Every $f$ in $D(M,N)$ induces a map of $F(N)$ into
$F(M)$ defined by
$$
g\to g\circ f
$$
and this in turn {\em a map} $f^{T}$ of $T(M)$ into $T(N)$ defined by
the equation
\begin{equation*}\label{chap0:0.1.2}
(f^{T}(x))(g)=x(g\circ f),\quad g\in F(N).\tag{0.1.2}
\end{equation*}
Then we have the following commutative diagram
\[\label{chap0:0.1.3}
\vcenter{\xymatrix@=1.5cm{
T(M)\ar[r]^{f^{T}}\ar[d]_{p_{M}} & T(N)\ar[d]^{p_{N}}\\
M\ar[r]^{f} & N
}}\tag{0.1.3}
\]

\setcounter{section}{1}
\setcounter{subsection}{3}
\subsection{}\label{chap0:0.1.4}
Let us note that if $f^{T}$ restricted to $T_{m}(M)$ is one-one then
we can choose a local coordinate system $(x^{1},\ldots,x^{e})$ for $N$
at $n=f(m)$ such that $\{x^{i}\circ f\}$ $(i=1,\ldots,d)$ form a local
coordinate system at $m$ for $M$. Also if $f^{T}$ restricted to
$T_{m}(M)$ is onto $T_{n}(N)$ then we can choose a local coordinate
system $(x^{1},\ldots,x^{d})$ for $M$ at $m$ and a local coordinate
system $(y^{1},\ldots,y^{e})$ for $N$ at $n$ such that
$x^{i}=y^{i}\circ f$, $i=1,\ldots,0$.

\subsection{}\label{chap0:0.1.5}
We call a manifold $N$ a {\em sub manifold} of $M$ if 
\begin{enumerate}
\renewcommand{\labelenumi}{\theenumi)}
\item $N\subset M$,\pageoriginale

\item the topology on $N$ is induced by that on $M$,

\item to each point $p$ of $N$ there is a chart $(U_{p}, r_{p})$ in
  the atlas of $M$ such that for some positive integer $k$
$$
r_{p}(N\cap U_{p})=\left\{(x_{i})\in
r_{p}(U_{p})\,\big|\,x_{k+1}=\ldots=x_{d}=0\right\}.
$$
\end{enumerate}

\subsection{}\label{chap0:0.1.6}
Under these conditions a vector $x$ of $T_{m}(M)$ will be in
$i^{T}(T_{m}(N))$ (where $i:N\to M$ denotes the injection) if and only
if $x(\varphi\circ i)=0\forall \varphi\in F(N)$ implies that $x=0$.

In case the manifold $M$ is an open sub manifold $A$ of a finite
dimensional real vector space $V$, we identify $T(A)$ with $A\times V$
as follows:

\subsection{}\label{chap0:0.1.7}
For $y\in V$, $f\in F(A)$, $a\in A$ set
$$
\zeta^{-1}_{a}(y)\cdot (f)=\lim\limits_{t\to
  0}\left(\dfrac{f(a+ty)-f(a)}{t}\right).
$$

We see that $\zeta^{-1}_{a}(y)\in T_{a}(M)$ and that the map
$\zeta^{-1}_{a}$ is an isomorphism of $V$ with $T_{a}(M)$. We denote
its inverse by $\zeta_{a}$ and define a map from $T(A)$ onto $V$ by
setting
\begin{equation*}\label{chap0:0.1.8}
\zeta(x)=\zeta_{p_{A}(x)}(x).\tag{0.1.8}
\end{equation*}

Then the identification is given by the map
\begin{equation*}\label{chap0:0.1.9}
T(A)\ni x\to (p_{A}(x), \zeta(x))\in A\times V.\tag{0.1.9}
\end{equation*}

Given $(U,r)$ the maps $\zeta\circ r^{T}$ and $(p_{r(U)},\zeta)\circ
r^{T}$ are called the {\em principal part} and the {\em trivialising
  map} respectively.

(See\pageoriginale Lang \cite{19} : p. 49).

The use of these maps instead of the explicit use of the local
coordinates has the advantages of the latter without its
tediousness. So, more explicitly, given $(U,r)$ we write
\begin{align*}
x &\mathop{=}_{\mho} (a,b)\\
a &= (p_{r(U)}\circ r^{T})(x)\quad \text{and}\quad b=(\zeta\circ
r^{T})(x).
\end{align*}
Generally we take $(a^{1},\ldots,a^{d})$, $(b^{1},\ldots,b^{d})$ as
the coordinate representations of $a$, $b$.

Now, we can consider the collection $(T(U),(p_{r(U)},\zeta)\circ
r^{T})$ corresponding to the charts $(U,r)$ of $M$ as an atlas on
$T(M)$ and thus define a structure of a manifold on $T(M)$. With this
structure on $(T(M))$ we have $p_{M}\in D(T(M),M)$ and furthermore if
$N$ is a manifold and $f\in D(M,N)$ then $f^{T}\in D(T(M),T(N))$.

\setcounter{subsection}{9}
\subsection{}\label{chap0:0.1.10}

Since $M$ is Hausdorff and hence there are functions with arbitrarily
small support we see that the class of all sections of $T(M)$, i.e.,
$$
\left\{X\in D(M,T(M)) \, \big|\, p\circ X=\id_{M}\right\}
$$
can be identified with $\mathscr{C}(M)$, the set of all derivations of
the $\mathbb{R}$-algebra $F(M)$ into itself. Elements of
$\mathscr{C}(M)$ are {\em called vector fields} on $M$.

\subsection{}\label{chap0:0.1.11}

Now let us take $(U,r)$. Suppose that a vector $(a,b)$ is given in
$U$. In $(a,b)$ we vary the first coordinate $a$ over $U$ keeping $b$
fixed. Thus, clearly, we get a vector field over
$U$. Now \pageoriginale we take 
functions $\varphi$ on $M$ such that
\begin{itemize}
\item[(i)] $0\leq \varphi\leq 1$, and

\item[(ii)] $\varphi$ is zero outside $U$ and is $1$ in a
  neighbourhood of $a$ in $U$.
\end{itemize}

Then we set
$$
X_{\ell c}=
\begin{cases}
\varphi(x)\cdot X(x) & \text{for \ } x\in U\\
0_{x} & \text{for \ } x\not\in U
\end{cases}
$$

This vector field so obtained is called a {\em locally constant vector
  field} at $a$.

If $f\in D(M,V)$ is a map of a manifold $M$ into a finite dimensional
real vector space, we define, besides $f^{T}$, {\em a map} $df$,
called {\em the differential of} $f$, from $T(M)$ to $V$ by setting
\begin{equation*}\label{chap0:0.1.12}
df = \zeta\circ f^{T}.\tag{0.1.12}
\end{equation*}
Further if $M$ is a vector space $V'$, then {\em denoting the Jacobian
  of} $f$ by $\Df$ we have (with the $\Df$ of \cite{36}):

\setcounter{subsection}{12}
\subsection{}\label{chap0:0.1.13}
$$
Df\circ \zeta=df.
$$
Let us write down explicitly the expression for $df$ and $Df$. Let
$V$ be $n$-dimensional with a basis
$\{\overline{e}_{1},\ldots,\overline{e}_{n}\}$ and corresponding
coordinate system $(\overline{x}^{1},\ldots,\overline{x}^{n})$ and, as
usual, let $(U,r)$ be a chart of $M$ with local coordinate system
$(x^{1},\ldots, x^{d})$. Then $f$ on $U$ is given by $n$
$C^{\infty}$-functions $\overline{f}^{i}=\overline{x}_{i}\circ f$
$(i=1,\ldots,n)$ on $U$. The canonical local bases for
$\mathscr{C}(U)$ and $\mathscr{C}(V)$ are
$$
\left\{\left(\dfrac{\partial}{\partial x^{i}}\right)\right\}\quad\text{and}\quad
\left\{\left(\dfrac{\partial}{\partial \overline{x}^{i}}\right)\right\}\text{
  \ respectively}. 
$$\pageoriginale
Then we have
\begin{equation*}\label{chap0:0.1.14}
f^{T}\left(\dfrac{\partial}{\partial
  x^{i}}\right)=\sum_{j}\dfrac{\partial \overline{f}^{j}}{\partial
  x^{i}}\dfrac{\partial}{\partial \overline{x}^{j}},i=1,\ldots,d.\tag{0.1.14}
\end{equation*}
Now is customary let us write the contra variant (tangent) vector as a
row vector. Now suppose that $df$ takes a contra variant vector with
coordinates $(x_{1},\ldots,x_{d})$ to one with
$(\overline{x}_{1},\ldots,\overline{x}_{n})$, i.e.
\begin{equation*}\label{chap0:0.1.15}
(df)\left(\sum_{i}x_{i}\dfrac{\partial}{\partial
    x^{i}}\right)=\sum_{i}\overline{x}_{i}\dfrac{\partial f}{\partial
    \overline{x}^{i}}.\tag{0.1.15}
\end{equation*}
Then the coordinates are related by the matrix equation
\begin{equation*}
\text{i.e.}\quad (x_{1},\ldots,x_{n})=
\begin{pmatrix}
\dfrac{\partial \overline{f}^{1}}{\partial x^{1}} & \dfrac{\partial
  \overline{f}^{2}}{\partial x^{1}} & \ldots & \dfrac{\partial
  \overline{f}^{n}}{\partial x^{1}}\\[5pt]
\vdots & \vdots & & \vdots\\[5pt]
\dfrac{\partial \overline{f}^{1}}{\partial x^{i}} & \dfrac{\partial
  \overline{f}^{2}}{\partial x^{i}} & \ldots & \dfrac{\partial
  \overline{f}^{1}}{\partial x^{i}}\\[5pt]
\vdots & \vdots & & \vdots\\[5pt]
\dfrac{\partial \overline{f}^{1}}{\partial x^{d}} &
\dfrac{\partial\overline{f}^{2}}{\partial x^{d}} &\ldots &
\dfrac{\partial \overline{f}^{n}}{\partial x^{d}}
\end{pmatrix}
\cdot (\overline{x}_{1},\ldots,\overline{x}_{n}); X=J\cdot X
\tag{0.1.16}\label{chap0:0.1.16}
\end{equation*}
Hence the Jacobian is $J$. 

\section{Forms}\label{chap0:sec2}

We consider $\mathscr{C}(M)$ as an $F(M)$ module and denote its {\em
  dual} by $\mathscr{C}^{\ast}(M)$. 
The \pageoriginale elements of
$\mathscr{C}^{\ast}(M)$ are called the {\em differential forms} on
$M$. 

\subsection{}\label{chap0:0.2.1}

Now we define the {\em cotangent bundle} $T^{\ast}(M)$ to be
$$T^{\ast}(M)=\bigcup\limits_{a\in M}\bigcup\limits_{x^{\ast}\in
  T_{a}(M)}\{x^{\ast}\},$$ 
and define a manifold structure on
$T^{\ast}(M)$ in a way analogous to that on $T(M)$. For $\omega \in
T^{\ast}(M)$, $X\in \mathscr{C}(M)$ by evaluation at $a$ in $M$ we see
that $\omega(X)(a)$ depends only on $X(a)$ and that we can identify
$\mathscr{C}^{\ast}(M)$ with 
\begin{equation*}\label{chap0:0.2.2}
\left\{\omega \in D(M,T^{\ast}(M))|p_{M}\circ \omega
=\id_{M}\right\}=\mathscr{C}^{\ast}(M)\tag{0.2.2} 
\end{equation*}

More generally, $\mathfrak{M}$ being any unitary module over a
commutative ring $A$ we write $L^{s}(\mathfrak{M})$
(resp. $E^{s}(\mathfrak{M})$) for the $A$-module of all multilinear
(resp. alternating multilinear) forms of degree $s$ on
$\mathfrak{M}$. There is a map, called {\em multiplication}, from
$L^{s}(\mathfrak{M})\times L^{s'}(\mathfrak{M})$ into
$L^{s+s'}(\mathfrak{M})$ defined by $(\omega,\sigma)\to \omega\cdot
\sigma$, where $(\omega\cdot
\sigma)(X_{1},\ldots,X_{s},X_{s+1}\ldots,X_{s+s'})=\omega
(X_{1},\ldots,X_{s})\cdot \sigma (X_{s+1},\ldots,X_{s+s'})$, and a
map, called {\em exterior multiplication}, from
$E^{s}(\mathfrak{M}) \times E^{s'}(\mathfrak{M})$ into
$E^{s+s'}(\mathfrak{M})$, obtained by alternating the latter one.

{\em Now we set}
\begin{align*}
\mathscr{L}^{s}(M) &= L^{s}(\mathscr{C}(M))\\
\text{\em and}\quad \mathscr{E}^{s}(M) & E^{s}(\mathscr{C}(M))
\end{align*}
considering $\mathscr{C}(M)$ as an $F(M)$-module, and 
\begin{align*}
L^{s}(T(M)) &= \bigcup_{a\in M}\bigcup_{x\in L^{s}(T_{a}(M))}\{x\}\\
E^{s}(T(M)) &= \bigcup_{a\in M}\bigcup_{x\in E^{s}(T_{a}(M))}\{x\}
\end{align*}
considering \pageoriginale the $T_{a}(M)$ as $\mathbb{R}$-modules. We
can define a 
manifold structure on $L^{s}(T(M))$ and also on $E^{s}(T(M))$ in a way
analogous to that on $T^{\ast}(M)$. By evaluation at $a$ of $M$ we can
show that for $\omega\in\mathscr{L}^{s}(M)$
$$
\omega(X_{1},\ldots,X_{s})(a)\quad\text{depends only on}\quad
(X_{1}(a),\ldots,X_{s}(a)) 
$$
and hence we can identify $\mathscr{L}^{s}(M)$ and
$\mathscr{E}^{s}(M)$ with
\begin{align*}
\left\{\omega \in D(M,L^{s}(T(M))\,|\, p_{M}\circ\omega
= \id_{M}\right\}\tag{0.2.3}\label{chap0:0.2.3}\\ 
\text{and}\qquad \left\{\omega\in
D(M,E^{s}(T(M)) \,|\, p_{M}\circ\omega=\id_{M}\right\} 
\end{align*}
respectively, and call them {\em $s$-forms on} $M$ and {\em
  $s$-exterior forms on} $M$ respectively. Now for $\omega\in
\mathscr{L}^{s}(M)$ and $a\in M$, $\omega(a)$ makes sense and we often
        {\em denote it by} $\omega_{a}$, and we have a similar
        convention for the elements of $\mathscr{E}^{s}(M)'s$.

\setcounter{subsection}{3}
\subsection{}\label{chap0:0.2.4}

Given a chart $(U,r)$, with the notation
$$
[i]=i_{1}<\ldots<i_{s}, dx_{[i]}=dx_{i_{1}}\wedge \ldots \wedge dx_{i_{s}}
$$
locally we can write any $s$-exterior form as
\begin{equation*}\label{chap0:0.2.5}
\sum_{[i]}\omega_{[i]}dx_{[i]},\quad\text{where}\quad \omega_{[i]}\in
F(r(U)).\tag{0.2.5} 
\end{equation*}

Every $f\in D(M,N)$ {\em induces a map} $f^{\ast}$ of multilinear
forms on $N$ into those on $M$. We have, by definition
\begin{equation*}\label{chap0:0.2.6}
(f^{\ast}\omega)(x_{1},\ldots,x_{s})
  = \omega(f^{T}(x_{1}), \ldots,f^{T}(x_{s}))\quad\text{for}\quad  
  \omega\in \mathscr{L}^{s}(N)\tag{0.2.6}
\end{equation*}
and $x_{1},\ldots,x_{s}\in T_{a}(M)$. Thus $f$ takes
$\mathscr{L}^{s}(N)$ into $\mathscr{L}^{s}(M)$ and furthermore it
takes elements of $\mathscr{E}^{s}(N)$ into those of
$\mathscr{E}^{s}(M)$ and hence can be considered as a map of
$\mathscr{E}^{s}(N)$ into $\mathscr{E}^{s}(M)$. $f^{\ast}$ is
a \pageoriginale homomorphism of $R$-modules having the following
properties: 

\smallskip
\noindent
\begin{tabular}{@{}lll}
& {i)} $f^{\ast}(\omega\cdot\sigma)=f^{\ast}(\omega)\cdot
  f^{\ast}(\sigma)$ for $\omega\in \mathscr{L}^{s}(N)$, $\sigma\in
  \mathscr{L}^{s'}(N)$, \\
(0.2.7)\label{chap0:0.2.7} & {ii)}
  $f^{\ast}(\omega\wedge\sigma)=f^{\ast}(\omega)\wedge 
  f^{\ast}(\sigma)$, for $\omega\in \mathscr{E}^{s}(N)$,
  $\sigma\in\mathscr{E}^{s'}(N)$ \\
& {iii)} for $g\in D(N,L)$, $L$ a manifold, $(g\circ
  f)^{\ast}=f^{\ast}\circ g^{\ast}$.
\end{tabular}

We define {\em a map $d$, called exterior differentiation}, from
$\mathscr{E}^{s}(M)$ into $\mathscr{E}^{s+1}(M)$ by setting
\begin{align*}\label{chap0:0.2.8}
&
  (d\omega)(X_{0},\ldots,X_{s})=\sum^{s}_{i=0}(-1)^{i}X_{i}(\omega(X_{0},\ldots,\widehat{X}_{i},\ldots,X_{s}))+\tag{0.2.8}\\
& \sum_{0\leq i<j\leq
  s}(-1)^{i+j}\omega([X_{i},X_{j}],X_{0},\ldots,\widehat{X}_{i},\ldots,\widehat{X}_{j},\ldots,X_{s}) 
\end{align*}
where $[X,Y]=X\circ Y-Y\circ X$ for $X$, $Y\in \mathscr{C}(M)$,
$X_{0},\ldots,X_{s}\in \mathscr{C}(M)$ and the element under $\wedge$
is to be deleted from the sequence of elements. This map has the
following properties:

\smallskip
\noindent
\begin{tabular}{lll}
& {i)} it is $\mathbb{R}$-linear\\
(0.2.9)\label{chap0:0.2.9} & {ii)} $d\circ d=0$\\
&  {iii)} $d(\omega\wedge\sigma) =
(d\omega) \wedge \sigma+(-1)^{\text{degree of}\omega} \omega \wedge
d\sigma$\\
& {iv)} $d\circ f^{\ast}=f^{\ast}\circ d$ \ for \ $f\in D(M,N)$, 
  $N$ a manifold.
  \end{tabular}

As an example we note
\begin{equation*}\label{chap0:0.2.10}
(d\omega)(X,Y)=X(\omega(Y))-Y(\omega(X))-\omega([X,Y])\tag{0.2.10}
\end{equation*}
for $\omega\in\mathscr{C}^{\ast}(M)$; $X$, $Y\in\mathscr{C}(M)$.

For $X\in\mathscr{C}(M)$, $\omega\in\mathscr{L}^{s}(M)$ we define the
{\em interior product} $i(x)\omega\in \mathscr{L}^{s-1}(M)$ by the
equation
\begin{align*}\label{chap0:0.2.11}
 (i(X)\omega)(X_{1}, \ldots, X_{s-1}) = \omega(X,X_{1}, \ldots,
  X_{s-1})
 \forall X_{1},\ldots,X_{s-1}\in\mathscr{C}(M). \tag{0.2.11}
\end{align*}

\setcounter{subsection}{11}
\subsection{}\label{chap0:0.2.12}\pageoriginale

We call multilinear forms on $(\mathscr{C}(M))^{s}\times
(\mathscr{C}^{\ast}(M))^{t}$ tensors of type $(s,t)$ and denote the
$F(M)$-module of such tensors by $\mathscr{L}^{s}_{t}(M)$.

\section{Integration}\label{chap0:sec3}

\subsection{}\label{chap0:0.3.1}

a.1. Let $V$ be a real vector space of finite dimension $d$. We call
any nonzero element of $E^{d}(V)$ an {\em orientation on} V A basis
$\{x_{1},\ldots,x_{d}\}$, in that order, is {\em positive} with
respect to the orientation $S$ on $V$ is
\begin{equation*}\label{chap0:0.3.2}
S(x_{1},\ldots,x_{d})>0.\tag{0.3.2}
\end{equation*}

\setcounter{subsection}{2}
\subsection{}\label{chap0:0.3.3}

Let $V$ be a vector space. $V$ together with a symmetric positive
definite bilinear form $g$ on $V$, i.e. a bilinear form such that
$$
g(x,y)=g(y,x),\quad\text{and}\quad g(x,x)>0\forall x\neq 0
$$
is called a $g$-{\em euclidean} (or simply {\em euclidean} if there is
no possible confusion) {\em space}. Sometimes we describe this
situation as ``$V$ is provided with a {\em euclidean structure}
$g$''. Two vectors $x$ and $y$ are said to be {\em orthogonal
  relative} to $g$ if $g(xy)=0$. A basis $\{e_{1},\ldots,e_{d}\}$ of
$V$ is called an {\em orthogonal basis relative} to $g$ or simply an
orthogonal basis of $(V,g)$ if $e_{1},\ldots,e_{d}$ are orthogonal
i.e. $g(e_{i},e_{j})=0$ if $i\neq j$. It is called an {\em orthonormal
  basis relative} to $g$ if, further,
$$
g(e_{i},e_{i})=1.
$$

\subsection{}\label{chap0:0.3.4}

\setcounter{example}{3}
\begin{example*}
Let $V$ be provided with a euclidean structure and be oriented by $S$
(i.e. $S$ is an orientation on $V$). Then it admits a {\em
canonical}\pageoriginale 
orientation $S_{V}$ defined by the equation
$$
S_{V}(x_{1},\ldots,x_{d})=1
$$
for every orthonormal basis $\left\{x_{1},\ldots,x_{d}\right\}$,
positive relative to $S$.
\end{example*}

\setcounter{subsection}{4}
\subsection{}\label{chap0:0.3.5}

2.~{\em A volume $t$ on} $V$ is a map $V^{d}\to \mathbb{R}$ such that
it is the absolute value of an orientation $S:t=|S|$, i.e.,
$$
t(x_{1},\ldots,x_{d})=|S(x_{1},\ldots,x_{d})|\quad\text{for}\quad
x_{1},\ldots,x_{d}\quad\text{in}\quad V,.
$$

\subsection{}\label{chap0:0.3.6}
\setcounter{example}{1}
\begin{example}\label{chap0:exam0.3.2}% example 0.3.2
Let $V$ be $g$-euclidean. Then it admits a {\em canonical volume}
$t_{V}$ defined by
$t_{V}(x_{1},\ldots,x_{d})=(\det(g(x_{i},x_{j})))^{1/2} \, \forall
x_{1},\ldots,x_{d}\in V$.

b.~Let $M$ be a manifold. We call an element $\sigma\in
\mathscr{E}^{d}(M)$ an {\em orienting} or {\em volume form} if
$\forall \mathfrak{m}\in M:\sigma_{m}$ is an orientation of
$T_{m}(M)$. In that case we also say that $\sigma$ orients $M$ or that
$M$ is oriented by $\sigma$.
\end{example}

\subsection{}\label{chap0:0.3.7}

\setcounter{example}{6}
\begin{example*}
On $\mathbb{R}^{d}$ there is an orientation $\tau$, called {\em the
  canonical orientation}, given by
\begin{equation*}
\tau=d u^{1}\wedge\ldots\wedge d u^{d}.\tag{0.3.8}\label{chap0:0.3.8}
\end{equation*}
We consider $\mathbb{R}^{d}$ always with this orientation.
\end{example*}

Now there is the notion of a diffeomorphism between oriented manifolds
{\em preserving orientation}. If $M$ is oriented then every open
sub-manifold of $M$ is oriented in a natural way. Now let $(U,r)$ be a
chart of an oriented manifold $M$. Then on $U$ there is an
induced orientation and on $r(U)\subset \mathbb{R}^{d}$ there is an induced
orientation. We say that $(U,r)$ is {\em positive} if $r$ preserves
orientation between $U$ and $r(U)$.

{\em If\pageoriginale $M$ is oriented}, using partition of unity, we
can define a 
{\em notion of integration for elements} of $\mathscr{E}^{d}$, denoted
by $\int_{M}\omega$ where $\omega\in\mathscr{E}^{d}(M)$, with some
good properties some of which we proceed to state. The integral
$\int_{M}\omega$ exists if $\omega$ has compact support. Further if
$M$ is an open sub manifold of an oriented manifold $N$ and
$\overline{M}$ is compact and $\omega$ is the restriction of an
element of $\mathscr{E}^{d}(N)$ then also $\int_{M}\omega$ exists. Let
$M$, $N$ be oriented manifolds and $f$ be a diffeomorphism of $N$ with
$M$ preserving orientation, and let
$\omega\in\mathscr{E}^{d}(M)$. Then if $\int_{M}\omega$ exists so does
$\int_{N}f^{\ast}\omega$, and 
\begin{equation*}
\int\limits_{N}f^{\ast}\omega=\int\limits_{M}\omega.\tag{0.3.9}\label{chap0:0.3.9}
\end{equation*}

\setcounter{subsection}{9}
\subsection{}\label{chap0:0.3.10}


\begin{lemma*}
Let $f\in D(N,M)$ where $N$, $M$ are oriented and let
$\omega\in\mathscr{E}^{d}(M)$ be an orienting form for $M$ such that
$\int_{M}\omega$ exists. Suppose that $f$ is subjective, preserves
orientation and that $f^{T}_{n}$ is an isomorphism $\forall n\in
N$. Then:
$$
\int\limits_{N}f^{\ast}\omega\geq \int\limits_{M}\omega.
$$
\end{lemma*}

%raghu gv 7
\begin{proof}
Under the assumptions on $f^{T}$ it follows that $f$ is a local
diffeomorphism. Now let us take an open covering $\{V_{i}\}$ of $N$
such that $f$ restricted to $V_{i}$ is a diffeomorphism. Then
$\{U_{i}=f(V_{i})\}$ form a covering of $M$ since $f$ is onto. Let us
take a partition of unity $\{\varphi_{i}\}$ on $N$ subordinate to the
covering $\{V_{i}\}$. Let us define $\overline{\varphi}_{i}$ on $M$ by
\begin{align*}
\overline{\varphi}_{i} &= \varphi_{i}\circ f^{-1}\quad\text{on}\quad
U_{i}\\
&= 0\quad\text{otherwise}.
\end{align*}
Then\pageoriginale\, $\sum\limits_{i}\overline{\varphi}_{i}\geq 1$ since
a point in $M$ has at least one inverse. Now
\begin{align*}
\int\limits_{N}f^{\ast}\omega &= \sum_{i}\int\limits_{N}\varphi_{i}\cdot
f^{\ast}\omega=\sum_{i}\int\limits_{M}\overline{\varphi}_{i}\omega\quad\text{by (\ref{chap0:0.3.9})}\\
&\geq \int\limits_{M}\omega\quad\text{since}\quad \sum
\overline{\varphi}_{i}\geq 1.
\end{align*}
\end{proof}

\setcounter{subsection}{10}
\subsection{}\label{chap0:0.3.11}

c.~We call a domain $D$ in $M$ a {\em nice domain} if $\forall m\in
b(D)=\overline{D}-D$ ($\overline{D}$ being the closure of $D$), there
is an open neighbourhood $U$ of $m$ in $M$ and a $\varphi\in F(M)$
such that
$$
d\varphi_{m}\neq 0\quad\text{and}\quad U\cap D=\varphi^{-1}(]-\infty,0[).
$$

\subsection{}\label{chap0:0.3.12}

Then it follows that $b(D)$ is a sub manifold of $M$ of dimension
$d-1$. The notion of orientation can be extended to manifolds with
good boundary. Then one sees that $b(D)$ is oriented in a natural way
if $D$ is. Now let $\omega$ be a $d-1$ form on $M$ and let us denote
by $i$ the injection of $b(D)$ into $M$.

Then Stokes theorem can be stated as follows:

\subsection{}\label{chap0:0.3.13}

If $D$ is a nice domain of $M$ such that $\overline{D}$ is compact
then the Stokes' formula holds:
$$
\int\limits_{D}d\omega=\int\limits_{b(D)}i^{\ast}\omega.
$$
Let us note that if $b(D)$ is empty, then $\int_{D}d\omega=0$.

d.~On a manifold $M$ a {\em positive odd $d$-form} is a map $\omega$
from $\mathscr{C}^{d}$ with values in the set of functions on $M$ such
that it is everywhere, locally, the absolute value of a local
$d$-exterior form. A {\em volume element} $\theta$ on $M$ is a
positive odd $d$-form such that, for every \pageoriginale $m$,
$\theta_{m}$ is a 
volume of $T_{m}(V)$. The volume element $\tau$ the absolute
value of the canonical orientation on $\mathbb{R}^{d}$, is called the
{\em canonical volume element} of $\mathbb{R}^{d}$.

\subsection{}\label{chap0:0.3.14}

On a manifold there is a notion of integration for positive odd
$d$-forms and this allows us to define the integral of a positive odd
$d$-form $\omega$, which we denote by $\int_{M}\omega$.

\subsection{}\label{chap0:0.3.15}

If $f\in D(N,M)$ is a diffeomorphism, $\omega$ a positive odd $d$-form
of $M$ then we can, in a natural way, associate to $\omega$ a positive
odd $d$-form $f^{\ast}\omega$ on $N$. Then if $\int_{M}\omega$ exists
so does $\int_{N}f^{\ast}\omega$ and further we have
$$
\int\limits_{M}\omega=\int\limits_{N}f^{\ast}\omega.
$$

\subsection{}\label{chap0:0.3.16}

\setcounter{remark}{15}
\begin{remark*}
if $M$ is a manifold oriented by $\omega$ then $|\omega|$ is a
positive odd $d$-form and if $\int_{M}\omega$ exists then
$\int_{M}|\omega|$ exists and 
$$
\int\limits_{M}\omega=\int\limits_{M}|\omega|
$$
\end{remark*}

e.~Let $E$ be a differentiable fibre bundle over $M$
$$
p:E\to M,
$$
with $E$ and $M$ oriented, so that the fibres are oriented in a
natural way. Let $\omega\in \mathscr{E}^{d}(M)$,
$\varphi\in\mathscr{E}^{f}(E)$ ($f$ being the dimension of a fibre)
and $i_{m}:p^{-1}(m)\to E$ be the injection of the fibre $p^{-1}(m)$
into $E$. Using a partition of unity and Fubini's theorem we have

\setcounter{subsection}{16}
\subsection{Integration along fibres}\label{chap0:0.3.17}

$$
\int\limits_{E}\varphi\wedge (p^{\ast}\omega)=\int\limits_{m\in
M}(\int\limits_{p^{-1}(m)}i^{\ast}_{m}(\varphi))\omega 
$$
provided \pageoriginale that $M$ and the fibres are compact.

\section{}\label{chap0:sec4}

a)~{\bf Double tangent bundle.}

\subsection{}\label{chap0:0.4.1}

We have the following commutative diagram;
\[
\xymatrix@R=2cm@C=2.5cm{
T(T(M))\ar[r]^{p^{T}}\ar[d]_{p_{T(M)}=p'} & T(M)\ar[d]^{p}\\
T(M)\ar[r]^{p} & M
}
\]
In the case of $(U,r)$, a chart of $M$, the situation in detail is
given by the following commutative diagram:
{\fontsize{9}{11}\selectfont
\begin{equation*}
\vcenter{
\xymatrix@C=.1cm{
T(T(U))\ar[ddd]\ar[rr]^{((p,\zeta)\circ r^{T})^{T}} & & T(r(U)\times
\mathbb{R}^{d})\ar[ddd]\ar[dr] & \\
 & \txt{$(r(U)\times
  \mathbb{R}^{d})(\mathbb{R}^{d}\times\mathbb{R}^{d})$\\$((a,b),(c,d))$\\$(a,b,c,d)$}\ar@{<->}[ur]^-{(p_{1},\zeta_{1})}\ar[ddr]_{p'}
& & \txt{$T(r(U))\times
  T(\mathbb{R}^{d})$\\$((p,\zeta),(p,\zeta))$}\ar[d]\\
 & & & (r(U)\times \mathbb{R}^{d})\times(\mathbb{R}^{d}\times
\mathbb{R}^{d})\ar[dl]^{p^{T}}\\
T(U)\ar[rr]^{(p,\zeta)\circ r^{T}} & & r(U)\times\mathbb{R}^{d} & 
}
}\tag{0.4.2}\label{chap0:0.4.2}
\end{equation*}}\relax
where \pageoriginale the arrows with two heads denote isomorphisms and
we identify 
corresponding spaces.

\setcounter{subsection}{2}
\subsection{}\label{chap0:0.4.3}
Generally we take $(a_{1},\ldots,a_{d})$, $(b_{1},\ldots,b_{d})$,
$(c_{1},\ldots,c_{d})$, $(d_{1},\ldots,d_{d})$ as coordinate
representatives of $a$, $b$, $c$, $d$ if
$z\displaystyle{\mathop{=}_{\cup}}(a,b,c,d)$. 

\subsection{}\label{chap0:0.4.4}
Let $\Phi$ be a $C^{\infty}$ function on $T(U)$. Then let us denote
$\Phi\circ (r,\zeta\circ r^{T})^{-1}$ on $T(r(U))$ by
$\overline{\Phi}$ and the canonical coordinates in $T(r(U))$ by
$(x^{1},\ldots,x^{d}$, $y^{1},\ldots,y^{d})$. Then
\begin{align*}
z(d\Phi) &= (a,b,c,d)(d\Phi)=\tag{0.4.5}\label{chap0:0.4.5}\\
         &= \sum^{d}_{i=1}\dfrac{\partial\Phi}{\partial
  x^{i}}((a);(b))c_{i}+\sum^{d}_{i=1}\dfrac{\partial\Phi}{\partial
  y^{i}}((a);(b))d_{i}. 
\end{align*}

As a particular case when $\varphi=d\Phi$, $\varphi\in C^{\infty}(U)$
we have
\begin{gather*}
z(d\varphi)=(a,b,c,d)(d\overline{\varphi})\\
=\sum^{d}_{i=1}\sum^{d}_{j=1}\dfrac{\partial^{2}\overline{\Phi}}{\partial
  x^{j}\partial x^{i}}(a)b_{i}c_{j}+\sum \dfrac{\partial
  \overline{\varphi}}{\partial x^{i}}(a)d_{i}.
\end{gather*}

Now let
$$
T(T(U))\ni z\mathop{=}_{\cup}(a,b,c,d).
$$
Then we have
\begin{equation*}\label{chap0:0.4.6}
p'(z)\mathop{=}_{\cup}(a,b)\quad\text{and}\quad p^{T}(z)\mathop{=}_{\cup}(a,c).\tag{0.4.6}
\end{equation*}

\setcounter{subsection}{6}
\subsection{}\label{chap0:0.4.7}


\begin{example*}
Let $\phi\in F(M)$ and let $\phi\circ r^{-1}=\phi$.
\end{example*}
Then $\phi\in F(r(U))$ and we have
\begin{align*}
z(d\phi)& =(a,b,c,d)(d\phi)\\
&= D\ophi_{a}(d)+\overset{2}{D}\ophi_{a}(b,c)\quad\text{by \ (0.4.5)}\tag{0.4.8}\label{chap0:0.4.8} 
\end{align*}\pageoriginale
where $D\ophi$ is the Jacobian of $\ophi$. Note that
$\overset{2}{D}\ophi$ is symmetric\break (see Dieudonne \cite{36} : p. 174).

\setcounter{subsection}{8}
\subsection{}\label{chap0:0.4.9}

b)~{\bf Vertical vectors.} For $x$ in $T(M)$ we define the space of
{\em vertical vectors} $V_{x}$ of $T(M)$ at $x$ to be
$(p^{T}_{x})^{-1}(0)$. (From \eqref{chap0:0.4.6} it follows that
$Y{\displaystyle{\mathop{=}_{\cup}}}(a,b,c,d)$ is vertical if and only
if $c=0$). From the injection
$$
i_{m}:T_{m}(M)\to T(M)\quad\text{at}\quad m=p(x)
$$
and the fact that $p(T_{m}(M))=m$ we conclude that
$$
(p^{T}\circ i^{T}_{m})(T_{x}(T_{m}(M)))=0
$$
and hence that
$$
i^{T}_{m}(T_{p(x)}(M))\subset V_{x}.
$$
But since $p^{T}_{x}$ is surjective the kernel of $p^{T}_{x}$ has
dimension $d$ and since the dimension of $i^{T}(T_{p(x)}(M))$, is $d$
we have
\begin{equation*}
V_{x}=i^{T}_{m}(T_{p(x)}(M)).\tag{0.4.10}\label{chap0:0.4.10}
\end{equation*}

For $y\in T(T(M))$ set $x=p_{T(M)}(y)$ and for $y$ vertical set
\begin{equation*}
\xi (y)=(\zeta\circ (i^{T}_{m})^{-1})(y).\tag{0.4.11}\label{chap0:0.4.11}
\end{equation*}
\begin{equation*}
\vcenter{\xymatrix@R=1.5cm@C=2.5cm{
T_{x}(T_{p(x)}(M))\ar[d]_{i^{T}_{m}}\ar[r]^{\zeta_{x}} & T_{p(x)}(M)\\
V_{x}\ar[ur] &
}}\tag{0.4.12}
\end{equation*}

\setcounter{subsection}{12}

\subsection{}\label{chap0:0.4.13}\pageoriginale

If $y{\displaystyle\mathop{=}_{\cup}}(a,b,c,d)$ then since $y$ is
vertical $c=0$, and it follows that\break
$\xi(y){\displaystyle\mathop{=}_{\cup}}(a,d)$.

Note that $\xi$ is an isomorphism between $V_{x}$ and $T_{m}(M)$.
$$
\displaylines{
\rlap{On}\hfill T(M)\mathop{\times}_{M}T(M)=\left\{(x,y)\in
T(M)\times T(M) \,|\, p(x)=p(y)\right\}\hfill
}
$$
there is canonical manifold structure. Now we define a map
$$
\xi^{-1}:T(M)\times T(M)\to T(T(M))
$$
by the equation
\begin{equation*}
\xi^{-1}(x,y)=\xi^{-1}_{x}(y) \quad\text{where}\quad 
\xi^{-1}_{x}(y)\tag{0.4.14}\label{chap0:0.4.14}  
\end{equation*}
is uniquely defined by the conditions
\begin{itemize}
\item[1)] $\xi(\xi^{-1}_{x}(y))=y$

\item[2)] $\xi^{-1}_{x}(y)\in V_{x}$.
\end{itemize}

\setcounter{subsection}{14}
\subsection{}\label{chap0:0.4.15}

If $x{\displaystyle{\mathop{=}_{\cup}}}(a,b)$, and
$y{\displaystyle{\mathop{=}_{\cup}}}(a,c)$ then it follows that
$$
\xi^{-1}_{x}(y)\mathop{=}_{\cup}(a,b,0,c)
$$


\setcounter{subsection}{15}
\subsection{}\label{chap0:0.4.16}

\begin{lemma*}
If $\omega\in F(T(M))$, restricted to $T_{m}(M)$ is linear for every
$m$ of $M$, and $z$ is a vertical vector of $T(M)$, then
$$
z(\omega)=\omega(\xi(z)).
$$
\end{lemma*}

\begin{proof}
The proof follows from the definitions and that, for a linear map $f$
in a vector space, one has $Df=f$.
\end{proof}

\subsection{}\label{chap0:0.4.17}

\begin{lemma*}
If $z\in T(T(M))$ and 
$$
z(d\varphi)=0\quad \forall \varphi\in F(M)
$$\pageoriginale
then
$$
z=0
$$
\end{lemma*}

\begin{proof}
With the notation of \ref{chap0:0.4.1} a) let
$z{\displaystyle{\mathop{=}_{\cup}}}(a,b,c,d)$. Using \ref{chap0:0.4.3}
we get, for every linear function
$$
z(d\phi)=D\phi_{a}(d)=\phi_{a}(d)=0
$$ 
and hence $d$ is zero; and hence for every quadratic function
$$
z(d\phi)=(D^{2}\phi_{a})(b,c)=0
$$
and hence $c$ is zero.
\end{proof}

\setcounter{subsection}{17}
\subsection{}\label{chap0:0.4.18}

{\em Let $h_{\theta}$ denote the map of $T(M)$ which takes every
vector $x$ into $\theta x$.} Then, with the usual notation relative to
$(U,r)$ we have
$$
h_{\theta}(a,b)\mathop{=}_{\cup}(a,\theta b)
$$
and
$$
h^{T}_{\theta}(a,b,c,d)\mathop{=}_{\cup}(a,\theta\cdot
b,c,\theta\cdot d).
$$

c.~{\bf The canonical involution on $T(T(M))$.}

\setcounter{subsection}{18}
\subsection{}\label{chap0:0.4.19}


\begin{theorem*}
There is {\em an involution} $z\to \overline{z}$ of $T(T(M))$ with the
following properties:
\begin{itemize}
\item[{\rm 1)}] $p^{T}(\overline{z})=p'(z)$, 

\item[{\rm 2)}] $p'(\overline{z})=p^{T}(z)$

\item[{\rm 3)}] $\overline{z}(d\phi)=z(d\phi) \, \forall \, \in F(M)$
\end{itemize}
and is {\em uniquely determined} by these conditions.
\end{theorem*}

\begin{proof}
In the case of $(U,r)$ the map
$$
z\mathop{=}_{\cup}(a,b,c,d)\to (a,c,b,d)\mathop{=}_{\cup}\overline{z}
$$
has the properties $1$, $2$ and $3$\pageoriginale thanks to
formula \eqref{chap0:0.4.2} 
and the symmetry of $D^{2}\phi$. Conversely, if
$z{\displaystyle{\mathop{=}_{\cup}}}(a,b,c,d)$ and
$z'{\displaystyle{\mathop{=}_{\cup}}}(a',b',c',d')$,
\begin{align*}
p^{T}(z') &= p'(z)\Leftrightarrow a=a'\quad\text{and}\quad b=c'\\
p'(z') &= p^{T}(z)\Leftrightarrow a=a'\quad\text{and}\quad b'=c
\end{align*}
and $z'(d\phi)=z(d\phi)\,\forall \phi\in F(M)$ together with $a=a'$,
$b=c'$, $b'=c$ gives $d=d'$. Because of this uniqueness the problem
becomes local which has a solution. The definition being intrinsic the
involution is {\em canonical}.
\end{proof}

\setcounter{subsection}{19}
\subsection{}\label{chap0:0.4.20}


\begin{remark*}
$\overline{z}=z\Leftrightarrow p'(z)=p^{T}(z)$.
\end{remark*}

d.~{\bf Another vector bundle structure on $T(T(M))$.}

%\setcounter{definition}{20}

\subsection{}\label{chap0:0.4.21}

\begin{defi*}
If $\theta\in\mathbb{R}$, $z$, $z'\in T(T(M))$ and
$p^{T}(z)=p^{T}(z')$
$$
\displaylines{
\rlap{set}\hfill z\oplus z'
=\left(\overline{\overline{z}+\overline{z'}}\right)\hfill\cr
\rlap{and}\hfill \theta\odot z
=\overline{\theta\cdot \overline{z}}\hfill 
}
$$
\end{defi*}

\noindent
Relative to $(U,r)$ if $z{\displaystyle{\mathop{=}_{\cup}}}(a,b,c,d)$
and $z'{\displaystyle{\mathop{=}_{\cup}}}(a',b',c',d')$ the formula are
$$
z\oplus z' \mathop{=}_{\cup}\,(a,b+b',c,d+d')
$$
and 
$$
\theta\odot z\mathop{=}_{\cup}(a,\theta b,c,\theta d).
$$
With this definition if $f$, $g$ are curves (for a definition of a
curve and related notions see \S 5)~in $T(M)$ such that $p\circ
f=p\circ g$ we have
$$
(f+g)'=f'\oplus g'\quad\text{and}\quad (\theta\cdot f)'=\theta\odot f'.
$$
This last definition of $\oplus$ and $\odot$ holds good for any
vector bundle $E\xrightarrow{p}M$.


\subsection{}\label{chap0:0.4.22}


\begin{lemma*}
For \pageoriginale $\phi\in F(M)$, $\theta\in\mathbb{R}$ and $z$,
$z'\in T(T(M))$ such that $p^{T}(Z)=p^{T}(z')$,
we have
$$
(z\oplus z')(d\phi)=z(d\phi)+z'(d\phi)
$$
and
$$
(\theta\odot z)(d\phi)=(\theta\cdot z)(d\phi).
$$
\end{lemma*}

\begin{proof}
The first part follows from the definition of $\oplus$ and property
3)~of the involution $-$, and similarly the latter.
\end{proof}

e.~{\bf The canonical forms $\mu$ and $d\mu$.} 

We have the following commutative diagram:
\begin{equation*}
\vcenter{\xymatrix@R=1.5cm@C=2.5cm{
T(T^{\ast}(M))\ar[d]_{p_{T(M)}=p''}\ar[r]^{(p^{\ast})^{T}} &
T(M)\ar[d]^{p}\\
T^{\ast}(M)\ar[r]_{p^{\ast}} & M
}}\tag{0.4.23}
\end{equation*}
where $p^{\ast}$ is the natural projection from $T^{\ast}(M)$ on
$M$. For $z\in T(T^{\ast}(M))$ we denote $p''(z)(p^{T}(z))$ by
$\mu(z)$. Then
\begin{equation*}\label{chap0:0.4.24}
\mu\in \xi^{1}(T^{\ast}(M))=\mathscr{C}^{\ast}(T^{\ast}(M)).\tag{0.4.24}
\end{equation*}
To describe locally the situation above we have a diagram similar to
the one in \eqref{chap0:0.4.2}. With a similar notation, if
$z=(a,\beta,c,\delta)$ then $p''(z)=(a,\beta)$, $p^{\ast}(z)=(a,c)$
and hence

\begin{equation*}
\mu(z)=\beta(c)=\sum_{i}\beta_{i}c_{i}\tag{0.4.25}\label{chap0:0.4.25}
\end{equation*}
To \pageoriginale compute $d\mu$, let
$z{\displaystyle{\mathop{=}_{\cup}}}(a,\beta,c,\delta)$ and
$z'{\displaystyle{\mathop{=}_{\cup}}}(a,\beta',c,\delta')$, and let
$Z$, $Z'$ be local vector fields on $T$ with constant principal parts
such that
\begin{itemize}
\item[i)] at $(a,\beta)$ they are equal to $z$ and $z'$ respectively
and

\item[ii)] $[Z,Z']=0$ in a neighbourhood of $(a,\beta)$.
\end{itemize}

Then we have \eqref{chap0:0.2.10}:
$$
d\mu (Z,Z')=Z\mu(Z')-Z'\mu(Z)
$$
around $(a,\beta)$. Hence
\begin{equation*}
d\mu(z,z')=\delta(c')-\delta'(c).\tag{0.4.26}\label{chap0:0.4.26}
\end{equation*}

From this follows the

\setcounter{subsection}{26}
\subsection{}\label{chap0:0.4.27}


\begin{lemma*}
$d\mu$ is a non degenerate element of $\mathscr{E}^{2}(T^{\ast}(M))$.
\end{lemma*}

\section{Curves}\label{chap0:sec5}

\subsection{}\label{chap0:0.5.1}

\begin{defi*}
A {\em curve} in $M$ is an open interval $I$ together with an $f\in D(I,M)$.
\end{defi*}

We denote, generally, a curve by $(I,f)$ and when no confusion is
possible we omit $I$. Generally, whenever we consider a curve we
assume that $0\in I$. A curve can also be viewed as a point set $E$
obtained as the image under $f$ of an interval $I$. Hence we sometimes
say ``the curve $E$ is parametrised by $f$'', ``the curve $f$ is
parametrised by $t\in I$'' and by these we simply mean that the curve
under consideration is $(I,f)$.

\setcounter{subsection}{1}
\subsection{}\label{chap0:0.5.2}
\pageoriginale
When we take a closed interval $[a,b]$ and say $f$ {\em is a curve
from} $[a,b]$ to $M$ we mean that there is an open interval $I\supset
[a,b]$ such that $f\in D(I,M)$. We denote by $P$ the element of
$\mathscr{C}(\mathbb{R})$ such that $\zeta\circ P=1$, so that
$\left\{P=\dfrac{d}{\dt}\right\}$ is a basis of
$\mathscr{C}(\mathbb{R})$, dual to the basis $dt$ of
$\mathscr{C}^{\ast}(\mathbb{R})$. Let us agree to denote the
restriction of $P$ to an interval by $P$ itself.

\subsection{}\label{chap0:0.5.3}

The curve $f^{-1}(t)=f(b-(t-a))$ is called the {\em inverse of the
curve} $f$ or the curve $f$ described in the opposite way.

\setcounter{definition}{3}

\subsection{}\label{chap0:0.5.4}

\begin{defi*}
If $f$ is a curve in $M$ we define the {\em speed} or the {\em tangent
vector} $f'$ to $f$ by $f'=f^{T}\circ P$.
\begin{equation*}
\vcenter{\xymatrix@R=2cm@C=4cm{
 & T(T(M))\ar@<-3pt>[d]_{p^{T}}\ar@<+3pt>[d]^{p'}\\
T(I)\ar[ur]^{(f')^{T}}\ar[r]_>>>>>>>>>>>{f^{T}} & T(M)\ar[d]^{p}\\
I\ar[u]^{P}\ar[uur]^>>>>>>>>>>>>>>>>>>>>{f''}\ar[ur]_{f'}\ar[r]_{f} & M
}}\tag{0.5.5}\label{chap0:0.5.5}
\end{equation*}
\end{defi*}

\setcounter{subsection}{6}

\subsection{}\label{chap0:0.5.7}


\begin{remarks*}
If $N$ is a manifold and $g\in D(M,N)$ and $f$ is a curve in $M$ then
$g\circ f$ is a curve in $N$ and
$$
(g\circ f)'=g^{T}\circ f'.
$$
Sometimes this will be used as a geometric device for computation.
\end{remarks*}

\setcounter{subsection}{7}
\subsection{}\label{chap0:0.5.8}
\pageoriginale
To compute $x(\phi)$ for $x\in T(M)$ and $\phi\in F(M)$ one can start
with a curve $f$ in $M$ such that $f'(0)=x$ and observe
$$
x(\phi)=f(0)(\phi)=(f^{T}\circ P)_{0}(\phi)=P(\phi\circ
f)\,\big|_{0}=\dfrac{d}{\dt}(\phi\circ f)\,\big|_{0}
$$

b)~We call the speed of the tangent vector $f'$ of $f$ {\em
acceleration $f''$ of} $f$ and thus 
$$
f''=(f')'=(f')^{T}\circ P=(f^{T})^{T}\circ P^{T}\circ P.
$$

\setcounter{subsection}{8}

\subsection{}\label{chap0:0.5.9}

\begin{remarks*}
We have $\overline{f''}=f''$, for
$$
p^{T}\circ f''=(p\circ f^{T}\circ P)^{T}\circ P=f^{T}\circ P=p'\circ
f'' 
$$
and (\ref{chap0:0.4.20}) now gives the result.
\end{remarks*}

\setcounter{subsection}{9}
\subsection{}\label{chap0:0.5.10}

We have $\dfrac{d^{2}(\phi\circ f)}{\dt^{2}}=f''(0)(d\phi)$. In fact
\begin{align*}
& f''(0)(d\phi)=((f')^{T}\circ P)(0)(d\phi)=P(d\phi(f')) \, \Big|_{0}\\
& =\dfrac{d}{\dt}(d\phi)(f^{T}\circ
P)\, \Big|_{0}=\dfrac{d}{\dt}\left(\dfrac{d}{\dt}(\phi\circ
f)\right)\,\Big|_{0}\\
&= \dfrac{d^{2}}{\dt^{2}}(\phi\circ f)\, \Big|_{0}
\end{align*}

\subsection{\em Change of parameter.}\label{chap0:0.5.11}

For $a$, $\theta\in \mathbb{R}$, {\em we define maps} $\tau_{a}$ and
$k_{\theta}$ by setting
$$
\tau_{a}(t)=a+t\quad\text{and}\quad k_{\theta}(t)=\theta\cdot t,
$$
for any $t$ in $\mathbb{R}$. It follows, directly, from the definition
that
$$
(f\circ \tau_{k})'=f'\circ \tau_{k}, (f\circ
k_{\theta})'=\theta(f'\circ k_{\theta})
$$
and\pageoriginale
$$
(f\circ k_{\theta})''=(h^{T}_{\theta}\circ f''\circ k_{\theta}).
$$
If $\phi$ is a diffeomorphism of an interval $I'$ with $I$ then
$f\circ\phi$ is a curve. This new curve is called the curve obtained
from the curve $(I,f)$ by re-parametrisation by $\phi$. This situation
is sometimes described as ``$\phi$ re-parametrises $f$.''

\section{Flows}\label{chap0:sec6}

In this article we fix, once for all, a manifold $L$ and a vector
field $X\in\mathscr{C}(L)$ of $L$.

\qquad a.~{\bf Integral curves.}

\subsection{}\label{chap0:0.6.1}

\begin{defi*}
An {\em integral} or {\em integral curve} of $X$ is a curve $f\in
D(I,L)$ such that
$$
f'=X\circ f.
$$
From the fundamental existence theorem in the theory of differential
equations we know that through any point of $L$ there is an integral
curve of $X$. Set
\begin{equation*}\label{chap0:0.6.2}
\psi=\{(t,m)\in\mathbb{R}\times L|\exists I\supset
[0,t]\quad\text{and an}\quad f\in D(I,L)\tag{0.6.2}
\end{equation*}
such that i)~$f$ is an integral of $X$, ii)~$f(0)=m\}$. {\em For $m\in
L$ we define $t^{+}(m)$ by the equation}
\begin{equation*}
t^{+}(m)=\sup\left\{t\in\mathbb{R}|(t,m)\in\psi\right\},\tag{0.6.3}
\end{equation*}
and similarly $t^{-}(m)$. We can see that $t^{+}$ (resp. $t^{-}$) is
lower (resp. upper) semi continuous on $L$. Since $L$ is Hausdorff we
can see that there exists a unique integral curve $f_{m}$ of $X$
defined over $]t^{-}(m)$, $t^{+}(m)[$ with $f(0)=m$, and that it is
maximal.
\end{defi*}

\setcounter{subsection}{3}
\subsection{}\label{chap0:0.6.4}
\pageoriginale
b.~{\em Flow.} For $t\in ]t^{-}(m)$, $t^{+}(m)[$

\subsection{}\label{chap0:0.6.5}
set
$$
\gamma(t,m)=f_{m}(t)=\gamma_{t}(m).
$$
Then $\gamma\in D(\psi,L)$ and in open sets of $L$ where $\gamma_{t}$,
$\gamma_{s}$ and $\gamma_{t+s}$ are defined we have
\begin{equation*}
\gamma_{t}\circ \gamma_{s}=\gamma_{t+s}\tag{0.6.6}\label{chap0:0.6.6}
\end{equation*}
and, in particular, $\gamma_{t}$ is a local diffeomorphism.

\setcounter{subsection}{6}
\subsection{}\label{chap0:0.6.7}

The family $\{\gamma_{t}\}$ is also called the local one parameter
group of transformations generated by $X$. {\em We call $\gamma$ the
flow of $X$.}

\subsection{}\label{chap0:0.6.8}
Now let us suppose that a map $f$ from a manifold $M$ onto $N$ is a
diffeomorphism and $X$ and $Y$ are vector fields in $M$ and $N$
respectively such that
$$
f^{T}\circ X=Y.
$$
Then if $\gamma_{M}$ and $\gamma_{N}$ are flows of $X$ and $Y$
respectively we have
$$
(\id_{R},f)\circ \gamma_{M}=\gamma_{N}.
$$

\qquad c.~{\bf Lie derivative.} Let $\omega\in \mathscr{L}^{r}(L)$,
$m\in L$ and $t\in \mathbb{R}$ be sufficiently small. Then 
$$
(\gamma_{t})^{\ast}\omega_{\gamma(t,m)}\in \mathscr{L}^{r}(T_{m}(L))
$$
for every $t$ and depends differentiably on $t$. So it makes sense to
set
\begin{equation*}
(\theta(X)\cdot\omega)(m)=\frac{d}{\dt}((\gamma_{t})^{\ast}\omega_{\gamma(t,m)})|_{t=0}.\tag{0.6.9}\label{chap0:0.6.9} 
\end{equation*}
Note \pageoriginale that
$$
\theta(X) \cdot \omega\in \mathscr{L}^{r}(L),
$$
and that $\theta(X) \cdot \omega$ is called {\em the Lie derivative}
of 
$\omega$ {\em with respect to $X$.} If $\omega\in \mathscr{E}^{r}(L)$,
it is easy to see that $\theta_{X}\omega\in \mathscr{E}^{r}(L)$.

\setcounter{subsection}{9}
\subsection{}\label{chap0:0.6.10}

We can see that
\begin{align*}
& \theta_{X}\omega(X_{1},\ldots,X_{r})=X(\omega(X_{1},\ldots,X_{r}))-\sum^{r}_{i=1}\omega(X_{1},\ldots,[X,X_{i}],\ldots,
X_{r})\\
&\forall X_{1},\ldots,X_{r}\in\mathscr{C}(L).
\end{align*}

\subsection{}\label{chap0:0.6.11}

Further we recall that
$$
\theta(X)\omega=0\Leftrightarrow \gamma^{\ast}_{t}(\omega)=\omega\,\forall t. 
$$
In this case we say that $\omega$ {\em is invariant by $X$} or {\em
under the flow of $X$.} We also have
\begin{equation*}
\theta(X)=i(X)\circ d+d\circ i(X).\tag{0.6.12}\label{chap0:0.6.12}
\end{equation*}


