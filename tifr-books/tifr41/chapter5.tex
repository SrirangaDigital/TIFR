\chapter{The structure of the Clifford Functor}\label{chap5} %Chapter 5

In\pageoriginale this chapter we introduce the category,
$\underline{\underline{\rm Quad}}(k)$ of quadratic forms on projective
$k$-modules, and the hyperbolic functor,
$\mathbb{H}:\underline{\underline{P}}\to
\underline{\underline{\rm Quad}}$. This satisfies the conditions of
chapter \ref{chap1} to yield an exact sequence, 
$$
K_1 \underline{\underline{P}} \to K_1  \underline{\underline{\rm Quad}}
\to K_0 \Phi \mathbb{H}\to K_0 \underline{\underline{P}}\to K_0
\underline{\underline{\rm Quad}} \to \text{Witt}(k)\to 0, 
$$ 
where Witt $(k)$ = coker $(K_0 \mathbb{H})$ is the classical ``Witt
ring'' over k.  

The Clifford algebra is constructed as a functor from $
\underline{\underline{\rm Quad}}$ to k-algebras, graded mod 2, and the
main structure  theorem (\S \ref{chap5:sec3}) asserts that the
Clifford algebra are 
(graded) azumaya algebras, in the sense of Chapter \ref{chap4} and that the
diagram  
\[
\xymatrix{
\underline{\underline{P}} \ar[r]^{\mathbb{H}}\ar[d]_{\wedge} &
\underline{\underline{\rm Quad}}  \ar[d]^{C1}\\
\underline{\underline{FP}}_2 \ar[r]_{\rm END} &
\underline{\underline{Az}}_2 
}
\]
commutes up to natural isomorphism. Here $\wedge$ denotes exterior
algebra, graded mod 2 by even and odd degrees. The proof is achieved
by a simple adaptation of arguments in Bourbaki \cite{key2}. 

This commutative diagram leads to a map of exact sequences 
{\fontsize{9pt}{11pt}\selectfont
\[
\xymatrix@C=.5cm{
K_1 \underset{=}{P} \ar[r] \ar[d] & K_1 \underline{\underline{\rm
    Quad}} \ar[r] \ar[d] & K_0 \Phi \mathbb{H} \ar[r] \ar[d]& K_0
\underset{=}{P} \ar[r] \ar[d] & K_0 \underline{\underline{\rm Quad}}
\ar[r] \ar[d] & \text{Witt} (k) \ar[r] \ar[d] & 0\\
K_1 \underline{\underline{FP}}_2  & K_1 \underline{\underline{Az}}_2
\ar[r] & K_0 \Phi END \ar[r] & K_0 \underline{\underline{FP}}_2 \ar[r]
& K_0 \underline{\underline{Az}}_2 \ar[r] & Br_2 (k) \ar[r] & 0
}
\]
}\relax

This\pageoriginale map of exact sequences is the ``generalized Hasse-Wall
invariant.''  

In \S \ref{chap5:sec4} we indicate briefly what the construction of the
spinor norm looks like in this generality.   

\section{Bilinear modules}\label{chap5:sec1}%Section 1

We shall consider modules over a fixed commutative ring $k$, and we
shall abbreviate,  
$$
\otimes = \otimes _k, \Hom = \Hom _k, M^* = \Hom (M, k). 
$$ 
Bil $(P \times Q)$ denotes the module of $k$-bilinear maps, $P \times
Q \to k$.  

Let $P$ be a $k$-module. If $x \in P$ and $y \in P^*$ write
$$
\langle y, x \rangle_P =  y (x). 
$$ 
If $f : P \to Q$ then $f^\ast : Q^\ast \to P^\ast$ is defined by  
$$
\langle f^\ast y, x \rangle_Q = \langle y, f X \rangle_P ~ (x \in P, y
\in Q^\ast). 
$$
There are natural isomorphisms
\begin{gather*}
\Hom (P, Q^*) \xleftarrow{s} \Bil(P \times Q) \xrightarrow{d} \Hom (Q,
P^\ast)\\ 
s_B \longleftarrow B \longrightarrow d_B
\end{gather*}
defined by
$$
\langle S_B x, y \rangle_Q = B (x, y) = \langle d_B y, x \rangle_P (x 
\in P, y \in Q). 
$$
 
Applying this to the natural pairing 
$$
\langle~ , ~ \rangle_P : P^* \times P \to k,  
$$
we obtain the natural homomorphism 
$$
d_P : P \to P^{\ast \ast} ; \langle d_P x,  y \rangle _{P*} = \langle y , x
\rangle_P. 
$$
We\pageoriginale call $P$ \textit{reflexive} if $d_P$ is an
isomorphism, and we will then often identify $P$ and $P^{**}$ via
$d_P$.  

Suppose $B \epsilon \Bil(P \times Q)$, $x \in P$, and $y \epsilon Q$. Then
$$
\langle d_B y, x \rangle_P = \langle s_B x, y \rangle_Q = \langle d_Q
y, s_B x \rangle _{Q^\ast}= \langle s^\ast_B d_Q y, x \rangle_P. 
$$ 
From this and the dual calculation we conclude:
\begin{equation*}
d_B = s^\ast_B d_Q \text{ and } s_B = d^\ast_B d_P. \tag{1.1}\label{eq1.1}
\end{equation*}
We call $B$ \textit{non-singular} if $d_B$ and $s_B$ are
isomorphisms. In view of (\ref{eq1.1}) this implies $P$ and $Q$ are
reflexive. Conversely, if $P$ and $Q$ are reflexive, and if $d_B$ is
an isomorphism, then (\ref{eq1.1}) shows that $s_B$ is also. 

A pair $(P, B)$, $B \in \Bil(P \times P)$, is called a
\textit{bilinear module}. $f : P_1 \to P_2$ is a \textit{morphism}
$(P_1, B_1) \to (P_2, B_2)$ if $B_2(fx, fy) = B_1(x, y)$ for $x$, $y
\epsilon P_1$. We define 
$$
(P_1, B_1) \perp (P_2, B_2) = (P_1 \oplus P_2, B_1 \perp B_2) 
$$
and 
$$
(P_1, B_1) \otimes (P_2, B_2) = (P_1 \otimes P_2, B_1 \otimes B_2) 
$$
by $(B_1 \perp B_2)((x_1, x_2)$, $(y_1, y_2))= B_1(x_1, y_1)+B_2 (x_2,
y_2)$, and $(B_1 \otimes B_2)(x_1 \otimes x_2, y_1 \otimes y_2)=B_1
(x_1, y_1) B_2(x_2, y_2)$. Identifying $(P_1 \oplus P_2)^* = P^*_1
\oplus P^*_2$ we have $d_{B_1 \perp B_2} = d_{B_1} \oplus
d_{B_2}$. Moreover, $d_{B_1 \otimes B_2}$ is $d_{B_1} \otimes d_{B_2}$
followed by the natural map $P^*_1 \otimes P^*_2 \to (P_1 \otimes
P_2)^*$. The latter is an isomorphism if one of the $P_i$ is finitely
generated and projective. 

If\pageoriginale $(P, B)$ is a bilinear module we shall write $B^\ast(x,
y) = B(y, x)$. If $P$ is reflexive and we identify $P = P^{\ast\ast}$ then
(\ref{eq1.1}) shows that $d_{(B^\ast)} = s_B = (d_B)^\ast$. We call
$(P, B)$ or $B$ \textit{symmetric} if $B = B^\ast$. For any $B$,
$B+B^\ast$ is clearly symmetric.  

If $(P, B)$ is a symmetric bilinear module we have a notion of
\textit{orthogonality}. Specifically, if $U$ is a subset of $P$, write  
$$
P^U = \{ x \in P|B(x, y)= 0 \; \forall y \in U \}. 
$$

When $P$ is fixed by the context we will sometimes write
$$
U^\perp = P^U.
$$
The following properties are trivial to verify:
\begin{align*}
& U^\perp \text{ is a submodule of } P. \\
& U \subset V \Rightarrow V^\perp \subset U^\perp \\
& U \subset U^{\perp \perp} \\
& U^\perp = U.^{\perp \perp \perp}
\end{align*}

We say $U$ and $V$ are \textit{orthogonal} if $U \subset V^\perp$, and
we call a submodule $U$ \textit{totally isotropic} if $U \subset
U^\perp$, i.e. if $B(x, y)=0$ for all $x$, $y \epsilon U$. The
expression $P = U \perp V$ denotes the fact that $P$ is the direct sum
of the orthogonal submodules $U$ and $V$. 

\setcounter{lemma}{1}
\begin{lemma}\label{chap5:lem1.2}%Lem 1.2
Let $(P, B)$ be a non-singular symmetric bilinear module. If $U$ is a
direct summand of $P$ then $U^\perp$ is also a  direct summand, and
$B$ induces a non-singular pairing on $U \times (P/U^\perp)$. 
\end{lemma}

\begin{proof}
Since\pageoriginale $0 \to U \to P \to P/U \to 0$ splits so does $0
\to (P/U)^\ast \to P^\ast \to U^\ast \to 0$. By hypothesis $d_B : P
\to P^\ast$ is an isomorphism, 
so $U^\perp = d^{-1}_B (P/U)^\ast$ is a direct summand of $P$. Moreover
the composite $P \xrightarrow{d_B} P^\ast \to U^\ast$ is surjective, with
kernel $U^\perp$, so $B$ induces an isomorphism $(P/U^\perp) \to
U^\ast$. Since $U$ and $(P/U^\perp)$ are reflexive this implies the
pairing on $U \times (P/U^\perp)$ is non-singular (see(1.1)). 
\end{proof}

\begin{lemma}\label{chap5:lem1.3} %Lem 1.3
 Let $f: (P_1, B_1) \to (P_2, B_2)$ be a morphism of symmetric
 bilinear modules, and suppose that $(P_1, B_1)$  is
 non-singular. Then $f$ is a monomorphism, and 
$$
P_2 = fP_1 \perp P^{(fP_1)}_2 
$$
\end{lemma}

\begin{proof}
If $x \epsilon \ker f$ then $0 = B_2 (fx, fy) = B_1(x, y)$ for all $y
\in P_1$ so $x =0$ because $B_1$ is non-singular. Now use $f$ to
identify $P_1 \subset P_2$ and $B_1 = B_2 |P_1 \times P_1$. Then $P_1
\cap P^{P_1}_2 = 0$ because $B_1$ is non-singular. If $x \in P_2$
define $h: P_1 \to k$ by $h(y) = B_2 (x \cdot y)$. Since $B_1$ is
non-singular $h(y) = B_1(x_1, y)$ for some $x_1 \epsilon P_1$, and
then we have $x = x_1 + (x-x_1)$ with $x-x_1 \in P^{P_1}_2$. 
\end{proof}

\begin{lemma}\label{chap5:lem1.4}%Lem 1.4
 Let $(P, B)$ be a non-singular symmetric bilinear module and suppose
 that $U$ is a totally isotropic direct summand of $P$. 
\begin{enumerate}[(a)]
\item  We can write $P = U^\perp \oplus V$, and, for any such $V$, $W
  =U \oplus V$ is a non-singular bilinear submodule of $P$. Hence $P =
  W \perp P^W$.

\item $V \approx U^\ast$, so if $U$ is finitely generated and projective
  then so is $W$, and $[W : k] =2[U : k]$. 

\item  If $ B = B_0 + B^\ast_0$ and if $B_0 (x, x) = 0$ for all $x
  \epsilon U$ then we can choose $V$ above so that $B_0 (x, x) =0$ for
  all $x \epsilon V$ also. 
\end{enumerate}
\end{lemma}

\begin{proof}%Proof
\begin{enumerate}[(a)]
\item According\pageoriginale to Lemma \ref{chap5:lem1.2}, $P =
  U^\perp \oplus V$, 
  and for any such 
  $V$, $B$ induces a non-singular form on $U \times V$. Thus $B$
  induces isomorphisms $f: U \to V^*$ and $g: V \to U^*$. If $B_1
  = B|W \times W$ then $d_{B_1}: U \oplus V \to U^* \oplus V^*$ is
  represented by a matrix $\left( \begin{smallmatrix} 0 & g \\ f &
    d_{B_2} \end{smallmatrix} \right)$, where $B_2 = B|V \times
  V$. Evidently $d_{B_1}$ is an isomorphism. Lemma \ref{chap5:lem1.3}
  now implies $P   = W \perp P^W$. 

\item is clear

\item Identifying $U = U^{\ast \ast}$ and $V=V^{\ast\ast}$, the symmetry of $B$
  implies $f^\ast = g$. Let $B_3 = B_0 | V \times V$, where $B = B_0 +
  B^\ast_0$ (by hypothesis), and set $k = f^{-1} d_{B_3}: V \to
  U$. Then for $v \in V$ we have  
$$
B(v, hv)= \langle fhv, v \rangle_V = \langle ff^{-1} d_{B_3} v, v 
\rangle_V = B_3 (v, v) = B_0 (v, v). 
$$
\end{enumerate}
\end{proof}

Let $t: V \to U \oplus V$ by $t(v) = v-h(v)$. Then if $V_1 = tV$ it
is still clearly true that $P = U^\perp \oplus V_1$ (in fact, $W = U
\oplus V_1$). We conclude the proof by showing that $B_0(v, v)=0$ for
$v \epsilon V_1$. 
Suppose $v \epsilon V$. Then 
\begin{align*}
B_0 (tv, tv) & = B_0(v-hv, v-hv) = B_0(v,
v) + B_0(hv, hv)-\\
& - B_0 (v, hv) - B_0 (hv, v). 
\end{align*}

Since $hv \in U$ and $B_0 (x, x) = 0$ for $x \in U$, by hypothesis,
and since $B= B_0 + B^\ast_0$, we have $B_0(tv, tv)=B_0 (v, v) - B_0(v,
hv)- B^\ast_0 (v, hv)= B_0 (v, v) - B(v, hv)$. This vanishes
according to the calculation above, so lemma \ref{chap5:lem1.4} is proved. 

Let $P$ be a module and $B \in \Bil (P \times P)$. We define the
function 
$$
q = q_B : P \to k; q(x) = B(x, x).
$$
$q$ has\pageoriginale the following properties:
\begin{equation*}
q(ax) = a^2 x  \quad (a \in k, x \in P), \tag{1.5}\label{eq1.5}
\end{equation*}

If $B_q(x, y) = q(x+y) - q(x) -q(y)$, then $B_q \in \Bil (P \times
P)$. Indeed, direct calculation shows that $B_q = B+B^*$. 

\setcounter{lemma}{5}
\begin{lemma}%lem 1.6
 Suppose $P$ is finitely generated and projective, and that $q$: $P
 \to k$ satisfies (\ref{eq1.5}). Then there is a $B \in \Bil (P \times P)$
 such that $q =q_B$. In particular, $B_q = B+B^*$. 
\end{lemma}

\begin{proof}
If $P$ is free with basis $(e_i)_{1 \leq i \leq n}$ then
$q(\sum \limits_i a_i e_i) = \sum\limits_i a^2_i q(e_i) +
\sum\limits_{i < j} a_i a_j B_q (e_i, e_j)$. Set $b_{ii} = q(e_i)$, 
$b_{ij} = B_q (e_i, e_j)$ for $i < j$, and $b_{ij}=0$ for $i >j$.
Then $q (\sum\limits_i a_i e_i) = \sum\limits_{i, j}a_i a_j
b_{ij} = B (\sum\limits_i a_i e_i, \sum\limits_i a_i e_i)$, where
$B(\sum\limits_i a_i e_i, \sum\limits_i c_i e_i) = \sum\limits_{i,
  j} a_i c_j b_{ij}$.  

In the general case choose $P'$ so that $F= P \oplus P'$ is free and 
extend $q$ to $q_1$ on $F$ by $q_1(x, x') = q(x)$ for $(x, x') \in P
\oplus P'$. If $q_1 = q_{B_1}$ then $q = q_B$ where $B = B_1 \big| P
\times P$.  

We define a \textit{quadratic form} on a module $P$ to be a function
of the form $q_B$ for some $B \in \Bil (P \times P)$. $B$ is then
uniquely determined modulo ``alternating forms,'' i.e. those $B$ such
that $B(x, x) = 0$ for all $x \in P$. We shall call the pair $(P, q)$
\textit{a quadratic module}, and we call it \textit{non-singular} if
$B_q$ is non-singular. $f: P_1 \to P_2$ is a \textit{morphism}
$(P_1, q_1) \to (P_2, q_2)$ of quadratic modules if $q_2(fx)=q_1(x)$
for all $x \epsilon P_1$. Evidently $f$ then induce a morphism $(P_1,
B_{q_1}) \to (P_2, B_{q_2})$ of the associated bilinear modules. 

If\pageoriginale $f$ is an isomorphism we call $f$ an
\textit{isometry}. If $q_i = 
q_{B_i}$ then we define $q_1 \perp q_2 = q_{B_1 \perp B_2}$ on $P_1
\oplus P_2$, and $q_1 \otimes q_2 = q_{B_1 \otimes B_2}$ on $P_1
\otimes P_2$. It is easily checked that these definition are
unambiguous. 
\end{proof}

\section{The hyperbolic functor}\label{chap5:sec2}%Sec 2

Let $P$ be a $k$-module and define
$$
B^P_0 \in \Bil (( P \oplus P^\ast) \times (P \oplus P^\ast)) \text{ by }  
B^P_0 ((x_1, y_1), (x_2, y_2)) = \langle y_1, x_2 \rangle_P,
$$
and let $q^P = q_{B^P_0}$ be the induced quadratic form:
$$
q^P (x, y) = \langle y, x \rangle_P \qquad (x \in P, y \in P^*). 
$$
Let $B^P = B^P_0 + (B^P_0)^\ast$ be the associated bilinear form, $B^P =
B_{q^P}$. Then 
$$
B^P (( x_1, y_1), (x_2, y_2)) = \langle y_1, x_2 \rangle_P + \langle 
y_2, x_1 \rangle_P. 
$$
If $d_P: P \to P^{\ast\ast}$ is the natural map then it is easily
checked that  
$$
d_{B^P}: P \oplus P^\ast \to (P \oplus P^\ast)^\ast = P^\ast \oplus P^{\ast\ast} 
$$
is represented by the matrix
$$
\begin{pmatrix} 
0 & 1_{P^\ast} \\
d_P & 0 
\end{pmatrix}.
$$
Consequently, $B^P$ \textit{is non-singular if and only if $P$ is
  reflexive}. If, in this case, we identify $P = P^{**}$ then the
matrix above becomes $ \left( \begin{smallmatrix} 0 & 1_{P^*} \\ 1_P &
  0 \end{smallmatrix} \right)$. 

We\pageoriginale will write
$$
\mathbb{H} (P) = (P \oplus P^\ast, q^P)
$$
and call this quadratic module the \textit{hyperbolic form} on $P$.

Suppose $f: P \to Q$ is an isomorphism of $k$-modules. Define 
\begin{gather*}
\mathbb{H}(f)  = f \oplus (f^\ast)^{-1} : \mathbb{H} (P) \to
\mathbb{H} (Q). \\ 
q^Q (\mathbb{H} (f) (x, y)) = q^Q (fx, (f^\ast)^{-1} y) = \langle
(f^{-1})^\ast y, fx \rangle_Q \\ 
= \langle y, f^{-1} fx \rangle_P = q^P (x, y), \text{ so }
\mathbb{H}(f) \text{ is an isometry}. 
\end{gather*}

If we identify $(P_1 \oplus P_2)^\ast = P^\ast_1 \oplus P^\ast_2$ so
that  
$$
\langle (y_1, y_2), (x_1, x_2) \rangle_{P_1 \oplus P_2} = \langle
y_1, x_1 \rangle_{P_1} + \langle y_2, x_2 \rangle_{P_2} 
$$
then the natural homomorphism
$$
f : \mathbb{H} (P_1) \perp \mathbb{H} (P_2) \to \mathbb{H} (P_1 \oplus 
P_2), 
$$
$f((x_1, y_1)$, $(x_2, y_2)) = ((x_1, x_2)$, $(y_1, y_2))$. is an
isometry. 

Summarizing the above remarks, $\mathbb{H}$ \textit{is a product
  preserving func\-tor} (in the sense of chapter \ref{chap1}) \textit{from
  (modules, isomorphisms}, $\oplus$) to (\textit{quadra\-tic modules,
  isometries,} $\perp$). We now characterize non-singular hyperbolic
forms. 

\begin{lemma}\label{chap5:lem2.1}%lem 2.1
A non-singular quadratic module $(P, q)$ is hyperbolic if and only if
$P$ has a direct summand $U$ such that $q|U =0$ and $U = U^\perp$. In
this case $(P, q) \approx \mathbb{H}(U)$ (isometry). 

Suppose\pageoriginale $P$ is finitely generated and projective. 
If $U$ is a direct summand such that $q|U =0$ and $[P : k] \leq 2 [ U:
  k]$ then $(P, q) \approx \mathbb{H}(U)$. 
\end{lemma}

\begin{proof}%Proof
If $(P, q) \approx \mathbb{H}(U) = (U \oplus U^\ast, q^U)$ then the
non-singularity of $(P, q)$ implies $U$ is reflexive, and it is easy
to check that $U \subset U \oplus U^\ast$ satisfies $q^U |U =0$ and $U =
U^\perp$. 
\end{proof}

Conversely, suppose given a direct summand $U$ of $P$ such that $q|U
=0$ and $U = U^\perp$. Write $q=q_{B_0}$, so that $B_q = B_0 +
B^\ast_0$. According to Lemma \ref{chap5:lem1.4} we can write $P =
U^\perp \oplus V = U 
\oplus V$ and $B_q$ induces a non-singular pairing on $U \times
V$. Moreover we can arrange that $B_0 (v, v) =0$ for all $v \in V$,
i.e. that $q|V =0$. Let $d: V \to U^*$ be the isomorphism induced by
$B_q$; $\langle dv, u  \rangle_U = B_q (v, u)$ for $u \in U$, $v \in
V$. 

Let 
$$
f= 1_U \oplus d : P = U \oplus V \to U \oplus U^*. 
$$
This is an isomorphism, and we want to check that 
$$
q^U((u, dv)) = q(u, v) \text{ for } u \epsilon U, v \epsilon
V. q^U((u, dv)) = \langle dv, u \rangle_U = B_q (v, u), 
$$
while $q(u, v) = q(u) + q(v) + B_q(u, v) = B_q (u, v)$, since $q/U =0$
and $q/V=0$. 

The last assertion reduces to the preceding ones we show that $U =
U^\perp$. Lemma \ref{chap5:lem1.2} shows that $U^\perp$ is a direct
summand of rank 
$[ U^\perp : k] = [ P : k]- [ U : k ]\leq [ U : k]$, because, by
assumption, $[P : k] \leq 2 [ U : k]$. But we also have $q/U = 0$ so
$U \subset U^\perp$, and therefore $U = U^\perp$, as claimed. 

\begin{lemma}\label{chap5:lem2.2} %Lem 2.2
A quadratic module $(P, q)$ is non-singular if and only if
$$
(P, q) \perp (P, -q) \approx \mathbb{H} (P), 
$$
provided $P$ is reflexive.
\end{lemma}

\begin{proof}%Proof
$P$ reflexive\pageoriginale implies $\mathbb{H}(P)$ is non-singular, and hence
  likewise for any orthogonal summand. 

Suppose now that $(P, q)$ is non-singular. Then so is $(P, q) \perp
(P, -q) = (P \oplus P, q_1 = q \perp (-q))$. 

Let $U = \{(x, x) \epsilon P \oplus P| x \in P\}$. Then $q_1/U = 0$,
and $U$ is a direct summand of $P \oplus P$, isomorphic to $P$. If $U
\varsubsetneqq U^\perp$ we can find a $(0, y) \epsilon U^\perp$, $y
\neq 0$. Then, for all $x \in P$, 
\begin{align*}
0 & = B_{q_1} ((x,x), (0, y)) = q_1 (x, x+y) - q_1 (x, x) - q_1 (0, y) \\
& = q(x) - q(x+y) + q(y) \\
& = - B_q (x, y).
\end{align*}
Since $B_q$ is non-singular this contradicts $y \neq 0$. Now the
Lemma follows from Lemma \ref{chap5:lem2.1}. 
\end{proof}

\begin{lemma}\label{chap5:lem2.3}% lem 2.3
 Let $P$ be a reflexive module and let $(Q, q)$ be a non-singular
 quadratic module with $Q$ finitely generated and projective. Then 
$$
\mathbb{H} (P) \otimes (Q, q) \approx \mathbb{H} (P \otimes Q). 
$$
\end{lemma}

\begin{proof}
The hypothesis on $Q$ permits us to identify $(P \otimes Q)^\ast = P^\ast
\otimes Q^\ast$, so it follows that $(W, q_1) = \mathbb{H} (P) \otimes
(Q, q)$ is non-singular. We shall apply Lemma \ref{chap5:lem2.1} by taking 

$U = P \otimes Q \subset W = (P \otimes Q) \oplus (P^\ast \otimes Q)$. If
$\sum x_i \otimes y_i \epsilon U$, then $q_1 (\Sigma x_i \otimes y_i)
= \Sigma q^P (x_i) q (y_i) + \sum_{i < j} B_{q_1} (x_i \otimes y_i,
x_j \otimes y_j) = \sum_{i < j} B^P (x_i, x_j) B_q (y_i, y_j) =0$,
because $q^P/P =0$ in $\mathbb{H}(P)$. Thus $U \subset U^\perp$, and
to show equality it suffices clearly to show that  
$(P^* \otimes Q) \cap U^\perp = 0$. If $\Sigma x_i  \otimes y_i 
\in U$ and $\Sigma w_j \otimes z_j \in (P^* \otimes Q) \cap
U^\perp$ then $0 = B_{q_1} ( \Sigma x_i \otimes y_i$, $ \Sigma w_j \otimes z_j)
= \sum \limits_{i, j} B^P (x_i, w_j) B_q (y_i, z_j)$. 
\end{proof}

Since\pageoriginale $(P^\ast \otimes Q)^\ast = P \otimes Q^\ast$ ($P$
is reflexive) the 
non-singularity of $q$ guarantees that all linear functionals on $P^\ast
\otimes Q$ have the form $\sum_i B^P(x_i, )\break B_q (y_i, )$, so $\Sigma
w_j \otimes z_j$ is killed by all linear functionals, hence is
zero. We have now shown $U = U^\perp$ so the lemma follows from Lemma
\ref{chap5:lem2.1}. 

A \textit{quadratic space} is a non-singular quadratic module $(P, q)$
with $P$ finitely generated and projective, i.e. $P \epsilon \obj
\underset{=}P$, the category of such modules. We define the category 
$$
\underline{\underline{\rm Quad}} = \underline{\underline{\rm Quad}} (k)
$$
with
\begin{align*}
& \text{ objects } : \text{ quadratic spaces } \\
& \text{ morphisms } : \text{ isometries } \\
& \text{ product } : \perp 
\end{align*}
The discussion at the beginning of this section shows that
$$
\mathbb{H} : \underset{=}P \to \underline{\underline{\rm Quad}}
$$
is a product preserving functor of categories with product (in the
sense of chapter \ref{chap1}), and Lemma \ref{chap5:lem2.1} shows that
$\mathbb{H}$ is 
cofinal. We thus obtain an exact sequence from Theorem
\ref{chap1:thm4.6} of chapter \ref{chap1}. We summarize this: 

\setcounter{prop}{3}
\begin{prop}% prop 2.4
 The hyperbolic functor
$$
\mathbb{H} : \underset{=}P \to \underline{\underline{\rm Quad}}
$$
is a cofinal functor of categories with product. It therefore
  induces (Theorem \ref{chap1:thm4.6} of chapter \ref{chap1}) an exact
  sequence  
$$
K_1 \underset{=}P \to K_1 \underline{\underline{\rm Quad}} \to K_0
\Phi \mathbb{H} \to K_0 \underset{=}P \to K_0
\underline{\underline{\rm Quad}} \to {\rm Witt}(k) \to 0, 
$$
where we define Witt $(k) =$ coker $(K_0 \mathbb{H})$.
\end{prop}

We\pageoriginale close this section with some remarks about the
multiplicative structures. Tensor products endow $K_0
\underline{\underline{\rm Quad}}$ with a  
commutative multiplication, and Lemma \ref{chap5:lem2.3} shows that
the image of $K_0 
\mathbb{H}$ is an ideal, so Witt $(k)$ also inherits a
multiplication. The difficulty is that, if 2 is not invertible in $k$,
then these are rings without identity elements. For the identity
should be represented by the form $q(x) = x^2$ on $k$. But then
$B_q(x, y) = 2xy$ is not non-singular unless 2 is invertible. 

Here is one natural remedy. Let $\underline{\underline{\rm Symbil}}$
denote the category of non-singular symmetric bilinear forms, $(P, B)$
with $P \epsilon \obj \underset{=}P$. If $(P, B) \in
\underline{\underline{\rm Symbil}}$ and 
$(Q, q) \epsilon \underline{\underline{\rm Quad}}$ define 
\begin{equation*}
(P, B) \otimes (Q, q) = (P \otimes Q, B \otimes q), \tag{2.5}\label{eq2.5}
\end{equation*}
where $B \otimes q$ is the quadratic form $q_{B \otimes B_0}$, for
some $B_0 \epsilon \Bil (Q \times Q)$ such that $q = q_{B_0}$. It is
easy to see that $B \otimes q$ does not depend on the choice of
$B_0$. Moreover, the bilinear form associated to $B \otimes q$ is $(B
\otimes B_0) + (B \otimes B_0)^\ast = (B \otimes B_0) + (B^\ast \otimes
B^\ast_0) = B \otimes (B_0 \otimes B^\ast_0)= B \otimes B_q$, because $B =
B^\ast$. Since $B$ and $B_q$ are non-singular so is $B \otimes B_q$ so
$(P \otimes Q, B \otimes q) \in \underline{\underline{\rm Quad}}$. 

If $a \epsilon k$ write $\langle a \rangle$ for the bilinear module
$(k, B)$ with $B(x, y) = axy$ for $x, y \epsilon k$. If a is a unit
then $\langle a  \rangle \epsilon \underline{\underline{\rm Symbil}}$. 

Tensor\pageoriginale products in $\underline{\underline{\rm Symbil}}$
make $K_0 \underline{\underline{\rm Symbil}}$ a commutative ring, with
identity $\langle 1 \rangle$, and (\ref{eq2.5}) 
makes $K_0 \underline{\underline{\rm Quad}}$ a $K_0
\underline{\underline{\rm Symbil}}-$module. The
``forgetful'' functor $\underline{\underline{\rm Quad}}$ $\to
\underline{\underline{\rm Symbil}}$, $(P, 
q) \longmapsto (P, B_q)$, induces a $K_0$ \,
$\underline{\underline{\rm Symbil}}$-homomorphism $K_0
\underline{\underline{\rm Quad}}\to K_0 
\underline{\underline{\rm Symbil}}$, so its image is an ideal. The\break
hyperbolic forms generate a $K_0 \underline{\underline{\rm Symbil}}$
submodule, image $K_0 \mathbb{H})$, of\break $K_0 \underline{\underline{\rm
    Quad}}$, so Witt $(k)$ is a $K_0   
\underline{\underline{\rm Symbil}}$-module. This follows from an
analogue of Lemma \ref{chap5:lem2.3} for the operation (\ref{eq2.5}) 

Similarly, the hyperbolic forms, $(P \oplus P^*, B^P)$, generate an
ideal in $K_0 \underline{\underline{\rm Symbil}}$ which annihilates
$Witt (k)$. Lemma \ref{chap5:lem2.2} says that $\langle 1 \rangle
\perp \langle -1 \rangle$ also annihilates Witt $(k)$.  


\section{The Clifford Functor}\label{chap5:sec3}%Sec 3

If $P$ is a $k$-module we write
$$
T(P) = (k) \oplus (P) \oplus (P \otimes P) \oplus \ldots \oplus 
(P^{\otimes n}) \oplus \ldots 
$$
for its tensor algebra. If $(P, q)$ is a quadratic module then its
\textit{Cliford algebra} is 
$$
Cl (P, q) = T(P) / I (q), 
$$ 
where $I(q)$ is the two sided ideal generated by all $x \otimes x -
q(x)(x \in P)$. If we grade $T(P)$ by even and odd degree (a $(
\mathbb{Z} /2 \mathbb{Z})$-grading) then $x \otimes x - q(x)$ is
homogeneous of even degree, so 
$$ 
Cl (P, q) = Cl_0 (P, q) \oplus Cl_1 (P, q) 
$$
is a graded algebra in the sense of chapter \ref{chap4}, \textit{We will
  consider} $Cl (P, q)$ \textit{to be a graded algebra}, and this must
be borne in mind when we discuss tensor products. 

The\pageoriginale inclusion $P \subset T(P)$ induces a $k$-linear map
$$
C_P : P \to Cl (P, q)
$$
such that $C_P(x)^2 = q(x)$ for all $x \in P$, and $C_P$ is clearly
universal among such maps of $P$ into a $k$-algebra. 

$Cl$ evidently defines a functor from quadratic modules, and their
morphisms, to graded algebras, and their homomorphisms (of degree
zero). Moreover it is easy to check that both $T$ and $Cl$ commute
with base change, $k \to K$. The next lemma says $Cl$ is ``product
preserving,'' in an appropriate sense. 

\begin{lemma}\label{chap5:lem3.1}% lem 3.1
There is a natural isomorphism of (graded) algebras, 
$$
Cl ((P_1, q_1) \perp (P_2, q_2)) \approx Cl (P_1, q_1) \otimes Cl
(P_2, q_2). 
$$
\end{lemma}

\begin{proof}%proof 
$P_i \to P_1 \otimes P_2 \xrightarrow{C_{P_1 \oplus P_2}} Cl ((P_1,
  q_1) \perp (P_2, q_2))$ induces an algebra homomorphism, 
$$
f_i : Cl (P_i, q_i) \to Cl ((P_1, q_1) \perp (P_2, q_2)), i= 1, 2. 
$$
\end{proof}

If $x_i \in P_i$ then $(f_1 x_1 + f_2 x_2)^2 = (q_1 \perp q_2
)(x_1, x_2) = q_1 x_1 + q_2 x_2  = (f_1 x_1)^2 + (f_2 x_2)^2$, so $f_1
x_1$ and $f_2 x_2$, being of odd degree, commute (in the graded
sense). Therefore so also do the algebras they generate, $\im f_1$
and $\im f_2$. Hence $f_1$ and $f_2$ induce an algebra homomorphism 
$$
F : Cl (P_1, q_1) \otimes Cl (P_2, q_2) \to Cl ((P_1, q_1) \perp (P_2,
q_2)), 
$$
and this is clearly natural. To construct its inverse let  
$$
g : P_1 \oplus P_2 \to Cl ((P_1, q_1) \perp (P_2, q_2)) \text{ by }
g(x_1, x_2) = C_1 x_1 \otimes 1 + 1 \otimes C_2 x_2, 
$$
where\pageoriginale $C_i = C_{P_i}$. If $g$ extends to an algebra
homomorphism from 
$Cl \perp ((P_1, q_1) \perp (P_2, q_2))$ it will evidently be inverse
to $F$, since this is so on the generators, $C_{P_1 \oplus P_2}(P_1
\oplus P_2)$, and $C_1 P_1 \otimes 1+ 1 \otimes C_2 P_2$, respectively.
To show that $g$ extends we have to verify that $g(x_1, x_2)^2 = (q_1
\perp q_2) (x_1, x_2)$. $g(x_1, x_2)^2 = (C_1 x_1)^2 \otimes 1 + 1
\otimes (C_2 x_2)^2 + C_1 x_1 \otimes C_2 x_2 + (-1)^{(\deg C_1 x_1)
  (\deg C_2 x_2)} (C_1 x_1 \otimes C_2 x_2) = q_1 x_1 + q_2 x_2 = (q_1
\perp q_2 ) (x_1 , x_2)$. 

\setcounter{examples}{1}
\begin{examples}% exmmp 3.2
If $q$ is the quadratic form $q(x) = ax^2$ on $k$, denote this
quadratic module by $(k, a)$. Then $C1 (k, a) = k\langle a \rangle =
k1 \oplus kx$, $x^2 = a$. Thus $C1 (\mathbb{R}, -1) \approx \mathbb{C}
= \mathbb{R} 1 \oplus \mathbb{R}i$, for example. 
\begin{equation*}
C1 ((k, a) \perp (k, b)) \approx C1 (k, a) \otimes C1 (k, b) \approx
(\frac{a, b}{k}). \tag{3.3}\label{eq3.3}
\end{equation*}

The latter denotes the $k$-alegbra with free $k$-basis 1, $x_a$, $x_b$,
$y$, where $\deg x_a = \deg x_b = 1$, $x^2_a =a$, $x^2_b = b$, $y =
x_a x_b = -x_b x_a$. The degree zero component is  
$$
\left( \frac{a, b}{k}\right)_0 = k[y], y^2 = -ab. 
$$
For example, as a graded $\mathbb{R}$-algebra, $\mathbb{C}
\otimes_{\mathbb{R}} \mathbb{C} \approx ( \dfrac{-1-1}{k})$, the
standard quaternion algebra (plus grading). 
\begin{equation*}
\mathbb{H} (k) = (k \oplus k^\ast, q^k).  \tag{3.4}\label{eq3.4}
\end{equation*}

Let $e_1$ be a  basis for $k$ (e.g. $e_1 = 1$)
and $e_2$ the dual basis for $k^\ast$. Writing $q=q^k$ we have $q(a_1 e_1
+ a_2 e_2) = a_1 a_2$. Hence $C1 (\mathbb{H}(k))$ is generated by
elements $x_1$ and $x_2$ (the images of $e_1$ and $e_2$) of degree 1
with the relations $x^2_1 = 0 = x^2_2$ and $x_1 x_2 + x_2 x_1 = 1$. In
$\mathbb{M}_2 (k)$ the matrices $y_1 = \left( \begin{smallmatrix} 0 &
  1 \\ 0 & 0 \end{smallmatrix}\right)$ and $ y_2=
\left( \begin{smallmatrix} 0 & 0 \\ 1 & 0 \end{smallmatrix}\right)$
satisfy these relations, so\pageoriginale there is a homomorphism 
\begin{align*}
Cl (\mathbb{H}(k)) &\to \mathbb{M}_2 (k) \\
x_i &\mapsto y_i
\end{align*}

It is easy to check that this is an isomorphism. This isomorphism is
the simplest case of Theorem below, which we prepare for in the
following lemmas. 
\end{examples}

\setcounter{lemma}{4}
\begin{lemma}\label{chap5:lem3.5}%Lem 3.5
 Let $P$ be a $k$-module. There is a $k$-linear map $P^\ast \to \Hom_k$
 $(T(P), T(P))$, $f \mapsto d_f$, where $d_f$ is the unique map of
 degree $-1$ on $T(P)$ such that $d_f (x \otimes y) = f(x) y - x
 \otimes d_f (y)$ for $x \epsilon P$, $y \in T (P)$. Moreover $d^2_f =0$
 and $d_f d_g + d_g d_f = 0$ for $f$, $g \epsilon P^*$. If $q$ is a
 quadratic form on $P$ then $d_f I (q) \subset I(q)$, so $d_f$ induces
 a $k$-linear map, also denoted $d_f$, of degree one on $Cl(P, q)$. 
\end{lemma}

\begin{proof}%Proof
$d_f$ is defined on $P^{\otimes (n+1)} = P \otimes P^{(\otimes n)}$ by
  induction on $n$, from the formula given. This shows uniqueness, and
  that  
$$
d_f (x_0 \otimes \ldots \otimes x_n) = \sum_{0 \leq i \leq n} (-1)^i
fx_i (x_0 \otimes \ldots \hat{i} \ldots x_n). 
$$
\end{proof}

Hence 
\begin{align*}
d^2_f (x_0 \otimes \ldots \otimes x_n) & = \sum_{0 \leq j < i \leq n}
(-1)^{i +j} (fx_i) (fx_j) (x_0 \otimes \ldots \hat{j} \ldots \hat{i}
\ldots \otimes x_n) \\ 
& + \sum_{0 \leq j < i \leq n} (-1)^{i +j-1} (fx_i) (fx_j) (x_0 \otimes
\ldots \hat{i} \ldots \hat{j} \ldots \otimes x_n) \\ 
& = 0.
\end{align*}

It\pageoriginale is easy to check that $f \mapsto d_f$ is $k$-linear,
so we have $0 = (d_{f +g})^2 = (d_f + d_g)^2$, and hence $d_f d_g +
d_g d_f =0$ for $f$, $g \epsilon P^*$.  

The formula above shows that if $x$, $x, y \in \, hT (P)$ (the set of 
homogeneous elements) then  
$$
d_f (x \otimes y) = d_f x \otimes y + (-1)^{\partial x} x \otimes d_f
y. 
$$
If $q$ is a quadratic form on $P$ write $u(x) = x \otimes x - q(x)$
for $x \epsilon P$. Since $u(x)$ has even degree the formula above
shows that $d_f (u(x) \otimes y) = d_f u (x) \otimes y + u (x) \otimes
d_f y$, and $d_f(u(x)) = f(x) x - f(x) x =0$, so $d_f (u(x) \otimes y)
\in I(q)$. If $v \in hT (P)$ then $d_f (v \otimes u(x) \otimes y) =
d_f v \otimes u (x) \otimes y + (-1)^{\partial v} v \otimes d_f (u(x)
\otimes y) \in I (q)$. Since $I(q)$ is additively generated by all
such $v \otimes u(x) \otimes y$ it follows that $d_f I (q) \subset I
(q)$. 

\begin{lemma}%Lem 3.6
If $B \in \Bil (P \times P)$ there is a unique $k$-linear map
$\lambda_B : T(P) \to T(P)$ satisfying  
\begin{enumerate}[{\rm(i)}]
\item $\lambda_B (1) = 1$

\item $\lambda_B L_x = (L_x + d_{B(x,\; )}) \lambda_B $ for $x \in P$. 
\end{enumerate}
(Here $L_x$ denotes left multiplication by $x$ in $T(P).) \lambda_B$
also has the following properties: 
\begin{enumerate}[{\rm(a)}]
\item $\lambda_B$ preserves the ascending filtration on $T(P)$
  and induces the identity map on the associated graded module. 

\item For $f \epsilon P^*$, $\lambda_B d_f=d_f \lambda_B$.

\item $\lambda_0 = 1_{T(P)}$ and $\lambda_{B+B'} = \lambda_B \circ
  \lambda_{B'}$ for $B$, ${B'} \epsilon $ Bill $(P \times P)$. 

\item If\pageoriginale $q$ is a quadratic form on $P$, then $\lambda_B
  I(q) = I(q-q_B)$, and $\lambda_B$ induces an isomorphism $Cl (P,
  q) \to Cl(P, q-q_B)$ of filtered modules. 
\end{enumerate}
\end{lemma}

\begin{proof}
Writing $xy$ in place of $x \otimes y$ in $T(P)$, (ii) reads:
$$
\lambda_B (xy) = x \lambda_B (y) + d_{B(x,\; )}(\lambda_B (y)) (x
\epsilon P, y \epsilon T(P)). 
$$
Starting with $\lambda_B (1) = 1$ this gives an inductive definition
of $\lambda_B$ on $P^{(\otimes n)}$, since the right side is
$k$-bilinear in $x$ and $y$. Moreover (a) follows also from this by
induction on $n$. 

(b)~~ We prove that $\lambda_B d_f = d_f \lambda_B$ by induction on $n$,
the case $n = 0$ being clear (from (a)). For $x \in P$, $y
\in h T(P)$, 
\begin{align*}
\lambda_B d_f (xy) & = \lambda_B ((fx) y - x(d_f y))\\
& =(fx) (\lambda_B y) - (x( \lambda_B (d_f y) + d_{B(x, \; )} (\lambda_B
(d_f y)))\\ 
d_f \lambda_B (xy) & = d_f (x (\lambda_B y ) + d_{B(x, \; )} (\lambda_B y ))\\
& = (fx) (\lambda_B y) - x(d_f(\lambda_B y)) + d_f d_{B(x, \; )}(\lambda_B y)
\end{align*}
Their equality follows from $d_f \lambda_B y=\lambda_B d_f y$
(induction) and the fact (Lemma \ref{chap5:lem3.5}) that $d_f
d_{B(x, \;  )}=-d_{B(x, \; )}d_f$. 

(c)~~ If $B = 0$ then $d_{B(x, \; )} = 0$ for all $x$ so (ii) reads
$\lambda_0 L_x = L_x \lambda_0$, and $1_{T(P)}$ solves this equation
for $\lambda_0$. We prove $\lambda_B 0 \lambda_B' = \lambda_{B+B'}$ by
checking (i) (which is clear) and (ii): 
\begin{align*}
\lambda_B \circ \lambda_{B'} (xy) & = x(\lambda_B \circ \lambda_{B'}
y) +d_{B+B' (x, \; )}(\lambda_B  \circ \lambda_{B'} y).\\ 
\lambda_B \lambda_{B'} (xy) & = \lambda_B (x (\lambda_{B'} y) +
d_{B'(x, \; )}(\lambda_{B'} y))\\ 
 & = x \lambda_B \lambda_{B'} y + d_{B(x, \; )}(\lambda_B \lambda_{B'} y) +
d_{B'(x, \; )} (\lambda_B \lambda_{B'} y) \\
& = x(\lambda_B \lambda_{B'} y) + (d_{B(x, \; )}+d_{B' (x, \; )}) (\lambda_B
  \lambda_{B'} y). \\
\text{and } & d_{B(x, \; )} + d_{B' (x, \; )} = d_{(B +B') (x, \; )}.  
\end{align*}

(d)~~ Let\pageoriginale $I = \{ u \in T(P) | \lambda_B (u) \epsilon I
(q- q_B) \}$. $\lambda_B (xu) = x (\lambda_B u) + d_{B(x, \; )}
(\lambda_B u)$, so, thanks to Lemma \ref{chap5:lem3.5}, I is a left
ideal. $\lambda_B ((X^2 - (qx))y) = x\lambda_B (xy) + d_{B(x, \; )}
\lambda_B (x y)$  
\begin{align*}
& - (qx) (\lambda_B y ) = x(x \lambda_B y + d_{B(x, \; )} (\lambda_B x))
  + d_{B(x, \; )} (x \lambda_B y + d_{B(x, \; )} \lambda_B y)\\
& - (qx) (\lambda_B y) = x^2 (\lambda_B y) + x d_{B(x, \; )} (\lambda_B
  y) + B(x, x) (\lambda_B y) - xd_{B(x, \; )} (\lambda_B y)\\ 
& - (qx) (\lambda_B y) ~~~ (\text{we have used } d^2_{B(x, \; )} =0;
  \text{ Lemma \ref{chap5:lem3.5}}) =\\
& = (x^2 - (qx - q_B x)) \lambda_B y \in I (q - q_B). 
\end{align*}

Thus $I$ is a left ideal containing all $(x^2 - qx)y$, so it contains
$I(q)$. We have proved  
$$
\lambda_B I(q) \subset I(q - q_B) = \lambda_B \lambda_{-B} I(q - q_B)  
\subset \lambda_B (I(q - q_B - q_{-B}) = \lambda_B I(q), 
$$
using (c). Now (a) implies $\lambda_B$ induces an isomorphism $Cl(P,
q) \to Cl(P, q- q_B)$ of filtered modules. 
\end{proof}

\setcounter{coro}{6}
\begin{coro}\label{chap5:coro3.7}%Corlry 3.7
Giving $Cl (P,q)$ the filtration induced by the ascending filtration on
$T(P)$, the structure of $Cl (P, q)$ as a filtered module is
independent of $q$. In particular, taking $q = 0$, we have an
isomorphism 
$$
Cl(P, q) \approx \Lambda (P)
$$
of filtered modules.
\end{coro}

\begin{proof}%Proof
Writing $q = q_B$ for some $B \in $ Bil $(P \times P)$ we obtain an
isomorphism $Cl(P, q) \to Cl (P, 0) = \Lambda (P)$, induced by
$\lambda_B$. 
\end{proof}

\begin{coro}%Corlry 3.8
$C_P: P \to Cl (P, q)$ is a monomorphism. If $U$ is a direct summand
  of $P$ then the map  
$$
Cl (U, q/ U) \to Cl (P, q), 
$$
induced by the inclusion $U \subset P$, is a monomorphism.
\end{coro}

\begin{proof}%proof
The\pageoriginale first assertion follows from the commutativity of  
\[
\xymatrix{
Cl (P,q) \ar[rr]^{\lambda_B} & & \wedge (P) \\
& P\ar[ur] \ar[ul]^{C_P)} &
}
\]
and the fact that $P \to \wedge (P)$ is a monomorphism. Let $B' = B/ U
\times U$, so $q / U = q_B'$. Then it is easily checked that  
\[
\xymatrix{
Cl (P, q) \ar[r]^{\lambda_B} & \wedge (P) \\
Cl (U, q') \ar[u] \ar[r]_{\lambda_{B'}} & \wedge (U) \ar[u]
}
\]
is commutative, so the second assertion follows since $\wedge (U) \to 
\wedge (P)$ is injective. 
\end{proof}

$\wedge (P) = T(P) / I(), I(0)$ being the (homogeneous) ideal
generated by all $x \otimes x$, $x \epsilon P$. 
$$
\wedge (P) = k \oplus \wedge^1 P \oplus \wedge^2 P \oplus \ldots
$$
and $\wedge^1 P \approx P$. Lemma \ref{chap5:lem3.1} givens a natural
isomorphism  
$$
\wedge (P \oplus Q) \approx \wedge (P) \oplus \wedge (Q) 
$$
of $(\mathbb{Z}/2 \mathbb{Z})$- graded algebras. $\wedge$ therefore
defines a product preserving functor  
$$
\wedge : \underset{=}{P} \to \underline{\underline{FP}}_2,
$$
if, for $P$ finitely generated and projective, we view $\wedge (P)$ as
a faithfully projective module, graded modulo 2. Similarly, by virtue
of Lemma \ref{chap5:lem3.1}, the Clifford algebra defines a product preserving
functor, 
$$
Cl : \underline{\underline{Quad}} \to (\text{graded algebras, } \otimes )
$$

\setcounter{theorem}{8}
\begin{theorem}\label{chap5:thm3.9}%Thm 3.9
If\pageoriginale $(P, q) \in \obj \underline{\underline{Quad}}$, then
$Cl(P, q) \in \obj \underline{\underline{Az}}_2$, i.e. it is a graded
azumaya algebra. The resulting functor $Cl : \underline{\underline{Quad}} \to
\underline{\underline{Az}}_2$ renders the diagram  
\[
\xymatrix{
\underset{=}{P} \ar[r]^{\mathbb{H}} \ar[d]_{\wedge} &
\underline{\underline{Quad}} \ar[d]^{Cl}\\
\underline{\underline{FP}}_2 \ar[r]_{END} & \underline{\underline{Az}}_2
}
\]
commutative up to natural isomorphism, i.e. for $P$ finitely generated
and projective, 
$$ 
Cl (\mathbb{H} (P)) \approx \text{ END } (\wedge (P))
$$
as graded algebras. 
\end{theorem}

\setcounter{coro}{9}
\begin{coro}%Corlry 3.10
There is a natural map of exact sequences.
{\fontsize{9pt}{11pt}\selectfont
\[
\xymatrix@C=.5cm{
K_1 P \ar[r] \ar[d] & K_1 \underline{\underline{Quad}} \ar[r] \ar[d] &
K_0 \Phi \mathbb{H} \ar[r] \ar[d] & K_0 \underset{=}{P} \ar[r] \ar[d]
& K_0 \underline{\underline{Quad}} \ar[r] \ar[d] & \text{ Witt } (k)
\ar[r] \ar[d] & 0\\
K_1 \underline{\underline{FP}}_2 \ar[r]& K_1 \underline{\underline{Az}}_2
\ar[r] & K_0 \Phi END \ar[r] & K_0 \underline{\underline{FP}}_2 \ar[r]&
K_0\underline{\underline{Az}}_2 \ar[r] & Br_2 (k) \ar[r] & 0
}
\]
}\relax

In Theorem 4.6 of Chapter \ref{chap4} we exhibited an exact sequence 
$$
0 \to Br(k) \to Br_2 (k) \to Q_2 (k) \to 0,
$$
where $Q_2 (k)$ was ``the group of graded quadratic extensions of 
$k$.'' The map above assigns to the class of $(P, q)$ in Witt $(k)$ and
element $\beta$ of $Br_2 (k)$. The projection of $\beta$ in $Q_2(k)$
corresponds, in the classical case when $k$ is a field, to the
discriminant, if char $k \neq 2$, and the Arf invariant if char $k
=2$. The remaining contribution from $Br(k)$ is essentially the Hasse
invariant. 
\end{coro}

\setcounter{proofof}{8}
\begin{proofof}%prf of 3.9
We\pageoriginale want to construct a natural isomorphism
$$
\varphi_P : Cl (\mathbb{H} (P)) \to \text{ END } (\wedge (P)). 
$$
for $P \in  \obj \underline{P}$. Suppose this is done. Then if
$(P, q) \in \obj \underline{\underline{Quad}}$, we have $(P, q) \perp (P, -q)
\approx \mathbb{H} (P)$ (Lemma \ref{chap5:lem2.2}), so $Cl(P, q)
\otimes Cl (P, -q) 
\approx Cl(P, q) \perp (P, -q))$ (Lemma \ref{chap5:lem3.1}) $\approx
Cl (\mathbb{H} 
(P)) = END (\wedge(P))$, by assumption. Therefore, by criterion (6)
of Theorem \ref{chap4:thm4.1}, Chapter \ref{chap4}, $Cl(P, q)$ is a
graded azumaya algebra. Thus we only have to construct $\varphi_P$.  

$\mathbb{H}(P) = (P \oplus P^\ast, q^P)$ with $q^P (x, y) = \langle
y, x \rangle_P$
 
\noindent
$= y(x)$ for $(x, y) \in P \otimes P^\ast$. Define 
$$
P \otimes P^\ast \to END (\wedge (P))
$$
by $(x, y) \mapsto L_x + d_y$. Then, using Lemma \ref{chap5:lem3.5},
$(L_x + d_y)^2 = L_{x^2} + L_x d_y + d_y L_x + d^2_y = L_x d_y + d_y L_x$,
because $x^2 = 0$ in $\wedge (P)$ and $d^2_y = 0$. If $u \in \wedge
(P)$ then $(L_x d_y + d_y L_x) u = xd_y (u) + d_y (xu) = xd_y (u) + y
(x) u - x d_y(u) = y(x) u$. Thus $(L_x + d_y)^2$ is multiplication by
$y (x) = q^P (x, y)$ on $\wedge (P)$, i.e. $(L_x + d_y)^2 = q^P (x,
y)$ in END $(\wedge (P))$. Thus we have defined an algebra
homomorphism   
$$
\varphi_P : Cl (\mathbb{H} (P)) \to END (\wedge (P)),
$$
and since $L_x + d_y$ has degree 1, it is a homomorphism of graded
algebras. 

Suppose $f: P_1 \to P_2$ is an isomorphism. Then on $\wedge (P_2)$,
$L_{f(x)} = \wedge (f) L_x \wedge(f)^{-1}$ and $d_{f^\ast}-1_y (x_2) =
(f^{*-1}y) (x_2) = y (f^{-1} x_2)$, so $\wedge (f) d_y \wedge (f)^{-1}
x_2) = \wedge (f) d_y (f^{-1} x_2) = \wedge (f) y (f^{-1} x_2) = d_{f*
  -1_y} (x_2)$ for $x_2 \in P_2$, because $y(f^{-1} x_2)$ 
has degree zero in $\wedge (P_2)$. Therefore\pageoriginale $\wedge (f)
(L_x + d_y) \wedge (f)^{-1} = L_{f(x)} + d_{f*-1_{(y)}}$ so it follows
that $\varphi_P$ is natural, recalling that $\mathbb{H} (f) = f \oplus
f^{*-1}$.  

Next we will show that, for $P = P_1 \otimes P_2$ the following
diagram is commutative:  
\[
\xymatrix{
Cl (\mathbb{H} (P_1 \oplus P_2)) \ar[r]^{\varphi_{P_1 \oplus P_2}} 
\ar[d]_{\approx}  & END (\wedge (P_1 \oplus P_2)) \ar[d]^{\approx}\\
Cl (\mathbb{H} (P_1) \perp \mathbb{H} (P_2)) \ar[d]_{\approx}& END
(\wedge P_1 \otimes \wedge P_2)\\
Cl (\mathbb{H} (P_1)) \otimes Cl (\mathbb{H} (P_2))
\ar[r]_{\varphi_{P_1} \otimes \varphi_{P_2}} & END (\wedge P_1)
\otimes END (\wedge P_2) \ar[u]_{\approx}
}
\]

To see this, we trace the images of $((x_1, x_2)$, $(y_1, y_2))$  
$$
\in (P_1 \otimes P_2) \oplus (P^\ast_1 \otimes P^\ast_2 ) \subset Cl (
\mathbb{H} (P_1 \oplus P_2)): 
$$
\[
\xymatrix{
((x_1, x_2), (y_1, y_2)) \ar@{|->}[r] \ar@{|->}[d] & L_{(x_1, x_2)} +
  d_{(y_1, y_2)} \ar@{|->}[d]\\
((x_1, y_1), (x_2, y_2)) \ar@{|->}[d] &  
\text{$\begin{matrix}
L_{x_1 \otimes 1} + L_{1 \otimes x_2}\\ 
+ d_{y_1} \otimes 1_{\wedge P_2} + 1_{\wedge P_1}
    \otimes d_{y_2}
  \end{matrix}$}\\
((x_1, y_1) \otimes 1) + (1 \otimes (x_2, y_2)) \ar@{|->}[r] &
\text{$\begin{matrix}
((L_{X_1} + d_{y_1}) \otimes 1_{\wedge P_2})\\
(1_{\wedge P_1} \otimes (L_{x_2} + d_{y_2}))
  \end{matrix}$}\ar@{|->}[u]
}
\]
Since\pageoriginale all of these algebras are faithfully projective
$k$-modules we conclude that $\varphi_{P_1 \oplus P_2} $ is an
isomorphism $\Leftrightarrow \varphi_{P_1} \oplus \varphi_{P_{2}}$ is an
isomorphism $\Leftrightarrow \varphi_{P_1}$ and $\varphi_{P_2}$ are
isomorphisms. (In Chapter \ref{chap2} we showed that the functor $Q
\otimes$ is faithfully exact for $Q$ faithfully projective.)  

Now given $P_1$ we choose $P_2$ so that $P_1 \oplus P_2 \approx k
\oplus \cdots \oplus k$, and then the problem of showing that
$\varphi_{P_1}$ is an isomorphism reduces to the special case $P_1 =
k$. We do this case now by a direct calculation.  

$\mathbb{H} (k) \approx (ke_1 \oplus ke_2, q)$ with $q(a_1 e_1 + a_2
e_2) = a_1 a_2$. Here $ke_2 = (ke_1)^\ast$ and $e_2$ is the dual basis to
$e_1$, i.e. $e_2(e_1) = 1$. Therefore $\wedge (ke_1) = k[e_1] = k1
\oplus ke_1$ with $e^2_1 = 0$, and $d_{e_2}(1) = 0$, $d_{e_2}(e_1)
=1$. Moreover, $L_{e_1}(1) = e_1$ and $L_{e_1} (e_1) = 0$. 

$Cl (\mathbb{H} (ke_1)) = k[e_1, e_2]$ with $e^2_1 = 0 = e^2_2$, and
$e_1 e_2 + e_2 e_1 = 1$, because $1 = q(e_1 + e_2) = (e_1 + e_2)^2$. 
$$
\varphi_k : Cl (\mathbb{H} (ke_1)) \to END (\wedge (ke_1)) 
$$
is defined by $\varphi_k(e_1) = L_{e_1}$ and $\varphi_k (e_2)=
d_{e_2}$. With respect to the basis 1, $e_1 $ for $\wedge (ke_1)$
these endomorphisms are represented by the matrices
$\left(\begin{smallmatrix} 0 & 0 \\ 1 & 0 \end{smallmatrix}\right)$
and $\left(\begin{smallmatrix} 0 & 1 \\ 0 &
  0 \end{smallmatrix}\right)$, respectively, and these clearly
generate $\mathbb{M}_2 (k)$. Thus $\varphi_k$ is surjective. On the
other hand Corollary \ref{chap5:coro3.7} says that, as a module, $Cl
(\mathbb{H}(k)) 
\approx \wedge (k \oplus k^\ast)$, a free module of rank four (because
$\wedge (k \oplus k^\ast ) \approx \wedge (k) \otimes (k^\ast)$). A
surjective homomorphism of free modules of the same finite rank must
be an isomorphism, so $\varphi_k$ is an isomorphism as claimed. 
\end{proofof}

\section{The orthogonal group and spinor norm}\label{chap5:sec4} %Section 4

We assume\pageoriginale here that $\spec \; (k)$ is connected. Suppose
$(P, q)$ is a quadratic space (i.e. $\in \obj \
\underline{\underline{\rm Quad}} \ (k)$) and that $[P:k] = n$. If $n$ is
odd then $2 \in U (k)$; otherwise reduce $(P, q)$ modulo a maximal
ideal containing $2k$, and we contradict the fact that non-singular
forms over fields of char 2 have even dimension. We propose to use the
Clifford algebra,  
$$
A = Cl (P, q) = A_\circ  \oplus A_1
$$
to study the \textit{orthogonal group}
$$
\Omega = \Omega (P, q),
$$
i.e. the group of isometries of $(P, q)$.

We take the position from Chapter \ref{chap4} that everything is
graded. Thus   
\begin{equation*}
\Pic (k) = \Pic |k| \oplus \mathbb{Z} / 2 \mathbb{Z}
\tag{4.1}\label{eq4.1} 
\end{equation*} 
is the group of invertible $k$-modules. The first summand describes
the underlying ungraded module ($|k|$-module) and the $\mathbb{Z} / 2
\mathbb{Z}$ summand designates the degree (0 or 1) in which it is
concentrated. 

Write
$$
G(A) = Aut_{k-alg} (A), 
$$
and $hU(A)$ for the homogeneous units of $A$. Recall that if $u \in h
U(A)$ then $\alpha_u \in G(A)$ is defined by $\alpha_u (a) =  
\begin{cases} 
ua u^{-1} \text{ if } \partial u = 0\\ 
ua' \nu^{-1} \text{ if } \partial u = 1
\end{cases}$. Here, for $a = a_0 + a_1$, $a' = a_0 - a_1$. Thus, for
homogeneous a, we can write this as  
$$
\alpha_u (a) = (-1)^{\partial u \partial a} u a u^{-1} (u \in h U (A),
a \in hA). 
$$

According\pageoriginale to Theorem \ref{chap5:thm3.9} $A$ is azumaya
$k$-algebra. Therefore Theorem \ref{chap4:thm5.2} of chapter
\ref{chap4} gives us an exact sequence   
\begin{gather*}
1 \to U (k) \to U (A_\circ) \to G(A) \to Pic (k)\tag{4.2}\label{eq4.2}\\ 
u \rightarrow \alpha_u 
\end{gather*}

To apply this we first embed $\Omega$ in $G(A)$. Indeed, since the
Clifford algebra is a functor of $(P, q)$ there is a canonical
homomorphism, $\alpha \mapsto C(\alpha)$, of $\Omega$ into $G(A)$. If
we identify $P \subset A$ (in fact $P \subset A_1)$ then $C(\alpha)$
\textit{is the unique algebra automorphism of $A$ such that}
$C(\alpha)(x) = \alpha(x)$ for $x \in P$. For example, the
automorphism $a \mapsto a'$ described above is just $C(-1_P)$. We
will use this monomorphism to identify $\Omega$ with a subgroup of
$G(A)$. We can characterise it: 
$$
\Omega = \{\alpha \in G(A) \big| \alpha P \subset P\}. 
$$
For if $\alpha P \subset P$ then for $x \in P$ we have
$q(\alpha x) = (\alpha x)^2 = \alpha (x^2) = \alpha (qx) = qx$, so
$\alpha $ induces an isometry, $\alpha': P \to P$. Evidently then
$\alpha = C(\alpha')$.  

Next we introduce the \textit{Clifford group} 
$$
\Gamma = \{u \in h U (A) | \alpha_u \in \Omega \}.  
$$
and the \textit{special Clifford group}
$$
\Gamma_\circ = \Gamma \cap A_\circ = \{u \in U(A_\circ)\big| \alpha_u
\in \Omega\}.  
$$
If $u \in U (k) \subset U(A_\circ)$ then $\alpha_u = 1$, so $U(k)
\subset \Gamma_\circ$. Therefore the exact sequence (\ref{eq4.2})
induces a sub-exact sequence,   
\[
\xymatrix@R=0.5cm{
1 \ar[r] & U(k) \ar[r] \ar@{=}[d] & U(A_\circ) \ar[r]
\ar@{}[d]|{\bigcup} & G(A) \ar[r] \ar@{}[d]|{\bigcup} & \Pic (k)
\ar@{=}[d]\\ 
1 \ar[r] & U(k) \ar[r] & \Gamma_\circ \ar[r] & \Omega \ar[r] & \Pic
(k)  
}\tag{4.3}\label{eq4.3}
\]
 Now\pageoriginale $\Pic (k) = \Pic |k| \otimes \mathbb{Z}/2
 \mathbb{Z}$ (see (\ref{eq4.1})), so we obtain homomorphisms  
 $$
 \Omega \rightarrow \Pic (k) 
 $$
 and 
 $$
 \Omega \rightarrow \mathbb{Z}/2 \mathbb{Z}, 
 $$
 the second being the first followed by projection on the second
 factor. We shall write  
 \begin{align*}
S \Omega & = \ker (\Omega \to \mathbb{Z}/2 \mathbb{Z})\\
\bigcup & \\
VS \Omega & = \ker (\Omega \to \Pic (k)),
 \end{align*} 
 the \textit{special}, and \textit{very special orthogonal groups}
 (of $(P,q))$, respectively. With this notation we can extract from
 (\ref{eq4.3}) an exact sequence 
 \begin{equation*}
1 \to U(k) \to \Gamma_\circ \to vS \Omega \to 1. \tag{4.4}\label{eq4.4}
 \end{equation*} 
 If $x \in P$ then $x^2 = qx $ in $A$, therefore also in
 $A^\circ$, so the identity map on $P$ extends to an isomorphism $A \to
 A^\circ$, or, in other words, an antiautomorphism of $A$. We shall denote
 it by $a \mapsto \tilde{a}$. All $\alpha \in \Omega$ commute
 with this antiautomorphism; just check it on $P$. For a$\in A$ we
 will define its \textit{conjugate}, $\bar{a}$, by  
 $$
 \bar{a} = \tilde{a}' = \tilde{a'}
 $$ 
 and write 
 $$
 Na= a\bar{a}.
 $$
 The last remark shows that $\alpha (\bar{a}) = \overline{\alpha(a)}$
 for $\alpha \varepsilon\Omega$. 

 Let\pageoriginale 
 $$
 \mathfrak{n} = \{a \in A \big| Na \in k \}.
 $$
 If $x \in P$ then $\bar{x}=\tilde{x'}=x'=-x $ so $Nx = - x^2 = -q(x)
 \in k$, and $P\subset \mathfrak{n}$. 
 
 Suppose $a$, $b \in \mathfrak{n}$. Then $N(ab) = (ab)\bar{ab}=
 ab\bar{b}\bar{a} = aN(b)\bar{a} = a \bar{a}N(b) = N(a)N(b)$, because
 $N(b) \in U(k)$. 
$$
 a,b \in \mathfrak{n} \Rightarrow ab \in \mathfrak{n}
 \text{ and } N(ab) =  N(a) N(b). 
$$
 
 Next suppose $ u \in \Gamma$. Then for $x \in P$ we have $\alpha_u(x)
 = (-1)^{\partial u} uxu^{-1}$, or $u'x = \alpha_u (x)u$. Therefore
 $\bar{u}\overline{\alpha_u(x)} = \bar{x} \bar{u'}$, and
 $\overline{\alpha_u(x)} = \alpha_u (\bar{x})$ with $\bar{x}
 =-x$. Setting $y = \alpha_u(\bar{x})$ we have $\bar{u}y =
 \alpha^{-1}_u(y) \bar{u}'$, so, by definition, $\bar{u}\in \Gamma$
 and $\alpha_{\bar{u}} = \alpha^{-1}_u$. In particular
 $\alpha_{u\bar{u}} = \alpha_u\alpha_{\bar{u}} = 1$ so $N(u) =
 u\bar{u}\in \ker (\Gamma \to G(A)) = U(k)$. Summarizing, we
 have proved: 
 
If $u \in \Gamma$ then $\bar{u}\in \Gamma$ and $\alpha_{\bar{u}} =
\alpha^{-1}_u$. 
 
Moreover $N(u) \in U(k)$ (i.e. $\Gamma\subset \mathfrak{n}$) so 
 $$
 u^{-1} = N(u)^{-1}\bar{u}. 
 $$
 Thus $N$ defines a homomorphism 
 $$
 N : \Gamma \to U(k). 
 $$
 We now introduce the groups 
 $$
{\rm Pin} = \ker(\Gamma \xrightarrow{N}U(k))
 $$
 and\pageoriginale 
 $$
 {\rm Spin} = \ker (\Gamma_\circ \xrightarrow {N}U(k)). 
 $$
 If $u \in U(k)$ then $\bar{u} = u$ so $N(u) = u^2$. Therefore if we
 apply $N$ to the exact sequence (\ref{eq4.4}) we obtain a commutative
 diagram with exact rows and columns: 
\[
\xymatrix{
& 1 \ar[d] & 1 \ar[d] & 1 \ar[d] & \\
1 \ar[r] & {}_2U(k) \ar[r] \ar[d] & {\rm Spin} \ar[d] \ar[r] & V S \Omega'
\ar[r] \ar[d] & 1 \\
1 \ar[r] & U(k) \ar[r] \ar[d] & \Gamma_0 \ar[r] \ar[d]^N & VS \Omega
\ar[r] \ar[d]^{\sigma} & 1\\
1 \ar[r] & U(k)^2 \ar[r] \ar[d] & U(k) \ar[r] & U(k)/ U(k)^2 \ar[r] &
1\\
& 1 & &  
}\tag{4.5}\label{eq4.5}
\]

Here  \, ${}_{2}U(k)$ denotes units of order 2 (square roots of
1). $\sigma:VS \Omega \to U(k)/ U(k)^2$ is called the \textit{spinor
  norm}, and its kernel, $VS \Omega'$, the \textit{spinorial kernel}. 
 
So far we have the following subgroups, with indicated successive
quotients, of $\Omega$: 
\begin{align*}
 & \overset{\Omega}{\underset{V S \Omega}{\Bigg| \; \Bigg \}}} \subset
  \Pic (k) = \Pic |k| \oplus \mathbb{Z} / 2 \mathbb{Z}\\ 
 & {\underset{V S \Omega'}{\Bigg| \; \Bigg \}}} \xrightarrow{\sigma} U(k)
  / U(k)^2 \tag{4.6}\label{eq4.6}\\ 
 &{\underset{1}{\Bigg| \; \Bigg \}}} \approx \Gamma_\circ / U(k) \subset
  U(A_\circ) / U(k). 
\end{align*}\pageoriginale

Of course the bulk of the group is the bottom layer. We shall now
investigate this for small values of $n = [p: k]$.  

\textit{$n = 1$} (so $2 \in U (k)$). $A_\circ = k$, $A_1 = P$, and
$\Gamma_\circ = U(k)$. $\Omega = \{\pm 1\}$ in this case. 
 
 \textit{$n = 2$} $A_1= P$ so $\Gamma_\circ = U(A_\circ)$. $A_\circ$ is a
 quadratic extension of $k$ (in the sense of chapter \ref{chap4}, \S
 \ref{chap4:sec3},) so 
 $\Gamma_\circ$ is abelian group. If $(P,q) = \mathbb{H}(k)$ then $A_\circ =
 k \times k$ so $\Gamma_\circ = U (k) \times U(k)$ and $V S
 \Omega\approx U (k)$.  
 
\textit{$n = 3$} (so $2 \in U(k)$). Then $A_1 = P \oplus L_1$, where
$L_1$ is the degree one term of $L = |A|^{|A|}$ the centre of the
ungraded algebra $A$. $A_\circ$ is a ``quaternion $|k|$-algebra,"
i.e. azumaya $| k |$-algebra of rank 4, and $N(a)\in k$ for all $a \in
A_\circ$. If $u \in U (A_\circ)$ then. Since $N(u) = u\bar{u}
\in U (k)$,\pageoriginale we have $u^{-1} = N
(u)^{-1}\bar{u}$. Therefore, for $a \in A$,    
\begin{align*}
\overline{\alpha_u (a)} = \overline{u a u^{-1}} &= \bar{u}^{-1}
\bar{a} \; \bar{u} = (N(u) u^{-1})^{-1} \bar{a} (N(u)u^{-1})\\ 
&= u \bar{a} u^{-1} = \alpha_u (\bar{a}). 
\end{align*}
Consequently $\alpha_u$ leaves invariant the eigenspaces of ${}^-$; these
behave nicely because $\overset{=}{a} = a$ and $2 \in U(k)$. 

Now $\bar{x} = - x$ for $x \in P$. If we localize $k$ then $P$ has an
orthogonal basis, $e_1, e_2, e_3$, and it is easy to see that $L_1 =
ke_1 e_2 e_3$, $\overline{e_1 e_2 e_3} = (-1)^3 e_3 e_2 e_1 =
(-1)^{3+2+1} e_1 e_2 e_3$. Therefore, under the action of $^-$, $A_1 =
P \oplus L_1$ is the eigenspace decomposition. In summary we have
observed that $u \in U(A_\circ) \Rightarrow Nu \in U(k) \Rightarrow
\alpha_u$ leaves the eigenspaces of $^-$ invariant $\Rightarrow
\alpha_u P \subset p$, i.e. $u \in \Gamma_\circ$. Therefore  
$$
\Gamma_\circ = U (A_\circ) \text{ and } V S \Omega = U (A_\circ) / U (k).
$$
In case $A_\circ = \mathbb{M}_2 (k)$ we have $\Gamma_\circ = GL_2
(k)$, the norm $N$ is just the determinant, and   
$$
VS \Omega = PGL_2 (k) = GL_2 (k) / U (k). 
$$

\textit{$n = 4$}. In this case $L = A_\circ^{A_\circ}$ is a quadratic
extension of $k$ (in the sense of Chapter \ref{chap4}, \S
\ref{chap4:sec3}), and $A_\circ$ is a quaternion 
$L$-algebra. The norm $N$ takes values in $L$. In case 2 $\in U(k)$ a
calculation like that for the case $n = 3$ (localize $k$ and
diagonalize $(P, q)$ first) shows that 
$$
\Gamma_\circ = \{u \in U (A_\circ) \big| Nu \in U(k)\}. 
$$
and this is probably in general. In case $A_\circ \mathbb{M}_2 (L)$ then
$N: U(A_\circ) = GL_2 (L) \to U(L)$ is just the determinant. Hence $SL_2
(L) \subset \Gamma_0$ and $\Gamma_\circ/ SL_2 (L)\approx U(k)$, in this
case. $VS \Omega = \Gamma_\circ/_{U(k)} \supset SL_2 (L)/_2 U(k)$,
i.e. modulo element of order 2 in\pageoriginale the centre, and modulo
this subgroup $ V S \Omega$ lands in $U(k)/U(k)^2$. Note that if $L =
k \times k$ then $ S L_2(L) = SL_2(k)\times SL_2 (k)$.  

Suppose $k$ happens to be a Dedekind ring of arithmetic type in a
global field. Then one knows that $\Pic (k)$ is finite (finiteness of
class number) and $U(k)$ is finitely generated (Dirichlet unit
theorem). Hence $VS\Omega = \Gamma_0 / U(k)$ is of finite index in
$\Omega$. The discussion above shows, therefore, that the finite
generation of $\Omega$ is equivalent to the finite generation of
$\Gamma_0$, and that for $n \leq 4$ this is ``usually'' equivalent to
the finite generation of $U(A_0)$. The point is that $U(A_0)$ is often
an easily recognized linear group. 

One can similarly use this procedure to reduce the study of normal
subgroups of $\Omega$ to those of $U(A_0)$, at least in many cases, for
$n\leq 4$. 

\begin{thebibliography}{99}
\bibitem{key1} {Auslander M. and Goldman O.}\pageoriginale -[1] The Brauer
  group of a  
  commutative ring, Trans. A.M.S., 97 (1960), pp. 367-409.

\bibitem{key2} {Azumaya G.} -[1] On maximally central algebras, Nagoys
  Math. J., Vol. 2 (1951), pp. 119-150. 

\bibitem{key3} {Bass H.} -[K] K-theory and stable algebra I.H.E.S, $n^0$
  22 (1964), pp. 489-544. 

-[2] The Morita theorems, mimeographed notes.

\bibitem{key4} {Bourbaki N.} -[1] Alg\'ebre, Chapitre VIII, Hermann,
  Paris,   1958. 

-[2] Alg\'ebre, Chapitre IX, Hermann, Paris, 1959. 

\bibitem{key5} {Freya P.} -[1] Abelian categories, Harper and Row, New
  York, 1964. 

\bibitem{key6} {Gabriel P.} -[1] Des cat\'egories ab\'elinnes,
  Bull. Soc. Math. France, 90(1962), pp. 323-448. 

\bibitem{key7} {Heller A}. -[1] Some exact sequences in algebraic
  K-theory, Topology, Vol.3 (1965), pp.389-408. 

\bibitem{key8} {Maclane S.} -[1] Natural associativity and commutativity,
  Rice University Studies, 49(1963), pp.28-46. 

\bibitem{key9} {Mitchell B.} -[1] The full imbedding theorem,
  Amer. J. Math., Vol. 86 (1964), pp. 619-637. 

\bibitem{key10} {Morita K.} -[1] Duality for modules and its applications to
  the theory of rings with minimum condition. Science reports of the
  Tokyo Kyoiku Daigaku, Vol. 6 Sec A (1958), pp. 83-142. 

\bibitem{key11} {O'Meara O.T.} -[1] Introduction to quadratic forms,
  Springer-Verlag,1963. 

\bibitem{key12} {Rosenberg A. and Zelinsky D.} -[1] Automorphisms of
  separable algebras, Pacific J. Math., Vol. 11 (1961), pp. 1109-1117. 

\bibitem{key13} {Wall C.T.C} -[1] Graded Brauer groups, Crelle's J.,
  Vol. 213 (1964), pp. 187-199. 

\bibitem{key14} {Witt E.} -[1] Theorie der quadratischen Formen in
  beliebigen K\"orpern, Crelle's J., Vol.176 (1937), pp. 31-44. 
 \end{thebibliography} 

