
\chapter*{Introduction}
\addcontentsline{toc}{chapter}{Introduction}

In order\pageoriginale to construct a general theory of (non-singular)
quadratic forms and orthogonal groups over a  commutative ring $k$,
one should first investigate the possible generalizations of the basic
classical tools (when $k$ is a field). These are    
\begin{enumerate}
\renewcommand{\theenumi}{\Roman{enumi}}
\renewcommand{\labelenumi}{(\theenumi)}
\item Diagonalization (if char $k \neq 2$), and Witt's theorem. 

\item Construction of the classical invariants: dimension,
  discriminant, Hasse invariant. 
\end{enumerate}

This course is mostly concerned with the algebraic apparatus which is
\textit{preliminary} to a generalization of II, particularly of the
Hasse invariant. Consequently, quadratic forms will receive rather
little attention, and then only at the end. It will be useful,
therefore, to briefly outline now the material to be covered and to
indicate its ultimate relevance to quadratic forms. 

We define a \textit{quadratic module} over $k$ to be a pair $(P, q)$
with $P \in \underset{=}P$, the category of finitely generated
projective $k$-modules, and with $q: P \to k$ a map satisfying
$q(ax) = a^2 q (x) \;  (a \epsilon k, a \epsilon P)$ and such that $(x, y)
\longmapsto q(x + y) - q(x) -q(y)$ is a bilinear form. This
form then induces a homomorphism $P \to P^* = \Hom_k (P, k)$ (by
fixing a variable), and we call $(P, q)$ \textit{non-singular} if $P
\to P^*$ is an isomorphism. 

If $(P_1, q_1)$ and $(P_2, q_2)$ are quadratic modules, we have the 
``orthogonal sum'' $(P_1, q_1) \perp (P_2, q_2) = (P_1 \oplus P_2,
q)$, where $q (x_1, x_2) = q_1(x_1) + q_2(x_2)$. 

Given \pageoriginale $P \in \underset{=}P$, in order to find a $q$ so
that $(P, q)$ 
in non-singular we must at least have $P \approx P^*$. Hence for
arbitrary $P$, we can instead take $P \oplus P^*$, which has an
obvious isomorphism, ``$\left(\begin{smallmatrix} 0 & 1_{P^*}
  \\ 1_P & 0 \end{smallmatrix}\right)$'', with its dual. Indeed this
is induced by the bilinear form associated with the \textit{hyperbolic
  module} 
$$
\mathbb{H} (P) = (P \oplus P^*, q_{_P}), 
$$
where $q_p(x, f) = f(x) (x \epsilon P, f \epsilon P^*)$. The following
statement is easily proved: 

$(P, q)$ is non-singular $\Leftrightarrow (P, q) \perp (P, -q) \approx
\mathbb{H}(P)$. 

Let $\underset{=}Q$ denote the category of non-singular quadratic
modules and their isometrics. In $\underset{=}P$ we take only the
\textit{isomorphisms} as morphisms. Then we can view $\mathbb{H}$ as
the \textit{hyperbolic functor} 
$$
\mathbb{H} : \underset{=}P \to \underset{=}Q, 
$$
where, for $f: P \to P'$, $\mathbb{H}(f)$ is the isometry $ f \oplus
f^{*^{-1}}: \mathbb{H} (P) \to \mathbb{H} (P')$. Moreover, there is a
natural isomorphism 
$$
\mathbb{H} (P \oplus P') \approx \mathbb{H} (P) \perp \mathbb{H} 
(P'). 
$$

With this material at hand I will now begin to describe the course. In
chapter \ref{chap1} we establish an exact sequence of Grothendieck groups of
certain categories, in an axiomatic setting. Briefly, suppose we are
given a category $\mathscr{C}$ in which all morphisms are isomorphisms
\pageoriginale 
(i.e. a groupoid) together with a product $\perp$ which has the formal
properties of $\perp$  and $\oplus$ above.  We then make an abelian
group out of $\obj \mathscr{C}$ in which $\perp$ corresponds to $+$; it
is denoted by $K_0 \mathscr{C}$. A related group $K_1 \mathscr{C}$, is
constructed using the automorphisms of objects of $\mathscr{C}$. Its
axioms resemble those for a determinant. If $H: \mathscr{C} \to
\mathscr{C}'$ is a product preserving functor (i.e. $H (A \perp B) =
HA \perp HB)$, then it induces homomorphisms $K_i H : K_i \mathscr{C}
\to K_i \mathscr{C}'$, $i = 0$, 1. We introduce a relative category
$\Phi H$, and then prove the basic theorem: 

There is an exact sequence
$$
K_1 \mathscr{C} \to K_1 \mathscr{C}' \to K_0 \Phi H \to K_0 
\mathscr{C} \to K_0 \mathscr{C}', 
$$
provided $H$ is ``cofinal''. \textit{Cofinal} means: given $A'
\epsilon \mathscr{C}'$, there exists $B' \to \mathscr{C}'$ and $C
\epsilon \mathscr{C}$ such that $A' \perp B' \approx HC$. This theorem
is a special case of results of Heller [1]. 

The discussion above shows that the hyperbolic functor satisfies all
the necessary hypotheses, so we obtain an exact sequence 
$$
[\mathbb{H}] \quad K_1 \underset{=}P \to K_1 \underset{=}Q \to K_0
\Phi \mathbb{H} \to  K_0 \underset{=}P \to K_0 \underset{=}Q \to
\text{ Witt } (k) \to 0. 
$$
Here we define  Witt $(k)$ = coker $(k_0 \mathbb{H})$. It corresponds
exactly to the classical ``Witt ring'' of quadratic forms (see
Bourbaki \cite{key2}). The $K_i \underset{=}P$, $i = 0$, 1 will be described
in chapter \ref{chap1}. $K_1 \underset{=}Q$ is related to the stable structure
of the orthogonal groups over $k$. 

The classical\pageoriginale Hasse invariant attaches to a quadratic
form over a 
field $k$ an element of the Brauer group Br $(k)$. It was given an
intrinsic definition by Witt \cite{key1} by means of the Clifford
algebra. This necessitates a slight artifice due to the fact that the
Clifford algebra of a form of odd dimension is not central
simple. Moreover, this complication renders the definition unavailable
over a commutative. ring in general. C.T.C. Wall \cite{key1} proposed a
natural and elegant alternative. Instead of modifying the Clifford
algebras he enlarged the Brauer group to accommodate them, and he
calculated this ``Brauer-Wall'' group $BW(k)$ when $k$ is a
field. Wall's procedure generalizes naturally to any $k$. In order to
carry this out, we present in chapters \ref{chap2}, \ref{chap3}, and
\ref{chap4}, an exposition 
of the Brauer-Wall theory. 

Chapter \ref{chap2} contains a general theory of equivalences of categories of
modules, due essentially to Morita \cite{key1} (see also Bass
\cite{key2}) and Gabriel 
\cite{key1}. It is of general interest to algebraists, and it yields, in
particular, the Wedderburn structure theory in a precise and general
form. It is also a useful preliminary to chapter \ref{chap2}, where we deal
with the Brauer group Br $(k)$ of azumaya algebras, following the work
of Auslander-Goldman \cite{key1}. In chapter \ref{chap4} we study the category
$\underset{=}A z_2$ of graded azumaya algebras, and extend Wall's
calculation of $BW(k)$, giving only statements of results, without
proofs. 

Here \pageoriginale we find a remarkable parallelism with the
phenomenon witnessed 
above for quadratic forms. Let $\underset{=}{FP}_2$ denote the
category of ``faithfully projective'' $k$-modules $P$ (see chapter \ref{chap1}
for definition), which have a grading modulo $2: P = P_0 \oplus
P_1$. Then the full endomorphism algebra $END(P)$ (we reserve End for
morphisms of degree zero) has a natural grading modulo 2, given by
maps homogeneous of degree zero and one, respectively. 

Matricially, $\left(\begin{smallmatrix} a & b \\ c &
  d \end{smallmatrix}\right) = \left(\begin{smallmatrix} a & 0 \\ 0 &
  d \end{smallmatrix}\right) + \left(\begin{smallmatrix} 0 & b \\ c &
  0 \end{smallmatrix}\right)$. These are the ``trivial'' algebras in
$\underset{ =}{Az}_2$; that is $BW(k)$ is the \textit{group} of
isomorphism classes of algebras in $\underset{ =}{Az}_2$, with respect
to $\otimes$, modulo those of the form $END(P)$. It is a group because
of the isomorphism 
$$
A \otimes A^* \approx END (A), 
$$
where $A^*$ is the (suitably defined) opposite algebra of $A$, for $A
\in \underset{ =}{Az}_2$. Morever, $A$ is faithfully projective as a
$k$-module. Finally we note that 
$$
END : \underset{=}{FP}_2 \to \underset{=}{Az}_2 
$$
is a functor, if in both cases we take homogeneous isomorphisms as
morphisms. For, if $f: P \to P'$ and $e \in END(P)$, then $END(f) (e)=
fef^{-1} \in END(P')$. Moreover, there is a natural isomorphism 
$$
END(P \otimes P') \approx END(P) \otimes END(P'). 
$$
Consequently,\pageoriginale we again obtain an exact sequence:
{\fontsize{10pt}{12pt}\selectfont
$$
[END] : ~ K_1 \underset{ =}{FP}_2 \to K_1 \underset{ =}{Az}_2 \to K_0
\Phi END \to K_0 \underset{ =}{FP}_2 \to K_0 \underset{ =}{Az}_2 \to
BW(k) \to 0. 
$$}\relax

Chapter \ref{chap5} finally introduces the category $\underset{ =}{Q}$ of
quadratic forms. The Clifford algebra is studied, and the basic
structure theorem for the Clifford algebra is proved in the following
form: The diagram of (product-preserving) functors 
\[
\xymatrix@R=1.5cm@C=1.5cm{
\underset{=}{P} \ar[r]^{\mathbb{H}} \ar[d]_{\wedge} & \underset{=}{Q}
\ar[d]^{\text{Clifford}}\\ 
\underset{=}{FP}_2 \ar[r]_{\rm END} & \underset{=}{Az}_2
}
\]

commutes up to natural isomorphism. Here $\wedge$ denotes the exterior
algebra, graded modulo 2 by even and odd degrees. 

This result simultaneously proves that the Clifford algebras lie in
$\underset{ =}{Az}_2$, and shows that there is a natural homomorphism
of exact sequences 
{\fontsize{9pt}{11pt}\selectfont
\[
\xymatrix@R=0.9cm@C=0.43cm{
[\mathbb{H}] \ar[d] \ar@{}[r]|{:} & K_1 \underset{=}{P}\ar[d] \ar[r] &
K_1 \underset{=}{Q} \ar[r] \ar[d] & K_0 \Phi \mathbb{H} \ar[d] \ar[r]
& K_0 \underset{=}{P}\ar[r] \ar[d]_{\rm{\underline{Grassman}}} & K_0 \underset{=}{Q}
\ar[r] \ar[d]_{\rm{\underline{Clifford}}} & {\rm Witt}(k)\ar[r]
\ar[d]^{\rm{\underline{Hasse-Wall}}} & 0 \\ 
{\rm END} \ar@{}[r]|{:} & K_1 \underset{=}{FP}_2 \ar[r] & K_1
\underset{=}{Az}_2 \ar[r] & K_0 \Phi {\rm END} \ar[r] & K_0
\underset{=}{FP}_2 \ar[r] & K_0 \underset{=}{Az}_2 \ar[r] & BW(k)
\ar[r] & 0
}
\]}\relax

This commutative diagram is the promised generalization of the Hasse
invariant. 
