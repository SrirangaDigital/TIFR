\chapter{The exact sequence of algebraic $K$-theory}\label{chap1}%chap I

The\pageoriginale exact sequence of Grothendieck groups constructed in
Bass [$K$, Chapter \ref{chap3}] is obtained here in an axiomatic
setting. The same is 
done in a considerably, more general setting by A. Heller in Heller
\cite{key1}. A special case of the present version was first worked out by
S. Chase (unpublished). 

In the last sections we shall describe the Grothendieck groups of
certain categories of projective modules. 


\section{Categories with product, and their
  functors}\label{chap1:sec1}%sec 1 

If $\mathscr{C}$ is a category, we shall denote by $\obj \mathscr{C}$,
the class of all objects of $\mathscr{C}$, and by $\mathscr{C}(A, B)$,
the set of all morphisms $A \to B$, $A$, $B \epsilon \obj
\mathscr{C}$. We shall assume the isomorphism classes in our
categories to form sets.  

A \textit{groupoid} is a category in which all morphisms are isomorphisms.

\begin{defi*}
A category with product is a groupoid $\mathscr{C}$, together
with a ``product'' functor 
$$
\perp : \mathscr{C} \times \mathscr{C} \to \mathscr{C},  
$$
which is assumed to be ``coherently'' associative and commutative in
the sense of MacLane \cite{key1}. 
\end{defi*}

That is, we are given isomorphisms of functors
$$
\perp \circ (1_\mathscr{C} \times \perp ) \approx \perp \circ (\perp \times
1_\mathscr{C}) : \mathscr{C} \times \mathscr{C} \times \mathscr{C} \to
\mathscr{C} 
$$
and\pageoriginale 
$$
\perp \circ T \approx \perp : \mathscr{C} \times \mathscr{C} \to
\mathscr{C}, 
$$
where $T$ is the transposition on $\mathscr{C} \times \mathscr{C}
$. Moreover, these isomorphisms are compatible in the sense that
isomorphisms of products of several factors obtained from these by a
succession of three-fold reassociations, and two-fold permutations, are
all the same. This permits us to write, unambiguously, expressions
like $A_1 \perp \cdots \perp A_n = \overset n {\underset{i=1}
  \perp} A_i$. 

A functor $F: (\mathscr{C}, \perp) \to (\mathscr{C}', \perp')$ of
categories with product is a functor $F: \mathscr{C} \to \mathscr{C}'$
which  ``preserves the product''. More precisely, there should be an
isomorphism of functors $F \circ \perp \approx \perp' \circ (F \times F):
\mathscr{C} \times \mathscr{C} \to \mathscr{C}'$, which is compatible,
in an obvious sense, with the associativity and commutativity
isomorphisms in the two categories. 

Hereafter all products will be denoted by the same symbol $\perp$
(except for special cases where there is a standard notation) and we
will usually write $\mathscr{C}$ instead of $\mathscr{C}$, $\perp$). 

\begin{examples*}
\begin{enumerate}[1)]
\item Let $k$ be a commutative ring and let $\underset{=}P$ denote the
  category of finitely generated projective modules over $k$ with
  isomorphisms as morphisms. It is a category with product if we set
  $\perp = \oplus$. 

\item The full subcategory $\underset{=}{FP}$ of $\underset{=}P$ with
  finitely generated faithful projective modules as objects. Here we
  set $\perp = \otimes_k$. 

\item The full\pageoriginale subcategory $\underset{=}{Pic}$ of
  $\underset{=}{FP}$ 
  whose objects are finitely generated projective modules of rank
  1. We set $\perp = \otimes_k$.  

\item The category $\underset{=}Q$ of quadratic modules over $k$ with
  isometries as morphisms. We take $\perp$ to be the orthogonal sum of
  two quadratic modules.  

\item The category $\underset{=}{Az}$ of Azumaya algebras over $k$
  (see Chapter \ref{chap3}). Here take $\perp = \otimes_k$. 
\end{enumerate}
\end{examples*}

Let $\mathscr{C}(k)$ denote one of the categories mentioned above, and
let $k \to k'$ be a homomorphism of rings. Then $k' \otimes_k$ induces
a functor $\mathscr{C}(k) \to (k')$ preserving product. 

If we neglect naturality conditions, then a category with product is
one whose (isomorphism classes of) objects are a commutative
semi-group. The \textit{Grothendieck group} is got by formally
introducing inverses and making this semi-group into a group. 

\begin{defi*}%defi 0
Let $\mathscr{C}$ be a category with product. The \textit{Grothendieck
  group} of $\mathscr{C}$ is defined to be an abelian group $K_0
\mathscr{C}$, together with a map 
$$
()_{\mathscr{C}}: \obj \mathscr{C} \to K_0 \mathscr{C}, 
$$
which is universal for maps into abelian groups satisfying
\begin{align*}
K_0.& ~ \text{ if } A \approx B, \text{ then } (A)_{\mathscr{C}} =
(B)_\mathscr{C},\\ 
K_1. & ~ (A \perp B)_{\mathscr{C}} = (A)_{\mathscr{C}} + (B)_{\mathscr{C}}. 
\end{align*}
\end{defi*}

In other words if $G$ is an abelian group and $\varphi: \obj
\mathscr{C} \to G$ a map satisfying $K0$ and $K1$, then there exists
a unique homomorphism of groups $\psi : K_0 \mathscr{C} \to G$ such
that $\varphi = \psi 0 (~ )_{\mathscr{C}}$. 

Clearly\pageoriginale $K_0 \mathscr{C}$ is unique. We can construct $K_0
\mathscr{C}$ by reducing the free abelian group on the isomorphism
classes of $\obj \mathscr{C}$ by relations forced by $K1$. 

When $\mathscr{C}$ is clear from the context, we shall write
$(~)$. instead of $(~)_\mathscr{C}$. 

\begin{prop}\label{chap1:prop1.1}% prop 1.1
\begin{enumerate}[(a)]
\item  Every element of $K_0 \mathscr{C}$ has the form $(A)- (B)$ for
  some $A$, $B \epsilon \obj \mathscr{C}$. 

\item $(A) = (B) \Leftrightarrow$ there exists $C \in \obj
  \mathscr{C}$ such that $A \perp C \approx B \perp C$. 

\item If $F: \mathscr{C} \to \mathscr{C}'$ is a functor of categories
  with product, then the map 
$$
K_0 F: K_0 \mathscr{C} \to K_0 \mathscr{C}', 
$$
 given by $(A)_{\mathscr{C}} \longmapsto (FA)_{\mathscr{C'}}$ is well
  defined and makes $K_0$ a functor into abelian groups. 
\end{enumerate}
\end{prop}

We defer the proof of this proposition, since we are going to prove it
in a more general form (proposition \ref{chap1:prop1.2} below). 

\begin{defi*}%defi 0
A {\em composition} on a category $\mathscr{C}$ with product is a
sometimes defined composition $\circ$ of objects of $\mathscr{C}$, which
satisfies the following condition: if $A \circ A'$ and $B \circ B'$ are
defined ($A$, $A'$, $B$, $B' \in \obj \mathscr{C}$), then so also is
$(A \perp B) \circ (A' \perp B')$, and 
$$
(A \perp B) \circ (A' \perp B') = (A \circ A') \perp (B \circ B'). 
$$
\end{defi*}

When this structure is present, we shall require the functors to
preserve it: $F(A \circ B) = (FA) \circ (FB)$. 

\begin{defi*}%defi 0
Let\pageoriginale $\mathscr{C}$ be a category with product and composition.
\end{defi*}

The \textit{Grothendieck group} of $\mathscr{C}$ is defined to be an
abelian group $K_0 \mathscr{C}$, together with a map 
$$
(~ )_{\mathscr{C}} : \obj \mathscr{C} \to K_o \mathscr{C},
$$
which is universal for maps into abelian groups satisfying $K0$, $ K1$
and 

$K2$. if $A \circ B$ is defined, then $(A \circ B)_{\mathscr{C}} =
(A)_{\mathscr{C}} + (B)_{\mathscr{C}}$. 

If composition is never defined, we get back the $K_0$ defined earlier.

As before we write $(~)$ instead of $(~)_{\mathscr{C}}$ when
$\mathscr{C}$ is clear from the context. 

We shall now generalize proposition \ref{chap1:prop1.1}.

\begin{prop}\label{chap1:prop1.2} % prop 1.2
Let $\mathscr{C}$ be a category with product and composition. 
\begin{enumerate}
\renewcommand{\theenumi}{\alph{enumi}}
\renewcommand{\labelenumi}{(\theenumi)}
\item Every element of $K_0 \mathscr{C}$ has the form $(A) - (B)$
  for some $A$, $B \in \obj \mathscr{C}$. 

\item $(A) = (B) \Leftrightarrow$ there exist $C$, $D_0$, $D_1$,
  $E_0$, $E_1 \in \obj \mathscr{C}$, such that $D_0 \circ D_1$ and $E_0 \circ
  E_1$ are defined, and 
$$ 
A \perp C \perp (D_0 \circ D_1) \perp E_0 \perp E_1 \approx B \perp C
\perp D_0 \perp D_1 \perp (E_0 \circ E_1). 
$$

\item  If $F : \mathscr{C} \to \mathscr{C}'$ is a functor of
  categories with product and composition, then the map 
$$
K_0 F: K_0 \mathscr{C} \to K_0 \mathscr{C}',
$$
given by $(A)_\mathscr{C} \longmapsto (FA)_\mathscr{C'}$, is well
  defined and makes $K_0$ a functor into abelian groups. 
\end{enumerate}
\end{prop}

\begin{proof}
\begin{enumerate}[(a)]
\item Any\pageoriginale element of $K_0 \mathscr{C}$ can be written as
$$ \sum\limits_i (A_i) - \sum\limits_j (B_j) = ( \underset{i}\perp A_i
  ) - (\underset{j} \perp B_j).
$$
 
\item Let us denote by [$A$] the isomorphism class containing $A
  \epsilon \obj \mathscr{C}$, and by $M$ the free abelian group
  generated by these classes. A relation $\sum [A_i] = \sum [B_j]$ in
  $M$ implies an isomorphism $\perp A_i \approx \perp B_j$ in
  $\mathscr{C}$. 

Now, if $(A) = (B)$, then we have a relation of the following type in  
$M$: 
\begin{align*}
[A] - [B] = & \sum \{ [ C_{h0} \perp C_{h1}] - [C_{h0}] - [Ch_{h1}] \} \\
+ & \sum \{ [ C'_{i0}] + [C'_{i1}] - [C'_{i0} \perp C'_{i1}] \} \\
+ & \sum \{ [ D_{j0}] + [D_{j1}] - [D_{j0} \circ D_{j1}] \} \\
+ & \sum \{ [ E_{l0} \circ E_{l1}] - [E_{l0}] - [E_{l1}] \}, \\
\end{align*}
or 
\begin{gather*}
[A]  + \sum \{ [ C_{h0}] +[C_{h1}]\} + \sum[C'_{i0} \perp C'_{i1}]\\
+
\sum[D_{j0} \circ D_{j1}] + \sum \{[E_{l0}] + [E_{l1}]\}  \\ 
= [B]  + \sum[ C_{h0} \perp C_{h1}] + \sum \{[C'_{i1}]\\ + [C'_{i1}]\}+
\sum\{[D_{j0}] +[D_{j1}] \} + \sum [E_{l0} \circ E_{l1}].  \\ 
\end{gather*}

This implies an isomorphism
$$
A \perp C \perp (D_0 \circ D_1) \perp E_0 \perp E_1 \approx B \perp C 
\perp D_0 \perp D_1 \perp (E_0 \circ E_1). 
$$
where\pageoriginale
\begin{align*}
C & = (~ \underset{h}\perp C_{h0}) \perp (\underset{h}\perp C_{h1})
\perp (\underset{i}\perp C'_{i0}) \perp (\underset{i}\perp C'_{i1}),
\\ 
D_0 & = \underset{j}\perp D_{j0}, \quad   E_0 = \underset{l}\perp
E_{10},\\ 
D_1 & = \underset{j}\perp D_{j1}, \quad   E_1 = \underset{l}\perp
E_{11}.
\end{align*}

The other implication is a direct consequence of the definition of
$K_0 \mathscr{C}$. 

\item The map $\obj \mathscr{C} \to K_0 \mathscr{C}'$ given by $A
  \longmapsto (FA)_{\mathscr{C'}}$ satisfies $K0$, $K1$ and $K2$. This
  gives rise to the required homomorphism $K_0 \mathscr{C} \to K_0
  \mathscr{C'}$. The rest is straightforward. 
\end{enumerate}
\end{proof}

Now let $\mathscr{C}$ be simply a category with product. For $A \in
\obj \mathscr{C}$, we write 
$$
G(A) = \mathscr{C} (A, A),
$$
the group of automorphisms of $A$. (Recall that $\mathscr{C}$ is a
groupoid.) If $f: A \to B$, we have a homomorphism 
$$
G(f): G(A) \to G(B),
$$
given by $G(f) (\alpha) = f \alpha f^{-1}$.

We shall now construct, out of $\mathscr{C}$, a new category $\Omega
\mathscr{C}$. We take $\obj \Omega \mathscr{C}$ to be the collection of
all automorphisms in $\mathscr{C}$. If $\alpha \in \obj \Omega
\mathscr{C}$ is an automorphism of $A \in \mathscr{C}$, we shall
sometimes write $(A, \alpha)$ instead of $\alpha$, to make $A$
explicit. A morphism $(A, \alpha) \to (B, \beta)$ in $\Omega
\mathscr{C}$ is a morphism $f : A \to B$ in $\mathscr{C}$ such that
the\pageoriginale diagram 
\[
\xymatrix{
A \ar[r]^f \ar[d]_{\alpha} & B \ar[d]^{\beta}\\
A \ar[r]_f & B
}
\]
is commutative, that is, $G(f) (\alpha) = \beta$. We define a product
in $\Omega \mathscr{C}$ by setting $(A, \alpha) \perp (B, \beta) = (A
\perp B, \alpha \perp \beta)$. There is a natural composition 0 in
$\Omega \mathscr{C}$: if $\alpha$, $\beta \epsilon \obj \Omega
\mathscr{C}$ are automorphisms of the same object in $\mathscr{C}$,
then we take $\alpha \circ \beta$ to be the usual of morphisms. The
compatibility of $\perp$ and $0$ in $\Omega \mathscr{C}$ is the
identity 
$$
(\alpha \perp \beta) \circ (\alpha' \perp \beta') = ( \alpha \circ \alpha')
\perp (\beta \circ \beta'), 
$$
which simply expresses the fact that $\perp$ is a functor (of two
variables). 

\begin{defi*}%defi 0
If $\mathscr{C}$ is a category with product, we define
$$
K_1 \mathscr{C} = K_0 \Omega \mathscr{C}.
$$
\end{defi*}

Let $F : \mathscr{C} \to \mathscr{C}'$ be a functor of categories with
product. Then $F$ induces $\Omega F: \Omega \mathscr{C} \to \Omega
\mathscr{C}'$, preserving product and composition, so we obtain
homomorphisms 
$$
K_i F : K_i \mathscr{C} \to K_i \mathscr{C}' \qquad i=0,1.
$$

We propose now to introduce a relative group to connect the above into
a 5-term exact sequence. 

First\pageoriginale we construct the relative category $\Phi F$ with
respect to the 
functor $F$. Objects of $\Phi F$ are triples $(A, \alpha, B)$, $A$, $B
\in \obj \mathscr{C}$ and $\alpha $ :  $FA \to FB$. A morphism $(A,
\alpha, B) \to (A', \alpha', \beta')$ in $\Phi F$ is a pair $(f, g)$
of morphisms $f$: $A \to A'$ and $g$: $B \to B'$ in $\mathscr{C}$ such
that 
\[
\xymatrix{
FA \ar[r]^{Ff} \ar[d]_{\alpha} & FA' \ar[d]^{\alpha'} \\
FB \ar[r]_{Fg} & FB'
}
\]
is a commutative diagram. We define product and composition in $\Phi
F$, by setting 
\begin{align*}
 (A, \alpha, B) \perp (A', \alpha', B') & = (A \perp A', \alpha \perp
  \alpha', B \perp B'),\\ 
 (B, \beta, C) \, \circ \, (A, \alpha, B) & = (A, \beta \alpha, C). 
\end{align*}

We shall see in \S \ref{chap1:sec4} that under some restriction on $F$, the
Grothendieck group of this relative category $\Phi F$ fits into an
exact sequence involving the $K'_i s$ of $\mathscr{C}$ and
$\mathscr{C}'$. 

We record here a few facts about $K_0 \Phi F$ which we shall need later:

\setcounter{remark}{2}
\begin{remark}% rem 1.3
\begin{enumerate}[(a)]
\item $(A, 1_{FA}, A)_{\Phi F} = 0$ for any $A \in \obj
  \mathscr{C}$. This follows from the fact $(A, 1_{FA}, A) \circ (A,
  1_{FA}, A) = (A, 1_{FA}, A)$ in $\Phi F$. 

\item $(A, \alpha, B)_{\Phi F} = -(B, \alpha^{-1}, A)_{\Phi F}$ for
  any $(A, \alpha, B) \in \obj \Phi F$. This follows from (a) and the
  equation $(B, \alpha^{-1}, A) 0 (A, \alpha, B) = (A, 1_{FA}, A)$. 

\item Any\pageoriginale element of $K_0 \Phi F$ can be written as $(A, \alpha,
  B)_{\Phi F}$. For, by proposition \ref{chap1:prop1.2} (a), any
  element of $K_0 \Phi 
  F$ can be written as $(A, \alpha, B)_{\Phi F} - (A', \alpha',
  B')_{\Phi F}$. But this equals $(A \perp B', \alpha \perp
  \alpha'^{-1}, B \perp A')_{\Phi F}$, in view of (b) above, and the
  axiom $K1$. 
\end{enumerate}
\end{remark}

We close this section with a lemma about permutations that will be
needed in \S \ref{chap1:sec4}. Consider a permutation $s$ of $\{ 1, \ldots,
n\}$. The axiom of commutativity for $\perp$ gives us, for any
$A_1, \ldots, A_n$, a well defined isomorphism 
$$
A_1 \perp \cdots \perp A_n \xrightarrow{\approx} A_{s(1)} \perp \cdots 
\perp A_{s(n)},  
$$
which we shall also denote by $s$. If $\alpha_i : A_i \to B_i$, then
the diagram 
\[
\xymatrix@C=2cm{
A_1 \perp \ldots \perp A_n \ar[r]^{\alpha_1 \perp \ldots \perp
  \alpha_n} \ar[d]_s & B_1 \perp \ldots \perp B_n \ar[d]^s\\
A_{s(1)} \perp \ldots \perp A_{s(n)} \ar[r]_{\alpha_{s(1)} \perp
  \ldots \perp \alpha_{s(n)}} & B_{s(1)} \perp \ldots \perp B_{s(n)}
}
\]
is commutative, that is 
\begin{equation*}
s(\alpha_1 \perp \cdots \perp \alpha_n) = (\alpha_{s(1)} \perp \ldots
\perp \alpha_{s(n)})s. \tag{1.4}\label{eq1.4} 
\end{equation*}

Suppose now that $\alpha_i : A_i \to A_{i+1}$, $1 \leq i \leq n -1$,
and $\alpha_n : A_n \to A_1$. Let $s(i) = i-1 (\mod n)$. and set
$\alpha = \alpha_1 \perp \cdots \perp \alpha_n$. Then $(A_1 \perp
\cdots \perp A_n, s_\alpha) \epsilon \obj \Omega \mathscr{C}$. If  
\begin{gather*}
\beta = (1_{A_1} \perp \alpha^{-1}_1 \perp \cdots \perp 
(\alpha_{n-1}\cdots \alpha_1)^{-1}),\\
A_1 \perp A_2 \perp \cdots \perp
A_n  \to A_1 \perp \cdots \perp A_1. 
\end{gather*}
then\pageoriginale $\beta: (A_1 \perp \cdots \perp A_n, s \alpha) \to
(A_1 \perp 
\cdots \perp A_1, \beta s \alpha \beta^{-1})$ in $\Omega
\mathscr{C}$. Now $\alpha \beta^{-1} = (\alpha_1 \perp \alpha_2
\alpha_1 \perp \cdots \perp (\alpha_n \cdots \alpha_1))$, and by
(\ref{eq1.4}) 
above, $\beta s = \alpha^{-1}_1 \perp (\alpha_2 \alpha_1)^{-1} \perp
\cdots \perp (\alpha_{n-1} \cdots \alpha_1)^{-1} \perp
1_{A_1}$. Consequently: 

\setcounter{lemma}{4}
\begin{lemma}\label{chap1:lem1.5}  % lem 1.5
Suppose $\alpha_i: A_i \to A_{i+1}$, $1 \leq i \leq n-1$ and $\alpha_n$: $A_n
\to A_1$. Let $s$ denote the permutation $s(i) = i-1 (\mod n)$. Then
in $\Omega \mathscr{C}$ 
\begin{gather*}
(A_1 \perp \cdots \perp A_n, s(\alpha_1 \perp \cdots \perp \alpha_n))\\
\approx (A_1 \perp \cdots \perp A_1, ~ 1_{A_1} \perp \cdots \perp' A_1
\perp (\alpha_n - \alpha_1)) 
\end{gather*}

In particular, if $\alpha : A \to B$ and $\beta : B \to C$, then 
$$
(A \perp B, t(\alpha \perp \alpha^{-1})) \approx (A \perp A, 1)
$$
and
$$
(A \perp B \perp C, s(\alpha \perp \beta \perp (\beta \alpha)^{-1}))
\approx (A \perp A \perp A, 1) 
$$
in $\Omega \mathscr{C}$, where $t$ and $s$ are the transposition and
the three cycle, respectively. 
\end{lemma}


\section{Directed categories of abelian groups}\label{chap1:sec2}%sec 2

In the next section we shall see that $K_1 \mathscr{C}$ can be
calculated as a kind of generalized direct limit. We discuss in this
section some necessary technical preliminaries. 

In this section $\mathscr{G}$ will denote a category of abelian
groups. Also, we shall assume that $\mathscr{G}$ is a set. 

\begin{defi*}%defi 0
A {\em direct limit} of $\mathscr{G}$ is an abelian group $
\underset{\to}{\mathscr{G}}$ together with a family of homomorphisms $f_A
: A \to \underset{\to}{\mathscr{G}}$. $A \in \obj \mathscr{G}$,
such\pageoriginale that the diagram 
\[
\xymatrix{
A \ar[rr]^f \ar[dr]_{f_A} && B \ar[dl]^{f_B}\\
& \underset{\to}{\mathscr{G}} & 
}
\]
is commutative for any morphism $f: A \to B$ and
$\underset{\to}{\mathscr{G}}$ is universal for this property. 
\end{defi*}

Clearly $\underset{\to}{\mathscr{G}}$ is unique. Also, it follows that
$\underset \to {\mathscr{G}}$ is the sum of its subgroups $f_A(A)$. We
can describe $\underset \to {\mathscr{G}}$ as the  quotient of $
\underset {A \, \in \, \obj \mathscr{G}} \oplus A$ by the subgroup generated
by the elements of the type $f(a)-a$, where $f$ is any morphism $A \to
B$ and $a \in A$. 

\begin{lemma}\label{chap1:lem2.1}%lem 2.1
Let $\mathscr{G}$ be such that given two objects $A$, $B$, there
exists an object $C$, with $\mathscr{G}(A, C)$ and $\mathscr{G}(B, C)$
non-empty. Then $\underset \to {\mathscr{G}} = \bigcup _{f_A (A)}$. 
\end{lemma}

\begin{proof}
Any element of $\underset{\to}{\mathscr{G}}$ can be written as a
finite sum $\sum f_{A_i} (a_i)$, $a_i \in A_i$. To establish our
assertion, it is enough to express an element of the type $f_A (a) +
f_B(b)$ as $ f_C(c)$ for some $C$ and $c \in C$. We choose $C$ such
that there are morphisms $g:A \to C$, $h : B \to C$. Then $c = g(a) + h(b)$
serves our purpose. 
\end{proof}

It follows in particular that, if $\mathscr{G}$ has a ``finial''
object, that is, an object $C$ such that $\mathscr{G} (A, C) \neq
\phi$ for every $A \in \obj \mathscr{G}$, then $f_C : C \to \underset
\to {\mathscr{G}}$ is surjective. Let $N$ be the subgroup of $C$
generated by all elements of the type $f_1(a) - f_2(a)$, $f_i \in
\mathscr{G} (A, C)$, $a \in A$. Clearly\pageoriginale $N \subset \ker
f_C$ and this 
induces a map $C/N \to \underset \to {\mathscr{G}}$. On the other
hand, all morphisms $A \to C$ induce the same map $A \to C/N$, and the
latter are clearly compatible with the morphisms $A \to B$. The
universal mapping property gives now a map $\underset \to
{\mathscr{G}} \to C/N$ which is easily checked to be the inverse of
$C/N \to \underset \to {\mathscr{G}}$. Thus $C/N \to \underset \to
{\mathscr{G}}$ is an isomorphism, that is, $N = \ker f_C$. 

\begin{defi*}%defi 0
$\mathscr{G}$ is called {\em directed}, if 
\begin{enumerate}
\renewcommand{\labelenumi}{(\theenumi)}
\item given $A, B \in \obj \mathscr{G}$, there exists $C \in \obj
  \mathscr{G}$, such that $\mathscr{G}(A, C)$ and $\mathscr{G}(B, C)$
  are both non-empty. 

\item given $f_i : A \to B$, $i=1, 2$, there exists $g : B \to C $ such
  that $gf_1 = gf_2$. 
\end{enumerate}
\end{defi*}

We note that lemma \ref{chap1:lem2.1} is valid for directed
categories. 

\begin{lemma}\label{chap1:lem2.2} %lem 2.2
Let $\mathscr{G}$ be directed and let $f_A(a) = 0$ for some $A \in \obj
\mathscr{G}$ and $a \in A$. There exists then a morphism $g : A \to B$
such that $g(a) = 0$. 
\end{lemma}

\begin{proof}
Since $f_A(a) = 0$, we have, in the direct sum of the $C's$, $a = \sum
\limits_i \pm (f_i (c_i)-c_i)$. Since only a finite number of terms
appear in the relation, we can find a $C$ into which all the
intervening groups map. In particular, if $\mathscr{G}'$ is the full
subcategory of $\mathscr{G}$ whose objects are those which have a map into
$C$, then $\mathscr{G}'$ has $C$ as a final object and we have
$f'_A(a) =0$, where $f'_A : A \to \underset \to {\mathscr{G}}'$. Now
it follows from the last paragraph that if $f : A \to C$, then there
is a family of pairs $f_{1i}$, $f_{2i} : B_i \to C$, and $b_i \in
B_i$, $1 \leq i \leq m$, such that $f(a)= \sum \limits^m_{i=1}
f_{1i}(b_i) - f_{2i}(b_i)$. Since $\mathscr{G}$ is directed,\pageoriginale it
follows easily, by induction on $m$, that there exists an $h : C \to
B$ such that $hf_{1i} = hf_{2i}$, $1 \leq i \leq m$. Then $hf(a) =
0$. 
\end{proof}

\begin{defi*}%defi 0
A subcategory $\mathscr{G}'$ of a directed category $\mathscr{G}$ is
called {\em dominating}, if 
\begin{enumerate}
\renewcommand{\labelenumi}{(\theenumi)}
\item given $A \in \obj \mathscr{G}$, there exists $A' \in \obj
  \mathscr{G}'$ and a map $A \to A'$ in $\mathscr{G}$, 

\item given $f_i : A'_i \to B$ in $\mathscr{G}$, $ i =  1, 2$, with
  $A'_i \in \mathscr{G}'$, there exists $g : B \to C'$ with $C' \in
  \obj \mathscr{G}'$ such that $gf_i \in \mathscr{G}'$, $i = 1, 2$. 
\end{enumerate}
\end{defi*}

We note first that $\mathscr{G}'$ is also directed. For given $A'_1$,
$A'_2 \in \obj \mathscr{G}'$, we can find $f_i : A'_i \to B$ in
$\mathscr{G}$. There exists then $a g : B \to C'$ with $C' \in \obj
\mathscr{G}'$ such that $gf_i$ is a morphism in $\mathscr{G}'$, $i =
1, 2$. Next suppose $f_1$, $f_2 : A' \to B'$ are maps in
$\mathscr{G}'$. There exists $g : B' \to C$ in $\mathscr{G}$ such that
$gf_1 = gf_2$. We can find $h : C \to D'$ with $hg$ in
$\mathscr{G}'$. Thus we have a morphism $hg$ in $\mathscr{G}'$ with
$(hg) f_1 = (hg)f_2$. 

\setcounter{prop}{2}
\begin{prop}\label{chap1:prop2.3} % prop 2.3
If $\mathscr{G}'$ is a dominating subcategory of the directed category
$\mathscr{G}$, then the natural map $\varphi : \underset \to
{\mathscr{G}'} \to \underset \to {\mathscr{G}}$ is an isomorphism. 
\end{prop}

\begin{proof}
Write $f'_{A'} : A' \to \underset \to {\mathscr{G}'}$ and $f_A : A \to
\underset \to {\mathscr{G}}$ for the canonical maps. If $A' \in \obj
\mathscr{G}'$, $A \in \obj \mathscr{G}$ and $A'
\displaystyle{\mathop{\underline{\quad}}^g} A$ is a 
map in either direction in $\mathscr{G}$, then 
\[
\xymatrix@C=1.5cm@R=1.5cm{
A' \ar[r]^g \ar[d]_{f'_{A'}} \ar[dr]_{f'_{A'}} & A \ar[d]^{f_A} \\
\underset{\to}{\mathscr{G}'} \ar[r]_{\varphi} & \underset{\to}{\mathscr{G}}
}
\]
is commutative.
\end{proof}

Given\pageoriginale $A$, we can find $g : A \to A'$ by (1), so that $\im
f_A \subset 
\Iim f_{A'} = \im \varphi f_{A'} \subset \im \varphi$. It follows from lemma
\ref{chap1:lem2.1}, that $\varphi$ is surjective. Suppose now $x \in \ker
\varphi$. Since $\mathscr{G}'$ is directed, lemma \ref{chap1:lem2.1}
is applicable to 
$\mathscr{G'}$ and we can write $x = f_{A'}(a')$. By lemma
\ref{chap1:lem2.2} there 
exists $g : A' \to A$ such that $g(a') = 0$. Choose $h : A \to B'$ in
$\mathscr{G}$ such that hg is in $\mathscr{G'}$. Thus $hg(a') = 0$,
so that $x = f'_{A'}a' = f'_{B'} hg (a') = 0$, which shows that
$\varphi$ is injective. 


\section{$K_1 \mathscr{C}$ as a direct limit}\label{chap1:sec3}%sec 3   

Let $\mathscr{C}$ be a category with product. If $A$ is an object of
$\mathscr{C}$, we write $G(A)$ for its automorphism group, [A] for the
isomorphism class of $A$, and $G[A]$ for the abelianization of
$G(A)$, that is, the quotient of $G(A)$ by its commutator
subgroup. This notation is legitimate because any two isomorphisms $A
\to B$ induce the same isomorphism $G[A] \to G[B]$, since $G(A) \to
G(B)$ is unique up to inner automorphisms. 
 
 We now propose to construct a directed category $\mathscr{G}$ of
 abelian\break groups, in the sense of \S \ref{chap1:sec2}. The objects of
 $\mathscr{G}$ 
 are the $G [A]$, $A \in \obj \mathscr{C}$. As for morphisms in
 $\mathscr{G}$, we set $\mathscr{G} (G[A],G[B]) = \phi$ if there
 exists no $A'$ with $A \perp A' \approx B$. Otherwise let $h : A \perp A'
 \to B$ be an isomorphism for some $A'$. We have a homomorphism $G(A)
 \to G(B)$ given by $\alpha \mapsto G(h) (\alpha \perp 1_{A'})$. This
 induces a homomorphism $f : G[A] \to G[B]$ which is independent of
 the isomorphism $h$ chosen and depends only on the
 isomorphism\pageoriginale class 
 $[A']$ of $A'$. The homomorphism $f$ will be denoted by $G[A] \perp
 [A']$. Now we define $\mathscr{G}(G [A], G [B])$ to be the set of all
 homomorphisms $G[A] \to G[B]$ which are of the form $G[A] \perp [A']$
 for some $A'$ with $A \perp A' \approx B$. 
 
 We define composition of morphisms in $\mathscr{G}$ by
 $$
 (G[B] \perp [B'] ) (G [A] \perp [A']) = G [ A] \perp [A' \perp B'],
 $$
 where $A \perp A' \approx B$.
 
 Since $\mathscr{G} (G [A_i], G[A_1 \perp A_2])$ is not empty for $i =
 1, 2$, $\mathscr{G}$ satisfies the condition (1) in the definition
 of a directed category. To verify (2), suppose given $f_1$, $f_2:
 G[A] \to G[B]$, 
 
 $f_i = G[A] \perp [A'_i]$. Then $[A \perp A'_i] =[B]$ so if we set  
 
 $g = G[B] \perp [A]: G[B] \to G[B \perp A]$, we have  
$$
gf_i = (G[B] \perp [A]) ( G[A] \perp [A'_i]) = G[A] \perp [A'_i \perp
   A] = G[A] \perp [B], 
$$
which is independent of $i$.

\begin{defi*}%defi 0
A functor $F : \mathscr{C}' \to \mathscr{C}$ of categories with
product is called {\em cofinal} if every $A \in \obj \mathscr{C}$
``divides'' $FB$' for some $B' \in \obj \mathscr{C}'$, that is, if $A
\perp A_1 \approx FB'$ for some object $A_1$ of $\mathscr{C}$. A
subcategory $\mathscr{C}' \subset \mathscr{C}$ is called {\em cofinal
} if the inclusion functor is. 
\end{defi*}

\begin{theorem}\label{chap1:thm3.1}%the 3.1
Let $\mathscr{C}$ be a category with product. 
\begin{enumerate}
\renewcommand{\labelenumi}{(\theenumi)}
\item Let $\mathscr{G}$ be the directed category of abelian groups
  constructed\break above. There is a canonical isomorphism 
$$
\underset{\to}{\mathscr{G}} \xrightarrow{\approx} K_1 \mathscr{C}.  
$$

\item Let $\mathscr{C}'$ be a full cofinal subcategory of
  $\mathscr{C}$. The inclusion of $\mathscr{C}'$ in $\mathscr{C}$
  induces an isomorphism  
$$
K_1 \mathscr{C}' \xrightarrow{\approx} K_1 \mathscr{C}.
$$
\end{enumerate}
\end{theorem}

\begin{proof}
\begin{enumerate}[(a)]
\item If\pageoriginale $\alpha \in G (A)$ let $(\alpha)$ denote its
  image in $K^{\mathscr{C}}_1$.  

Since $(\alpha \beta) = (\alpha) + (\beta)$, the map $\alpha \mapsto
(\alpha)$ is a homomorphism $G(A) \to K_1 \mathscr{C}$ which, since
$K_1 \mathscr{C}$ is abelian, induces $g_{[A]} : G [A] \to K_1
\mathscr{C}$. In particular, since $(1_{A'}) = 0$, we have $(\alpha \perp
1_{A'}) = (\alpha) + (1_{A'}) = (\alpha)$ and this implies that the
$g_{[A]}$ actually define a map of the directed category $\mathscr{G}$
into $K_1 \mathscr{C}$. Hence we have a homomorphism
$\underset{\to}{\mathscr{G}} \to K_1 \mathscr{C}$. To construct its
inverse we need only observe the obvious fact that the map assinginig
to each $\alpha$ its image $(via G(A) \to G[A] \to
\underset{\to}{\mathscr{G}})$ in $\underset{\to}{\mathscr{G}}$ satisfies
the axioms defining $K_1$, so that by universality, we get the desired
homomorphism $K_1 \mathscr{C} \to \underset{\to}{\mathscr{G}}$. 

\item If $A'$ and $B'$ are two objects of $\mathscr{C}' \subset
  \mathscr{C}$, the symbols $G(A'), G[A']$ and $G[A'] \perp [B']$ are
  unambiguous since $\mathscr{C}'$ is full in $\mathscr{C}$. Let
  $\mathscr{G}'$ be the directed category associated with
  $\mathscr{C}'$. Evidently $\mathscr{G}' \subset \mathscr{G}$, and we
  need only show  that $\mathscr{G}'$ is a dominating subcategory of
  $\mathscr{G}$ (in the sense of \S \ref{chap1:sec2}), for then we can invoke
  proposition \ref{chap1:prop2.3}. 
\end{enumerate}
\end{proof}

Given $G[A] \in \obj \mathscr{G}$, choose $A \perp B \approx C', B
\in \obj \mathscr{C}, C' \in \obj \mathscr{C}'$. This is possible
because $\mathscr{C}'$ is cofinal in $\mathscr{C}$. Then $G[A] \perp
[B] : G [A] \to G[C']$, and $G[C'] \in \obj \mathscr{G}'$. This
verifies condition $1)$ for $\mathscr{G}'$ to be dominating in
$\mathscr{G}$. Condition 2) requires that if $f_1$, $f_2 : G[A'] \to
G[B]$, $A' \in \obj \mathscr{C}'$, then there exists $g: G[B] \to
G[C']$ such that $gf_i$ is a morphism in $\mathscr{G}', i = 1,2$. Let
$f_i = G[A'] \perp [A_i]$. 

Choose\pageoriginale $D \in \obj \mathscr{C}$ so that $B \perp D
\approx D' \in \obj 
\mathscr{C}'$. Set $C' = A' \perp D'$ and let $g = G[B] \perp [A'
  \perp D]$. Then $gf_i = (G [B] \perp [A' \perp D]) (G[A'] + [A_i]) =
G[A'] \perp[A_i \perp A' \perp D] = G[A'] \perp [B \perp D]= G[A']
\perp [D']$ which is a morphism in $\mathscr{G}$. 

\begin{defi*}%defi 0
An object $A$ of $\mathscr{C}$ is called \textit{basic} if the
sequence $A^n = A \perp \cdots \perp A$ ($n$ factors) is cofinal; that
is, every $B \in \obj \mathscr{C}$ divides $A^n$ for some $n$. 
\end{defi*} 

If $A$ is basic the full subcategory $\mathscr{C}'$ whose objects are
the $A^n$, $n \geq 1$, is a full cofinal subcategory (with product) to
which we may apply the last theorem. If we assume that $A^n \approx
A^m \Longrightarrow n = m$, then $\mathscr{G}'$ is an ordinary direct
sequence of abelian groups. The groups are $G[A^n]$, $n \geq 1$, and
there is a unique map, $G[A^n] \to G[A^{n+m}]$, namely $G[A^n] \perp
[A^m]$, which is induced by $\alpha \mapsto \alpha 1_{A}m$. These are
only non-identity morphisms in $\mathscr{G}'$. 

In this case we can even make a direct system from the $G(A^n)$, by  
$$
G(A^n) \to G(A^{n+m}) ; \alpha \mapsto \alpha \perp 1_{A}m.
$$
If we write
$$
G(A^{\infty}) = \lim \limits_{\to} G(A^n)
$$
then it is clear that
$$
\lim \limits_{\to} G[A^n] = G(A^{\infty}) / [G(A^{\infty}), G(A^{\infty}].
$$

\begin{theorem}\label{chap1:thm3.2}%theo 3.2
Suppose\pageoriginale that $A$ is a basic object of $\mathscr{C}$, and
that $A^n \approx A^m$ implies $n =m$.  
\begin{enumerate}
\renewcommand{\theenumi}{\alph{enumi}}
\renewcommand{\labelenumi}{(\theenumi)}
\item $K_1 \mathscr{C}$ is the direct limit
\begin{align*}
K_1 \mathscr{C} & \approx \lim \limits_{\to} (G [A^n]; G[A^n] \perp
[A^m] : G[A^n] \to G[A^{n+m}])\\ 
& \approx G(A^{\infty}) / [G(A^{\infty}, G(A^{\infty})]. 
\end{align*}

\item If $\alpha$, $\beta \in G(A^n)$, then $(\alpha) = (\beta) $ in
  $K_1 \mathscr{C} \Leftrightarrow$ there exist $\gamma \in G(A^m)$
  and $\delta_1$, $\delta_2$, $\varepsilon_1$, $\varepsilon_2 \in
  G(A^p)$, for some $m$ and $p$, such that 
$$
\alpha \perp \gamma(\delta_1 \delta_2) \perp 1_{A^p} \perp
\varepsilon_1 \perp \varepsilon_2 
$$
and 
$$
\beta \perp \gamma \perp \delta_1 \perp \delta_2 \perp (\varepsilon_1
\varepsilon_2) \perp 1_{A^p} 
$$
are conjugate in $G(A^{n + m + 4p})$.

\item $(\alpha) = 0$ in $K_1 \mathscr{C} \Leftrightarrow$ there exists 
  $\gamma \in G(A^m)$ for some $m$, such that  
$$
\alpha \perp \gamma \perp \gamma^{-1} 
$$
is a commutator. Moreover, $\alpha^2 \perp 1_{A^{2m}}$ is a product of
two commutators. 
\end{enumerate}
\end{theorem} 

\begin{proof}
\begin{enumerate}[(a)]
\item Follows directly from Theorem \ref{chap1:thm3.1} and the
  preceding remarks. 

\item The implication $\Leftarrow$ is clear.

For $\Rightarrow$, we apply Proposition \ref{chap1:prop1.2} (b) to the category
$\mathscr{C}'$ consisting\pageoriginale of $A^n$, and use part (a)
above to obtain 
$\gamma$, $\delta_1$, $\delta_2$, $\varepsilon_1$, $\varepsilon_2$
such that  
$$
\bar{\alpha} = \alpha \perp \gamma \perp (\delta_1 \delta_2) \perp
\varepsilon_1 \perp \varepsilon_2 
$$
and 
$$
\bar{\beta} = \beta \perp \gamma \perp \delta_1 \perp \delta_2 \perp
(\varepsilon_1 \varepsilon_2) 
$$
are isomorphic Write $n= n (\alpha)$ if $\alpha \in G(A^n)$, and
similarly for $\beta, \gamma, \ldots$. Our hypothesis shows that
$n(\alpha)$ is well defined and that  
\begin{align*}
n(\alpha) + n(\gamma) + & n(\delta_1 \delta_2) + n (\varepsilon_1) + n
(\varepsilon_2)\\ 
& = n(\beta) + n(\gamma) + n(\delta_1) + n (\delta_2) + n
(\varepsilon_1 \varepsilon_2). 
\end{align*}
Since $n(\alpha) = n (\beta)$, $n (\varepsilon_1) = n(\varepsilon_2) =
n(\varepsilon_1 \varepsilon_2)$ and $n(\delta_1) = n(\delta_2)=
n(\delta_1 \delta_2)$, we conclude that $n(\delta_i) = n
(\varepsilon_i);$ call this integer $p$, and write $m = n(\gamma)$. 
\end{enumerate}
\end{proof}

Since $\bar{\alpha} \approx \bar{\beta}$, we have $\bar{\alpha} \perp
1_{A^p} \approx \bar{\beta} \perp 1_{Ap}$. Both of these are in
$G(A^{n + m + 4p})$, and we can conjugate by suitable permutations of
the factors to obtain  
$$
\alpha' = \alpha \perp \gamma \perp (\delta_1 \delta_2) \perp 1_{Ap}
\perp \varepsilon_1 \perp \varepsilon_2 
$$
and
$$
\beta' = \beta \perp \gamma \perp \delta_1 \perp \delta_2 \perp
(\varepsilon_1 \varepsilon_2) \perp 1_{Ap}. 
$$
Now,\pageoriginale two elements of $G(A^{n+ m + 4p})$ are isomorphic
if and only if 
they are conjugate (recall the definition, in \S \ref{chap1:sec1}, of
isomorphism 
in $\Omega \mathscr{C})$. Therefore $\alpha'$ and $\beta'$ are
conjugates. This completes the proof of (b). 

Moreover, $\beta'^{-1} \alpha' = (\beta^{-1} \alpha) \perp 1_{A^m}
\perp \delta_2 \perp \delta^{-1}_2 \perp \varepsilon^{-1}_2 \perp
\varepsilon_2$ is a commutator. Conjugating by a permutation of
factors, we find that $(\beta^{-1} \alpha) \perp 1_{A^m} \perp
(\delta_2 \perp \varepsilon_2) \perp (\delta_2 \perp
\varepsilon_2)^{-1}$ is a commutator. Since we could have chosen $m =
2m'$, we could take $\gamma_1 = 1_{A^m} \perp \delta_2 \perp
\varepsilon_2$, and a further conjugation shows that 
$$
(\beta^{-1} \alpha) \perp \gamma_1 \perp \gamma^{-1}_1 
$$
is a commutator. Assuming $\beta = 1_{A^n}$ we have proved the first
part of $(c) : \alpha \perp \gamma_1 \perp \gamma^{-1}_1$ is a
commutator. Since $\alpha \perp \gamma_1 \perp \gamma^{-1}_1$ is
conjugate to $\alpha \perp \gamma_1^{-1} \perp \gamma_1$, their
product $\alpha^2 \perp 1_{A} 2m_1, m_1 = n(\gamma_1)$, is a product
of two commutators. This proves the last assertion in (c). 


\section{The exact sequence}\label{chap1:sec4}%sec 4

Throughout this section $F : \mathscr{C} \to \mathscr{C}'$ denotes a
cofinal functor of categories with product. 

We define
$$
d : K_0 \Phi F \to K_0 \mathscr{C}
$$
to be the homomorphism induced by the map $(A, \alpha, B) \mapsto
(A)_{\mathscr{C}} - (B)_{\mathscr{C}}$ from\pageoriginale  $\obj \Phi F$ to $K_0
\mathscr{C}$. This is clearly additive with respect to $\perp$ to $0$
in $\Phi F$ so it does define a homomorphism $d$. The composite of $d$ and
$K_0 F : K_0 \mathscr{C} \to K_0 \mathscr{C}'$ sends $( A,
\alpha,B)_{{\Phi} F}$ to $(FA)_{\mathscr{C}'} - (FB)_{\mathscr{C}'}$,
which is zero, since $FA$ and $FB$ are isomorphic. 

Suppose $(A)_{\mathscr{C}} - (B)_{\mathscr{C}} \in \ker K_0 F$. Using
Proposition \ref{chap1:prop1.1}, we can find a $C' \in \mathscr{C}'$
and an $\alpha : 
Fa \perp C' \to FB \perp C'$. Cofinality of $F$ permits us to choose
$C' = FC$ for some $C \in  \obj \mathscr{C}$. Then $d$ maps $(A \perp
C, \alpha, B \perp C)$ into $(A)_{\mathscr{C}} -
(B)_{\mathscr{C}}$. Thus we have proved that the sequence 
$$
K_0 \Phi F \xrightarrow{d} K_0 \mathscr{C} \xrightarrow{K_0 F} K_0
\mathscr{C}' 
$$
is exact.


Let $\mathscr{C}_1$ denote the full subcategory of $\mathscr{C}'$ whose
objects are all $FA$,\break $A \in \obj \mathscr{C}$. By Theorem
\ref{chap1:thm3.1} (b), we 
have an isomorphism 
$$
\theta : K_1 \mathscr{C}_1 \to K_1 \mathscr{C}'.  
$$
Let 
$$
d_1 : K_1 \mathscr{C}_1 \to K_0 \Phi F  
$$
be the homomorphism induced by the map $(FA, \alpha) \mapsto (A,
\alpha, A)_{\Phi F}$ from $\obj \Omega \mathscr{C}_1$ to $K_0 \Phi
F$. This map is additive with respect to $\perp$ and 0 in $\Omega
\mathscr{C}_1$, so that $d_1$ is well defined. The composite $d \circ d_1$
sends $(FA, \alpha)_{\Omega \mathscr{C}_1}$ to $(A)_{\mathscr{C}} -
(A)_{\mathscr{C}} = 0$. Thus $d \circ d_1 = 0$. 

We\pageoriginale define now a homomorphism 
$$
d' : K_1 \mathscr{C}' \to K_0 \Phi F 
$$
by setting $d' = d_1 \circ \theta^{-1}$.

Clearly $d \circ d' = 0$. Suppose $(A, \alpha, B)_{\Phi F} \in \ker
d$. Then, by Proposition \ref{chap1:prop1.1}(b), there is an
isomorphism $\beta : A 
\perp C \to B \perp C$ for some $C \in  \obj \mathscr{C}$. We have
then a commutative diagram  
\[
\xymatrix@R=1.5cm@C=1.5cm{
FA \perp FC \ar[r]^{\alpha \perp 1_{FC}} \ar[d]_{F 1_{A\perp C}} & FB
\perp FC \ar[d]^{F \beta^{-1}}\\
FA \perp FC \ar[r]_{\alpha'} & FA \perp FC
}
\]
for a suitable $\alpha'$, showing that the triples $(A \perp C, \alpha
\perp 1_{FC}, B \perp C)$ and $(A \perp C, \alpha', A \perp C)$ are
isomorphic. Thus  
$$
(A, \alpha, B)_{\Phi  F} = (A \perp C, \alpha \perp 1_{FC}, B \perp
C)_{\Phi F} = (A \perp C, \alpha', A \perp C)_{\Phi F}. 
$$
The third member is the image of $(F(A \perp C), \alpha')_{\Omega
  \mathscr{C}_1}$ by $d_1$.  

Hence the sequence 
$$
K_1 \mathscr{C}' \xrightarrow{d'} K_0 \Phi F \xrightarrow{d} K_0
\mathscr{C} 
$$
is exact.

Next\pageoriginale we note that $d' \circ K_1 F = 0$. This follows from
the fact that 
$d' \circ K_1 F$ sends $(A, \alpha)_{\Omega \mathscr{C}}$ to $(A, F,
\alpha, A)_{\Phi F}$ and the triples $(A, F \alpha, A)$ and $(A,
1_{FA}, A)$ are isomorphic in view of the commutative diagram 
\[
\xymatrix{
FA \ar[r]^{F \alpha} \ar[d]_{F \alpha} & FA \ar[d]^{F1_A} \\
FA  \ar[r]_{1_{FA}} & FA
}
\]

\setcounter{theorem}{5}
\begin{theorem}\label{chap1:thm4.6}%theo 4.6
If $F : \mathscr{C} \to \mathscr{C}'$ is a cofinal functor of
categories with product, then the sequence  
$$
K_1 \mathscr{C} \xrightarrow{K_1 F} K_1 \mathscr{C}' \xrightarrow{d'}
K_0 \Phi F \xrightarrow{d} K_0 \mathscr{C} \xrightarrow{K_0 F} K_0
\mathscr{C}' 
$$
is exact.
\end{theorem}

We have only to show that $\ker d' \subset \im K_1 F$. For this we need
an effective criterion for recognizing the triples $(A, \alpha, B)$
with the property $(A, \alpha, B)_{{\Phi} F} = 0$. This is given in
Lemma \ref{chap1:lem4.7} below, for which we now prepare. 

In $\Omega \mathscr{C}'$ let $\mathscr{E}$ denote the smallest class of 
objects such that  
\begin{enumerate}
\renewcommand{\theenumi}{\roman{enumi}}
\renewcommand{\labelenumi}{(\theenumi)}
\item $\alpha \approx \beta$, $\alpha \in \mathscr{E} \Rightarrow
  \beta \in \mathscr{E}$ 

\item $\alpha$, $\beta \in \mathscr{E} \Rightarrow \alpha \perp \in
  \mathscr{E}$ 

\item $\alpha$, $\beta \in \mathscr{E}$, $\alpha \circ \beta$ defined
  $\Rightarrow \alpha \circ \beta \in \mathscr{E}$  

\item $(FA, 1_{FA})$, $(FA \perp FA, t) \in \mathscr{E}$ for all $A$,
  $t$ being the transposition.  

These\pageoriginale properties imply the following:

\item If $\alpha \in \mathscr{E}$, then $(\alpha)_{\Omega
  \mathscr{C}'} \in \im K_1 F \subset \ker d'$. 

We need only note in $(iv)$, that $''t = Ft''$, with the obvious abuse
of notation. 

\item $(FA \perp \cdots \perp FA, s) \in \mathscr{E}$ for any
  permutation s. 

Using $(i)$, $(ii)$, $(iii)$ and $(iv)$, this reduces easily to the 
fact that transpositions generate the symmetric group. 

\item If $\alpha : FA \to FB$ and $\beta : FB \to FC$, then
$$
(FA \perp FB, t) (\alpha \perp \alpha^{-1})) \in \mathscr{E}
$$
and 
$$
(FA \perp FB \perp FC, s(\alpha \perp \beta (\beta \alpha)^{-1})) \in
\mathscr{E}, 
$$
where $t$ and $S$ are the appropriate transposition and 3-cycle
respectively.  
\end{enumerate}

This statement follows from $(i)$, $(iv)$ and Lemma \ref{chap1:lem1.5}. 

Now in $\Phi F$ we call an object of the form $(A, \alpha, A)$ an
\textit{automorphism}. We call it  \textit{elementary} if $(FA,
\alpha) \in \mathscr{E}$. For any $\alpha = (A, \alpha, B)$, we write 
$$
\alpha \sim 1
$$
if $\alpha \perp 1_{FC} \approx \varepsilon$ for some $c \in
\mathscr{C}$ and some elementary automorphism $\varepsilon$. 

We also write 
$$
\alpha \sim \beta
$$
if and only if $\alpha \perp \beta^{-1} \sim 1$.

\setcounter{lemma}{6}
\begin{lemma}\label{chap1:lem4.7} %lemm 4.7
For\pageoriginale $\alpha, \beta \in \Phi F$, $(\alpha)_{\Phi F} =
(\beta)_{\Phi F} 
\Leftrightarrow \alpha \sim  \beta$. In particular, $(\alpha)_{\Phi F} = 0
\Leftrightarrow \propto \sim 1$. 
\end{lemma}

Before proving this lemma, let us use it to finish the 

\setcounter{proofofthm}{5}
\begin{proofofthm}%proop of the lemma 4.6
Given $(FA, \propto )$ such that $(\alpha)_{\Omega \mathscr{C}'} \in
\ker d'$ we have to show that $(\alpha)_{\Omega \mathscr{C}'} \in \im
K_1 F$. The hypothesis means that $(\alpha)_{\Phi F} = 0$, so that by
Lemma \ref{chap1:lem4.7}, there is a $c \in \obj \mathscr{C}$, an elementary
automorphism $\varepsilon = (E, \varepsilon, E)$, and an isomorphism
$(f, g) : \alpha \perp 1_{FC} \to \varepsilon$. This means that the
diagram  
\[
\xymatrix@R=1.7cm@C=2cm{
FA \perp FC \ar[r]^{\alpha \perp 1_{FC}} \ar[d]_{F\bar{f}} & FA \perp
FC \ar[d]^{Fg} \\
FE \ar[r]_{\varepsilon} & FE
}
\]
is commutative. Hence $\alpha \perp 1_{FC} = Fg^{-1} Ff(Ff)^{-1}
\varepsilon Ff = F(g^{-1 }f) \varepsilon'$, where $\varepsilon' =
(Ff)^{-1} \varepsilon Ff \approx \varepsilon $ in $\Omega
\mathscr{C}'$. By properties (i) and (v) above,
$(\varepsilon')_{\Omega \mathscr{C}'} \in \im K_1 F$, so we have
$(\alpha)_{\Omega \mathscr{C}'} = (\alpha \perp 1_{FC})_{\Omega
  \mathscr{C}'}= (F(g^{-1} f))_{\Omega \mathscr{C}'} +
(\varepsilon)_{\Omega \mathscr{C}'} \varepsilon \im K_1 F$, as
required. 
\end{proofofthm}

\setcounter{proofoflemma}{6}
\begin{proofoflemma}%proo  of lemma 4.7
 If $\alpha \sim \beta$, then $(\alpha)_{\Phi F} = (\beta)_{\Phi F}$
 by virtue of (v) above. For the converse, we will prove: 
\begin{enumerate}
\renewcommand{\theenumi}{\alph{enumi}}
\renewcommand{\labelenumi}{(\theenumi)}
\item $\sim$ is an equivalence relation 

\item $\perp$ induces a structure of abelian group on $M = \obj \Phi
  F/ \sim$. 

\item $\alpha \circ \beta \sim \alpha \perp \beta$ whenever $\alpha \circ
  \beta$ is defined. 
\end{enumerate}
\end{proofoflemma}

Once\pageoriginale shown, these facts imply that the map obj $\Phi F \to M$
satisfies the axioms for $K_0 \Phi F$, so it induces a homomorphism
$K_0 \Phi F \to M$, which is evidently surjective. Injectivity follows
from the first part of the proof above  
\begin{enumerate}[(1)]
\item If $\alpha$ and $\beta$ are elementary automorphisms, then so
  are $\alpha^{-1}$, $\alpha \perp \beta$, and $\alpha \circ \beta$ (if
  defined). 

This is obvious.

\item If $\beta \sim 1$ and $ \alpha \perp \beta \sim 1$, then $\alpha
  \sim 1$. 

For, by adding an identity to $\beta$, we can find elementary
automorphisms $\varepsilon_1 = (E_1, \varepsilon_1, E_1)$ and
$\varepsilon = (E, \varepsilon, E)$ and an isomorphism $(f, g) :
\alpha \perp \varepsilon_1 \to \varepsilon $. Thus  
\[
\xymatrix{
FA \perp FE_1 \ar[r]^{\alpha \perp \varepsilon_1} \ar[d]_{Ff} & FB
\perp FE_1 \ar[d]^{Fg}\\
FE \ar[r]_{\varepsilon} & FE
}
\]
commutes. Set $\varepsilon_2 = (A \perp E_1, 1_{FA} \perp
\varepsilon^{-1}_1, A \perp E_1)$; $\varepsilon_2 = 1_{FA} \perp
\varepsilon^{-1}_1$ is clearly elementary. Set $\varepsilon'_2 = (E,
Ff \varepsilon_2 Ff^{-1}, E)$. Since $Ff : (FA \perp FE_1,
\varepsilon_2) \to (FE, \varepsilon'_2)$ in $\Omega \mathscr{C}'$,
$\varepsilon'_2$ is also an elementary automorphism. Moreover, we have   
$$
(f, g) : (\alpha \perp \varepsilon_1) \circ \varepsilon_2 \to \varepsilon
\circ \varepsilon'_2, 
$$
clearly, and $(\alpha \perp \varepsilon_1) \circ \varepsilon_2 = \alpha
\perp 1_{FE_1}$. Since $\varepsilon \circ \varepsilon'_2$ is elementary,
we have shown $\alpha \sim 1$, as claimed. 

If\pageoriginale $\alpha = (A, \alpha, B)$ and $\beta = (B, \beta, C)$, then  

\setcounter{enumi}{3}
\item $\alpha \perp \alpha^{-1} \sim 1$,

and 

\item $\alpha \perp \beta \perp (\beta \alpha)^{-1} \sim 1$.
\end{enumerate}

For,
$$
(1_{A \perp B}, t) : \alpha \perp \alpha^{-1} \to (A \perp B, t(\alpha
\perp \alpha^{-1}), A \perp B) 
$$
and 
\begin{gather*}
(1_{A \perp B \perp C},s) : \alpha \perp \beta \perp (\beta
\alpha)^{-1}\\ 
\to (A \perp B \perp C, s(\alpha \perp \beta \perp (\beta
\alpha)^{-1}, A \perp B \perp C), 
\end{gather*}
and the latter are elementary by property (vii) of $\mathscr{E}$.

Now, for the proof of (a), we note that (4) $\Rightarrow$ reflexivity,
$(1) \Rightarrow$ symmetry, and (3) plus (4) $\Rightarrow$
transitivity. The statements $(b)$ and $(c)$ follow respectively from
(1) and (5). 


\section{The category \underline{\underline{$P$}}}\label{chap1:sec5}%sec 5

Let $k$ be a commutative ring. We define $\underset{=}{P}(k)$ (or
$\underset{=}{P}$) to be the category of finitely generated projective
k-modules and their isomorphisms, with product $\oplus$. 

The groups $K_i \underset{=}{P}$ are denoted in $[ K, \S 12 ]$ by
$K_i(k)$. Strictly speaking, the definitions do not coincide since the
$K_i(k)$ are defined in terms of exact sequences, and not just
$\oplus$. Of course this makes no difference for $K_0$ since all
sequences split. For $K_1$, however, the exact sequences of
automorphisms $0 \to \alpha' \to \alpha \to \alpha'' \to 0$ need not
split. In terms of matrices this means that $\alpha$ has the form  
$$
\alpha =
\begin{pmatrix}
\alpha' & \beta \\
0 & \alpha''
\end{pmatrix}
$$
It\pageoriginale is clear that $\alpha$ can be written in the form $\alpha =
(\alpha' \oplus \alpha'') \varepsilon'$, where $\varepsilon$ is of the
form $\left(\begin{smallmatrix}id & \gamma \\0 &
  id \end{smallmatrix}\right)$, and the equivalence of the two
definitions results from the fact that $(\varepsilon)_{\Omega
  \underset{=}{P}} = 0$ in $K_1 \underset{=}{P}$. The last fact is seen
by adding a suitable identity automorphism to $\varepsilon$ to put it
in $GL(n, k)$, for some $n$, and then writing the result as a product
of elementary matrices (see (5.3) below). 

We summarize now some results from $[K]$.

The tensor product $\oplus_k$ is additive with respect to $\oplus$ so
that it induces on $K_0 \underset{=}{P}$ a structure of commutative
ring. 

If $\mathscr{Y} \in  \spec (k)$ and $P \in \underset{=}{P}$, then
$P_{\mathscr{Y}}$ is a free $k_{\mathscr{Y}}$- module and its rank is
denoted by $rk_p (\mathscr{Y})$. The map 
$$
rk_p : \text{ spec } (k) \to \mathbb{Z},
$$
given by $\mathscr{Y} \to rk_p (\mathscr{Y})$, is continuous, and is
called the \textit{rank} of $P$. Since $rk_{P \oplus Q} = rk_P + rk_Q$
and $rk_{P \oplus Q} = rk_P rk_Q$, we have a \textit{rank
  homomorphism} 
$$
rk : K_0 \underset{=}{P} \to C,
$$
where $C$ is the ring of continuous functions spec $(k) \to \mathbb{Z}$.
\begin{itemize}
\item[(5. 1)] The rank homomorphism $rk$ is split by a ring
  homomorphism $C \to K_0 \underset{=}{P}$, so that we can write  
$$
K_0 \underset{=}{P} \approx \oplus \tilde{K}_0 \underset{=}{P},
$$
where $\tilde{K}_0 \underset{=}{P} = \ker (rk)$. $\tilde{K}_0
\underset{=}{P}$ is a nil ideal. 

This\pageoriginale result is contained in [$K$, Proposition 15.4].

\item[(5.2)] \textit{Suppose $\max (k)$ the space of maximal ideals of
  $k$, is a noetherian space of dimension $d$. Then}  
\begin{enumerate}[(a)]
\item If $x \in K_0 \underset{=}{P}$ and $rk (x) \geq d$, then $x =
  (P)_{\underset{=}{P}}$ for some $P \in \underset{=}{P}$. 

\item If $rk (P)_{\underset{=}{P}} > d$ and if $(P)_{\underset{=}{P}}
  = (Q)_{\underset{=}{P}}$, then $P \approx Q$.  

\item $(\tilde{K}_0\underset{=}{P})^{d + 1} =0$.
\end{enumerate}

Since $k$ is a basic object for $\underset{=}{P}$ in the sense of \S
\ref{chap1:sec3}, we deduce immediately from Theorem
\ref{chap1:thm3.2} and [$K$, Theorem \ref{chap1:thm3.1} and
  Proposition 12.1], that   

\item[(5.3)] There is a natural isomorphism 
$$
K_1 \underset{=}{P} \approx GL(k) / [GL(K), GL(k)], 
$$
where $GL(k) = \lim \limits_{\to} GL(n,k) (= Aut K^n)$ with respect
to the maps $\alpha \mapsto \begin{pmatrix} \alpha & 0 \\ 0 &
  I_m \end{pmatrix}$ from $GL(n, k)$ to $GL(n+m , k) \cdot [GL(k) , GL(k)]
= E(k)$, the group generated by all elementary matrices in $GL(k)$, we
have also $E(k) = [E(k), E(k)]$. The determinant map det : $GL(K) \to
U(k)$ is split by $U(k) \to GL(k)$ (defined via $GL(1, K)$). Thus we
have a natural decomposition 
$$
K_1 \underset{=}{P} \approx U(k) \oplus SK_1 \underset{=}{P}, 
$$
where $SK_1 \underset{=}{P}\approx SL(k) / E(k) = SL(k)/[SL(k), SL(k)]$.

We\pageoriginale have also the following interesting consequence of
Theorem \ref{chap1:thm3.2}.  

\item[(5.4)] If $\alpha \in [GL(n, k)$, $GL(n, k)]$, then for some $m$
  and some $\gamma \in GL\break (m, k)$,
 $\left(\begin{smallmatrix} 
\alpha &  0 & 0 \\ 
0 & \gamma & 0 \\ 
0 & 0& \gamma^{-1}
 \end{smallmatrix}
  \right)$ is a commutator in $GL(n + 2m, k)$ and
  $\left(\begin{smallmatrix} \alpha^2 & 0 \\ 0 &
    I_{2m} \end{smallmatrix} \right)$ is a product of two
  commutators. 
\end{itemize}


\section{The category \underline{\underline{$FP$}}}\label{chap1:sec6}%%% 6

Let $k$ be a commutative ring.  

\begin{prop}\label{chap1:prop6.1}%% 6.1
 The following conditions on a $k$-module $P$ are equivalent:  
\begin{enumerate}[(a)]
\item $P$ is finitely generated, projective, and has zero
  annihilator. 

\item $P$ is finitely generated, projective, and has every where
  positive rank (that is $P_{\mathscr{Y}} \neq 0$ for all $\mathscr{Y}
  \in  \spec (k))$.  

\item There exists a module $Q$ and an $n > 0$ such that $P \otimes_k
  Q \approx k^n$. 
\end{enumerate}
\end{prop}

\begin{proof}
The equivalence $(a) \Leftrightarrow (b)$ is well known.
\end{proof}

$(b) \Rightarrow (c)$. The module $P$ is ''defined over'' a finitely
generated subring $k_0$ of $k$. By this we mean that there exists a
finitely generated projective $k_0$-module $P_0$ such that $P \approx
k \otimes_{k_0} P_0$. To see this, we express $P$ as the cokernel of an
idempotent endo-morphism of a free $k$-module $k^n$. Let $\alpha$ be
the matrix of this endomorphism with respect to the canonical basis of
$k^n$. We take for $k_0$, 
the\pageoriginale subring of $k$ generated by the entries of
$\alpha$. It is easily 
seen that $P_0$ can be takes to be the cokernel of the endomorphisms of
$k^n_0$ determined by $\alpha$.  

  So we can assume that $k$ is noetherian with $\dim \max (k)  = d <
\infty$. Let  $x = (P)_{\underline{\underline{P}}} \in K_0 \underset{=}{P}$. Then
$rk(x)$ is a positive continuous functions spec $(k) \to \mathbb{Z}$,
and it takes only finitely many values, since spec $(k)$ is quasi -
compact. Hence we can find $y \in C$ (in the notation of (5.1)) such
that $rk(x)y = m > 0$ (the constant function $m$). Now $x = rk (x) -
z'$ with $z'$ nilpotent, so that $xy = m - z$ with $z = y z' $ 
nilpotent. It follows that $n = m^h = wxy $ for some $h > 0$ and
$\omega \in K_0 \underset{=}{P}$\,; for instance we can take $h \geq d +
1$, in view of (5.2) $(c)$. By enlarging $h$ we can make $rk (wy)  >
d$. Then we have $wy = (Q)_{\underset{=}{P}}$ for some $Q$ by (5.2)
$(a)$, it follows that $P \otimes _k Q \approx k^n$.  

$(c ) \Rightarrow (a)$. Assume $P \otimes_k Q\approx k^n$. There is a
finite set of elements $x_1, \ldots, x_p \in P$ such tht $P \otimes_k
Q  = \sum\limits^p_{ i = 1}  x_i \otimes Q$. We have then a
homomorphisms $f : F \to P, F$ a free $k$-module of finite rank, such
that $f \otimes 1_Q : F \otimes _K Q \to P \otimes_k Q$ is surjective
and therefore splits. Hence $f \otimes 1_{ Q \otimes P} :F \otimes Q
\otimes P \to P \otimes Q \otimes P$ is surjective and splits. Thus $P
\otimes k^n (\approx P \otimes Q \otimes P)$, being a direct summand
of $F \otimes k^n (\approx F \otimes Q \otimes P)$, is finitely generated and
projective. It follows that $P$ is finitely generated and projective.   

That $P$ has zero annihilator is clear. 

\begin{remark*}%rema 0
The\pageoriginale argument in $(b) \Rightarrow (c)$ can be used to show, more
precisely, that if $P$ is a finitely generated projective
  $k-$module of constant rank $r > 0$, then $ P \otimes_k Q \approx
  k^{r^{ d +1}}$ for some projective $k$-module $Q$ and some $d \geq
  0$. If $\max(k) $ is a noeterian space of finite dimension, then
  this number can be chosen for $d$.  
\end{remark*} 

Modules satisfying (a), (b) and (c) above will be called
\textit{faithfully projective}. They are stable under $\otimes (= 
\otimes_k)$. The faithfully projective modules together with their
isomorphisms form a category  
$$
\underline{\underline{FP}} (k) \;\; (\text{or } \underline{\underline{FP}}) 
$$
with product $\otimes$, in the sense of \S \ref{chap1}. Condition (c) in
the proposition above shows that \textit{the free modules are cofinal
  in $\underline{\underline{FP}}$}. We propose now to calculate the groups
$K_i \underline{\underline{FP}}$.  

We write 
$$
Q \otimes_z K_0 \underline{\underline{P}} = (Q \otimes_{\mathbb{Z}}
C)  
\oplus (Q \otimes_{\mathbb{Z}}\tilde{K}_0 \underline{\underline{P}}) 
$$
in the notation of (5.1). Thus $\mathbb{Q} \otimes_{\mathbb{Z}} C$ is the
ring continuous functions from $\spec (k)$ (discrete)
$\mathbb{Q}$. Let $U^+ (\mathbb{Q} \otimes_{\mathbb{Z}} K_0
  \underline{\underline{P}})$ denote the unit whose 
``rank'' (= projection on $\mathbb{Q} \otimes C$) is a positive
  function.   

\setcounter{theorem}{1}
\begin{theorem}%theo 6.2
$K_0 \underset{=}{FP} \approx U^+ (\mathbb{Q} \otimes_{\mathbb{Z}} K_0
  \underset{=}{P})$\pageoriginale 
$$
\approx U^+ (\mathbb{Q} \otimes_{\mathbb{Z}} C ) \oplus (\mathbb{Q}
\otimes_{\mathbb{Z}}\tilde{K_0}\underset{=}{P}).  
$$
\end{theorem}

\begin{example*}%exam 0
Suppose $\spec (k)$ is connected, so that $C = \mathbb{Z}$. The  
$$
K_0 \underline{\underline{FP}} \approx \text{ (positive rationals) }
\oplus (\mathbb{Q} \otimes_{\mathbb{Z}} \tilde{K_0} \underset{=}{P}),  
$$
the direct sum of free abelian group and a vector space over
$\mathbb{Q}$.  
\end{example*}

\begin{proof}
If $P$ is faithfully projective, then $P \otimes Q \approx k^n$ for
some $n > 0$, so that $(P)_{\underset{}{P}} (Q){\underset{=}{P}} = n$ in
$K_0{\underset{=}{P}}$. It follows that $1 \otimes (P)_{
  \underset{=}{P}} \in U^+ (\mathbb{Q} \otimes_{\mathbb{Z}} K_0
\underset{=}{P})$, and this homomorphism, being multiplicative with
respect to $\otimes$, defines a homomorphism  
\begin{equation*}
K_0 \underline{\underline{FP}} \to U^+ (\mathbb{Q} \otimes _{\mathbb{Z} } K_0
\underset{=}{P}). \tag{6.3} 
\end{equation*}
\end{proof}

We first show that this map is surjective. Any element of the right
hand side can be written as $\dfrac{1}{n} \otimes x$, $x \in  K_0
\underset{=}{P}$, and $rk(x)$ is a positive function of $\spec \; (k)$
into $\mathbb{Z}$. Since $x$ is defined over a finitely generated
subring of $k$, we can assume without loss of generality, that $k$ is
finitely generated with $\max(k)$ of dimension $d$, say. By increasing
$n$ by a multiple we can make $rk(x)$ exceed $d$, so that $x = (P)
_{\underset{=}{P}}$ so some $P \in \underset{=}{P}$ by (5.2)(a). Clearly $P \in
\underline{\underline{FP}}$. Thus $\dfrac{1}{n} \otimes x = ( 1 \otimes (k^n)
_{\underset{=}{P}})^{-1} (1 \otimes (P)_{\underset{=}{P}})$ is in the
image of (6.3).  


Next\pageoriginale we prove the injectivity of (6.3). Suppose $1 \otimes
(P)_{\underset{=}{P}} = 1 \otimes (Q)_{\underset{=}{P}}$. Then, for
some integer $n > 0$, $n((P)_{\underset{=}{P} } - (Q)_{\underset{=}{P}})
= 0$, so that $(k^n \otimes_k P)_{\underset{=}P} = (k^n \otimes_k
Q)_{\underset{=}{P}}$. By choosing $n$ large we can make rank
$(k^n \otimes _k P)$ large and then invoke (5.2)(b) to obtain $k^n
\otimes P \approx k^n \otimes Q$. Hence $(P)_{\underline{\underline{FP}}} =
(Q)_{\underline{\underline{FP}}}$. This establishes the first
isomorphism in the theorem.   

To prove the second isomorphism, we note that 
$$ 
U^+ (\mathbb{Q} \otimes_{\mathbb{Z}} K_0 \underset{=}{P}) = U^+
(\mathbb{Q} \otimes_{\mathbb{Z}} C) \times (1 + (\mathbb{Q} 
\otimes_{\mathbb{Z}} \tilde{K}_0 \underset{=}{P})),  
$$
and, since $\mathbb{Q} \otimes_{\mathbb{Z}}\tilde{K}_0 P$ is a nil algebra
over $\mathbb{Q}$, we have an isomorphism 
$$
\exp : \mathbb{Q} \otimes_{\mathbb{Z}} \tilde{K}_0 P \to 1 +
(\mathbb{Q} \otimes_{\mathbb{Z}} \tilde{K}_0 P).   
$$
In order to compute $K_1 \underline{\underline{FP}}$, we prove a
general lemma  about direct limits. Let  
$$
L = (W_n, f_{n , nm} : W_n \to W_{nm} )_{n, m, \in \mathbb{N}}   
$$ 
be a direct system of abelian groups, indexed by the positive integers,
ordered by divisibility. We introduce an associated direct system  
$$
L' = (W_n, f'_{ n , \; nm} : W_n \to W_{nm}), 
$$
where $f'_{n, nm} = mf_{n, \; nm}$, and a homomorphism 
$$
(n. 1 _{W_n}) : L \to L' 
$$
of\pageoriginale direct systems. For the latter we note that
\[
\xymatrix@R=1.5cm@C=1.5cm{
W_n \ar[r]^{n\cdot 1_{W_n}} \ar[d]_{f_{n,nm}} & W_n \ar[d]^{mf_{n,nm}
  = f^1_{n,nm}}\\
W_{nm} \ar[r]_{nm1_{W_{n,m}}} & W_{nm}
}
\]
is commutative. $L'$ is a functor of $L$. We have an exact sequence of
direct systems 
\begin{equation*}
L \to L' \to L'' \to 0, \tag{6.4}
\end{equation*}
where $L'' = (W_n / W_{nm}, f''_{n,nm})$ is the cokernel of $L \to
L'$.

\setcounter{lemma}{4}
\begin{lemma}\label{chap1:lem6.5} %%% 6.5
With the notation introduced above, the exact sequences
$$
\varinjlim  L \to \varinjlim L' \to \varinjlim L'' \to 0   
$$
and
$$
\varinjlim L \otimes (\mathbb{Z} \to \mathcal{Q} \to \mathcal{Q} /
\mathbb{Z} \to 0)  
$$
are isomorphic. (here $\otimes = \otimes_{\mathbb{Z}}$.). 
\end{lemma}

\begin{proof}
Let $E= ( \mathbb{Z}_n , e_{n, nm})$ with $\mathbb{Z}_n = \mathbb{Z}$
and $e_{n, \; nm} = 1_{\mathbb{Z}}$ for all $n$, $m \in
\mathbb{N}$. Evidently the exact sequence of direct systems 
$$
L \to L' \to L'' \to 0 
$$ 
and 
$$
L \otimes (E \to E' \to E'' \to 0) 
$$
are\pageoriginale isomorphic. The lemma now follows from the fact that 
$\lim\limits_{\to} E = \mathbb{Z}$, $\lim\limits_{\to}E' =
\mathbb{Q}$, and standard properties of direct limit. 
\end{proof}

Theorem \ref{chap1:thm3.1} allows us to compute $K_1
\underline{\underline{FP}}$ using only the free modules. Let  
$$
W_n = GL (n,k) / [GL(n,k), GL(n,k)]
$$  
and let $f_{n,nm}$ and $g_{n,nm}$ be the homomorphisms $W_n \to
W_{nm}$, induced respectively by $\alpha \mapsto
\left(\begin{smallmatrix} \alpha_{I_n} && 0 \\ & \ddots \\ 0 &&
  In  \end{smallmatrix}\right)$ and $\alpha \to
\left(\begin{smallmatrix} \alpha & &0 \\ & \ddots  \\ 0 &
  &\alpha \end{smallmatrix}\right)$ from $GL(n,k)$ to $GL(nm,
k)$. Then it follows from theorem \ref{chap1:thm3.1} that  
$$
K_1 \underline{\underline{P}} = \varinjlim (W_n, f_{n,nm})
$$
and 
$$
K_1 \underline{\underline{FP}}= \varinjlim (W_n, g_{n,nm}).
$$

\begin{lemma}\label{chap1:lem6.6}% \lem 6.6
If $\alpha \in GL(n,k)$ and if $nm \ge 3$, then  
$$
\begin{pmatrix}
 \alpha^m & & &  0 \\
 & I_n & &  \\ 
& & \ddots & \\
0 & &  &  I_n  
\end{pmatrix} 
\equiv 
\begin{pmatrix} 
\alpha & & & 0 \\ 
 & \alpha && \\ 
& & \ddots & \\
0 & & & \alpha  
\end{pmatrix} 
\mod [GL(n,k), GL(n,k)]. 
$$
(See [$K$, Lemma 1.7]).
\end{lemma}

It follows from lemma \ref{chap1:lem6.6}, that $g_{n, nm}= f'_{n, nm}=
mf_{n, nm}$, 
and hence, using lemma \ref{chap1:lem6.5}, we have the following 

\setcounter{theorem}{6}
\begin{theorem}% \thm 6.7
$K_1 \underline{\underline{FP}} \approx \mathbb{Q} \otimes_{\mathbb{Z}} K_1
  \underset{=}{P}$\pageoriginale  
$$
\approx(\mathbb{Q} \otimes_{\mathbb{Z}} U(k)) \oplus (\mathbb{Q}
\otimes_{\mathbb{Z}} SK_1 \underset{=}{P}). 
$$ 
\end{theorem}

If we pass to the limit before abelianizing, we obtain the groups
$$
GL_{\otimes}(k) = \varinjlim (GL(n,k), \alpha \mapsto \alpha \otimes
I_m )_{n,m \in \mathbb{N}}, 
$$
which consists of matrices of the type 
$$
\begin{pmatrix}
\alpha_{ \ddots} && 0 \\ 
& \alpha\\ 
0 && \ddots
\end{pmatrix}
$$
where $\alpha$ is in $GL(n,k)$ for some $n$.  The centre of this group
consists of scalar matrices (the case $n=1$) and is isomorphic to
$U(k)$. We write 
$$
PGL(k) = GL_{\otimes}(k)/ \text{ centre } = GL_{\otimes}(k) / U(k). 
$$
Now
$$
K_1 \underline{\underline{FP}} = GL_\otimes (k) / [GL_\otimes (k),
  GL_\otimes (k)] = (\mathbb{Q} \otimes_{\mathbb{Z}} U(k)) \oplus
(\mathbb{Q} \otimes_{\mathbb{Z}} SK_1 \underset{=}{P})  
$$
and we have projective on the first summand
$$
{\det}' : K_1 \underline{\underline{FP}} \to \mathbb{Q}
\otimes_{\mathbb{Z}} U(k),   
$$
which is induced by the determinant. Explicitly, if $\alpha \in GL
(n,k)$, then 
$$
{\det}' \begin{pmatrix}
\alpha_{ \ddots} && 0 \\ & \alpha\\ 0 && \ddots
\end{pmatrix} = \frac{1}{n} \otimes \det \alpha.
$$

This\pageoriginale evaluates ${\det}'$, in particular, on elements of
the centre (the case. $n=1$); so we see easily that: 
\begin{equation*}
 \text{ coker } (U(k) \to K_1 \underline{\underline{FP}}) \tag{6.8}\label{eq6.8}
\end{equation*}
\begin{align*}
&=  (\mathbb{Q} / \mathbb{Z} \otimes_{\mathbb{Z}} U(k)) \oplus (\mathbb{Q}
  \otimes_{\mathbb{Z}} SK_1 \underset{=}{P})\\ 
&=  PGL(k) / [PGL(k), PGL (k)] \\
&= \varinjlim  PGL (n,k) / [PGL(n,k), PGL (n,k)],
\end{align*}
\textit{where the maps are induced by the homomorphisms $\alpha
\mapsto \alpha \otimes I_m$ from $GL(n,k)$ to
$GL(nm,k)$.} 


\section{The category
  \underline{\underline{$\Pic$}}}\label{chap1:sec7}%%% sec 7 

$\underline{\underline{\Pic}} (k)$ (or $\underline{\underline{\Pic}}$)
is the full subcategory of $\underline{\underline{FP}}$ consisting of projective
$k$-modules of rank one, with $\otimes_k$ as product. We shall denote
$K_{0} \underline{\underline{\Pic}}$ by Pic $(k)$.   

A module $P$ in $\underline{\underline{\Pic}}$ satisfies
$$
P \otimes_k P^{*} \approx k,
$$ 
where $P^*= \Hom_k (P,k)$. So any object of
$\underline{\underline{\Pic}}$, in particular $k$, is cofinal. Theorem
\ref{chap1:thm3.1} then shows that  
$$
K_1 \underline{\underline{\Pic}} \approx \text{ Aut } (k) \approx U(k).
$$

The inclusion $\underline{\underline{\Pic}} \subset
\underline{\underline{FP}}$ induces 
homomorphisms  
\begin{equation*}
\Pic (k) \to K_0 \underline{\underline{FP}} \tag*{$(7.1)_0$}\label{eq7.1_0}
\end{equation*}
and\pageoriginale 
\begin{equation*}
U(k) \to K_1 \underline{\underline{FP}}. \tag*{$(7.1)_1$}\label{eq7.1_1}
\end{equation*}

The latter is induced  by $U(k)= GL(1,k) \subset GL_\otimes (k)$,
which identifies $U(k)$ with the centre of $GL_\otimes (k)$. So the
co-kernel is $PGL(k)$. Thus, we have from (\ref{eq6.8}), 
\begin{align*}
\text{coker } (7.1)_1  &\approx ( Q / \mathbb{Z} \otimes_\mathbb{Z} U(k))
\oplus (\mathbb{Q} \otimes_\mathbb{Z} SK_1 \underset{=}{P})
\tag{7.2}\label{eq7.2}\\   
&\approx PGL(k) / [PGL(k), PGL(k)],
  \end{align*}  
  and 
  \begin{align*}
\ker (7.1)_1 = & \text{ the torsion subgroup of } U(k) \\
& \text{(that is, the roots of unity in $k$)}.
  \end{align*}  
  
  The last assertion follows from the fact that \ref{eq7.1_1} is the natural
  map $U(k) \to \mathbb{Q} \otimes_\mathbb{Z} U(k)$ followed  by the
  inclusion   of the latter into $K_1 \underline{\underline{FP}} = (\mathbb{Q}
  \otimes_\mathbb{Z} U(k))   \oplus (\mathbb{Q} \otimes_\mathbb{Z} SK_1
  \underline{\underline{P}})$.   

(7.3) \textit{The kernel of the natural map \ref{eq7.1_0}; $\Pic (k) \to K_0
\underline{\underline{FP}}$ is the torsion subgroup of $\Pic (k)$.}

\begin{proof}
If $(L)_{\underline{\underline{\Pic}}} \in \ker$ \ref{eq7.1_0},
  then $L \otimes_k P 
\approx k \otimes_k P \approx  P$ for some $P \in
\underline{\underline{FP}}$. By 
$(6.1)(c)$, we can choose $P$ to be $k^n$, in which case we have $L
\otimes \cdots \otimes L\approx k^n$. Taking $n^{\rm th}$ exterior powers,
we get $L \otimes \cdots \otimes L\approx k$, so that
$(L)_{\underline{\underline{\Pic}}}$ is a torsion element in $\Pic (k)$. 
  \end{proof}  
  
  Conversely, suppose $(L)_{\underline{\underline{\Pic}}}$ has  order $n$, that is
  that $L \otimes \cdots \otimes L \approx k$. We have to show that
  $(L)_{\underline{\underline{FP}}} =0$. This amount\pageoriginale to showing that
  $L \otimes P   \approx P$ for some $P$ in
  $\underline{\underline{FP}}$. It is immediate that we 
  can take for $P$, the module  $k \oplus L \oplus L^{\otimes 2}
  \oplus \cdots \oplus L^{\otimes (n-1)}$, where $L^{\otimes i}$
  denotes the $i$-fold tensor product of $L$ with itself. 
  

