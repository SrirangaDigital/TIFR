\chapter{The Brauer group of a commutative ring}\label{chap3} %cha III

In\pageoriginale this chapter we prove the fundamental theorem on
Azumaya algebras, 
following largely the paper of Auslander Goldman \cite{key1}. In \S
\ref{chap3:sec4} we 
obtain Rosenberg and Zelinsky's	generalization of the Skolem-Noether
theorem (see \cite{key1}). Finally we introduce the Brauer group $Br (k)$ of
a commutative ring $k$. The functor End $: \underline{\underline{FP}}
\rightarrow \underline{\underline{Az}}$ is cofinal, in the sense of
chapter \ref{chap1}, and we obtain an exact sequence 
$$
K_1\underline{\underline{FP}} \to K_1\underline{\underline{Az}} \to 
K_o \Phi \text{ End } \to K_0 \underline{\underline{FP}} \to K_0
\underline{\underline{Az}} \to Br (k) \to 0. 
$$
We have computed the groups $K_i \underline{\underline{FP}}$ in
chapter \ref{chap1}, and we further show here that $K_0 \Phi$ End $\approx \Pic
(k)$. The final result is that the functors
$\underline{\underline{\Pic}} \to \underline{\underline{FP}} \to
\underline{\underline{Az}} $ yield an exact sequence 
$$
U(k) \to K_1 \underline{\underline{FP}} \to K_1
\underline{\underline{Az}} \to \Pic (k) \to K_0
\underline{\underline{FP}} \to K_0 \underline{\underline{Az}} \to Br
(k) \to 0, 
$$
from which we can extract a short exact sequence 
$$
0 \to (\mathbb{Q}/ \mathbb{Z} \otimes_{\mathbb{Z}} U (k)) \oplus
(\mathbb{Q} \otimes_{\mathbb{Z}} sK_1P) \to K_1
\underline{\underline{Az}} \to^t \Pic (k) \to 0, 
$$
the last group being the torsion subgroup of $\Pic (k)$. This gives a
fairly effective calculation of $K_1 \underline{\underline{Az}}$. 


\section{Separable Algebras}\label{chap3:sec1}%sec 1

Let $k$ be a commutative ring. If $A$ is a $k-$algebra, we write
$A^0$ for the opposite algebra of $A$, and $A^e = A \otimes_k A^0$. A
two-sided\pageoriginale $A-$ module $M$ can be viewed as a left $A
\otimes_k A^0$ module: We define the scalar multiplication by  
$$
(a \otimes b)x = axb, \quad x \in M, a,b \in A.
$$
In particular, $A$ is a left $A^e-$ module, in a natural manner, and
we have an exact sequence 
\begin{equation*}
0 \to J \to A^e \to A \to 0 \tag{1.1}
\end{equation*}
of $A^e-$ linear maps (where $a \otimes b \in A^e$ goes to $ab \in
A$). If needed, we shall make the notation more explicit by writing 
$$
A^e = (A/k)^e,
$$
and  
$$
J = J (A) = J (A/k).
$$
We define $k-$linear map
$$
\delta : A \to J, 
$$
by setting $\delta(a) = a \otimes 1 - 1 \otimes a$. 

\setcounter{lemma}{1}
\begin{lemma}%lem 1.2
Im $\delta$ generates $J$ as a left ideal, and $\delta$ satisfies
  $\delta(ab) = a (\delta b) + (\delta a)b$. 
\end{lemma}

\begin{proof}
Clearly $\im \delta \subset J$. If $x = \sum a_i \otimes b_i \in J$,
that is, if $\sum a_i b_i = 0$, then $x= \sum a_i \otimes b_i - \sum
a_i b_i \otimes 1 = \sum (a_i \otimes 1) ((1 \otimes b_i) - (b_i
\otimes 1)) = - \sum a_i \delta b_i$. Finally,\pageoriginale $\delta (ab) = ab
\otimes 1-1 \otimes ab = (a \otimes 1) (b \otimes 1-1 \otimes b) + (a
\otimes 1) (1 \otimes b) - (1 \otimes b) (1\otimes a) = a (\delta b) +
(1 \otimes b) \delta a = a(\delta b) + (\delta a)b$. 
\end{proof}

\setcounter{coro}{2}
\begin{coro}\label{chap3:coro1.3} %coro 1.3.
If $M$ is a left $A^e-$ module and $N$ is a right $A^e -$module,
there are natural isomorphisms 
$$
\Hom_A e(A, M) \approx \{ x \in M \big| ax = xa, \text{ for all } a \in A \},
$$
 and 
$$
N \otimes_{A^e} A \approx N/ \text{ (Submodule generated by } ax - xa,
a \in A, x \in N). 
$$
\end{coro}

\begin{proof}
Since $A \approx A^e /J$, $\Hom_{A^e} (A, M) \approx \{ x \in M \big|
Jx = 0 \} = \{ x \in M \big| (\delta a)x = 0 \forall a \in A \} = \{ x
\in M \big| ax = xa \forall a \in A \}$. The other part is trivial. 
\end{proof}

For a two-sided $A-$module $M$ we shall denote the subgroup $\{x \in
M \big| ax = xa \forall a \in A \}$ by $M^A$. Note that if $A$ is a
subalgebra of a $k-$algebra $B$, then $B^A$ is just the centralizer of
$A$ in $B$. In particular, $A^A = $ centre $A$. 

We denote by Der$_k(A, M)$ the $k-$module of all $k-$derivations of
$A$ into $M$, that is, $k-$linear maps $d : A \to M$ satisfying $d(ab)
= ad(b) + (da)b$, $a$, $b \in A$. If $f: M \to N$ is $A^e-$linear,
then $d \mapsto f d$ defines a $k-$linear map Der$_k (A, M) \to Der_k
(A,N).$. For example, if $x \in M$ and if $f : A^e \to M$ is
defined by $f(1) = x$, then the composite 
$$
A \xrightarrow{\delta} J \hookrightarrow A^e \xrightarrow{f}M 
$$
is a derivation, called the \textit{inner derivation} $d_x$ defined
by $x$. Thus, if $a \in A$, $d_x (a) = (\delta a) x = ax - xa$. 

\setcounter{prop}{3}
\begin{prop}\label{chap3:prop1.4}% prop 1.4
 For\pageoriginale an $A^e$-module $M$, the map $f \mapsto f \delta$
 defines an isomorphism 
$$
\Hom_{A^e} (J,M) \to {\rm Der}_k(A, M),
$$
 with inner derivations corresponding to those $f$ which can be
 extended to $A^e$. 
\end{prop}

\begin{proof}
Since im $\delta$ generated $J$, we have $f \delta = 0 \Rightarrow f =
0$. 
\end{proof}

Suppose $d \in {\rm Der}_k (A, M)$. We can define a $k-$linear map $f : A^e
\to M$ by setting $f (\sum a_i \otimes b_i) = - \sum a_i d(b_i)$. This
satisfies $f \delta a = f(a \otimes 1- 1 \otimes a) = - ad (1) +
1d(a)$ for all $a \in A$. But $d(1) = d(1^2) = 1d(1) + d(1)1 = 2d(1)$
so that $d(1) = 0$. Thus $f \delta = d$. It remains to show that $f/J$
is $A^e-$linear. If $x = \sum a_i \otimes b_i \in J$, we must show
that $f((a \otimes b)x) = (a \otimes b )f(x)$. But $f((a \otimes b)x)
= f(\sum aa_i \otimes b_i b) = - \sum aa_i d(b_i b)= - \sum aa_i (b_i
db + d(b_i)b) = (a \otimes b)f(x)$. 

The derived functors of $M \mapsto M^A$ are called the
\textit{Hochschild cohomology groups} of A with coefficients in
$M$. We denote them by $H^i(A, M)$. By virtue of (1.3), $H^i (A, M)
\approx \Ext^i_{A^e} (A, M)$. The exact sequence (1.1) gives us an
exact sequence 
\begin{gather*}
0 \to \Hom_{A^e} (A, M) \to \Hom_{A^e} (A^e, M)\\ 
\to \Hom_{A^e} (J, M) \to
\Ext^1_{A^e} (A, M) \to 0, 
\end{gather*}
which we can rewrite, using (1.3) and (1.4), to obtain:

\begin{prop}% props 1.5
There is an exact sequence 
$$
0 \to M^A \to M \to Der_k (A, M) \to H^1 (A, M) \to 0. 
$$
so that\pageoriginale $H^1 (A,M) = $ the $k-$module of $k-$derivations
of $A$ into $M$, modulo the $k-$submodule of inner derivations.  
\end{prop}

If $C = A^A = $ centre $A$, then $C \otimes 1 \subset $centre $A^e$, 
so we can view the above exact sequence as a sequence of $C-$modules 
and $C- $linear maps. 

\setcounter{propanddef}{5}
\begin{propanddef}% pro and def 1.6 
A $k-$algebra $A$ is called separable, if it satisfies the following
conditions, which are equivalent: 
\begin{description}
\item[(1)] A is a projective $A^e-$module.

\item[(1)$_{\rm bis}$] $M \mapsto M^A$ is an exact functor on
  $A^e -$ modules. 

\item[(1)$_{\rm ter}$] $(A^e)^A \to A^A \to 0$ is exact. 

\item[(2)] If $M$ is an $A^e-$module, then every $k-$derivation $A \to
  M$ is inner.  

\item[(2)$_{\rm bis}$] the derivation $\delta : A \to J$ is
  inner. 
\end{description}
\end{propanddef}

\begin{proof}
Since $M^A \approx \Hom_{A^e}(A, M)$, the implications $(1)
\Leftrightarrow (1)_{\rm bis} \Rightarrow (1)_{\rm ter}$ are clear. If
$\Hom_{A^e} (A,A^e) \to \Hom_{A^e}(A, A) \to 0$ is exact, then $1_A$ factors
through $A^e$, so that $A$ is $A^e$ projective, thus proving
$(1)_{\rm ter} \Rightarrow (1)$. 
\end{proof}

$(1) \Leftrightarrow (2)$ by virtue of the identifications
$\Ext^1_{A^e}(A, M) = H^1(A, M) = $ derivations modulo inner
derivations. Also the implication $(2) \Rightarrow (2)_{\rm bis}$ is
obvious. Finally, proposition \ref{chap3:prop1.4} shows that $\delta$
is inner $\Leftrightarrow 1_J$ extends to a homomorphism $A^e
\rightarrow J$ that is, $\Leftrightarrow$ the exact sequence (1.1)
splits. This proves $(2)_{\rm bis} \Rightarrow (1)$.  

\setcounter{coro}{6}
\begin{coro}% coro 1.7
If\pageoriginale $A/k$ is separable with centre $C$, then for an
$A^e-$module $M$, there is a split exact sequence of $C-$models,  
$$
0 \to M^A \to M \to Der_k (A, M) \to 0.
$$
In particular, $C$ is a $C-$direct summand of $A$.
\end{coro}

This follows directly from (1.5) and the definition above. 

\begin{coro}% corol 1.8
If $A \to B$ is an epimorphism of $k-$algebras, with $A/k$ separable,
then $B/k$ is separable, and centre $B$= image of centre $A$. 
\end{coro}

\begin{proof}
If $M$ is a two-sided $B-$module, then evidently $M^B = M^A$, so that
$M \mapsto M^B$ is an exact functor, that is, $B$ is separable. Also,
$A^A \to B^A = B^B \to 0$ is exact. 
\end{proof}


\section{Assorted lemmas}\label{chap3:sec2}%secc 2
The reader is advised to skip this section and use it only for
references.  

\begin{lemma}[Schanuel's lemma] %lem 2.1
If $0 \to N_i \to P_i \xrightarrow{f_i} M \to
    0$ are exact with $P_i$ projective, $ i = 1, 2$, then $P_1
    \oplus N_2 \approx P_2 \oplus N_1$. 
\end{lemma}

\begin{proof}
If $P_1 \prod\limits_M P_2 = \{(x_1, x_2) \in P_1 \oplus P_2 \big| f_1
(x_1) = f_2 (x_2) \}$, then the coordinate projections give us maps
$P_1 \prod_M P_2 \to P_i$, $i = 1, 2$, and\pageoriginale a commutative
diagram   
\[
\xymatrix{
& & 0 \ar[d] & 0 \ar[d]& \\
& & N_2 \ar[d] \ar@{=}[r] & N_2 \ar[d] & \\
0 \ar[r] & N_1 \ar[r] \ar@{=}[d] & P_1 \Pi_M P_2 \ar[d] \ar[r] & P_2
\ar[r] \ar@{-}[d] & 0 \\
0 \ar[r] & N_1 \ar[r] & P_1 \ar[r] \ar[d] & M \ar[r] \ar[d] & 0\\
& & 0 & 0 & 
}
\]
with exact rows and columns, as is easily checked. Since the $P_i$ are
projective, we conclude that $P_1 \oplus N_2 \approx P_1 \prod
\limits_M P_2 \approx P_2 \oplus N_1$. 
\end{proof}

Let A be a ring. An $A-$ module $M$ is called \textit{finitely
  presented}, if there exists an exact sequence 
$$
F_1 \to F_0 \to M \to 0
$$
of $A-$linear maps with $F_i$ a finitely generated free $A-$ module,
$i= 0, 1$. 

\setcounter{coro}{1}
\begin{coro} %corol 2.2
\begin{enumerate}[(a)]
\item If $ 0 \to M' \to M \to M'' \to 0 $ is an exact sequence of
  $A-$modules, with $M$ and $M''$ finitely presented, then $M'$ is 
  finitely generated. 

\item If\pageoriginale $A$ is commutative, and $M$ and $N$ are
  finitely presented $A-$modules, then so is $M \otimes_A N$.  

\item If $A$ is an algebra over a commutative ring $k$, and if $A$ is
  finitely presented as a $k-$module, then $A$ is finitely presented
  as an $A^e-$module. 
\end{enumerate}
\end{coro}

\begin{proof}
\begin{enumerate}[(a)]
\item Case I. Suppose $M$ is projective. Then the result follows
  easily from the definition and Schanuel's lemma. 
\end{enumerate}

\medskip
\noindent
\textbf{General Case.} Let $f : P \to M$ be surjective with $P$
finitely generated and projective, and let $f'' : P \to M''$ be the
epimorphism obtained by composing $f$ with $M \to M''$. We have a
commutative diagram 
\[
\xymatrix{
& 0 \ar[r] \ar[d]^{f'} & P \ar[d]^f \ar@{=}[r] & P \ar[r] \ar[d]^{f''}
  & 0 \\
0 \ar[r] & M' \ar[r] & M \ar[r] & M'' \ar[r] & 0
}
\]
with exact rows. The exact sequence
$$
0 = \ker f' \to \ker f \to \ker f'' \to \text{ coker } f' = M' \to \text{ 
  coker } f = 0 
$$
shows that  $M'$ is finitely generated, since by case I, ker $f''$ is.
\begin{enumerate}
\item[(b)] follows easily from right exactness.

\item[(c)] If $A$ is finitely presented as a $k-$ module, then $A^e$
  is finitely presented as a $k-$module, This implies that $J$ is
  finitely generated as a $k-$module and a fortiori as an $A^e-$
  module.	 
\end{enumerate}
\end{proof}

\setcounter{lemma}{2}
\begin{lemma}% lem 2.3
Let\pageoriginale $K_i$ be a commutative $k-$ algebra and let $M_i$,
$N_i$ be \break $K_i -$modules, $i = 1, 2$. There is a natural isomorphism  
$$
(M_1 \otimes_k M_2) \otimes_{K_1 \otimes_k K_2} (N_1 \otimes_k N_2) 
\xrightarrow{\approx} (M_1 \otimes_{K_1} N_1) \otimes_k (M_2
\otimes_{K_2} N_2) 
$$
given by $(m_1 \otimes m_2) \otimes (n_1 \otimes n_2) \mapsto (m_1
\otimes n_1) \otimes (m_2 \otimes n_2)$. If the $M_i$ and $N_i$ are
$K_i-$algebras, then the above map is an isomorphism of $K_1 \otimes_k
K_2 -$algebras. 
\end{lemma}

\begin{proof}
Straightforward.
\end{proof}

\setcounter{coro}{3}
\begin{coro}\label{chap3:coro2.4} % corol 2.4
\begin{enumerate}[(a)]
\item If $K_i$ is a commutative $k$-algebra, and $A_i$ a $K_i$-alge\-bra,
  $i = 1, 2$, then (2.3) defines a natural isomorphism  
$$
(A_1/K_1)^e \otimes_k (A_2 /K_2)^e \approx (A_1 \otimes_k A_2 /K_1
  \otimes_k K_2)^e 
$$ 

\item If $K$ and $A$ are $k-$algebras, $K$ commutative, then $(K
  \otimes_k A/K)^e \approx K \otimes_k (A/k)^e$. 
\end{enumerate}
\end{coro}

\begin{proof}
\begin{enumerate}[(a)]
\item In (2.3) we take $M_i = A_i$ and $N_i = A^0_i$. Evidently $(A_1
  \otimes_k A_2)^0 = A^0_1 \otimes_k A^0_2$. 

\item Set $A_1 = K_1 = K$, $A_2 = A$ and $K_2 = k$ in $(a)$.
\end{enumerate}
\end{proof}

\setcounter{lemma}{4}
\begin{lemma} %lem 2.5
Let $K_i$ be a commutative $k$-algebra, let $A_i$ be a $K_i$-algebra,
and let $M_i$ and $N_i$ be $A_i$-modules, $i = 1, 2$.\,\hbox{The $k$-bilinear
map}~$(f_1, f_2)\break \mapsto f_1 \otimes f_2$ defines a $K_1 
\otimes_k K_2$-homomorphism 
$$
\Hom_{A_1} (N_1, M_1) \otimes_k \Hom_{A_2} (N_2, M_2) \rightarrow
\Hom_{A_1 \otimes_k A_2} (N_1 \otimes_k N_2, M_1 \otimes_k M_2). 
$$
It\pageoriginale is an isomorphism in either of the following situations:
\begin{enumerate}[(i)]
\item $N_i$ is a finitely generated projective $A_i-$module,
  $i = 1, 2$. 

\item $N_1$ and $M_1$ are finitely generated projective
  $A_1-$modules, $A_1$ is $k-$flat, and $N_2$ is a finitely
  presented $A_2-$module. 
\end{enumerate}
\end{lemma}

\begin{proof}
The first assertion is clear.
\begin{enumerate}[(i)]
\item By additivity we are reduced to the case $N_i = A_i$, $i = 1,
  2$, and then the assertion is clear. 

\item By additivity again we can assume that $N_1 = M_1 = A_1$.
\end{enumerate}

Write $SN_2 = A_1 \otimes_k \Hom_{A_2}(N_2, M_2)$, $TN_2 = \Hom_{A_1
  \otimes_k A_2} (A_1 \otimes_k N_2,\break A_1 \otimes_k M_2)$. We have a
map $SN_2 \to TN_2$. This is a isomorphism for $N_2 = A_2$, and
therefore, for $N_2 = A^{(n)}_2$. Let now $A^{(n)}_2 \to A^{(m)}_2 \to
N_2 \to 0$ be an exact sequence. $S$ and $T$ being left exact
contravariant functors in $N_2$, we obtain a commutative diagram 
\[
\xymatrix{
0 \ar[r] & SN_2 \ar[r] \ar[d] & SA^{(m)}_2 \ar[r] \ar[d] & SA^{(n)}_2
\ar[d]\\
0 \ar[r] & TN_2 \ar[r] & TA^{(m)}_2 \ar[r] & TA^{(n)}_2
}
\]
with  exact rows. The second and third vertical maps being
isomorphisms, if follows that the first one is also an isomorphism. 
\end{proof}

\setcounter{coro}{5}
\begin{coro} %corol 2.6
If $K_i$ is a commutative $k-$algebra, and if $A_i /K_i$ is a
separable algebra, $i = 1, 2$, then $A_1 \otimes_k A_2 / K_1
\otimes_k K_2$ is a separable\pageoriginale
 algebra with centre = (centre $A_1$)
$\otimes_k $ (centre $A_2$). More generally, if $M_i$ is an $(A_i /
K_i)^e$ module, then the natural map 
$$
M^{A_1}_1 \otimes_k M^{A_2}_2 \to (M_1 \otimes_k M_2)^{A_1 \otimes_k
  A_2} 
$$
is an isomorphism.
\end{coro}

\begin{proof}
We have, by hypothesis, (2.4), and (2.5), an isomorphism 
\begin{align*}
M^{A_1}_1 \otimes_k M^{A_2}_2 & = \Hom_{(A_1/K_1)} e(A_1, M_1) \otimes_k
\Hom_{(A_2/K_2)} e(A_2, M_2) \to \\
&\quad \Hom_{(A_1 /K_1)^e \otimes_k (A_2 /
K_2)^e} (A_1 \otimes_k A_2, M_1 \otimes_k M_2)\\
& = (M_1 \otimes_k
M_2)^{A_1 \otimes_k A_2} 
\end{align*}
 Applying this to $(A_1 /K_1)^e \otimes_k (A_2 / K_2)^e = (A_1
 \otimes_k A_2/K_1 \otimes_k K_2)^e \to A_1 \otimes_k A_2 \to 0$ we get
 a commutative diagram	 
\[
\xymatrix{
((A_1 \otimes_k A_2/ K_1 \otimes_k K_2)^e)^{A_1 \otimes_k A_2} \ar[r]
  & (A_1 \otimes_k A_2)^{A_1 \otimes_k A_2} \\
((A_1/k_1)^e)^{A_1} \otimes_k ((A_2/K_2)^e)^{A_2} \ar[r] \ar[u] & (A^{A_1}_1
  \otimes_k A^{A_2}_2) \ar[u]
}
\]
in which the vertical maps are isomorphisms and the lower horizontal
map is surjective, by hypothesis and right exactness of
$\otimes_k$. It follows that the upper map is also surjective, and
this finishes the proof, using criterion (1.6)(1)$_{\rm ter}$ for
separability. 
\end{proof}


\begin{coro} %Cor 2.7
If\pageoriginale $A_i / k$ is a separable algebra, $i =1, 2$, then
$(A_1 \otimes_k A_2)/k$ is separable with centre $A_1 \otimes_k$
centre $A_2$ as its centre.  
\end{coro}

\begin{coro} % Cor 2.8
Suppose $K$ and $A$ are $k$-algebras. Suppose further that $K$ is
$k$-flat. 
\begin{enumerate}[(a)]
\item If $M$ and $N$ are $A$-modules with $N$ finitely presented, then 
$$
K \otimes_k \Hom_A (N, M) \to \Hom_{k \otimes_{k} A}(K \otimes_k N, K
\otimes_k M) 
$$
is an isomorphism.

\item If $K$ is commutative, and if $A$ is finitely presented as
  an$A^e$-module then for an $A^e$-module $M$, the map 
$$
K \otimes_k (M^A) \to (K \otimes_k M)^{K \otimes_k A}  
$$
\textit{is an isomorphism}.
\end{enumerate}
\end{coro}

\begin{proof}%Proof
The statement $(a)$ follows from (2.5) (ii) with $N_1= M_1= A_1 = K_1
=K$, and $K_2 =k$. The statement (b) follows from (a) by substituting
$A^e$ for $A$, and $A$ for $N$.  
\end{proof}

\begin{coro}% Cor 2.9
Let $K$ and $A$ be $k$-algebras, with $K$ commutative.
\begin{enumerate}[(a)]
\item $A/_k$ separable $\Rightarrow K \otimes_k A/K$ is separable with
  centre $(K \otimes_k A) = K \otimes_k$ (centre $A$). 

\item If  $K$ is faithfully $k$-flat and if $A$ is a finitely
  presented $A^e$-module, then $(K \otimes_k A)/K$ separable
  $\Rightarrow A/ k$ separable. 
\end{enumerate}
\end{coro}

\begin{proof}%Proof
\begin{enumerate}[(a)]
\item is\pageoriginale a special case of (2.6).

\item Suppose $K \otimes_k A/K$ is separable. Corollary
  \ref{chap3:coro2.4} implies 
  that $(K \otimes_k A/K)^e \to K \otimes_k A$ is isomorphic to $K
  \otimes_k ((A/k)^e \to A)$. Then (2.8)(b) further implies that
  $((K \otimes_k A/k)^e)^{K \otimes_k A} \to (K \otimes_k
  A)^{K \otimes_k A}$ is isomorphic to $K \otimes_k ((( A/k)^e)^A \to
  A^A)$. Therefore, by hypothesis, $K \otimes_k ((( A/k^e)^A \to A^A)$
  is surjective. Since $K$ is faithfully $k$-flat, this implies that
  $((A/k)^e)^A \to A^A$ is surjective, so that $A/k$ is separable
  (see (1.6)(1) ter).  
\end{enumerate}
\end{proof}

\begin{example*}%Example
If $K$ is a noetherian local ring, in (2.9)(b) we can take $K$ to be
completion of $k$.  
\end{example*}

\begin{coro}\label{chap3:coro2.10} %Corlry 2.10
If $A/k$ is a finitely $A^e$-presented $k$-algebra, then $A/k$ is
separable $\Leftrightarrow A_{\mathscr{M}} / k_{\mathscr{M}}$  is
separable for all maximal ideals $\mathscr{M}$ of $k$. 
\end{coro}

\begin{proof}%Proof
Take $K = \prod\limits_{\mathscr{M}} k_\mathscr{M}$ in
(2.9)(b). Alternatively, repeat the proof of (2.9)(b) and at the end
use the fact that a $k$-homomorphism $f$ is surjective
$\Leftrightarrow f_{\mathscr{M}}$ is surjective for all $\mathscr{M}$.  
\end{proof}

\begin{coro}%Cor 2. 11
Suppose $A_i$ is a $k$-algebra and that $P_i$ is a finitely generated
projective $A_i$-module, $i=1, 2$. Then  
$$
\End_{A_1}(P_1) \otimes_k \End_{A_2}(P_2) \to \End_{A_1 \otimes_k 
  A_2}(P_1 \otimes_k P_2) 
$$
is an algebra isomorphism.
\end{coro}

\begin{proof}%Proof
Set $N_i = M_i=P_i$ and $K_i = k$ in (2.5) (i).
\end{proof}

\begin{coro} %Cor 2. 12
Suppose\pageoriginale $P_1$ and $P_2$ are finitely generated
projective $k$-modules. Then 
$$
\End_k (P_1) \otimes_k \End_k (P_2) \to \End_k (P_1 \otimes_k P_2)
$$
is an algebra isomorphism.
\end{coro}

\begin{proof}%Proof
Set $A_i = k$ in (2. 11).
\end{proof}

\setcounter{prop}{12}
\begin{prop}\label{chap3:prop2.13} %Prop 2. 13
Let $P$ be a finitely generated projective $k$-module. Then $A= \End_k
(P)$ is a separable algebra with centre $k/ ann P$. 
\end{prop}

\begin{proof}%Proof
Both centre $A$ and $B$ ann $P$ commute with localization, and hence
we can use (2.10) to reduce to the case when $P$ is free, say $P
\approx k^{(n)}$, so that $A \approx \mathbb{M}_n (k)$. Denoting the
standard matrix algebra basis by $(e_{ij})$, we set $e=
\sum\limits^{n}_{i=1} e_{i 1} \otimes e_{1 i} \epsilon A^e$. Then
$(e_{rs} \otimes 1) e = \sum e_{rs}e_{i 1} \otimes e_{1 i} = e_{r 1}
\otimes e_{1 s}$ and $( 1 \otimes e_{rs}) e= \sum e_{i 1} \otimes e_{1i}
   e_{rs} = e_{r1} \otimes e_{1 s}$. Hence $e \epsilon
(A^e)^A$. Under $(A^e)^A \to A^A$, $e$ maps into $\sum e_{i1} e_{1 i}
= \sum_{e_{ii}}=1$. Hence $(A^e)^A \to  A^A$ is surjective. Thus $A$
is separable, by (1.6)(1)$_{\rm ter}$. Theorem (\ref{chap2:thm6.1})(5)
of chapter \ref{chap2} implies that centre $\mathbb{M}_n (k) = k$. 
\end{proof}

\setcounter{lemma}{13}
\begin{lemma}\label{chap3:lem2.14} %Lemma 2.14
Let $f: M \to M$ be a $k$-endomorphism, $k$ a commutative
ring. Suppose that $M$ is either noetherian or finitely generated and
projective. Then, if $f$ is surjective, it is an automorphism. 
\end{lemma}

\begin{proof}%Proof
If ker $f \neq 0$, then $f$ surjective implies that ker $f^n$ is a
strictly ascending chain of submodules, an impossibility if $M$ is
noetherian, If $M$ is projective, then $ M \approx M \oplus$ ker $f$
and localization\pageoriginale shows that $\ker f =0$ if $M$ is
finitely generated.  
\end{proof}

\setcounter{prop}{14}
\begin{prop}\label{chap3:prop2.15} %Prop 2.15
Let  $k$ be a local ring with maximal ideal $\mathscr{M}$, and let $A$
be a $k$-algebra, finitely generated as  a $k$-module. Suppose that
either $k$ is noetherian or that $A$ is $k$-projective. Then if  $A/
\mathscr{M} A$ is a separable $(k / \mathscr{M})$-algebra, $A$ is a
separable $k$-algebra. 
\end{prop}

\begin{proof}%Proof
Consider $\delta : A \to J = J(A/k)$. Let $k'$ denote reduction modulo
$\mathscr{M}$; e. g. $k' = k / \mathscr{M}$. Then $\delta$ induces a
$k'$-derivation $\delta': A' \to J'$, where $J'$ is a two-sided
$A'$-module. By hypothesis and criterion (1.6)(2), $\delta'$ must be
inner, $\delta' (a') = a' e' - e'a' = ae' - e' a = \delta (a) e'$, for
some $e' \epsilon J'$, coming from say $e \epsilon J$. 
It follows that $\delta (a) e \equiv \delta (a) \mod \mathscr{M} J$,
so (1.2) implies $J =  Je + \mathscr{M} J$. The exact sequence  
$$
0 \to J \to A^e \to A \to 0
$$ 
shows that $J$ is noetherian if $k$ is, and that $J$ is $k$-projective
and finitely generated if $A$ is. Hence we can apply lemma
\ref{chap3:lem2.14} to the 
$k$-homomor\-phism $J \xrightarrow{e} J$, provided the latter is
surjective. But this follows from Nakayama's lemma, since $J = Je +
\mathscr{M} J$. 

Now the composite $J \hookrightarrow A \xrightarrow{e} J$ is an
automorphism of $J$, so that $J$ is an $A^e$-direct summand of
$A^e$. This proves that $A$ is $A^e$ -projective, as required. 
\end{proof}

\setcounter{lemma}{15}
\begin{lemma} %Lem 2.16
Let $f: P \to M$ be a $k$-homomorphism with $P$ finitely  generated
and projective. Denote the functor $\Hom_k ( ,k)$ by
$^*$. Then\pageoriginale $f$ has a left inverse $\Leftrightarrow f^*:
M^* \to P^*$ is surjective. If 
$M$ is finitely presented, then (coker $f^*)_\mathscr{M} = $ coker
$(f_{\mathscr{M}})^*$, so that $f$ has a left inverse $\Leftrightarrow
f_{\mathscr{M}}$ does for all maximal ideals $\mathscr{M}$.  
  \end{lemma}  

  \begin{proof}%Proof
$f$ left invertible $\Rightarrow f^*$ right invertible $\Rightarrow f^*$
    surjective $\Rightarrow f^*$ right invertible (because $P^*$ is
    projective) $\Rightarrow f^{**}: P^{**} \to M^{**}$ left
    invertible. The commutative square 
\[
\xymatrix{
P \ar[r]^f \ar[d]_{\approx} & M \ar[d] \\
P^{\ast\ast} \ar[r]_{f^{\ast\ast}} & M^{\ast\ast}
}
\]
 shows that $f^{**}$ left invertible $\Rightarrow f$ left invertible.  
  \end{proof}   
  
  The natural homomorphism
  $$
  (M^*)_{\mathscr{M}} =  ( \Hom_k (M,K))_\mathscr{M} \to
  (M_\mathscr{M})^* = \Hom_{k_{\mathscr{M}}} (M_\mathscr{M}, 
  k_\mathscr{M}) 
  $$
  is an isomorphism for $M$ finitely presented, by (2.8)(a). Hence
  since $P$ is also finitely presented, we have  $(f^*)_{\mathscr{M}}
  \approx (f_{\mathscr{M}})^*$ in this case, so that by exactness of
  localization, (coker $f^*$) $_{\mathscr{M}} \approx$ coker
  $(f_\mathscr{M})^*$. 

\setcounter{coro}{16}
  \begin{coro}\label{chap3:coro2.17}  %Corlry 2.17
If $A$ is a faithfully $k$-projective $k$-algebra, then $k$ is a
direct summand of $A$. 
  \end{coro}  

  \begin{proof}%Proof
We want $k \to A$ to have a left inverse, and (2. 16) plus our
hypothesis makes it sufficient to prove this for $k$ local,
say\pageoriginale with 
maximal ideal $\mathscr{M}$. Then $1 \epsilon A/ \mathscr{M} A$ is a
part of a $k /\mathscr{M}$-basis for  $A/ \mathscr{M}A$, so Nakayama's
lemma implies that $1 \epsilon A$ is a part of a $k$-basis of $A$. 
  \end{proof}  

\setcounter{prop}{17}
  \begin{prop}\label{chap3:prop2.18} %Prop 2.18
Let $A$ and $B$ be $k$-algebras with $A$ a faithfully projective
$k$-module. Then $A \otimes_k B /k$ separable $\Rightarrow B/ k$
separable. 
  \end{prop}

  \begin{proof}%Proof
$A$ is $k$-projective implies that $A^e$ is $k$-projective. Hence $(A
    \otimes_k B)^e \approx A^e \otimes_k B^e$ is
    $B^e$-projective. Thus, if $A \otimes_k B$ is $(A \otimes_k
    B)^e$-projective, then it is $B^e$-projective. Corollary
    \ref{chap3:coro2.17} and 
    our hypothesis implies $A \approx k \oplus A'$ as a $k$-module, so
    that $A \otimes_k B \approx B \oplus (A' \otimes_k B)$ as a
    $B^e$-module. Thus $B^e$-projectivity of $A \otimes_k B \Leftarrow
    B^e$-projectivity of $B$.  
    \end{proof}    

    \begin{prop}\label{chap3:prop2.19} %Prop 2. 19
Suppose $A$ is a $K$-algebra and that $K$ is a $k$-algebra. Then
\begin{enumerate}[(1)]
\item $A/K$ and $K/k$ separable $\Rightarrow A/k$ separable. 

\item $A/k$ separable $\Rightarrow A/k$ separable.
\end{enumerate}

If $A$ faithfully $K$-projective, then $A/k$ separable $\Rightarrow
K/k$ separable. 
    \end{prop}    

    \begin{proof}%Proof
\begin{enumerate}[(1)]
\item $K/ k$ separable means that 
$$
0 \to J (K /k) \to  K^e \to K \to 0
$$
splits as an exact sequence of $K^e$-modules. Hence
$$
0 \to (A/k)^e \otimes_{K^e} J(K/k) \to (A/k)^e \to (A/k)^e
\otimes_{k^{e}} K  \to 0 
$$
splits\pageoriginale as an exact sequence of $(A/k)^e$-modules, so that
$(A/k)^e 
\otimes_{K^e}K$ is a projective $(A/k)^e$-module. it follows easily
from corollary \ref{chap3:coro1.3} that $(A \otimes_k A^o)
\otimes_{K^e} K \approx 
A \otimes_K A^o = (A/ K)^e$. Hence if we further assume that $A$ is
$(A/K)^e$-projective, it follows from the projectivity of $(A/K)^e$
over $(A/k)^e$, remarked above, that $A$ is $(A/k)^e$-projective. 

\item In the commutative diagram with exact rows and columns 
\[
\xymatrix{
(A/k)^e \ar[r] \ar[d] & A \ar[r] \ar@{=}[d] & 0 \\
(A/K)^e \ar[r] \ar[d] & A \ar[r] & 0\\
0 & & 
}
\]
if the top splits, then so much the bottom. Suppose $A$ is faithfully
$K$-projective. Then $(A/k)^e$ is $(K/k)^e$-projective, so that $A$ is
$(K/k)^e$-projective, assuming that $(A/k)$ is separable. By corollary
\ref{chap3:coro2.17} $K$ is a $K$-direct summand, of $A$, hence a
$(K/k)^e$-direct 
summand, so we conclude that $K$ is $(K/k)^e$-projective, as claimed. 
\end{enumerate}
    \end{proof}    

    \begin{prop}\label{chap3:prop2.20} %Prop 2.20
\begin{enumerate}[(a)]
\item If  $A_1$ and $A_2$ are $k$-algebras, then $A_1 \times A_2 /k$
  is separable $\Leftrightarrow A_1 / k$ and $A_2 /k$ are. 

\item If $A_i$ is a $k_i$-algebra, $i=1, 2$, then $A_1 \times A_2 /
  k_1 \times k_2$ is separable $\Leftrightarrow A_1 / k_1$ and $A_2$
  and $A_2 / k_2$ are. 
\end{enumerate}
    \end{prop}  

\begin{proof}%Proof
\begin{enumerate}[(a)]
\item $(A_1 \times A_2)^e$\pageoriginale $ =(A_1 \otimes_k A^0_1) \times (A_2
  \otimes_k A^0_2) \times (A_1 \otimes_k A^0_2) \times (A_2 \otimes_k
  A^0_1) =  A^e_1 \times A^e_2 \times B$, and $A_1 \times A_2$ is an
  $(A_1 \times A_2)^e$-module annihilated by $B$. As such it is the
  direct sum of the $A^e_i$-modues $A_i$. Thus $A_1 \times A_2$ is
  $(A_1 \times A_2)^e$ -projective $\Leftrightarrow A_i$ is
  $A^e_i$-projective  

\item Any $k_1 \times k_2$-module or algebra splits canonically into a
  product of one over $k_1$ and one over $k_2$. In particular $(A_1
  \times A_2 / k_1 \times  k_2)^e =  (A_1 / k_1 )^e \times (A_2 /
  k_2)^e$, so $A_1 \times A_2$ is $(A_1 \times A_2 / k_1 \times
  k_2)^e $-projective $\Leftrightarrow A_i$ is $(A_i /
  k_i)^e$-projective.  
\end{enumerate}
    \end{proof}    
    

\section{Local criteria for separability}\label{chap3:sec3} % Sec 3
    
  \begin{theorem}\label{chap3:thm3.1}%Thm 3.1
Let $A$ be a $k$-algebra, finitely generated as a $k$-module. Suppose
either that $k$ is noetherian or that $A$ is a projective
$k$-module. Then the following statements are equivalent:  
\begin{enumerate} [(1)]
\item $A/k$ is separable.

 \item For each maximal ideal $\mathscr{M}$ of $k$, $A/ \mathscr{M} A$ 
  is a semi-simple $k / \mathscr{M}$-algebra whose centre is a 
  product of separable field extensions of $k / \mathscr{M}$. 

\item For any homomorphism $k \to L$, $L$ a field, $L \otimes_k A$ is
  a semi-simple algebra. 
    \end{enumerate}    
  \end{theorem}

    We will deduce this form the following special case:

\setcounter{theorem}{1}
    \begin{theorem}\label{chap3:thm3.2}%Thm 3.2
Let $A$ be a finite dimensional algebra over a field $k$. The
following statements are equivalent: 
 \begin{enumerate}[(1)]
\item $A/k$ separable 

\item $L \otimes_k A$ is semi-simple for all field extensions
  $L/k$.
 
\item For\pageoriginale some algebraically closed field $L /k$, $L
  \otimes_k A$ is $\varepsilon$ product of full matrix
    algebras over $L$. 

\item $A$ is semi-simple and centre $A$ is  a product
  of separable field extensions of $k$. 
 \end{enumerate}
    \end{theorem}

    We first prove that $(3.2) \Rightarrow (3.1)$: 
    
    $(1) \Rightarrow (3)$. If $k \to L$, then $L \otimes_k  A/L$ is
    separable, by (2.9), and we now apply $(1) \Rightarrow (4)$ of
    (3.2). 
    
    $(3) \Rightarrow (2)$. Apply $(2) \Rightarrow (4)$ of (3.2),
    where $L$ ranges over field extensions of $k / \mathscr{M}$. 
    
    $(2) \Rightarrow (1)$. From $(4) \Rightarrow (1)$ of (3.2) we
    know that $A/ \mathscr{M} A$ is a separable $(k/
    \mathscr{M})$-algebra so the hypothesis on $A$ and proposition
    \ref{chap3:prop2.15} imply that  $A_{\mathscr{M}} / k_{\mathscr{M}}$ is
    separable, for all $\mathscr{M}$. (1) now follows from corollary
    \ref{chap3:coro2.10}. 

\setcounter{proofofthm}{1}
    \begin{proofofthm}%%%% 3.2
$(1) \Rightarrow (2)'$. Since $A/k$ separable implies that $(L
      \otimes_k A) /L$ is separable, it suffices to show that $A/k$
      separable $\Rightarrow A$ is semi-simple. Let $M, N$ be left
      $A$-modules. Then $\Hom_m (M, N)$ is a two-sided $A$-module,
      i.e., an $A^e$-module, and $Hom_{A^e}(A, Hom_k (M,N))=
      \Hom_A(M,N)$ clearly (see (1.3)). Since $k$ is a field,
      $\Hom_k(M, ~)$ is an exact functor. Since $A$ is
      $A^e$-projective (by assumption), $\Hom_{A^e}(A, ~)$ is
      exact. Hence $\Hom_A(M, ~)$ is an exact functor, so every
      $A$-module is projective. Proposition \ref{chap2:prop6.7} of
      Chapter \ref{chap2} now implies that $A$ semi-simple.  
    
    $(2) \Rightarrow (3)$.\pageoriginale This follows from the structure of
      semi-simple rings plus the fact that there are no non-trivial
      finite dimensional division algebras over an algebraically
      closed field. 
    
    $(3) \Rightarrow (1)$. By assumption, $L \otimes_k A$ is a product
      of full matrix algebras over $L$. Proposition
      \ref{chap3:prop2.13} and \ref{chap3:prop2.20} 
      imply that $L \otimes_k A/L$ is separable.  Since $L$ is
      faithfully $k$-flat ($k$ is a field!), (2.9)(b) implies that
      $A/k$ is separable. 
    
    The proof of $(1)  \Leftrightarrow (4)$ will be based upon the
    next two lemmas, which are special cases of the theorem. 
    \end{proofofthm}

\setcounter{lemma}{2}
    \begin{lemma}\label{chap3:lem3.3} % lemma 3.3
$A$ finite field extension $C/k$ is separable as a $k$-algebra
      $\Leftrightarrow$ it is separable as a field extension of $k$. 
    \end{lemma}   

    \begin{proof}%Proof
If $k \subset K \subset C$, then $C/k$ is separable, in either sense
$\Leftrightarrow C/K$ and $K/k$ are, in the same sense. This follows
from proposition \ref{chap3:prop2.18} in one case, and from field
theory in the 
other. An induction on degree therefore reduces the lemma to the case
$C=  k [X]/ (f(X))$. Let $L$ be an algebraic closure of $k$, and write
$f(X) = \prod\limits_{i}(X- a_i)^{e_i}$ in $L[X]$, with $a'_i s$
distinct. Then $C$ is a separable field extension $\Leftrightarrow L
\otimes_k C =L [X] / f (X) L[X]$ has no nilpoint elements
$\Leftrightarrow L \otimes_k C$ is a product of copies of $L
\Leftrightarrow L \otimes_k C/L$ is separable $\Leftrightarrow C$ is a
separable algebra over $k$, by $(1) \Leftrightarrow (3)$ of (3.2),
which we have already proved. 
    \end{proof}    
    
    We now prove the implication $(1) \Leftrightarrow (4)$ of
    (3.2). We have already proved that $A/k$ is separable implies that
    $A$ is semi-simple. Hence the centre $C$ of $A$ must be a finite
    product of field\pageoriginale extension of $k$. in particular $A$ is a
    faithfully projective $C$-module, so by proposition
    \ref{chap3:prop2.19}, $A/k$ 
    separable $\Rightarrow C/k$ separable. The last part of (4)
    now follows from proposition \ref{chap3:prop2.20} and lemma
    \ref{chap3:lem3.3} above.  

    \begin{lemma}\label{chap3:lem3.4} % Lemma 3.4
Let $k$ be a field, and suppose that  $A$ is a finite dimensional
$k$-algebra, simple and central (i.e.centre $A=k$). If $B$ is any
$k$-algebra, every two-sided ideal of $A \otimes_k B$ is of the form
$A \otimes_k J$ for some two sided ideal $J$ of $B$. 
    \end{lemma}    

\begin{proof}%Proof 
According to theorem \ref{chap2:thm6.1} chapter \ref{chap2} there is a
division algebra $D$ 
and an $n >0$ such that $A \approx \mathbb{M}_n (D)= D \otimes_k
\mathbb{M}_n (k)$. Theorem \ref{chap2:thm4.4} of chapter \ref{chap2} contains
the lemma when 
$A= \mathbb{M}_n (k)$, in which case $A \otimes_k B= \mathbb{M}_n (B)
= \End_B(B^{(n)})$. It therefore suffices to prove the lemma for
$A=D$, a division algebra. If $(e_i)$ is a $k$-basis for $B$, then $(1
\otimes e_i)$ is a left $D$-basis for $D \otimes_k (B)$. Let $I$ be a
two-sided ideal of $D \otimes_k B$. Then $I$  is a $D$-subspace of $D
\otimes_k B$, and it is (clearly) generated by the ``primordial''
elements of $I$ with respect to the basis $(1 \otimes e_i)$, i.e. by
those elements $x=  \sum(d_i \otimes 1) ( 1 \otimes e_i) \neq 0$ of
$I$ such that $S(x) = \{i | d_i \neq 0\}$ does not properly contain
$S(y)$ for any $y \neq 0$ in $I$, and such that the least one $d_i
=1$. If $x$ such an element and if $d \neq 0$ is in $D$, then $x(d
\otimes 1 ) \in I$, because $I$ is a two-sided ideal. Now $x(d \otimes
1) = \sum (d_i \otimes e_i) (d \otimes 1) = \sum d_i d \otimes e_i$ so
$S(x(d \otimes 1))= S(x)$. Subtracting $(d' \otimes 1)x$ from
$x(1\otimes d)$
will therefore render $S((d' \otimes 1) x-x(1 \otimes d))$ a proper
subset of $S(x)$, for a suitable $d'  \epsilon D$. 
    \end{proof}    
    
    Since\pageoriginale $x$ is primordial, this implies $(d' \otimes 1) x=x(d
    \otimes 1)$, i.e. that $\sum d' d_i \otimes e_i = \sum d_i d
    \otimes e_i$. Some $d_i =1$ so we have $d' = d$. Moreover, $d_i  d
    = dd_i$ for all $i$. By assumption centre $D =k$, so $x \epsilon k
    \otimes B= 1 \otimes B$. Setting $1 \otimes J = I\cap(1 \otimes
    B)$, we therefore have $I= D \otimes J$. 
    
    We shall now prove the implication $(4) \Rightarrow (1)$ of
    theorem \ref{chap3:thm3.2}. Let $C=$ centre $A$. To show that
    $A/k$ is separable 
    Proposition \ref{chap3:prop2.19} makes it sufficient to show that
    $A/C$ and $C/k$ 
    are separable. In each case, moreover, proposition
    \ref{chap3:prop2.20} reduces 
    the problem to the case when $C$ is a field, Separability of
    $C/k$ then results from the hypothesis and lemma
    \ref{chap3:lem3.3}. Let $L$ be 
    an algebraic closure of $C$. Then it follows from lemma
    \ref{chap3:lem3.4} that 
    $L \otimes_C A$ is simple, hence a full matrix algebra over
    $L$. Separability of $A/C$ now follows from the implication $(3)
    \Rightarrow (1)$ which we have proved. Thus the proof of  the
    theorem \ref{chap3:thm3.2} is complete. 

  \section{Azumaya algebras}\label{chap3:sec4} % Sec 4

 \begin{thmanddef}\label{chap3:thm4.1}%Thm and Def 4.1
An azumaya algebra is a $k$-algebra $A$ satisfying the following
conditions, which are equivalent: 
 \begin{enumerate}[(1)]
\item $A$ is a finitely generated $k$-module and $A/k$
  is central and separable. 

\item $A/k$ is central and $A$ is a  generator as
  an $A^e$-module. 

\item $A$ is a faithfully projective $k$-module, and the
  natural representation $A^e \to \End_k (A)$ is an
  isomorphism. 

\item The\pageoriginale bimodule $_{{A}^e} A_k$ is invertible (in the
  sense of definition \ref{chap2:def3.2} of chapter \ref{chap2}),
  i.e. the functors  
%\begin{align*}
%(N & \; \longmapsto \; A \otimes_k N) \\
%k- {\rm mod} & \; \longrightarrow \; A^e -{\rm mod} \\
%& \; \longleftarrow \; \\
%(M^A  & \; \longleftarrow \; M)
%\end{align*}

\[
\xymatrix@R=-1.3cm{
(M^A & M \ar@{|->}[l])\\
k-{\rm mod} \ar@<1ex>[r] & A^e-{\rm mod} \ar@<1ex>[l]\\
(N \ar@{|->}[r] & A \otimes_k N)
}
\]
are inverse equivalences of categories.

\item $A$ is a finitely generated projective $k$-module, and for all
  maximal ideals $\mathscr{M}$ of $k$, $A/ \mathscr{M}A$ is a central
  simple $k / \mathscr{M}$ -algebra. 

\item There exists a $k$-algebra $B$ and a faithfully projective
  $k$-module $P$ such that $A \otimes_k B \approx \End_k (P)$. 
 \end{enumerate}      
 \end{thmanddef}

 \begin{proof}
$(1) \Rightarrow (2)$. Let $\mathfrak{M}$ be a maximal two-sided ideal of $A$,
   and set $\mathscr{M} = \mathfrak{M} \cap k$. According to (1.8) and our
   hypothesis $A/\mathfrak{m}$ is a separable $k$-algebra with centre $k /
   \mathscr{M}$. Since $A /\mathfrak{M}$ does not have two-sided ideals, its
   centre is a field. Thus $\mathscr{M}$ is a maximal ideal of $k$, so
   $A/ \mathscr{M} A$ is a central separable algebra over $k /
   \mathscr{M}$, and it follows from theorem \ref{chap3:thm3.2} that
   $A/ \mathscr{M} 
   A$ is simple. Consequently $m = \mathscr{M}A$. Applying this to
   $A^e$, which, by (2.7), is also a separable $k$-algebra, we
   conclude that every maximal two-sided ideal of  $A^e$ is of the
   form $\mathscr{M} A^e$ for some maximal ideal $\mathscr{M}$ of $k$.  
  \end{proof}        

  Viewing\pageoriginale $A$ as a left $A^e$-module we have the pairing
  $g_A$: $A 
  \otimes_k \Hom_{A^e}\break (A, A^e) \to A^e$, and its image is a two
  -sided ideal which equals $A^e \Leftrightarrow A$ is a generator as
  an $A^e$-module (see(5.6) of chapter \ref{chap2}). If $im g_A \neq A^e$, then
  im $g_A$ is contained in some maximal two-sided ideal of $A^e$, so,
  according to the paragraph above, im $g_A \subset \mathscr{M}
  A^e$, for some maximal ideal $\mathscr{M}$ of $k$. Now lemma
  \ref{chap2:lem5.7} of 
  chapter \ref{chap2}, plus our hypothesis that $A$ is $A^e$-projective, imply
  that $A=(im g_A)A \subset \mathscr{M}A$. But from (1.7), $k=$ centre
  $A$ is a direct summand of $A$. So $A = \mathscr{M} A \Rightarrow k
  = \mathscr{M}$, which is a contradiction. 
  
  $(2) \Rightarrow (3)$. $A$ is a generator of $A^e-\mod$, so the pairing
  $$
  g_{A}: \Hom_{A^e}(A, A^e) \otimes_k A \to  A^e
  $$ 
  is surjective. It follows now from theorem \ref{chap2:thm4.3} and
  proposition \ref{chap2:prop5.6} 
  of chapter \ref{chap2} that $A$ is a finitely generated projective
  $k$-module, and that $A^e \to \End_k (A)$ is an isomorphism. Since
  $A$ is a faithful $k$-module ($k$ being centre of $A$), corollary
   \ref{chap2:coro5.10} of chapter \ref{chap2} implies that it is
   faithfully projective.  
  
  $(3) \Rightarrow (4)$. This follows directly from proposition
  \ref{chap2:prop5.6}  and definition \ref{chap2:def3.2} of chapter
  \ref{chap2}.  
  
  $(4) \Rightarrow (1)$. is trivial once we note that centre $A=
  \Hom_{A^e} (A,A)$. 
  
  $(1) \Rightarrow (5)$ follows from $(1) \Rightarrow (3)$ of theorem
  \ref{chap3:thm3.1}. 
  
  $(5) \Rightarrow (1)$. Theorem  \ref{chap3:thm3.1} shows that $A/k$
  is separable. 
  
  Let $C=$centre $A$. Then $C$ is a $C$-direct summand of $A$, and
  hence a finitely\pageoriginale generated projective $k$-module, since $A$ is
  so. We have a homomorphism $k \to C$, and (1.8) implies that $k/
  \mathscr{M} \to C / \mathscr{M}C$ is an isomorphism for all maximal
  ideals $\mathscr{M}$ of $k$. This implies that $k \to C$ is
  surjective, since the cokernel is zero modulo all maximal ideals of
  $k$, and hence it splits, because $C$ is projective. The kernel of
  $k \to C$ is also zero, since it is zero modulo all maximal ideals
  of $k$. Thus $k \to C$ is an isomorphism. 
  
  $(3) \Rightarrow (6)$. Take $B =A^0$ and $P=A$.
  
  $(6) \Rightarrow (1)$. $\End_k (P)$ is faithfully projective, since
  $P$ is. Since $A \otimes_k B \approx \End_k (P)$, it follows from
  proposition \ref{chap1:prop6.1} of chapter \ref{chap1} that $B$ is faithfully
  projective. Proposition \ref{chap3:prop2.13} says that $\End_k (P)/k$
  is  central 
  and separable, so that, by proposition \ref{chap3:prop2.18}, $A/k$ is
  separable. Similarly $B/k$ is separable. It follows from (2.7),
  that (centre $A$)$\otimes_k$(centre $B$) = centre $\End_k
  (P)=k$. Hence centre $A$ has rank 1, as a projective $k$-module, and
  so centre $A=k$, since $k$ is a direct summand of $A  \otimes_k B$
  and therefore of centre $A$. 
  
\setcounter{coro}{1}
  \begin{coro}%Corol 4.2
If $A/k$ is an azumaya algebra, then $\mathscr{U} \mapsto \mathscr{U}
A$ is a bijection from the ideals of $k$ to the two-sided ideals of
$A$. 
\end{coro}   

   \begin{proof}%Proof
This follows from theorem \ref{chap2:thm4.4} of chapter \ref{chap2},
since two-sided ideals of $A$ are simply $A^e$-submodules of $A$.  
   \end{proof}   

   \begin{coro} % Coro 4.3
Let $A \subset B$ be $k$-algebras with $A$ azumaya. Then the natural
map $A \otimes_k B^A \to B$ is an isomorphism. 
   \end{coro}   

   \begin{proof}%Proof
This is a special case of the statement (1) of theorem \ref{chap3:thm4.1}. 
   \end{proof}   

\begin{coro}%Coro 4.4
Every\pageoriginale endomorphism of an azumaya algebra is an automorphism. 
\end{coro}   

\begin{proof}%Proof
Suppose $f$: $A \to A$ is an endomorphism of an azumaya algebra
$A/k$. By (4.2), ker $f = \mathscr{U} A$ for some ideal $\mathscr{U}$
of  $k$ and hence ker $f=0$. Therefore (4.3) implies $A \approx f (A)
\otimes_k A^{f (A)}$. Counting ranks we see that $A^{f(A)}=k$. 
\end{proof}   

\begin{coro} %Cor 4.5
The homomorphism 
$$
\Pic_k (k) \longrightarrow \Pic_k (A),
$$
induced by $L \mapsto A \otimes_k L$, is an isomorphism.
 \end{coro}   

  \begin{proof}%Proof
This follows (see (4) of (4.1)) from the fact that $A \otimes_k:k-
\mod \to A^e -\mod$ is an equivalence which converts $\otimes_k$ into
$\otimes_A$; the latter is just the identity  
$$
(A \otimes_k M) \otimes_A (A \otimes_k N) \approx (A \otimes_A A) 
\otimes_k (M \otimes_k N) \approx A \otimes_k (M \otimes_k N). 
$$
   \end{proof}

 \begin{coro}[Rosenberg-Zelinsky] %Coro 4.6
 If $A/k$ is an azumaya algebra, then there is an exact sequence
$$
0 \to In {\rm Aut}(A) \to {\rm Aut}_{k-alg} (A)
\xrightarrow{\varphi_A}  \Pic (k),  
$$
where $im \varphi_A= \bigg\{(L)| A\otimes_k L\approx A $ as a left
$A$-module $\bigg \}$ 
      \end{coro}      

\begin{proof}%Proof
This follows immediately from (4.5) and proposition \ref{chap2:prop7.3}
of chapter \ref{chap2}
\end{proof}    

\begin{coro}%Coro 4.7
If $A/k$ is an azumaya algebra of rank $r$ as a projective $k$-module,
then $Aut_{k-alg}(A) / In  Aut (A)$ is an abelian group\pageoriginale
of exponent $r^d$ for some $d > 0$. 
\end{coro}  
    
\begin{proof}%Proof
Let $A \otimes_k L \approx A$ as a left $A$-module, hence as a
$k$-module. The remark following  proposition \ref{chap1:prop6.1} of
chapter \ref{chap1}  provides
us with a $k$-module $Q$ such that $Q \otimes_k A \approx k^{(r^d)}$ for
some $d > 0$. So $L^{(r^d)} \approx k^{(r^d)}$. Taking $r^d$th
exterior powers we have $L^{\otimes r^d} \approx k$. By virtue of
(4.6), the corollary is now proved. 
 \end{proof}       

\begin{coro}[Skolem-Noether]%Coro 4.8
 If   Pic $(k) =0$, then all automorphisms of an azumaya
  $k$-algebra are inner. 
       \end{coro}       

       \begin{coro} %Coro 4.9
If $A$ is a $k$-algebra, finitely generated as a module over its
centre $C$, then $A/k$ is separable $\Leftrightarrow A/C$ and $C/k$
are separable. 
       \end{coro}       

       \begin{proof}%Proof
In view of (2.19) it is enough to remark that, if $A/k$ is separable,
then $A$ is faithfully $C$-projective. This follows from $(1)
\Leftrightarrow (3)$ of theorem \ref{chap3:thm4.1}. 
       \end{proof}       

\setcounter{propanddef}{9}
\begin{propanddef} % Prop 4.10
Call two azumaya $k$-algebras $A_1$ and $A_2$ similar, if they satisfy
the following conditions, which are equivalent: 
\begin{enumerate}[(1)]
\item $A_1 \otimes_k A^o_2 \approx \End_k (P)$ for some
  faithfully projective $k$-module $P$. 

\item $A_1 \otimes_k \End_k (P_1) \approx A_2 \otimes_k \End_k (P_2)$
for some faithfully projective $k$ modules $P_1$ and
    $P_2$. 

\item $A_1 - \mod$ and $A_2- \mod$ are equivalent
  $k$-categories. 

\item $A_1 \approx \End_{A_2} (P)$ for some faithfully
         projective right $A_2$-module $P$. 
         \end{enumerate}         
       \end{propanddef}

         \begin{proof}%Proof
$(1) \Rightarrow (2)$.\pageoriginale $A_2 \otimes_k \End_k (P) \approx
           A_1 \otimes_k 
           A_2 \otimes_k A^o_2 \approx A_1 \otimes_k  \End_k (A_2)$. 

$(2) \Rightarrow (3)$. Since $A_i \otimes_k \End_k (P_i)
           \approx \End_{A_i}(A_i \otimes_k (P_i)$ (see (2.8) $(a)$),
         and since $A_i \otimes_k P_i$ is a faithfully projective
         $A_i$-module, it follows from theorem \ref{chap2:thm4.4} and
         proposition \ref{chap2:prop5.6} of chapter \ref{chap2}, that
         $A_i -  \mod$ is $k$-equivalent to $(A_i\otimes_k \End_k
         (P_i) - \mod$, $i=1, 2$.  

$(3) \Rightarrow (4)$ follows proposition \ref{chap2:prop4.1} and
         theorem \ref{chap2:thm4.4} of chapter \ref{chap2}. 
         
$(4) \Rightarrow (1)$. $A_1 \otimes_k A^0_2 \approx \End_{A_2}(P) 
\otimes_k A^o_2 \approx \End_{A_1 \otimes_k A^0_2} (P  \otimes_k
A^0_2)$. Now $P  \otimes_k A^0_2$ is faithfully projective $A_2
\otimes A^0_2$-module, so $P  \otimes_k A^0_2 \approx A_2 \otimes_k
Q$, where $Q =  (P  \otimes_k A^o_2)^{A_2}$ is faithfully projective
$k$-module. Hence $A_1 \otimes_k A^0_2 \approx \End_{A_2^e}(A_2
\otimes_k (Q)$, since $A_2 \otimes_k$: $k - \mod \to A^e_2 - \mod$ is
an equivalence.    
\end{proof}

It follows from this proposition that similarly is an equivalence
relation between azumaya algebras, and that $\otimes_k$ induces a
structure of abelian group on the set of similarity classes of azumaya
$k$-algebras. We shall call this  group the \textit{Brauer group of}
$k$, and hence denote it by $Br(k)$. The identity element in $Br(k)$
is the class of $k$, and the inverse of the class of an azumaya
$k$-algebra $A$ is the class of $A^0$. 
         
If $K$ is a commutative $k$-algebra, then $A \mapsto K \otimes_k A$
induces a homomorphism $Br(k) \to Br(K)$, by virtue of (2.9)(a), and
this makes $Br$ a functor from commutative rings to abelian groups. 
         

\section{Splitting rings}\label{chap3:sec5} %Sec 5
         
If\pageoriginale $P$ is a projective $k$-module, denote its rank by
$[P : k]$. This 
is a function spec $(k) \to \mathbb{Z}$. If $L$ is a commutative
$k$-algebra, denote by $\varphi_L$ the natural map spec $(L) \to $spec
$(k)$. Then, for a projective $k$-module $P$, we have $\varphi_L o
[P:k]= [P \otimes_k L : L]$. If  $A/k$ is an Azumaya algebra, denote
its class in $Br(k)$ by  $(A)$. 

\begin{theorem} %Thm 5.1
Let $A/k$ be an azumaya algebra.
\begin{enumerate}[(a)]
\item If $L \subset A$ is a maximal commutative subalgebra,
  then $A \otimes_k L  \approx \End_L (A)$ as $L$-algebras, viewing
  $A$ as a right $L$-module. Hence if  $A$ is $L$-projective, then
  $(A) \in \ker (BR(k) \to Br (L))$, and $\varphi_L o [A : k] = [A :
    L]^2$. If also $L$ is $k$-projective, then $\varphi_L o  [L :k] =
  [A : L]$. If $L/k$ is separable, $A$ is automatically
  $L$-projective. 

\item Suppose $L$ is a commutative faithfully $k$-projective
  $k$-algebra, and suppose $(A) \in \ker (Br (k) \to  Br (L))$. Then
  there is an algebra $B$, similar to $A$, which contains $L$ as a
  maximal commutative subalgebra. If $\End_k (L)$ is projective as a
  right $L$-module, then so is $B$. 
\end{enumerate}     
\end{theorem}

\begin{proof}
\begin{enumerate}[(a)]
\item $\End_L (A)$ is the centralizer $B$ in $\End_k (A)= A \otimes_k
  A^0$ of $l \otimes L \subset A \otimes A^0$. Since $A = A \otimes l
  \subset A \otimes L \subset B$, it follows from (4.3), that $B =A
  \otimes_k B^A$. Now $B^A \subset (A \otimes_k A^0)^A= l \otimes
  A^0$, and $B^A$ commutes with $l \otimes L$, a maximal commutative
  subalgebra of $l \otimes A^0$. Hence $B^A =l  \otimes L$, so $B = A
  \otimes_k L$, as claimed. 
  
If\pageoriginale $A$ is $L$-projective, then $\varphi_L \circ [A : k] = [A
  \otimes_k L:L]= [\End_L (A) : L]= [A : L]^2$. If, further, $L$ is
$k$-projective, $\varphi_L \circ [A : k] = [A \otimes_k L :  L]= [A
  \otimes_L (L \otimes_k L) : L] = [A :L]$.$[L \otimes_k L :L]=[A 
  : L]$. $(\varphi_L \circ [ L : k])$, so that $[A :L] = \varphi_L \circ [L
  : k]$.  
  
Suppose $E = (0 \to J \to L^e \to L \to 0)$ splits. Then $A \otimes_L
E$ also splits. So $A = A \otimes_L L$ is projective over $A \otimes_L
(L \otimes_k L^0) =  A \otimes_k L$. But $A \otimes_k L /L$ is an
azumaya algebra, so $A \otimes_k L$ is projective over $L$. Hence $A$
is $L$-projective. 

\item If $(A) \epsilon \ker (Br (k) \to Br(L))$, then $A^0 \otimes_k L
  \approx \End_L (P)$ for some faithfully projective $L$-module
  $P$. Using the isomorphism to identity, we have $A^o \otimes_k L =
  \End_L (P) \subset \End_k (P) =D$. Let $B=D^{A^o}$. Then (4.3) implies
  that $D=A^0 \otimes_k B$. Since $L$ is faithfully $k$-projective, so
  also is $P$. So $D /k$ is a trivial azumaya algebra and $(B) =(A)$
  in $Br (k)$. Clearly $L = l \otimes L \subset B$. Since $B
  =D^{A^o}$, $B^L = D^{A_o} \cap D^L$. Further, $D^L =\End_k (P)^L =
  \End_L (P)$, so $B^L = \End_L (P)^{A_o}$. Since $\End_L (P) = A^o
  \otimes_k L$, we have $\End_L (P)^{A_0}=$ centre $\End_L (P) = L$. Now
  $D=A^o \otimes_k B$ with $L \subset B$, so $D$ is locally (with
  respect to $k$) a direct sum of copies of $B$ as an 
  $L$-module. Hence $B$ is $L$-projective as soon as we show that $D
  = \End_k (P)$ is. Again we localize (with respect to $k$), whereupon
  $P$ becomes $L$-free and $I$ becomes $k$-free. Then We can write $P
  =P_0 \otimes_k L$, $P_0$ a free $k$-module, and we have $D= \End_k
  (P) = \End_k (P_0) \otimes_k \End_k (L)$. By hypothesis, $\End_k
  (L)$ is right $L$-projective, so $D$ is $L$-projective.  
\end{enumerate}
\end{proof}


\section{The exact sequence}\label{chap3:sec6} % Section 6
 
We\pageoriginale now make out of the azumaya $k$-algebras, a category
with product, in the sense of chapter \ref{chap1}. We write   
$$
\underset{=}{Az} = \underset{=}{Az}(k)
$$
for the category  whose objects are azumaya $k$-algebras, whose
morphisms are algebra isomorphisms, and with product $\otimes_k$. 
 
Recall that the category $\underline{\underline{FP}} =
\underline{\underline{FP}}(k)$ (see 
\S \ref{chap1:sec6} of chapter \ref{chap1}) of faithfully projective
$k$-modules also has $\otimes_k$ as product. More over, (2.12) says
that functor  
$$
\End= \End_k : \underset{=}{FP} \to \underset{=}{Az}
$$
preserves products. (If $f:P \to Q$ in $\underset{=}{FP}$, then $f$ is
an isomorphism, and End $(f):  \End (P) \to \End (Q)$ is defined by
End $(f) (e) = fef^{-1}$.) Theorem \ref{chap3:thm4.1} (6) asserts that End is a
cofinal functor, so we have the five term exact sequence from theorem
\ref{chap1:thm4.6} of chapter \ref{chap1}: 
\begin{equation*}
K_1 \underset{=}{FP} \xrightarrow{K_1 \End}  K_1 \underset{=}{Az} \to K_o 
\Phi \End \to K_c \underset{=}{FP} \xrightarrow{K_o \End} K_o
\underset{=}{Az}. \tag{6.1}\label{eq6.1} 
\end{equation*} 
 
It follows immediately from (4.10)(2), that 
\begin{equation*}
\text{ coker } (K_0 \End ) = Br (k). \tag{6.2}\label{eq6.2}
\end{equation*} 
 
Consider the composite functor
\begin{equation*}
\underline{\underline{\Pic}} \overset{I}\hookrightarrow
\underline{\underline{FP}} 
\xrightarrow{\End} \underline{\underline{Az}}, \tag{6.3}\label{eq6.3}  
\end{equation*} 
which sends every  object of $\underline{\underline{\Pic}}$ to the
algebra $k \in 
\underline{\underline{Az}}$. Hence the composites\pageoriginale ($K_i$
End) $\circ (K_i I) = 0$ for $i = 
0, 1$. We will now construct a connecting homomorphism $K_1
\underline{\underline{Az}} \to K_0 \underline{\underline{\Pic}} = \Pic
(k)$ and use it to identity (\ref{eq6.1}) with the sequence we will thus
obtain from (\ref{eq6.3}).  

Recall that $K_1\underline{\underline{Az}}$ is derived from the
category $\Omega\underline{\underline{Az}}$, whose objects are pairs
$(A, \alpha)$, $A \in \underline{\underline{Az}}$, $\alpha \in Aut_{
  k - alg}(A)$. Let ${}_1A_\alpha$ denote the invertible two-sided
$A$-module constructed in lemma \ref{chap2:lem7.2} of chapter
\ref{chap2}. We have $~_1 
A_\alpha \approx A \otimes_k L_\alpha$, where $L_\alpha = (~_1
A_\alpha)^A$, according to theorem (\ref{chap3:thm4.1})(4). In this
way we have a map   
$$
\obj \Omega \underline{\underline{Az}} \to \obj
\underline{\underline{\Pic}}, 
$$
given by $(A, \alpha) \mapsto L_\alpha = ({}_1 A_\alpha)^A$. If $f :
(A, \alpha) \to (B, \beta) $ is an isomorphism in $\Omega
\underline{\underline{Az}}$, then $f$ induces (by restriction) an
isomorphism $L_\alpha \to L_\beta$, thus extending the map above to a
functor. If $\alpha$, $\beta \in \Aut_{k-alg} (A)$, then we have
from (II, (7.2)(2)) a natural isomorphism  
$$
~_1 A_{\alpha \beta} \approx ~_1 A_\alpha \otimes_{A^1} A_\beta.  
$$
Since $A \otimes _k : k - \mod \to A - \mod$ converts $\otimes_k$ into
$\otimes_A$, it follows that $L_{ d \beta} \approx L_\alpha \otimes_k 
L_\beta$. Finally, given $(A, \alpha)$ and $(B, \beta)$, we have  
\begin{align*}
L_{\alpha \otimes \beta} & = (_1 ( A \otimes B)_{\alpha \otimes
  \beta})^{ A \otimes_k B} \\ 
& = (~_1 A_\alpha \otimes_{k ~ 1} B_\beta )^{ A \otimes_k B}\\
& = (_1 A_\alpha)^A \otimes_{k}(_1 B_\beta )^B \\
& = L_\alpha \otimes_k L_\beta. 
\end{align*}

We\pageoriginale have thus proved: 


\setcounter{prop}{3}
\begin{prop}% proposition 6.4
$( A, \alpha ) \mapsto L_\alpha = (_1 A_\alpha )^A $ defines a
  functor  
$$
J : \Omega \underline{\underline{Az}} \to \underline{\underline{\Pic}} 
$$ 
of categories with product, and it satisfies, for $\alpha$, $\beta \in
Aut_{k - \log} (A)$, $A \in \underline{\underline{Az}}$, 
$$
L_{\alpha \beta} \approx L_\alpha \otimes_k L_\beta. 
$$

Now we define a functor. 
$$
T : \underline{\underline{\Pic}} \to \Phi \End  
$$
by setting $T(L) = (L, \alpha _L, k)$, where $\alpha_L$ is the unique
$k$-algebra isomorphism $\End_k (L) \approx k \to \End_k (k)\approx
k$. Clearly $T$ preserves products.  

Suppose $(P, \alpha, Q) \in \Phi$ End. Thus $\alpha : A = \End (P) 
\to B = \End (Q)$ is an algebra isomorphism. $\alpha$ permits us to
view left $B$-modules as left $A$-modules. Since $P \otimes_k : k -
\mod A- \mod$ is an equivalence, the inverse being $\Hom_A (P, ~): 
A - \mod \to k - \mod$, we can apply this functor to the $B-$, hence
$A$-module, $Q$ and obtain a $k$-module  
$$
L = \Hom_A (P, Q)
$$
such that $Q \approx P \otimes_k L$ as a left $A$-module. It follows
that $L_\alpha \in \underline{\underline{\Pic}}$.\break If $(f, g)$: $(P,
\alpha, Q) \to (P', \alpha', Q')$ is in $\Phi \End$, then the map $\Hom
_{\End (P)}\break (P, Q) \to \Hom_{\End (P')}(P', Q')$, given by $e \mapsto
ge f^{-1}$, is in\pageoriginale $\underline{\underline{\Pic}}$, thus
giving us a functor  
$$
S  : \Phi \End \to \underline{\underline{\Pic}}. 
$$
Moreover, $S$ preserves products because 
\begin{align*}
& \Hom_{\End (P \otimes_k P')}(P \otimes_k P', Q \otimes_k Q' ) \\ 
& \approx \Hom_{\End (P) \otimes_k \End (P')}(P \otimes_k P', Q
  \otimes_k Q' ) \tag*{{\text by (2.12)}}\\ 
& \approx \Hom_{\End (P)} (P,Q) \otimes_k \Hom_{\End (P')} (P',Q)
\tag*{\text{by (2.5)(i).}}
.\end{align*} 

If $L \in \underline{\underline{\Pic}}$, then $STL = S(L, \alpha _L, L)
= \Hom_k (k, L) \approx L$.  
 \end{prop}
 
We have now proved all but the last statement of 

\begin{prop}% Proposition 6.5
There are product-preserving functors 
\[
\xymatrix@C=1.5cm{
\underline{\underline{\Pic}} 
 \ar@<0.6ex>[r]^T
& \Phi \End \ar@<0.6ex>[l]^S 
}
\]
defined by $T L = (k, \alpha_L, L)$ and $S(P, \alpha, Q) = \Hom_{\End
  (P)} (P, Q)$, such that $ST \approx
\Id_{\underline{\underline{\Pic}}}$. If $(P, \alpha, R) \in \Phi \End$,
then $S(P, \beta \alpha, R) \approx S (Q, \beta, R) \otimes_k S(P,
\alpha, Q)$.   
\end{prop}

\begin{proof}% proof
The prove the last statement we note that composition defines a homomorphism 
$$
\Hom_{\End (Q)} (Q, R) \otimes_k \Hom_{\End (P)} (P, Q) \to \Hom_{\End 
  (P)}(P,R).  
$$
The\pageoriginale module above are projective and finitely generated over
$k$. Therefore it is enough to check that the map is an isomorphism
over residue class fields $k / \mathscr{M}$.  
\end{proof}

\begin{prop}\label{chap3:prop6.6} %%% Proposition 6.6
$S$ and $T$ define inverse isomorphisms 
$$
\Pic (k) = K_0 \underline{\underline{\Pic}} \leftrightarrows K_0 \Phi
\underline{\underline{\End}}. 
$$
\end{prop}

\begin{proof}
$ST \approx \Id_{\underline{\underline{\Pic}}}$ so it suffices to show
  that $K_0 S$ is injective. If $K_0 S\break (P, \alpha, Q) = k$, then $Q
  \approx P \otimes_k k = P$ as a left $\End (P)$-module. Let $f : Q
  \to P$ be such an isomorphism. This means that for all $e
  \in \End(P)$ and $q \in Q$, $f (\alpha (e) q) = ef (q)$, that is,
  that   
\[
\xymatrix@=1.4cm{
\End(P) \ar[r]^{\alpha} \ar[d]_{\End (1_P)} & \End (Q) \ar[d]^{\End
  (f)}\\
\End(P) \ar[r]_{1_{\End (P)}} & \End (P)
}
\]
commutes. Thus $(1_P, f): (P, \alpha, Q) \to (P, 1 _{\End(p)}, P)$
in $\Phi \End$, so\break $(P, \alpha, Q)_{\Phi \End} = 0$ in $K_0 \Phi \End$.  
\end{proof}


\setcounter{theorem}{6}
\begin{theorem}\label{chap3:thm6.7}% theorem 6.7
The sequence of functors 
$$
\Omega \underline{\underline{\Pic}}\xrightarrow{\Omega I} \Omega
\underline{\underline{FP}} \xrightarrow{\Omega \End} \Omega
\underline{\underline{Az}}\xrightarrow{J}\underline{\underline{\Pic}}
\xrightarrow{I} \underline{\underline{FP}}
\xrightarrow{\End}\underline{\underline{Az}} 
$$ 
of categories with product defines an exact sequence which is the top
row of the following commutative diagram:  
{\fontsize{10pt}{12pt}\selectfont
\[
\xymatrix@C=.45cm{
U(k) \ar[r] & K_1 \underline{\underline{FP}} \ar[r] \ar@{=}[d] & K_1
\underline{\underline{Az}} \ar[r] \ar@{=}[d] & \Pic (k) \ar[r]
\ar@<0.6ex>[d]^{K_0 T} & K_0 \underline{\underline{FP}} \ar[r] \ar@{=}[d] & K_0
\underline{\underline{Az}}  \ar[r] \ar@{=}[d] & Br (k) \ar[r] & 0\\
& K_1 \underline{\underline{FP}} \ar[r] & K_1
\underline{\underline{Az}} \ar[r] & K_0 \Phi \End \ar[r]
\ar@<0.6ex>[u]^{K_0 S}
& K_0 \underline{\underline{FP}} \ar[r] & K_0
\underline{\underline{Az}} & & 
}
\]}\relax

 The bottom row is the exact sequence of theorem \ref{chap1:thm4.6} of
 chapter \ref{chap1} for 
 the functor End : $\underline{\underline{FP}} \to
 \underline{\underline{Az}} \cdot K_0 S$ and $K_0 T$ are the isomorphisms
 of proposition \ref{chap3:prop6.6}.  
\end{theorem}

\begin{proof}
We\pageoriginale first check commutativity: If $L
\in\underline{\underline{\Pic}}$, 
then $K_0 I (L)_{\underline{\underline{\Pic}}} =
(L)_{\underline{\underline{FP}}}$, while $K_0 T(L)_{
  \underline{\underline{\Pic}}} = (L, \alpha_L, L)_{\Phi \End }$ is sent
to $(L)_{\underline{\underline{FP}}}
(k)^{-1}_{\underline{\underline{FP}}} =
(L)_{\underline{\underline{FP}}}$.  
\end{proof}

Now that the diagram commutes, exactness of the top row follows from that of
the bottom row, wherever the isomorphisms imply this. At $K_0
\underline{\underline{Az}}$ and $Br (k)$ exactness has already been
remarked in (\ref{eq6.2}) above. At $K_1 \underline{\underline{FP}}$ the
composite is clearly trivial. So it remains only to show that $\ker(K_1
\End) \subset \Iim K_1 I$.  

Now we know that the free modules $k^n$ are cofinal in
$\underline{\underline{FP}}$, and hence (by cofinality of $\End$) that
the matrix algebras $\mathbb{M}_n (k) = \End(k^n)$ are cofinal in
$\underline{\underline{Az}}$. Hence we may use them to compute $K_1$
as a direct limit (theorem \ref{chap1:thm3.1} of chapter \ref{chap1}).  
  
Write $GL_n (\underline{\underline{FP}}) = GL(n, k) = Aut_k (k^n)$,
and $GL_n(\underline{\underline{Az}}) = Aut _{ k -alg}\break (\mathbb{M}_n
(k))$. We have the ``inner automorphism homomorphism''  
$$
f_n : GL_n (\underline{\underline{FP}} \to GL_n
(\underline{\underline{Az}}) 
$$
with ker $f_n = $ centre $GL_n (\underline{\underline{FP}} = GL_1
(\underline{\underline{FP}}) = U(k))$, and $\im f_n \approx PGL\break (n,
k)$.  

Tensoring with an identity automorphism defines maps  
$$
GL_n (\underline{\underline{FP}})  \to GL_{nm}
(\underline{\underline{FP}}) 
$$
and\pageoriginale
$$
GL_n (\underline{\underline{Az}}) \to GL_{nm}
(\underline{\underline{Az}}) 
$$
making $(f_n)_{n \in \mathbb{N}}$ a map of directed systems. Writing  
\begin{align*}
GL ( \underline{\underline{FP}}) & = \lim_\to GL_{n}
(\underline{\underline{FP}}),\\ 
GL ( \underline{\underline{Az}}) & = \lim_\to GL_{n}
(\underline{\underline{Az}}), 
\end{align*}
and 
$$
f = \lim_{ \to } f_n : GL(\underline{\underline{FP}}) \to
GL(\underline{\underline{Az}}),  
$$
we see that $K_1$ End is just the ablianization of $f$. In \S
\ref{chap1:sec6} of chapter \ref{chap1} we computed  
\begin{align*}
K_1 \underline{\underline{FP}} & = GL(\underline{\underline{FP}})/
[GL(\underline{\underline{FP}}), GL(\underline{\underline{FP}})] \\ 
& = (\mathbb{Q} \otimes_{ \mathbb{Z}} U(k)) \oplus (\mathbb{Q}
\otimes_{\mathbb{Z}} SK_1 \underline{\underline{P}}).   
\end{align*}
Moreover $K_1I$ is induced by the inclusion $U(k) = GL_1
(\underline{\underline{FP}}) \to GL (\underline{\underline{FP}})$. $K_1
I$ is the map $K_1 \underline{\underline{\Pic}} = U(k) -
\mathbb{Z}\otimes_\mathbb{Z} U(k) \to \mathbb{Q} \otimes_{\mathbb{Z}}
U(k) \subset K_1 \underline{\underline{FP}}$, so coker $(K_1 I) =
(\mathbb{Q} / \mathbb{Z} \otimes_{\mathbb{Z}} U (k)) \otimes ( \mathbb{Q}
\otimes_\mathbb{Z} SK_1 P = PGL(k)/ [ PGL (k), PGL(k)]$. Thus
exactness at $K_1 \underline{\underline{FP}}$ means that: the
inclusion $PGL(k) \subset GL(\underline{\underline{Az}})$ induces a
monomorphism 
$$ 
PGL(k) / [PGL (k), PGL (k)] \to GL (\underline{\underline{Az}}) / [
  GL(\underline{\underline{Az}}), GL(\underline{\underline{Az}})].  
$$ 

Suppose $\alpha$, $\beta \in GL_n (\underline{\underline{Az}}) = Aut_{
  k - alg} (\mathbb{M}_n (k))$. Let $\tau : k^n \otimes k^n \to k^n
\to k^n$ be the transposition. Write $E(\tau)$ for the
corresponding\pageoriginale 
inner automorphism of $\mathbb{M}_{N^2} (K)$. Then $E(\tau) (\alpha
\otimes 1_{\mathbb{M}_n (k)}) E(\tau)^{-1} = 1_{\mathbb{M}_n (k)}
\otimes \alpha$ commutes with $\beta \otimes 1_{\mathbb{M}_n
  (k)}$. Now $\tau$ is just a permutation of the basis of $k^n
\otimes_k  k^n $, so it is a product of elementary matrices, provided
it has determinant $+ 1$ (which happens when $\dfrac{1}{2}n (n - 1)$
is even). For example, if we restrict our attention to values of $n$
divisible by 4, then $\tau$ is a product of elementary matrices,
hence lies in $[ GL_{n^2},  GL_{n^2}(k)]$, so $E (\tau) \in [
  PGL_{n^2} (k), PGL_{n^2}(k)]$. It follows that, for $n$ divisible by
4, the image of $GL_n(\underline{\underline{Az}})$ in
$GL_{n^2}(\underline{\underline{Az}}) / [ PGL_{n^2} (k),
  PGL_{n^2}(k)]$ is abelian. Note that $PGL_{n^2} (k ) =$ In Aut
$(\mathbb{M}_n(k))$ is normal in $GL_n(\underline{\underline{Az}})$,
hence so also is $[ PGL_n^2 (k), PGL_n^2 (k)]$, so the factor group
above is defined. Finally, since the $n$ divisible by 4 are cofinal
$\mathbb{N}$ we can pass to the limit to obtain  
$$
[ GL(\underline{\underline{Az}}), GL(\underline{\underline{Az}})]
\subset [ PGL(k), PGL(k)],  
$$
are required.  \hfill{Q.E.D.}

\setcounter{prop}{7}
\begin{prop}% Proposition 6. 8
In the exact sequence of theorem \ref{chap3:thm6.7}, 
\begin{gather*}
\ker (U(k) \to K_1 \underline{\underline{FP}} ) = \text{ the
  torsion subgroup of } U(k), \\ 
\ker (\Pic(k) \to K_0 \underline{\underline{FP}} ) = \text{ the
  torsion subgroup of } \Pic(k),  
\end{gather*}
Hence\pageoriginale there is an exact sequence 
$$
 0 \to (\mathbb{Q} / \mathbb{Z} \otimes_{\mathbb{Z}} U(k))
  \oplus (\mathbb{Q} \otimes_{\mathbb{Z}} SK_1 
 \underline{\underline{P}}) \to K_1 \underline{\underline{Az}} \to
 \left({\substack{\text{the torsion} \\ \text{subgroup
     of} \\ \Pic (k)}}\right) \to 0. 
 $$
  This sequence splits (not naturally) as sequence of abelian groups.  
 \end{prop}

\begin{proof}
For $K_1 I : U(k) = K_1 \underline{\underline{\Pic}} \to K_1
\underline{\underline{FP}}$ we have from chapter \ref{chap1}, \S
\ref{chap1:sec7},   
$$
\ker K_1 I = \text{ the torsion subgroup of } U(k),
$$
and
$$
\text{coker } K_1 I = (\mathbb{Q} / \mathbb{Z} \otimes_{\mathbb{Z}}
U(k)) \oplus (\mathbb{Q} \otimes_{\mathbb{Z}} SK_1
\underline{\underline{P}}). 
$$
From the same source we have 

ker $K_0 I$ =  the torsion subgroup of  $\Pic (k)$. 

The last assertions now follow from the exact sequence plus the fact
that the left hand term is divisible, hence an injective
$\mathbb{Z}$-module.  
\end{proof}

