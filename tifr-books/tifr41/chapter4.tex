\chapter{The Brauer-Wall group of graded Azumaya
  algebras}\label{chap4} %Chapter 4  

This\pageoriginale chapter contains only a summary of results, without
proofs. They are included because of their relevance to the following
chapter, on Clifford algebras.   

\section{Graded rings and modules}\label{chap4:sec1} %section 1

All graded objects here are graded by $\mathbb{Z} / 2 \mathbb{Z }$. A
ring $A = A_0 \oplus A_1$ is graded if $A_i A_j \subset A_{ i +j} (i,
j \in \mathscr{Z} / 2 \mathscr{Z})$, and an $A$-module $M = M_0 \oplus
M_1$ is graded if $A_i M_j \subset M_{i + j}$. (We always assume
modules to be left modules unless otherwise specified.) If $S$ is a
subset of a graded object, $hS$ will denote the homogeneous elements
of $S$, and $\partial x = $ degree of $x$, for $x \in h S$.  

If $A$ is a graded ring, then $|A|$ will denote the underlying
ungraded ring. If $A$ is ungraded, then $(A)$ denotes the graded ring
with $A$ concentrated in degree zero. An $A$-module is graded or not
according as $A$ is or is not. If $M$ is an $A$-module (A graded) we
write $|M|$ for the underlying $|A|$-module. If $A$ is not graded, we
write $(M)$ for the $(A)$-module with $M$ concentrated in degree zero.  

Let $A$ be a graded ring. For $A$-modules $M$ and $N$, 
$$
{\rm HOM}_A (M, N)
$$
is the graded group of additive maps from $M$ to $N$ defined by: $f
\in h HOM_A (M, N) \Leftrightarrow (i) f$ is homogeneous of degree
$\partial f$ (i.e. $f(M_i) \subset N_{ i + \partial f})$; and (ii) fax
= $(-1)^{\partial f \partial a} a fx (a \in hA, x \in M)$.  

The\pageoriginale degree zero term of ${\rm HOM}_A (M, N)$ is denoted by
${\rm Hom}_A (M, N)$.  

Let $A'$ denote the graded group $A$ with new multiplication  
$$
a \cdot b = (- 1)^{\partial a \partial b} ab \qquad (a, b, \in hA). 
$$
If $M$ is an $A$-module let $M'$ denote the $A'$-module with $M$ as
the underlying graded group and operators defined as  
$$
a \cdot  x = (-1)^{\partial a \partial x} ax \qquad (a \in hA, x \in h M). 
$$
Then it is straightforward to verify that 
$$
{\rm HOM}_A (M, N) = \Hom_{|A'|}(|M'|, |N'|), 
$$
an equality of graded groups. 

$A$-mod refers to the category with $A$-modules as objects and
homomorphisms of degree zero (i.e. $ \Hom _A (, )$) as morphisms.  

\begin{lemma}%lemm .1.1
The following conditions on an $A-$module $P$ are
equivalent:  
\begin{enumerate}[(1)]
\item $\Hom_A (P, ~)$ is exact on $A-\mod$.

\item $\Hom_{|A|}(|P|, ~)$ is exact on $|A|-\mod$. 

\item $\Hom_A(P, ~)$ is exact on $A-\mod$. 

\item $P$ is a direct summand of $A^{(I)} \oplus (\tau A)^{(J)}$ for
  some $I$ and $J$, where $\tau A$ is the $A$-module $A$ with grading
  shifted by one.  
\end{enumerate}
\end{lemma}

This lemma tells us that the statement ``$P$ is $A$-projective'' is
unambiguous.  

If\pageoriginale $S \subset A$ (graded), we define the
\textit{centralizer} of $S$ 
in $A$ to be graded subgroup $C$ such that $c \in h C \Leftrightarrow
cs = (-1)^{ \partial c \partial s}sc$ for all $s \in h S$. It is
easy to see that $C$ is actually a subring of $A$. We say that two
subrings of $A$ \textit{commute}, if each lies in the centralizer of the
other. If $B_1$ and $B_2$ are subrings generated by sets $S_1$ and
$S_2$, respectively, of homogeneous elements, then $B_1$ and $B_2$
commute $\Leftrightarrow S_1$ and $S_2$ commute. We write  

$A^A =$ CENTRE $(A) =$ centralizer of $A$ in $A$. 

\noindent
The degree zero term will be denote centre $A$. One must not confuse
centre $A$, centre $|A|$, and CENTRE $(A)$. They are all distinct in
general.  

Let $k$ be a commutative ring which is graded, but concentrated in
degree zero. Even though $k$ and $|k|$ are not essentially different,
$k - \mod$ and $|k|$-mod are. $A $ $k$-algebra is a graded ring $A$
and a homomorphism 
$k \to A$ of graded rings such that the image $k 1$ lies in $A^A$, and
hence in centre $A$.  

If $A^1$ and $A^2$ are $k$-algebras and if $M^i$ is an $A^i$-module $i
= 1, 2$ we define the $k$-module $M^1 \otimes M^2$ by  
$$
(M^1 \otimes M^2)_n = (M^1_0 \otimes M^2_n ) \oplus (M^1_1 \otimes
M^2_{n +1}).  
$$
We define an action of $A^1 \otimes A^2$ on $M^1 \otimes M^2$ by  
$$
(a_1 \otimes a_2) (x_1 \otimes x_2) = (-1)^{ \partial a_2 \partial
  x_1} a_1 x_1 \otimes a_2 x_2 
$$
for $a_i \in h A^i $, $x_i \in hM^i$, $ i = 1, 2$. This makes $A^1
\otimes A^2$ a $k$-algebra (for $M^i = A^i$) and $M^1 \otimes M^2$ an
$A^1 \otimes A^2$-module. The subalgebras $A^1 \otimes 1$
and\pageoriginale $1 
\otimes A^2$ commute, and the pair of homomorphisms $A^i \to A^1
\otimes A^2$ is universal for pairs of homomorphisms $f^i : A^i \to B$
of $k$-algebras such that im $f^1$ and im $f^2$ commute. In practice
it is useful to observe that: if $A^i$ is generated by homogeneous
elements $S^i$, and if $f^1 (S^1)$ and $f^2 (S^2)$ commute, then $f^1
(A^1)$ and $f^2 (A^2)$ commute.  


\section{Separable algebras}\label{chap4:sec2}% section 2

In this section also $k$ denotes a commutative ring concentrated in degree
zero. If $A$ is a $k$-algebra, then $A^0$ denotes the opposite
algebra, and we write  
$$
A^* = (A ' )^0 = (A^0)'
$$
for the algebra with graded group $A$ and multiplication 
$$
a \times b = (-1) ^{ \partial a \partial b_{ba}} \qquad (a, b \in hA).  
$$
A right $A$-module $M$ will be considered a left $A^*$-module by
setting  
$$
ax = (-1)^{\partial a \partial x}xa \qquad (a \in h A^*, x \in h M). 
$$
If $M$ is a left $A$-, right $B$-module such that $(ax)b = a (xb)$ and
$tx = xt$ for all $a \in A$, $x \in M$, $b \in B$, $t \in k$, then we
view $M$ as an $A \otimes_k B^*$-module by  
$$
(a \otimes b) x = (-1)^{\partial b \partial x } axb \qquad (a \in 
hA, b \in h B, x \in hM).  
$$
In particular, two-sided $A$-modules will be identified with modules 
over  
$$
A^e = A \otimes_k A^*
$$
We\pageoriginale have an exact sequence 
\begin{gather*}
0 \to J \to A^e \to A \to 0\\
(a\otimes b)  \mapsto ab 
\end{gather*}
of $A^e$-modules. We call $A$ a \textit{separable} $k$-algebra if $A$
is $A^e$-projective. This means that the functor  
\begin{gather*}
A^e - {\rm mod} \to k - {\rm mod} \\ 
M \mapsto M^A = {\rm HOM}_{A^e} (A, M)
\end{gather*}
is exact. 

The stability of separability and CENTRES under base change and tensor
products all hold essentially as in the ungraded case. In particular
${\rm END}_k (P) = {\rm HOM}_k (P, P)$ is separable with CENTRE $k /
ann P$, for $P$ a finitely generated projective $k$-module. Moreover :   

\begin{prop}% proposition 2.1
Let $A$ be finitely generated as a $k-$module and suppose either that
$k$ is noetherian or that $A$ is $k-$projective. Then $A$ is
separable $\Leftrightarrow (A / \mathscr{M} A) / (k / \mathscr{M})$
is separable for all maximal ideals $\mathscr{M}$ of $k$.  
\end{prop}

Suppose now that $k$ is a field. If $a \in k$. write $k<a> = k[X] /
(X^2- a)$, with grading $k. 1 \oplus k \cdot x$, $x^2 = a$. It can be
shown that if char $k \neq 2$ and if $a \neq 0$, then $k < a > $ is 
separable with CENTRE $k$. Moreover  
$$
k < a > \otimes_k k < b > \approx \left( \frac{ a, b}{k}\right),  
$$
the $k$-algebra with generators $\alpha$, $\beta$ of degree one
defined by relations: $\alpha^2 = a$, $\beta^2 = b$, $\alpha\beta 
 = - \beta \alpha$. 
 
\setcounter{theorem}{1}
\begin{theorem}% theorem 2.2
Let\pageoriginale $A$ be a finite dimensional $k$-algebra, $k$ a
field. Then the following conditions are equivalent:  
\begin{enumerate}[(1)]
\item $A/k$ is separable. 

\item $A = \Pi A_i$, where $A_i$ is a simple (graded) $k$-algebra and
  $A^{A^i}_i$ is a separable field extension of $k$, concentrated in
  degree zero.  

\item For some algebraically closed field $L \supset k$, $L \otimes_k A$
  is a product of algebras of the types  

\item[(i)] $END_L(P)$, $P$ a finite dimensional $L$-module, and  

\item[(ii)] $L < 1 > \otimes _L {\rm END}_L (P)$, $P$ a finite dimensional
  $L$-module with $P_1 = 0$.  
\end{enumerate}
If char $k = 2$, then type (ii) does not occur. 
\end{theorem}

\setcounter{coro}{2}
\begin{coro}% corollary 2.3
Let $k$ be any commutative ring and $A$ a $k$-algebra finitely
generated as a $k$-module. Suppose either $k$ is noetherian or that
$A$ is $k$-projective. Then if $A/k$ is separable, $|A|$, $|A_0|$,
$|A^A|$, and $|A^{A_0}|$ are separable $|k|$-algebras. 
\end{coro}


\section{The group of quadratic extensions}\label{chap4:sec3} %section 3

A \textit{quadratic extension} of $k$ is a separable $k$-algebra $L$
which is a finitely generated projective $k$-module of rank 2. By
localizing and extending 1 to a $k$-basis of $L$ we see that $|L|$
is commutative. 
 
\begin{prop}% Proposition 3. 1
If $L/k$ is a quadratic extension, then there is a unique $k$-algebra
automorphism $\sigma = \sigma (L)$ of $L$ such that $L^\sigma = k$.  
\end{prop}

\begin{prop}% proposition 3.2
If\pageoriginale $L^1$ and $L^2$ are quadratic extensions of $k$, then
so also is  
$$
L^1 \ast L^2 = (L^1 \otimes_k L^2)^{\sigma_1 \otimes \sigma_2},  
$$
where $\sigma _i = \sigma (L^i)$. Further, $*$ induces on the
isomorphism classes of quadratic extensions the structure of an abelian
group,  
$$
Q_2(k). 
$$
\end{prop}

If we deal with $|k|$-algebras, then we obtain a similar group, 
$$
Q(k)
$$
of ungraded quadratic extensions. Each of these can be viewed as a
graded quadratic extension of $k$, concentrated in degree zero, and
this defines an exact sequence  
$$
0 \to Q(k) \to Q_2 (k) \to T, 
$$
where $T=$ continuous functions spec $(k) \to \mathbb{Z} / 2
\mathbb{Z}$, and right hand map is induced by $L  \mapsto [ L_1 : k] $
= the rank of the degree one term, $L_1$, of $L$. In particular, if
Spec $(k)$ is connected, we have  
$$
0 \to Q (k) \to Q_2 (k) \to \mathbb{Z} / 2 \mathbb{Z}, 
$$ 
and the right hand map is surjective $\Leftrightarrow 2 \in U (k)$. In
this case $L = k < u >$ is a quadratic extension for $u \in U (k)$. $L
= k \cdot 1 \oplus k \cdot x$ with $x^2 = u$, and $\sigma (x) = - x $
for $\sigma = \sigma (L)$. If $u_1$, $u_2 \in U(k)$, then $k < u_1 >
\otimes_k k < u_2 > = \left(\dfrac{u_1, u_2}{k}\right)$ has $k$-basis
$1$, $x_1$, $x_2$, $x_3 = x_1 x_2 = - x_2 x_1$ with $x^2_1 = u _1$,
$x^2_2 = u _2$, $x^2_3 = -u _1u _2$. If $\sigma _i = \sigma (k
<u_i>)$, then $\sigma_1 \otimes \sigma_2$ sends $x_1 \mapsto - x_1$,
$x_2 \mapsto - x_2$, $x_3 \mapsto x_3$.  

It\pageoriginale follows that 
$$
k < u_1 > \ast k < u_2 > = k[-u_1 u_2],
$$
where $k[u]= k[X]/ (X^2-u)$, concentrated in degree zero.

\begin{prop}% proposition 3.3
 Suppose $2 \in U (k)$. Then
\begin{enumerate}[(a)]
\item the sequence $0 \to Q (k) \to Q_2 (k) \to \mathbb{Z}/2
  \mathbb{Z} \to 0$ is exact; and  

\item there is an exact sequence 
$$
U(k) \xrightarrow{2} U(k) \to Q(k) \to \Pic (k) \xrightarrow{2} \Pic
(k), 
$$
where the map in the middle are defined by $u \mapsto k[u]$ and $L
\mapsto (L/k)$, respectively. 
\end{enumerate}
\end{prop}

Next suppose that char $k = 2$.

\begin{prop}%proposition 3.4
Suppose $k$ is a commutative ring of characteristic 2. Then if $a
\in k$, $k[a]= k[X]/(X^2 + X+a)$ is a quadratic extension,
concentrated in degree zero. $k[a]=k \cdot 1 + k \cdot x $ with $x^2 + x + a =
0$, and $\sigma (x) = x+1$. $Q_2 (k) \approx Q[k]$ and there is an
exact sequence  
$$
k \xrightarrow{\wp}k \to Q (k) \to 0,
$$
where $\wp (a)=a^2+a$, and $a \mapsto k[a]$ induces $k \to Q[k]$.
\end{prop}


\section{Azumaya algebras}\label{chap4:sec4}% section 4
$k$ is a commutative ring concentrated in degree zero. 

\begin{thmanddef}\label{chap4:thm4.1}% theorem and definition 4.1
$A$ is an  azumaya $k$- algebra if it satisfies the
   following conditions, which are equivalent: 
\begin{enumerate}[(1)]
\item $A$ is a finitely generated $k$- module, $A^A= k$, and
  $A/k$ is separable. 

\item $A^A = k$ and  $|A|$ is a generator as an $|A^e|$-module.

\item $A$ is\pageoriginale a faithfully projective $k$-module and $A^e
  \to END_k (A)$ is an isomorphism. 

\item The functors
\[
\xymatrix@R=-1.3cm{
 (A \otimes_k N) &  N \ar@{|->}[l]) \\
 A^e -{\rm mod} \ar@<1ex>[r] &  k-{\rm mod} \ar@<1ex>[l]\\
 (M  \ar@{|->}[r] & M^A )
}
\]
are inverse equivalences of categories.

\item For all maximal ideals $\mathscr{M} \subset k$, $A/
  \mathscr{M} A$ is simple, and CENTRE $ (A/ \mathscr{M} A)= k/
  \mathscr{M}$. 

\item There exists a $k$-algebra $B$ and a faithfully projective
  $k$-module $P$ such that $A \otimes_k B \approx END_k (P)$. 
\end{enumerate}
\end{thmanddef}

\setcounter{coro}{1}
\begin{coro}% corollary 4.2
Let $A$ and $B$ be $k$-algebras with $A$ azumaya. Then
  $\mathfrak{b} \mapsto A \otimes \mathfrak{b}$ is a bijection from
  two-sided ideals of $B$ to those of $A \otimes_k B$. 
\end{coro}

\begin{coro}% corollary 4.3
If $A \subset B$ are $k$-algebras with $A$ azumaya, then $B
  \approx A \otimes_k B^A$. 
\end{coro}

Call two azumaya algebras $A$ and $B$ \textit{similar} if $A
\otimes_k  B^* \approx END_K (P)$ for some faithfully projective
module $P$. With multiplication induced by $\otimes_k$, the similarity
classes form a group, denoted 
$$
Br_2 (k), 
$$
and called the \textit{Brauer-Wall group} of $k$.

\setcounter{theorem}{3}
\begin{theorem}% theorem 4.4
If $A$ is an azumaya algebra, define $ L(A)= A^{A_0}$. Then $L
  (A)$ is a quadratic extension of $k$, and $L (A \otimes_k B) 
  \approx L(A) \ast L(B)$. $A \mapsto L(A)$ induces an exact sequence  
$$
0 \to Br (k) \to Br_2 (k) \to Q_2 (k) \to 0.
$$
\end{theorem}


\section{Automorphisms}\label{chap4:sec5}%Section 5

If\pageoriginale $A$ is a $k$-algebra $a=a_0 + a_1 \in A$
write $\sigma(a) 
= a' = a_0 - a_1$. Let $U(A)$ denote the group of units in $A$, and
$hU(A)$ the subgroup of homogeneous units. If $u \in hU(A)$, we define
the \textit{inner automorphism}, $\alpha_u$, by   
$$
\alpha_u (a) = u \sigma^{\partial u}(a) u^{-1}.  
$$
This is clearly an algebra automorphism of $A$, and a simple
calculation shows that $\alpha_{uv}= \alpha_u \alpha_v$. Thus we have
a homomorphism  
$$
hU(A) \to Aut_{k-alg}(A) ; u \mapsto \alpha_u. 
$$ 
The kernel consists of those $u$ such that $u \sigma^{\partial u}(a) =
au $ for all $a \in A$. Taking a homogeneous, $\sigma (a) =
(-1)^{\partial a}a$, so the condition becomes $(-1)^{\partial u
  \partial a} ua = au$, for all $a \in hA$, i.e. $u
\in$  CENTRE$(A)=A^A$.  

Thus we have an exact sequence 
\begin{equation*}
1 \to hU (A^A)\to hU(A) \to Aut_{k-alg} (A).\tag{5.1}\label{eq5.1}
\end{equation*}
Now just as in the ungraded case one can prove:

\setcounter{theorem}{1}
\begin{theorem}\label{chap4:thm5.2}% theorem 5.2
Let $A$ be an azumaya $k$-algebra. If $\alpha \in
Aut_{k-alg}(A)$, let $1^A\alpha$ denote the $A^e$-module $A$ with
action $a \cdot x \cdot b= ax \alpha (b)$ for $a,x,b \in A$. Then $L_\alpha =
(_{1}A_\alpha)^A$ is an invertible $k$-module, and $\alpha \mapsto
L_\alpha$ induces a homomorphism $g$ making the sequence $1 \to U(k)
\to U(A_0) \to Aut_{k-alg}(A) \xrightarrow {g} \Pic (k)$		 
exact. $\im  g = \{(L) | A \otimes_k L \approx A$ as left $A$-modules\}
\end{theorem}

Here\pageoriginale $U(k) = U(A^A) \subset U(A_0) \subset hU(A)$, and
the left hand 
portion of the sequence is induced by (\ref{eq5.1}) above. Pic $(k)$ is the
group of  ``graded invertible $k$-modules''. If $u$ is a unit of
degree one in $A$ then $L_{\alpha_u}$ is just $k$, but concentrated in
degree one. This explains why we have $U(A_0)$, and not $hU(A)$, in
the exact sequence.  

This theorem will be applied, in Chapter \ref{chap5}, \S
\ref{chap5:sec4}, to the study of 
orthogonal groups. We conclude with the following corollary:   

\setcounter{coro}{2}
\begin{coro}% corollary 5.3
 $Aut_{k-alg}(A)/$ (inner automorphisms) is a group of exponent $r^d$
  for some $d>0$, where $r=[A:k]$. 
\end{coro}


