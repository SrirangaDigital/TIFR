\thispagestyle{empty}
\phantom{a}
\begin{center}
{\Large\bf Lectures on Modular Functions of\\[8pt] One Complex Variable}
\vskip 1cm

{\bf By}
\medskip

{\large\bf H. Maass}
\vfill

{\bf Notes by}
\medskip

{\large\bf Sunder Lal}
\vfill

{\bf Tata Institute Of Fundamental Research}

{\bf Bombay}

{\bf 1964}

{\bf (Revised 1983)}

\end{center}
\eject

~\thispagestyle{empty}
\vfill

\begin{center}
Dedicated to\\[0.75cm]
{\large\bfseries Hans Petersson}
\end{center}

\vfill
\eject
\thispagestyle{empty}

\begin{center}
{\bf Author}
\medskip

{\large\bf H. Mass}

Am Pferchelhang 5

D-6900 Heidelberg 

Federal Republic of Germany
\vfill

\copyright Tata Institute Of Fundamental Research, 1983
\vfill 


\parbox{0.7\textwidth}{No part of this book may be reproduced in any 
form by print, microfilm or any other means without written permission
from the Tata Institute of Fundamental Research, Colaba, Bombay 400 005}
\vskip 1cm    

\parbox{0.7\textwidth}{Printed by M. N. Joshi at The Book Centre Limited,
Sion East, Bombay 400 022 and published by the Tata Institute of Fundamental Research,
Bombay}
\vfill

\centerline{{\bf Printed In India}}
\end{center} 

\eject
\thispagestyle{empty}
\chapter*{Preface To The Revised Edition}


Thanks are due to the Editor of the Tata Institute Lecture Notes in
Mathematics for the suggestion to reissue my Lecture Notes of 1963,
which provided an opportunity to touch up the text with improvements
in several respects. In collaboration with Professor S. Raghavan I
could expand the Errata from its original four pages to sixteen
pages. He took upon himself the task of inserting the necessary
modifications in the notes. A truly onerous undertaking! I am indebted
to Professor E. Grosswald for valuable proposals in this
connection. We have now a substantially revised version which we hope,
is much more readable. However, what is perfect in this world! I
therefore seek the indulgence of the reader for any possible error in
the revised text.

Finally mention should be made of Dr. C.M. Byrne, Mathematics
Editor of Springer-Verlag, for her encouraging support at crucial
stages of the task and for making me occasionally forget, in a
charming manner, that decisions of the publishers are based on prosaic
calculations.

It is a pleasure to thank everyone who has been involved in this
project.
\vskip 1cm

\noindent
Heidelberg
\medskip

\noindent
November 15, 1983\hfill {\large\bf H. Maass}\qquad 




\vfill
\eject

\thispagestyle{empty}


\chapter*{Preface To The First Edition}


These are notes of lectures which I gave at the Tata Institute of
Fundamental Research in 1962/63. They provide an introduction to the
theory of modular functions and modular forms and may be described as
elementary, in as much as basic facts from the theory of functions of
a complex variable and some properties of the elementary
transcendental functions form the only prerequisites. (It must be
added that I have counted the Whittaker functions among the elementary
transcendental functions). It seemed to me that the investigations of
Siegel on discrete groups of motions of the hyperbolic plane with a
fundamental region of finite volume form a particularly suitable
introduction, since they make possible a simple characterization of
groups conjugate to the modular group by a minimal condition.

My thanks are due to Mr.~Sunder Lal for his careful preparation of
these notes.
\vskip 1cm

\hfill {\large\bf Hans Maass}\qquad





