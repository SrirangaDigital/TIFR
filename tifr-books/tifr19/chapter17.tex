
\chapter{Convergence Theorems}\label{p4:chap8}%chap VIII

\setcounter{section}{33}
\section{Domination Principle}\label{p4:chap8:sec34}%sec 34

We\pageoriginale suppose only the
axioms \ref{p4:chap1:sec1:axiom1}, \ref{p4:chap1:sec1:axiom2}
and \ref{p4:chap1:sec1:axiom3} on the space $\Omega$.  

\begin{defn}\label{p4:chap8:sec34:def22}%defi 22
  Let $v \ge 0$ be a superharmonic function in $\Omega$.
\end{defn}

\begin{enumerate}[(a)]
\item the support of $v$ is the complement of the largest open set in
  which $v$ is harmonic; 
\item the best harmonic minorant of $v$ in an open set $\omega \subset
  \Omega$ is $R^{C \omega}_v$ or $\bar{H}^\omega_v$. 
\end{enumerate}

We shall now introduce a new axiom and prove some equivalent forms of
the same. 

\noindent
\textit{Axiom $D$} (form $\alpha$) : If $v$ is a potential in
$\Omega$, which is locally bounded in $\Omega$ and harmonic in a
domain $\delta \subset \bar{\delta} \subset \Omega$; any other potential which
is $\ge  v$ on $C \delta$ is $\ge  v $ on $\delta$ as well. 

\noindent
(Equivalent forms) ($\beta$): For any relatively compact open set
$\omega \subset \Omega$, and any superharmonic function $\ge 0$ which
is locally bounded in $\Omega$, the best harmonic minorant in $\omega$
is equal to the greatest one. 

($\alpha$) (Domination Principle): If $v$ is any locally bounded
potential $\ge 0$, for any superharmonic function $w \ge 0$, the fact
that $w \ge v$ on the support of $v$, implies $w \ge v$ everywhere. 

Let us assume the existence of a potential $> 0$ (otherwise $\alpha,
\beta$ and $\gamma$ are trivial). Let us prove the equivalences. First
of all $\beta$ and $\gamma$ imply $\alpha$. 

Let\pageoriginale us suppose $(\alpha)$ and prove $(\beta)$. We may assume in the
hypothesis of $(\beta)$ that $v$ is a potential. Let us introduce a
relatively compact open set $\omega_o$ in $\Omega$ such that
$\bar{\omega} \subset \omega_o$ and observe that
$\hat{R}^{\omega_o}_v$ is a potential with the same harmonic majorant
as $v$ in $\omega$. We may also suppose $\omega$ to be a domain
because of the properties of the minorants involved. 

We observe that the greatest harmonic minorant of $v$ in $\omega$ is
inf. $R_v^{C \delta}$ for all $\delta \subset \bar{\delta} \subset
\omega$, and its continuation by $v$ is a nearly superharmonic
function $V \le v$. Let $w$ be any superharmonic function $\ge 0$ in
$\Omega$, and $w \ge v$ on $C \omega$. Then $w_1 = \inf (v, w)$ is
superharmonic $\ge 0$ and $\le v. ~ w_1$ is a potential $ \ge \hat{V}$
on $C \omega$, therefore $R^{C \omega}_v \ge \hat{V}$ on
$\omega$. Hence $R^{C \omega }_v = V$ on $\omega$. 

Let us suppose $\alpha )$ and $\beta )$ and prove $\gamma$ ). Let $v$
be a locally bounded potential $> 0$ with support $S$ and $w$ a
superharmonic function $\ge 0$ and $w \ge v$ on $S$. Consider any
superharmonic function $w_1 \ge 0$ with $w_1 \ge v$ on $C \omega$
where $\omega \subset \bar{\omega} \subset \Omega$. Then $w +
\inf. (w_1, v) \ge v$ on $S \cup C \omega$; therefore $w + \inf.(w_1,
v) \ge R^{S \cup C \omega}_v$ and $w + R^{c \omega}_v \ge R^{S \cup C
  \omega}_v$. By $\beta )$, the second member $= v$ in $\omega \cap
CS$. Now $R^{C \omega}_v \to 0$ according to the directed set of $
\omega \subset \bar{\omega} \subset \Omega$. Hence $w \ge v$. 

\setcounter{remark}{0}
\begin{remark}\label{p4:chap8:sec34:rem1} % rem 1
  If we suppress in $\alpha ), \beta), \gamma)$ the hypothesis that
  $v$ is locally bounded, these properties still remain equivalent
  with the same arguments, but they are no longer true in the
  classical case. 
 \end{remark}

\begin{remark}\label{p4:chap8:sec34:rem2} % rem 2
  When the constant $1$ is superharmonic, $D)$ implies that for any\pageoriginale
  locally bounded potential $v, v \le \sup.  v$ on $S$ (the support of
  $v)$. (Maximum principle). 
\end{remark}

\noindent
\textbf{Local Character of $(D)$}. Using the continuation theorem
(\ref{p4:chap4:sec21:thm14}, $n^o$ 21) we see that whenever there
exists a potential $> 0$ on 
$\Omega$, axiom $D$ for $\Omega$, (in the form $\beta$) implies the
same property for any regular subspace or even any subspace and
therefore for a neighbourhood of  each point. The converse seems
true. We know the proof only in the case when $\Omega$ has a countable
base for open sets. This converse has been proved as a consequence of
a further study by R.M. Herve. 

\section{First application of axiom \texorpdfstring{$D$}{D}}\label{p4:chap8:sec35}%sec 36
 
\begin{thm}\label{p4:chap8:sec35:thm26} % the 26
  Let $v$ be a superharmonic function in $\Omega$, $\ge 0$, locally
  bounded and with support $S$. We suppose the axioms
  \ref{p4:chap1:sec1:axiom1}, \ref{p4:chap1:sec1:axiom2},
  \ref{p4:chap1:sec1:axiom3}  and
  $D$ on $\Omega$. For any point $y$  on the boundary of $S$, 
  \begin{gather*}
    \lim. \sup. v(x) = \lim. \sup. v(x)\\
    x \in CS, x \to y \quad x \in \partial S, x \to y.
  \end{gather*}
\end{thm} 
 
Therefore $v$ is continuous in $\Omega$ at $y$, if the restriction of
$v$ to $S$ is continuous at $y$. 
 
\begin{proof} % pro
  We shall use the following remark (that is true independent of $D$).
\end{proof} 
 
Let $v$ be a superharmonic function $v(x_o)$ finite (at the point
$x_o$). Consider the regular domains $\omega$ containing $x_o$, and the
set $e \subset \partial \omega$ where $v > v (x_o) +
\varepsilon$. Then $\int \varphi_e d \rho^\omega_{x_o} \to 0$
according to the filter corresponding to the directed set of these
$\omega$ ($\varphi_e$ being the characteristic function of $e$). This\pageoriginale is an
immediate consequence of $\int v d \rho^\omega_{x_o} \to v(x_o)$. 

Observe now that in Theorem \ref{p4:chap8:sec35:thm26}, the inequality
\begin{gather*}
  \lim. \sup v(x) \ge \lim. \sup. v(x)\\
  x \in CS, x \to y \quad x \in \partial S, x \to y
\end{gather*}
is obvious. Suppose strict inequality holds good, introduce a number
$\lambda$ in between the numbers. Let $\delta$ be the open set of $CS$
where $v > \lambda$,  and $\delta'$ the intersection of $\delta$ with
a regular domain $\omega \ni y$. In $\delta'$, $v =
\bar{H}_v^{\delta'}$ according to $D (\beta)$. Let us decompose $v$ as
$v = \varphi_1 + \varphi_2$ where $\varphi_1 = v $ on $C \delta, 0$ on
$\delta$ and $\varphi_2 = 0$ or $C \delta$ and $v$ and $\delta$. Now,
$v = \bar{H}^{\delta'}_v \le \bar{H}^{\delta'}_{\varphi_1} +
\bar{H}^{\delta'}_{\varphi_2} $ and $\varphi_1 \le \lambda$ on
$\partial \delta'$ for $\omega$ small enough. Given $\lambda' >
\lambda$, $\bar{H}^{\delta'}_{\varphi_1} \le \lambda'$ when $\omega$
is small enough (because this envelope is majorised by a harmonic
function which would be defined in the neighbourhood of $y $ and equal
to $\dfrac{\lambda + \lambda'}{2}$ at $y$). On the other hand, we have
in $\delta'$ 
$$
\bar{H}^{\delta'}_{\varphi_2} \le \bar{H}^{\omega}_{\varphi_2} =\int
\varphi_2 d \rho^\omega_x 
$$

This integral for $x = y$ tends to $0$ according to the considered
directed set of the $\omega$. Therefore $\underset{x \in \delta' x \to
  y}{\lim. \sup.} \bar{H}^{\delta'}_{\varphi_2} x)$ is arbitrarily small
for a suitable $\omega$. We conclude that $\underset{x \in C S, x \to
  y}{\lim. \sup.}. v (x) \le \lambda'$ therefore $\le \lambda$. This
is a contradictional. Hence the theorem. 

\setcounter{remark}{0}
\begin{remark}\label{p4:chap8:sec35:rem1} % rem 1
  As\pageoriginale a particular case, if $S = \{y\}, v$ is continuous at $y$.
\end{remark}

\begin{remark}\label{p4:chap8:sec35:rem2} % rem 2
  We want to emphasize that only with the axioms
  1, 2, 3 and, the
  axiom $D$ implies that any locally bounded potential whose
  restriction on the support is continuous, is itself continuous on
  $\Omega$. It is remarkable that when $\Omega$ satisfies the second
  axiom of countability the converse is true (due to R.M. Herve). 
\end{remark}
 
\section{Convergence theorems}\label{p4:chap8:sec36}%theo 36
 
\begin{thm}[convergence of sequences]\label{p4:chap8:sec36:thm27} % the 27
  We suppose 1, 2, 3 and $D$ and a countable base for open sets on
  $\Omega$. Let $v_n$ be fa decreasing sequence of superharmonic
  functions $\ge 0$. The limit $v$ which is nearly superharmonic is
  different form the regularised function $\hat{v}$ on a polar set
  (i.e., $v$ is quasi-superharmonic).  
\end{thm}

\begin{proof} 
  We may suppose the existence of a finite continuous potential $V_o >
  0$ (otherwise, it is trivial). Let us suppose first $v_1$ to be
  upper bounded and $\epsilon >  0$. Let $\alpha$ be the set of points $x$
  where $v(x) - \hat{v} (x) > \varepsilon$. It is enough to prove that
  $\alpha$ is polar, for, that will imply $\{x : v (x) > \hat{v}
  (x)\}$ is polar. As $\alpha$ is a borelian set in the compact
  metrizable space $\bar{\Omega}, \alpha$ is a K-analytic set, and we
  have only to see that every compact subset $K$ of $\alpha$ is polar
  (cf. Prop. \ref{p4:chap7:sec33:prop22}). 
\end{proof}

We introduce an open set $\omega$ such that $K \subset \omega \subset
\bar{\omega} \subset \Omega$ then, $v_n (x) \ge H^{\omega - K}_{v_n}
(x) \ge H^{\omega - K}_v (x)$ in $\omega - K$. Hence $v(x)\geq 
H_v^{\omega-k}(x); v (x) ~\geq  H_v^{\omega-K}(x) ~ H^{-K}_v (x)$ in
$\omega-K$. According to 
axiom $D$ (form), $H^{\omega - K}_{\hat{v}}$\pageoriginale is the greatest
harmonic minorant of $\hat{v}$ in $\omega - K$, therefore (by the
above inequality) equal to $H^{\omega - K}_v$. Hence $H^{\omega -
  K}_{v  - \hat{v}} = 0$. 

Now $v - \hat{v} > \varepsilon$ on $K$. Let us choose $\lambda > 0$
such that $\lambda V_o < \varepsilon$ on $K$ and define $\varphi$ on
$\partial (\omega - K)$ such that $\varphi = 0$ on $\partial \omega$
and $\varphi = \lambda V_o $ on $K$. Then 
$$
(R^K_{\lambda V_o})_\omega = H^{\omega - K}_ \varphi \le H^{\omega -
  K}_ {v - \hat{v}} \text {on } \omega - K. 
$$

Hence $(R^K_{V_o})_\omega = 0$ on $\omega - K ; K$ is polar in
$\omega$, therefore in $\Omega$. 

In the general case, for every positive integer $p$, we introduce
$v^p_n = \inf. (v_n, p V_o)$ which is locally bounded and decreases as
$n$ increases. Form the particular case we deduce, $w_p = \hat{w}_p$
quasi everywhere, if $w_p = \lim\limits_{n \to \infty}. v^p_n$, for
every $p$. Now $v = \lim\limits_{n \to \infty}. v_n = \lim\limits_{n
  \to \infty}. w_p$. 


Therefore quasi-everywhere in $\Omega, v = \lim\limits_{p \to
  \infty}. \hat{w}_p$; the limit on the right hand side is
superharmonic $(\le v_1)$. 

\begin{coro*} 
  With the same hypothesis, let $v_n \ge 0$ be a sequence of
  hyperharmonic functions in $\Omega$. Then $\inf. v_n, \lim\limits_{n
    \to \infty}. \inf. v_n$ are quasi-superhar\-monic or equal
  everywhere to $+ \infty$ (and nearly hyperharmonic). 
\end{coro*}

In fact the function $w^p_n = =inf. (v_n, v_{n+1}, \ldots,  v_{n+p})$
is hyperharmonic, decreasing as $p$ increases, and the limit as $p \to
\infty$ is a function $w_n$ which is nearly hyperharmonic, and
quasi-superharmonic or equal to $+ \infty$. Now we observe that $w_n$
is increasing and the corollary follows. 

\begin{thm}[General Lower Envelope]\label{p4:chap8:sec36:thm28} 
  With\pageoriginale the same hypothesis on $\Omega$ (Axioms
  \ref{p4:chap1:sec1:axiom1}, \ref{p4:chap1:sec1:axiom2},
  \ref{p4:chap1:sec1:axiom3} and $D$ and
  second axiom of countability) let us consider a family of
  superharmonic functions $v_i \ge 0$. The lower envelope $v =
  \inf\limits_i. v_i$, which is a nearly superharmonic function, equals
  $\hat{v}$ quasi-everywhere in $\Omega$. 
\end{thm}

\begin{proof} % pro
  We know by the topological lemma of Choquet that there exists a
  sequence $v_{\alpha_n}$ in the family such that, the lower
  regularised functions of $\inf. v_{\alpha_n}$ and $\inf. v_i$ are the
  same. Then 
  $$
  \inf. \hat{} v_{\alpha_n} = \inf. \hat{ } \inf. v_i \le \inf. v_{\alpha_n}.
  $$
\end{proof}

\noindent
Hence the theorem.

\section{Application to reduced functions}\label{p4:chap8:sec37} %sec 37

\begin{remark*} % rem
  The axioms \ref{p4:chap1:sec1:axiom1},
\ref{p4:chap1:sec1:axiom2} and \ref{p4:chap1:sec1:axiom3} the first axiom of countability for
  $\Omega$ (there exists at each point a countable base of
  neighbourhoods ) imply for any polar set $E$ and a point $x_o \in
  CE$, an associated function finite at $x_o$. 
\end{remark*}

For, if $U_n$ is a countable base of neighbourhoods of $x_o$, there is
for $E \cap C U_n$ an associated function $v_n$ such that $v_n <
\dfrac{1}{n^2}, \sum v_n$ satisfies the required condition. 

\begin{prop}[Properties of the reduced function]\label{p4:chap8:sec37:prop23} % pro 23
  Let $\Omega$ satisfy the axioms \ref{p4:chap1:sec1:axiom1},
  \ref{p4:chap1:sec1:axiom2}, \ref{p4:chap1:sec1:axiom3} and $D$ and second axiom
  of countability. Let $E \subset \Omega$ and $\varphi \ge 0$, a
  function on $E$ majorised by a superharmonic function $V \ge 0$  in
  $\Omega$. 
\end{prop}

\begin{enumerate} [(i)]
\item $\hat{R}^E_\varphi = R^E_\varphi$ quasi-everywhere
\item on\pageoriginale $CE, R^E_\varphi = \hat{R}^E_\varphi$
\item $\hat{R}^E_\varphi$ is the smallest superharmonic function $\ge
  0$ which is  $\ge \varphi$ quasi-everywhere on $E$. $\hat{R}^E_V = V
  q.e$ on $E$.  
\end{enumerate}

In fact if a superharmonic function $v \ge 0$ in $\Omega$ satisfies $v
\ge \varphi$ on $E$ except on a polar set $\alpha$. Let us introduce
$w$ associated to $\alpha$. Then $v + \varepsilon w \ge \varphi$ on
$E$, therefore $\ge R^E_\varphi$ everywhere. Now if we take as $v$ the
function $\hat{R}^E_{\varphi_E} $ and choose $w$ finite at a point
$x_o \in CE$, we have $\hat{R}^E_\varphi (x_o) \ge R^E_\varphi(x_o)$,
hence the property $(ii)$. 

On the other hand for any $v,  v \ge R^E_\varphi$ at any point where
$w$ is finite, i.e. quasi-everywhere. Hence $v \ge \hat{R}^E_\varphi$
quasi-everywhere. 
