\chapter{Superharmonic and Hyperharmonic Functions}\label{p4:chap2}%chap II

\setcounter{section}{5}
\section{}\label{p4:chap2:sec6}% section 6

\begin{defn}\label{p4:chap2:sec6:def4} % definition 4
  Let\pageoriginale the space $\Omega$ satisfy the fundamental axioms. A function
  $v$ defined on an open set $\omega_o$ of $\Omega$ is a hyperharmonic
  function if it satisfies 
  \begin{enumerate}[(i)]
  \item $v$ is lower semi continuous
  \item $v > - \infty$
  \item for any regular open set (or domain) $ \omega \subset
    \bar{\omega} \subset \omega_o$,  
  \end{enumerate}
  $v(x) \ge \int vd ~ \rho^\omega_x$ for every $x$ in $\omega$.
\end{defn}

A function $u$ such that $-u$ is hyperharmonic, is called hyperharmonic.

\textbf{First properties.} On any open set $\omega_o$ of $\Omega$, the
hyperharmonic functions satisfy: 
\begin{enumerate}[(1)]
\item If $v_1$ and $v_2$ are hyperharmonic functions then $\lambda_1
  v_1, \lambda_2 ~ v_2$ and $\lambda_1 ~ v_1 + \lambda_2 ~ v_2$ are
  again hyperharmonic functions for any finite $\lambda_1 > 0,
  \lambda_2 > 0$. The same is true of $\inf (v_1,v_2)$. 
\item If $(v_i)_{i \in I}$ is any family of hyperharmonic functions
  directed for increasing order, then the upper envelope of this
  family is again hyperharmonic. (This is a consequence of the
  possibility of interchanging the operations of taking supremum and
  integration for $d ~ \rho^\omega_x$). 
\item~
\begin{thm}\label{p4:chap2:sec6:thm2}%%2
  Any\pageoriginale hyperharmonic function on a domain $\omega$,
  taking the value $+ \infty$ on any open subset of $\omega$ is
  identically $+ \infty$ in $\omega$.  
\end{thm}
\end{enumerate}

Consider the set of points $A$ of $\omega$ in whose neighbourhood $v =
+ \infty$. $A$ is a non-empty open subset of $\omega$. Suppose this is
not the whole of $\omega$. There exists a non empty connected
component $A_1$ of $A$ and a boundary point $x_o$ of $A_1$ in
$\omega$. For any regular domain $\omega_1 \subset \bar{\omega}_1
\subset \omega$ containing the point $x_o$,  
$$
v(x_o) \ge ~ \int ~ v ~ d ~ \rho^\omega_x
$$

We may choose one such $\omega_1$, not containing $A_1$; then there
exists a point of the boundary of $\omega_1$ in $A_1$ and so $v = +
\infty$ in a neighbourhood of that point. Any non-empty open set of
the boundary of $\omega_1$ has non-zero $d ~ \rho^{\omega_1}_{x_o}$-
measure. It follows that $v(x_o) = + \infty$. Further the summability
being independent of any particular point of $\omega_1$, $v(x) = +
\infty$ for every $x \in \omega_1$. Hence $x_o$ is in $A$. This is a
contradiction. Therefore $v = + \infty$ everywhere in $\omega$. 

This leads us to the following definition:

\begin{defn}\label{p4:chap2:sec6:def2} % definition 5
  {\em Superharmonic functions:}

  A hyperharmonic function on an open set $\omega_o$, which takes
  finite values at least at one point of each of the components of
  $\omega_o$ is called a superharmonic function. A function $u$ such
  that $-u$ is superharmonic is called subharmonic. 
\end{defn} 

A\pageoriginale superharmonic function is $d ~\rho^{\omega_1}_x$ - summable for
every $x$ belonging to a regular open set $\omega_1$ contained in
$\omega_o$ with $\bar{\omega}_1 \subset \omega_o$. 

\begin{defn}\label{p4:chap2:sec6:def6}% definition 6
  {\em Hyper and super h-harmonic functions.} If $h$ is a finite
  continuous function $> 0$ on $\Omega$, we can consider the hyper
  h-harmonic functions and the corresponding condition is, 
  $$
  v(x) \ge \int ~ v ~ d ~ \rho'^\omega_x = \int ~ v(y)
  \frac{h(y)}{h(x)} ~ d ~\rho^\omega_x 
  $$
\end{defn}

We see that the new functions are the quotients by $h(x)$ of the hyper
and superharmonic functions. 
 
%%% Missing section7
\setcounter{section}{7}
\section{Minimum principle}\label{p4:chap2:sec8}%sec 8

\begin{thm}[i]\label{p4:chap2:sec8:thm3-i} % theorem 3(i)
  If $v$ is a hyperharmonic function $\ge 0$ in a domain, then $v = 0$
  everywhere or $v > 0$ everywhere. An equivalent form is the
  following: 

  If a hyperharmonic function on an open set has  minimum zero at a
  point $x$ then it is zero in some neighbourhood of $x$. 
  
  Let us prove first form. Suppose $v > 0$ at a point: then $v > 0$ in a
  neighbourhood of the point. Then $nv$ tends to a hyperharmonic
  function which is $+ \infty$ in an open set and hence
  everywhere. Therefore $v > 0$ everywhere in the domain. 
\end{thm}

\setcounter{thm}{2}
\begin{thm}[ii]\label{p4:chap2:sec8:thm3-ii} % theorem 3 (ii)
  Suppose\pageoriginale in an open set there exists a harmonic\footnote{or even
    only superharmonic (constantinescu-Cornea-Loeb.see add.chapter)}
  function $h > \varepsilon > 0$ (as it is the case with regular open
  sets). Any hyperharmonic function $v$ in $\omega$ which has at every
  point on the boundary a limits inferior $\ge 0$ is itself $\ge 0$
  everywhere on $\omega$. 
\end{thm}

\begin{proof}
  It is enough to consider a domain $\omega$. $\dfrac{v}{h}$ continued
  by zero on the boundary points of $\omega$ is $> - \infty$ and lower
  semi continuous on closure of $\omega$. If $v$ were not $\ge 0$ in
  $\omega$, let $k < 0$ be the infimum of  $\dfrac{v}{h}$ in
  $\bar{\omega}$.  $k$ is attained at some point $x_o$ in
  $\omega$. Now, in $\omega ~ \dfrac{v}{h} - k$ is hyper-h-harmonic
  $\ge 0$ and equal to zero at $x_o$. Consequently $\dfrac{v}{h} - k =
  0$ or $ v = kh $ in $\omega$, which contradicts the fact that $\lim
  \inf ~ v \ge 0$ at any boundary point. 
\end{proof}

In case, the constants are harmonic, the hyperharmonic function $v$ satisfies:
$$
v \ge \inf [ \lim.\inf ~ v \text{ at any boundary point }].
$$

We have already proved that this property holds for harmonic functions
(cl.Ch.$I.3$). 

\section{Local criterion}\label{p4:chap2:sec9}%sec 9

\begin{thm}\label{p4:chap2:sec9:thm4}% theorem 4
  The condition $(iii)$ of the definition of hyperharmonicity of a
  function $v$ on an open set $\omega_o$ can be replaced by a weaker
  (locally) one viz: for any $x_o$ in $\omega_o$ there exists for
  every $v$ a base regular neighbourhoods\pageoriginale $\omega'$ such that
  $\bar{\omega} \subset \omega_o$ and $v (x_o) \ge \int ~ v ~ d ~
  \rho^{\omega'}_{x_o}$.
\end{thm}

We shall name the functions characterised by the condition of the
above theorem [besides (i) and (ii) of Def. 4] as $N$ - functions
on $\omega_\circ$. These functions have properties similar to those of
hyperharmonic functions; some of which we shall see before proving the
theorem. 
\begin{enumerate}[1)]
\item If $v $ is a $N$ -function $\ge 0$ in an open set $\omega, nv$ and
  $\lim ~ nv$ are also $N$ - functions in $\omega$. 
\item If a $N$-functions is $+ \infty$ in a domain $\omega \subset
  \omega_\circ$ it is $+ \infty$ at every boundary point of $\omega$ in
  $\omega$. 
  
  These two properties follow with arguments similar to those in
  \S  \ref{p4:chap2:sec6}.
\item If $v \geq 0$ in a domain $\omega, v > 0$ or $v = 0$ everywhere
  in $\omega$.

  Infact the set $\delta$ where $v > 0$ is open. Suppose it is a
  non-empty set different from $\omega$. There exists a connected
  component $\delta_1$ of $\delta$ which is non-empty; $\lim. nv$ is a
  N-function and is $+ \infty$ on $\delta_1$. Therefore on $\partial
  \delta_1 \cap \omega$ (non-empty); that implies $v > 0$ on this
  set.\footnote{In a shorter way, if $x_o ~ \in ~ \partial \delta_1
    \cap \omega$, the hypothesis imply directly, for a suitable
    $\omega' v(x_o) \ge \int ~ vd ~ \rho^{\omega'}_{x_o} > 0$.} This
  is a contradiction. 

\item Let $\omega$  be an open set such that there exists a harmonic
  function $h > \varepsilon > 0$ on $\omega$. If $v$ is a N -function
  on $\omega, \dfrac{v}{h}$ is an analogous function but with the
  measure $d ~ \rho'^{\omega}_x$ instead of $d ~ \rho^\omega_x$, as in
  Theorem \ref{p4:chap2:sec8:thm3-ii}(ii)  it follows that, $\lim, \inf. v \ge 0$ at all
  the boundary\pageoriginale points of $\omega_o$ implies $v \ge 0$ in $\omega$. 
\end{enumerate}

\begin{remark*}% remark
  The property 4) can be proved in this way also. We may suppose
  $\omega_o$ to be the whole space $\Omega$ and constants to be
  harmonic. We have to prove that, if $w$ satisfies axioms for an N -
  function the condition $\lim.\inf. w \ge 0$ at the Alexandroff point
  $\mathscr{O}$ of $\Omega$ implies $w \ge 0$. If $\inf. w$ were equal
  to $k < 0$, we introduce $w_1 = w - k \ge 0$ which is equal to zero
  at a point $x_o$ in $\Omega$. For any $x_1$ where $w_1 = 0$ there
  exists a regular neighbourhood $\omega$ such that $\int ~ w_1 ~ d ~
  \rho^\omega_{x_1} \le 0$; this implies $w_1 = 0$ on $\partial
  \omega$. Let us consider the family $\Phi$ of open neighbourhoods
  $\delta_i$ of $x_o$ in $\Omega$ such that $\underset{x \in
    \delta_i}{\lim.\inf.} w_1(x) = 0$ at any boundary point. There is
  a largest one, the union $\delta_o$ (because in $\bar{\Omega},
  \partial \delta_o \subset \cup_i \partial \delta_i)$. The point
  $\mathscr{O}$ is not obviously in $\partial \delta_o$. There exists
  a point $z \in \partial \delta_o$ different from $\mathscr{O}$ and
  $x_o$. Now there is a regular neighbourhood $\delta'$ of $z$, such
  that $w_1 = 0$ on $\partial \delta'$ and $\delta' \cup \delta_o$
  belongs to $w ~ \Phi$; this contradicts the maximality of
  $\delta_\circ$. Hence the result. 
\end{remark*}

\noindent \textbf{Proof of the theorem.} 
  Given a regular open set $\omega \subset \bar{\omega} \subset
  \omega_o$, let us consider $V = v - \int \theta ~ d ~ \rho^\omega_x$
  where $\theta$ is any finite continuous function such that $\theta
  \le v$ on the boundary $\partial \omega$ of $\omega$. $V$ satisfies
  all the three axioms of $N$ -functions in $\omega$ and $\lim.\inf. V
  \ge 0$ at all boundary points of $\omega$. Therefore $V \ge 0$. We
  conclude that $v (x) \ge \int ~ v ~ d ~ \rho^\omega_x$. 

\section{Elementary case of ``balayage''}\label{p4:chap2:sec10}%sec 

\begin{thm}\label{p4:chap2:sec10:thm:5} % theorem 5
  If\pageoriginale $v$ is a hyperharmonic function in $\omega_o$ and if $\omega$ is
  a regular open set with $\omega \subset \bar{\omega} \subset
  \omega_o$, the function $E^\omega_v$ which is equal to $v$ outside
  $\omega$ and to $\int ~ v ~ d ~ \rho^\omega_x$ in $\omega$ is
  hyperharmonic in $\omega_o$. [Similar result holds by replacing
    ``hyperharmonic'' by ``superharmonic'']. 
\end{thm}

The local conditions are fulfilled. The lower semi-continuity of the
new function follows from the proposition $3$ (no.$3$) applied to
$\int  (-v)d ~ \rho^\omega_x$. To any $x$, let us associate the
regular neighbourhoods $\delta$ (with $\bar{\delta} \subset \omega_o$)
such that $\bar{\delta} \subset C \bar{\omega}$ if $x \in C
\bar{\omega} $ or $\bar{\delta} \subset \omega$ of $x \in
\omega$. They will be used in the condition $(iii)$ and $E^\omega_v$
satisfies the conditions of the local criterion. 

\text{Another proof} (without using the local criterion). Consider any
$\omega'$ regular $(\bar{\omega}' \subset \omega_o)$ and any finite
continuous function $\theta$ on $\partial\omega'$ satisfying $\theta
\le E^\omega_v \le v$.   

First $ \qquad \qquad \int ~\theta ~ d ~ \rho^{\omega'}_x \le ~ \int ~
v ~ d ~ \rho^{\omega'}_x \le v$ 

Using prop.\ref{p4:chap1:sec3:prop3}, we see that in $\omega \cap
\omega', \int \theta ~ d ~ \rho^{\omega'}_x - \int ~ v ~ d ~
\rho^\omega_x$ has at any boundary point a $\lim.\sup \le 0$,
therefore is $\le 0$. Hence we have in 
$\omega'$ 
$$
\displaylines{\hfill 
  \int ~ \theta ~ d ~ \rho^{\omega'}_x
  ~ \le ~ E^\omega_v\hfill \cr
  \text{therefore}\hfill 
  \int ~ E^\omega_v ~ d ~ \rho^{\omega'}_x ~ \le  ~ E^\omega_v.\hfill }
$$
