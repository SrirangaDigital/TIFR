\chapter{Some Examples of Dirichlet Problem}\label{p4:chap6}%chap VI

\setcounter{section}{25}
\section{}\label{p4:chap6:sec26}% 26

Some\pageoriginale well known Dirichlet problems are particular cases of the
following one: 

Let $\Omega$ be dense in a space $\varepsilon \neq
\Omega$. $\mathscr{L}_1$ be the sets of the intersections with
$\Omega$ of the neighbourhoods of the points of the boundary $B=
\varepsilon - \Omega$. (We may also consider a subset $\mathscr{L}'_1$
corresponding to s subset $B'$ of $B$). A function on $\mathscr{L}_1$
(or $\mathscr{L}'_1$) is considered as well a function on $B$ (or
$B'$), because of the obvious one-one correspondence. Let $\sum_1$ be
the set of all hyperharmonic functions on $\Omega$ (we may also
consider a subset $\sum_2$ as in $n^\circ 22$). Let us suppose $\sum_1$
(or $\sum_2$) and $\mathscr{L}_1$ (or $\mathscr{L}'_1$) are
associated. Then Theorem $18$ holds but we have to study further in
every case the resolutive and absolutely resolutive functions, the
integral representation (for which it is for instance interesting to
find sets of resolutive functions from which the Daniell continuation
gives the general integral) and finally the behaviour at the boundary
of the generalised solution.  

\noindent
\textit{First example}(with regular domain $\omega$)

\begin{prop}\label{p4:chap6:sec26:prop:16}%%% 16
  $\omega$ is considered as the fundamental space. $\bar{\omega}=\varepsilon$.
\end{prop}

The envelope $\bar{\mathscr{H}}_f$ corresponding to the previous
$\mathscr{L}_1$ and $\sum_1$ (obviously associated) has been already
introduced $(n^0 15)$ denoted by $\bar{H}\omega_f$\pageoriginale and seen to be
equal to $\bar{\int} f d \rho^\omega_x$. Therefore the resolutive
functions are the $d \rho^\omega_x$- summable functions (hence
absolutely resolutive) and for these function $\mathscr{H}_f (x) =
\int f d \rho^\omega_x$. The Daniell integral of Theorem
\ref{p4:chap5:sec25:thm18} is the Radon integral for the harmonic
measures $d \rho^\omega_x$.  

\section[Second Example...]{Second Example (with any relatively compact\break domain $\omega$)}\label{p4:chap6:sec27}

\begin{thm}\label{p4:chap6:sec27:thm19}
  We suppose the existence of a potential $> 0$ (besides axioms
  \ref{p4:chap1:sec1:axiom1}, \ref{p4:chap1:sec1:axiom2} $\&$
  \ref{p4:chap1:sec1:axiom3}). We consider $\bar{\omega}$ as the space
  $\varepsilon$ and 
  the corresponding $\mathscr{L}_1$ and $\sum_1$ that are
  associated. The corresponding envelope $\mathscr{H}_f$ has been
  studied and denoted by $\bar{H}^\omega_f$ and we may write also
  $\underbar{H}_f$, $H^\omega_f$ instead of
  $\underline{\mathscr{H}}_f$ and $\mathscr{H}_f$. (We suppress index
  $\omega$ for simplicity). 
\end{thm}

Now the finite continuous functions $\theta$ on $\omega$ are
resolutive \footnote{Result and proof due to R.M.Herve.} (hence
absolutely) and $\mathscr{H}_\theta (x)$ for such a $\theta$ defines a
Radon integral that we denote $\int \theta d \mu_x.  d \mu_x$ is
called the harmonic measure (identical to $d \rho^\omega_x$ if
$\omega$ is regular). The $d \mu_x$ summability and the sets of $d
\mu_x$-measure are independent of $x \in \omega$. Any resolutive
function is absolutely resolutive. $\bar{\int} f d \mu_x \leq
\bar{H}^\omega_f (x)$ and hence resolutive functions $f$ are summable
and $H^\omega_f (x) = \int f d \rho^\omega_x$. 

Moreover if $\Omega$ has a countable base for open sets, $\bar{\int} f
d \mu^\omega_x = \bar{H}^\omega_f (x)$ and the $d \mu_x$-summable
function are resolutive (the\pageoriginale Daniell integral of Theorem $18$ is the
$d \mu_x$- integral). 

Let $V_1$ be a finite continuous potential $> 0$ in $\Omega$. As
$\bar{H}^\omega_{V_1} \leq V_1$, $\substack{\lim\sup\limits\\{x \to
y}}$. $\bar{H}^\omega_{V_1}(x)$ at any boundary point $y$ is
$\leq V_1 (y)$, therefore $\bar{H}^\omega_{V_1} \leq
\underbar{H}^\omega_{V_1}$ on $\omega$ ; hence the equality. The same
holds good for the difference of such $V_1$. Now any finite continuous
function $f$ on $\partial \omega$ can be approximated by differences
$\omega$ of such potentials (Theorem \ref{p4:chap4:sec21:thm15}) 
$$
w-\varepsilon \leq f \leq w + \varepsilon
$$ 
Then \qquad $H_w - \varepsilon \bar{H}_1 \leq \underbar{H}_f \leq
\bar{H}_f \leq H_w + \varepsilon \bar{H}_1$. 

Hence the resolutivity of $f$.

The properties concerning the $d \mu _x$-summability, the $d \mu_x$
measure zero are immediate consequences of the definitions of the
Radon integral ($\bar{\int}$ and $\underline\int$) and of axiom
\ref{p4:chap1:sec1:axiom3}. The resolutivity implies absolute resolutivity, because of the
fact that the envelopes $\bar{H}_f$ and $\underbar{H}_f$ remain the
same by changing $\sum_1$ into $\sum_2$. (See Prop. 15 (iii)). 

For any $f$ on $\partial \omega$, we remark finally (see the proof of
Theorem \ref{p3:chap4:sec12:thm9}) that if $v$ is any superharmonic function satisfying
$\psi' = \lim$. $\inf. v \geq  \underset{> - \infty}f$, at every
boundary point, $\bar{H}^\omega_f \leq \bar{H}^\omega_\psi \leq v$,
therefore $\bar{H}_f = \Inf  \bar{H}_\psi$ for all $\psi$ lower
semi-continuous $\geq f$ and $> - \infty$. On the other hand if
$\theta$ is a continuous function $\leq$ such a $\psi,  H_\theta =
\int \theta d \mu_x \leq \underbar{H}_\psi$. 

Hence\pageoriginale $\bar{\int}f d \mu_x \leq \int \psi d \mu_x \leq
\underbar{H}_\psi$, therefore $\bar{\int} f d \mu_x \leq \bar{H}_f$. 

Suppose now $\Omega$ has a countable base for open sets, therefore is
metrizable, or only that on the subspace $\partial \omega$ any open
set is a $K_\sigma$ set. Then any lower semi-continuous function $\psi
> -\infty$, is the limit of an increasing sequence of finite continuous
functions $\theta_n,  H_{\theta_n} \to \bar{H}_\psi$. Hence $\int
\theta_n d \mu_x \to \int \psi d \mu_x$ therefore $\bar{H}_\psi = \int
\psi d \mu_x$ and using a previous remark $\bar{H}_f (x) = \bar{\int}f
d \mu_x$. 

In the last hypothesis, the Daniell continuation from $H_\theta$ for
finite continuous functions $\theta$ gives the general representation
of Theorem $18$. 

\section{Extension to a relatively compact open set \texorpdfstring{$\omega$}{omega}}\label{p4:chap6:sec28} %section 28 
 

Hypothesis regarding $\Omega$ is assumed to be the same as in $n^0 27$.

We consider for any $f$ on $\partial \omega$, the envelope
$\bar{H}^\omega_f$ already introduced and the other one equal to $-
\bar{H}^\omega_{-f}$. We recall that for any connected component
$\omega_i$ of $\omega, \bar{H}^{\omega_i}_f = \bar{H}^\omega_f$ on
$\omega_i$. (We may suppress the indices $\omega$, $\omega_i$ when
there is on ambiguity.) The resolutivity and absolute resolutivity are
defined by the same condition for all $\omega_i$, and the
corresponding generalised solution is denoted by $H^\omega_f$. 

For a finite continuous function $\theta$ on $\partial \omega$ and any
fixed $x \in \omega$, $H_\theta (x)$ is a Radon integral $\int \theta
d \mu^\omega_x$. The measure $d \mu^\omega_x$, when considered in
$\Omega$ is identical to the corresponding one on the component
containing $x$. For any $f$ on $\partial \omega, \bar{\int}f d
\mu^\omega_x \leq \bar{H}^\omega_f (x)$ and the\pageoriginale equality holds good at
least when $\Omega$ has a countable base for open sets. 

Negligible sets $e$ on $\partial \omega$ are defined by the condition
$\bar{H}^\omega_{\varphi_e}= 0$ on $\omega$. The existence of a
superharmonic function $\geq 0$ which tends to $+ \infty$ at every
point of $e < \partial \omega$, implies that $e$ is negligible; the
converse is true when $\omega$ is a countable union of domains. 

A weaker condition is $\underbar{H}^\omega_{\varphi_e}=0$. This
implies that a bounded harmonic function which tends to zero at any
boundary point outside such an $e$ is identically zero. 

\begin{prop}[Variation of
    $\omega$]\label{p4:chap6:sec28:prop17}% proposition 17 
  Let $F$ be a continuous function on
  $\partial \omega$. Given $\varepsilon > 0$ and a compact set $K
  \subset \omega$, there exists a compact set $K_1$ satisfying $K
  \subset K_1 \subset \omega$ and such that for any open set
  $\omega_1, K_1 \subset \omega_1 \subset \omega$ the inequality
  $\bigg | H^\omega_F - H^{\omega_1}_F \bigg | < \varepsilon$ holds on
  $K$. 
\end{prop}

In fact, given $\varepsilon' > 0$ we may choose a superharmonic
function $v$ and a subharmonic function $u$ (on $\omega$), satisfying 
\begin{align*}
  \lim. \inf. v & \geq F \text{ at all points of } \partial \omega,
  v-H^\omega_F < \varepsilon' \text{ on }K \\ 
  \text{ and } \qquad \lim. \sup. u & \leq  F \text{ at all points of
  } \partial \omega, H^\omega_F -u < \varepsilon' \text{ on }K. 
\end{align*}

Let $h$ be a harmonic function $> 0 $ on an open set containing $\bar{\omega}$.

Let $v_1$ and $u_1$ be respectively the continuations of $v$ and $u$
by their $\lim.  \inf $ and $\lim.  \sup$. $\partial \omega$.  On
$\bar{\omega}$, the 
inequality $v_1 - F + \varepsilon' h > 0 $ on $\partial \omega$
implies the same inequality in an open set\pageoriginale containing $\partial
\omega$. We have a similar property for $u$. Hence there exists in
$\omega$ a compact set $K_1$ containing $K$ such that on $\omega -
K_1$, 
$$
v-F+\varepsilon' h > 0,  \qquad u-F-\varepsilon' h < 0. 
$$

Therefore the condition $K_1 \subset \omega_1 \subset \omega$ implies
on $\omega_1$, 
$$
\displaylines{\hfill 
  v \geq H^{\omega_1}_{F- \varepsilon'_h} = H^{\omega_1}_F -
  \varepsilon'h \text{ and } u \leq H^{\omega_1}_{F+\varepsilon' h}=
  H^{\omega_1}_F + \varepsilon' h \hfill \cr
  \text{i.e.,} \bigg | H^\omega_F - H^{\omega_1}_F \bigg | \leq
  \varepsilon' h + \varepsilon' \hfill }
$$
on $K$ and hence the proposition.

\begin{coro*}
  For any increasing directed set $S$ of open sets $\omega_1 \subset
  \omega$ (ordered by inclusion) such that any compact set $K \subset
  \omega$ is contained in some set of $S, d \mu^{\omega_1}_x$ (for a
  fixed $x \in \omega$) converges vaguely to $d \mu^\omega_x$
  according to the corresponding filter. 
\end{coro*}

\section[Behaviour of the generalised solution...]{Behaviour of the generalised solution at the\hfill\break boun\-dary
Regular boundary points${}^1$}\label{p4:chap6:sec29} %%sec29
\footnotetext[1]{In the note \cite{2} the question was
  studied as an application of thin sets. A part of the study, as it
  is developed here, needs only a weaker hypothesis.} 
 
\begin{defn}\label{p4:chap6:sec29:def20}% definition 20
  A point $x_0 \in \partial \omega$ (for the open set $\omega \subset
  \bar{\omega} \subset \Omega$) is said to be regular, if for every
  finite continuous function $\theta$ on $\partial \omega$, 
  $$
  H^\omega_\theta (x) \to \theta (x_0). (x \in \omega,  x \to x_0 ).
  $$
\end{defn}

\begin{prop}\label{p4:chap6:sec29:prop18} % proposition 18
  $\omega$ is regular if and only if all boundary points are regular.\pageoriginale
\end{prop}

\begin{prop}\label{p4:chap6:sec29:prop19} % proposition 19
  If $f$ is an upper bounded function on $\partial \omega, x_0$ a
  regular boundary point then, 
  $$
  \lim_{x \in \omega, }.\sup_{x \to x_0}. \bar{H}^\omega_f (x) \leq
  \lim_{y \in \partial \omega,}.\sup. \underset{y \to
    x_0}f(y). 
  $$
\end{prop}

\begin{proof}
  If the right hand side $\lambda$ is $< + \infty$, let $\lambda' >
  \lambda$ and $\theta$ a finite continuous function on $\partial
  \omega$, such that $\theta (x_0) < \lambda'$, $\theta (y) \geq f(y)$
  on $\partial \omega$. Then $\bar{H}_f \leq H_\theta,
  \underset{x \to {x_0}}{\lim.\sup\limits}. \bar{H}_f \leq \theta(x_0) \leq
  \lambda'$. 
\end{proof}

Hence the proposition.

\noindent
\textbf{Corollaries}
\begin{enumerate}[(i)]
\item for any bounded and resolutive function $f$ which is continuous
  at a regular boundary point $x_0, H_f (x) \to f(x_0)$. 
\item For any superharmonic function $V \geq 0, \hat{R}^{CW}_V(x_0)$
  if $x_0$ is a regular point.  
\end{enumerate}

In fact $\bar{H}^\omega_V = R^{CW}_V$ in $\omega$ (no. $15$) and
$\lim_{x \in \omega}. \inf_{x \to x_0}. \underbar{H}_V (x) \geq V(x_0)
\qquad (\text { Prop. } 19) $

There fore 
\begin{align*}
 \lim_{x \to}. \inf_{x_0}. R^{C \omega}_V
  & \geq V(x_0),\\ 
  \hat{R}^{C \omega}_V & \geq V(x_0).
\end{align*}
As $\hat{R}^{C \omega}_V \leq V$ the result is true.

\noindent
\textbf {Criteria of Regularity} (we assume axioms
\ref{p4:chap1:sec1:axiom1}, \ref{p4:chap1:sec1:axiom2},
\ref{p4:chap1:sec1:axiom3} on 
$\Omega$ and the existence of\pageoriginale a potential $> 0$ on $\Omega$). 

\begin{thm}\label{p4:chap6:sec29:thm20} % theorem 20
  $x_0 \in \partial \omega$ is a regular point if and only if for every
  finite continuous potential $V, \hat{R}^{C \omega}_V (x_0) \geq
  V(x_0)$. 
\end{thm}

\begin{proof}
  The necessary part has already been proved. Conversely the
  inequality implies in particular, 
  $$
  {\underset{x \to x_0,x \in \omega}{\lim.\inf. R^{C\omega}_V}}\geq  V(x_0)
  $$ 
  As $V \geq R^{C \omega}_V = H^\omega_V$ in $\omega$, we deduce
  $H_V(x) \to V(x_0)$ and because of the approximation theorem $(15)$
  we conclude $H_\theta (x) \to \theta (x_0)$ for any finite
  continuous function $\theta$ on $\partial \omega$. 
\end{proof}

\begin{coro*}
  If $\omega_1 \subset \omega_2 \subset \bar{\omega}_2 \subset
  \Omega$, ~  $x_0 \in \partial \omega_1 \cap \partial \omega_2$ the
  regularity of $x_0 \in \partial \omega_2$ implies the regularity for
  $\omega_1$ (for $\hat{R}^{C \omega_1}_V \geq \hat{R}^{C
    \omega_2}_V$)  
\end{coro*}

\begin{prop}[Local character of
    Regularity]\label{p4:chap6:sec29:prop20} %%%prop 20).  
  If $x_0 \in \partial \omega$ is regular for $\omega$, it is
  regular for $\omega \cap \delta$ for any open neighbourhood $\delta$
  of $x_0$; conversely if $x_0$ is regular for one such $\omega \cap
  \delta$, it is regular for $\omega$. 
\end{prop}

The first part follows from the last corollary.

As regards the converse we consider the function $V'$ defined in
$\bar{\omega}$ as equal to $V$ in $\partial \omega$ and to
$H^\omega_V$ in $\omega$. $V'$ on $\partial \omega'(\omega' = \omega
\cap \delta$) is resolutive [Th. \ref{p3:chap4:sec12:thm10}, Ch. $IV$ and and analogous
  result]. Now $H^\omega_V = H^{\omega'}_{V'}$ on $\omega'$ and now it
follows [from Cor. $1$, Prop. \ref{p4:chap6:sec29:prop19}] that $H^{\omega'}_{V'} \to V'$ at
$x_0 \in \partial \omega$, therefore, $H^\omega_V \to V$ at $x_0 \in
\partial \omega$ i.e.,\pageoriginale the point $x_0$ is regular on $\partial
\omega$. 

\begin{thm}\label{p4:chap6:sec29:thm21}   % theorem 21
  Let $V_0$ be a potential $> 0$, finite and continuous at a point $x_0
  \in \partial \omega$. For $x_0$ to be regular point of $\omega$ it is
  necessary and sufficient that for every neighbourhood $\sigma$ of
  $x_0$, $\hat{R}^{\sigma \cap C \omega}_{V_0} (x_0) = V_0 (x_0)$. 
\end{thm}

\begin{proof}
  We can suppose $\sigma$ to be compact.
\end{proof}

Suppose $x_0$ is a regular point. Let $\sigma$ be a compact
neighbourhood of $x_0$ and $\Omega_1$ a relatively compact open set
such that $(\sigma \cap \bar{\omega}) \subset \Omega_1$. Let
$\Omega'_1 = \Omega_1 - (\sigma \cap C \omega)$ and $g$ the function
equal to $V$ on $\partial (\sigma \cap C \omega_1)$ and zero on
$\partial \Omega_1$. Obviously $R^{\sigma \cap C \omega}_{V_0} \geq
\bar{H}^{\Omega'_1}_g \geq \underbar{H}^{\Omega'_1}_g$ in
$\Omega'_1$. Since $x_0$ is also a regular point of $\Omega'_1,
\lim. \inf. \underbar{H}^{\Omega'_1}_{g} (x)\geq V_0 (x_0)$. 
$$
\displaylines{\hfill 
  x \in \Omega'_1, x  \to x_0 \hfill \cr
  \text{ Hence }\hfill \hat{R}^{\sigma \cap C \omega}_{V_0}(x_0) 
  \geq V_0 (x_0).\phantom{Hence}\hfill }
$$

Conversely, if possible let $x_0$ be irregular. Then there exists a
finite continuous potential $V > 0$ such that $\hat{R}^{C \omega}_V
(x_0) < V(x_0)$. We may choose $\lambda > 0$ such that $\hat{R}^{C
  \omega}_V (x_0) < \lambda V_0(x_0) < V(x_0)$. 
For a suitable compact neighbourhood $\sigma $ of $x_0$, $\lambda V_0
< V$ on $\sigma$. 

Therefore, $\hat{R}^{\sigma \cap C \omega}_{\lambda V_0} \leq
\hat{R}^{\sigma \cap C' \omega}_V (x_0) < \lambda V_0(x_0)$. Hence $\hat{R}^{\sigma
  \cap C \omega}_{V_0} V_0(x_0)$. This is clearly a contradiction. The
Theorem is proved. 

\begin{thm}\label{p4:chap6:sec29:thm22} % theorem 22
  Suppose\pageoriginale there exists on $\omega$, or only on $\omega \cap \delta_0$
  for an open neighbourhood $\delta_0$ of $x_0 \in \delta \omega$, a
  superharmonic function $w > 0$, which tends to zero when $x \to
  x_0$. Then $x_0$ is regular for $\omega$. Then converse is true when
  $\omega$ is a countable union of domains (for instance when  $\Omega$
  has a  countable base for open sets). 
\end{thm}

\begin{proof}
  Let $\theta$ be any finite continuous function on $\partial \omega$,
  we have to prove that $H_\theta (x) \to \theta (x_0)~ (x \in
  \omega)$. We may suppose $\theta \geq 0$ and actually $\theta = 0$
  in a neighbourhood of $x_0$, for, the general case can be deduced
  from this one. 
\end{proof}

Let us introduce a regular domain $\delta$ containing $x_0$, and such
that $\bar{\delta}\subset \delta_0$ and $\theta = 0$ on $\partial
\omega \cap \bar{\delta}$. Let $\theta_0$ be the continuation of
$\theta$ by the function $H^\omega_\theta$ defined on $\omega$ and we
remark $H^\omega_\theta = \bar{H}^{\omega \cap \delta}_\theta$
(Theorem \ref{p3:chap4:sec12:thm10}). 

We may choose a compact set $\sigma$ on $\partial \delta \cap \omega$
such that $\sigma' = (\partial \delta \cap \omega) - \sigma $
satisfies $\int \limits_{\sigma'} d \rho^\delta_{x_0} < \varepsilon$
(for a $\varepsilon > 0$). With a suitable $\lambda > 0$, the function
$w_1 = \lambda w + (\sup. \theta_0) \int \limits_{\sigma'} d
\rho^\delta_x$ is in $\omega \cap \delta$ a superharmonic function $>
0$ whose $\lim. \inf$. at the boundary is $\geq \theta_0$. Hence $w_1
\geq \bar{H}^{\omega \cap \delta}_{\theta_0}$. 

On the other hand, $\lim. \sup. (w_1) \leq \varepsilon
(\sup. \theta_0)$. The same holds for $\bar{H}^{\omega \cap
  \delta}_{\theta_0}$ or $H^\omega_\theta$. As $\varepsilon$ is
arbitrary, $H^\omega_\theta (x) \to 0 ~(x \in \omega,  x \to x_0)$. 

The\pageoriginale converse with the additional hypothesis is an application of
prop. \ref{p3:chap4:sec12:prop12} (extended) because the set of
components $\omega_i$ is 
countable. If $V$ is a finite continuous potential $> 0$, which is not
harmonic in every $\omega_i,  V - H^\omega_V > 0$ in $\omega$ and
tends to zero when $x \in \omega$ tends to $x_0$. 

\begin{coro*}
  If $x_0 \in \partial \omega$ is regular for $\omega$, it is exterior
  or regular to every component $\omega_i$ of $\omega$; the converse
  is true $\big \{ \omega_i \big\}$ is  countable. \footnote{This
    restriction is unnecessary.} 
\end{coro*}

The first part is an immediate consequence of the definition of
regularity. As regards the converse we introduce superharmonic
functions $w_i > 0$ ($w_i \leq $ a fixed finite continuous potential $>
0$) such that, in case where $x_0 \in \partial \omega_i, w_i \to 0$
$(x \in \omega_i, x \to x_0)$. Then $\sum \dfrac{1}{n^2} w_i$ is
superharmonic $> 0$ in $\omega$, and tends to zero $(x \in \omega, x
\to x_0)$. 

We shall study later the set of irregular points as an application of
a theory which needs a further axiom. 

\section{Third Example of Dirichlet Problem}\label{p4:chap6:sec30}% section 30

The argument of Theorem \ref{p4:chap6:sec27:thm19} can be generalised
as follows: We have 
\begin{enumerate}[(i)]
\item a compact and metric space $\varepsilon$ (instead of the latter
  condition we may assume that any open set in $\varepsilon - \Omega$
  is a $K_\sigma$-set. 
\item for any superharmonic function $v$ on $\Omega$, the condition
  for every $y \in \varepsilon - \Omega$, $\underset{x \to
    y}{\lim. \inf.} v(x) \geq 0 ~ (x \in \Omega)$ implies $v \geq
  \Omega$. 
\end{enumerate}
(This\pageoriginale condition is satisfied for instance when there exists a harmonic
function $h > \varepsilon > 0$ on $\Omega$). 

We essentially take $\mathscr{L}_1$ and $\sum_1$ we suppose further
that the finite continuous functions on $\varepsilon - \Omega$ are
resolutive. 

%\setcounter{Proposition}{20}
\begin{prop}\label{p4:chap6:sec30:prop21} % proposition 21
  Under the above conditions, for any finite continuous function
  $\theta$ on $\varepsilon - \Omega$, $\mathscr{H}_\theta(x)$ defines
  a Radon integral $\int \theta d \mu_x$; then $\bar{\mathscr{H}}_f
  (x)\break =\bar{\int} f d \mu_x$. The set of resolutive functions is
  identical with that of $d \mu_x$- summable functions (which is
  independent of $x \in \Omega$); hence resolutive functions are
  absolutely resolutive and for such functions $f, \mathscr{H}_f (x) =
  \int f d \mu_x$. The Daniell continuation of $\mathscr{H}_\theta
  (x)$ gives the general Daniell integral of Theorem
  \ref{p4:chap5:sec25:thm18}, which is here identical to the previous
  Radon integral.  
\end{prop}
