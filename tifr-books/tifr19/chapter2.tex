\part{General Capacities of Choquet and Capacitability}\label{p2} %%II

\chapter{Capacitability}\label{p2:chap1}

\section{True capacity and capacitability}\label{p2:chap1:sec1}

Let\pageoriginale us give some notions and results of the theory of capacity,
developed farther and under somewhat more general conditions by
$G$. Choquet. It was inspired by classical capacity but is is now a
general basis tool in analysis. We introduce here some new terms
characterizing notions that will be used often. Let $E$ be a Hausdorff
space. 

\begin{defn} \label{p2:chap1:sec1:def1} %%1
  A real valued set function $\varphi$ (finite or not ) defined on the
  class of all subsets of $E$ will be called a \textit{ true capacity
  } if it satisfies the following conditions: 
  \begin{enumerate}[\rm(i)]
  \item $\varphi$ is an increasing function.
  \item For any increasing sequence $\{ A_n \}$ of subsets of $E$ 
    $$
    \varphi(\cup A_n) \text{ or } \varphi(\lim_{n \to \infty} A_n)
    =\lim_{n \to \infty} \varphi (A_n) \text{ or } \sup \varphi(A_n) 
    $$
  \item For any decreasing sequence $\{K_n \}$ of compact subsets of $E$ 
    $$
    \varphi(\cap K_n) \text{ or } \varphi(\lim_{n \to \infty} K_n)
    =\lim_{n \to \infty} \varphi (K_n) \text{ or } \inf \varphi(K_n) 
    $$
    A first example of true capacity would be an outer measure induced
    by a positive\footnote{positive measure (or set function) will
      mean measure (or set function) $\geq 0$. We understand positive
      in the sense $\geq 0$, but in order to avoid any trouble, we
      shall write the symbol $\geq 0$ instead of the word.} measure on
    a locally compact Hausdorff space. (See any book on measure
    theory). 
    
    As for the notations we shall follow $G$. Choquet. Consequently 
  \end{enumerate}
\end{defn}

$K-$set\pageoriginale will stand for compact set, $K_{\sigma \delta}$- set for a set
which is countable union of $K$-sets and $K_{\sigma \delta}$ - sets
for a set which is countable intersection of $K_{\sigma}$-sets. 

\begin{defn}\label{p2:chap1:sec1:def2}% def 2
  A $K$-analytic set of $E$ is one which is the continuous image of a
  $K_{\sigma \delta}$ set contained in a compact Hausdorff space. 
\end{defn}

Any class of subsets of $E$, closed under countable unions and
intersections is called a Borelian field. In particular the smallest
Borelian field containing the compact sets of $E$ is called the $K$-
Borelian field, the elements of this family being termed $K$-Borel
sets. It is noted that a $K$-Borelian set is a $K$-analytic set and
that in a complete separable metric space any borelian set (in the
ordinary sense) is homeomorphic to a $K$-Borelian set, therefore is a
$K$-analytic set. 

\begin{defn} \label{p2:chap1:sec1:def3}%def 3
  Any set $A$ of $E$ is said to be $\varphi$- capacitable if 
  $$
  \varphi(A) = \sup \{ \varphi (K): \text{ for compact sets } K
  \subset A \}. 
  $$
\end{defn}

In order to prove that $K$-analytic sets contained in compact sets are
capacitable with respect to any true capacity, we need some steps; we
first examine how an inverse image transforms a true capacity. 

\setcounter{Lemma}{0}
\begin{Lemma}\label{p2:chap1:sec1:lem1}% lem 1
  Let $f$ be a continuous map of a Hausdorff space $F$ into another
  Hausdorff space $E$. Let $\varphi$ be a true capacity on $E$. 
The set function $\chi$ defined on $F$ by $\chi (A)= \varphi [f(A)]$
is a true capacity on $F$; and the image $f(A)$ of a
$\chi$-capacitable set $A$ of $F$ is $\varphi$- capacitable. 
\end{Lemma}

\begin{proof} %pro
  $\chi$ is an increasing set function. If $\{ A_n \}$ is an increasing
  sequence\pageoriginale of sets of $F, f(A_n)$ is increasing, therefore 
  \begin{align*}
    \chi (\cup A_n) = \varphi[f(\cup A_n)]& = \varphi[ \cup f(A_n)]\\
    & = \lim_{n \to \infty} \varphi [f(A_n)]\\
    & = \lim_{n \to \infty} \chi (A_n)
  \end{align*}
\end{proof}

Finally if $K_n$ is any decreasing sequence of compact sets of $F$
with elementary topological considerations $f(\cap K_n) = \cap f(K_n)$
and therefore  
$$
\chi (\cap K_n) = \varphi[f(\cap K_n)] = \varphi[\cap f(K_n)] =
\inf.\varphi(f(K_n)) = \inf. \chi (K_n). 
$$

In order to complete the lemma we have to show that the image of a
$\chi$-capacitable set by $f$ is $\varphi$-capacitable. Let $A$ be a
$\chi$- capacitable set and suppose $\chi (A) > - \infty$. For any
$\lambda < \chi (A)$, there exists a compact set $K \subset A$ such
that $\chi(K) > \lambda$. Then $f(K) \subset f(A)$ and further
$\varphi[f(K)]> \lambda$ whereas $\lambda < \varphi [ f(A)]$. This
being true for every $\lambda < \varphi (f(A))$ it follows that $f(A)$
is $\varphi$- capacitable. 

\begin{Lemma}\label{p2:chap1:sec1:lem2}%lem 2
  Every $K_{\sigma \delta}$ set of $E$ is capacitable for any true
  capacity $\varphi$ on $E$. 
\end{Lemma}  

\begin{proof} %pro
  Let $A$ be a $K_{\sigma \delta}$  set of $E$. By definition 
  $$
  A = \bigcap^{\infty}_{i=1} \bigcup^{\infty}_{j=1} A^i_j
  $$
  where\pageoriginale $A^i_j$ are compact sets of $E$. Without loss of generality we
  can assume that $A^i_j$ are increasing with $j$ for every index
  $i$. Now for $\lambda < \varphi (A)$ (if $\varphi(A) > - \infty)$ we
  are in search of a compact set $K$ contained in $A$ such that
  $\varphi (K) > \lambda$. Then the theorem will be a consequence of
  the increasing property of $\varphi$. 
\end{proof}

To this end consider the increasing sequence $A^1_n \cap A$ whose limit
is $A$. By the condition (ii) of the definition of $\varphi$, 
$$
\varphi(A) = \lim_{n \to \infty} \varphi (A^1_n \cap A)
$$

Hence there exists an integer $p_1$ such that $ \varphi (A^1_n \cap A)
> \lambda$. Next consider the increasing sequence $A^1_p{_1} \cap
A^2_n \cap A$ which increases to $A^1_p{_1} \cap A$. The same argument
enables us to find an integer $p_2$ such that $\varphi(A^1_p{_1} \cap
A^2_p{_2} \cap A)> \lambda$. It is easy to see that proceeding on the
same lines, we get by induction a sequence $\{ p_j \}$ of integers
satisfying for every $j$, 
$$
\varphi(A^1_p{_1} \cap \ldots \ldots \cap A^j_p{_j} \cap A) > \lambda
$$

Put $B_i = A^1_{p^1} \cap \ldots \ldots \cap A^i_{p}$. The sets $B_i$
are compact, decreasing and $\cap B_i \subset A$.  

$\varphi(\cap_i B_i) = \lim\limits_i \varphi (B_i) \geq
\lambda$. Hence the lemma is proved. 

\begin{thm} \label{p2:chap1:sec1:thm1}% the 1
  In a Hausdorff space $E$ any $K$-analytic set $A$ contained in\pageoriginale some
  compact set of $E$ is capacitable for every true capacity $\varphi$
  on $E$. 
\end{thm}

\begin{proof} % pro
  We may assume for simplicity that the space $E$ itself is compact.
By definition there exists a $K_{\sigma \delta}$ set $B$ in a compact
space $F$ and a map $f$ onto $A$ defined and continuous on $B$. 

Let $\Gamma$ be graph of $f$. We shall prove first that $\Gamma$ is a
$K_{\sigma \delta} $ set of $F \times E$. Since $f$ is continuous and
$E$ is Hausdorff, it is known that  $\Gamma$ is a closed subset of $B
\times E$.  $\Gamma$ is the intersection of a closed $C$ of $F \times
E$ with $B \times E$. On the other hand $B$ being a $K_{\sigma \delta}
$ set of $ F, B \times E$ is a $K_{\sigma \delta} $ set of $F \times
E$, as $E$ is compact. 
Hence $\Gamma$ is itself a $K_{\sigma \delta} $ set. For any $H
\subset F \times E, \varphi$(proj$_E H$) defines a true capacity
$\chi$ on $F \times E$ (Lemma \ref{p2:chap1:sec1:lem1}); 
$\Gamma$ is $\chi$-capacitable (Lemma
\ref{p2:chap1:sec1:lem2}); therefore its projection $A$ on $E$ is
$\varphi$-capacitable (Lemma \ref{p2:chap1:sec1:lem1}). 
\end{proof}

\begin{remark*} %rem 
  The theorem is true when $A$ is supposed to be contained just in a
  $K_{\sigma}$ set, say $\cup K_n$. It is enough to remark that $A
  \cap K_n \subset K_n$, therefore capacitable and $A \cap K_n \to
  A$. 
\end{remark*}
