\chapter{Reduced Functions and Polar Sets}\label{p4:chap7}

\setcounter{section}{30}
\section{}\label{p4:chap7:sec31}% section 31

We\pageoriginale suppose only the
axioms \ref{p4:chap1:sec1:axiom1}, \ref{p4:chap1:sec1:axiom2}
$\&$ \ref{p4:chap1:sec1:axiom3}, and go to complete first the 
properties of the reduced functions, $R^E_v$ for $\Omega$ or
$(R^E_v)_\omega$ for an open set $\omega \subset \Omega$, relative to
the set $E$ and the superharmonic function $v \geq 0$ (on $\Omega$ or
$\omega$). 

\setcounter{Lemma}{0}
\begin{Lemma}\label{p4:chap7:sec31:lem1}% lemma 1
  $(R^{E \cap \omega}_v)\omega \leq R^{E \cap \omega}_v \leq R_v^{E}\leq R^{E \cup
    C \omega}_v$ and the difference between the extreme terms is
  majorised by $R^{C \omega}_v$ in $\omega$. 
\end{Lemma}

The only non-trivial part is the last result. It is obvious if $\omega
= \Omega$. If not, let $x_0$ be any point in $\omega$. Then for any
$\lambda > (R^{E \cap \omega}_v)_\omega (x_0)$ and $\lambda_1 > R^{C
  \omega}_v (x_0)$, there exist superharmonic functions $w \geq 0$ and
$w_1 \geq 0$ respectively in $\omega$ and $\Omega$ such that, 
\begin{align*}
  w \geq v ~ \text{ on } E \cap \omega & \quad \text{ and } \quad
  w(x_0) \leq \lambda \\ 
  w_1 \geq v ~ \text{ on } C \omega & \quad \text{ and } \quad
  w_1(x_0) \leq \lambda _1.  
\end{align*}

Let us define $w'$ equal to $v$ on $C \omega$ and to $\inf.(w_1 +w,v)$
on $\omega$.This is a superharmonic function $\geq 0$, on $\Omega$ and
this $w' \geq v$ on $E \cap C \omega$; therefore $w' \geq R^{E \cup C
  \omega}_v$. Hence $R^{E \cup C \omega}_v(x_0) \leq w' (x_o) \leq w'
(x_0) \leq \lambda + \lambda_1$. This being true of any $\lambda > (R_v 
^{E \cap \omega})_\omega (x_0)$ and $\lambda_1 > R^{C \omega}_v(x_0)$,
we get 
$$
R^{E \cup C \omega}_v (x_0) \leq (R^{E \cap \omega}_v)_\omega (x_0) +
R^{C \omega}_v (x_0) 
$$

Now\pageoriginale it is obvious that the inequality is true at all points $x_0 \in \omega$.

\begin{coro*}
  If $v$ is a potential, $(R^{E \cap \omega}_v)_\omega \to R^E_v$ at
  any point, according to the directed set $\omega \subset
  \bar{\omega} \subset \Omega$. 
\end{coro*}

In fact, $R^{C \omega}_v$ tends to the greatest harmonic minorant of
$v$, i.e.  zero. 

\begin{thm}\label{p4:chap7:sec31:thm23}% theorem 23
  Let $V > 0$ be a finite continuous superharmonic function in $\Omega$
  and $x_0
  \in \Omega$. Then $R^E_v(x_0)$ defines on compact sets $E$ a strong
  capacity such that for any set $E$, the corresponding outer capacity
  is $R^E_v(x_0)$. 
\end{thm}

\begin{proof}
  The case in which there exists no potential $> 0$ is obvious. Hence
  we assume the existence of a finite continuous potential $V_0 > 0$. 

  Let us first prove that $R^E_V(x_0)= \inf. R^\omega_V(x_0)$ for all
  open sets $\omega$ containing $E$. Let $v$ be a superharmonic function
  $\geq 0$ such that $v \geq V$ on $E$ and $R^E_V(x_0)+\varepsilon \geq
  v(x_0)$, (for some choice of $\varepsilon > 0$). The inequality $v
  \geq V(1-\varepsilon)$ holds in an open set $\omega \supset E$,
  therefore $\dfrac{v}{1-\varepsilon} \geq R^\omega_V$. Then 
  $$
  \displaylines{\hfill 
    (1- \varepsilon)R^\omega_V (x_0) \leq v(x_0) \leq R^E_V (x_0) +
    \varepsilon \hfill \cr
    \text{ i.e., } \hfill R^\omega_V (x_0) - R^E_V (x_o) \leq
    \varepsilon (1+ V(x_0))\hfill } 
$$
  Hence the $R^E_V(x_0) = \inf. R^\omega_V(x_0)$ as required.
\end{proof}

Now, suppose $E$ is a compact set. $R^E_V(x_0)$. is an increasing
function of $E$. The above property shows that this function is
continuous on the right. We shall show that this set function is
strongly\pageoriginale subadditive, i.e., 
$$
R_V^{E_1 \cup E_2}+  R^{E_1 \cap E_2}_V \le R^{E_1}_V + R^{E_2}_V
$$

We may suppose that $V$ is a potential by taking inf. $(V, \lambda
V_o)$ instead of $V$, where $\lambda$ is so chosen that $\lambda V_o
\ge V$ on $E_1 \cup E_2$. (The in-equality will be unaltered.) The
inequality is first of all true for all points of $E_1 \cup E_2$; for
instance if $x_o \in E$, 
$$
R_V^{E_1 \cup E_2} = V =   R^{E_1}_V ,   ~R^{E_1 \cap E_2}_V  \le  R^{E_2}_V.
$$

If we take two superharmonic functions $v_1 \ge 0$ and $v_2 \ge 0$
such that  $v_1 \ge V$ on $E_1$, $v_2 \ge V$ on $E_2$ then it is
enough to prove that 
$$
R_V^{E_1 \cup E_2}+  R^{E_1 \cap E_2}_V \le v_1 + v_2.
$$

Hence consider in $C (E_1 \cup E_2)$ the function 
$$
D = v_1 + v_2 -\left(R_V^{E_1 \cup E_2}+  R^{E_1 \cap E_2}_V\right).
$$ 

The reduced functions being everywhere upper semi-continuous, we have
for $x \in C (E_1 \cup E_2)$ and $x \to y \in \partial (E_1 \cup E_2)$ 
$$
\lim.  \sup.  R_V^{E_1 \cup E_2}+   R^{E_1 \cap E_2}_V  \le  R^{E_1
  \cup E_2}_V(y) + R_V^{E_1 \cap E_2} (y) \le  R^{E_2}_V (y) +
R^{E_2}_V (y) 
$$
and hence, $\lim. \inf. D \ge 0$.

Now in $C (E_1 \cup E_2), D \ge - 2V$, hence by
Prop. \ref{p4:chap4:sec20:prop9}(ii) 
$n^\circ.  20), D \ge 0$. This proves the strong subadditivity. 

We\pageoriginale shall see that for any open set $\omega \subset \Omega$,
$$
R^\omega_V = \sup_{K \subset \omega}. R^K_V \quad (K \text { compact)},
$$
which implies that $R^\omega_V (x_o)$ is the corresponding inner
capacity of $\omega ~ ^.$, and finally that $R^E_V (x_o)$ is the outer
capacity for any set $E$. 

In fact, $\hat{R}^K_V \le V ; \hat{R}^K_V$ tends, according to the
directed set of $K \subset \omega$, to a superharmonic function $\ge
0$, which is equal to $V $ on $\omega$, therefore majorises
$R^\omega_V$. We conclude, 
$$
\sup_{K \subset \omega}. R^K_V = \sup_{K \subset
  \omega}. \hat{R}^K_{\bar{V}} = R^\omega_V = \hat{R}^\omega_V. 
$$

\section{Polar sets}\label{p4:chap7:sec32}%sec 32

\begin{defn}\label{p4:chap7:sec32:def21} % def 21
  A set $E$ is said to be polar in the open set $\omega_ \circ \subset
  \Omega$, if there exists in $\omega_\circ$ a superharmonic function
  $v \ge 0$ (or equivalently a potential), called associated to $E$,
  equal to $+ \infty$ at least on $E \cap \omega_o$. 
\end{defn}

  The term ``quasi-everywhere in $\omega_\circ$'' ``will mean except
  for a set polar in $\omega_\circ$''.  

  The property of a set being polar in $\omega_\circ$ is equivalent to
  its being polar in every component of $\omega_\circ$. 
  
  A set will be said to be polar from inside if every compact subset is polar.

\medskip
  \noindent
  \textbf{First Properties of Polar sets in $\Omega$:}\pageoriginale
  
  \begin{enumerate} [(i)]
  \item For any open set $\omega$, relatively compact, a polar set $E$
    on $\partial \omega$ is negligible for $\omega$, because
    $\bar{H}^\omega_{\varphi_E}$ is majorised by any function
    associated to $E$. 

    $CE$ is therefore dense (for a polar set $E$); any hyperharmonic
    function is determined by its values outside a polar set $E$ (at
    $x \in E$, $v(x) \lim \int vd \rho^\omega_x $ according to the
    filter $\mathscr{F}_x$ of section of the decreasing directed set
    of all regular domains containing $x$).  
     
    A function will be said to quasi-hyperharmonic or quasi
    superharmonic if it is quasi-everywhere equal to a hyper
    (\resp super) harmonic function; in both the cases the latter
    function is unique. 
  \item Any countable union of polar sets is polar.

    We choose each $E_n$ an associated function $v_n$ such that $\int
    v_n d \rho^{\omega_o}_{x_o} = \dfrac{1}{n^2} (\omega_o$ regular,
    $x_o \in \omega_o)$, then $\sum v_n$ is associated to $\cup E_n$. 
  \item If $E$ is a closed set polar in $\Omega$, then $CE$ is
    connected. 
    
    If not, $CE = \omega_1 \omega_2,  \omega_1 \neq \phi,  \omega_2
    \neq \phi, \omega_1 \cap \omega_2 = \phi$ and $\omega_1$ and
    $\omega_2$ open sets. Let $w$ be an associated function (to
    $E$). By changing the value of $w$ into $+ \infty$ on $\omega_2$
    we get a hyper harmonic function $w' (w'$ satisfies the local
    criterion). $w' = + \infty$ on $\omega_2$ and it is not
    identically $+  \infty$, this is a contradiction. 
  \item Let $E$ be a closed polar set in $\Omega$. A superharmonic
    function $v$ in $\Omega - E$, supposed to be lower bounded on
    every compact subset of $\Omega$, can be uniquely continued on $E$
    to a superharmonic function. 
  \end{enumerate}

Let\pageoriginale $w$ be associated to $E$. Define the sequence $v_n$ of
superharmonic functions as : $v_n = + \infty$ on $E$ and $v_n = v +
(1/n) w$ in $CE$ uniformly lower bounded on every compact set. Then
inf. $v_n$ is a nearly superharmonic function $V$, and $V = v $ at any
point where $w$ is finite. We deduce, $\int V d \rho ^\omega_x = \int
v d\rho^\omega_x$ for any $\omega$ regular, $\bar{\omega} \subset
C(E)$. Hence $\hat{V} = v $ on $E, v$ is the continuation needed. 

\noindent
\textbf{Consequence}. If $h$ is harmonic in $\Omega - E$ and bounded
on every compact set of $\Omega$, there exists a unique harmonic
continuation of $h$ on $\Omega$. 

\section{Criterion}\label{p4:chap7:sec33}%sec 33

\begin{thm}\label{p4:chap7:sec33:thm24}%theo 24
  If $E$ is a polar set in $\Omega$, and $w$ a superharmonic function
  $> 0$ in $\Omega$, then $\hat{R}^E_w = 0$ everywhere and $R^E_w = 0
  ~ q.e.$ Conversely, if for one superharmonic function $w > 0$,
  $\hat{R}^E_w$=0 
  every where, or if there exists $x_o \in \Omega$ such that
  $\hat{R}^E_w (x_o) = 0$ then $E$ is a polar set. 
\end{thm}
  
 Suppose $E$ is a polar set, $v$ an associated function and $x_o$ such
 that $v(x_o) < + \infty$. Then $\lambda v \ge w$ on $E$ for any
 $\lambda > 0$, therefore $R^E_w (x_o) = 0$. Hence $R^E_w = 0$ q.e.,
 therefore on a dense set. It follows $\hat{R}^E_w = 0$. 

Conversely, suppose $R^E_w(x_o) = 0$. There exists superharmonic
function $v_n > 0$ such that $v_n \ge w$ on $E$ and $v_n (x_o) <
\dfrac{1}{n^2}$. Then $\sum v_n$ is superharmonic $> 0$ and $ = +
\infty$ on $E$. 

If\pageoriginale we suppose only $\hat{R}^E_w = 0$ at  a point, therefore
everywhere, $R^E_w$ is a nearly superharmonic function whose
regularised function is zero. Therefore $\bar{\int} R^E_w d
\rho^\omega_x = 0$ for any regular domain $\omega$ we deduce that
$R^E_w = 0$ nearly everywhere, therefore at least at one point. 

\begin{coro*}
  For any set $F$ and a polar set $E$, and any superharmonic function
  $v \ge 0$, $\hat{R}^{E \cup F}_V ~ ~ = \hat{R}^F_v$. 
\end{coro*}

\begin{thm}[Local character Theorem]\label{p4:chap7:sec33:thm25} % them 25
  Suppose there exists a potential $> 0$ in $\Omega$. Let $E$ be a set
  which is locally polar (i.e. for some neighbourhood $\delta$ of every
  point $E \cap \delta$ is polar in $\delta$). Then $E$ is a polar set
  in $\Omega$. 
\end{thm}

\begin{proof}
  Let us suppose that $V_o$ is a continuous potential $> 0$. 
  \begin{enumerate} [a)]
  \item Suppose first $E \subset \bar{E} \subset \delta$ where
    $\delta$ is a regular domain, and $E$ polar in $\delta$. If $v$ is
    a superharmonic function in $\delta$, associated to $E$, we know
    (Theorem \ref{p4:chap4:sec21:thm14}) that there exists in $\Omega$ a potential which is
    equal to $v$ on an open set containing $\bar{E}$ upto a harmonic
    function. This shows $E$ is polar in $\Omega$. 
  \item Let us suppose that $E$ is relatively compact. To every $x \in
    \Omega$, we associate a neighbourhood $\delta$ such that $E \cap
    \delta$ is polar in $\delta$. Let us introduce an open set
    $\delta''$ and a regular domain $\delta'$ such that $\delta''
    \subset \bar{\delta}''\subset \delta' \subset \delta$. $E \cap
    \delta''$ is polar in $\delta'$ therefore in $\Omega$. We cover
    $\bar{E}$ by a finite union of such $\delta''$ that we call
    $\delta''_i$. The sets $\delta''_i \cap E$ are polar in $\Omega$,
    and so also is $E$. 
  \item (general case) We consider $E$ as the union of two sets $E_1$
    and $E_2$ whose exteriors are not empty. Suppose $x_o \notin
    \bar{E}_1$. For any relatively compact\pageoriginale set $\omega \ni x_o, E \cap
    \bar{\omega}$ is polar in $\Omega [(b)]$, therefore $E_1 \cap
    \omega$ is polar in $\omega$ and $\hat{R}^{E_1 \cap \omega}_{V_o}
    = 0 $; $R^{E_1 \cap \omega}_{V_o} (x_o) = 0$. As $R^{E_1 \cap
      \omega}_{V_o} \to R^{E_1}_{V_o}$ according to the directed
    family of the considered sets $\omega$ (lemma $1$, cor. $1$) we
    conclude $R^E_{V_o}(x_o) = 0$. Now the theorem follows
    immediately. 
  \end{enumerate}
\end{proof}

\begin{prop}\label{p4:chap7:sec33:prop22} %sec  22
  A $K$-analytic set $E_o$ (in the sense of Choquet) which is polar from
  inside is polar. 
\end{prop}

Suppose $E_o$ non-empty and $V_o > 0$ a continuous potential in
$\Omega$. For any compact set $K, E_o \cap K$ is a $K$-analytic set, the
capacity $R^E_{V_o}(x_o)$ is equal to zero for any compact set $E
\subset E_o \cap K (x_o \notin E_o \cap K)$ therefore also for $E_o
\cap K $; as $E_o \cap K$ is polar for any $K, E_o$ is polar. 
