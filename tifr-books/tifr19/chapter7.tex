\chapter{\texorpdfstring{$G$}{G}-capacity\texorpdfstring{${}^1$}{1}}\label{p3:chap4}


\setcounter{section}{8}
\section{\texorpdfstring{$G$}-capacity and \texorpdfstring{$G$}-capacity measures}\label{p3:chap4:sec9}%sec 9

For\pageoriginale any lower semi-continuous kernel $G \geq 0$ on a
locally compact 
Hausdorff space $E$, we shall define $G$ and $G^*$\footnote{In the
  chapter \ref{p3:chap4} and \ref{p3:chap5} we develop some of the
  ideas of Choquet $[8,9]$. This chapter also contains some new
  results like theorems \ref{p3:chap4:sec11:thm8},
  \ref{p3:chap4:sec12:thm10} and proposition   
 \ref{p3:chap4:sec12:prop12}.} -capacities which will be used to
characterise exceptional sets 
in other and similar convergence theorems.  

\begin{defn}\label{p3:chap4:sec9:def12}%def 12
  The $G$- capacity of any compact set $K$ is defined by $G$-cap $(K) =
  \sup.  \{\mu (K):S \mu \subset K, G \mu \leq 1~\text{ on }~E\}$. 
\end{defn}

\setcounter{Lemma}{3}
\begin{Lemma}\label{p3:chap4:sec9:lem4}%lem 4
  Let us consider measures $\mu_i (i \in I)$ on a compact set $K$,
  such that $\mu_i (K)$ is bounded, and $\mu_i$ converges (in vague
  topology) to $\mu $ according to a filter on $I$ and satisfy $G
  \mu_i \leq 1$. Then $\mu _ i (K) \to \mu (K)$  and $G \mu \leq 1$.  
\end{Lemma} 

For if $f \in \mathscr{K}(E), \int f d \mu _i \to \int f d \mu $ and
if $0 \leq f (y) \leq G(x, y )$ for a fixed $x, \int f d \mu _1 \leq 1
\Rightarrow \int f d \mu \leq 1$. Hence follows the inequality $\int
G(x, y) d \mu (y) \leq 1$. 

\begin{prop}\label{p3:chap4:sec9:prop7}%pro 7
  If for a compact set $K$, $G$ - cap.  $(K)$ is finite, there exists
  measures $\mu$, called $G$-capacitary measures, such that $G \mu \leq
  1$ and $\mu (K)  = G- cap.  (K)$.  
\end{prop}

\begin{proof}
  It is possible to find a sequence of measures $\mu_n$ on $K$ such
  that $G \mu_n \leq 1$ and $\mu_n(K) \to G$-cap $(K)$. Since $\mu_n
  (K)$ is bounded, there exists a subsequence $\mu _{ n_p}$ which
  converges to a measures $\mu$ on $K$. The\pageoriginale previous lemma asserts
  that this $\mu$ satisfies the required conditions. 
\end{proof}

\begin{remark*}
  The set of $G$-capacity measures of $K$ is a convex and compact set. 
\end{remark*}

\begin{thm}\label{p3:chap4:sec9:thm7}%the 7
  The function $G$-cap. is a weak capacity on the compact sets of $E$.
\end{thm}

\begin{proof}
  The $G$-cap. is obviously an increasing function. It is
  subadditive. Let $K_1$ and $K_2$ be a compact sets, $\mu$ any
  measure on $K_1 \cup K_2$ such that $G \mu \leq 1$; then  
  $$
  \displaylines{\hfill 
  \mu (K_1 \cup K_2) \leq \mu (K_1)+\mu (K_2) \leq G
  -\text{cap}. (K_1) + G - \text{cap}. (K_2)\hfill \cr
  \text{and}\hfill  
  G-\text{cap}. (K_1 \cup K_2) \leq G-\text{cap}. (K_1) +
  G-\text{cap}. (K_2)\hfill} 
  $$

  Finally let us prove that $G$- cap. is continuous to the right on
  compact sets. Assuming that the contrary is true,  we start from a
  compact set $K$ with $G$-cap. $(K) < + \infty$ and we can find, for
  a suitable $\varepsilon > 0$ and any open set $\omega$ containing
  $K$, a compact set $K_\omega $ such that  
  $$
  K \subset K _\omega \subset \omega \text{  and }
  G-\text{cap}. (K_\omega) > G-\text{cap}. (K) + \varepsilon. 
  $$

  There exists on $K_\omega$  a measure $\mu _\omega$ such that $\mu_{
    \omega } (K_\omega) = G - cap (K_\omega) + \varepsilon$ and $G \mu
  _\omega (x) \leq 1$.  We may restrict ourselves to open sets
  $\omega$ contained in a compact neighbourhood $K_1$ of $K$. The
  family of such open sets is directed for decreasing order. We take a
  filter, finer than the filter of\pageoriginale sections,  according to which
  $\mu_\omega$ converges vaguely to a measure on $K_1$. Using a
  functions $f \in \mathscr{K}(E)$ whose support $S_f$ is in $C K$, we
  see that $\int f d \mu_\omega = 0 $ when $\omega \cap S_f  = \phi$
  therefore $\int f d \mu = 0 $, that is $S _\mu \subset K$. By
  lemma \ref{p3:chap4:sec9:lem4}, 
   $\mu (K) = G - cap (K) + \varepsilon $, also $G \mu \leq 1$, 
  and hence $\mu (K) \leq G - \text{ cap}. (K)$. This is contradiction. The
  proof is complete.  
\end{proof}

\begin{prop}\label{p3:chap4:sec9:prop8}%proposition 8
  Let $\alpha$ be the set where a potential $G \mu \geq\lambda > 0$.  Then
  the inner $G^*$-capacity of $\alpha$ is $\dfrac{\mu (E)}{\lambda}$.  
\end{prop}

Let us introduce a compact set $K \subset \alpha $ and a measure $\nu$
on $K$ such that $G^*\nu \leq 1$ 
$$
\nu (K) \leq \int \frac{G \mu } {\lambda} = \frac{1}{\lambda} \int G^*
\nu d \mu \leq \frac{\mu (E)}{\lambda} 
$$ 

Therefore $G^*$-cap $(K) \leq \frac{\mu (E)}{\lambda}$ and inner
$G^*$-cap $(\alpha)\leq \dfrac{\mu (E)}{\lambda}$  
\begin{prop}\label{p3:chap4:sec9:prop9}%proposition 9
  The inner capacity $\varphi_*$ induced by the weak capacity
  $\varphi (K) = G - cap (K)$ on $E$ is subadditive on measurable
  subsets of $E$.  
\end{prop} 

\begin{proof}
  Let $A$ and $B$ be two measurable sets. Let $K$ be a compact set
  contained in $A \cup B $ such that $\varphi (K) >a$ for a choice 
  $a< \varphi _* (A \cup B)$. Let $\mu$ be a measure on $K$ such that
  $\mu (K) > a$ and $G \mu \leq 1$. For $\epsilon > 0$, we can find
  compact sets $K, K''$,
  respectively contained in $K \cap A$ and $K \cap B$ and such that
  $\mu (K \cap A)\leq \mu (K') + \varepsilon / 2$ and $\mu (K \cap B)
  \leq  \mu (K'') + \varepsilon / 2$. By considering the restriction of
  $\mu$ to $K'$ and $K''$ we obtain $\mu (K') \leq \varphi _* (K \cap
  A)$ and $ \mu (K'') \leq \varphi _* (K \cap B)$. Now we have  
  \begin{align*}
    a < \mu (K) & \leq ( K \cap A ) + \mu ( K \cap B ) \\
    & \leq \mu (K') + \mu (K'') + \varepsilon \\
    & \leq \varphi_* ( K \cap A ) +\varphi_* ( K \cap B ) + \varepsilon \\
    & \leq \varphi_* (A) + \varphi_* (B) + \varepsilon  
  \end{align*}
  $ \varepsilon > 0$\pageoriginale being arbitrary and $a$ being any number $ <
  \varphi_* ( A \cup B ) $ the subadditivity follows. 
\end{proof}

\section{\texorpdfstring{$G$}-capacity and \texorpdfstring{$G$}-negligible sets}\label{p3:chap4:sec10}

A $G$-negligible compact set has $G$-capacity zero as can be easily
seen. The converse is true $G$ satisfies the weak maximum principle or
if $E$ is compact and $G$ is regular. For if a compact set $K$ of
capacity zero is not $G$-negligible, we  may find ( Lemma
\ref{p3:chap1:sec3:lem1}) a measure
$ \mu \neq  0 $ on $K$ such that the  restriction to $ S \mu $ of $ G
\mu $ is finite and continuous. The hypothesis implies that $ G \mu $
is  bounded in the whole space and hence  $ G- cap (K) \neq 0 $, which
is a contradiction. Hence we deduce: 

\begin{prop}\label{p3:chap4:sec10:prop10}  % prop 10 
  For a regular kernel $G$ on $E$, $G$-negligible sets and sets of {\em
    inner } $G$-capacity zero are the same if either $E$ is compact or
  $G$ in addition satisfies weak maximum principle. 
\end{prop}

\section{\texorpdfstring{$G$}-capacity and strong sub-additivity}\label{p3:chap4:sec11}

In order to make the  $G-$capacity strongly subadditive we need some
more conditions on the kernel. This leads us to define two new
principles. 

\begin{defn}\label{p3:chap4:sec11:def13} % Definition 13
  {\em Principle of equilibrium ( for open sets ) -$G$-satisfies} this
  principle if for any relatively compact open set $ \omega $, there
  exists a measure $ \mu $ on $ \bar{\omega} $ such that $ G \mu \leq
  1 $ on $E$ and $ G \mu = 1 $ on $ \omega$. Without the conditions  $
  S \mu  \bar{\omega}$ and $ G \mu \leq 1 $ everywhere we shall speak
  of the `` weak principle of equilibrium''. 
\end{defn}

\begin{defn}\label{p3:chap4:sec11:def14} % definition 14
  {\em Weak Domination Principle} -$G$ satisfies this\pageoriginale principle if for
  any two measures $ \mu $ and $ \nu $, with compact supports and
  bounded  $G$- potentials, the condition $G \nu \geq  G \mu $ on $
  S \mu $ implies $ G \nu \geq G \mu $ everywhere. 
\end{defn}

\begin{prop}\label{p3:chap4:sec11:prop11} % proposition 11
  Let $ \mu_1 $ and $ \mu_2 $ be two measures with compact
  supports. Let $G^*$ satisfy the the weak principle of
  equilibrium. Then $ G \mu_1 \geq G \mu_2 $ implies $ \mu_1 (E) \geq
  \mu_2 (E)$. 
\end{prop}

\begin{proof}
  Let us  introduce a relatively compact open set $ \omega $
  containing  \break $S \mu_1 \cup ~ S \mu_2 $. There exists a measure $\nu$
  such that $G^*  \nu = 1$ on $\omega$. Now, 
  \begin{align*}
    \int G \mu_1 d \nu &\geq \int G \mu_2 d \nu \\
    \text{ Therefore }\qquad \quad \quad 
    \int G^* \nu d \nu_1 & \geq \int G^* \nu d \mu_2 \hspace{3cm}
  \end{align*}
  and hence the result.
\end{proof}

\begin{thm}\label{p3:chap4:sec11:thm8} % theorem 8
  {\em ( of strong subadditivity ) }.
\end{thm}

Let  $G$ satisfy the principle of  equilibrium and weak domination
principle, further let $G^*$ have the weak principle of
equilibrium. Then the  $G$-cap. is strongly subadditive. 

\begin{proof}
  Let $\varphi$ be the $G$-capacity. For any two compact sets $K_1$
  and $K_2$ on has to verify the condition, 
  $$
  \varphi (K_1 \cup K_2) + \varphi (K_1 \cap K_2) \leq \varphi (K_1) +
  \varphi (K_2). 
  $$
\end{proof}

Since\pageoriginale $\varphi$ is continuous to the  right, we may introduce
relatively compact  open sets  $ \omega_i ( i = 1,2 ) $ such that  $
K_i \subset \omega_i $ and  $ \varphi ( \bar{\omega}_i ) \leq \varphi
(K_i) + \varepsilon $. By the equilibrium principle, there exists on $
\bar{\omega}_i $ measure $ \mu_i $ such that $ G \mu_i \leq 1 $
everywhere and $ G \mu_i  = 1 $ on $ \omega$; hence $ \mu_i (
\bar{\omega}_i ) \leq \varphi ( \bar{\omega}_i )$. Let $\nu_1$ and
$\nu_2$ be two measures on $ K_1 \cup K_2 $ and $ K_1 \cap K_2 $
respectively such that $ G \nu_i \leq 1 $. Now, $ G \mu_i = 1 $ on
$K_i$ therefore  $ G \mu_1 \ge G \nu_1 $ on $K_1$ and  $ G \mu_2 \ge G
\nu_2 $ on $ K_2 $. As $G$ satisfies the weak domination principle, $
G \mu_2 \ge G \nu_2 $ everywhere. Hence on $ K_1, G \mu_1 + G \mu_2
\geq G \nu_1 + G\nu_2 $. The same inequality holds good on $K_2$ by a
similar argument. Now applying the weak domination principle of $G$
once again, the last inequality is  true everywhere as it holds on $
K_1 \cup K_2 $. This in turn gives, (Prop.\ref{p3:chap4:sec11:prop11}),  
$$
\mu_1 (E) + \mu_2 (E) \geq \nu_1 (E) + \nu_2 (E)
$$

Hence
$$
\varphi (K_1) + \varphi (K_2) + 2 \varphi \geq \varphi (K_1
\cup K_2) + \varepsilon (K_1 \cap K_2). 
$$
This inequality being true for any arbitrary $ \varepsilon > 0 $, the
strong subadditivity follows. 

\section{\texorpdfstring{$G$}-polar sets}\label{p3:chap4:sec12} %section 12.

\begin{defn}\label{p3:chap4:sec12:def15} %definition 15
  A relatively compact set $\alpha $ of $E$ is called a $G$-polar
  set (with respect to the kernel $G$) if there exists a measure $\mu$
  (called associated measure) with compact support such that $ G \mu =
  + \infty $ on $ \alpha $. 
\end{defn}

Any\pageoriginale set is called $G$-polar, if its intersection with every compact
set is $G$-polar. 

\begin{thm}\label{p3:chap4:sec12:thm9} % theorem 9
  A relatively compact $G$-polar set has outer $G^*$-capacity zero. 
\end{thm}

Let $\alpha$ be such a set and $\mu$ an associated measure. Let $
\omega_n = \big \{ x : G \mu > n \big \} $. By proposition $8$, we
know that the inner  $G^*$-cap. $ ( \omega_n ) \leq \dfrac{1}{n} \mu
(E) $. It follows that outer $G^*$-cap. $(\alpha)$ is zero. 

The converse of this  theorem is not true without some additional hypothesis. 

\begin{defn}\label{p3:chap4:sec12:def16} % definition 16
  A kernel $G$ satisfies  {\em complete equilibrium principle } if for
  every relatively compact open set $\omega$, there exists a measure
  $\mu \geq 0$, on the boundary $\partial \omega$ with the property  $
  G \mu \leq 1 $ everywhere, $ G \mu = 1 $ on $ \omega $. 
\end{defn}

\begin{prop}\label{p3:chap4:sec12:prop12} % prop 12
  Let $G$ be a kernel satisfying complete equilibrium  principle and
  further $G (x,y)$ be a continuous function in $y$, for every $x$, in
  the complement of the set $ \big\{ x \big\} $. Then for any
  relatively compact open set $\omega$, there exists on $\partial
  \omega$ a measure  $\mu$ satisfying $ G \mu \leq 1 $ everywhere,  $G
  \mu = 1 $ on $\omega$ and $ \mu ( \partial \omega ) \leq $ inner
  $G$-cap. $ (\omega) $. Moreover, if $G$ satisfies the ( weak )
  domination principle and $G^*$ the weak equilibrium principle we may
  replace the latter inequality by the equality.  
\end{prop}

Let us consider open sets $ \omega \underset{i}{\subset} \bar{\omega}
\underset{i}{\subset} \omega $ and $\mu_i$ on $ \partial \omega_i$
such that $G \mu_i \leq 1$, $ G \mu_i = 1$ on $\omega_i$. The set  of
such $\omega_i$ is directed for increasing order and by taking a
suitable filter finer than the filter of sections,we get a vague limit
$\mu$ of $\mu_i$. As $ \mu_i (E) \leq $ $G$-cap $ (\bar{\omega}_i)
\leq $ inner $G$-cap $(\omega)$, we deduce from Lemma $4$, $\mu(E)\leq
$ inner $G$-cap. $(\omega)$, $G \mu \leq 1 $. Moreover, by arguments
similar to the one we have used before, $ S \mu \subset \partial
\omega$\pageoriginale and if   $ x \in \omega $, $ \int G (x,y) d \mu_i (y) \to
\int ~ G (x,y) d \mu (y) $, therefore $G \mu = 1 $ on $\omega$. If  $
\lambda < $ inner $G$-cap, $ (\omega) $, we can find a compact set $ K
\subset \omega $ and a measure $\nu$ on $K$ such that $G \mu \leq 1$
and  $\nu (E) \geq \lambda $. If $ \omega_i \supset K $ then  $ G \mu_1
\geq G \nu $ on $K$ and therefore (by domination principle of $G$)
everywhere. Now follows by Prop.11, $\mu_1 (E) \geq \nu (E) \geq
\lambda$, and  the final assertion.  

\begin{thm}\label{p3:chap4:sec12:thm10}% otheorem 10.
  Let $G$ be a kernel satisfying the complete equilibrium principle
  and $G (x,y)$ for any $x$ a continuous function of $y$ outside $
  \big \{ x \big \} $. Then a relatively  compact set $\alpha$ with
  outer $G$-cap. $(\alpha) = 0 $ is a $G$-polar set.  
\end{thm}

Let $ \big \{ \omega_n \big \} $ be a decreasing sequence of
relatively compact open sets containing $\alpha$, such that inner
$G$-cap. $ ( \omega_n ) < \dfrac{1}{n^2} $. There exists on the
boundary $ \partial \omega_n $ a measure $\mu_n$ such that $ G \mu_n =
1 $ on $ \omega_n $ and $ \mu_n (E) \leq \dfrac{1}{n^2} $. Now $ \sum
\mu_n $ defines a measure $[$ by the condition $ ( \sum \mu_n ) (f) =
  \sum \mu_n (f) $ for any $f$ in $  \mathscr{K} (E) ] $. It is easily
seen that  $ \sum \mu_n $ has compact support which is contained in $
\bar{\omega}_1 $ and that $ \sum G \mu_n = G ( \sum \mu_n ) $. Hence
the $G$-potential of this measure equals $ + \infty $ on $ \alpha$ 

\begin{coro*}
  If the kernel $G$ is finite and continuous with respect to each
  variable when $ x \neq y $, and if $G$ and $G^*$ satisfy complete
  principle of equilibrium, then the $G$ and $G^*$  polar sets and the
  sets of outer $G$ and $G^*$-capacity zero are all the same. 
\end{coro*}
