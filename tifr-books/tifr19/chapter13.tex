\chapter[Sets of Harmonic and Superharmonic Functions...]{Sets of Harmonic and Superharmonic Functions Lattices and
  Potentials}\label{p4:chap4} 

\setcounter{section}{13}
\section{Saturated set of hyperharmonic functions}\label{p4:chap4:sec14}

\begin{defn}\label{p4:chap4:sec14:def11}%def 11
  A\pageoriginale set $\mathscr{U}$ of hyperharmonic functions defined on an open
  set $\omega_0 \subset \Omega$ is said to be saturated if it
  satisfies the following conditions: 
  \begin{enumerate}[\rm (i)]
  \item if $u_1,  u_2 \in \mathscr{O}$ then inf ($u_1,  u_2$) belongs
    to $\mathscr{O}$ 
  \item for any $u \in \mathscr{O}$ and any regular domain $\omega
    \subset \bar{\omega} \subset \omega_0$, the functions
    $E^{\omega}_u$, (equal to $u$ outside $\omega$ and to $\int ud
    \rho^{\omega}_{x}$ in $\omega$), again belongs to $\mathscr{O}$. 
  \end{enumerate}
\end{defn}

The intersection of any set of saturated sets of hyperharmonic
functions is again saturated. Hence there exists a least saturated set
containing any given set $S$ of hyperharmonic functions; it is called
the saturated extension $S^*$ of $S$. 

We have a similar definition and property for the saturated set of
hyperharmonic functions. 

\begin{thm}\label{p4:chap4:sec14:thm8}%them 8
  The lower envelope of any saturated set $\mathscr{O}$ of
  hyperharmonic functions on a domain $\delta$ is $+ \infty,  -
  \infty$ or harmonic. 
\end{thm} 

\begin{proof}
  Let $\omega$ be any regular domain with $\bar{\omega} \subset
  \delta$. $E^\omega_u \le u$ for every $u \in \mathscr{O}$ and $\inf
  \limits_{u \in \mathscr{O}} E^\omega_u \le \inf \limits_{u \in
    \mathscr{O}} u$. But in $\omega$, $E^\omega_u$ form a directed\pageoriginale set
  for decreasing order, of functions which are equal to $+ \infty$ or
  harmonic. Hence the infimum (lower envelope) is either $+ \infty$ or
  $- \infty$ or harmonic in $\Omega$. We see that the sets where the
  envelope in question is $+ \infty$, $- \infty$ or finite are
  disjoint open set of $\delta$. It follows that the lower envelope is
  $+ \infty$, $- \infty$ or finite (and therefore harmonic in this
  case) in the domain $\delta$. 
\end{proof} 
 
A similar theorem holds good in the case of hyperharmonic functions.
 
\section{Examples}\label{p4:chap4:sec15}

\begin{enumerate}[a)]
\item Give $E \subset \Omega$ and a function $\varphi \ge 0$ on $E$,
  then the set of all hyperharmonic functions $\ge 0$ which are $\ge
  \varphi$ on $E$ form a saturated set on $C \bar{E}$. Hence $R^E$ is
  equal in every component of $C \bar{E}$ to $+ \infty$ or a harmonic
  function. 
\item Let $\omega$ be an open set $\subset \Omega$ and $f$ any real
  (finite or not) valued function on the boundary $\partial \omega$ of
  $\omega$ in the topology of $\bar{\Omega}$ (compactified space). 
\end{enumerate}

\begin{defn}\label{p4:chap4:sec15:def12}% def12
  $\bar{H}^\omega_f $  is the lower envelope in $\omega$ of all
  hyperharmonic functions on $\omega$ satisfying $\lim. \inf. v \ge_{> -
    \infty}f$ at all points of $\partial \omega$. 
\end{defn}

\setcounter{prop}{3}
\begin{prop}\label{p4:chap4:sec15:prop4} %pro4
  In any connected component $\omega_i$ of $\omega
  ~\bar{H}^{\omega_i}_f = \bar{H}^\omega_f = + \infty$, $- \infty$ or a
  harmonic function. 
\end{prop}

\begin{thm}\label{p4:chap4:sec15:thm9}% theo9
  If $\omega$ is a regular open set $\bar{H}^\omega_f (x) = \int f d
  \rho^\omega_x$. 
\end{thm}

\begin{proof}
  If $f$ is a finite continuous function $\theta$ on $\partial
  \omega$, the behaviour\pageoriginale of $\int \theta d \rho^\omega_x$ at the
  boundary implies $\int \theta \rho^\omega_x \ge
  \bar{H}^\omega_\theta (X)$. On the other hand, for any $v$
  satisfying the boundary condition corresponding to $\theta$, $v- \int
  \theta d \rho^\omega_x$ has a lim.inf. $\ge 0$ at the
  boundary. Therefore $v \ge \int \theta d \rho^\omega_x$ and $\int
  \theta d \rho^\omega_x \ge \bar{H}^\omega (x)$ 
\end{proof}

Suppose $f$ is a lower semi-continuous function (lower bounded )
$\psi$. At the boundary $\lim. \inf. \int \psi d \rho^\omega_x \ge
\psi$, therefore $\int \psi d \rho^\omega_x \ge \bar{H}^\omega_\psi
(x)$. We see that for continuous $\theta \le \psi$,
$\bar{H}^\omega_\theta(x) = \int \theta d \rho^\omega_x \le
\bar{H}^\omega_\psi (x)$; this is true of all $\theta \le \psi$. Hence
$\int \psi d \rho^\omega_x \ge \bar{H}^\omega_\psi(x)$. 

Now for any function $f$, introduce the $v$ whose envelope is
$\bar{H}^\omega_f$ and $\psi(y) = \lim. \inf$. $V$ at the boundary
points $y$, $v \ge \bar{H}^\omega_\psi \ge
\bar{H}^\omega_f$. Therefore $\bar{H}^\omega_f$ is the lower envelope
of $\bar{H}^\omega_{f_i}$ for all $f_i$ which are lower
semi-continuous, lower bounded and $\ge f$. But $\int f d \rho^\omega_x=
\inf \limits_{i}$. $\int f_i d \rho^\omega_x$. Hence the theorem. 

\begin{prop}\label{p4:chap4:sec15:prop5}%pro5
  Suppose that for any hyperharmonic function $v$ on $\omega$, the
  condition $\lim$. $\inf$. $v \ge 0$ implies $v \ge 0$. Then if
  $\bar{v}$ is equal to $v$ on $\partial \omega \cap \Omega$ and to
  zero at the Alexandroff point of $\Omega$. Then $\bar{H}^\omega_v =
  R^{C \omega}_v$ on $\omega$. 
\end{prop}

Let $w$ be any hyperharmonic function in $\omega$ satisfying
$\lim$. $\inf$. $w \ge \bar{v}$ or the boundary. $\Inf$. ($w$,$v$)
continued by $v$ is hyperharmonic $\ge 0$ in $\omega$ and majorises
$R^{C \omega}_v$. Conversely any hyperharmonic function $w \ge 0$ in
$\Omega$, majorising $v$ on $\partial \omega \cap \Omega$ satisfies in
$\Omega \lim$. $\inf$. $w \ge \bar{v}$ at any boundary point,
therefore is $\ge \bar{H}_v$. 

\begin{thm}\label{p4:chap4:sec15:thm10}
  Let $\omega'$ be an open set $\subset \omega$, $f$ a function on
  $\partial \omega$ (in $\bar{\Omega}$), $F$\pageoriginale equal to $f$ on $\partial
  \omega$ and to $\bar{H}^\omega_f$ on $\Omega$. Then
  $\bar{H}^\omega_f = \bar{H}^\omega_F$ on $\omega'$. 
\end{thm}

We may suppose $\omega$, $\omega'$ connected. First $\bar{H}^\omega_f
\bar{H}^{\omega'}_F$. Now if $\bar{H}^\omega_f = - \infty$, the theorem
is obvious. 

If $\bar{H}^\omega_f = + \infty$, any hyperharmonic function on
$\omega'$, whose $\lim$. $\inf$. at the boundary is $F _{> - \infty}$
forms with the continuation $+ \infty$, a hyperharmonic function in
$\omega$ satisfying $\lim \inf \ge f $ at the $\ge f_{> - \infty}$ at
the boundary. Hence $\bar{H}^{\omega'}_F = + \infty$. Now suppose
$\bar{H}^\omega_f$ finite. Let $v$ be a hyperharmonic function on
$\omega'$ satisfying the boundary condition $\lim$. $\inf$. $v^1 \ge
F$; the function $\inf$. ($v^1$, $\bar{H}^{\omega}_f$) in $\omega'$,
continued by $\bar{H}^\omega_f$ is a hyperharmonic function $V$ in
$\omega$. Now let $v$ be any hyperharmonic function in $\omega$
satisfying the boundary condition $\lim$. $\inf v \ge f$. Let us study
at the boundary of $\omega$ the function $U = V + v -
\bar{H}^\omega_f$; by considering the sets where $V = v' (\text{on}~ \omega)$
and where $V = \bar{H}^\omega_f(\text{on}~ \omega)$, we conclude
$\lim$. $\inf$. $U \ge f$ therefore $U \ge \bar{H}^\omega_f$. Hence $V
\ge \bar{H}^\omega_f$, $v' \ge \bar{H}^\omega_f$ and finally
$\bar{H}^{\omega'}_f \ge \bar{H}^\omega_f$. 

\begin{remark*}
  Form this general theorem we may deduce, a shorter proof of property
  $(e) n^0 12$. 
\end{remark*}

\section{Harmonic minorants and majorants}\label{p4:chap4:sec16} % sec 16

Let $\mathscr{U}$, $\mathscr{H}$ be any two set of hypoharmonic and
hyperharmonic functions respectively on an open set $\omega \subset
\Omega$ such that for any $u \in \mathscr{U}$ and $v \in \mathscr{H}$,
we have $u \le v$. From this we deduce that for any $u \in
\mathscr{U}^*$ and any $v \in \mathscr{H}^*,  u \le v$ or
$\sup\limits_{u \in \mathscr{U}}.u_* \inf \limits_{v \in
  \mathscr{H}}.v_*$ 

For\pageoriginale the set of $u \in \mathscr{U}^*$ which are $\le $ one fixed $v \in
\mathscr{H}$ form a saturated set containing $\mathscr{U}$ and is
therefore identical to $\mathscr{U}^*$. Now the functions of
$\mathscr{H}^* $ which are $\ge$ one fixed $u \in \mathscr{U}^*$ form
a saturated set identical to $\mathscr{H}^*$ (by the same argument). 

\noindent
\textbf{Particular case.} Given $\mathscr{H}$, Let $\mathscr{U}$ be
the set of all hypoharmonic functions which are $\le$ every $v \in
\mathscr{H}$, then 
\begin{enumerate}[a)]
\item $\mathscr{U} = \mathscr{U}^*$ and does not change when
  $\mathscr{H}$ is replaced by $\mathscr{H}^*$ 
\item $V = \inf \limits_{v \in \mathscr{H}^*}.v$ is equal to $+
  \infty$, $- \infty$ or harmonic in any connected component of
  $\omega$. 
\end{enumerate}

If $V < + \infty$, it is the greatest hypoharmonic minorant of
$\mathscr{H} or \mathscr{H}^*$. $V$ is finite if and only if there
exists a superharmonic function in $\mathscr{H}$ and a subharmonic
minorant of $\mathscr{H}$; the $V$ is the greatest harmonic minorant
of $\mathscr{H} or of \mathscr{H}^*$. 

\begin{remark*}
  If $v$ is superharmonic, and $u$ subharmonic, the condition $u \le v$
  implies that the greatest harmonic minorant of $v$ majorises the
  smallest harmonic majorant of $u$. 
\end{remark*}

\section{Lattice on harmonic functions}\label{p4:chap4:sec17}% sec 17

\begin{thm}\label{p4:chap4:sec17:thm11}% them 11
  The set of harmonic functions $\ge 0$ on an open set $\omega$ is a
  lattice for the natural order(and even a complete lattice). 
\end{thm}

(A partially ordered set will be called a upper semi lattice (moreover
complete) if the set of two elements (\resp. in addition any nonvoid
upper bounded set) has a smallest majorant; and a complete lower semi
lattice\pageoriginale if the set of two elements and any nonvoid lower bounded set
has a greatest minorant. A lattice satisfying both the conditions is
called a complete lattice). 

We have only to note that if $u_1$, $u_2$ are harmonic functions $\ge
0$, $u_1 + u_2$ and $0$ are harmonic majorant and minorant of $u_1$
and $u_2$ respectively. 
\begin{coro*}
  The vector space of functions on $\omega$ which can be expressed as
  differences of two harmonic functions $\ge 0$ is for the natural
  order, a Riesz space and further a complete lattice. (See Bourbaki
  Integration, Chap. II). 
\end{coro*}

\section{Lattice on superharmonic functions}\label{p4:chap4:sec18} % sec 18

\begin{prop}\label{p4:chap4:sec18:prop6}
  The set of superharmonic functions $\ge 0$ on $\omega$ is an upper
  semi-lattice for the natural order (and further a complete lattice).  
\end{prop}

Give two superharmonic functions $v_1 \ge 0$, $v_2 \ge 0$ then $v_1 +
v_2$ is a superharmonic majorant. Given a set $\mathscr{H}$ of
superharmonic functions $v$, suppose that the set $W$ of superharmonic
majorants $w$ is non-empty. The lower envelope of $W$ is nearly
superharmonic functions $w_0$. We deduce now that $\hat{w}_0 \ge v$
from the inequality $w_0 \ge $ any $v\in \mathscr{H}$. Then $w_0$ is
the smallest superharmonic majorant of $\mathscr{H}$. The passage to
lattice-property is well known. 

\begin{defn}\label{p4:chap4:sec18:def13} % def 13
  {\em Specific order}: Let us call specific order for super harmonic
  functions on $\omega$, the order $<$ in which $v_1 < v_2$ signifies
  $v_2 = v_1 +$ (superharmonic functions $\ge 0$); this implies the
  natural order.  
\end{defn}

\begin{thm}\label{p4:chap4:sec18:thm12}% them 12
  The\pageoriginale set of superharmonic functions $\ge 0$, on $\omega$ is an upper
  semi-lattice for the specific order (and further a complete
  lattice). 
\end{thm}

If $v_1$, $v_2$ are superharmonic functions $\ge 0$, then $v_1 + v_2$
(respectively $0$) is a specific majorant (minorant). We have to prove
that if some $v_i$ are superharmonic and have a common specific
majorant (we consider all the specific majorants $w$) there is a
smallest one. 

Let $W = \inf \limits_{w}$. $w$ (infimum in the natural sense). If $w
= v_i + \omega'_i$ ( $\omega'_i$ superharmonic $\ge 0$), then let $W_i =
\inf\limits_{w} \omega'_i \ge 0$. 

Then $W = v_i + W'_i$; $W$ and $W'_i$ are nearly superharmonic functions
and $\hat{W} = v_i + \hat{W}_i > v_i$. Therefore $\hat{W}$ is a
specific majorant of the $w_i$ : $\hat{W} = W$. 

We have to see that any fixed specific majorant $w_0$ is $> W$. Let us
follow R.M. Herve and prove first that the functions $U$, equal to
$w_0 - W$ where it is defined and equal to $+ \infty$ where $w_0 = W =
+ \infty$, is a nearly superharmonic functions $\ge 0$. For any
regular domain $\omega$, we have to see that 
$$
U(x) \ge \bar{\int} Ud \rho^{\omega}_x (x \in \omega) ~\text{or}~ w_0 (x) -
\bar{\int} Ud \rho^{\omega}_x \geq W (x)  
$$

Let us consider on $\partial \omega$ any function $\psi$, which is
$dp^\omega_x$ summable, lower bounded, lower semi continuous and $\ge
- U$. Note that in $\omega$, $f(x) = \int \psi d \rho^\omega_x$ has a
lim. inf at the boundary which is $\ge - U$. It\pageoriginale will be enough to
prove that $w_0 + f \ge W$. Observe that $w_0 + f$ in $\omega$ has a
limit inferior $\ge W$ at the boundary; therefore the function
$\alpha_{\varepsilon}~(\varepsilon > o)$ equal to inf. ($w_0 + f +
\varepsilon w_0$, $W$) in $\omega$ and continued by $W$ is
superharmonic. By a similar argument the function $\beta_{\varepsilon}
$ equal to $(( w_0)^i + f + \varepsilon w_0$, $w'_i$) and continued by
$W'_i$ is superharmonic. Moreover $\beta_{\varepsilon} \ge 0$ (see
Theorem \ref{p4:chap2:sec8:thm3-ii}(ii)). 

But $\alpha_{\varepsilon} = v_i + \beta_{\varepsilon}$, therefore
$\alpha_{\varepsilon} > v_i$, $\alpha_{\varepsilon} \ge W$, $w_0 + f +
\varepsilon w_0 \ge W $ in $\omega$, then also $w_o + f \ge W$. 

Now $U$ is a nearly superharmonic function; $w_0 = W + U$ nearly
everywhere $w_0 = W + \hat{U}$, $w_0 > W$. 

\section[Vector space of differences of...]{Vector space of differences of superharmonic\break function $\ge
  0$}\label{p4:chap4:sec19} % sec 19

Let us consider now the pairs ($u$, $v$) of superharmonic functions
$\ge 0$ on $\omega$. We define an equivalence relation, in the set of
pairs ($u$, $v$), denoted by ($u_1$, $v_1$) $\sim$ ($u_2$, $v_2$) by
the condition 

$u_1 + v_2 = u_2 + v_1$ which is equivalent to say that $u_1 - v_1 =
u_2 - v_2$ nearly everywhere; every difference being defined nearly
everywhere. 

We denote the equivalence class containing ($u$, $v$) by [$u$, $v$]
and we say the function $u - v$, defined nearly everywhere, is
associated to [$u$, $v$]. These equivalence classes form a vector
space $S$ over the real number field, if we introduce the obvious
operations corresponding to the usual operations for associated
functions, nearly everywhere. 

More\pageoriginale precisely,
\begin{align*}
  \lambda [u,v ] &  =[\lambda u, \lambda v] ~\text{if}~ \lambda \text{ is real
    and }\ge 0 ~(0.  \infty \text{ being } 0) \\ 
  \lambda [u,v ] &  =[- \lambda u, - \lambda v] \text { if } \lambda
  \text{ is real and }<  0 \\ 
       [u_1, v_1] & + [u_2,  v_2] = [u_1 + u_2,  v_1 + v_2]
\end{align*}

(A) ~ The \textit{natural} order on $S$, denoted [$u_1$, $v_1$] $\ge$
[$u_2$, $v_2$], is defined by $u_1 + v_2 \ge u_2 + v_1$ or $u_1 - v_1
\ge u_2 - v_2$ nearly everywhere. 

The corresponding ``positive cone" is the set of [$u$, $v$] such that
$u - v\ge 0$ nearly everywhere; it is not the set of the [$u$, $0$]. 
 
Let us observe that, in the ordinary sense, nearly everywhere
$$
\sup (u_1 - v_1,  u_2 - v_2) = u_1 + u_2 - \inf (u_2 + v_1,  u_1 + v_2).
$$

Therefore, there is for the natural order in $S$ a $\sup$ ([$u_1$,
  $v_1$], [$u_2$, $v_2$]) and an associated function is $\sup$ ($u_1 -
v_1$, $u_2 - v_2$ )  nearly everywhere. (Note that if $v_1 = v_2 = 0
$, it is different form the supremum which exists in the subset of the
[$u,0$]). 

Similar result hold for the infimums. In general we see, with this
natural order that if $X$, $Y \in S$ and $X'$, $Y' $ are any associated
functions then, $X'^{+}$, $X'^-$, $|X'|$, $\sup$ ($X'$, $Y'$), $\inf$
($X'$, $Y'$) (lofineun nearly everywhere in the ordinary sense ) are
associated to $ X^+$, $X^-$, $|X|$, $\sup$ ($X$, $Y$ ), and $\inf$
($X$, $Y$) respectively. Hence  

\begin{prop}\label{p4:chap4:sec19:prop7}% prop 7
  The\pageoriginale vector space $S$ with the natural order is a Riesz space. 

  (B) ~ More important is the {\em specific order } $(> )$ defined by
  [$u_1$, $v_1$] $>$ [ $u_2$, $v_2$ ] if and only if [ $u_1$, $v_1$ ]
  - [ $u_2$, $v_2$ ] $=$ [$w$, $0$]  for some $w \geq 0 $ (superharmonic
  ). or equivalently,  
  $$
  u_1 - v_1 = u_2 - v_2 + w  \text{ nearly everywhere}.  
  $$
  This  is the order corresponding to the choice of the ``positive
  cone'' $S^+ $ of the [$u$, $0$].  
\end{prop}

From the general theory of order vector spaces [Bourbaki Integration
  Ch, II], we obtain the,  

\begin{thm}\label{p4:chap4:sec19:thm13} % them 13
  The vector space $S$ with the specific order is a Riesz space, (and
  actually a complete lattice, i.e. is ``complement reticule''). 
\end{thm}

Note that the correspondence [$u$, $0$] $\leftrightarrow u $ is an
isomorphism which allows the identification of these notations. For
the sake of shortness we may use [$u$, $v$] and $u-v$ equivalently.  

\section{Potentials}\label{p4:chap4:sec20} % sec 20

Any superharmonic function which has a harmonic minorant possesses a
greatest harmonic minorant as well.  

\begin{prop}\label{p4:chap4:sec20:prop8} % prop 8
  If $v$ is a superharmonic function $\geq 0$, $R^{CE}_v $,
  $R_v^{C\overset{0}E}$ tend to the greatest harmonic minorant of $v$,
  following the filter of sections of the increasing directed family of
  the relatively compact sets\pageoriginale $E$ of $\Omega$ 
\end{prop}

In fact $R^{CE}_v$ is harmonic in $\overset{\circ}E$ and tends to a
harmonic minorant of $v$. If there exist harmonic minorants $> 0$, for
any such function $u$, we have $u \le R^{C \overset{\circ}E}_v$ on
$\overset{\circ}E$ (since any superharmonic function in $\Omega$ which is
$\ge v$ outside $\overset{\circ}E$ must be $\ge u $ in $\overset{\circ}E$
because of the behaviour of $v-u$ at the boundary of $\overset{0}E
$ and by Theorem \ref{p4:chap2:sec8:thm3-ii}(ii)).  

\begin{remark*}
  If $v_1 = v_2$ outside a compact set, the harmonic minorants of
  $v_1$ and $v_2$ in $\Omega$ are the same. 
\end{remark*}

\begin{defn}\label{p4:chap4:sec20:def14} % def 14
  A superharmonic function $v \ge 0$ in $\Omega$ is called a positive
  potential, briefly a potential, if its greatest harmonic minorant is
  zero. 
\end{defn}

Same definition for a potential in an open set $\omega \subset \Omega$.

\noindent
\textbf{Immediate Properties.}

If $v$ is a potential, $\lambda v (\lambda > 0)$ is also a potential. Any
superharmonic function $w$ such that $0 \le w \le v$ is a
potential. The infimum and the sum of two potentials $v_1$ and $v_2$
are also potentials (the latter follows because of $R^{CE}_{v_1 + v_2}
\ge R^{CE}_{v_1} + R^{CE}_{v_2}$). 

\begin{prop}\label{p4:chap4:sec20:prop9} % prop 
  Suppose $v$ is hyperharmonic in an open set $\omega \subset \Omega$,
  and satisfies 
  \begin{enumerate}[\rm (i)]
  \item at any point of $\Omega \cap \partial \omega$,
    $\lim$. $\inf$. $v \ge 0$. 
  \item There exists a potential $V$ in $\Omega$ such that $v \ge - V$
    in $\Omega$. Then $v \ge 0$. 
  \end{enumerate}
\end{prop}	

For, $\inf$. ($v$, $0$) continued by $0$ is a superharmonic function
$v_1$ in $\Omega$ and $v_1 \ge - V$ in $\Omega$. Therefore $v_1$
majorises the smallest\pageoriginale harmonic majorant of $-V$. 

\noindent
\textbf{Difference of Potentials.}
If $X'$ and $Y'$ are equal nearly everywhere to different of
potentials (in other words if $X'$, $Y' \in S$) it is the same for
$X'^{+}$, $X'^{-}$, $X'$, $\sup$ ($X'$, $Y'$) and $\inf$ ($X'$, $Y'$). 

In the space $S$, the subset consisting of [$u$, $v$] with potentials
$u$, $v$, is a subspace $S'$ and is closed for the operations sup and
inf (of two elements), $|*|$, $(*)^+ $ and $(*)^-$; and is hence a
Riesz space for both the orders. Similar results hold if we consider
only the finite continuous functions.	 

\section{Existence of a Potentials \texorpdfstring{$> 0$}{0}: Consequences}\label{p4:chap4:sec21}  % sec 21

The case where there exists no such potential in a domain is rather
trivial; it is easy to prove, in this case, that the superharmonic
functions $\ge 0$ are all harmonic and proportional. 

\textit{Suppose there exists a potential $V > 0$} (i.e. there exist
non-harmonic superharmonic functions $> 0$ on $\Omega$) 
\begin{enumerate}[(i)]
\item On any relatively compact open set $\omega \subset \Omega$ there
  exists a harmonic function $> \varepsilon > 0 : \hat{R}^{C \omega}_v
  > 0$ is one 
\item 

~
\vskip -1.28cm \phantom{a}
  \begin{prop}\label{p4:chap4:sec21:prop10} % prop 10
    If $E$ is a relatively compact set, for any superharmonic functions
    $v \ge 0, \hat{R}^{E}_{v}$ is a potential. 
  \end{prop}

  It is obvious if $v$ is bounded on $\bar{E}$, because for a suitable
  $\lambda > 0$, $\lambda V > v$ on $E$ and  
  $$
  \lambda V \ge R^E_{\lambda V} \ge R^E_v \ge \hat{R}^E_v.
  $$
  In the general case, $v = w$ (potential ) $+ h$ (harmonic $\ge 0).$
  $$
  \hat{R}^E_v \le \hat{R}^E_h + \hat{R}^E_w.
  $$

  Where\pageoriginale $\hat{R}^E_h$ is a potential according to particular case and
  $\hat{R}^E_w$ is a potential because it is $\le w$. 
  
  As a complement, we take a regular domain $\omega$, and $E \subset
  \bar{E} \subset \omega$; then $(\hat{R}^E_v)_\omega \rightarrow 0$ at
  the boundary. 
  
  Let us introduce an open set $\omega_1$ such that $\bar{E} \subset
  \omega_1 \subset \bar{\omega}_1 \subset \omega (\hat{R}^{\omega_1}_v)
  - \int (\hat{R}^{\omega_1}_v) d \rho^\omega_x$ is in $\omega$ a
  superharmonic function $w \ge 0$ tending to zero at the boundary of
  $\omega$; therefore $\lambda w$, for a suitable $\lambda > 0$, is $\ge
  v$ on $\bar{E}$. Hence $\lambda w \le (R^E_v) \omega$. 
\item 

~
\vskip -1.24cm \phantom{a}
  \begin{prop}\label{p4:chap4:sec21:prop11} % prop 11
    There exists a finite continuous potential $V_0 > 0$.
  \end{prop}
  
  We first construct a locally bounded potential $> 0$; we introduce
  two open sets $\omega'$, $\omega''$ such that $\bar{\omega}' \subset
  \omega'' \subset \bar{\omega}'' \subset \Omega\,  V'' = \hat{R}^{C
    \omega''}_V$ is a potential $> 0$ harmonic in $\omega'' V ' =
  \bar{R}^\omega_{V''}$ is a potential $> 0$, harmonic in $C \omega'$
  and bounded on any compact set of $\Omega$. We use a finite covering
  of $\bar{\omega}'$ with regular domains $\omega_i$ and replace $V'$
  by $\int V' d \rho^{\omega_1}_x$ in $\omega_1$ to get
  $V'_1$. Successively replacing $V'_i$ by $\int V'_i d \rho^{i+1}_x$ in
  $\omega_{i + 1}$ (and this process is finite) we finally get a
  continuous potential $V_0 > 0$. 
\item 

~
\vskip -1.24cm \phantom{a}
  \begin{prop}\label{p4:chap4:sec21:prop12} % prop 12
    Give any domain $\omega$, there exists a finite continuous
    potential $> 0$ which is not harmonic in $\omega$.  
  \end{prop}
  
  Let $\alpha$ be an open set such that $\alpha : \bar{\alpha} \subset
  \bar{\omega}$. The potential $\hat{R}^\alpha_0$ is equal to $V_0$ in
  $\alpha$ and hence $> 0$ in $\omega$. By considering a covering of
  $\bar{\alpha}$ by means of regular domains $\delta_i$ we deduce as
  in Prop.\ref{p4:chap4:sec21:prop11} a finite continuous potential $w \le V_0$ such that $w
  = \hat{R}^\alpha_{V_0}$ outside $\cap \delta _i$. Hence $w > 0$ and
  harmonic outside\pageoriginale $\cup \delta_i$. Moreover $w$ is not harmonic in
  $\omega$, for otherwise $w$ would be harmonic in $\Omega (\le V)$
  and  therefore identically zero. 
\end{enumerate}

\begin{extension}
  Give a countable union of disjoint domains $\omega_i$, there exists
  a finite continuous potential $w > 0$, which is not harmonic in all
  $\omega_i$. 
\end{extension}

If $V_i$ is a potential corresponding to $\omega_i$ (for each i) got
by the previous argument, $\sum \lambda_i V_i$ where $\lambda_i > 0$
and $\sum \lambda_i < + \infty$ serves the purpose. 

\begin{thm}[Continuation Theorem] \label{p4:chap4:sec21:thm14} % them 14 
  Suppose there exists a potential $V > 0$ in $\Omega$. Let $V$ be a
  superharmonic function, $v \ge 0$ in a regular domain $\delta$. For
  any open set $\omega \subset \bar{\omega} \subset \delta$, there
  exists a potential in $\Omega$ which is equal to $v$ in $\omega$,
  upto a harmonic function in $\omega$. 
\end{thm}

\begin{proof}
  We know that $(\hat{R}^\omega_v) \delta$ is superharmonic $\ge 0$,
  harmonic in $\delta - \bar{\omega}$ equal to $v$ in $\omega$ and
  tends to zero at the boundary of $\delta$. 
\end{proof}

Let us introduce and open set $\omega_1$ such that $\omega \subset
\omega_1 \subset \bar{\omega}_1 \subset \delta$ and a finite
continuous potential $V_0 > 0$ in $\omega$ which is not harmonic in
$\delta$. $V_0 - H^\delta_{V_0}$ is superharmonic $> 0$ in $\Omega$  and
tends to zero at the boundary. For a suitable $\lambda > 0$, $\lambda
(V_0 - H^\delta_{V_0}) \ge (\hat{R}^\omega_{v}) \delta$ on $\partial
\omega_1$. Hence the inequality holds on $\delta - \bar{\omega_1}$. 

We deduce that the function equal to $\lambda V_0$ in $C \delta$ and
to $\lambda H^{\delta_{V_0}} + (\hat{R}^\omega_v) \delta$ in $\delta$
is a potential and meets the requirements of the theorem. 

\begin{extension}
  R.M. Herve has proved this theorem independently for any open\pageoriginale set
  $\omega$ by
  a different method. An adaptation of the previous proof (using
  Theorem \ref{p3:chap4:sec12:thm9}) may also be given. 
\end{extension}

\begin{thm}[Approximation Theorem 
  (R.M. Herve)]\label{p4:chap4:sec21:thm15} %% thm 15 
If there exists a potential
  $V 0$, any finite continuous function $f$ on a compact set $K$ can be
  uniformly approximated on $K$ by the difference of two finite
  continuous potentials on $\omega$.  
\end{thm}

\begin{proof}
  Consider all such differences $X$. If $V_0$ is a fixed finite
  continuous potential $V_0$ ~~ 0, the quotients $X/V_0$ form a real
  vector space of finite continuous functions. Moreover this vector
  space contains constants and $X/ V_0$ as well.	 
\end{proof}

In order to apply the theorem of Stone on approximation and obtain the
approximation of $f /V_0$ by means of some $X/V_0$ on $K$(and then the
approximation of $f$ by $X$) we should verify that $X/{V_0}$ separate
points of $K$. Let $x_0 \neq y_0$ in $K$ and a regular domain $x_0 $
and $y_0$. Let $W$ be a finite continuous potential non harmonic in ;
if $\dfrac{W(x_0)}{V_0(x_0)} = \dfrac{W(y_0)}{V_0(y_0)}$ we replace
$W$ by $W_1 = Wd _x $ in. We get a finite continuous potential such
that $\dfrac{W_1(x_0)}{V(x_0)}\neq \dfrac{W_1(y_0)}{V(y_0)}$. 
