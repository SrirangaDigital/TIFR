

\chapter{A generalization of the Morse Lemma}\label{chap2}

As\pageoriginale We Saw in Chapter \ref{chap1}, the problem is to find the local zero set of the {\em reduced mapping}, which is a mapping of class $C^{m}(m \geq 1)$ in a neighbourhood of the origin in the $(n+1)$-dimensional dspace $X_{1} = \Ker DG(0)$ into the $n$-dimensional space $Y_{1}$ (a given complement of $Y_{2} = Range DG(0)$).

We shall develop an approach which is analogous to the one we used in the case $n = 1$ (Morse lemma). The first task is to find a suitable generalization of the {\em Morse condition}.

\section[A Nondegeneracy Condition For.........]{A Nondegeneracy Condition For Homogeneous Polynomial Mappings.}\label{chap2-sec1}

Let $q : \mathbb{R}^{n+1} \to \mathbb{R}^{n}$ be a polynomial mapping, homogeneous of degree $k \geq 1$ (i.e. $q = (q_{\alpha})_{\alpha=1, n}$ where $q_{\alpha}$ is a polynomial, homogeneous of degree $k$ in $(n+1)$ variables with real coefficients).

\begin{definition}\label{chap2-def1.1}
We shall say that the polynomial mapping $q$ verifies the condition of $\mathbb{R}$-nondegeneracy (in short, $\mathbb{R}$-N.D.) if, for every non-zero solution $\widetilde{\xi} \epsilon \mathbb{R}^{n+1}$ of the equation $q(\widetilde{\xi}) = 0$, the mapping $Dq(\xi) \epsilon \mathscr{L} (\mathbb{R}^{n+1},\break \mathbb{R}^{n})$ is onto.
\end{definition}

As is homogeneous, its zero set in $\mathbb{R}^{n+1}$ is a cone in $\mathbb{R}^{n+1}$ with vertex at the origin. Actually, we have much more precise information. First, observe that
\begin{equation*}
q(\widetilde{\xi}) = \frac{1}{k} Dq(\widetilde{\xi}) \cdot \widetilde{\xi},\tag{1.1}\label{chap2-eq1.1}
\end{equation*}
for\pageoriginale every $\widetilde{\xi} \epsilon \mathbb{R}^{n+1}$ {\em (Euler's theorem)}. Indeed, from the homogeneity of $q$, write
$$
q(t \widetilde{\xi}) = t^{k}q(\widetilde{\xi})
$$
and differentiate both sides with respect to $t$; then
$$
Dq(t\widetilde{\xi}) \cdot \widetilde{\xi} = kt^{k-1} q(\widetilde{\xi}).
$$

Setting $t = 1$, we get the identity (\ref{chap2-eq1.1}).

Let $\widetilde{\xi} \epsilon \mathbb{R}^{n+1} - \{0\}$ be such that $q(\widetilde{\xi}) = 0$. Then, it follows that
$$
\mathbb{R} \widetilde{\xi} \subset \Ker Dq(\widetilde{\xi}).
$$

But $Dq(\widetilde{\xi}) \epsilon \mathscr{L} (\mathbb{R}^{n+1}, \mathbb{R}^{n})$ is {\em onto} by hypothesis. Hence, $\dim \Ker Dq\break(\widetilde{\xi}) = 1$, so that
\begin{equation*}
\Ker Dq(\widetilde{\xi}) = \mathbb{R} \widetilde{\xi}.\tag{1.2}\label{chap2-eq1.2}
\end{equation*}

\begin{theorem}\label{chap2-thm1.1}
Let the polynomial mapping $q$ verify the condition $(\mathbb{R} - N.D.)$. Then, the zero set of $q$ in $\mathbb{R}^{n+1}$ is made up of a finite number $v$ of lines through the origin.
\end{theorem}

\begin{proof}
Since the zero set of $q$ is a cone with vertex at the origin, its zero set is a union of lines. To show that there is a finite number of them, it is equivalent to showing that their intersection with the unit sphere $S_{n}$ in $\mathbb{R}^{n+1}$ consists of a finite number of points. Clearly, the set
$$
\{\widetilde{\xi} \epsilon S_{n} ; q(\widetilde{\xi}) = 0 \},
$$
is\pageoriginale closed in $S_{n}$ (continuity of $q$) and hence {\em compact}. To prove that it is finite, it suffices to show that it is also {\em discrete}. Let then $\widetilde{\xi} \epsilon S_{n}$ such that $q(\widetilde{\xi}) = 0$. By hypothesis, $Dq(\widetilde{\xi}) \epsilon \mathscr{L}(\mathbb{R}^{n+1}, \mathbb{R}^{n})$ is {\em onto} and we know that $\Ker Dq(\widetilde{\xi}) = \mathbb{R} \widetilde{\xi}$ (cf. (\ref{chap2-eq1.2})). Then, the restriction of $Dq(\widetilde{\xi})$ to any complement of the space $\mathbb{R}\widetilde{\xi}$ is an isomorphism to $\mathbb{R}^{n}$. In particular, observe that the points of the sphere $S_{n}$ have the following property: for every $\widetilde{\zeta} \epsilon S_{n}$, the  tangent space $T_{\widetilde{\zeta}} S_{n}$ of $S_{n}$ at $\widetilde{\zeta}$ is nothing but $\{\widetilde{\zeta}\}^{\perp}$. hence, for every $\widetilde{\zeta} \epsilon S_{n}$, we can write
$$
\mathbb{R}^{n+1} = \mathbb{R} \widetilde{\zeta} + T_{\widetilde{\zeta}}S_{n}.
$$
In particular, we deduce
\begin{equation*}
Dq(\widetilde{\xi}) \epsilon \text{ Isom } (T_{\widetilde{\xi}} S_{n}, \mathbb{R}^{n}).
\end{equation*}

As $q$ is regular, the Inverse function theorem shows that there is no solution other than $\widetilde{\xi}$ for the equation $q(\widetilde{\xi}) = 0$ near $\widetilde{\xi}$ on $S_{n}$.
\end{proof}

\begin{comment}\label{chap2-com1.1}
The above theorem does {\em not} prove that there is any line in the zero set of $q$. Actually, the situation when $\nu = 0$ can perfectly occur.
\end{comment}

\begin{comment}\label{chap2-com1.2}
It is tempting to try to get more information about the number $\nu$ of lines in the zero set of $q$. Of course, it is not possible to expect a formula expressing $\nu$ in terms of $q$ but one can expect an upper bound for $\nu$. It can be shown (under the condition ($\mathbb{R}$. N.D.)) that the inequality
\begin{equation*}
\nu \leq k^{n},\tag{1.3}\label{chap2-eq1.3}
\end{equation*}
{\em always holds}.\pageoriginale This estimate is an easy application of the {\em generalized Bezout's theorem} (see e. g. Mumford \cite{26}). Its statement will not be given here because it requires preliminary notions of algebraic geometry that are beyond the scope of these lectures.
\end{comment}

We shall give a flavour of the result by examing the simplest case $n = 1$. Let $q$ be a homogeneous polynomial of degree $k$ in two variables. More precisely, given a basis $(\widetilde{e}_{1}, \widetilde{e}_{2})$ of $\mathbb{R}^{2}$, write
$$
\widetilde{\xi} = \xi_{1} \widetilde{e}_{1} + \xi_{2}\widetilde{e}_{2}, \xi_{1}, \xi_{2} \epsilon \mathbb{R}.
$$
then
\begin{equation*}
\begin{cases}
& q(\widetilde{\xi}) = \sum\limits_{s=0}^{k} a_{s} \xi_{1}^{k-s} \xi_{2}^{s},\\
& a_{s} \epsilon \mathbb{R}, 0 \leq s \leq k.
\end{cases}\tag{1.4}\label{chap2-eq1.4}
\end{equation*}

It is well-known that such a polynomial $q$ can be deduced from a unique $k$-linear symmetric form $Q$ on $\mathbb{R}^{2}$ by
$$
q(\widetilde{\xi}) = Q(\widetilde{\xi}, \cdots , \widetilde{\xi}),
$$
where the argument $\widetilde{\xi} \epsilon \mathbb{R}^{2}$ is repeated $k$ times. In particular,
$$
a_{s} = \binom{k}{s} Q (\widetilde{e}_{1}, \cdots, \widetilde{e}_{1}, \widetilde{e}_{2}, \cdots , \widetilde{e}_{2}),
$$
where the argument $\widetilde{e}_{1}$ (respectively $\widetilde{e}_{2}$) is repeated $s$ times (respectively $k-1$ times). Now, the basis $(\widetilde{e}_{1}, \widetilde{e}_{2})$ can be chosen so that $a_{k} \neq 0$. Indeed
$$
a_{k} = q (\widetilde{e}_{2}, \cdots, \widetilde{e}_{2}) = q(\widetilde{e}_{2})
$$
and\pageoriginale $\widetilde{e}_{2}$ can be taken so that $q(\widetilde{e}_{2}) \neq 0$ (since $q \nequiv 0), \widetilde{e}_{1}$ being any vector in $\mathbb{R}^{2}$, not collinear with $\widetilde{e}_{2}$. If so, observe that the local zero set of $q$ contains no element of the form $\xi_{2} \widetilde{e}_{2}, \xi_{2} \neq 0,$ since $q(\xi_{2} \widetilde{e}_{2}) = a_{k} \xi_{2}^{k}$. In other words, each nonzero solution of the equation $q(\widetilde{\xi}) = 0$ {\em has a nonzero} component $\xi_{1}$. Dividing then (\ref{chap2-eq1.4}) by $\xi_{1}^{k}$, we find
$$
q(\widetilde{\xi}) = 0 \Leftrightarrow \sum\limits_{s=0}^{k} a_{s} \left(\frac{\xi_{2}}{\xi_{1}}\right)^s = 0.
$$

Setting $\tau = \frac{\xi_{2}}{\xi_{1}}$ and since $\tau$ is real whenever $\xi_{1}$ and $\xi_{2}$ are, we see that $\tau$ must be a {\em real root} of the polynomial
$$
a(\tau) = \sum\limits_{s=0}^{k} a_{s} \tau^{s}.
$$

Conversely, to each real root $\tau$ of the above polynomial is associated the line $\{\xi_{1} \widetilde{e}_{1} + \tau \xi_{1} \widetilde{e}_{2} ; \xi_{1} \epsilon \mathbb{R} \}$ of solutions of the equation $q(\widetilde{\xi}) = 0$. Here, the inequality $\nu \leq k$ follwos from the {\em fundamental theorem of algebra.}

\begin{remark}\label{chap2-rem1.1}
Writing
$$
a(\tau) = \sum\limits_{s=0}^{k} a_{s} \tau^{s} = a_{k} \prod_{s=1}^{k} (\tau - \tau_{s})
$$
where $\tau_{s}, 1 \leq s \leq k$ are the $k$ (not necessarily distinct) roots of the polynomial $a(\tau)$ and replacing $\tau$ by $\xi_{2} | \xi_{1}$ with $\xi_{1} \neq 0$, we find
$$
q(\widetilde{\xi}) = a_{k} \prod_{s=1}^{k} (\xi_{2} - \tau_{s} \xi_{1}),
$$
as relation which remains valid when $\xi_{1} = 0$.
\end{remark}

\begin{comment}\label{chap2-com1.3}
Recall\pageoriginale that a continuous {\em odd} mapping defined on the sphere $S_{m-1} \subset \mathbb{R}^{m}$ with values in $\mathbb{R}^{n}$ always vanishes at some point of $S_{m-1}$ when $m > n$. Here, with $m = n + 1$ we deduce that $\nu \geq 1$ when $k$ is {\em odd}. When $k$ is {\em even}, it can be shown (cf. Buchner, Marsden and Schecter \cite{5}) taht $\nu$ is {\em even too} (possibly 0 however).
\end{comment}

\begin{remark}\label{chap2-rem1.2}
Any small perturbation of $q$ (as a homogeneous polynomial mapping of degree $k$) still verifies the condition ($\mathbb{R}$ - N.D.) and its local zero set id made of the {\em same number of lines} (Hint: let $Q$ denote the finite dimensional space of homogeneous polynomials of degree $k$. Consider the mapping $(p, \widetilde{\zeta}) \epsilon Q \times S_{n} \to p(\widetilde{\zeta}) \epsilon \mathbb{R}^{n}$ and note that the derivative at $(q, \widetilde{\xi})$ with respect to $\widetilde{\zeta}$ is an isomorphism when $q(\widetilde{\xi}) = 0$.)
\end{remark}

\begin{comment}\label{chap2-com1.4}
  Condition ($\mathbb{R}$-N.D.) ensures that the zero set of $q$ in $\mathbb{R}^{n+1}$ is made of a finite number of lines through the origin. The converse is {\em not} true\footnote{Incidentlly, we have shown that the zero set of $q$ is a finite union of lines, when $n = 1$, {\em with no assumption other than} $q \nequiv 0$.}. Actually, the condition ($\mathbb{R}$-N.D.) also shows that each line in the zero set is ``simple'' in the way described in Remark \ref{chap2-rem1.2}. When the condition ($\mathbb{R}$.N.D.) does not hold but the zero set of $q$ is still made up of a finite number of lines, some of them are ``multiple'', namely, split into {\em several lines} or else {\em disappear} when replacing $q$ by a suitable small perturbation.
\end{comment}

\section{Practical Verification of the Condition ($\mathbb{R}$-N.D.).}\label{chap2-sec2}

The above considerations leave us with two basic questions:
\begin{enumerate}
\item[(i)] How\pageoriginale does one check the condition ($\mathbb{R}$-N.D.) for a given mapping $q$?

\item[(ii)] If the condition ($\mathbb{R}$-N.D.) holds, how can we compute (approximations of) the lines in the zero set of $q$?
\end{enumerate}

Here we shall give a {\em partial answer} to these questions. We begin with the case $n = 1$. As we know, in every system of coordinates $\widetilde{\xi} = \xi_{1} \widetilde{e}_{1} + \xi_{2}\widetilde{e}_{2}$ of $\mathbb{R}^{2}$ such that $q(\widetilde{e}_{2}) \neq 0$ (such a system always exists when $q \nequiv 0$), one has
\begin{equation*}
q(\widetilde{\xi}) = a_{k} \prod_{s=1}^{k} (\xi_{2} - \tau_{s}\xi_{1}),\tag{2.1}\label{chap2-eq2.1}
\end{equation*}
where $a_{k} = q(\widetilde{e}_{2}) \neq 0$ and $\tau_{s}, 1 \leq s \leq k$ are the $k$ roots of the polynomial 
\begin{equation*}
a(\tau) = \sum\limits_{s=0}^{k} a_{s} \tau^{s}.\tag{2.2}\label{chap2-eq2.2}
\end{equation*}

If so, each line in the zero set of $q$ is of the form
\begin{equation*}
\{\xi_{1} \widetilde{e}_{1} + \tau_{s} \xi_{1} \widetilde{e}_{1} ; \xi_{1} \epsilon \mathbb{R} \},\tag{2.3}\label{chap2-eq2.3}
\end{equation*}
where $\tau_{s}$ is a real root of $a(\tau)$. Saying that $q$ verifies the condition ($\mathbb{R}$-N.D.) amounts to saying that the derivative $Dq(\widetilde{\xi}) \epsilon \mathscr{L}(\mathbb{R}^{2}, \mathbb{R})$ is {\em onto} (i.e. not equal to zero) at each point $\widetilde{\xi} \epsilon \mathbb{R}^{2} - \{0\}$ such that $q(\widetilde{\xi}) = 0$. For $\widetilde{\xi}, \widetilde{\zeta} \epsilon \mathbb{R}^{2}$,
\begin{equation*}
Dq(\widetilde{\xi}) \cdot \widetilde{\zeta} = a_{k} \sum\limits_{s=1}^{k} \left[ \prod_{\sigma \neq s} (\xi_{2} - \tau_{\sigma} \xi_{1})\right] (\zeta_{2} - \tau_{s} \zeta_{1}).\tag{2.4}\label{chap2-eq2.4}
\end{equation*}

Since $q(\widetilde{\xi}) = 0$, there is an index $s_{0}$ such that $\xi_{2} - \tau_{s_{0}} \xi_{1} = 0$. Thus
$$
Dq(\widetilde{\xi}) \cdot \widetilde{\zeta} = a_{k} \left[\prod_{\sigma \neq s_{0}} (\tau_{s_{0}} \xi_{1} - \tau_{\sigma} \xi_{1}) \right] (\zeta_{2} - \tau_{s_{0}} \zeta_{1}).
$$

Clearly,\pageoriginale the linear form $\widetilde{\zeta} \epsilon \mathbb{R}^{2} \to \zeta_{2} - \tau_{s_{0}} \zeta_{1}$ is onto (i.e. non-zero). Thus, $Dq(\widetilde{\xi})$ will be onto if and only if
\begin{equation*}
a_{k} \prod_{\sigma \neq s_{0}} (\tau_{s_{0}} - \tau_{\sigma}) \xi_{1} \neq 0.\tag{2.5}\label{chap2-eq2.5}
\end{equation*}

But from the choice of the system of coordinates, we know that $\xi_{1} \neq 0$. Hence, $Dq(\widetilde{\xi})$ will be onto if and only if
$$
\tau_{\sigma} \neq \tau_{s_{0}} \text{ for } \sigma \neq s_{0},
$$
i.e. $\tau_{s_{0}}$ is a {\em simple root} of the polynomial $a(\tau)$.

To sum up, when $n = 1$, the mapping $q$ will verify the condition ($\mathbb{R}$-N.D.) {\em if} and only if given a system of coordinates $(\widetilde{e}_{1}, \widetilde{e}_{2})$ in $\mathbb{R}^{2}$ such that $q(\widetilde{e}_{2}) (= a_{k}) \neq 0$, {\em each real root} of the polynomial $a(\tau) = \sum\limits_{s=0}^{k} a_{s} \tau^{s}$ is {\em simple}.

In the particular case when $k = 2, a(\tau)$ is a quadratic polynomial.
\begin{enumerate}
\item[(i)] If its discriminant is $< 0$, it has no real root (then, each of them is simple) : ($\mathbb{R}$-N.D.) {\em holds}.

\item[(ii)] If its discriminant is zero, it has a double real root: ($\mathbb{R}$-N.D.) {\em fails to hold.}

\item[(iii)] If its discriminant is $> 0$, it has two simple real roots: ($\mathbb{R}$-N.D.) {\em holds.}
\end{enumerate}

Note that the above results give a way for finding (approximations of) the lines in the zero set of $q$. It suffices to use an algorithm\pageoriginale for the computation of the roots of the polynomial $a(\tau)$. However, it is not easy to check whether a given root of a polynomial is simple, by calculating approximations to it through an algorithm and the first question is not satisfactorily answered.

Observe that it is of course {\em sufficient}, for the condition ($\mathbb{R}$-N.D.) to hold , that {\em every root (real or complex)} of the polynomial $a(\tau)$ is simple.

\begin{definition}\label{chap2-def2.1}
If every root (real or comples) of $a(\tau)$ is simple, we shall say that the polynomial $q$ satisfies the condition of $\mathbb{C}$-nondegeneracy (in short, $\mathbb{C}$-N.D.)
\end{definition}

Definition \ref{chap2-def2.1} can be described by saying that the polynomial $a(\tau)$ and its derivative $a'(\tau)$ {\em have} no common root.

Now recall the following result from elementary algebra: let $a(\tau)$ and $b(\tau)$ be two polynomials (with complex coefficients) of degrees {\em exactly} $k$ and $\ell$ respectively, i.e.
\begin{equation*}
\begin{cases}
& a(\tau) = \sum\limits_{s=0}^{k} a_{s}, \tau^{s}, a_{k} \neq 0,\\
& b(\tau) = \sum\limits_{s=0}^{\ell} b_{s} \tau^{s}, b_{\ell} \neq 0.
\end{cases}\tag{2.6}\label{chap2-eq2.6}
\end{equation*}

Then, a {necessary and sufficient condition} for $a(\tau)$ and $b(\tau)$ to have a common root in $\mathbb{C}$ is that {\em the} $(k + \ell) \times (k + \ell)$ {\em determinant}
\begin{equation*}
\mathscr{R} =
\begin{vmatrix}
a_{k} & \cdots &  a_{0} &&&\\
&  a_{k} & \cdots &   a_{0}&&\\
&&\quad \cdots &&& \\
&& a_{k} & \cdots &  a_{0}&\\
b_{\ell} & \cdots & b_{0}\\
& b_{\ell} & \cdots & b_{0}\\
 && \cdots & \\ 
&&b_{\ell} & \cdots & b_{0}&\\
\end{vmatrix}
\end{equation*}\pageoriginale

in which there are $\ell$ rows of ``$a$'' entries and $k$ rows of ``$b$'' entries, {\em is} 0 (note that this statement is false in $\mathbb{R}$). The determinant $\mathscr{R}$ is called the {\em resultant} (of Sylvester) of $a(\tau)$ and $b(\tau)$. In particular, when $b(\tau) = a'(\tau)$ (so that $\ell = k-1), \mathscr{R}$ is called the discriminant, denoter by $\mathscr{D}$, of $a(\tau)$. It is a $(2k-1) \times (2k-1)$ determinant. Elementary properties and further developments about resultants can be found in Hodge and Pedoe \cite{16} or Kendig \cite{18}.

Again, we examine the simple case when $k = 2$. If so, 
$$
a(\tau) = a_{2}\tau^{2} + a_{1}\tau + a_{0}, a_{2} \neq 0,
$$
so that
$$
a'(\tau) = 2a_{2}\tau + a_{1}.
$$

Now, from the definitions
\begin{equation*}
\mathscr{D} =
\begin{vmatrix}
a_{2} & a_{1} & a_{0}\\
2a_{2} & a_{1} & 0\\
0 &  2a_{2} & a_{1}
\end{vmatrix}
=-a_{2}(a_{1}^{2} - 4a_{0}a_{2}).
\end{equation*}\pageoriginale
The quantity $a_{1}^{2} - 4a_{0}a_{2}$ is the usual discriminant and, as $a_{2} \neq 0$ we conclude
\begin{equation*}
\mathscr{D} \neq 0 \Leftrightarrow a_{1}^{2} - 4a_{2}a_{0} \neq 0\tag{2.8}\label{chap2-eq2.8}
\end{equation*}

\begin{remark}\label{chap2-rem2.1}
When $k = 2$, the condition ($\mathbb{R}$ - N.D.) is characterized by saying that the discriminant is $\neq 0$ too. Hence, when $k = 2$
\begin{equation*}
(\mathbb{R}-N.D.) \Leftrightarrow (\mathbb{C}-N.D.),\tag{2.9}\label{chap2-eq2.9}
\end{equation*}
but this is no longer true for $k \geqq 4$.
\end{remark}

We have found that the condition $(\mathbb{C}-N.D.)$ holds $\Leftrightarrow \mathscr{D} \neq 0$. The advantage of this stronger assumption is that it is immediate to obtain the discriminant $\mathscr{D}$ in terms of the coefficients $a_{s}$'s, hence from $q$.

{\em Expression of the conditions $(\mathbb{R}-N.D.)$ and $(\mathbb{C}-N.D.)$ in any system of coordinates}: We know to express the coorditions $(\mathbb{R}-N.D.)$ and $(\mathbb{C}-N.D.)$ in a system of coordinates $\widetilde{e}_{1}, \widetilde{e}_{2}$ such that $q(\widetilde{e}_{2}) \neq 0$. Actually, we can get such an expression in {\em any} system of coordinates. Indeed, assume $q(\widetilde{e}_{2}) = 0$. Then, the coefficient $a_{k}$ vanishes and we have
\begin{equation*}
q(\widetilde{\xi}) = \sum\limits_{s=0}^{k-1} a_{s} \xi_{1}^{k-1} \xi_{2}^{s}.\tag{2.10}\label{chap2-eq2.10}
\end{equation*}

It\pageoriginale follows that the line $\{\xi_{2} \widetilde{e}_{2}, \xi_{2} \epsilon \mathbb{R} \}$ is in the zero set of $q$. Away from the origin on this line, the derivative $Dq(\widetilde{\xi})$ must be onto (i.e. $\neq 0$). An immediate calculation shows, for $\widetilde{\xi} = \xi_{2} \widetilde{e}_{2}$, that
\begin{equation*}
Dq(\widetilde{\xi}) \cdot \widetilde{\zeta} = a_{k-1} \xi_{2}^{k-1} \zeta_{1},\tag{2.11}\label{chap2-eq2.11}
\end{equation*}
for every $\widetilde{\zeta} \epsilon \mathbb{R}^{2}$. As $\widetilde{\xi}$ is $\neq 0$ if and only if $\xi_{2}$ is $\neq 0$, we have
$$
Dq(\widetilde{\xi}) \text{ is onto } \Leftrightarrow a_{k-1} \neq 0.
$$

Now, for any solution $\widetilde{\xi}$ of the equation $q(\widetilde{\xi}) = 0$ which is not on the line $\mathbb{R}\tilde{e}_{2}$, we must have $\xi_{1} \neq 0$. Arguing as before, we get
$$
q(\widetilde{\xi}) = \xi_{1}^{k} \sum\limits_{s=0}^{k-1} a_{s} \left(\frac{\xi_{2}}{\xi_{1}}\right)^{s}
$$
and all this amounts to solving the equation,
$$
\sum\limits_{s=0}^{k-1} a_{s} \left(\frac{\xi_{2}}{\xi_{1}}\right)^2 = 0.
$$

Setting $\tau = \xi_{2} / \xi_{1}$ and
$$
a(\tau) = \sum\limits_{s=0}^{k-1} a_{s} \tau^{s}
$$
the method we used when $a_{k} \neq 0$ shows that the condition $(\mathbb{R}-N.D.)$ is equivalent to the fact that {\em each real root} of $a(\tau)$ is simple. The condition $(\mathbb{C}-N.D.)$ being expressed by saying that {\em every root} of $a(\tau)$ is simple, can be written as
$$
\mathscr{D} \neq 0,
$$
where $\mathscr{D}$ is the discriminant of $a(\tau)$.

\begin{remark}\label{chap2-rem2.2}
As $a_{k} = 0$, the polynomial $a(\tau)$ must be considered as a\pageoriginale polynoimal of degree $k-1$, namely $\mathscr{D}$ is a $(2k-3) \times (2k-3)$ determinant (instead of $(2k-1) \times (2k-1)$ when $a(\tau)$ is of degree $k$). If $a(\tau)$ is considered as a polynomial of degree $k$ with leading coefficient equal to zero, the determinant we obtain is {\em always} zero and has no significance.
\end{remark}

Now, we come back to the general case when $n$ is arbitrary. Motivated by the results when $n = 1$, it is interesting to look for a generalization of the condition $(\mathbb{C}-N.D.)$. Let $Q$ be the $k$-linear symmetric mapping such that
$$
q(\widetilde{\xi}) = Q(\widetilde{\xi}, \cdots, \widetilde{\xi}),
$$
where the argument $\widetilde{\xi}$ is repeated $k$ times. Then $Q$ has a {\em canonical extension} as a $k$-linear symmetric mapping from $\mathbb{C}^{n+1} \to \mathbb{C}^{n}$ (so that linear means $\mathbb{C}$-linear here). Indeed, $\mathbb{C}^{n+1}$ and $\mathbb{C}^{n}$ identify with $\mathbb{R}^{n+1} + i \mathbb{R}^{n+1}$ and $\mathbb{R}^{n} + i \mathbb{R}^{n}$ respectively. Now, take k elements $\widetilde{\xi}^{(1)} , \cdots , \widetilde{\xi}^{(k)}$ in $\mathbb{C}^{n+1}$. These elements can be written as
$$
\widetilde{\xi}^{(s)} = \widetilde{u}^{(s)} + i\widetilde{v}^{(s)}, \widetilde{u}^{(s)}, \widetilde{v}^{(s)} \epsilon \mathbb{R}^{n+1}, 1 \leq s \leq k.
$$

Define the extension of $Q$ (still denoted by $Q$) by
$$
Q(\widetilde{\xi}^{(1)}, \cdots, \widetilde{\xi}^{(k)}) = \sum\limits_{j=1}^{k} i^{k-j} \sum\limits_{P \epsilon \mathscr{P}_{j}} Q(\widetilde{u}^{(P(1))}, \widetilde{u}^{(P(j))}, \widetilde{v}^{(P(j+1))}, \widetilde{v}^{(P(k))}),
$$
where $\mathscr{P}_{j}$ denotes the set of permutations of $\{1, \cdots, k\}$ such that 
$$
P(1) <\cdot< P(j)
$$
and
$$
P(j+1) <\cdot< P(k).
$$

It\pageoriginale is easily checked that this defines an extension of $Q$ which is $\mathbb{C}$-linear with respect to each argument and symmetric. An extension of $q$ to $\mathbb{C}^{n+1}$ is then
\begin{equation*}
q(\widetilde{\xi}) = Q(\widetilde{\xi}, \cdots, \widetilde{\xi}).\tag{2.12}\label{chap2-eq2.12}
\end{equation*}

\begin{remark}\label{chap2-rem2.3}
In practice, if $q = (q_{\alpha})_{\alpha = 1, \cdots, n}$ and
$$
\widetilde{\xi} = \xi_{1}\widetilde{e}_{1} + \cdots + \xi_{n+1} \widetilde{e}_{n+1},
$$
where $(\widetilde{e}_{1}, \cdots, \widetilde{e}_{n+1})$ is a basis of $\mathbb{R}^{n+1}$ with $\xi_{1}, \cdots, \xi_{n+1} \epsilon \mathbb{R}$, each $q_{\alpha}$ is a polynomial, homogeneous of degree $k$ with real coefficients. The extension (\ref{chap2-eq2.12}) is obtained by simply replacing each $\xi_{j} \epsilon \mathbb{R}$ by $\xi_{j} \epsilon \mathbb{C}$.
\end{remark}

\begin{definition}\label{chap2-def2.2}
We shall say that $q$ verifies the condition of $\mathbb{C}$-non - degenreacy (in short, $(\mathbb{C}-N.D.)$) if, for every non-zero solution $\widetilde{\xi}$ of the equation $q(\widetilde{\xi}) = 0$, the linear mapping $Dq(\widetilde{\xi}) \epsilon \mathscr{L} (\mathbb{C}^{n+1}, \mathbb{C}^{n})$ (complex derivative) is onto.
\end{definition}

Of course, when $n = 1$, Definition \ref{chap2-def2.2} coincided with Definition \ref{chap2-def2.1}. Whenever the mapping $q$ verifies the condition $(\mathbb{C}-N.D.)$, it verifies the condition $(\mathbb{R}-N.D.)$ as well: This is immediately checked after observing for $\widetilde{\xi} \epsilon \mathbb{R}^{n+1}$ that the restriction to $\mathbb{R}^{n+1}$ of the complex derivative of $q$ at $\widetilde{\xi}$ is nothing but its real derivative. A method for checking the conditions $(\mathbb{R}-N.D.)$ and $(\mathbb{C}-N.D.)$ when $n = 2$ is described in Appendix 1 where we also make some comments on the general case, not quite completely solved however.

\section[A Generalization of the Morse Lemma.....]{A Generalization of the Morse Lemma for Mappings from $\mathbb{R}^{n+1}$ into $\mathbb{R}^{n}$ : ``Weak'' Regularity Results.}\label{chap2-sec3}

From\pageoriginale now on, the space $\mathbb{R}^{n+1}$ is equipped with its euclidean structure. Let $\mathscr{O}$ be an open neighbourhood of the origin in $\mathbb{R}^{n+1}$ and
$$
f : \mathscr{O} \to \mathbb{R}^{n},
$$
a mapping of class $C^{m}, m \geq 1$. Assume there is a positive integer $k, 1 \leq k \leq m$ such that
\begin{equation*}
D^{j}f(0) = 0 \quad 0 \leq j \leq k-1\tag{3.1}\label{chap2-eq3.1}
\end{equation*}
(in particular $f(0) = 0$). For every $\widetilde{\xi} \epsilon \mathbb{R}^{n+1}$, set
\begin{equation*}
q(\widetilde{\xi}) = D^{k}f(0) \cdot
(\widetilde{\xi})^{k}.\tag{3.2}\label{chap2-3.2}
\end{equation*}

Our purpose is to give a precise description of the zero set of $f$ around the origin (local zero set). We may limit ourselves to seeking {\em nonzero} solution only. For this, we first perform a {\em transformation of the problem}. Let $r_{0} > 0$ be such that the closed ball $\overline{B}(0, r_{0})$ in $\mathbb{R}^{n+1}$ is contained in $\mathscr{O}$. The problem will be solved if, for some $r, 0 \leq r \leq r_{0}$ and every $0 < |t| < r$, we are able to determine the solutions of the equation
$$
f(\widetilde{x}) = 0, ||\widetilde{x}|| = |t|.
$$

It is immediate that $\widetilde{\xi}$ is a solution for this system
if and only if we can write  
$$
\widetilde{x} = t \widetilde{\xi}
$$
with\pageoriginale
\begin{equation*}
\begin{cases}
& 0 < |t| < r, \widetilde{x} \epsilon S_{n},\\
& f(t \widetilde{x}) = 0
\end{cases}
\end{equation*}
where $S_{n}$ is the unit sphere in $\mathbb{R}^{n+1}$. Also it is not
restrictive to assume that $f$ is defined in the whole space
$\mathbb{R}^{n+1}$ (Indeed, $f$ can always be extended as a $C^{m}$
mapping outside $\overline{B} (0, r_{0})$).

Now, let us define
$$
g : \mathbb{R} \times \mathbb{R}^{n+1} \to \mathbb{R}^{n},
$$
by
\begin{equation*}
g(t, \widetilde{\xi}) = k \int_{0}^{1} (1-s)^{k-1} D^{k}f(st
\widetilde{\xi}) \cdot (\widetilde{\xi})^{k} ds,\tag{3.3}\label{chap2-eq3.3}
\end{equation*}
a fromula showing that $g$ is of class $C^{m-k}$ in $\mathbb{R} \times
\mathbb{R}^{n+1}$. 

\begin{lemma}\label{chap2-lem3.1}
Let $g$ be defined as above. Then
\begin{align*}
g(t, \widetilde{\xi}) & = \frac{k!}{t^{k}} f(t \widetilde{\xi}) \text{
for } t \neq 0, \widetilde{\xi} \epsilon
\mathbb{R}^{n+1},\tag{3.4}\label{chap2-eq3.4}\\
g(0, \widetilde{\xi}) & = D^{k} f(0) \cdot (\widetilde{\xi})^{k}
\text{ for } \widetilde{\xi} \epsilon \mathbb{R}^{n+1}.\tag{3.5}\label{chap2-eq3.5}
\end{align*}
\end{lemma}

\begin{proof}
The relation $g(0, \cdot) = q$ follows from the definitions
immediately. Next for $\widetilde{x} \epsilon \mathbb{R}^{n+1}$, write
the Taylor expansion of order $k-1$ of $f$ about the origin. Due to
(\ref{chap2-eq3.1}),
$$
f(\widetilde{x}) = \frac{1}{(k-1)!} \int_{0}^{1} (1-s)^{k-1} D^{k} f(s
\widetilde{x}) \cdot (\widetilde{x})^{k} ds.
$$

With $\widetilde{x} = t \widetilde{\xi}, t \neq 0$ and comparing with
(\ref{chap2-eq3.3}), we find
$$
g(t, \widetilde{\xi}) = \frac{k!}{t^{k}} f(t \widetilde{\xi}).
$$
\end{proof}

From\pageoriginale Lemma \ref{chap2-lem3.1} the problem is equivalent
to
\begin{equation*}
\begin{cases}
& 0 < |t| < r, \widetilde{\xi} \epsilon S_{n},\\
& g(t, \widetilde{\xi}) = 0.
\end{cases}
\end{equation*}

In what follows, we shall solve the equation (for small enough $r >
0$)
\begin{equation*}
\begin{cases}
& t \epsilon (-r, r), \widetilde{\xi} \epsilon S_{n},\\
& g(t, \widetilde{\xi}) = 0.
\end{cases}\tag{3.6}\label{chap2-eq3.6}
\end{equation*}

Its solutions $(t, \widetilde{\xi})$ with $t \neq 0$ will provide the
nonzero solutions of $f(\widetilde{x}) = 0$ verifying
$||\widetilde{x}|| = |t|$ through the simple relation $\widetilde{x}
= t \widetilde{\xi}$. Of course , the trivial solution $\widetilde{x}
= 0$ is also obtained as $\widetilde{x} = 0 \widetilde{\xi}$ with
$g(0, \widetilde{\xi}) = q(\widetilde{\xi}) = 0$ unless this equation
has no solution on the unit sphere. Thus all the solutions of
$f(\widetilde{x}) = 0$ such that $0 < ||\widetilde{x}|| < r$ are
given by $\widetilde{x} = t\widetilde{\xi}$ with $(t,
\widetilde{\xi})$ solution of (\ref{chap2-eq3.6}) provided that the
zero set of $q$ does not reduce to the origin.

From now on, we assume that the mapping $q$ verifies the condition
$(\mathbb{R}-N.D.)$ and we denote by $\nu \geq 0$ the number of lines
in the zero set of $q$, so that the set
\begin{equation*}
\{\widetilde{\xi} \epsilon S_{n} ; q(\widetilde{\xi}) = 0 \},\tag{3.7}\label{chap2-3.7}
\end{equation*}
has exactly $2\nu$ elements. We shall set
$$
\{\widetilde{\xi} \epsilon S_{n} ; q(\widetilde{\xi}) = 0 \} =
\{\widetilde{\xi}_{0}^{1}, \cdots, \widetilde{\xi}_{0}^{2\nu}\},
$$
with an obvious abuse of notation when $\nu = 0$. This set is stable
under\pageoriginale multiplication by $-1$, so that we may assume that
the $\widetilde{\xi}_{0}^{j}$'s have been arranged so that
\begin{equation*}
\widetilde{\xi}_{0}^{\nu+j} = -\widetilde{\xi}_{0}^{j}, 1 \leq j \leq \nu.\tag{3.8}\label{chap2-eq3.8}
\end{equation*}

\begin{lemma}\label{chap2-lem3.2}
(i) Assume $\nu \geq 1$ and for each index $1 \leq j \leq 2\nu$, let
  $\sigma_{j} \subset S_{n}$ denote a neighbourhood of
  $\widetilde{\xi}_{0}^{j}$. Then, there exists $0 < r \leq r_{0}$
  such that the conditions $(t, \widetilde{\xi}) \epsilon (-r, r)
  \times S_{n}$ and $g(t, \widetilde{\xi}) = 0$ together imply
$$
\widetilde{\xi} \epsilon \bigcup\limits_{j=1}^{2\nu} \sigma_{j}.
$$

(ii) Assume that $\nu = 0$. Then, there exists $0 < r < r_{0}$ such
that the equation $g(t, \widetilde{\xi}) = 0$ has no solution in the
set $(-r, r) \times S_{n}$.
\end{lemma}

\begin{proof}
(i) We argue by contradiction : If not, there is a sequence
  $(t_{\ell},\break \widetilde{\xi}_{\ell})_{\ell \geq 1}$ with $\lim\limits_{\ell
    \to \infty} t_{\ell} = 0$ and $\widetilde{\xi}_{\ell} \epsilon
  S_{n}$ such that
$$
g(t_{\ell}, \widetilde{\xi}_{\ell}) = 0
$$
and
$$
\widetilde{\xi}_{\ell} \notin \bigcup\limits_{j=1}^{2\nu} \sigma_{j}.
$$ 

From the {\em compactness} of $S_{n}$ and after considering a
subsequence, we may assume that there exists $\widetilde{\xi} \epsilon
S_{n}$ such that $\lim\limits_{\ell \to + \infty} \widetilde{\xi}_{\ell} =
\widetilde{\xi}$. By the continuity of $g, g(0, \widetilde{\xi}) =
0$. As $g(0, \cdot) = q, \widetilde{\xi}$ must be one of the elements
$\widetilde{\xi}_{0}^{j}$, which is impossible since
$$
\widetilde{\xi}_{\ell} \notin \bigcup\limits_{j=1}^{2\nu} \sigma_{j},
$$ 
for\pageoriginale every $\ell \geq 1$ so that the sequence
$(\widetilde{\xi}_{\ell})$ cannot converge to $\widetilde{\xi}$.

(ii) Again we argue by contradiction. If there is a sequence
$(t_{\ell}, \widetilde{\xi}_{\ell})_{\ell \geq 1}$ such that
$\lim\limits_{\ell \to + \infty} t_{\ell} = 0, \widetilde{\xi}_{\ell}
\epsilon S_{n}$ and $g(t_{\ell}, \widetilde{\xi}_{\ell}) = 0$, the
continuity of $g$ and the compactness of $S_{n}$ show that there is
$\widetilde{\xi} \epsilon S_{n}$ verifying $q(\widetilde{\xi}) = g(0,
\widetilde{\xi}) = 0$ and we reach a contradiction with the hypothesis
$\nu = 0$.
\end{proof}

\begin{remark}\label{chap2-rem3.1}
From Lemma \ref{chap2-lem3.2}, the equation $f(\widetilde{x}) = 0$ has
then {\em no solution} $\widetilde{x} \neq 0$ in a sufficiently small
neighbourhood of the origin when $\nu = 0$; in other words, the local
zero set of $f$ {\em reduces to the origin}.

We shall then focus on the main case when $\nu \geq 1$. For this we
need the following lemma.
\end{remark}

\begin{lemma}\label{chap2-lem3.3}
The mapping g verifies
$$
g \epsilon C^{m-k} (\mathbb{R} \times \mathbb{R}^{n+1}, \mathbb{R}^{n})
$$
and the partial derivative $D_{\widetilde{\xi}}g(t, \widetilde{\xi})$
exists for every pair $(t, \widetilde{\xi}) \epsilon \mathbb{R} \times
\mathbb{R}^{n+1}$. Moreover
$$
D_{\widetilde{\xi}}g \epsilon C^{m-k} (\mathbb{R} \times
\mathbb{R}^{n+1}, \mathscr{L}(\mathbb{R}^{n+1}, \mathbb{R}^{n}))
$$
\end{lemma}

\begin{proof}
We already know that $g \epsilon C^{m-k}$. Besides, the existence of a
partial derivative $D_{\widetilde{\xi}}g(t, \widetilde{\xi})$ at every
point $(t, \widetilde{\xi}) \epsilon \mathbb{R} \times
\mathbb{R}^{n+1}$ is obvious from the relations (\ref{chap2-eq3.4})
and (\ref{chap2-eq3.5}), from which we get 
\begin{equation*}
D_{\widetilde{\xi}}g(t, \widetilde{\xi}) = \frac{k!}{t^{k-1}}
Df(t\widetilde{\xi}) \text{ if } t \neq 0,\tag{3.9}\label{chap2-eq3.9}
\end{equation*}
and
\begin{equation*}
D_{\widetilde{\xi}}g(0, \widetilde{\xi}) = Dq(\widetilde{\xi}) =
kD^{k}f(0) \cdot (\widetilde{\xi})^{k-1}.\tag{3.10}\label{chap2-eq3.10}
\end{equation*}\pageoriginale

First, assume that $k = 1$. Then
\begin{align*}
D_{\widetilde{\xi}}g(t, \widetilde{\xi}) & = Df(t, \widetilde{\xi}),\\
D_{\widetilde{\xi}}g(0, \widetilde{\xi}) & = Df(0)
\end{align*}
and the assertion follows from the fact that $Df$ is $C^{m-1}$, by
hypothesis. Now, assume $k \geq 2$ and write the Taylor expansion of
$Df$ of order $k - 2$ about the origin. For every $\widetilde{x}
\epsilon \mathbb{R}^{n+1}$ and due to (\ref{chap2-eq3.1})
$$
Df(\widetilde{x}) = \frac{1}{(k-2)!} \int_{0}^{1} (1-s)^{k-2}
D^{k}f(s\widetilde{x}) \cdot (\widetilde{x})^{k-1} ds. 
$$

With $\widetilde{x} = t\widetilde{\xi}$,
$$
Df(t\widetilde{\xi}) = \frac{t^{k-1}}{(k-2)!} \int_{0}^{1} (1-s)^{k-2}
D^{k}f(st\widetilde{\xi}) \cdot (\widetilde{\xi})^{k-1} ds.
$$

From (\ref{chap2-eq3.9}) and (\ref{chap2-eq3.10}), the relation
\begin{equation*}
D_{\widetilde{\xi}}g(t, \widetilde{\xi}) = k(k-1) \int_{0}^{1} (1-s)^{k-2}
D^{k}f(st\widetilde{\xi})\cdot(\widetilde{\xi})^{k-1} ds
\end{equation*}
holds for every $t \epsilon \mathbb{R}$ and every $\widetilde{\xi}
\epsilon \mathbb{R}^{n+1}$. Hence the result, since the right hand
side of this identity is of class $C^{m-k}$.
\end{proof}

Finally, let us recall the so-called {\em ``strong''} version of the
Implicit function theorem (see Lyusternik and Sobolev \cite{22}).

\begin{lemma}\label{chap2-lem3.4}
Let U, V and W be real Banach spaces and $F = (F(u, v))$ a mapping
defined on a neighbourhood $\mathscr{O}$ of the origin in $U \times V$
with values\pageoriginale in $W$. Assume $F(0) = 0$ and
\begin{enumerate}
\item[(i)] F is continuous in $\mathscr{O}$,

\item[(ii)] the derivative $D_{v}F$ is defined and continuous in
  $\mathscr{O}$,

\item[(iii)] $D_{v}F(0) \epsilon Isom (V, W)$.
\end{enumerate}

Then, the zero set of $F$ around the origin in $U \times V$ coincides
with the graph of a continuous function defined in a neighbourhood of
the origin in $U$ with values in $V$.
\end{lemma}

\begin{remark}\label{chap2-rem3.2}
In the above statement, $F$ is {\em not} supposed to be $C^{1}$ and the
result is {\em weaker} than in the usual Implict function theorem. The
function whoce graph is the zero set of $F$ around the origin is found
to be poly continuous (instead of $C^{1}$). The proof of this
``strong'' version is the same as the proof of the usual statement
after observing that the assumptions of Lemma \ref{chap2-lem3.4} are
sufficient to prove continuity.
\end{remark}

We can now state an important result on the structure of the solutions
of the equation (\ref{chap2-eq3.6}).

\begin{theorem}\label{chap2-thm3.1}
Assume $\nu \geq 1$; then, there exists $r > 0$ such that the equation
$$
g(t,\widetilde{\xi}) = 0, (t, \widetilde{\xi}) \epsilon (-r, r) \times S_{n}
$$
is equivalent to
$$
t \epsilon (-r, r), \widetilde{\xi} = \widetilde{\xi}^{j} (t),
$$
for some index $1 \leq j \leq 2\nu$ where, for each index $1 \leq j
\leq 2\nu$, the function\pageoriginale $\widetilde{\xi}^{j}$ is of class $C^{m-k}$
from $(-r, r)$ into $S_{n}$ and is uniquely determined. In particular,
$$
\widetilde{\xi}^{j}(0) = \widetilde{\xi}_{0}^{j}, 1 \leq j \leq 2\nu,
$$
and
$$
\widetilde{\xi}^{v+j}(t) = -\widetilde{\xi}^{j} (-t),
$$
for every $1 \leq j \leq 2\nu$ and every $t \epsilon (-r, r)$.
\end{theorem}

\begin{proof}
We first solve the equation $g(t, \widetilde{\xi}) = 0$ around the
solution $(t = 0, \widetilde{\xi} = \widetilde{\xi}_{0}^{j})$ for each
index $1 \leq j \leq 2\nu$ separately. Let us then fix $1 \leq j \leq
2\nu$. As we in Chapter \ref{chap2}, \ref{chap2-sec1}, the condition
$(\mathbb{R}-N.D.)$ allows us to write
\begin{equation*}
\mathbb{R}^{n+1} = \Ker Dq (\widetilde{\xi}_{0}^{j}) \oplus
T_{\widetilde{\xi}_{0}^{j}} S_{n}.\tag{3.11}\label{chap2-eq3.11}
\end{equation*}

From (\ref{chap2-eq3.5}),
$$
Dq(\widetilde{\xi}_{0}^{j}) = D_{\widetilde{\xi}}g(0, \widetilde{\xi}_{0}^{j})
$$
and (\ref{chap2-eq3.11}) shows that
\begin{equation*}
D_{\widetilde{\xi}}g(0, \widetilde{\xi}_{0}^{j}) |_{
T_{\widetilde{\xi}_{0}^{j}} S_{n}} \epsilon Isom
(T_{\widetilde{\xi}_{0}^{j}} S_{n},
\mathbb{R}^{n}).\tag{3.12}\label{chap2-eq3.12}
\end{equation*}

Let then $\theta_{j}^{-1} (\theta_{j} = \theta_{j}(\xi'))$ be a chart
around $\widetilde{\xi}_{0}^{j}$, centered at the origin of
$\mathbb{R}^{n}$ (i.e. $\theta(0) = \widetilde{\xi}_{0}^{j}$) and set
$$
\hat{g}(t, \xi') = g(t, \theta_{j}(\xi')).
$$

Then $D_{\xi}, \hat{g}$ is defined around $(t = 0, \xi' = 0)$ and
$$
D_{\xi}, \hat{g}(t, \xi') = D_{\widetilde{\xi}}g(t, \theta_{j}(\xi'))
\cdot D\theta_{j}(\xi')
$$

From\pageoriginale Lemma \ref{chap2-lem3.3}, it follows that $\hat{g}$
and $D_{\xi}, \hat{g}$ are of class $C^{m-k}$ around the origin in
$\mathbb{R} \times \mathbb{R}^{n}$. Besides,
$$
\hat{g}(0) = g(0, \widetilde{\xi}_{0}^{j}) =
q(\widetilde{\xi}_{0}^{j}) = 0
$$
and combining (\ref{chap2-eq3.12}) with the fact that $D\theta_{j}(0)$
is an isomorphism of $\mathbb{R}^{n}$ to $T_{\widetilde{\xi}_{0}^{j}}
S_{n}$ (recall that $\theta_{j}^{-1}$ is a chart), one has
$$
D_{\xi}, \hat{g}(0) \epsilon Isom (\mathbb{R}^{n}, \mathbb{R}^{n}).
$$

If $m - k \geq 1$, the Implicit function theorem (usual version)
states that the zero set of $\hat{g}$ around the origin is the graph
of a (necessarily unique) mapping
$$
t \to \xi'(t)
$$
of class $C^{m-k}$ around the origin verifying $\xi'(0) = 0$. If $m-k
= 0$, the same result holds by using the ``strong'' version of the
Implict function theorem (Lemma \ref{chap2-lem3.4}). The zero of $g$
around the point $(0, \widetilde{\xi}_{0}^{j})$ is then the graph of
the mapping
$$
t \to \theta_{j}(\xi'(t)),
$$
of class $C^{m-k}$ around the origin, which is the desired mapping
$\widetilde{\xi}^{j}(t)$. In particular, given any sufficiently small
$r > 0$ and any sufficiently small neighbourhood $\sigma_{j}$ of
$\widetilde{\xi}_{0}^{j}$ in $S_{n}$, {\em there are no solutions of
  the equation $g(t, \widetilde{\xi}) = 0$ in $(-r, r) \times
  \sigma_{j}$ other than those of the form $(t, \widetilde{\xi}^{j}
  (t))$}.

At this stage, note that the above property is not affected by
arbitrarily shrinking $r > 0$. In particular, the same $r > 0$ can be
chosen\pageoriginale for every index $1 \leq j \leq 2\nu$. Further,
applying Lemma \ref{chap2-lem3.2} with disjoint neighbourhoods
$\sigma_{j}, 1 \leq j \leq 2\nu$, we see after shrinking $r$ again, if
necessary, that there in {\em no} solution of the equation $g(t,
\widetilde{\xi}) = 0$ with $|t| < r$ outside $\bigcup\limits_{j=1}^{2\nu}
\sigma_{j}$. In other words, we have $\widetilde{\xi}^{j} (t) \epsilon
\sigma_{j}$, $1 \leq j \leq 2\nu$ and the solutions of $g(t,
\widetilde{\xi}) = 0$ with $|t| < r$ are exactly the $2\nu$ pairs $(t,
\widetilde{\xi}^{j} (t)), 1 \leq j \leq 2\nu$. These pairs are
distinct, since the neighbourhoods $\sigma_{j}$ we have been taken
disjoint.

Finally, writing $g(-t, \widetilde{\xi}^{j}(-t)) = 0$ for $|t| < r$
and observing that $g(-t, -\widetilde{\xi}) = g(t, \widetilde{\xi})$,
we find $g(t, -\widetilde{\xi}^{j}(-t)) = 0$ for $|t| < r$. Since
$-\widetilde{\xi}^{j}(0) = -\widetilde{\xi}_{0}^{j} =
\widetilde{\xi}_{0}^{\nu+j}$ for $1 \leq j \leq \nu$
(cf. (\ref{chap2-eq3.8})), we deduce that the function
$-\widetilde{\xi}^{j}(-t)$ has the property characterizing
$\widetilde{\xi}^{\nu+j}(t)$ and the proof is complete.
\end{proof}

\begin{corollary}\label{chap2-coro3.1}
Under our assumptions, the equation $f(\widetilde{x}) = 0$ has no
solutions $(\widetilde{x}) \neq 0$ around the origin in
$\mathbb{R}^{n+1}$ when $\nu = 0$. When $\nu \geq 1$ and for $r > 0$
small enough, the solutions of the equation $f(\widetilde{x}) = 0$
with $||\widetilde{x}|| < r$ are given by $\widetilde{x} =
\widetilde{x}^{j}(t), 1 \leq j \leq \nu$, where the functions
$\widetilde{x}^{j} \epsilon C^{m-k} ((-r, r), \mathbb{R}^{n+1})$ are
defined through the functions $\widetilde{\xi}$, $1 \leq j \leq \nu$
of Theorem \ref{chap2-thm3.1} by the formula
$$
\widetilde{x}^{j}(t) = t\widetilde{\xi}^{j}(t), t \epsilon (-r, r).
$$

In addition, the functions $\widetilde{x}^{j}, 1 \leq j \leq \nu$,
are differentiable at the origin with 
$$
\frac{d\widetilde{x}^{j}}{dt} = \widetilde{\xi}^{j}, 1 \leq j \leq \nu.
$$
\end{corollary}

\begin{proof}
We\pageoriginale already observed in Remark \ref{chap2-rem3.1} that
the local zero set of $f$ reduces to $\{0\}$ when $\nu = 0$. Assume then
$\nu \geq 1$. We know that the solution of $f(\widetilde{x}) = 0$
with $||\widetilde{x}|| < r$ are of the form $\widetilde{x} =
t\widetilde{\xi}$ with $|t| < r$, $\widetilde{\xi} \epsilon S_{n}$,
$g(t, \widetilde{\xi}) = 0$. From Theorem \ref{chap2-thm3.1}, $r > 0$
can be chosen so that
$$
\widetilde{x} = \widetilde{x}^{j}(t) = t\widetilde{\xi}^{j}(t), 1
\leq j \leq 2\nu.
$$

Like $\widetilde{\xi}^{j}$, each function $\widetilde{x}^{j}$ is of
class $C^{m-k}$. In addition, for $t \neq 0$,
$$
\frac{\widetilde{x}^{j}(t)}{t} = \widetilde{\xi}^{j}(t),
$$
so that $\dfrac{d\widetilde{x}^{j}}{dt}(0)$ exists with
$$
\frac{d\widetilde{x}^{j}}{dt}(0) = \lim_{t \to 0}
\widetilde{\xi}^{j}(t) = \widetilde{\xi}_{0}^{j}.
$$

Finally, from the relation $\widetilde{\xi}^{\nu+j}(t) =
-\widetilde{\xi}^{j}(t)$, for $1 \leq j \leq \nu$, we get
$$
\widetilde{x}^{\nu + j}(t) = \widetilde{x}^{j}(t), 1 \leq j \leq \nu,
$$
and the solution of $f(x) = 0$ are given through the first $\nu$
functions $\widetilde{x}^{j}$ only.
\end{proof}

\section{Further Regularity Results.}\label{chap2-sec4}

With Corollary \ref{chap2-coro3.1} as a starting point, we shall now
show, {\em without any additional assumption, that the functions
  $\widetilde{x}^{j}$ are actually of class $C^{m-k+1}$ at the and of
  class $C^{m}$ away from it}. This latter assertion is the simpler one
to prove.

\begin{lemma}\label{chap2-lem4.1}
After\pageoriginale shrinking $r > 0$ if necessary, the functions
$\widetilde{x}^{j}, 1 \leq j \leq \nu$ are of class $C^{m}$ on $(-r,
r) - \{0\}$.
\end{lemma}

\begin{proof}
From the relation $\widetilde{x}^{j}(t) = t\widetilde{\xi}^{j}(t)$, it
is clear that the functions $\widetilde{x}^{j}$ and
$\widetilde{\xi}^{j}$ have the same regularity away from the
origin. Thus, we shall show that the functions $\widetilde{\xi}^{j}$
are of class $C^{m}$ away from the origin. Recall that
$\widetilde{\xi}^{j}$ is characterized by 
\begin{equation*}
\begin{cases}
& \widetilde{\xi}^{j}(t) \epsilon S_{n}\\
& g(t, \widetilde{\xi}^{j}(t)) = 0,\\
& \widetilde{\xi}^{j}(0) = \widetilde{\xi}_{0}^{j},
\end{cases}
\end{equation*}
for $t \epsilon (-r, r)$. Also recall (cf. (\ref{chap2-eq3.12}))
$$
D_{\widetilde{\xi}}g(0, \widetilde{\xi}_{0}^{j}) =
Dq(\widetilde{\xi}_{0}^{j}) \epsilon Isom (T_{\widetilde{\xi}_{0}^{j}}
S_{n}, \mathbb{R}^{n}).
$$

Then, by the continuity of $D_{\widetilde{\xi}}g$ (Lemma
\ref{chap2-lem3.3}), there is an open neighbourhood $\sigma_{j}$ of
$\widetilde{\xi}_{0}^{j}$ in $S_{n}$ such that, after shrinking $r >
0$ if necessary,
\begin{equation*}
D_{\widetilde{\xi}}g(t, \widetilde{\xi}) \epsilon Isom
(T_{\widetilde{\xi}}S_{n}, \mathbb{R}^{n})\tag{4.1}\label{chap2-eq4.1}
\end{equation*}
for every $(t, \widetilde{\xi}) \epsilon (-r, r) \times
\sigma_{j}$. By shrinking $r > 0$ again and due to the continuity of
the function $\widetilde{\xi}^{j}$, we can suppose that
$\widetilde{\xi}^{j}$ takes its values in $\sigma_{j}$ for $t \epsilon
(-r, r)$. Let then $t_{0} \epsilon (-r, r), t_{0} \neq 0$. Thus
$$
g(t_{0}, \widetilde{\xi}^{j}(t_{0})) = 0,
$$
and, from (\ref{chap2-eq4.1}),
$$
D_{\widetilde{\xi}}g(t_{0}, \widetilde{\xi}^{j}(t_{0})) \epsilon Isom
(T_{\widetilde{\xi}^{j} (t_{0})} S_{n}, \mathbb{R}^{n}).
$$\pageoriginale

As $t_{0} \neq 0, g(t, \widetilde{\xi})$ is given by
(\ref{chap2-eq3.4}) for $t$ around $t_{0}$ and $\widetilde{\xi} \epsilon
\mathbb{R}^{n+1}$ and thus the mapping $g$ has the same regularity as
$f$ (i.e. is of class $C^{m}$) around the point $(t_{0},
\widetilde{\xi}^{j}(t_{0}))$. By the Implicit function theorem, we
find that the zero set of $g$ around $(t_{0}, \widetilde{\xi}(t_{0}))$
in $(-r, r) \times S_{n}$ coincides with the graph of a unique
function $\widetilde{\zeta} = \widetilde{\zeta}(t)$ of class $C^{m}$
around $t_{0}$, such that $\widetilde{\zeta}^j (t_{0}) =
\widetilde{\xi}(t_{0})$. But from the {\em uniqueness}, we must have
$\widetilde{\zeta}(t) = \widetilde{\xi}^{j}(t)$ around $t = t_{0}$ so
that the function $\widetilde{\xi}^{j}(\cdot)$ is of class $C^{m}$
around $t_{0}$ for every $t_{0} \neq 0$ in $(-r. r)$.
\end{proof}

To prove the regularity $C^{m-k+1}$ of the function
$\widetilde{x}^{j}$ at the origin, we shall introduce the mapping
$$
h : \mathbb{R} \times \mathbb{R}^{n+1} \to \mathbb{R}^{n},
$$
defined by
\begin{equation*}
h(t, \widetilde{\xi}) = Df(t\widetilde{\xi}) \cdot \widetilde{\xi} -
\int_{0}^{1} Df(st \widetilde{\xi}) \cdot \widetilde{\xi} ds,\tag{4.2}\label{chap2-eq4.2}
\end{equation*}
if $k = 1$ and by
\begin{equation*}
h(t, \xi) = k \int_{0}^{1} \frac{d}{ds} [-s(1-s)^{k-1}]
D^{k}f(st\widetilde{\xi}) \cdot (\widetilde{\xi})^{k} ds,\tag{4.3}\label{chap2-eq4.3}
\end{equation*}
if $k \geq 2$. Since $f$ is of class $C^{m}$, it is clear, in any case,
that $h \epsilon C^{m-k}(\mathbb{R} \times \mathbb{R}^{n+1},
\mathbb{R}^{n})$. In addition, with $t = 0$ in the definition of
h((\ref{chap2-eq4.3})), we find
\begin{equation*}
h(0, \widetilde{\xi}) = 0 \text{ for every } \widetilde{\xi} \epsilon
\mathbb{R}^{n+1}.\tag{4.4}\label{chap2-eq4.4}
\end{equation*}

\begin{lemma}\label{chap2-lem4.2}
For\pageoriginale any $(t, \widetilde{\xi}) \epsilon (\mathbb{R} -
\{0\}) \times \mathbb{R}^{n+1}$, the partial derivative $(\partial
  g/\partial t)\break (t, \widetilde{\xi})$ exists and
\begin{equation*}
\frac{\partial q}{\partial t} (t, \widetilde{\xi}) = \frac{1}{t}h(t, \widetilde{\xi}).\tag{4.5}\label{chap2-eq4.5}
\end{equation*}
\end{lemma}

\begin{proof}
The existence of $\dfrac{\partial q}{\partial t} (t, \widetilde{\xi})$
for $t \neq 0$ immediate from the relation (\ref{chap2-eq3.4}) from
which we get
$$
\frac{\partial q}{\partial t} (t, \widetilde{\xi}) = \frac{k!}{t^{k}}
(Df(t\widetilde{\xi}) \cdot \widetilde{\xi} - \frac{k}{t} f(t\widetilde{\xi})).
$$
Hence
\begin{equation*}
\frac{t\partial q}{\partial t}(t, \widetilde{\xi}) =
\frac{k!}{t^{k-1}} (Df(t\widetilde{\xi}) \cdot \widetilde{\xi} -
\frac{k}{t} f(t\widetilde{\xi})).\tag{4.6}\label{chap2-eq4.6}
\end{equation*}

If $k = 1$, the relation (\ref{chap2-eq4.6}) becomes
$$
\frac{t\partial q}{\partial t} (t, \widetilde{\xi}) =
Df(t\widetilde{\xi}) \cdot \widetilde{\xi} - \frac{1}{t} f(t\widetilde{\xi}).
$$

Writing
$$
f(t\widetilde{\xi}) = t\int_{0}^{1} Df(st\widetilde{\xi}) \cdot
\widetilde{\xi} ds,
$$
the desired relation (\ref{chap2-eq4.5}) follows from
(\ref{chap2-eq4.2}).

Now, assume $k \geq 2$. Recall the relations
\begin{align*}
f(t \widetilde{\xi}) & = \frac{t^{k}}{(k-1)!} \int_{0}^{1} (1-s)^{k-1}
D^{k}f(st\widetilde{\xi}) \cdot (\widetilde{\xi})^{k} ds,\\
Df(t\widetilde{\xi}) & = \frac{t^{k-1}}{(k-2)!} \int_{0}^{1}
(1-s)^{k-2} D^{k}f(st\widetilde{\xi}) \cdot (\widetilde{\xi})^{k-1} ds,
\end{align*}
that we have already used in the proofs of Lemma \ref{chap2-eq3.1} and
Lemma \ref{chap2-lem3.3} respectively. After an immediate calculation,
(\ref{chap2-eq4.6}) becomes
$$
t \frac{\partial q}{\partial t} (t, \widetilde{\xi}) = k \int_{0}^{1}
[(k-1)(1-s)^{k-2}] D^{k}f(st\widetilde{\xi}) \cdot
(\widetilde{\xi})^{k} ds.
$$\pageoriginale

But
$$
(k-1)(1-s)^{k-2} - k(1-s)^{k-1} = \frac{d}{ds} \left[-s(1-s)^{k-1} \right],
$$
so that (\ref{chap2-eq4.5}) follows from the definition
(\ref{chap2-eq4.3}) of $h$.
\end{proof}

\begin{theorem}\label{chap2-thm4.1}
(Structure of the local zero set of $f$): Assume $\nu \geq 1$. For
  sufficiently small, $r > 0$ the local zero set of $f$ in the ball
  $B(0, r) \subset \mathbb{R}^{n+1}$ consists of exactly $\nu$ curves
  of class $C^{m-k+1}$ at the origin and class $C^{m}$ away from the
  origin. These curves are tangent to a different one from among the
  $\nu$ lines in the zero set of $q(\widetilde{\xi}) = D^{k}f(0) \cdot
  (\widetilde{\xi})^{k}$ at the origin.
\end{theorem}

\begin{proof}
In Lemma \ref{chap2-lem4.1}, we proved that the functions
$\widetilde{x}^{j}$ and $\widetilde{\xi}^{j}$ are of class $C^{m}$
away from the origin. Since $m \geq 1$, we may differentiate the
identity
$$
\widetilde{x}^{j}(t) = t\widetilde{\xi}^{j}(t),
$$
to get
\begin{equation*}
\frac{d\widetilde{x}^{j}(t)}{dt} = t
\frac{d\widetilde{\xi}^{j}(t)}{dt} + \widetilde{\xi}^{j}(t), 0 < |t| <
r.\tag{4.7}\label{chap2-eq4.7} 
\end{equation*}

Also, we know that $\widetilde{x}^{j}$ is differentiable at the origin
with 
$$
\frac{d\widetilde{x}^{j}}{dt} (0) = \widetilde{\xi}_{0}^{j}.
$$

We shall prove that the function $(d\widetilde{x}^{j} / dt)$ is of
class $C^{m-k}$ at the origin. Let $\sigma_{j} \subset S_{n}$ be the
open neighbourhood of $\widetilde{\xi}_{0}^{j}$ considered in
Lemma \ref{chap2-lem4.1},\pageoriginale so that
$D_{\widetilde{\xi}}g(t, \widetilde{\xi}) \epsilon Isom
(T_{\widetilde{\xi}}S_{n}, \mathbb{R}^{n})$ for every $(t,
\widetilde{\xi}) \epsilon (-r, r) \times \sigma_{j}$. In other words,
the mapping $\left[D_{\widetilde{\xi}}g(t, \widetilde{\xi}
  |_{T_{\widetilde{\xi}}S_{n}})\right]^{-1}$ is an isomorphism of
$\mathbb{R}^{n}$ to the space $T_{\widetilde{\xi}}S_{n} \subset
\mathbb{R}^{n+1}$ for every pair $(t, \widetilde{\xi}) \epsilon (-r,
r) \times \sigma_{j}$ and hence can be considered as a one-to-one
linear mapping from $\mathbb{R}^{n}$ into $\mathbb{R}^{n+1}$ (with
range $T_{\widetilde{\xi}}S_{n} \subset \mathbb{R}^{n+1}$). A simple
but crucial observation is that {\em the regularity $C^{m-k}$ of the
  mapping $D_{\widetilde{\xi}}g$ (Lemma \ref{chap2-lem3.3}) yields the
regularity $C^{m-k}$ of the mapping}
$$
(t, \widetilde{\xi}) \epsilon (-r, r) \times \sigma_{j} \to
\left[D_{\widetilde{\xi}}g(t, \widetilde{\xi})
  |_{T_{\widetilde{\xi}}S_{n}}\right]^{-1} \epsilon \mathscr{L}
(\mathbb{R}^{n}, \mathbb{R}^{n+1}).
$$

This is easily seen by considering a chart of $S_{n}$ around
$\widetilde{\xi}_{0}^{j}$ and can be formally seen by observing that
the dependence of the tangent space $T_{\widetilde{\xi}}S_{n}$ on the
variable $\widetilde{\xi} \epsilon S_{n}$ is $C^{\infty}$ while taking
the inverse of an invertible linear mapping is a $C^{\infty}$
operation. Setting, for $(t, \widetilde{\xi}) \epsilon (-r, r) \times
\sigma_{j}$, 
$$
\varphi^{j} (t, \widetilde{\xi}) = \left[D_{\widetilde{\xi}}g(t,
  \widetilde{\xi}) |_{T_{\widetilde{\xi}}S_{n}}\right]^{-1} h(t, \widetilde{\xi})
$$
and noting that $h \epsilon C^{m-k} (\mathbb{R} \times
\mathbb{R}^{n+1}, \mathbb{R}^{n})$, we deduce
$$
\varphi^{j} \epsilon C^{m-k} ((-r, r) \times \sigma_{j}, \mathbb{R}^{n+1}).
$$

Note from (\ref{chap2-eq4.4}) that $\varphi^{j}(0, \widetilde{\xi}) =
0$ for $\widetilde{\xi} \epsilon \sigma_{j}$. On the other hand, by
implict differentiation of the identity $g(t, \widetilde{\xi}^{j}(t))
= 0$ for $0 < |t| < r$, (the chain rule applies since $g$ is
differentiable with respect\pageoriginale to $(t, \widetilde{\xi})$ at
any point of $(\mathbb{R} - \{0\}) \times \mathbb{R}^{n+1}$),
$$
\frac{\partial q}{\partial t} (t, \widetilde{\xi}^{j}(t)) +
D_{\widetilde{\xi}}g(t, \widetilde{\xi}^{j}(t)) \cdot
\frac{d\widetilde{\xi}^{j}}{dt} (t) = 0.
$$

But $(d\widetilde{\xi}^{j} / dt)(t) \epsilon
T_{\widetilde{\xi}^{j}(t)}S_{n}$, since $\widetilde{\xi}^{j}$ takes
its values in $S_{n}$. Hence
$$
\frac{d\widetilde{\xi}^{j}}{dt} (t) = -\left[D_{\widetilde{\xi}}g(t,
  \widetilde{\xi}^{j} (t))
  |_{T_{\widetilde{\xi}^{j}(t)}S_{n}}\right]^{-1} \frac{\partial
  q}{\partial t} (t, \widetilde{\xi}^{j} (t)).
$$

With Lemma \ref{chap2-lem4.2}, this yields
$$
\frac{d\widetilde{\xi}^{j}}{dt} (t) = -\frac{1}{t} \varphi^{j} (t,
\widetilde{\xi}^{j}(t)), 0 < |t| < r
$$  
and the relation (\ref{chap2-eq4.7}) can be rewritten as
\begin{equation*} 
\frac{d\widetilde{x}^{j}}{dt} (t) = -\varphi^{j}(t,
\widetilde{\xi}^{j}(t)) + \widetilde{\xi}^{j}(t), 0 < |t| < r.\tag{4.8}\label{chap2-eq4.8}  
\end{equation*}

But (cf. Corollary \ref{chap2-coro3.1})
$$
\frac{d\widetilde{x}^{j}}{dt} (0) = \widetilde{\xi}_{0}^{j},
$$
whereas
$$
-\varphi^{j}(0, \widetilde{\xi}(0)) + \widetilde{\xi}^{j}(0) =
-\varphi^{j}(0, \widetilde{\xi}_{0}^{j}) + \widetilde{\xi}_{0}^{j} = \xi_{0}^{j}.
$$
since $\varphi^{j}(0, \widetilde{\xi}) = 0$ for $\widetilde{\xi}
\epsilon \sigma_{j}$. Thus the identity (\ref{chap2-eq4.8}) holds for
every $t \epsilon (-r, r)$. As its right hand side is of class
$C^{m-k}$, we obtain 
$$
\frac{d\widetilde{\xi}^{j}}{dt} \epsilon C^{m-k} ((-r, r), \mathbb{R}^{n}).
$$

As a last step, it remains to show that the curves generated by the
functions $\widetilde{\xi}^{j}, 1 \leq j \leq 2\nu$ have the same
regularity as the functions $\widetilde{x}^{j}$ themselves. From an
elementary result of differential geometry,\pageoriginale it is
sufficient to prove that
$$
\frac{d\widetilde{x}^{j}}{dt}(t) \neq 0, t \epsilon (-r, r).
$$

But this is immediate from (\ref{chap2-eq4.8}) by observing that
$\widetilde{\xi}^{j}(t)$ and $\varphi^{j} (t,\break \widetilde{\xi}^{j}(t))$
are orthogonal, since $\varphi^{j}(t, \widetilde{\xi}^{j}(t)) \epsilon
T_{\widetilde{\xi}^{j}(t)} S_{n}$. Hence
$$
||\frac{d\widetilde{x}^{j}(t)}{dt}|| \geq ||\xi^{j}(t)|| = 1, t
\epsilon (-r, r)
$$
and the proof is complete.
\end{proof}

\begin{comment}\label{chap2-com4.1}
When $k = 1$, the condition $(\mathbb{R}-N.D.)$ amounts to saying that
$Df(0)$ is onto. In particular, $\nu = 1$ and the local zero set of $f$
is made up of exactly one curve of class $C^{m}$ away from the origin
and also $C^{m}$ at the origin: the conclusion is the same as while
using the {\em Implict function theorem.}
\end{comment}

\begin{comment}\label{chap2-com4.2}
Assume now $n = 1$ and $k = 2$. The condition $(\mathbb{R}-N.D.)$ is
the Morse condition and we know that $\nu = 0$ or $\nu = 2$. The
statement is noting but the weak form of the Morse Lemma.
\end{comment}

\begin{comment}\label{chap2-com4.3}
In the same direction see the articles by Magnus \cite{24}, Buchner,
Marsden and Schecter \cite{5} Szulkin \cite{40} among others. Theorem
\ref{chap2-thm4.1} is a particular case of the study made in Rabier
    \cite{29}. More generally, the following extension (which, however, has
    no major application in the nondegenerate cases we shall consider
    in Chapter \ref{chap3}) does not requires $Df(0)$ to vanish. Such
    a result is important in generalizations of the desingularization
    process we shall describe in Chapter \ref{chap5}.
\end{comment}

\begin{theorem}\label{chap2-thm4.2}
Let\pageoriginale f be a mapping of class $C^{m}, m \geq 1$, defined
on a neighbouhood of the origin in $\mathbb{R}^{n+1}$ with values in
$\mathbb{R}^{n}$. Let us set
$$
n_{0} = n - \dim  Range Df(0),
$$
so that $0 \leq n_{0} \leq n$ and the spaces $\Ker Df(0)$ and
$\mathbb{R}^{n}$/Range $Df(0)$ can be identified with
$\mathbb{R}^{n_{0}+1}$ and $\mathbb{R}^{n_{0}}$ respectively. Let
$\pi$ denote the canonical projection operator from $\mathbb{R}^{n}$
onto $\mathbb{R}^{n}$/Range $Df(0)$. We assume that
\begin{equation*}
\begin{cases}
& f(0) = 0\\
& \pi D^{j}f(0) = 0, 1 \leq j \leq k-1
\end{cases}
\end{equation*}
for some $1 \leq k \leq m$ and that the mapping
{\fontsize{10}{12}\selectfont
$$
q_{0} : \widetilde{\xi} \epsilon \Ker Df(0) \simeq \mathbb{R}^{n_{0}+1}
\to q_{0} (\widetilde{\xi}) = \pi D^{k}f(0) \cdot
(\widetilde{\xi})^{k} \epsilon \mathbb{R}^{n} / Range Df(0) \simeq
\mathbb{R}^{n_{0}}
$$}
verifies the condition $(\mathbb{R}-N.D.)$. Then the zero set of the
mapping consists of a finite number $\nu \leq k^{n_{0}}$ of lines
through the origin in $\Ker  Df(0)$ and the local zero set of the
mapping f consists of exactly $\nu$ curves of class $C^{m}$ away from
the origin and of class $C^{m-k+1}$ at the origin. These $\nu$ curves
are tangent to a different one from among the lines of the zero of the
mapping $q_{0}$ at the origin.

For a proof, cf. \cite{29}. Theorem 3.2.
\end{theorem}

\section[A Generalization of the Strong Version.....]{A Generalization of the Strong Version of the Morse
  Lemma.}\label{chap2-sec5} 

To complete this chapter, we shall see that the local zero
set\pageoriginale of $f$ can be deduced from the zero set of the
polynoimal mapping $q(\widetilde{\xi}) = D^{k}f(0) \cdot
(\widetilde{\xi})^{k}$ through an origin-preserving $C^{m-k+1}$ local
diffeomorphism of the {\em ambient space} $\mathbb{R}^{n+1}$ under
some additional assumptions on the lines in the zero set of $q$.

\begin{theorem}\label{chap2-thm5.1}
Under the additional assumptions that $\nu \leq n + 1$ and the $\nu$
lines in the zero set of q are linearly independent (a vacuous
condition if $\nu = 0$), there exists an origin-preserving local
diffeomorphism $\phi$ of class $C^{m-k+1}$ in $\mathbb{R}^{n+1}$ that
transforms the local zero set of the mapping q into the local zero set
of the mapping f. Moreover, $\phi$ is of class $C^{m}$ away from the
origin and can be taken so that $D\phi(0) \equiv I$.
\end{theorem}

\begin{proof}
If $\nu = 0$, we can choose $\phi = I$. If $\nu \geq 1$, the local
zero set of $f$ is the image of the $\nu$ functions
$\widetilde{x}^{j}(t)$, $1 \leq j \leq \nu$, $|t| < r$. These
functions are of class $C^{m-k+1}$ at the origin and of class $C^{m}$
away from it, with $(d\widetilde{x}^{j} / dt)(0) =
\widetilde{\xi}_{0}^{j}$, $1 \leq j \leq \nu$.

Let us denote by $(\cdot | \cdot)$ the usual inner product of
$\mathbb{R}^{n+1}$. We may write
$$
\widetilde{x}^{j}(t) = \alpha_{j}(t) \widetilde{\xi}_{0}^{j} +
\widetilde{\zeta}^{j}(t), 
$$
where
$$
\alpha_{j}(t) = (\widetilde{x}^{j}(t) | \widetilde{\xi}_{0}^{j}).
$$

Therefore, both functions $\alpha_{j}$ and $\widetilde{\zeta}^{j}$ are
of class $C^{m-k+1}$ at the origin\pageoriginale and of class $C^{m}$
away from it. Besides,
\begin{equation*}
\begin{cases}
& \alpha_{j}(0) = 0 \epsilon \mathbb{R}, \frac{d\alpha_{j}}{dt} (0) =
  1,\\
& \widetilde{\zeta}^{j} (0) = 0 \epsilon \mathbb{R}^{n+1},
  \frac{d\widetilde{\zeta}^{j}}{dt} (0) = 0 \epsilon
  \mathbb{R}^{n+1}.
\end{cases}\tag{5.1}\label{chap2-eq5.1} 
\end{equation*}

From (\ref{chap2-eq5.1}), after shrinking $r$ if necessary, we deduce
that the function $\alpha_{j}$ is a $C^{m-k+1}$ diffeomorphism from
$(-r, r)$ to an open interval $I_{j}$ containing zero. We shall set
$$
(-\rho, \rho) = \bigcap\limits_{j=1}^{\nu} I_{j}, \rho > 0
$$
so that each function $\alpha_{j}^{-1}$ is well defined, of class
$C^{m-k+1}$ in $(-\rho, \rho)$ and of class $C^{m}$ away from the
origin,

As the $\nu$ vectors $\widetilde{\xi}_{0}^{j}$, $1 \leq j \leq \nu$
are linearly independent we can find a bilinear form $a(\cdot ,
\cdot)$ on $\mathbb{R}^{n+1}$ such that
\begin{equation*}
a(\widetilde{\xi}_{0}^{i}, \widetilde{\xi}_{0}^{j}) = \delta_{ij}
\text{(Kronecker delta)}, 1 \leq i, j \leq \nu\tag{5.2}\label{chap2-eq5.2}
\end{equation*}

For $||\widetilde{\xi}|| < \rho$, let us define
$$
\phi(\widetilde{\xi}) = \widetilde{\xi} + \sum\limits_{j=1}^{\nu}
\widetilde{\zeta}_{0}^{j_{\circ}} \alpha_{j}^{1_{\circ}}
a(\widetilde{\xi}, \widetilde{\xi}_{0}^{j})
$$
clearly, $\phi$ is of class $C^{m-k+1}$ at the origin and of class
$C^{m}$ away from it and further $\phi(0) = 0$, $D\phi(0) =
I$. Therefore $\phi$ is an origin preserving local diffeomorphism of
$\mathbb{R}^{n+1}$ having the desired regularity properties. Let us
show that $\phi$ transforms the local zero set of $q$ into the local
zero set of : For $|\tau| < \rho$ and $1 \leq i \leq \nu$
\begin{align*}
\phi(\tau \widetilde{\xi}_{0}^{i}) & = \tau \widetilde{\xi}_{0}^{j} +
\sum\limits_{j=1}^{\nu} \widetilde{\zeta}^{j} (\alpha_{j}^{-1}(a(\tau
\widetilde{\xi}_{0}^{i}, \widetilde{\xi}_{0}^{j})))\\
& = \tau \widetilde{\xi}_{0}^{i} + \sum\limits_{j=1}^{\nu}
\widetilde{\zeta} (\alpha_{j}^{-1}(\tau) \delta_{ij}) = \tau
\widetilde{\xi}_{0}^{i} + \widetilde{\zeta}^{i} (\alpha_{i}^{-1}
(\tau))\\
& = \widetilde{x}^{i} (\alpha_{i}^{-1} (\tau)).
\end{align*}\pageoriginale
\end{proof}

\begin{comment}\label{chap2-com5.1}
Assume $k = 2$, $n = 1$. Then the condition $(\mathbb{R}-NM.D.)$ is
equivalent to the Morse condition and $\nu = 0$ or $\nu = 2$. In any
case, $\nu \leq n + 1 = 2$ and, if $\nu = 2$, the two lines are
distinct. Thus, they are linearly independent and Theorem
\ref{chap2-thm5.1} coincides with the ``strong'' version of the Morse
lemma (Theorem \ref{chap1-thm3.1} of Chapter \ref{chap1}).
\end{comment}

\begin{remark}\label{chap2-rem5.1}
If the diffeomorphism $\phi$ is only required to be $C^{1}$ at the
origin, the result is true {\em regardless of the number of lines} in
the zero set of $q(\leq k^{n}$ because of the condition
$(\mathbb{R}-N.D.)$) and without assuming of course that they are
linearly independent. The condition also extends to mapping from
$\mathbb{R}^{n+p}$ into $\mathbb{R}^{n}$, $p \geq 1$. Under this form,
there is also a {\em stronger result} when a {\em stronger assumption}
holds : Assume for every $\widetilde{\xi} \epsilon \mathbb{R}^{n+1} -
\{0\}$ that the derivative $Dq(\widetilde{\xi})$ is onto. Then, the
diffeomorphism $\phi$ can be taken so that
$$
f(\phi(\widetilde{\xi})) = q(\widetilde{\xi}),
$$ 
for $\widetilde{\xi}$ around the origin (see Buchner, Marsden and
Schecter \cite{5}). For $k = 1$, the assumption reduces to sayong that
$Df(0)$ is onto. For $k = 2$ and $n = 1$, it reduces to the Morse
condition again.
\end{remark}

