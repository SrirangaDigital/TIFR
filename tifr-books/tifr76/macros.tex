\usepackage{graphicx,xspace,fancybox}
%\usepackage[section]{placeins}
\newcounter{pageoriginal}
\marginparwidth=10pt
\marginparsep=10pt
\marginparpush=5pt
%\renewcommand{\thepageoriginal}{\arabic{pageoriginal}}
\newcommand{\pageoriginale}{\refstepcounter{pageoriginal}\marginpar{\footnotesize\xspace\textbf{\thepageoriginal}}
} 
\let\pageoriginaled\pageoriginale

\newcounter{mysection}[chapter]


\newtheorem{corollary}{Corollary}[section]       
\def\thecorollary{\themysection.\arabic{corollary}}
\newtheorem{lemma}{Lemma}[section]
\def\thelemma{\themysection.\arabic{lemma}}

\newtheorem{theorem}{Theorem}[mysection]

\newtheorem{lem}{Lemma}[section]
\newtheorem{proposition}{Proposition}[mysection]
\newtheorem{sublemma}{Lemma}[section]
\newtheorem{subtheorem}{Theorem}[section]
\newtheorem{autothm}{AUTOMORPHISM THEOREM}[section]
\newtheorem{prop}{Proposition}[section]
\newtheorem{subprop}{Proposition}[subsection]
\newtheorem{subdefin}{Definition}[subsection]
\newtheorem{cor}{Corollary}
\newtheorem{coro}{Corollary}[section]
\newtheorem{subcoro}{Corollary}[subsection]
\newtheorem{defn}{Definition}
\newtheorem{gausslemma}{Gauss Lemma}[section]
\newtheorem{Theorem}{THEOREM}[section]



\newtheoremstyle{remark}{10pt}{10pt}{ }%
{}{\bfseries}{.}{ }{}
\theoremstyle{remark}
\newtheorem{remark}{Remark}[mysection]
\newtheorem{exercise}{Exercise}[section]
\newtheorem{definition}{Definition}[mysection]
\newtheorem{remarks}{Remarks}[section]
\newtheorem{comment}{COMMENT}[mysection]
\newtheorem{subremark}{Remark}[section]
\newtheorem{problem}{Problem}[section]
\newtheorem{example}{Example}[section]
\newtheorem{subexample}{Example}[section]
\newtheorem{exam}{Example}[section]
\newtheorem{Definition}{Definition}[section]
\newtheorem{thmm}{Theorem}[section]
\newtheorem{examples}{Examples}[section]
\newtheorem{Note}{Note}[section]
\newtheorem{note}{Note}[section]
\newtheorem{notation}{Notation}[section]
\newtheorem{claim}{Claim}[section]
\newtheorem{application}{Application}[section]
\newtheorem{proofoflemma}{Proof of Lemma}[section]
\newtheorem{proofofsublemma}{Proof of Lemma}[subsection]
\newtheorem{proofofprop}{Proof of Proposition}[section]
\newtheorem{proofoftheorem}{Proof of theorem}[subsection]
\newtheorem{proofofthm}{Proof of Theorem}[section]
\newtheorem{proofcontd}{Proof continued}[subsection]
	
\newtheoremstyle{uremark}{10pt}{10pt}{ }%
{}{\bfseries}{{\boldmath$_{m}$}.}{ }{}
\theoremstyle{uremark}
\newtheorem{uproblem}{Problem}[section]



\newtheoremstyle{nonum}{}{}{\itshape}{}{\bfseries}{\kern -2pt{\bf.}}{ }{#1 \mdseries
{\bf #3}}
\theoremstyle{nonum}

\newtheorem{lemma*}{Lemma}	
\newtheorem{theorem*}{Theorem}	
\newtheorem{irreducibilitythm*}{IRREDUCIBILITY THEOREM}
\newtheorem{genirreducibilitythm*}{GENERIC IRREDUCIBILITY THEOREM}
\newtheorem{embedthm*}{EMBEDDING THEOREM}
\newtheorem{prop*}{Proposition}	
\newtheorem{claim*}{Claim}
\newtheorem{conjecture*}{CONJECTURE}
\newtheorem{coro*}{Corollary}

\newtheoremstyle{mynonum}{}{}{ }{}{\bfseries}{\kern -2pt{\bf.}}{ }{#1 \mdseries
{\bf #3}}
\theoremstyle{mynonum}
\newtheorem{remark*}{Remark}	
\newtheorem{remarks*}{Remarks}	
%\newtheorem{defi*}{Definition}
\newtheorem{exer*}{Exercise}	
\newtheorem{example*}{EXAMPLE}	
\newtheorem{examples*}{EXAMPLES}	
\newtheorem{note*}{Note}
\newtheorem{proofofproposition}{Proof of proposition}
\newtheorem{corollaryoflemma}{Corollary of lemma}
\newtheorem{proofofthelemma}{Proof of the Lemma}
\newtheorem{defi*}{Definition}


\newtheoremstyle{dashthm}{}{}{\itshape}{}{\bfseries}{$'$.~}{ }{}
\theoremstyle{dashthm}
\newtheorem{dashprop}{Proposition}

\newtheoremstyle{dashexam}{}{}{\itshape}{}{\bfseries}{$'$.~}{ }{}
\theoremstyle{dashexam}
\newtheorem{dashexam}{Example}[section]


\def\ophi{\overset{o}{\phi}}


\def\oval#1{\text{\cornersize{2}\ovalbox{$#1$}}}


\newcommand*\mycirc[1]{%
  \tikz[baseline=(C.base)]\node[draw,circle,inner sep=.7pt](C) {#1};\:
}


\def\p{\partial}
\def\ora#1{\overrightarrow{#1}}

\DeclareMathOperator*{\Int}{Int}
\DeclareMathOperator*{\Min}{Min}
\DeclareMathOperator{\Div}{div}
\DeclareMathOperator{\grad}{grad}
\DeclareMathOperator{\Au}{Au}
\DeclareMathOperator{\supp}{supp}
\DeclareMathOperator{\tr}{tr}
\DeclareMathOperator{\card}{card}
\DeclareMathOperator{\Dp}{Dp}
\DeclareMathOperator{\Dv}{Dv}

\DeclareMathOperator{\Ker}{Ker}
\DeclareMathOperator{\meas}{meas}


\def\uub#1{\underline{\underline{#1}}}
\def\ub#1{\underline{#1}}
\def\oob#1{\overline{\overline{#1}}}
\def\ob#1{\overline{#1}}


\font\bigsymb=cmsy10 at 4pt
\def\bigdot{{\kern1.2pt\raise 1.5pt\hbox{\bigsymb\char15}}}
\def\overdot#1{\overset{\bigdot}{#1}}

\makeatletter
%\renewcommand\subsection{\@startsection{subsection}{2}{\z@}%
%                                     {-3.25ex\@plus -1ex \@minus -.2ex}%
%                                     {-1.5ex \@plus .2ex}%
%                                     {\normalfont}}%

\renewcommand\section{\@startsection{section}{1}{\z@}%
                                   {-3.5ex \@plus -1ex \@minus -.2ex}%
                                   {2.3ex \@plus.2ex}%
                                   {\stepcounter{mysection}\normalfont\Large\bfseries}}

\renewcommand\thesection{\thechapter.\@arabic\c@section}
\renewcommand\themysection{\@arabic\c@mysection}

\renewcommand\thesubsection{\thesection\@alph\c@subsection}

\renewcommand{\@seccntformat}[1]{{\csname the#1\endcsname}\hspace{0.3em}}
\makeatother

\def\fibreproduct#1#2#3{#1{\displaystyle\mathop{\times}_{#3}}#2}
\let\fprod\fibreproduct

\def\fibreoproduct#1#2#3{#1{\displaystyle\mathop{\otimes}_{#3}}#2}
\let\foprod\fibreoproduct


\def\cf{{cf.}\kern.3em}
\def\Cf{{Cf.}\kern.3em}
\def\eg{{e.g.}\kern.3em}
\def\ie{{i.e.}\kern.3em}
\def\iec{{i.e.,}\kern.3em}
\def\idc{{id.,}\kern.3em}
\def\resp{{resp.}\kern.3em}


\def\bA{\mathbf{A}}
\def\bB{\mathbf{B}}
\def\bC{\mathbf{C}}
\def\bD{\mathbf{D}}
\def\bE{\mathbf{E}}
\def\bF{\mathbf{F}}
\def\bG{\mathbf{G}}
\def\bH{\mathbf{H}}
\def\bI{\mathbf{I}}
\def\bJ{\mathbf{J}}
\def\bK{\mathbf{K}}
\def\bL{\mathbf{L}}
\def\bM{\mathbf{M}}
\def\bN{\mathbf{N}}
\def\bO{\mathbf{O}}
\def\bP{\mathbf{P}}
\def\bQ{\mathbf{Q}}
\def\bR{\mathbf{R}}
\def\bS{\mathbf{S}}
\def\bT{\mathbf{T}}
\def\bU{\mathbf{U}}
\def\bV{\mathbf{V}}
\def\bW{\mathbf{W}}
\def\bX{\mathbf{X}}
\def\bY{\mathbf{Y}}
\def\bZ{\mathbf{Z}}




\makeatletter
\def\cleardoublepage{\clearpage\if@twoside \ifodd\c@page\else
    \thispagestyle{empty}\hbox{}\newpage\if@twocolumn\hbox{}\newpage\fi\fi\fi}

\renewcommand\tableofcontents{%
    \if@twocolumn
      \@restonecoltrue\onecolumn
    \else
      \@restonecolfalse
    \fi
    \chapter*{\contentsname
        \@mkboth{%
           \contentsname}{\contentsname}}%
    \@starttoc{toc}%
    \if@restonecol\twocolumn\fi
    }
\makeatother

\marginparsep=10pt
\marginparwidth=18pt

\renewcommand{\thefigure}{\arabic{section}.\arabic{figure}}

\renewcommand\chaptermark[1]{\markboth{\thechapter. #1}{}}
\renewcommand\sectionmark[1]{\markright{\thesection. #1}}


\newtheoremstyle{alph}{}{}{\itshape}{}{\bfseries}{\kern -2pt{\bf:}}{
      }{#1 \mdseries{\bf #2}
{\bf #3}}
\theoremstyle{alph}
\newtheorem{alphremark}{Remark}
\renewcommand{\thealphremark}{\Alph{chapter}.\arabic{alphremark}}

\newtheorem{alphlemma}{Lemma}
\renewcommand{\thealphlemma}{A\arabic{chapter}.\arabic{alphlemma}}

\newtheorem{alphtheorem}{Theorem}
\renewcommand{\thealphtheorem}{A\arabic{chapter}.\arabic{alphtheorem}}

\newtheorem{alphcorollary}{Corollary}
\renewcommand{\thealphcorollary}{A\arabic{chapter}.\arabic{alphcorollary}}

\newtheorem{alphrem}{Remark}
\renewcommand{\thealphrem}{A\arabic{chapter}.\arabic{alphrem}}




