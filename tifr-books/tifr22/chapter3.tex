
\chapter{Transmutation in the Irregular Case}\label{part1:chap3} %chap 3
 
 \textbf{ Introduction.} Our\pageoriginale aim in this chapter is to obtain a
 transmutation operator for a differential operator with regular
 coefficients. In order to reduce the mixed problem relative to the
 operator  
 $$
 \Lambda + \frac{\partial ^2 }{\partial t^2} + \frac{2  p + 1}{t}
 \frac{\partial} {\partial t}, p ~\text{real or complex},  
 $$
 where $\Lambda =-  \Delta ( \Delta$ being the Laplacian in the
 variables $x_1,  \ldots,  x_n$) to the mixed problem relative to the
 operator  
 $$
 \Lambda + \frac{\partial ^2}{\partial t^2} 
 $$
 we shall construct an operator will transmute the operator 
 $$
 L_p = D^2 + \frac{2 p + 1}{x}D
 $$
 into the operator $D^2$. The difficulty in this case arises due to
 the presence of the coefficient $\dfrac{1}{x}$ which has a singularity
 at $x = 0$. If we seek the solution of the problem in $x \geq a$,
 where $a > 0$,  the method of the preceding chapter is perfectly
 valid without any change. But the important case is precisely the one
 in which $a = 0$.   
 
 We\pageoriginale shall determine isomorphisms $B_p,  \mathscr{B}_p$(of certain space
 which will be precisely specifies in the sequel ) which satisfy  
 $$
 D^2 B_p = B_p L_p ; ~\mathscr{B} _p D^2 = L_p \mathscr{B}_p. 
 $$
 The operator $B_p$ for $-1 <  \Rep< - \dfrac{1}{2}$ and the operator
 $\mathscr{B}_p$ for $\Rep > - \dfrac{1}{2}$ are
 classical. $\mathscr{B}_p$ is the Poisson's operator and $B_p$ is the
 derivative of the Sonine operator.  
 
 \section{The operator $\mathscr{B}_p$ for $\Rep> -
   \dfrac{1}{2}$}\label{part1:chap3:sec1}
 
\markright{\thesection. The operator $\mathscr{B}_p$ for $\Rep> - 1/2$}

 It can be forseen that the operator $\mathscr{B}_p$  is defined in
 terms of the solution for $y > 0$ of the partial differential
 equation  
 $$
 \Phi _{xx} - \Phi_{yy} + \frac{2 p + 1}{x} \Phi_x = 0
 $$
 with the conditions $\Phi (0, y) = g^* (y)$ being an even function
 $g^* (y) =g(y)$ for $ y > 0$ and $g^* (y) = g (-y)$ for $y < 0$, and
 $\Phi _x (0, y) = 0$ Now we define $\mathscr{B}_p [ g(x)] = \Phi (x,
 0)$.  
 
 Changing the variables $(x, y)$ to the variables $(s, t)$ be means of
 the formulae  $s \sqrt{2} = y + x$ and $t \sqrt{2} = y - x$, we see
 by simple computation that $u ( s, t  ) = \Phi \left(\dfrac{s - t}{2},
 \dfrac{s + t}{2}\right)$ is a solution of the partial differential equation  
 $$
 u_{st} - \frac{p + \frac{1}{2} }{s - t}u_s + \frac{p + \frac{1}{2}}{s
   - t }  u_t = 0 
 $$
 with\pageoriginale the conditions $u(s, s) = g^* (s \sqrt{2})$
 $$
 (u_s - u_t ) _{ s = t} = 0 
 $$\eject
 For $\alpha = p + \dfrac{1}{2}$, Poisson's solution has the form 
 $$
 u (s, t ) = \frac{\Gamma (2 \alpha)}{ [ \Gamma (\alpha)]^2} \int^1_0
 g^* \left\{ [ s + (t - s )\rho ]\sqrt{2}\right\} (1 - \rho)^{\alpha -
   1 }\rho ^{ \alpha - 1} d \rho  
 $$
 valid for Re$ \alpha > 0 ~(\text{or}~ \Rep> - \dfrac{1}{2})$.
Then 
\begin{multline*}
     \mathscr{B} [ g(x)] = \Phi (x, 0)  = u
   (\frac{x}{\sqrt{2}}, \frac{- x}{\sqrt{2}})\\
   = \frac{\Gamma (2 \alpha )}{[ \Gamma(\alpha )]^2} \int^1_0 g^* [ x
     (1 - 2 \rho )] (1 - \rho )^{ \alpha - 1} \rho^{\alpha - 1 } d
   \rho  
\end{multline*}
 setting $1 - 2 \rho = r $, 
 \begin{align*}
   \mathscr{B}_p [ g (x) ] &= \frac{\Gamma (2 \alpha )}{[ \Gamma
       (\alpha) ]^2 }  \frac{1}{2^{ 2 \alpha - 2 }} \int_0^1 (1 - r^
   2)^{\alpha - 1} g^* (rx) dr \\ 
   & = \frac{2 \Gamma ( p + 1)}{\sqrt{\pi} \Gamma (p +
     \frac{1}{2})}\int_0^1 (1 - t^2 )^{ p - \frac{1}{2}} g (tx)
   dt. \\ 
   (\text{ Using } \Gamma (2 \alpha ) & = \left.\frac{\Gamma (\alpha )\Gamma (
     \alpha +\frac{1}{2})2^{ 2 \alpha - 1 }}{\sqrt{\pi}} \right). 
 \end{align*}

Note that $\mathscr{B} _ p [ g (0)] = 1$. 

\setcounter{proposition}{0}
\begin{proposition}\label{part1:chap3:sec1:prop1}%proposition 1
  The\pageoriginale mapping $f \to \mathscr{B}_p f$ is a linear continuous map of
  $F^2 = \left\{ f / f \in \mathscr{E}^2 ( x \geq 0),  f ' (0) = 0
  \right\}$ into itself and satisfies  
  $$
  \mathscr{B} _p D^2 f = L_p \mathscr{B} _p f  \text{ for any  } f \in F^2. 
  $$
  As for $\Rep> - \frac{1}{2}$, differentiation under sign of
  integration is permissible, $\mathscr{B}_p f \in \mathscr{E}^2 (x
  \geq 0)$. Further $\left\{ \dfrac{d}{dx} \mathscr{B}_p [f]\right\}_{
    x = 0} = f' (0)\, k$ ($k$ being some constant),  
  $$
  = 0. 
  $$
  Hence $\mathscr{B} _p f \in F^2 $. Evidently $\mathscr{B}_p$ is
  linear.  In order to prove continuity, since $F^2$ has a metrizable
  topology induced by that of $\mathscr{E}^2 (x\geq 0) $, it is
  sufficient to show that if a sequence $\{f_n \}, n = 1, 2, \ldots,
  f_n \in F^2$. converges to $0$ in $F^2$, then $\mathscr{B}_p f_n$
  converges to zero in $F^2$. But $f_n \to 0$  in $F^2$ implies $f_n
  \to 0$ uniformly on each compact set, in particular on the compact
  set $[ 0, 1]$ from which it follows that  $\mathscr{B}_p f _n \to
  0$. It remains to verify that $\mathscr{B}_p$ satisfies the given
  condition. Writing $\beta _p = \frac{2 \Gamma ( p +1) }{\sqrt{\pi}
    \Gamma \left(p + 1/2 \right)} $ 
  \begin{align*}
  \frac{1}{\beta_p} & \left \{ L_p \mathscr{B} _p f - \mathscr{B} _p D^2
  f \right \} = \frac{1}{\beta _p} \int_0^1 \bigg \{ t^2 (1 - t ^2 )^{
    p - \frac{1}{2}} f'' (tx) \\
    & \qquad + \frac{(2 p + 1)t}{x} (1 - t^2 )^{p - \frac{1}{2}} f' (tx) - (1
    - t^2) ^{p -\frac{1}{2}} f'' (tx)\big\}  dt \\ 
    & = \frac{2 p + 1}{x} \int^1 _0 t (1 - t^2) ^{ p - \frac{1}{2}} f'
    (tx) dt - \int_0^1 (1 - t^2 )^{p + \frac{1}{2}} f'' (tx) dt.\\ 
    & = \frac{2 p + 1}{x} \left\{- \frac{1}{2} \left[ \frac{(1 -
        t^2)^{ p + \frac{1}{2}}}{p + \frac{1}{2}} f' (tx) \right]^1_0
    + \frac{1}{2} \int^1_0 \frac{(1-t^2)^{p + \frac{1}{2}}}  x f''
    (tx)dt \right\}\\  
    & \quad - \int^1_0 (1 - t^2 )^{ p + \frac{1}{2}} f'' (tx)
    dt  ~\text{(integrating the first integral by parts)}.\\
    & = 0.
  \end{align*}
\end{proposition} 

\begin{remark*}
  Let\pageoriginale $J_p (x) $ denote the classical Bessel function and let 
  $$
  j_p (x) = 2^p \Gamma (p + 1) x^{-p} J_p (x) 
  $$
  $j_p (x) $ is in $F_2$ and is the unique solution of the
  differential equation $\dfrac{d^2 y}{dx^2} + \dfrac{2 p + 1}{x}
  \dfrac{dy}{dx} + s^2 y = 0 $ with the  conditions $(y)_{x=0} = 1$\pageoriginale
  and $\left(\dfrac{dy}{dx}\right)_{ x = 0} = 0$. Now $\cos sx \in F^2$. Let
  $\mathscr{B}_p (\cos sx ) =  g(x)$. Using $L_p \mathscr{B}_p =
  \mathscr{B}_p D^2$, we get  
  \begin{align*}
    L_p \mathscr{B} _p [ \cos sx ] & = L_p [ g(x)] = \mathscr{B}_p D^2
    (\cos sx) \hspace{2cm}\\ 
    & = -s^2 \mathscr{B}_p [ \cos sx ]  = - s^2 g(x).\\
    \text{ i.e.,} \hspace{2cm} L_p g + s^2 g & = 0.  
  \end{align*}
\end{remark*}

Further $g(0) = \cos 0 = 1$ and $g' (0) = 0$ since $g \in F^2 $. This
shows that  
$$
\mathscr{B}_p (\cos sx) = g (x) j_p (sx).  
$$
Hence we obtain the classical formula, 
$$
j_p (sx) = \frac{ 2 \Gamma (p  + 1)}{\sqrt{ \pi }\Gamma (p +
  \frac{1}{2})} \int_0^1 (1 -t^2)^{ p - \frac{1}{2}}\cos (stx) dt.  
$$
The operator $\mathscr{B}_p$ was considered by Poisson in this
particular question of the transformation of the consine into the
function $j_p$. \textit{The operator $B_p$ for $-1< \Rep< -
  \dfrac{1}{2}$} 

For $f \in \mathscr{E}^ 0 (x \geq 0)$, the definition of $B_p$ is 
\begin{align*}
  B_p [ f(x) ] & = b_p x \int^x _0 (x^2 - y^2 )^{\frac{-p-3}{2}}\cdot y^{2p + 1} f(y)
  dy\\
  &  = b_p \int_0^1 t^{ 2p+1} (1 - t^2)^{-p-3/2} f (tx) dt \\ 
  \text{ where } \qquad \qquad   1 / b_p & = \frac{1}{2\sqrt{\pi}}
  \Gamma (p + 1) \Gamma \left(- p - \frac{1}{2}\right) 
 \end{align*} 
 
\begin{proposition}\label{part1:chap3:sec1:prop2} % propositon 2
  For\pageoriginale $- 1 < \Rep< - \dfrac{1}{2}$ and $f \in F^2, f \to B_p f$ is a
  linear continuous map of $F^2$ into itself satisfying  
  $$
  D^2 B_p f = B_p L_p f. 
  $$
\end{proposition}
 
The proof of the fact that $B_p$ is a linear continuous map of $F^2$
into itself is analogous to the one we have given in Proposition
\ref{part1:chap3:sec1:prop1}. We verify that $B_p$ satisfies the given condition, again as in
Proposition \ref{part1:chap3:sec1:prop1}, by integration by parts  
\begin{align*}
& \frac{1}{b_p} \left\{ B_p L_p - D^2 B_p \right\} f(x) 
   = \int^1_0 t^{ 2 p + 1} (1 - t^2)^{- p -3 / 2}\\ 
& \hspace{4cm} \left\{ f'' (tx) +
  \frac{2 p + 1}{tx}  f' (tx) - t^2 f'' (tx) \right\}dt \\ 
  & = \int^1_0 t^{ 2 p + 1} (1 - t^2)^{-p- \frac{1}{2}}) f'' (tx) +
  \frac{2 p + 1}{x}\int^1_0 t^{2p} (1 - t^2)^{-p-3 /2}) f' (tx) dt\\ 
  & = \left[ t^{ 2 p + 1} (1 - t^2)^{ - p - \frac{1}{2}}f' (tx) 
    \right ]_0^1 - \int^1 _0 \frac{f' (tx)}{x} \frac{d}{dt}\\ 
  & \hspace{2cm}\left(\frac{t^{
      2 p + 1}}{(1 - t^2)^{p + \frac {1}{2}}}\right)dt  + \frac{2 p +1}{x}
  \int_0^1 t^{2p}  \frac{f' (tx) }{(1 - t^2)^{p+ 3/2}} dt =0.   
\end{align*}

\heading{The\pageoriginale Sonine operator $\bar{B}_p$ for $-1 < \Rep<
  \dfrac{1}{2}$}.  

$ \bar{B}_p $ is defined for every $ f \in \mathscr{E}^0 (x \geq 0) $by 

\begin{align*}
  \bar{B}_p f & = x \bar{b}_p \int^1_0 \frac{t^{ 2 p + 1}}{(1 - t^2)^{
      p + \frac{1}{2}}} f (tx)  dt   \\
  & = \bar{b}_p \int^x_0  y ^{ 2p + 1} (x^2 - y^2)^{- p- \frac{1}{2}}
  f (y) dy\hspace{2cm} \\ 
  \text{where}\hspace{2cm}\bar{b}_p & = \frac{\sqrt{\pi}}{\Gamma (p +
    1) \Gamma (- p + \frac{1}{2})}.  
\end{align*}
The integral converges if $-1< \Rep< \frac{1}{2}$ and we can
differentiate under the sign fo integration  
$$
\frac{d}{dx}\bar{B}_p [ (f (x) )]  = B_p [ f (x) ]\text{ for every } f
\in \mathscr{E}^0 (x \geq 0)  
$$ 

\heading{Relation between $\mathscr{B}_p$ and $\bar{B}_p$}
 
When $- \dfrac{1}{2} < \Rep< \dfrac{1}{2}$, both $\mathscr{B}$ and
$\bar{B}_p$ are defined and it is easy to prove by direct computation
Abel's functional equation  
$$
\bar{B}_p \mathscr{B}_p [ f(x)] = \int^x _0 f (y) dy. 
$$
In fact   
\begin{align*}
\bar{B}_p \mathscr{B}_p [ f (x) ] & = \bar{b}_p
  \int^x_0 y^{ 2 p +1}(x^2 - y^2)^{ - p - \frac{1}{2}}\beta
  _p y^{-2p}   \int^y_0 f(z) (y^2 - z^2)^{p - \frac{1}{2}} dz\\ 
  & = \bar{b}_p \beta_p \int^y_0 dy \int^y_0 y (x^2 - y^2 )^{ - p -
    \frac{1}{2}} f (z)dz \\ 
  & = \bar{b}_p \beta_p \int^x_0 [ f (z) \int^x_z (x^2 - y^2)^{ - p -
      \frac{1}{2}} (y^2 - z^2 )^{p-\frac{1}{2}} ydy ] dz  
\end{align*}
Setting\pageoriginale $x^2 \sin ^2 \theta + z^2 \cos^2 \theta = y^2 $,  we have 
\begin{align*}
x^2 - y^2 & = (x^2 - z^2 )\cos ^2 \theta,  y^2 - z^2 = (x^2 - z^2)
\sin^2 \theta\\ 
y dy & = 2(x^2 - z^2) \sin \theta \cos \theta d \theta
\end{align*}
\begin{align*}
  \int^x_z (x^2 - y^2)^{ - p - \frac{1}{2}} & (y^2 - z^2)^{ p  - \frac{1}{2}} ydy\\ 
  & = \int^{\frac{\pi}{2}} _ 0 \cos ^{ - 2 p} \theta \sin^{2p} \theta
  d \theta = \frac{1}{2} B \left(- p + \frac{1}{2}, p + \frac{1}{2}\right) \\ 
  & = \frac{1}{2} \frac{\Gamma \left(- p + \frac{1}{2}\right)\Gamma  \left(p +
    \frac{1}{2}\right)}{\Gamma (1)} 
\end{align*}
Hence $\bar{B}_p \mathscr{B}_p [ f (x) ] = \int^x _0 f (z) dz $ so
that $D \bar{B}_p \mathscr{B}_p [ f(x)]  = f(x)$.  

\section{Continuation of the operator $B_p$}\label{part1:chap3:sec2}

For any $f \in \mathscr{E}(x \geq 0) = \mathscr{E} ^\infty (x \geq
0)$, we define  
\begin{align*}
  T_1 [ f (t, x ) ] & = D_t [ t f (tx)], \\
  T_2 [ f(t, x )] & = D_t [ t^3 T_1 \{ f (t, x)\} ]. \\
  \text{ In general } \hspace{2cm} T_n [ f (t, x ) ] & = D_t \{ t^3
  T_{ n -1} [ f 
    (t, x )] \}.\hspace{2cm} 
\end{align*}

\begin{lem}\label{part1:chap3:sec2:lem1}% lemma 1
  $T_n [ f (t, x )] = t^{ 2n - 2} g_n (t, x)$\pageoriginale where $g_n (t, x )$ is
  an indefinitely differentiable function in $[ 0, 1] \times [ 0,
    \infty ) $.  
\end{lem}

The proof of the lemma is trivial and is based on indication on $n$. 

For $n = 1$, we have only to set $g_1 (t, x ) = D_t[ tf (tx)]
$. Assume that the lemma is true for $n-1$ so that  
$$
T_{ n -1} [ f (t, x)] = t^{ 2n - 4} g_{ n - 1} (t, x ). 
$$
Define 
$$
g_n (t, x ) = 3g _{ n-1} (t, x) + (2n - 4) g_{(n -1)} (t, x) + t D_t
\, g_{ n - 1} (t, x).  
$$
By definition 
\begin{align*}
  & T_n  [ f (t, x) ] = 3 t^2 T_{ n -1} [ f (t, x)]  + t^3 D_t T_{ n
    -1} [ f (t, x)] \\ 
  & = 3t^2 t^{ 2n - 4} g_{ n - 1} (t, x ) + t^3 \left \{ (2n - 4) t
  ^{2n -5} g_{ n -1} (t, x) + t^{ 2n - 4 }D_t\, g_{ n-1} (t, x)\right
  \}.\\ 
  & = t^{ 2n - 2} g_n (t, x). 
\end{align*}

\begin{coro*}
  The integral $\int^1_0 t^{2p - (2 n -3)} (1 - t^2)^{-p + \left(\dfrac{2n -
      3}{2}\right)} T_n f (x, t) dt $ converges for $- 1 < \Rep< n -
  \dfrac{1}{2}$.  
\end{coro*}

The\pageoriginale corollary is immediate since the integral can be written as

\noindent $\int^1 _0 t^{ 2 p +1} (1 - t^2)^{-p + \dfrac{2 n -3}{2}} g_n (t, x)$.  

\setcounter{proposition}{0}
\begin{proposition}\label{part1:chap3:sec2:prop1} %proposition 1
  For $-1 < \Rep< - \dfrac{1}{2}$, and for $f \in \mathscr{E} (x \geq 0)$, 
  \begin{multline*}
  B_p [ f(x) ] = \frac{(-1)^n b_p}{(2 p + 1) (2 p - 1)\cdots (2 p -
    \overline{2n - 3})}\\
  \int^1_0 \frac{t^{2 p - (2n -3)}}{(1 - t^2)^{p -
      \frac{2 n -3}{2}}} T_n f (t, x) dt 
  \end{multline*}
  The proof is based on induction on $n$ and the following formula
  which is obvious:  
\begin{align*}
  \frac{d}{dt} \left [ \frac{t^{2 p - \lambda}}{(1 - t^2)^{p
        -\frac{\lambda}{2}}}\right] & =  \frac{(2 p - \lambda) t^{2 p
      - \lambda - 1}} 
       {(1 - t^2)^{p - \lambda / 2 + 1}}\tag{1}\label{part1:chap3:sec2:eq1}\\
       \text{\rm for any } \qquad \lambda B_p [ f(x) ] & = b_p \int^1
       _0 t^{ 2 p + 1} (1 - t^2)^{ - p -3/2} f (tx) dt\hspace{1cm}   
\end{align*}
\end{proposition}

Let \qquad $n = 1$. If $p  + \dfrac{3}{2} = p - \dfrac{\lambda}{2}+1$,
\quad i.e.,\quad  $\lambda = - 1$ 
$$
\frac{t^{2 p}}{(1 - t^2)^{p + 3 /2}} = \frac{1}{(2 p +1)} \frac{d}{dt}
\left(\frac{t^{ 2 p +1}}{(1 - t^2)^{ p + \frac{1}{2}}}\right) 
$$
so that 
\begin{align*}
  B_p [ f(x) ] & =  \frac{ b_p}{ 2 p + 1} \int^1_0 \frac{d }{dt} \left(\frac{
    t^{2 p + 1}}{ (1 - t^2)^{p + \frac{1}{2}}}\right) t f (tx)dt\\ 
  & = -\frac{b_p}{2p+1} \int^1_0 \frac{t^{2p+1}}{(1-t^2)^{p + \frac{1}{2}}}
  T_1 \big[ f(t, x) \big ] dt 
\end{align*}
(integrating\pageoriginale by parts, the integrated part being zero since $\Rep+
\frac{1}{2} < 0$ and $2\Rep + ~ 2 > 0$). Thus the formula to be
proved holds for  $n=1$. Assuming it for $n-1$, we establish it for
$n$ 
\begin{multline*}
  B_p \big[ f(x) \big] = \frac{(-1)^{n-1}b_p}{(2p + 1)(2p - 1) \ldots
  (2p-2n+5)}\\ 
  \int^1_0 \frac{t^{2p-(2n-5)}}{(1-t^2)^{p-\frac{2n-5}{2}}}
  T_{n-1} [f(t, x)] dt  
\end{multline*}

Using (\ref{part1:chap1:sec1:eq1}) with $p-\dfrac{2n-5}{2} = p- \dfrac{\lambda}{2} + 1$
i.e., ~ $\lambda = 2n-3$, the integral on the right hand side equals 
\begin{align*}
  \frac{1}{2p-2n+3} & \int^1_0 \frac{d}{dt} \left[
    \frac{t^{2p-2n+3}}{(1-t^2)^{p- \frac{2n-3}{2}}} \right] t^3
  T_{n-1} [f(t, x)] dt \\ 
  &= \frac{1}{2p-2n+3} \left\{ \left[ \frac{t^{2p+2}}{(1-t^2)^{p-
        \frac{2n-3}{2}}} g_{n-1} (t, x) \right]^1_0\right.\\ 
  & \hspace{3cm}\left. -\int^1_0
  \frac{t^{2p-2n+3}}{(1-t^2)^{p- \frac{2n-3}{2}}}  T_n \left[ f(t, x)
    \right] dt \right \} \\ 
  &= \frac{1}{2p-2n+3} \int^1_0 \frac{t^{2p-2n+3}}{(1-t^2)^{p-
      \frac{2n-3}{2}}} T_n  \big[ f(t, x) \big] dt, 
\end{align*}
the\pageoriginale integrated part being zero since $-1 < \Rep < - \dfrac{1}{2}$. We write 
\begin{align*}
  B^n_p \big[ f(x)\big] & = b^{(n)}_p \int^1_0
  \frac{t^{2p-(2n-3)}}{(1-t^2)^{p- \frac{2n-3}{2}}} T_n  \big[ f(t, x)
    \big] dt \tag{2}\label{part1:chap3:sec2:eq2} \\
  \text{where}\quad  
  b^{(n)}_p & = (-1)^n  \frac{b_p}{(2p+1) (2p-1) (2p-3) \cdots
    (2p-2n+3)} \tag{3}\label{part1:chap3:sec2:eq3}
\end{align*}

The integral is convergent for $-1 < \Rep < n - \dfrac{1}{2}$ so that
under this condition, we can differentiate under the sign of
integration and  $B^n_p f \in \mathscr{E}$. We obtain for each $n$ a
function which assigns to each $p$ in $-1 < \Rep <
n-\dfrac{1}{2}$, a map  $B^n_p$ of  $\mathscr{E}$ into itself which
coincides with $B_p$ if  $-1 < \Rep < -\dfrac{1}{2}$. It is
easy to see that $B^n_p$ is a linear map of $\mathscr{E}$ into
itself. In order to show that it is  continuous, as $\mathscr{E}$ is
metrizable, it is enough to prove that if a sequence $\big\{ f_j
\big\}_{j = 1, 2, \ldots}$ tends to zero in $\mathscr{E}$, then $B^n_p
f_j$ tends to $0$ in $\mathscr{E}$. We have 
$$
\displaylines{\hfill 
  \left| D^r_x B^n_p \big[ f_j (x) \big] \right| \leq b^{(n)}_p M(r, x,
  j) \int^1_0 \frac{t^{2p-(2n-3)}}{(1-t^2)^{p-\frac{2n-3}{2}}} dt
  \hfill \cr
  \text{where} \hfill M(r, x, j) = \sup\limits_{1 \leq t \leq 1} \left| D^r_x
  T_n \left[ f_j (t,x) \right] \right|  \hfill }
$$

Now\pageoriginale $T_n \big[ f_f (t, x) \big]$ is a polynomial in $t, x$ with
coefficients which are derivatives of order $\leq n$ of $f_j$ and
$f_j$ together with all its derivatives converge to zero on each
compact subset. Hence $M(r, x, j) \to 0$ as $j \to \infty$ uniformly
for $x$ on each compact subset $i. e$. $B^n_p f_j \to 0$ in
$\mathscr{E}$. Thus we obtain a function $p \to B^n_p$ on  $-1  < \Rep
< - \dfrac{1}{2}$ with values in $\mathscr{L} (\mathscr{E},
\mathscr{E})$, the  space of linear continuous maps of $\mathscr{E}$
into itself. We intend to prove that this function is analytic and can
be continued in the whole complex plane into a function which ia
analytic  in the half plane $\Rep > -1$ and meromorphic in $\Rep  <
-1$ with a  sequence of  poles lying on the real axis. Before proving
this continuation theorem we give first the definition of a vector
valued analytic function and some of its properties which follow
immediately from the definition. 

\begin{defi*}
  Let 0 be an open subset of the complex plane and $E$ a locally
  convex vector space. A function $f : 0 \to E$ is called analytic if
  for every $e'$ in the topological dual $E'$ of $E$ $( i. e$. the
  space of linear continuous forms on $E )$ the function $z \to < f
  (z), e' > $ is analytic in $0$ where $<,  >$ denotes the scalar
  product between $E$ and $E'$.  
\end{defi*}

\begin{lem}\label{part1:chap3:sec2:lem2}%lemma 2.
  If $E$ is  locally convex vector space in which closed convex
  envelope of a compact set is compact, a function $f : 0 \to E$ is
  analytic if $f$ is continuous and for every $e'$  in a total set
  $M'$ of $E'$, $z \to < f (z), e' > $ is analytic in $0$. 
\end{lem}

Let\pageoriginale $C$ be any simple closed curve lying entirely in $0$, enclosing
region ( open connected set ) contained in $0$. For $e' \in M'$, the
function $z \to < f (z), e' > $ is analytic in $0$, so that $\int_c <
f (z), e' > dz = 0$ i. e. $< \int_c f (z) dz$, $e' > = 0$. $\int_c
f(z) dz$ is the integral of the continuous $E$-valued function $f$
over the compact set $C$ and is an element of $E$ since $E$ has the
property that the closed convex envelope of any compact subset is
compact. ( For integration of a vector valued function, refer to
$N$. Bourbaki, Elements de Mathematique, Integration, Chapter
III). Hence $\int_c f (z)dz = 0$ since $M'$ is total, so that $ <
\int_c f (z) dz, ~ e' > = \int_c < f (z), e' > dz = 0$ for every $e'
\in E'$. Also as  $f$ is continuous it follows that $z \to < f (z), ~
e' >$ is continuous. This proves that $z \to < f (z), e' > $ is
analytic for every $e' \in E'$ since the choice of $C$ was arbitrary. 

\begin{proposition}\label{part1:chap3:sec2:prop2}%proposi 2.
  Suppose that $-1 < \Rep < n -\dfrac{1}{2}$. Then 
  \begin{enumerate}[\rm a)]
  \item $B^n_p f \in \mathscr{E} (x \geq 0)$ for every $f \in
    \mathscr{E} (x \geq 0)$ 
  \item The mapping $f \to B^n_p f$ is linear continuous of
    $\mathscr{E}$ into itself. 
  \item The function $p \to B^n_p$ on the strip $-1 < \Rep < n -
    \dfrac{1}{2}$ with values in $\mathscr{L}_s (\mathscr{E},
    \mathscr{E} )$ is analytic  where $\mathscr{L}_s ( \mathscr{E},
    \mathscr{E} )$ is the  space of linear  continuous maps of
    $\mathscr{E}$ into $\mathscr{E}$ endowed with the topology of
    simple convergence. 
  \end{enumerate}
\end{proposition}

We have to prove only $(c)$. We first observe that any linear
continuous form on $\mathscr{L}_s (E, F)$ is given by finite linear
combination\pageoriginale of forms of the type $u \to < ue, f' >$ with $e \in E$ and
$f' \in F'$. The theorem will be proved if we show that the map $p\to
< B^n_pf, T >$  is analytic in $-1 < \Rep < n -\dfrac{1}{2}$ where $f$
is any element of $\mathscr{E}$ and $T$ any element of $ \mathscr{E}'$;
i.e. for  fixed $f$ we have to show  that the map $p \to ~ B^n_pf$
is  analytic with values in $\mathscr{E}$. Now $\mathscr{E}$ ia a
Hausdorff complete locally  convex vector space and therefore closed
convex envelope of each  compact subset of $\mathscr{E}$ is compact (
refer to $N$. Bourbaki, Espaces Vectoriels Topologiques, Ch.II, \S
4, Prop.2). Further it is easy to see that $p \to~ B^n_pf$ is
continuous and that the set  $\big\{ \delta_x \big\}_{x \geq 0}$,
where $\delta_x$ is the Dirac measure with support at $x$, is total in
$\mathscr{E}'$. Applying the lemma, we see that in order to prove
analyticity of the function $p \to~ B^n_pf$ we have only to prove that
the function $p \to < B^n_pf, \delta_x >$ i.e. the function  $p \to ~
B^n_p f (x)$ (for $f$ and $x$ fixed ) is analytic in $-1 < \Rep < n
-\dfrac{1}{2}$. 

Observing that $b_p = \dfrac{2 \sqrt{\pi}}{\Gamma (p+1) \Gamma
  (-p-\dfrac{1}{2})}$ is an entire function with zeros at $ p = -1,
-2, -3, \ldots$ and $p = -\dfrac{1}{2}, \dfrac{1}{2}, \dfrac{3}{2},
\ldots $ we see that $b^n_p$ in (\ref{part1:chap3:sec2:eq3}) is an entire function. The
integral in (\ref{part1:chap3:sec2:eq2}) on the other hand converges for $-1 < \Rep < n -
\dfrac{1}{2}$ and therefore is  analytic in the same region, Hence $p
\to ~ B^n_p f (x)$ is analytic  in the strip $-1 < \Rep < n-
\dfrac{1}{2}$. 

\begin{coro*}
  The functions $B^m_p$ and $B^n_p$ where $m$ and $n$ are two
  distinct\pageoriginale positive integers are identical in the intersection of
  their domains of definition. 
\end{coro*}

We have in fact two analytic functions $B^m_p$ and $B^n_p$ which
coincide with $B_p$ in $-1 < \Rep < -\dfrac{1}{2}$ which is common to
their domains of definitions and therefore the two functions coincide
everywhere in the domain  which is the  intersection of  their domains
of definition due to analyticity. It follows from  the corollary that
we have  a unique analytic function $B_p$ defined for $\Rep > -1$.  

\begin{remark*}
  We have $B_{-1/2}=$ identity.
\end{remark*}

For $n=1$,
$$
B_p (f) = \frac{-b_p}{2p+1} \int^1_0 
\frac{t^{2p+1}}{(1-t^2)^{p+\frac{1}{2}}} \frac{d}{dt} \big[ tf (tx)
\big]dt 
$$
and if $p = -\dfrac{1}{2}$, $\dfrac{-b_p}{2p+1} = 1 $ and 
$$
B_{-\frac{1}{2}} \big[ f ~ (x) \big] = \int^1_0 \frac{d}{dt} \big[ t ~
  f(tx) \big] dt = f (x).  
$$

\heading{Continuation of $B_p$ for $\Rep < -\dfrac{1}{2}$.}

We define
\begin{align*}
  U_1 \big[ f (t, x) \big]  &= D_t \big[ (1-t^2)^{\frac{1}{2}} f (tx) \big]. \\
  U_2 \big[ f (t, x) \big] &= D_t \big[ (1-t^2)^{3/2} ~ f(tx) \big]. 
\end{align*}

In\pageoriginale general \hspace{.5cm} $ U_n \big[ f(t, x) \big] = D_t \big[
  (1-t^2)^{3/2} U_{n-1} f(tx) \big]$. 

\begin{lem}\label{part1:chap3:sec2:lem3}%lemma 3.
  For every $f$ in $\mathcal{E} (x \geq 0 ) = \mathscr{E}$ we have 
\noindent 
\begin{align*}
    & U_n \big[ f (t, x) \big]  = (1-t^2)^{3/2} D_t U_{n-1} f (t, x) -3t
    (1-t^2)^{\frac{1}{2}} U_{n-1} \big[ f (t, x) \big]
    \tag{4}\label{part1:chap3:sec2:eq4} \\
    &\text{\rm and }\quad  
    U_n \big[ f (t, x) \big]  = (1-t^2)^{\frac{n-2}{2}} h_n (t, x)
    \tag{5} \label{part1:chap3:sec2:eq5}
\end{align*}
  where $ h_n (t, x)$ is indefinitely differentiable in $\big[ 0, 1
    \big] \times \big[ 0, \infty \big)$. 
\end{lem}

Relation (\ref{part1:chap3:sec2:eq4}) is
evident. (\ref{part1:chap3:sec2:eq5}) can be proved by induction on $n$. It
is true for $n=1$ if we set  
$$
h_1 (t, x) = -t f (tx) + x (1-t^2) f' (t, x).
$$

Assuming it for $n-1$, it is easy to verify that
(\ref{part1:chap3:sec2:eq5}) holds for $n$ if 
$$
h_n (t, x) = -nth_{n-1} (t, x) + (1-t^2) D_t h_{n-1} (t, x) 
$$
\begin{coro*}
  The integral $\int^1_0 \frac{t^{2p+n+1}}{(1-t^2)^{p+
      \frac{n+1}{2}}} ~ U_n \big[ f (x, t) \big] dt$ converges for
  $-1-\frac{n}{2} < \Rep < - \frac{1}{2}$.  
\end{coro*}

\begin{proposition}\label{part1:chap3:sec2:prop3}%proposi 3.
  If $ -1 < \Rep < - \dfrac{1}{2}$,
  \begin{multline*}
    B_p \big[ f (x) \big] = \frac{(-1)^n b_p}{(2p+2) (2p+3) \cdots
      (2p+n+1)}\\ 
    \int^1_0 \frac{t^{2p+n+1}}{(1-t^2)^{p+\frac{n+1}{2}}}
    U_n \big[ f (t, x) \big] dt 
  \end{multline*}
  The\pageoriginale proof of this proposition is analogous to that of
  Proposition \ref{part1:chap3:sec2:prop1}. 
\end{proposition}

We prove it by induction on $n$ and by using formula
(\ref{part1:chap3:sec2:eq1}).

For $-1 < \Rep < -\dfrac{1}{2}$,
$$
B_p \big[ f(x) \big] = b_p ~ \int^1_0 t^{2p+1}
(1-t^2)^{-p-\frac{3}{2}} f (tx) dt 
$$
Using\pageoriginale (\ref{part1:chap3:sec2:eq1}) with $2p+1 = 2p- \lambda - 1$ i. e. $\lambda = -2 $, we have,
$$
\frac{t^{2p+1}}{(1-t^2)^{p+2}} = \frac{1}{2p+2} \frac{d}{dt} (
\frac{t^{2p+2}}{(1-t^2)^{p+1}}) 
$$
so that
\begin{align*}
  B_p \big[ f ~ (x) \big] &= \frac{b_p}{2p+2} \int^1_0 \frac{d}{dt} (
  \frac{t^{2p+2}}{(1-t^2)^{p+1}}) (1-t^2)^{\frac{1}{2}} f(tx) dt \\ 
  &= \frac{-b_p}{(2p+2)} \int^1_0 \frac{t^{2p+2}}{(1-t^2)^{p+1}} U_1
  \big[ f (t, x) \big] dt,\\ 
\end{align*}
the integrated part being zero since $-1 < \Rep < \dfrac{1}{2}$. The
formula is proved for $n=1$. We assume it for $n-1$ 
\begin{multline*}
  B_p \big[ f (x) \big] = \frac{(-1)^{n-1} b_p}{(2p+2) (2p+3) \cdots
  (2p+n)}\\ 
  \int^1_0 \frac{t^{2p+n}}{(1-t^2)^{p+n/2}} U_{n-1} [f(t, x)] dt 
\end{multline*}
Using (\ref{part1:chap3:sec2:eq1}) with $2p+n = 2p-\lambda -1$ i. e. $\lambda = -(n+1)$, we get
$$
\frac{d}{dt} \left( \frac{t^{2p+n+1}}{(1-t^2)^{p+\frac{n+1}{2}}}\right) =
(2p+n+1) \frac{t^{2p+n+2}}{(1-t^2)^{p+\frac{n+3}{2}}}, 
$$
so that  
\begin{multline*}
  B_p \big[ f (x) \big] = \frac{-(-1)^{n-1} b_p}{(2p+2)\cdots (2p+n+1)}
  \int^1_0 \frac{t^{2p+n+1}}{(1-t^2)^{p+\frac{n+1}{2}}}\\ 
  D_t \left\{
  (1-t^2)^{\frac{3}{2}} U_{n-2} f (t, x) \right\} dt 
\end{multline*}
we shall now set 
\begin{align}
  n_{B_{p}} \big[ f (x) \big] &= (n)_{b_{p}}
  \frac{t^{2p+n+1}}{(1-t^2)^{p+\frac{n+1}{2}}} U_n \big[ f (t, x)
    \big] dt  \tag{6}\label{part1:chap3:sec2:eq6}\\ 
  (n)_{b_{p}} &= \frac{(-1)^n b_p}{(2p+2) (2p+3)\cdots (2p+n+1)}
  \tag{7}\label{part1:chap3:sec2:eq7}  
\end{align} 
$(n)_{b_{p}}$ is a meromorphic function of $p$, with poles at the
points\break  
$p = \dfrac{-3}{2}, ~ \dfrac{-5}{2}, \ldots$.

\begin{proposition}\label{part1:chap3:sec2:prop4}%proposi 4.
Suppose that $p$ satisfies
\begin{equation}
  -1 - \frac{n}{2} < \Rep < - \frac{1}{2} \tag{8}\label{part1:chap3:sec2:eq8}
\end{equation}
and does \textit{not} assume any of the values 
\begin{equation}
  -\frac{3}{2}, ~ \frac{-5}{2}, ~ \ldots \tag{9}\label{part1:chap3:sec2:eq9} 
\end{equation}
\end{proposition}

Then\pageoriginale 
\begin{enumerate}[a)]
\item ${}^nB_{p}f \in \mathscr{E}$ for  each  $f \in \mathscr{E}$
\item The mapping $f \to {}^nB_{p}f$ is linear continuous from
  $\mathscr{E}$ into $\mathscr{E}$. 
\item The function $p \to {}^n B_{p}$ is meromorphic in the strip
  defined by (\ref{part1:chap3:sec2:eq8}) with poles situated at the
  points  given by (\ref{part1:chap3:sec2:eq9}).   
\end{enumerate}

We omit the proof of $a)$ and $b)$  since it is exactly similar to
that of Proposition \ref{part1:chap3:sec2:prop2}. The proof of $c)$
reduces as in Proposition \ref{part1:chap3:sec2:prop2}, to showing
that the function  
$$
p \overline{~~~~~~~}  ~ ^{(n)}b_{p}  \int^1_0
\frac{t^{2p+n+1}}{(1-t^2)^{p+\frac{n+1}{2}}} U_n \left[ f (t, x)
  \right] dt 
$$
is meromorphic in the strip (\ref{part1:chap3:sec2:eq8}) with poles at
the points (\ref{part1:chap3:sec2:eq9}), 
which is obvious since the integral converges in
(\ref{part1:chap3:sec2:eq8}) and is therefore
analytic in (\ref{part1:chap3:sec2:eq8}) and $n_{b_{p}}$ is
meromorphic in (\ref{part1:chap3:sec2:eq8}) with poles given by
(\ref{part1:chap3:sec2:eq9}).  

\begin{coro*}
  If $m, n$ are two distinct positive integers, the two functions
  $m_{B_{p}}$, $n_{B_{p}}$ coincide in the common part of their
  domains of definition. 
\end{coro*}

This corollary is immediate like the corollary of Proposition
\ref{part1:chap3:sec2:prop2}. Consequently we have a unique function
$B_{p}$ defined for $\Rep < - \dfrac{1}{2}$. 

Propositions\pageoriginale (\ref{part1:chap3:sec2:prop2}) and
(\ref{part1:chap3:sec2:prop4}) give finally the continuation theorem for
the operator $B_p$. 

\begin{theorem*}
  The function $p - B_p $ defined initially in $-1 < -\Rep <
  -\dfrac{1}{2}$ with values in $\mathscr{L} ( \mathscr{E},
  \mathscr{E})$ endowed with the  topology of simple convergence can
  be continued in the whole  plane into a meromorphic function. The
  poles of this function are situated at the points $\dfrac{-3}{2},
  \dfrac{-5}{2}, \dfrac{-7}{2},  \ldots $. 
\end{theorem*}

\begin{remark*}
  The notion of an analytic function with values in a locally convex
  vector space $E$ depends only on the system of bounded subsets of
  $E$. $\mathscr{E}$ being complete it is easy to see (in view of
  Theorem 1, page 21, Ch.III,  Espaces vectorieles  Topologiques
  by $N$. Bourbaki) that the space $\mathscr{L} (\mathscr{E},
  \mathscr{E})$ when furnished with the topology of simple
  convergence has the same system of bounded sets as when furnished
  with the topology of  uniform convergence on the system of bounded
  sets of $\mathscr{E}$. Hence in the theorem we can replace the
  topology of simple convergence by the topology of uniform
  convergence on bounded subsets of $\mathscr{E}$ or by any other
  locally convex topology which lies between these two topologies. 
\end{remark*}

Let $\mathscr{E}_*$ and $\mathscr{D}_\circ$ be subspaces of
$\mathscr{E}_{(x \geq 0)}$ defined by  
\begin{align*}
  \mathscr{E}_* &= \left\{ f \,| f \in \mathscr{E}(x \geq 0), ~ f^{2n+1}
  (0) = 0 \text{ for } n \geq 0 \right\} \\ 
  \mathscr{D}_\circ &= \left\{ f | f \in (x \geq 0), ~ f^n (0) = 0
  \text{ for } n \geq 0 \right \}  
\end{align*}
When\pageoriginale $-1 < \Rep < -\dfrac{1}{2}$, we have
$$
D^r B_p \big[ f (x) \big] = b_p \int^1_0 t^{2p+1+r}
{(1-t^2)}^{-p-\frac{3}{2}} f^r (tx) dt 
$$
so that 
\begin{equation}
  D^r B_p \big[ f (0) \big] = b_{p, r} f^r (0)
  \tag{10}\label{part1:chap3:sec2:eq10} 
\end{equation}
where
$$
b_{q, r}= \frac{\sqrt{\pi} \,\Gamma \left(p+ \dfrac{r}{2} +1\right)}{\Gamma
  \left(\frac{r+1}{2}\right) \Gamma (p+1)} 
$$

This shows that when $r$ is even, $p \to b_{p, r}$ is an entire
function of $p$ and when $r$ is odd it is a meromorphic function with
poles at $\frac{-3}{2}, ~ \frac{-5}{2},  \ldots$ 
(\ref{part1:chap3:sec2:eq10}) is therefore
true for all $p$ not equal to the exceptional values $\frac{-3}{2},
\frac{-5}{2},  \ldots $. Then $B_p [f]$ is in $\mathscr{E}_*$ or in
$\mathscr{D}_\circ$ according as $f$ is in $\mathscr{E}_*$ or
$\mathscr{D}_\circ$. Therefore for $p \neq \frac{-3}{2}, \frac{-5}{2},
\ldots,  B_p \in \mathscr{L} (\varepsilon_*, \varepsilon_*)$( or $\in
\mathscr{L}(\mathscr{D}_o,  \mathscr{D}_o )$) and $p \to B_p$ is
meromorphic with poles at $p= \dfrac{-3}{2},\ldots$. Actually the
following stronger result holds. 

\begin{theorem*}
  The function $p \to B_p$ is an entire analytic function with values
  in $\mathscr{L}(\varepsilon_*, \varepsilon_*)$ (also in
  $(\mathscr{D}_o, \mathscr{D}_o )$). 
\end{theorem*}

We have seen more than once, that in order to investigate the
analyticity of the function $p \to B_p$, it is sufficient to do the
same for the function $p \to B_p [f (x)]$, where $f \in \mathscr{E}$
and $x \geq 0$ are  arbitrarily chosen and are fixed. In this case we
study\pageoriginale the behaviour of the function $p - B_p [f(x)]$ where $f \in
\mathscr{E}^*$ and $x \geq 0$, in the neighbourhood of the point $p_o
= - \dfrac{2m+1}{2}$, (a supposed singularity of the function). 

By Taylor's formula,
$$
f(tx) = \sum^N_0 \frac{x^n t^n}{n!} f^n(0) +  \frac{x^{N+1}}{N}
\int^t_o f(t \xi) f^{N+1}( \xi x) d \xi, 
$$
$N$ being an arbitrary integer.

Suppose first that $ -1 < \Rep < - \dfrac{1}{2}$. Then 
\begin{multline*}
  B_p [f(x)] = \frac{\sqrt{\pi}}{\Gamma (p+1)} \sum^N_o \frac{\Gamma
    \left(p+ \frac{n}{2} +1\right)}{\Gamma \left(\frac{n}{2}+
    \frac{1}{2}\right)  n !} x^n ~f^n (0) \\
  +b_p \frac{x^{N+1}}{N !} \int^1_o t^{2 p+1} (1-t^2)^{-p -
    \frac{3}{2}} dt \int^1_o (t- \xi)^N  f^{N+1} (\xi x) d \xi.  
\end{multline*}
The right hand side of this formula is well defined when
$-\dfrac{N+3}{2} < \Rep < - \dfrac{1}{2}$ and depends analytically on
$p$ and therefore coincides with the continuation of $B$ already
obtained. Choosing $N$ such that $- \dfrac{N+3}{2}< p_o$, the integral
on the right is analytic at the point $p_o$. Hence we have only to
consider the finite sum at $p_o$. The function $\Gamma (p +
\dfrac{n}{2} +1)$ has poles at the points $p$ such that $p+
\dfrac{n}{2} +1 = - \mu, \mu$ a positive integer. It has a pole at
$p_0$ if $\dfrac{n}{2} =-  \mu -1 - p_0 = m - \mu -
\dfrac{1}{2}$ i.e. $n = 2 (m -\mu) -1$.\pageoriginale But when $n$ is odd, $f^n(0) =
0$ since $f \in \mathscr{E}_*$. The finite sum is therefore analytic
at $p_0$ and  $B_p [f (x)]$ has false singularity at $-
\dfrac{2m+1}{2}$ and the proof of the theorem is complete. 

\begin{theorem*}
  The formula 
  $$
  D^2 B_p f = B_p L_p f ~\text{holds for every}~ f \in
  \mathscr{E}^*
  $$ 
  and for every complex number $p$. $L_p f$ is in
  $\mathscr{E}_*$ if $ f \in \mathscr{E}_*$ so that $L_p \in
  \mathscr{L}( \mathscr{E}_*, \mathscr{E}_*)$ and it is easy to verify
  that $p \to L_p$ is an entire function with value is $\mathscr{L}(
  \mathscr{E}_*, \mathscr{E}_*)$. The two entire functions $p \to B_p
  L_p$ and $p \to D^2 B_p$ coincide by
  Proposition \ref{part1:chap3:sec1:prop2}, \S 1 in $-1 <
  \Rep <  - \dfrac{1}{2}$ and are therefore identical in the whole
  plane. 
\end{theorem*}

\begin{remark*}
  It is necessary to suppose that $f \in \mathscr{E}_*$. If $f$ is
  only in $\mathscr{E}, L_p f$ is not in $\mathscr{E}$ (always). 
\end{remark*}

\section{Continuation of $\mathscr{B}_p$}\label{part1:chap3:sec3}% sec 3

We now consider the extension of the operator $\mathscr{B}_p$
initially defined for $\Rep  > - \dfrac{1}{2}$ by  
\begin{equation*} 
  \mathscr{B}_p [f(x)] = \beta_p \int^1_o (1-t^2) ^{p- \frac{1}{2}}
  f(tx) dt. \tag{I}\label{part1:chap3:sec3:eqI}
\end{equation*}
  where
  $$
  \beta_p = \frac{2 \Gamma (p+1)}{\sqrt{\prod} \Gamma (p+
    \frac{1}{2})}
  $$

Changing\pageoriginale the variable $t= \sin \Theta$,
$$
\mathscr{B}_p [f (x)] = \beta_p  \int^{\frac{\prod}{2}}_o f(x \sin \theta)
\cos^{2p} \theta d \theta.  
$$

Let $M^1_p [f(x, \theta)] = f(x \sin \theta)$
$$
+ \frac{1}{(2p +1) (2p+2)} \frac{d}{d \theta} \left\{ \sin \theta
\frac{d}{d \theta}[\sin \theta f ( x \sin \theta )] \right\} 
$$
and
$$
N^1_p [ f (x, \theta )] = \frac{1}{2p+1} \frac{d}{d \theta } \left\{
\sin \theta  f(x \sin \theta ) \right\} 
$$

We determine by induction, the functions
\begin{multline*}
  M^n_p [f(x, \theta  )] = M^{n-1}_p [f(x, \theta  )] +  \frac{1}{2p
    +2n} \frac{d}{d \theta  }\\ 
  \left\{\sin \theta  \frac{d}{ d \theta }
  \sin \theta  N^{n-1}_p [f(x, \theta  ] \right\}  
  +  \frac{1}{(2p+ 2n-1)(2p + 2n)} \frac{d}{d \theta }\\ \left\{ \sin
  \theta  \frac{d}{d \theta } [\sin \theta  M^{n-1}_p (f(x, \theta 
    ))\right\} 
\end{multline*}
and 
$$
N^n_p [f(x, \theta )] = N^{n-1}_p [ f(x, \theta ) ] +  \frac{1}{2p+
  2n-1} \frac{d}{d \theta } \left\{ \sin \theta  M^{n-1}_p [f (x, \theta 
  )] \right\} 
$$

\setcounter{proposition}{0}
\begin{proposition}\label{part1:chap3:sec3:prop1}%proposi 1.
  For $\Rep > - \dfrac{1}{2}, f \in \mathscr{E} (x \geq 0)$ and for any $n$, 
  \begin{multline*}
    \mathscr{B}_p [ f(x)] = \beta_p \int^{\frac{\pi}{2}}_o \cos^{2p
      + 2n} \theta  M^n_p [ f(x, \theta  )] d \theta \\ 
    + \beta_p
    \int^{\frac{\pi}{2}}_0 \cos^{2p + 2n + 1} \theta  N^n_p f(x,
    \theta  ) d \theta  \tag{1}\label{part1:chap3:sec3:eq1} 
  \end{multline*}
  
  The\pageoriginale proof of this proposition is elementary and is
  based on induction 
  on $n$ and the process of integration by parts. 
  \begin{multline*}
  \mathscr{B}_pf (x)= \beta_p \int^{\frac{\pi}{2}}_{0} f (x \sin\theta  )
  \cos^{2p+2}  \theta  d \theta \\ 
  + \beta_p \int^{\frac{\pi}{2}}_{o} f (x
  \sin \theta ) \sin^2 \theta  \cos^{2p} \theta  d \theta . 
  \end{multline*}
\end{proposition}

But as $\sin \theta  \cos^{2p} \theta  = \dfrac{d}{d \theta }
(\dfrac{\cos ^{2p +1} \theta }{- (2p+1)})$ integrating by parts the
second integral, we get, 
\begin{multline*}
  \mathscr{B}_pf (x) = \beta_p \int^{\frac{\pi}{2}}_{o} f (x \sin \theta  )
  \cos^{2p+2} \theta  d \theta \\ 
  + \beta_p
  \int^{\frac{\pi}{2}}_{0}\frac{1}{2p+1} \frac{d}{d \theta }  ( \sin
  \theta  f (x \sin \theta )) \cos^{2p+1} \theta  d \theta  
\end{multline*}

Now the second integral in this equation equals
\begin{multline*}
  \int^{\frac{\pi}{2}}_{0}\frac{1}{2p+1} \frac{d}{d \theta }  ( \sin
  \theta  f (x \sin \theta  )) \cos^{2p+3} \theta  d  \theta  \\ 
  +  \int^{\frac{\pi}{2}}_{0}\frac{1}{2p+1} \frac{d}{d \theta }  ( \sin
  \theta  f (x \sin \theta  )) \cos^{2p+1} \theta  \sin^2 \theta  d
  \theta \\ 
  = \int^{\frac{\pi}{2}}_{0}\frac{1}{2p+1} \frac{d}{d \theta }  ( \sin
  \theta  f (x \sin \theta  )) \cos^{2p+3} \theta  d \theta  \\ 
  +  \int^{\frac{\pi}{2}}_{0}\frac{1}{(2p+1)(2p+2)}\cos^{2p + 2}\theta 
  \frac{d}{d \theta } \bigg\{  \sin \theta  \frac{d}{d \theta } ( \sin
  \theta  f( x \sin \theta  ))\bigg\} d \theta  
\end{multline*}
so\pageoriginale that
\begin{multline*}
  \mathscr{B}_pf(x) = \beta_p \int^{\frac{\pi}{2}}_{0} \cos^{2p +2}
  \theta  \bigg\{f (x \sin \theta  )\\ 
  + \frac{1}{(2p+1)(2p+2)}\frac{d}{d
    \theta }(\sin \theta  \frac{d}{d \theta } (\sin \theta  f (x \sin
  \theta  )))\bigg\} d \theta \\ 
  + \beta_p \int^{\frac{\pi}{2}}_{0} \cos^{2p+3} \theta  \frac{1}{2p+1}
  \frac{d}{d \theta }  ( \sin \theta  f (x \sin \theta  ))  d \theta  
\end{multline*}

Hence formula (\ref{part1:chap3:sec3:eq1}) holds for $n=1$. Assuming it for $n-1$, it can be
verified for $n$ by integration by parts. 

We define for every integer $n > 0$,
\begin{multline*}
  \mathscr{B}^n _p f (x) = \beta_p \int^{\frac{\pi}{2}}_{0} \cos^{2p +
    2n} \theta  M^n_p f (x, \theta  ) d \theta  +\\ 
  \beta_p \int^{ \frac{\pi}{2}}_{o} \cos^{2p+2n+1} \theta  N^n_p f (x,
  \theta  )d \theta  
\end{multline*}

The two integrals converge for $\Rep > - \dfrac{1}{2}- n$.

\begin{proposition}\label{part1:chap3:sec3:prop2}%proposi 2.
  If\pageoriginale $p$ satisfies
  \begin{equation*}
    \Rep > - \frac{1}{2}- n \tag{2}\label{part1:chap3:sec3:eq2}
  \end{equation*}
    and
  \begin{equation*}  
    p \neq -1, -2, -3, \ldots \tag{3}\label{part1:chap3:sec3:eq3}
  \end{equation*}
\end{proposition}

Then
\begin{enumerate}[a)]
\item For $f \in \mathscr{E} (x \geq 0)= \mathscr{E}, \mathscr{B}^n _p f
  \in \mathscr{E}$. 
\item The mapping $f \to \mathscr{B}^n_p f$ is linear continuous of
  $\mathscr{E}$ into itself. 
\item The function $p \to \mathscr{B}^n _p$ is meromorphic in the half
  plane defined by (\ref{part1:chap3:sec3:eq2}) with values in
  $\mathscr{L} (\mathscr{E}, 
  \mathscr{E})$ the poles being situated at the points
  (\ref{part1:chap3:sec3:eq3}). We shall
  give only the outline of the proof since  it is analogous to that of
  Proposition (\ref{part1:chap3:sec3:prop2}), \S 2. The function $\beta _p = \dfrac{2 \Gamma
    (p+1)}{\sqrt{\pi} \Gamma (p + \dfrac{1}{2})}$ is a meromorphic
  function of $p$ which vanishes  at $- \dfrac{1}{2}, - \dfrac{3}{2},
  \ldots, $ and poles at $-1,-2,-3, \ldots$ and for $x, \theta $ fixed,
  $M^n_p f ( x, \theta )$,  and $N^n_p f (x, \theta )$ are meromorphic
  function with poles at $- \dfrac{1}{2}, -1, \dfrac{-3}{2}, -2,
  \ldots$. Hence for $p$ fixed and $\neq -1, -2, \ldots, $ the
  function $(x, \theta ) \to \beta _p M^n _p (x, \theta )$ and $(x,
  \theta ) \to \beta_p N^n _p f (x, \theta )$ are indefinitely
  differentiable in $[0, \infty) \times (0, \dfrac{\pi}{2})$ so that
  $(a)$ is true.  The proof for $(b)$ is similar to that of $(b)$ of
  proposition \ref{part1:chap3:sec2:prop2}, \S 2. For $f,x$ and
  $\theta $ fixed, $M^n_p f (x, 
  \theta )$ and $N^n_p f (x, \theta)$ are meromorphic functions with
  poles at $-1, -2, \ldots$ so that $p \to \mathscr{B}_pf (x)$ is
  meromorphic $\Rep > - \dfrac{1}{2}-n$ with poles at
  (\ref{part1:chap3:sec3:eq3}). Further
  since $p \to \mathscr{B}_p$ is continuous in the
  region\pageoriginale given by (\ref{part1:chap3:sec3:eq2}) and
  (\ref{part1:chap3:sec3:eq3}), $(c)$ follows. 
 \end{enumerate} 

\begin{coro*}
  If $m, n$ are two distinct positive integers, the functions
  $\mathscr{B}^m_p$ and $\mathscr{B}^m_p$ coincide in the intersection
  of their domains  of  definitions. 
\end{coro*}

In fact the two meromorphic functions coincide with $\mathscr{B}_p$
in the half plane $\Rep > - \dfrac{1}{2}$ which is common to their
domains of definitions and hence  they coincide everywhere in the
intersection of their domains of definition. 

We have proved the following
\begin{theorem*}
  The function $p \to \mathscr{B}_p$ defined in $\Rep > -
  \dfrac{1}{2}$ by (\ref{part1:chap3:sec3:eqI}) with values in $\mathscr{L}(\mathscr{E},
  \mathscr{E})$ can be continued analytically into a function
  meromorphic  in the whole plane with poles at $-1, -2, -3, \ldots$. 
\end{theorem*}

\begin{rem}\label{part1:chap3:sec3:rem1}%remark 1.
  $\mathscr{B}_{- \frac{1}{2}= }$ identity.
\end{rem}

\begin{rem}\label{part1:chap3:sec3:rem2}%remark 2.
  For $Rep > - \dfrac{1}{2}$, we have
  \begin{align*}
  D^r \mathscr{B}_p f(x) & = \beta _p \int^{1}_{\circ} t^r (1-t^2)^{p -
    \dfrac{1}{2}}f ^r (tx) dt\hspace{2cm} \\
  \text{and} \qquad  
    [D^r \mathscr{B }_p f (x) ]_{x=0} & = \beta _{p,r}f^r (0)
    \tag{4}\label{part1:chap3:sec3:eq4} 
  \end{align*}
  where $\beta_{p,r}= \dfrac{\Gamma (p+1) \Gamma
    \left(\dfrac{r+1}{2}\right)}{\sqrt{\pi} \Gamma \left(\dfrac{r}{2}+
    p+1\right)}$ is a
  meromorphic  function of $p$ with poles at $-1,-2, \ldots $. By
  analytic continuation, the equation\pageoriginale $(I)$ holds for $p \neq -1, -2,
  \ldots$. Then if $f\in \mathscr{E}_{*} (\resp \mathscr{D}_o$),
  $\mathscr{B}_{p} f \in \mathscr{E}_{*} (\resp \mathscr{D}_o)$ and $p
  \to \mathscr{B}_p$ is meromorphic with values in
  $\mathscr{L}(\mathscr{E}_{*}, \mathscr{E}_{*}) (\resp 
  \mathscr{L}(\mathscr{D}_o,  \mathscr{D}_o))$. 
\end{rem}

\begin{theorem*}
  For any $p \neq -1, -2,  \ldots$ we have
  $$
  \mathscr{B}_p D^2 f = D^2 \mathscr{B}_p f,  f \in \mathscr{E}_{*}.
  $$
\end{theorem*} 

By Proposition \ref{part1:chap3:sec1:prop1}, \S 1, the given equation holds for  $ \Rep >
-\dfrac{1}{2}$ and hence for all $p \neq -1, -2, \ldots $ by analytic
continuation. 

\begin{theorem*}
  For any complex $p \neq -1, -2, \ldots $ the operators $B_p$ and
  $\mathscr{B}_p$ are isomorphisms of $\mathscr{E}_*$ (resp
  $\mathscr{D}_o$) onto itself which are inverses of each other. 
\end{theorem*} 
 
 We have seen that $D\bar{B}_p \mathscr{B}_p = I $(identity) for $-
 \dfrac{1}{2} < \Rep < \dfrac{1}{2}$. Also for $p$ in the same region
 $D\bar{B}_p = B_p$ so that $B_p \mathscr{B}_p = I$ for $-
 \dfrac{1}{2}< \Rep < \dfrac{1}{2}$ and the same holds for all $p \neq
 -1, -2, -3$, by analytic continuation. Similarly $\mathscr{B}_p B_p
 =I$ for $p \neq -1, -2,  \ldots$. 
 
