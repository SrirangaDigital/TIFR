
\chapter{Mean Periodic Function in $R^2$}\label{part2:chap2}%chap 2

We\pageoriginale shall study function of two variables mean periodic with respect to
two distributions, and discuss problems \ref{part2:chap1:sec1:prob1}
and \ref{part2:chap1:sec1:prob2} stated in Chapter 
1 (page \pageref{page68}). The general case being difficult we shall give
solutions of the problems in some special cases when the distributions
are of particular type with the spectrum satisfying certain
conditions. 

\begin{defi*}%defi
  A continuous function $F$ on $R^2$ is said to be mean periodic with
  respect to $T_1, T_2 \in \varepsilon'(R^2)$ if it verifies
  simultaneously the convolution equations 
  $$
  T_1 \ast F = 0, T_2 \ast F = 0.
  $$
\end{defi*}

The Fourier- Laplace transforms
$$
M_i (\lambda) = \langle T_i, e^{<\lambda, x>} \rangle
$$
where $x = (x_1, x_2)\, R^2, \lambda = (\lambda_1, \lambda_2) \in C^2,
<\lambda, x> = \lambda_1 x_1 + \lambda_2 x_2$ of $T_i (i = 1,2)$ are entire
functions of $\lambda_1, \lambda_2$ of exponential type. The \textit{
  spectrum } $(\sigma)$ is defined by $M_i (\lambda) = 0, i =
1,2,$. In all that follows we shall restrict ourselves to the case
when i) $(\sigma)$ is countable and (ii) $(\sigma)$ is simple i.e. $\alpha
\in (\sigma)$ is a simple zero of $M_1 (\lambda)$ and 
$M_2 (\lambda )$\pageoriginale and the Jacobian 
\begin{align*}
  D(\lambda ) & = \begin{vmatrix}
    \dfrac{\partial M_1}{\partial \lambda_1}\quad \dfrac{\partial
      M_1}{\partial \lambda_2} \\ {} & {} \\ \dfrac{\partial
      M_2}{\partial \lambda_1} \quad \dfrac{\partial M_2}{\partial
      \lambda_2} \end{vmatrix} ~\text{does not vanish at any}~\alpha \in
  (\sigma ). ~\text{We define}\\  
  t_{\alpha}(\lambda ) & = \frac{1}{D(\alpha) (\lambda_1 - \alpha_1)
    (\lambda_2 - \alpha_2)} \quad \frac{J(M, M)}{J(\lambda,  \alpha
    )}\\
  \text{where}\quad & \hfill \\
  \frac{J(M, M)}{J(\lambda,  \alpha )} & = \begin{vmatrix} M_1
  (\lambda_1,  \lambda_2 ) - M_1(\alpha_1, \lambda_2 )~~ M_1 (\alpha_1,
  \lambda_2) - M_1(\alpha_1, \alpha_2 )\\ 
         {} & {} \\ 
  M_2 (\lambda_1,
  \lambda_2 ) - M_2 (\alpha_1, \lambda_2 )~~ M_2 (\alpha_1, \lambda_2) -
  M_2(\alpha_1, \alpha_2)\end{vmatrix}
\end{align*}
is the determinant of Jacobi. It is clear that
\begin{align*}
  t_\alpha (\beta ) &= 0 \quad \text{ for } \quad \alpha,  \beta \in
  (\sigma ),  \alpha \neq \beta \\ 
  &=1 \quad \text{ for } \quad \alpha, \beta \in (\sigma ),  \alpha =
  \beta 
\end{align*}
For $\alpha_\in (\sigma )$, $t_\alpha (\lambda )$ is an entire
function of $\lambda$ of exponential type and by Paley-Wiener theorem,
$t_\alpha (\lambda)$ is the Fourier Laplace transform of a
distribution $T_\alpha \in \mathscr{E}^1 (R^2)$. $\{ T_\alpha
\}_{\alpha \in (\sigma )} $ together with the exponentials\break $\left
\{e^{\langle \alpha,  x} \rangle \right \}_{\alpha \in (\sigma )}$
mean periodic with respect to $T_1,  T_2$ form a biorthogonal system,
i. e. 
\begin{equation*}
\begin{aligned}
  &\langle T_\alpha,  e^{<\beta,  x >} \rangle = 0 \text{ for }
  \alpha,  \beta \in (\sigma ), \alpha \neq \beta \\ 
  &\langle T_\alpha,  e^{<\alpha,  x >} \rangle = 1
  ~~\text{ for } ~~\alpha \in (\sigma )  
\end{aligned}\tag{1}\label{part2:chap2:eq1}
\end{equation*}
If\pageoriginale $F$ is mean periodic with respect to $T_1, T_2$ and if we suppose
the existence of an expansion of $F$ in a series of mean periodic
exponentials  
\begin{equation*}
  F(x) \sim \sum_{\alpha \in (\sigma )} c_\alpha \quad  e^{< \alpha,
    x >} \tag{2}\label{part2:chap2:eq2} 
\end{equation*}
we have formally
\begin{equation*}
  c_\alpha = \langle T_\alpha,  F \rangle \tag{3}\label{part2:chap2:eq3}
\end{equation*}

Let $s_\alpha (\lambda ) = \dfrac{1}{D(\alpha} \dfrac{J(M,
  M)}{J(\lambda,  \alpha )}$ so that  
$$
s_\alpha (\lambda ) = (\lambda_1 - \alpha_1) (\lambda_2 - \alpha_2 )
t_\alpha (\lambda ) 
$$
For $\alpha$ fixed in $(\sigma )$, $s_\alpha (\lambda )$ is an entire
function of $\lambda$ of exponential type. Let $S_\alpha$ be the
distribution in $\mathscr{E}'(R^2)$ whose Fourier-Laplace image is
$s_\alpha (\lambda )$. Denoting $\dfrac{\partial}{\partial x_i}$ by
$D_i (i = i, 2)$, we have 
\begin{align*}
  \langle (D_1 + \alpha_1 ) (D_2 + \alpha_2 ) T_\alpha,  e^{\lambda x}
  \rangle & = (D_1 D_2 + \alpha_1 D_2 + \alpha_2 D_1 + \alpha_1 \alpha_2)
  T_\alpha,  e^{\lambda x}\rangle\\
  &= \langle T_\alpha,  (D_1 D_2 - \alpha_1 D_2 - \alpha_2 D_1 +
  \alpha_1 \alpha_2) e^{\lambda x} \rangle 
\end{align*}
(by\pageoriginale the definition of the derivative of a distribution
$\langle D^p T$, $\phi \rangle = (-1)^p$ $\langle T, D^p \phi
\rangle$, refer to `theory des distributions' by L. Schwartz) 
$$
= \langle T_\alpha,  (\lambda_1 - \alpha ) (\lambda_2 - \alpha_2
)e^{\lambda x}\rangle = (\lambda_1 -\alpha_1) (\lambda_2 - \alpha_2)
t_\alpha (\lambda) 
$$
Hence  
\begin{equation}
  S_\alpha = (D_1 + \alpha_1 ) (D_2 + \alpha_2 ) T_\alpha
  \tag{4}\label{part2:chap2:eq4} 
\end{equation}
(\ref{part2:chap2:eq4}) is equivalent to 
\begin{equation}
  e^{- \langle \alpha,  x \rangle} D_1 D_2 \left[ e^{< \alpha,  x >}
    T_\alpha \right] = S_\alpha \tag{5}\label{part2:chap2:eq5} 
\end{equation}
For
$$
\langle e^{-\langle \alpha,  x\rangle} \left[ D_1 D_2
  ~e^{\langle \alpha,  x\rangle} T_\alpha \right],  \phi \rangle =
\langle T_\alpha,  e^{<\alpha,  x>} D_1 D_2 e^{-\langle \alpha,
  x\rangle} \phi \rangle 
$$

(by the definition of the multiplicative product of a distribution by
a function and the derivative of a distribution; refer to `Theory des
distributions') 
\begin{align*}
  &= \langle T_\alpha,  (\alpha_1 \alpha_2 - \alpha_2 D_1 -\alpha_1 D_2
  + D_1 D_2 ) \phi \rangle \\ 
  &= \langle (D_1 + \alpha_1 ) (D_2 + \alpha_2 ) T_\alpha,  \phi > = <
  S_\alpha,  \phi \rangle 
\end{align*}

Let $G(x)$ be any solution of the partial differential equation
\begin{align*}
  e^{\langle \alpha,  x\rangle} D_1 D_2 \left[ e^{-\langle \alpha, x
      \rangle} G(x) \right] & = F(x) \tag{6}\label{part2:chap2:eq6}\\ 
  \text{or}\hspace{2cm}  DG & = F
  \tag*{$(6')$}\label{part2:chap2:eq6'}
\end{align*}
where\pageoriginale $D = \dfrac{\partial^2}{\partial x_1 \partial x_2}
-\alpha_1 
\dfrac{\partial}{\partial x_2} - \alpha_2 \dfrac{\partial}{\partial
  x_1} + \alpha_1 \alpha_2 $ is a differential operator with constant
coefficients. For $F \in \mathscr{E}$, we can choose $G \in
\mathscr{E}$. We can actually take $G$ to be  
\begin{equation}
  G(x_1, x_2) = \int_{a_1}^{x_1} \int_{a_2}^{x_2} e^{\alpha_1 (x_1 -
    \xi_1 ) + \alpha_2 (x_2 - \xi_2 )} F (\xi_1,  \xi_2 ) d \xi_1 d
  \xi_2 \tag{7}\label{part2:chap2:eq7} 
\end{equation}
where $a_1, a_2$ are arbitrary.

Thus we obtain from
\begin{align*}
  \langle T_\alpha,  F \rangle &= \langle T_\alpha,  e^{\langle \alpha,
    x \rangle} D_1 D_2 \left[ e^{- \langle\alpha,  x\rangle} G(x)
    \right] \rangle \\ 
  & = e^{-\langle \alpha,  x \rangle} D_1 D_2 \left[ e^{\langle \alpha,
      x \rangle} T_\alpha \right],  G \rangle = \langle S_\alpha, G
  \rangle, 
\end{align*}
the formula\label{page79}
\begin{equation}
  c_\alpha = < S_\alpha,  \int_{a_1}^{x_1} \int_{a_2}^{x_2}
  e^{\alpha_1 (x_1 - \xi_1 ) + \alpha_2 (x_2 - \xi_2 )} F (\xi_1,
  \xi_2 ) d \xi_1 d \xi_2 > \tag{8}\label{part2:chap2:eq8} 
\end{equation}
which is a natural generalization of the formula know in the case of $R^1$.

\begin{remark*}
  The\pageoriginale distributions $S$ are completely explicit. We have
  $$
  S_\alpha (\lambda ) = \frac{1}{D(\alpha )} M_1 (\lambda_1,
  \lambda_2 ) M_2 (\alpha_1,  \lambda_2 ) - M_2 (\lambda_1,  \lambda_2
  ) M_1 (\alpha_1,  \lambda_2) 
  $$
  and $S_\alpha = \dfrac{1}{D(\alpha )} \left[ T_1 * \sum_2 - T_2 *
    \sum_1 \right]$ where $T_1$ and $T_2$ are the given distributions
  and $\sum_2$ for instance is a distribution in the variable $x_2$
  determined by its Fourier-Laplace image $M_2 (\alpha_1,  \lambda_2
  )$. 
\end{remark*}

\begin{theorem*}
  The mean periodic exponentials $\left\{ e^{<\alpha,  x >}
  \right\}_{\alpha \in (\sigma )}$ form a free system of functions in
  $\mathscr{E}$. 
\end{theorem*}

In fact they form a biorthogonal system with the distributions\break 
$\left\{ T_\alpha \right\}_{\alpha \in (\sigma )}$ in $\mathscr{E}'$
and it is clear from (\ref{part2:chap2:eq1}) that no $e^{\langle\alpha, x \rangle}$ is
in the closed subspace generated by $\left\{ e^{<\beta,  x
  >} \right\}_{\substack{\beta \in (\sigma)\\ \beta \neq
    \alpha}}$. 

\begin{remark*}
  It is possible to choose the solution $I_\alpha(f) = G$ of
  (\ref{part2:chap2:eq6})
  such that for $F \in \mathscr{E}$, the function $\lambda \to I_\alpha
  (F)$ from $C^2$ to $\mathscr{E}$ is an entire function. In fact it
  suffices to take $G$ as in (\ref{part2:chap2:eq7}). In this case 
  $$
  c_\alpha \quad e^{\langle\alpha, x \rangle} =
  \frac{e^{\langle\alpha, x \rangle}}{D(\alpha )} \langle S_\alpha,
  I_\alpha (F) \rangle  
  $$
  $\lambda \to S_\lambda$\pageoriginale is an entire function of $\lambda$ with
  values in $\mathscr{E}'$ and $\lambda \to I_\lambda (F)$ is an
  entire function of $\lambda$ with values in $\mathscr{E}$. Hence 
  $$
  \mathscr{F} (x, \lambda ) = \frac{e^{\langle\lambda,  x \rangle}
    \langle S_\lambda,  I_\lambda (F) \rangle}{M_1 (\lambda ) M_2
    (\lambda )} 
  $$
  is a meromorphic function of $\lambda$ in which the `local residues'
  in the sense of Poincare are precisely $c_\alpha \quad e^{< g
    \alpha,  x >}$. 
\end{remark*}

The Problem is now as follows.

If $F$ is given in $\mathscr{E}$ mean periodic with respect to $T_1$
and $T_2$ then is $F$ then sum of the series $\sum\limits_{\alpha \in
  (\sigma )} e_\alpha \quad e^{< \alpha,  x >}$? 

If for $x$ fixed, the integrals of $\mathscr{F}(x, \lambda )$ over a
system of varieties in $C^2$ tending to infinity have for limit $F(x)$
and if the global Cauchy theorem were true in $C^2$, then the answer
to this question would be in affirmative. In this manner, the problem
is equivalent to the global Cauchy theorem in $C^2$ and the answer is
completely unknown. 
