\chapter{Some theorems on  stratification}\label{chap2}

\markright{\thechapter. Some theorems on  stratification}

\setcounter{lemma}{0}
\begin{lemma}\label{chap2-lem1}
  If\pageoriginale $V$ is a  $ \mathbb{C}$-analytic set in an open set $U
  \subset   \mathbb{C}^n$ and  if $f_1,\ldots, f_r$ are holomorohic
  functions  on $U$ such that the germs of $f_1,\ldots , f_r$ at any
  point $b$ in $U$  generate  the ideal $I_b$ of $V_b$, then $V$ is a
  submanifold  of dimension $p$  in  a neighbourhood  $U_0 \subset
  U$ of any point $a_0$ if and,only if rank $\left(\dfrac{\partial
    f_i}{\partial z_{j}}\right)_{z} = n-p $ for $z$ in $U' \cap V $ where
  $U'$ is a certain neighbourhood of $a_0$. 
\end{lemma}

\begin{proof} % pro 
  If  $a_\circ$ is a  simple  point  of  dimension $p$, then there
  exists a neighbourhood $U_0$ of $a_0$ and holomorphic functions
  $g_1,\ldots , g_{n-p}$ on $U_0$ such that rank $\left(\dfrac{\partial
    g_i}{\partial z_j}\right)_{z} = n-p$ for  any $z$ in $U_0$ and the
  germs  $(g_{i})_{z}$ generate  the ideal $I_z$ for $z$ in
  $U_0$. Clearly $g_i = \sum\limits _{j}  \lambda_{ij} f_{j}$ and
  hence rank $\left(\dfrac{\partial f_i}{\partial z_{j}}\right)_{z} \ge n-p$ for
  $z$ in $U_0$. Since $(g_{i})_{z}$ also  generate  $I_z$, we  have
  conversely, rank $\left(\dfrac{\partial g_i}{\partial z_{j}}\right)_{z} \ge$
  rank $\left(\dfrac{\partial f_i}{\partial z_{j}}\right)_{z} \ge n-p$ i.e. rank
  $\left(\dfrac{\partial f_i}{\partial z_{j}}\right)_{z}  =n-p$  for $z$ in
  $U_0$. Conversely if  rank  $\left(\dfrac{\partial f_i}{\partial
    z_{j}}\right)_{z} = n-p$ for $z$  in $U'$, $U'$ being a neighbourhood  of
  $a_0$, we can find  a subset $\{f_1,\ldots ,f_{n-p}\}$  of  $ \{
  f_1,\ldots ,f_r \}$ such that rank $\left(\dfrac{\partial f_i}{\partial
    Z_{j}}\right)(a)_{\substack{i\leq n-p\\j \leq n}}=
  n-p$. Consider $V' = \{z \epsilon U'\mid f_i(z)=0,i\leq n-p \}$.  

 
Let $U'' \subset U'$ be a neighbourhood of $a_0$ such that $U''
\cap V \subset V'\cap  U'' \dim V' = p$ and\pageoriginale $V' \cap U''$ is a
manifold. Hence by Proposition  \ref{chap1-prop2} of Chapter
\ref{chap1}  if $ V \cap U'' 
\underset{\neq}{\subset}  V' \cap U''$, $\dim V=p' < p$  in $U'$, and if
$b$ is a  simple point  of $V$  in $U''$, rank  $\left(\dfrac{\partial
  f_i}{\partial z_j}\right) (b) =n-p'$ by the converse proved above. Thus
rank $\left(\dfrac{\partial f_i}{\partial z_j}\right) (b) =n-p'> n-p$
and we have a contradiction and this proves the lemma. 
\end{proof}

\begin{lemma}\label{chap2-lem2} %lem 2
  If  $V$  is an  irreducible analytic set and if $M$  is the set of
  its  simple points  and if $ \dim V= p $ then $\dot{V} = V-M$    is
  an analytic set and $\dim  \dot{V} < p$. 
\end{lemma}

\begin{proof} %pro 
  For any $a_\circ \epsilon  V$, there exists a neighbourhood  $U$ and
  holomorphic functions $f_1,\ldots , f_r$ on $U$ (by Theorem
  \ref{chap1-thm6}, Chapter \ref{chap1})  such that the germs
  $(f_{i_{b}})$ generate the 
  ideal $I_b$ at any point  $b$ in $U$. By the above lemma a point
  $b$ in $U$ is a simple  point  if and only if rank $\left(\dfrac{\partial
    f_i}{\partial z_j}\right)_{b} = n-p$ i.e. a point  $z$ is  in  $U \cap
  \dot{V}$   if  and only if determinants of all submatrices of order
  $\ge n-p$ of $\left(\dfrac{\partial f_i}{\partial z_{j}}\right)_{z}$ are
  zero. Since $\dfrac{\partial f_i}{\partial z_j}$  and hence the
  determinants are holomorphic on  $U$, the lemma is proved.    
\end{proof}

\begin{lemma}\label{chap2-lem3} % lem 3
 Let $V = U V_i$ be an analytic set when $V_i$ are its irreducible
 components and let the maximum dimension of $V=p$. Then if $M = \{ z
 \epsilon V \mid z$ is a simple point  of dimension $p \}$ then the
 set $V_1= V-M$ and the set $\dot{V}$ of  singular points of $V$ are
 analytic sets of dimension $ < p$. 
\end{lemma}

\begin{proof} % pro 3
  It\pageoriginale follows  from the remark of Chapter \ref{chap1}
  that 
$$
\dot{V}
  \bigcup\limits_{i} \dot{V}_{i} \cup  \left(\bigcup\limits_{i\neq j}
  V_i \cap V_j\right).
$$ 

By Lemmma \ref{chap2-lem2} 
  proved above, $\dot{V}_i$ is an
  analytic  set and hence it follows that $\dot{V}$ is an analytic set
  of dimension $< p$. Also $V_1 = V-M = \left(\bigcup\limits_{\dim V_i <
    p} V_i\right)\cup \left(\bigcup\limits_{i\neq j}V_i
  \cap V_j\right) 
  \cup \left(\bigcup\limits_{\dim V_j=p}\dot{V}_j\right)$ is an analytic  set
  of dimension $< p$. 
\end{proof}

\setcounter{proposition}{0}
\begin{proposition}\label{chap2-prop1} %pro 1
If $V$ is any $\mathbb{C}$-analytic set there exists a strict
partition $V = \bigcup M_i $ of $V$.  
\end{proposition}

\begin{proof} %pro 
  Let $M_1$ be the  set of simple  points of maximum dimesion $p$ say,
  of $V$ and $M_1 = \cup M^{p}_{i}$,$ M^{p}_{i}$ being  the connected
  components of $M_1$. Then  $\bar{M}^{p}_{i}$ is an analytic  set and by the
  Lemma \ref{chap2-lem2} above, $V-M_1$ is an analytic set of $\dim < p$. Hence
  $\dot{M}^{p}_{i} = \bar{M}^{p}_{i} \cap (V-M_1)$ is  an analytic set. Now
  consider $V_1 = V-M_1\cdot \dim V_1 < p$. Let $M_2$ be the set  of
  simple  points of maximum dimension $p_1$, of  $V_1$. Then if
  $M_2 = \bigcup M^{p_{1}}_{i} M^{-p_{1}}_{i}$ and
  $\dot{M}^{p_{1}}_{i}$ are analytic sets and  $V-M_2$ is  an analytic
  set of dimension $< p_1$ and so on. We finally get $V= \bigcup
  \limits ^{P}_{i=0} (\underset{\nu}{\bigcup} M^{i}_{\nu}) $ and this
  is clearly a strict  partition  into  manifolds. This strict
  partition is a canonical one.  
\end{proof}

\setcounter{remark}{0}
\begin{remark}\label{chap2-rem1} %rem 1
  If $V$ is a  $ \mathbb{C} $ -analytic space then $V$ has a strict
  partition  into  manifolds. 
\end{remark}

\begin{proof} %pro 
  Let $V _k = \bigcup\limits_{\substack{V_{\alpha}\subset V
      \\ \text{and }\dim V_{\alpha}\le k}} V_{\alpha}$,
  $V_{\alpha}$ being  irreducible  analytic sets contained in  $V$. Then
  $V_k$  has  a strict  partition into  manifolds and if $V_k=
  \underset{i}{U}M_{ki}$ is  the partition, since $\{ V_{\alpha}\}$
  are locally  finite, we define $M_i = \underset{k \to
    \infty}{\varinjlim} M_{ki}~$ and it can be easily verified that $V
  = U M_i$ is\pageoriginale a strict partition in to manifolds. 
\end{proof}

\setcounter{examples}{0}
\begin{examples*} %Exa 1
\begin{enumerate}
\item $\text{Let}~ V \subset \mathbb{C}^{3}_{xyt} ~\text{be given by} $
    $$ 
    V= \left\{ z \epsilon  \mathbb{C}^{3}_{xyt} \big| x (x^2-y^2 -t) =
    0, V(x^2-y^2-t) = 0 \right\}. 
    $$

    Then clearly if $M_1 = \mathbb{C}_{t}$ and  $M_2 = V - \mathbb{C}_{t},
    V= M_1 \cup M_2$ is the canonical strict partition into
    manifolds. But this is not a stratification since $\dot{M}_2 =
    \{0\}$ is not a union of manifolds in the partition. 

  \item Let $V = V_1 \cup  V_2 \cup V_3$,  where
    \begin{alignat*}{4}
      && V &= \left\{ z \epsilon \mathbb{C}^{5}_{xytrs} \big| x^2-y^2-t^2 =
    0,r= s=0 \right\},\\
    && V_2  & =  \left\{ z \epsilon  \mathbb{C}^{5}_{xytrs} \big| x=y = r =
    0  \right\} \\
    \text{and}&\hspace{1cm} & V_3& = \left\{z \epsilon \mathbb{C}^{5}_{xytrs}
    \big| t=y=s=0 \right\}. 
    \end{alignat*}
\end{enumerate}

Then if 
\begin{multline*}
  M^0 _1 = \{0\}, M^2 _1 = V_1- \{0\}, M^2_2  = V_2-\{0\},\\ 
  M^2 _3 = V_3
  -\{0\},  V = M^0_1 \cup \left(\bigcup \limits ^{3}_{i=1} M^2_i\right) 
\end{multline*}
is the  canonical strict partition into manifolds  and clearly this is
a stratification. 
\end{examples*}

\begin{notation*} %Not 
  In what follows, a partition into manifolds of an analytic set $V$
  shall be written as $V = \cup M^k_\nu$ is a manifold of dimension
  $k$ and $M^k  = \cup M^k_\nu$ is the union of all manifolds of
  dimension $k$. 
\end{notation*}

\begin{lemma}\label{chap2-lem4} %lem 4
  Let $V$ be an analytic set and $V = \cup M^h_\nu$ be a strict
  partition into manifolds and $V= \cup S^k_\mu$, a
  stratification.Then the following are equivalent. 
\begin{enumerate}[(1)]
\item $S^{h}_{\nu} \cap M^{k}_{\mu} \neq \phi \Rightarrow S^{h}_{\nu}
  \subset \overline {M}^{k}_{\mu}$\pageoriginale
\item $ S^{h}_{\nu} \cap M^{k}_{\mu} \neq \phi \Rightarrow S^{h}_{\nu}
  \subset  M^{k}_{\mu}$ (i.e. $\{ S^{k}_{\mu }\}$ is a refinement
    of  $\{ M^{k}_{\nu}\}$). 
\end{enumerate} 
\end{lemma}

\begin{proof}
Obviously (2) $\Rightarrow$ (1). Conversely suppose that (1) holds.  
Let $z \in M^{k}_{\mu} \cap S^{h}_{\nu}$. Then there is a
  neighbourhood $U$ of $z$ such that $U \cap M^{k}_{\mu} = U \cap
  \overline{M}^{k}_{\mu}$. Since $S^{h}_{\nu} \subset
  \overline{M}^{k}_{\mu}$, it follows that $U \cap S^{h}_{\nu} \subset
  U \cap M^{k}_{\mu} $. If $A = \{ z \in S^{h}_{\nu}\big| z \in
  M^{k}_{\mu}\}$ then this proves that $A$ is open in
  $S^{h}_{\nu}$. If $z_{\nu} \in A $ and $z_{\nu} \to z_{o} \in
  S^{h}_{\nu}$, let if possible, $z_{0} \notin M^{k}_{\mu}$ and let
  $z_{0} \in M^{l}_{\lambda}$, $(l,\lambda) \neq (k, \mu)$. Then by the
  same argument as above there is a neighbourhood $U_{0}$ of $z_{0}$
  such that $U_{0} \cap S^{h}_{\nu} \subset U_{0} \cap
  M^{l}_{\lambda}$ but then there are $z_{\nu} \in M^{l}_{\lambda}
  \cap M^{k}_{\mu}$ and we have a contradiction. This proves that $A$
  is closed and hence that lemma is proved. 
\end{proof}

\setcounter{definition}{0}
\begin{definition}\label{chap2-defin1} %def 1
 For an analytic set $V$, a stratification $V = \cup S^{k}_{\mu}$ is
  defined to be a narrow stratification if for every open set $U$ of
  $V$, the connected components of $S^{k}_{\mu} \cap U $ form a
  stratification of $V \cap U$.   
\end{definition}

\begin{remark}\label{chap2-rem2} % rem 2
  An arbitrary stratification may not be a narrow stratification. For
  example, let $V = \mathbb{C}^{2}_{xy}$. Consider the stratification
  $M_{1} = \big\{ z \in \mathbb{C}^{2}_{xy}\big| x = 0 \big\}$ and
  $M_{2} = V -M_{1}$. Consider the point $ a = (1,1) \in M_{2}$. Let
  $f: \mathbb{C}^{2}_{xy} \to \mathbb{C}^{n}$ be a bolomorphic map.$n$
  sufficiently large, such that  
  \begin{enumerate}[(1)]
  \item $f$ is proper
  \item $f(z_{1}) \neq f(z_{2})$ if $(z_{1},z_{o}) \neq (0,a)$
  \item $f(a) = f(0)$\pageoriginale and 
  \item rank $(df)(z) = 2$ for any $z \in \mathbb{C}^{2}_{xy}$.
  \end{enumerate}    
\end{remark} 

Then $ f(\mathbb{C}^{2}_{xy})$ is an analytic set and if $M_{1} =
f(M_{1}), N_{2} = f(M_{2} - \{a\})$ then $ f(\mathbb{C}^{2}_{xy}) =
N_{1} \cup N_{2}$ is a stratification of $ f( \mathbb{C}^{2}_{xy})$
but it is not a narrow stratification. For if $U$ is a sufficiently
small neighbourhood of $f(a)$, then $f^{-1}(U) = U_{1}\cup U_{2},
U_{1} \cap U_{2} = \phi$, where $U_{1}$ and $U_{2}$ are neighbourhoods
of $0$ and $a$ respectively and hence $N_{2} \cap U$ has two connected
components $f(U_{1}\cap M_{2})$ and $f(U_{1} \cap M_{2})$ and $U \cap
f(V) = (N_{1} \cap U) \cup f (U_{1} \cap M_{2}) \cup f(U_{2}\cap
M_{2})$ is not a stratification.  

\begin{proposition}\label{chap2-prop2} %pro 2
  If $V$ is a $\mathbb{C}$- analytic set and $V = \cup M^{i}_{r}$ is a
  strict partition into manifolds there exists a narrow stratification
  $V = \cup S^{k}_{\mu}$ which is a refinement of $(M^{i}_{r})$.   
\end{proposition}

\begin{proof} %pro 
  We shall assume that there exist integers $n_{0} > n_{1} > . >n_{k}$
  with the following properties 
  \begin{enumerate}[(1)]
  \item for every $i\le k$, there exists an analytic set $V_{i+1}
    \subset V $ with $V_{0} = V$, $\dim. V_{i+1} < n_{i}$ and $\dim
    V_{i+1} = n_{i+1}$ if $i < k$ with  

  \item $V - V_{i+1} = \bigcup \limits _{j=o}^{i}  S^{n_{j}}_{\nu}$
    where $(S^{n_{j}}_{\nu})$ is a locally finite family of connected
    manifolds and $ S^{h}_{r} \cap S^{k}_{s} = \phi $ if $(h,r) \neq
    (k,s).\overline{S}^{h}_{r}$ and  $\dot{S}^{h}_{r}$ are analytic sets for
    $h \ge n_{i}$ and if for 
    $h\ge k\ge n_{i}$ and $U$ open, $S^{h}_{r,r'}$, $S^{k}_{s,s'}$ are
    connected components of $S^{h}_{r} \cap U$ and $S^{k}_{s} \cap U$
    respectively, then $S^{h}_{r,r'} \cap S^{k}_{s,s'} \neq \phi
    \Rightarrow S^{h}_{r,r'} \subset S^{k}_{s,s'}$,
 
  \item for $h \ge n_{i}, S^{h}_{r} \cap \overline{M}^{k}_{s} \neq \phi
    \Rightarrow S^{h}_{r} \subset \overline{M}^{k}_{s}$. 
  \end{enumerate}
  For $k = 0$, the above statement is trivial. Assuming that the result
  holds for $k = r-1$ we shall prove it for $k =r$.      
\end{proof}  

Let\pageoriginale $\dim V_r = n_r < n_{r-1}$ and let $M^n r = \cup
M^{n_r}_\nu$  be 
the set of simple points of $V_r$ of dimension $n_r$, where
$M^{n_r}_\nu$ 
are its connected components. By Lemma \ref{chap2-lem3} above,
$\overline{M}^{n_r}_\nu$  and $V_n - M^{n_r}$ are analytic sets. We
define $W^i_{r+1}, i = 1,2$ as follows. 

$W^1_{r+1}$,  is the set of $ z\in \bar{M}^{n_{r}}$ such that 
there is a neighourhood $U$ of $z$ and irreducible component $\sum$ of
$U\cap \overline{S}^h_\nu$ with $h \geq n_r$ such that $0 \leq
\dim_z \sum \cap \overline{M}^n _r < n_r$; $W^2_{r+1}$ is the set of $
z \epsilon \overline{M}^{n_r}$ such that there is a neighbourhood $U$ of
$z$ and an irreducible component $\sum_1$ of $U \cap
\overline{M}^k_\mu, k$ and $\mu$ arbitrary, such that $0 \leq \dim_z U
\cap \sum_1 \cap \overline{M}^{n_r}< n_r$.  

Let $W_{r+1} = W'_{r+1} \cup W^2_{r+1}$. Since $(M^k_\mu)$ and
$(S^h_\nu)$, $h > n_r$ are locally finite, $W'_{r+1}, W^2_{r+1}$ and
hence $W_{r+1}$ are anaytic sets. More over by Proposition \ref{chap1-prop2} of
Chapter \ref{chap1}, $\dim W_{r+1}< n_r$. Hence $S^{n_r}_\nu = M^{n_r}_\nu -
W_{r+1}$ are connected manifolds and $\overline{S}^{n_r}_\nu =
\overline{M}^{n_r}_\nu$ and $\dot{S}^{n_r}_\nu = (W_{r+1}\cap
\bar{M}^{n_r}_\nu) \cup \dot{M}^{n_r}_\nu$ are analytic sets. Also if
in an open set $U$ we have $U \cap S^{n_r}_{\nu,\nu'} \cap
\overline{S}^k_{\mu,\mu'}\neq \phi$ for $k \geq n_r$ then $\dim U \cap
\overline{S}^{n_r}_{\nu,\nu'} \cap \overline{S}^k_{\mu,\mu'} = n_r$ by
definition of $W'_{r+1}$. Hence $S^{n_r}_{\nu,\nu'}\subset
\overline{S}^k_{\mu,\mu'}$ i.e. the property $(2)$ is satisfied for $k
= r$. Moreover if $S^{n_r}_\nu \cap \overline{M}^{k}_\mu \neq \phi$
for some $k$, it follows in the same way from the definition of
$W^2_{r+1}$, that $\dim \overline{S}^{n_r}_\nu \cap \overline{M}^k_\mu
= n_r$ and hence $S^{n_r}_\nu \subset \overline{M}^k_\mu$. Hence we
prove by induction the existence of a sequence $n_0 > n_1 > \cdots >
n_k$ satisfying the above three properties and hence there is a norrow
stratification $V = \cup S^k_\mu$ which in fact satisfies the
condition of Lemma $4$ and hence it is a refinement of $(M^k_\mu)$.  

\begin{remark*} % rem 3
  \begin{enumerate}[(3)]
    \setcounter{enumi}{2}
  \item Let $\wedge_1 ,\ldots, \wedge_k$ be strict partitions of
    $\nu, \wedge_i$ given by $V =\bigcup\limits_{h,\nu}M^{i,h}_\nu$
    for each $i$. Then there exists a stratification $V = \cup
    S^k_\mu$\pageoriginale 
    of $V$ which is a refinement of $\wedge_i$ for each $i$. In the
    above proof we have only to change $W^2_{r+1}$ to $\bigcup \limits
    ^i_{i=1}{W'}^i_{r+1}$, where ${W'}^i_{r+1}$ is the set of points $z$
    in $\overline{M}^{n_r}$ for which there is a neighbourhood $U_z$
    and a connected $\sum$ component of some $U_z \cap
    \overline{M}^{i,h}_\nu$ such that $0 \leq \dim_z \sum \cap
    \overline{M}^{n_r} < n_r$. 

\item[(4)] Proposition \ref{chap2-prop2} can also be proved without using Lemma
    \ref{chap2-lem4}. We have only to change $W^2_{r+1}$ to
    $W'^{2}_{r+1} = \bigg\{z 
    \epsilon \overline{M}^{n_r}\mid$ There is a neighbourhood $U$ of
    $z$ and a component $\sum_1$ in $U$ of some
    $\overline{M}^k_\mu$such that $0\leq \dim_z \overline{M}^{n_r}
    \cap \sum_1 < n_r$ or a comnonent $\sum'_1$ in $U$ of some
    $\dot{M}^k_\mu$ such that $0 \leq \dim_z \sum'_1 \cap
    \overline{M}^{n_r} < n_r\bigg\}$. Then it follows immediately that
    if $V = \bigcap S^k_\nu$ is the stratification obtained as in the
    proof of Proposition \ref{chap2-prop2}, $S^k_\nu \cap M^h_\mu \neq
    \phi \Rightarrow S^k_\nu \subset M^h_\mu$ 
  \end{enumerate}
\end{remark*}

\begin{lemma}[Whitney]\label{chap2-lem5} %Lem 5
  Let $V$ be an analytic set of constant
  dimension $p, V \subset \mathbb{C}^n$ and let $a \epsilon V$. Then
  there exists a neighbourhood $U$ of a and finite number of vector
  fields $v^1,\ldots,v^q$ defined on $U$ such that 
  \begin{enumerate}[\rm(i)]
  \item $v^k(z) = 0$, $1 \leq k \leq q$, if $z$ is a singular point of
    $V \cap U$ and 
  \item $v^1(z),\ldots, v^q(z)$ span the tangent space $T(V,z)$ if $z$
    is a simple point of $V\cap U$. We give here two proofs of this
    lemma. 
  \end{enumerate}
\end{lemma}

\noindent
\textbf{1st Proof.}\, By Cartan's coherence theorem, there exists a
neighbourhood $U$ of a and a finite number of holomorphic functions
$f_1,\ldots,f_q$ such that the germs of $f_1,\ldots,f_q$ at any point
$b$ in $U$, generate the ideal of germs of holomorphic functions at $b$,
vanishing on $V_b$. It follows from Lemma \ref{chap2-lem1} that $z$ in $V$ is a
simple point of $V$ if and only if rank $\left(\dfrac{\partial
  f_i}{\partial z_j}\right)_z = n - p = r$.\pageoriginale 

In what follows $\lambda=(\lambda_1,\ldots, \lambda_n)$, $1 \leq
\lambda_1 < \cdots < \lambda_r \leq n~ \nu = (\nu_1,\ldots, \nu_r)$, $1
\leq \nu_{1} < \nu_2 < \cdots < \nu_r \leq r$, and $\mu=(\mu_1,\ldots,
\mu_{r+1})$, $1\leq \mu_1 < \mu_2 < \cdots < \mu_{r+1} \leq n$. Also, we put
$D_{\lambda \nu} = \det \left[\left(\dfrac{\partial f \lambda_j}{\partial z
    \nu_i}\right)\right]$ and $\mu^{(i)} = (\mu_1,\ldots,\hat{\mu}_i,\ldots,
\mu_{r+1})$ [a hat over a term means that the term is omitted]. 

We now define vectors $v^{\lambda \mu}$ as follows
\begin{alignat*}{4}
  &&v^{\lambda \mu} & = (v_k^{\lambda \mu})\epsilon \mathbb{C}^n
  \quad\text{where}\\ 
  &&v^{\lambda \mu}_k & = 0\quad\text{if}\quad k \notin \mu\\
  \text{and} &\hspace{2cm}& & = (-1)^{i-1} D_{\lambda
    \mu}(i)\quad\text{if}\quad k = \mu_i.\hspace{3cm} 
\end{alignat*}

Then obviously $v_{(z)}^{\lambda \mu}  = 0$ for any $z$ where rank
$\left(\dfrac{\partial f_i}{\partial z_j}\right)_z < r$ i.e. for any $z$ in the
set of singular points of $V \cap U$.  

We shall prove the condition (ii) for the vectors $(v^{\lambda
  \mu})$. Let $z$ be a simple point of $V$. Then $v^{\lambda
  \mu}\in T(V,z)$ if and only if $\langle df_j(z), v^{\lambda
  \mu}(z)\rangle = 0$ for $1 \leq j \leq q$. 

But $\langle df_j(z), v^{\lambda \mu}(z)\rangle$
\begin{align*}
  & = \sum^n_{k=1} \left(\frac{\partial f_j}{\partial z_k}\right) (z)
  v_k^{\lambda \mu}(z)\\ 
  & = \sum^{r+1}_{i=1} \dfrac{\partial f_j}{\partial z \mu_i}
  (z)(-1)^{i-1} D_{\lambda \mu} (i)  
\end{align*}
and this is nothing but the determinant

$$
\begin{bmatrix}
  \dfrac{\partial f_j}{\partial z \mu_1} & \dfrac{\partial f_j}{\partial
    z \mu_2} \cdots & \dfrac{\partial f_j}{\partial z \mu_{r + 1}}\\[10pt] 
  \dfrac{\partial f\lambda_1}{\partial z \mu_1} & \dfrac{\partial
    f\lambda_1}{\partial z \mu_2} \cdots & \dfrac{\partial
    f\lambda_1}{\partial z \mu_{r+1}}\\[5pt] 
  \vdots & \vdots   &  \vdots\\[5pt]
  \dfrac{\partial f\lambda_r}{\partial z \mu_1} & \dfrac{\partial
    f\lambda_r}{\partial z \mu_2} & \dfrac{\partial
    f\lambda_r}{\partial z \mu_{r+1}}
\end{bmatrix}
$$\pageoriginale

If $j \in \lambda$ this is clearly zero and if $j \notin \lambda$ this
is the determinant of a submatrix of order $(r+1)$ of the matrix
$\left(\dfrac{\partial f_i}{\partial z_j}\right)$ and hence is zero. Thus
$(v^{\lambda \mu})\epsilon T(V,z)$ for each pair $(\lambda,\mu)$. It
now remains to prove that these vectors span $T(V,z)$. The dimension
of $T(V,z) = \dim_z V = p$. Now there exists a pair $(\lambda,\nu)$
and a neighbourhood $U'$ of $z$ such that $D_{\lambda \nu}(\zeta)
\neq 0$ for $\zeta \epsilon U'$. Then define for each $\rho \notin
\nu$, $\mu^\rho$ as the $(r+1)$-tuple which contains the integers
$(\nu_1,\ldots,\nu_r,\rho)$. Then $v_\rho^{\lambda \mu^\rho} = \pm D
_{\lambda \nu}$. Hence if $v^1,\ldots,v^p$ are the vectors defined by
$v^{\lambda \mu^\rho}$, $\rho \notin \nu$, $v^1,\ldots v^p$ are linearly
independent since their projections on $\mathbb{C}_{z \lambda'_1},\ldots
, z_{\lambda'_{n-r}}\lambda'_{i}\notin \nu$ are independent and hence span
$T(V,z)$. 

\medskip
\noindent
\textbf{2nd Proof (R. Narasimhan).} We use hero the properties of sheaves of
germs of holomorphic functions on a manifold. Let $V$ be an analytic
set of constant dimension in an open set $\Omega$ in
$\mathbb{C}^n$. Let $S \subset V$ be the set of singular points of
$V$. Let $F$ be the sheaf (on $\Omega$) of germs of holomorphic
mappings $g = (g_1,\ldots, g_n)$ into $\mathbb{C}^n$ such that a)
$g(y) = 0$ for $y \in S$ and b) $\sum
g_i(y)\dfrac{\partial}{\partial z_i}\in T(V,y)$ for $y\in
V-S$. We have only to prove that $F$ is coherant. By H. Cartan's coherence
theorem, i.e. Theorem \ref{chap1-thm6} of chapter \ref{chap1}, for any
$a \in V$, 
there is a neighbourhood $W(\subset \Omega)$ of $a$ and holomorphic
functions $f_1,\ldots f_r$ on $W$ such\pageoriginale that the germs of
$f_1,\ldots,f_r$ at any point $b$ in $W$ generate the  ideal $I_b$ of
$V_b$. Then for a holomorphic map $g:W \rightarrow \mathbb{C}^n$, 
$g_b \in F_b$ (the stalk of $F$ at b)for every $b$ if and only if
$(1)\sum \limits^n_{i=1} g_i(z)\dfrac{\partial f_j}{\partial z_i}
(z)=0, j=1,2,\ldots,r,$ if $z \epsilon V \cap W$ and $(2) g_i(z) = 0,
i = 1,2,\ldots, n$ if $z \epsilon S \cap W$. 

Now, if $\varphi_1,\ldots, \varphi_k$ are holomorphic, then the sheaf
of $(\alpha_1,\ldots,\alpha_k)$ such that $\sum \alpha_i \varphi_i
= 0$ on an analytic set $A$ is a quotient of the sheaf of relations
between $(\varphi_1,\ldots,\varphi_k, \psi_1,\ldots,\psi_{l})$ where the
$(\psi_j)$ generate the ideal sheaf of $A$ and so is coherent. Since
the intersection of finitely many coherent sheaves is again coherent,
our lemma follows. 

\begin{proposition}\label{chap2-prop3} %pro 3
  Let $V$ be an analytic set, $W$, a $\mathbb{C}$-analytic manifold
  and $f:V \rightarrow W$ a holomorphic map. Then there exists a narrow
  stratification $V = \cup S^k_\nu$ of $V$ such that if $f_{k,\nu}=f\mid
  S^k_\nu$, rank $df_{k,\nu}$ constant on $S^k_\nu$. 
\end{proposition}

Such a stratification will be called a stratification consistent with $f$.

\begin{proof} %pro 
  We assume that there exist integers $\dim V =n_0 > n_1 >\cdots >
  n_k$ such that for each $i \leq k$, there an analytic set $V_{i+1}
  \subset V$ with the following properties. 
\begin{enumerate}[(1)]
\item $\dim V_{i+1}< n_i$ and $\dim V_{i+1} = n_{i+1}$ if $i < k$.

\item $V - V_{i+1} = \cup S^h_\nu$, $h \ge n_i$, $S^h_\nu$ being
  connected manifolds of dimension $h$, $S^h_\nu \cap S^k_\mu = \phi$ if
  $(h,\nu) \neq (k,\mu)$ for which $S^h_\nu$, $\dot{S}^h_\nu$ are analytic sets;
  furthermore,  
  $$ 
  S^h_\nu \cap \overline{S}^k_\mu \neq \phi \Rightarrow S^h_\nu
  \subset \overline{S}^k_\mu. 
  $$

\item if\pageoriginale $f_{h,\nu}=f\mid S^h_\nu,$ rank of $df_{h\nu}$
  is constant an 
  $S^h_\nu$. The above statement is trivial for $k = 0$.  
\end{enumerate}

Assuming the existence of the above sequence for $k = r-1$, we shall
prove it for $k=r$. Let $\dim V_r = n_r$ and $M^{n_r} =$ the set of simple
points of $V_r$ of dimension $n_r$, and let $M^{n_r} - \cup
M^{n_r}_{\nu},M^{n_r}_\nu$ being the connected components of
$M^{n_r}$. By Lemma \ref{chap2-lem3} above,
$\overline{M}^{n_r}_\nu$, $\dot{M}^{n_r}_\nu$ and $V_r-M^{n_r}$ are
analytic sets. We defire $W^1_{r+1,\nu}$ and $W^2_{r+1,\nu}$ as
follows. $W^1_{r+1,\nu} = \bigg\{z\epsilon \overline{M}^{n_r}_\nu
\mid$  There exists a neighbourhood $U$ of $z$ and a connected
component $\sum$ of $S^k_\mu$ with $k \geq n_r$ such that $0 \leq
\dim_z U \cap \sum \cap \overline{M}^{n_r}_\nu < n_r \bigg\}$. 

Let $f^{n_r}_\nu = f \mid M^{n_r}_\nu$ and let the maximum rank of
$df^{n_r}_\nu=k_r$ on $M^{n_r}_\nu$. Then $W^2_{r+1,\nu} = \bigg\{z
\epsilon M^{n_r}_\nu \mid \text{rank} (df^{n_r}_\nu)(z)< k_r\bigg\}$. 

Clearly by the same argument as in Proposition \ref{chap2-prop2},
$W^1_{r+1,\nu}$ is 
an analytic subset of $\overline{M_\nu^{n_r}}$ and $\dim W^1_{r+1,\nu}
< n_r$. We shall now prove that $(W^2_{r+1,\nu}\cup M^{n_r}_\nu)$ and
hence $\overline{W^2}_{r+1,\nu}$ is an analytic set of dimension $<
n_r$. Since this problem is local we assume $V \subset
\mathbb{C}^n$. By Lemma \ref{chap2-lem5} above, for every $a \epsilon
\overline{M_\nu^{n_r}}$ there exists a neighbourhood $U \subset
\mathbb{C}^n$ and vector fields $v^1,\ldots,v^q$ on $U$ such that
$v^i(z) = 0$ for $z$ in $\dot{M}_\nu^{n_r} U$ and $(v^i)$ span
$T(M_\nu^{n_r}, z)$, if z is a simple point of $\overline{M_\nu^{n_r}}
\cap U$. Further, if $z$ is a simple point of $\overline{M}_\nu^{n_r}$
we may choose $U$ sufficiently small so that after a holomorphic
change of coordinates, if $f = (f_i)$ and $df (z)$, the transformation
$T(V,z)\rightarrow T(W,f(z))$. then if  
\begin{align*}
  \langle df (z), v^j(z) \rangle & = w^j (z)\quad \text{are vectors in T(W,
    f(z)), then}\\ 
  \text{rank} \quad(df_\nu^{n_r})(z) & = \text{dimension of the space
    spanned by } w^{j}(z). 
\end{align*}

Since\pageoriginale $(w^j_i(z))$ are holomorphic functions on $U$, there exist
holomorphic functions $h_1,\ldots, h_k$ on $U$ such that  
$$
U \cap W^2_{r+1,\nu} = \bigg\{z \epsilon M_\nu^{n_r}\big| h_1(z)=
\cdots = h_k(z) = 0\bigg\} 
$$
where $(h_i)$ are holomorphic on $U$. Let $U \cap \dot{M}_\nu^{n_r} =
\bigg\{z \epsilon U \mid g_1 (z)=\ldots = g_l(z) = 0\bigg\}$ where
$(g_i)$ are holomorphic on $U$. Then  
$$
U \cap (W^2_{r+1,\nu} \cup \dot{M}_\nu^{n_r}) =\bigg\{z \epsilon
\overline{M}_\nu^{n_r}\cap \big| h_i(z) g_j(z)=0, i \leq k,i \leq l
\bigg\} 
$$
 and therefore $W^2_{r+1,\nu}\cup \dot{M}_\nu^{n_r}$ is analytic. Also
 $\dot{M}_\nu^{n_r}$ is an analytic set and hence
 $\overline{W^2_{r+1,\nu}}= clos (W^2_{r+1,\nu}\cup \dot{M}^{n_r}_\nu
 -\dot{M}^{n_r}_\nu)$ is an analytic set by Proposition
 \ref{chap1-prop4} of Chapter 
 \ref{chap1}. Let $W_{r+1,\nu}=W^1_{r+1,\nu}\cup \overline{W^2}_{r+1,\nu}$ and
 let $S^{n_r}_\nu =M^{n_r}_\nu - W_{r+1,\nu}$. Then
 $\overline{S^{n_r}_\nu} = \overline{M^{n_r}_\nu}$ and
 $\dot{S}_\nu^{n_r} = \dot{M}_\nu^{n_r} \cup W_{r+1,\nu}$ are analytic
 sets and by the definition of $W^1_{r+1,\nu}$, condition $(2)$ of the
 induction hypothesis is satisfied. Further it follows from the
 definition of $W^2_{r+1,\nu}$ that rank $df_\nu^{n_r}$= constant on
 $S_\nu^{n_r}$ where $f^{n_r}_\nu = f\mid S^{n_r}_\nu$ and hence the
 proposition is proved by induction. 
\end{proof}

\setcounter{remark}{5}
\begin{remark}\label{chap2-rem6} %rem 6
  If $V$ is an analytic set, $W$ an analytic manifold and $f:V
  \rightarrow W$ a holomorphic map and if $b \in W$, $Z = V \cap f^{-1}
  (b)$ is an analytic set. Moreover if $V = \cup M^i_\nu$ is a
  stratification of $V$. consistent with the restrictions of $f$, then
  the connected components $S^i_{\nu,j}$ of $Z \cap M^i_\nu$ form a
  strict partition of $Z$. 
\end{remark}

\begin{proof}
By the constant rank therorem in Chapter \ref{chap1}, $S^i_{\nu,j}$ is
a manifold. 

Let $Z \cap \overline{M}^i_\nu = \cup V_{\alpha}$, $V_\alpha$ being
irreducible components of $Z \cap \overline{M}_\nu^i$.  

Then if $V_{\alpha} \cap S^i_{\nu,j} \neq \phi$, then $V_{\alpha} \cap
\dot{M}^i_\nu$ is proper analytic subset of\pageoriginale $V_{\alpha}$
and $V_{\alpha}\cap \dot{M}^i_\nu = V_{\alpha} - V_{\alpha} \cap \dot{M}^i_\nu$ is
connected and dense in $V_{\alpha}$. Hence $\overline{S}^i_{\nu,j}=
\bigcup\limits_{V_{\alpha}} \cap S^i_{\nu,j}\neq \phi V_{\alpha}$ is an
analytic set and so is $\dot{S}^i_{\nu,j}=\overline{S}^i_{\nu,j}\cap
(\bigcup \limits_{k \neq j} \overline{S}^i_{\nu,k} \cup
\dot{M}^i_\nu)$. But the $S^i_{\nu,j}$ do not, in general, form a
stratification of $Z$, as shown by the following: 
\end{proof}

\setcounter{example}{2}
\begin{example}\label{chap2-exam3} %Exa 3
  Let $V$ be the analytic set in $ \mathbb{C}^3_{xyt}$ given by $V =
  \bigg\{z \epsilon  \mathbb{C}^3_{xyt}\big| y = x^2
  \bigg\}$. Consider the stratification $M^1 =  \mathbb{C}_t$ and $M^2
  = V -  \mathbb{C}_t$. It is consistent with the restrictions of the
  holomorphic map $f:  \mathbb{C}^3_{xyt}\rightarrow  \mathbb{C}$
  given by $f(z) = vt. f^{-1}(0) = Z =  \mathbb{C}_t \cup (M^2 \cap
  t=0)$ and this is clearly not a stratification. 
\end{example}

\begin{definition}\label{chap2-defin2} %Def 2
  If $V$ is an analytic set in $\Omega \subset  \mathbb{C}^n$, a
  function $f:V \rightarrow  \mathbb{C}$ is said to be strongly
  holomorphic if for every $a \epsilon V$, there is a neighbourhood $U
  \subset  \mathbb{C}^n$ and a holomorphic function $F$ on $U$ such
  that $F \mid V \cap U = f \mid V \cap U$.  
\end{definition} 

\begin{proposition}\label{chap2-prop4} %Pro 4
  Let $V$ and $W$ be analytic manifolds and $f:V \rightarrow W$ a
  strongly holomorphic map. Then there exists a stratification $V =
  \cup S^k_\mu$ of $V$ such that rank df = constant on $S^k_\mu$.  
\end{proposition}

\begin{proof} %pro 
  Let $L_\circ = V$ and let max rank df on $V=r_\circ$. Let $L_1 =$
  the set of points of $V$ such that rank $df < r_\circ$ on
  $L_1$. Then if $\dim V = n_\circ, L_1$ is an analytic set of
  dimension $n_1 < n_0$. Let max rank $df = r_1 < r_\circ$ on $L_1$
  and so on. We get a finite sequence 
  $$
  V = L_0 \supset L_1 \supset \cdots \supset L_k, \dim L_k = 0.
  $$
  max rank $df = r_i $  on $ L_i, \dim L_i = n_i$,
  $$
  n_0 >n_1 > \cdots > n_k = 0, r_0 > r_1 > \cdots > r_k.
  $$\pageoriginale 
  Let $\wedge_i$ be a strict partition of $V$ as follows.


$\overline{(V-L_i)}$ is an analytic set and $L_i$ is an analytic
set. Let $\overline{(V-L_i)} = \cup M^i_\nu$ and $L_i = \cup S^i_\nu$
be the respective stratifications of $\overline{(V-L_i)}$ and
$L_i$. Then $V=(\cup M^i_\nu - L_i)\cup(\cup S^i_\nu)$ is a strict
partition of $V$ and we define $\wedge_i$ to be that strict
partition. Then, by Remark \ref{chap2-rem2} following Proposition
\ref{chap2-prop2} above, we 
have a stratification $V = \cup S^k_\mu$ which is a refinement of each
$\wedge_i$. If $S^k_\mu \subset L_i$ and if $S^k_\mu \cap L_{i+1} \neq
\phi$, then $S^k_\mu \subset L_{i+1}$ since the stratification is a
refinement of $\wedge_{i+1}$. If $S^k_\mu \subset L_i$. and $S^k_\mu
\cap L_{i+1} = \phi$. rank $df \leq r_i$ on $S^k_\mu$ and since
$S^k_\mu \cap L_{i+1} = \phi$, rank $df = r_i$ on $S^k_\mu$, i.e. $V =
\cup S^k_\mu$ is a stratification with the required properties. 
\end{proof}

\begin{remark}\label{chap2-rem7} %rem 7
  The above proposition can also be proved directly by changing
  $W^2_{r+1,\nu}$ in the proof of Proposition \ref{chap2-prop3}, to
  $$
  \displaylines{\hfill 
    W^{'2}_{r+1,\nu}  = \bigg\{z \epsilon \overline{M_\nu^{n_r}}\big|
    ~\text{rank}~ (df)(z) <  \max ~\text{rank}~(df) ~\text{on}~
    \overline{M_\nu^{n_r}}\bigg\}\hfill \cr 
    \text{and then by defining}\hfill \cr 
    \hfill W_{r+1,\nu}  = W'_{r+1,\nu} \cup
    W'^2_{r+1,\nu} ~\text{and} S^{n_r}_r = M^{n_r}_\nu
    -W_{r+1,\nu}.\hfill }
  $$
\end{remark}

We expect to prove in the next chapter the following important theorem
of Whitney. 

\begin{theorem*} %the 
  Let $V$ be an analytic set of dimension $k$ and $M$ manifold of
  dimension $m<k$ such that $M \subset V$ and $\overline{M}$ is an
  analytic set. Then there exist analytic sets $W_a, W_b$ of
  dimensions $< m,W_a, W_b \subset \overline{M}$ such that if $z
  \epsilon M-W_a$. the pair $(M,V)$ satisfies the condition (a)
  [stated in Chapter \ref{chap1}] of Whitney at 
  $az$\pageoriginale and if a $z \in M- W_{b}$, the pair $(M,V)$ satisfies the
  condition $(b)$ of Whitney at $z$. 
\end{theorem*}

If we assume the above theorem it is easy to prove the Whitney's
theorem stated in Chapter \ref{chap1}. We use the same reduction process as
above. 

\begin{theorem*} %The 
  For an analytic set $V$, there exists stratification $V = \cup
  S^{k}_{\mu}$ which is $(a)$ and $(b)$ regular. 
\end{theorem*} 

\begin{proof}
  We prove by induction the existence of $a$ someone of positive integers
  $$
  n_{0} > n_{1}> \dots > n_{k}
  $$
  such that for each $i\leq k$, there exists an analytic set $V_{i+1}$
  in $V$ such that $\dim V_{i+1}<
  n_{i}$ and $\dim V_{i+1} = n_{i+1}$ if $i< k$. which
  have. further following properties. 
\begin{enumerate}[(1)]
\item $V-V_{i+1} = \cup S^{k}_{\mu}$, $k \geq n_{i}$ where 
  $S^{k}_{\mu}$ are connected manifolds with $S^{k}_{\mu} \cap
  S^{h}_{\nu} =\emptyset$ if $(k \mu) \neq (h, \mu)$, $S^{k}_{\mu}\cap
  \bar{S}^{h}_{\nu}\neq \emptyset \Rightarrow
  S^{k}_{\mu} \subset S^{-h}_{\nu}$,  and  $\bar{S}^{k}_{\mu}$
    and $\dot{S}^{k}_{\mu}$ are analytic sets.
 
\item For $h>k\geq n_{i}$, if $S^{h}_{\nu} \subset \bar{S}^{k}_{\mu}$,
  the pair $(S^{h}_{\nu}, \bar{S}^{k}_{\mu})$ is (a) and (b) regular. 
\end{enumerate}

Assuming the existence of such $a$ sequance for $K=r-1$. let the
dimension of $V_{r} = n_{r}< n_{r-1}$. Let  $M^{nr} = \cup
M^{n_r}_{\nu}$ be the set of simple points of $V_{r}$ of 
  dimension  $n_{r}$, $M^{n_r}_{\nu}$ being its connected of
components. Define the set $W^{1}_{r+1,\nu}$ as in proposition
\ref{chap2-prop3}. Now if 
$M^{nr}_{\nu}\subset \bar{S}^{k}_{\mu}$, there are analytic sets  $W^{k,
  \mu}_{a, \nu}$  and  $W^{k,\mu}_{b,\nu}$ in
$\overline{M^{n_r}_{\nu}}$ such that for any $z\in M^{n_r}_{\nu} - W^{k,
  \mu}_{a,\nu}$, $(M^{n_r}_{\nu}, \bar{S}^{-k}_{\mu})$ is (a) regular
at $z$ and\pageoriginale for  $z\in
M^{n_r}_{\nu} - W^{k, \mu}_{b, \nu}(M^{n_r}_{\nu}, \bar{S}^{k}_{\mu})$  is
(b) regular at $z$. Let $W^{2}_{r+1,\nu}= \bigcup\limits_{M^{n_r}_\nu
  \subset \bar{S}^k_\mu}  (W^{k,\mu}_{a,\nu} \subset W^{k,
  \mu}_{b,\nu})$. Then  $W^{2}_{r+1, \nu}$ is an analytic set of
dimension $< n_{r}$.  Let $W_{r+1, \nu} = W^{1}_{r+1, \nu} \cup
W^{2}_{r+1, \nu}$  and  $S^{n_r}_{\nu} = M^{n_r}_{\nu} - W_{r+1,
  \nu}$. Then clearly conditions (1) and (2) of the induction
hypothesis are satisfide and the theorem is proved by induction. 
\end{proof}

\setcounter{prop'}{2}
\begin{prop'} % pro 3
In proposition \ref{chap2-prop3} we can form a stratification which is also a
  whitney stratification. 
\end{prop'}

We have only to take, for $W_{r+1, \nu}$, $W_{r+1, \nu} \cup W^{2}_{r+1,
  \nu}$,  where  $W^{2}_{r+1, \nu}$ is as in the above proof. 
