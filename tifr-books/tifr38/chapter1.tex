\chapter{Preliminaries}\label{chap1} %% chap1

\markright{\thechapter. Preliminaries}

In\pageoriginale this course of lectures we shall deal with various
decompositions of 
$\mathbb{C}$-analytic set into manifold and then with some of their
applications. 

We shall first state some definitiens and a few theorem without\break proofs
on holomorphic funtionns and $\mathbb{C}$-analytic sets which we shall
use in what follows. 

\begin{definitions}\label{chap1-defins1} 
  Let $\Omega$ be an open set in $\mathbb{C}^n$. $A$ complex valued
  function $f$, defined on $\Omega$ is side to he holomorphic on
  $\Omega$, if for every $\zeta$ in $\Omega$, there exists
  $(\rho_{i}), \rho_{i} > 0, 1 \leq i \leq n, a_{\alpha} \in
  \mathbb{C}^n$, such that 
  \begin{align*}
    f(z) &= f(\zeta) + \sum\limits_{\mid \alpha \mid \geq 1}
    a_{\alpha}((z-\zeta))^{\alpha}~\text{for}~ \mid z_{i}- \zeta_{i}
    \mid < \rho_{i}\\ 
    \text{ \rm where }~~\alpha &= (\alpha_{1}, \ldots , \alpha_{n})
    \in \mathbb{N}^{n},\\ 
    ((u))^{\alpha} &= u^{{\alpha}_{1}}_{1}, \dots , u^{{\alpha}_{n}}_{n},\\
    \mid \alpha \mid &= \alpha_{1} + \alpha_{2} + \ldots + \alpha_{n}. 
  \end{align*}

We denote the set of holomorphic functions on $\Omega by \mathscr{O} \Omega$
\end{definitions}

\begin{theorem}[Hartogs]\label{chap1-thm1} % the 1
  A complex valued function $f$ is holomorphic
  on $\Omega$ if and only if the partial derivatives $\dfrac {\partial
    f}{\partial z_{i}}$ exist at each point of $\Omega$. 
\end{theorem}

If $V$ is an analytic manifold and $a\in V$, we denote by $T(V,a)$
the tangent space to $V$ at a. 

\setcounter{definition}{1}
\begin{definition}\label{chap1-defin2} %def 2
  Let\pageoriginale $V^{n}$ and $W^{m}$ be two $\mathbb{C}$-analytic manifolds and
  $f : V^{n} \rightarrow W^{n}$ an analytic map. For $a \in V^{n}$,
  let $(df)(a)$ denote the linear map $T(V^{n},a) \rightarrow T(W^{m},
  f(a))$, Then we define rank $(df)(a)$ as the dimension of the image
  of $T(V^{n}, a)$ by this map. 
\end{definition}

\begin{remark*}%rem 
  If $(x_{1}, \ldots , x_{n})$ denote local coordinates in a
  neighbourhood $U$ of a and if $f(x) = (f_{1}(x) , \ldots ,
  f_{m}(x))$ for $x \in U$, in local coordinates in a
  neighbourhood $U$ of $f(a)$, then rank $(df)(a) = \text{rank}$ of
  the matrix $(\dfrac{\partial f_{i}}{\partial x_{j}}
  (a))_{\substack{{1 \leq i \leq m}\\ {1 \leq j \leq n}}}$ and it is
  independent of the coordinate neighbourhoods chosen. We now state the
  important. 
\end{remark*}

\begin{theorem}[Constant rank theorem]\label{chap1-thm2} %the 2
  Let $V^{n} \text{ and } W^{m}$ be two $\mathbb{C}$ analytic
  manifolds and $f : V^{n} \rightarrow W^{n}$ be a holomorphic
  map. Let the rank $(df)(x) = r$, a constant, for $x \in$ an open
  set $\Omega \subset V^{n}$. Then for every $a \in \Omega$, there
  exist neighbourhoods $U$ of $a$, $V$ of $f(a)$, open cubes $Q_{1}
  \subset \mathbb{C}^{n},Q_{2} \subset \mathbb{C}^{m}$ and
  biholomorphic maps $u:U \rightarrow Q_{1}, v:V \rightarrow Q_{2}$
  such that if $g = v ~o~ f~ o~ u^{-1}$, we have $g(x_{1}, \ldots ,
  x_{2}, 0, \ldots , 0)$. 
\end{theorem}

Let $\Omega$ be an open set in $\mathbb{C}^{n}$ such that $0 \in
\Omega$. Then we denote the inductive limit,  $
\underset{\Omega}{\varinjlim} \mathscr{O}^{n}_{\Omega}$, by
$\mathscr{O}^{n}_{o}$ or by $\mathscr{O}^{n}$. It is clear that
$\mathscr{O}^{n}$ is a ring and we call it the ring of germs of
holomorphic functions at $0$. 

We shall assume the following properties of $\mathscr{O}^{n}$.
\begin{enumerate}
\item $\mathscr{O}^{n}$ is isomorphic to the ring of convergent power
  series in $n$ variables with complex coofficients. 
\item $\mathscr{O}^{n}$ is a local ring.
\item $\mathscr{O}^{n}$\pageoriginale is an integral domain.
\item $\mathscr{O}^{n}$ is a noetherian ring.
\item $\mathscr{O}^{n}$ is factorial.
\end{enumerate}

\begin{definition}\label{chap1-defins3}\footnote{We use the term
    analytic set for an 
    analytic subvariety of 
    an analytic manifold and the term analytic space for a space that
    is locally an analytic set.} Given a $\mathbb{C}$ -analytic
  manifold $M^{n}$ a subset $V$ of $M^{n}$ is defined to be a
  $\mathbb{C}$-analytic set if for every $a \in M^{n}$ there exists a
  neighbourhood $U$ in $M^{n}$ and a finite number of holomorphic
  functions $\bigg\{ f_{i} \bigg\}, 1 \leq i \leq m$ on $U$ such that
  $U \cap V = \{ z \in U \mid f_{i} (z) = 0, 1 \leq i \leq m\}$. 

  (4). A point a in $V$ is said to be simple if there exists a
  neighbourhood $U_{a}$ of a such that $U_{a} \cap V$ is an analytic
  submanifold of $U_{a}$. 
\end{definition}

\begin{remark}\label{chap1-rem1}  %rem 1
  An analytic set on a $\mathbb{C}$ -analytic manifold is closed. 
\end{remark}

Unless otherwise stated, in what follows, an analytic manifold and an
analytic set will mean a $\mathbb{C}$ -analytic manifold and a
$\mathbb{C}$ -analytic set respectively.  

\begin{notation*}
  If  $z = (z_{1}, \ldots , z_{n})$ is a point in $\mathbb{C}^{n}$, $z'$
  will denote the point\break $(z_{1}, \ldots , z_{n-1})$ in
  $\mathbb{C}^{n-1}$. A poly-disc $D^{n}$ in $\mathbb{C}^{n}$
  will be $D_{1}\times \cdots \times D_{n}$ where $D_{i}$ are
  discs in $\mathbb{C}$. $\mathscr{M}^{n}$ will denote the maximal
  ideal (i.e. the 
  ideal of germs vanishing at $0$) of $\mathscr{O}^{n}$ and
  $\mathscr{O}^{n-1}$ will denote the ring of germs of holomorphic
  functions at $0'$  in $\mathbb{C}^{n-1}$. We identify
  $\mathscr{O}^{n-1}$ with a subring of $\mathscr{O}^{n}$. 
\end{notation*}

\setcounter{definitions}{4}
\begin{definitions}\label{chap1-defins5} 
  A\pageoriginale distiniuished polynomial  in $z_{n}$ of degree $n$
  is a polynomial 
  $z^{p}_{n} + \sum\limits_{k=1}^{p} a_k (z')z_{n}^{p-k} $, where
  $a_{k}(z')$ are holomorphic functions in $z'$ on an open
  neighbourhood of $0' \text{ in } \mathbb{C}^{n-1} \text{ and }
  a_{k}(0') = 0$.  
\end{definitions}

\begin{theorem}[Weierstrass preparation theorem]\label{chap1-thm3}  
  Let $f \in \mathscr{M}^n$, $f\neq 0$. Then  
  \begin{enumerate}
  \item There exists a basis $(z_{1}, \ldots , z_{n})$ of 
    $\mathbb{C}^{n}$ such that $f(0', z_{n})$ does not vanish identically
    in any neighbourhood of $z_{n} = 0$ in $\mathbb{C}$.
 
  \item With respect to any basis satisfying condition (1) above, there
    exists a unique distinguished polynomial $P$ in $\mathscr{O}^{n-1}
    [z_{n}]$ such that $f = gP$ for some $g$ in $\mathscr{O}^{n}$ and
    $g \notin \mathscr{M}^{n}$.
 
  \item If $f(z) =  \sum \limits^{\infty}_{1} a_{k} (z)$ in a
    neighbourhood of $0$, where $a_{k}(z)$ are homogeneous polynomials
    of degree $k$, and if $p$ is the least 
integer such that $a_{p}(z)\not\equiv 0$, then $p$ is the minimum
degree of a distinguished polynomial $P$ for which 
    $$
    f = g P \text{~ where~ } g \in \mathscr{O}^{n} \text{~ and~ } g
    \not\in \mathscr{M}^{n}. 
    $$
  \end{enumerate}
\end{theorem}

\begin{theorem}[The division theorem]\label{chap1-thm4} 
  Let $f \in \mathscr{M}^{n}$ and suppose the basis for $\mathbb{C}^{n}$ so
  chosen that the condition (1) of Theorem 3 above is satisfied. Then
  if $f = u P$, where $P$ is a distinguished polynomial of degree $p$
  in $\mathscr{O}^{n-1}_{z'}[z_{n}]$ and $u \in
  \mathscr{O}^{n}$, $u \not\in \mathscr{M}^{n}$, then for any $g$ in
  $\mathscr{O}^{n}$, there exists $h$ in $\mathscr{O}^{n}$ and
  $r(z',z_{n})$ in $\mathscr{O}^{n-1}_{z'} [z_{n}]$, with
  degree of $r < p$ such that 
  $$
  g = h f + r
  $$
  and\pageoriginale the $h$ and $r$ are unique.
\end{theorem}

\setcounter{definition}{5}
\begin{definition}\label{chap1-defin6} 
  An analytic set $V$ is said to be irreducible if, $ V = V_{1} \cup
  V_{2}$, where $V_{1} \text{ and } V_{2}$ are analytic sets, implies
  either $V = V_{1}$ or $V = V_{2}$. 
\end{definition}

Let $V$ be an analytic set $\subset M, M$ being an analytic
manifold. Then the germ of $V$ at a point $a \in M$ is defined to be
$\underset{\Omega}{\varinjlim} V \cap \Omega$ where $\Omega$ is a
neighbourhood of $a$ in $M$. The germ at a of an analytic set $V$ is
said to be irreducible if a has a fundamental system of neighbourhoods
$U$ such that $U \cap V$ is irreducible. The following Proposition is
an easy consequence of the property $4$ of $\mathscr{O}^{n}$. 

\setcounter{proposition}{0}
\begin{proposition}\label{chap1-prop1}  %% 1
  The germ of an analytic set $V$ at a can be written uni\-quely as 
$V_{a} = \bigcup\limits_{1 \leq i \leq r} V_{ia}$, where $V_{ia}$ are
  irreducible germs of analytic sets $V_{ia}\not\subset
  \bigcup\limits_{j \neq i} V_{ja}$ for any $i$. 
\end{proposition}

\setcounter{remarks}{1}
\begin{remarks}\label{chap1-rems2} 
  If $I$ is the ideal in $\mathscr{O}_{a}$ of germs of holomorphic
  functions vanishing on $V_{a}, I$ is prime if and only if $V_{a}$ is
  irreducible.  
\end{remarks}

3.~ Here we give an example of an irreducible analytic set $V$ and a
point $a \in V$ such that $V_{a}$ is reducible. Let $V = \{ z \in
\mathbb{C}^{2}_{xy} \mid x^{3} + y^{3} - xy = 0 \}$. Then since the
set of simple points of $V$ is connected, $V$ is irreducible. Consider
the points $x = x(t) = \dfrac{t}{1+t^{3}}$ and $y = y(t) =
\dfrac{t^{2}}{1+t^{3}}$ for $t$ sufficiently small. Then the points
$(x(t), y(t))$ are in $V$. Further, $f(x,t) \equiv  (1+ t^{3}) x(t) - t =
0$ \text{ gives } $f_{t}(0,0) = -1$ and hence  by the implicit
function theorem the equation $x =   \dfrac{t}{1+t^{2}}$ can be solved
for $t$, for\pageoriginale sufficiently small $x$, i.e. there exists
$\varepsilon > 0$. such that for $|x|< \varepsilon, t = t(x)$ is an analytic function
of $x$. Thus the analytic set $V_{1}$ defined by $V_{1} = \left\{ z \in
\mathbb{C}^{2}_{xy}\big| |x| < \varepsilon ,y =
\dfrac{[t(x)]^{2}}{1+[t(x)]^{3}}\right\}$ is contained in $V$ and
similarly $V_{2} = \left\{ z \in \mathbb{C}^{2}_{xy}\big| |x| <
\varepsilon' ,y =  \dfrac{[t'(x)]^{2}}{1-[t'(x)]^{3}}\right\}$ where
$t'(x)$ is a solution of $(1-t^{3})x+t=0$, in $V$. Hence there is a
neighbourhood $U$ of $0$ such that $U \cap V = (U \cap V_{1}) \cup (U
\cap V_{2})$ and thus $V_{o}$ is reducible. 

We now recall the important theorem of local representation of an
analytic sec and some of its consequences that we shall need later. 

\begin{theorem}\label{chap1-thm5}  % the 5
  Let $I$ be a prime ideal in $\mathscr{O}^{n}, I \neq \{0\}, I \neq
  \mathscr{O}^{n}$. Then there exist  
  \begin{enumerate}[(a)]
  \item a basis $(z_{1},\ldots,z_{k},\ldots,z_{n})$ for
    $\mathbb{C}^{n}$, in integer $k \ge 0$ and a fundamental system of
    neighbourhoods $D^{n} = D^{k} \times D^{n-k}$ of $0, D^{k} \subset
    C^{k}_{z_{1},\ldots,z_{k}} D^{n-k} \subset
    \mathbb{C}^{n-k}_{z_{k+1},\ldots,z_{n}}$' and if $\mathscr{O}^{k}$
    denotes the ring of germs of holomorphic functions at $0''$ in
    $\mathbb{C}^{k}_{z_{1},\ldots, z_{k}}$, 

  \item there exist polynomials $P_{k+1}[x], Q_{k+j}[x],
    \overset{\sim}{Q}_{k+j}[x], 2 \le j \le n-k $ in $
    \mathscr{O}^{k}[x]$ with $\deg Q_{k+j}, \deg
    \overset{\sim}{Q}_{k+j}< \deg P_{k+j}$. such that $I$ is generated
    by a finite number of holomorphic functions $f_{1}, \ldots ,f_{r}$
    on $D^{n}$ and if $S$ is the analytic set defined as the set of
    zeros of these functions in $D^{n}$, then following are satisfied      
  \end{enumerate}  

  \begin{enumerate}
  \item $\mathscr{O}^{K} \cap I = \{O\}$\pageoriginale

  \item If $\eta:  \mathscr{O}^{n}\to  \mathscr{O}^{n}/I$ is the
    natural map, the quotient field of $(\mathscr{O}^{n}/I)$ is
    generated by $\eta (z_{k+1})$ over the quotient field of $
    \mathscr{O}^{k}$. 

  \item $P_{k+1}[x]$ is the minimal polynomial of $\etaup(z_{k+1})$
    over $ \mathscr{O}^{k}$ and if $\delta = $ discriminant of
    $P_{k+1}$ over $ \mathscr{O}^{k}$, then $\delta z_{k+j}-Q_{k+j}
    [z_{k+1}]$ and 
    $$ 
    \dfrac{\partial P_{k+1}}{\partial x} [z_{k+1}]
    z_{k+j} - \overset{\sim}{Q}_{k+j}[z_{k+1}]
   $$ 
   are in $I$. 

  \item For every $z' \in D^{k}$ with $\delta (z') \neq 0$, there
    exist precisely $p$ points $(p$ = degree of $P_{k+1}[x]) (z',
    z^{i})$ in $S$, 
    $$
    z^{i} = \dfrac{Q_{k+j} [z^{i}_{k+1}]}{\delta(z')} =
    \dfrac{\overset{\sim}{Q}_{k+j}[z^{i}_{k+1}]}{P'_{k+1}[z_{k+1}]}
    $$ 
    where $(z^{i}_{k+1})_{1 \,\le\, i\, \le\, p}$ are the roots of
    $P_{k+1}[x] = 0$. 

  \item The points $S' = \{z \in S | \delta (z') \neq O\}$ are simple
    points of dimension $k$ of $S$ and $S'$ is connected and dense in
    $S$ and $\pi :S' \cap D^{n}\to (D	^{k} \cap \{ z'| \delta (z')
    \neq 0\})$ is a covering.  

  \item The projection $\pi:V \cap D^{n} \to D^{k}$ is proper and open.
  \end{enumerate} 
\end{theorem}
 
If $0$ is in an anaytic set $V$ and if $V_{0}$ is irreducible, let $I
=$ the ideal of germs at $0$ of holomorphic functions vanishing on
$V_{0}$. Then coordinate system $(z_{1},\ldots,z_{n})$ at $0$ which
satisfies the conditions (1)--(6) of the above theorem with respect
to $I$, is said to be proper for $V_{0}$. 

\begin{theorem}[H. Cartan]\label{chap1-thm6} 
  If\pageoriginale $S$ is an analytic set in an open set $U \subset
  \mathbb{C}^{n}$, 
  for any $a_0 \in U$, there exist a neighbourhood $U_{o}$ and
  $a$ finite number of holomorphic functions $f_{1},\ldots,f_{r}$ on
  $U_{o}$ such that for any point $b$ in $U_{o}$, the germs of
  $f_{1},\ldots,f_{r}$ at $b$ generate the ideal $I_{b}$ associated to
  $S_{b}$  over  $ \mathscr{O}^{n}_{b}$.   
\end{theorem}

\setcounter{definitions}{7}
\begin{definitions}\label{chap1-defins8} 
  If a is a simple point of $V$, let $U$ be a neighbourhood of a such
  that $U \cap V$ is an analytic submanifold of $\mathbb{C}^{n}$. Then
  the dimension of $V$ at a, denoted by $\dim_{a} V$ is defined to be
  the dimension of the submanifold $U \cap V$. 
\end{definitions}

9.~ For any point $\zeta$ in $V$, the dimension of $V$ at $\zeta$,
denoted by $\dim _{\zeta} V$ is defined by $ 
\underset{U_{\zeta}}{\varinjlim} \bigg( \underset{\text{point in}
  U_{\zeta} \cap V} {\overset {\text{Sup}}{\text{z is a simple}}} ~~~
\overset{\dim_z V}{\empty}  \bigg)$, where $U_{\zeta}$ is a
neighnourhood of~$\zeta$. 

\begin{proposition}\label{chap1-prop2} %pro 2
  If $V$ is an irreducible analytic set, $V'$ is another analytic set
  and $V' \underset{\neq}{\subset} V$, then $\dim V' < \dim V$. 
\end{proposition}   

\begin{proposition}\label{chap1-prop3}  %pro 3
  If $V$ is an analytic set, $V \subset \Omega \subset
  \mathbb{C}^{n}$, and if $0 \in V$ and $V_{o}$ is irreducible and if
  $\dim_{o} V =k$ and if $\{x_{1},\ldots,x_{n} \}$  is a coordinate
  system in a neighbourhood $U$ of $0$ such that $\{x\in U|  x_{1} =
  \ldots = x_{k} = 0 \} \cap V = \{0\}$ then by a linear change of
  coordinates, we can find a coordinate system $\{y_{1},\ldots,
  y_{n}\} y_{1} = x_{1}, \ldots ,y_{k} = x_{k}$ such that
  $\{y_{1},\ldots, y_{n}\}$ is a proper coordinate system for $V_{0}$,
  at $0$.   
\end{proposition}
 
\setcounter{remark}{2}
\begin{remark}\label{chap1-rem3}%rem 3
  If an analytic set $V = \cup V_{i}$, where $V_{i}$ are distinct
  irreducible analytic sets, the simple points of $V$ are theorem
  simple points $z$ of $V_{i}$, for each $i$, such that $z \notin
  V_{j}$ for $j \neq i$. 
\end{remark}

\begin{theorem}[Hilbert's Nullstellensatz]\label{chap1-thm7}  %the 7
  For\pageoriginale  any  ideal $I$ in $ \mathscr{O} ^n$, there exists
  an integer $n  = n(I)$ such that if $f \in  \mathscr{O}^n$ and if
  $f$ vanishes  on 
  the germ of  the analytic set $S_I$ defined by $I$, then $f^n \in
  I$.  
\end{theorem}

\begin{proposition}\label{chap1-prop4}  %pro 4
  If  $V$ is an analytic set and $W \subset V$ is also an analytic
  set, $\overline{(V-W)}$ is an analytic set and
  $\dim. \overline{(V-W)} \cap W < \dim \overline{(V-W)}$. 
\end{proposition}

\begin{proposition}\label{chap1-prop5}  %pro 5
  If $V$ and $W$ are analytic sets in  an open  set $\Omega$ in
  $\mathbb{C}^n$, then $V \cap W$ is an analytic set and $\dim (V \cap
  W) \ge \dim  V+ \dim W-n$. 
\end{proposition}

We now state the various types of decomositions  of an analytic set
that we shall consider. 

{\bf (i)}  \textit{Strict partitions  into manifolds}

\setcounter{definition}{9}
\begin{definition}\label{chap1-defin10} %def 10
  An analytic set $V$ is said to be partitioned  strictly in to manifolds if 
  $$
  V = \underset{i}{\bigcup} M_i, \text{  where  }
  $$
  \begin{enumerate}
  \item $M_i$ are analytic submanifolds of $\mathbb{C}^n$ with $M_i
    \cap M_j = \phi $for $i \neq  j $  and 
  \item if $\dot{M}_i = \overline{M}_i - M_i, \dot{M}_i,
    \overline{M}_i$ are analytic sets. 
  \end{enumerate}
\end{definition}

{\bf (ii)}  \textit{Canonical  strict partitions into manifolds}.

We shall prove in Chapter $II$ that any analytic set $V$ can be
canonically strictly partitioned into manifolds $V = \bigcup M_i$, 
where, if $M_i$ is a manifold of maximum  dimension $p$ say, then it
is a connected  component of  the set of simple points of $V$, if
dimension $p $ and 
$$
p =  \underset{a \in V}{\text{max}}  (\dim _a
V).
$$ 

{\bf (iii)}  \textit{Stratifications}\pageoriginale

\begin{definition}\label{chap1-defin11}  %def 11
  A strict partition into manifolds is a stratification  if and only if 
  \begin{enumerate}[1.]
  \item  $\overline{M}_{i} - M_{i} = \dot{M}_i  =  \bigcup \limits _{j
    \epsilon J'_{i}}  M_{j} $ \text{ for some subset}
    $J'_{i}$ of $T$ or 

  \item $M_{i} \cap \overline{M}_{j} \neq \phi  \Rightarrow  M_{i}
    \subset  \overline{M}_{j}$ 
  \end{enumerate}
\end{definition}

\begin{examples}\label{chap1-exams1}  %Exa 1
  Let $V$  be the  analytic set in    $\mathbb{C}^3  xyt$ given by
  $V=\big\{z \in \mathbb{C}^3 \text{xyt} \big| t^2
  =x^2=y^2\big\}$. Then $V$  is a cone and if $D$ is a generator of
  the  cone, let $M_{1} = D, M_{2} = V-D$. Then $V= M_1 \cup M_2$ is a
  stratification. 
  However, it is clear that this stratification is not uniquely
  determined as we may choose any generator for $D$  and that either
  of two such  stratifications is not  finer than the other. 
\end{examples}

{\bf (iv)  Whitney  Stratifications}

Given a stratification $V = \bigcup M_i$, let $M_i \subset
\overline{M}_j$. Let $z_0$ be in $M_i$ and consider a sequence of
points $\{z_\nu \} \epsilon  M_j, z_\nu \rightarrow z_0$. If
$T(M_j,z)$ denotes  the tangent space  at $z$ for any $z$ in $M_j$,
let 
$T(M_j,z_\nu) \rightarrow T $ (in a  natural sense of  Grassmann
manifold that  we shall  describe  later) as $z_\nu \rightarrow
z_0$. The  the pair $(M_i, M_j)$ is said to be \textit{$(a)$ regular}
at $z_0$, according to Whitney , if , for any such limit $T$, 
$T \supset T (M_i,z_0)$. This is clearly not  the case in an arbitrary
stratification. Consider 

\setcounter{example}{1}
\begin{example}[Whitney]\label{chap1-exam2}  % Exa 2
  Let $V$  be the analytic set  in $\mathbb{C}^{3}_{xyt}$
  given by $V = \big\{z \epsilon  \mathbb{C}^{3}_{xyt} \big| y^2 =
  tx^2\big\}$ and  the stratification $M_1 =  \mathbb{C}_t$ and  $M_2
  = V - \mathbb{C}_t$. Consider points $z_\nu$ on $ \mathbb{C}_x$  such
  that $z_\nu  \rightarrow 0$. It is clear\pageoriginale that if $T$
  is the limit 
  of $T (M_2,z_\nu)$, $T \not\supset T (M_1,0)$. However if we have the
  stratification given by  
  $M_0 =  M_{1}=\mathbb{C}_{t}- \{0\}$, $M_2= V - \mathbb{C}_t$  then  the
  pairs $(M_0,M_1),  (M_1,M_2)$ are $(a)$  regular at all points of
  $M_\circ$  and $M_1$ respectively. 
\end{example}

 But in a set other than  a complex analytic set, it may not be
 possible to obtain a substratification which is $(a)$ regular, i.e.
 for which  all pairs $(M_i,M_j)$ of strata, $M_i \subset
 \overline{M}_j$,  are $(a)$  regular.  
 
 \begin{example}\label{chap1-exam3}  % Exa 3
   Let  $f:\mathbb{R}^{3}_{xyt} \rightarrow \mathbb{R}$  be the
   continous function given  by $f (x,y,t) = t \sin \left(\dfrac{x}{t}\sin
   \dfrac{1}{t}\right) -y$ if  $t \neq 0$ and  $f(x,y,0) = -y$. We define
   the set $V$ in $\mathbb{R}^{3}_{xyt}$ by $V= \big\{(x,y,t)\epsilon
   \mathbb{R}^{3}_{xyt} \big|  f(x,y,t) = 0\big\}$. Consider the
   partition  $M^1 = \mathbb{R}_x$ and $M^2 = V-\mathbb{R}_x$, of
   $V$. 
 \end{example}
 
We have $t \dfrac{d}{dt} \left(\dfrac{1}{t} \sin \dfrac{1}{t}\right) =
-\dfrac{1}{t} \left(\sin \dfrac{1}{t} + t \cos \dfrac{1}{t}\right)$.  Let $t_\nu
$ be a sequence of points such that $t_\nu \rightarrow 0$ and $\left[t
  \dfrac{d}{dt} \left(\dfrac{1}{t} \sin
  \dfrac{1}{t}\right)\right]_{t=t_{\nu}} =0$. Then 
clearly $\tan \dfrac{1}{t_{\nu}} = -\dfrac{1}{t_{\nu}} \text{ and
} \begin{vmatrix} \sin \dfrac{1}{t_{\nu}} \end{vmatrix} =
\dfrac{1}{\sqrt{1+t^2_\nu}} \rightarrow 1$.   
We suppose, without  loss of generality, that $\sin \dfrac{1}{t_{\nu}}
\rightarrow 1$. Let $x_{\nu,k} = \dfrac{2\pi kt_{\nu}}{\sin
  \dfrac{1}{t_\nu}}$  and $z_{\nu,k} = (x_{\nu,k},0,t_\nu) \epsilon
V$. We have, by direct computation, 
$$
\frac{\partial f}{\partial x} (z_{\nu,k}) = \sin \frac{1}{t_{\nu}},
\frac{\partial f}{\partial y} (z_{\nu,k}) = -1, \frac{\partial
  f}{\partial t} (z_{\nu,k}) =0. 
$$

Let $z_\circ = (x_0,0,0)$ and $x_{\nu} = z_{\nu,k (\nu)}$ where $k(\nu)$
is the largest integer\pageoriginale such that  $k(\nu) \leq
\dfrac{x_\circ}{2 \pi t \nu} \sin \dfrac{1}{t\nu}$. Then clearly $ z_\nu \in V $ and
$z_\nu \rightarrow z_0$. $T(\nu,z_{\nu,_{k(\nu)}})$  has a limit,
namely, the plane  orthogonal to the vector $(1, -1,0)$. This  plane
does not contain  $\mathbb{R}_x$ and hence a condition corresponding
to the condition $(a)$ is not satisfied  in this example. 
 
However, it follows from  the whitney's  theorem that we shall prove,
that such a situation cannot arise in an  $\mathbb{C}$- analytic
case. 
 
If we assume that the stratification is $(a)$ regular and if $M_i
\subset \overline{M}_j, M_i$ and $M_j$ being  two strata, let $z_o$ be
in $M_i$ and $\{z_\nu\}$ be a sequence in $M_j, z_\nu \rightarrow z_0
$. Let $\{\zeta_\nu\}$ be a sequence in $M_i, \zeta_\nu \rightarrow
z_0$ and $\{\lambda_\nu\}$, a sequence of complex numbers such that
$\lambda_\nu(z_\nu -\zeta_\nu)$, is convergent to (a vector) $v$
say. Then the pair  $(M_i,M_j)$ is said to be \textit{(b) regular}
(according to  Whitney) if every such  $v \epsilon T = Lim T
(M_j,z_\nu)$. 

\begin{example}[Whitney]\label{chap1-exam4}  %Exa 4
  Consider the analytic set
  $$ 
  V = \{z \epsilon \mathbb{C}^{3}_{xyt} \mid t^2 (x^2-y^2)
  +x^3-y^4=0\}.
$$
Let
$$
M_1=\mathbb{C}_t,M_2=V-\mathbb{C}_t, V = M_1
  \cup M_2
$$
  be the stratification. Consider the points $z= \left(-\dfrac{1}{\nu^2},0,
  \dfrac{1}{\nu}\right)$ on $M_2$  and  $\zeta_\nu =
  \left(0,0,\dfrac{1}{\nu}\right)$ on $M_1$.  Clearly $ z_\nu,  \zeta_\nu
  \rightarrow  0$ and  $\nu ^2 (z_\nu- \zeta_\nu) \rightarrow
  (-1,0,0)$ and obviously the pair $(M_1, M_2)$ is not (b) regular at
  $0$ for the normal to the surface  $f(x_1,y_1,t)= t^2(x^2-y^2) +
  x^3-y^4 =0$ at $\left(-\dfrac{1}{\nu^2},0,  \dfrac{1}{\nu}\right)$ is parallel
  to $\left(1,0,- \dfrac{2}{\nu}\right)$ and this tends to  $(1,0,0)$ at $\nu
  \rightarrow \infty$. 
\end{example} 

We give here a example of set $V$ which is not $\mathbb{C}$ -analytic
and a stratificaton $V = \cup M_i,M_1 \subset \overline{M}_2$ and the
pair $(M_1,M_2)$ is not ${\bf (b)}$ regular at any point of $M_1$.  

\begin{example}\label{chap1-exam5}  %Exa 5
  Let\pageoriginale $ V \subset \mathbb{R}^{3}_{xyt}$ be given  as
  follows. Consider 
  the parameters $ \rho,\theta $ in $\mathbb{R}^2$ where $x=\rho \cos
  \theta, y= \rho \sin \theta$. Then $V= \{(x_1,y_1,t) \mid \rho
  =e^\theta, -\infty \leq \theta \leq1\}$. Let $M_1 = \mathbb{R}_{xy} =
  \{(x,y,t ) \mid x= y= 0 \}$ and $M_2 =V-M_1$.  Then for any point
  $(0,0,t_0)$ on $M_1$, consider $z_\nu = \{(z,y,t)\mid \theta = -2
  \pi \nu,t=t_0\}$,  i.e. $z_\nu = (e^{-2 \pi \nu}, 0,t_0)$. Let
  $\zeta_\nu = (0,0,t_0), \lambda_\nu = e^{2 \pi
    \nu}$, i.e. $v=(1,0,0)$. Now  the planes tangent to $M_2$ are all
  orthogonal to  $(1,0,0)$ and the  pair $(M_1,M_2)$ is not $(b)$
  regular at any point $(0,0,t_0)$ of $M_1$.  
\end{example}

However, we shall prove that such a situation does not arise in
$\mathbb{C}$-analytic sets. We shall prove the following  

\begin{theorem*}[(Whitney)] %the 8
  Every stratification of a  $ \mathbb{C} $ -analytic set admits a
  substratification which is (a) and (b) regular. 
\end{theorem*}

