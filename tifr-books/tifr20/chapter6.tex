\chapter{Holomorphic Connections}\label{chap6}%chap 6

\section{Complex vector bundles}\label{chap6:sec1}%sec 6.1

\setcounter{defn}{0}
\begin{defn}\label{chap6:sec1:def1}%defi 1
  A\pageoriginale {\em complex vector bundle} is a differentiable vector bundle $E$
  over a manifold $V$ with a differentiable automorphism $J : E \to E$
  such that 
  \begin{enumerate}[1)]
  \item $JE_x \subset E_x$ for every $x \in V;$
  \item $J^2 Y = - Y$ for every $Y \in E$.
  \end{enumerate}
\end{defn}

Let $P$ be a differentiable principal bundle over $V$ with group
$G$. Let us assume given a left representation $s \to s_L$ of $G$ in a
complex vector space $L$, such that each $s_L$ is a complex
automorphism of $L$. Then a vector bundle $E$ associated to $P$ with
typical fibre $L$ can be made into a complex vector bundle by setting
$J_q (\xi, v) = q (\xi, \sqrt - 1 v)$ for every $\xi \in P$ and
$v \in L, q$ being the usual projection $P \times L \to E$.
Conversely any complex vector bundle $E$ can be obtained in the above
way. In fact, for $x \in V$ we define $P_x$ to be the vector space of
all linear isomorphisms $\alpha : \mathbb{C}^n \to E_x$ such that
$\alpha (\sqrt - 1 v) = J \alpha (v)$ for $v \in \mathbb{C}^n$. Then as
in Ch. $4$, one can provide $P = \cup P_x$ with the structure true of
a principal bundle over $V$ with group $GL (n, \mathbb{C}). E$ is
easily seen to be a complex bundle associated to $P$ with fibre
$\mathbb{C}^n$ with respect to the obvious representation of $GL (n,
\mathbb{C})$ in $\mathbb{C}^n$. 

If\pageoriginale $E$ is a complex vector bundle, the $\mathscr{U}(V)-$ module
$\varepsilon (V)$ of differentiable sections of $E$ can be provided
with the structure of an $\mathscr{U}_\mathbb{C} (V)-$ module
$\mathscr{U}_\mathbb{C} (V)$ being the algebra of complex-valued
differentiable functions on $V)$ by setting $(f + ig) \sigma = f
\sigma  + J g \sigma$ for $f, g \in \mathscr{U}(V)$. Conversely if $I$
is an endomorphism of the $\mathscr{U}(V)-$ module of sections of a
differentiable vector bundle $E$ over $V$ such that $I^2 \sigma = -
\sigma$ for every section $\sigma$, then there exists one and  only
one automorphism $J$ of $E$ such that $J (\sigma x) = (I \sigma)x$
(for every section $\sigma$ and $x \in V$) and $J$ makes of $E$ a
complex vector bundle. 

\section{Almost complex manifolds}\label{chap6:sec2}%sec 6.2

\begin{defn}\label{chap6:sec2:def2}%def 2
  An {\em almost complex manifold} is a differentiable manifold $V$
  for which the tangent bundle has the structure of a complex vector
  bundle, i.e., there exists an $\mathscr{U}(V)-$ endomorphism $I$ of
  $\mathscr{C} (V)$ such that $I^2 = -$ (Identity). 
\end{defn}

The $\mathscr{U}_\mathbb{C} (V)$- module $\mathscr{C}_\mathbb{C} (V)$
of complex vector fields (defined as formal sums $X_1 + i X_2, X_1,
X_2 \in \mathscr{C}(V)$) can be identified with the module of
derivations of $\mathscr{U}_\mathbb{C} (V)$ over $\mathbb{C}$. The
$\mathscr{U}(V)-$ endomorphism $I$ of $\mathscr{C}(V)$ can then be
extended to an $\mathscr{U}_\mathbb{C} (V)$- endo-morphism of
$\mathscr{C}_\mathbb{C} (V)$ by setting $I(X_1 + iX_2) = IX_1 + iIX_2$
for $X_1, X_2 \in \mathscr{C} (V)$. 

\begin{defn}\label{chap6:sec2:def3}%defi 3
  A vector field $X \in \mathscr{C}_\mathbb{C} (V) $ is said to be of
  {\em type $(1, 0)$} (\resp. of {\em type $(0,1)$}) if $IX - iX$
  (\resp. $IX = - iX$). 
\end{defn}

We shall denote by $\mathscr{C}_{(1, 0)}(V), \mathscr{C}_{(0, 1)}(V)$
the $\mathscr{U}_\mathbb{C} (V)$- modules of vector fields of type
(1, 0) and type (0, 1) respectively. 

Clearly\pageoriginale one has $\mathscr{C}_\mathbb{C} (V) = \mathscr{C}_{(1, 0)}(V)
\oplus \mathscr{C}_{(0 , 1)}(V)$. Moreover, any vector field $X$ of
type $(1, 0)$ can be expressed in the form $X_1 - iIX_1$ with $X_1 \in
\mathscr{C}(V)$. On the other hand, $X - iIx$ is of type $(1, 0)$ for
every $X \in \mathscr{C}_\mathbb{C} (V)$. Similarly every vector field
of type (0, 1) can be expressed in the form $X_1 + iIX_1$ with $X_1
\in \mathscr{C}(V)$ and $X + iIX$ is of type (0, 1) for every $X \in
\mathscr{U}_\mathbb{C} (V)$. 

Let $V$ be an almost complex manifold and $E$ a complex vector bundle
over $V$. A \textit{complex form of degree $p$ on $V$ with values in
  $E$} is a multilinear form of degree $p$ on the
$\mathscr{U}_\mathbb{C} (V)$ - module  $\mathscr{C}_\mathbb{C} (V)$
with values in $\varepsilon (V)$. Any real form can be extended in one
and only one way to a complex form. On the other hand, every complex
form is an extension of a real form. 

Every holomorphic manifold $V$ carries with it a canonical almost
complex structure. With reference to this a vector field on $V$ is of
type (1, 0) if and only if its expression in terms of any system
$(Z_1, \ldots, Z_n)$ of local coordinates does not involve
$\dfrac{\partial}{\partial \bar{Z}_1}, \ldots,
\dfrac{\partial}{\partial \bar{Z_n}}$. Hence it follows that $[
  \mathscr{C}_{(1, 0)}(V), \mathscr{C}_{(1, 0)}$ $(V) ] \subset
\mathscr{C}_{(1, 0)} (V)$. Conversely an almost complex structure on
$V$ comes from a holomorphic structure on $V$ if $[ \mathscr{C}_{(1,
    0)}(V), \mathscr{C}_{(1, 0)}$ $(V) ] \subset \mathscr{C}_{(1, 0)}
(V)$ cf. Newlander and Nirendberg, Annals of Math. $1957$. Let $X, Y
\in \mathscr{C}_\mathbb{C}(V)$. Then $X - iIX, Y - iIY \in
\mathscr{C}_{(1, 0)} (V)$ and the above condition requires that $[X -
  iIX, Y - iIY] \in \mathscr{C}_{(1, 0)} (V)$. In other words, 
\begin{align*}
  [X,Y] - [IX, IY] & = -  I [IX, Y] -  I [X, IY].\\
  F(X, Y) & = [X, Y] + I [IX, Y] + I [X, IY] - [IX, IY]
\end{align*}
is\pageoriginale easily seen to be $\mathscr{C}_{\mathbb{C}} (V)-$ bilinear. Hence
we have: 

An almost complex structure on $V$ is induced by a holomorphic
structure if and only if $F(X, Y) = 0$ for every $X, Y \in
\mathscr{C}_{\mathbb{C}} (V)$. 

\section{Derivation law in the complex case}\label{chap6:sec3}%sec 6.3

We wish to study derivation laws in the module of sections of a
complex vector bundle over an almost complex manifold (mostly
holomorphic manifold). This can be done in the algebraic set-up of
Ch.1. 

Let $A$ be a commutative algebra over the field of real numbers and
$A_{\mathbb{C}}$ its complexification. Let $C$ be the $A_{\mathbb{C}}$
module of derivations of $A_\mathbb{C}$. We assume given on $C$ and
$A_\mathbb{C}$ - endomorphism $I$ such that $I^2 = - $ (Identity). If
we define $\bar{X} \in C$ by $\bar{X}a = \overline{X \bar{a} }$ for $a
\in A$, then we also assume that $I$ satisfies $I\bar{X} = \bar{IX}$. $X
\in C$ is said to be of type $(1, 0)$ (\resp. type $(0, 1)$) if $IX =
iX$ (\resp. $IX = - iX$). Let $M$ be an $A_{\mathbb{c}}$- module. A
multilinear form $\alpha$ of degree $p$ on $C$ with values in $M$ is
said to be of type $(r, s)$ if $r + s = p$ and $\alpha (X_1, \ldots
X_p) = 0$ whenever either more than $r X'_i s$ are of type $(1, 0)$
or more than $s X^{'}_i s$ are of type $(0, 1)$. We shall denote the
submodule of $\mathscr{F}^{p} (C, M)$ consisting of all forms of
type $(r, s)$ by $\mathscr{F}^{r, s} (C, M)$. Then it is easy to see
that $\mathscr{F}^p (C, M) = \sum\limits_{r + s = p} \mathscr{F}^{r,
  s} (C, M)$, the sum being direct. Let us denote the module of
alternate forms of type $(r, s)$ on $C$ with values in $M$ by
$\mathscr{U}^{r, s} (C, M)$. Let $M_1, M_2, M_3$ be three
$A_{\mathbb{C}}$- modules with a bilinear product $M_1 \times M_2 \to
M_3$. It is then easy to\pageoriginale see that if $\alpha \in \mathscr{U}^{r, s}
(C, M_1), \beta \in \mathscr{U}^{r' , s'} (C, M_2)$, then $\alpha
\wedge \beta \in \mathscr{U}^{r + r', s + s'} (C, M_3)$. 

We shall now make the additional assumption that $F[X, Y] = [X, Y] + I
[IX, Y] + I [X, IY] - [IX, IY] = 0$ for every $X, Y \in C$. (This
corresponds to the case when the base manifold $V$ is holomorphic). A
derivation law $D$ in an $A_\mathbb{C}$ module $M$ gives rise to an
exterior derivation $d$ in $\mathscr{U}^p (C, M)$. Using the explicit
formula for $d$ (Ch.\ref{chap1:sec7}) and the fact that $F \equiv 0$, it is
easily proved that for $\alpha \in \mathscr{U}^{r, s} (C, M), d
\alpha$ is the sum of a form $d '\alpha$ of type $(r + 1, s)$ and a
form $d'' \alpha$ of type $(r, s+ 1)$. The curvature form $K \in
\mathscr{U}^2 (C, \Hom_{A_\mathbb{C}} (M, M)$ is a sum of three
components $K^{2, 0}, K^{1, 1}$ and $K^{0 , 1}$. For $\alpha \in
\mathscr{U}^{p, q} (C. M)$ we have $d^2 \alpha = K \wedge \alpha$
(Lemma \ref{chap1:sec9:lem5}, Ch. 1.9) and calculating the components
of type $(p, q + 2)$ we obtain 
$$
d''^2 \alpha = K^{0, 2} \wedge \alpha. 
$$

Now we make the further assumption that $K^{0, 2} = 0$. It follows
that $d''^2 = 0$. For every integer $p$, form the sum
$\sum\limits_{q}\mathscr{U}^{ p, q} (C, M) = \mathscr{T}_p$. Then we have a
complex 
$$
\mathscr{U}^{p, 0} (C, M) \xrightarrow{d^{''}}\mathscr{U}^{p, 1} (C,
M) \xrightarrow{d^{''}} \cdots \mathscr{U}^{p, q} (C, M)
\xrightarrow{d^{''}} \cdots 
$$

The complex $\mathscr{T}_p$ of course depends on the derivation law $D$. Let
$\hat{D}$ be another derivation law in $M$. Then $\hat{D} = D
+ \omega$ where $\omega \in \Hom_{\mathbb{C}} (C, \Hom_{A
  \mathbb{C}}(M, M))$\pageoriginale (see Ch. \ref{chap1:sec10}). We shall say that two
derivation laws $D, \hat{D}$ are \textit{equivalent} if $\omega$
is of type $(1, 0)$. The boundary operator $d^{''}$ in the complex
$\mathscr{S}_p$ remains the same when $D$ is replaced by an equivalent
derivation  law $\hat{D}$. For, if $D = \hat{D} + \omega$, we
have (Ch.\ref{chap1:sec10}) $\hat{d} \alpha = d \alpha + \omega \wedge \alpha$
for $\alpha \in \mathscr{U}^{p, q} (C, M)$. Hence on computing the
components of type $(p, q +  1)$, one obtains $\hat{d}'' \alpha = d''
\alpha$. Thus we have associated to every equivalence class of
derivation laws, a complex $\mathscr{T}_p$ (for every integer $p$). 

A derivation law $D$ in $M$ induces a derivation law $\underline{D}$
in $\Hom_{A_\mathbb{C}}$ $(M, M) = $ End $M$(Ch.\ref{chap1:sec3}). Then it is easy
to see that 
$$
\underbar{K} (X, Y) \rho = K (X, Y)\circ \rho - \rho \circ K (X, Y)
$$
for $X, Y \in C, \rho \in$ end M. Hence if $K^{0, 2} = 0,
\underline{K}^{0, 2}$ is also $= 0$. Moreover, if $D, \hat{D}$ are
equivalent, so are $\underline{D}$, $\underline{\hat{D}}$. Hence to an
equivalence class of derivation laws in $M$ corresponds a complex
$\underline{\mathscr{T}}_p$ (for every integer $p$):  
$$
\mathscr{U}^{p, o} (C, ~\text{End}~ M) \xrightarrow{\underline{d}''}
\mathscr{U}^{p, 1} (C, ~\text{End}~ M) \xrightarrow{\underline{d}''} \cdots
\mathscr{U}^{p, q} (C, ~\text{End}~ M) \xrightarrow {\underline{d}''} \cdots 
$$

Now $K^{1, 1}$ is an element of $\mathscr{U}^{1, 1}(C, E nd M)$. Since
we have $dK = 0$ (Lemma \ref{chap1:sec9:lem6}, Ch.1.9) and $K^{0, 2}= 0$, we get $d''
K^{1, 1} = 0$, i.e., $K^{1, 1}$ is a cocycle of the complex
$\underline{\mathscr{T}}$. If $D$ is replaced by an equivalent derivation
law $\hat{D}$, we have (Ch.\ref{chap1:sec10}) $\hat{K} = K + d \omega + \omega
\wedge \omega$ where $\hat{D} = D + \omega$.\pageoriginale But $\omega \in
\mathscr{U}^{1, 0} (C, End M)$. Hence $\hat{K}^{1, 1} = K^{1,1} + d''
\omega$. In other word, the cohomology class of $K^{1, 1}$ in the above
complex $\underline{\mathscr{T}}_1$ is uniquely fixed. 

\section{Connections and almost complex structures}\label{chap6:sec4}%Sec 6.4

Let $P$ be a differentiable principal bundle over an almost complex
manifold $V$ with complex Lie\pageoriginale group $G$. Let $\mathscr{Y}$ be the Lie
algebra of $G$ identified with the space of real tangent vectors at
$e$. We denote by $I_G$ the endomorphism $\mathscr{Y} \to \mathscr{Y}
$ given by the complex structure of $G$ and by $I$ the almost complex
structure of $V$. Obviously we have 
$$
s(I_G a)^{-1}s = I_G (sas ^{-1}) \text{ for } s \in G, a \in \mathscr{Y}.
$$

Let $\xi \in P$ and $p \xi = x \in V$. Then we have the exact sequence
$$
0 \to \mathfrak{N}_\xi \to T_\xi \to T_x \to 0
$$
where $\mathfrak{N}_\xi$ is the space of vectors at $\xi$ tangential to the
fibre. If a connection is given on $P$, then we may define as almost
complex structure on $P$ by carrying over the complex structure on
$T_x$ to $S_\xi$ (the space of horizontal vectors at $\xi$) and that
of $\mathscr{Y}$ to $\mathfrak{N}_\xi$ (by virtue of the isomorphism $a \to \xi
a$). This is easily seen to define an almost complex structure on $P$
so that we have the 

\setcounter{proposition}{0}
\begin{proposition}\label{chap6:sec4:prop1}%prop 1
  Let $\gamma$ be a connection form on a differentiable principal
  bundle $P$ over an almost complex manifold $V$ with complex Lie
  group $G$. Then there exists one and only one almost complex
  structure on the manifold $P$ such that  
  \begin{enumerate}[1)]
  \item $\gamma (I_p d \xi ) = I_G \gamma (d \xi)$
  \item $pI_p d \xi = I pd \xi $   for   $\xi \in P$.
  \end{enumerate}

  In virtue of the general procedure given in Ch.\ref{chap6:sec2}, $\gamma$ is
  defined on complex vector fields by setting $\gamma (iX) = I_G
  \gamma (X)$ for $X \in \mathscr{C}(P)$. The connection form is
  obviously of type $(1, 0)$ with respect to the induced almost
  complex structure on $P$. Moreover, the $P \times G \to P$ given by
  the action of $G$ on $P$ is almost complex. We shall call a
  principal bundle $P$ over an almost complex manifold $V$ with
  complex Lie group $G$ an \textit{almost complex principal bundle} if
  $P$ has an almost complex structure such that the projection $p$ and
  the map $(\xi, s) \to \xi s$ of $P \times G \to P$ are both almost
  complex (i.e., compatible with the almost complex structure). We have
  seen above that $P$ with the almost complex structure induced by a
  connection form $\gamma$ on $P$ is an almost complex principal
  bundle. Conversely, let $P$ be an almost complex principal
  bundle. If $\gamma = \gamma^{1, 0} + \gamma^{0 , 1}$ is a connection
  form on $P$ then $\gamma^{1, 0}$ is again a connection form. In
  fact, 
\begin{align*}
  \gamma^{1, 0} (\xi a) & = \frac{1}{2} \gamma^{1, 0} ((\xi a + iI_p
  \xi a) + (\xi a - iI_p \xi a) )\\ 
  & = \frac{1}{2} \gamma (\xi a - iI_p \xi a)\\
  & = \frac{1}{2} (a - I_G \gamma (I_P \xi a)) = \frac{1}{2} (a - I_G
  \gamma (\xi I_G a))\\ 
  & = a
\end{align*}
\end{proposition}

Similarly\pageoriginale it may be shown that $\gamma^{1, 0} (d \xi s) = s^{-1}
\gamma^{1, 0} (d \xi)s$. Moreover we have 
\begin{align*}
  \gamma^{1, 0} (I_P d \xi) & = \frac{1}{2} \gamma^{1, 0} (I_P d \xi +
  id \xi + I_P d\xi - id \xi)\\ 
  & = \frac{1}{2} \gamma (I_P d \xi + id \xi)\\
  & = I_G \gamma^{1, 0} (d \xi).
\end{align*}

On the other hand, $pI_P  d \xi = I_p d \xi$ by assumption. Hence by
prop.\ref{chap6:sec4:prop1}, Ch.6.4, $\gamma^{1, 0}$ induces the given
almost complex structure on $P$. 

Let $\gamma, \hat{\gamma}$ be two connection forms on an almost
complex principal bundle $P$. Then $\gamma - \hat{\gamma}$ is a G-form
(Ch.\ref{chap5}) of degree $1$ on $P$ with values in $\mathscr{Y}$. It may be
identified with a differential form of degree $1$ on $V$ with values
in the adjoint bundle. Using the fact that every vector field on $V$
of type (1,0) is the projection of a vector field on $P$ of type
(1,0) it is easy to see that this identification is
type-preserving. In other words, the type of $G$-forms on $P$ with
values in $\mathscr{Y}$ depends only on the almost complex structure
on $V$. We say that $\gamma , \hat{\gamma}$ are \textit{equivalent} if
$\gamma - \hat{\gamma}$ is of type (1,0). The following proposition
in then immediate. 

\begin{proposition}\label{chap6:sec4:prop2}%prop 2
  Two connections $\gamma , \hat{\gamma}$ on $P$ induce the same
  almost complex structure on $P$ if and only if $\gamma$ and
  $\hat{\gamma}$ are equivalent. 
\end{proposition}

On the other hand we have seen that every almost complex structure on
$P$ which makes of it an almost complex principal bundle is\pageoriginale induced by
a connection of type $(1, 0)$. We have thus set up one-one
correspondence between almost complex bundle structure on $P$ and
equivalence classes of connections. 

We will hereafter assume that the base manifold $V$ is holomorphic and
investigate when a connection form $\gamma$ induces a holomorphic
structure on $P$. Then curvature form $K$ of $\gamma$ is a G-form of
degree $2$ with values in $\mathscr{Y}$. Let $K = K^{2, 0} + K^{1, 1}
+ K^{0, 2}$ be its decomposition. 

\begin{proposition}\label{chap6:sec4:prop3}%prop 3
  A connection form $\gamma$ on $P$ induces a holomorphic structure on
  $P$ if and only if $K^{0, 2}= 0$. 
\end{proposition}

In fact, if $\gamma$ induces a holomorphic structure on $P$, then $d''
\gamma = 0$. Since $K = d \gamma + [\gamma, \gamma]$ and $\gamma$ is
of type $(1, 0)$ for the induced complex structure, $K^{0, 2} =
0$. Conversely, let $K^{0, 2} = 0$. We have then to show that 
$$
F_P (X, Y) = [X, Y] + I_P [_P X, Y] + I_P [X, I_P Y] - [I_P X, I_P Y] = 0
$$
for any two vector fields $X, Y$ on $P$. Since $F_P$ is a tensor, it
is sufficient to prove that $F_P (X, Y) = 0$ for projectable vector
fields $X, Y$. Then $I_P X, I_P Y$ are also projectable, and we have
$P^F P (X, Y)=  F_V (P^X, P^Y) = 0$ since $V$ is holomorphic. On the
other hand, 
\begin{align*}
  \gamma (F(X, Y)) & = \gamma ([X, Y] - [I_P X, I_P Y]) + I_G \gamma
  ([I_P X, Y] + [X, I_P Y])\\ 
  & = \gamma ([X, Y] - [I_P X, I_P Y] + [iI_P X, Y] + [X, iI_PY])\\
  & = \gamma ([X + iI_PX, Y + iI_PY]).
\end{align*}

But\pageoriginale since $K^{0, 2} = 0, K(X + iI_P X, Y + iI_PY ) = 0$.

i.e., $(X + iI_PX) \gamma (Y + iI_PY) - (Y + iI_PY) \gamma (X + iI_PX)
- \gamma ([X + iI_PX, Y + iI_PY]) = 0$. 

Since $\gamma$ is of type $(1, 0)$, we obtain $\gamma (F(X, Y)) = 0$
which gives $F(X, Y) = 0$.  

\begin{defn}\label{chap6:sec4:def4}%def 4
  A differentiable principal bundle $P$ over a holomorphic manifold
  $V$ with a complex Lie group $G$ is said to be a {\em holomorphic
    principal} bundle if $P$ is a holomorphic manifold such that the
  projection $p : P \to V$ and the map $(\xi , s )\to \xi s$ of $P
  \chi G \to P$ are holomorphic. 
\end{defn} 

If a connection form $\gamma$ exists on $P$ with $K^{0, 2} = 0$, then
the induced almost complex structure on $P$ is holomorphic and its is
obvious that this makes of $P$ a holomorphic principal bundles. 

\section{Connections in holomorphic bundles}\label{chap6:sec5}%sec 6.5

Let $P$ be a holomorphic principal bundle over $V$ with group $G$. The
definitions of holomorphic vector bundles with respect to a
holomorphic representation of $G$ in a complex vector space $L$ are
given in analogy with the differentiable case. 

All the results of Chapter 2 can be carried over to holomorphic
bundles with obvious modifications. Let $\mho (U)$ be the algebra of
holomorphic functions on an open subset $U$ of $V$ and $\mho$ the
sheaf of holomorphic functions on $V$. The notion of a sheaf of
$\mho$- modules is defined as in Chapter $4$. Moreover, there exists a
functor from the category of holomorphic vector bundles over $V$ to
the category\pageoriginale of locally free sheaves of $\mho$ modules which takes
exact sequences to exact sequences, and vice versa. 

Let $G$ be a complex Lie group, $P$ a holomorphic principal bundle
with group $G$ over a holomorphic manifold $V$, and $E$ a holomorphic
vector bundle over $V$ associated to $P$ with typical fibre $L$. Any
connection form $\gamma$ on $P$ defines a derivation law in the module
of sections of $E$ (Ch.\ref{chap5}). 

\begin{proposition}\label{chap6:sec5:prop4}%prop 4
  If the connection form $\gamma$ is of type $(1, 0)$, then a
  differentiable section $\sigma$ of $E$ over an open subset $U$ of
  $V$ is holomorphic if and only if $d'' \sigma = 0$. 
\end{proposition}

If $ \tilde{\sigma}$ is the $G$-function on $P$ with values in $L$
corresponding to $\sigma$ under our usual isomorphism, it is easy to
prove (as in the differentiable case) that $\sigma$ is holomorphic if
and only if $\tilde{\sigma}$ is holomorphic. But this is equivalent to
saying that $d '' \tilde{\sigma} = 0$ for the canonical derivation law
in the $\mathscr{U}_\mathbb{C}(P)$ module of $L$-valued functions on
$P$. Since the derivation law $D$ induced by $\gamma$ in
$\mathscr{L}(P)$ was defined by $D_X f = X f + \gamma (X)_L f$ (in the
usual notation) and $\gamma$ is of type $(1,0)$, the $d''$ for $D$ and
the canonical derivation law are the same. Since the isomorphism
$\alpha \to \tilde{\alpha}$ is type-preserving, we have $(d''
\tilde{\alpha}) = \tilde{d}'' \tilde{\alpha}$ where $\tilde{d}''$
corresponds to the derivation law $D$ in $\mathscr{L}_{G}(P)$. Hence
$d'' \tilde{\sigma} = 0$. 

Let $U$ be an open subset of $V$ over which $E$ is trivial and let
$\sigma_1,\ldots$, $\sigma_n$ be holomorphic sections on $U$ which form
a basis for the $\mathscr{U}(U)-$ module $\varepsilon (U)$ of sections
of $E$ over $U$. Then for any\pageoriginale section $\sigma = \sum f_i \sigma_i$
over $U$ we have 
\begin{align*}
  d'' \sigma & = \sum (d'' f_i) \wedge \sigma_i + \sum f_i \wedge d''
  \sigma_i ~\text{(Ch.\ref{chap1:sec8})}\\ 
  & = \sum d'' f_i \wedge \sigma_i \text{
    Prop. \ref{chap6:sec5:prop4}; Ch. 6.5}\\ 
  & = \sum (d'' f_i) \sigma_i
\end{align*}

In other words, $d''$ depends only on the manifold $V$ and \textit{
  not } on the principal bundle $P$. Moreover it is obvious that
$d''^2 \sigma = 0$. 

However, this can be proved algebraically. In fact, if $\gamma,
\hat{\gamma}$ are of type $(1, 0)$, so is $\gamma - \hat{\gamma} =
\omega$. From this it follows that that induced derivation laws are
equivalent and the corresponding $d''$ are the same
(Ch. \ref{chap6:sec3}). Furthermore, since $K^{0, 2} = 0$ (which is a consequence
of prop. \ref{chap6:sec4:prop3}. Ch.6.4 $\gamma$ being of type (1,
0)) we have  seen $d''^2 = 0$ in Ch.\ref{chap6:sec3}. We have
therefore a complex  
$$
\varepsilon^{p, 0} (U) \xrightarrow{d''} \varepsilon^{p, 1} (U)
\xrightarrow{d''} \cdots \varepsilon^{p, q}(U) \xrightarrow{d''}
\cdots 
$$
for every open subset $U$ of $V$, where $\varepsilon^{p, q} (U)$ is
the module of differential forms of type $(p, q)$ on $U$ with values
in $E$. This defines a complex $S^P (E)$ of sheaves over $V$: 
$$
\varepsilon^{p, 0} \xrightarrow{d''} \varepsilon^{p, 1}
\xrightarrow{d''} \cdots \cdots \varepsilon^{p, q} \xrightarrow{d''}
\cdots \cdots 
$$
where $\varepsilon^{p, q}$ denotes the sheaf of differential forms of
type $(p, q)$ on $V$ with values in the vector bundle $E$. Let
$\mathscr{U}^{p, q}$ be the sheaf of\pageoriginale complex valued differentiable
functions of type $(p, q)$ on $V, \mho$ the sheaf of holomorphic
functions on $V, \varepsilon_h$ the sheaf of holomorphic sections of
$E$. Then $\varepsilon^{p, q}$ is isomorphic to the sheaf
$\mathscr{U}^{p, q} \underset{\mho}\otimes \varepsilon_h$ and
therefore is a fine sheaf. Moreover if $\mho^p$ is the sheaf of
holomorphic forms of degree $p$ on $V$ then the sequence 
$$
0 \to \mho^p \underset{\mho}\otimes \varepsilon_h \to \varepsilon^{p,
  0} \xrightarrow{d''} \varepsilon^{p, 1} \to \cdots 
$$
is exact Dolbeault theorem [12]). Therefore the complex
$\mathscr{T}^p(\varepsilon)$ is a fine resolution of the sheaf $\mho^p
\underset{\mho}\otimes \varepsilon_h$ and if $S^p (\varepsilon)$ is
the complex of sections 
$$
\varepsilon^{p, 0} (V) \xrightarrow{d''} \varepsilon^{p, 1} (V)
\xrightarrow{d''} \cdots \varepsilon^{p, q}(V) \xrightarrow{d''}
\cdots  
$$

Then
$$
H^q (S^p (\varepsilon)) \simeq H^q (C, \mho^p \otimes \varepsilon_h)
$$
for every pair of integers $p, q > 0$.

Now assume $E$ to be the adjoint bundle ad $(P)$ to $P$. We shall
construct a canonical cohomology class in $H^1 (S^1 (ad(P)))$. Let
$\gamma$ be a connection form of type (1, 0) on $P$ and $K = K^{2,0}
+ K^{1,1}$ its curvature form. As a $G$-form on $P$ with values in the
Lie algebra $\mathscr{Y}, K$ corresponds to an alternate form $\chi$
of degree 2 on $V$ with values in $ad(P)$. Since $d'' K^{1,1} = 0$ 
(Ch. \ref{chap6:sec3}), we have $d'' \chi ^{1,1} = 0$. In other words,
$\chi^{1,1}$ as a 1-cocycle of the complex $S^1 (ad(P))$. 

Moreover\pageoriginale if $\hat{\gamma}$ is another connection of type $(1, 0),
\hat{\chi}^{1,1}$ differs from $\chi^{1,1}$ by a coboundary of the
complex $S^1 (ad(P))$. We have therefore associated a unique
1-cohomology class in $S^1 (ad(P))$ to the holomorphic bundle $P$. To
the cohomology class of $\chi^{1,1}$ corresponds an element of $H^1
(V, \mho^1 \underset{\mho}\otimes (ad (P))_h)$. This will be referred
to as the \textit{Atiyah class} of the principal bundle $P$. 

Regarding the existence of a holomorphic connection on the bundle $P$
(i.e., a connection such that $\gamma$ is a holomorphic form) , we have
the 

\setcounter{theorem}{0}
\begin{theorem}\label{chap6:sec5:thm1}%theo 1
  There exists a holomorphic connection in a holomorphic bundle $P$ if
  and only if the Atiyah class $a(P)$ of the bundle is zero. 
\end{theorem}

In fact, if $\gamma$ is a connection form on $P$ of type $(1, 0),
\gamma$ is holomorphic if and only if $d'' \gamma = 0$. Since $K^{0,
  2} = 0$, we see that $d'' \gamma = K^{1,1} = 0$ and hence
$\chi^{1,1} = 0$. 

Conversely, if $a(P) = 0$ and $\gamma$  a connection form of type
$(1, 0)$ on $P$, we have $\chi^{1, 1} \sim 0$. Hence there exists a
form $\alpha \in (ad P)^{1, 0}(V)$ such that $d'' \alpha =
\chi^{1,1}$. Then $\hat{\gamma} = \gamma - \tilde{\alpha}$ is a
connection form on $P$ such that $d'' \hat{\gamma} = 0$. 

\begin{coro*}%corolry
  There always exists a holomorphic connection on a holomorphic bundle
  $P$ over a Stein manifold $V$. 
\end{coro*}

In fact, the sheaf $\mho^1 \underset{\mho}\otimes (ad \mathscr{G})_n$
is a coherent sheaf and therefore $a(P)=0$, since $V$ is a Stein
manifold. By Theorem \ref{chap6:sec5:thm1}, there exists a holomorphic connection on
$P$. 

\setcounter{section}{6}
\section{Atiyah obstruction}\label{chap6:sec7}%sec 6.7

Let\pageoriginale $0 \to \mathscr{F}^1 \to \mathscr{F} \to
\mathscr{F}'' \to 0$ be 
an exact sequence of locally free sheaves of $\mho$ modules over a
holomorphic manifold $V$. Since $\mathscr{F}'$ is locally free, the
corresponding sequence 
$$
0 \to \Hom_\mho (\mathscr{F}'' , \mathscr{F}') \to \Hom_\mho
(\mathscr{F},\mathscr{F}') \to \Hom_\mho (\mathscr{F}', \mathscr{F}')
\to 0 
$$
is exact. This gives rise to the exact sequence  
\begin{gather*}
0 \to H^o (V, \Hom_\mho (\mathscr{F}'', \mathscr{F}')) \to \cdots \to
H^o (V, \Hom_\mho \mathscr{F}',\mathscr{F}')) \\
\to H^1 (V, \Hom_\mho (\mathscr{F}'', \mathscr{F}') \to \cdots 
\end{gather*}
$H^o (V, \Hom_\mho (\mathscr{F}', \mathscr{F}'))$ is then the module of
sections of the sheaf $\Hom_\mho$ $(\mathscr{F}',\mathscr{F}')$. The
image of the identity section of $\Hom_\mho (\mathscr{F}'\mathscr{F}')$
is called the \textit{obstruction} to the splitting of the given
sequence. 

Let $\mathcal{K}_h, \mathscr{T}_h, \mathscr{C}_h$ be the sheaves of
holomorphic vector fields on $P$ tangential to the fibres, of all
holomorphic invariant vector fields on $P$ and of all holomorphic
vector fields on $V$ respectively. Then we have the exact sequence 
$$
0 \to \mathcal{K}_h \to \mathscr{T}_h \to \mathscr{C}_n \to 0
$$

Let $b(P)$ be the obstruction to the splitting of this exact
sequence. $b(P)$ is then a class in $H^1 (V, \Hom_\mho (\mathscr{C}_h,
\mathcal{K}_h))$. We shall call this the \textit{Atiyah obstruction
  class.} 

\begin{theorem}\label{chap6:sec7:thm2}%the 2
  The\pageoriginale necessary and sufficient condition for a holomorphic connection
  to exist on $P$ is that $b(P) = 0$. 
\end{theorem}

In fact, $b(P) = 0$ if and only if the above sequence splits and the
rest of the proof is exactly as for the differentiable case. 

\begin{theorem}\label{chap6:sec7:thm3}%the 3
  There exists a canonical isomorphism $\rho$ of the sheaf
  $\mathcal{R} = \mho^1 \underset{\mho}\otimes (ad P)_h$ onto
  $\Hom_\mho (\mathscr{C}_h , \mathcal{K}_h)$ such that $\rho a (P) = -
  b(P)$. 
\end{theorem}

We shall define $\rho _U  : \mathcal{R} (U) \to \Hom_{\mho (U)}
(\mathscr{C}_h (U), \mathcal{K}_h (U))$ for every open subset $U$ of
$V$. Every $\omega \in \mathcal{R}(U)$ is a differential form on $U$
with values i the adjoint bundle and $\tilde{\omega}$ is a $G$-form on
$p^{-1}(U)$ with values in $\mathscr{Y}$. For every $X \in
\mathscr{C}_h (U)$ and $\xi \in p^{-1}(U)$, we define $(\rho_U
(\omega) (X))_\xi = \dfrac{1}{2} \xi (\omega X_\xi - iI_G \omega
X_\xi)$. It is easily seen that $\rho_U$ is an isomorphism and that it
defines an isomorphism 
$$
\rho : \mathcal{R}\to \Hom_\mho (\mathscr{C}_h, \mathcal{K}_h)
$$

Let $\{U_i\}$ be a covering of $V$ by means of open sets $U_i$, over
each of which $P$ is holomorphically trivial. We shall compute $a(P),
b(P)$ as cocycles of the above covering with values in the sheaves
$\mathcal{R}, \Hom_\mho (\mathscr{C}_h,\break \mathcal{K}_h)$ with respect to
the above covering. Let $\gamma_i$ be holomorphic connections on
$p^{-1}(U_i)$ and $\gamma$ a differentiable connection on
$P$. $K^{1,1}$ is the element $d'' \gamma $ in $(ad P)^{1,1} (V)$. If
$\gamma = \gamma_i + \alpha_i$ over $p^{-1}(U_i)$ with $\alpha_i \in
(ad P)^{1, 0}(V)$, then $K^{1,1}$ is the $d''-$ image of $\alpha_i$ on
$U_i$. For, $d'' \gamma = d'' \tilde{\alpha}_i = d'' \tilde{\alpha}_i$
(in our\pageoriginale usual notation). Hence $a(P)$ is represented in $H^1 (X,
\mathcal{R})$ with respect to the above covering by the cocycle
$\alpha_i - \alpha_j = - (\gamma_i - \gamma_j)$ on $U_i \cap U_j$. 

On the other hand,in the exact sequence
$$
0 \to \Hom_\mho (\mathscr{C}_n , \mathcal{K}_h) \to \Hom_\mho
(\mathscr{I}_h,\mathcal{K}_h) \to \Hom_\mho (\mathcal{K}_h,
\mathcal{K}_h) \to 0  
$$
the identity section of $\Hom_\mho (\mathcal{K}_h, \mathcal{K}_h)$ can
be lifted on $U_i$ into the map $\Gamma_i \in \Hom_\mho (\mathscr{T}_h,
\mathcal{K}_h)$ where $\Gamma_i (X)_\xi = \dfrac{1}{2} \xi (\gamma_i
(X_\xi) - iI_G \gamma (X_\xi))$ for $X \in \mathscr{T}_h, \xi \in
P$. Hence the obstruction class is represented by $\lambda_{ij}$ on
$U_i \cap U_j$ defined by 
$$
\lambda_{i, j} (pd \xi) = \frac{1}{2} \xi ((\gamma_i - \gamma_j) \,(d
\xi)-iI_G (\gamma_i - \gamma_j )(d \xi)) 
$$
for every $d \xi \in T_\xi$. This represents the cocycle of the class
$b(P)$ for the covering $\{U_i\}$. Obviously $\rho (a(P)) = - b(P)$. 

\section{Line bundles over compact Kahler manifolds}\label{chap6:sec8}%sec 6.8

Let $P$ be a holomorphic principal bundle over a manifold $V$ with
group $G = C^* (C^* = GL (1, C))$. We shall compute the Atiyah class
$a(P)$ for such a bundle. Let $\{U_i\}$ be a covering of $V$ over each
of which $P$ is trivial with holomorphic sections $\sigma_i$ on $U_i$
and a set of holomorphic transition functions $\{m_{ij}\}$. Since the
adjoint representation is trivial, the adjoint bundle is also trivial
and the $G$-functions and $G$-forms on $P$ are respectively functions and
forms on $P$ which are constant on each fibre. Let $\gamma$ be a
connections form of type\pageoriginale $(1, 0)$ on $P$. It is easy to see that $d''
(\tilde{\sigma}_i ^* \gamma) = d'' \gamma$. But $\sigma^*_j \gamma -
\sigma^*_i \gamma = m_{ij} ^{-1} dm_{ij} = d (\log m_{ij})$. Thus the
forms $d (\log m_{ij})$ form a cocycle representing the Atiyah class
for the covering $\{U_i\}$. 

Finally, when the base manifold is compact Kahler $H^1 (V, \mathcal{K}
= \mho^1)$ may be identified (\cite{30}) with a subspace of $H^2 (V, C)$
by Dolbeault's theorem. The sequences 
\begin{equation*}
  0 \to Z \to \mho \xrightarrow{e^{2 \pi i}} \mho^* \to 0
  \tag{1}\label{chap6:sec8:eq1} 
\end{equation*}
and
\begin{equation*}
  0 \to \mathbb{C} \to \mho \xrightarrow{d} \mho^1 \to 0
  \tag{2}\label{chap6:sec8:eq2}  
\end{equation*}
where $Z, \mathbb{C}$ are constant sheaves and $\mho^*$ is the sheaf
of nonzero holomorphic functions, can be imbedded in the commutative
diagram 
\[
\vcenter{\xymatrix@R=.3cm{0 \ar[r]& Z\ar[dd]_j\ar[r] & \mho \ar[dd]_i\ar[r]&
  \mho^\ast\ar[dd]^{\frac{1}{2 \pi i}\, d \log} \ar[r] & 0\\
  & & & &  \ar@{}[d]\\
0 \ar[r]& \mathbb{C} \ar[r] & \mho \ar[r] & \mho' \ar[r] & 0
}}\tag{3}\label{chap6:sec8:eq3}
\]
where $j, i$ are respectively the inclusion and identity maps. We get
consequently the commutative diagram 
\[
\xymatrix{H^1 (V, \mho^\ast)\ar[d]_\delta \ar[r]^\alpha &
  H^2(V,Z)\ar[d]^\beta\\
H^1 (V, \mho') \ar[r]^\gamma & H^2 (V, C)}
\]
where\pageoriginale $\alpha$ is the connecting homomorphism of the exact sequence
(\ref{chap6:sec8:eq1}), $\beta, \delta$ maps induced by diagram
(\ref{chap6:sec8:eq3}), and $\gamma$ the
injection given by Dolbeault's theorem (\cite{29}). If $\{m_{ij}\}$ are the
multiplicators for a covering $(U_i)$ of $V, \delta (m_{ij})$ is given
by $\dfrac{1}{2 \pi i} d \log (m_{ij})$. The bundle $P$ may be
regarded as an element of $H^1 (V, \mho^*)$ and its image by $\alpha$
is the first integral Chern class of $P$ and its image by $\beta$ is
the Chern class $C(P)$ with complex coefficients. By the commutativity
of the diagram, we have $a(P) = 2 \pi i C(P)$. 

In particular, we obtain the result that if there exists a holomorphic
connection in a line bundle over a compact Kahler manifold, then its
Chern class with complex coefficients $=0$. It has been proved under
more general assumptions on the group of the bundle that the existence
of holomorphic connections implies the vanishing of all its Chern
classes \cite{2}. 


\begin{thebibliography}{99}
\bibitem{1}{W. Ambrose and I Singer}\pageoriginale - A theorem on holonomy -
  Trans. Amer. Math. Sec. 75(1953) 428 -443 
\bibitem{2}{M.F.Atiyah} - Complex analytic connections in fibre
  bundles - Trans. Amer. Math. Sec. 85(1957) 181-207 
\bibitem{3}{''} - On the Krull-Schmidt theorem with applications to
  sheaves - Bull. Soc. Math. 84(1956) 307-317 
\bibitem{4}{N. Bourbaki} - Algebra multilineare
\bibitem{5}{E. Cartan} - Les groups d' holonomie des espaces
  generalises- Acta Methematica 48(1926) 1-42 
\bibitem{6}{H. Cartan} - La transgression dans un groups de Lie etdans
  un espace fibre principal - COll.  Top. Bruxelle (1950)-57-71 
\bibitem{7}{`'} - Notions d' algebre differentielle; applications aux
  groups de Lie et aux varietes ou opere un groupe de Lie (Espaces
  fibres ) - Coll. Top. Bruxelles (1951) 15 - 27 
\bibitem{8}{S.S.Chern} - Differential geometry of fibre bundles-
  Proc. Int. Cong. Math. (1950) 397 - 411  
\bibitem{9}{S.S.Chern}\pageoriginale - Topics in differential geometry - Institute
  for Advanced Study, Princeton, 1951 
\bibitem{10}{''} - Notes on differentiable manifolds, Chicago
\bibitem{11}{C. Chevalley} - Theory of Lie groups, Princeton, 1946
\bibitem{12}{P. Dolbeault} - Forms differentielles et cohomologie
  surune variete analytique complex I Annals of Math, 64(1956) 83 -
  130 
\bibitem{13}{C.H. Dowker} - Lectures on Sheaf theory, Tata Institute
  of Fundamental Research, Bombay, 1956 
\bibitem{14}{C.Ehresmann} - Sur les arcs analytiques d'un espace de
  Cartain - C.R.Acad. Sciences 206(1938) 1433 - 1436 
\bibitem{15}{A. Frolicher and A. Nijenhuis} - Theory of vector -
  valued differential forms - Indagations Mathematicae 18(1956) 338 -
  359 
\bibitem{16}{''} - Some new cohomolgy invariants for complex manifolds
  II - Indagationes Mathematicas 18 (1956) 553 - 564 
\bibitem{17}{R. Godement} - Theorie des faisceaux, Paris, 1958 
\bibitem{18}{A. Grothendieck} - A general theory of fibre spaces with
  structure sheaf - University of Kanasa report No. 4, 1955 
\bibitem{19}{S. Kobaysashi} - Theory of connections, University of
  Washington, 1955 
\bibitem{20}{J.L. Koszul}\pageoriginale - Multiplicateurs et classes
  caracteristiques -Trans. Amer. Math. Soc. 89 (1958) 256 - 267 
\bibitem{21}{A. Lichnerowicz} - Theorie globale des connexions et des
  groupes d' holonomie, Paris, 1955 
\bibitem{22}{B. Malgrange} - Lectures on the theory of several complex
  variables - Tata Institute of Fundamental Research, Bombay, 1958 
\bibitem{23}{S. Murakami} - Algebraic study of fundamental
  characteristic classes of sphere bundles - Osaka Math. Journal
  8(1956) 187 - 224 
\bibitem{24}{S. Nakano} - On complex analytic vector bundles -
  Jour. Math. Soc. Japan 7(1955) 1 - 12 
\bibitem{25}{''} -  Tangential vector bundles and Todd canonical
  systems of an algebraic variety - Mem. Coll. Sci. Kyoto 29(1955)
  145 - 149 
\bibitem{26}{K. Nomizu} - Lie groups and differential geometry Pub, of
  the Math. Soc, of Japan, 2, 1956 
\bibitem{27}{G. de Rham} - Sur la reductibilite dun espace de Riemann-
  Comm. Math. Helv. 26(1952) 328-344 
\bibitem{28}{''} - Varietes differentiables, Paris, 1955
\bibitem{29}{N. Steenrod} - The topology of fibre bundles, Princeten,
  1951 and 1957 
\bibitem{30}{A. Weil} - Varietes Kahleriennes, Paris, 1958
\bibitem{31}{H.Yamabe} - On an arewise connected subgroup of a Lie
  group - Osaka Math. Journal 2(1950) 13 - 14. 
\end{thebibliography}
