
\chapter{Differentiable Bundles}\label{chap2} % chap 2

\section{}\label{chap2:sec1} % section 2.1

We\pageoriginale give in this chapter, mostly without proofs, certain
definitions and results on fibre bundles which we require in the sequel. 
 
\setcounter{defn}{0}
\begin{defn}\label{chap2:sec1:def1} %definition 1
  \em{ A differentiable principal fibre bundle } is a manifold $P$ on
  which a Lie group $G$ acts differentiable to the right, together
  with a differentiable map $p$ of $P$ onto a differentiable manifold
  $X$ such that  

  \textbf{P.B.} for every $ x_0 \in X $, there exist an open
  neighbourhood $U$ of $x_0$ in  $X$ and a diffeomorphism $\gamma$ of $
  U \times G \to p^{-1} (U) $ ( which is an open submanifold of $P)$
  satisfying $ p \gamma ( x,s ) = x , \gamma ( x, st ) = \gamma (x,s)
  t $ for $ x \in U $ and  $ s, t \in G $. 
\end{defn} 
 
 $X$ shall be called the \textit{ base }, $p$ the \textit{projection}
and $P$ the bundle. For any $ x \in X , p^{-1}(x) $ shall be called
the  \textit{fibre over $x$ } and for $ \xi \in P $, the fibre over $
p \xi $ is the \textit{fibre through} $\xi$.  
 
The following properties follow immediately from the definition:
 
\begin{enumerate}[a)]
\item Each fibre is stable under the action of $G$, and $G$ acts with
  out fixed points on $P$, i.e. if $ \xi s = \xi $ for some $ \xi \in
  P$ and  $ s \in G $, then $s = e$; 
\item $G$ acts transitively on each fibre, i.e. if $ \xi, \eta $ are
  such that\pageoriginale $ p \xi = p \eta $, then there exists $ s \in G $ such
  that $ \xi = \eta$; 
\item For every  $x_0 \in X$, there exist an open neighbourhood $V$
  of $x_0$ and a  differentiable map $ \sigma : V \to P $ such that $
  p \sigma (x) = x$ for every $ x \in V $. We have only to choose for
  $V$ the neighbourhood $U$ of condition  (P.B), and define for $ x
  \in V, \sigma (x) = \gamma ( x,e ) $. A continuous (\resp
  differentiable) map $ \sigma : V \to P $ such that  $ p 
  \sigma (x) = x $ for every $x \in V$ is called a \textit{
    cross-section \resp differentiable cross-section }) over $V$. 
\item For every $ x_0 \in X $, there exist an open neighbourhood $V$
  of $x_0$ and a differentiable map $\rho$ of $p^{-1} (V) $ into $G$
  such that $ \rho ( \xi s ) = \rho (\xi) s $ for every $ \xi \in
  p^{-1} (V) $ and $ s \in G $. Choose for $V$ as before the
  neighbourhood $U$ of (P.B.). If $ \pi $ is the canonical
  projection $ U \times G \to G $, the map $ \rho = \pi \circ \gamma^{-1}
  $ of  $ p^{-1} (V) $ into  $G$ satisfies the required condition. It
  is also obvious that $\rho$ is bijective when restricted to any
  fibre. 
\end{enumerate}   
   
Conversely we have the following
\begin{proposition}\label{chap2:sec1:prop1}%proposition 1
  Let $G$ be a Lie group acting differentiably to the right on a
  differentiable manifold $P$. Let $X$ be  another differentiable
  manifold and $p$ a  differentiable map $ P \to X $. If conditions
  $(b) , (c), (d)$ are fulfilled, then $P$ with $ p: P \to X $ is a
  principal bundle over $X$. 
\end{proposition}   
   
For every $ x_0 \in X $, we can find an open neighbourhood $V$ of
$x_0$, a differential map $ \sigma : V \to P $ and a homomorphism $
\rho : p^{-1} (V) \to G $ such that $ p \sigma (x) = x, \rho (\xi s )
= \rho (\xi) s $ and moreover\pageoriginale $\rho \sigma (x) = e $ for every $ x
\in V , \xi \in p^{-1} (V) $. We  define  $ \gamma : U \times G \to P
$ by setting $ \gamma (x,s) = \sigma (x)s $ for $ x \in V ,  s \in G
$. If $\theta$ is the map $ p^{-1} (V) \to V \times G $ defined by $
\theta : \xi \to ( p \xi ,\rho  \xi ) $ it is easy to verify that  $
\theta \gamma = \gamma \theta = $ Identity, using the fact  that  $
\rho \sigma (x) = e $ for every $ x \in U $. Both $\theta$ and $\gamma$
being differentiable, our assertion is proved. 
   
\section{Homomorphisms of bundles}\label{chap2:sec2} %section 2.2

\begin{defn}\label{chap2:sec2:def2}%definition 2
  A \em{homomorphism} $h$ of a differentiable principal bundle $P$
  into another bundle $P'$ ( with the same group $G)$ is a
  differentiable map $ h : P \to P ' $ such that $ h ( \xi s ) = h
  (\xi) s $ for every $ \xi \in P $, $ s \in G $.  
\end{defn}

It is obvious that points on the same fibre are taken by $h$ into
points of $P'$ on the same fibre. Thus the homomorphism $h$ induces a
map $\underbar{h} : X \to X' $ such that the diagram 
\[
\xymatrix{P \ar[d]_p\ar[r]^h & P'\ar[d]^{p'}\\
X \ar[r]^h & X'}
\]
is commutative. The map $ \underbar{h} : X \to X' $ is easily seen to
be differentiable. This is called the \textit{projection} of $h$. 

\begin{defn}\label{chap2:sec2:def3} %definition 3
  A homomorphism $ h : P \to P'$ is said to be an
  \textit{isomorphism} if there exists a homomorphism $ h' : P' \to P
  $ such that $hoh'$l $h' oh$ are identities on $P'$, $P$ respectively. 
\end{defn}

\begin{proposition}\label{chap2:sec2:prop2} %proposition 2
  If $P$ and  $ P '$ are differentiable principal bundles  with\pageoriginale the
  same base $X$ and group $G$, then every homomorphism $h: P \to P'$
  whose projection $\bar{h}$ is a diffeomorphism of $X$ onto $X$, is
  an isomorphism.  
\end{proposition}

In the case when $P$ and $P'$ have the same base $X$, an isomorphism
$h : P \to P'$ for which $\bar{h}$ is identity will be called an
isomorphism $\bar{over} X$.  

\section{Trivial bundles}\label{chap2:sec3}

If $G$ is a Lie group and $X$ a differentiable manifold, $G$ acts on
the manifold $X \times G$ by the rule $(x, s) t = ( x , st) $. $X
\times G$ together with the natural projection $X \times G \to X $ is
a principal bundle. Any bundle isomorphic to the above is called
\textit{ trivial principal bundle}.
  
\begin{proposition}\label{chap2:sec3:prop3} %proposition 3
  Let $P$ be a principal bundle over $X$ with group $G$. Then the
  following statements are equivalent:  
  \begin{enumerate}[ 1)]
  \item $P$ is a trivial bundle. 
  \item There exists a differentiable section of $P$ over $X$. 
  \item There exists a differentiable map $\rho : P \to G$ such that
    $\rho ( \xi s) = \rho (\xi )s$ for every $\xi \in P, s \in G$.  
  \end{enumerate}
\end{proposition}

\section{Induced bundles}\label{chap2:sec4}

Let $P$ be a principal bundle over $X$ with group $G$. Let $q$ be a
differentiable map of a differentiable manifold $Y$ into $X$. The
subset $P_q$ of $Y \times  P$ consisting of points $(y, \xi)$ such
that $q(y) = p(\xi)$ is a closed submanifold. There is also a
canonical map $p_q : P_q \to Y$ defined by $P_q (y, \xi) = y$. The
group $G$ acts on $P_q$ with\pageoriginale the law $(y , \xi )s = (y, \xi s)$. It
is easy to see that $P_q$ together with $p_q$ is a principal bundle
with base $Y$ and group $G$. This is called the bundle induced from
$P$ by the map $q$. There then exists clearly a canonical homomorphism
$h : P_q \to P$ such that  
\[
\xymatrix{P_q \ar[d]_{p_q} \ar[r]^h & P\ar[d]^p\\
Y \ar[r]^{h= q} & X}
\]
is commutative. $h$ is defined by $h(y , \xi ) = \xi$. By proposition
(\ref{chap2:sec3:prop3}). Ch.2.3, the principal bundle $P_q$ is trivial if and only if
there exists a differentiable cross section for $P_q$ over $Y$. This
is equivalent to saying that there exists a differentiable map
$\lambda : Y \to P$ such that $p = p \circ  \lambda $, i.e., the diagram  is
commutative. 

\[
\xymatrix{& P\ar[d]_p\\
Y \ar[ur]^\lambda \ar[r]_q & X
}
\]

We now assume that $q$ is surjective and everywhere of $ rank = \dim X
$. If $P_q$ is trivial we shall say that $P$ is \textit{trivialised}
by the map $q$.

Let $q$ be a differentiable map $Y \to X$  which trivialises
$P$. Consider the subset $Y_q$ of $Y \times Y$ consisting of points
$(y, y')$ such that $q(y) = q(y')$. This is the graph of an
equivalence relation in $Y$. Since $q$ is of $rank = \dim X, Y_q$
is a closed submanifold of $Y \times Y$.

Let $\lambda$ be a map $Y \to P$ such that $q = p \circ \lambda $. If $(y,
y')\in Y_q$,\pageoriginale then $\lambda(y), \lambda (y')$ are in the
same fibre and 
hence there exists $m(y, y')\in G$ such that $\lambda (y') = \lambda
(y) m (y, y')$. Thus we have a map $m : Y_q \to G$ such that for $(y,
y')\in Y_q$, we have $\lambda (y') = \lambda (y) m (y ,
y')$. This map is easily seen to be differentiable. Obviously we
have, for $(y, y'), (y', y'') \in  Y_q , m (y, y') m (y' , y'') = m
(y, y'')$.  

\begin{defn}\label{chap2:sec4:def4} %definition 4
  Let $Y$ be differentiable manifold and $q$ a differentiable map of
  $Y$ onto $X$ everywhere of $rank = \dim X$. The manifold $Y_q$ is
  defined as above. Any differentiable map $m : Y_q  \to G$ is said to
  be a \textit{ multiplicator  }with value in $G$ if it satisfies  
  $$
  m (y, y') ~m(y' , y'') = m(y, y'') ~\text{ for }~ (y, y') , (y'.  y'')\in Y_q.
  $$

  We have seen that to every trivialisation of $P$ by $q$ corresponds a
  multiplicator with values in $G$. However, this depends upon the
  particular lifting $\lambda$ of $q$. If $\mu$ is another such
  lifting with multiplicator $n$, there exists a differentiable map
  $\rho : Y \to G$ such that $m(y', y) \rho (y) =\rho (y') n (y' , y)$
  for every $(y, y') e Y_q$. Accordingly, we define an
  equivalence relation in the set of multiplicators in the following
  way:  

  The multiplicators $m, n$ are {\em equivalent } if there exists a
  differentiable map $\rho : Y \to G$ such that $m(y, y') \rho (y') =
  \rho (y) n (y, y')$ for every $(y, y') \in Y_q$. Hence to every
  principal bundle $P$ trivialised by $q$ corresponds a class of
  multiplicators $m(P)$. It can be proved that if $m (P) = m(P')$,
  then $P$ and $P'$ are isomorphic over $X$. Finally, given a
  multiplicator $m$, there exists bundle a $P$ trivialised by $q$ for
  which $m(P) = m$. In fact, in the space $Y \times G$,
  introduce\pageoriginale an 
  equivalence relation $R$ by defining $(y, s ) \sim (y' , s')$ if $(y,
  y') \in Y_q$ and $s' = m (y' , y) s$. Then the quotient $(Y \times
  G)/R$ can be provided with the structure of a differentiable
  principal bundle over $X$ trivialised by $q$. The multiplicator
  corresponding to the map $\lambda : Y \to (Y \times G)/R$ defined by
  $y \to (y, e)$ is obviously $m$.  
\end{defn}

\section{Examples}\label{chap2:sec5}

\begin{enumerate}[1)]
\item Given a principal bundle $P$ over $X$, we may take $Y = P$ and
  $q = p$. Then $\lambda = $Identity is a lifting of $q$ to $P$. The
  corresponding multiplicator $m$ is such that $y' = y ~m (y, y')$
  where $p(y) = p (y')$ 
\item Let $(U_i)_{ i \in I}$ be an open cover of $X$ such that there
  exists a cross section $\sigma _i $ of $P$ over each $U_i$. Take for
  $Y$ the open submanifold of $X \times I$ (with $I$ discrete)
  consisting of elements $(x, i)$ such that $x \in
  u_i$. Define $q (x , i) = x $. This is obviously surjective and
  everywhere of rank $= \dim X$. Then $P$  is trivialised by $q$, since
  the map $\lambda (x, i) = \sigma_i (x)$ of $Y \to P$  is a lifting
  of $q$. The manifold $Y_q$ may be identified with the submanifold
  of $X \times I \times I$ consisting of elements $x, i, j$ such that
  $x \in \cup_i \cap \cup_j$. If $m$ is the corresponding
  multiplicator, we have $\lambda 
  (x, i ) = \lambda (x, j ) m (x, j, i )$. This can be written as
  $\sigma_i (x)  = \sigma _j (x) m_{ji}(x)$ where the multiplicator
  $m$ is looked upon as a family of maps $m_{ji} : U_j \cap U_i \to G$
  such that $m_{ji} m_{ik}(x) = m_{jk}(x)$ for every $x \in U_i \cap
  U_j \cap U_k$. Such a family of maps is called a set of \textit{
    transition functions}. Two sets of transition functions\pageoriginale $\{
  m_{ij}\}, \{ n_{ij}\}$  are equivalent if and only if there exists a
  family of differentiable maps $\rho _i : U_i \to G$ such that
  $m_{ji}(x) \rho_i (x)  = \rho_j (x) n_{ji} (x)$ for every $x \in U_i \cap
  U_j$. Conversely, given a set of transition functions $\{m_{ij}\}$
  with respect to a covering $U_i$ of $V$, we can construct a bundle
  $P$ over $V$ such that $P$ is trivial over each $U_i$ and there
  exists cross - sections $\sigma _i$ over $U_i$ satisfying $\sigma _i
  (x) =\sigma _j (x) m_{ij} (x)$ for every $x \in U_i \cap U_j$ 
\end{enumerate} 

Let ${G}$ be the sheaf of germs of differentiable functions
on $X$ with values in $G$. The compatibility relations among transition
functions 
$$
\text{ viz }. m_{ji} (x) m _{ik} (x) = m _{j k }(x) \textit{ for every
} x \in U_i \cap U_j \cap U_k  
$$
only state that a set of transition functions is a $1-$cocycle of the
covering $( U_i) _{ i \in I}$ with values in the sheaf
$\underbar{G}$. Two such cocycles are equivalent (in the sense of
multiplicators) if and only if they differ by a coboundary. In other
words, the set of equivalent classes of transition functions for the
covering $(U_{i})_{ i \in I}$ is in one-one correspondence with
$H^1(( U_i) _{ i \in I}, G)$. It will be noted that there is no group
structure in $H^1 (( U_i)_{i \in I}, G)$ in general. It can be proved
by passing to the direct limit that there is a one- on correspondence
between classes of isomorphic bundles over $X$ and elements of $H^1
(X, \underbar{G})$.  

\section{Associated bundles}\label{chap2:sec6}

Let $P$ be a differentiable principal fibre bundle over $X$
with\pageoriginale group 
$G$. Let $F$ be a differentiable manifold on which $G$ acts
differentiably to the right. Then $G$ also acts on the manifold $P
\times F$ by the rule $(\xi, u) s = (\xi s, us)$ for every $s \in G$.  

\begin{defn}\label{chap2:sec6:def5} %definition 5
  A {\em differentiable bundle } with fibre type $F$ {\em associated
    to $P $ } is a differentiable manifold $E$ together with a
  differentiable map $q : P \times F \to E $  such that $(P \times F,
  q)$ is a principal fibre bundle over $E$ with group $G$.  
\end{defn}

Let $G$ act on a differentiable manifold $F$ to the right. Then we can
construct a differentiable bundle associated to $P$ with fibre $F$.
We have only to take $E = \dfrac{( P \times F)} {G}$ under the action
of $G$ defined as above and $q$ to be the canonical projection $P
\times F \to E$. The differentiable structure in $E$ is determined by
the condition: $(P \times F, q) $ is a differentiable principal bundle
over $E$.  

Now let $E$ be a differentiable bundle associated to $P$ with fibre type
$F$ and group $G$. Then there exists a canonical map $p_E : E \to X$
such that $p_E q (\xi , u) = p \xi$ for $(\xi , u ) \in P \times F$
where $p, q$ are respectively the projections $P \to X$ and $P \times
F \to E$. $X$ is therefore called the base manifold of $E$ and $p_E$ the
projection of $E$. For every $x \in X, p_E^{-1}(x)$ is called the
fibre over $x$.  Let $U$ be an open subset of $X$. A continuous (
resp.  differentiable ) map $\sigma : U \to E$ such that $p_E \sigma
(x) = x$  for every $x \in U$ is called a \textit{ section (resp
  differentiable section)}  of $E$ over $U$.  

Let $\sigma$ be a differentiable section of $P$ over an open subset of
$U$ of $X$. This gives rise to a diffeomorphism $\gamma$ of $U \times
F$ onto $p_E^{-1} (U)$ defined\pageoriginale by $\gamma (x, v) = q (\sigma (x) , v)$
for $x \in U, v \in F$. On the other hand, we also have $p_E \gamma (x
, v) = x$. In particular , if $P$ is trivial, there exists a global
cross-section $\sigma$ (Prop.\ref{chap2:sec3:prop3}, Ch. 2.3) and
hence $\gamma$ is a diffeomorphism of $U \times F$ onto $E$.  

We finally prove that all fibres in $E$ are diffeomorphic with $F$. In
fact for every $z \in P$. the map $F \to E$ (which again we denote by
$z$) defined by $z (v) = q ( z , v )$ is a diffeomorphism of $F$ Onto
$p^{-1}_E (p (z))$. Such a map $z : F \to E$ is called a
\textit{frame} at the point $x = p(z)$. Corresponding to two different
frames $z, z ' $ at the same point $x$, we have two different
diffeomorphisms $z, z' $ of $F$ with $p^{-1}_E (x)$. If $s \in G$ such
that $z ' s = z $, then the we have $z (v) = z' ( vs^{-1})$. 

\begin{examples*}
  \begin{enumerate}[(1)]
  \item Let $V$ be a connected differentiable manifold and $P$ the
    universal covering manifold of $V$. Let $p : P \to V$ be a
    covering map. Then the fundamental group $\pi_1 $ of $V$ acts on
    $P$ and makes of  $P$ a principal bundle over $V$ with group
    $\pi_1$. Moreover, any covering manifold is a bundle over $V$
    associated to the universal covering manifold of $V$ with
    discrete fibre. On the other hand, any Galois covering of $V$ may
    be regarded as a principal bundle over $V$ with a quotient of $\pi
    _1 $ as group. 
  \item Let $B$ be a closed subgroup of a Lie group $G$. Then $B$ is
    itself a Lie group and it acts to the right on $G$ according to
    the following rule: $G \times  B \to G$ defined by $( s, t) \to
    st$. Consider the quotient space $V = {G}/{B}$ under the
    above action. There exists one and only one structure of a
    differentiable manifold on $V$ such that $G$ is a differentiable
    bundle over $V$ with group $B$ (\cite{29}). It is
    moreover\pageoriginale easy to 
    see that left translations of $G$ by elements of $G$ are bundle
    homomorphisms of $G$ into itself. The projections of these
    automorphisms to the base space are precisely the translations of
    the left coset space ${G}/{B}$ by elements of $G$.  
  \item Let $G$ be a Lie group and $B$ a closed subgroup. Let $H$ be a
    closed subgroup of $B$. As in (2), ${B}/{H}$ has the structure
    of a differentiable manifold and $B$ acts on ${B}/{H}$ to the
    right according to the rule:   
    $$
    q (b) b' = b^{-1}  q (b) = q(b'^ {-1} b), 
    $$
    where $b, b' \in B$ and $q$ is the canonical projection $B \to
    {B}/{H}$. We define a map $r : G\times {B}/{H} \to
    {G}/{H} $ by setting $ r ( s, bH ) = sb H$. It is easy to see
    that this makes $G \times {B}/{H}$ a principal bundle over
    $\dfrac{G}{H}$. In other words, ${G}/{H}$ is a bundle
    associated to $G$ with base ${G}/{B}$ and fibre ${B}/{H}$.  
  \end{enumerate} 
\end{examples*}

\section{Vector fields on manifolds}\label{chap2:sec7}

Let $V$ be a differentiable manifold and $\mathscr{U} (V)$ the
algebra of differentiable functions on $V$. At any point $\xi \in V$,
a tangent vector $U$ is a map $U : \mathscr{U} (V) \to R$ satisfying
$U (f + g ) = Uf + Ug; U f = 0 $ when $f $ is constant; and $U (fg)  =
(Uf) g (\xi) + f (\xi)  (Ug) $ for every $f, g \in \mathscr{U}
(V)$. The tangent vectors $U$ at a point $\xi$ form a vector space
$T_{ \xi}$. A vector field $X$ is an assignment to each $\xi$ in $V$
of a tangent vector $X _\xi$ at $\xi$. A vector field $X$ may also be
regarded as a map of $\mathscr{U} (V)$ into the algebra of real valued
functions on $V$ by setting $(Xf)(\xi) = X_\xi f$. A vector field $X$
is a said to be differentiable\pageoriginale if $X \mathscr{U}(V)\subset \mathscr{U}
(V)$. Hence the set of differentiable vector fields on $V$ is the
module $\mathscr{C}(V)$ of derivations of $\mathscr{U}(V)$.  

If $p$ is a differentiable map from a differentiable manifold $V$ into
another manifold $V'$, we define a map $p^* : \mathscr{U} (V') \to
\mathscr{U}(V)$ by setting $p^*  f = fop $. Furthermore, if $\xi \in
V$, then a linear map of $T_\xi$ into $T_{ p \xi }$ (which is again
denoted by $p$) is defined by $(pU) g = U(p^* g)$.  

Now let $G$ be a Lie group acting differentiably to the right on a
manifold $V$. As usual, the action is denoted $(\xi , s ) \to \xi
s$. For every $(\xi , s ) \in  V \times G$, there are two inclusion
maps  
\begin{align*}
  V \to V \times G & \text{ defined by } \eta \to (\eta , s ); \text{ and }\\
  G \to V \times G & \text{ defined by } t  \to (\xi , t )
\end{align*}

These induce injective maps $T_\xi \to T_{ ( \xi , s)}, T_s \to T_{
  (\xi, s)}$ respectively. The image of $d \xi \in T_\xi$ in $T_{ (
  \xi, s)}$ is denoted by $(d \xi , s )$. The image of $ds  \in T_s$
in $T_{(\xi, s)}$ is denoted by $(\xi , ds)$. We set $(d \xi, ds ) =
(d \xi, s) + (\xi, ds )$. The image of $d \xi \in T_\xi$ by the map
$\eta \to \eta s $ of $V$ into $V$will be denoted by $d \xi
s$. Similarly, the image of  vector $ds \in T_s$ by the map $t \to \xi
t $ will be denoted $\xi  ds$. Therefore the image of the vector $(d
\xi ,  ds) \in T_{(\xi , s )}$ by the map $(\xi, s )\to \xi s $ is $d
\xi s + \xi ds$. In particular, the group $G$ acts on itself and such
expressions as dst and tds will be used in the above sense. The
following formulae are easy to verify:  
\begin{enumerate}[1)]
\item $(d \xi s ) t = (d \xi ) (st)$\pageoriginale 
\item $( \xi ds ) t =  \xi  (dst)$
\item $(\xi t ) ds =  \xi  (tds)$ for $\xi \in V$ and $s, t \in G$. 
\end{enumerate}

Let $G$ be a Lie group with unit element $e$. The space $T_e$ of
vectors at $e$ will be denoted by $\mathscr{Y}$. A vector field $X$ on
$G$ is said to be left invariant if $sX_t = X_{st}$ for every $s , t
\in G$. Every left invariant vector field is differentiable and is
completely determined by its value at $e$. Given $a \in \mathscr{Y}$,
we define a left invariant vector field $I_a$ by setting $(I_a)_s =
sa$.  This gives a natural isomorphism of $\mathscr{Y} $ onto the
vector space of left invariant vector fields on $V$. There is also a
similar isomorphisms of $\mathscr{Y}$ onto the space of right
invariant vector fields. In the same way, when $G$ acts on a manifold
$V$ to the right, for every $a \in \mathscr{Y}$, we define a vector
field $Z_a$ on $V$, by setting $(Z_a)_\xi = \xi $ a for $\xi \in
V$. Thus $\mathscr{Y} $ defines a vector space of vector fields on
$V$.  

\section{Vector fields on differentiable principal bundles}\label{chap2:sec8}

Let $P$ be a differentiable principal over a manifold $V$ with group
$G$. Then the projection $p:P \to V $ gives rise to a homomorphism
$p^* : \mathscr{U}(V) \to \mathscr{U } (P)$. Since $p$ is onto , $p^*$
is injective. This defines on every $\mathscr{U}(P)$- module a
structure of an $\mathscr{U}(V)$module. It is clear that any element
$h \in p^* \mathscr{U} (V)$ is invariant with respect to the action of
$G$ on $P$. Conversely, if $h \in \mathscr{U}(P)$, whenever $h$ is
invariant with respect to $G$ then $h \in p^* \mathscr{U}(V)$. In fact
if $f \in \mathscr{U}(V)$ is defined by setting $f(x) = h (z)$ where
$z$ is any\pageoriginale element in $p^{-1}(x)$, then $f$ coincides locally with the
composite of a differentiable cross- section and $h$, and is
consequently differentiable. For every $\xi \in P$, there exists a
natural linear map of $\mathscr{Y}$ into $T_\xi$ taking  $ a \in
\mathscr{Y}$ onto $\xi a \in T_{ \xi}$. It is easy to verify that the
sequence of linear maps  
$$
(0) \to\mathscr{Y } \to T_\xi \to T_{p (\xi)} \to  (0)
$$
is exact. The image of $\mathscr{Y}$ in $T_\xi$, i.e. the space of
vectors $\xi$ a with $a \in \mathscr{Y}$ is the space of vectors
\textit{ tangential } to the fibre $\xi G$ and will be denoted by
$\mathfrak{N}_\xi$.  

\begin{defn}\label{chap2:sec8:def6} %definition 6
  A vector field $X$ on $P$ is said to be {\em tangential to the
    fibres } if for every $x \in P, p (X_\xi) = 0 $.  
\end{defn}

A vector field $X$ is tangential to the fibres if and only if $X_\xi
\in \mathfrak{N}_\xi$ for every $\xi \in P$. It is immediate than an
equivalent condition for a vector field $X$ to be tangential to the
fibres is that $X(p^* \mathscr{U} (V)) = (0)$. We denote the set of
vector fields tangential to the fibres by $\mathfrak{N}$. Then
$\mathfrak{N}$ is an $\mathscr{U}(P)$-submodule  of $\mathscr{C}
(P)$. We have the following  

\begin{proposition}\label{chap2:sec8:prop4} %proposition 4
  If $a_ 1 , \ldots , a_r$  is a base for $\mathscr{Y}$, then $Z_{a_l}
  , \ldots Z_{a_r}$ is a basis for $\mathfrak{N}$ over $\mathscr{U}
  (P)$.  
\end{proposition}

In fact, for every $a \in \mathscr{Y} , (Z_{a} )_{\xi} = \xi a \in
\mathfrak{N} _\xi $, i.e., $Z _a \in \mathfrak{N}$. Moreover, if $f_1
, \ldots , f_r \in \mathscr{U}(P)$ are such that $\sum\limits^r _{ i =
  1} f_i Z_{a _i} = 0 $, then $\sum\limits^r_{ i = 1 } f_i (\xi )
(\xi a_i) = 0$ for every $\xi \in P$. Since $\left\{ \xi a_i \right\}^r
_{ i = 1 }$ form a bias for $\mathfrak{N}_\xi , f_i (\xi) = 0 $ for
$i=1 , 2,  \ldots , r$ and $Z_{a_i}, \ldots Z_{a _r}$\pageoriginale are linearly
independent. On the other hand, if $X \in \mathfrak{N}$, at each point
$\xi \in P$, we can write $X_\xi = \sum \limits^r _{ i = 1} f_i (\xi )
(\xi ) \xi a_i$, and hence $X = \sum \limits^r_{ i = 1} f_i Z_{a _i}$
where the $f_i$ are scalar functions on $P$. Let $\xi \in P$ and $g_1,
\ldots , g_r \in \mathscr{U}(P)$ such that $(Z_{ a_i})_\xi g_j  =
\delta _{ij}$ for every $i. j$; then the functions $f_i$ are solutions
of the system of linear equations:  
$$
X g_j = \sum^{r}_{ i = 1}f_1 (Z_{a_i} g_j). 
$$

Since the coefficient are differentiable and the determinant $|
Z_{a_i} g_j| \neq 0$ in a neighbourhood of $\xi , f_i$ are differentiable
at $\xi$.  

\section{Projections vector fields}\label{chap2:sec9}  

Let $P$ be a principal bundle over $V$ and $X$ a vector field on
$P$. It is in general not possible to define the image vector field
$pX$ on $V$. This is however possible if we assume that for all $\xi$
in the same fibre , the image $pX_\xi$ in the same. This yields the
following.  

\begin{defn}\label{chap2:sec9:def7}%definition 7
  A vector field $X$ on $P$ is said to be {\em projectable} if $p
  (X_\xi) = p (X_{\xi_s})$ for\pageoriginale every $\xi \in P, s \in G$.  
\end{defn}

\begin{proposition}\label{chap2:sec9:prop5}%proposition 5
  A vector field $X$ on $P$ is projectable if and only if $X p^*
  \mathscr{U}$ $(V) \subset p^* \mathscr{U} (V)$.  
\end{proposition}

This follows from the fact that 
$$
X_\xi (p^* f)= p(X_\xi) f = p(X_{\xi s}) f = X_{\xi s} (p^* f)
$$
 for every $f \in \mathscr{U}(V)$. 

\begin{proposition}\label{chap2:sec9:prop6} %proposition 6
  A vector field $X$ is projectable if and only if $Xs- X$ is
  tangential to the fibre, ie. $Xs - X \in \mathfrak{N}$ for every $s
  \in G$.  
\end{proposition} 
 
In fact, if $X$ is projectable , we have 
\begin{align*}
  p (Xs - X)_\xi & = p ( X _{ \xi s^{-1}} s - X_\xi )\\
  & = p (X_{\xi s^{-1}}) - p (X_\xi )\\
  & = 0. 
 \end{align*} 
 
Hence $(X s - X )_\xi \in \mathfrak{N} (\xi)$ for every $\xi \in P$. 
 
The converse is also obvious from the above. 

\begin{defn}\label{chap2:sec9:def8} %definition 8
  The {\em projection} $pX$ of a projectable vector field $X$ is
  defined by $(pX)_{p\xi} = p X_\xi$ for every $\xi \in P$.  
\end{defn}  
 
 Since $p^* \{ (pX )f \}  = X (p^* f)$ for every $f \in \mathscr{U}
 (V)$ we see that $p X$ is a differentiable vector field on $V$. We
 shall denote the space of projectable vector field by $\wp$. It is
 easy to see that if $X, Y \in \wp$, then $X + Y \in \wp$ and $p (X
 +Y) = pX + pY$. Moreover $\wp$ is a submodule of $\mathscr{C}(P )$
 regarded as an $\mathscr{U}(V)$-module (but \textit{not} an $
 \mathscr{U}(P)$-submodule). For $f \in \mathscr{U} (V)$ and
 $X \in \wp$, we have $p ((p^*f)X) = f (pX)$. Thus $p : \wp \to
 \mathscr{C}(V)$ is an $\mathscr{U}(V)$-homomorphism and the kernel is
 just the module $\mathfrak{N}$ of vector fields on $P$ tangential to
 the fibre. Furthermore, for every $X, Y \in \wp$, we have $[X , Y]
 \in \wp$ and $p [ X, Y] = [pX, pY]$.  

\begin{proposition}\label{chap2:sec9:prop7} %proposition 7
  If $V$ is paracompact, every vector field on $V$ is the image of a
  projectable  vector field on $P.$ i.e., $\wp \to \mathscr{C}(V) $ is
  surjective.\pageoriginale  

  Let $x \in V$ and $U$ a neighbourhood of $x$ over which $P$ is
  trivial. It is clear that any vector field $\underbar{X}$ in $U$ is
  the projection of a vector field $X$ in $p^{-1}(U)$. Using the fact
  that $V$ is paracompact we obtain that there exists a locally finite
  cover $(U_i)_{ i \in I}$ of $V$ and a family of projectable vector
  fields $X_i \in \wp$ such that $ p X_i  = \underbar{X}_i$ on $U_i$
  where $X_i$ coincides on $U_i$ with a given vector field $X$ on
  $V$. Let $(\varphi_i )_{ i \in I}$ be a differentiable partition of
  unity for $V$ with respect to the above cover. Then $X =
  \sum\limits_{ i \in I}(p^*\varphi_i)X_i$ is well-defined  and is in
  $\wp$ with projection $pX = \sum\limits_{i \in I}\varphi _ i p X_i =
  \underbar{X}$.  
\end{proposition}
 
\begin{example*}
  Let $G$ be a Lie group acting differentiably to the right on a
  differentiable manifold $V$. The action of $G$ on $V$ is given by
  the map $p: V \times G \to V$. Consider the manifold $V \times G$
  with the above projection onto $V$. $G$ acts to the right on $V
  \times G$ by the rule  
  $$
  (x, s )t = (xt, t ^{-1} s) \text{ for every } x \in V, s, t \in G. 
  $$
  
  The map $\gamma : V \times G \to V \times G$ defined by  $\gamma (x,
  s) = ( xs,  s^{-1})$ for $x \in V, s \in G$ is a diffeormorphism. We
  also have  
  $$
  \gamma (x , st) = \gamma ( x , s) t 
  $$

  This show that $V \times G$ is a trivial principal bundle over $V$
  with group $G$ and projection $p$. A global cross -section is given by
  $\sigma : x \to (x, e)$. 
\end{example*} 
 
 Let $\mathscr{Y} = T_e$ be the space of vectors at $e$ and $I_a$ the
 left invariant\pageoriginale vector field on $G$ whose value at $e$ is $a$. Let
 $(0, I_a)$ be the vector field on $V \times G$ whose value at $(x
 ,s)$ is $( x, sa)$. Then for every $a \in \mathscr{Y}, (0, I_a) $ is
 projectable; as a matter of fact , it is even right invariant. For, 
\begin{align*}
  (O, I_a)_{(x , s)}t & = (x, sa ) t \\
  & = (x , t^{-1} sa)\\
  & = (0, I_a) _{(x, t) t^{-1}s}\\
  &  = (0, I_a)_{ (x, s) t}
\end{align*} 
 
 We moreover see that $p(0, I_a ) = Z_a $ since we have $p (x, sa ) =
 xsa = (Z_a)_{xs}$.  
 
We define a bracket operation $[a , b]$ in $\mathscr{Y}$ by setting
$[I_a , I_b] = I_{[ a, b]}$. Then we have in the above situation  
 $$
 [ (0, I_a) , (0, I_b)] = (0, I_{[a, b]}). 
 $$
 Then it follows that 
\begin{align*}
  [Z_a , Z_b] & = [ p (0, I_a) , p (0, I_b)]\\
  & = p [ (0, I_a) , (o, I_b)]\\
  & = Z_{ [ a , b]}. 
\end{align*} 

