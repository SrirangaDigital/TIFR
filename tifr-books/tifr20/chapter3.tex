
\chapter{Connections on Principal Bundles}\label{chap3}% \chapter 3

\section{}\label{chap3:sec1} % section 3.1

A\pageoriginale connection in a principal bundle $P$ is, geometrically speaking, an
assignment to each point $\xi$ of $P$ of a tangent subspace at $\xi$
which is supplementary to the space $\mathfrak{N}_\xi$. This
distribution should be differentiable and invariant under the action
of $G$. More precisely, 

\setcounter{defn}{0}
\begin{defn}\label{chap3:sec1:def1}% definition 1
  A {\em{connection}} $\Gamma$ on the principal bundle $P$ is a
  differentiable tensor field of type $(1,1)$ such that 
  \begin{enumerate}[1)]
  \item $\Gamma (X) \subset \mathfrak{N}$ ~for every~ $X \in \mathscr{C} (P)$
  \item $\Gamma (X) = X$ ~for every~ $X \in \mathfrak{N}$
  \item $\Gamma (X) = \Gamma(X)s$ ~for every~ $s \in G$.
  \end{enumerate}
\end{defn}

Thus, at each point $\xi, \Gamma$ is a projection of $T_\xi$ onto
$\mathfrak{N}_\xi$. Condition $3)$ is equivalent to $\Gamma (d \xi)s =
\Gamma (d \xi s)$ for every $\xi \in P, d \xi \in T_\xi$ and $s \in
G$. 

\begin{examples*}% example
  \begin{enumerate}[1)]
  \item If $G$ is a discrete group, the submodule $\mathfrak{N}$ of
    $\mathscr{U}(P)$ is $(0)$. Thus any projection of $\mathscr{C}(P)$
    onto $\mathfrak{N}$ has to be $0$. Hence the only connection on a
    Galois covering manifold is the $(0)$ tensor.  
  \item Let $V$ be a differentiable manifold and $G$ a Lie
    group. Consider the trivial principal bundle $V \times G$ over
    $V$. The tensor $\Gamma$ defined by  
    $$
    \Gamma (dx, ds) = (x, ds)
    $$
    is easily seen to be a connection.
  \item Let $P$ be a Lie group and $G$ a closed subgroup. Consider the
    principal\pageoriginale bundle $P$ over $P/G$. Let $\wp$ be the tangent space at
    $e$ to $P$, and $\mathscr{Y}$ that at $e$ to $G$. Then
    $\mathscr{Y}$ can be identified with a subspace of $\wp$. For $a
    \in \mathscr{Y}$, the vector field $Z_a$ is the left invariant
    vector field $I_a$ on $P$. We shall now assume that in $\wp$,
    there exists a subspace $\mathfrak{M}$ such that  
    \begin{enumerate}[1)]
    \item $\wp = \mathscr{Y} \oplus \mathfrak{M}$
    \item for every $s \in G, s^{-1} \mathfrak{M} s \subset \mathfrak{M}$.
    \end{enumerate}
    ($\mathfrak{M}$ is only a subspace supplementary to $\mathscr{Y}$,
    which is invariant under the adjoint representation of $G$ in
    $\wp$. Such a space always exists if we assume that $G$ is compact
    or semisimple). 
  \end{enumerate}
\end{examples*}

Under the above conditions, there exists a connection $\Gamma$ and
only one on $P$ such that $\Gamma(\mathfrak{M}) = (0)$ and $\Gamma(tb)
= t (\Gamma b)$ for every $b \in \wp$ and $t \in G$. The kernel of
$\Gamma$ is the submodule of $\mathscr{C}(P)$ generated by the left
invariant vector field $I_a$ on $P$ with $a \in
\mathfrak{M}$. Moreover, $\Gamma$ is left invariant under the action
of $G$. Conversely, every left invariant connection corresponds to
such an invariant subspace $\mathfrak{M}$ supplementary to
$\mathscr{Y}$ in $\wp$. 

\setcounter{theorem}{0}
\begin{theorem}\label{chap3:sec1:thm1}% theorem 1
  If $V$ is paracompact, for every differentiable principal bundle $P$
  over $V$, there exists a connection on $P$. 
\end{theorem}

We have seen that there exists a connection on a trivial bundle. Using
the paracompactness of $V$, we can find a locally finite open cover
$(U_i)_{i \in I}$ of $V$ such that $P$ is trivial over each $U_i$ and
such that there exist tensor fields $\Gamma_i$ of type $(1,1)$ on $P$
whose restrictions to $P^{-1}(U_i)$ are connections. Set $\theta_{ij}
= \Gamma_i - \Gamma_j$ for\pageoriginale every $i,j \in I$. Then the $\theta_{ij}$
satisfy the following equations: 
$$
\theta_{ij} ~ d ~ \xi \in \mathfrak{N}_\xi, \theta_{ij} (\xi ~ a) =
0,\text{ and } \theta_{ij} (d \xi s) = (\theta_{ij} d \xi) s 
$$
for every $\xi \in p^{-1}(U_i \cap U_j), d \xi \in T_\xi, a \in
\mathscr{Y}, s \in G$. 

Now, let $(\varphi_i)_{i \in I}$ be a partition of unity with respect
to the above covering, i.e., support of $\varphi_i \subset U_i$ and
$\sum \varphi_i = 1$.  Denote $p^* \varphi_i$ by
$\tilde{\varphi_i}$. Then $\tilde{\varphi}_i$ is a partition of unity
with respect to the covering $p^{-1}(U_i). \tilde{\varphi}_k
\theta_{ik}$ is a tensor on $P$ for every $k$ and hence 
$$
\zeta_i = \sum_k \tilde{\varphi}_k \theta_{ik}
$$
is a tensor having the following properties:
$$
\zeta_i d \xi \in \mathfrak{N}_\xi, \zeta_i \xi ~ a = 0 \text{ and }
\zeta_i(d \xi s ) = (\zeta_i d \xi)s. 
$$

The tensor $\Gamma_i - \zeta_i$ is easily seen to be a connection on
$p^{-1}(U_i)$. Since $\tilde{\varphi}_k$ is a partition of unity, it
follows that $\Gamma_i - \zeta_i, \Gamma_j - \zeta_j$ coincide on
$p^{-1}(U_i \cap U_j)$ for every $i,j \in I$. Therefore the tensor
field $\Gamma $ on $P$ defined by $\Gamma = \Gamma_i - \zeta_i$ on
$p^{-1}(U_i)$ is a connection on $P$. 

\section{Horizontal vector fields}\label{chap3:sec2} % section 3.2

Let $\Gamma$ be a connection on $P$. By definition, $\Gamma$ is a map
$\mathscr{C}(P) \rightarrow \mathfrak{N}$. The kernel $\mathfrak{J}$
of this map is called the module of \textit{horizontal vector
  fields}. It is clear that $\Gamma$ maps $\wp$ into itself. Hence we
have $\wp = \mathfrak{N} \oplus ( \wp \cap \mathfrak{J})$. It is
easily seen that the vector fields belonging to the
$\mathscr{U}(V)$-module $\wp \cap \zeta$ are invariant under the
action\pageoriginale of $G$. We have defined the projection $p$ of $\wp$ onto
$\mathscr{C}(V)$. The kernel of this projection is $\mathfrak{N}$
since the restriction of $p$ to $\wp \cap \mathfrak{J}$ is bijective. 

We shall now see that the module $\mathscr{C}(F)$ of all vector fields
on $P$ is generated by projectable vector fields. 

\begin{theorem}\label{chap3:sec2:thm2}% theorem 2
  Let $V$ be a paracompact manifold and $P$ a principal bundle over
  $V$. Then the map $\eta : \mathscr{U}(P)
  \underset{\mathscr{U}(V)}{\otimes} \wp \cap \mathfrak{J} \rightarrow
  \mathfrak{J}$ defined by $\eta (\sum f_s \otimes X_s) = \sum f_s X_s
  $ is bijective. 
\end{theorem}

The proof rests on the following
\setcounter{lem}{0}
\begin{lem}\label{chap3:sec2:lem1} % lemma 1
  $\mathscr{C}(V)$ is a module of finite type over $\mathscr{U}(V)$.
\end{lem}

In fact, by Whitney's imbedding theorem (\cite{28}) there exists a
regular, proper imbedding of each connected component of $V$ in
$R^{2n+1}$ where $n = \dim V$. This gives us a map $f : V \rightarrow
R^{2n + 1}$ defined by $y \rightarrow (f,(y),\ldots, f_{2n+1}(y))$
which is of maximal rank. Let $(U_j)_{j \in J}$ be an open covering of
$V$ such that on every $U_j, (df_i)_{i = 1,2,\ldots 2n+1}$ contains a
base for the module of differential forms of degree $1$ on $U_j$. Let
$\mathfrak{S}$ be the set of maps $\alpha : [1,n] \rightarrow
[1,2n+1]$ such that $\alpha(i+1) < \alpha(i)$ for $i = 1,2,\ldots
(n-1)$. We shall denote by $U_\alpha$ the union of the open sets of
the above covering in which $df_{\alpha_1}\ldots df_{\alpha_n}$ are
linearly independent. Thus we arrive at a finite cover
$(U_\alpha)_{\alpha \in \mathscr{C}}$ having the above property. Using
a partition of unity for this cover, any form $\omega$ on $V$ can be
written as a linear combination of the $df_i$. Using the map
$f$, we can introduce a Riemannian\pageoriginale metric on $V$ which gives an
isomorphism of $\mathscr{C}(V)$ onto the module of differential forms
of degree $1$. This completes the proof of the lemma. 

\begin{proofofthetheorem} 
  Let $X_1, \ldots X_{2n+1}$ be a set of
  generators for the module $\mathscr{C}(V)$ and $(U_\alpha)_{\alpha \in
    \mathfrak{S}}$ a finite covering such that for every \break $\alpha,
  X_{\alpha(1)}, \ldots X_{\alpha(n)}$ form a base for the module of
  vector fields on $U_\alpha$. We shall now prove that the map $\eta$ is
  injective. 
\end{proofofthetheorem}

Let $(\varphi_\alpha)_{\alpha \in \mathfrak{S}}$ be a partition of
unity for this covering and $X_s$ be vector fields $\in \wp \cap
\zeta$ such that $p X_s = X_s$. Then $\varphi_\alpha X_s =
\sum\limits_{i =1}^n g^i_{\alpha,s} X_{\alpha(i)}$, with
$g^i_{\alpha,s} \in \mathscr{U}(V)$ and hence $\varphi_\alpha X_s =
\sum\limits_{i=1}^n ~g^i_{\alpha, s}X_{\alpha(i)}$ using the structure
of $\mathscr{U}(V)$-module on $\mathscr{C}(P)$. If $u =
\sum\limits_{s=1}^{2n+1} ~ f_s \otimes X_s$ with $f_s \in
\mathscr{U}(P)$ is such that $\eta (u) = 0$, then $\eta
(\varphi_\alpha ~u) = 0$ for every $\alpha$. Therefore $\sum\limits_s
~f_\alpha  ~ g_{\alpha,s}^i = 0 $ on $U_\alpha$ and consequently on
$V$. But $\varphi_\alpha ~ u = \sum\limits_{s,i} ~ f_s ~
g_{\alpha,\beta}^i \otimes X_{\alpha (i)}$; it follows that $u =\sum
\varphi_\alpha ~ u = 0$. The proof that $\eta$ is surjective is
similar. 

\begin{coro*}% corollary
  If $V$ is paracompact, the module of horizontal vector fields on $P$
  is generated over $\mathscr{U}(P)$ by the projectable and horizontal
  vector fields. 
\end{coro*}

\section{Connection form}\label{chap3:sec3} % section 3.3

Le $P$ be a principal bundle over $V$ and $\Gamma$ a connection on
$P$. If $d \xi \in T_\xi$, then $\Gamma ~ d \xi \in \mathfrak{N}_\xi $
and $a \rightarrow \xi a$ is an isomorphism of $\mathscr{Y}$ onto
$\mathfrak{N}_\xi$. We define a differential form $\gamma$ on $P$ with
values in\pageoriginale $\mathscr{Y}$ by setting $\gamma(d \xi) = a$ where $\Gamma
(d \xi) = \xi ~ a$. In order to prove that $\gamma$ is differentiable,
it is enough to prove that $\gamma$ takes differentiable vector fields
into differentiable functions with values in $\mathscr{Y}$. We have
$\gamma(Z_a) = a$ for every $a \in \mathscr{Y}$ and $\gamma(X) = 0$
for every $X \in \mathfrak{J}$. Since the module $\mathscr{C}(P)$ is
generated by $\mathfrak{J}$ and the vector fields $Z_a, \gamma(X)$ is
differentiable for every differentiable vector field $X$ on $P$. Thus
corresponding to every connection $\Gamma$ on $P$, there exists one
and only one form with values in $\mathscr{Y}$ such that $\Gamma (d
\xi) = \xi \gamma (d \xi)$ for every $d ~ \xi \in T_\xi$. It is easily
seen that $\gamma$ satisfies 
\begin{enumerate}[1)]
\item $\gamma(\xi a) = a$ for every $\xi \in P$ and $a \in \mathscr{Y}$
\item $\gamma(d \xi s ) = s^{-1}\gamma (d \xi) s $ for $d \xi \in
  T_\xi$  and $s \in G$. 
\end{enumerate}

A $\mathscr{Y}$- valued form on $P$ satisfying 1) and 2) is called
a \textit{connection form}. Given a connection form $\gamma$ on $P$,
it is easy to see that there exists one and only one connection
$\Gamma$ for which $\gamma$ is the associated form, i.e., $\Gamma(d
\xi) = \xi \gamma (d \xi) $ for every $d \xi \in T_\xi$. 

\section{Connection on Induced bundles}\label{chap3:sec4} % section 3.4

Let $P,P'$ be two principal bundles over $V,V'$ respectively. Let $h$
be a homomorphism of $P'$ into $P$. If $\gamma$ is a connection form
on $P$, the form $h^* \gamma$ on $P'$ obviously satisfies conditions
(1) and (2) and is therefore a connection form  on $P'$. 

In particular, if $P'$ is the bundle induced by a map $q : V'
\rightarrow V$, then $\gamma$ induces a connection on $P'$. 

Let $P$ be a differentiable principal bundle over $V$, and $q$ a\pageoriginale map $Y
\rightarrow V$ which trivialises $P, \rho$ being a lifting of $q$ to
$P$ with $m$ as the multiplicator. As in Chapter 2.4, we denote by
$Y_q$ the subset of $Y \times Y$ consisting of points $(y,y')$ such
that $q(y) = q(y')$. Now $\omega = \rho^* \gamma$ is a differential
form on $Y$. We have $\rho(y') = \rho(y) \, m (y,y')$ for every $(y,y')
\in Y_q$. Differentiating we obtain for every vector $(dy,dy') $ at
$(y,y') \in Y_q$ 
$$
\rho(dy') = \rho(dy) m (y,y') + \rho (y) m (dy,dy') 
$$

Since $\omega(dy') = \gamma(\rho dy')$ we have
\begin{align*}
  \gamma (\rho dy') & = m(y,y')^{-1} \omega (dy) m (y,y') + \gamma
  (\rho(y) m (y,y') m (y,y')^{-1} m(dy,dy'))\\ 
  & = m(y,y')^{-1} \omega (dy) m (y,y') + m(y,y')^{-1} m(dy,dy')
\end{align*}

Hence $m(dy,dy') = m(y,y') \omega (dy') - \omega(dy) m (y,y')$.

Conversely if $\omega$ is a differential form on $Y$ satisfying 
$$
m(dy,dy') = m(y,y') \omega (dy') - \omega (dy) m (y,y')
$$
for every $(y,y') \in Y_q$, then there exists one and only one
connection form $\gamma$ on $P$ such that $\omega = \rho^* \gamma$. 

In particular, when the trivialisation of $P$ is in terms of a
covering $(U_i)_{i \in I}$ of $V$ with differentiable sections
$\sigma_i (Ch.2.5)$, the connection form $\gamma$ on $P$ gives rise to
a family of differential forms $\omega_i = \sigma_i^* \gamma$ on
$U_i$. From the equations $\sigma_i(x) = \sigma_j (x) m_{ji}(x)$
defining\pageoriginale the transition functions $m_{ji}$, we obtain on
differentiation, 
$$
\displaylines{\hfill  
  \omega_i(dx) = m_{ji}(x)^{-1} \omega_j(dx) m_{ji}(x) + m_{ji} 
  (x)^{-1} m_{ji}(dx) \text{ for } x \in U_i,\hfill\cr  
  \text{ i.e., }\hfill m_{ji}(dx) = m_{ji}(x) \omega_i (dx) - \omega_j 
  (dx) m_{ji}(x). \hfill }
$$

Conversely given a family of differentiable forms $\omega_i$ on the
open sets of a covering $(U_i)_{i \in I}$ of $V$ satisfying the above,
there exists one and only one connection form $\gamma$ on $P$ such
that $\omega_i = \sigma_i^* \gamma$ for every $i$. 

\section{Maurer-Cartan equations}\label{chap3:sec5} % section 3.5

To every differentiable map $f$ of a differentiable manifold $V$ into
a Lie group $G$, we can make correspond a differential form of degree
$1$ on $V$ with values in the Lie algebra $\mathscr{Y}$ of $G$ defined
by $\alpha (\xi) (d \xi) = f(\xi)^{-1} f(d \xi)$ for $\xi \in V$ and
$d \xi \in T_\xi$. We shall denote this form by $f^{-1} df$. This is
easily seen to be differentiable. 

If we take $f : G \rightarrow G$ to be the identity map, then we
obtain a canonical differential form $\omega$ on $G$ with values in
$\mathscr{Y}$. Thus we have $\omega (ds) = s^{-1} ds \in
\mathscr{Y}$. Also $\omega(tds) = \omega(ds)$ for every vector $ds $
of $G$. That is, $\omega$ is a left invariant differential
form. Moreover, it is easy to see that any scalar left invariant
differential form on $G$ is obtained by composing $\omega$ with
elements of the algebraic dual of $\mathscr{Y}$. In what follows, we
shall provide $A \otimes \mathscr{Y}$ with the canonical derivation
law (Ch.\ref{chap1:sec2}).

\setcounter{proposition}{0}
\begin{proposition}[Maurer-Cartan]\label{chap3:sec5:prop1}% proposition 1
  The canonical form $\omega$ on $G$ satisfies $d \omega +
  [\omega,\omega] = 0$.\pageoriginale 
\end{proposition}

We recall that the form $[\omega,\omega]$ has been defined by
$[\omega,\omega] (X,Y) = [\omega(X),\omega(Y)]$ for every vector
fields $X,Y \in \mathscr{C}(G)$. Since the module of vector fields
over $G$ is generated by left invariant vector fields, it suffices to
prove the formula for left invariant vector fields $X = I_a, Y = I_b ~
a,b \in \mathscr{Y}$. We have 
$$
d(\omega) (I_a,I_b) = I_a \omega(I_b) - I_b \omega(I_a) -
\omega([I_a,I_b]) 
$$

But $[I_a,I_b] = I_{[a,b]}$ and $\omega[I_a,I_b] = \omega(I_{[a,b]}) =
[a,b] = [\omega(I_a),\omega(I_b)]$ since $\omega(I_a) = a, \omega(I_b)
= b$. 

\begin{coro*}% corollary
  If $f$ is a differentiable map $V \rightarrow G$, then the form
  $\alpha = f^{-1} df$ satisfies $d \alpha + [\alpha,\alpha] = 0$. 
\end{coro*}

In fact, $\alpha (d \xi) = f^{-1}(\xi) f (d\xi) = \omega(f(d\xi)) =
(f^*\omega) (d\xi)$ and hence $\alpha = f^* \omega$. The above
property of $\alpha$ is then an immediate consequence of that of
$\omega$. 

Conversely, we have the following
\begin{theorem}\label{chap3:sec5:thm3}% theorem 3
  If $\alpha$ is a differential form of degree $1$ on a manifold $V$
  with values in the Lie algebra $\mathscr{Y}$ of a Lie group $G$
  satisfying $d \alpha + [\alpha, \alpha] = 0$, then for every $\xi_o
  \in V$
  and a differentiable map $f : U \rightarrow G$ such that $f^{-1}
  (\xi)f(d\xi) = \alpha(d\xi)$ for every vector $d \xi$ of $U$. 
\end{theorem}

Consider the form $\beta = p_1^* \alpha - p^*_2 \omega$ on $V \times
G$ where $p_1 : V \times G \rightarrow V$\pageoriginale and $p_2 : V \times G
\rightarrow G$ arc the two projections and $\omega$ the canonical left
invariant form on $G$. If $\beta$ is expressed in terms of a basis $\{
a_1, \ldots a_r\}$ of $\mathscr{Y}$, the component scalar differential
forms $\beta_i$ are everywhere linearly independent. This follows from
the fact that on each $p^{-1}_1(\xi), \beta$ reduces to $p^*_2
\omega$. We shall now define a differentiable and involutive
distribution on the manifold $V \times G$. Consider the module
$\mathfrak{M}$ of vector fields $X$ on $V \times G$ such that $\beta_i
(X) = 0$  for every $\beta_i$. We have now to show that $X,Y \in
\mathfrak{M} \Rightarrow [X,Y] \in \mathfrak{M}$. But this is an
immediate consequence of the relations $ d ~ \alpha = - [ \alpha,
  \alpha]$ and $d \omega = [ \omega, \omega]$. 

By Frobenius' theorem (see \cite{11}) there exists an integral
submanifold $W$ (of $\dim = \dim ~ V$) of $V \times G$ in a
neighbourhood of $(\xi_o,e)$. Since for every vector $(\xi_o,a) \neq
0$ tangent to $p_1^{-1} \xi_o, \beta (\xi_o,a) = a \neq 0$ there exist a
neighbourhood $U$ of $\xi_o$ and a differentiable section $\sigma$ into
$V \times G$ over $U$ such that $\sigma(U) \subset W$. Define
$f(\xi) = p_2 \sigma(\xi)$. By definition of $W$, it follows that
$\beta \sigma (d \xi) = 0$. This means that  
$$
\displaylines{\hfill 
  \alpha p_1 \sigma (d \xi) - \omega p_2 \sigma (d \xi) = 0\hfill \cr
  \text{ i.e., }\hfill  \alpha (d \xi) = \omega(p_2 \sigma(d \xi)) = (f^*
  \omega) d \xi = f^{-1} (\xi) f (d\xi).\hfill } 
$$

\begin{remark*}% remark
  When we take $G =$ additive group of real numbers, the above theorem
  reduces to the Poincare's theorem for $1$-forms. Concerning the
  uniqueness of such maps $f$, we have the  
\end{remark*}

  \begin{proposition}\label{chap3:sec5:prop2}% proposition 2
    If $f_1, f_2$ are two differentiable maps of a connected manifold
    $V$ into a Lie group $G$ such that $f^{-1}_1 df_1 = f^{-1}_2
    df_2$, then there\pageoriginale exists an element $s \in G $ such
    that $f_1 = f_2 s$. 
  \end{proposition}

Define a differentiable function $s : V \rightarrow G$ by setting
$$
s(\xi) = f_1 (\xi) f^{-1}_2 (\xi) \text{ for every } \xi \in V.
$$

Differentiating this, we get 
$$
s(d \xi) f_2 (\xi) + s (\xi) f_2 (d \xi) = f_1 (d \xi).
$$

Hence $s(d \xi) = 0$. Therefore $s$ is locally a constant and since
$V$ is connected $s$ is everywhere a constant. 

Regarding the existence of a map $f$ in the large, we have the following

\begin{theorem}\label{chap3:sec5:thm4}
  If $\alpha$ is a differential form of degree $1$ on a simply
  connected manifold $V$ with values in the Lie algebra $\mathscr{Y}$
  of a Lie group $G$, such that $d \alpha + [\alpha, \alpha] = 0$,
  then there exists a differentiable map $f$ of $V$ into $G$ such that
  $f^{-1} df = \alpha$. 
\end{theorem}

The proof rests on the following
\begin{lem}\label{chap3:sec5:lem2}% lemma 2
  Let $m_{ij}$ be a set of transition functions with group $G$ of a
  simply connected manifold $V$ with respect to a covering $(U_i)$. If
  each $m_{ij}$ is locally a constant, then there exists locally
  constant maps $\lambda_i : U_i \rightarrow G$ such that $m_{ij}(x) =
  \lambda_i (x)^{-1} \lambda_j (x)$ for every $x \in U_i \cap U_j$. 
\end{lem}

In fact, there exists a principal bundle $P$ over $V$ with group $G$
considered as a discrete group and with transition functions $m_{ij}$
(Ch.\ref{chap2:sec5}). $P$ is then a covering manifold over $V$ and hence
trivial.\pageoriginale Therefore the maps $\lambda_i : U_i \rightarrow G$ exist
satisfying the conditions of the lemma (Ch.\ref{chap2:sec5}). 

\begin{proofofthetheorem} 
  Let $(U_i)_{i\in I}$ be a covering of
  $V$ such that on each $U_i$, there exists a differentiable function
  $f_i$ satisfying $f_i(x)^{-1} f_i(dx) = \alpha(dx)$ for every $x \in U_i$. We
  set $m_{ij}(x) = f_i (x) f_j (x)^{-1}$. It is obvious that the
  $m_{ij}$ form a set of locally constant transition functions. Let
  $\lambda_i$ be the maps $U_i \rightarrow G$ of the lemma. Then
  $\lambda_i(x) f_i (x) = \lambda_j(x) f_j(x)$ for every $x \in U_i \cap
  U_j$. The map $f : V \rightarrow G$ which coincides with $\lambda_i
  f_i$ on each $U_i$ is such that $f^{-1}df = \alpha$. 
\end{proofofthetheorem}

\section{Curvature forms}\label{chap3:sec6}% section 3.6

\begin{defn}\label{chap3:sec6:def2}% definition 2
  Let $\gamma$ be a connection form on a principal bundle $P$ over
  $V$. Then the alternate form of degree $2$ with values in the Lie
  algebra $\mathscr{Y}$ of the structure group $G$ defined by $K = d
  \gamma + [\gamma, \gamma]$ is said to be the {\em{curvature form}}
  of $\gamma$. A connection on $P$ is said to be {\em{flat}} if the
  form $K \equiv 0$. 
\end{defn}

\begin{theorem}\label{chap3:sec6:thm5}% theorem 5
  The following statements are equivalent:
  \begin{enumerate}[a)]
  \item The connection $\gamma$ is flat i.e., $d \gamma + [\gamma,
    \gamma] = 0$. 
  \item If $X,Y$ are two horizontal vector fields on $p$, so is [$X,Y$].
  \item For every $x_o \in V$, there exists an open neighbourhood $U$
    of $x_o$ and a differentiable section $\sigma$ on $U$ such that
    $\sigma^* \gamma = 0$. 
  \end{enumerate}
\end{theorem}

\begin{proof}% proof
  \begin{enumerate}[a)]
  \item $\Rightarrow b)$ is obvious from the definition.
  \item $\Rightarrow c)$
  \end{enumerate}
\end{proof}

Let\pageoriginale $a_1,a_2,\ldots a_n$ be a basis of $\mathscr{Y}$ and let $\gamma =
\sum \gamma_i a_i$ where $\gamma_i$ are scalar differential
forms. Since $\gamma_i (Za_j) = \delta_{ij}$, it follows that $\{
\gamma_1,\ldots \gamma_r\}$ form a set of differential forms of
maximal rank $r$. From $b)$, it follows that the module $\mathfrak{J}$
of horizontal vector fields is stable under the bracket
operation. Hence $\mathfrak{J}$ forms a  distribution on $P$ which is
differentiable and involutive. Let $\xi_o \in P$ such that $p(\xi_o) =
X_o$ and let $W$ be an integral manifold $W$ of $\dim = \dim V$ in a
neighbourhood of $\xi_o$. As in Th.\ref{chap3:sec5:thm3}, Ch.3.5, $W$
is locally the image 
of a differentiable section $\sigma$ over an open neighbourhood of
$x_o$. Since all the tangent vectors of $W$ are horizontal, we have
$\gamma \sigma(dx) = 0$ for every vector $dx$ on $U$. 

\noindent c)~ $\Rightarrow a)$ 

Let $\sigma $ be a differentiable section over an open subset $U$ of
$V$ such that $\sigma^* \gamma = 0$. For every $\xi \in p^{-1}(U)$,
define a differentiable function $\rho : p^{-1} (U) \rightarrow G$ by
the condition $\xi = \sigma (p(\xi)) \rho(\xi)$. Differentiating this,
we obtain 
$$
d \xi = \sigma p (d \xi) \rho(\xi) + \sigma p(\xi) \rho(d \xi)
$$

Hence $\gamma (d \xi) = \rho^{-1} (\xi) \gamma (\sigma p (d \xi)) \rho
(\xi) + \rho(\xi)^{-1} \rho (d \xi)$ by conditions $(1)$ and $(2)$ for
connection forms. i.e. $\gamma(d \xi) = \rho (\xi)^{-1} \rho (d \xi)$
and $a)$ follows from cor. to prop.\ref{chap3:sec5:prop1}, Ch.3.5. 

\begin{theorem}\label{chap3:sec6:thm6}%theorem 6
  If there exists a flat connection on a principal bundle $P$ over\pageoriginale a
  simply connected manifold $V$, then $P$ is a trivial
  bundle. Moreover, a differentiable cross-section $\sigma$ can be
  found over $V$ such that $\sigma^* \gamma = 0$. 
\end{theorem}

By $Th.4, Ch.3.5$, there exists an open covering $(U_i)_{i \in I}$ and
cross sections $\sigma_i : U_i \rightarrow P$ such that $\sigma^*_i
\gamma = 0$. Let $\sigma_i (x) m_{ij}(x) = \sigma_j(x)$ where the
corresponding transition functions are $m_{ij}$. Then we have 
$$
\gamma (\sigma_i(dx) m_{ij} (x) + \sigma_i(x) m_{ij}(dx)) = \gamma
\sigma_j (dx). 
$$

Therefore, $m_{ij}(dx) = 0$, i.e. the $m_{ij}$ are locally
constant. Lemma \ref{chap3:sec5:lem2}, Ch.3.5 then gives a family $\lambda_i$ of
locally constant maps $U_i \rightarrow G$ such that 
$$
m_{ij} = \lambda^{-1}_i \lambda_j
$$

It is easily seen that $\sigma_i \lambda^{-1}_i = \sigma_j
\lambda^{-1}_j$ on $U_i \cap U_j$. Define a cross section $\sigma$ on
$V$ by setting $\sigma = \sigma_i \lambda^{-1}_i$ on every $U_i$. Then
we have 
\begin{align*}
  \gamma (\sigma(dx)) & = \gamma(\sigma_i(dx)\lambda^{-1}_i (x))\\
  & = \lambda_i (x) (\gamma \sigma_i(dx)) \lambda^{-1}_i (x)\\
  & = 0 \text{ for } x \in U_i.
\end{align*}

Hence the bundle is trivial and $\sigma^* \gamma = 0$.

\begin{proposition}\label{chap3:sec6:prop3}% proposition 3
  If $X$ is a vector field on $P$ tangential to the fibre, then\pageoriginale
  $K(X,Y) = 0$ for every vector field $Y$ on $P$. 
\end{proposition}

In fact, since $\wp = \wp \cap \mathfrak{J}\oplus \mathfrak{N}$ and
$\wp$ generates $\mathscr{C}(P)$, it is enough to prove the assertion
for $Y \in \mathfrak{N}$ and $Y \in \wp \cap \zeta$. In the first case
$X = Z_a $ and $Y = Z_b$, we have 
\begin{align*}
  K(Z_a,Z_b) & = (d \gamma + [\gamma,\gamma]) (Z_a,Z_b)\\
  & = Z_a \gamma (Z_b) - Z_b \gamma (Z_a) - \gamma [Z_a,Z_b] +
  [\gamma(Z_a),\gamma(Z_b)]\\ 
  & = - \gamma (Z_{[a,b]}) + [a,b]\\
  & = 0
\end{align*}

In the second case, $Y$ is invariant under $G$ and it is easy to see
that $[Z_a,Y] = 0$. Since $\gamma(Y)= 0$, we have 
\begin{align*}
  K(Z_a,Y) & = (d \gamma + [\gamma,\gamma]) (Z_a,Y)\\
  & = Z_a \gamma(Y) - Y \gamma(Z_a) - \gamma[Z_a,Y] + [ \gamma(Z_a),
    \gamma(Y)]\\ 
  & = 0.
\end{align*}

\begin{proposition}\label{chap3:sec6:prop4}% proposition 4
  $$
  K (d_1 \xi s, d_2 \xi s) = s^{-1} K(d_1 \xi,d_2 \xi)s ~\text{ for }~
  \xi \in P, d_1 \xi, d_\alpha \xi  \in T_\xi ~\text{ and }~ s \in G. 
  $$
\end{proposition}

By Prop.\ref{chap3:sec6:prop3}, Ch.3.6, it is enough to consider the case when $d_1 \xi,
d_2 \xi \in \mathfrak{J}_\xi$. Extend $d_1 \xi, d_2 \xi$ to horizontal
vector fields $X_1, X_2$ respectively. Then we have 
\begin{align*}
  K(d_1 \xi s, d_2 \xi s) & = K(X_1 s, X_2 s) (\xi)\\
  & = - \gamma [X_1 s, X_2 s] (\xi)\\
  & = - \gamma ([X_1, X_2] s) (\xi)\\
  & = - \gamma ([X_1, X_2]_\xi s)\\
  & = - s^{-1} \gamma ([X_1,X_2]_\xi)s\\
  & = s^{-1} K(X_1,X_2) (\xi) s\\
  & = s^{-1} (d_1 \xi, d_2 \xi)s \text{ for every } s \in G.
\end{align*}\pageoriginale

\section{Examples}\label{chap3:sec7}% section 3.7

\begin{enumerate}
\item Let $\dim V = 1$. Then $\mathscr{C}(V)$ is a face module
  generated by a single vector field. But $\mathscr{C}(V)$ is
  $\mathscr{U}(V)$-isomorphic to $ \mathfrak{J} \cap \wp$ and hence
  $\mathfrak{J} =\mathscr{U} (P)\bigotimes \limits_{\mathscr{U}(V)}
  \mathfrak{J} \cap \wp$ is also $\mathscr{U}(P)$ - free, and of rank
  $1$. Since $K$ is alternate, $K \equiv 0$. In other words, if the
  base manifold of $P$ is a curve, any connection is flat. 
\item Let $G$ be a closed subgroup of a Lie group $H$ and  let $
  \mathscr{Y}, \mathscr{F}$ be their respective LIe algebras. We have
  seen that (Example 3, Ch.\ref{chap3:sec1}) if $\mathfrak{M}$ is a subspace of
  $\mathscr{F}$ such that $\mathscr{F} = \mathfrak{M} \oplus
  \mathscr{Y}$ and $s^{-1} \mathfrak{M} s \subset \mathfrak{M}$ for
  every $s \in G$, then the projection $\Gamma : \mathscr{F}
  \rightarrow \mathfrak{M}$ gives rise to a connection which is left
  invariant by element of $G$. Denoting by $I_a$ the left invariant
  vector fields on $H$ whose values at $e$ is $a \in \mathscr{F}$, we
  obtain 
  \begin{align*}
    K(I_a,I_b) & = - \gamma([I_a,I_b]) + [\gamma(I_a), \gamma(I_b)]\\
    & = - \gamma I_{[a,b]} + [\Gamma_e a, \Gamma_e b]\\
    & = - \Gamma_e ([a,b]) + [\Gamma_e (a), \Gamma_e(b)].
  \end{align*}
\end{enumerate}

Thus\pageoriginale $K$ is also left invariant for elements of $G$. Moreover $\Gamma$
is flat if and only if $\Gamma_e$ is a homomorphism of $\mathscr{J}$
onto $\mathscr{U}$. This again is true if and only if $\mathfrak{M}$
is an ideal in $\mathscr{J}$. 

