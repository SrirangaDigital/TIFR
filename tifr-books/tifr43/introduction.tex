\chapter*{Introduction}\pageoriginale

\addcontentsline{toc}{chapter}{Introduction}

\markboth{Introduction}{Introduction}

The recent efflorescence in the theory of polyhedral manifolds due to Smale's handle-theory, the differential obstruction theory of Munkres and Hirsch, the engulfing theorems, and the work of Zeeman, Bing and their students - all this has led to a wide gap between the modern theory and the old foundations typified by Reidemeister's {\rm Topologie der Polyeder} and Whitehead's ``Simplicial spaces, nuclei, and $m$-groups''. This gap has been filled somewhat by various sets of notes, notably Zeeman's at I.H.E.S.; another interesting exposition is Glaser's at Rice University. 

Well, here is my contribution to bridging the gap. These notes contain:

\begin{enumerate}
\renewcommand{\labelenumi}{(\theenumi)}
\item The elementary theory of finite polyhedra in real vector spaces. The intention, not always executed, was to emphasize geometry, avoiding combinatorial theory where possible. Combinatorially, convex cells and bisections are preferred to simplexes and stellar or derived subdivisions. Still, some simplicial technique must be slogged through.

\item A theory of ``general position''  (i.e., approximation of maps by ones whose singularities have specifically bounded dimensions), based on ``non-degeneracy''. The concept of $n$-manifold is generalized in the most natural way for general-position theory by that of $ND(n)$-space - polyhedron $M$ such that every map from an $n$-dimensional polyhedron into $M$ can be approximated by a non-degenerate\pageoriginale map (one whose point-inverse are all finite). 

\item A theory of ``regular neighbourhoods'' in arbitrary polyhedra. Our regular neighbourhoods are all isotopic and equivalent to the star in a second-derived neighbourhoods are all isotopic and equivalent to the star in a second-derived subdivision (this is more or less the definition). Many applications are derived right after the elementary lemma that ``locally collared implies collared''. We then characterize regular neighbourhoods in terms of Whitehead's ``collapsing'', suitably modified for this presentation. The advantage of talking about regular neighbourhoods in arbitrary polyhedra becomes clear when we see exactly how they should behave at the boundaries of manifolds.

After a little about isotopy (especially the ``cellular moves'' of Zeeman), our description of the fundamental techniques in polyhedral topology is over. Perhaps the most basic topic omitted is the theory of block-bundles, microbundles and transversality.

\item Finally, we apply our methods to the theory of handle - presentations of $PL$-manifolds \`a la Smal\'es theory for differential manifolds. This we describe sketchily; it is quite analogous to the differential case. There is one innovation. In order to get two handles which homotopically cancel to geometrically cancel, the ``classical'' way is to interpret the hypothesis in terms of the intersection number of attaching and transverse spheres, to reinterpret this geometrically, and then to embed a two-cell over which a sort of Whitney move can be made to eliminate a pair of intersection. Our method, although rather ad-hoc, is more direct, avoiding the algebraic complication of intersection numbers (especially\pageoriginale unpleasant in the non-simply-connected case) as well as any worry that the two-cell might cause; of course, it amounts to the same thing really. This method is inspired by the engulfing theorem. [There are, by the way, at least two ways to use the engulfing theorem itself to prove this point].
\end{enumerate}

We do not describe many applications of handle-theory; we do obtain Zeeman's codimension 3 unknotting theorem as a consequence. This way of proving it is, unfortunately, more mundane than ``sunny collapsing''.

We omit entirely the engulfing theorems and their diverse applications. We have also left out all direct contact with differential topology.

\bigskip
\centerline{*\hspace{2cm} *\hspace{2cm} *}
\bigskip

Let me add a public word of thanks to the Tata Institute of Fundamental Research for giving me the opportunity to work on these lectures for three months that were luxuriously free of the worried, anxious students and administrative annoyances that are so enervating elsewhere. And many thanks to Shri Ananda Swarup for the essential task of helping write these notes.

\vskip 1cm

\begin{flushright}
{\large\bf John R. Stallings}

Bombay

March,\quad 1967
\end{flushright}
