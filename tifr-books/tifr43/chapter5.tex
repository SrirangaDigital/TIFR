

\chapter{General Position}\label{chap5}

We\pageoriginale intend to study $PL$-manifolds is some detail. There are certain basic techniques which have been developed for this purpose, one of which is called ``general position''. An example is the assertion that ``if $K$ is a complex of dimension $k$, $M$ a $PL$-manifold of dimension $>2 k$, and $f:K\to M$ is any map, then $f$ can be approximated by imbeddings''. More generally we start with some notions ``a map $f:K\to M$ being generic'' and ``a map $f:K\to M$ being in ``generic position'' with respect to some $Y\subset M$''. This ``generic'' will be usually with reference to some minimum possible dimensionality of ``intersections'', ``self intersections'' and ``nicety of intersections''. The problem of general position is to define useful generic things, and then try to approximate nongeneric maps by generic ones for as large a class of $X$'s, $Y$'s and $M$'s as possible (even in the case of $PL$-manifolds, one finds it necessary to prove general position theorems for arbitrary $K$).

It seems that the first step in approximating a map by such nice maps is to approximate by $a$ so called nondegenerate map, that is a map $f:K\to M$ which preserves dimensions of subpolyhedra.

Now it happens that a good deal of `general position' can be obtained from just this nondegeneracy, that is if $Y$ is the sort of polyhedron in which maps from polyhedra of dimension $\leq$ some $n$ can be approximate by nondegenerate maps, then they can be approxiamted by\pageoriginale nicer maps also. And the class of these $Y$'s is much larger than that of $PL$-manifolds.

We call such spaces Non Degenerate $(n)$-spaces or $ND(n)$-spaces. The aim of this chapter is to obtain a good description of such spaces and prove a few general position theorems for these spaces.

\section{Nondegeneracy}\label{chap5-sec5.1}

\begin{proposition}\label{chap5-prop5.1.1}
The following conditions on a polyhedral map $f:P\to Q$ are equivalent:
\begin{itemize}
\item[\rm(a)] For every subpolyhedron $X$ of $P$,

$\dim f(X)=\dim X$.

\item[\rm(b)] For every subpolyhedron $Y$ of $Q$,

$\dim f^{-1}(Y)\leq \dim Y$.

\item[\rm(c)] For every point $x\in Q$, $f^{-1}(x)$ is finite.

\item[\rm(d)] For every line segment $[x,y]\subset P$, $x\neq y$, $f([x,y])$ contains more than one point.

\item[\rm(e)] For every $\mathscr{P}$, $\mathcal{Q}$ with respect to which $f$ is simplicial, $f(\sigma)$ has the same dimension as $\sigma$, $\sigma\in \mathscr{P}$. 

\item[\rm(f)] There exists a presentation $\mathscr{P}$ of $p$, on each cell of which $f$ is linear, and one-to-one.
\end{itemize}
\end{proposition}

\begin{proof}
Clearly
\begin{align*}
& {\rm(a)}~\Longrightarrow {\rm(d)}\\
& {\rm(b)}~\Longrightarrow {\rm(c)}~\Longrightarrow {\rm(d)}\\
& {\rm(e)}~\Longrightarrow {\rm(f)}
\end{align*}

To\pageoriginale see that (a) $\Longrightarrow$ (b):

Consider a subpolyhedron $Y$ of $Q$; then $f(f^{-1}(Y))\subset Y$.\break $\Dim (f^{-1}(Y))=\dim f(f^{-1}(Y))$ by (a) and as $f(f^{-1}(Y))\subset Y$, $\dim f(f^{-1}*Y))\leq \dim Y$. Hence $\dim (f^{-1}(Y)\leq \dim Y$.

To see the (d) $\Rightarrow$ (e):

Let $\sigma\in\mathscr{P}$. If $f(\sigma)$ has not the same dimension as that of $\sigma$, two different vertices of $\sigma$ say $v_{1}$ and $v_{2}$ are mapped onto the same vertex of $f(\sigma)$ say $v$. Then $[v_{1},v_{2}]$ is mapped onto a single point $v$, contradicting (d).

Finally (f) $\Rightarrow$ (a):

To See this, first observe that if $f$ is linear and one-to-one- on a cell $C$, then it is linear one one-to-one on $\overline{C}$ also. Thus if $A$ is a polyhedron in $\overline{C}$, $\dim f(A)=\dim A$. But, $X=\bigcup\limits_{C\in\mathscr{P}}(X\cap \overline{C})$, and $\dim X=\Max\limits_{C\in\mathscr{P}}(\dim X\cap \overline{C})$. It follows that $\dim f(X)=\dim X$.

Thus we have
\[
\xymatrix@R=.5cm{
 & b\ar@{=>}[dr] & \\
a\ar@{=>}[ur] & & c\ar@{=>}[d]\\
f\ar@{=>}[u] & & d\ar@{=>}[dl]\\
& e\ar@{=>}[ul] &
}
\]
and therefore all the conditions are equivalent.
\end{proof}

\begin{definition}\label{chap5-defi5.1.2}
We shall call a polyhedral map $f$ which satisfies any of the six equivalent conditions of proposition \ref{chap5-prop5.1.1}. {\em a nondegenerate} map. 
\end{definition}

Note\pageoriginale that a nondegenerate map may have various ``foldings''; in other words it need not be a local embedding.

\begin{ex}\label{chap5-ex5.1.3}
\begin{enumerate}
\renewcommand{\labelenumi}{(\theenumi)}
\item If $f:P\to Q$ is a polyhedral map, and $P=P_{1}\cup \ldots \cup P_{k}$, $P_{i}$ is a subpolyhedron of $P$, $1\leq i\leq k$, and if $f/P_{i}$ is nondegenerate, then $f$ is nondegenerate.

\item If $f:P\to Q$ is nondegenerate, and $X\subset P$ a subpolyhedron, then $f/X$ is also nondegenerate.

[Hint~: Use 1.C].
\end{enumerate}
\end{ex}

\begin{exprop}\label{chap5-exprop5.1.4}
The composition of two nondegenerate maps is a nondegenerate map.
\end{exprop}

\begin{exprop}\label{chap5-exprop5.1.5}
If $f:P_{1}\to Q_{1}$, and $g:P_{2}\to Q_{2}$ are nondegenerate, then $f\ast g:P_{1}\ast P_{2}\to Q_{1}\ast Q_{2}$ is nondegenerate.


In Particular conical extensions of nondegenerate maps are again nondegenerate.

[Hint: Consider presentations with respect to which $f$, $g$ are simplicial and use 1 $f$.].
\end{exprop}

Let $f:P\to Q$ be a polyhedral map, $\mathscr{S}$ and $\mathscr{Z}$ triangulations of $P$ and $Q$ with reference to which $f$ is simplicial. $\mathscr{S}_{k}$ and $\mathscr{Z}_{k}$ as usual denote the $k^{\text{th}}$ skeletons of $\mathscr{S}$ and $\mathscr{Z}$. Let $\theta$, $\eta$ be centerings of $\mathscr{S}$, $\mathscr{Z}$ respectively such that $f(\theta\sigma)=\eta(f\sigma)$ for $\sigma\in \mathscr{S}$. Let $\mathscr{S}^{k}$ and $\mathscr{Z}^{k}$ denote the dual skeletons with respect to these centerings. Then

\begin{ex}\label{chap5-ex5.1.6}
\begin{enumerate}
\renewcommand{\theenumi}{\alph{enumi}}
\renewcommand{\labelenumi}{(\theenumi)}
\item $f(\mathscr{S}_{k})\subset \mathscr{Z}_{k}$.

$f$ is nondegenerate if and only if $f^{-1}(\mathscr{Z}_{k})\subset \mathscr{S}_{k}$. 

\item $f^{-1}(\mathscr{Z}^{k})\subset \mathscr{S}^{k}$\pageoriginale

$f$ is nondegenerate if and only if

$f(\mathscr{S}^{k})\subset \mathscr{Z}^{k}$.

\item Formulate and prove the analogues of (a)

and (b) for regular presentations.
\end{enumerate}
\end{ex}


\section{$ND(n)$-spaces. Definition and Elementary properties}\label{chap5-prop5.2} 

\begin{definition}\label{chap5-defi5.2.1}
A polyhedron $M$ is said to be an {\em $ND(n)$-space (read Non-Degenerate $(n)$-sace)} if and only if:

for every polyhedron $X$ of dimension $\leq n$, and any map $f:X\to M$ and any $\epsilon>0$, there is an $\epsilon$-approximation to $f$ which is nondegenerate. 
\end{definition}

This property is a polyhedral invariant:

\begin{proposition}\label{chap5-prop5.2.2}
If $M$ is and $ND(n)$-space, and $\mathcal{L}:M\to M'$ a polyhedral equivalence, then $M'$ is also $ND(n)$.
\end{proposition}

\begin{proof}
Obvious.
\end{proof}

Before we proceed further, it would be nice to know such spaces exist. Here is an example:

\begin{proposition}\label{chap5-prop5.2.3}
An $n$-cell is an $ND(n)$-space.
\end{proposition}

\begin{proof}
By \ref{chap5-prop5.2.2}, it is enough to prove for $\overline{A}$, where $A$ is an open convex $n$-cell in $\mathbb{R}^{n}$. Let $f:X\to \overline{A}$ be any map from a polyhedron $X$ of dimension $\leq n$. First choose a triangulation $\mathscr{S}$ of $X$, such that $f$ is linear on each simplex of $\mathscr{S}$. Let $v_{1},\ldots,v_{r}$ be the vertices of $\mathscr{S}$. First we alter the map $f$ a little to a $f'$ so that $f'(v_{1}),\ldots,f'(v_{r})$ are all in $A=$ Interior of $\overline{A}$. This is\pageoriginale clearly possible: We just have to choose points near $f(v_{i})$'s in the interior and extend linearly. Next, by \ref{chap1-ex1.2.12} of Chapter \ref{chap1}, we can choose $y_{1},\ldots,y_{r}$ so that $y_{i}$ is near $f'(v_{i})$ and $y_{i}$'s are in general position, that is any $(n+1)$ or less number of points of $y$'s is independent. If we choose $y$'s near enough $f'(v_{i})$'s, the $y$'s will be still in $\overline{A}$, that is why we shifted $f(v_{i})$'s into the interior. Now we define $g(v_{i})=y_{i}$ and extend linearly on simplexes of $\mathscr{S}$ to a get a map $X\to M$, which is non-degenerate by \ref{chap5-prop5.1.1} (f). And surely if $f(v_{i})$ and $y_{i}$ are near enough, $g$ will be good approximation to $f$.
\end{proof}

The next proposition says, roughly, that an $ND(n)$-space is locally $ND(n)$.

\begin{proposition}\label{chap5-prop5.2.4}
If $\mathscr{S}$ is any simplicial presentation of an $ND(n)$-space, and $\sigma\in \mathscr{S}$, then $|St(\sigma,\mathscr{S})|$ is an $ND(n)$-space. 
\end{proposition}

\begin{proof}
If $x$ is a point of $\sigma$, then $|St(\sigma,\mathscr{S})|$ is a cone with vertex $x$ and base $\p \sigma\ast|Lk(\sigma,\mathscr{S})|$ which is a link of $x$ in $M$; and $|St(\sigma,\mathscr{S})|-\p \sigma\ast|Lk(\sigma,\mathscr{S})|$ is open in $M$. If $f:X\to |St(\sigma,\mathscr{S})|$ is any map from a polyhedron $X$ of dimension $\leq n$ and $\epsilon >0$, we first shink it towards $x$ by a map $f'$ say so that $f'(X)\subset |St(\sigma,\mathscr{S})|-\p \sigma \ast|Lk(\sigma,\mathscr{S})|$ so that $\rho(f,f')<\epsilon/2$. Now $N=M-(|St(\sigma,\mathscr{S})\text{min}-\p \sigma\ast|Lk(\sigma,\mathscr{S})|)$ is a subpolyhedron of $M$, and $f'(X)\cap N=\emptyset$. Therefore $\rho(f'(X),N)>\delta>0$. Let $\eta=\min(\delta,\epsilon/2)$. Since $M$ is $ND(n)$, we can obtain an $\eta$-approximation to $f'$, say $g$ which is nondegenerate. $g$ is an $\epsilon$-approximation to $f$ and $g(X)\cap N=\emptyset$, $g(X)\subset M$.\pageoriginale Therefore $g(X)\subset \mid St(\sigma,\mathscr{S})|$. Hence $|St(\sigma,\mathscr{S})|$ is an $ND(n)$-space.
\end{proof}

Next we establish a sort of ``general position'' theorem for $ND(n)$-space.

\begin{theorem}\label{chap5-thm5.2.5}
Let $M$ be an $ND(n)$-space, $K$ a subpolyhedron of $M$ of dimension $\leq k$. Let $f:X\to M$ be a map from a polyhedron $X$ of dimension $\leq n-k-1$. Then $f$ can be approximated by a map $g:X\to M$ such that $g(X)\cap K=\emptyset$.
\end{theorem}

\begin{proof}
Let $D$ be a $(k+1)$-cell. Let $f':D\times X\to M$ be the composition of the projection $D\times X\to X$ and $f$, that is $f'(a,x)=f(x)$ for $a\in D$, $x\in X$. By hypothesis, $\dim (D\times X)\leq n$.  Hence $f'$ can be approximated by a map $g'$ which is nondegenerate. The dimension of ${g'}^{-1}(K)\leq k$. Consider $\pi({g'}^{-1}(K))$; (where $\pi$ is the projection $D\times X\to D$), this has dimension $\leq k$; hence it cannot be all of the $(k+1)$-dimensional $D$. Choose some $a \in D-\pi({g'}^{-1}(K))$. Then $g'(a\times X)\cap K=\emptyset$. We define $g$ by, $g(x)=g'(a,x)$, for $x\in X$. Since $f(x)=f'(a,x)$, and $g'$ can be chosen to be as close to $f'$ as we like, we can get a $g$ as close to $f$ as we like. 
\end{proof}

We can draw a few corollaries, by applying the earlier approximation theorems.

\begin{ex}\label{chap5-ex5.2.6}
If $M$ is $ND(n)$, $K$ a subpolyhedron of $M$ of dimension $ \leq k$, then the pair $(M,M-K)$ is $(n-k-1)$-connected.
\end{ex}

[Hint: It is enough to consider maps $f:(D,\p D)\to (M,M-K)$, and show that such an $f$ is homotopic to a map $g$ by a homotopy which is fixed on $\p D$, and with $g(D)\subset M-K$. First, by \ref{chap5-thm5.2.5}, one can get\pageoriginale a very close approximation $g_{1}$ to $f$ with $g_{1}(D)\subset M-K$. Then since $g_{1}|\p D$ and $f|\p D$ are very close, there will be a small homotopy $h$ (\ref{chap3-sec3.2.3}) in a compact polyhedron in $M-K$ with $h_{0}=f|\p D$, $h_{1}=g_{1}|\p D$. Expressing $D$ as the identification space of $\p D\times I$ and $D_{1}$ (a cell with $\p D_{1}=\p D\times 1)$ at $\p D\times 1$ and patching up $h$ and the equivalent of $g_{1}$ on $D_{1}$, we get a map $g:D\to M$, with $g|\p D=f|\p D$, $g(D)\subset M-K$ and $g$ close to $f$. Then there will a homotopy of $f$ and $g$ fixed on $\p D$].

As an application this and \ref{chap5-prop5.2.4} we have:

\begin{proposition}\label{chap5-prop5.2.7}
If $\mathscr{S}$ is a simplicial presentation of an $ND(n)$-space and $\sigma\in \mathscr{S}$, then $|Lk(\sigma,\mathscr{S})|$ is $(n-\dim \sigma-2)$-connected. 
\end{proposition}

\begin{proof}
For by \ref{chap5-prop5.2.4}, $|St(\sigma,\mathscr{S})|=\overline{\sigma}\ast|Lk(\sigma,\mathscr{S})|$ is $ND(n)$, and by \ref{chap5-ex5.2.6}, $(|St(\sigma,\mathscr{S})|, |St(\sigma,\mathscr{S})|-\overline{\sigma})$ is $(n-\dim \sigma-1)$-connected, thus giving that $|St(\sigma,\mathscr{S})|-\overline{\sigma}$ is $(n-\dim \sigma-2)$-connected. But $|Lk(\sigma,\mathscr{S})|$ is a deformation retract of $|St(\sigma,\mathscr{S})|-\overline{\sigma}$.
\end{proof}

\section{Characterisations of $ND(n)$-spaces}\label{chap5-sec5.3}

We now introduce two more properties: the first an inductively defined local property called $A(n)$ and the second a property of simplicial presentations called $B(n)$ and which is satisfied by the simplicial presentations of $ND(n)$-spaces. It turns out that if $M$ is a polyhedron and $\mathscr{S}$ a simplicial presentation of $M$, then $M$ is $A(n)$ if and only if $\mathscr{S}$ is $B(n)$. Finally, we complete the circle by showing that $A(n)$-space have an approximation property which\pageoriginale is somewhat stronger than that assumed for $ND(n)$-spaces. $A(n)$ shows that $ND(n)$ is a local property. $B(n)$ is useful in checking whether a given polyhedron is $ND(n)$ or not. Using these, some more descriptions and properties of $ND(n)$-spaces can be given.

\begin{definition}[The property $A(n)$ for polyhedra]\label{chap5-defi5.3.1}
~

Any polyhedron is $A(0)$.

If $n\geq 1$, a polyhedron $M$ is $A(n)$ if and only if the link of every point in $M$ is a $(n-2)$-connected $A(n-1)$.
\end{definition}

\begin{definition}[The property $B(n)$ for simplicial presentations]\label{chap5-defi5.3.2}
~

A simplicial presentation $\mathscr{S}$ is $B(n)$, if and only if for every $\sigma\in \mathscr{S}$, $|Lk(\sigma,\mathscr{S})|$ is $(n-\dim\sigma-2)$-connected.

\end{definition}

By \ref{chap5-prop5.2.7}, we have

\begin{proposition}\label{chap5-prop5.3.3}
If $M$ is $ND(n)$, then every simplicial presentation $\mathscr{S}$ of $M$ is $B(n)$.
\end{proposition}

The next to propositions show that $A(n)$ and $B(n)$ are equivalent (ignoring logical difficulties).

\begin{proposition}\label{chap5-prop5.3.4}
If $M$ is $A(n)$, then every simplicial presentation of $M$ is $B(n)$. 
\end{proposition}

\begin{proof}
The proof is by induction on $n$. For $n=0$, the $B(n)$ condition says that certain sets are $(\leq -2)$-connected, i.e.\@ any $\mathscr{S}$ is $B(0)$, agreeing with the fact that any $M$ is $A(0)$. Let $n>0$, and assume the proposition for $m<n$.

Let $|\mathscr{S}|=M$, $\sigma\in \mathscr{S}$ and $\dim \sigma=k$.

If $k=0$, then by the condition $A(n)$, the link of the element of $\sigma$, which can be taken to be $|Lk(\sigma,\mathscr{S})|$ is $(n-2)$-connected.\pageoriginale 

If $k>0$, let $x$ be any point of $\sigma$. Then a link of $x$ in $M$ is $\p \sigma\ast|Lk(\sigma,\mathscr{S})|$, which is $A(n-1)$ by hypothesis.

Hence by inductive hypothesis, its simplicial presentation $\{\p \sigma\}\ast Lk(\sigma,\mathscr{S})$ satisfies $B(n-1)$. If $\tau$ is any $(k-1)$-dimensional face of $\sigma$,
$$
|Lk(\sigma,\mathscr{S})|=|Lk(\tau,\{\p \sigma\}\ast Lk(\sigma,\mathscr{S}))|
$$
which is $((n-1)-(k-1)-2)$-connected i.e.\@ $(n-k-2)$-connected since $\{\p \sigma\}\ast Lk(\sigma,\mathscr{S})$ is $B(n-1)$.
\end{proof}

\begin{proposition}\label{chap5-prop5.3.5}
If a polyhedron $M$ has a simplicial presentation $\mathscr{S}$ which is $B(n)$, then $M$ is $A(n)$.
\end{proposition}

\begin{proof}
The proof is again by induction. For $n=0$, it is the same as in the previous case. And assume the proposition to be true for all $m<n>0$.

Let $x\in M$. Then $x$ belongs to some simplex $\sigma$ of $\mathscr{S}$, and a link of $x$ in $M$ is $\p \sigma\ast|Lk(\sigma,\mathscr{S})|$. We must show that this is an $(n-2)$-connected $A(n-1)$.

As per connectivity, we note (setting $k=\dim \sigma$) that $\p \sigma$ is a $(k-1)$-sphere; and by $B(n)$, $|Lk(\sigma,\mathscr{S})|$ is $(n-k-2)$-connected. As the join with a $(k-1)$-sphere rises connectivity by $k$, $\p \sigma\ast|Lk(\sigma,\mathscr{S})|$ is $(n-2)$-connected.

To prove that $\p \sigma\ast |Lk(\sigma,\mathscr{S})|$ is $A_{n-1}$, it is enough to show that $\{\p \sigma\}\ast Lk(\sigma,\mathscr{S})=\mathscr{S}'$ say is $B(n-1)$; for then by induction it would follow that $|\mathscr{S}'|=\p \sigma\ast|Lk(\sigma,\mathscr{S})|$ is $A(n-1)$. Take a typical simplex $\mathcal{L}$ of $\mathscr{S}'$. It is of the form $\beta\gamma$,\pageoriginale $\beta\in \{\p \sigma\}$, $\gamma\in Lk(\sigma,\mathscr{S})$, with $\beta$ or $\gamma=\emptyset$ being possible. Now $Lk(\mathcal{L},\mathscr{S}')=Lk(\beta,\{\p \sigma\})\ast Lk(\gamma,Lk(\sigma,\mathscr{S}))$. 

Let $a$, $b$, $c$ be the dimensions of $\alpha$, $\beta$, $\gamma$ respectively. $a=b+c+1$. Remember that $\dim \sigma=k$. Therefore $|Lk(\beta,\{\p \sigma\})|$ is a $(k-b-2)$-sphere. Now $|Lk(\gamma,Lk(\sigma,\mathscr{S}))|=|Lk(\gamma\sigma,\mathscr{S})|$; and by $B(n)$ assumption, this is $(n-(c+k+1)-2)$ connected. Hence the join of $|Lk(\beta,\{\p \sigma\})|$ and $|Lk(\gamma,Lk(\sigma,\mathscr{S}))|$ which is $|Lk(\alpha,\mathscr{S}')|$ is 
$$
[(n-(c+k+1)-2)+(k-b-2)+1]\text{-connected}
$$
that is $((n-1)-a-2)$-connected.

Thus $\mathscr{S}'$ is $B(n-1)$, and therefore by induction $|\mathscr{S}'|=\p \sigma\ast\break|Lk(\sigma,\mathscr{S})|$, a link of $x$ in $M$ is a $(n-2)$-connected $A(n-1)$. Hence $M$ is $A(n)$.
\end{proof}

We need the following proposition for the next theorem.

\begin{proposition}\label{chap5-prop5.3.6}
Let $\mathscr{P}$ be a regular presentation of an $A(n)$-space $M$ and $\eta$ be any centering of $\mathscr{P}$. Let $A$ be any element of $\mathscr{P}$, and $\dim A=k$. Then
$$
|\lambda A|\text{~ is an~ }(n-k-2)\text{-connected}\quad A_{n-k-1}
$$
and $|\delta A|$ is a contractible $A_{n-k}$.
\end{proposition}

\begin{proof}
We know that $\lambda A$ is the link of a $k$-simplex in $d\mathscr{P}$. Since $d\mathscr{P}$ satisfies $B(n)$, $|\lambda A|$ is $(n-k-2)$-connected.

If $k=0$, $\lambda A$ is the link of a point and therefore $A(n-1)$, since $M$ is $A(n)$. 

If $k>0$, then $\p A\ast |\lambda A|$ is a link of a point in $M$, and\pageoriginale so is $A(n-1)$. Take a $(k-1)$-simplex $\sigma$ of $d\mathscr{P}$ in $\p A$; then $\lambda A$ is $Lk(\sigma,d\{\p A\}\ast\lambda A)$ which (by induction on $k$), we know to be a presentation of an $A((n-1)-(k-1)-1)$-space.

To prove that $|\delta A|$ is $A(n-k)$, we prove that $\delta A$ is $B(n-k)$. Consider its vertex $\eta A$, then $Lk(\eta A,\delta A)=\lambda A$, and $|\lambda A|$ is $(n-k-2)$-connected. For a simplex $\sigma \in \lambda A$, we have $|Lk(\sigma,\delta A)|=\subset |(Lk(\sigma,\lambda A))|$ which is contractible. For a simplex $\tau=\sigma\{\eta A\}$, $\sigma\in \lambda A$, $Lk(\tau,\delta A)=Lk(\sigma,\lambda A)$. If $\tau$ has dimension $t$, $\sigma$ has dimension $(t-1)$; and so $|\lambda A|$ being $A(n-k-1)$, $|Lk(\sigma,\lambda A)|$ is $((n-k-1)-(t-1)-2)$-connected, i.e.\@ $|Lk(\tau,\delta A)|$ is $((n-k)-t-2)$-connected. This shows that $\delta A$ satisfies $B(n-k)$.
\end{proof}

\begin{theorem}\label{chap5-thm5.3.7}
Let $M$ be an $A(n)$-space, $Y\subset X$ polyhedra of dimension $\leq n$, and $f:X\to M$ a map such that $f|Y$ is non-degenerate. Given any $\epsilon>0$, there is an $\epsilon$-approximation $g$ to $f$ such that $g$ is nondegenerate and $g|Y=f|Y$. 
\end{theorem}

\begin{proof}
The proof will be by induction on $n$. If $n=0$, we take $g=f$, since any map on a $0$-dimensional polyhedron is nondegenerate.

So assume $n>0$, that the proposition with $m$ instead of $n$ to be true for all $m<n$.

Without loss of generality we can assume that $f$ is polyhedral.\break Choose simplicial presentations $\mathscr{Z}\subset \mathscr{S}$, $\mathscr{M}$ of $Y$, $X$ and $M$ such that $f$ is simplicial with respect to $\mathscr{S}$ and $\mathscr{M}$; and such that the diameter of the star of each simplex in $\mathscr{M}$ is less than $\epsilon$. Let $\theta$, $\eta$ be centerings of $\mathscr{S}$ and $\mathscr{M}$ with $f(\theta\sigma)=\eta(f\sigma)$\pageoriginale for all $\sigma\in\mathscr{S}$. Then clearly $f^{-1}(\mathscr{M}^{k})\subset \mathscr{S}^{k}f(\mathscr{Z}^{k})\subset \mathscr{M}^{k}$ and the diameter of $|\delta \rho|$ is less than $\epsilon$ for every $\rho\in \mathscr{M}$.

Consider an arrangement $A_{1},\ldots,A_{r}$ of simplexes of $\mathscr{M}$ so that\break $\dim A_{i}\geq \dim A_{i+1}$, for $1\leq i\leq k$. The crucial fact about such an arrangement is, for each $i$, (*) $\lambda A_{i}$ is the union of $\delta A_{j}$ for some $j$'s less than $i$.

We construct an inductive situation $\sum_{i}$ such that
\begin{enumerate}
\renewcommand{\labelenumi}{(\theenumi)}
\item $X_{i}=f^{-1}(|\delta A_{1}|\cup \ldots \cup |\delta A_{i}|)$

\item $Y_{i}=X_{i}\cap Y$

\item $g_{i}:X_{i}\to M$, a nondegenerate map

\item $g_{i}(f^{-1}|\delta A_{i}|)\subset |\delta A_{i}|$

\item $g_{i}|X_{i-1}=g_{i-1}$

\item $g_{i}|Y_{i}=f|Y_{i}$ 
\end{enumerate}
\noindent
$\mathscr{M}^{n}$ is the union of certain $A_{i}$'s in the beginning, say $A_{i}$'s with $i\leq \ell$. $f^{-1}(\mathscr{M}^{n})\subset \mathscr{S}^{n}$ and $|\mathscr{S}^{n}|$ is $0$-dimensional. Hence $f|f^{-1}(|\delta A_{1}|\ldots |\delta A_{\ell}|)$ is already nondegenerate. If we take this to be $g_{\ell}$ all the above properties are satisfied an we have more than started the induction. Now let $i>\ell$ and suppose that $g_{i-1}$ is defined, that is we already have the situation $\sum_{(i-1)}$.

It follows from (4) and (5), that for $j<i$, $g_{i-1}(f^{-1}|\delta A_{j}|)\subset |\delta A_{j}|$, and hence from (*) that $g_{i-1}$ maps $f^{-1}(|\lambda A_{i}|)$ into $|\lambda A_{i}|$.

Also\pageoriginale this shows that if $x\in f^{-1}(|\delta A_{j}|)$ then both $g_{i-1}(x)$ and $f(x)$ are in $|\delta A_{j}|$, which has diameter $<\epsilon$, and so $g_{i-1}$ is an $\epsilon$-approximation to $f|X_{i-1}$.

There are ow two cases.
\begin{case}
$\dim A_{i}=k\geq 1$.
\end{case}

Look at $|\delta A_{i}|$. This is a contractible $A(n-k)$. Let $X'=f^{-1}(|\delta A_{i}|)$ and $Y'=(Y\cap X')\cup f^{-1}(|\lambda A_{i}|)$. The maps $f$ on $Y\cap X'$ and $g_{i-1}$ on $f^{-1}(|\lambda A_{i}|)$ agree where both are defined by $\sum_{i-1}(6)$, and are nondegenerate by hypothesis and induction. Hence patching them up we get a nondegenerate map $f':Y'\to |\delta A_{i}|$. Since $|\delta A_{i}|$ is contractible $f'$ can be extended to a map (still denoted by $f'$) of $X'$ to $|\delta A_{i}|$. Since $X'\subset |\mathscr{S}^{k}|$, $\dim X'\leq n-k$, and $|\delta A_{i}|$ is $A(n-k)$, there is a nondegenerate map $f'':X'\to |\delta A_{i}|$ such that $f''|Y'=f'|Y'$, by using the theorem for $(n-k)\leq n-1$. 

We now define $g_{i}$ to be $g_{i-1}$ on $X_{i-1}$ and $f''$ on $X'$; these two maps agree where both are defined, namely $f^{-1}(|\lambda A_{i}|)$. Thus $g_{i}$ is well defined and is nondegenerate as both $f''$ and $g_{i-1}$ are nondegenerate. And clearly all the six conditions of $\sum_{i}$ are satisfied.
\begin{case}%2
$\Dim A_{i}=0$.
\end{case}

Let $B_{1},\ldots,B_{s}$ be the vertices of $\mathscr{S}$ which are mapped onto $A_{i}$. Then
\begin{gather*}
f^{-1}(|\delta A_{i}|)=|\delta B_{1}|\cup\ldots |\delta B_{s}|.\\
\text{Let } \qquad X'=|\lambda B_{1}|\cup\ldots \cup |\lambda B_{s}|=|\mathscr{S}^1|\cap f^{-1}(|\delta A_{i}|)
\end{gather*}
$X'$\pageoriginale is of dimension $\leq n-1$, and contains $f^{-1}(|\lambda A_{i}|)$. Let $Y'=f^{-1}(|\lambda A_{i}|)$. Here the important point to notice is, that $Y\cap X'\subset Y'$. This is because $f|Y$ is nondegenerate: $Y\cap X'\subset \mathscr{Z}^{1}$, so $f(Y\cap X')\subset \mathscr{M}^{1}$. It is also in $|\delta A_{i}|$ and therefore $f(Y\cap X')\subset |\mathscr{M}^{1}|\cap |\delta A_{i}|=|\lambda A_{i}|$. 

We first extend $g_{i-1}|Y'$ to $X'$ and then by conical extension to $f^{-1}\break(|\delta A_{i}|)$. $g_{i-1}$ maps $Y'$ into $|\lambda A_{i}|$ and is non-degenerate on $Y'$. Since $|\lambda A_{i}|$ is $(n-2)$-connected, and $\dim X'\leq n-1$, $g_{i-1}|Y'$ can be extended to a map $f'$ of $X'$ into $|\lambda A_{i}|~|\lambda A_{i}|$ is also $A(n-1)$. Hence by the inductive hypothesis we can approximate $f'$ by a nondegenerate map $f''$ such that $f''|Y'=f'|Y'=g_{i-1}|Y'$. 

Hence $f''||\lambda B_{j}|,1\leq j\leq s$ is nondegenerate and maps $|\lambda B_{j}|$ into $|\lambda A_{i}|$. We extend this to a map $h_{j}:|\delta B_{j}|\to |\delta A_{i}|$, by mapping $B_{j}$ to $A_{i}$ and taking the join is clearly nondegenerate. Since $|\delta B_{j}|\cap |\delta B_{j'}|\subset |\lambda B_{j}|\cap |\lambda B_{j'}|\subset X'$, if $j\neq j'$, $h_{j}$'s agree whereever their domains of definition overlap. Similarly $h_{j}$ and $g_{i-1}$ agree where both are defined. We now define $g_{i}$ to be $g_{i-1}$ on $X_{i-1}$ and $h_{j}$ on $|\delta B_{j}|$. Thus $g_{i}$ is defined on $X_{i-1}\cup f^{-1}(|\delta A_{i}|)=X_{i}$ and is nondegenerate since $g_{i-1}$ and $h_{j}$'s are nondegenerate. It obviously satisfies conditions 1-5 of $\sum_{i}$, to see that it satisfies (6) also: Let $\sigma$ is any simplex of $d\mathscr{Z}$ in $\delta B_{j}$, if $B_{j}$ is not a vertex of $\sigma$ there is nothing to prove; if $B_{j}$ is a vertex of $\sigma$, write $\sigma=\{B_{j}\}\sigma'$. Both $h_{j}$ and $f$ agree on $\overline{\sigma}'$ and $B_{j}$ and on $\overline{\sigma}$ both are joins, hence both are equal on\pageoriginale $\overline{\sigma}$. Then (6) is also satisfied and we have the situation $\sum_{i}$.  
\end{proof}

This theorem shows in particular that $ND(n)$ is a local property; and that $ND(n)$-spaces have stronger approximation property than is assumed for them.

The following propostions, which depend on the computations of links are left as exercises.

\begin{exprop}\label{chap5-exprop5.3.8}
$C(X)$ and $S(X)$ are $ND(n)$ if and only if $X$ is an $(n-2)$-connected $ND(n-1)$.
\end{exprop}

Thus the $k^{\text{th}}$ suspension of $X$ is $ND(n)$ if and only if $X$ is an $(n-k-1)$-connected $ND(n-k)$-space.

\begin{exprop}\label{chap5-exprop5.3.9}
Let $\mathscr{S}$ be a simplicial presentation of $X$. Then $X$ is $ND(n)$ if and only $|Lk(v,\mathscr{S})|$ is $(n-2)$-connected $ND(n-1)$ for each vertex $v$ of $\mathscr{S}$.
\end{exprop}

\begin{exprop}\label{chap5-exprop5.3.10}
If $\mathscr{S}$ is a simplicial presentation of an $ND(n)$-space, and $0\leq k\leq n$, then the skeleton $\mathscr{S}_{k}$ is $ND(k)$, and the dual skeleton $\mathscr{S}^{k}$ is $ND(n-k)$.
\end{exprop}

Thus the class of $ND(n)$-spaces is much larger than the class of $PL$ $n$-manifolds, which incidentally are $ND(n)$ by the $B(n)$-property.

The results of this section can be summarised in the following proposition:

\begin{proposition}\label{chap5-prop5.3.11}
The following conditions on a polyhedron $M$ are equivalent:
\begin{enumerate}
\renewcommand{\labelenumi}{\rm(\theenumi)}
\item $M$ is $ND(n)$

\item $M$ is $A(n)$

\item a\pageoriginale simplicial presentation of $M$ is $B(n)$

\item every simplicial presentation of $M$ is $B(n)$

\item there exists a simplicial presentation $\mathscr{S}$ of $M$ such that $|LK(v,\mathscr{S})|$ is $(n-2)$-connected $A_{n-1}$ for all $v\in \mathscr{S}$ and $\dim v=0$

\item $M$ satisfies the approximation property of theorem \ref{chap5-thm5.3.7}. 

\item $M\times I$ is $ND(n+1)$.
\end{enumerate}
\end{proposition}

\section{Singularity Dimension}\label{chap5-sec5.4}

\subsection{Definitions and Remarks}\label{chap5-sec5.4.1}
Let $P$ and $M$ be two polyhedra,\break $\dim P=p$, $\dim M=m$, $p\ m$, and $f:P\to M$  a nondegenerate map. Ed define the singularity of $f$ (or the 2-fold singularity of $f$) to be set $\{x\in P|f^{-1}f(x)$ contains at least 2 points$\}$, and denote it by $S(f)$ or $S_{2}(f)$. By triangulating $f$, it can be seen easily that $S(f)$ is a finite union of open cells, so that $\overline{S(f)}$ is a subpolyhedron of $p$.

Similarly, we define the {\em $r$-fold singularity of $f$} for $r\geq 3$, to be the set $\{x\in P|f^{-1}f(x)$ contains at least $r$ points$\}$. This will be denoted by $S_{r}(f)$. As above $S_{r}(f)$ is a finite union of open cells, so that $\overline{S_{r}(f)}$ is a subpolyhedron of $P$. Clearly $S{2}(f)\supset S_{3}(f))\ldots$; and $S_{r}(f)$ are empty after a certain stage; since $f$ is nondegenerate.

The number $(m-p)$ is usually referred to as the {\em codimension}; and the number $r(p)-(r-1)m$, for $r\geq 2$ is called the {\em $r$-fold point dimension} and is denoted by $d_{r}$ (see e.g.\@ Zeeman ``Seminar on combinatorial Topology'', Chapter VI). Clearly $d_{r}=d_{r-1}-(m-p)$. 

It will be convenient to use the notions of dimension and imbedding\pageoriginale in the following cases: (1) dimension of $A$, where $A$ is a union of open cells. In this case the dim. $A$ denotes the maximum of the dimensions of the open cells comprising $A$ and is the same as the dimension of the polyhedron $\overline{A}$. (2) Imbedding $f$ of $C\to M$, when $C$ is an open cell and $M$ a polyhedron. This will be used only when $f$ comes from a polyhedral embedding of $\overline{C}$. In such a case $f(C)$ will be the union of a finite member of open cells. And if $A\subset M$ is some finite union of open cells, then $f^{-1}(A)$ will be finite union of open cells and one can talk of its dimension etc..

A nondegenerate map $f:P\to M$ will be said to be in {\em general position} if
$$
\dim (S_{r}(f))\leq d_{r},\quad \text{for all}\quad r
$$

If $p=m$, this means nothing more than that $f$ is nondegenerate, so usually $p<m$. 

\setcounter{proposition}{1}
\begin{proposition}\label{chap5-prop5.4.2}
Let $\mathscr{P}$ be a regular presentation of a polyhedron $P$ such that for every $C\in \mathscr{P}$, $f|C$ is an embedding. Let the cells of $\mathscr{P}$ be $C_{1},\ldots,C_{t}$, arranged so that $\dim C_{i}\leq \dim C_{i+1}$, $1\leq i\leq t$, and let $P_{i}$, $i\leq t$ be the subpolyhedron of $P$ whose presentation is $\{C_{1},\ldots,C_{i}\}$. Then 
\begin{itemize}
\item[\rm(i)] 
\begin{tabbing}
  $S_{2}(f|P_{i})$ \= = $S_{2}(f|P_{i-1})\cup \{C_{i}\cap f^{-1}(f(P_{i-1}))\}$\\[5pt]
 \>\qquad $\cup \{P_{i-1}\cap f^{-1}(f(C_{i}))\}$
\end{tabbing}

\item[\rm(ii)] \begin{tabbing}
$S_{r}(f|P_{i})$ \= = $S_{r}(f|P_{i-1})$\\[4pt]
\>\quad $\cup\{C_{i}\cap f^{-1}(f(S_{r-1}(f|P_{i-1}))\}$\\[4pt]
\>\quad $\cup\{S_{r-1}(f|P_{i-1})\cap f^{-1}(f(C_{i}))\}$ 
\end{tabbing}
\end{itemize}
\end{proposition}

This\pageoriginale is obvious. If we write $P=S_{1}(f)$, (compatible with the definition of $S_{r}$'s, then $S_{1}(f\ P_{i})$ would be just $P_{i}$, and only (ii) be written (with $r\geq 2$) instead of (i) and (ii).

The proposition is useful in inductive proofs. For example, to check
that a nondegenerate $f$ is in general position, it is enough check
for each little cell $C_{i}$, that $\dim C_{i}\cap
f^{-1}(f(S_{r-1}(f|P_{i-1}))\leq d_{r}$. If we have already checked
upto the previous stage; since $f$ is non-degenerate
$f^{-1}f(S_{r-1}\break(f|P_{i-1}))$ will of dimension $d_{(r-1)}$, and then
we will have to verify that $f\ C_{i}$ intersects
$f(S_{r-1}(f|P_{i})))$ in codimension $\geq (m-p)$ or that $(C_{i})$
intersects $f^{-1}f(S_{r-1}(f|P_{i}))$ in codimension $\geq m-p$, (We
usually say that {\em $A$ intersects $B$ in codimension $q$} if $\dim
(A\cap B)=\dim B-q$. Similarly the expression `{\em $A$ intersects $B$
  in codimension $\geq q$}' is used to denote $\dim (A\cap B)\leq \dim
B-q$). The aim of the next few propositions is to obtain presentations
on which it would be possible to inductively change the map, so that
$f(C_{i})$ will intersect the images of the previous singularities in
codimension $\geq (m-p)$. Proposition \ref{chap5-prop5.4.7} and
\ref{chap5-prop5.4.9} are ones we need; the others are auxilary to
these. 

\begin{ex}\label{chap5-ex5.4.3}
Let $A$, $B$, $C$, be three open convex cells, such that $A\cap B$ is a single point and $C\supset A\cup B$. Then $\dim C\geq \dim A+\dim B$. 
\end{ex}

\noindent
[{\bf Hint:} First observe that if $A'$ and $B'$ are any two {\em intersecting} open cells then $L_{A'}\cap L_{B'}=L_{A'\cap B'}$, where $L_{X}$ denotes the linear manifold spanned by $X$. Applying this to the above situation 
\begin{align*}
\dim C=\dim L_{C}\geq \dim L_{(A\cup B)} &= \dim L_{A}+\dim L_{B}-\dim (L_{A}\cap L_{B})\\
&= \dim L_{A}+\dim L_{B}-\dim (L_{A\cap B})\\
&= \dim L_{A}+\dim L_{B},\text{~ since~ } A\cap B
\end{align*}\pageoriginale
is a point.]

\begin{proposition}\label{chap5-prop5.4.4}
Let $A$ be an open convex cell of dimension $n$, and $\mathfrak{a}$ a regular presentation of $\overline{A}$ with $A\in\mathfrak{a}$. If $L$ is any linear manifold such that $\dim L\cap A=k\geq \theta$, then there is a $B\in\mathfrak{a}$, of dimension $\leq n-k$, with $B\cap L\neq \emptyset$. Further, if $A'$ is any cell of $\mathfrak{a}$ contained in $\p A$, we can require that $\overline{A'}\cap B=\emptyset$. 
\end{proposition}

\begin{proof}
If $k=0$, we can choose $A$ itself to be $B$. If $k>0$, consider the regular presentation $\mathscr{C}=\{C\cap L|C\cap L\neq \emptyset, C\in \mathfrak{a}\}$ of $\overline{A} \cap L$. $\mathscr{C}$ must have more than one $0$-cell. Choose one of these $0$-cells of $C$. It must be the form $B\cap L$ for some $B\in \mathfrak{a}$. We would like to apply \ref{chap5-prop5.4.2}, for $B$, $L\cap A$ and $A$. But $B$ and $L\cap A$ do not intersect. Since we are interested in the dimension of $B$, the situation can be remedied as follows: Let $D$ be an $n$-cell, such that $\overline{A}\subset D$. $L\cap D$ is again $k$-dimensional. Since $B\subset D$, $B\cap L\subset D\cap L$, and as $B\cap L$ is nonempty, $B$ and $D\cap L$ intersect. $B\cap (D\cap L)$ cannot be more than one point since $B\cap (D\cap L)\subset B\cap L$ which is just a point. Applying \ref{chap5-prop5.4.2} to $B$, $D\cap L$ and $D$ we have $n=\dim D\geq \dim B+\dim (D\cap L)=\dim B+k$, or, $\dim B\leq n-k$.


To see the additional remark, observe that all the vertices of $\mathscr{C}$ cannot be in $\overline{A'}$, for then $L\cap \overline{A}\subset \overline{A'}$, contrary to the hypothesis that $L\cap A$ is nonempty. Hence we can choose a $0$-cell $B\cap L$, $B\in \mathfrak{a}$ of $\mathscr{C}$ not in $\overline{A'}$. Since $\mathfrak{a}$ is a regular presentation $B\cap \overline{A'}=\emptyset$.
\end{proof}

This just means that if $L$ does not intersect the cells of $\mathfrak{a}$\pageoriginale of $\dim \leq \ell$, then dimension of the intersection is $<n-\ell$, or codimension of intersectoin is $>\ell$. Using the second remark of \ref{chap5-prop5.4.4} we have:

\begin{corollary}\label{chap5-coro5.4.5}
Let $\mathscr{P}$ be a regular presentation, containing a full subpresentation $\mathcal{Q}$ (which may be empty). Let $\mathscr{P}_{k}=\{\mathscr{C}\in \mathscr{P}-\mathcal{Q},\dim C\leq k\}$. If $L$ is any linear manifold which does not intersect $\mathscr{P}_{k}$, then $\dim (L\cap(\mathscr{P}-\mathcal{Q})\leq n-k-1$, where $n=\dim(\mathscr{P}-\mathcal{Q})$.
\end{corollary}

\begin{proposition}\label{chap5-prop5.4.6}
Let $A$ be a closed convex cell of dimension $\geq k+q$, let $S$ be a $(k-1)$-sphere in $\p A$: and $B_{1},\ldots,B_{r}$ be a finite number of open convex cells of dimension $\leq q-1$ contained in the interior of $A$. Further, let $\mathscr{S}$ be a simplicial presentation of $S$. Then there is an open dense set $U$ of interior $A$ such that if $a\in U$, $\sigma\in \mathscr{S}$, then the linear manifold $L_{(\sigma,a)}$ generated by $\sigma$ and `$a$' does not intersect any of the $B_{i}$'s.
\end{proposition}

\begin{proof}
For any $\sigma\in\mathscr{S}$, consider the linear manifolds $L_{(\sigma,B_{i})}$ generated by $\sigma$ and $B_{i}$, for $1\leq i\leq r$. $\Dim L_{(\sigma,B_{i})}\leq k+q-1$. Hence $U_{\sigma}=\text{int\,}A-\bigcup\limits_{i}L_{(\sigma,B_{i})}$ is an open dense subset of int $A$. If $a$ is any point of $U_{\sigma}$, then $L_{(\sigma,a)}$ does not intersect any of the $B_{i}$'s; for if there were $a$ $B_{j}$ with $L_{(\sigma,a)}\cap B_{j}\neq \emptyset$, let $b\in L_{(\sigma,a)}\cap B_{j}$. $L_{(\sigma,b)}\subset L_{(\sigma,a)}$ and is of the same dimension as $L_{(\sigma,a)}$, since $b$ is in the interior of $A$. Thus $a\in L_{(\sigma,b)}\subset L_{(\sigma,B_{j})}$ contrary to the choice of $a$. Therefore if we take $U=\bigcap\limits_{\sigma\in\mathscr{S}}\bigcup_{\sigma}$, $U$ satisfies our requirements.
\end{proof}

\begin{proposition}\label{chap5-prop5.4.7}
Let\pageoriginale $A$ be a closed convex cell of dimension $\geq k+q$, let $S$ be a $(k-1)$-sphere contained in $\p A$, and let $\{B_{1},\ldots,B_{r}\}$ be a finite number of open convex cells in int $A$. Then there is an open dense subset $U$ of int $A$ such that if $a\in U$, then $S\ast a$ intersects each of the $B_{i}$'s in codimension $\geq q$.
\end{proposition}

\begin{proof}
Let $\mathscr{S}$ be some simplicial presentation of $S$. First let us consider one $B_{i}$. Let $\mathscr{B}_{i}$ be a regular presentation of $\overline{B}_{i}$ containing a full subpresentation $\mathscr{X}_{i}$ covering $\overline{B}_{i}\cap \p A$. Let $\mathscr{B}_{q-1}=\{C\in\mathscr{B}_{i}-\mathscr{X}_{i},\dim C\leq q-1\}$. By \ref{chap5-prop5.4.6}, there is an open dense subset of int $A$ say $U_{i}$ such that if $a\in U_{i}$, $\sigma\in \mathscr{S}$, then $L_{(\sigma,a)}$ does not intersect any of the elements of $\mathscr{B}_{q-1}$. By \ref{chap5-coro5.4.5}, $\dim L_{(\sigma,a)}\cap (\mathscr{B}_{i}-\mathscr{X}_{i})\leq n_{i}-q$, where $n_{i}=\dim B_{i}$. Hence $\dim (S\ast a\cap B_{i})\leq n_{i}-q$. Therefore if we take $U=\bigcap_{j}U_{j}$, where $U_{j}$ constructed as above for each of $B_{j}$'s, then $U$, satisfies the requirements of the proposition.
\end{proof}

\begin{proposition}\label{chap5-prop5.4.8}
Let $\sigma$ be a $k$-simplex, $\Delta$ a closed convex $q$-cell; $\mathscr{P}$ a regular presentation of $\overline{\sigma}\ast\Delta$. Then there exists an open dense subset $U$ of $\Delta$, such that if $a\in U$, the linear manifold $L_{(\sigma,a)}$ spanned by $\sigma$ and $a$, does not intersect any cell $C\in \mathscr{P}$ satisfying $C\cap \overline{\sigma}=\emptyset$ and $\dim C\leq q-1$. 
\end{proposition}

\begin{proof}
Let $C\in\mathscr{P}$, with $C\cap \overline{\sigma}=\emptyset$ and $\dim C\leq q-1$. The linear manifold $L_{(\sigma,C)}$ has dimension $\leq k+q$, while $L_{(\sigma,\Delta)}$ has dimension $k+q+1$. Therefore $L_{(\sigma,C)}\cap \Delta$ has dimension $\leq q-1$ and so $U_{C}=\Delta-L_{(\sigma,C)}$ is open and dense in $\Delta$. Define $U$ to\pageoriginale be the intersection of all the $U_{C}$. If $a\in U$, and there were some $C$ of $\mathscr{P}$ of dimension $\leq q-1$, $C\cap \overline{\sigma}=\emptyset$, with $L_{(\sigma,a)}\cap C\neq \emptyset$, choose $b\in C\cap L_{(\sigma,a)}$; since $b\not\in \overline{\sigma}$, $\dim L_{(\sigma,b)}=k+1=\dim L_{(\sigma,a)}$ and so $L_{(\sigma,a)}=L_{(\sigma,b)}$ i.e. $L_{(\sigma,a)}\subset L_{(\sigma,C)}$, or, $a\in L_{(\sigma,C)}$ contrary to the choice of $a$.
\end{proof}

\begin{proposition}\label{chap5-prop5.4.9}
Let $S$ be a $(p-1)$-sphere, $\Delta$ a closed convex $q$-cell, $\mathscr{P}$ a regular presentation of $S\ast\Delta$. Then there exists
\begin{enumerate}
\renewcommand{\labelenumi}{\rm(\theenumi)}
\item a regular refinement $\mathscr{P}'$ of $\mathscr{P}$

\item a point $a\in \Delta$

\item a regular presentation $\mathcal{Q}$ of $S\ast a$ such that
\begin{enumerate}
\renewcommand{\theenumii}{\alph{enumii}}
\renewcommand{\labelenumii}{\rm(\theenumii)}
\item $\mathcal{Q}$ contains a full subpresentation $\mathscr{S}$ covering $S$,

\item Each $C\in \mathcal{Q}-\mathscr{S}$ is the intersection of a linear manifold with a (unique) cell $E_{C}\in\mathscr{P}'$, if $C\neq C'$, $E_{C}\neq E'_{C}$, and if $C<C'$, then $E_{C}<E'_{C}$

\item $\dim C\leq \dim E_{C}-q$, for all $C\in \mathcal{Q}-\mathscr{S}$.
\end{enumerate}
\end{enumerate}
\end{proposition}

\begin{proof}
Let $\mathfrak{a}$, $\mathscr{B}$ be simplicial presentations of $S$, $\Delta$; and let $\mathscr{P}'$ be a common simplicial refinement of $\mathfrak{a}\ast\mathscr{B}$ and $\mathscr{P}$. Since $\mathfrak{a}$ is full in $\mathfrak{a}\ast \mathscr{B}$, there is a subpresentation, say $\mathscr{S}$, of $\mathscr{P}'$ covering $S$. If $\sigma\in \mathfrak{a}$, $\overline{\sigma}\ast\Delta$ is covered by a subpresentation in $\mathfrak{a}\ast\mathscr{B}$, hence there is a subpresentation of $\mathscr{P}'$, say $\mathscr{P}'_{\sigma}$, covering $\overline{\sigma}\ast\Delta$. Applying \ref{chap5-prop5.4.8} to $\mathscr{P}'_{\sigma}$, we get an open dense subset $U_{\sigma}$ of $\Delta$. Let $U$ be the intersection of the sets $U_{\sigma}$ for $\sigma\in \mathfrak{a}$. Let $a\in U$. Obviously `$a$' is in an (open) $q$-simplex of\pageoriginale $\mathscr{P}'$ contained in $\Delta$. Hence `$a$' belongs to a $q$-simplex of $\mathscr{B}$, call it $\rho$.

We define $\mathcal{Q}$ to be union of $\mathscr{S}$, $\{a\}$, and all nonempty intersections of the form $L_{(\sigma,a)}\cap E$, for $\sigma\in \mathfrak{a}$, $E\in \mathscr{P}'-\mathscr{S}$. It is clear that $L_{(\sigma,a)}\cap E=\sigma\{a\}\cap E$. Moreover $\overline{E}\cap S=\overline{F}$, $F\in\mathscr{S}$ ($F$ may be empty) since $\mathscr{S}$ is full in $\mathscr{P}'$. This immediately gives that $\mathcal{Q}$ is a regular presentation, using the fact that $\p (A\cap B)$ is the disjoint union of $\p A\cap B$, $A\cap \p B$, $\p A\cap \p B$, for open convex cells $A$, $B$ with $A\cap B\neq \emptyset$. Moreover $\mathscr{S}$ is full in $\mathcal{Q}$. If $C\in \mathcal{Q}$ is of the form $C=L_{(\sigma,a)}\cap E$, we write $E$ as $E_{C}$. By definition each $C\in \mathcal{Q}-\mathscr{S}$ is the intersection of $E_{C}$ with a linear manifold, and if $C'<C$, $C'\in\mathcal{Q}-\mathscr{S}$, $E_{C'}<E_{C}$ since $\mathscr{P}'$ is regular. Since $L_{(\sigma,a)}$ does not intersect any $(\leq q-1)$-dimensional face $E$ of $E_{C}$ with $E\cap \overline{S}=\emptyset$, by \ref{chap5-coro5.4.5} $\dim L_{(\sigma,a)}\cap E_{C}\leq \dim E_{C-q}$. It remains to verify that if $C_{1}\neq C_{2}, C_{1}$, $C_{2}\in \mathcal{Q}-\mathscr{S}$, then $E_{C_{1}}\neq E_{C_{2}}$. Let $C_{1}=L_{(\sigma,a)}\cap E_{C_{1}}$, $C_{2}=L_{(\tau,a)}\cap E_{C_{2}}$; $\sigma$, $\tau\in \mathfrak{a}$, $E_{C_{1}}$, $E_{C_{2}}\in\mathscr{P}'-\mathscr{S}$, $C_{1}\neq \emptyset \neq C_{2}$. If $\sigma=\tau$, and $C_{1}\neq C_{2}$, clearly $E_{C_{1}}\neq E_{C_{2}}$. If $\sigma\neq \tau$, then $C_{1}$ cannot be equal to $C_{2}$. In this case $E_{C_{1}}\subset \sigma\rho$, $E_{C_{2}}\subset \tau \rho$, ($\rho$ defined in the first paragraph of the proof). But $\sigma\rho$ and $\tau\rho$ are disjoint, hence $E_{C_{1}}\neq E_{C_{2}}$.
\end{proof}

\begin{remark*}
In the above proposition $\mathscr{P}'$ can be taken any presentation of $S\ast \Delta$ refining $\mathscr{P}$ and a join presentation of $S\ast\Delta$.
\end{remark*}

\begin{proposition}\label{chap5-prop5.4.10}
Let $M$ be an $ND(n)$-space. Let $X\subset P$ be polyhedra\pageoriginale such that $P=X\cup C$, $C$ a closed convex cell, and $X\cap C=\p C$, and $\dim P=p\leq n$. Let $f:P\to M$ be a map such that $f/X$ is in general position. Then there exists an arbitrary close approximation $g$ to $f$ such that $g$ is in general position and $g/X=f/X$.
\end{proposition}

\begin{proof}
If $p=n$, any nondegenerate approximation of $f$ would do. So let $p<n$. In particular $\dim C\leq p<n$.

\smallskip
\noindent
{\bf Step A.}~ Let $D$ be an $(\leq n)$-dim-cell containing $\p C$ in its boundary, and such that
\begin{enumerate}
\renewcommand{\labelenumi}{(\theenumi)}
\item $D=\p C\ast \Delta$, $\Delta$ a closed convex $(n-p)$-cell

\item $\Delta \cap C$ is a single point `$d$' in the interior of both $C$ and and $\Delta$ so that $C=d\ast\p C$ 

\item $D\cap P=C$.
\end{enumerate}

This is clearly possible (upto polyhedral equivalence by considering $P\times 0$ in $V\times W$, (where $V$ is the vector space containing $P$, $W$ an $(n-p)$-dimensional vector space), and taking an $(n-p)$-cell $\Delta$ through $d\times 0$ in $d\times W$, for some $d\in C-\p C$ etc. The join of the identity on $\p C$ and the retraction $\Delta \to d$ gives a retraction $r:D\to C$. Thus $(f/C)\cdot r$ is an extension of $f/C$. Since $M$ is an $ND(n)$-space, $(f/C)\cdot r$ can be approximated by a non-degenerage map, say $h$, such that $h/\p C=f/\p C$. Let us patch up $f/X$ and $h$, and let this be also called $h$; now $h$ maps $X\cup D=P\cup D$ into $M$ and is nondegenerate. Triangulate $h$ so that the triangulation of $X\cup D$ with reference to which $h$ is simplicial contains a subpresentation $\mathscr{D}$ which refines a join presentation of $\p C\ast \Delta$ We apply\pageoriginale \ref{chap5-prop5.4.9} now, $\mathscr{D}$ will be $\mathscr{P}'$ there and we obtain, a point $a\in \Delta$, a presentation $\mathscr{B}$ (what was called $\mathcal{Q}$ there) of $\p C\ast a$. Each cell $B$ of $\mathscr{B}$ not in $\p C$, is the intersection of a unique $E_{B}$ of $\mathscr{D}$ with a linear manifold, if $B'<B$ then $E_{B'}<E_{B}$ and $\dim B\leq \dim E_{B}-(n-p)$. 

\medskip
\noindent{\bf Step B.}~ Let $B_{1},\ldots,B_{r}$ be the elements of $\mathscr{B}$ not in $\p C$, arranged so that $\dim B_{i}\leq \dim B_{i+1}$; for $1\leq i<r$. Let $X_{i}=X\cup B_{1}\cup\ldots\cup B_{i}$; $X_{i}$ is a polyhedron. We define a sequence of embeddings $\mathscr{L}_{i}:X_{i}\to X\cup D$, such that
\begin{enumerate}
\renewcommand{\labelenumi}{(\theenumi)}
\item $\mathscr{L}_{i}|X$ is the identity embedding of $X$ in $X\cup D$

\item $\mathscr{L}_{i}$ is an extension of $\mathscr{L}_{i-1}$

\item $\mathscr{L}_{i}(B_{i})\subset E_{B_{i}}$

\item $h\mathscr{L}_{i}$ is in general position.
\end{enumerate}

We shall construct the $\mathscr{L}_{i}$'s one at a time begining with $\mathscr{L}_{0}:X\to X\cup D$, the inclusion, $h\cdot \mathscr{L}_{0}=f/X$, is in general position, and we can start the induction.

Suppose $\mathscr{L}_{i-1}$ is already constructed. Then $\dim S_{r}(h\mathscr{L}_{i-1})\leq d_{r}$; and by (2), (3), $\mathscr{L}_{i-1}$ embeds $\p B_{i}$ in $\p E_{B_{i}}$; consider $h^{-1}(h\mathscr{L}_{i-1}(S_{r}(h\mathscr{L}_{i-1})))$ intersected with $E_{B_{i}}$. Since $h$ is nondegenerate, these consists of a finite number of open convex cells of dimension $\leq d_{r}$. We apply \ref{chap5-prop5.4.7} to this situation with $q=n-p$, $A=\overline{E}_{B_{i}}$, $S=\mathscr{L}_{i-1}(\p B_{i})$ and $\{B_{1},\ldots,\}$ of \ref{chap5-prop5.4.7} standing for the open cells of $h^{-1}(h\mathscr{L}_{i-1}(S_{r}(h\mathscr{L}_{i-1}))$ intersected with $E_{B_{i}}$ for all $r\geq 1$. By \ref{chap5-prop5.4.7}, we can choose a point in $E_{B_{i}}$ say $e_{i}$ ($a$ of \ref{chap5-prop5.4.7}) so that $\mathscr{L}_{i-1}(\p B_{i})\ast e_{i}$ intersects\pageoriginale all these (i.e.\@ for all $r\geq 1$) in codimension $\geq (n-p)$. The join of $\mathscr{L}_{i-1}|\p B_{i}$ and the 
map of a point $b_{i}$ of $B_{i}$ to $e_{i}$ gives the required extension on $B_{i}$. 

Then $\dim\{\mathscr{L}_{i}B_{i}\cap h^{-1}(h\mathscr{L}_{i-1}(S_{r}(h\mathscr{L}_{i-1}))\}\leq d_{r}-(n-p)$, equivalently $\dim \{(h\mathscr{L}_{i}B_{i})\cap h\mathscr{L}_{i-1}(S_{r}(h\mathscr{L}_{i-1}))\}d_{r+1} \leq d_{r+1}$, that is $\dim \{(h\mathscr{L}_{i}B_{i})\cap h\mathscr{L}_{i}(S_{r}(h\mathscr{L}_{i}|X_{i-1}))\}d_{r+1}$, since $h\mathscr{L}_{i}$ is an extension of $h\mathscr{L}_{i-1}$.

Since
\begin{align*}
S_{r+1}(h\mathscr{L}_{i}) &= S_{r+1}(h\mathscr{L}_{i-1})\\
&\quad \cup \{B_{i}\cap (h\mathscr{L}_{i})^{-1}(h\mathscr{L}_{i}(S_{r}(h\mathscr{L}_{i}|X_{i-1})))\\
&\quad \cup \{S_{r}(h\mathscr{L}_{i}|X_{i-1})\cap (h\mathscr{L}_{i})^{-1}(h\mathscr{L}_{i})(B_{i})\}
\end{align*}
and since $h\mathscr{L}_{i-1}$ is already in general position, $\dim S_{r+1}(h\mathscr{L}_{i})\leq d_{r+1}$. At the last stage, we get an imbedding $\mathscr{L}_{r}$ of $X\cup \p C\ast a$ in $X\cup D$, such that $h\mathscr{L}_{r}$ is an general position.

That $h\mathscr{L}_{r}$ can be chosen as close to $f$ as we like is clear.
\end{proof}

\begin{theorem}\label{chap5-thm5.4.11}
Let $M$ be an $ND(n)$-space, $X\subset P$ polyhedra $\dim p\leq n$ and $f:P\to M$ a map such that $f|X$ is in general position. Then there exists an arbitrary close approximation $g$ to $f$ such that $g|X=f|X$, and $g$ is in general position. 
\end{theorem}

\begin{proof}
Let $\mathscr{P}$ be a regular presentation of $P$ with $X$ covered by a subpresentation $\mathscr{X}$. Let $(\mathscr{P}-\mathscr{X})=\{A_{1},\ldots,A_{r}\}$ be arranged so that\pageoriginale $\dim A_{i}\leq \dim A_{i+1}$, $1\leq i<r$. Let $P_{i}=X\cup A_{1}\cup\ldots\cup A_{i}$, $X_{i}=P_{i-1}$. Apply proposition \ref{chap5-prop5.4.10} successively to $(P_{1},X_{1})\ldots,(P_{r},X_{r})$.

This requires the following comment: We must use our approximation theorem, which for $M$ and $\epsilon>0$ gives $\delta(\epsilon)>0$, such that for any $Y\supset Z$, $h_{1}:Y\to M$, $h_{2}:Z\to M$, if $h_{2}$ is polyhedral, and $h_{1}|Z$ is a $\delta(\epsilon)$-approximation to $h_{2}$, then there is $h_{3}:Y\to M$, a polyhedral extension of $h_{2}$, which is an $\epsilon$-approximation to $h_{1}$.

We want $g$ to be an $\epsilon$-approximation to $f$.

Define $\epsilon_{r}=\epsilon,\epsilon_{i-1}=\delta\left(\dfrac{\epsilon_{i}}{2}\right)$. 

Denote $f|P_{i}$ by $f_{i}$. We start with $g_{0}=f_{0}=f|X$. Suppose $g_{i-1}$ is defined on $P_{i-1}$ such that $g_{i-1}$ is in general position and is an $\epsilon_{i-1}$ approximation to $f_{i-1}$. Then we first extend $g_{i-1}$ to $P_{i}$ say $f'_{i}$ so that $f'_{i}$ is an $\epsilon_{\frac{i}{2}}$ approximation to $f_{i}$ (this is possible since $\epsilon_{i-1}=\delta(\epsilon i/2)$) by the approximation theorem. Then we use \ref{chap5-prop5.4.10} to get an $\epsilon_{i/2}$ approximation $g_{i}$ to $f'_{i}$ such that $g_{i}$ is in general position and $g_{i}|P_{i-1}=g_{i-1}$. $g_{i}$ is an $\epsilon_{i}$-approximation to $f_{i}$ and is in general position. $g_{r}$ gives the required extension.
\end{proof}

By the methods of \ref{chap5-prop5.4.10}, the following proposition can be proved:

\begin{proposition}\label{chap5-prop5.4.12}
Let $M$ be $ND(n)$; $\dim p\leq n$, $P=X\cup C$, where $C$ is a closed $p$-cell, $X\cap C=\p C$. Let $f:P\to M$ be a map,\pageoriginale such that $f|X$ is nondegenerate; and call $\dim X=x$. Then there is a nondegenerate approximation $g:P\to M$, arbitrarily close to $f$, such that $g|X=f|X$, and $S(g)=S(f|X)$ plus (a finite number of open convex cells of $\dim \leq \Max (2p-n,p+x-n)$)
\end{proposition}

\noindent
{\bf Sketch of the proof:}~ First we proceed as in Step A of \ref{chap5-prop5.4.10}. Now $p=\dim C$. In Step B) instead of 4) we write
$$
\dim\{B_{i}\cap (h\mathscr{L}_{i})^{-1}(h\mathscr{L}_{i-1}(X_{i-1}))\} \leq \Max (2p-n,p+x-n).  
$$

And in the proof instead of the mess before, we have only to bother about $h^{-1}(h\mathscr{L}_{i-1}(X_{i-1}))$, intersected with $E_{B_{i}}$.
\begin{gather*}
h^{-1}(h\mathscr{L}_{i-1}(X_{i-1}))=h^{-1}(h\mathscr{L}_{i-1}(X))\cup h^{-1}(h\mathscr{L}_{i-1}(B_{i}\cup\ldots\cup B_{i-1})=\\
=h^{-1}(f(X))\cup h^{-1}(h\mathscr{L}_{i-1}(B_{1}\cup\ldots\cup B_{i-1})).
\end{gather*}

Now the only possibility of $h^{-1}(f(X))$ intersecting $E_{B_{i}}$ is when $E_{B_{i}}\subset h^{-1}(f(X))$ since $h$ is simplicial. Since $\dim B_{i}\leq \dim E_{B_{i}}-(n-p)$, it already intersects in the right codimension. And the intersections with second set can be made minimal as before.\hfill$\Box$

\begin{theorem}\label{chap5-thm5.4.13}
Let $M$ be $ND(n)$; $X\supset P$; $\dim p\leq n$, $f:P\to M$ a map such that $f|X$ is an imbedding. Then arbitrary close to $f$ is a map $g:P\to M$, such that $g|X=f|X$ and calling $x=\dim X$, $p=\dim P-X$,
$$
\dim S(g)\leq \Max(2p-n,p+x-n).
$$
\end{theorem}

\begin{proof}
This follows from \ref{chap5-prop5.4.12}, as \ref{chap5-thm5.4.11} from \ref{chap5-prop5.4.10}. 
\end{proof}

This theorem is useful in proving the following embedding theorem\pageoriginale for $ND(n)$-spaces.

\begin{theorem}[Stated without proof]\label{chap5-thm5.4.14}
Let $M$ be a $ND(n)$-space, $P$ a polyhedron of dimension $p\leq n-3$ and $f:P\to M$ a $(2p-n+1)$-connected map. Then there is a polyhedron $Q$ in $M$ and a simple homotopy equivalence $g:P\to Q$ such that the diagram
\[
\xymatrix@=1.5cm{
P\ar[r]^{g}\ar[dr]_{f} & Q\ar[d]^-{\text{\rm inclusion}}\\
 & M
}
\]
is homotopy commutative.
\end{theorem}

The method of Step A in \ref{chap5-prop5.4.10}, gives;

\begin{proposition}\label{chap5-prop5.4.15}
Let $M$ be an $ND(n)$-space, and $P$ a polyhedron of dimension $p\leq n$ and $f:P\to M$ be any map. Then $\exists$ a regular presentation $\mathscr{P}$ of $P$, simplicial presentation $\mathscr{M}$ of $M$ and an arbitrary close approximation $g$ to $f$ such that, for each $C\in \mathscr{P}$, $g|(C)$ is a linear embedding, and $g(C)$ is contained in a simplex $\sigma_{C}$ of $\mathscr{M}$ of dimension = dimension $C+(n-p)$ and $g(\p C)=\p (gC)\subset \p \sigma_{C}$. Moreover $\mathscr{M}$ can be assumed to refine a given regular presentation of $M$. 
\end{proposition}

Also a relative version of \ref{chap5-prop5.4.15} could be obtained.\hfill$\Box$

And from this and \ref{chap5-thm5.4.13}.

\begin{theorem}\label{chap5-thm5.4.16}
Let $f:P\to M$ be a map from a polyhedra $P$ of $\dim =p$ into an $ND(n)$-space $M$, $p\leq n$, and let $Y$ be a subpolyhedron of $M$ of dimension $y$. Then there exists an arbitrary close approximation $g$ to $f$ such that
$$
\dim (g(P)\cap Y)\leq p+y-n.
$$
\end{theorem}

And\pageoriginale a relative version of \ref{chap5-thm5.4.16}.\hfill$\Box$

\setcounter{subsection}{16}
\subsection{}\label{chap5-sec5.4.17}
It should be remarked that the definition of `general position' in 
\ref{chap5-sec5.4.1} is a definition of general position, and other definitions are possible, and theorems, such as above can be proved. Here we formulate another definition and a theorem which can be proved by the methods of \ref{chap5-prop5.4.10}.

A dimensional function $d:P\to \{0,1,\ldots\}$ is a function defined on a polyhedron, with non-negative integer values, such that there is some regular presentation $\mathscr{P}$ of $P$ such that for all $C\in \mathscr{P}$, $x\in C$, $d(x)\geq \dim C$, and $d$ is constant on $C$.

We say $d_{1}\leq d_{2}$, if for all $x\in P$, $d_{1}(x)\leq d_{2}(x)$.

If $f:P\to M$ is a nondegenerate map, and $d$ a dimensional function, and $k_{1},\ldots,k_{s}$ non-negative integers, we define
\begin{align*}
& Sd(f;k_{1},\ldots,k_{s})
 = f^{-1}\{m\in M|\exists \text{~ distinct points}\\
&\qquad x_{1},\ldots,x_{s}\in P,\text{~ such that}\\
&\quad\;\; d(x_{i})\leq k_{i},\text{~ and~ }f(x_{i})=m\text{~ for all~ } i\}. 
\end{align*}

It is possible that such a set is a union of open simplexes, and hence its dimension is easily defined.

A map $f:P\to M$ is said to be {\em $n$-regular} with reference to a dimensional function $d$ on $P$ if it is nondegenerate and 
$$
\dim Sd(f;k_{1},\ldots,k_{s})\leq k_{1}+\cdots+k_{s}-(s-1)n.
$$
for all $s$, and all $s$-tuples of non-negative integers.

If $\dim P\leq n$, and since we have $f$ nondegenerate then it is\pageoriginale possible to show that a map $f$ is $n$-regular if it satisfies only a finite number of such inequalities, namely those for which all $k_{i}\leq n-1$ and $s<2n$.

The theorem that can be proved is this;

\begin{theorem*}
Let $X\subset P$, $f:P\to M$, where $M$ is $ND(n)$ and $\dim P\leq n$. Let $d_{X}$ and $d_{P}$ be dimensional functions on $X$ and $P$, with $d_{X}\leq d_{P}|X$. Suppose $f|X$ in $n$-regular with reference to $d_{X}$. Then $f$ can be approximated arbitrarily closely by $g:P\to M$ with $g|X=f|X$ and $g$ $n$-regular with reference to $d_{P}$.
\end{theorem*}

The proof is along the lines of theorem \ref{chap5-thm5.4.11}. We find a regular presentation $\mathscr{P}$ of $P$ with a subpresentation covering $X$, and such that $d_{X}$ and $d_{P}$ are constant on elements of $\mathscr{P}$. We utilise theorem \ref{chap5-prop5.4.10} to get $g$ on the cells of $\mathscr{P}$ one at a time; in the final atomic construction, analogous to part (B) of \ref{chap5-prop5.4.10}, we will have
$$
S\subset \p E
$$
where $S$ is a $(k-1)$-sphere. $E$ a cell of dimension $\geq k+q$, where $q=n-p$ (the cell we are extending over is a $p$-cell, on which $d_{P}$ is constant $\geq p$). We have to insert a $k$-cell that will intersect such things as
$$
h^{-1}(S_{d_{p}}(\phi_{i-1};k_{1},\ldots,k_{s}))
$$
in dimension
\begin{gather*}
\dim S_{d_{p}}(\phi_{i-1};k_{1},\ldots,k_{s})-q\\
\leq k_{1}+\cdots+k_{s}+d_{P}(p\text{-cell})-\text{s.n.}
\end{gather*}

We can do this for our situation; this inequality will imply $g_{i}$\pageoriginale is $n$-regular.\hfill$\Box$

Finally we can define on any polyhedron $P$ a canonial dimensional function $d$:
\begin{align*}
d(x) &: \text{Min}\Big\{\dim (\text{Stary in [Star of $x$ in $P$]}\\[4pt]
&\qquad y\in\ \text{~[Star of $x$ in $p$]}\Big\}
\end{align*}

A function $n$-regular with reference to this $d$ will be termed, perhaps, in general position, it being understood that the target of the function is $ND(n)$. Thus:

\begin{coro*}
If $X\subset P$, $\dim P\leq n$, $f:P\to M$, $M$ a $ND(n)$-space, and if $f|X$ is in general position then $f|X$ can be extended to a map $g:P\to M$ in general position such that $g$ closely approximates $f$. 
\end{coro*}

\setcounter{proposition}{17}
\begin{conclusion}
Finally, it should be remarked, that the above `general position' theorems, interesting though they are; are not delicate enough for many applications in manifolds. For example, one need: If $f:X\to M$ a map of a polyhedron $X$ into a manifold, and $Y\subset M$, the approximation $g$ should be such that not only $\dim (g(X)\cap Y)$ is minimal, but also should have $\overline{S_{r}(g)}$ intersect $Y$ minimally e.g.\@ if $2x+y<2n$, $\overline{S(g)}$ should not intersect $Y$ at all. The above procedure does not seem to give such results. If for example we know that $Y$ can be moved by an isotopy of $M$ to make its intersections minimal with some subpolyhedra of $M$, then these delicate theorems can be proved. This is true in the case of manifolds, and we refer to Zeeman's notes for all those theorems.
\end{conclusion}




