\thispagestyle{empty}
\begin{center}
{\Large\bf Lectures on}\\[5pt]
{\Large\bf Three-Dimensional Elasticity}
\vskip 1cm

{\bf By}\\[5pt]
{\large\bf P.~G.~Ciarlet}
\vfill

{\bf Tata Institute of Fundamental Research}

{\bf Bombay}

{\bf 1983}
\end{center}
\eject

\thispagestyle{empty}
\begin{center}
{\Large\bf Lectures on}\\[5pt]
{\Large\bf Three-Dimensional Elasticity}
\vskip 1cm

{\bf By}

{\large\bf P.~G.~Ciarlet}
\vfill

{Lectures delivered at the}\\[4pt]
{\bf Indian Institute of Science, Bangalore}\\[10pt]
{under the}\\[10pt]
{\bf T.I.F.R.I.I.Sc. Programme in Applications of}\\[4pt]
{\bf Mathematics}

\vfill

{\bf Notes by}\\[4pt]
{\large\bf S. Kesavan}
\vfill


{Published for the}\\[4pt]
{\bf Tata Institute of Fundamental Research}\\[5pt]
{\bf Springer-Verlag}\\[5pt]
{Berlin Heidelberg New York}\\[4pt]
{\large\bf 1983}
\end{center}
\eject

\thispagestyle{empty}
\begin{center}
{\large\bf \copyright Tata Institute of Fundamental Research, 1983}
\vfill

\rule{\textwidth}{.5pt}

ISBN 3-540-12331-8 Springer-Verlag, Berlin. Heidelberg. New York

ISBN 0-387-12331-8 Springer-Verlag, New York, Heidelberg. Berlin

\rule{\textwidth}{.5pt}
\vfill

\parbox{0.7\textwidth}{No part of this book may be reproduced in any 
form by print, microfilm or any other means without written permission
from the Tata Institute of Fundamental Research, Colaba, 
Bombay 400 005}
\vfill 

Printed by M. N. Joshi at The Book Centre Limited,

Sion East, Bombay 400 022 and published by H. Goetze,

Springer-Verlag, Heidelberg, West Germany

\vfill


{\bf Printed in India}
\end{center}
\eject

~
\thispagestyle{empty}

\vfill

\hfill {\em These Lecture Notes are dedicated to}

\hfill {\em Professor K.G. Ramanathan}

\vfill

\eject


\chapter*{Avant-Propos}

\addcontentsline{toc}{chapter}{Avant-Propos}


When\pageoriginale studying any physical problem in Applied Mathematics, three
essential stage are involved. 
\begin{enumerate}
\item Modelling: An appropriate mathematical model, based on the
  physics or the engineering of the sitution, must be found. Usualluy
  these models are given  \textbf {a pariori} by the physicists or
  the engineers themselves. However, mathematicians can also play an
  important role in this process especially considering the increasing
  emphasis on non - linear models of physical problems. 
\item Mathematical study of the model: A model usally involves a set
  of ordinary' or partial differential equations or an (energy)
  functional to be minimized. One of the first tasks is to find a
  suitable functional space in which to study the problem. Then comes
  the study of existence and uniqueness or non -uniqueness of
  solutions. An important feature of linear theories is the existence
  of unique solutions depending continuoussly on the data (Hadamard's
  definition of well - posed problems). But with non-linear problems,
  non-uniqueness is a prevealent phenomenon. For instance,
  bifuracation of solutions is of special interest. 
\item Numerical analysis of the model: By this is meant the
  description of, and the mathematical analysis of, approximation
  schemes, which \textbf{can} be run on a computer in a `reasonable'
  time to get `reasonably accurate' answers. 
\end{enumerate}

In the following set of lectures the first two of the above aspects
will be studied with reference to the theory of elasticity in three
dimensions. 

In\pageoriginale  the first chapter a non-linear system of partial differential
equations will be established as a mathematical model of
elasticity. The non-linearity will appear in the highest order terms
and this is an important source of difficulties. An energy functional
will be established and it will be seen that the equations of
equilibrium can be obtained as the Euler equations starting from the
energy functional. 

Existence results will be studied in the second chapter. The two
important tools will be the use of the implicit function theorem and
the theory of J. BALL. 
