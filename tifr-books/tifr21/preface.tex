\thispagestyle{empty}
\begin{center}
{\Large\bf Lectures on}\\[5pt] 
{\Large\bf Topics In The Theory of Infinite Groups}
\vskip 1cm

{\bf  By}
\medskip

{\large\bf  B.H. Neumann}
\vfill

{\bf  Notes by}
\medskip

{\large\bf  M. Pavman Murthy}
\vfill

\parbox{0.7\textwidth}{No part of this book may be 
reproduced in any form by print, microfilm or any 
other means without written permission from the 
Tata Institute of Fundamental Research, 
Colaba, Bombay 5}
\vfill


{\bf Tata Institute of Fundamental Research, Bombay}

{\bf 1960}

{\bf (Reissued 1968)}
\end{center}

\eject


\chapter{Preface}

As\pageoriginale the title of this course of lectures suggests, my aim was not a
systematic treatment of infinite groups. Instead I have tried to
present some of the methods and results that are new and look
promising, and that have not yet found their way into the books of
Kurosh, Specht, Zassenhaus, Marshall Hall, Jr. The contents of
Chapters 8, 10, 11, 12 were mostly still unpublished at the time of
the lectures; those of Chapters 8 and 12 have recently appeared. All
through the lectures I have drawn attention to the numerous problems
that still defy our efforts at solution. The Theory of Groups is still
very much alive today.

This course was delivered during the monsoon term, 1959, and extended
over 36 lectures. I enjoyed every one of them. I am profoundly
grateful to Professor K. Chandrasekharan for inviting me to spend this
term at the Tata Institute of Fundamental Research. I also wish to
record my gratitude to Mr. Pavman Murty, who took the notes and
prepared them for circulation.
\bigskip


\begin{flushright}
{\large\bf B.H. Neumann.}

The University,\\
MANCHESTER, 13,\\
ENGLAND.\\
December, 1959.
\end{flushright}
