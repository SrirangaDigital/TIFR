
\chapter{Embedding of Nilpotent and Soluble Groups}\label{chap11}

\section {}\label{chap11:sec1}%1

Let\pageoriginale $G$ be a group and $A \subset G,B \subset G$ be any two subsets of
$G$. We define the \textit{ commutator subgroup} $[A,B]$ of these
subsets as 
$$
[A,B]=gp\left(\bigg \{ [a,b]\big | a \in A, b \in B \bigg \} \right).
$$

In particular $[G,G]$ is the \textit{derived group} of $G$. A normal
series of the form 
$$
G=G_0 \geq G_1 \geq G_2 \geq \cdots
$$
is called a \textit {descending central series} if 
\begin{gather*}
  G_i \Delta G, i=1,2, \ldots, \text{ and }\\
  G_i/G_{i+1} \leq \text{ centre } (G_i/ G_{i+1}), i=0,1,2, \ldots.
\end{gather*}

It is immediate that
$$
G_i/ G_{i+1} \leq \text{ centre }(G/ G_{i+1})
$$
if and only if 
$$
[G,G_i]\leq G_{i+1}.
$$

In\pageoriginale general a descending central series may not become stationary in a
finite number of steps. We call a group $G$ \textit{ nilpotent of
  classes} $n$ if $G$ has a descending central series with 
$$
G_n= \bigg\{ 1\bigg \}.
$$

The normal series
$$
\displaylines{\hfill 
  G=G_0 \geq G_1 \geq G_2 \geq \cdots\hfill \cr
  \text{where} \hfill 
  G_{i+1}=[G_i,G], i=0,1,2, \ldots\hfill }
$$
is called the \textit{lower central series.} One can show that the
terms of the lower central series are verbal subgroups of $G$ and
hence fully invariant in $G$. A group $G$ is nilpotent of class $n$ if
and only if the $n^{\text{th}}$ term in the lower central series is the trivial
group. Further if $n$ is the least integer such that the $n^{\text{th}}$ term of
the lower central series is the trivial group, then $G$ is nilpotent
of class $n$ but not of class $n-1$. 

A series of the form
$$
\{ 1\} =H_\circ \leq H_1 \leq H_2 \cdots
$$
is called an \textit{ascending central series series} if 
\begin{gather*}
  H_i \Delta G, i=1,2, \ldots,  \text{ and }\\
  H_{i+1}/H_i \leq \text{ centre }(G/H_i)
\end{gather*}
or\pageoriginale equivalently if 
$$
\bigg[G,H_{i+1}\bigg]\leq H_i.
$$

Obviously, in a nilpotent group there is an ascending central series
terminating in $G$ in a finite number of steps. The ascending central
series 
$$
\displaylines{\hfill 
  \{ 1\}=H_0 \leq H_1 \leq H_2 \leq \cdots\hfill \cr
  \text{with}\hfill 
  H_{i+1}/H_i = ~\text{centre}~ (G/H_i)\hspace{2.3cm}\hfill }
$$
is called the \textit{upper central series} of $G$. In general, the
upper central series does not become stationary in a finite number of
steps and even if it does it may not end in $G$. But the upper central
series of  a nilpotent group reaches $G$ in a finite number of
steps. Further the upper and the lower central series of a nilpotent
group (broken off as soon as the former has reached $G$ and the
latter 1) have the same length. 

Obviously every nilpotent groups is soluble and the length of
solubility does not exceed the class of nilpotency. It\pageoriginale is not
difficult to prove that 
\setcounter{theorem}{0}
\begin{theorem}\label{chap11:sec1:thm1} %Thm 1
  A group of order $p^n$ where $p$ is a prime and $n>1$ is nilpotent
  of class $n-1$. 
\end{theorem}

The proofs of the above statements including Theorem $1$ are straight
forward (see eg. Krosh $(1956)$, Chapter XV,p.211.) 

\section{}\label{chap11:sec2} %sec 2

We have seen in the last chapter that an amalgam of two abelian groups
(i.e. nilpotent groups of class $1$) is embeddable in and abelian
group. We now ask: 

Can every amalgam of two nilpotent (soluble) groups be embedded in a
nilpotent (soluble) group? 

In general, the answer to this question is 'no'. In fact, there is an
amalgam of an abelian group $A$ and a nilpotent group $B$ of class
$c=2$ which cannot be embedded in any nilpotent group. (J.Wiegold,
1959) 

The following example shows that an amalgam of two nilpotent groups
need not even be embeddable in a soluble group. 

Let
\begin{alignat*}{6}
  K&= gp(g,h;\quad  &g^5&=h^5=1,\quad  &[g,h]&=1),\\
  A &=gp(H,a;       &g^a&=gh, h^a=h, &a^5&=1),\\
  B&= gp(H,b; &g^b&=g, h^b=g^{-1}h, &b^5&=1).
\end{alignat*}

Clearly,\pageoriginale
$$
|H|=5^2,|A|=|B|=5^3.
$$

By Theorem \ref{chap11:sec1:thm1}, $A$ and $B$ are nilpotent groups of class
2. Consider the amalgam of the groups $A$ and $B$ with the
amalgamated subgroup $H$. From the definition of $A$ and $B$ we
readily confirm that 
$$
H \Delta A, H \Delta B.
$$

We now prove that the amalgam of $A$ and $B$ with the amalgamated
subgroup $H$ is not embeddable in any soluble group. Let $G$ be a
group embedding the amalgam and 
$$
p=gp(A,B)\leq G.
$$

We first note that
$$
H \Delta P.
$$

Let $\Gamma$ be the group of all automorphisms of $H$ induced by the
inner automorphisms of $P$. It is well known that 
$$
\Gamma \cong P/N,
$$
where $N$ is the centralizer of $H$ in $P$. (The set $N$ of all
elements in $P$ which commute with every element of $H$ is group; the
group $N$ is called the \textit{centralizer} of $H$ in $P$. Since $H$
is normal in $P$, one easily verifies that $N \Delta P$.) 

Now,\pageoriginale
$$
\Gamma =gp(\alpha, \beta),
$$
where $\alpha, \beta$ are the automorphisms of $H$ induced by the
inner automorphisms $\varphi_a,\varphi_b$ of $P$ given by 
$$
\displaylines{\hfill 
  x^{\varphi_a}a=a^{-1}xa, \text{ for every } x \in P\hfill \cr
  \text{and}\hfill  
  x^{\varphi_b}=b^{-1}xb, \text{ for every } x \in P.\hfill }
$$

The group $H$ being abelian and of exponent 5 can be considered as
a vector space over the prime field $GF(5)$ of characteristic 5. In
fact $H$ becomes a two dimensional vector space over $GF(5)$ with
$(g,h)$ as a basis. Thus the endomorphisms ring of $H$ is the ring of
all $2 \times 2$ matrices over $GF(5)$. Let us now take the matrix
representations of the automorphisms $\alpha, \beta$ of $H$. Now
writing the operations of $H$ additively, we have 
\begin{align*}
  g^\alpha &=g^2=g+h\\
  h^\alpha &=h^a=h.
\end{align*}

Thus $\alpha$ corresponds to the matrix
$$
\begin{pmatrix}
  1&1\\0&1
\end{pmatrix}
$$
Similarly\pageoriginale it is easy to see that $\beta$ corresponds to the matrix
$$
\begin{pmatrix}
  1&0\\-1&1
\end{pmatrix}
$$

The multiplicative group $M$ generated by the matrices $\alpha, \beta$
is precisely the group of all $2 \times 2$ matrices with determinant
$1$ over $GF(5)$. The group $\Gamma$ is isomorphic to this group
$M$. We identify $\Gamma$ with $M$. The group $M$ is well known and is
called the binary icosahedral group (see Coxeter and Moser,
1957, p.69). The binary icosahedral group has order 120. Its centre
is cyclic of order 2, and the factor group of the centre is the
icosahedral group (or alternating group $A_5$ of degree 5). Thus $M$
is not soluble. This prove that $P$ and therefore $G$ is not
soluble. Thus the amalgam of two nilpotent group of class 2 need not
even be embeddable in a soluble group. 

\section{}\label{chap11:sec3} %Sec 3

In this section we shall impose some conditions on  the amalgamated
subgroup $H$ to achieve a 'good' embedding of the amalgam of nilpotent
or soluble groups. 

\begin{theorem}\label{chap11:sec3:thm2} %Thm 2.
  Let $A,B$ be two nilpotent groups of class $c$ (soluble groups of
  length $\ell$). The amalgam of $A$ and $B$ with an amalgam subgroup
  $H$ can be embedded in a nilpotent group of class $c$ (soluble group
  of length $\ell$) if $H$ is central both in\pageoriginale $A$ and in $B$. 
\end{theorem}

\begin{proof}
  Let $P$ be the generalised direct product of the amalgam. Then
  $$
  P=A \times B/N,N \Delta A \times B.
  $$

  Now the direct product of two groups (and indeed the Cartesian
  product of an arbitrary family of groups) that are nilpotent of
  class $c$ (soluble of length $\ell$) is itself a nilpotent of class
  $c$ (soluble of length $\ell$)> In fact the nilpotent groups of
  class $c$, and also the soluble groups of length $\ell$, form a
  variety. [For soluble groups, of cf. Chapter $7$, for nilpotent
    groups we omit the proof]. It follows that $A \times B$, and then
  also $P$, is nilpotent of class $c$ (soluble length $\ell$). 
\end{proof}

If $H$ is central in $A$ but not necessarily central in $B$ then
Wiegold's example (see Section \ref{chap11:sec2}) shows that we cannot in general
hope for an embedding in a nilpotent group. But in the case of
solubility the situation is different as is shows by the following
theorem. 

\begin{theorem}\label{chap11:sec3:thm3}% Thm 3
  If $A$ is soluble of length $\ell,B$ soluble of length $m$ and if
  $H$ is central in $A$ then the permutational product $P$ of the
  amalgam (irrespective of the transversal chosen) is soluble of
  length $n \leq \ell+ m-1$. 
\end{theorem}

\begin{proof}
  Let $S,T$ be transversals of $H$ in $A$ and $B$ respectively and
  $K=S \times T \times H$. Let $P$ be the permutational product of the
  amalgam\pageoriginale corresponding of the transversals $S,T$. For every $f \in
  B^S$, we define a mapping $\gamma (f)$ of $K$, called a {\em
    quasi-multiplication}, as follows: 
  $$
  (s,t,h)^{\gamma(f)}=(s,t,h)^{\rho(f(s))},(s,t,h)\in K.
  $$

  In other words, $\gamma (f)$ coincides with $\rho (f(s))$ on all
  those elements of $K$ whose first coordinate is $s$. Thus  
  \begin{gather*}
    (s,t,h)^{\gamma(f)}(s^*,t^*,h^*), \text{ where}\\
    s^*=s,t^*h^*=thf(s).
  \end{gather*}

  Consider the mapping $\eta$ of the Cartesian power $B^S$ into, and
  in fact onto, the set $\Gamma$ of all quasi-multiplications, $\eta$
  being defined by 
  $$
  f^ \eta= \gamma(f),f \in  B^S.
  $$

  First we prove that
  $$
  \gamma (fg)=\gamma (f) \gamma(g), \text{ for } f,g \in  B^S.
  $$

  Let $(s,t,h)$ be an arbitrary element of $K$. Then
  \begin{align*}
    (s,t,h)^{\gamma(fg)} &=(s,t,h)^{\rho(fg(s))}= (s,t,h)^{\rho(f(s)).g(s))}
    &= (s,t,h)^{\rho (f(s)) \rho(g(s))}. \\
    &= \bigg ( (s,t,h)^{\gamma(f)} \bigg) ^{\rho(g(s))}.
  \end{align*}
\end{proof}

Now\pageoriginale since $s$ is not altered after applying $\gamma(f)$, we have
$$
\bigg ( (s,t,h)^{\gamma(f)} \bigg) ^{\rho(g(s))}=\bigg (
(s,t,h)^{\gamma(f)} \bigg) ^{\gamma(g)}= (s,t,h)^{\gamma(f)\rho(g(s))\circ
  \gamma(g)}. 
$$

Therefore
$$
\gamma(fg)=\gamma(f) \gamma(g).
$$

This proves that $\eta$ is a homomorphism, and the homomorphic image
is a group. In particular, this proves that the quasi-multiplications
are permutations on the set $K$. Now if 
\begin{align*}
  \gamma(f) &= L, \text{ then }\\
  (s,t,h)^{\gamma(f)} &=(s,t,h) \text{ for every } (s,t,h) \in  K;
\end{align*}
that is 
\begin{align*}
  thf(s)& = th, \text{ for every } s \in  S; i.e.,\\
  f(s)& = 1, \text{ for every } s \in  S.
\end{align*}

Thus\pageoriginale the kernel of $\eta$ is trivial; that is, $\eta$ is an
isomorphism. Thus 
$$
B^S \cong \Gamma.
$$

Further,
$$
\rho (B)\leq \Gamma
$$

For,
\begin{align*}
  \rho (b)& =\gamma(f_b), b \in  B, \text{ where }\\
  &\quad f_b \in  B^S \text{ is such that}\\
  f_b(s)& = b \text{ for every }s \in  S;
\end{align*}
in other words $f_b$ is in the diagonal of $B^S$.

Consider the group
$$
\Delta =\Gamma \cap P.
$$

We claim that
$$
\Delta \Delta P.
$$

Let $a \in  A, \gamma (f) \in  \Gamma, f \in 
B^S$  and  
$$
\rho (a^{-1}) \gamma (f) \rho (a) = \rho (a)^{-1} \gamma (f) \rho (a)
= \gamma'. 
$$

Let\pageoriginale $(s,t,h) \in  K$ and 
$$
sa^{-1}= \bar{s} \bar{h}, \bar{s} \in  S, \bar{h} \in 
H. 
$$

Thus $\bar{s}$, $\bar{h}$ are completely determined by $a$ and
$s$. Then  
$$
(s, t, h)^{\rho(a^{-1})} =(\bar{s}, t, h \bar{h});
$$
for, $H$ being central in $A$, we have 
$$
sha^{-1} = sa^{-1} = \overline{s} \overline{h}h.
$$

Now 
\begin{gather*}
  (\bar{s}, t, h\bar{h})^{\gamma (f)}= (\bar{s},t,h\bar{h})^{\rho
  (f(\beta\bar{s}))} = (\bar{s}, t_1, h_1), \text{ where } \\ 
  t_1 h_1 = th \bar{h}f(s). 
\end{gather*}

Finally we have,
\begin{gather*}
  (\bar{s},  t_1,  h_1)^{\rho (a)} = (s_1, t_1, \bar{h}_1),  \text{ where } \\
  s_1 \bar{h}_1 =\bar{s} h_1 a.
\end{gather*}

Now,  again since $H \le $ centre  $(A)$, we have 
$$
s_1 \bar{h}_1 =\bar{s}h_1 a = (\bar{s} a)h_1 = s({\overline{h}}^{-1} h_1)
$$

Therefore,\pageoriginale
\begin{gather*}
  s= \bar{s} ; \text{ and } \\
  \bar{h}_1 =  \overline{h}^{-1}h_1.
\end{gather*}

Further
\begin{align*}
  t_1 \bar{h}_1 &= t_1 \overline{h}^{-1}h_1
  &= (t_1 h_1) \overline{h}^{-1} = (th\bar{h}f (\bar{s})) \overline{h}^{-1} \\
  &= th(\bar{h}f(\bar{s}) \overline{h}^{-1})
\end{align*}

Now, since $\bar{s},\bar{h}$ are completely determined by $s$ and $a$,
the function $f'$ defined by  
$$
f'(s)= \bar{h}f(\bar{s}) \overline{h}^{-1}
$$ 
is well defined and is in $B^S$. We have 
\begin{gather*}
  (s,t,h) ^{\rho (a^{-1}) \gamma (f) \rho (a)} =  (s_1, t_1, \bar{h}_1)
  ; \text{ and } \\ 
  s_1 = s, t_1 \bar{h}_1 = thf' (s). 
\end{gather*}

Therefore\pageoriginale 
$$
\rho(a^{-1}) \gamma (f) \rho (a) = \gamma (f') \in  \Gamma.
$$

It is now immediate that 
$$
\rho(a^{-1}) \Delta \rho (a) = \Delta. 
$$

Since $\rho (B) \le \Delta$, we have 
$$
\rho (b^{-1}) \Delta (b) = \Delta, b \in  B.
$$

Hence,
$$
\Delta \Delta P.
$$

Further,
$$
P/ _Delta \cong \rho (A) / \rho (A) \cap \Delta
$$

Now $\rho (A) / \rho (A) \cap \Delta$ is soluble of length $\ell$ and
$\Delta \le B^s$ is soluble of length $m$. Therefore $P$ is soluble of
length $\ell +m$. This almost proves Theorem \ref{chap11:sec3:thm3}; but we still want
to improve the bound for the soluble length of $P$. Consider now, 
$$
\rho(a^{-1}) \rho (b) \rho (a) \in  \Gamma.
$$

By what we have proved, it follows that 
$$
\rho(a^{-1}) \rho (b) \rho (a) = \gamma (f')
$$
where\pageoriginale $f' \in  B^S$ is defined by 
\begin{gather*}
  f'(s) = \bar{h} b \overline{h}^{-1} \text{ where }\\
  s \overline{a}^{-1} = \overline{sh}
\end{gather*}

Define $g \in  B^S$ by
\begin{align*}
  g(s) & = \bar{h},  s \in  S \text{ and }\\
  & \quad f_b \in  B^s \text{ by }\\
  f_b(s) & = b \text{ for every} s \in  S.
\end{align*}

Then,
$$
\rho(a^{-1}) \rho (b) \rho (a) = \gamma (gf_b g^{-1}).
$$

Therefore,
\begin{align*}
[ \rho (b), \rho(a)] & = \rho(b^{-1}) \rho(a^{-1}) \rho(b) \rho(a) =
\gamma (f^{-1}_{b} g {f_b} g^{-1})\\ 
& = \gamma [f_b, g^{-1}] \in \Gamma' ,
\text{ for all } a \in  A, b \in  B. 
\end{align*}

Therefore,
$$
[\rho (A), \rho (B)] \le \Gamma'. 
$$

It is not difficult to show that if a group $G$ is generated by its
subgroups\pageoriginale $G_1, G_2$ then its derived group is  
$$
G'= G'_1 G'_2 [G_1, G_2];
$$
hence   
\begin{gather*}
  P' =[\rho (A), \rho (A)]~ [\rho (B), \rho (B)]~ [\rho (A), \rho (B)] \\ 
    \le \rho (A') \Gamma'.  
\end{gather*}

Again,
$$
P'' = \rho (A'') \Gamma '' [\rho (A'), \Gamma '] \le \rho (A '') \Gamma '. 
$$

Continuing in this fashion we arrive at  $P^{(\ell)} \leq \rho
(A^{(\ell)}{\Gamma'}$, 
where $P^{(\ell)}$, $A(\ell)$ denote the $\ell$th derived groups of
$P$ and $A$ respectively. Now  since $A$ is soluble of length  $\ell$,
we have  
$$
A^{(\ell)}= \{ 1\}.
$$

Hence,
$$
P^{(\ell)} \le \Gamma '. 
$$

Therefore,
$$
P^{(\ell + m-1)} \le  (\Gamma ')^{(m-1)} = \Gamma^{(m)} = \{ 1 \},
$$
as\pageoriginale $\Gamma \cong B^s$ is soluble of length $m$.  Therefore te group
$P$ is soluble of length $\ell + m-1$. 

This proves our assertion.
