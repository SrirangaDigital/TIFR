
\chapter{The Problems of Heinz Hopf}\label{chap12} % chapter XII 

\section{}\label{chap12:sec1} % \sec 1

More\pageoriginale than  twenty five years ago, Heinz Hopf formulated
the following 
two problems which are closely related. These problems arose out of a
topological problem, which we do not formulate here  (cf. B.H. Neumann,
1953). 

\noindent\textbf{First Hopf Problem}. Can a finitely generated group be
isomorphic to one of its proper factor groups? 

\noindent\textbf{Second Hopf Problem}. If $G$ is a finitely generated group and
$H$ an epimorphic image of $G$, and $G$ an epimorphic image of $H$,
are $G$ and $H$ necessarily isomorphic? 

We now take following definition
\begin{defi*}
  $A$ group $G$ is a \textit{Hopf group} if $G$ is not isomorphic to
  ant of its proper factor groups. 
\end{defi*}

In virtue of this definition, the First Hopf Problem can be
reformulated as follows: 

Is every finitely generated group  Hopf group?

There are examples of non-finitely generated groups which are not Hopf
group. For instance, one can easily verify  that the direct power or
the cartesian  power of any group $G \neq 1$ over any infinite index
set $I (e,g. I= \{ 1,2,3, \cdots\})$ is not a Hopf group. The Prufer
group $Z (p^{\infty})$ (see Ch.8, Section \ref{chap8:sec2} Corollary 3) is\pageoriginale
also a non-Hopf group. 

A negative answer to the second problem implies an affirmative answer
to the first. In other words the existence of two non-isomorphic
finitely generated groups which are epimorphic images of each other
implies  the existence of a finitely generated non-Hopf group. For let
$G$ be a finitely generated group and $H$ any group and let $\theta$
and $\psi$ be endomorphisms of $G$ onto $H$ and $H$ onto $G$
respectively. Then the composite map $\theta \psi$ is an epimorphism
of $G$ onto $G$. Now if $\psi$ has a non-trivial kernel  then $\theta
\psi$ also has a not-trivial kernel. Let $N$ be the kernal of $\theta
\psi$. Then it follows that  
$$
G \cong G/N, 
$$   
that is, $G$ is not a Hopf group. On the other hand, the existence of
a finitely generated non-Hopf group does not by itself solve the
second Hopf Problem. 

It is known that all finitely generated free groups are Hopf groups
(Magnus ($1935$); see also Kurosh,  $(1956) 39, p. 59)$. Magnus
$(1935)$ also proved that the finitely generated reduced free groups
of the variety of nilpotent groups of class $c$ are Hopf
groups. Reinhold Baer made an example of finitely generated non Hopf
group. Though he later withdrew this as containing a mistake, it
suggested the possibility of finding such a group. B.H. Neumann
$(1950)$ thereupon constructed a $2$ generator non-Hopf group; this
has an\pageoriginale infinite number of defining relations. Graham Higman $(1951)$
constructed a finitely related $3$-generator non-Hopf group. Using the
group of Graham Higman $(1951)$, B.H. Neumann $(1953)$ gave an example
of $3$-generator finitely related groups $G$ and $H$ which are
epimorphic images of each other, but are not isomorphic. Thus the
first and the second Hopf Problems have been solved
now. 

Let $G$ be a non-Hopf group, Then there exists a non-trivial normal
subgroup $N$ of $G$ such that  
$$
G \cong G/N. 
$$ 

Let $\varphi$ denote the isomorphism of $G/N$ onto $G$, and $\theta$
the canonical epimorphism of $G$ onto $G/N$. The mapping 
$$
\psi = \theta \varphi 
$$ 
is an epimorphism of $G$ onto $G$. Let $N_1$ be the kernal of the
mapping $\psi$. Then  
$$
N_1 = \{ 1\}^{\psi -1} =  \bigg( \{ 1\}^{\varphi ^{-1}}\bigg)^{\theta
  -1}  = \{ 1\}^{\theta ^{-1}} =N. 
$$

Consider now the epimorphism $\psi^2$ of $G$ onto $G$. By an easy
application  of well-known isomorphism theorems one finds that the
kernal $N_2$ of $\psi^2$ is such that 
$$
N_1 <N_2, N_2 /N_1 \cong N.
$$

More\pageoriginale generally, if $N_r$ is the kernel of the epimorphism $\psi$, we have 
$$
N_{r-1} < N_r, N_r/ N_{r-1} \cong N.
$$

Thus,
$$
N_1 <  N_2 < N_3 < \cdots 
$$
is a strictly  ascending of normal subgroups. As this is not possible
is a group satisfying the maximal condition for normal subgroups, we
have  
\setcounter{theorem}{0}
\begin{theorem}\label{chap12:sec1:thm1}%the 1
  A group satisfying the maximal condition for normal subgroups is a
  Hopf group.  
\end{theorem}

We know that a finitely generated nilpotent group satisfies the
maximal condition for subgroups. (See Kurosh, 1956, Ch. XV,
p. 232) Hence we have: 
\setcounter{corollary}{0}
\begin{corollary}\label{chap12:sec1:coro1}%cor 1
  A finitely generated nilpotent group is a Hopf group.
\end{corollary}

Again, by a theorem of $P$. Hall $(1954^b)$ already quoted (in Chapter
8, p. 141) a finitely generated metabelian group satisfies the
maximal condition for normal subgroups. Thus we have: 
\begin{corollary}\label{chap12:sec1:coro2}%coro 2
  A finitely generated metabelian group is a Hopf group.
\end{corollary}

This is the best possible result so far as soluble length of soluble
Hopf groups is concerned; we shall later make an example of a finitely
generated non-Hopf group which is soluble of length 3 (see section
\ref{chap12:sec4} of this Chapter). 

\section{}\label{chap12:sec2}%sec 2

\setcounter{defn}{0}
\begin{defn}\label{chap12:sec2:def1} % def 1
  A\pageoriginale subgroup $G$ of a group $H$ is an \em{E-subgroup} of $H$ if for
  every normal subgroup $R$ of $G$, there exist a normal subgroup $S$
  of $H$ such that  
  $$
  S \cap G=R. 
  $$
\end{defn}

The above definition is equivalent to the following:
\begin{defn}\label{chap12:sec2:def2} % def 2
  $A$ subgroup $G$ of a group $H$ is an $E$-subgroup of $H$ if for
  every normal subgroup $R$ of $G$, we have  
  $$
  R^H \cap G=R,
  $$
  where $R^H$ is the normal closure of $R$ in $H$. It is clear that if
  $R^H \cap G=R$, then we can take $R^H$ as the $S$ of Definition
  (\ref{chap12:sec2:def1}); conversely, if there is a normal subgroup $S$ of $H$ such
  that $S \cap G=R$, then $R \le S$, hence $R^H \le S^H =S$, and  
  $$
  R \le R^H \cap G \le S \cap G = R ;
  $$
  thus also $R^H \cap G=R$. We give yet another equivalent definition
  of an E-subgroup: 
\end{defn} 

\begin{defi*}\label{chap12:sec2:def3} %3
  $A$ subgroup $G$ is an $E$-subgroup of $H$ if every epimorphism
  $\theta$ of $G$ onto a group $G_1 = G^{\theta}$ can be extended to
  an epimorphism $\theta^*$ of $H$ onto a group $H_1$ containing
  $G_1$. 
\end{defi*} 
 
Let $G \le H$ satisfy the conditions of Definition (\ref{chap12:sec2:def2}).
 
 Let $\theta$ be any epimorphism of $G$ onto a group $G_1$ and $R$ be
 its kernel.\pageoriginale Then  
 $$
 R= \{ 1\}^{\theta^{-1}} \Delta G.
 $$
 
 Therefore
 $$
 R^H \cap G = R.
 $$
 
 Now
 $$
 GR^H / R^H \cong G/R^H \cap G= G/R \cong G_1. 
 $$ 
 
Let $\theta^*$ be the natural map of $H$ onto $H/R^H$. By identifying
$G_1$ with $GR^H/H$ canonically $\theta^*$ becomes an extension
$\theta$. 
 
Conversely, assume the conditions of Definition (\ref{chap12:sec2:def3}). Let $R \Delta
G$, and let $\theta$ be the canonic epimorphism of $G$ onto $G_1 =
G/R$. Extend $\theta$ to an epimorphism $\theta^*$ of  $H$ onto a
group $H_1$ containing $G_1$, and let the kernel of $\theta^*$ be
$S$. As $R^{\theta^*}= R^{\theta}$ is trivial, $R \le S \cap G$. Now
if $s \in  S \cap G$ then 
$$
1=s^{\theta^*}= s^{\theta}, 
$$ 
and thus $s \in  R$. It follows that $S \cap G \le R$, and hence 
$$
S \cap G =R. 
$$
 
This is the condition of Definition (\ref{chap12:sec2:def1}), which we already know to be
equivalent to Definition (\ref{chap12:sec2:def2}). Thus all the three definitions are
equivalent. 
 
If\pageoriginale $H$ is the direct product of two groups $F$ and $G$ then $F$ and
$G$ are E-subgroups of $H$. More generally, if $H$ is the direct
product or the Cartesian product of a family of groups, say
$\{G_i\}_{i \in   I}$, then each factor $G_i$ is an E-subgroup
of $H$. If $H$ is any group and $Z(H)$ its center, then any subgroups
of $Z(H)$ is an E-subgroup of $H$. This follows from the fact that
every subgroup of $Z(H)$ is a normal subgroup of $H$. Further,  if $H$ is a simple group then a proper non-trivial
subgroup of $H$ is not an $E$-subgroup of $H$. We now prove the
following: 
\begin{theorem}\label{chap12:sec2:thm2} % \thm 2
  The relation ``$E$- subgroup of'' is transitive; in other words, if
  $G$ is an $E$-subgroup of $H$, and $H$ an $E$-subgroup of $K$, then
  $G$ is an $E$-subgroup of $K$. 
\end{theorem} 

\begin{proof}
  Let,
  $$
  R \Delta G.
  $$
\end{proof}

 Then since $G$ is an $E$-subgroup $H$, there is an $S \le  H$ such that 
 $$
 S \Delta H, S \cap G =R.
 $$
 
 Now since $H$ is an E-subgroup of $K$, there is a $T \le  K$ such that
 $$
 T \Delta K, T \cap H=S.
 $$
 
 We have 
 $$
 G \cap T = G \cap H \cap T = G \cap S= R.
 $$
 
 This\pageoriginale proves that $G$ is an $E$-subgroup of $K$.

 \section{}\label{chap12:sec3} % \sec 3
 
 Let $A,B$ be any two groups. Let 
 $$
 P= A W r B.
 $$
 
 We shall now prove that the coordinate subgroups $A_b \le A^B, b
 \in  B$ (that is, 
 $$
 A_b = \left\{ f \bigg | f  \in  A^B, f(y) =1,  \text{ for all
 } y \neq b \right\}) 
 $$
 and the diagonal $A^\Delta \le A^B$ are $E$-subgroups of $P$.
 
 Let $\varphi $ be any epimorphism of $A$ onto a group  $A_o$. Let 
 $$
 P_o = A_o W r B.
 $$
 
 Consider the mapping $\varphi^*$ of $P$ onto $P_o$ defined as
 follows. For every $p =bf \in  P$, with $b \in  B, f
 \in  A^B$, 
 \begin{gather*}
   p^{\varphi^*}= (bf)^{\varphi^*} = bf_o, \text{ where } f_o
   \in  A^B_o \text{ and } \\ 
   f_o (y) =(f(y))^\varphi,  y \in  B.
 \end{gather*} 
 
 We claim  that $\varphi^*$ is an epimorphism of $P$ onto $P_o$. Let
 $p= bf, p' =b' f' \in  P, b,b' \in  B, f, f'
 \in  A^B$. 
 
 Then\pageoriginale 
 \begin{align*}
   p^{\varphi^*} & = b f_o, p'^{\varphi^{*}} =  b'  f'_o, ~\text{ where }\\
   f_o(y) & =(f(y))^\varphi, y \in  B, , ~\text{ and }\\
   f'_o (y)& = (f' (y))^\varphi, y \in  B.
 \end{align*}
 
 Now
 $$
 \displaylines{\hfill 
   pp' =(bf)(b' f ' ) = bb'. f^{b'} f'; \qquad \text{ and } \hfill \cr
   \text{therefore,}\hfill  
   (pp')^{\varphi^*}= bb'. h,  \text{ where } h \in  A^B_o\hfill \cr
   \text{and} \hfill  
   h(y) = (f^{b'}f' (y))^{\varphi}\hfill \cr
   \hfill = (f^{b'}(y). f' (y))^\varphi = (f^{b'}(y))^\varphi
   (f'(y))^{\varphi},\hfill \cr
   \hfill =(f(yb^{-1}))^\varphi (f'(y))^\varphi,  \text{ for all } y
   \in  B.\hfill} 
$$
 
 On the other hand, 
 $$
 p^{\varphi^*} p'^{\varphi^*}= (bf_o) (b' f'_o) =bb' f^{b'}_o f'_o. 
 $$

Now\pageoriginale 
\begin{align*}
  f^{b'}_\circ f'_\circ (y) &= f^{b'}_\circ (y).  f'_\circ(y)\\
   & = f_\circ (yb'^{-1}) f'_\circ(y) = (f(yb^{-1}))^{\varphi} (f' (y))^{\varphi}.
\end{align*}

Thus,
$$
(pp')^{\varphi *} = p^{\varphi *} p'^{\varphi *}.
$$

This proves that $\varphi^*$ is a homomorphism. It is easy to see that
it maps $P$ onto $P_0$ ; hence it is an epimorphism, as claimed. 

Let $\theta$ be an epimorphism of $A_b$(or $A^{\Delta})$ onto a group
$A_0$ and $\psi$ be isomorphism of $A$ onto $A_b$ (or
$A^{\Delta}$). Then the epimorphism  
$$
\varphi = \psi \theta
$$
of $A$ onto $A_0$ gives rise to a mapping $\varphi^*$ of $P$ onto
$P_0$. If the group $A_0$ is identified with $A_{ob}$(or
$A^{\Delta}_0)$, it follows without difficulty that $\varphi^*$ is an
extension of $\theta$. This proves: 
\setcounter{Lemma}{0}
\begin{Lemma}\label{chap12:sec3:lem1}% lem 1
  In a wreath product the coordinate subgroups and the diagonal
  subgroup are $E$-subgroups. 
\end{Lemma}

In Chapter \ref{chap8} we proved that a countable group $G$ can be embedded in
a 2-generator group $H$. We shall now prove that the embedding\pageoriginale
procedure given there embeds $G$ as an $E$-subgroup of $H$. In the rest
of this Chapter we shall use the notation of Chapter \ref{chap8}. 

Let us briefly recall the embedding procedure of Chapter \ref{chap8}. 

We started with a  countable group $G$ where 
\begin{align*}
  G & = gp(\big\{ a_i \big\}_{i \in  I}) \text{ and }\\
  I & = \big\{ 1,2,3, \ldots \big\}.
\end{align*}

We then formed the wreath product 
\begin{align*}
  P& = GW r C, \text{ where }\\
  C& = gp(c) ;
\end{align*}
and we embedded $G$ as the diagonal subgroup $G^{\Delta}$,
$$
G^{\Delta} \leq G \leq P.
$$

We then formed the wreath product 
$$
Q = PWr B,
$$
where $B$ was any group containing elements $b_i, i \in  I$,
with the property, 
$$
b_i \neq 1, b_i \neq b_j,  b_i b_j \neq 1, b_i b_j \neq b_k.
$$

Then\pageoriginale we realised $G$ as a subgroup $G^*$ of 
$$
H = gp(q,B),  q \in  P^B.
$$

In fact 
\begin{align*}
  G^* & = gp(\big\{ h_i \big \}_{i \in  I}), \text{ where }\\
  h_i & [q^{b_i}, q ] \in  P^B.
\end{align*}

(For the definitions of $q$ and$h_i$, see Chapter \ref{chap8}.)

Now be Lemma \ref{chap12:sec3:lem1}, $G^{\Delta}$ is E-subgroup of $P$. Further, $G^*$ is
a subgroup of the coordinate subgroup $P_1 \leq Q$, where  
$$
P_1 = \Bigg\{f \Bigg| f \in  P^B, f(y) = 1 \text{ for all } y
\neq 1, y \in  B \Bigg\} 
$$
and $G$ is mapped onto $G^*$ under the natural isomorphism of $P$ onto
$P_1$. Therefore 
$$
G^* \text{ is an E-subgroup of } P_1.
$$

Again by Lemma \ref{chap12:sec3:lem1}, $P_1$ is an $E$-subgroup of $Q$. Hence by the
transitivity property of $E$-subgroups, $G^*$ is an $E$-subgroup of
$Q$. Now, since  
$$
G^* \leq H \leq Q,
$$
it suffices for our purpose to show the following simple lemma.

\begin{Lemma}\label{chap12:sec3:lem2}%2
  If\pageoriginale $G \leq H \leq K$, and if $G$ is an $E$-subgroup of
  $K$, then $G$ is an $E$-subgroup of $H$. 
\end{Lemma}

For if $R \Delta G$, there is a subgroup $T \Delta K$ with 
$$
T \cap G =R;
$$
put $S = T \cap H$: then $S \Delta H$and 
$$
S \cap G = T \cap H \cap G = T \cap G = R.
$$

Applying this lemma to $G* \leq H \leq Q$, we obtain the stated result:
\begin{corollary}
  $G^*$ is an $E$-subgroup of $H$.
\end{corollary} 
 
 Let us now take $G$ to be the free group of countably infinite rank
 presented by  
 \begin{align*}
   G& = gp( \big\{ a_i \big\}_{i \in  I}; \phi), \text{ where }\\
   I & = \big\{ \ldots 2,-1,0,1, \ldots \big \}.
 \end{align*}
 
 Take $B$ to be the infinite cyclic group 
 \begin{align*}
   B & = gp(b), \text{ and }\\
   b_i & = b^{3i -1}
 \end{align*}
 
 Then 
 $$
 h_i \Bigg[q^{b^{3i -1}}, q \Bigg] = \Bigg[b^{1-3i}, qb^{3i -1}, q \Bigg].
 $$ 
 
 Identifying\pageoriginale the group $G$ with 
 $$
 \displaylines{\hfill 
   G* = gp( \bigg\{ h_i \bigg\}_{i \in  I})\hfill \cr
   \text{we have}\hfill  G \leq H = gp(q, b).\hfill }
 $$
 
 Let now $F$ be the free group
 $$
 F = gp(s,t;\phi).
 $$
 
 Define $E \leq F$ as 
 \begin{align*}
   E &  = gp( \bigg\{ e_i \bigg\}_{i \in  I}), \text{ where }\\
   e_i & = \Bigg[s^{ 1-3i }, ts^{3i -1}, t \Bigg], i \in  I.
 \end{align*}
 
 We prove:
 \begin{theorem}\label{chap12:sec3:thm3}%3
   The subgroup $E$ is an $E$-subgroup of $F$.
 \end{theorem} 

 \begin{proof}
   Let $\theta$ be the epimorphism of $F$ onto $H$ defined by 
   $$
   s^{\theta} = b, t^{\theta} = q.
   $$

   Then we have 
   \begin{gather*}
     e^{\theta}_i  = h_i, i \in  I, \text{ and }\\
     E^{\theta} = G.
   \end{gather*}
 \end{proof}
 
 Since\pageoriginale $G$ is freely generated by $\bigg\{ h_i \bigg\}_{i \in 
   I}$, it follows that $E$ is also freely generated by $\bigg\{ e_i
 \bigg\}_{i \in  I}$. Hence the restriction of $\theta$ to $E$
 is an isomorphism. Let  
 $$
 R \Delta E.
 $$
 
 Then 
 $$
 R^{\theta} = R_0 \Delta G.
 $$
 
 Now since $G$ is an $E$-subgroup of $H$, there is a $S_0 \Delta H$ such that 
 $$
 G \cap S_0 = R_0.
 $$
 
 Let $S = S_0^{\theta^{-1}}$. Then 
 $$
 S \Delta F.
 $$
 
 We have 
 $$
 (S \cap E)^{\theta} \leq S^{\theta} \cap E^{\theta} = S_0 \cap G =
 R_0 = R^{\theta}. 
 $$
 
 This gives 
 $$
 S \cap E \leq R,
 $$
 as the restriction of $\theta$ to $E$ is one- one. But evidently also
 $R \leq S \cap E$; hence, 
 $$
 S \cap E =R.
 $$
 
This\pageoriginale proves that $E$ is an $E$-subgroup of $F$.

It may be of interest to remark that Theorem \ref{chap12:sec3:thm3} is equivalent to the
embedding theorem proved in Chapter \ref{chap8}. For a countable group $G$ is
an epimorphic image of $E$ by an epimorphism, say $\theta$. Now since
$E$ is an $E$-subgroup of $F, \theta$ can be extended to an epimorphism
$\theta^*$ of $F$. Then  
$$
G \leq F^{\theta^*}, \text{ and }
$$
$F^{\theta^*}$ is generated by $2$ elements. The following is an
unsolved problem in this context. 

\noindent \textbf{Unsolved problem. }
  Is there a free infinite rank in the group 
  $$
  F =gp(s,t; s^p = t^q = 1) ?
  $$

For $p \geq 2, q \geq 6$, the answer (unpublished) to this question is `yes'.

For $p = 2, q =3$, we have the following interesting problem:
\begin{prob*}
  Has the modular group an E-subgroup that is free of infinite rank?
\end{prob*}

\section{Finitely generated soluble non-Hopf group}\label{chap12:sec4}%sce 4

The object of this section is to construct a finitely generated
soluble non-Hopf group. In this section also we shall use\pageoriginale the notation
of Chapter \ref{chap8}. Using the embedding procedure of Chapter \ref{chap8}, we embed
the free abelian group of countably infinite rank into a 3-generator
group $H$ by suitably choosing the group $B$; and then we prove that a
certain factor group of $H$ is a non-Hopf group. 

Let $G$ be the free abelian group of countably infinite rank presented by 
$$
\displaylines{\hfill 
  G = gp(\big\{ a_i \big\}_{i \in  I}; \bigg[a_i, a_j \bigg]= 1,
  i, j, \in  I),\hfill \cr 
  \text{where}\hfill  
  I = \bigg\{\ldots -1,0, 1,2, \ldots \bigg\}.\hspace{2.3cm}\hfill}
$$

As in Chapter \ref{chap8}, we embed $G$ as the diagonal subgroup $G^{\Delta} $
of $G^C$ in  
\begin{align*}
  P & = G \,Wr\, C, \text{ where }\\
  C & = gp(c).
\end{align*}

We now take the group $B$ to be the free abelian group of rank 2
presented by  
$$
B= gp (b, b' ; [b, b']=1)
$$
and form the wreath product 
$$
Q = PW r B.
$$

Choose\pageoriginale the elements $b_i$ (see Chapter 8) as 
$$
b_i = b^i b', i \in  I.
$$

One easily verifies that these $b_i$ satisfy the inequalities:
$$
b_i \neq 1, b_i \neq b_j, b_i b_j \neq 1, b_i b_j \neq b_k.
$$

Therefore, as in Chapter 8, the group $G$ is embedded as the subgroup
$G*$ of $H$, where  
$$
H = gp(q,B) = gp (q,b,b') \leq Q
$$
and $G^* = gp(\big\{ a_i \big\}_{i \in  I}) \leq H$.

We recall that $q \in  P^B$ is defined by 
\begin{align*}
  q(1) & =c,\\
  q(b^{-1}_i) & = g_i, i \in  I,\\
  q(y) & = 1 \text{ otherwise },
\end{align*}
where 
\begin{gather*}
  g_i \in  G^C \text{ is defined by }\\
  g_i(c^n) = a^{-n}_i,  n \in  I; \text{ and further },\\
  h_i = [q^{b_i}, q], i \in  I.
\end{gather*}

Consider\pageoriginale the mapping of $B$ onto $B$ defined by 
\begin{align*}
  b^{\beta}& = b \text{ and }\\
  b'^{\beta} & = bb'.
\end{align*}

It is easy to verify that $\beta$ is an automorphism of $B$. We want
to extend  $\beta$ to an automorphism  $\beta^*$ of $Q$. (Our procedure
is applicable to an arbitrary automorphism of $B$, but we require it
only for the particular  $\beta$ we have specified.) To do this we
first extend  $\beta$ to an automorphism of $P^B$ as follows: 

For every $f \in  P^B$, define $f^{\beta^*} \in  P^B$ by 
$$
f^{\beta^*} (y) = f(y^{\beta^{-1}}),  \text{ for all } y \in  B.
$$

It is easy to verify that $\beta^*$ is one-one.

Now if $f_1, f_2 \in  P^B$, then for every $y \in  B$, we have 
$$
(f_1f_2)^{\beta_*}(y) = f_1 f_2 (y^{\beta^{-1}}) = f_1(y^{\beta^{-1}})
f_2(y^{\beta^{-1}}) =  f^{\beta^*}_1 (y).  f_2^{\beta^*}(y). 
$$

Hence, 
$$
(f_1 f_2)^{\beta*} = f^{\beta*}_1 f^{\beta*}_2 ;
$$
it follows that ${\beta^*}$ is an automorphism of $P^B$. 

We now extend ${\beta^*}$ to $Q$ and denote the extension of ${\beta^*}$
also by ${\beta^*}$. For every $q_o = b_o f \in  Q$, with ba
$\in  B, f \in  P^B$, define\pageoriginale  
\begin{gather*}
  q^{\beta*}_0 \qquad Q \text{ as follows }:\\
  q^{\beta*}_0 = (b_0 f)^{\beta*} = b^{\beta}_\circ f^{\beta*}.
\end{gather*} 

 If
\begin{gather*}
  q_0 = b_0 f, b_0 \in  B, f \in  P^B \text{ and }\\
  q'_0 = b'_0 f', b'_0 \in  B, f' \in  P^B
\end{gather*}
are arbitrary elements of $Q$, then 
\begin{align*}
  (q_0q'_0)^{\beta*} = (b_0 fb'_0 f')^{\beta*} = (b_0 b'_0
  f^{b'_0}f')^{\beta*}& = (b_0 b'_0)^{\beta*}
 .  (f^{b'_0}f')^{\beta*}\\ 
  & = (b_0 b'_0)^{\beta*} (f^{b'_\circ})^{\beta*} f'^{{\beta*}}
\end{align*}
and 
$$
q^{\beta*}_0 q^{\beta*}_0 = (b_0 f)^{\beta*} (b'_0 f ')^{\beta*} =
b^{\beta*}_0 f^{\beta*}. b'^{\beta*}_0 f'^{\beta*} = b^{\beta*}_0
b'^{\beta*}_0 (f^{\beta*})^{b'_0}f'^{\beta*}. 
$$

But 
$$
  (f'^{b'_o})^{\beta*} (y) = f^{b'_0}(y^{\beta-1}) = f(y^{\beta-1}
  b'^{-1}_0) \text{ for all } y \in  B,
$$
and
\begin{multline*}
  (f^{\beta*})^{b'^{\beta*}_0}(y) = f^{\beta*} (y(b'^{{\beta*}}_0)^{-1})
  = f^{\beta*} (y(b'^{-1}_0)^{\beta*})\\ 
  = f((yb'^{-1
    {\beta}_0})^{\beta-1}) = f(y^{\beta-1}b'^{-1}_0),
\end{multline*}
for\pageoriginale all $y \in  B$.

Therefore
$$
(f^{\beta*})^{b'_0} = f^{b'_0})^{\beta*}.
$$

Hence 
$$
(q_0 q')^{\beta*} = q^{\beta*}_0 q'^{\beta*}.
$$

Again one can easily verify that ${\beta*}$ is one-one and onto; that
is, ${\beta*}$ is an automorphism of $Q$. 

Next, let $\gamma$ be the automorphism of $G$ defined by 
$$
a^{\gamma}_i = a_{i + 1} (i \in  I).
$$

We want to extend $\gamma$ to an automorphism $\gamma*$ of $Q$. (Our
procedure is applicable to an arbitrary automorphism of $G$, but we
require it only for the particular $\gamma$ we have specified.) Define
the mapping $\gamma^+$ of $G^C$ onto $G^C$ as follows: 

If $f \in  G^C$, then $f^{\gamma +} \in  G^C$, and 
$$
f^{\gamma+} (c^n) = (f(c^n)) \text{ for } n \in  I.
$$ 

A\pageoriginale straight forward verification shows that $\gamma^+$ is an
automorphism of $G^C$. We now extend $\gamma^+$ to $P$ by putting  
$$
(c^tf)^{\gamma^+} = c^t f^{\gamma^+}, c \in  C, f \in 
G^C, t \in  I. 
$$

Let $p_1 = c^t f,  P_2 = c^u f'$ belong to $P$. Then 
\begin{gather*}
  (p_1 p_2)^{\gamma ^{+}} = (c^{t+u } f^{c^{u} f'})^{\gamma^{+} } =
  c^{t+u} (f^{c^{u}}f' )^{\gamma^{+}}\\ 
  = c^{t + u} (f^c{^u})^{\gamma^{+}} f'^{\gamma^{+}}; \text{ and }\\
  p_1^{\gamma ^{+}} p_2^{\gamma^{+}} = c^t f^{\gamma^{+}} c^u
  f'^{\gamma^{+}} = c^{t+u} (f^{\gamma^{+}})^{c^u} f'^{\gamma^{+}}. 
\end{gather*}

But
\begin{align*}
  (f^{c^u})^{\gamma^{+}} (c^n) & = (f^{c^{u}}(c^n))^{\gamma} = (f(c^{n - u}))^{\gamma}\\
  = f^{\gamma^{+}} (c^{n-u}) & = (f^{\gamma^{+}})^{c^u} (c^n),
  \text{ for all } n \in  I. 
\end{align*}

Therefore 
$$
(f^{c^{u}})^{\gamma^{+}} = (f^{\gamma^{+}})^{c^{u}}.
$$

Hence\pageoriginale
$$
(p_1 p_2)^{\gamma^{+}}  = p_1^{\gamma^{+}} p_2^{\gamma^{+}}
$$

It is obvious that the extended mapping ${\gamma^{+}}$ is one-one and
onto. Thus ${\gamma^{+}}$ is an automorphism of $P$. 

We now extend ${\gamma^{+}}$ to an automorphism ${\gamma^{+}}$ of
$Q$. We first define ${\gamma^{+}}$ on $P^B$ as follows. For any $y
\in  P^B, g^{\gamma^{*}} \in  P^B$ and  
$$
g^{\gamma*} (y) = (g(y))^{\gamma^{+}}
$$

One easily verifies that ${\gamma*}$ is an automorphism of $P^B$. We
now extend ${\gamma*}$ to $Q$ by putting 
$$
(b_0g)^{\gamma*} = b_0 g^{\gamma*}, \text{ for } b_0 \in  B, g
\in  P^B. 
$$

Let 
\begin{gather*}
  q_1 = b_0 g, q_2 = b'_0 g', \text{ be in } Q \text{ with }\\
  b_0, b'_0 \in  B, g,g' \in  P^B. \text{ Then }\\
  (q_1 q_2)^{\gamma*} = (b_0 b'_0 g^{b'_{0}} g')^{\gamma*} = b_0 b'_0
  (b^{b'_0} g')^{\gamma*}\\ 
  = b_0 b'_0 (g^{b'_0})^{\gamma*} g'^{\gamma*}; \text{ and }\\
  q_1^{\gamma*} q_2^{\gamma*} = b_\circ g'^{\gamma*}. b'_0 b'_0 (g^{\gamma*})^
  {b'_0} g'^{\gamma*}.
\end{gather*}

But,\pageoriginale
$$
(g^{b'_0})^{\gamma*}(y) = (g^{b'_0}(y))^{\gamma^{+}} =
(g(yb'^{-1}_0))^{\gamma^{+}} = g^{\gamma*}
(g^{b'_0}(g^{\gamma*})^{b_0} (y) \text{ for all } y \in  B. 
$$

That is to say,
\begin{gather*}
  (g^{b'_0})^{\gamma*} = (g^{\gamma*})^{b'_0}; \text{ that is },\\
  (q_1 q_2)^{\gamma*} = q_1^{\gamma*} q_2^{\gamma*}.
\end{gather*}

One easily sees that ${\gamma*}$ is one-one and onto. Hence,
${\gamma*}$ is   an automorphism of $Q$. 

It will be noticed that the procedure of extending ${\gamma^{+}}$by
${\gamma*}$ is the same as that of extending ${\gamma}$ to
${\gamma^{+}}$; in fact it applies to wreath products in general. Now,
however, we being to\pageoriginale use the particular automorphisms $\beta,
{\gamma}$ we had chosen and the automorphisms $\beta*,  {\gamma*}$
constructed from them. 

Now consider automorphism $\beta*  {\gamma*}$ of $Q$. For any $b_0
\in  B$, we have $b^{\beta*  {\gamma*}}_0 =
(b^{\beta}_0)^{\gamma*} = b^{\beta}_0$; that is $\beta*  {\gamma*}$ is
an extension of $\beta$. Further. 
$$
q^{\beta*  {\gamma*}} (y) = \left(q^{\beta* }\right)^{{\gamma*}} (y) =
\left(q^{\beta*}(y)\right)^{\gamma^{+}} = \left(q(y^{\beta -1})\right)^{\gamma^{+}}. 
$$

Therefore 
$$
\displaylines{\hfill 
  q^{\beta*  {\gamma*}}  (1) = (q(1^{\beta -1}))^{\gamma^{+}} =
  (c)^{\gamma^{+}} = c\hfill \cr
  \text{and}\hfill 
  (q^{\beta*\gamma*}(b^{-1}_i)) = (q((b_i^{-i})^{\beta^{-1}}))^{\gamma^{+}} =
  (q((b_i^{\beta -1})^{-1}) ))^{\gamma^{+}}.\hfill } 
$$

But
$$
\displaylines{\hfill 
  b_i^{\beta -1} = (b^i b')^{\beta -1} = (b^i)^{\beta -1} (b')^{\beta
    -1} = b^i b^{-1} b' = b^{i -1} b' = b_{i - 1}, \hfill \cr
  \text{so that} \hfill 
  q^{\beta*\gamma*}(b^{-1}_i) = (q((b_{i-1}))^{\gamma^{+}} =
  (g_{i-1})^{\gamma^{+}}.\hfill } 
$$

As\pageoriginale
$$
g_{i -1}^{\gamma^{+}} (c^n) = (g_{i -1}(c^n))^{\gamma} = (a^{-n}_{i
  -1})^{\gamma} = a^{-n}_i, \text{ for all } n \in  I, 
$$
it follows that 
$$
q^{\beta*\gamma*}(b^{-1}_i) = g^{\gamma^{+}}_{i -1} = g_i,  i  \in  I.
$$

Now, since $b_i$ permute among themselves upon applying $\beta$, we have 
$$
q^{\beta*\gamma*} (y) = 1, y \neq 1, b^{-1}_i, i \in  I.
$$

Therefore 
$$
q^{\beta*\gamma*} =q.
$$

This shows that ${\beta*\gamma*}$ maps $H= gp(b, b', q)$ onto
itself. Let $\alpha$ be the restriction of ${\beta*\gamma*}$ to $H$,
so that $\alpha$ is an auto-morphism of $H$. We have  
\begin{multline*}
  h^{\alpha}_i = [q^{b_i}, q]^{\alpha} = \left[q^{\alpha^{b^{\alpha}_i}},
  q^{\alpha}\right]= \left[q^{b^{\alpha}_i}, q \right] = \left[q^{b_{i
        + 1}}, q\right] = h_{i +   1}\\ 
  (\text{ for } b^{\alpha_i}  
  =( b^i b')^{\alpha} = (b^i)^{\beta} b'^{\beta} = b^i bb = b^{i +1 }
  b' = b_{i + 1}). 
\end{multline*}

Consider $R \leq G$, where 
$$
R = gp(\ldots,  a_{-1}, a_0)
$$

In\pageoriginale the identification of $G$ with $G^*, R$ is identified with 
$$
R^* = gp(\ldots,  h_{-1}, h_0).
$$

Trivially,
$$
R^* \Delta G^*.
$$

Now by the corollary of Lemma \ref{chap12:sec3:lem2}, $G^*$ is an $E$-subgroup of
$H$. There is therefore an  
\begin{gather*}
  S^* \Delta H, \text{ such that }\\
  R^* = S^* \cap G^*.
\end{gather*}

Let
$$
K= H/ S^*.
$$

Then 
$$
K = H/ S^* \cong H^{\alpha} / S^{*^{\alpha}} = H/ S^{*^{\alpha}}. 
$$

Now 
$$
S^* \geq gp(\ldots, h^{\alpha}_{-1},h^{\alpha}_{0}) = gp (\ldots,
h_{-1}, h_0, h_1). 
$$

But,\pageoriginale
\begin{gather*}
  h_1 \notin R^* = S^* \cap G^*, \text{ so that }\\
  h_1 \notin S^*
\end{gather*}

Thus $S^*$ is strictly contained in $S^{*^{\alpha}}$. We have 
$$
H/S^{*^{\alpha}} \cong H/S^*/S^{*^{\alpha}}/S^*.
$$

Thus
$$
K = H/S^{*^{\alpha}} \cong H/S^*/S^{*^{\alpha}}/S^* = K/N,
$$
where $N = S^{*^{\alpha}}/S^*$ is not trivial. Evidently $K$ is a
3-generator group. Further by Corollary \ref{chap12:sec1:coro2},
p. 141 of Chapter \ref{chap8}, since 
$G$ is abelian, $H$ and therefore $K$ is soluble of length 3. Thus we
have proved: 
\begin{theorem}\label{chap12:sec4:thm4}%the 4
  The group
  $$
  K= H/ S^*
  $$
  is a 3-generator non-Hopf group, soluble of length 3.
\end{theorem}

\begin{thebibliography}{99}
\bibitem{1}{Baer, Reinhold (1953)}\pageoriginale - Das Hyperzentrum einer Gruppe
  $III$. Math. Zeitschr, 59, 299-338 
\bibitem{2}{Birkhoff, Garret(1935)}- On the structure of abstract
  algebras. Proc Cambridge Philos. Soc. 31, 433-454. 
\bibitem{3}{Boone, W. W. $(1954^a)$}- Certain simple uusolvable
  problems of group theory
  I.Proc. K. Nederl. Akad. Wetensch. Amsterdam $(A)$ 57, 231-237. 

  $(1954^b)$- Certain simple unsolvable problems of group
  theory II. Proc. K. Nederl. Akad. Wetensch. Amsterdam $(A)$57,
  492-497. 

  $(1955^a)$- Certain simple unsolvable problems of group
  theory III. Proc. K. Nederl. Akao. Wetensch. AMsterdam $(A)$ 58,
  252-256. 

  $(1955^b)$- Certain simple unsolvable problems of group
  theory IV. Proc. K. Nederl. Akad. Wetensch. Amsterdam $(A)$58,
  571-577. 

  (1957)- Certain simple unsolvable problems of group
  theory VI. Proc. K. Nederl. Akad. Wetensch. Amsterdam $(A)$60,
  22-27.
 
  (1958)- The word
  problem. Proc. Nat. Acad. Sc. U.S.A.44, 1061-1065. 

  (1959)- The word problem. Annals of Math. (70) No.2,
  207-265. 
\bibitem{4}{Britton J.L.(1956)}- Soution of the word problem for
  certain types of groups $I$. Proc. Glasgow Math. Assoc. 3, 45-54. 

  (1957)- Soution of the word problem for certain types
  of groups $II$. Proc. Glasgow Math. Assoc. 3., 68-90. 

  (1958) - The word problem for
  groups. Proc. Lond. Math. Soc. (3) 8, 493-506. 
\bibitem{5}{Burnside, W.(1902)}\pageoriginale - On an unsettled question in the
  theory of discontinuous groups. Journal of Pure and Applied
  Maths. (Lond.) 33, 230-238. 
\bibitem{6}{Coxeter, H.S.M.and W.O.J.Moser (1957)} - Generators and
  relations for discrete groups. (Ergebnisse der Mathematik undihrer
  Grenzgebiete) Springer-Verlag, Berlin, Gottingen, Heidelberg. 
\bibitem{7}{Hall, P. (1933)} - A contribution to the theory of groups
  of prime order. Proc. London Math. Soc. (2) 36, 29-95. 

  (1954$^a$) - The splitting properties of relatively
  free groups. Proc. London Math. Soc.(3) 4, 343-356. 

  (1954$^b$) - Finiteness conditions on soluble
  groups. Proc. London Math. Soc. (3)4, 419-436. 
\bibitem{8}{Hall, P. and Graham Higman (1956)}- On the p-length of
  p-soluble groups and reduction theorems for Burnside's
  problem. Proc. London Math. Soc. (3) 6, 1-42. 
\bibitem{9}{Hall, Marshall Jr.(1959)}- Solution of the Burnside problem
  for exponent six. Illinois J. Math. Vol.2, No.4B(1958) 764-786. 
\bibitem{10}{Higman, Graham (1951)}- A finitely related group with an
  isomorphic proper factor group. J. London Math. Soc. 26, 59-61. 

  (1956)- On finite groups of exponent
  five. Proc. Cambridge Philos. Soc. 52, 381-390. 
\bibitem{11}{Higman Graham and B.H. Neumann (1952)}- Groups as groupoids
  with one law. Publ. Math. Debrecen 2, 215-221. 
\bibitem{12}{Higaman Graham, B.H. Neumann and Hanna Neumann(1949)}-
  Embedding theorems for groups. J. London Math. Soc. 24, 247-254. 
\bibitem{13}{Hirsh, K. A.(1938$^a$)}- On infinite soluble groups
  I. Proc. London Math. Soc. (2) 44, 53-60. 

  (1938$^b$)- On infinite soluble groups II. Proc. London
  Math. Soc. (2) 44, 336-344. 
\bibitem{14}{Hirsch K.A. (1946)}\pageoriginale - On infinite soluble
  groups III. Proc. London Math. Soc. (2) 49, 184-194. 

  (1952)- On finfinite soluble groups IV. J. London
  math. Soc. 27, 81-85. 

  (1954)- On infinite soluble groups V. J. London
  Math. Soc. 29, 250-254. 
\bibitem{15}{Kaloujnine, Leo(1948)}- La struture des p-groupes des sylow
  des groupes symetriques finis. Ann. Sci. Ecole Norm. Sup. (3) 65,
  239-276. 
\bibitem{16}{Kaloujnine, Leo and Marc Krasner (1950)}- Produit complet
  des groupes de permutations et probleme d'extension des roupes,
  Acta. Sci. Math. Szeged. Vol.13, 208-230. 

  (1951) - Produit complet des groupes de permutations et
  probleme d' extension des groupes, Acta Sci. Math. Szeged. Vol. 14,
  39-66,69-82. 
\bibitem{17}{Kostriken, A. I.(1955)}- Solution of the restricted
  Burnside problem for exponent five, Izv. Akad. Nauk SSSR,
  Ser., at. 19, 233-244(Russian) 

  (1959)- On Burnside's problems. Izv. Akad. Nauk SSSR
  Ser Mat. (23), 3-34(Russian). 
\bibitem{18}{Kurosh, A.G. (1955)}- Theory of groups. Chelsea Publ, Co.,
  New York, N.Y., Vol. I. 

  (1956)- Theory of groups. Chelsea Publ. Co., New York,
  N.Y., Vol. II. 
\bibitem{19}{Levi, F. and B.L.van der Waerden (1933)}- \"Uber eine
  besondere Klasse von Gruppen Abh. Math. Sem. Hamburg, 154-158. 
\bibitem{20}{Lyndon, R.C. (1952)}- Two notes on nilpotent
  groups. Proc. Amer. Math. Soc. 3, 579-583.
\bibitem{21}{Magnus, Wilhelm (1932)}- Oas Identitatsproblem fou Gruppen
  miteiner definierenden Relation, Math. Ann. 106, 295-307. 

  (1935)- Beziehungen zwischen Gruppen und Idealen in
  einem speziellen Ring. Math. Ann. 111, 259-280. 
\bibitem{22}{Neumann, B.H.($1937^a$)}\pageoriginale - Identical
  relations in groups I. Math. Ann. 114, 506-525. 

  (1937$^b$)- Some remarks on infinite groups. J. London
  Math. Soc. 12, 120-127. 

  (1950)- A two-generator group isomorphic to a proper
  factor group. J. London Math. Soc. 25, 247-248. 

  (1953)- On a problem of Hopf. J. London Math.Soc. 28,
  351-353. 

  (1954)- An essay on free products of guoups with
  amalgamations. Phil. Trans. Roy. Soc. London. A, 246, 503-545. 

  (1956)- On some finite groups wiht trivial
  multiplicator. Publ. Math. Debrecan 4, 186-194. 
\bibitem{23}{Neumann, Hanna (1957)}- Generalised free products with
  amalgamated subgroups I. Amer. J. MATH. 70, 590-625. 

  (1949)- Generalised free products with amalgamated
  subgroups II. Amer. J. Math. 71, 491-540. 

  (1950)- Generalised free sums of cyclical
  groups. Amer. J. Math. 72. 671-685. 

  (1951)- On an amalgam of abelian groups. J. London
  Math. Soc. 26, 228-232. 

  (1956)- On varieties of groups and their associated
  near rings. Math. Math. Zeitschr. 65, 36-69. 

\bibitem{24}{Neumann, B.H. and Hanna Neumann(1950)}- A remark on
  generalised free products. J. London Math. Soc. 25, 202-204. 

  (1950-51)- Zwei Klassen charakteristischer unterguppen
  und ihre factor gruppen. Math. Nach. 4, 106-125. 

  (1953)- A contribution to the embedding theory of group
  amalgams. Proc. London Math. Soc. (3), 3, 245-256. 

  (1959)- Embedding theorems for groups. J. London
  Math. Soc. 34, 465-479. 
\bibitem{25}{Newman, H.H.A. (1942)}\pageoriginale - On theories with a
  combinatorial 
  definition of ``equivalence''. Annals of Math, 43, 223-243. 
\bibitem{26}{Novikov, P.S.(1952)}- On the algorithmic unsolvability of
  the identity problem. Doklady Akad. Nauk SSSR 85, 709-712(Russian). 

  (1955)- On the algorithmic unsolvability of the word
  problem in group theory. Trudy Mat. Inst. im. Steklov No.44, Izdat,
  Akad. Nauk. SSSR 143 p.p. (Russian). = Amer. Math. Transl. seies 2,
  Vol.9 (1958)1-122.

  (1959)- On periodic group. Doklady Akad. Nauk. SSSR
  127, 749-752 (Russian). 
\bibitem{27}{Polya, George(1937)}- Kombinatorische Anzahlbestimmungen
  fur Group, Graphen und Chemische Verbindungen Acta Math. 68,
  145-253. 
\bibitem{28}{Sanov, I.N.(1940)}- Solution of Burnside problem for
  exponent four. Ucenye Zapiski Leningrad Univ. 55, 16-170 (Russian). 

  (1947)- On the Burnside Problem. Dklady
  Akad. Nauk. SSSR 57, 759-761(Russian). 
\bibitem{29}{Schiek, Helmut(1956)}- Ahnlichkeitsanalyse von
  Gruppenrelationen. Acta Math. (96), 157-252. 
\bibitem{30}{Schreier, Otto(1927)}- Die Untergruppen der freien
  Gruppen. Abh. Math. Sem. Hamburg, 5, 161-183. 
\bibitem{31}{Specht Wilhelm(1932)}-Eine Verallgemeinerung der
  symmetrischen Gruppen. Schriften
  d. Math. sem. u.d. Inst. F. angew. Math. d. Univ. Berlin 1,4,1. 
\bibitem{32}{Tartakovskiiu, V. A.($1949^a$)}- The sieve method in the
  theory of group. Mat. Sb. N. S. 25(67), 3-50 (Russian). =
  Amer. Math. Soc. Transl. No. 60(1952), 

  ($1949^b$)\pageoriginale -
  Application of the sieve 
  method to the solution of the word problem in certain types of
  group. Math. Sb. W. S. 25(67), 251-274(Russian). =
  Amer. Math. Soc. Transl. No.60(1952), 63-92. 

  ($1949^c$)- Solution of the word problem for groups
  with a k-reducible basis for $k > 6$, Izv. Akad. Nauk. SSSr
  Ser. Mat. 13, 43-494 (Russian). =
  Amer. Math. Soc. Transl. No. 60(1952), 93-110. 

  (1952)- On primitive composition. Mat. Sb. N.S. 30(72),
  39-52 (Russian). 
\bibitem{33}{Wiegold, James(1959)}-Nilpotent products of groups with
  amalgemations. Publ. Math. Debrecen 6, 131-168. 
\bibitem{34}{Wielandt, Helmut(1939)}- Eine Verallgemeinerung der
  Untergruppen. Math. Zeitscher. 45, 209-244. 
\bibitem{35}{Zassenhaus, Hans J.(1958)}- The theory of group. Chelsea
  Publ. Co., New York.
\end{thebibliography}

