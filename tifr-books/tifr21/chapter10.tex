
\chapter{Permutational Products}\label{chap10}%cha X

\section{Permutational products and Scbreier's Theorem}\label{chap10:sec1}%sec 1

In\pageoriginale this chapter we shall introduce another product called the
``permutational product'' of an amalgam. The permutational product of
an amalgam will be used in giving an alternative proof of Schreier's
Theorem. The embedding group we are going to construct will be in
general different from the generalised product of the amalgam in
question. Once we construct a group embedding the amalgam the
existence of the generalised free product follows Theorem $2$ of the
preceding chapter.  

It the following we will be considering an amalgam of two groups. This
is just for convenience. The same proof carries over the case of an
amalgam of an arbitrary family of groups with a single amalgamated
subgroup.  

Let the amalgam consist of groups $A, B$ with the amalgamated subgroup
$H, H \leq A, H \leq B$. We choose arbitrary left transversals $S, T$
of $H$ in $A$ and $B$ respectively. Thus every $a \in  A$ and
$b \in  B$ can be uniquely written as  
\begin{align*}
  a & = sh, ~ s \in  S,  h \in  H \\
  b & = th_1,  ~ t \in  T,  h_1 \in  H. 
\end{align*}

Consider\pageoriginale the product set $K = S \times T \times H$. Our object is to
realise the embedding group as a permutation group of the set $K$. For
every $a \in  A, b \in  B$ we define mappings $\rho_
{(a)}, \rho_{(b)}$ of $K$ into $K$ as follows. 
 
Let $k = (s, t, h) \in  K, s \in  S, t
\in  T, h \in  H$. Then put  
$$
\displaylines{\hfill 
  k^{ \rho_ (a) } = k ' = (s', t', h'),\hfill \cr 
  \text{where} \hfill  s ' h' = s ~ h ~ a, \hfill \cr
  \text{and} \hfill t' = t. \hfill } 
$$

Similarly we define  
$$
\displaylines{\hfill 
  k^{ \rho_ {(b)} } = k'' = (s'', t'', h''), \hfill \cr
  \text{where} \hfill  s'' h'' = t ~ h ~ b, \hfill \cr
  \text{and} \hfill s'' = s. \hfill }
$$

If $h^* \in  H$, then $\rho_{(h^*)}$ can be defined in two ways,
once considering it as an element of $A$ and the other time
considering it as an element of $B$. But,  
$$
\displaylines{\hfill s~ h ~ h^* = s ' ~ h', \hfill \cr
  \text{implies} \hfill s = s', h' = hh^*. \hfill }
$$

Similarly\pageoriginale 
$$
\displaylines{\hfill t h h^* = t'' ~ h'', \hfill \cr 
  \text{implies}\hfill  t = t'', h'' = h h^*. \hfill }
$$

Therefore, whatever way we compute $k^{\rho(h^*)}$, we get 
$$
k^{\rho (h^*)} = ( s, t, hh^*); 
$$
hence our definition of $\rho$ is unambiguous. Also, $\rho (h^*)$ when
applied to any $(s, t, h )\in  K$ leaves $s, t$ unaltered.  

Conversely one can easily verify that if $\rho (x), x \in  A$
or $B$, does not alter $s, t $ when applied to a triplet $(s, t, h)
\in  K$, then $x \in  H$.  

Now consider the mapping $\rho_A$ of $A$ into the semigroup of all
mappings of $K$ into itself defined by  
$$ 
a^{\rho_ A} = \rho(a),  \text{ for all } a \in  A. 
$$

Let $a, a_1 \in  A, ( s, t, h) \in  K $. We have, 
$$
\displaylines{\hfill 
  ( s, t, h ) ^{\rho_ (a)} =  (s', t', h' ), \hfill \cr
  \text{with } \hfill  s' h' = s ~ h ~ a, \hfill \cr
  \text{and} \hfill  t' = t; \hfill \cr
  \text{and further}\hfill  
  ( s',  t',  h' )^{\rho (a_1)} =  (s'', t'', h'' ),\hfill \cr 
  \text{where } \hfill  s'' h'' = s'  ~ h' ~ a_1, \hfill \cr
  \text{and} \hfill t'' = t'. \hfill } 
$$

But\pageoriginale 
$$
sh (aa_1) = (sha )a_i = s' h' a_i = s'' h''. 
$$

Therefore, 
$$
(s, t, h )^{ \rho_(aa_1)} = (s'', t'', h'') = (s, t, h)^{ \rho_ (a)\rho_ (a_1)}. 
$$

As this is true for all $(s, t, h) \in  K$, we have 
$$
\rho (aa_1) = \rho (a ) \rho (a_1). 
$$

Again as this is true for all $a, a_1, \in  A$ it follows that
the mapping $\rho_A$ is a homomorphism. Therefore the homomorphic
image $A^{\rho_ A} = \rho (A)$ is a group. Hence every $\rho (a), a
\in  A$ has a two sided inverse; that is, $\rho (a)$ is a
permutation group of $K$. Further if  
$$
\rho(a) = L, a \in  A, 
$$
then for every $(s, t, h) \in  K$, we have 
$$
(s, t, h) ^{ \rho_ (a)} = (s, t, h).  
$$

Therefore,\pageoriginale 
\begin{gather*}
  \text{ sha = sh ; that is  } \\
  a = 1 
\end{gather*}
that is $\rho_A$ is an isomorphism of $A$ onto $A^{\rho_ A}  = \rho
(A)$. Similarly the mapping $\rho _B$ of $B$ into the semigroup of all
mapping of $K$ into itself defined by  
$$
b^{\rho_ B} = \rho_ (b), \text{ for all } b \in  B
$$
is a monomorphism; that is $B$ is isomorphic to $B^{ \rho_ B} = \rho
(B)$. Denote by $P$ the permutation group of $K$ generated by $\rho (A)$
and $\rho_ (B)$,  
$$
P = gp (\rho (A),  \rho (B)). 
$$

It is evident that $P$ contains isomorphic copies of $A$ and $B$
namely $A^{\rho_ A}$ and $B ^{ \rho_ B}$. We claim that  
$$
\displaylines{\hfill 
  A^{\rho_ A}\cap B^{ \rho_ B} = H ^{\rho_ A} = H^{\rho_B}. \hfill \cr
  \text{Let}\hfill  
  \rho_{(a)} = \rho_{(b)}, a \in  A, b \in  B.\hfill } 
$$

Then\pageoriginale $\rho_{(a)} $ fixes both $s, t $ of any $(s, t, h) \in 
K$. Therefore,  
$$
a \in  H. 
$$

Similarly, 
$$
b \in  H. 
$$

Now, since $\rho_A $ is an isomorphism and 
$$
\rho_{(a)} = \rho_{(b)}, 
$$
it follows that 
\begin{gather*}
  a = b \in  H ; \text{ thus } \\
  A^{\rho_ A}\cap B^{\rho_ B} = H ^{\rho_A} = H^ {\rho_B}.
\end{gather*}

Hence $P$ embeds the amalgam. This proves Schreier's Theorems. We call
$P$ a \textit{ permutational product } of the amalgam. This proof of
Shchreier's theorem immediately leads us to the following corollary.  

\begin{coro*}%coro
  An amalgam of two finite groups is embeddable in a finite group.
\end{coro*}

In this context we mention the following unsolved problem. 

\noindent \textbf{Unsolved problem.} 
  If an amalgam of $n (n > 2)$ finite groups embeddable in a group, is
  it embeddable in a finite group?  
 
\section{}\label{chap10:sec2} % Section 2

Now\pageoriginale we shall consider amalgams of abelian groups. We ask the following
question: If an amalgam of $n$ abelian groups is embeddable in a
group, is it embeddable in an abelian group ?  

For $ n = 2, 3, 4$, the answer to this question is `yes'; for $n =
5$. `no' ( see Hanna Neumann, $1951$ and B. H.  Neumann and Hanna
Neumann, 1953). For $n = 2$, we shall prove the assertion.  

We shall start with a more general situation. Let $A$ and $B$ be any
two groups and $H, H_1$ be isomorphic subgroups of $A$ and $B$
respectively, and let $H, H_1$ be contained in the centres of $A$ and
$B$,  
$$
H \leq \text{ centre } (A), H_1  \leq \text{ centre } (B). 
$$

Let $\theta$ be an isomorphism of $H$ onto $H_1$. Consider the direct
product $A \times B$ of $A$ and $B$. We shall denote an arbitrary
element of $A \times B$ by  
$$
a \times b, \text{ with } a \in  A,  b \in  B. 
$$

Consider $N \subseteq A \times B$, defined by 
$$
N = \bigg\{ h^{-1} \times h_1 \bigg | h_1 = h ^\theta \bigg\}. 
$$

Now if $x = h^{-1} \times h_1, y = h'^{-1} \times h'_1 \in  N$ with 
\begin{align*}
  h_1 & = h^\theta, h'_1 = h '^\theta, \text{ then }\\
  xy^{-1} & = (h^{-1} \times h_1) (h'^{-1} \times h_1 )^{-1} = (h^{-1}
  \times h_1) (h' \times h_1^{'-1}) \\ 
  & = h^{-1} h' \times h_1 h_1'^{-1}\\
  & = (hh'^{-1})^{-1} (\text{ As }, h, h' \in  \text{ centre } (A))\\
  & = (hh'^{-1})^{-1} \times h_1 h_1^{-1}. 
\end{align*}

But,\pageoriginale 
$$
(hh'^{-1})^\theta = h^\theta (h^\theta )^{-1} = h_1 h_1^{'-1}. 
$$

Therefore, 
\begin{gather*}
  xy^{-1} \in  N ; \text{ that is }\\
  N \leq A \times B. 
\end{gather*}

It is easy to verify that 
$$
\displaylines{\hfill 
  N \leq ~\text{ centre }~ (A \times B )\hfill \cr
  \text{and therefore} \hfill N \Delta A \times B. \hfill}
$$

Consider now the quotient group $A \times B /N$. We claim that the
mapping $\pi$ of $A$ into $A \times B /N $ defined by  
$$
a^\pi = (a \times 1) N \in  A \times B/N, a \in  A. 
$$
is\pageoriginale a monomorphism. That it is a homomorphism is easy to verify. Now if  
\begin{gather*}
  a \times 1 \in   N, \text{ then } \\
  a \times 1  = h^{-1} \times h ^\theta, \text{ for some } h \in  H; 
\end{gather*}
that is, 
$$
a = h^{-1}, 1 = h^\theta. 
$$

Since $\theta $ is an isomorphism, 
\begin{gather*}
  1 = h^\theta \text{ implies } h = 1 ; \text{ that is } \\
  a = h^{-1} = 1. 
\end{gather*}

Hence $\pi$ has a trivial kernel; that is $\pi$ is a
monomorphism. Similarly the mapping $\pi_1$ of $B$ into $A \times B
/N$ defined by  
$$
b^{\pi_1} = (1\times b)N \in  A \times B /N, b \in  B
$$
is a monomorphism. Thus the groups $A$ and $B$ are monomorphically
embedded in $A \times B/N$. We assert that $A \times B /N$ embeds the
amalgam in question. To see this we have only to prove  
$$
A^{\pi}\cap B^{\pi_1} = H^{\pi} = H_1^{\pi_1}.
$$

Now for any $h \in  H$, we have 
\begin{gather*}
  (h \times 1)( 1 \times h^\theta )^{-1} = h \times ( h^\theta )^{-1}
  \in  N : \text{ that is }\\ 
  (h \times 1) N = (1 \times h^\theta ) N. 
\end{gather*}

Making\pageoriginale $h$ run through all the elements of $H$, we get 
$$
H^{\pi}= H_1^{\pi}. 
$$

It is immediate that 
$$
H^\pi = H_1^{\pi_1} \subseteq A^\pi \cap B ^{\pi_1}.
$$

Conversely if $x \in  A^\pi \cap B^{\pi _1}$, then 
$$
x = (a \times 1) N = (1 \times b) N  \text{ for some  } a  \in 
A, b \in  B.   
$$

This gives 
\begin{gather*}
  a \times b^{-1} \in  N ; \text{ that is } \\
  a = h, b = h^\theta \text{ for some } h \in  H ; \text{ that is }  \\
  x = (h \times 1)N \in  H^{\pi} =  H_1 ^{\pi_1}.
\end{gather*}

Hence 
$$
A^\pi   \cap B^{\pi _1} \subseteq H^{\pi } = H^{\pi _1}_1. 
$$

Combining this with the above inclusion we have 
$$
A^\pi  \cap B^{\pi _1} = H^\pi  = H^{\pi _1}_1. 
$$

This\pageoriginale proves that $A \times B /N$ embeds the amalgam. It is evident that 
$$
A \times B /N = A^{\pi } B^{\pi _1}, 
$$
and every element of $A^{\pi }$ commutes with every element of $B^{
  \pi_1}$. We call $A \times B/N$ a ``generalised direct product'' of
the amalgam. [This is also called a ``central product'' by some
  authors.] 

Let $A$ and $B$ any two groups each containing an isomorphic copy of a
group $H$. Without loss of generality we can take $H \leq A, H \leq
B$. Let $A$ and $B$ embedded monomorphically in a group $G$. We shall
identify these monomorphic images with $A$ and $B$ respectively and
take  
$$
A \leq G, B \leq G. 
$$

We call $G$ a \textit{ generalised direct product } of the amalgam of
$A$ and $B$ with the amalgamated subgroup $H$ if  
\begin{enumerate}[(i)]
\item $G = AB$
\item $A \cap B = H$
\item Every element of $A$ commutes with every element of $B$ 
\end{enumerate}

In particular when $H$ is the trivial group we get the usual direct
product. In order that a generalised direct product of the amalgam in
question may exists necessary that  
$$
H \leq \text{ centre } (A), H \leq \text{ centre } (B).
$$

This\pageoriginale is immediate from (iii). We have also proved that the condition
is sufficient.  

Let $G$ be a generalised direct product of the amalgam consisting of
groups $A$ and $B$ with an amalgamated subgroup $H$. Consider the
mapping $\varphi$ of $A \times B /N$ onto $G$ defined by  
$$
(( a \times b)N) ^{\varphi}  = a b \quad G. 
$$

Once can easily verify that $\varphi$ is an isomorphism. In other
word, the generalised direct product of an amalgam in unique upto an
isomorphism. Thus we can speak of \textit{ the } generalised direct
product of an amalgam. In contract to this, the permutational product
of an amalgam is in general not unique. We shall soon make an
example. We summarise the results proved above in the following;  
\setcounter{theorem}{0}
\begin{theorem}\label{chap10:sec2:thm1}%theo 1
  The generalised direct product of an amalgam consisting of groups $A$
  and $B$ with an amalgamated subgroup $H$ exists if and only if  
  $$
  H \leq \text{ centre }(A),H \leq \text{ centre }(B);
  $$
  and it is unique to an isomorphism. 
\end{theorem}

Taking $A$ and $B$ to be abelian groups we have, 

\begin{coro*}%coro 
  An amalgam of two abelian groups is embeddable in an abelian group; 
\end{coro*}

\section{ }\label{chap10:sec3} %Section 3

Consider\pageoriginale again the amalgam of two groups $A$ and $B$ with an
amalgamated subgroup $H$ with, $H \leq \text{ centre  }(A),H \leq
\text{ centre }(B)$. As before, we choose transversals $S, T$ of $H$
in $A$ and $B$ respectively and form permutational product $P$ on the
set $K = S \times T \times H$. We show now that in this case every
$\rho (a) \rho (A)$ commutes with every element $\rho (b) \in 
\rho (B)$. Let $(s, t, h)\in  K$,  $\rho (a) \in \rho (A), \rho
(b) \in  (B)$. Then,  
\begin{align*}
  (s, t, h )^{ \rho (a)} & = (s_1,  t_1, h_1), ~\text{ with }\\
  s_1 h_1 & = sha,  t = t_1 ; \\
  (s_1,  t_1,  h_1)^{ \rho (b)}  & = (s_2, t_2, h_2), ~\text{ with }\\
  t_2 h_2 & = t _1 h_1 b , s_2=s_1 
\end{align*}

Now from the above equations it follows that 
$$
thb = th (h_1^{-1} t_1^{ -1}t_2 h_2) = t_1 h (h^{-1}_1 t^{-1}_1 t_2  h_2). 
$$
But $H \leq$ centre $(B)$. Therefore, 
$$
thb = t_2 t_1 t_1 ^{-1} hh_1^{-1} h_2 = h_2 hh^{-1}_1 h_2. 
$$
Hence\pageoriginale we have, 
$$
(s, t,  h)^{\rho (b)} = (s,  t_2,  hh_1^{-1} h_2 )
$$
Again from the above equations and since $H \leq $ centre $(A)$, we have 
$$
s (hh^{-1}_1 h_2 ) a = ( sha ) h_1^{-1} h_2 = (s_1 h_1 ) = s_1 h_2 = s_2 h_2. 
$$

Therefore
\begin{align*}
  (s, t, g) ^{\rho (b) \rho (a)} & = (s, t _2,  hh^{-1}_1 h_2)^{\rho
    (a) }= (s_2,  t_2,  h_2) ; \\ 
  &= (s, t, h)^{\rho (a) \rho (b)}. 
\end{align*}

As this is true for all $(s, t, h)\in  K$, we get 
$$
\rho (a) \rho (b) = \rho (b) \rho (a). 
$$

This is true for all $a \in  A, b \in  B$. Since 
$$
P = gp(\rho (A), \rho (B))
$$
embeds the amalgam, we have proved that $P$ is the generalised direct
product of the amalgam. Thus we have:  
\begin{theorem}\label{chap10:sec3:thm2} % them 2.
  If\pageoriginale $H$ is central in both $A$ and $B$, then the permutational
  product of the amalgam is the generalised direct product.  
 \end{theorem}
 
The uniqueness of the generalised product gives:
\begin{coro*}
  Under the assumptions of the above theorem the permutational product
  does not depend upon the transversals chosen. 
\end{coro*} 
 
Incidentally, not that in this case the permutational product is not
the generalised free product. For in the generalised free product $a
\in  A - H$ and $b \in  B-H$ do not commute. 
 
In general, the permutational product depends upon the transversals
chosen. We give here an example. Take $A,B$ to be groups isomorphic to
$S_3$ and $H$ to be a subgroup of order $2$ of $S_3$. For $A, B, H$ we
give the following presentations. 
\begin{align*}
  A =& gp~(p,r; p^3 = r^2 = (p r)^2 = 1), \\
  B =& gp~(p,r; q^3 = r^2 = (q r)^2 = 1), \\
  H =& gp~(p,r; p^2 = 1).
\end{align*}
 
For the transversals of $H$ in $A$ and $B$, first we choose
$$
S_1 =\bigg \{1,  p, p^2 \bigg\}, T_1 =\bigg \{1, q, q ^2\bigg\}
$$
We rename the elements of $K_1 = S_1 \times T_1 \times H$, for
convenience: 
\begin{alignat*}{6}
  (1, 1, 1) & = 1; &(p, 1,1) &= 4; \quad &(p^2,  1, 1) & = 7; \\
  (1, q, 1) & = 2; &(p, q,1) &= 5; &(p^2,  q, 1) &= 8; \\
  (1, q^2, 1) & = 3; &(p, q^2,1) &= 6; &(p^2,  q^2, 1) &= 9; \\
  (1, 1, r) & = 1'; &(p, 1,r) &= 4'; &(p^2,  1, r) &= 7'; \\
  (1, q, r) & = 2'; \quad &(p, q, r) &= 5';\quad  &(p^2,  q, r) &= 8'; \\
  (1, q^2, r) & = 3'; &(p, q^2, r) &= 6'; &(p^2,  q^2, r) &= 9'.
\end{alignat*}
 
By\pageoriginale a straightforward  computation one obtains:
\begin{align*}
  \rho(p) & = (147) (258)(369) (1' 7' 4') (2' 8' 5') (3' 9' 6'),\\
  \rho(r) & = (11') (22') (33') (44') (55') (66') (77') (88')(99'), \\
  \rho(q) & = (123) (456) (789) (1' 3' 2') (4' 6' 5') (7' 9' 8').
\end{align*}
 
One can easily verify that 
$$
[\rho (p), \rho(q)] = L.
$$ 	

Thus $\rho(p)$ and $\rho(q)$ generate a group of order $9$. Let $p_1$
denote the permutational product of the amalgam, 
$$
P_1 = gp (\rho(p), \rho(q), \rho(r)).
$$

It is not difficult to verify that $P_1$ is an extension of $gp
(\rho(p), \rho(q))$ by $gp(\rho(r)$). Thus\pageoriginale  
$$
|P_1| = 18.
$$

Now we choose different transversals and form the permutation
pro\-duct. Choose 
$$
S_2 = \bigg\{ r, p, p^2 \bigg\}, T_2 = T = \bigg\{1, q, q^2 \bigg\}
$$
to form the permutational product. Let 
$$
K_2 = S_2 \times T_2 \times H.
$$

As before we rename the elements of $K_2$
\begin{alignat*}{6}
  (r, 1,1) & = 1;\quad & (r,q, 1) &= 2; &(r, q^2,  1) &= 3; \\
  (p, 1,1) & = 4 ; &(p,q, 1)& = 5;& (p, q^2,  1)& = 6; \\
  (p^2, 1,1) & = 7 ;\qquad  &(p^2,q, 1) &= 8;\qquad  & (p^2, q^2, 1)& = 9; \\
  (r, 1,r) & = 1'; &(r, q, r)& = 2'; &(r, q^2,  r)& = 3'; \\
  (p, 1,r) & = 4'; &(p, q, r)& = 5'; &(p, q^2,  r)& = 6'; \\
  (p^2, 1,r) & = 7'; &(p^2, q, r) &= 8'; &(p^2, q^2,  r)& = 9'. 
\end{alignat*}

As we have not changed the transversal $T_1$, of $H$ in $B, \rho(q)$
and $\rho(r)$\pageoriginale are not altered. The only generator that is altered is
$\rho(p)$. One can again compute it without difficulty: 
$$
\rho(p) = (17'4') (28' 5') (39' 6') (1' 47) (2' 58) (3' 69)
$$

Now
$$
[\rho(p), \rho(q)] = (132) (456) (1'2'3') (7'8'9').
$$

Thus $gp (\rho(p)$, $\rho(q))$ is not elementary abelian; in fact it
turns out to be a group order $81$. The group $P_2$, the permutational
product with the above choice of transversals is given by  
\begin{gather*}
  P_2 = gp (\rho(p), \rho(q),  \rho(r)); and \\
  P_1 \neq P_2.
\end{gather*}

Thus, in general, by selecting different transversals we get different
permutational products. If we choose the transversals $\{r, p, p^2
\},  \{r,  q, q^2 \}$ of $H$ in $A,B$ respectively, the
corresponding permutational product $P_3$ we get, is a group of order
$9$; in fact it is the direct product of the alternating group, $A_9$
and a group of order $2$ and therefore not soluble, whereas $S_3$ is
metabelian. Thus the permutational product of two metabelian group
with an amalgamated sub ground need not even be soluble. 

\section{}\label{chap10:sec4}%sec 4

Consider\pageoriginale now the amalgam of any two groups with an amalgamated
subgroup, say $H$. We have already seen that if $H$ is central both in
$A$ and in $B$, then the permutational product does not depend upon
the transversals of $H$ chosen in $A$ and $B$. We now prove that if
$H$ is central in $A$, then the permutational product is independent
of the transversal of $H$ in $B$ we choose to form the product. More
precisely we have 
\begin{theorem}\label{chap10:sec4:thm3}%the 3
  Let $H \le $ center $(A), S$ a transversal of $H$ in $A$.If $T, T'$
  are any two transversals of $H$ in $B$, then the permutational
  product $P$ on the set $K = S \times T \times H$ and the
  permutational product $P'$ on $K' = S \times T' \times H$ are
  isomorphic.  
\end{theorem}

\begin{proof}
  Let $\rho'(a)$, $\rho(b) \,a \in  A$, $b \in  B$, denote
  the permutations on the set $K'$ corresponding to permutations
  $\rho(a), \rho(b)$ on the set $K$. Consider the mapping $\varphi$ of
  $K$ into $K'$,  
  $$
  (s, t, h)^{\varphi} = (s, t', h'), s \in  S, h, h'
  \in  H, t, \in  T, t' \in  T' 
  $$
  defined by  
  $$
  t' h' = th.
  $$
\end{proof}

It is obvious that $\varphi$ is $1-1$ and onto and 
$$
\displaylines{\hfill 
  (s, t', h')^{\varphi - 1} = (s, t, h), s \in  S, t \in 
  T, t' \in  T', h, h' \in  H, \hfill \cr
  \text{where} \hfill th = t' h'.\hfill }
$$

Now\pageoriginale for $a \in  A$, let us compute
$$
(s, t', h')^{\varphi^{-1}}(a \varphi), (s, t', h') \in  K'.
$$

We have
\begin{align*}
  (s,t', h')^{\varphi ^{-1}} & = (s, t, h), ~\text{where} \\
  th & = t'h',t \in  T, h \in  H;~\text{ and} \\
  (s,t,h)^{\varphi(a)} & = (s_1,  t_1, h_1), ~\text{where} \\
  s_1 h_1 & = sha, t = t_1,  s_1 \in  S, h_1 \in  H; ~\text{and} 
\end{align*}
finally,
\begin{align*}
  (s_1, t_1, h_1)^{\varphi} &= (s_1, t'_1,  h'_1), ~\text{where} \\
  t'_1 h'_1 & = t_1 h_1,  t'_1 \in  T',  h'_1 \in  H.
\end{align*}

Now,
$$
t'_1 h'_1 = t_1 h_1 =th_1 = th.h^{-1}h_1 = t' h' h^{-1} h_1.
$$

Therefore,\pageoriginale 
$$
t'_1 = t',  h'_1 = h' h^{-1} h_1.
$$

Using the hypothesis that $H \le $centre $(A)$, we get
\begin{align*}
  (sa) h & = sha = s_1 h_1; ~\text{that is} \\
  sa & = s_1 h_1 h^{-1} ~\text{and} \\
  sh'a = (sa)h' & = (s_1 h_1 h^{-1})h' = s_1 (h_1 h^{-1} h') = s_1 h'_1.
\end{align*}

Thus 
$$
(s,t',h')^{\varphi^{-1} \rho (a) \varphi} = (s_1, t'_1,  h_1) = (s_1,
t', h'_1) = (s, t', h')^{\rho'(a)}. 
$$

As this is true for all $(s, t',  h') \in  K'$, we have
$$
\varphi^{-1} \rho (a) \varphi = \rho' (a).
$$

Now consider $\varphi^{-1} \rho(b) \varphi$, for $b \in  B$. We have
$$
\displaylines{\hfill
  (s, t' h')^{\varphi -1} = (s,t,h), th = t' h';\hfill \cr
  \text{and} \hfill (s, t, h)^{\rho(b)} = (s,t_1, h_1),\hfill \cr 
  \text{where} \hfill 
  t_1 h_1 = thb, t_1 \in  T, h_1 \in  H: \hfill \cr
  \text{and} \hfill (s, t, h_1)^{\varphi} = (s, t'_1,  h'_1),\hfill
  \cr 
  \text{where} \hfill  t'_1 h'_1 = t_1 h_1,  t_1 \in  T',  h'_1
  \in  H.\hfill }  
$$

But,\pageoriginale
$$
t' h' b = thb = t_1 h_1 = t'_1 h'_1.
$$

Therefore, 
$$
(s,t', h')^{\varphi -1 \rho(b) \varphi} = (s, t'_1, h'_1) = (s, t', h')^{\rho(b)}.
$$

Again as this is true for all $(s, t', h') \in  K'$, we get
$$
\varphi^{-1}\rho(b) \varphi = \rho'(b).
$$

It is obvious that the mapping $\eta$ of $\rho(A) \cup \rho (B)$ into
$\rho'(A) \cup \rho'(B)$ defined by 
$$
\rho (x)^\eta = \varphi^{-1} \rho (x) \varphi, x \in  \rho (A) \cup \rho (B),
$$
is $1 -1 $ and 'onto '. Further if 
$$
u (\rho (x_1),  \ldots,  \rho(x_n) = L),
$$
is a relation in $P_1$, with
$$
x_i \in  \rho(A) \cup \rho(B), i=1, \ldots, n,
$$
then
\begin{multline*}
  u(\rho (x_1)^\eta, \ldots,  \rho (x_n^\eta) = (u(\rho (x_1), \ldots, 
  \rho (x_n))^\eta) = \varphi^{-1}\\ 
  u (\rho(x_i), \ldots,  \rho
  (x_n)) \varphi = \varphi^{-1} \ell \varphi = L. 
\end{multline*}

Therefore\pageoriginale by von Dyck's Theorem, $\eta$ can be extended to an
isomorphism of $P$ onto $P'$, as $\rho(A) \cup \rho (B)$ and $\rho
'(A) \cup \rho' (B)$ generate $P$ and $P'$ respectively. This proves
the theorem.	 

\section{}\label{chap10:sec5}%sec 5

We shall now consider questions of the form:

If the groups $A$ and $B$ have the property $\mathcal{P}$, can the
amalgam be embedded in a group $P$ with the property $\mathcal{P} ?$ 

Let $\mathcal{P}$ be a property of groups. We say that a group $G$ has
the property $\mathcal{P}$ \textit{locally} if every finite set of
elements of $G$ is contained in a subgroup of $G$ having
$\mathcal{P}$. In particular, a group $G$ is \textit{locally finite}
if every finitely generated subgroup of $G$ is finite. Similarly we
can speck if \textit{locally soluble} and \textit{locally nilpotent }
groups. 

A locally finite group is obviously periodic. For a long time nothing
was known about the converse of this statement; but the recent results
of Novikov provide example of periodic groups that are not locally
finite. 

Consider an amalgam of groups $A, B$ with an amalgamated subgroup
$H$. We ask if $A$ and $B$ are locally finite, can the amalgam be
embedded in a locally finite group? The answer to this question,\pageoriginale in
general, is `no'. But if we impose certain `good' conditions on $H$
such an embedding can be achieved. The answer to the above question is
`yes' if $H$ is finite or $H$ is central both in $A$ and $B$. (See
Theorems \ref{chap10:sec5:thm4} and \ref{chap10:sec5:thm5}.) If $H$ is central only in $A$ or $B$,  the
answer is not completely known; for a partial result, see the end of
this Chapter.  

We can repeat the same question replacing ``locally finite''  by
``periodic''; that is, we ask: if $A$ and $B$ are periodic, can the
amalgam be embedded in a periodic group? Again, in general, the
answer is 'no'. If $H$ is central in both $A$ and $B$, the answer is
'yes'. Nothing is known in the case when $H$ is finite or when $H$ is
central only in $A$ or $B$. 

We now give an example to show that if $A$ and $B$ are locally finite,
the amalgam need not even be embeddable in a periodic group. 

Let $C$ be a periodic abelian group in which the orders of the
elements is unbounded. For instance we can take $C$ to be the Prufer
$p^\infty$ -group. Let  
$$
H = C \times C.
$$

Take
$$
\displaylines{\hfill 
  A = gp (H, a; a^4 = 1, (c,d)^a = (d^{-1}, c), ~\text{for all}~ (c,d)
  \in  H) \hfill \cr
  \text{and}\hfill  
  B =  gp (H, b; b^3 = 1, (c,d)^b = (c^{-1}d, c^{-1}), ~\text{for all}
  (c,d) \in  H).\hfill } 
$$

$A$\pageoriginale is the splitting  extension of $H$ by the cyclic group (of order
$4$) generated by a; similarly, $B$ is the splitting extension of $H$
by the cyclic group (of order $3$) generated by $b$. It is not
difficult to show (cf. the lemma in the next section) that an
extension of a locally finite group by a locally finite group (or as
we also say a locally finite-by-locally finite group) is itself
locally finite. Thus $A$ and $B$ are locally finite. Their
intersection is, of course, 
$$
A \cap B = H.
$$

Let $P$ be any group embedding the amalgam, consider the element
$$
p = ab \in  P.
$$

We have for any $(c, d) \in  H$, 
$$
(c, d)^p = (c, d)^{ab} = (d^{-1}, c)^b = (dc, d).
$$

It is easy to verify that 
$$
(c, d)^{p^n} = (d^n c,  d), ~\text{for}~ n=1,2,3, \ldots. 
$$

If\pageoriginale $p$ were of finite order, say $m$, then 
$$
(c, d)^{p^m} = (d^m c, d) = (c,d);
$$
that is,
\begin{align*}
  d^m c & = : \text {that is} \\
  d^m & =, ~\text{for all}~ d \in  H.
\end{align*}

This contradicts our choice of $H$. Therefore $P$ is not periodic.

\section{}\label{chap10:sec6}%sec 6

In this section we shall give two sufficient conditions for the
amalgam of two locally finite groups $A$ and $B$ with an amalgamated
subgroup $H$ to be embeddable in a locally finite group. 

\begin{theorem}\label{chap10:sec5:thm4}%the 4
  The amalgam of two locally finite groups $A$ and $B$ with an
  amalgamated subgroup $H$ is embeddable in a locally finite group if
  $H$ is central both in $A$ and $B$. 
\end{theorem}

\begin{proof}
  Since 
  $$
  H \le ~\text{centre}~ (A), H \le ~\text{centre}~ (B),
  $$
  the generalised direct product $P$ of the amalgam exists. We claim
  that $P$ is locally finite. One can prove this directly. But, we
  shall deduce it from a more general lemma. 
\end{proof}

\begin{lemma*}
  An extension of a locally finite group by a locally finite group is
  a locally finite group. 
\end{lemma*}

\begin{proof}
  Let\pageoriginale $P$ be an extension of a locally finite group $A$ by a locally
  finite group $B$, so that  
  $$
  A \triangle P,  P/A \cong B.
  $$

  Let $p_1, \ldots,  p_n$ be arbitrary elements of $P$, where $n$ is
  any positive integer and 
  $$
  G = gp(p_1, \ldots,  p_n).
  $$
\end{proof}

Consider the canonical mapping $\varphi$ of $P$ onto $P/A$. We have
$$
|G^\varphi | = | gp (p^{\varphi}_1, \ldots,  p^{\varphi}_n) | < \infty
,\text{ since} 
$$
$P/A$ is isomorphic to $B$ and thus locally finite. If $\varphi_0$ is
the restriction of $\varphi$ to $G$, we have 
$$
\varphi^{-1}_0 \bigg\{ 1 \bigg \} = G \cap A, G^{\varphi_0} = G^{\varphi}.
$$

Therefore,
$$
G/G \cap A \cong G^{\varphi}.
$$

Now since $G$ is finitely generated and $G \cap A$ has finite index in
$G$, by a theorem of Schreier (1927, cf, e.g Kurosh 1956, p. 36)
also $G \cap A$ is finite generated. Therefore the local finiteness of
$A$ implies that the group $G \cap A$ is finite. Now, since $G \cap A$
and $G / G \cap A$ are finite, $G$ itself is finite. This prove that
$P$\pageoriginale is locally finite. Now to complete that proof of
Theorem \ref{chap10:sec5:thm4}, we
have only to remark that the generalised direct product $P$ of the
amalgam is an extension of $A$ by a factor group of $B$ (namely by
$B/H$). 

\begin{theorem}\label{chap10:sec5:thm5}%the 5
  The amalgam of locally finite groups $A$ and $B$ with an amalgamated
  subgroup $H$ is embeddable in a locally finite group if $H$ is
  finite. 
\end{theorem}

\begin{proof}
  Choose transversals $S, T$ of $H$ in $A$ and $B$ respectively and
  form the permutational product $P$ of the amalgam on the set	$K = S
  \times T \times H$. Let, 
  $$
  G = gp(p_1,  \ldots,  p_r), p_i \in  P, i=1, \ldots.  r
  $$
  be any finitely generated subgroup of $P$. Each $p_i$ is a word in
  the elements of $\rho (A)$ and $\rho(B)$. Let $\rho(a_i), i = 1,
  \ldots,  m$ and  
  $$
  \rho (b_i), i=1, \ldots, \ell,  a_i \in  A, b_i \in  B
  $$
  occur when $p_i$ are expressed as words in the elements of $\rho(A)$
  and $\rho(B)$. Let, 
  \begin{align*}
    A_1 & = gp (a_1, \ldots,  a_m,  H) \text{and }\\
    B_1 & = gp (b_1, \ldots,  b_m,  H); \text{so that} 
  \end{align*}
  $$
  G = gp(p_1, \ldots,  p_r) \le gp (\rho(A_1), \rho(B_1), \rho(H)) = P_1 (say).
  $$
\end{proof}

Now\pageoriginale since $H$ is finite and $A$ and $B$ are locally finite, the groups
$A_1$ and $B_1$ are finite. Further 
$$
A_1 \cap B_1 = H.
$$

We now define $K_{ab} \subseteq K, a \in  A, b \in  B$ by
$$
K_{ab} = \bigg\{ (s,t,h) \big | s \in  aA_1,  t \in  b
B_1 \bigg\} 
$$ 
Since $A_1, B_1$ and $H$ are finite, each $K_{ab}$ is finite; in fact,
$$
\big|K_{ab}\big| = \big| S \cap aA_1 \big| \big| T \cap bB_1 \big|
\big| H \big| =\big| A_1:H \big| \big| B_1:H \big| \big|H\big| =
\frac{\big|A_1\big|~ \big|B_1\big|}{\big|H\big|}. 
$$

Further for $a, c \in  A, b, d \in  B$ either
\begin{gather*}
  K_{ab} \cap K_{cd} = \phi \text{or} \\
  K_{ab} = K_{cd}.
\end{gather*}

For, if $(s, t, h) \in  K_{ab} \cap K_{cd}$, then
$$
\displaylines{\hfill 
  s \in  aA_1 \cap cA_1,  t \in  bB_1 \cap dB;\hfill \cr
  \text{hence} \hfill  aA_1 = cA_1, bB_1 = dB_1,\hfill \cr
  \text{and}\hfill K_{ab}=K_{cd}.\hfill }
$$

Now\pageoriginale since every $(s,t,h)\in  K_{st}$, it follows that
$K=\bigcup\limits_{a \in  A,b \in  B}K_{ab}$. We claim that
every $K_{ab}$ admits $P_1$. For let $(s,t,h)\in  K_{ab}$;
then, for $i=1, \ldots, m$, 
\begin{align*}
  (s,t,h)^{\rho (a_i)}& =(s_1,t_1,h_1) \text{ where}\\
  s_1 h_1 & = sha_i,t_1=t;
\end{align*}
Thus
$$
\displaylines{\hfill 
  s^{-1}s_1=h a_i h_1^{-1}\in  A_1,\hfill \cr
  \text{and}\hfill sA_1=s_1A_1.\hspace{1.2cm}\hfill }
$$

But $(s,t,h)\in  K_{ab}$. Therefore
$$
s_1A_1=sA_1=aA_1; \text{ that is}, s_1 \in  aA_1.
$$

Moreover, $t_1=t \in  bB_1$. Hence
$(s,t,h)^{\rho(a_i)}\in  K_{ab}$. Similarly it can be proved
that 
$$ 
(s,t,h)^{\rho(a_i)}\in  K_{ab}, \text{ for every
}(s,t,h)\in  K_{ab}. 
$$

It\pageoriginale is also obvious that, for every $h' \in  H$,
$$
(s,t,h)^{\rho(h')}\in  K_{ab}, (s,t,h)\in  K_{ab}.
$$

Thus for every $a \in  A,b \in  B$, the elements of
$P_1$ restricted to $K_{ab}$ are permutations of the set
$K_{ab}$. Hence $P_1$ is a subgroup of the Cartesian product of
symmetric groups of permutations on the sets $K_{ab}$. Now since all
of the sets $K_{ab}$ have the same cardinal, the group $P_1$ can be
regarded as a subgroup of a Cartesian power of the group $S(F)$
where $S(F)$ is the
symmetric group of permutations on a set $F$ of cardinal
$|K_{ab}|$. Now the group $P_1$ is finitely generated. The following
lemma proves that the group $P_1$ is finite. 

\begin{lemma*}
  Let $E$ be a finite group, $Y$ any set and $Q$ a finitely generated
  subgroup of $E^\gamma$. Then $Q$ is finite. 
\end{lemma*}

\begin{proof}
  Let $Q=gp(q_1, \ldots, q_n)\in  E^Y$, with $q_i \in 
  E^Y$. Consider the $n$-tuples 
  $$
  (q_1(y), \ldots, q_n(y)),y \in  Y.
  $$

  Since $F$ is finite, there can only be a finite number of distinct
  such $n$-tuples. In fact the number $N$ of distinct n-tuples cannot
  exceed $|E^n|$. Let $y_1,\ldots, y_N \in  Y$. be such that 
  $$
  (q_1(y), \ldots, q_n(y)),i=1,\ldots,N
  $$
  are\pageoriginale $N$ distinct $n$-tuples. Let
  $$
  Y_0= \bigg \{ y_1,\ldots, y_N \bigg \}
  $$
  Consider the mapping $\theta$ of $Q$ into $E^{Y_0}$ defined by 
  $$
  q^\theta=q_0,
  $$
  where $q_0$ is the restriction of $q$ to $Y_0$. It is easy to verify
  that is a homomorphism. In fact $\theta$ is a homomorphism. For let
  $q \in  Q$ belongs to the kernel of $\theta$, that is
  $q^\theta=e_0$, where $e_0$ is the neutral element of $E^{\gamma_0}$;
  that is  
  $$
  e_0(y_i)=1, i=1, \ldots, N.
  $$
\end{proof}

If $q=u(q_1, \ldots, q_n)$, then
$$
q(y)=u(q_1(y), \ldots, q_n (y)).
$$

Now there exists a $y_j,1\leq j \leq n$ such that
$$
q_i(y)=q_i(y_j),i=1, \ldots, n.
$$

Now 
$$
u(q_1(y_j),\ldots,q_n(y_j))=1, \text{ since } q^\theta =e_0.
$$

Therefore\pageoriginale
$$
q(y)=u(q_1(y_j),\ldots,q_n(y_j))=1;
$$
as $y$ was an arbitrary element of $Y$, we see that $q$ is the unit
elements of $E^Y$. Thus the kernel of $\theta$ is trivial, in other
words, $\theta$ is a homomorphism. Now $\theta$, being isomorphic to a
subgroup of the finite group $E^{Y_0}$, is finite. This completes the
proof of the lemma. 

Thus we have proved that every finitely generated subgroup $P_1$ of
$P$ is finite; that is $P$ is locally finite. 

Observing that in the proof of the above theorem the transversal $S,T$
were arbitrary we have: 
\begin{coro*}
  Every permutational product of the amalgam of two locally finite
  groups with an amalgamated subgroup is locally finite if the
  amalgamated subgroup is finite. 
\end{coro*}

If $A$ and $B$ are locally finite and if $H$ is central in $A$ and of
countable index in $A$, it can be proved that there is an embedding
(in a permutation product with a suitable transversal $S$ of $H$ in
$A$) in a locally finite group. We shall, however, not prove this. 

