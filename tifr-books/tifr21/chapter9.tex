
\chapter{Generalised Free Products of Groups with
  Amalgamations}\label{chap9} %chap IX 

\section{}\label{chap9:sec1} %1

In\pageoriginale this chapter we shall consider the question under what conditions a
given family of groups with prescribed intersections can be embedded
in a group. More precisely the problem is the following. 

Let $\big\{ G_i \big \}_{i \in I}$ be a family of groups and
$\big\{ H_{i j} \big \}_{i \in I}$ be a given family of
subgroups of $G_i$, for every $i \in I$. We then ask: does
there exist a group $P$ and monomorphism $\theta_i$ of $G_i$ into $P$
for every $i \in I$ with the property 
$$
G^{\theta_i}_i \cap G_j^{\theta_j} = H^{\theta_i}_{ij} =
H^{\theta_j}_{ji}, \text{ for all } i,j \in I? 
$$

Certain conditions are necessary for the existence of such a group.

First we note that 
$$
H_{i i} = G_i.
$$

Since $H_{ij}$ and $H_{ji}$ are to be mapped onto the same subgroup of
$P$, they must be isomorphic. In fact $\varphi_{ij}$, the restriction
of $\theta_i \theta_j^{-1}$ to $H_{ij}$, must be an isomorphism of
$H_{ij}$ onto $H_{ji}$. It is immediate that  
$$
\varphi_{ij} \varphi_{ji} = \ell,
$$
the\pageoriginale identity map of $H_{ij}$ onto itself. Further,
$$
G^{\theta_i}_i \cap G_j^{\theta_j} \cap G^{\theta_k}_k =
H^{\theta_i}_{ij} \cap H^{\theta_i}_{ik} = H^{\theta_i}_{ji} \cap
H^{\theta_j}_{jk} = H^{\theta_k}_{ki} \cap H^{\theta_k}_{kj}. 
$$

Thus the three intersections,
$$
H_{ij} \cap H_{ik}, H_{ji} \cap H_{jk}, H_{ki} \cap H_{kj}
$$
must be mapped onto one and the same subgroup
$$
G^{\theta_i}_i \cap G^{\theta_j}_j \cap G^{\theta_k}_k.
$$

Now
\begin{multline*}
  (H_{ij} \cap H_{ik})^{\varphi_{ij}} =(H_{ij} \cap H_{ik})^{\theta_i
    \theta^{-1}_j}= (H^{\theta_i}_{ij} \cap
  H^{\theta_i}_{ik})^{\theta^{-1}_j}\\ 
  = \left(H^{\theta_{ji}^{\theta_j}} \cap
  H^{\theta_{jk}^{-1}}\right)^{\theta_{j}^{-1}} =  H_{ji} \cap H_{jk}. 
\end{multline*}

The mapping $\varphi_{ij} \varphi_{jk}$ is an isomorphism of $H_{ij}
\cap H_{ik}$ onto $H_{ij} \cap H_{ik}$; in fact 
$$
\varphi_{ij} \varphi_{jk} = \varphi_{ik} \text{ on } H_{ij} \cap H_{ik}.
$$

We can similarly write down further necessary conditions which arise
from the fact that the the intersection of more than three groups
$H_{ij}$, $H_{ik}$, $\ldots$\pageoriginale are to be mapped onto one and the same
intersection of groups $G^{\theta_i}_i,  G^{\theta_j}_j, \ldots $. But
once the necessary conditions in terms of the intersection of three
groups are satisfied other such conditions involving more than three
groups are automatically satisfied. For instance, say four groups
$G_i$, $G_j$, $G_k$, $G_{\ell}$. Then  
\begin{align*}
  (H_{ij} \cap H_{ik} \cap H_{i \ell})^{\theta_i} &= H^{\theta_i}_{ij}
  \cap H^{{\theta_i}}_{ik} \cap H^{{\theta_i}}_{i \ell} \\
  &= (H^{\theta_i}_{ij} \cap H^{\theta_i}_{ik}) \cap
  (H^{\theta_i}_{ij} \cap H^{\theta_i}_{i \ell})\\ 
  &=(H^{\theta_i}_{ji} \cap H^{\theta_j}_{jk})  (H^{\theta_j}_{ji}
  \cap H^{\theta_j}_{i \ell})\\ 
  & = H^{\theta_i}_{ji} \cap H^{\theta_j}_{jk} \cap H^{\theta_j}_{j
    \ell}, \text{ and so on }. 
\end{align*}

It is easy to verify that 
$$
\varphi_{ij} \varphi_{ik} \varphi_{k \ell} = \varphi_{i \ell} \text{
  on } H_{ij} \cap H_{ik} \cap H_{ i \ell}. 
$$

Thus we have proved 
\setcounter{theorem}{0}
\begin{theorem}\label{chap9:sec1:thm1}%theo 1
  In order that $\big\{ G_i \big\}_{i \in I}$ be embeddable in
  a group with prescribed intersections $\big\{ H_{ij} \big\}_{(i, j)
    \in I \times I}$ it is necessary that there be
  isomorphisms $\varphi_{ij}$ of $H_{ij}$ onto $H_{ji}$ satisfying the
  following conditions. 
\end{theorem}

\begin{enumerate}[(1)]
\item $ \varphi_{ij} \varphi_{ji}= \ell$,\pageoriginale the identity
  map of $H_{ij}$ onto itself  
\item $\varphi_{ij}$ maps $H_{ij} \cap H_{ik}$ onto $H_{ji} \cap H_{jk}$
\item $\varphi_{ij} \varphi_{jk}= \varphi_{ik}$ on $H_{ij} \cap
  H_{ik}$, for all $i,j,k \in I$. 
\end{enumerate}

Here after we shall refer to the family of subgroups $\big\{ H_{ij}
\big\}_{(i, j) \in I \times I}$ satisfying the necessary
conditions of Theorem \ref{chap9:sec1:thm1} as the family of \textit{
  amalgamated subgroups}. 

Let $\big\{ G_{i} \big\}_{i \in I}$ be a family of groups with
amalgamated subgroups \break $\big\{ H_{ij} \big\}_{(i, j)  I \times I}$ and
et  
$$
G_i = gp (E_i ; R_i)
$$
be a presentation of $G_i$ with generators $E_i$ and a set of defining
relations $R_i$. Let  
\begin{align*}
  H_{ij} & = gp(D_{ij}), \text{ where }\\
  H_{ij} & = \big\{ d_{i j \nu} \big\},
\end{align*} 
$\nu$ running over some index set. Since $H_{ij}$ and $H_{ji}$ are
isomorphic, we can choose generators $D_{ij}$ in such a way that  
\begin{align*}
  D^{\varphi_{ij}}_{ij}& = D_{ji} \text{ and }\\
  d^{\varphi_{ij}}_{ij \nu}& = d_{i \nu}.
\end{align*}

Thus\pageoriginale $\nu$ runs over one and the same index set for $D_{ij}$ and
$D_{ij}$. Without loss of generality we can take  
$$
\bigcup_j D_{ij} \subseteq E_i.
$$

Now for every $i \in I$, we take a set $E^*_i$ with
$|E_i|=|E^*_i|;$ that is there is a $1-1$ and onto map $\theta^*_i$ of
$E_i$ onto $E^*_i$. Let $R^*_i$ be the set of all relations defined by 
$$
R^*_i = \Bigg\{ r \left(e^{\theta^*_i}_1, \ldots,  e^{\theta^*_i}_n\right) =1 \Big|
(r(e_1, \ldots,  e_n) = 1) \in R_i,  e_1, \ldots, e_n
\in E_1 \Bigg\} 
$$
Let 
$$
P^* = gp\left(\bigcup_{i \in I} E_i^{\theta^*_i}; \bigcup_{i
  \in I} R^*_i,  d^{\theta^*_i}_{i j \nu}= d^{\theta^*_i}_{i j
  \nu}, \text{ for all } i,j,  \nu \right) 
$$ 

We shall refer to the  collection $\big\{ G_{i} \big\}_{i \in
  I}$ with amalgamated subgroups $\big\{ H_{ij} \in \big\}_{(i, j)  I
  \times I}$ as an \textit{ amalgam }. If there exists a group $P$
embedding the family $\big\{ G_{i} \big\}_{i \in I}$ with
amalgamated $\big\{ H_{ij} \in \big\}_{(i, j)  I \times I}$, we say that
``$P$ embeds the amalgam''.

\begin{theorem}\label{chap9:sec1:thm2}%the 2
  If there exists a group $P$ embedding the amalgam, then the group
  $P^*$ also embeds the amalgam and if $\theta_i$ is the corresponding
  canonical monomorphism of $G_i$ into $P$, then there is a
  homomorphism $\varphi$ of $P^*$ into $P$, mapping  
  $$
  G^*_i = gp(E^{\theta^*_i}_i ) \leq P^*
  $$
  isomorphically\pageoriginale onto $G^{\theta^*_i}_i$ such that 
  $$
  (G^*_i \cap G^*_j)^{\varphi} = G_i^{{\theta_i}} \cap
  G^{\theta_j}_j = H^{\theta_i}_{ij} = H^{\theta_j}_{ji}. 
  $$ 
\end{theorem}

\begin{proof}
  Define the mapping $\varphi$ of $\bigcup \limits_{i \in I}
  E^{\theta^*_i}_i$ into $P$ by  
  $$
  (e^{\theta^*_i}_i)^{\varphi} = e^{\theta_i}_i, \text{ for } e_i
  \in R_i \text{ and } 
  $$
  where ${\theta_i}$ is the embedding monomorphism of $G_i$ into
  $P$. We claim that $\varphi$ can be extended to a homomorphism of
  $P^*$ into $P$. 
\end{proof}

For let 
$$
r_1(e^{\theta^*_i}_{i_{1}},\ldots,  e^{\theta^*_i}_{i_{n}})= 1 \text{
  we a relation in } R^*_i. 
$$

Then  
$$
\displaylines{\hfill 
  \left(r_i \left(e^{\theta^*_i}_{i_{1}},\ldots, e^{\theta^*_i}_{i_{n}}\right)\right) =
  r_i \left(e^{\theta^*_i \varphi}_{i_{1}},\ldots,  e^{\theta^{*_i}
      \varphi}_{i_{n}}\right) = r_i \left(e^{\theta_i}_{i_{1}},\ldots
 ,  e^{\theta_i}_{i_{n}}\right)=1, \hfill \cr
  \text{since }\hfill 
  r_1(e_{i_{1}}, \ldots,  e_{i{n}}) = 1\hfill }
$$
is a relation in $R_i$ and $\theta_i$ is a monomorphism of $G_i$ into $P$.

Further,\pageoriginale
\begin{align*}
  d^{\theta^*_i}_{i j \nu}& = d^{\theta^*_j}_{i j \nu} \text{ implies }\\
  d^{\theta_i}_{i j \nu}& = d^{\theta_j}_{i j \nu}.
\end{align*}

For, \qquad \qquad \qquad $d^{\varphi_{ij}}_{i j \nu}  =
d^{\theta_i \theta_{j}^{-1}}_{i j \nu} = d_{j i \nu }$. 
\medskip

Thus the defining relations of $P^*$ go over to relations of $P$ upon
applying $\varphi$ and therefore $\varphi$ can be extended to a
homomorphism of $P^*$ into $P$, by von Dyck's Theorem. We shall denote
this homomorphism also by $\varphi$. Again an application of von
Dyck's theorem shows that $\theta^*_i$, the mapping of $E_i$ into
$P^*$, can be extended to a homomorphism of $G_i$ into $P^*$, which we
also denote by $\theta^*_i$. It is obvious that  
$$
G^{\theta^*_i}_i = G^*_i.
$$

We claim that (since $P$ embeds the amalgam) $\theta^*_i$ is a
monomorphism of $G_i$ into $P^*$. By the definition of  
$$
\theta^*_i \varphi = \theta_i \text{ on } G_i.
$$

Therefore the kernel of $\theta^*_i$ is contained in that of
$\theta_i$. But $\theta_i$, being a monomorphism has trivial
kernel. Therefore $\theta^*_i$ has a\pageoriginale trivial kernel; that is
$\theta^*_i$ is a monomorphism of $G_i$ into $P^*$. To show $P^*$
embeds the amalgam we have only to prove that  
$$
G^{\theta^*_i}_i \cap G^{\theta^*_j}_j = H^{\theta^*_i}_{ij} =
H^{\theta^*_j}_{ji}, i,j \in I. 
$$

Now,
\begin{align*}
  H^{\theta^*_i}_{ij} & = \left(gp( \big\{ d_{i j \nu} \big\})\right)^{\theta^*_i}
  = gp \left( \big\{ d^{\theta^*_i}_{i j \nu} \big\}\right)\\ 
  & = gp \left( \big\{ d^{\theta^*_j}_{ j i \nu} \big\}\right) =
  \left(gp( \big\{ d_{i j 
    \nu} \big\})\right)^{\theta^*_j} = H^{\theta^*_j}_{ji} ; \text{ and }\\ 
  H^{\theta^*_i}_{ij} & = H^{\theta^*_j}_{ji}  \subseteq
  G^{\theta^*_j}_{i} \cap  G^{\theta^*_j}_{j}. 
\end{align*}

Let $h^* \in G^{\theta^*_i}_{i}  \cap G^{\theta^*_j}_{j}$. Then,
$$
h^{*^ \varphi} \in ( G^{\theta^*_i}_{i} \cap
G^{\theta^*_j}_{i})^{\varphi} \subseteq G^{\theta^*_i \varphi}_{i}
\cap G^{\theta^*_j \varphi}_{j} = G^{\theta^i}_{i} \cap
G^{\theta_j}_{j}. 
$$

Now $P$ embeds the amalgam. Hence 
$$
h^{*^ \varphi} \in G^{\theta_i}_{i} \cap G^{\theta_j}_{j} =
H^{\theta_i}_{ij} = H^{\theta_j}_{ji} 
$$

Therefore there is a unique $h \in H_{i j}$ such that 
$$
h^{*{^\varphi}} = h^{\theta_i} \in H^{\theta_i}_{ij}.
$$

Now\pageoriginale ${\theta^*_i}$ is an isomorphism of $G_i$ onto $G^*_i$. Therefore 
\begin{gather*}
h^{\ast^{{\theta_i}^{\ast^{-1}}}} = h ; \text{ that is }\\
  h^{\theta^*_i} \in H^{\theta^*_i}_{ij} ; \text{ that is to say }\\
  G^{\theta^*_i}_{i} \cap G^{\theta^*_j}_{j} \subseteq H^{\theta^*_i}_{ij}
\end{gather*} 

Combining this with the reversed inclusion which we have already
proved we have  
$$
G^{\theta^*_i}_{i} \cap G^{\theta^*_j}_{j} = H^{\theta^*_i}_{ij} =
H^{\theta^*_j}_{ji}. 
$$

This proves that $P^*$ embeds the amalgam. Further $\varphi$
restricted to $G^{\theta^*_i}_{i}$ is precisely the mapping
${\theta^{*-1}_i} {\theta_i}$ and hence is $1-1$; that is the mapping
$\varphi$ restricted to $G^{\theta^*_i}_{i}$ is a monomorphism. 

We shall refer to $P^*$ as the ``canonic group'' of the amalgam and
${\theta^*_i}$ as the ``canonic homomorphism'' of $G_i$ into $P*$. 

If $P^*$ embeds the amalgam, we call $P^*$ the \textit{ generalised
  free product } of the amalgam. The name `` generalised free
product'' is justifiable as there is a homomorphism of $P^*$ into any
group that\pageoriginale embeds the amalgam. 

\section{}\label{chap9:sec2}%2

If all the amalgamated subgroups $H_{ij}$ are trivial, then the
cartesian product of $\{G_i \}$ is one that embeds the
amalgam. Therefore the corresponding $P^*$ also embeds the amalgam;
this is  known as the \textit{ free product } of the family of groups
$\{G_i \}$. Free products occur naturally in applications of group
theory. For instance the free group  
$$
G = gp(E ; \phi)
$$
is the free product of infinite cyclic groups $\big\{ gp(e_i)
\big\}_{e_i \in E}$. Another example of a free product is the
following. Consider the set $M$ of all linear transformations of the
form 
$$
z^{\varphi}= \frac{az+b}{cz+d} \text{ with } ad-bc= \pm 1, \text{
  where } 
$$
$a,b,c,d$ are rational integers. It is not difficult to verify that
$M$ is a group with composite of maps as the multiplication. The group
$M$ is known as the \textit{ modular group }. This group $M$ is the
free product of two cyclic groups of order $2$ and $3$ generated by
$\alpha $ and $\beta$ respectively where   
\begin{align*}
  z^{\alpha} & = - \frac{1}{z}; \alpha^2 = \ell \text{ and }\\
  z^{\beta } & = - \frac{1}{z-1}; \beta^3 = \ell \text{ thus }\\
  M & = gp(\alpha, \beta; \alpha^2 = \beta^3 = 1).
\end{align*}

(See\pageoriginale Coxeter and Moser $1957 pp. 85-88$; the group is there called the
\textit{ projective } modular group in $2$- dimensions.) 

The generalised free products, too, appear naturally in topology. For
instance the clover knot group (i.e. the fundamental group of the
residual space in $S^3$ of the clover knot) is the generalised free
product of two infinite cyclic groups say gp(a) and gp(b) with
gp($a^2$) and $gp(b^3)$ as the amalgamated subgroups. More generally
any torus knot group is the generalised free product of two infinite
cyclic groups gp(a) and ga(b) with gp($a^m$) and $gp(a^n)$
amalgamated. 

\section{}\label{chap9:sec3}%3

We can make examples of amalgams which cannot be embedded in any
group. Consider the following amalgam. 

Let, 
$$
G_1 = gp(g_1, h_1, k_1 ; h^2_1 = k^2_1=1, h_1, k_1 =1,  g^2_1 =
1,h^g_1 =k_1). 
$$

This is known the ``dihedral group'' of order 8 and it is precisely
the group of automorphisms of a square. (One can describe $G_1$ also
as the wreath product of two cyclic groups of order 2). We select the
alternating group $A_4$ as $G_2$, presented as follows. 
$$
G_2 = gp(g_1, h_2, k_2; g^3_2 =1,  h^{g_2}_2 = k_2, k^{g_2}_2 = h_2 k_2.
$$

We\pageoriginale take $C_6$, the cyclic group of order $6$, as $G_3$ and give it
rather an ``unorthodox'' presentation; that is  
$$
G_3 = gp( c,d; c^2 = d^3 = 1; [c, d ] =1)
$$
The following we take as the amalgamated subgroups.
\begin{alignat*}{4}
  H_{12}& = gp(h_1, k_1) \leq G_1,&\qquad & H_{21} = gp(h_2, k_2) \leq G_2\\
  H_{13} & = gp(g_1) \leq G_1,  && H_{31} = gp(c) \leq G_3\\
  H_{23} & = gp(g_2) \leq G_2,  && H_{32} = gp(d) \leq G_3.
\end{alignat*} 

The amalgamating isomorphisms $\varphi_{ij}$ are to map $h_1$ on $h_2,
k_1$ on $k_2, g_1$ on $c, g_2$ on $d$. 

It this amalgam can be embedded in a group, then the canonic group
$P^*$ also embeds the amalgam. Now $P^*$ is the group generated by  
$$
g_1, h_1, k_1, g_2, h_2, k_2, c, d
$$
with defining relations consisting of the defining relations of $G_1,
G_2, G_3$ and the following amalgamating relations. 
$$
h_1 = h_2, k_1= k_2, g_1=c, g_2=d.
$$

Now in $P^*$, we have,
\begin{align*}
  h_1 & = h^{[c, d]}_1 = h^{[g_1, g_2]}_1= h^{g^{-1}}_1 g^{-1}_2 g_1
  g_2 = k^{g^{-1}_1 g_1 g_2}_1 = k^{g^{-1}_1 g_1 g_2}_2\\ 
  & = h^{g_1 g_2}_2 = h^{g_1 g_2}_1 = k^{ g_2}_1 = k^{ g_2}_2 = h_2 k_2 = h_1 k_2 ; 
\end{align*}
that is\pageoriginale 
\begin{align*}
  k_2& = 1 \text{ and therefore }\\
  h_2 & = 1.
\end{align*}

Hence,
$$
h_1 = h_2 = k_1 = k_2 =1.
$$

Thus,
$$
P^* \cong C_0
$$
and therefore cannot embed the amalgam ; that is to say the amalgam we
have considered cannot be embedded in any group. Note that the amalgam
satisfies the necessary conditions. Thus an amalgam is not always
embeddable in a group.  

\section{}\label{chap9:sec4}%4

We can impose certain conditions on the amalgam to make it embeddable
in a group. A special case when all the amalgamate subgroups coincide
with a single group was first studied by Schreier (1927). 

\begin{theorem}[Schreier, 1927]\label{chap9:sec4:thm3}%the 3
  If\pageoriginale all the amalgamated subgroups coincide (with a single group) then
  the amalgam is embeddable. 
\end{theorem}

Before proceeding to prove the theorem we make a definition. Let $G$
be any group and $H \leq G$. Then $G$ can be written as the union of
disjoint left cosets of $H$. We choose one representative for each of
these left cosets. The set of all such representatives is called a
\textit{ left transversal } of $H$ in $G$. Similarly a right
transversal of $H$ in $G$ can be defined. If $S$ is a left transversal
of $H$ in $G$, then  
\begin{gather*}
  G= SH, \text{ with the property }\\
  S' \subseteq S, G = S' H \text{ implies } S' = S.
\end{gather*}

(We call $|S|$ the index of $H$ in $G$; notation $|S| =|G :
H|)$. Every element $g \in G$ can be written as  
$$
g = sh, \text{ with } s \in S, h \in H.
$$

Moreover, this representation is unique. For if 
$$
g= sh = s'h', s \neq s', s,s' \in S, h, h' \in H,
$$
then 
\begin{gather*}
  s^{-1}s' = hh'^{-1} \in H; \text{ that is }\\
  sH= s'H; \text{ and therefore by our }
\end{gather*}
choice\pageoriginale of $S$,
$$
s = s', h = h'.
$$

It is often convenient to choose the transversal in such a way that
$H$ is represented by 1. 

We shall give two proofs of Theorem \ref{chap9:sec4:thm3}. The second proof will be
given in the next chapter, and applied there to give further embedding
theorems. 

\noindent \textbf{First proof of Theorem 3.} Let $\{ G_i \}_{i
  \in I}$ be a family of groups and let $G_i$ contain an
isomorphic copy of a given group $H$, for every $i \in
I$. Without loss of generality we can think of all these isomorphic
copies as identified with each other, i.e., 
$$
H \leq G_i, i \in I.
$$

We call the $G_i$ the {\em constituents } of the amalgam. We choose a
left transversal $S_i$ of $H$ in $G_i$, for each $i \in I$, and
we here represent $H$ always by $1$; thus $1 \in S_i$, for all
$i \in I$. Now we pick out certain words in the elements of
$G_i$. We call  
$$
w = s_1 s_2 \ldots s_n h 
$$
a {\em normal word} if it satisfies the following three conditions:
\begin{enumerate}[1)]
\item Each $s_{\nu}, \nu = 1, \ldots, n$, is a representative $\neq 1$
  belonging to one of the left transversals we have chosen, say
  $S_{i(\nu)}$; 
\item $i(\nu) \neq i (\nu + 1)$, for $\nu = 1, \ldots,n$, in other
  words, no two  consecutive\pageoriginale $s_{\nu}$ appearing in $w$
  belong to the same set of representatives. 
\item $h \in H$.
\end{enumerate}

We call $n$ the \textit{ length } of the normal word $w$ and denote it
by $\ell (w)$. In particular $n$ may be zero also. In fact  
$$
\ell (w) = 0
$$
if and only if $w \in H$. We denote by $w_o$ the normal word
consisting of the identity element alone. Let $W$ be the set of all
normal words. Consider the mapping $\rho (g)$ of $W$ into $W$, for all
$g \in \bigcup \limits_i G_i$, defined as follows. 

For $g \in G_k, k \in I$ and 
$$
w = s_1 s_2 \cdots s_n h \in W
$$
we put 
$$
w^{\rho (g) } = w',
$$
where $w'$ is defined as follows.
\begin{enumerate}[(i)]
\item If $n > 0, i(n) = k $; that is $s_n$ lies in the same group
  $G_k$ as $g$, then $s_n hg$ is a certain element of $G_k$ and are be
  uniquely written as  
  $$
  s_n hg = s' h', s' \in S_k, h' \in H.
  $$
\end{enumerate} 
 
We then put 
$$
\displaylines{\hfill  
  w' = s_1 s_2 \cdots s_{n-1}s' h', \text{ if } s' \neq 1\hfill \cr 
  \text{and} \hfill   
  w' = s_1 s_2 \cdots s_{n-1}h',  \text{ if } s' = 1.\phantom{and}\hfill }
$$

(ii)~\pageoriginale
If $n=0$ or $n > o$ and $i(n) \neq k$; that is if $s_n \notin G_k$, we
represent $hg \in G_k$ as  
  $$
  \displaylines{\hfill 
    hg = s' h', s' \in S_k, h' \in H\hfill \cr
    \text{and write}\hfill 
    w' = s_1 s_2 \cdots s_n s' h', \text{ if } s' \neq 1;\hfill \cr
    \text{and}\hfill  
    w' = s_1 s_2 \cdots s_n h' \text{ if } s' \neq 1.\hfill }
  $$

  Thus $w' = w^{\rho (g)}$ is defined for every $w \in W$ and
  it is easy to verify that $w'$ is again a normal word. If $g$ is
  contained in more than one constituents $G_i$, say $g \in G_j
  \cap G_k $, then $g \in H$ and we can define $w'$ according to
  $(i)$ or $(ii)$. Now, if  
  $$
  s_n hg = s' h', \text{ then }
  $$
  since $g \in H$,
  $$
  \displaylines{\hfill s_n  = s'\hfill \cr
    \text{and}\hfill gh = h'.\hfill }
  $$
  
  Therefore\pageoriginale according to $(i)$, we have
  $$
  w' = s_1 s_2 \cdots s_{n-1}s' h' =  s_1 s_2 \cdots s_{n} hg.
  $$
  
  On the other hand, computing $w'$ according to $(ii)$ we have, as 
  \begin{gather*}
    hg = h',\\
    w' = s_1 s_2 \cdots s_{n} h'=  s_1 s_2 \cdots s_{n}hg.
  \end{gather*}
  
  Thus we get the same $w'$ whichever way we compute it. 
  
  We shall now show that
  $$
  \rho (gg') = \rho (g) \rho (g'), \text{ for } g,g' \in G_k.
  $$
  
  Put
  $$
  w^{\rho (g)} = w',  w'^{\rho (g')} = w'', w^{\rho (gg')} = w^*.  
  $$

(1)~ If $n > 0, i(n) = k$; then
  $$
  s_n hg = \in G_k.
  $$
  
  We write 
  \begin{gather*}
    s_n hg = s' h',  s' \in S_k,  h' \in H   \text{ and }\\
    s' h' g' = s'' h'', s'' \in S_k, h'' \in H.
  \end{gather*}

  Now,\pageoriginale 
  $$
  \displaylines{\hfill 
    w'' = s_1 s_2 \cdots s_{n-1}s'' h'' \text{ if } s'' \neq 1 \hfill \cr
    \text{and} \hfill  
    w'' = s_1 s_2 \cdots s_{n-1} h'' \text{ if } s'' = 1.\hfill}
  $$

  On the other hand let 
  $$
  s_n h (gg') = s^* h^*, s^* \in S_k, h^* \in H.
  $$
  
  Then  
  $$
  \displaylines{\hfill 
    w^* = s_1 s_2 \cdots s_{n-1}s^* h^* \text{ if } s^* \neq 1 \hfill \cr
    \text{and} \hfill  
    w^* = s_1 s_2 \cdots s_{n-1} h^* \text{ if } s^* = 1.\phantom{and}\hfill }
  $$
  
  But, 
  $$
  s_n h (gg') = (s_n hg)g' = s' h' g' = s'' h''.
  $$
  
  Therefore,
  \begin{gather*}
    s^* = s'', h^* = h'' ; \text{ that is }\\
    w'' = w^*.
  \end{gather*}
  
  Notice\pageoriginale that it does not matter whether $s' = 1$ or not as $i (n-1)
  \neq k$ and in either case we have to consider $s' h' g'$ to compute
  $w''$.
 
(2)~ If $n=0 $ or if $n > 0, i(n) \neq k$, write
  $$
  \displaylines{\hfill 
  hg = s' h', s' \in S_k, h' \in H\hfill \cr
  \text{and} \hfill 
  s' h' g' = s'' h'', s'' \in S_k, h'' \in
  H.\phantom{and}\hfill }
  $$
  
  Then, 
  $$
  \displaylines{\hfill 
    w'' = s_1 s_2 \cdots s_{n}s'' h'' \text{ if } s'' \neq 1\hfill \cr
    \text{and} \hfill  
    w'' = s_1 s_2 \cdots s_{n} h'' \text{ if } s'' = 1.\phantom{and a}\hfill }
  $$

  On the other hand if we put 
  $$
  \displaylines{\hfill 
    h(gg') = s^* h^*, s^* \in S_k, h^* \in H, \hfill \cr
    \text{we have} \hfill 
    w^* = s_1 s_2 \cdots s_{n}s^* h^* \text{ if } s^* \neq 1\hfill \cr
    \text{and} \hfill  
    w^* = s_1 s_2 \cdots s_{n} h^* \text{ if } s^* = 1. \hfill }
  $$
  
  But,\pageoriginale 
  $$
  s^* h^* = h(gg') = (hg) g' = s' h' g' = s'' h''.
  $$
  
  Therefore
  \begin{align*}
    s^* = s'', h^* & = h''; \text{ that is }\\
    w^* & = w'.
  \end{align*}

  In this case also not does not matter whether $s' =1$ or $s' \neq 1$.
  
  Thus we proved that in both the cases
  $$
  w^{\rho (gg')} = \bigg\lgroup w^{\rho (g)}\bigg\rgroup^{\rho (g')}
  $$
  
  As this is true for all $w \in W$, we have
  $$
  \rho (gg') = \rho (g) \rho (g').
  $$

  Again, as this true for all $g, g' \in G_k$ we conclude that
  mapping $\rho_k$ of $G_k$ into the semigroup of the mappings of $W$
  into itself, defined by 
  $$
  g^{\rho_k} = \rho (g), \text{ for all } g \in G_k
  $$ 	
  is a homomorphism; and therefore the image $G_k^{\rho_k}$ is a
group. Hence every $\rho (g), g \in G_k$ has a two sided
inverse, that is $\rho (g)$ is a permutation of $W$, for all $g
\in G_k$. We claim that $\rho$ is an isomorphism\pageoriginale of $G_k$ onto
$G_k^{\rho _{k}}$. For, if 
\begin{gather*}
  \rho (g) = L,  g \in G_k, \text{ then }\\
  w_o ^{\rho (g)} = w_o
\end{gather*}

But,
\begin{gather*}
  w_o ^{\rho (g)} = sh, \text{ where }\\
  g = sh,  s \in S_k, h \in H.
\end{gather*}

Therefore, 
$$
\displaylines{\hfill 
  sh = 1; \qquad \text{ that is }\hfill \cr
  \hfill s= 1, h=1\hfill \cr
  \text{i,e.,}\hfill  g=1.\hfill }
$$

Hence the kernel of $\rho_k$ is trivial; that is to say, $\rho _k$
is an isomorphism of $G_k$ onto $G_k^{\rho_k}$. Let $\sum$ denote the
group of permutations of $W$ generated by the $G_i^{\rho i}, i
\in I$; that is  
$$
\sum = gp (\bigg \{ G_i^{\rho i}\bigg\}).
$$

By what we have just proved all groups $G_i$ are embedded in by the
isomorphism $\rho_i$. Now let 
$$
\displaylines{\hfill 
  g^{\rho_i} = g'^{\rho_k}, g \in G_i,  g' \in G_i,  g'
  \in G_k. i \neq k; \hfill \cr
  \text{that is} \hfill \rho (g) = \rho (g').\hfill }
$$

Let\pageoriginale 
\begin{gather*}
  g= sh \text{ with } s \in S_i, h \in H, \text{ and }\\
  g'= s' h' \text{ with } s' \in S_k, h' \in H.
\end{gather*}

Then,
$$
w_o^{\rho (g)} = sh \text{ or } h, \text{ according  as }s \neq 1
\text{ or } s=1. 
$$

Similarly,
$$
w_o^{\rho (g')} = s' h' \text{ or } h' \text{ according as }s' \neq 1
\text{ or }s' = 1. 
$$

But 
$$
w_o^{\rho (g)} = w_o^{\rho (g')}.
$$

Therefore
\begin{gather*}
  s= s' = 1, h = h' ; \text{ that is }\\
  g = g' = h.
\end{gather*}

Hence 
$$
g^{\rho_i} = g'^{\rho_k} \in H^{\rho_i} = H^{\rho_k},
$$

Thus,\pageoriginale
$$
G_k ^{\rho_k} \cap G_i^{\rho_i} = H^{\rho_i} = H^{\rho_k}, \text{ for
  all } i, k \in I. 
$$

This proves that the group $\sum$ embeds the amalgam under consideration.

\section{}\label{chap9:sec5}%sec 5

We shall now turn $W$ into a group isomorphic to $\sum$ by defining a
suitable multiplication in $W$, in fact $\sum$ will turn out to be the
right-regular permutation representation of the group $W$, we are to
define. Consider the mapping $\eta$ of $\sum$ into $W$ defined by 
$$
\sigma^{\eta} = w_o ^{\sigma}, \text{ for every }\sigma \in \sum.
$$

Let 
$$
w= s_1 s_2 \cdots s_n h
$$
be any normal word in $W$. Put
$$
\sigma = \rho_{(s_1)}\rho_{(s_2)}\cdots \rho_{(s_n)}\rho_{(h)}.
$$

It is easy to verify that
$$
w^{\sigma}_o = s_1 s_2 \cdots s_n \cap = w
$$

Thus the mapping is `onto' $W$. Now we shall show that $\eta$ is $1-1$.

We\pageoriginale shall first prove the following lemma.

\begin{lemma}%lemma
  Let
  $$
  \sigma = \rho_{(g_1)} \cdots \rho_{(g_m)} \in \sum, \text{
    with } g_{\mu}G_{i (\mu)} (1 \leq \mu \leq m). 
  $$
  and $i (\mu) \neq i (\mu + 1)$. Then the length of the normal word
  $w^{\sigma}_o$ is $m$ if $m >1$ and further 
  $$
  w^{\sigma}_o = s_1 s_2 \cdots s_m h \text{ with } s_{\mu}\in
  S_{i (\mu)}, 1 \leq \mu \leq m, h \in H. 
  $$
\end{lemma}

If $m=1$, then the length of $w^{\sigma}_o$ is $0$ or $1$ according
$g_1$ is in $H$ or not. 

\begin{proof}
  The proof of the lemma will be by induction. For $m=2$, we have
  $$
  \sigma = \rho (g_1) \rho (g_2), g_1 \in G_{i (1)},  g_2
  \in G_{i (2)}, i (1) \neq i(2). 
  $$

  Now,
  $$
  \displaylines{\hfill 
    w_o^{\rho (g_1)}  = s_1 h_1 \text{ with } 1 \neq s_1 \in
    S_{i (1)}, h \in H \hfill \cr 
    \text{and}\hfill g_1 = s_1 h_1;\hfill \cr
    \text{and}\hfill 
    w_o^{\sigma}  = (w_o ^{\rho (g_1)})^{\rho (g_2)} = s_1 s_2 h_2, 1 \neq
    s_2 \in S_{i (2)}, h \in H\hfill \cr 
    \text{and}\hfill  h_1 g_2 = s_2 h_2. \hfill }
  $$
\end{proof} 

Thus\pageoriginale
$$
\ell (w_o ^{\sigma}) =2.
$$

Assume the lemma to be true for all or with $2 \leq r < m$. Let $m >
2$. Put 
$$
\hat{\sigma} = \rho (g_1)\rho (g_2) \cdots \rho (g_{m-1}).
$$

Then by induction hypothesis
$$
w_o^{\hat{\sigma}} = s_1 s_2 \cdots s_{m-1} h', s_{\mu} \in
G_{i (\mu)}, 1 \leq \mu \leq m-1,  h' \in H. 
$$

Now,
\begin{align*}
  w_o ^{\sigma} & = \bigg\lgroup w_o ^{\hat{\sigma}}\bigg\rgroup
  ^{\rho (g_m)} = (s_1 s_2 \cdots  s_{m-1} h) ^{\rho (g_m)}\\ 
  & = s_1 s_2 \cdots s_{m-1} s_m h, \text{ where }\\
  h' g_m &= s_m h,  1 \neq s_m \in S_{i (m)}, h \in H.
\end{align*}

This completes the induction and proves that
$$
\ell (w_o ^{\sigma}) = m \text{ if } m > 1.
$$

If $m=1$, then
$$
\sigma = \rho(g_1).
$$

It\pageoriginale is obvious that $\ell (w_o) = 0  $ or $1$ according as $g_1$ is in
$H$ or not. 

Now if,
$$
w_o ^{\sigma} = w_o ^{\sigma}, \sigma, \sigma' \in \sum,
$$
then 
$$
w_o^{\sigma \sigma'^{-1}} = w_o.
$$

Choose $m$ to be the least positive integer such that 
$$
\sigma \sigma'^{-1} = \rho (g_1) \cdots \rho (g_m), \text{ with } g_i
i G_{i (\mu)}, 1 \leq \mu \leq m. 
$$

If $m > 1$, then $i (\mu)\neq i (\mu + 1)$. For, otherwise $\sigma
\sigma'^{-1}$ can be ``shrunk'' by amalgamating $g_{\mu}$ and $g_{\mu
  +1}$. Hence by the above lemma it follows that 
$$
\ell (w_o^{\sigma \sigma'^{-1}}) = m > 1.
$$

Since 
$$
w_o^{\sigma \sigma'^{-1}} = w_o,
$$
this is absurd.

Therefore, $ m = 1 $; thus let
$$
\sigma \sigma'^{-1} = \rho (g_1), g_1 \in G_{i (1)}.
$$

Then\pageoriginale
\begin{gather*}
  w_o ^{\sigma \sigma'^{-1}} = s_1 h_1, \text{ where }\\
  g_1 = s_1 h_1 \text{ with } s_1 \in S_{i (1)}, h_1 \in H.
\end{gather*}

But
\begin{gather*}
  w_o ^{\sigma \sigma'^{-1}} = w_o; \text{ and therefore }\\
  s_1 = 1, h_1 = 1; \text{ that is }\\
  g_1 = 1 : \text{ that is to say }\\
  {\sigma \sigma'^{-1}} = \rho (g_1) = L.
\end{gather*}

Hence 
$$
\sigma = \sigma'.
$$

Thus the mapping $\eta$ of $\sum$ onto $W$ is $1-1$. We now put a
group structure on $W$ in the following way. 

Define
\begin{gather*}
  wow'  = w_o ^{\sigma \sigma^{-1}}, \text{ where }w,w' \in W\\
  \text{ and } \qquad w_o ^{\sigma } = w, w_o ^{\sigma'} = w'.
\end{gather*}

One can easily verify that $W$ is turned into a group with this
multiplication\pageoriginale and that the group $\sum$ is the right regular
permutation representation of the group $W$. Further, $W$ being
isomorphic to $\sum$ also embeds the amalgam. To return to our old
notation, $W$ will be renamed $P$. We shall identify the groups
$G_k^{\rho_k \eta}$ with $G_k$. Under this identification,  
$$
G_k \leq P, \text{ for all } k \in I.
$$

Further 
$$
G_i \cap G_j = H \text{ in } P.
$$

Let
$$
w = s_1 s_2 \cdots s_n h \in W, s_{\mu } \in S_{i
  (\mu)} 1 \leq \mu \leq n, h \in H. 
$$

It is easy to verify that
$$
w = s_1 \circ s_2 \circ \cdots \circ s_n \circ h.
$$

If
$$
w' = s'_1 s'_2 \cdots s'_m h',
$$
then in general the usual product of the words $w$ and $w'$ is not a
normal word. If 
\begin{align*}
  \sigma & = \rho(s_1) \rho(s_2) \cdots \rho(s_n) \rho(h) \text{ and }\\
  \sigma' & = \rho(s'_1)\rho(s'_2) \cdots \rho(s'_m) \rho(h'), \text{ then }\\
  w_o w' & = w_o^ {\sigma \sigma'} = (s_1 s_2 \cdots s_n h)^{\sigma'} = w^{\sigma'}.
\end{align*}

Now,\pageoriginale
$$
\sigma \sigma' = \rho(s_1) \cdots \rho(s_n) \rho(h) \rho(s'_1) \cdots
\rho(s'_m) \rho(h'). 
$$

If $s_n$ and $s'_1$ do not be in the same constituent, then  by our lemma
$$
w_o ^{\sigma \sigma'}= w_o^{\rho(s_1) \cdots \rho(s_n h) \rho(s'_1) \cdots \rho(s'_m h')}
$$
is of length $n + m$ and 
$$
 wow' = w_o ^{\sigma \sigma'} = s_1 s_2 \cdots s_n
s^{(1)}s^{(2)} \cdots s^{(m)}h^{(m)} 
$$
where
\begin{gather*}
  h s'_1 = s^{(1)} h^{(1)}\\
  h^{(1)} s'_2 = s^{(2)}h ^{(2)}\\
  \cdots \cdots \cdots \cdots \cdots\\
  \cdots \cdots \cdots \cdots \cdots \\
  h^{(m-1)}s'_m = s^{(m)} h^{(m)}.
\end{gather*}

On the other hand if $s'_1$ and $s_n$ are in the same constituent, we\pageoriginale
amalgamate $s_n h$ and $s'_1$ and write 
$$
s_n hs'_1 = s^{(1)} h ^{(1)}
$$
and proceed as in the above case.	

We now proceed to show that $\sum$ is the generalised free product of
the amalgam. Let 
$$
u \underline{(\rho (g))} = L
$$
be a relation in $ \sum $; we show that it follows from the relations
of the groups $\rho (G_i)$. We write the relation in the form  
$$
\rho (g_1) \rho (g_2) \cdots \rho (g_n) = L,
$$
where $g_{\nu} G_{i (\nu)}, \nu = 1,  \ldots, n$. [This can be done
  because $(\rho (g))^{-1} = \rho (g^{-1})$.] If $n=1$, then we have
$\rho (g_1) = L$, and we have seen already that this implies $g_1 =
1$, so the relation is trivial. Assume then that $n > 1$. We claim
that there are two successive elements $g_{\nu}, g_{\nu +1}$ out of
the same constituent, that is $i (\nu) = i (\nu +1)$; for if
not, then 
$$
\ell (w_o^{\rho (g_1) \rho (g_2) \cdots \rho (g_n)}) = n > 1,
$$
by our lemma, and this contradicts the assumed relation. Now in $G_{i
  (\nu)}$, there is a product $g^*$, say of $g_{\nu}$ and $g_{\nu
  +1}$, so that 
$$
g_\nu g _{ \nu + 1} = g^*
$$
is\pageoriginale a relation in $G_{i (\nu)}$ ; thus also 
$$
\rho_{(g_\gamma)} \rho _{(g_{ \gamma + 1})} = \rho_ {(g^*)}
$$
is a relation in $\rho_{ (G_{i(\nu)})}$, that is a consequence of the
defining relations of $\rho_{ (G_{i(\nu)})}$. By means of this
relation we can now reduce the given relation to a shorter one,  
$$
 \rho (g_1) \cdots \rho ( g _{ \gamma - 1}) \rho (g^*) \rho (g_{\nu +
   2})\cdots \rho (g_n) = \ell. 
$$

By an easy induction one deduces that the given relation follows from
the defining relations of the constituent groups. This proves the
theorem:  
\begin{theorem}\label{chap9:sec5:thm5} %theorem 4
  The group $\sum$ and hence $P$ is the generalised free product of
  the family of groups $\big\{G_i \big\}_{ i \in I}$ with all
  the amalgamating subgroups coinciding with $H$.  
\end{theorem}

We immediately have the following consequence: 

\begin{coro*}%coro 
  The group $\sum$ (and hence $P$) does not depend upon the
  trans\-versals $S_i$ of $H$ in $G_i$ that have been chosen.  
\end{coro*}
