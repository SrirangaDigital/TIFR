\chapter{Calculus of Wiener Functionals}\label{chap1}%%% 1

\section{Abstract Wiener Space}%%% 1.1
  

\setcounter{pageoriginal}{4}
Let\pageoriginale $W$ be a separable Banach space and let $B(W)$ be
the Borel field, i.e.,  topological $\sigma$-field. Let
$\overset{\ast}{W}$ be the dual of $W$.    

\begin{definition}%1.1
  A probability measure $\mu$ on $(W,B(W))$ is said to be a {\em
    Gaussian measure} if the following is satisfied: 
  
  For every $n$ and $\ell_1, \ell_2, \ldots, \ell_n$ in
  $\overset{\ast}{W}, \ell_1 
  (W), \ell_2(W), \ldots, \ell_n(w)$, as random variables on ($W,B(W),
  \mu$) are Gaussian distributed i.e.,  $\exists~ V= (v_{ij})^n_i$, $j=1$ and
  $m \epsilon  \mathbb{R}^n$ such that $(v_{ij}) \ge 0$ and symmetric
  and for every $c= (c_1, c_2, \ldots, c_n) \epsilon  \mathbb{R}^n$, 
  $$
  \int\limits_{W} \exp \left\{ \sum_{i=1}^d \sqrt{-1} c_i \ell_i
  (w)\right\}  \mu(dw)= \exp \left\{\sqrt{-1}< m,c > - \frac{1}{2}<
  Vc, c > \right\}  
  $$
  where $<.,.>$ denotes the $\mathbb{R}^n$ -inner product.
\end{definition}  
  
  We say that $\mu$ is a \textit{mean zero Gaussian measure} if $m =
  0$, or equivalently, 
  $$
  \int\limits_{W} \ell(w) \mu (dw)=0  \quad \text{for every} \quad
  \ell \epsilon   \overset{\ast}{W}. 
  $$
  
Let $S(\mu)$ denote the support of $\mu$. For Gaussian measure,
$S(\mu)$ is a closed linear subspace of $W$ and hence without loss of
generality, we can assume $S(\mu)=W$ (otherwise, we can restrict the
analysis to $S(\mu)$). 

\setcounter{theorem}{0}
\begin{theorem}\label{chap1:thm1.1}%%%% 1.1
  Given\pageoriginale a mean zero Gaussian measure $\mu$ on
  $(W,B(W))$, there exists a unique separable Hilbert space $H \subset
  W$ such that the inclusion  map $i$: $H \to W$ is continuous, $i(H)$
  is dense in $W$ and   
  \begin{equation*}
    \int\limits_{W} e^{\sqrt{-1} \ell(w)} \mu (dw) =e^{-
      \frac{1}{2}|\ell |_H^2} \tag{1.1}\label{eq1.1} 
  \end{equation*}
  where $|.|_H$ denotes the Hilbert space $H$-norm.
\end{theorem} 

\begin{remark}%%% 1
  $H \subset W$ implies  $\overset{\ast}{W} \subset H^*=H$ and for $h
  \epsilon  H, 
  \ell  \epsilon  \overset{\ast}{W}, \ell(h)$ is given by $\ell(h) = <
  \ell, h   >_H$. 
\end{remark}  

\begin{remark}%%% 2
  Condition (\ref{eq1.1}) is equivalent to 
  \begin{equation*}
    \int\limits_{W} \ell(w) \ell' (w) \mu (dw)= < \ell, \ell' >_H
    \text{ for every } \ell, \ell' \epsilon  \overset{\ast}{W}. \tag*{$(1.1)'$} 
  \end{equation*}
\end{remark}

\begin{remark}%%%% 3
  The triple $(W,H, \mu)$ is called an {\em abstract Wiener space}.
\end{remark}  

\medskip
\noindent{\textbf{Sketch of proof of Theorem \ref{chap1:thm1.1}:}} 
  Uniqueness follows from the the fact that $H= \bar{W}^{*^{|.|_H}}$.

\medskip
\noindent{\textbf{Existence:}}
 By definition of Gaussian measure, $\overset{\ast}{W}
\subset L_2 (\mu)$. Let $\tilde{H}$ be the completion of
$\overset{\ast}{W}$ under 
$L_2$-norm. Let $j: \overset{\ast}{W} 3 \ell \to \ell (w) \epsilon  \tilde{H}$;
then $j$ is one-one linear, continuous and has dense range. The
continuity of $j$ follows from the fact that (Fernique's theorem):
there exists $\alpha > o$ such that  
$$
\int\limits_{W} e^{\alpha || w ||^2} \mu (dw) < \infty.
$$ 

Now consider $j^*$, the dual map of $j$,
$$
j^*: \tilde{H}^* = \tilde{H} \to W^{**} \supset W.
$$

It\pageoriginale can be shown that $j^*(\tilde{H}) \subset
\omega$. Take $H=j^{*} 
(\tilde{H})$ and for $\bar{f}, \bar{h}$ in $H$, define 
$$
< \bar{f}, \bar{h}> = < f,h> \text{ where } \bar{f}= j^{*}(f),
\bar{h}= j^*(h). 
$$ 

\begin{example}[Wiener space]\label{chap1:exam1.1}%%% 1.1
   Let $W=W^r_o$ and $\mu: r$- dimensional Wiener measure.

$H=\{h = (h^i(t))^r_{i=1} \epsilon  W^r_o: h^i(t)$ are absolutely
continuous on $[o, T]$ with square integrable derivative $\dot{h}^i(t),
1 \le i \le r \}$ 

For $h=(h^i(t))^r_{i=1}, g = (g^i (t))^r_{i=1}$, define the inner product
$$
<h, g> = \sum_{i=1}^{r} \int\limits_{o}^{T} \dot{h}^i (s) \dot{g}^i (s) ds.
$$
\end{example}

Then $H$ is a separable Hilbert space and $(W,H, \mu)$ is an abstract
Wiener space which is called \textit{$r$-dimensional Wiener space}. 

\begin{example}\label{chap1:exam1.2}%% 1.2
Let $I$ be a compact interval in $\mathbb{R}^d$ and
$$
K(x,y)=(k^{ij}(x,y))^r_{i, j=1}
$$ 
where $k^{ij}(x,y) \epsilon 
C^{2m}(I \times I)$, and satisfies the following conditions: 
\begin{enumerate}[(i)]
\item $k^{ij}(x,y)=k^{ij}(y,x) ~ \forall~ x, y \epsilon  I ~1 \le i,j \le r$.
\item For any  $c_{ik} \epsilon  \mathbb{R}$, $i=1$, $w, \cdots, r$,
  $k=1, 2, \ldots, n$, $n \epsilon  N$,  $\sum\limits_{k,\ell=1}^{n}$
  $\sum\limits_{i,j=1}^{r} k^{ij}$ $(x_k, x_\ell) c_{ik}c_{j \ell} \ge 0$,
  $\forall~ x_k \epsilon  I$, $k=1, 2, \ldots, n$. 
\item for $| \alpha |=m$, there exists $o < \delta \le 1$ and $c > o$ such that 
  $$
  \displaylines{\hfill 
  \sum_{i=1}^{r}\left[k^{(\alpha) ii}(x,x) +
    k^{(\alpha)ii}{(y, y)}-2k^{(\alpha)ii}(x,y)\right] \le c| x-y|^{2
    \delta} \hfill \cr
  \text{where}\hfill 
  k^{(\alpha) ij}(xy)= D^\alpha_x D^\alpha_y k^{ij}(x,y).\hfill } 
  $$\pageoriginale
\end{enumerate}
\end{example}

(As usual, $\alpha = (\alpha_1, \alpha_2, \ldots, \alpha_d)$ is a
multi-index, $|\alpha| = \alpha_1 + \cdots + \alpha_d$ and  
$$
D^{\alpha}_x= \frac{\partial|\alpha |}{\substack{\alpha_1
    \\\partial^\alpha{x_1}} \substack{\quad \\ \cdots}\substack{\alpha_d\\
  \partial^\alpha{x_d}}}. 
$$
Now, for $f \epsilon  C^m(I \to R^r),f=(f^1, f^1, \ldots, f^r)$, define
$$
|| f ||_{m, \epsilon }= \sum_{i=1}^r \sum_{|\alpha| \le m} ||
D^\alpha f^i ||_{\epsilon }, 
$$
where
$$
|| f^i ||_{\epsilon }= \max_{x \epsilon  I}|f^i(x)| +
\sup_{\substack{x \neq y \\ x,y \epsilon  I}}
\frac{|f^i(x)-f^i(y)|}{|x-y|^\epsilon } 
$$

Let
$$
C^{m, \epsilon }(I \to \mathbb{R}^r)= \{w \epsilon  C^m(I \to
\mathbb{R}^r): || w ||_{m, \epsilon } < \infty\}. 
$$

$W=(C^{m, \epsilon }, ||. ||_{m, \epsilon })$ is a Banach space.

\begin{fact}
  For any $\epsilon , o \le \epsilon  < \delta, \exists~$ a mean
  zero Gaussian measure on $W$ such that  
  $$
  \int\limits_{W}w^i(x)w^j (x) \mu (dw) =k^{ij}(x,y) i, j= 1,2, \ldots, r.
  $$  
\end{fact}

Then by theorem \ref{chap1:thm1.1} it follows that there exists a
Hilbert space $H 
\subset W$ such that $(W, H, \mu)$ is an abstract Wiener space. In
this case, $H$ is the reproducing kernel Hilbert space associated
with the kernel $K$, which is defined as follows: 

For\pageoriginale $x=\left(x_1, x_2, \ldots, x_n\right), x_k \epsilon
I, \lambda= ( 
\lambda_1, \lambda_2, \ldots, \lambda_n)$,  
$$
\displaylines{\hfill 
  \lambda_k= \left(\lambda^i_k\right)^r_{i=1} \epsilon  \mathbb{R}^r, \text{
    define } W_{[x, \lambda]}(y)=(W^i_{[x, \lambda]}(y))^r_{i=1}  \hfill\cr
  \text{by}\hfill   
  W^i_{[x,\lambda]}(y)= \sum_{j=1}^r \sum_{k=1}^n k^{ij}(y,
  x_k). \lambda^i_k,\hfill\cr 
  \text{and let}\hfill 
  S=\left\{W_{[x, \lambda]}: x = (x_1, x_2, \ldots, x_n), x_k \epsilon 
  I, \lambda= (\lambda_1, \ldots,
  \lambda_n),\vphantom{\lambda^i_k\mathbb{R}^r}\right.\hfill \cr 
  \hfill \left.\lambda_k = (\lambda^i_k)^r_{r=1} \epsilon  \mathbb{R}^r
  \text{ and } n \epsilon  \mathbb{N}\right\}. \hfill }
$$

For $W_{[x, \lambda]}, W_{y, \nu} \epsilon  S$, when $x= (x_1, x_2,
\ldots, x_{n_1})$, $\lambda=(\lambda_1, \ldots, \lambda_{n_1})$,
$y=(y_1, \ldots,  y_{n_2})$, $\nu = (\nu, \ldots, \nu_{n_2})$, define
the inner product by  
$$
< W_{[x, \lambda]}, W_{x,\nu} >= \sum_{k=1}^{n_1} \sum_{\ell=1}^{n_2}
\sum_{i,j=1}^r k^{ij}(x_k, y_\ell) \lambda^i_k \nu^j_\ell; 
$$
then $(S, <.,.>)$ is an inner product space and the reproducing
kernel Hilbert space $H$ is the completion of $S$ under this inner
product. 

\section{Einstein-Uhlenbeck Operators and Semigroups}%%% 1.2

Let $(W, H, \mu)$ be an abstract Wiener space and $(S,B(S))$ a
measurable space. A map $x: W \to S$ is called an \textit{$S$-valued
  Wiener functional}, if it is $B(W)|B (S)$-measurable. Two $S$-valued
Wiener functionals $x,y$ are said to be equal and denoted by $x=y$ if
$x(w)=y(w)\,a.a.w\,(\mu)$. For the moment, we consider mainly the case $S
=\mathbb{R}$. 

\medskip
\noindent{\textbf{Notation}}
  $L_p= L_p(W,B(W), \mu),1 \le p < \infty$.\pageoriginale 


\begin{definition}%%% 1.2  
$F: W \to \mathbb{R}$ is a polynomial, if  $\exists~ \, n \epsilon 
  \mathbb{N}$ and  $\ell_1, \ell_2, \ldots,\break \ell_n \epsilon  \overset{\ast}{W}$
  and $p(x_1, \ldots, x_n)$, a real {\em polynomial} in $n$ variables
  such that  
$$
F(w)= p(\ell_1(w), \ell_2(w), \ldots, \ell_n (w)) ~\forall ~ w \epsilon  W.
$$
\end{definition} 
 
In this expression of $F$, we can always assume that
$\{\ell_i\}^n_{i=1}$ is an $ONS$ in the sense defined below. We define
degree $(F)=$ degree $(P)$ which is clearly  independent of the choice
of $\{\ell_i\}$. We denote by $\mathcal{P}$ the set of such polynomial
and by $\mathcal{P}_n$ the set of polynomial of degree $\le n$. 

\begin{fact}
  $\mathcal{P} \subset L_p, 1 \le p < \infty$ and the inclusion is dense 
\end{fact}

\begin{definition}%%% 1.3
  A finite or infinite collection $\{\ell_i\}$ of elements in
  $\overset{\ast}{W}$ is 
  said to be an {\em orthonormal system} $(ONS)$ if $< \ell_i, \ell_j
  >_H= \delta_{ij}$. It is said to be an {\em orthonormal basis}
  $(ONB)$ if it is an $ONS$ and $L(\ell_1, \ell_2, \ldots)^{|.|_H}=H$,
  where $L(\ell_1, \ell_2, \ldots)$ is the linear span of $(\ell_1,
  \ell_2, \ldots)$.  
\end{definition}  


\medskip
\noindent{\textbf{Decomposition of $L_2$:}}
 We now represent $L_2$ as an
  infinite direct sum of subspaces and this decomposition is called
  the \textit{Wiener-Chaos decomposition} or the \textit{Wiener-Ito
    decomposition}. 
  
  Let $C_o=\{$ constants $\}$
  
  Suppose $C_o, C_1, \ldots, C_{n-1}$ are defined. Then we define
  $C_n$ as follows: 
  $$
  C_n= \bar{\mathcal{P}}_n^{||~~~|| L_2 } \circleddash [C_o \oplus C_1
    \oplus \cdots \oplus C_{n-1}] 
  $$\pageoriginale
  i.e.,  $C_n$ is the orthogonal complement of $C_o \oplus \cdots \oplus
  C_{n-1}$ in $\bar{\mathcal{P}}_n || ~~|| L_2$. Since $\mathcal{P}$
  is dense in $L_2$, it follows that  
  $$
  L_2= C_o \oplus C_1 \cdots \oplus C_n \oplus \cdots
  $$

\medskip
\noindent{\textbf{Hermite Polynomials:}}
 The Hermite polynomials are defined as 
  $$
  H_n(x)= \frac{(-1)^n}{n!} e^{x^2/2} \frac{d^n}{dx^n}(e^{-x^2/2}), n
  = 0,1,2, \ldots 
  $$
  
They have the following properties:
\begin{enumerate}
  \item $H_o(x)=1$ 

  \item $\sum\limits_{n=o}^{\infty}t^n H_n(x) =e^{-(t^2/2)+tx}$

  \item $\dfrac{d}{dx} H_n(x) =H_{n-1}(x)$

  \item $\int\limits_{\mathbb{R}} H_n (x) H_m(x) \dfrac{1}{\sqrt{(2
      \pi)}}e^{-x^2/2} dx = \dfrac{1}{n!} \delta_{n,m}$. 
\end{enumerate}
  
Let $\Lambda = \{a=(a_1, a_2, \dots)|\,a_i \,\epsilon  z^+, a_i =0$
expect for a finite numbers of $i' s\}$.  
  
For $a \epsilon  \Lambda, a! \triangleq \prod\limits_{i} (a_i!), |a|
\triangleq \sum\limits_{i} a_i$. Let us fix an $ONB (\ell_1, \ell_2,
\ldots)$ in $\overset{\ast}{W}$. Then for $a \epsilon  \Lambda$, we define   
  $$
  H_a(w) \triangleq \prod_{i=1}^\infty H_{a_i}(\ell_i(w)).
  $$ 
 
Since $H_o(x) \equiv 1$ and $a_i =0$ expect for a finite number of
$i's$, the above product is well defined. We note that $H_a(.)
\epsilon  \mathcal{P}_n$ if $|a| \le n$. 

\setcounter{proposition}{1}
\begin{proposition}%%%% 1.2
   \begin{enumerate}[\rm (i)]
   \item $\left\{ \sqrt{a!} H_a(w): a \epsilon  \Lambda \right\}$ is
     an $ONB$ in $L_2$.\pageoriginale  

   \item $ \left\{ \sqrt{a!} H_a(w): a \epsilon  \Lambda,
     |a|=n\right\}$ is an $ONB$ in $C_n$. 
   \end{enumerate}
\end{proposition}  

\begin{proof}
Since $\{\ell_i\}$ is an $ONB$ in $\overset{\ast}{W}$, $\{\ell_i (w)\}$ are
$N(0.1)$, $i.i.d$. random variables on $W$. Therefore, 
\begin{align*}
  \int\limits_{W} H_a(w) H_b(w) \mu (dw) & = \prod_{i=1}^{\infty}
  \int\limits_{W} H_{a_i} (\ell_i (w))H_{b_i} (\ell_i(w)) \mu (dw)\\ 
  &= \prod_{i=1}^{\infty} \int\limits_{\mathbb{R}}
  H_{a_i}(x)H_{b_i}(x) \frac{1}{\sqrt{(2 \pi)}} e^{-x^2/2} dx\\ 
  &=  \prod_{i} \frac{1}{a_i !} \delta_{a_i, b_i} = \frac{1}{a!}
  \delta_{a,b}. 
\end{align*}
Since $\mathcal{P}$ is dense in $L_2$, the system $\left\{ \sqrt{a!}
H_a(w);  a \epsilon  \Lambda\right\}$ is complete in $L_2$. 
\end{proof}

Let $J_n$ denote the orthogonal projection from $L_2$ to $C_n$. Then
for $F \epsilon  L_2$, we have $F= \sum\limits_{n} J_n F$. In
particular, if $F \epsilon  \mathcal{P}$, then the above sum is
finite and $J_n F \epsilon  \mathcal{P}, ~\forall~ n$. 

\begin{definition}%%% 1.4
  The function $F: W \to \mathbb{R}$ is said to be a smooth
    functional, if ~$\exists~ n \epsilon  \mathbb{N}, \ell_1 \ell_2, \ldots,
  \ell_n \epsilon \overset{\ast}{W}$, and $f \epsilon  C^\infty(\mathbb{R}^n)$,
  with polynomial growth order of all derivatives of $f$, such that  
  $$
  F(w)=f(\ell_1(w), \ell_2(w), \ldots \ell_n(w)) ~\forall~ w
  \epsilon   W. 
  $$  
\end{definition}

We denote by $S$ the class of all smooth functionals on $W$. 

\begin{definition}%%% 1.5
  For\pageoriginale $F(w) \epsilon  S$ and $t \ge o$, We define
  $(T_tF)(w)$ as follows: 
  \begin{equation*}
    (T_tF)(w) \triangleq \int\limits_{W} F(e^{-t}w+ \surd(1-e^{-2t})
    u)\mu(du) \tag{1.2}\label{eq1.2} 
  \end{equation*}
\end{definition}

\medskip
\noindent{\textbf{Note (i):}}
If $F \epsilon  S$ is given by 
$$
F(w)=f(\ell_1(w), \ldots  \ell_n(w)), f \epsilon  C^\infty(\mathbb{R}^n)
$$
for some $ONS$  $\{\ell_1, \ell_2, \ldots \ell_n\} \subset
\overset{\ast}{W}$, then  
\begin{equation*}
  (T_tF)(w) = \int\limits_{\mathbb{R}^n} f(e^{-t } \xi + \surd
  (1-e^{-2t}) \eta) \frac{1}{(\sqrt{2 \pi})^n}e^{-(|\eta|^2) /2}d \eta
  \tag{1.3}\label{eq1.3}  
\end{equation*}
 where $ \xi = (\ell_1(w), \ldots, \ell_n (w)) \epsilon
 \mathbb{R}^n$.

\medskip
\noindent{\textbf{Note (ii):}}
The above definition can be also be used to define $T_tF$ when $F
\epsilon  L_p$. 

\medskip
\noindent{\textbf{Properties of $T_tF$:}}
\begin{enumerate}[(i)]
\item $F \epsilon   S \Rightarrow T_tF \epsilon  \mathcal{S}$
\item $F \epsilon  \mathcal{P} \Rightarrow T_t F \epsilon  \mathcal{P}$
\item For $f$, $G \epsilon  \mathcal{S}$ 
  $$
  \int\limits_{W}(T_tF)(w)G(w) \mu(dw)= \int\limits_{W} F(w) (T_tG)(w) \mu (dw)
  $$
\item $T_{t+s}F(w)=T_t(T_s F)(w)$
\item If $F \epsilon  \mathcal{S}$, $F= \sum\limits_{n} J_nF$, then 
  $$
  T_t F= \sum_{n}e^{-nt}(J_nF)
  $$
\item $T_t$ is a contraction on $L_p$, $1 \le p < \infty$.
\end{enumerate}

\begin{proof}
(i)\pageoriginale and (ii) are trivial and (iii) and (iv) follow easily from
  (v). Hence we prove only (v) and (vi). 
\end{proof}

\medskip
\noindent{\textbf{Proof of (v):}}
   Let $\ell \in \overset{\ast}{W}$ and 
  $$
  F(w) = E^{\sqrt{-1}} \ell (w) +\frac{1}{2} |\ell|^2_H. 
  $$

Then 
\begin{align*}
  T_t F(w) & =  \int\limits_W \exp \left[\sqrt{-1}e^{-t}\ell (w)+ \sqrt{-1}
    \surd  (1-e^{-2t})\ell (u)+ \frac{1}{2}|\ell |^2_H\right] \mu (du)\\ 
  & = e^{\sqrt{-1}} e^{-t} \ell (w) + \frac{1}{2} | \ell |^2_H
  \int\limits_W e^{\sqrt{-1}} \surd (1-e^{-2t}) \ell (u)_{\mu (du)}
  \\ 
  & = e^{\sqrt{-1}} e^{-t}\ell (w) + \frac{1}{2}e^{e^{2t}|\ell |^2_H}.
\end{align*}

Let 
\begin{align*}
  \lambda & = (\lambda_1, \lambda_2, \ldots, \lambda_N )\in
  \mathbb{R}^\vee, N \in \mathbb{N}\\ 
  \ell & = \lambda_1 \ell_1 +\cdots + \lambda_N \ell_N, \{ \ell_i
  \}^N_{i=1} \text { an ONS }.  
\end{align*}

Let 
$$
F(w) = e^{\sqrt{-1}}\ell (w)+ \frac{1}{2}|\ell|^2_H.
$$

Then 
\begin{align*}
  &F(w)  =  \prod^N_{i=1} e^{\sqrt{-1}}\lambda_i \ell_i(w) -
  \frac{1}{2}(\sqrt{-1}\lambda_i)^2 \\ 
  & = \sum^{\infty}_{m_1, \ldots, m_N=0}(\sqrt{-1} \lambda_1)^{m_1}
  \cdots (\sqrt{-1} \lambda_N)^{m_N} 
  \times H_{m_1}(\ell_1(w))\cdots H_{m_N}(\ell_N(w)).
\end{align*}

Applying $T_t$ to both sides of the above equation, we have 
{\fontsize{10pt}{12pt}\selectfont
\begin{multline*}
  e^{\sqrt{-1}} e^{-t}\ell (w) +\frac{1}{2} e^{2t}|\ell |^2_H= T_t  F(w) 
  = \sum^\infty_{m_1,\ldots m_N = 0}(\sqrt{-1}\lambda_1)^{m_1}\ldots
  (\sqrt{-1}\lambda_N)^{m_N}\\
 \times T_t \left(\prod^{m_N}_{i = 1} H_{m_i}(\ell_i (.))\right) (w).
\end{multline*}}\relax\pageoriginale

Hence 
\begin{gather*}
  T_t \left(\prod^N_{i = 1} H_{m_i}(\ell _i(.))\right) (w) = \prod^N _{i = 1}
  e^{-tm_i}H_{m_i}(\ell_i (w))\\ 
  = e ^{-t} \sum^N _{i = 1} m_i \prod^N _{i=1} H_{m_i}(\ell_i)(w))
\end{gather*}
implies 
$$
(T_t H_a)(w) = e^{-|a| t}H_a(w). 
$$

If $P \epsilon  P $, then $F = \sum_n J_n F$ where $J_n F
\epsilon  C_n$. Then since  
$$
\left\{\sqrt{a!}H_a (w): a \epsilon  \wedge, |a| = n\right\}
$$
is an $ONB$ for $C_n$, we finally have 
$$
(T_t F)(w) = \sum_n e^{-nt}(J_n F)(w).
$$

\medskip
\noindent{\textbf{Proof of (vi):}}
  Let $P_t (w, du)$ denote the image measure $\mu \circ \phi^{-1}_{t, w}$
  of the map $\phi_{t,w}: W \to W$ 
  $$
  \phi_{t, w}(u) = e^{-t} w + \surd (1-e^{-2t})u.
  $$

Then 
$$
(T_t F)(w) = \int P_t (w, du) F(u), F \epsilon  L_p.
$$

First\pageoriginale let $F$ be a bounded Borel function on $W$. Then $F
\in L_p$ and  
\begin{align*}
 || T_t F ||^p _{L_p} & = \left\{\int\limits_W | \int\limits_W P_t (w, du)
 F(u)|^P_\mu (dw)\right\}\\ 
 & \leq   \left\{\int\limits_W | \int\limits_W P_t (w, du)
 F(u)|^P_\mu (dw)\right\} \\ 
 & = < 1, T_t (|F|^P) >_{L_2}\\
 & = < 1, |F|^P >_{L_2} (\because T_t 1 = 1)\\
 & = || F ||^p_P.
\end{align*}

Hence $|| T_t F||_{L_p} \leq || F ||_{L_p}$ holds for any bounded
Borel function $F$. In the general case, for any $F \in L_p$, we
choose $F_n$, bounded Borel functions, such that $F_n \to F$ in
$L_p$. Then  
\begin{gather*}
  || T_t F_n ||_{L_p} \leq || F_n||_{L_p} ~\forall~ n, \\
  => || T_t F||_{L_p} \leq || F ||_{L_p}. 
\end{gather*}

Actually $T_t$ has a stronger contraction known as
\textit{hyper-contractivity}:  

\setcounter{theorem}{2}
\begin{theorem}[Nelson]\label{chap1:thm1.3} %Theorem 1.3
 Let $1 \leq p <\infty, t > 0$ and $q(t) = e^{2t}(p-1)
  + 1 > p$. Then for $F \epsilon  L_{q(t)}$,  
  $$
  || T_t F|| _{q(t)} \leq || F||_p.
  $$
\end{theorem}

\begin{remark*} %Remark 
  The semigroup $\{ T_t: t \geq o \}$ is called the {\em Ornstein -
  Uhlenbeck Semigroup}.  
\end{remark*}

\medskip
\noindent{\textbf{Some Consequence of the
    Hyper-Contractivity:}}\pageoriginale 
\begin{enumerate}
\renewcommand{\labelenumi}{\theenumi)}
\item  $J_n: L_2 \to C_n$ is a bounded operator on $L_p, 1 < p < \infty$. 
  \begin{proof}
    Let  $p > 2$. Choose $t$ such that $e^{2t} + 1 = p$.Then by Nelson's
    theorem, we have  
    $$
    || T_t F||_p \leq || F ||_2.
    $$

    In particular 
    $$
    || T_t J_n F|| _p \leq || J_n F ||_2 \leq ||F||_2 \leq ||F||_p. 
    $$

    But 
    $$
    || T_t J_n F|| _p = e^{-nt}|| J_n F|| _p;
    $$
    hence 
    $$
    ||J_n F|| _p \leq e^{nt}|| F ||_p.
    $$
    
    For $1 < p < 2$, Considering the dual map $J^*_n$ of $J_n$ and
    applying the previous case, we get  
    $$
    ||J^*_n F ||_p \leq e^{nt} || F||_p.
    $$
  \end{proof}    
    But, for $F \in P, J^*_n = J_n$. Hence, by denseness of $p$, the
    results follows.  

\item Let $V_n = C_0 \oplus \ldots C_1 \oplus 
  C_n$($V_n$ are called \textit{ Wiener chaos of order} $n$). Then,
  for every $1 \leq p, q < \infty, ||. ||_p$ and $||. ||_p$ are
  equivalent on $V_n$, i.e.,  for every $F \in V_n, \exists~ C_{p, q, n}
  > 0 $ such that  
  $$
  || F ||_q \leq C_{p, q, n}|| F ||_p.
  $$
In\pageoriginale particular, for $F \in V_n, || F ||_p < \infty, 1 < p < \infty$.
\end{enumerate}

\begin{proof}
  Easy and omitted.
\end{proof}

\begin{definition}[Ornstein-Uhlenbeck Operator]%defi 1.6
We define the generator $L$ of the semigroup $T_t$, which is called
\textit{Ornstein-Uhlenbeck Operator}, as follows: 
\end{definition}

For $F \epsilon  P$, define
$$
L(F) = \frac{d}{dt} T_tF|_{t=0} = \sum_n (-n) J_n F.
$$

Note that $L$ maps polynomials into polynomials. $L$ can also be
extended, as an operator on $L_P$, as the infinitesimal generator of a
contraction semigroup on $L_P$. The extension of $L$ will be given in
later sections. In particular, for $L_2$, let 
$$
D(L) = \left\{F \epsilon  L_2: \sum_n || J_n F ||^2_2 < \infty \right\}
$$
and for $F \epsilon  D (L)$, define
$$
L(F) = \sum_n (-n) J_n F.
$$

In it easily seen that $L$ is a self-adjoint operator on $L_2$.

\begin{definition}[Fr\'echet derivative]%defin 1.7
For $F \epsilon  P$ and $w \epsilon  W$, define
$$
DF (w) (u) = \frac{\partial F}{\partial t} (w + tu)|_{t = o} ~\forall~
u~ \epsilon ~ W. 
$$
\end{definition}

For each $w \epsilon  W, DF(w)$, which is called the
\textit{Fr\'echet derivative of $F$ at} $w$, is a continuous
linear functional on $W$ i.e.,   

$DF(w)\epsilon  \overset{\ast}{W}$.\pageoriginale More precisely,
$DF(w)$ is given as follows:  

Let $\{\ell_i\}$ be an ONS is $\overset{\ast}{W}$ and $F = p (\ell_1
(w),\ldots, \ell_n (w))$, then 
$$
DF(w) (u) = \sum_{i = 1}^{n} \partial_i p (\ell_1 (w),\ldots,
\ell_n(w)). \ell_i(u), 
$$
which we can also write as
$$
DF(w)  = \sum_{i = 1}^{n} \partial_i p (\ell_1 (w),\ldots, \ell_n(w)). \ell_i.
$$

For $F \epsilon  P$, the Fr$\acute{e}$chet derivative at $w$ of
order $k > 1$ is defined as  
\begin{gather*}
  D^kF(w) (u_1,u_2, \ldots, u_k)= \frac{\partial^k}{\partial t_1
  .. \partial t_k } F (w +t_1 u_1 + \cdots + t_k u_k)|_{t_1 = .. = t_k
  = 0}\\ 
  \text{ for } u_i \epsilon  W, 1 \le i \le k. 
\end{gather*}

Explicitly, if $F(w) = p (\ell_1(w), \ldots, \ell_n (w)))$, then
{\fontsize{10pt}{12pt}\selectfont
\begin{gather*}
  D^k F(w) = \sum^{n}_{i_1 = 1} .. \sum^{n}_{i_k = 1} \partial_{i_1},
  \partial_{i_2} \cdots \partial_{i_k} P(\ell_1(w), \ell_2(w), \ldots,
  \ell_n (w))). \times \ell^{\ell}_{i_1} \otimes .. \otimes \ell_{i_k}
\end{gather*}}\relax
where
$$
\ell_{i_1} \otimes .. \otimes \ell_{i_k} (u_1, u_2, \ldots, u_k)
\overset{\Delta}= \ell_{i_1} (u_1),\ldots, \ell_{i_k} (u_k). 
$$

Note that for each $w, D^k F (w) \epsilon  
\underbrace{\overset{\ast}{W} \otimes \cdots \otimes \overset{\ast}{W}}_{k ~\text{times}}$
where
$$
\underbrace{\overset{\ast}{W} \otimes \cdots \otimes \overset{\ast}{W}}_{k ~\text{times}} \overset{\Delta}= \left\{V: 
    \underbrace{W x \cdots x W}_{k ~\text{times}} \to \mathbb{R} | V \text{ is
  multilinear and continuous}\right\}. 
$$

\begin{definition}[Trace Operator]\label{chap1:def1.8}\pageoriginale% defi 1.8 
Let $\{h_i\}$ be an $ONB$ in $H$. For $V \epsilon  \overset{\ast}{W}
\otimes \overset{\ast}{W}$ 
we define the \textit{trace} of $V$ with respect to $H$, denoted as
trace$_HV$ by 
$$
\text{ trace }_HV = \sum^{\infty}_{i = 1} V(h_i, h_i).
$$
\end{definition}

Note that the definition is independent of the choice of $ONB$ and for 
$V \epsilon  \overset{\ast}{W} \otimes \overset{\ast}{W}$, trace$_HV$
exists and trace$_H(.)$ is 
a continuous function on $\overset{\ast}{W} \otimes
\overset{\ast}{W}$. 

\begin{remark*}
  For $\ell_1, \ell_2 \epsilon  \overset{\ast}{W}$,
  \begin{align*}
    \text { trace }_H \ell_1 \otimes \ell_2 & = \sum_{i} \ell_1 (h_i)
    \ell_2 (h_i) = \sum_{i} < \ell_1, h_i > _H < \ell_2, h_i > _H\\ 
    & = < \ell_1, \ell_2 >  _H.
  \end{align*}
\end{remark*}

\begin{theorem}%the 1.4
  If $F \epsilon  P$, then
  \begin{equation*}
    LF (w) = \text {trace}_H D^2 F~(w) - DF~ (w)~ (w),~ \text{ for } ~w
    ~\epsilon  ~W. \tag{1.3}\label{thmeq1.3}  
  \end{equation*}
\end{theorem}

\begin{proof}
Let $\{\ell_1, \ell_2, \ldots, \ell_n\}$ be an $ONS$ in $\overset{\ast}{W}$ and 
\begin{equation*}
  F(w) = p (\ell_1 (w), \ell_2(w),\ldots, \ell_n (w)).\tag*{$\Box$}
\end{equation*}

By the remark, we see that
\begin{multline*}
  \text{ RHS of  (\ref{thmeq1.3}) } = \sum\limits^{n}_{i=1} \partial_i
  \partial_i 
  p (\ell_1 (w), \ldots, \ell_n (w))\\  
  -\sum^{n}_{i = 1} \partial_i p (\ell_1 (w),
  \ldots, \ell_n (w)). \ell_i (w). 
\end{multline*}

Now let $\xi = (\ell_1 (w), \ldots, \ell_n (w))$, then
\begin{align*}
  \frac{d}{dt} T_tF(w) 
 &= \frac{d}{dt} \int\limits_{\mathbb{R}^n} p
  (e^{-t}\xi + \surd (1 -e^{-2t})n) (2
  \pi)^{-n/2} e^{\frac{-|\eta|^2}{2}}d \eta\\ 
 & = - \int\limits_{\mathbb{R}^n} \sum^{n}_{i = 1} e^{-t} \xi_i
  \partial_i p (e^{-t}\xi + \surd (1 -e^{-2t})n) (2
  \pi)^{-n/2}e^{\frac{-|\eta|^2}{2}}d \eta\\  
 & + \int\limits_{\mathbb{R}^n} \sum^{n}_{i = 1} \partial_i p(e^{-t}
  \xi + \surd (1-e^{-2t})n) \frac{\eta_i e^{-2t} (2\pi)^{-n /2}}{\surd
    (1 - e^{-2t})}e^{\frac{-|\eta|^2}{2}}d \eta \\ 
 & - \int\limits_{\mathbb{R}^n} \sum^{n}_{i = 1}  e^{-t} \xi_i
  \partial_i p (e^{-t} \xi + \surd (1-e^{-2t})n) (2
  \pi)^{-n/2}e^{\frac{-|\eta|^2}{2}}d \eta\\ 
 & -\int\limits_{\mathbb{R}^n} \sum^{n}_{i = 1} \partial_i p(e^{-t}\xi
  + \surd (1 - e^{-2t})n) \frac{e^{-2t} (2 \pi)^{- n / 2}}{\surd (1 -
    e^{-2t})} \times \partial_i (e^\frac{-|\eta|^2}{2})d \eta. 
\end{align*}\pageoriginale

Integrating the second expression by parts, we get
$$
\frac{d}{dt}T_tF(w) = - \sum^{n}_{i = 1} \xi_i e^{-t}T_t (\partial_i
p)(\xi) + \sum^{n}_{i = 1} e^{-2t}T_t (\partial^2_i p)\xi. 
$$

Hence we have
$$
LF (w) = \lim\limits_{t \to 0} \frac{d}{dt} T_tF(w) = RHS.
$$
\end{proof}

\begin{definition}[Operator $\delta$]%defi 1.9
  Let $P_{\overset{\ast}{W}}$ be the totality of functions $F(w): W
  \to \overset{\ast}{W}$ which 
  can be expressed in the form 
  $$
  F(w) = \sum^{n}_{ i =1} F_i (w) \ell_i
  $$
  for some $n \epsilon  \mathbb{N}, \ell_i \epsilon _i \overset{\ast}{W}$ and $F_i
  (w) \epsilon  p, i = 1,2,\ldots, n$. $F \epsilon  P_{\overset{\ast}{W}}$ is
  called \textit{$a~ \overset{\ast}{W}$-valued polynomial}. The linear operator
  $\delta: P_{\overset{\ast}{W}} \to P_W$ is defined as follows: 
  \begin{gather*}
    \text{ Let } \ell_1, \ell_2, \ldots, \ell_n, \ell ~\epsilon ~ \overset{\ast}{W}
    \text { and }\\ 
    F(w) = p (\ell_1 (w), \ldots, \ell_n (w))\ell.
  \end{gather*}
\end{definition}

Define\pageoriginale
$$
\delta F (w) = \sum_{i=1}^{n} \partial_i p(\ell (w), \ldots, \ell_n
(w)) < \ell_i, \ell > _H 
-p (\ell_1 (w), \ldots, \ell_n (w))\ell (w)
$$
and extend the definition to every $F \epsilon  P_{W}*$ by linearity.

\setcounter{proposition}{4}
\begin{proposition}%pro 1.5
  \begin{enumerate}[(i)]
  \item For every $F \epsilon  p$, $\delta (DF) = LF$. More
    generally if $F_1, F_2 \epsilon  p$, then 
    \begin{equation*}
      \delta (F_1 . DF_2) = < DF_1, DF_2 >_H + F_1 . L
      (F_2).\tag{1.4}\label{eq1.4} 
    \end{equation*}
  \item {\em (Formula for integration by parts)}

    In $F \epsilon  P$ and $G \epsilon  P_{W}*$, then
    \begin{equation*}
      \int\limits_{W} < G, DF >_{H} (w) \mu (dw) = -\int\limits_{W} \delta
      G (w) F (w) \mu (dw) \tag{1.5}\label{eq1.5}  
    \end{equation*}
    which says that $\delta = - D*$.
  \end{enumerate}
\end{proposition}

\noindent
\textit{Proof.}
  (i) follows easily from definitions. (ii) We may assume
  \begin{align*}
    G(w) & = p (\ell _1 (w), \ldots, \ell_n (w))\ell
    & F(w)  = q (\ell _1 (w), \ldots, \ell_n (w))
  \end{align*}
  where $\{ \ell_i\}$ is ONS in $\overset{\ast}{W}$. Then
  \begin{align*}
    < G, DF >_H &= \sum_{i=1}^{n} (\partial_i\, q)p < \ell_i, \ell >_H\\
    \delta G.F &= \sum_{i=1}^{n} (\partial_i\, p).q < \ell_i, \ell > _H
    - p.q \, \ell(w). \tag*{$\Box$} 
  \end{align*}

So\pageoriginale we have to prove that
\begin{multline*}
  \int\limits_{\mathbb{R}} \sum_{i=1}{n} (\partial_i q (\xi)). p(\xi)<
  \ell_i, \ell > _H e^{\frac{-|\xi|^2}{2}} d\xi\\ 
  =- \int\limits_{\mathbb{R}^n} \sum_{i=1}^{n} \left[(\partial _i p (\xi))q
    (\xi) < \ell_i, \ell > _H -p (\xi) q(\xi) < \ell _i, \ell >
    \xi_i\right]e^{\frac{-|\xi|^2}{2}} d\xi 
\end{multline*}
which follows immediately by integrating the $LHS$ by parts

\begin{proposition}[Chain rule]%pro 1.6
Let $P (t_1, \ldots, t_n)$ be a polynomial and $F_i \epsilon  P$,
for $i= 1,2, \ldots, n$. Let $F = P (F_1, F_2 \ldots F_n)
\epsilon  P$. Then  
$$
DF (w) = \sum_{i=1}^{n} \partial_i P(F_1 (w), F_2 (w), \ldots, F_n
(w)). DF_i (w) 
$$ 
and
\begin{multline*}
  LF (w)  = \sum_{i, j=1}^{n}\partial_i \partial_j P (F_1 (w), \ldots
 , F_n (w)). < DF_i DF_j> _H\\ 
 + \sum_{i=1}^{n} \partial _i P(F_1 (w) m \ldots, F_n (w)) \times LF_i (w).
\end{multline*}
\end{proposition}

\begin{proof}
   Easy.
\end{proof} 

\section{Sobolev Spaces over the Wiener Space}%%% 1.3

\begin{definition}%def 1.10
Let $F \epsilon  p, 1 < p < \infty, -\infty < s < \infty$. Then 
$$
|| F||_{p,s} \overset{\Delta}=|| (I-L)^{s/2} F ||_p
$$
where
$$
(I-L)^{s/2} F \overset{\Delta}{=} \sum_{n=0}^{\infty}(1+n)^{s/2} J_n F
\epsilon  P. 
$$
\end{definition}

\begin{proposition}%pro 1.7
  \begin{enumerate}[\rm (i)]
  \item If\pageoriginale $p \leq p'$ and $s \leq s'$, then
    $$
    || F||_{p,s} \leq || F||_{p's'} ~\forall~ F \epsilon p.
    $$

  \item $\forall~ 1 < p < \infty, -\infty < s < \infty, || . || _{p,s}$
    are compatible in the sense that if, for any $(p,s)$, $(p', s')$
    and $F_n \epsilon  p, n = 0,1,2, \ldots, || F_n ||_{p,s} \to
    0$ and $|| F_n - F_m ||_{p',s'}\to 0$ as $n,m \to \infty$, then
    $|| F_n || _{p',s'}\to 0$ as $n \to \infty$ 
  \end{enumerate}
\end{proposition}

\begin{proof}
\begin{enumerate}
\renewcommand{\theenumi}{\roman{enumi}}
\renewcommand{\labelenumi}{(\theenumi)}
\item Since, for fixed $s, || F ||_{p,s} \leq || F ||_{p',s'}$ if $p'
  > p$, it is enough to prove 
  $$ 
  || F ||_{p,s} \leq || F ||_{p,s'} ~\text{for}~ s' \geq s.
  $$

To prove this, it is sufficient to show that for $\alpha > o$,
$$
|| (I - L)^{-\alpha}F||_p \leq || F||_p ~\forall~ F \epsilon  P.
$$

We know that $|| T_t F ||_p \leq || F ||_p$. From the Wiener-Chaos
representation for $T_t F$ and $(I - L)^{-\alpha} F$, we have  
$$
(I - L)^{-\alpha}F = \frac{1}{\Gamma (\alpha)} \int\limits_{o}^{\infty}
e^{-t} t^{\alpha -1} T_t F dt. 
$$

Hence
\begin{align*}
  || (I-L)^{-\alpha} F||_p & \leq \frac{1}{\Gamma (\alpha)}
  \int\limits_{o}^{\infty} e^{-t} t^{\alpha -1} || T_t F||_p dt\\ 
  & \leq || F || _p
\end{align*}
which proves the result.

\item  Let\pageoriginale $G_n = (I-L)^{s' /2} F_n \epsilon
  \mathcal{P}$. Therefore  $|| G_n - G_m ||_{p'} \to 0$ as $n, m \to
  \infty$. Therefore, $\exists ~G 
\epsilon  L_p$, such that $|| G_n - G||_{p'} \to 0$. But 
$$
|| F_n ||_{p,s}\to 0 \Rightarrow || (I -L)^{1/2 (s-s')} G_n ||_p \to 0.
$$

Enough to show $G=0$. Let $H \epsilon  P$. Then $(I-L)^{1/2 (s' -s)}
H \epsilon  P$. Noting that $P \subset L_q$ for every $1< q<
\infty$, we have 
\begin{align*}
  \int\limits_W G.H d\mu & = \lim_{n \to \infty} \int\limits_{W} G_n H d\mu \\
  & = \lim_{n \to \infty} \int\limits_{W} (I - L )^{1/2 (s-s')} G_n
  (I- L)^{1/2 (s'-s)} H d \mu\\ 
  & = 0.
\end{align*}
Since $\mathcal{P}$ is hence in $L_p ~\forall~ q, G =0$.
\end{enumerate}
\end{proof}

\begin{definition}%def 1.11
Let $1 < p < \infty, - \infty < s < \infty$. Define
$\mathbb{D}_{p,s}= $ the completion of $\mathcal{P}$ by the norm
$||~~||_{p,s}$. 
\end{definition} 

\begin{fact}
  \begin{enumerate}[\rm 1)]
  \item $\mathbb{D}_{p,o} = L_p$.
  \item $\mathbb{D}_{p',s'} \hookrightarrow \mathbb{D}_{p,s}$ if $p \leq
    p', s \leq s'$. 

    Hence we have the following inclusions:
    
    Let $o < \alpha < \beta, o < p < q < \infty$. Then
    \begin{gather*}
      \mathbb{D}_{\substack{p,\beta\\ \circlearrowleft}} \hookrightarrow
      \mathbb{D}_{\substack{p,\alpha \\ \circlearrowleft}} \hookrightarrow
      \mathbb{D}_{p,o} = L_p \hookrightarrow
      \mathbb{D}_{\substack{p,-\alpha \\ \circlearrowleft}}
      \hookrightarrow  \mathbb{D}_{\substack{p,-\beta\\ \circlearrowleft}}\\ 
      \mathbb{D}_{q, \beta} \hookrightarrow  \mathbb{D}_{q, \alpha}
      \hookrightarrow  \mathbb{D}_{q, o} = L_q \hookrightarrow
      \mathbb{D}_{q,-\alpha} \hookrightarrow  \mathbb{D}_{q, -\beta} 
    \end{gather*}

  \item  Dual\pageoriginale of $D_{p, s}  \equiv D'_{p, s} = D_{q, -s}$ where
    $\dfrac{1}{p} + \dfrac{1}{q} = 1$, under the standard
    identification $(L_2)' = L_2$. 
  \end{enumerate}
\end{fact}

This follows from the following facts:

Let $A = (I - L)^{-s/2}$. Then the following maps are isometric isomorphisms:
\begin{align*}
  A: & L_p \to \mathbb{D}_{p, s}\\
  A: & \mathbb{D}_{q, - s} \to L_q
\end{align*}
and hence
$$
\overset{\ast}{A}: (\mathbb{D}_{p, s})' \to L_q
$$
is also an isometric isomorphism if $\dfrac{1}{p} + \dfrac{1}{q} = 1$.

Also, from the relation
$$
\int\limits_w F(w) G (w) \mu (dw) = \int\limits_W (I - L)^{s / 2} F(w)
(I-L)^{-s/2} G (w) \mu (dw), 
$$
it is easy to see that $\mathbb{D}_{q, - s} \subset (\mathbb{D}_{p,
  s})'$, isometrically. 


\begin{definition} % def 1.12
  \begin{align*}
    \mathbb{D}_{\infty} & = \bigcap_{p, s} \mathbb{D}_{p, s}\\
    \mathbb{D}_{- \infty} & = U_{p, s} \mathbb{D}_{p, s}\\
    ({\rm Hence } \quad ~~ \mathbb{D}'_{\infty} & = U \mathbb{D}'_{p, s}
    = \mathbb{D}_{- \infty}. )
  \end{align*}

Thus $\mathbb{D}_ \infty$ is a complete countably normed space and
$\mathbb{D}_{- \infty} $ is its dual. 
\end{definition}

\begin{remark*} % rem
  Let $S (\mathbb{R}^d)$ be the Schwartz space of rapidly decreasing
  $C^ \infty - $ functions, $H_{p, s}$ the (classical) Sobolev
  space\pageoriginale 
  obtained by completing $S (\mathbb{R}^d)$ by the norm  
  $$
  || f ||_{p, s} = || ( | x |^2 - \triangle)^{s/2} f ||_p, f \epsilon 
  S(\mathbb{R}^d) 
  $$
  where $\triangle$ denotes the Laplacian. Then it is well-known that 
  \begin{align*}
    \bigcap_{p, s} H_{p, s} & = ~ =~ \bigcap_{s} H_{2, s}\\
    U_{p, s} \; H_{p, s} & = ~ = ~ U_s \; H_{2, s}.
  \end{align*}
\end{remark*} 
 
Thus every element in $\bigcap\limits_{p, s} H_{p, s}$ has a
continuous modification, actually $a ~ C^ \infty$ - modification. But
in our case, the analogous results are not true. 

First, in our case, $\bigcap_s \mathbb{D}_{2, s} \neq
\mathbb{D}_\infty$. Secondly, $\exists~ F\epsilon  \mathbb{D}_ \infty$
which has no continuous modification on $W$, as the following example
shows.  

\begin{example}\label{chap1:exam1.3}%example 1.3
  Let $W = W^2_o = \left\{ w \epsilon  C ([0, 1] \to \mathbb{R}^2), w(0)
  = 0  \right\} \mu = P \equiv 2 - \dim$. Wiener measure. Let, for $w =
  (w_1, w_2) \epsilon  W$, 
  $$
  F(w) = \frac{1}{2} \left\{ \int\limits_{o}^{1} w_1 (s) dw_2 (s) -
  \int\limits_{o}^{1} w_2 (s) dw_1 (s) \right\} 
  $$
  (stochastic area of Levy) where the integrals are in the sense of
  It$\hat{o}$'s stochastic integrals. 
 \end{example} 

Then $F \epsilon  C_2 \subset \mathbb{D}_ \infty$. But $F$ has no
continuous modification: suppose $\exists~ \,\hat{F} (w)$, continuous and
such that $\hat{F} (w) = F(w)\, a.a. \,w(p)$. Let 
$$
\hat{ \hat {F}} (w) = \frac{1}{2} \left[\int\limits_o^1 (w_1 (s)
  \dot{w}_2 (s) - w_2 (s) \dot{w}_1 (s)) ds \right] 
$$
for\pageoriginale $w \epsilon  C^2_o ([0, 1] \to \mathbb{R}^2)$. Note that
$\hat{\hat{F}}$ has no continuous extension to $W^2_o$. On the other
hand, we have the following fact: For $\delta > o$, 
$$
\displaylines{\hfill
P \left\{ | F (w) - \hat{\hat{F}} (\phi) | < \delta | || w - \phi || <
\epsilon  \right\} \to 1\hfill\cr 
\text{as}\hfill
\epsilon  \downarrow o, ~\forall~ \phi \epsilon  C^2_o ([0, 1] \to
\mathbb{R}^2).\hfill} 
$$

Hence
$$
\hat{F} \equiv \hat{\hat{F}} \text {  on  } C^2_o ([0, 1] \to
\mathbb{R}^2), \text { a contradiction. } 
$$

\begin{definition}%defini 1.13
  Let $F \epsilon  P$. Then
  $$
  D^k F(w) \epsilon  \underbrace{W^*
    \otimes \cdots \otimes W^*}_{K ~\text{times}} 
  $$
  and we define the \textit{Hilbert-Schmidt norm} of $D^k F(w) $ as 
  $$
  | D^k F(w) |^2_{HS} = \sum_{i_1, \ldots, i_k=o}^{\infty} \left\{ D^k
  F(w) \left[h_{i_1}, \ldots, h_{i_k}\right] \right\}^2 
  $$
  where $\{ h_i\}^{\infty}_{i = 1}$ is an $ONB$ in $H$.
\end{definition}

\begin{remark*} % rem
  \begin{enumerate} [1)]
  \item The definition is independent of the $ONB$ chosen.
  \item If $k = 1$, then $| DF (w) |^{2}_{HS} = |DF [w] |^2_H$.
  \end{enumerate}
\end{remark*}

\setcounter{theorem}{7}
\begin{theorem}[Meyer]\label{chap1:thm1.8} % the 1.8
  For $1 < p < \infty, k \epsilon  Z^+$, there exist $A_{p, k} > a_{p,
    k} > 0 $ such that 
  \begin{equation*}
    a_{p, k} || | D^k F |_{HS} ||_p \le || F ||_{p, k} \le A_{p, k} (||
    F||_p + || | D^k F |_{HS} ||_p ) \tag{1.5}\label{thmeq1.5}  
  \end{equation*}
  for every $F \epsilon  \mathcal{P}$.
\end{theorem}

Before\pageoriginale proving this result, let us consider the
analogous result in 
classical analysis, which can be stated as: 

For $1 < p < \infty$, there exists $a_p > 0$ such that 
\begin{equation*}
  a_p || \frac{\partial^2 f}{\partial x_i \partial x_j} ||_p \le ||
  \Delta f ||_p, ~\forall~ f \epsilon  \mathcal{S} (\mathbb{R}^d),
  \tag{1.6}\label{eq1.6}  
\end{equation*}
where $\mathcal{S} (\mathbb{R}^d)$ denotes the Schwartz class of $C^
\infty$ - rapidly decreasing functions. 

\medskip
\noindent{\textbf{Proof of (\ref{eq1.6}):}}
Let $p = 2$, then
\begin{align*}
  || \frac{\partial^2 f}{\partial x_i \partial x_j} ||_2 &= || \xi_i \xi
  _j \hat{f} (\xi) ||_2, ~{\rm where}~ \hat{f} (\xi) =
  \int\limits_{\mathbb {R}^d} e^{\sqrt{-1} \xi. x} f (x) dx\\ 
  & \le  ~ C_p || | \xi |^2 ~ \hat{f} (\xi) ||^2_2\\
  & =  ~ C_p || \triangle f ||_2.
\end{align*}

For the general case, we need Calderon-Zygmund theory of singular
integrals or Littlewood-Paley inequalities. We here consider the
Littlewood-Paley inequalities. 

Consider the semigroups $P_t$ and $Q_t$ defined as follows:
$$
 \displaylines{\hfill P_t = e^{t \triangle},\hfill\cr
   \text{i.e.,}\hfill (P_t f)\hat{}(\xi) = e^{-t| \xi |^2 } \hat{f}
  (\xi), f \epsilon \mathcal{S} (\mathbb{R}^d)\hfill\cr 
   \text{and}\hfill Q_t = e^{-t (- \triangle)^{1/2}}\hfill\cr
   \text{i.e.,} \hfill (Q_t f)\hat{ } (\xi) = e^{-t | \xi | } \hat{f} (\xi), f
   \epsilon  \mathcal{S} (\mathbb{R}^d) \hfill\cr
   \text{where}\hfill
   \hat{f} (\xi) = \int\limits_{\mathbb{R}^d} e^{\sqrt{-1 } \xi. x }
   f(x) dx.\hfill} 
$$

The\pageoriginale transition from $P_t$ to $Q_t$ is called
\textit{subordination of Bochner} and is given by  
$$
Q_t = \int\limits_{o}^{\infty} P_s \mu_t (ds)
$$
where $\mu_t$ is defined as
$$
\int\limits_{o}^{\infty} e^{- \lambda s} \mu_t (ds) = e^{- \sqrt{\lambda t}}.
$$

Note that $Q_t$ can also be expressed as
$$
\displaylines{\hfill
  Q_t f(x) = \int\limits_{\mathbb{R}^d} \frac{c_n
    t}{(t^2+|x-y|^2)^{(d+1)/2}}  f(y) dy\hfill \cr
  \text{where}\hfill
  c^{-1}_{n} = \int\limits_{\mathbb{R}^d} \frac{1}{(1+ | y
    |^2)^{(d+1)/2}} dy.\hfill} 
$$

Now, we define \textit{Littlewood-Paley functions} $G_f$ and $G_f
\to$, $f \epsilon  \mathcal{S} (\mathbb{R}^d)$ as: 
$$
\displaylines{\hfill
  G_f (x) = \left[\int\limits_{o}^{\infty} t \left\{| \frac{\partial}{\partial t}
    Q_t f (x) |^2 + \sum^{d}_{i = 1} Q_t f(x) |^2 \right\} dt
    \right]^{1/2} \hfill \cr
  \text{and}\hfill
  G_{f^\to} (x) = \left[\int\limits_o^\infty \left\{ t |
    \frac{\partial}{\partial t} Q_t f (x) |^2 \right\} dt
    \right]^{1/2}.\hfill}  
$$

\begin{fact}{\em (Littlewood-Paley Inequalities):}
  For $1 < p < \infty, \exists\, o < a_p < A_p $ such that 
  \begin{equation*}
    a_p || G_f (x) ||_p \le || f ||_p \le A_p || G_{f^\to} (x) ||_p,
    ~\forall~ f \epsilon  \mathcal{S} (\mathbb{R}^d).\tag{1.7}\label{eq1.7}  
  \end{equation*}
\end{fact}

Define\pageoriginale the operator $R_j$ by
$$
(R_j f)\hat{} (\xi) = \frac{\xi_j}{| \xi |} \hat{f} (\xi)
$$
$R_j$ is called the \textit {Riesz transformation}. In particular,
when $d= 1$, it is called \textit{Hilbert transform}. It is clear that  
$$
\frac{\partial^2}{\partial x_j \partial x_j} f(x) = R_i R_j \triangle f(x).
$$

\begin{fact}
For $1 < p < \infty, \exists\, o < a_p < \infty$ such that 
\begin{equation*}
  a_p || R_j f ||_p \le || f ||_p.\tag{1.8}\label{eq1.8} 
\end{equation*}
\end{fact}

Note that (\ref{eq1.6}) follows from (\ref{eq1.8}). Hence we prove
(\ref{eq1.8}). We have 
\begin{align*}
  (R_j Q_t f) \hat{} (\xi) & = \frac{\xi _ j}{| \xi |} e^{-t | \xi | }
  \hat{f} (\xi) \\ 
  & = (Q_t R_j f) \hat{} (\xi).
\end{align*}

Also
$$
\sqrt{-1} \frac{\partial}{\partial t} R_j (Q_t f) (x) =
\frac{\partial}{\partial x_j} Q_t f (x). 
$$

Hence we get 
$$
G_{\overset{\to}{R_j f}} \le G_f,
$$
which gives (\ref{eq1.8}), by using (\ref{eq1.7}). Now, we come to
Meyer's theorem. 

\setcounter{proofoftheorem}{7}
\begin{proofoftheorem}%%% 1.8
\begin{step}\label{chap1:step1}%%% 1
  Using the $0 - U$ semigroup $T_t$, we define $Q_t $ by
  $$
  \displaylines{\hfill
  Q_t = \int\limits_o^\infty T_s \mu_t (ds)\hfill\cr
  \text{where}\hfill
  \int\limits_{o}^{\infty} e^{- \lambda s} \mu_t (ds) = e^{-
    \sqrt{\lambda}t}.\hfill }
  $$
\end{step} 
\end{proofoftheorem}\pageoriginale

Note that
$$
Q_t = \sum_{n = o}^{\infty} e^{- \sqrt{n} t} J_n.
$$

$F \epsilon  \mathcal{P}$, we define $G_F$ and $\psi_F$ as follows:
$$
\displaylines{\hfill
G_F (w) = \left[ \int\limits_{o}^{\infty} t ( \frac{\partial}{\partial t}
  Q_t F (w))^2 dt \right]^{1/2}\hfill\cr 
\text{and}\hfill
\psi_F (w) = \left[ \int\limits_0^\infty \left\{T_t (< DT_t F, DT_t F >^{1/2}_H
  ) (w) \right\}^2 dt \right]^{1/2}.\hfill} 
$$

Then the following are true:

For $1 < p < \infty, \exists\, o < c_p < C_p < \infty$ such that
\begin{gather*}
  c_p || F ||_p \le || G_F ||_p \le C_p || F ||_p,\\
  c_p || F ||_p \le || \psi_F ||_p \le C_p || F ||_p, ~\forall~ F
  \epsilon  \mathcal{P} \text { such that } J_o F = 0. \tag{1.9}\label{eq1.9}   
\end{gather*}

\begin{proof}
  Omitted.
\end{proof}

\begin{step}[An $L_p$-multiplier theorem]\label{chap1:step2}%%% 2
A linear operator $T_{\phi}: \mathcal{P} \to \mathcal{P}$ is said to
be \textit{given by a multiplier} $\phi = (\phi (n)), $ if 
$$
T_{\phi} F = \sum_{n = 1}^{\infty} \phi (n) J_n F, ~\forall~ F
\epsilon  \mathcal{P}. 
$$

Note\pageoriginale that the operators $T_t, Q_t$ and $L$  are given by the
multipliers $e^{nt}, e^{-\sqrt{n} t}$ and $(-n)$ respectively. 
\end{step}

\begin{fact}{\em (Meyer-Shigekawa):}
  If $\phi (n) = \sum\limits_{k = o}^{\infty} a_k
  \left(\dfrac{1}{n^\alpha}\right)^k$, $\alpha \ge o$ for $n \ge n_o$ for some
  $n_o$ and $\sum\limits_{k = o}^{\infty} | a_k |
  \left(\dfrac{1}{n^\alpha_o}\right)^k < \infty$, then $\exists\, c_p$ such that  
  \begin{equation*}
    || T_\phi F ||_p \le c_p || F ||_p, ~\forall~ F \epsilon
    \mathcal{P}.\tag{1.10}\label{eq1.10}  
  \end{equation*}
\end{fact}

Note that the hypothesis in the above fact is equivalent to: there
exists $h(x)$ analytic, i.e., $h(x) = \sum a_k x^k$, near zero such
that 
$$
\phi (n) = h \left(\frac{1}{n^\alpha}\right) \text { for } n \ge n_o.
$$

\medskip
\noindent{\textbf{Proof of (\ref{eq1.10}):}}
  First, we consider the case $\alpha = 1$. We have 
  \begin{align*}
    T_\phi & = \sum_{n = o}^{n_o -1} \phi (n) J_n + \sum_{n =
      n_o}^{\infty} \phi (n) J_n\\ 
    & = T_\phi^{(1)} + T_\phi^{(2)}.
  \end{align*}

We know that $T_\phi^{(1)}$ is $L_p$-bounded as a consequence of
hyper contractivity, i.e., 
$$
|| T^{(1)} F ||_p \le c_p || F ||_p.
$$

Hence it is enough to show that
$$
|| T_\phi^{(2)} F || \le c_p || F ||_p.
$$

\medskip
  \begin{equation*}
    \textbf{Claim: }\quad || T_t (I - J_o - J_1 - \cdots - J_{n_o -1}) F ||_p
    \le C e^{-n_o       t} || F ||_p.\tag{1.11}\label{eq1.11}   
  \end{equation*}

  Let $p > 2$. Choose $t_o$ such that $p = e^{2_{t_o}} + 1$. Then by
  Nelson's theorem, 
  \begin{align*} 
    || T_{t_o} T_t & (I - J_o - J_1 - \cdots - J_{n_o -1}) F ||^2_p\\
    & \quad \le   ||T_t (I - J_o - J_1 - \cdots - J_{n_o -1}) F ||^2_2\\
    & \quad =   || \sum_{n = n_o}^{\infty} e^{-nt} J_n F ||^2_2\\
    & \quad =  \sum_{n = n_o}^{\infty} e^{-2n_o t} || J_n F ||^2_2\\
    & \quad \le  e^{-2n_o t} || F ||_p^2.
  \end{align*}\pageoriginale

Therefore
$$
|| T_t (I - J_o - J_1 - \cdots - J_{n_o -1})F ||_p \le C e^{-n_o t} || F ||_p
$$
where $C = e^{n_o t_o}$.

For $1 < p < 2$, the result (\ref{eq1.11}) follows by duality. Define
$$
R_{n_o} = \int\limits_o^\infty T_t(I - J_o - J_1 - \cdots - J_{n_o -1}) dt.
$$

From (\ref{eq1.11}), we get 
$$
|| R_{n_o} F ||_p \le C \frac{1}{n_o} || F ||_p
$$
and it is clear that  
{\fontsize{10pt}{12pt}\selectfont
\begin{align*}
  R_{n_o}^2 F & =\int\limits_o^\infty \int\limits_{o}^\infty T_t (I -
  J_o - J_1 - \cdots - J_{n_o -1}) T_s (I - J_o - \cdots - J_{n_o -1})
  Fdtds\\ 
  & =\int\limits_o^\infty \int\limits_{o}^{\infty} T_{t + s} (I - J_o -
  J_1 - \cdots - J_{n_o -1}) F dtds. 
\end{align*}}\relax

Hence
$$
|| R^2_{n_o} F ||_p \le C. \frac{1}{n^2_o} || F ||_p
$$
and repeating this, we get
$$
|| R^k_{n_o} F ||_p \le C. \frac{1}{n^k_o} || F ||_p.
$$

Also,\pageoriginale note that if $F \epsilon  C_n, n \ge n_o$
\begin{align*}
  R_{n_o} F & = \int\limits_{o}^{\infty} T_t J_n F dt\\
           & = \frac {1}{n} J_n F
\end{align*}
and
$$
R^k_{n_o} F = \frac{1}{n^k} J_n F.
$$

Therefore
$$
T_\phi^{(2)} F = \sum_{n = n_o}^{\infty} \sum_{k = o}^{\infty} a_k
R_{n_o}^{k} J_n F = \sum_{k = 1}^{\infty} a_k R_{n_o}^k F. 
$$

Hence
$$
|| T_\phi^{(2)} F ||_p \le U \left(\sum_k | a_k |
\left(\frac{1}{n_o}\right)^k\right) || F ||_p 
$$
which gives the result.

For the general case, i.e., $o < \alpha < 1$, define
$$
Q^\alpha_t = \sum e^{-n^\alpha t} J_n F = \int\limits_o^\infty T_s \mu_t^{(\alpha)} (ds)
$$
where
$$
\int\limits_{o}^{\infty} e^{-\lambda s} \mu_t^{(\alpha)} (ds) = e^{- \lambda^\alpha t}.
$$

As in the case $\alpha = 1$, write
$$
T_\phi = T_\phi^{(1)} + T_\phi^{(2)}.
$$

In this case also, we see that $T_\phi^{(1)}$ is $L_p$ -
bounded. Using (\ref{eq1.11}), 
\begin{align*}
  || Q_t^{(\alpha)} & (I - J_o - J_1 - \cdots - J_{n_o -1}) F ||_p\\
  & \qquad \le C \int\limits_{o}^{\infty} || F ||_p e^{-n_o s}
  \mu_t^{(\alpha)} (ds)\\ 
  & \qquad = C e^{-n_o^\alpha t} || F ||_p. 
\end{align*}

Define\pageoriginale
$$
R_{n_o} = \int\limits_o^\infty Q_t^{(\alpha)} (I - J_o - J_1 - \cdots
- J_{n_o -1}) dt 
$$
and proceeding as in the case $\alpha = 1$, we get that $T_\phi^{(2)}$
is also $L_p$ - bounded. Hence the proof of (\ref{eq1.10}). 

\begin{remark*}
  (Application of $L_p$ - Multiplier Theorem)
\end{remark*}

Consider the semigroup $\{ Q_t \}_{t \ge o}$. For $F \epsilon 
\mathcal{P}$, we have 
$$
Q_t F = \sum_{n = o}^{\infty} e^{- \sqrt{n} t} J_n F.
$$

The generator $C$ of this semigroup is given by 
$$
CF = \sum_{n = o}^{\infty} (- \sqrt{n}) J_n F, F \epsilon  \mathcal{P}.
$$

If we define $||| . |||_{p, s} $ for $F \epsilon  \mathcal{P}$ by
$$
||| F |||_{p, s} = || (I - C)^s F ||_{\dot{p}}, 1< p < \infty, -
\infty < s < \infty 
$$
where $(I - C)^s F = \sum_{n = o}^{\infty} (I + \sqrt{n})^s J_n F$,
then $|| ~ ||_{p, s}$ is equivalent to $||| . |||_{p, s}$, $\forall~ 1
< p < \infty, -\infty < s < \infty$. i.e., $\exists ~ a_{p, s }$, 
$A_{p, s }$, $o < a_{p, s} < A_{p, s} < \infty \ni a_{p, s} ||| F |||_{p,
  s} \le || F ||_{p, s} \le A_{p, s} ||| F |||_{p, s}$. 

\begin{proof}
  Let $T_\phi F = \sum\limits_{n = o}^{\infty} \phi (n) J_n F, F
  \epsilon  \mathcal{P}$, where 
  \begin{align*}
    \phi (n) & = \left(\frac{1+ \surd n}{\sqrt{1 + n}}\right)^s, - \infty
    < s < \infty \\ 
    & = h \left(\left(\frac{1}{n}\right)^{1/2}\right)
  \end{align*}
  with  $h(x)  = \left(\frac{1 + x}{\sqrt{(1 + x^2)}}\right)^s$  which
  is analytic near the origin. 
\end{proof}

Note that $T_\phi^{-1} = T_{\phi^{-1}}$ where $\phi^{-1 } (n) =
\dfrac{1}{\phi (n)} = h^{-1}
\left(\left(\dfrac{1}{n}\right)^{1/2}\right)$ with $h^{-1} (x) 
= \dfrac{1}{h (x)}$ also analytic near the origin. Thus
both\pageoriginale $T\phi$ and 
$T_\phi^{-1}$ are bounded operators on $L_p$. Further,  
$$
\displaylines{\hfill
(I - C)^s F = (I - L)^{s/2} T_\phi F = T_\phi (I -L)^{s/2} F\hfill \cr
\text{and}\hfill
(I - L)^{s/2} F = T_\phi^{-1} (I -C)^s F = T_{\phi^{-1}} (I - C)^s F.\hfill}
$$

Hence our result follows easily from the fact that
$$
|| T_\phi F ||_p \le C_p || F ||_p \text { and } || T_{\phi^{-1}} F
||_p \le C_p || F ||_p. 
$$

To proceed further, we need the following inequality of Kchinchine.


\medskip
\noindent{\textbf{Kchinchine's Inequality:}}
 Let $(\Omega, F,
p)$ be a probability space. Let $\{ \gamma_m (\omega) \}^\infty_{m =
  1}$ be a sequence of i.i.d. random variables on $\Omega$ with
$P(\gamma_m = 1) = P(\gamma_m = -1) = 1/2$, i.e., $\{ \gamma_m (\omega)
\}$ is a coin tossing sequence. 
\begin{enumerate}[a)]
\item If $\{ a_m \}$ is a sequence of real numbers, then, $\forall~ 1 <
  p < \infty, \exists\, o < c_p < C_p < \infty$ independent of $\{
  a_m\}$ such that 
  \begin{align*}
    c_p \left(\sum_{m = 1}^{\infty} |a_m |^2\right)^{p/2} & \le E 
    \left( | \sum_{ m =  1}^{\infty} a_m \gamma_m (\omega) |^p\right)\\  
    & \le C_p \left( \sum_{m=1}^{\infty} |a_m
    |^2\right)^{p/2}.\tag{1.12}\label{eq1.12}  
  \end{align*}

\item If $\{ a_{m, m'} \}$ is a (double) sequence of real numbers,
  then, $\forall~ 1 < p < \infty, \exists~ o < c_p < C_p < \infty$
  independent of $\{ a_{m, m'} \}$ such that  
  \begin{align*}
    c_p \left(\sum_{m, m'}^{\infty} | a_{m, m'} |^2 \right)^{p/2} &
    \le E \left[ \left\{
      \sum_{m' = 1}^{\infty} (\sum_{m' = 1}^{\infty} a_{m', m}\,
      \gamma_m (\omega))^2  \right\}^{p/2} \right]\\ 
    & \le C_p \left(\sum_{m, m = 1}^{\infty} | a^2_{m, m'}
    |\right)^{p/2}.\tag{1.13}\label{eq1.13}  
  \end{align*} 

\item  Let\pageoriginale $((a_{mm'})) \geq o$ i.e.,  for any finite $m_1< m_2<
  \cdots <m_n$, the matrix $((a_{m_i m_j}))_{1 \leq i, j \leq n}$ is
  positive definite. Then, $\forall~ 1 < p < \infty, \exists\, o < C_p <
  c_p < \infty$ independence of $(a_{mm'})$ such that 
  \begin{align*}
    c_p \left(\sum _i a_{ii}\right)^{p/2} &\leq E \left[\left(\sum
      _{i, j}a_{ij}\gamma _i (\omega ) \gamma_j (\omega
      )\right)^{p/2}\right] \\
    & \leq C_p \left(\sum_i a_{ii}\right)^{p/2}. \tag{1.14}\label{eq1.14} 
  \end{align*}
\end{enumerate}

\begin{step}\label{chap1:step3}%3
  (Extension of L-P inequalities to sequence of functionals).

  Let $F_n \in \mathcal{P}, n=1,2, \ldots$ with $J_o F_n=0$. Then 
  $$
  || \surd \left(\sum^{\infty} _{{n=1}}(F_n)^2\right) ||_p \leq A'_p || \surd
  \left(\sum^{\infty} _{{n=1}} G^2_{F_n}\right) ||_p, ~\forall~ 1 < p < \infty. 
  $$
\end{step}

\begin{proof}
  Let $\{ \gamma _i (\omega)\}$ be a coin tossing sequence on a
  probability space $(\Omega, F, P)$. 
\end{proof}

Let $\chi (\omega,w) = \sum_i \gamma (\omega)F_i (w), \omega \in
\Omega_1, w \epsilon  W$. 

We first consider the case when $F_n \equiv 0, ~\forall~ n \geq
N$. (Hence the above sum is finite). Then the general case can be
obtained by a limiting argument. By Kchinchine's inequality, $\exists$
constants $c_p, C_p$ independent of $w$ such that 
\begin{align*}
  c_p \left(\sum_i F_i (W)^2\right)^{p/2}& \leq E|X(\omega,w)|^p\\
  &\leq C_p\left(\sum_i F_i (W)^2\right)^{p/2} ~\forall~ = w \epsilon  W.
\end{align*}

Integrating w.r.t. $\mu$, we get
\begin{align*}
  c_p || \left(\sum _i F^2_i\right)^{1/2} ||^p_p &\leq E \left\{ || X(\omega, W)
  ||^p_p \right\} \tag{1.15}\label{eq1.15}  \\ 
  &\leq C_p || \left(\sum _i F^2_i\right)^{1/2} ||_p^p.
\end{align*}

But\pageoriginale by step \ref{chap1:step1}, we have
\begin{equation*}
  || \chi (\omega, .) ||_p \leq A_p || G_X (\omega,.) ||_p ~\forall~
  \omega \epsilon  \Omega . \tag{1.16}\label{eq1.16} 
\end{equation*}

Now
\begin{align*}
  (G_{\rho (\omega,.)})^2 &= \left[\int \limits^\infty _o t
    \left[\frac{d}{dt}Q_t \left(\sum_i \gamma_i (\omega )F_i
      (.)\right)\right]^2 dt\right]\\  
  &= \sum_{i, j} \gamma _i (\omega )\gamma_j (\omega )a_{ij},
\end{align*}
where
$$
a_{i j}= \int\limits^\infty _o t \left(\frac{d}{dt} Q_t F_i\right)
\left(\frac{d}{dt} Q_t F_j\right) dt.
$$

Also
\begin{align*}
  \sum _i a_{ij} & = \sum_i \int\limits ^t_o t \left(\frac{d}{dt} Q_t
  F_i\right)^2 dt \\ 
  &= \sum_i G^2_{F_i}.
\end{align*}

Then Kchinchine's inequality $(c)$ implies
\begin{align*}
  c_p \left(\sum_i G_{F_i}(W)^2\right)^{p/2} & \leq E | G_{X(., w)}|^p\\
  & \leq C_p \left(\sum_i G_{F_i}(W)^2\right)^{p/2}
\end{align*}
where $o < c_p <C_p < \infty$.

Integrating over $\mu$, we get
\begin{equation*}
  c_p || \surd \left(\sum_i G^2_{F_i}\right) ||^p_p \leq E || G_{\chi (., .)} ||
  ^p_p \leq C_p || \surd \left(\sum_i G^2_{F_i}\right)
  ||^p_p. \tag{1.17}\label{eq1.17}  
\end{equation*}
(\ref{eq1.15}), (\ref{eq1.16}) and (\ref{eq1.17}) together prove step
\ref{chap1:step3}. 

\begin{step}[Commutation relations involving $D$]\label{chap1:step4}%%%% 4
Let $\{ \ell_i \} ^\infty_{i=1}\subset \overset{\ast}{W} \subset H, \{
\ell_i\}$ an 
$ONB$ in $H$. Let $D_iF = < DF, \ell_i > $, for $F \epsilon 
\mathcal{P}$. Then $D_i F \epsilon  \mathcal{P}, ~\forall~ i$. Further,  
$$
< DF, DF >_H = \sum _i (D_i F)^2 = | DF |^2_{HS}.
$$\pageoriginale

In fact,
$$
| D^k F |^2_{HS}= \sum_{i_1, \ldots,i_k} (D_{i_1} (D_{i_2}( \cdots
\cdots ( D_{i_k} (F)) \cdots ))^2. 
$$
\end{step}

Let 
\begin{align*}
  T_\phi &= \sum^{\infty}_{n=o}\phi (n)J_n,\\
  T_{\phi +} &= \sum^{\infty}_{n = o}\phi (n + 1)J_n.
\end{align*}

\begin{fact}%4
  $\forall~ i = 1, 2,\ldots, D_i T_\phi = T_{\phi}+D_i$.
\end{fact}

\noindent
\textit{Proof.}
We have seen that the set $\left\{ \sqrt{a} H_a (w), a \epsilon  A \right\}$ is
an $ONB$ in $L_2$. Therefore it suffices to prove 
\begin{equation*}
  D_i T_\phi H_a= T_ {\phi} + D_i H_a, ~\forall~ a \epsilon
  \Lambda. \tag*{$\Box$} 
\end{equation*}

If $a=(a_1,a_2, \ldots ..)$ with $a_i>o$, then let $a(i) = (a_1, a_2,
\ldots,a_{i-1}, a_i -1,a_{i+1}, \ldots)$. From  $H_a
(w)=\underset{i}{\prod} H_{a_i}(\ell_i (w))$, it can be easily seen that 
\begin{equation*}
  D_i H_a =
  \begin{cases}
    H_{a (i)} &\text{ if } a_i > o \\ 
    0 & \text{ if } a_i = o
  \end{cases}
\end{equation*}

Note that, if $| a | = n$,
$$
T_\phi H_a =\phi (n) H_a \left(\therefore H_a \epsilon  C_n\right)
$$
implies
$$
D_i T_\phi H_a = \phi (n) D_i H_a.
$$

If\pageoriginale $a_i >o$, then $D_i H_a = H_{a(i)}$ where $ | a (i) | = n-1$. Therefore
\begin{align*}
  D_i T_ \phi H_a &= \phi (n)H_{a(i)}.\\
  &=T_ {\phi +} H_{a(i)} = T_{\phi}+ D_i H_a.
\end{align*} 
 
 If $a_i = 0$, this relation still holds since both sides are zero.

 \begin{coro*}
   $T_t D_i F = e^t D_i T_t F, ~\forall~ i$ and hence  
   $$
   Q_i D_i F = D_i \int\limits^\infty_0 \mu_t (ds) e^s T_s F, ~\forall~
   i, ~\forall~ F \epsilon  \mathcal{P}. 
   $$
 \end{coro*}

\begin{step}%%% 5
  Now we use the previous steps to get the final conclusion.
\end{step} 
 
In the following $c_p$, $C_p$, $a_p$, $A_p$ are all positive constants
which may change in some cases, but which are all independent of the 
function $F$. 
 
 $1< p < \infty$ is given and fixed. First we shall prove 
\begin{equation*}
  c_p || < DF, DF >^{1/2}_H || _p \leq || CF ||_p \leq C_p || < DF, DF
  >^{1/2}_H ||_p \tag{1.18}\label{eq1.18}
\end{equation*}
where
$$
  C= \lim_{t \to o} \frac{Q_t - I}{t}  
  \text{i.e.,}  \quad  Cf = \sum_n  \left(-\sqrt{n}\right) J_n F.
 $$
 
 From corollary of step \ref{chap1:step4}, we have 
 \begin{gather*}
   T_t D_iF =e^t D_i T_t F, ~\forall~ F \epsilon  \mathcal{P}. \\
   T_t \left\{ \left(\sum_i f^2_i\right)^{1/2} \right\} \geq
   \left[\sum_i \left(T_t f_i\right)^{1/2}\right],
   ~\forall~ f_i \epsilon  \mathcal{P} 
 \end{gather*} 
  implies
 \begin{align*}
  T_t \left\{ \left(\sum_i ( D_i F)^2\right)^{1/2} \right\} &\geq
  \left[\sum_i (T_t D_i F)^{2}\right]^{1/2}\\ 
  & \geq e^t \left[\sum_i (D_t T_i F)^{2}\right]^{1/2}\\
\text{i.e } \hspace{1cm} T_t \surd (< DF, DF >_H) & \geq e^t \surd (<
DT_t  F, DT_t F>_H).  \hspace{1cm} 
 \end{align*}
  
  Changing\pageoriginale $F$ by $T_t F$,
  $$
  T_t (\surd (< DT_t F, DT_t F >_H)) \geq e^t \surd (<DT_{2t} F,
  DT_{2t}F >_H). 
  $$
    Now
\begin{align*}
   \psi _F & {\overset{\Delta}=} \left[\int\limits_o ^\infty \left\{ T_t(\surd
     (< DT_t F, DT_t F>_H)) \right\}^2 dt\right]^{1/2}\\ 
   & \geq \left\{ \int\limits^\infty _o e^{2t} < DT_{2t} F, DT_{2t} F
   >_H dt \right\}^{1/2}\\
   & = \text{ const}. \left\{ \int \limits^ \infty _o e^{t} < DT_{t}F, D
   T_{t} F >_H dt \right\}^{1/2}. 
\end{align*}  
  
Therefore, by the Littlewood-Paley inequality (Step \ref{chap1:step1}),
  \begin{equation*}
    || F || _p \geq C_p || \left\{ \int \limits ^ \infty _oe^{t} < DT_{t}
    F,  DT_{t} F >_H dt \right\}^{1/2} ||_p . \tag{1.19}\label{eq1.19} 
  \end{equation*}
  
  Substituting $T_uF$ for $F$ in (\ref{eq1.19}),
$$
e^{u/2} || T_uF || _p \geq C_p || \left\{ \int \limits ^ \infty
_oe^{s} < DT_{s} F, DT_{s} F >_H ds \right\}^{1/2} ||_p. 
$$  
  Therefore
{\fontsize{9pt}{11pt}\selectfont
\begin{align*}
  \int \limits^ \infty _o   e^u || T_u F ||_p & du\geq C_p
  \int\limits^\infty_o e^{u/2}|| \left\{ \int \limits ^ \infty _u e^{s}<
    DT_{s}F, DT_{s} F >_H ds \right\}^{1/2} ||_p du\\ 
  &\geq C_p || \int \limits^\infty_o e^{u/2} \left\{ \int \limits^
  \infty_u e^{s} < DT_{s}F, DT_{s} F>_H ds \right\}^{1/2} du||_p \\ 
  & \geq C_p || \left\{ \int\limits^\infty_o ds \left[ \int^\infty_o
    e^{u/2} T_{\{u \leq s\}} du \times e^{s/2} \surd (< DT_sF, DT_s
    F>_H)\right]^2 \right\}^{1/2}||_p\\
  &= C_p || \left\{ \int \limits^\infty _o \left[2(e^{s} - e^{s/2}) \surd
    ( < DT_{s} F, DT_s F >_H)\right]^2 ds \right\}^{1/2} ||_p \\ 
  & \geq 2C_p || \left[ \int \limits ^ \infty _o e^{2s} < DT_{s} F,DT_s F
    >_H ds \right]^{1/2} ||_p\\ 
  & \hspace{2cm}-2C_p || \left[ \int \limits ^ \infty _o e^{s} <
    DT_{s} F, DT_s F >_H ds \right]^{1/2} ||_p. 
\end{align*}}\relax

  
  Hence\pageoriginale by (\ref{eq1.19}),
  $$
  || \left[ \int \limits ^ \infty _o e^{2s}< DT_{s} F, DT_s F>_Hds
  \right]^{1/2} ||_p \leq d_p || F ||_p + A_p \int \limits ^ \infty _o e^u
  || T_u F || _p du.
  $$ 
  
By step \ref{chap1:step2}, we know that if $|| (J_o + J_1) F
  || = 0$, then  
 $$
 || T_u F || _p \leq C_p e^{-2u} || F ||_p.
 $$
  
 Therefore, if $(J_o+J_1)F=0$,
  
 \begin{equation*}
    || F ||_p \geq C_p || \left\{ \int \limits ^\infty _o e^{2s} < DT_s F,
    DT_s F >_H ds \right\} ^{1/2}||_p. \tag{1.20}\label{eq1.20} 
 \end{equation*}
  
  Suppose $F \epsilon  \mathcal{P}$ satisfies $(J_o + J_1) F =0 $. By
  step \ref{chap1:step3},  
  $$
  || <DF, DF>_H^{1/2} || _p = || \left\{ \sum ^\infty _{i=1} (D_iF)^2
  \right\}^{1/2} ||_p 
  $$  
\begin{align*}
  & \leq C_p || \left\{ \sum^\infty _{i=1} (G_{D_i}F)^2 \right\}^{1/2} ||_p\\
  &= C_p || \left\{ \sum ^\infty _{i=1} \int \limits ^\infty _o t
  (\frac{d}{dt} Q_t D_i F)^2 dt \right\}^{1/2} ||_p . \tag{*} 
\end{align*}
	
By step \ref{chap1:step4},

$Q_t D_i F=D_i \tilde{Q}_t F$ where $\tilde{Q}_t F = \sum\limits _n
e^{-\surd (n-1)t} J_n$ implying  
$$
\displaylines{\hfill
  \frac{d}{dt} Q_t D_i F = D_i \left(\frac{d}{dt}\tilde{Q}_t\right) = D_i
  \tilde{Q}_t ~\text{CRF}\hfill \cr 
  \text{where}\hfill
  RF = \sum^ \infty _{n =1} \surd (1- \frac{1}{n}) J_n F.\hfill }
$$

Hence
$$
\displaylines{(*)\hfill
  =C_p || \left\{ \int\limits^\infty _o t < D\tilde{Q}_t CRF, D\tilde{Q}_t
  CRF >_H dt \right\} ^{1/2} ||_p \hfill(**) }
$$
since\pageoriginale
\begin{align*}
  \tilde{Q}_t & =\int\limits ^ \infty _o \mu_t (ds) e^s T_s,\\
  &< D\tilde{Q}_t CRF, D\tilde{Q}_t CRF >_H^{1/2}\\
  & \leq \int \limits ^ \infty _o\mu_t (ds) e^s < DT_s CRF, DT_s CRF
  >_H^{1/2} ds \\ 
  & \leq \left[\int \limits ^ \infty _o\mu_t (ds) e^{2s} < DT_s CRF, DT_s
    CRF >_H ds \right]^{1/2}. 
\end{align*}

Since
$$
\int \limits^\infty_o t \mu_t (ds) dt = ds \left(\text{ follows from } \int
\limits^\infty_o \int \limits ^ \infty _o te^{- \lambda s}
        {\mu}_t(ds)dt = \frac{1}{\lambda}\right), 
$$
we have
\begin{align*}
  (**)& \leq C_p || \left\{ \int\limits^ \infty _o e^{2s} < DT_s CRF, DT_s
  CRF >_H ds \right\}^{1/2} || _p \\ 
  & \leq C_p || CRF || _p \leq C_p || CF || _p
  \tag{by (\ref{eq1.20}) and since~ $RC=CR$ and $|| R || _p < \infty$.}
\end{align*}

Hence we have obtained
  $$
  || < DF, DF >_H || _p \leq C_p || CF ||_p ~\text{if}~ (J_o + J_1)F = 0.
  $$ 

For $F \epsilon  C_o \oplus C_1$, it is easy to verify directly that
$$
|| < DF, DF >_H ^{1/2} || _p \leq C_p || CF || _p.
$$

Hence we have proved
\begin{equation*}
|| < DF, DF >_H ^{1/2} || _p \leq C_p || CF ||_p, ~\forall~ F
\epsilon  \mathcal{P}. \tag{1.21}\label{eq1.21} 
\end{equation*}


The converse inequality of (\ref{eq1.21}) can be proved by the following
duality arguments: we have for $F$, $G \epsilon \mathcal{P}$, 
\begin{align*}
  |\int \limits _W CF. Gd\mu| &= |\int CF (I-J_o) Gd\mu| \left(\because \int
  \limits_W CF d\mu = 0\right)\\ 
  &=|\int\limits _W CF. C \tilde{G}d\mu |~ \left[\tilde{G} =
    C^{-1}(I-J_o)G\right]\\ 
  &=|\int C^2F. \tilde{G} d \mu | = |\int <LF, \tilde{G}> d \mu |\\
  &=|\int <DF, \tilde{G} >_H d \mu| ~\left[ \vphantom{\int \limits_W}
    \because < DF,  \tilde{G}>_H\right.\\ 
  &= \left.\frac{1}{2} \left\{L (F \tilde{G}) - LF. \tilde{G}- F.L
    \tilde{G} \right\} 
  \text{ and } \int \limits_W LF =0 ~\forall~ F \epsilon  \mathcal{P}
  \right]\\ 
  & \leq \int |DF |_H | D \tilde{G}|_H d\mu \\
  & \leq || ~| DF |_H ||_p ||~ | D \tilde{G} |_H ||_q \left(\frac {1}{p} +
  \frac {1}{q} = 1\right)\\ 
  & \leq C_q || ~| DF |_H ||_p || ~|C \tilde{G}||_q   ~\text{ by
    (\ref{eq1.21}) }\\
  &= C_q || ~|DF|_H ||_p || (I-J_o)G || _q \\
  &\leq a_q || ~| DF|_H ||_p || G ||_q.
\end{align*}\pageoriginale

Hence taking the supremum w.r.t. $|| G || _q \le 1$, we have $|| CF ||
_p \leq a_p || ~| DF |_H ||_p$. The proof of (\ref{eq1.18}) is complete.  

Now we shall prove that
\begin{gather*}
  || \,| D^k F |_{HS} ||_p \leq C_p || C^k F ||_p ~\forall~ F
  \epsilon  \mathcal{P} \tag{1.22}\label{eq1.22} \\
  || \,|D^kF|_{HS} ||_p \leq C'_p || C^kF ||_p ~\forall~ F \epsilon  \mathcal{P}
  ~\text{ if }~ (J_o +J_1 + \cdots J_{k-1})F = 0\tag{1.23}\label{eq1.23} 
\end{gather*}

Then, since
$$
\displaylines{\hfill
  C_p || (I - C)^s F || _p \leq C'_p || (I - L)^{s/2} F || _p \leq C''_p
  || (I - C)^s F ||_p \hfill \cr
  \text{and}\hfill
  a_p || C^k F ||_p \leq || (I-C)^kF ||_p + || F ||_p,\hfill}
$$
Theorem \ref{chap1:thm1.8} follows at once.

\medskip
\noindent{\textbf{Proof of (\ref{eq1.22}):}} \pageoriginale (By
induction). Suppose (\ref{eq1.22}) holds for $1, 2, 
\ldots k$. Let $\{ \gamma_m (w) \}_{m \epsilon  N^k}$ be coin
tossing sequence indexed by $m=(i_1, i_2, \ldots,i_k)\epsilon 
\mathbb{N}^k$ on some probability space $(\Omega,F,P)$. Let $D_m =
D_{i_1}D_{i_2} \cdots D_{i_k}$. Then  
$$
|D^k F|^2_{HS}= \sum \limits_{m \epsilon  \mathbb{N}^k} \left\{ D_mF
\right\}^2. 
$$

Set
$$
X(\omega)=\sum \limits_{m \epsilon  \mathbb{N}^k} \gamma_m (\omega)D_m F.
$$

Then
$$
\displaylines{\hfill
  D_i \chi (\omega) = \sum \limits_{m \epsilon  \mathbb{N}^k} \gamma_m
  (\omega)D_i D_m F \hfill\cr
  \text{and}\hfill
  C\chi (\omega ) = \sum \limits_{m \epsilon  \mathbb{N}^k} \gamma_m
  (\omega)CD_m F.\hfill }
$$
we know that, by (i),
$$
|| \surd \left(\sum^\infty_{i = 1} | D_i X(\omega)|^2\right) ||_p \leq C_p || CX
(\omega) ||_p ~\forall~ \omega. 
$$

Therefore
\begin{equation*}
  E \left\{ || \surd \left(\sum^\infty _{i=1} | D_i X(\omega)
  |^2\right) ||_p^p \right\} \leq C_p E || CX(\omega ) ||
  ^p_p. \tag{1.24}\label{eq1.24}   
\end{equation*}

Therefore, by step \ref{chap1:step3},
\begin{align*}
  E \left\{ || \sum _i (D_i X(\omega ))^2 ||^p_p\right\} & \geq a_p || \surd
  \left(\sum_{i, m} (D_i D_m F)^2\right) ||^p_p \tag{1.25}\label{eq1.25}\\ 
  &= a_p || ~|D^{k+1} F |_{HS} || _p^p.
\end{align*}

On the other hand, by step \ref{chap1:step3},
\begin{align*}
  E || CX (\omega )|| ^p_p &= E || \sum_{m \epsilon  \mathbb{N}^k}
  \gamma_m(\omega ) (CD_m F) || ^p_p\\ 
  & \leq C_p || \left\{ \sum_{m \epsilon  \mathbb{N}^k} (CD_m F)^2
  \right\} ^{1/2} ||^p_p \\ 
  &= C_p || \left\{ \sum_{m \epsilon  \mathbb{N}^k}(D_m CR_k F)^2
  \right\} ^{1/2} ||^p_p\\ 
  & \hspace{2cm}~\text{( by step \ref{chap1:step4}, where)} R_k F =
  \sum ^\infty _{n = 
    k} \surd (1- \frac{k}{n})J_n F \\ 
  & = C_p || ~| D^k CR_k F |_{HS} ||_p^p\\
  & \leq A_p || C^{k+1}R_k F ||^p_p ~\text{ (by induction hypothesis)}\\
& \leq A'_p || C^{k+1}F ||^p_p ~(\therefore || R_k ||_p \leq a_p
  \text{ by step \ref{chap1:step2}}). 
\end{align*}\pageoriginale

This together with (\ref{eq1.24}) and (\ref{eq1.25}) proves that 
$$
|| ~| D^{k+1}F | _{HS} ||_p \leq C_p || C^{k+1}F ||_p
$$
i.e.,  (\ref{eq1.22}) holds for $k+1$ and the proof of (\ref{eq1.22})
is complete. (\ref{eq1.23}) can be proved in a similar manner.


\setcounter{corotothm}{7}
\begin{corotothm}%%% 1.8
  Let $F \epsilon  D_{p, k}$, $1 < p < \infty$, $k \epsilon  \mathbb{Z}^+$; then
  $D^\ell F \epsilon  L_2 (W \to H^{\otimes \ell})$ are defined for
  $\ell = 0, 1, \ldots k$, where 
$$
H^{\otimes \ell}= \underbrace{H
    \otimes \cdots \otimes H}_{\ell-\text{times}}
$$ 
is the Hilbert space of
  all continuous $\ell$-multilinear forms on $\underbrace{H \otimes
    \cdots \otimes H}_{\ell-times}$ with Hilbert-Schmidt norm. Note
  that $H^{\otimes o} = \mathbb{R}$ and $H^{\otimes 1} = H$. 
\end{corotothm}

\begin{proof}
  For\pageoriginale $F \epsilon  D_{p, k}, \exists~ F_n \epsilon
  \mathcal{P} \ni || F_n-F 
  ||_{p, k}\to 0$ which implies $\{ F_n \}$ is Cauchy in
  $\mathbb{D}_{p, k}$. Hence using Meyer's theorem, we get 
  $$
  || ~| D^\ell F_n-D^\ell F_m |_{HS} || \leq C || F_n-F_m || _{p, k} \to 0
  $$
  which gives the result.
\end{proof}

Recall that if $F \epsilon  \mathcal{P}_{\overset{\ast}{W}}$ then

$F(w) = \sum\limits_{i=1}^n F_i (w) \ell_i$ for some $n,\ell_i
\epsilon  \overset{\ast}{W} $ and $F_i \epsilon  \mathcal{P}$. 

For
$$
\displaylines{\hfill
  F(W)\sum_{i=1}^n F_i (w) \ell_i \epsilon
  \mathcal{P}_{\overset{\ast}{w}},\hfill\cr 
  \text{define}\hfill
  LF (w)=\sum_{i=1}^n LF_i (w) \ell_i\hfill \cr
  \text{and}\hfill 
  (1 - L)^{s/2} F (w) = \sum^{n}_{i=1}(1-L)^{s/2}F_i(w) \ell_i.\hfill}
$$

For $1< p < \infty$ and $- \infty$, $s < \infty$, define the norms $||
. ||^H_{p, s}$ on $\mathcal{P}_{\overset{\ast}{W}}$ by 
$$
|| F ||^H_{p, s}=|| \,|(I-L)^{s/2}F_i(w) |_H ||_p.
$$

Let $\mathbb{D}^H{p, s}$ denote completion of
$\mathcal{P}_{\overset{\ast}{W}} w. r. t$. the 
norm $|| . || ^H_{p, s}$. It is clear that $\mathbb{D}^H_{p,s} \subset L_p
(W \to H)$ for $s \geq 0$ and in fact $\mathbb{D}^H_{p, o}= L_p (W \to
H)$.
 
\setcounter{proposition}{8}
\begin{proposition}%proposition 1.9
  The operator $D: \mathcal{P} \to \mathcal{P}_{\overset{\ast}{W}}$
  can be extended as a continuous 
  operator from $\mathbb{D}_{p, s+1}$ to $\mathbb{D}^H_{p, s}$ for every
  $1< p < \infty,- \infty < s < \infty$. 
\end{proposition}

\begin{proof}
Let\pageoriginale $\{ \ell _i \} \subset \overset{\ast}{W}$ be a $ONB$
in $H$ and $F 
\epsilon  \mathcal{P}$. Now 
\begin{equation*}
  |(I-L)^{s/2} DF|_H = \left(\sum^\infty _{i=1} \left[(I - L)^{s/2}D_i
    F\right]^2\right)^{1/2}. \tag*{$\Box$} 
\end{equation*}

Using step \ref{chap1:step4} above, we get
{\fontsize{10pt}{12pt}\selectfont
\begin{align*}
  |(I-L)^{s/2} DF|_H&= \left(\sum^\infty _{i=1}\left\{ D_iR (I -
  L)^{s/2} F \right\}^2 \right)^{1/2} \text{ where } R=\sum  ^\infty _{i=1}
  \left(\frac{n}{n+1}\right)^{s/2}J_n\\ 
  &=|DR(I-L)^{s/2}F|_H.
\end{align*}}

Therefore
\begin{align*}
  || \,|(I-L)^{s/2} DF |_H ||_p &= || \,| DRI-L)^{s/2}F|_H ||_p \\
  & \leq C_p || R(I-L)^{(s + 1)/2}F || _p ~\text{(by Meyer's theorem)} \\
  &\leq C'_p || (I - L)^{(s+1)/2}F || _p ~\text{ (by $L_p$ multiplier
    theorem) }\\ 
  &=C'_p || F || _{p, s + 1}.
\end{align*}
$$
\displaylines{
\text{i.e.,}\hfill 
  || DF || _{p, s}^H \leq C'_p || F || _{p, s + 1}\hfill }
$$
from which the result follows by a limiting argument.
\end{proof}

From the above proposition, it follows that we can define the dual map
$D^*$ of $D$, as a continuous operator  
$$
\displaylines{\hfill 
  D^*: (\mathbb{D}^H_{p, s})' \to (\mathbb{D}_{p, s +1})'\hfill\cr
  \text{i.e.,}\hfill 
  D^*: \mathbb{D}_{p,  s + 1}^H \to \mathbb{D}_{p, s}, 1< p < \infty, -
  \infty < s < \infty.\hfill }
$$ 

And\pageoriginale we know that for $F \epsilon  \mathcal{P}$, $D^*F =
- \delta F$. Hence we have the following corollary. 

\begin{coro*}
  $\delta: \mathcal{P}_{\overset{\ast}{W}}\to \mathcal{P}$ can be
  extended as a continuous operator from 
  $\mathbb{D}^H_{P, s + 1}\to \mathbb{D}_{P, s}$ for every $1<p<
  \infty$, $-\infty < s < \infty$. 
\end{coro*}

\begin{proposition}%propo 1.10
  Let $F \epsilon  \mathbb{D}_{P, k}, G \epsilon  \mathbb{D}_{q,
    k}(\mathbb{D}^H_{q, k})$ for $k \epsilon  Z^+, 1 < p, q <
  \infty$ and let $ 1 < r < \infty$, such that
  $\dfrac{1}{p}+\dfrac{1}{q}=\dfrac{1}{r}$. Then $FG \epsilon 
  \mathbb{D}_{r, k}$ (resp. $\mathbb{D}^H_{r, k}$) and $\exists~ C_{p,
    q, k} > 0$ such that  
  \begin{gather*}
    || FG ||_{r, k} \leq C_{p, q, k} || F ||_{p, k} || G ||_{q, k}\\
    (\text{ resp. } || FG ||^H_{r, k} \leq C_{p, q, k} || F ||_{p, k}
    || G||^H_{q, k}). 
  \end{gather*}
\end{proposition}

\begin{proof}
Let $F, G \epsilon  \mathcal{P}$; then we have
$$
D(FG) = F. DG +G. DF
$$
Therefore
$$
|D[FG]|_H \leq |F||DG|_H + | G | | DF |_{H}.
$$
Similarly
$$
\displaylines{\hfill
  D^2 FG = FD^2G + 2DF \otimes DG + G. D^2F\hfill\cr
  \text{and}\hfill  
  | D^2 FG |_{HS} \leq | F | | D^2 G |_{HS} + 2 | DF |_H | DG |_H + | G
  | | D^2F |_{HS}.\hfill \Box} 
$$

In this way, we obtain for every $k = 1, 2, \ldots$,
$$
\sum^{k}_{\ell = o}|D^l(FG)|_{HS} \leq C_k\left(\sum^{k}_{\ell=o}|D^l F
|_{HS}\right) \left(\sum^{k}_{\ell=o}|D^\ell G |_{HS}\right). 
$$
\end{proof}

Applying H\"older's inequality, we get 
$$
|| \sum^{k}_{\ell=o}| D^\ell (FG) |_{HS} ||_r \leq C_k  ||
\sum^{k}_{\ell = o}| D^\ell F |_{HS}||_p. || \sum^{k}_{\ell=o}|D^\ell
G |_{HS}||_q. 
$$
Then\pageoriginale the result follows by using  Meyer's theorem. And
the case $G \epsilon  \mathbb{D}^H_{q, k}$ follows by similar
arguments.  

\begin{coro*}
  \begin{enumerate}[\rm (i)]
  \item $\mathbb{D}_\infty$ is an algebra and the map 
    $$
    \mathbb{D}_\infty \times \mathbb{D}_\infty \exists~ (F, G) \to FG
    \epsilon  \mathbb{D}_\infty 
    $$
    is continuous.

  \item If $F \epsilon  \mathbb{D}_\infty, G \epsilon 
    \mathbb{D}^H_\infty = \bigcap \limits_{p, s}\mathbb{D}^H_{p, s}$,
    then $ FG \epsilon  \mathbb{D}^H_\infty$ and the map $(F, G)\to
    FG$ is continuous. 
  \end{enumerate}
\end{coro*}

Hence we see that $\mathbb{D}_\infty$ is a nice space in the sense that 
\begin{align*}
  L &: \mathbb{D}_\infty \to \mathbb{D}_\infty \text{ is continuous }\\
  D &: \mathbb{D}_\infty \to \mathbb{D}_\infty^H \text{ is continuous }\\
  \delta &: \mathbb{D}^H_\infty \to \mathbb{D}_\infty \text{ is continuous }.
\end{align*}

\begin{proposition}\label{chap1:prop1.11}%propo 1.11
  \begin{enumerate}[\rm (i)]
  \item Suppose $f \epsilon  C^\infty (\mathbb{R}^n)$, tempered and
    $F_1, F_2,\break \ldots,F_n$ $\epsilon  \mathbb{D}_\infty$; then $F = f
    (F_1, F_2, \ldots, F_n) \epsilon  \mathbb{D}_\infty$ and  
    \begin{enumerate}[\rm (a)]
    \item $DF = \sum \limits^{n}_{i=1} \partial_i f(F_1, F_2, \ldots,
      F_n). DF_i$

    \item $LF = \sum \limits^{n}_{i, j=1} \partial_i \partial_j f(F_1,
      F_2, \ldots, F_n) < DF_i, DF_j >_H$  

      \hfill $+\sum \limits^{n}_{i=1}$
      $\partial_i f(F_1, F_2, \ldots, F_n). L(F_i)$. 
    \end{enumerate}

  \item For  $F, G \epsilon  \mathbb{D}_\infty$,
    $$
    \displaylines{\hfill
    < DF, DG>_H = \frac{1}{2} \left\{ L(FG) - LF. G-F. LG
    \right\}\hfill \cr
    \text{and hence}\hfill
    < DF, DG>_H \epsilon  \mathbb{D}_{\infty}.\hfill}
    $$

  \item If\pageoriginale $F, G, J \epsilon  \mathbb{D}_\infty$, then 
    \begin{multline*}
      < D < DF, DF >_H, DJ >_H  = < D^2 F, DG \otimes DJ >_{HS}\\
       + < D^2 G, DF \otimes DJ >_{HS}.
    \end{multline*}

  \item If $F \epsilon  \mathbb{D}_\infty, G \epsilon 
    \mathbb{D}^H_\infty$, then 
    $$
    \delta (FG) = < DF, G >_H + F. \delta G.
    $$
  \end{enumerate}
\end{proposition}

In particular, if $F, G \epsilon  \mathbb{D}_\infty$ then
$$
\delta (F. DG) = < DF, DG >_H + F. LG.
$$

These formulas are easily proved first for polynomials and then
generalized as above by standard limiting arguments. 


\section[Composites of Wiener Functionals...]{Composites of Wiener Functionals and\hfill\break Schwartz
  Distributions}%sec 1.4 

For $F=(F^1, F^2, \ldots, F^d): W \to \mathbb{R}^d$, we state two
conditions which we shall refer to frequently. 
\begin{equation*}
  F^i \epsilon  \mathbb{D}_\infty, i = 1, 2, \ldots d \tag{A.1}\label{eqA.1}
\end{equation*}
Setting
\begin{equation*}
  \sigma^{ij}= <DF^i,DF^j >_H \epsilon  \mathbb{D}_\infty, \int (\det
  \sigma )^{-p} (w) d \mu (w) < \infty ~\forall~ 1< p < \infty
  . \tag{A.2}\label{eqA.2}  
\end{equation*}

We note that $((\sigma_{ij}))\geq 0$.

\setcounter{lem}{0}
\begin{lem}\label{chap1:lem1}%lemma 1
  Let $F:W \to \mathbb{R}^d$ satisfy (\ref{eqA.1}) and
  (\ref{eqA.2}). Then $\gamma 
  = \sigma^{-1} \epsilon  \mathbb{D}_\infty$ and 
  $$
  D \gamma^{ij}= - \sum^d_{k, \ell=1}\gamma^{ik}\gamma^{j\ell}D\sigma^{k\ell}.
  $$
\end{lem}

\begin{proof}
Let $\epsilon  > 0$. Let

\qquad \qquad $ \sigma^{ij}_\epsilon  (w) = \sigma^{ij}(w) + \epsilon 
\delta_{ij} > 0$ \qquad  (i.e.,  positive definite). 
\end{proof}

Then\pageoriginale it can be easily seen that if $\gamma_\epsilon  =
\sigma^{-1}_{\epsilon }$, then $\exists~ f \epsilon  C^\infty
(\mathbb{R}^{d^2} ) \ni \gamma^{ij}_\epsilon  (w) =
f(\sigma^{ij}_\epsilon  (w))$. 

Then by proposition (\ref{chap1:prop1.11}), since $\sigma^{ij}_{\epsilon }
\epsilon \mathbb{D}_\infty,\gamma^{ij}_{\epsilon }
\epsilon \mathbb{D}_\infty$. Further, it follows from the dominated
convergence theorem that $\gamma^{ij}_{\epsilon } \to \gamma^{ij}$
in $L_p ~\forall~ 1 < p < \infty$. 

Next we show that $D^k \gamma^{ij}\epsilon  L_p (W \to H^{\otimes
  k}) ~\forall~ 1 < p < \infty$. 
Hence, by Meyer's theorem, $\gamma \epsilon  \mathbb{D}_{p,k}
~\forall~ 1 < p < \infty$ and $\forall~ k \epsilon  Z^+$ implying
$\gamma \epsilon  \mathbb{D}_\infty$. We have 
$$
\sum_j \gamma^{ij}_\epsilon  \sigma^{ik}_\epsilon  = \delta^{ik}.
$$

Therefore
$$
\displaylines{\hfill
  \sum_j \gamma^{ij}_\epsilon  D\sigma^{jk}_\epsilon  +\sum_j
  \sigma^{jk}_\epsilon  D \gamma^{ij}_\epsilon  = 0 \hfill\cr
  \text{implies}\hfill
  D \gamma^{ij}_\epsilon  =-\sum^d_{k,l=1}\gamma^{ik}_\epsilon 
  \gamma^{jl}_\epsilon  D \sigma^{kl}_\epsilon .\hfill} 
$$

Similarly, we get
$$
D^k \gamma^{ij}_\epsilon  = -\sum
\gamma_\epsilon .\gamma_\epsilon  \cdots \gamma_\epsilon 
D^{m_1}\sigma_\epsilon  \otimes \cdots \otimes
D^{m_k}\sigma_\epsilon  
$$
where $m_1 + \cdots +m_k = k$ and we have omitted superscripts in
$\sigma^{ij}_\epsilon , \gamma^{kl}_\epsilon $ etc. for
simplicity. Therefore, since 
$$
\displaylines{\hfill 
  \gamma^{ij}_\epsilon  \to \gamma^{ij} \text{ in }L_p,\hfill \cr
  \hfill D^k \gamma^{ij}_\epsilon  \to \sum \gamma . \gamma \cdots
  \gamma D^{m_1}\sigma \otimes \ldots \otimes D^{m_k}\sigma \hfill\cr
  \text{in}\hfill
  L_p(w \to H^{\otimes k}), ~\forall~ 1 < p < \infty\hfill\cr
  \text{implies}\hfill \cr
  \hfill D^k \gamma^{ij} \sum \gamma. \gamma \cdots \gamma D^{m_1}\sigma
  \otimes \cdots \otimes D^{m_k}\sigma \epsilon  L_p (W \to H^{\otimes
    k}). \forall~1 < p < \infty.\hfill} 
$$

\begin{lem}\label{chap1:lem2}%lemma 2
  Let\pageoriginale $F: W \to \mathbb{R}^d$ satisfy (\ref{eqA.1}) and
  (\ref{eqA.2}). 
  \begin{enumerate}[\rm 1)]
  \item Then, $\forall~ G \epsilon  \mathbb{D}_\infty$ and $\forall~i
    = 1,2,\ldots d. \exists~ l_i(G)\epsilon  \mathbb{D}_\infty$ which
    depends linearly on $G$ and satisfies 
    \begin{equation*}
      \int\limits_W (\partial_i \phi_o F). G \mu (dw)= \int\limits_W
      \phi_o F). l_i(G)d\mu, \tag{1.26}\label{eq1.26} 
    \end{equation*}
    $\forall~ \epsilon  S(\mathbb{R}^d)$. Furthermore, for any $1 \le
    r < q < \infty$, 
    \begin{equation*}
      \sup_{|| G ||_{q,1} \le 1}||l_i(G)||_r < \infty. \tag{1.27}\label{eq1.27}
    \end{equation*}

    Hence (\ref{eq1.26}) and (\ref{eq1.27}) hold for every $G \epsilon
    \mathbb{D}_{q,1}$. 

  \item Similarly, for any $G \epsilon  \mathbb{D}_\infty$, and $1
    \le i_1, i_2,\ldots i_k \le d, k \epsilon  \mathbb{N}, \exists
    ~l_{i_1 \ldots i_k}\break (G) \epsilon  \mathbb{D}_\infty$ which
    depends linearly on $G \ni$ 
    \begin{equation*}
      \int \limits_W (\partial_{i_1}\ldots \partial_{i_1} \phi o F). Gd\mu
      =\int\limits_W \phi oF l_{i_1} \ldots_{i_k}(G)d \mu, ~\forall~ \quad \phi
      \epsilon  S(\mathbb{R}^d) \tag*{$(1.26)'$}\label{eq1.26'} 
    \end{equation*}
    and for $1 \le r < q < \infty$,
    \begin{equation*}
      \sup_{|| G ||_{q,k} \le 1}||l_{i_1\ldots i_k} (G)||_r <
      \infty. \tag*{$(1.27)'$}\label{eq1.27'} 
    \end{equation*}
  \end{enumerate}
Hence again \ref{eq1.26'} and \ref{eq1.27'} hold for every $G \epsilon 
\mathbb{D}_{q,k}$. 
\end{lem}

\begin{proof}
Note that $\phi o F \epsilon  \mathbb{D}_\infty $ and
$$
D(\phi oF)= \sum^{d}_{i=1}\partial_i \phi oF\cdot DF^i.
$$
Therefore 
\begin{align*}
   < D(\phi oF)&,  DF^j >_H=\sum^{d}_{i=1}\partial_i \phi oF \cdot \sigma^{ij}\\
  \text{and}\hspace{2cm}
  \partial_i \phi oF &= \sum^{d}_{j=1}< D \phi oF, DF^j >_H
  \gamma^{ij}.\hspace{1cm}\tag*{$\Box$} 
\end{align*}
\end{proof}

Hence\pageoriginale
\begin{align*}
  \int \limits _W \partial_i \phi o F. G d\mu &= \sum^d_{j=1}\int
  \limits_W < D\phi o F, \gamma^{ij} GDF^j >_H d \mu \\ 
  &= -\sum^d_{j=1}\int\limits_W (\phi o F) \delta(\gamma^{ij} GDF^j) d\mu
\end{align*}
Let
\begin{align*}
  \ell_i (G)  &= -\sum^d_{j=1} \delta (\gamma^{ij}GDF^j) \\
  & = -\sum^d_{j=1}[ < D (\gamma^{ij}G), DF^j >_H+\gamma^{ij}G. LF^j]\\
  & = -\sum^d_{j=1} \left[\left\{ -\sum^d_{k, \ell =1} G \gamma^{ik} \gamma^{j\ell }
    < D \sigma^{k\ell}, DF^j > + \gamma^{ij} < DG, DF^j >_H
    \right\}\right.\\ 
    & \hspace{7cm} \left. \vphantom{\sum^d_{k, \ell =1}}+ \gamma^{ij}
    GLF^j\right].  
\end{align*}

Therefore
\begin{multline*}
| \ell_i (G) | \leq \sum^d_{j=1} \left[ \left\{ \sum^d_{k, \ell=1} | \gamma^{ik}
  \gamma^{j\ell} || D \sigma^{k\ell} |_H. | G | | DF^j |_H \right\}\right.\\ 
  \left. \vphantom{\sum^d_{j=1}}+ |\gamma^{ij} | | DF^j |_H | DG |_H +
  | \gamma^{ij} | | LF^j | . | G |\right]. 
\end{multline*}

Hence if $p$ is such that $\dfrac{1}{r} = \dfrac{1}{p} + \dfrac{1}{q}$, then 
\begin{align*}
|| l_i (G) ||_r &\le \sum^d_{j=1}\left[ \left\{\sum^d_{k, \ell=1}|| \gamma^{ik}
  \gamma^{j\ell} || DF^{j}|_H | D \sigma^{k\ell} |_H ||_p.||G||_q
  \right\}\right. \\ 
  & \left. +|| \,|\gamma^{ij}|| DF^{j}|_H ||_p ||\, |DG|_H ||_q 
  \vphantom{\sum^d_{j=1}} +||\, |\gamma^{ij}|| LF^{j}|\, ||_p .||
  G ||_q \right]. 
\end{align*}

Now taking supremum over $|| G ||_q + ||\, |DG |_H ||_q \le 1$, we get
(\ref{eq1.27}). 

2) The proof is similar to that of (1) and we note that
$$
\ell_{i_1\ldots i_k} (G) = \ell_{i_k}[ \ldots[\ell_{i_2}[\ell_{i_1} (G)]] \ldots ].
$$

Let\pageoriginale $\phi \epsilon  S = S(\mathbb{R}^d), -\infty < k < \infty$,
where $k$ is an integer. Let 
$$
\displaylines{\hfill
|| \phi ||_{T_{2K}}= || (1+ | x |^2 -\Delta)^k \phi ||_\infty\hfill\cr
\text{where}\hfill
|| f ||_\infty = \sup _{x \epsilon  \mathbb{R}^d} |f(x)|.\hfill }
$$

Let 
$$
{}_{\bar{S}}|| . ||_{T_{2k}}= T_{2k}.
$$

\begin{facts}
  \begin{enumerate}[\rm (1)]
  \item $S \subset \ldots \subset T_{2k} \subset \ldots \subset T_2
    \subset T_o = \{f \text{ cont }., f \to 0 \text{ as }\break | x | \to
    \infty \}$ 
    $$
    \subset T_{-2} \subset \ldots T_{-2k}.
    $$

  \item $\bigcap \limits_k T_k =S$

  \item $\bigcup \limits_k T_k =S'$.
  \end{enumerate}
\end{facts}

\setcounter{theorem}{11}
\begin{theorem}\label{chap1:thm1.12}%them 1.12
Let $F: W \to \mathbb{R}^d$ satisfy (\ref{eqA.1}) and (\ref{eqA.2}). Let $\phi
\epsilon  S \Leftrightarrow \phi oF \epsilon 
\mathbb{D}_\infty)$. Then, $\forall~ k \epsilon  \mathbb{N}$ and
$\forall~ 1 < p < \infty, \exists ~C_{k,p} > 0$ such that $|| \phi oF
||_{p,-2k} \le C_{p,k}|| \phi ||_{T_{-2k}} $ for all $\phi \epsilon 
S$. 
\end{theorem}

\begin{proof}
  Let $\psi = (1 + |x|^2 - \Delta)^{-k} \phi \epsilon  S$. Then for
  $G \epsilon  \mathbb{D}_\infty, \exists~ \eta_{2k}(G) \epsilon 
  \mathbb{D}_\infty$ such that 
  \begin{gather*} 
    \int\limits_W \left[(1 + | x |^2 - \Delta )^k \psi oF\right]. G \mu (dw) 
    = \int\limits_W \psi oF \left[\eta{2k}(G)\right] \mu (dw)
  \end{gather*}
  i.e., \qquad \quad  $\int\limits_W \phi oF. Gd\mu = \int \limits_W (1+| x
  |^2 - \Delta)^{-k} \phi oF. \eta_{2k}(G)d \mu$. 
\end{proof}

Therefore\pageoriginale
$$
| \int \limits_W \phi oF.Gd \mu | \le || \phi ||_{T_{-2k}} || \eta_{2k}(G) ||_1.
$$
Let
$$
K = \sup_{|| G ||_{q,2k}\le 1} || \eta_{2k}(G)||_1 < \infty,
$$
which follows easily from Lemma \ref{chap1:lem2}. Note that $\eta_{2
  k}(G)$ has a 
similar expression as $\ell_{i_1\ldots i_k} (G)$ only with some more
polynomials of $F$ multiplied. 

Then taking supremum over $|| G ||_{q, 2k} \le 1$ in the above
inequality, we get  
$$
|| \phi_o F ||_{p, -2k} \le K. || \phi ||_{T_{-2k}}.
$$

Since we can take any $q$ such that $\dfrac{1}{r} = 1 < \dfrac{1}{q} <
\infty$ and $\dfrac{1}{p}+\dfrac{1}{q}= 1,p(1 < p < \infty )$ can also
be chosen arbitrarily. 

\begin{coro*}
  We can uniquely extend $\phi \epsilon  S(\mathbb{R}^d) \to \phi oF
  \epsilon  \mathbb{D}_\infty$ as a continuous linear mapping $T
  \epsilon  T_{-2k} \to T(F) \epsilon  \mathbb{D}_{p,-2k}$ for
  every $k \epsilon  Z^+$ and $1 < p < \infty$. 
\end{coro*}

Indeed, the extension is given as follows:

$T \epsilon  T_{-2k}$ implies $\exists~ \phi_n S(\mathbb{R}^d)$ such
that $ || \phi_n -T ||_{T_{-2k}} \to 0$ which implies $\{\phi_n\}$ is
Cauchy in $T_{-2k}$ and hence, by Theorem \ref{chap1:thm1.12},
$\{\phi_n oF\}$ is 
Cauchy in $\mathbb{D}_{p, -2k}, 1 < p < \infty$ and hence we let
$T(F)= \lim \limits_{n \to \infty} \phi{n}o F$, limit being taken
$w.r.t$. the norm $|| \quad ||_{p, -2k}$. Note that $T(F)$ is uniquely
determined. 

\begin{definition}%defini 1.14
$T(F)$\pageoriginale is called the {\em composite} of $T \epsilon  T_{-2k}$ and
  $F$ satisfying (\ref{eqA.1}) and (\ref{eqA.2}). Note that, since $k$ is
  arbitrary, we have defined the composite $T (F)$ for every $T
  \epsilon  S'(\mathbb{R}^d)$ as an element in $\mathbb{D}_{-
    \infty}$. 
\end{definition}

\setcounter{proposition}{12}
\begin{proposition}%propo 1.13
If $T = f \epsilon  \hat{C} (\mathbb{R}^d)=T_o \subset S'
(\mathbb{R}^d)$, then $f(F)=foF$; the usual composite of $f$ and $F$. 
\end{proposition}

\begin{proof}
$T \epsilon  T_o$ implies there exists $\phi_n \epsilon  S$ such that
$$
|| \phi_n -f ||_{T_o} \to 0.
$$
Obviously, we get $|| \phi_n oF - foF ||_p \to 0$ for $1 < p <
\infty$. Hence the result follows by definition of $f(F)$. 
\end{proof}

\section{The Smoothness of Probability Laws}%sec 1.5

\setcounter{lem}{0}
\begin{lem}\label{chap1:addlem1}%lemma 1
  Let $\delta_y$ be the Dirac $\delta $- function at $y \epsilon 
  \mathbb{R}^d$. 
  \begin{enumerate}[\rm (i)]
  \item $\delta_y \epsilon  T_{-2m}$ if and only if $ m >\dfrac {d}{2}$.
  \item if $m > \dfrac{d}{2}$, then the map $ y \epsilon  \mathbb{R}^d
    \to \delta_y \epsilon  T_{-2m}$ is continuous. 
  \item if $m=\left[\dfrac{d}{2}\right]+1, k \epsilon  Z^+$, then $y
    \epsilon  \mathbb{R}^d \to \delta_y \epsilon  T_{-2m-2k}$ is
    $2k$ times continuously differentiable. 
  \end{enumerate}
\end{lem}

Equivalently,
$$
y \epsilon  \mathbb{R}^d \to D^\alpha \delta_y \epsilon  T_{-2m
  -2k}, \alpha \epsilon \mathbb{N}^d |\alpha | \le 2k 
$$
is continuous.
\begin{proof}
  Omitted.\pageoriginale
\end{proof}

\begin{coro*}
  Let $F$ satisfy (\ref{eqA.1}) and (\ref{eqA.2}) and $m
  =\left[\dfrac{d}{2}\right]+1, k 
  \epsilon  Z^+$; then $y \to \delta_y (F) \epsilon  \mathbb{D}_{p,
    -2m-2k}$ is $2k$ times continuously differentiable for every $1 < p
  < \infty$. In particular, we have the following:  
  
  For every $G \epsilon  \mathbb{D}_{q,2m+2k}$
  $$
  < \delta_y (F),G > \epsilon  C^{2k}(\mathbb{R}^d), ~\text{where}~ <
  \delta_y (F),G >
  $$
  denote the canonical bilinear form which we may write roughly as
  $E^\mu$ $(\delta_y(F).G)$. 
\end{coro*}

\begin{lem}\label{chap1:addlem2}%lemma 2
  Let $m=\left[\dfrac{d}{2}\right]+1$ and $1<q<\infty$. If $f \epsilon 
  C(\mathbb{R}^d)$ with compact support, then 
  $$
  \int \limits_{\mathbb{R}} f(y)< \delta_y F,G > dy = E^\mu (foF.G)
  $$
  for every $G \epsilon  \mathbb{D}_{q,2m}$.
\end{lem}

\begin{proof}
  Let
  $$
  \displaylines{\hfill 
  i=(i_1,i_2, \ldots i_d), \Delta^{(n)}_i = \left[\frac{i_1}{2^n},\frac{i_1
      +1}{2^n}\right] \times \ldots \times \left[\frac{i_d}{2^n},\frac{i_d
        +1}{2^n}\right] \hfill \cr
  \text{and}\hfill
   x^{(n)}_i =\left(\frac{i_1}{2^n},\frac{i_2}{2^n}, \ldots
   \frac{i_d}{2^n}\right) \quad \text{ where }\quad  i_k \epsilon
   Z.\hfill\Box } 
  $$

Note that $ | \Delta^{(n)}_i | = \left(\frac{1}{2^n}\right)^d$, where
$|.|$ denote the Lebesgue measure. For $f \epsilon  C(\mathbb{R}^d)$
with compact support, we have  
$$
\sum_i f \left(x^{(n)}_i\right) | \Delta^{(n)}_i | \delta_{x_i (n)} \to \int
\limits_{\mathbb{R}^d} f(x) \delta_x dx = f. 
$$

Note\pageoriginale that the above integral is $T_{-2m}$-valued and the
integration 
is in the sense of Bochner and hence the convergence is in
$T_{-2m}$. Therefore, we have  
$$
\sum_i f\left(x^{(n)}_i\right) | \Delta^{(n)}_i | \delta_{x_i}(F) \to
foF \text{ in } \mathbb{D}_{p,2m} 
$$
for $1< p < \infty$. In particular,
$$
< \sum_i f\left(x^{(n)}_i\right) | \Delta^{(n)}_i |\delta_{x_i}(F),G > \to
E(foF.G)~ \text{ for every } ~G \epsilon  \mathbb{D}_{q,-2m}. 
$$

But 
$$
< \sum_i f\left(x^{(n)}_i\right) | \Delta^{(n)}_i |\delta_{x_i}(F),G >
\to \int \limits_{\mathbb{R}^d}f(x)< \delta_x F,G > dx;. 
$$
hence the result.
\end{proof}

\setcounter{theorem}{13}
\begin{theorem}%them 1.14
  Let $F =(F^1,F^2, \ldots,F^d)$ satisfy the conditions (\ref{eqA.1}) and
  (\ref{eqA.2}). Let $m=\left[\frac{d}{2}\right]+ 1, k \epsilon  Z^+$
  and $1 < q < 
  \infty$. Set, for every $G \epsilon  \mathbb{D}_{q,2m+2k}$ 
  $$
  \mu^F_G(dx)=E^\mu (G(w): F(w) \epsilon  dx).
  $$
  Then $\mu^F_G(x)$ has a density $P^F_G(x) \epsilon 
  C^{2k}(\mathbb{R}^d)$ and $P^F_G(x)= < \delta_x(F), G>$. 
\end{theorem}

\begin{proof}
  Easily follows from Lemma \ref{chap1:addlem1} and Lemma \ref{chap1:addlem2}.
\end{proof}

\begin{remark*}
  By the above theorem, we see that if $G$
  $$
  G \epsilon  \mathbb{D}_{q,\infty}= \bigcap^\infty_{k=o}
  \mathbb{D}_{q,k} 1 < q < \infty, 
  $$
  then $\mu^F_G(dx)$ has a $C^\infty$- density. Further, if $G \equiv 1
  \epsilon  \mathbb{D}_\infty$, then the probability law of $F$: 
  $$
  \mu^F_1(dx) = \mu \{w: F(x) \epsilon  dx \}
  $$\pageoriginale
  has a $C^\infty$-density. But we have
  $$
  \mu^F_G(dx) = E^\mu(G|F =x) \mu^G_1(dx).
  $$
  
  Hence
  $$
  p^F_G(x) = E^\mu(G | F = x) p^F_1(x).
  $$
\end{remark*}
