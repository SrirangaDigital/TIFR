\thispagestyle{plain}
  
\chapter*{Introduction}


\addcontentsline{toc}{chapter}{Introduction}

\markboth{Introduction}{}
 
 Let\pageoriginale  $W^r_o$ be the space of all continuous functions
 $w=(w^k(t))^r_{k=1}$ from $[o, T]$ to $\mathbb{R}^r$, which vanish at
 zero. Under the supremum norm, $W^r_o$ is a Banach space. Let $P$ be
 the $r$-dimensional Wiener measure on $W^r_o$. The pair $(W^r_o, P)$
 is usually called ($r$-dimensional) Wiener space. 
  
 Let $A$ be a second order differential operator on $\mathbb{R}^d$ of
 the following form: 
\begin{equation*}
  A=\frac{1}{2} \sum_{i, j=1}^{d} a^{ij}(x) \frac{\partial^2}{\partial
  x^i \partial x^j}+ \sum_{i=1}^{d} b^i(x) \frac{\partial}{\partial
  x^i}+ c(x). \tag{0.1} 
\end{equation*} 
 where $a^{ij}(x)) \ge 0$, i.e.,  non-negative definite and symmetric.
 
 Now, let 
 $$
 a^{ij}(x) = \sum_{k=1}^{r} \sigma^i_k(x) \sigma^j_k(x)
 $$
 and consider the stochastic differential equation
 \begin{align*}
   d\chi^i(t)  & = \sum_{k=1}^r \sigma^i_k(X(t))dW^k(t) +b^i (X(t) dt,
   i=1,2, \ldots, d, \tag{0.2}\label{eq0.2} \\ 
   X(o)& =x, x \epsilon  \mathbb{R}^d. 
 \end{align*} 
 We know if the coefficients are sufficiently smooth, a unique
 solution exists for the above $SDE$ and a global solution exists if
 the coefficients have bounded derivative. 
 
 Let $X(t,x,w)$ be the solution of (\ref{eq0.2}). Then $t \to X(t, x, w)$ is
 a sample path of $A_o$-diffusion process, where $A_o =A-c(x)$. The
 map $x \to X(t,x,w)$, for fixed $t$ and $w$ from $\mathbb{R}^d$ to
 $\mathbb{R}^d$ is a diffeomorphism (stochastic flow\pageoriginale of
 diffeomorphisms), if the coefficient are sufficiently smooth. And the
 map $w \to X(t,x,w)$, for fixed $t$ and $x$, is a Wiener functional,
 i.e.,  a measurable function from $W^r_o$ to $\mathbb{R}^d$. 
 
 Consider the following integral on the Wiener space:
 \begin{equation*}
   u(t,x) =E\left[\exp \left\{\int_o^T c(X(s,x,w)) ds\right\}.f
     (X(t,x,w))\right] \tag{0.3}\label{eq0.3}
  \end{equation*}  
  where both $f$ and $c$ are smooth functions on $\mathbb{R}^d$ with
  polynomial growth order and $c(x) \le M< \infty$. Then $u(t,x)$
  satisfies 
  \begin{gather*}
    \frac{\partial u}{\partial t}=Au \tag{0.4}\label{eq0.4}\\
    u|_{t=0}=f
  \end{gather*}
and any solution of this initial value problem (\ref{eq0.4}) with polynomial
growth order coincides with $u(t, x)$ given by (\ref{eq0.3}). 
  
Suppose we take formally $f(x) = \delta_y(x)$, the Dirac
$\delta$-function at $y \epsilon  \mathbb{R}^d$ and set  
  \begin{equation*}
    p(t,x,y)= E \left[\exp \left\{ \int\limits_o^t c
      (X(s,x,w))ds\right\} \delta_y (X(t,x,w))\right];
    \tag{0.5}\label{eq0.5}  
  \end{equation*}  
  then we would have
  $$
  u (t,x)= \int\limits_{\mathbb{R}^d} p(t,x,y)f(y) dy
  $$
and $p(t,x,y)$ would be the fundamental solution of
(\ref{eq0.4}). (\ref{eq0.5}) 
is thus a formal expression for the fundamental solution  of
(\ref{eq0.4}), often used intuitively, but $\delta_y(X(t,x,w))$ has no
meaning as a Wiener functional. The purpose of these lectures is to
give a correct mathematical meaning to the formal expression
$\delta_y(t,x,w))$\pageoriginale by 
using concepts like `integration by parts on Wiener space', so that
the existence and smoothness of the fundamental solution, or the
transition probability density for (\ref{eq0.3}), can be assured through
(\ref{eq0.5}). This is a way of presenting \textit{Malliavin's calculus}, an
infinite dimensional differential calculus, introduced by Malliavin
with the purpose of applications to problems of partial differential
equations like (\ref{eq0.4}). 

