\chapter{Lecture 13}\label{chap13}%cha 13

\setcounter{section}{25}
\section{The Components of a Repartition}\label{chap13:sec26}%\sec 26

Our\pageoriginale next aim is to introduce the L-functions modulo an integral
divisor of an algebraic function field over a finite constant filed
and to obtain their functional equation. We shall develop the
necessary results for this in this and the next two lectures.  

Let $K$ be an algebraic function field over an arbitrary (not
necessarily finite) constant field $k$. For a repartition
$\mathscr{C} \in \mathscr{X}$ and a prime divisor
$\mathscr{Y}$, we define the component $\mathscr{C}^\mathscr{Z}$ of
$\mathscr{C}$ at $\mathscr{Z}$  to be the repartition defined by  
\begin{equation*}
  \mathscr{C}^\mathscr{Z} (\mathscr{U})=
  \begin{cases}
    \mathscr{C}_\mathscr{Z}& \text{ if }~\mathscr{U} ~\mathscr{Y}\\
    0 & \text{ if }~ \mathscr{Y} \text{is a prime divisor other than}~
    \mathscr{Y}. 
  \end{cases} 
\end{equation*}

The mapping is clearly a $k$-linear mapping of $\mathscr{X}$ into
itself. This induces a linear mapping of the dual of $\mathscr{X}$ into
itself, by which a differential $\omega$ is taken  to a liner map
$\omega \mathscr{Z} : \mathscr{X} \to \mathscr{X}$ given by  
$$
\omega^\mathscr{Y} (\mathscr{C}) = \omega (\mathscr{C}^\mathscr{Z})
$$
$\omega^\mathscr{Z}$ is called the \text{component} of the
differential $\omega$ at $\mathscr{Y}. \omega^\mathscr{Z}$ is not, in
general, a differential. For $X \in K$, we have 
$$
(X \omega)^\mathscr{Y} (\mathscr{C}) = (X \omega)
(\mathscr{C}^\mathscr{Z}) = \omega (X \mathscr{C}^\mathscr{Z}) =
\omega ((X \mathscr{C})^\mathscr{Y}) = \omega^\mathscr{Z} (X
\mathscr{C}) 
$$

We shall now prove a lemma which expresses a differential in terms of
its components. 

\begin{lemma*}
  If\pageoriginale $ \omega $ is a differential and $ \mathscr{C}$ a repartition, $
  \omega^\mathscr{Y} ( \mathscr{C} ) = 0 $ for all but  a finite
  number of prime divisors $ \mathscr{Y} $, and we have  
  $$
  \omega ( \mathscr{C} ) = \sum_{\mathscr{Y}} \omega^\mathscr{Y} ( \mathscr{C} ).
  $$
\end{lemma*}


\begin{proof}
  Let $ \mathscr{U} $ be a divisor which divides the differential $
  \omega $. Let\break $ \mathscr{Y}_1 , \ldots, \mathscr{Y}_n $ be the finite
  number of prime divisors for which either\break  $ v_{\mathscr{Y}_{i}}
  (\mathscr{U} ) \neq 0 $ or $ v_{\mathscr{Y}_{i}} (\mathscr{C})  < 0
  $. For $ \mathscr{U} \neq $ any of the $ \mathscr{Y}_i $, and any
  prime divisor $ \mathscr{Y} $, we have  
  \begin{multline*}
  v_{\mathscr{Y}(\mathscr{C}^{\mathscr{U}})}=v_\mathscr{Y} (
  \mathscr{C}^{\mathscr{U}} (\mathscr{Y} ))\\
  = \left. 
       \begin{cases}
          v_\mathscr{C} (\mathscr{Y}_{\mathscr{U}}) & \text{ if }
          \mathscr{U}= \mathscr{Y}\\
            v_\mathscr{Y} (0) & \text{ if } \neq
              \mathscr{Y}\mathscr{U}  
       \end{cases}\right\}
       = \left.
       \begin{cases}
        v_\mathscr{U}(\mathscr{C}) & \text{ if } \mathscr{Y} =
          \mathscr{U}\\ 
        \infty&  \text{ if } \mathscr{Y} \neq \mathscr{U}
       \end{cases}\right\}
       \ge v_\mathscr{Y} (\mathscr{U}), 
  \end{multline*}
  and therefore $ \mathscr{C}^\mathscr{U} \in  \Lambda (\mathscr{U})
  $, $ \omega^\mathscr{U} (\mathscr{C}) = \omega
  (\mathscr{C}^\mathscr{U}) = 0 $. 
\end{proof}

Also, if we put $ \mathscr{Y} = \mathscr{C} - \sum \limits^{n}_{i=1}
\mathscr{C}^{\mathscr{Y}_i} $, we have for any $\mathscr{Y} $, 
$$
v_\mathscr{Y} (\mathscr{Y}) = 
\left. \begin{cases}
   v_\mathscr{Y} (\mathscr{C}_\mathscr{Y}) & \text{ if } \mathscr{Y} \text{ is
     not any  of the}\mathscr{Y}_1\\
   v_\mathscr{Y} (0) = \infty & \text{ if } \mathscr{Y} \text{ is a
    certain } \mathscr{Y}_i
\end{cases}\right\}
 \ge v_\mathscr{Y} (\mathscr{U}),  
$$
and therefore $ \mathscr{Y} \in \Lambda  (\mathscr{U}) $. Hence,
\begin{gather*}
  \omega (\mathscr{C}) = \omega \left( \mathscr{C} - \sum^{n}_{i=1}
  \mathscr{C}^{\mathscr{Y}_i}\right) + \omega \left( \sum^{n}_{i=1}
  \mathscr{C}^{\mathscr{Y}_i}\right) = \omega \left( \sum^{n}_{i=1}
  \mathscr{C}^{\mathscr{Y}_i } \right) \\ 
  = \sum^{n}_{i=1} \omega \left( \mathscr{C}^{\mathscr{Y}_i} \right) =
  \sum^{n}_{i=1} \omega^{\mathscr{Y}_i} = \sum_{\mathscr{Y}} \omega
  \mathscr{Y} (\mathscr{C} ). \\ 
\end{gather*}

Our lemma is proved.

We shall now prove another useful

\begin{lemma*}
  Let\pageoriginale $\omega $ be a differential and $ \mathscr{C} $ a prime
  divisor. Then $ v_\mathscr{Y} (( \omega )) $ is the largest integer
  $m$ such that whenever $ X \in K $ and $ v_\mathscr{Y} (X) \ge -m $,
  we have $ \omega^\mathscr{Y} (X) = 0 $. 
\end{lemma*}

\begin{proof}
  Suppose first that $ X\in K $ with $ v_\mathscr{Y} (X) \ge
  -v_\mathscr{Y} (( \omega )) $. Then clearly the repartitions
  $x^\mathscr{Y} $ is in $ \Lambda (( \omega )^{-1} ) $ and therefore
  $ \omega^\mathscr{Y} (X) = \omega (X^\mathscr{Y} ) = 0 $. 
\end{proof}

Now, by the definition of $ ( \omega ) $, $ \omega $ does not vanish on
the space\break $ \Lambda (( \omega )^{-1}\mathscr{Y}^{-1})$. Hence there  exists a
repartition $ \mathscr{C} \in \Lambda (( \omega )^{-1}
\mathscr{Y}^{-1} ) $ such that $ \omega ( \mathscr{C} ) \neq 0 $. It
is evident that for $ \mathscr{U} \neq \mathscr{Y} $, $
\mathscr{C}^\mathscr{U} \in \Lambda (( \omega )^{-1})$ and therefore
$ \omega ( \mathscr{C} ) \neq 0 $. Hence,  
$$
0 \neq \omega (\mathscr{C}) = \sum_{\mathscr{U}} \omega (
\mathscr{C}^\mathscr{U}) = \omega (\mathscr{C}^\mathscr{Y})  
$$
Put $ X = \mathscr{C}^\mathscr{Y} $. Then,
$$
\displaylines{\hfill
  \omega^\mathscr{Y} (X) = \omega (X^\mathscr{Y} ) = \omega (
  \mathscr{C}^\mathscr{Y} ) \neq 0  \hfill \cr
  \text{and}\hfill  
  v_\mathscr{Y} (X) = v_\mathscr{Y} ( \mathscr{C} ) \ge  -v_\mathscr{Y}
  ((\omega))-1. \hfill }
$$

Thus, $ v_\mathscr{Y} (( \omega )) $ is the largest $m$ for which $
v_\mathscr{Y} (X) \ge -m $ implies that $ \omega^\mathscr{Y} (X)= 0
$.
