\chapter{Lecture 5}\label{chap5}

\setcounter{section}{8}
\section{Divisors}\label{chap5:sec9}%sce 9

Let\pageoriginale $K$ be an algebraic function field with constant field $k$. We make the
\begin{defi*}
  A {\em divisor} of $K$ is an element of the free abelian group
  generated by the set of places of $K$. The places themselves are
  called {\em prime divisors}. 
\end{defi*}

The group $\vartheta$ of divisors shall be written
multiplicatively. Any element $\mathscr{U}$ of the group $\vartheta$
of divisors can be written in the form 
$$
\mathscr{U} = \prod_{\mathscr{Y}} \mathscr{Y} v_{\mathscr{Y}(\mathscr{U})}
$$ 
where the product is taken over all prime divisors $\mathscr{Y}$ of
$K$, and the $v_\mathscr{Y}(\mathscr{U})$ are integers, all except a
finite number of which are zero. The divisor is denoted by $n$. 

We say that a divisor $\mathscr{U}$ is \textit{integral} if
$v_\mathscr{Y}(\mathscr{U}) \geq 0$, for every $\mathscr{Y}$, and that
$\underline{\mathscr{U}}$ \textit{divides} $\underline{\delta}$ if
$\delta \mathscr{U}^{-1}$ is integral. Thus, $\mathscr{U}$ divides
$\delta$ if and only if $v_\mathscr{Y} (\delta) \geq v_{\mathscr{Y}} (\mathscr{U})$
for every $\mathscr{Y}$. 

Two divisors $\mathscr{U}$ and $\delta$ are said to be
\textit{coprime} if $v_\mathscr{Y}(\mathscr{U} \neq 0$ implies that
$v_\mathscr{Y} (\delta ) = 0$. 

The \textit{degree} $d(\mathscr{U})$ of a divisor $\mathscr{U}$ is the
integer 
$$
d(\mathscr{U}) = \sum_\mathscr{Y} f_\mathscr{Y} v_\mathscr{Y}(\mathscr{U}),
$$
where\pageoriginale $f_\mathscr{Y}$ is the degree of the place $\mathscr{Y}$.

The map $\mathscr{U} \to d(\mathscr{U})$ is a homomorphism of the
group of divisors $\vartheta$ into the additive group of integers. The
kernel $\vartheta_\circ$ of this homomorphism is the subgroup of
divisors of degree zero. 

An element $X$ of $K$ is said to be \textit{divisible} by the divisor
$\mathscr{U}$ if $v_\mathscr{Y}(X) \geq v_\mathscr{Y}(\mathscr{U})$
for every $\mathscr{Y}$. Two elements $X$, $Y \in K$ are said to be
\textit{congruent modulo a divisor} $\mathscr{U}$ (written $X \equiv
(\mod \mathscr{U})$) if $X-Y$ is divisible by $\mathscr{U}$. 

Let $S$ be a set prime divisors of $K$. Then we shall denote by
$\Gamma (\mathscr{U}/S)$ the set of elements $X \in K$ such that
$v_\mathscr{Y} (X) \geq v_\mathscr{Y}(\mathscr{U})$ for every
$\mathscr{Y}$ in $S$. It is clear that $\Gamma (\mathscr{U}/S)$ is a
vector space over the constant field $k$, and also that if
$\mathscr{U}$ divides $\delta, \Gamma (\delta /S) \subset
\Gamma(\mathscr{U}/S)$. Also, if $S$ and $S^1$ are two sets of prime
divisors such that $S \subset S^1$, then $\Gamma (\mathscr{U}/S^1)
\subset \Gamma (\mathscr{U}/S))$. Finally, $\Gamma (\mathscr{U}/S) =
\Gamma(\delta /S)$ if $\mathscr{U}\delta^{-1}$ contains no
$\mathscr{Y}$ belonging to $S$ with a non-zero exponent. 

If $S$ is a set of prime divisor which is fixed in a discussion and
$\mathscr{U}$ a divisor, we shall denote by $\mathscr{U}_\circ$ the
new divisor got from $\mathscr{U}$ by omitting all $\mathscr{Y}$ which
do not occur in $S$: $\mathscr{U}_\circ = \prod \limits_{\mathscr{Y}
  \in S} \mathscr{Y}^{v_\mathscr{Y}(\mathscr{U})}$ 

\begin{theorem*}
  Let $S$ be a finite set of prime divisors and  $\mathscr{U}, \delta$
  two divisors such that $\mathscr{U}$ divides $\delta$. Then, 
  $$
  \dim_k \frac{\Gamma (\mathscr{U}/S)}{\Gamma(\delta/S)}=d(\delta_\circ)
  - d(\mathscr{U}_\circ) = d(\delta_\circ \mathscr{U}_\circ ^{-1}). 
  $$
\end{theorem*}

\begin{proof}
  By\pageoriginale our remark above, we may assume that $\mathscr{U}=
  \mathscr{U}_\circ$ and $\delta = \delta_\circ$. 

  Moreover, it is clearly sufficient to prove the theorem when $\delta =
  \mathscr{U} \mathscr{Y}$, where $\mathscr{Y}$ is a prime divisor
  belonging to $S$. For, if $\delta = \mathscr{U} \mathscr{Y}_1
  ..\mathscr{Y}_n$ we have 
  \begin{multline*}
    \dim_k \frac{\Gamma (\mathscr{U}/S)}{ \Gamma(\delta/S)} = \dim _k
    \frac{\Gamma (\mathscr{U}/S)} {\Gamma(\mathscr{U} \mathscr{Y}_1/S)}+\\
    \dim_k \frac{\Gamma (\mathscr{U}
      \mathscr{Y}_1/S)}{\Gamma(\mathscr{U}\mathscr{Y}_1 \mathscr{Y}_2 /S)}
    + \cdots + \dim_k \frac{\Gamma (\mathscr{U}_{\mathscr{Y}_1\cdots
        \mathscr{Y}_{n-1}}/S}{ \Gamma(\delta/S)}   
  \end{multline*}
  and $d(\delta) - d(\mathscr{U}) = d(\mathscr{Y}_1) + d(\mathscr{Y}_2)
  + \cdots + d(\mathscr{Y}_n)$. 
\end{proof}

  Hence, we have to prove that if $\mathscr{U}$ is a divisor such that
all the prime divisors occurring in it with non-zero exponents are in
$S$, and $\mathscr{Y}$ any prime divisor in $S$, we have 
$$
\dim_k \frac{\Gamma (\mathscr{U}/S)}{\Gamma
  (\mathscr{U}\mathscr{Y}/S)}= f_\mathscr{Y} = f. 
$$

By the theorem on independence of valuations, we may choose an element
$u \in K$ such\pageoriginale that 
$$
v_\mathscr{U}(u) = v_\mathscr{U}(\mathscr{U}) \text{ for all
}\mathscr{U} \text{ in } S. 
$$

If $X_1, X_2, \ldots X_{f+1}$ are any $f+1$ elements of $\Gamma
(\mathscr{U} /S)$, the elements $X_1 u^{-1}, \ldots X_{f+1}u^{-1}$ are
all in $\mathscr{U}_\mathscr{Y}$. But since the degree of
$k_\mathscr{U} = \mathscr{O}_{\mathscr{Y}/\mathscr{Y}}$ over $k$ is
$f$, we have 
$$
\sum^{f+1}_{i=1} a_i X_i u^{-1} \in \mathscr{Y}, a_i \in k, \text{ not
  all } a_i \text{ being zero }. 
$$

Hence, $\sum \limits^{f+1}_{i=1} a_i X_i \in \Gamma (\mathscr{U}
\mathscr{Y}/S), a_i \in k$, not all $a_i$ being zero, thus proving
that the dimension over $k$ of the quotient $\dfrac{\Gamma
  (\mathscr{U}/S)}{\Gamma (\mathscr{U}\mathscr{Y}/S)}$ is $\leq f$. 

Now, suppose $Y_1, \ldots Y_f$ are $f$ elements of
$\mathscr{O}_{\mathscr{Y}}$such that they are linearly independent
over $k$ modulo $\mathscr{Y}$. Choose$Y^1_i \in K$ such that  
$$
v_{\mathscr{Y}}(Y^1_i -Y_i) > 0, v_{\mathscr{U}}(Y^1_i) \geq 0 ~\text{
  for }~ \mathscr{U} \neq \mathscr{Y}, \mathscr{U} \in S. (i =1, f). 
$$

By the first condition, $Y^1_i \equiv Y_i (\mod \mathscr{Y})$, and
hence $Y_i$ and $Y^1_i$ determine the same element in
$k_\mathscr{Y}$. But by the second condition, the elements $u Y^1_i$
belong to $\Gamma(\mathscr{U}/S)$, and since $Y_1, \ldots Y_f$ are
linearly independent $\mod \mathscr{Y}$, no linear combination of $u
Y^1_1, \ldots uY^1_f$ with coefficients in $k$- at least one of which
is non-zero-can lie in $\Gamma(\mathscr{U} \mathscr{Y}/S)$. Thus,
$\dim_k \dfrac{\Gamma
  (\mathscr{U}/S)}{\Gamma(\mathscr{U}\mathscr{Y}/S)} \geq f$. 

\medskip
\noindent
This proves our theorem.
