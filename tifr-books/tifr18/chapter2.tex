\chapter{Lecture 2}\label{chap2}

\setcounter{section}{2}\label{chap2:sec3}
\section{(Contd.)}

In\pageoriginale this lecture, we shall establish the equivalence of the concepts of
valuation, place and valuation ring. 

Two places $\varphi_1 : K \rightarrow \sum_1 \cup (\infty)$ and
$\varphi_2 : K \rightarrow \sum_2 \cup (\infty)$ are said to be
\textit{equivalent} if there exists an isomorphism $\lambda$ of $\sum_1$,
onto $\sum_2$ such that $\varphi_2 (a) = \lambda \circ \varphi_1 (a)$ for
every $a$, with the understanding that $\lambda(\infty) = \infty$. 

This is clearly an equivalence relation, and thus, we can put the set
of places of $K$ into equivalence classes. Moreover, equivalent places
$\varphi_1$ and $\varphi_2$ obviously define the same valuation rings
$\mathscr{O}_{\varphi_1}$ and $\mathscr{O}_{\varphi_2}$. Thus, to
every equivalence class of places is associated a unique valuation
ring. 

Conversely, let $\mathscr{O}$ be any valuation ring and $\mathscr{Y}$
its maximal ideal. Let $\sum$ be the quotient field $\mathscr{O}/
\mathscr{Y}$ and $\eta$ the natural homomorphism of $\mathscr{O}$ onto
$\sum$. It is an easy matter to verify that the map $\varphi : K
\rightarrow \sum U \{\infty \}$ defined by 
\begin{equation*}
  \varphi(a) =
  \begin{cases}
    \eta (a) & \text{if}~ a  \in \mathscr{O} \\ 
    \infty & \text{if}~ a \notin \mathscr{O}
  \end{cases}
\end{equation*}
is a place, whose equivalence class corresponds to the given valuation
ring $\mathscr{O}$. 

Let $v_1$ and $v_2$ be two valuations of a field $K$ in the ordered
group $W_1$ and $W_2$. We shall denote the unit elements of both the
groups\pageoriginale by $1$, since it is not likely to cause confusion. We shall say
that $v_1$ and $v_2$ are \textit{equivalent} if $v_1 (a) > 1$ if and
only if $v_2(a) > 1$. 

Let $v_1$ and $v_2$ be two equivalent valuations. From the definition,
it follows, by taking reciprocals, that $v_1(a) < 1$ if and only if
$v_2(a) < 1$, and hence (the only case left) $v_1(a) = 1$ if and only
if $v_2(a) = 1$. Let $\alpha$ be any element of $W_1$. Choose $a \in
K$ such that $v_1(a) = \alpha$ (this is possible since $v_1$ is onto
$W_1$). Define $\sigma (\alpha) = v_2(a)$. The definition is independent
of the choice of $a$ since if $b$ were another element with $v_1(b) =
\alpha$, then $v_1(ab^{-1}) = 1$ so that $v_2(ab^{-1}) = 1$,
i.e. $v_2(a) = v_2(b) $. Thus, $\sigma$ is a  mapping from $W_1$ onto
$W_2$ (since $v_2$ is onto $W_2$). It is easy to see that $\sigma$ is
an order preserving isomorphism of $W_1$ onto $W_2$ and we have
$v_2(a) = (\sigma . v_1)(a)$ for every $a \in K^*$. Thus, we see that
the definition of equivalence of valuations can also be cast into a
form similar to that for places. 

Again, equivalence of valuations is an equivalence relation,  and we
shall that equivalence classes of valuations of a field $K$ correspond
cano\-nically and biunivocally to valuation rings of the field $K$. 

Let $v$ be a valuation and $\mathscr{O}$ be the set of elements $a$ in
$K$ such that $v(a) \le 1$. It is an immediate consequence of the
definition that $\mathscr{O}$ is a ring. Also, if $a \in K, v(a) > 1$,
then $v\left(\dfrac{1}{a}\right) < 1$ and hence $\dfrac{1}{a} \in
\mathscr{O}$. Thus, $\mathscr{O}$ is a valuation ring. Also,\pageoriginale if $v_1$
and $v_2$ are equivalent, the corresponding rings are the same. 

Suppose conversely that $\mathscr{O}$ is a valuation ring in $K$ and
$\mathscr{Y}$ its maximal ideal. The set difference $\mathscr{O}-
\mathscr{Y}$ is the set of units of $\mathscr{O}$ and hence a subgroup
of the multiplicative group $K^*$. Let $\eta : K^* \rightarrow K^* /
\mathscr{O}-\mathscr{Y}$ be the natural group homomorphism. Then $\eta
(\mathscr{O}^*)$ is obviously a semi-group and the decomposition $K^* /
\mathscr{O}-\mathscr{Y}= \eta (\mathscr{O}^*) \cup \{ 1\} U\cup\eta
(\mathscr{O}^*)^{-1}$ is disjoint, Hence, we can introduce an order in
the group $K^* /\mathscr{O}-\mathscr{Y}$ and it is easy to verify that
$\eta$ is a valuation on $K$ whose valuation ring is precisely
$\mathscr{O}$. 

Summarising, we have

\begin{theorem*}
  The valuations and places of a field $K$ are, upto equivalence, in
  canonical correspondence with the valuation rings of the field. 
\end{theorem*}
