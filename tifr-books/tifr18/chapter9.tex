\chapter{Lecture 9}\label{chap9}%chap 9

\setcounter{section}{15}
\section{Rational Function Fields}\label{chap9:sec16} %sec 16

In\pageoriginale this lecture, we shall consider some particular function fields and
find their canonical class, genus, etc. as illustrations of the general
theory. Let us first consider the rational function fields. 

Let $K = k (X)$ be a rational function field in one variable over
$k$. We shall first show that $k$ is precisely the field of constants
of  $K$. For later use, we formulate this in a more general form. 

\begin{lemma*}
  Let $K$ be a purely transcendental extension of a field $k$. Then
  $k$ is algebraically closed in $K$. 
\end{lemma*}

\begin{proof}
  Let $(x_i)_{i \in I}$ be any transcendence basis of $K$ over $k$ such
  that $K = k (x_i)$. Since any element of $K$ is a rational
  combination of a finite number of $x_i$, we may assume that $I$ is a
  finite set of integers $(1, \ldots n)$. 
\end{proof}

We proceed by induction. Assume first that $n = 1$. Let $\alpha$ be
any element of $k(x_1)$ algebraic over $k$. $\alpha$ may be written in
the form $\dfrac{f (x_1)}{g(x_1)}$, where $f$ and $g$ are polynomials
over $k$ prime to each other. 

We then have
$$
f(x_1) - \alpha g (x_1) = 0
$$

This proves that if $\alpha$ were not in $k$, $x_1$ is algebraic over
$k(\alpha)$, (since\pageoriginale the above polynomial for $x_1$ over $K(\alpha)$
cannot vanish identically ) and hence over $k$ which is a
contradiction. Hence $\alpha$ is in $k$. 

Suppose now that the lemma holds for $n - 1$ instead on $n$. If
$\alpha$ were an element of $k(x_1 , \ldots x_n)$ algebraic over $k$,
it is algebraic over $k(x_1, \ldots x_{n-1})$. By the first part, it
should be the $k(x_1, \ldots x_{n - 1})$, and hence by induction
hypothesis in $k$. Our lemma is proved. 

Let us return to the rational function field $K$. We have already seen
that the prime divisors are $(i)N_{\mathcal{X}}$, the prime divisor
corresponding to $\dfrac{1}{X}$, and $(ii) \mathscr{Y}_{p(X)}$, the
prime divisors corresponding to irreducible polynomials $p(X)$ in
$k[X]$. If $f(X)$ is a rational function of $X$ over $k$ having the
unique decomposition $p_1^{e_1} (X) \cdots p_r^{e_n}(X)$, the
principal divisor $(f(X))$ is clearly given by 
$$
(f(X)) = \prod^{r}_{\nu  = 1} \mathscr{Y}_{p_\gamma}^{e_\gamma} (X)
N_{\mathcal{X}}^{- \deg f} 
$$

It follows that the space $L(N^{-t}_X)$ for $t \ge 0$
consists precisely of all polynomials of degree $\le t$, and since
there are $t + 1$ such polynomials independent over $k$, and
generating all polynomials of degree $\le t, (1, X, \ldots X^t$ for
example), we deduce that 
$$
N(N^t_X E) = t + 1 .
$$

But by the corollary to the Riemann - Roch theorem, if
$d(N_X^t E) = td (N_{X}) = t > 2g - 2$, we should
have 
$$
N(N^t_{X} E) = d(N_{X}^t E) + 1 - g = t + 1 - g,
$$
and\pageoriginale hence, $g = 0$. Thus, there are no non-zero differentials of the
first kind. 

Since $d (N_{X}^{-2} E) = - 2 = 2g - 2 < 0$, it follows that
$N(N^{-2}_X E) = 0 = g$. and hence $W = n^{-2}_\mathcal{X} E$. 


\section[Function Fields of Degree Two...]{Function Fields of Degree Two Over a Rational Function
  Field}\label{chap9:sec17} %sec 17 

\markright{\it 17. Function Fields of Degree \ldots}

We start with a lemma which will be useful for later calculations.

\begin{lemma*}
  Let $K/_k$ be any algebraic function field and $X \in K$ any
  transcendental element. If $R(X) = \dfrac{f_1(X)}{f_2(X)}$ is any
  rational function of $X$, then $f_1$ and $f_2$ being prime to each
  other $(R(X)) = \dfrac{\mathfrak{z}f_1}{\mathfrak{z}f_2}
  N_X^{- \deg R(X)}$ and $\mathfrak{z}_{f_1}$ and
  $\mathfrak{z}_{f_2}$ are prime to each other and both are prime to
  $N_{X}$. Moreover, $[k(X) : k(R(X))] = \max (\deg f_1, \deg
  f_2)$. 
\end{lemma*}

\begin{proof}
  Let $f(X)$ be any polynomial over $k$ of the form
  $$
  f(X) = a_o + a_1 X + \cdots + a_t X^t.
  $$

  If $\mathscr{Y}$ is a prime divisor not dividing $N_{X}$,
  i.e., if $v_{\mathscr{Y}} (X) \ge 0$, we have $v_{\mathscr{Y}} (f(X))
  \ge \min\limits_{\nu = o} (\gamma v_{\mathscr{Y}}(X)) \ge o)$. If on the
  other hand, $\mathscr{Y}$ does occur in $N_{X}$, we have
  $v_{\mathscr{Y}} (X) < o$, and hence 
  $$
  v_{\mathscr{Y}} (f(X)) = \min\limits_{\nu = o -t} (\nu
  v_{\mathscr{Y}}(X)) = t v_{\mathscr{Y}} (X) 
  $$
  
  Thus, we see that $\mathfrak{z}_f$ is prime to $N_{X}$ and
  $N_{f (x)} = N^t_{X}$. 
\end{proof}

Hence, if $R(X) = \dfrac{f_1 (X)}{f_2 (X)}$, where $(f_1, f_2)  =1$, we have 
$$
(R(X)) = \frac{\mathfrak{z}f_1}{\mathfrak{z}f_2} N_{X}^{-\deg R(X)}
$$

We\pageoriginale assert that $\mathfrak{z}_{f_1}$ and $\mathfrak{z}_{f_2}$ are prime
to each other. For if not, let $\mathfrak{z}_{f_1}$ and
$\mathfrak{z}_{f_2}$ have a prime divisor $\mathscr{Y}$ in
common. Then $v_{\mathscr{Y}} (X) \ge o$. Find polynomials $g_1$ and
$g_2$ such that 
$$
f_1 g_1 + f_2 g_2 = 1.
$$

$\begin{aligned}
\text{Then}, \quad 
  0 &= v_{\mathscr{Y}}(1) = v_{\mathscr{Y}} (f_1 g_1 + f_2 g_2)\\
& \ge \min (v_{\mathscr{Y}} (f_1) + v_{\mathscr{Y}}(g_1),
v_{\mathscr{Y}} (f_2) + v_{\mathscr{Y}} (g_2)) > 0,
\end{aligned}$

a contradiction. Thus, $\mathfrak{z}_{f_1}$ and $\mathfrak{z}_{f_2}$
are prime to each other. 

To prove the last part of the lemma, we may assume without loss of
generality that $\deg f_1 \ge \deg f_2$ or that $\deg R(X) \ge 0$
(otherwise consider $\dfrac{1}{R(X)}$). It then follows that 
\begin{align*}
  \mathfrak{z}_{R(X)} & = \mathfrak{z}_{f_1 (X)}, [k(X) : k(R(X))] = [K :
    k(R(X))] / [K : k(X)]\\ 
  & = \frac{d(\mathfrak{z}_{R(X)})}{d (N_{X})} = \frac{d
  (\mathfrak{z}_{f_1 (X)})} {d(N_{X})} = \frac{d
  (\mathfrak{z}_{f_1 (X)})} {d(N_{X})} = \deg f_1. 
\end{align*}
Our lemma is proved.

It follows in particular that $k(x) = k(R(X))$ if and only if $\max
(\deg f_1$, $\deg f_2) = 1$, or $R(X) = \dfrac{\alpha X + \beta} {\gamma
  X + \delta }, \alpha, \beta, \gamma, \delta \in k$ and $\alpha
\delta - \beta \gamma \neq 0$. 

Now, let $k$ be a field of characteristic different from $2$, $X$ a
transcendental element over $k$ and $K$ a field of degree two over
$k(X)$ which is not got from an algebraic extension of $k$ (viz., $K$
should not be got by the adjunction to $ k(X)$ of elements algebraic
over $k$). Then $k$ is the constant field of $K$, for if it were not,
there exists an element $\alpha $ of $K$ which is algebraic over $k$
but not in\pageoriginale $k$. $\alpha$ cannot lie in $k(X)$, since $k$ is
algebraically closed in $k(X)$. Hence, $k(X, \alpha)$ should be an
extension of degree at least two over $k(X)$, and should therefore
coincide with $K$. But this contradicts our assumption regarding $K$. 

Now, $K$ can be get the adjunctions to $k(X)$ of an element $Y$ which
satisfies a quadratic equation 
$$
Y^2 + bY + c = 0, b, c \in k(X)
$$

Completing the square ( note that characteristic $k \neq 2$), we get 
$$
\left(Y + \frac{b}{2}\right)^2 + \left(c - \frac{b}{4}^2\right) = 0,
$$
and hence $k(X, Y) = k (X, Y^1)$ where $Y^1 = Y + \dfrac{b}{2}$
satisfies equation of the form $\gamma^{1^2} = R(X), R(X)$ being a rational
function of $X$. Let $R(X) = \prod\limits^{r}_{\nu - 1} p_\gamma ^{e
  \gamma}(X)$, where $p_\nu (X)$ are irreducible polynomials in $k[X]$
and $e_\nu$ are integers. Putting $e_\nu = 2g_\nu + \epsilon_\gamma$, where
$g_\nu$ are integers and $\epsilon_\nu = 0$ or $1$, and $Y'' =
\dfrac{Y^1}{\prod\limits_\nu p^\gamma g_gamma (X)}$, we see that $k (X, Y) = k (X,
Y^{''})$, and $Y^{''}$ satisfies an equation of the form 
$$
Y''^{2} = \prod^{r}_{1} p_{\gamma}^{\epsilon \nu}(X) = D(X),
$$
where $D(X)$ is a polynomial which is a product of different
irreducible polynomials. 

We shall therefore assume without loss in generality that $K = k (X,
Y)$ with $Y^2 = D(X)$ of the above form. Let us assume that $m$ is the
degree of $D$. 

Now,\pageoriginale let $\sigma$ be the automorphism of $K$ over $k(X)$ which is not
the identity. If $Z = R_1 (X) + Y R_2(X)$ is any element of $K,
Z^\sigma = R_1 (X) - YR_2 (X)$. To every prime divisor $\mathscr{Y}$
of $K$, let us associate a prime divisor $\mathscr{Y}^{\sigma}$ by the
definition 
$$
v_{\mathscr{Y}^\sigma}(Z) = v_{\mathscr{Y}}(\sigma^{-1} Z)
$$

This can be extended to an automorphism of the group $\vartheta$ of
divisors (see Lecture \ref{chap19}). We shall denote this automorphism again by
$\sigma$, and the image of a divisor $\mathscr{U}$ by
$\mathscr{U}^\sigma$. Since $X^\sigma = X, N^\sigma_{X} =
N_{X}$. 

Suppose now that $Z = R_1 (X) + YR_2 (X)$ is any element of
$L(N^{-t}_{X})$ (t any integer). Applying $\sigma$, we deduce
that $R_1 (X) - YR_2 (X)$ should also be an element of
$L(N_{X}^{-t})$. Adding, $2R_1 (X) \in L (N^{-t}_{X}),
R_1 (X) \in L (N^{-t}_{X})$. If $R_1 (X) = \dfrac{f_1 (X)}{g_1
  (X)}$, where $f_1$ and $g_1$ are coprime polynomials, we have
$(R_1(X)) = \dfrac{\mathfrak{z}_{f_1}}{\mathfrak{z}_{g_1}} N_{X}^
{-\deg R_1}$ divisible by $N^{-t}_{X}$, and hence we deduce
that $\mathfrak{z}_{g_1} = n$, and $g_1 (X)$ is a constant. Hence
$R_1(X)$ is a polynomial of degree $\le t$ (if $t < o, R_1 (X) = 0)$. 

Also, since both $Z$ and $Z^\sigma$ are in
$L(N^{-t}_{X})$. This implies as before that $R^2_1 - DR^2_2$
is a polynomial of degree $\le 2t$. Hence $DR^2_2$ is a polynomial of
degree $\le 2t$, and since $D$ is square free, $R_2$ is a polynomial
of degree $\le t - \dfrac{m}{2}$.

Conversely, by working back, we see that if $R_1$ is a polynomial of
degree $\leq t$ and $R_2$ a polynomial of degree $\leq t=
\frac{m}{2}$, $Z = R_1 + Y R_2 \in L (N^{-t}_{X})$ . Hence, we obtain 
$$
N(E N^t_{X}) = l(N^{-t}_{X}) = 
\begin{cases}
  0 &\text{ if }~ t < 0\\
  t + 1 & \text { if }~ 0 \le t \le \frac{m}{2} - 1 \text{ and } m \text{
    even } \\ 
  t + 1 & \text{ if }~ m ~\text{ and }~ 0 \le t \leq \frac{m - 1}{2}\\
  2t + 2 - \frac{m}{2} & \text{ if }~ m ~\text { even and }~ t \ge
  \frac{m}{2} \\ 
  2t + 2 - \frac{m + 1}{2}&  \text{ if }~ m ~\text{ odd and }~ t \ge
  \frac{m + 1 }{2}. 
\end{cases}
$$\pageoriginale\

Since $d(N_{X}) = [K : k(X)] = 2$, for $t > g - 1$, we have
$d(E N^t_{X}) = 2t > 2g - 2$ and hence 
$$
N (E N^t_{X}) = d (N^t_{X}) - g + 1  = 2t - g + 1
$$

Comparing with the above equations, we deduce that $g$ is
$\dfrac{m}{2} - 1$ if $m$ is even and $\dfrac{m - 1}{2}$ if $m$ is
odd. 

Thus, we obtain examples of fields of arbitrary genus over any
constant field. 

The canonical class of $K$ is $E N^{g -1}_{X}$. For,
$$
\displaylines{\hfill 
  d (N^{g - 1}_{X} E) = 2g - 2,\hfill \cr
  \text{and}\hfill  N(N^{g - 1}_{X} E) = 
  \begin{cases}
    g - 1 + 1 = g & \text{ if }~ g > 0\\
    0 = g & \text{ if }~ g = 0.
  \end{cases}\hfill }
$$

If $gg > o$, there exists a differential $\omega$ of the first kind
with $(\omega) = N^{g -1}_{X}$. Then clearly the differentials
$\omega, X \omega, \ldots, X^{g - 1}\omega$ are all of the first kind
and are linearly independent over $k$, and as they are $g$ in number,
they form a base over $k$ for all differentials of the first kind.  

\section{Fields of Genus Zero}\label{chap9:sec18}%sec 18

We\pageoriginale shall find all fields of genus zero over a constant field $k$.

First, notice that any divisor of degree zero of a field of genus zero
is a principal divisor. For let $C$ be a class of degree $0$. Then
since $d(C) > - 2 = 2g - 2, N(C) = d(C) - g + 1 =1$ and therefore $C =
E$.  

Now,
$$
d(W^{-1} ) = 2 > 2g - 2, N(W^{-1}) = 2 - g + 1 = 3,
$$
and therefore there exists three linearly independent integral
divisors $\mathscr{U}_1, \mathscr{U}_2, \mathscr{U}_3$ in the class
$W^{-1}$ (incidentally, this proves there exists integral divisors,
and consequently prime divisors of degree at most two). Let
$\dfrac{\mathscr{U}_1}{\mathscr{U}_2} = (X)$. Then clearly
$\mathscr{N}_X$ divides $\mathscr{U}_2$, and hence we obtain 
$$
[K : k(X)] = d(N_{X}) \le d(\mathscr{U}_2) = 2.
$$

Thus, any field of genus zero should be either a rational function
field or a quadratic extension of a rational function field. We have
the following 
\begin{theorem*}%theo 0
  The necessary and sufficient condition for a field of genus zero to
  be a rational function field is that it possess a prime divisor of
  degree $1$. 
\end{theorem*}

\begin{proof}
  If $K = k(X)$, the prime divisor $n_\mathcal{X}$ satisfies the
  requisite condition. 

  Conversely, let $\mathscr{Y}$ be a prime divisor of degree $1$ of
  $K$. Then, $N(\mathscr{Y} E) = d(\mathscr{Y}) + 1 = 2$, and
  therefore there are elements $X_1, X_2$\pageoriginale in $K$ linearly independent
  over $k$ such that $X_1 \mathscr{Y} = \mathscr{U}_1$ and $X_2
  \mathscr{Y} = \mathscr{U}_2 $ are integral divisors. Thus if $X =
  \dfrac{X_1}{X_2}, (X) = (\dfrac{X_1}{X_2}) =
  \dfrac{\mathscr{U}_1}{\mathscr{U}_2}$, and $d (N_{X}) \le d
  (\mathscr{U}_2) = d (\mathscr{Y}) = 1$. But $X$ is not in $k$, and
  hence $d(N_{X}) = 1 = [K : k (X)]$, from which it follows
  that $K = k(X)$. The theorem is proved. 
\end{proof}

\section{Fields of Genus One}\label{chap9:sec19}%sec 19

Let $K/k$ be an algebraic function field of genus $1$. Then, since
$N(W) = g = 1$ and $d(W) = 2g-2=0$, the canonical class $W$ coincides
with the principal class $E$. 

A function field of genus one which contains at least one prime
divisor of degree one is called an \textit{ elliptic function field.}
(The genus being one does not imply that there exists a prime divisor
of degree one. In fact, it can be proved easily that the field $R(X,
Y)$, where $R$ is the field of real numbers and $X, Y$ transcendental
over $R$ and connected by the relation $Y^2 + X^4 + 1 = 0$ has every
prime divisor of degree two). Let us investigate the structure of
elliptic function fields. 

Let $\mathscr{Y}$ be a prime divisor of degree one. Then, since $d
(\mathscr{Y}^2) = 2 > 2g - 2 = 0$, we have $l (\mathscr{Y}^{-2}) =
2$. Let $1, X$ be a basis of $L(\mathscr{Y}^{-2})$ over $k$. Since $X
\mathscr{Y}^2$ is integral, $N_{X}$ divides $\mathscr{Y}^2
. N_{X}$ cannot be $\mathscr{Y}$, since if it were, we obtain
$[K : k(X)] = d(N_{X}) = d(\mathscr{Y}) = 1, K = k(X)$ and
hence $g = 0$. Thus, $N_{X}$ should be equal to
$\mathscr{Y}^2$. 

Again, since $l(\mathscr{Y}^{-3}) = 3$, we may complete $(l, X)$ to a
basis $(1, X, Y)$ of $L(\mathscr{Y}^{-3})$ over $k$. $N_\gamma$ should
divide $\mathscr{Y}^3$, and since $Y$\pageoriginale is not an element of $L
(\mathscr{Y}^{-2}), N_\gamma$ does not divide $\mathscr{Y}^2$. Thus, $N_\gamma =
\mathscr{Y}^3$. 

The denominators of $1, X, Y, X^2, XY, X^3$ and $Y^2$ are respectively
$N$, $\mathscr{Y}^2$, $\mathscr{Y}^3$, $\mathscr{Y}^4, \mathscr{Y}^5,
\mathscr{Y}^6$ and $\mathscr{Y}^6$. Since the first six elements have
different powers of $\mathscr{Y}$ in the denominator, they are
linearly independent elements of $L(\mathscr{Y}^{-6})$. But
$l(\mathscr{Y}^{-6}) = 6$, and the seventh element, being in
$L(\mathscr{Y}^{-6})$ should therefore be a linear combination of the
first six. We thus obtain 
$$
Y^2 + \gamma X Y + \delta Y = \alpha_3 X^3 + \alpha_2 X^2 +
\alpha_2 X^2 + \alpha_1 X + \alpha_o, \gamma, \delta, \alpha_i \in k. 
$$

Now, if $Y$ where a rational function of $X$, writing $Y =
\dfrac{f(X)}{g(X)}$, $f(X)$, $g(X) \in k[X]$, $(f(X)$, $g(X)) = 1$, and
substituting in the above equation, we easily deduce that $g(X)$
should be a constant, and that $Y$ should be a polynomial in $X$ of
degree $\le 1$. But this would mean that $Y$ is divisible by
$\mathscr{Y}^2$, which we have already ruled out. Hence, $[k(X, Y) : k
  (X)] = 2 = d (N_{X}) = [K : k(X)]$, and consequently, $K =
k(X, Y)$. 

If the characteristic of $k$ is different from $2$, we may as in \S
\ref{chap9:sec16} find a $Z$ such that $K =k (X, Y)$, with 
$$
Z^2 = f(X)
$$
where $f(X)$ is a cubic polynomial in $X$ with non-repeating
irreducible factors. 

A partial converse of the above result is valid. Suppose that $K = k
(X, Z)$, where $X$ is transcendental over $k$ and $Z$ satisfies an\pageoriginale\
equation of the form $Z^2 = f (X)$, where $f(X)$ is a cubic polynomial
which we assume to be irreducible. Let $ch(k) \neq 2$. Then, the genus
of $K$ is one, by $\S 16$. Also, since $\deg f(X) = 3, Z^2 \in L
(N^{-3}_{X})$, and $N^2_z$ divides $N^3_{X}$. But
since $X$ is of degree $3$ over $k(Z)$, we have $d (N^2_z) = 2d (N_z)
= 2. [K : k (Z)] = 6 = 3. [K : k(X)] = 3d (N_{X})$, and
therefore $N_z= N^3_{X}$. From this, it is clear that there
exists a prime divisor $\mathscr{Y}$ with $d(\mathscr{Y}) = 1$ such
that $N_z = \mathscr{Y}^3, N_{X} = \mathscr{Y}^2$. 

Finally, suppose a field $K$ is of genus greater than $1$. Then, $N(W)
= g > o$ and $d(W) = 2g - 2 > o$, and hence we deduce the existence of
an integral divisor $\neq N$ of degree $2g - 2$. Thus, we have proved
that if $g > 1$, there always exists prime divisors of degree $\le 2g
- 2$. The minimal degree of prime divisors for a field of genus one is
not known. 
