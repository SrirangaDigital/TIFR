\chapter{Lecture 10}\label{chap10}%chap 10

\setcounter{section}{19}
\section{The Greatest Common Divisor of a Class}\label{chap10:sec20}%sec 20

We\pageoriginale wish to find when the greatest common divisor of all integral
divisors of a class is different from the unit divisor $N$. We assert
that this is impossible when $d(C) \ge 2g$. In fact, let $\mathscr{U}$ be an
integral divisor of the class $C$. Then we obtain 
{\fontsize{10pt}{12pt}\selectfont
\begin{gather*}
  d(C) - g + 1 = N(C) = N (C \mathscr{U}^{-1}) = d (C) - d
  (\mathscr{U}) + 1 - g + N(WC^{-1} \mathscr{U}),\\ 
  d(\mathscr{U}) = N(WC^{-1} \mathscr{U}) \le \max (0, d (WC^{-1}
  \mathscr{U}) + 1) 
\end{gather*}}\relax
and since $1 + d (WC^{-1} \mathscr{U}) \le 2g - 2 - 2g + 1 + d
(\mathscr{U}) = d (\mathscr{U}) - 1$, we should have $d(\mathscr{U}) =
0$ and $\mathscr{U} = N$.  

This is in a sense the best possible result. In fact, if there exists
a prime divisor $\mathscr{Y}$ of degree $1$ in the field, we have $d(W
\mathscr{Y}) = 2g - 2$, and hence. 
$$
N(W \mathscr{Y}) = d(W) + d(\mathscr{Y}) - g + 1 = g = N (W),
$$
which proves that $\mathscr{Y}$ divides all integral divisor of the
class $W \mathscr{Y}$. Nevertheless, if $g > 0 $, we can prove that the
greatest common divisor of the canonical class $W$ is $N$. For,
suppose $\mathscr{U} \neq N$ is an integral divisor of $W$. Then, 
\begin{align*}
  \dim L(\mathscr{U}^{-1}) & = N (E \mathscr{U}) = d (\mathscr{U}) + 1 - g +
  N(W \mathscr{U}^{-1})\\ 
  & = d(\mathscr{U}) + 1 - g + N (W) = d (\mathscr{U}) + 1 \ge 2.
\end{align*}

Thus, there exists a transcendental element $X \in L
(\mathscr{U}^{-1})$. The divisor $(X) \mathscr{U}$ is then
integral. Also, since $N(W) = g > o$, we can choose a differential\pageoriginale\
$\omega$ of the first kind. Then, $(X \omega) = ((X)
\mathscr{U}). ((\omega) \mathscr{U}^{-1})$ is also integral and
therefore $X \omega$ is also first kind. By repetition of the
argument, we obtain that $X^n \omega$  is of the first kind for all
positive integers $n$. But since $X$ is transcendental, $
\mathscr{N}_X \neq \mathscr{N}$, and hence $X^n \omega$ cannot be an
integral divisor for large $n$. This is a contradiction. Our assertion
is therefore proved.  

For fields of genus $g > o$, we shall improve the inequality $N(C) \le
\max (o , d (C) + 1)$. 
\begin{lemma*}
  If $g > o$ and $d (C) \ge o, C \neq E$, we have
  $$
  N (C) \le d (C)
  $$
\end{lemma*}

\begin{proof}
  We may obviously assume that $N(C) > o$. Then there exists an
  integral divisor $\mathscr{U}$ in $C$, and since $C \neq E,
  \mathscr{U} \neq \mathscr{N}$, and therefore $d (C) = d
  (\mathscr{U}) >o$. Since $G > o$, $\mathscr{U}$ cannot be a divisor of
  the class $W$,and therefore 
  \begin{align*}
    g & = N (W) > N (W \mathscr{U}^{-1}) = N (WC^{-1})\\
    N (C) & = d(C) - g+1 + N (WC^{-1}) \le  d(C) - g+1 + g-1 =d (C).
  \end{align*}
  
  Our lemma is proved
\end{proof}

\section[The Zeta Function  of Algebraic...]{The Zeta Function  of Algebraic Function Fields Over Finite
  Constant Fields}\label{chap10:sec21}%sec 21 

\markright{\it 21. The Zeta Function  of Algebraic Function \ldots}

In the rest of this lecture and the following two lectures, we shall
always assume that $k$ is a finite field of characteristic $p > o$ and with
$q=p^t$ elements,and that $K$ is an algebraic function field with\pageoriginale\
constant field $k$. 

If $\mathscr{Y}$ is any prime divisor of $K$, we shall call the number
of elements in the class field $k_\mathscr{Y}$ the \textit{norm of }
$\mathscr{Y}$. Since $[k_\mathscr{Y} : k] = d (\mathscr{Y})$, we see
that norm of $\mathscr{Y}$ (which we shall denote by $N_\mathscr{Y}$ )
is given by 
$$
N_\mathscr{Y} = q^{d (\mathscr{Y})}
$$

We may extend this definition to all divisors, by putting
$$
N \mathscr{U} = \prod_{\mathscr{Y}} (N
\mathscr{Y})^{v{_{\mathscr{Y}}}}{^{(\mathscr{U})}} = q^{d
  (\mathscr{U})} 
$$

Clearly we have for two divisors $\mathscr{U}$ and $\delta$
$$
N \mathscr{U} \delta = N \mathscr{U} . N \delta.
$$

Before introduction the zeta function, we shall prove an important
\begin{lemma*}
  For any positive integer $m$, the number of prime divisors of degree
  $\le m$ is finite. The number of classes of degree zero is
  finite. (The latter is called {\em the class number} of $K$ and is
  denoted by $h$).  
\end{lemma*}

\begin{proof}
  To prove the first part of the lemma, choose any transcendental
  element $X$ of $K$. Let $\mathscr{Y}$ be any prime divisor of $K$
  with $d (\mathscr{Y}) \le m$ and which does not divide
  $\mathscr{N}_X$. (We may neglect those $\mathscr{U}$ which divide
  $\mathscr{N}_X$, since they are finite in number.) Let
  $\mathscr{Y}^1$ be the restriction of $\mathscr{Y}$ to $k
  (X)$. $\mathscr{Y}^1$ must be a place on $k (X)$. Since
  $\mathscr{Y}^1 (X) \neq \infty$, there corresponds a unique
  polynomial $(p(X))$ which gives rise to $\mathscr{Y}^1$.  

  Now,\pageoriginale since $k_{\mathscr{Y}'} \subset k_\mathscr{Y}$, we obtain
  $$
  \deg (p (X)) = d (\mathscr{Y'}) \le d (\mathscr{Y}) \le m.
  $$
\end{proof}

Since the number of polynomials of degree $\le m$ over a finite field
is finite, and since there the are only a finite number of prime
divisors $\mathscr{Y}$ of $K$ which divide $\mathfrak{z}_{p
  ({X})}$ for a fixed $p (X)$, the first part of our theorem
is proved.  

To prove the second part, choose and fix an integral divisor
$\mathscr{U}_o$ such that $d (\mathscr{U}_o) \ge g$. If $C$ is any
class of degree zero, we have  
$$
N (C \mathscr{U}_o ) \ge d (C) + d (\mathscr{U}_o) - g+1 \ge 1,
$$
and hence there exists an integral divisor $\mathscr{U}$ in $C
\mathscr{U}_o$ such that $d (\mathscr{U}) = d (C\mathscr{U}_o) = d
(\mathscr{U}_o)$. But since 
$$
d (\mathscr{U}) = \sum d (\mathscr{Y}) v_\mathscr{Y} (\mathscr{U}), d
(\mathscr{Y}) \ge 1, v_\mathscr{Y} (\mathscr{U}) \ge 0,  
$$
and there are only a finite number of $\mathscr{Y}$ with $d
(\mathscr{Y}) \le d (\mathscr{U}_o)$, there are only a finite of
integral $\mathscr{U}$ with $d (\mathscr{U}) = d (\mathscr{U}_o)$ and
consequently only a finite number of $C$ with $d (C) = 0$.  

The lemma is completely proved. 

\begin{remark}\label{chap10:sec21:rem1}%rem 1
  Let $\rho$ denote the least positive integer which is the degree of
  a class. since $C \to d (C)$ is a homomorphism of the group
  $\mathfrak{R}$ of divisor classes into the additive group $Z$ of
  integers, we see that the degree of any class of the from $\nu \rho$
  where $\nu$ is an integer, and that to any $\nu$, there correspond
  precisely $h$ classes $ C_1^{(\gamma \rho)} ,\ldots,C_h ^{(\nu
    \rho)}$\pageoriginale  of degree $\nu \rho$.  
\end{remark}

\begin{remark}\label{chap10:sec21:rem2}%rem2
  The number of integral divisors in any $C$ is precisely \break
  $\dfrac {
    q^{N(C)} -1} {q-1}$. This is clear if $N(C) = 0$. If $N(C) > 0$, let
  $\mathscr{U}$ be any integral divisor of $C$. Then all integral
  divisors of $C$ are of the from $(X) \mathscr{U}$, where $X \in L
  (\mathscr{U}^{-1}), X \neq 0$. Also $(X) \mathscr{U} = (Y)
  \mathscr{U}$ if and only if $\left(\dfrac{X}{Y}\right) =\mathscr{N}$ or $X =
  a\gamma, a \in k^*$. Since the number of non zero elements
  of $L (\mathscr{U}^{-1})$ is $q^{N (C)} -1$ and the number of non
  zero elements of $k$ is $q-1$ our assertion follows.   
\end{remark}

Now, let $s = \sigma + it$ be a complex variable. For $\sigma > 1$, we
define \textit { the zeta function} of the algebraic function field
$K$ by the series  
$$
\zeta (s, K) =\sum_\mathscr{U} \frac{1}{(N \mathscr{U})^s} , s =
\sigma + it, \sigma > 1 
$$ 
the summation being extended over all integral divisors of the field
$K$. Since $\bigg|  \dfrac{1}{ (N \mathscr{U})^s}\bigg| = \dfrac{1}{(N
  \mathscr{U})^\sigma}$, and all the terms of the series are positive
when $s$ is real, the following calculations are valid first for $s >
1$ and the for complex $s$ with $\sigma > 1$. In particular, they
prove the absolute convergence of the series     
$$
\tau (s, K)
$$
$= \sum\limits_C$ (number of integral divisors in $C$). $q^{-sd
  (C)}$, the last summation being over all classes, 
$$
= \frac{1}{q-1} \sum_{C} (q^{N (C)} - 1) q^{-sd (C)};
$$
writing $d (C) = \nu \rho$, and noticing that $q^{N (C)} -1 = 0$ if $d
(C) < 0$, the\pageoriginale above expression becomes 
$$
\frac{1}{q-1} \sum^\infty_{\nu = o} q^{- \nu \rho s} \sum^h_{l=1} q^{N
  (C_l{^{(\nu \rho)}})} - \frac{h}{q-1} \sum^\infty_{ \nu = o} q^{-
  \nu \rho s} 
$$
Let us now put $U = q^{-s}$. Then $|U| = |q^{-s}| = q^{-\sigma} < 1$
since $\sigma > 1$, and we can sum the second geometric
series. Suppose now that $g > o$. We may then split the first sum into
two parts, the ranging over $o \le \nu \le \dfrac{2g-2}{\rho}$ and the
second over $\nu > \dfrac{2g-2}{\rho}$. (Since $d(w) = 2g-2, \rho$)
divides $2g-2$; or $\dfrac{2g-2}{\rho}$ is an integer). In the second
summation since $d (C_l^{(\nu \rho )}) = \nu \rho > 2g-2$, we may
substitute $N (C_l^{(\nu \rho)}) = \nu \rho - g+1$. We obtain the
expression  
\begin{multline*}
 \frac{1}{q-1} \sum^{\frac{2g-2}{\rho}}_{\nu = o} q^{- \nu \rho s}
 \sum^h_{l=1} q^{N (C_l^{(\nu \rho)})} + \frac{h}{q-1} \sum_{ \nu >
   \frac{2g-2}{\rho}} q^{-\nu \rho s} q^{\nu \rho -g+1} -
 \frac{h}{q-1}. \frac{1}{1-q^{s \rho}}\\ 
  =\tau (s,K) = \frac{1}{q-1} \sum^{\frac{2g-2}{\rho}}_{\nu = o}
  U^{\nu \rho} \sum^h_{l=1} q^{N (C_l^{(\nu \rho)})}\\ +
  \frac{hq^{1-g}}{q-1} \frac{(Ug)^{2g-2+\rho}}{ 1-(Uq)^\rho} -
  \frac{h}{q-1} \frac{1}{1-U^\rho} \tag{1}\label{chap10:sec:21:eq1} 
\end{multline*}

If $g = o, N(C) = d (C) - g+1$ for all $C$ with $d (C) > o$, and a
similar computation gives 
\begin{equation*}
  \tau (S, K) = \frac{hq}{q-1} \frac{1}{1-(Uq)^\rho} - \frac{h}{q-1}
  \frac{1}{1-U^\rho} \tag{2}\label{chap10:sec:21:eq2}  
\end{equation*}

(\ref{chap10:sec:21:eq1}) and (\ref{chap10:sec:21:eq2}) may be combined as follows
\begin{equation*}
  (q-1) \tau (s,K) = F (U) + R (U),\tag{3}\label{chap10:sec:21:eq3} 
\end{equation*}
where\pageoriginale $F(U)$ is the polynomial 
\begin{equation*}
  F (U) = \sum_{o \le d (C) \le 2g -2} q^{N (C)} U^{d (C)} \tag{4}\label{chap10:sec:21:eq4} 
\end{equation*}
and $R (U)$ the rational function
\begin{equation*}
  R(U) = hq^{1-g} \frac{(U q)^{\max (o,2g-2+\rho)}}{1-(Uq)^\rho} -
  \frac{h}{1-U^\rho} \tag{5}\label{chap10:sec:21:eq5}  
\end{equation*}

These formulae provide the analytic continuation of $\tau (s, K)$ to
the whole plane. The only possible poles are the values of $s$ for
which $U^\rho = 1$ or $(q^U)^\rho=1$.  

For a rational function field $K = k (X)$, since $g=o, h=1$ and $\rho
= 1$, we obtain  
$$
(q-1) \tau (s, K) = \frac{q}{1-Uq} - \frac{1}{1-U} =
\frac{q-1}{(1-Uq)(1-U)} 
$$

