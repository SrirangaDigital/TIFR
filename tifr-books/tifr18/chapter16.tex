\chapter{ Lecture 16}\label{chap16} %cha 16

\setcounter{section}{28}
\section{Characters Modulo \texorpdfstring{$\mathcal{F}$}{F}}\label{chap16:sec29}%\section 29

We\pageoriginale shall now introduce the characters of an algebraic function field
$K$ modulo an integral divisor $\mathcal{F}$ 
\begin{defi*}
  A chapter $\mathscr{X}$ modulo an integral divisor
    $\mathcal{F}$ is a homomorphism of $\mathfrak{R}_{\mathcal{F}}$ into
  the multiplicative group of complex numbers with absolute value one.   
\end{defi*}

If $\mathscr{U}$ is a divisor prime to $\mathcal{F}$, we put
$\mathcal{X} (\mathscr{U}) = \mathcal{X} (\mathscr{U} E
_{\mathcal{F}})$. If $\mathscr{U} = \delta / \mathscr{L}$, where $\delta$
and $\mathscr{L}$ are mutually coprime and integral and $\mathscr{L}$
coprime to $\mathcal{F}$ whereas $\delta$ is not, we define
$\mathcal{X} (\mathscr{U}) = 0$. $\mathcal{X}$ is this defined as a
complex valued function on a subset of $\mathscr{V}$, and is clearly
multiplicative (i.e., if $\mathcal{X} (\mathscr{U})$) and $\mathcal{X}
(\delta)$ are defined, $\mathcal{X} (\mathscr{U} \delta) = \mathcal{X}
(\mathscr{U} \mathcal{X} (\delta))$. Note that $\mathcal{X}$ is
defined in particular on all integral divisors of $K$. 

Let $\mathcal{X} $ be a character $\mod \mathcal{F}$ of $K$. An
integral divisor $\mathcal{F}^1$ is said to be a \text{modulus of
  definition} of $\mathcal{X}$ if for any $X \in K$  coprime
to $\mathcal{F}$ and $X \equiv 1 (\mod ^+ \mathcal{F^1})$ we have
$\mathcal{X} ((X)) = 1$. $\mathcal{F}$ itself is clearly a modulus of
definition. The reason for our terminology is provided by the
following  
\begin{theorem*}
  If $\mathcal{X} $ is a character $\mod \mathcal{F}$ and
  $\mathcal{F}^1$ a modulus of definition of $\mathcal{X}$, there
  exists a unique  character $\mathcal{X}^1$ of $K \mod \mathcal{F}^1$
  such that for any divisor $\mathscr{U}$ prime to $\mathcal{F}
  \mathcal{F}^1$,  
  $$
  \mathcal{X} (\mathscr{U}) = \mathcal{X}^1 (\mathscr{U})
  $$
\end{theorem*}

Conversely,\pageoriginale if $\mathcal{X}$ and $\mathcal{X}^1$ are two characters mod
$\mathcal{F}$ and $\mathcal{F}^1$ respectively such that whenever
$\mathscr{U}$ is coprime to $\mathcal{F} \mathcal{F}^1$, we have
$\mathcal{X} (\mathscr{U}) =\mathcal{X}^1 (\mathscr{U})$, then
$\mathcal{F^1}$ is a modulus of definition of $\mathcal{X}$ and
$\mathcal{X}^1$ is the character associated to $\mathcal{X}$ by the
first part of the theorem. 

\begin{proof}
  Suppose $\mathcal{F}^1$ is a modulus of definition of
  $\mathcal{X}$. If $C _{\mathcal{F}^1}$ is any  class modulo
  $\mathcal{F}^1$, we can find a divisor $\mathscr{U}$ in
  $C_{\mathcal{F}^1}$ coprime to $\mathcal{F}$ and we define  
  $$
  \mathcal{X}^1 (C_{\mathcal{F}^1}) = \mathcal{X}  (\mathscr{U}) .
  $$  
\end{proof}

The definition is independent of the choice of $\mathscr{U}$ in
$C_{\mathcal{F}^1}$. For ,if  $\delta \in C_{\mathcal{F}^1}$ and is
coprime to $\mathcal{F}$, there  exists an $X \in K$ such that
$\mathscr{U} \delta^{-1}= (X), X \equiv 1 (\mod^+ \mathcal{F}^1)$ and
$X$ coprime to $\mathcal{F}$. Hence  
\begin{gather*}
  \mathcal{X} (\mathscr{U}\delta^{-1}) = \mathcal{X} ((X)) =1 , \\
  \mathcal{X} (\mathscr{U})  = \mathcal{X} (\delta ).
\end{gather*}

So defined, $\mathcal{X} ^1$ is evidently a character modulo
$\mathcal{F}^1$ which satisfies the condition  
$$
\mathcal{X} (\mathscr{U}) = \mathcal{X}^1(\mathscr{U})
$$
if $\mathscr{U}$ is coprime to $\mathcal{F} \mathcal{F'}$. The
uniqueness follows from the  fact that  this definition of
$\mathcal{F}^1$ is forced upon us by the above condition.  

The first  of our  theorem is proved. 

To\pageoriginale prove the second part, suppose $\mathcal{X}$ and $\mathcal{X}^1$ are
two characters modulo $\mathcal{F}$ and $\mathcal{F}^1$ respectively
satisfying the above condition. Then, if $(X)$ is coprime to
$\mathcal{F}$ and $X \equiv 1 (\mathcal{F}^1), (X) $ is coprime to
$\mathcal{F} \mathcal{F}^1$, and therefore 
$$
\mathcal{X} ((X)) = \mathcal{X}^1 ((X)) = 1
$$

This proves that $\mathcal{F}^1$ is a modules of definition of
$\mathcal{X}$ and $\mathcal{X}^1$ the associated character $\mod
\mathcal{F}^1$. 

\begin{coro*}%Corlry
  $\mathcal{F}$ is a modulus of definition of $\mathcal{X}^1$.
\end{coro*}

Our next theorem states that to any given character, there exists a
`smallest' modulus of definition. To prove this, we require the
following 

\begin{lemma*}%Lem
  If $\mathcal{F}^1$ and $\mathcal{F}''$ are two moduli of definition
  of a character modulo $\mathcal{F}$, their greatest common divisor
  $\mathcal{F}'''$ is also a modulus of definition. 
\end{lemma*}

\begin{proof}
  Let $X$ be coprime to $\mathcal{F}$ and $X \equiv 1 (\mod
  \mathcal{F}''')$. Find a $Y \in K$ such that  
  \begin{align*}
    & v_\mathscr{Y} (XY - 1) \ge v_\mathscr{Y} (\mathcal{F}^1) + \big |
    v_\mathscr{Y} (X) \big | \text{ if } v_\mathscr{Y} (\mathcal{F}^1
    \mathcal{F}^{''' - 1} > 0,\tag{1}\label{chap16:sec29:eq1} \\ 
    & v_\mathscr{Y} (X (Y - 1)) \ge v_\mathscr{Y} (\mathcal{F}'') +
    \big | v_\mathscr{Y} (X) \big | \text{ if } v_\mathscr{F}
    (\mathcal{F}'' \mathcal{F}^{''' -1} > 0,\tag{2}\label{chap16:sec29:eq2} 
  \end{align*}
and $v_\mathscr{Y} (Y - 1) \ge v_\mathscr{Y}
    (\mathcal{F} \mathcal{F}' \mathcal{F}") + \big | v_\mathscr{Y} (X)
    \big |$ if  $\mathscr{Y}$  divides  $\mathcal{F}
    \mathcal{F}' \mathcal{F}"$  but does not belong to the first two
    categories. \hfill (3)

    Since $\mathcal{F}'''$ is the greatest common divisor of
    $\mathcal{F}'$ and $\mathcal{F}''$ the first  two\pageoriginale categories are
    mutually exclusive, and the third category is by definition exclusive
    of (\ref{chap16:sec29:eq1}) or (\ref{chap16:sec29:eq2}). 
\end{proof}

Now, one can easily verify that $(Y)$ is coprime to
$\mathcal{F}$. Hence, $(XY)$ is also coprime to $\mathcal{F}$. We now
assert that $Y \equiv 1$ $(\mod ^+ \mathcal{F}'')$ and $XY \equiv 1
(\mod^+ \mathcal{F}' )$. To verify the first, suppose $\mathscr{Y}$ is
a prime divisor dividing $\mathcal{F}''$. Then $\mathscr{Y}$ must occur
in one of the three categories. If $\mathscr{Y} $ belongs to
(\ref{chap16:sec29:eq1}), 
\begin{multline*}
  v_\mathscr{Y} (Y - 1) =  v_\mathscr{Y} (XY - 1 + 1 - X) - v_\mathscr{Y} (X)\\
  \ge \min (v_\mathscr{Y} (XY - 1), v_\mathscr{Y} (X - 1)) - v_\mathscr{Y}(X),
\end{multline*}
and since $v_\mathscr{Y} (X- 1) \ge v_\mathscr{Y} (\mathcal{F}''') =
v_\mathscr{Y} (\mathcal{F}'') > 0, v_\mathscr{Y} (X) = 0$, and the
right hand side of the inequality becomes 
$$
\ge \min (v_\mathscr{Y} (\mathcal{F}'), v_\mathscr{Y} (\mathcal{F}''))
= v_\mathscr{Y} (\mathcal{F}''). 
$$

If $\mathscr{Y}$ belongs to category (\ref{chap16:sec29:eq2}), we get
$$
v_\mathscr{Y} (Y - 1) \geq v_\mathscr{Y} (\mathcal{F}'') + \big |
v_\mathscr{Y} (X) \big | - v_\mathscr{Y} (X) \ge v_\mathscr{Y}
(\mathcal{F}''). 
$$

Finally, for $\mathscr{Y}$ in $(3)$, we get again
$$
v_\mathscr{Y} (Y - 1) \ge v_\mathscr{Y} (\mathcal{F}'').
$$

The second congruence $XY \equiv 1 \pmod {^+ \mathcal{F}'}$ can be
proved similarly. Since $\mathcal{F}'$ and $\mathcal{F}''$ are moduli
of definition, we deduce that $\mathcal{X}(Y) = \mathcal{X} (XY) =
1$. Hence $\mathcal{X} (X) = 1$. This completes the proof of the fact
that $\mathcal{F}'''$ is a modulus of definition. 

The\pageoriginale theorem we mentioned is a fairly easy consequence. 

\begin{theorem*}%Thm
  Let $\mathcal{X}$ be a character of $K \mod \mathcal{F}$. Then there
  exists a unique integral divisor $m$ such that it divides every
  modulus of definition of $\mathscr{X}$ and every integral divisor
  which is divisible by it is a modulus of definition. $m$ is called
  the \textit{conductor } of $\mathcal{X}$.  
\end{theorem*}

If $\mathcal{X}_1$, is the associated character to $m$, the conductor of
$\mathcal{X}_1$ is $m$ itself. 

\begin{proof}
  By the previous lemma, the g.c.d. of all moduli of definition of
  $\mathcal{X}$ is an integral divisor which satisfies all the
  conditions of the first part of the theorem. Let $\mathcal{X}_1$, be
  the associated character $\mod m$. If the conductor of
  $\mathcal{X}_1$, is $m_1$, it is clear that $m_1$ is also a  modulus
  of definition of $\mathcal{X}$. Hence $m$ divides $m_1$, and $m_1$
  divides $m$ since $m_1$ is the conductor of $\mathcal{X}_1$. Thus,
  $m = m_1$. Our theorem is proved. 
\end{proof}

A character $\mathcal{F}$ modulo $\mathcal{F}$ is said to be \textit{
  primitive } or \textit{ proper } if $\mathcal{F}$ is the conductor
of $\mathcal{X}$. 
