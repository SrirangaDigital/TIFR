\chapter{Lecture 1}\label{chap1}

\section{Introduction}\label{chap1:sec1}

We\pageoriginale shall be dealing in these lectures with the algebraic aspects of
the theory of algebraic functions of one variable. Since an algebraic
function $w(z)$ is defined implicitly by an equation of the form $f(z,
w) = 0$, where $f$ is a polynomial, it is understandable that the
study of such functions should be possible by algebraic methods. Such
methods also have the advantage that the theory can be developed in
the most general setting, viz. over an arbitrary field, and not only
over field of complex numbers (the classical case). 

\begin{defi*}
  Let $k$ be a field. An {\em algebraic function field} $K$ over $k$
  is a finitely generated extension over $k$ of transcendence degree
  at least equal to one. If the transcendence degree of $K/k$ is $r$,
  we say that it is a function field in $r$ variables. 
\end{defi*}

We shall confine ourselves in these lectures to algebraic function
fields of one variable, and shall refer to them shortly as `function
fields'. 

If $K/k$ is a function field, it follows from our definition that
there exists an $X$ in $K$ transcendental over $k$, such that $K/k(X)$
is a finite algebraic extension. If $Y$ is another transcendental
element of $k$, it should satisfy a relation $F(X, Y) = 0$, where $F$
is a polynomial over $K$ which does not vanish identically. Since\pageoriginale $Y$
is transcendental by assumption, the polynomial cannot be independent
of $X$. Rearranging in powers of $X$, we see that $X$ is algebraic
over $k(Y)$. Moreover, 
\begin{multline*}
  [K : k(Y)] = [K  : k(X,Y)]. [k(X, Y): k(Y)]\\ 
  \le [K : k(X)] . [k (X,  Y): k(Y)] < \infty  
\end{multline*}
and thus $Y$ also satisfies the same conditions as $X$. Thus, any
transcendental element of $K$ may be used as a variable in the place
of $X$. 

The set of all elements of $K$ algebraic over $k$ forms a subfield
$k'$ of $K$, which is called the \textit{field of constants } of
$K$. Hence forward, we shall always assume, unless otherwise stated,
that $k = k'$,  i.e., that $k$ is algebraically closed in $K$. 

\section{Ordered Groups}\label{chap1:sec2}

\begin{defi*}
  A multiplicative(additive) Abelian group $W$ with a binary relation
  $< (>)$ between its elements is said to be an ordered group if 
\end{defi*}

\begin{enumerate}[\rm 0(1)]
\item for $\alpha ,\beta \in W$, one and only one of the relations
  $\alpha < \beta, \alpha = \beta, \beta < \alpha \, (\alpha > \beta ,
  \alpha = \beta > \alpha)$ holds. 
\item $\alpha < \beta , \beta < \gamma \Rightarrow \alpha < \gamma\, 
  (\alpha > \beta , \beta > \gamma \Rightarrow \alpha > \gamma)$ 
\item $\alpha < \beta, \delta \in W \Rightarrow \alpha \delta < \beta
  \delta (\alpha > \beta, \delta \in W \Rightarrow \alpha + \delta >
  \beta + \delta)$ 
\end{enumerate}

We shall denote the identity  (zero) element by 1(0). In this and the
following section, we shall express all our results in multiplicative
notation. $\alpha > \beta$ shall mean the same thing as $\beta <
\alpha$. 

Let $W_0$ be the set $\{\alpha : \alpha \in W \alpha < 1\}$. $W_0$ is
seen to be a semi group by 0(2) and 0(3). Moreover, $W = W_0 \cup \{ 1\}
\cup W^{-1}_0$ is a disjoint\pageoriginale partitioning of $W$ (where $W^{-1}_0$ means
the set of inverses of elements of $W_0$). Conversely, if an Abelian
group $W$ can be partitioned as $W_0 \cup \{ 1 \} \cup W^{-1}_0$, where $W_0$
is a semi-group, we can introduce an order in $W$ by defining $\alpha
< \beta$ to mean $\alpha \beta^{-1} \in W_0$; it is immediately
verified that $0(1)$, $0(2)$ and $0(3)$ are fulfilled and that $W_0$ is
precisely the set of elements $< 1$ in this order. 

For an Abelian group $W$, the map $\alpha \rightarrow \alpha^n$ ($n$
any positive integer) is in general only an endomorphism. But if $W$
is ordered, the map is a monomorphism; for if $\alpha$ is greater than
or less  than 1, $\alpha^n$ also satisfies the same inequality. 

\section{Valuations, Places and Valuation Rings}\label{chap1:sec3}

We shall denote the non-zero elements of a field $K$ by $K^*$.
\begin{defi*}
  A {\em Valuation} of a field $K$ is a mapping $v$ of $K^*$ onto an
  ordered multiplicative (additive) group $W$ (called the {\em group
    of the valuation} or the {\em valuation group}) satisfying the
  following conditions: 
  \begin{enumerate}[\rm V(1)]
  \item For $a, b \in K^* , v(ab) = v(a)v (b)$ $(v(ab) = v(a) + v(b))$;
    i.e. $v$ is homomorphism of the multiplicative group $K^*$ onto
    $W$. 
  \item For $a , b, a+b \in K^* , v(a+b) \le \max (v(a), v(b))\, 
    (v(a+b)) \ge$ $\min (v(a)$ $v(b)))$ 
  \item $v$ is non-trivial; i.e., there exists an $a \in K^*$ with
    $v(a) \neq 1(v(a) \neq 0)$ 
  \end{enumerate}
\end{defi*}

Let us add an element $0(\infty)$ to $W$ satisfying the following 
\begin{enumerate}[(1)]
\item $0.0 = \alpha . 0 = 0. \alpha = 0$ for every $\alpha \in W
  (\infty + \infty = \alpha + \infty = \infty + \alpha = \infty)$, 
\item $\alpha >0$\pageoriginale for every $\alpha \in W (\alpha < \infty)$. If we
  extend a valuation $v$ to the whole of $K$ by defining $v(0) =
  0\,(v(0) = \infty$, the new mapping also satisfies $V(1), V(2)$ and
  $V(3)$. 
\end{enumerate}	

The following are simple consequences of our definition.
\begin{enumerate}[(a)]
\item For $a \in K $,  \quad $v(a) = v(-a)$. To prove this, it is
  enough by $V(1)$ to prove that $v(-1) = 1$. But $v(-1)$. $v(-1) = v(1) = 1$ by
  $V(1)$, and hence $v(-1) = 1 $ by the remark at the end of $\S 2$. 
\item If $v(a) \neq v(b), v(a+b) = \max (v(a), v(b))$. For let $v(a) <
  v(b)$. Then, $v(a+b) \le \max (v(a), v(b)) = v(b) = v(a+b-a) \le
  \max (v(a+b), v(a)) = v(a+b)$ 
\item Let $a_i \in K, (i = 1, \ldots n)$. Then an obvious induction on
  $V(2)$ gives $v (\sum\limits^n_1 a_i) \le \max\limits^n_{i=1}
  v(a_i)$, and equality holds if $v(a_i) \neq v(a_j)$ for $i \neq j$. 
\item If $a_i \in K, (i = 1, \ldots n)$ such that $\sum\limits^n_1 a_i
  = 0$, then $v(a_i) = v(a_j)$ for at least one pair of unequal
  indices $i$ and $j$. For let $a_i$ be such that $v(a_i) \ge v(a_k)$
  for $k \neq i$. Then $v(a_i) = v(\sum\limits^n_{\substack{k=1\\k
      \neq i}} a_k) \le \max^n\limits_{\substack {k = 1 \\ k \neq i}}
  (v(a_k)) = v(a_j)$ for some $j \neq i$, which proves that $v(a_i) = v(a_j)$.
\end{enumerate}

Let $\sum $ be a field. By $\sum (\infty)$, we shall mean the set
of elements of $\sum$ together with an abstract element $\infty$ with
the following properties.  

\noindent
$
\begin{aligned}
\alpha + \infty & = \infty + \alpha = \infty ~\text{for every}~\alpha \in \sum. \\
\alpha~ . ~ \infty & = \infty . \alpha = \infty ~\text{for every}~ \alpha \in
\sum, \alpha \neq 0. 
\end{aligned}
$

\noindent
$\infty + \infty $ and $0. \infty$ are not defined.
\begin{defi*}
  A\pageoriginale {\em place} of a field $K$ is a mapping $\varphi$ of $K$ into
  $\sum U (\infty)$ (where $\sum$ may be any field ) such that  
  \begin{enumerate}[\rm P(1)]
  \item $\varphi (a+b) = \varphi (a) + \varphi(b)$.
  \item $\varphi (ab) = \varphi (a). \varphi (b)$.
  \item There exist $a, b \in K $ such that $\varphi (a) = \infty$ and
    $\varphi (b) \neq 0$ or $\infty$. $P(1) and P(2)$ are to hold
    whenever the right sides have a meaning. 
  \end{enumerate}
\end{defi*}

From this it follows, taking the $b$ of $P(3)$, that $\varphi(1)
\varphi (b) = \varphi (b)$, so that $\varphi (1) = 1$, and similarly
$\varphi (0) = 0$. 

Consider the set $\mathscr{O}_{\varphi}$ of elements $a \in K$ such
that $\varphi (a) \neq \infty$. Then by P(1), P(2) and P(3),
$\mathscr{O}_{\varphi}$ is a ring which is neither zero nor the whole
of $K$, and $\varphi$ is a homomorphism of this ring into
$\sum$. Since $\sum$ is a field, the kernel of this homomorphism is a
prime ideal $\mathscr{Y}$  of $\mathscr{O}_{\varphi}$ 

Let $b$ be an element in $K$ which is not in
$\mathscr{O}_{\varphi}$. We contend that $\varphi (\dfrac{1}{b}) =
0$. For if this mere not true, we would get $1 = \varphi(1) = \varphi
(b)$. $\varphi (\dfrac{1}{b}) = \infty$, by $P(2)$. Thus $\dfrac{1}{b}
\in \mathscr{Y}$, and thus $\mathscr{Y}$ is precisely the set of
non-units of $\mathscr{O}_{\varphi}$. Since any ideal strictly
containing $\mathscr{Y}$ should contain a unit, we see that
$\mathscr{Y}$ is a maximal ideal and hence the image of
$\mathscr{O}_{\varphi}$ in $\sum$ is again a field. We shall therefore
always assume that $\sum$ is precisely the image of
$\mathscr{O}_{\varphi}$ by $\varphi$, or that $\varphi$ is a mapping
onto $\sum U (\infty)$. 

The above considerations motivate the 
\begin{defi*}
  Let $K$ be a field. A {\em valuation ring} of $K$ is a proper subring
  $\mathscr{O}$ of $K$ such that if  $a \in K^*$, at least one of the
  elements a\pageoriginale\, $\dfrac{1}{a}$ is in $\mathscr{O}$.  
\end{defi*}

In particular, we deduce that $\mathscr{O}$ contains the unity
element. Let $\mathscr{Y}$ be the set of non-units in
$\mathscr{O}$. Then $\mathscr{Y}$ is a maximal ideal. For, let $a \in
\mathscr{O}, b \in \mathscr{Y}$. If $a b \notin \mathscr{Y}, ab$ would
be a unit of $\mathscr{O}$, and hence $\dfrac{1}{ab} \in \mathscr{O}$.
This implies that $a \dfrac{1}{ab} = \dfrac{1}{b} \in \mathscr{O}$,
contradicting our assumption that $b$ is a non-unit of
$\mathscr{O}$. Suppose that $c$ is another element of
$\mathscr{Y}$. To show that $b - c \in \mathscr{Y}$, we may assume
that neither of them is zero. Since $\mathscr{O}$ is a valuation ring,
at least one of $\dfrac{b}{c}$ or $\dfrac{c}{b} $, say $\dfrac{b}{c}$,
is in $\mathscr{O}$. Hence, $\dfrac{b}{c} - 1 = \dfrac{b-c}{c} \in
\mathscr{O}$. If $b-c$ were not in $\mathscr{Y}, \dfrac{1}{b-c} \in
\mathscr{O}$, and hence $\dfrac{1}{b-c} \cdot \dfrac{b-c}{c} =
\dfrac{1}{c} \in \mathscr{O}$, contradicting our assumption that $c
\in \mathscr{Y}$. Finally, since every element outside $\mathscr{Y}$
is a unit of $\mathscr{O}, \mathscr{Y}$ is a maximal ideal in
$\mathscr{O}$. 
