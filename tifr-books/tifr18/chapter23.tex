\chapter{Lecture 23}\label{chap23}

\setcounter{section}{37}
\section{Genus of a Constant Field Extension}\label{chap23:sec38} %sec 38

The\pageoriginale notations will be the same as in the previous lecture; in
addition, we shall denote by $g_R$ the genus of a function field $R$
and by $F_R (\mathscr{U})$ the vector space over the constant field of
$R$ of elements divisible by the divisor $\mathscr{U}$. The dimension
of $F_R (\mathscr{U})$ will be denoted by $N_R
(\mathscr{U}). \Lambda_R (\mathscr{U})$ shall denote the vector space
of repartitions of $R$ divisible by the divisor $\mathscr{U}$. 

\setcounter{thm}{0}
\begin{thm}\label{chap23:sec38:thm1}%Thm 1
  If $\lambda _{L/K} = 1$, i.e., if $K$ and $l$ are linearly disjoint
  over $k, g_L \le g_K$. For any divisor $\mathscr{U}$ of $K$, a base
  of $F_K (\mathscr{U})$ is a part of a base of $F_L (\mathscr{U})$,
  and hence $N_K (\mathscr{U}) \le N_L (\mathscr{U})$.  
\end{thm}

\begin{proof}
  The last part immediately follows from the fact that $F_K
  (\mathscr{U}) \subset F_L (\mathscr{U})$, if we observe that $K$ and
  $l$ are linearly disjoint over $k$ and that $l =l_0$. Taking a
  divisor $\mathscr{U}$ of $K$ such that  
  \begin{align*}
  - d_K (\mathscr{U})  > 2g_K - 2, - & d_L (\mathscr{U}) > 2g_L -2, \quad
  \text{ we have} \\
    N_K (\mathscr{U}) & = d_K (\mathscr{U}) - g_K+1,\\
    N_L (\mathscr{U}) & = d_L (\mathscr{U}) - g_L +1,
  \end{align*}
  and it follows from the previous inequality that $g_L \le g_K$.
\end{proof}

\begin{thm}\label{chap23:sec38:thm2}%Thm 2
  If $l_0$ is separably generated over $k, g_K= g_L$ and a basis of
  $F_K (\mathscr{U})$ is also a base of $F_L (\mathscr{U})$ for any
  divisor $\mathscr{U}$ of $K$; hence $N_L (\mathscr{U}) = N_K
  (\mathscr{U})$.  
\end{thm}

\begin{proof}
  We\pageoriginale first consider the case $l_0 = k (u), u$ transcendental over
  $k$. Let $Z \in F_L (\mathscr{U})$. Then $Z$ can be written
  uniquely in the from 
  $$
  Z = \frac{F(u)}{G(u)} = \frac{\sum\limits^n _{\nu=o} a_\nu
    u^\nu}{u^m +\sum\limits_{\mu=o}^{m-1} b_\mu u^\mu}, a_\nu, b_\mu
  \in K 
  $$
  with $F$ and $G$ coprime. We shall show that $b_\mu \in K$.
\end{proof}

Let $\mathscr{K}$ be a prime divisor of $L$ lying over a prime divisor
$\mathscr{Y}$ of $K$. Then by the remark at the end of the previous
lecture, 
{\fontsize{10pt}{12pt}\selectfont
$$
v_\mathscr{K} (G (u)) = v_\mathscr{K} (u^m + b_{m-1} u^{m-1} +\cdots+
b_m) = \min (0, v_\mathscr{Y} (b_1) ,\ldots, v_\mathscr{Y} (b_m)) \le
0 
$$}\relax

Hence the only possible prime divisors which occur in the numerator
$\mathfrak{z}_{G (U)}$ of $G(u)$ are those which over $K$. But now,
since $Z \in F_L (\mathscr{U})$, the divisor  
$$
(Z) \mathscr{U}^{-1} = \frac{\mathfrak{z}_F
  \mathscr{N}_G}{\mathscr{N}_F \mathfrak{z}_G \mathscr{U}} 
$$
is integral, and since $\mathfrak{z}_G$ and $\mathscr{N}_G$ are
coprime, any prime divisor occurring in $\mathfrak{z}_G$ must divide
$\mathfrak{z}_F$. But since $F $ and $G$ are coprime, there exist
polynomials $F_1 (u)$ and $G_1 (u)$ with 
$$
F (u) F_1 (u) + G (u) G_1 (u) = 1.
$$
If $F_1 (u) = c_o + c_1 u +\cdots+c_t u^t $ and $\mathscr{U}$ a prime
divisor of $L$ variable over $K$, we would have  
$$
v_\mathscr{U} (F_1 (u)) \ge \min_\nu (v_\mathscr{U} (c_\nu) + \gamma
v_\mathscr{U} (u)) =0 
$$
since\pageoriginale $v_\mathscr{U} (u) = v_\mathscr{U} (V_\mathscr{U} (c_\nu)) =
0$. Similarly $v_\mathscr{U} (G_1 (u)) \ge 0$. Hence we have  
\begin{align*}
  0= v (1) & \ge \min (v_\mathscr{U} (F (u)) + v_\mathscr{U} (F_1(u)),
  v_\mathscr{U} (G (u)) +v_\mathscr{U} (G (u))\\ 
  & \ge \min (v_\mathscr{U} ((F (u))), v_\mathscr{U} (g (u))),
\end{align*}
and therefore $\mathfrak{z}_{G (u)}$ and $\mathfrak{z}_{F (u)}$ can
not have a common prime divisor. Therefore $\mathfrak{z}_{G (u)} =
\mathscr{N}$ and $G(u)$ is constant. 

It follows that for any prime prime divisor $\mathscr{Y}$ of $K$,
$$
v_\mathscr{Y} (Z) = v_\mathscr{Y} (F (u)) = \min_\nu v_\mathscr{Y}
(a_\nu) \ge v_\mathscr{Y} (\mathscr{U}) 
$$
and therefore $a_\nu \in F_K (\mathscr{U})$.

Thus, we see that $F_L (\mathscr{U})$ is the vector space generated
over $l=l_0$ by $F_K (\mathscr{U})$. from the liner disjointness of
$K$ and $l$, we deduce that $N_K (\mathscr{U}) = N_L (\mathscr{U})$.  

Next suppose $l =l_0 = k (\alpha)$ is finite separable (and therefore
simple) over $k$. Any element $Z \in F_L (\mathscr{U})$ can be
written uniquely in the form 
$$
Z = c_0 + c_1 \alpha +\alpha+ c_{n-1} \alpha^{n-1}, c_i \in K
$$
where $n$ is the degree of $\alpha$ over $k$ or $K$. 

Let $L_1$ be the smallest normal extension of $L$ over $K$. Taking
conjugates in the above equation over $K$, we obtain 
$$
Z^{(i)} = c_0 + c_1 \alpha^{(i)} +\cdots+ c_{n-1} \alpha^{(i)^{n-1}} (i=1 ,\ldots n)
$$

We\pageoriginale may solve for $c_k$ of obtain
$$
a_k=\frac{
  \begin{vmatrix} 
    1 \alpha^{(1)} & \cdots  &
    \alpha^{(1)^{k-2}}Z^{(1)} & \cdots & \alpha^{(1)^{k}} & \cdots&
    \alpha^{(1)^{n-1}} \\ 
    1 \alpha^{(n)} & \cdots  &
    \alpha^{(n)^{k-2}}Z^n & \cdots & \alpha^{(n)^{k}} & \cdots &
    \alpha^{(n)^{n-1}}  
  \end{vmatrix}}
  {\begin{vmatrix} 
      1 \alpha^{(1)} &
      \cdots \cdots \cdots \cdots \cdots \cdots \cdots \cdots \cdots
      \cdots  \cdots  & \alpha^{(1)^{n-1}}\\ 
      1 \alpha^{(n)} & \cdots
      \cdots \cdots\cdots \cdots  \cdots \cdots  \cdots \cdots  \cdots
      \cdots & \alpha^{(n)^{n-1}}  
  \end{vmatrix}} 
$$

The denominator is a constant $\neq 0$ and the numerator is a linear
combination of $Z^{(i)}$ with constant coefficients. Since
$\mathscr{U}$ is a divisor of $K$, it may be easily verified that every
conjugate $Z^{(i)}$ of $Z$ is divisible by $\mathscr{U}$ in
$L_1$. Hence the $c_k$ are divisible by $\mathscr{U}$ in $L_1$ and
hence in $K$. We have proved in this case also that $F_L
(\mathscr{U})$ is generated by $F_K (\mathscr{U})$ over $i$. The
equality $N_L (\mathscr{U}) = N_K (\mathscr{U})$ again follows from
the liner disjointness of $K$ and $l =l_o$ over $k$.  

The case of any separably generated extension $l_0$ now follows along
familiar lines. Any $Z \in F_L (\mathscr{U})$ is contained in
a field $L^1 = Kl^1$, where $l^1$ is a finitely separably generated
extension of $k$; hence it is an element of $F_{L^1} (\mathscr{U})$. But
in the case of a finitely separably generated extension, the theorem
follows by induction if we use the first two cases. Hence we again
obtain the fact that $F_L (\mathscr{U})$ is generated by $F_K
(\mathscr{U})$ over $l$, and the equality $N_L (\mathscr{U}) = N_K
(\mathscr{U})$.  

Choosing a divisor $\mathscr{U}$ with $-d_k (\mathscr{U}) > 2g_K -2,
-d_L (\mathscr{U}) > 2g_L -2$ 

\medskip
\noindent
$\begin{aligned}
  \text{we have} \qquad   - d_K (\mathscr{U})  &= N_K (\mathscr{U}) -
  g_K +1,\\
  \text{and} \qquad - d_L (\mathscr{U}) & = N_L (\mathscr{U}) - g_L +1.
\end{aligned}$\pageoriginale\
\medskip

But since $\lambda_{L/K} = 1$, it follows that $d_K (\mathscr{U}) =
d_L (\mathscr{U})$ and therefore $g_L = g_K$. The theorem is
completely proved. 

\begin{thm}\label{chap23:sec38:thm3}%Thm 3
  For any constant field extension $L$ of $K$, we have $g_L
  \lambda_{L/K} \le g_K$. (In particular $g_L \le g_K$).  
\end{thm}

\begin{proof}
  Since the genus is preserved for for a purely transcendental
  extension of the constant field, and since $\lambda_{L /K} = 1$for
  such an extension, it follows from the obvious formula
  $\lambda_{L/L^1} \lambda_{L^1/K} = \lambda_{L/K} L \supset L^1
  \supset K$ that it is enough to prove the theorem for algebraic
  extension  $l_0$ of $k$. 
\end{proof}

First assume that $l_0$ is a finite extension of $k$ with a basis
$\alpha_1 ,\ldots, \alpha_n$ over $k$. By lemma \ref{chap22:sec37:lem1} of the previous
lecture, $L/K$ is a finite extension of degree $n_o \le n$, and we may
assume that $\alpha_1 ,\ldots, \alpha_{n_0}$ form a basis of $L$
over $K$.   

Let us denote by $\mathscr{X}_K$ and $\mathscr{X}_L$ the vector spaces
of repartitions of $K$ and $L$ over $k$ and $l$ respectively. We define
a map $\sigma$ of the direct product $\prod\limits^{n_o}_{\nu=1}
\mathscr{X}_K \text{ of }\mathscr{X}_K$ by itself $n_o$ times into the
space $\mathscr{X}_L$ by defining the image of  $(\mathscr{C}_1
,\ldots, \mathscr{C}_{n_0}) \in \prod\limits_{\nu =1}^{n_0}
\mathscr{X}_K \text{ in } \mathscr{X}_L$ to be the repartition
$\mathscr{C}$ of $L$ defined by 
$$
\mathscr{C} (\mathscr{K}) = \sum^{n_o}_{\nu=1} \alpha_\nu
\mathscr{C}_\nu (\mathscr{Y}) 
$$
for any prime divisor $\mathscr{K}$ of $L$, where $\mathscr{Y}$ is the
prime divisor of $K$ lying\pageoriginale below $\mathscr{K}$. It is easy to verify
that $\mathscr{C}$ so defined is a repartition, and that $\sigma$ is a
k-isomorphism of $\prod\limits^{n_0}_{\nu=1} \mathscr{X}_K$ into
$\mathscr{X}_L$. Let the image under $\sigma$ be the subspace
$\mathscr{X}^o_L$ of $\mathscr{X}_L \mathscr{X}^o_L$ is a  vector
subspace of $\mathscr{X}_L$, if the latter is considered as a vector
space over $k$. 

If $y = \sum\limits^{n_0} _1 \alpha _\nu x_\nu, \, x_\nu \in K$,
is any element of $L$, the element $(x_1 ,\ldots,x_{n_0})$  of
$\prod\limits_{\nu=1}^{n_0} \mathscr{X}_K$ (keeping in mind that we
have identified $K (L)$ with a subspace of $\mathscr{X}_K
(\mathscr{X}_L)$ and we may use the same symbol for an element of $K
(L)$ and the corresponding repartition in $\mathscr{X}_K
(\mathscr{X}_L)$) clearly goes to the repartition $y$ of
$\mathscr{X}_L$. Hence, $\mathscr{X}^0_L \supset L$. We assert that
for any divisor $\mathscr{U}$ of $K$, we have     
$$
\mathscr{X}_L = \mathscr{X}_L^0+ \Lambda_L (\mathscr{U})
$$

To prove this, consider a repartition $\mathscr{C}$ of $L$. Let
$\mathscr{K}_1 ,\ldots, \mathscr{K}_r$ be the prime divisors of $L$
lying over a prime divisor $\mathscr{Y}$ of $K$. We can find an
element $y (\mathscr{Y})$ of $L$ satisfying   
$$
v_{\mathscr{K}{_{\nu}}} (y (\mathscr{Y}) - \mathscr{C}
(\mathscr{K}_\nu)) \ge v_{\mathscr{K}{_{\nu}}} (\mathscr{U}), ~ (\nu
= 1 ,\ldots r) 
$$

Define a repartition $\mathscr{Y}$ of $L$ by

$\mathscr{Y} (\mathscr{K}) = y (\mathscr{Y})$ if $\mathscr{K}$ lies
over $\mathscr{Y}$ and $v_{\mathscr{K}^1}(\mathscr{U}) < 0$ or if
$\mathscr{K}$ lies over $\mathscr{Y}$ and $v_\mathscr{K} (\mathscr{C})
\neq 0$ for some prime divisor $\mathscr{K}^1$ lying over
$\mathscr{Y}$, $\mathscr{Y} (\mathscr{K}) = 0$ otherwise. 

Clearly,\pageoriginale $\mathscr{C} - \mathscr{Y} \Lambda_L (\mathscr{U})$. We shall
show that $\mathscr{Y} \varepsilon \mathscr{X}^0_L$. Let $y
(\mathscr{Y}) = \sum\limits_{\nu=1}^{n_o} \alpha_\nu y^1_\nu
(\mathscr{Y}), y^1_\nu (\mathscr{Y}) \in K$. Define repartitions
$\mathscr{Y} ^1_\nu$ of $K (\nu = 1 ,\ldots n_0)$ by putting  

$\mathscr{Y}^1_\nu (\mathscr{Y}) = y^1 (\mathscr{Y}) \text{ if }
v_\mathscr{Y} (\mathscr{U}) \neq 0$ or $v_\mathscr{K} (\mathscr{C}) <
0$ for some $\mathscr{K}$ lying over $\mathscr{Y}$, $\mathscr{Y}^1_\nu
(\mathscr{Y}) = 0$ otherwise.  

We then have $\sigma ((\mathscr{Y}^1_1 ,\ldots, \mathscr{Y}^1_{n_o}))
= \mathscr{Y} \in \mathscr{X}_L^0$. Our assertion in proved. 

Now, if $\mathscr{N}$ denotes the unit divisor, we have
$$
\displaylines{\hfill 
\dim_k \frac{\mathscr{X}_k}{\Lambda_K (\mathscr{N}) + K}= g_K\hfill \cr
\text{and}\hfill 
\dim_l \frac{\mathscr{X}_L}{\Lambda_L {(\mathscr{N})+L}} = g_L\hfill }
$$

From the first equation,
$$
n_0 g_K = \dim_k \frac{\prod\limits_{\nu =1}^{n_0} \mathscr{X}
  _K}{\prod\limits_{\nu=1}^{n_0} \Lambda_K (\mathscr{N}) +
  \prod\limits_{\nu=1}^{n_0} K} 
$$
and applying $\sigma$
$$
n_0 g_K = \dim_K \frac{\mathscr{X}^0_L}{\Lambda_L^0 (\mathscr{N}) +L},
$$
where\pageoriginale $\Lambda_L^0 (\mathscr{U})$ is the $k$-subspace $\sigma
(\prod\limits_{\nu=1}^{n_0} \Lambda_K (\mathscr{U}))$ of
$\mathscr{X}_L$ for any divisor $\mathscr{U}$ of $K$.  

On the other hand,
\begin{align*}
  n g_L & = n\dim_l \frac{\mathscr{X}_L}{\Lambda_L (\mathscr{N}) +L}
  = \dim_k \frac{\mathscr{X}_L}{\Lambda_L (\mathscr{N})+ L}\\ 
  & = \dim_k \frac{\mathscr{X}_L^o + \Lambda_L (\mathscr{N}) +
    L}{\Lambda_L (\mathscr{N}) +L} = \dim_k \frac{\mathscr{X}_L^o}{
    \mathscr{X}_L^o \cap (\Lambda_L (\mathscr{N})+ L)}\\ 
  &=\dim_k \frac{\mathscr{X}^0_L}{\Lambda_L^o (\mathscr{N})+ L} -
  \dim_k \frac{\mathscr{X}_L^0 \cap (\Lambda_L
    (\mathscr{N})+L)}{\Lambda_L^o (\mathscr{N}) + L}
\end{align*}
$(\Lambda_L^0(\mathscr{N})$ is obviously a subspace of
$\mathscr{X}_L^0 \cap (\Lambda_L (\mathscr{N}) +L))$. Hence we deduce
that  
\begin{gather*}
  n g_L \le n_0 g_K\\
  g_L \le \frac{n_0}{n} g_K
\end{gather*}

But if $X$ is any element of $K$ transcendental over $k$, we obtain
\begin{align*}
  \lambda_{L/K} = \frac{d_K (\mathscr{N}_X)}{d_L (\mathscr{N}_X)} =
  \frac{\big[ K : k (X)\big]}{\big[ L:l(X)\big]} & = \frac{\big[K : k
      (X)\big]}{\big[ L : k (X)\big]} . \big[l(X) : k (X)\big]\\ 
  & = \frac{\big[ l: k\big]}{\big[ L:k\big]} = \frac{n}{n_o}
\end{align*}
and\pageoriginale our result is proved in the case of a algebraic extension $l_0$ of $k$.

To prove the theorem in the case of an arbitrary algebraic extension,
we shall show that there exists a finite extension $l^1_0$ of $k$ such
that for $L^1 = Kl^1_0$, we have $\lambda_{L/L^1} = 1$. It would then
follow from theorem $1$ that $g_L \leq g_{L^1}$ and our result would
follow. 

Let $X \in K$ be transcendental over $k$ and $\mathscr{N}_X$ the
denominator of $X$. We have 
\begin{align*}
  m &= d_K (\mathscr{N}_X ) = \bigg[ K : k(X) \bigg] ,\\
  m_\circ &= d_L (\mathscr{N}_X) = \bigg [L : l(X) \bigg] .
\end{align*}

A base $x_1, \ldots \ldots x_m$ of $K/{k(X)}$ spans $L$ over $l(X)$,
and hence we have $m - m_\circ$ relations 
$$
\sum^{m}_{\nu =1} x_\nu C_{\nu \mu} = 0, ~ \mu = 1, 2, \ldots\ldots , m-m_\circ
$$
with coefficients $C_{\nu \mu}$ in $l(X)$ such that the $m - m_\circ$
vectors 
$$
(C_{1 \mu}, \ldots\ldots C_{m \mu})~(\mu =1, \ldots\ldots ,
m-m_\circ)
$$ 
are linearly independent over $l(X)$. The rational
functions $C_{\nu \mu}$ of $X$ over $l$ have coefficients in a
finitely generated subfield $l^1_0 \supset k$ of $l$. Since $L^1 =
Kl^1_0$ is spanned by $x_1, \ldots x_m$ over $l^1_0(X)$ and
since the $C_{\nu \mu} $ are in $l^1_0(X)$, we deduce that 
$$
d_{L^1}(\mathscr{N}_X) \leq \bigg[L^1 : l^1_X (X)\bigg] \leq
m_0 = d_L (\mathscr{N}_X), 
$$
and\pageoriginale since we already have $\lambda_{L/L^1} \geq 1$, we deduce that
$\lambda_{L/_L1} = 1$. 

Our theorem is completely proved.
\begin{remark*}
  If $\lambda_{L/_K} > 2$, we can actually assert that
  $\lambda_{L/_{K}} g_L < g_K$. For suppose $\lambda_{L/_K} g_L =
  g_K$. Let $\omega$ be a non-zero differential of $K$. Then, we have 
  $$
  d_L((\omega)) = \frac{d_K((\omega))}{\lambda_{L/_K}}= \frac{2
    g_k -2}{\lambda_{L/_K}}, 
  $$
  and hence (since $d_L ((\omega))$ is an integer) $\lambda_{L/_K}$
  divides $2g_K -2$. But from the equation $\lambda_{L/K}g_L = g_K$,
  we deduce that $\lambda_{L/K}$ divides $g_K$ and hence $2g_K$. This
  implies that $\lambda_{L/K}$ divides $2, \lambda_{L/K} \leq 2$,
  which is a contradiction. Hence we have the strict inequality. 
\end{remark*}

If however $\lambda_{L/K} = 2$, we may have $2g_L = g_K$ as the
following example follows. 

Let $k$ be a field of characteristic $2$ and $\alpha_\circ, \alpha_1$
two elements of $k$ such that $\bigg [k \left(\alpha^{\frac{1}{2}}_0,
  \alpha^{\frac{1}{2}}_1\right) : k \bigg] = 4$. Then it can be seen easily
that if $X$ is a transcendental element over $k$, the polynomial $Y^2
- (\alpha_1 + \alpha_1 X^2)$ is an irreducible polynomial of $Y$ over
$k(X)$. Hence, if $Y$ is a root of the equation $Y^2 = \alpha_0 +
\alpha_1 X^2, [k (X, Y) : k(X)] = 2$. Put $K = k(X, Y)$. It can be
proved (see the example for $l \neq l_0 $ given in Lecture $21$)
that the constant fields of $k(X, Y)$ is $k$.  

Now, it can be deduced by taking valuations in the equation $Y^2 =
\alpha_0 + \alpha_1 X^2$ that $\mathscr{N}_Y = \mathscr{N}_X$. Hence
the elements  $1, X, X^2, \ldots X^n, Y, YX, \ldots$, $YX^{n-1}$ are all
elements of $K$ divisible by $\mathscr{N}^{-n}_X$.\pageoriginale Since they are
linearly independent, we have $l(\mathscr{N}^{-n}_X) ]\geq 2n+1$,
  and the Riemann-Roch theorem gives $g_K = 0$. 

Now, let $l_0$ be any extension of $k$ such that $\bigg[ l_0
  \left(\alpha^{\frac{1}{2}}_0, \alpha^{\frac{1}{2}}_1\right) : l_0
  \bigg] < 4$, and $L = Kl_0$. Since $g_L \lambda_{L/_K} \leq g_K
= 0$, we necessarily have $g_L = 0$. We shall show that
$\lambda_{L/_K} = 2$. In fact, since $\bigg[ l_0 \left(\alpha
  ^\frac{1}{2}_0, \alpha^{\frac{1}{2}}_1\right) : l_0 \bigg]
\leq 2$, there is a relation of the form 

$\beta \alpha^{\frac{1}{2}}_\circ + \gamma \alpha^{\frac{1}{2}}_1 =
\delta, \beta, \gamma ,\delta \in l_0$, not all zero. 

We may solve for $\alpha^{\frac{1}{2}}_0$ and
$\alpha^{\frac{1}{2}}_1$ from this and the equation 
$$
\alpha^{\frac{1}{2}}_0 + X \alpha^{\frac{1}{2}}_1 = Y,
$$
since $\beta X - \gamma \neq 0$, thus proving that
$\alpha^{\frac{1}{2}}_0, \alpha^{\frac{1}{2}}_1 \in l_0
(X,Y)$, $l=l_0
\left(\alpha^{\frac{1}{2}}_\circ,\alpha^{\frac{1}{2}}_1\right)$. Hence, $L =
l(X)$ and $d_L(\mathscr{N}_X) = 1$, Since $d_K
(\mathscr{N}_X) = 2, \lambda_{L/K} = 2$. 

One can in fact show that the above example covers the general case
when $g_L = g_K$ and $\lambda_{L/_K} > 1$. 

To prove this, we first observe that we must have $g_L = g_K = 0$, for
otherwise, we would obtain 
$$
g_L < \lambda_{L/K} g_L \leq g_K.
$$
If now, $W$ were the canonical class of $K$, $d(W^{-1}) = 2$ and
$N(W^{-1}) = 3$. Hence there exists an integral divisor $\mathscr{U}$
in the class $W^{-1}$ of degree $2$, and $N_K(\mathscr{U}^{-1})
=3$. Let $1, X, Y$ be a basis of $F_K(\mathscr{U}^{-1})$. Then $X$ is
not\pageoriginale a constant, and since $\mathscr{N}_X$ divides $\mathscr{U}$
and $d(\mathscr{U}) = 2$, we see that
$\mathscr{U}=\mathscr{N}_X$. Hence, 
$$
\bigg[ K : k (X) \bigg] = d (\mathscr{N}_X) = 2.
$$

Now, we assert that $Y \notin k (X)$. For if it were, we can write
\break \hbox{$Y= \dfrac{f_1(X)}{f_2(X)}$,} $f_1$ and $f_2$ being coprime
polynomials. Then,\break  $(Y) = \dfrac{\mathfrak{z} f_1}{\mathfrak{z}
  f_2}\mathscr{N}_X^{\deg f_2 - \deg  f_1}$, and since $(Y)
\mathscr{N}_X$ is integral, we 
deduce that $f_2$ is constant and $\deg f_1 = 1$. This contradicts our
assumption that $1, X, Y$ are linearly independent. Hence, $k(X, Y)
\neq k(X)$, and since $\bigg [ K : k(X)\bigg] = 2, K = k(X, Y)$. Also,
since $\lambda_{L/K} > 1$, $Y$ should be purely inseparable over
$k(X)$, and therefore satisfy an equation of the form 
$$
Y^2 = R(X),
$$
$R(X)$ being a rational function of $X$. Since $Y^2$ is divisible by
$\mathscr{N}^{-2}_X$, we deduce by an argument similar to the one
used above that $R(X)$ is a polynomial of degree at most two. Thus, 
$$
Y^2 = \alpha_\circ + \alpha_1 X + \alpha_2 X^2
$$
Since $X$ should also be purely inseparable over $k(Y)$, we deduce
that $\alpha_1 = 0$. 

Now, if $\bigg[ k \left(\alpha^{\frac{1}{2}}_{\circ},
  \alpha^{\frac{1}{2}}_{2}\right) : k \bigg]$ were not equal to $4$, it is
less than or equal to $2$. Hence we have a relation of the form 
$$
\beta \alpha^{\frac{1}{2}}_{\circ} + \gamma \alpha^{\frac{1}{2}}_{2}
= \delta, \beta, \gamma, \delta \in k. 
$$

This\pageoriginale together with the relation
$$
\alpha^{\frac{1}{2}}_{\circ} + \alpha^{\frac{1}{2}}_{2} X = Y
$$
proves that $\alpha^{\frac{1}{2}}_{\circ},
\alpha^{\frac{1}{2}}_{2}$, are both in $K$ and hence in $k$. This
would imply that $Y \in K (X)$, which if false. Hence, 
$$
\bigg[ k \left(\alpha^{\frac{1}{2}}_{\circ},
  \alpha^{\frac{1}{2}}_{2}\right) : k \bigg ] = 4. 
$$ 

Finally, suppose $\bigg[l_0 \left(\alpha^{\frac{1}{2}}_0,
  \alpha^{\frac{1}{2}}_{2}\right) : l_0 \bigg] = 4$. Then for any
subfield $l^1_0$ of $l_0$ containing $k$, we have $\bigg[
  l^1_0 \left(\alpha^{\frac{1}{2}}_0, \alpha^{\frac{1}{2}}_{2}\right)
  : l^1_0 \bigg ] = 4$. Hence the constant field of $Kl^1_0$
is $l^1_0$. This implies that (see Lecture \ref{chap22})
$\lambda_{L/_K}=1, a$ contradiction. 

Our assertion is proved.

If $g_L = g_K > 0$, we deduce from the equation
$$
g_L = \lambda_{L/_K} g_L = g_K
$$
that $\lambda_{L/_K} = 1$.

We now prove the following
\begin{theorem*}
  If $g_L = g_K$ and $\lambda_{L/_K}=1$, then for any divisor
  $\mathscr{U}$ of $K$ a basis of $F_K(\mathscr{U})$ over $k$ is also
  a basis of $F_L(\mathscr{U})$ over $l$; in particular
  $N_L(\mathscr{U}) = N_K(\mathscr{U})$. 
\end{theorem*}

\begin{proof}
  Let us denote by $l F_K(\mathscr{U})$ the vector space generated over
  $l$ by $F_K(\mathscr{U})$ in $L$. Clearly we have $lF_K(\mathscr{U})
  \subseteq F_L(\mathscr{U})$. Since $\lambda_{L/_K} = 1$, $l$ and $K$
  are linearly disjoint over $k$ and we obtain 
  $$
  N_K (\mathscr{U}) = \dim_l lF_K(\mathscr{U}) \leq \dim_l F_L
  (\mathscr{U}) = N_L(\mathscr{U}) 
  $$

  Now,\pageoriginale let $\mathscr{U}$ be any divisor with $d_K(\mathscr{U}) =
  d_L(\mathscr{U}) < 2 - 2g_K$. 
\end{proof}

Then we have
\begin{align*}
  N_K(\mathscr{U}) & + d_K(\mathscr{U}) = 1-g_K ,\\
  N_L(\mathscr{U}) & + d_L(\mathscr{U}) = 1-g_L,
\end{align*}
and since $d_K(\mathscr{U}) = d_L(\mathscr{U})$ and $g_K = g_L$, we obtain

\medskip
\noindent
$\begin{aligned}
  N_K (\mathscr{U}) & = N_L(\mathscr{U}),\\
  \text{and} \hspace{3cm}  lF_K(\mathscr{U}) & = F_L (\mathscr{U}).
\end{aligned}$
\medskip

To draw the same conclusion for an arbitrary divisor $\mathscr{U}$,
choose two divisors $\delta$ and $\mathcal{L}$ of $K$ such that $(i)$
the least common multiple of $\delta$ and $\mathcal{L}$ is
$\mathscr{U}$ and $(ii) d (\delta) < 2-2g_K, d(\mathcal{L}) <
2-2g_L$. This is clearly possible. We then have $F_K(\delta) \cap
F_K(\mathcal{L}) = F_K(\mathscr{U}), F_L(\delta) \cap F_L(\mathcal{L})
= F_L (\mathscr{U})$. 

Let $\alpha_1, \ldots , \alpha_m$ be a basis of
$F_K(\mathscr{U})$. Complete this to a basis $\beta_1, \ldots$,
$\beta_1, \alpha_1, \ldots, \alpha_m$ of $F_K(\delta)$ and to a basis
$\gamma_1, \ldots$, $\gamma_n, \alpha_1, \ldots, \alpha_m$ of
$F_K(\mathcal{L})$. We assert that $\alpha_1, \ldots, \alpha_m,
\beta_1, \ldots , \beta_1, \gamma_1, \ldots , \gamma_n$ are linearly
independent elements of $K$ over $k$. In fact, if we had a linear
relation 
$$
\sum a_i \alpha_i + \sum b_j \beta_j = \sum c_k \gamma_k, ~a_i, b_j, c_k \in k,
$$
since\pageoriginale the left side is an element of $F_K(\delta)$ and the right side
an element of $F_K(\mathcal{L}), \sum c_k \gamma_k$ is an element of
$F_K(\mathscr{U})$ and therefore $c_k=0, a_i = 0$ and $b_j = 0$. Hence
$\beta_1, \ldots, \beta_l, \alpha_1, \ldots, \alpha_m, \gamma_1,
\ldots, \gamma_n$ is also a set of linearly independent elements over
$l$. 

Now suppose $y$ is an element of $F_L(\mathscr{U}) = F_L (\delta) \cap
F_L (\mathcal{L})$. Since $F_L(\delta )$ has for basis $(\alpha_1,
\ldots, \alpha_m, \beta_1, \ldots \beta_l)$ over $1$ and $y \in
F_L(\delta)$, we have 
$$
y = \sum_i a_i \alpha_i + \sum_j b_j \beta_j, ~a_i , b_j \in l,
$$
and similarly, since $y \in F_L(\mathcal{L})$ and $F_L(\mathcal{L})$
has for basis $(\alpha_1, \ldots , \alpha_m, \gamma_1$, $\ldots,
\gamma_n)$, we have 
$$
y= \sum_j c_j \alpha_j + \sum_K d_k \gamma_k, ~c_j, d_k \in l.
$$

Equating the above two expressions for $y$, we obtain (since
$\alpha_1, \ldots$, $\alpha_m , \beta_1$, $\ldots$, $\beta_1 , \gamma_1,
\ldots, \gamma_n$ are linearly independent over $l$) 
$$
a_i = c_i~, ~b_j = d_k = 0.
$$

Thus $y \in l F_K (\mathscr{U})$ and hence we have $F_L(\mathscr{U}) =
l F_K (\mathscr{U})$. Again by linear disjointness of $l$ and $K$ over
$k$, we obtain 
$$
N_K(\mathscr{U}) = N_L (\mathscr{U})
$$

Our theorem is proved.


The converse of the above theorem is very easy to prove. If\break $N_K
(\mathscr{U}) = N_L (\mathscr{U})$ for all divisors $\mathscr{U}$, or
even only for a sequence of\pageoriginale divisors $\mathscr{U}$ with
$d_k(\mathscr{U}) \to -\infty$, we have the following equations for
$-d_k(\mathscr{U})$ sufficiently large 
\begin{align*}
  N_K (\mathscr{U}) &+ d_K (\mathscr{U}) = 1-g_K,\\
  N_L (\mathscr{U}) &+ d_L (\mathscr{U})=1-g_L.
\end{align*}

Hence we obtain
$$
\lambda_{L/K}= \frac{d_K(\mathscr{U})}{d_L (\mathscr{U})} = \frac{-N_K
  (\mathscr{U}) + 1-g_K}{-N_L(\mathscr{U}) +1-g_L}, 
$$
and letting $d_K(\mathscr{U}) \to - \infty$, and observing that the
right hand side has limit $1$, we obtain $\lambda_{L/_K}=1$. Hence
$d_K(\mathscr{U}) = d_L (\mathscr{U})$ for any divisor $\mathscr{U}$,
and we obtain $g_K = g_L$. 

\begin{coro*}
  If $g_L = g_K$ and $\lambda_{L/K} = 1$, the natural homomorphism of
  the class group $\mathscr{K}_K$ of $K$ into the class group
  $\mathscr{K}_L$ of $L$ is an isomorphism. Under this isomorphism,
  the canonical class of $K$ goes to the canonical class of $L$. 
\end{coro*}

\begin{proof}
  Let $\mathscr{U}$ be any divisor of $K$ which is a principal divisor
  in $L$. Then $d_L(\mathscr{U}) = 0$ and $N_L(\mathscr{U}) = 1$. By
  the above theorem, $d_K(\mathscr{U})=0$ and
  $N_K(\mathscr{U})=1$. This proves that $\mathscr{U}$ is a principal
  divisor of $K$, and thus the kernel of the homomorphism of
  $\mathscr{K}_K$ consists of the unit class alone. Thus, the map is
  an isomorphism. We shall use the same symbol for a class of $K$ and
  its image as a class of $L$. 
\end{proof}

Also, if $W_K$ is the canonical class of $K$, $d_L(W_K) = d_K (W_K) =
2 g_L-2$\pageoriginale and $N_L(W_K) = N_K(W_K) = g_L$, which proves that $W_K$ is
the canonical class of $L$. 

\section{The Zeta Function of an Extension}\label{chap23:sec39}%sec 39

Let $K/_k$ be an algebraic function field with a finite field of
constants $k$ containing $q$ elements. Let $k_f$ be the extension of
$k$ of degree $f$. Since $k$ is perfect, $k_f/k$ is a separable
extension, and hence the constant field of $L_f = K k_f$ is $k_f$. Let
$\mathscr{Y}$ be a prime divisor of $K$ and
$\mathscr{K}_1,\ldots,\mathscr{K}_h$ the prime divisors of $L_f$ lying
over $\mathscr{Y}$. Then, since $k_f/k$ is separable, the
$\mathscr{K}_i$ are unramified over $K$, and we have 
$$
\mathscr{Y} = \mathscr{K}_1 \ldots \ldots \mathscr{K}_h.
$$

Now, since $k_f/k$ is separable, we know that $L_{\mathscr{K}_i}=
k_\mathscr{Y}k_f$, and since the degrees of $K_\mathscr{Y}$ and $k_f$
over $k$ are respectively $d_K(\mathscr{Y})$ and $f$, the degree of
$L_{\mathscr{K}_i}$ over $k$ is l.c.m. $[d_K(\mathscr{Y}), f]$. Thus,  
$$
f d_{L_f} (\mathscr{K}_i) = \frac{f d_K(\mathscr{Y}}{(f, d_K(\mathscr{Y}))}
$$
But we know that
$$
f = \left[ k_f : k \right] = \left[L_f : K \right] = \sum^{h}_{i = 1}
d_{L_f/_K}(\mathscr{K}_i) = h d_{L_f/_K}(\mathscr{K}_1), 
$$
and using the relation   
$$
d_{L/K} (\mathscr{K}_1) d (\mathscr{Y}) = d_L (\mathscr{K}) [l;k], 
$$
we\pageoriginale deduce the formula 
$$
h=(f, d_K(\mathscr{Y}))
$$

An immediate consequence of the above formula is the following
\begin{theorem*}
  For algebraic function fields with finite constant fields, the least
  positive value of the degree of its divisors is $1$. 
\end{theorem*}

\begin{proof}
  Let $\rho$ be this least value. Take $f = \rho$ in the above formula.
\end{proof}

We obtain, since $\rho$ divides each $d_K (\mathscr{Y})$,
\begin{align*}
  h & = \rho \\
  \text{ and also } \hspace{3cm} d_{L_f}(\mathscr{K}_i) &=
  \dfrac{d_K(\mathscr{Y})}{\rho}.\hspace{3cm} 
\end{align*}

Substituting in the Euler product for $\zeta (s, L_f)$, we obtain
\begin{align*}
  \zeta (s, L_f) = \prod _{\mathscr{K}} \left(1 - q^{-sf
    d_{L_f}(\mathscr{K})}\right)^{-1} &= \prod_\mathscr{Y} \prod^h_{i=1}
  \left(1-q^{-sd_K(\mathscr{Y})}\right)^{-1} \\ 
  & = (\zeta (s, K))^h = (\zeta (s, K))^\rho
\end{align*}

Since both $\zeta (s, L_f)$ and $\zeta(s, K)$ have a pole of order $1$
at $s = 1$, we deduce that $\rho = 1$. 

Finally, we prove a theorem expressing the zeta function of a finite
constant field extension in term of the L-series of the ground field. 

\begin{theorem*}
  With the same notation as above, we have
  $$
  \zeta (s, L_f) = \prod^f_{\nu =1} L (s, \chi_{f, \nu}, K)
  $$
  where\pageoriginale $\chi_{f, \nu}$ is the character on the class group taking the
  value $e^{\frac{2 \pi i \nu}{f}}$ on all classes of degree $1$. 
\end{theorem*}

\begin{proof}
  \begin{align*}
  \zeta (s, L_f) & = \prod_\mathscr{K}\left(1-N_{L_f} \mathscr{K}^{-s}\right)^{-1}
  = \prod_{\mathscr{Y}} \prod_{\mathscr{K}/
    \mathscr{Y}} \left(1-q^{-sfd_{L_f}(\mathscr{K})}\right)^{-1} \\
    &= \left\{ \prod_\mathscr{Y}(1-N_K \mathscr{Y}^{-s \frac{f}{(f,
        d_K(\mathscr{Y}))}}\right\}^{-(f, d_K(\mathscr{Y}))}\\ 
    &= \prod_\mathscr{Y} \prod^f_{\nu =1}\left(1 - e^{\frac{2 \pi i
        \nu}{f}d_K(\mathscr{Y})} N_K \mathscr{Y}^{-s}\right)^{-1} 
  \end{align*}
(the last follows from the easily established formula $
  \prod\limits^r_{\nu = 1}\left(1-e^{\frac{2 \pi i \nu}{r}s}z\right) =
  \left(1-z^{\frac{r}{(r, s)}}\right)^{(r, s)}$ for positive integral $r, s$) 
  $$
  = \prod^f_{\nu=1} \zeta \left(s-\frac{2 \pi i \nu}{f \log q}, K\right) =
  \prod^f_{\nu=1} L(s, \chi_{f, \nu}, K) 
  $$
  
  The proof of the theorem is complete.
\end{proof}

\begin{thebibliography}{99}
\bibitem {key1}{Artin, E.} - Algebraic numbers and algebraic
  functions. Princeton 1950-51. 
\bibitem {key2}{Artin, E.} - Quadratische korper im gebiet der hoheren
  kongruenzen. I and II. Math. Zeit. 19 (1924) p. 153-206,
  207-246. 
\bibitem {key3}{Chevalley, C.} - Introduction to the theory of algebraic
  functions of one variable. Mathematical Surveys. Number
  $VI$. Amer. Math. Soc. 1951. 
\bibitem {key4}{Dedekind, R. \& Weber, H.} - Theorie der algebraischen
  funktionen einer veranderlichen J. reine angew. math. 92(1882)
  p. 181-290. 
\bibitem {key5}{Hensel, K. \& Landsberg, G.} - Theorie der algebraischen
  funktionen einer variablen und ihre anwendung auf algebraische
  kurven und Abelsche Integrale. Leipzing. 1902. 
\bibitem {key6}{Kronecker, L.} - Grundzuge einer arithmetischen theorie
  der algebraischen grossen. Werke Band 2. 
\bibitem {key7}{Riemann, B.} - Theorie der Abelschen functionen. J. Reine
  angew. Math. 54 (1857). Werke Zweite Auflage Erste Abtheilung. $VI$
  p. 88-142. 
\bibitem {key8}{Roch, G.} - \"{U}ber die Anzahl der wilkurlicher
  Konstanten in algebraischen Funktionen. J. reine angew. Math. 64
  (1865) p. 372-376. 
\bibitem {key9}{Schmid, H. L. \& Teichmuller, O.} - Ein neuer bewies fur
  die Funktion algeichung der
  L-reihen. Abh. Math. sem. Hans. univer. 15 (1943) p. 85-96. 
\bibitem {key10}{Schmidt, F. K.} - Analytische zahlentheorie in korpern
  der charakteristik p. Math. Zeit. 33 (1931) p. 1-32. 
\bibitem {key11}{Schidt, F.K.} - zur arithmetischen theorie algebrische
  funktionen $I$. Math. Zeit. 41 (1936) p. 415-438. 
\bibitem {key12}{Tate, J.} - Genus change in inseparable extensions of
  function fields. Proc. Amer. Math. Soc. 3 (1952) p. 400-406. 
\bibitem {key13}{Weil, A.} - Zur algebraischen theorie der algebraischen
  funktionen. J. reine angew. Math. 179 (1938) p. 129-133. 
\bibitem {key14}{Weissinger, J.} - Theorie der divieoren
  Kongruenzen. Abh. Math. sem. Hans. Univer. 12 (1938) p. 115-126. 
\end{thebibliography}
