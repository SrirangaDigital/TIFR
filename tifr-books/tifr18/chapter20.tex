\chapter{Lecture 20}\label{chap20} % lec 20

\setcounter{section}{33}
\section{Divisors in an Extension}\label{chap20:sec34}%Sec 34

Let\pageoriginale $L /l$ be an arbitrary extension  of $K /k$. We wish to imbed the
group of divisors of $K$ in the group of divisors of $L$ in such a way
that the principal divisor $(X)_K$ in $K$ of an element $X$ of $K$
goes into the principal divisor of the element $X$ in $L$.  
This condition may also be rewritten in the form   
$$
\prod_{i=1}^{m} \mathscr{Y}_i^{v_{\mathscr{Y}_i}{(X)}} \longrightarrow
\prod_{i=1}^m \prod_{j=1}^{h_i} \mathscr{K}_{ij}^{e_{L/
  _K}(\mathscr{K}_{ij}) v_{\mathscr{Y}_i}(X)} 
$$
where $\mathscr{Y}_i (i=1, \ldots , m)$ are the prime divisors of $K$
occurring in $X$ and $\mathscr{K}_{ij} (j=1, \ldots , h_i)$ are all the
prime divisors of $L$ lying over $\mathscr{Y}_i$ . This motivates the
following 

\begin{defi*}%Defn
  If $\mathscr{U}= \prod\limits_{i=1}^m \mathscr{Y}_i^{
  v_{\mathscr{Y}_i} {(\mathscr{U})}}$ is any divisor of $K$, we shall
  identity it with the divisor $\prod\limits_{i=1}^m
  \prod\limits_{j=1}^{h_i} \mathscr{K}_{ij}^{e_{L/K}(\mathscr{K}_{ij})
  v_{\mathscr{Y}i}(\mathscr{U})}$ in $L$. 
\end{defi*}

It is easy to see that this accomplishes an isomorphic imbedding of
the group $v_K$ of divisors of $K$ in the group $v_L$ of
divisors of $L$ which  takes the principal divisor $(X)_K$ to the
principal  divisor $(X)_L$. Hence we also get a homomorphism of the
class group $\mathscr{K}_K$ of $K$ the class group $\mathscr{K}_L$ of
$L$. From  now on, we shall use the same for  a divisor or class of
$K$ or its image as a divisor or class of $L$. 

We\pageoriginale have the following  theorem comparing  the degree in $K$ of a
divisor of $K$ and its degree in $L$ 

\begin{theorem*}%Thm
  There exists a positive rational number $\lambda_{L /K}$ depending
  only on $L$ and $K$ such that for any divisor $\mathscr{U}$ of $K$, 
  $$
  d_L (\mathscr{U}) = d_K (\mathscr{U})/\lambda_{L /K}.
  $$
\end{theorem*}

\begin{proof}
  Obviously it suffices to prove that for any two prime divisors
  $\mathscr{Y}$ and $\mathscr{U}$ of $K$, we have  
  $$
  \frac{d_L{ (\mathscr{Y})}}{d_K(\mathscr{Y})} =
  \frac{d_L(\mathscr{U}}{d_K(\mathscr{U})} 
  $$
\end{proof}

Assume on the contrary that we have 
$$
\displaylines{\hfill 
  \frac{d_L{ (\mathscr{Y})}}{d_K(\mathscr{Y})} <
  \frac{d_L(\mathscr{U}}{d_K(\mathscr{U})}\hfill \cr
  \text{i.e.,}\hfill 
  \frac{d_L{ (\mathscr{Y})}}{d_L(\mathscr{U})}  <
  \frac{d_K(\mathscr{Y})}{d_K(\mathscr{U})} \hfill }
$$

Then there is a positive rational $m /n$ such that 
$$
\frac{d_L{ (\mathscr{Y})}}{d_L(\mathscr{U})}  < \frac{m}{n} <
\frac{d_K(\mathscr{Y})}{d_K(\mathscr{U})} 
$$

It follows that for sufficiently large integral $t$, we have 
$$
\displaylines{\hfill 
  d_K(\mathscr{Y}^{nt} \mathscr{U}^{-mt}) = t(nd_k (\mathscr{Y}) -
  md_K(\mathscr{U})) > 2g_K -1 \hfill \cr
  \text{and}\hfill  
  d_L(\mathscr{Y}^{nt} \mathscr{U}^{-mt}) = t(nd_L (\mathscr{Y}) -
  md_L(\mathscr{U})) < 0 \hfill }
$$\pageoriginale\

It follows from the first inequality  and the Riemann-Roch theorem
that there exists an element $X \neq 0$ in $K$ divisible by the
divisor $\mathscr{U}^{mt} \mathscr{Y}^{-nt}$. Hence,  
$$
\displaylines{\hfill 
  (X)_K = \mathscr{U} \mathscr{U}^{mt} \mathscr{Y}^{-nt}, \mathscr{U}
  \text{ integral }\hfill \cr 
  \text{and}\hfill  
  d_L ((X)) = d_L (\mathscr{U}) - t(nd_L(\mathscr{Y}) - md_L
  (\mathscr{U})) > 0\hfill }
$$
by the second inequality. But this is impossible and our theorem is
proved.  

If $[L : K]< \infty$, the value of $\lambda_{L /K}$ is $\dfrac{[l :
    k]}{L : K}$. For, if $\mathscr{Y}$ be any prime divisor of $K$ and
$\mathscr{K}_i (i =1, \ldots, h)$ be the prime divisors of $L$ lying
over $\mathscr{Y}$, we have  
\begin{multline*}
  d_L(\mathscr{Y}) = \sum_{i=1}^h d_L (\mathscr{K}_i) e_{L
    /K}(\mathscr{K}_i)\\
  = \frac{1}{[l : k]} \sum_{i=1}^h d_{L /
    K}(\mathscr{K}_i) d_K (\mathscr{Y}) e_{L/K}(\mathscr{K}_i) =
  \frac{[L : K]}{[l : k]}d_L (\mathscr{Y}). 
\end{multline*}

Now let $L/l$ be an extension of $K/ k$ of finite degree and $L_1$
the smallest normal extension of $K$ containing $L$. Let $l_1$ be the
algebraic closure of $l$ in $L_1$. Clearly  $L_1 / l_1$ is an
algebraic function field with constant field $l_1$. Let $G= G(L_1/K)$
be the Galois group\pageoriginale of $L_1$ over $K$ and $H$ be the subgroup of all
automorphisms of $L_1$ fixing every element of $L$. Let $G/H$ be the
set of left cosets of $G$ modulo $H$. If $Y$ is an element of $L$, it
is well-known that the norm of $Y$ over $K$ is given by 
$$
N_{L / _K}(Y) = \left[ \prod_{\bar{\sigma} \in G/ H} \sigma Y
  \right]^{[L : K]_i}, 
$$
the product being over any set of representatives of the cosets in $G
/H$. This suggests the following definition for the norm of a divisor
of $L$. (As already explained, we shall use the same symbol for a
divisor of $L$ and its canonical image as a divisor of $L_1$).  If
$\mathscr{U}$ is a divisor of $L$, we put  
$$
\text{Norm}_{L /_K} \mathscr{U} =Nm_{L / _K}\mathscr{U} =
\left\{\prod_{\bar{\sigma \in G/ H}} \sigma \mathscr{U} \right\}^{[L :K]_i} 
$$

The definition is independent of the choice of the representative
$\sigma$ of $\bar{\sigma}$, since $\sigma \mathscr{U}= \mathscr{U}$ if
$\sigma \in H$. (More generally, if  $L_{1/ l_1}$ and $L_2 /_{l_2}$
are two extension of an algebraic function field and $\sigma$ an
isomorphism of $L_1$ onto $L_2$ mapping $l_1$ onto $l_2$ and fixing
every element of $K, \sigma$ maps every divisor $\mathscr{U}$ of $K$
considered as a divisor of $L_1$ onto $\mathscr{U}$ considered as a
divisor of $L_2$). 

We list below the essential properties of the norm 
\begin{enumerate}
\item The norm of a divisor of $L$ is a divisor of $K$. Hence the norm
  mapping is a homomorphism of $\vartheta_L$ into $\vartheta_K$  
\item If $\mathscr{K}$ is a prime divisor of $L$ lying over the prime
  divisor $\mathscr{Y}$ of $K$, we have  
  $$
  Nm_{L /K}\mathscr{K} = \mathscr{Y}^{d_{L/ K ^{(\mathscr{K})}}}
  $$\pageoriginale\
\item If $y \in L$,
  $$
  Nm_{L/ K}(y)_L = (Nm_{L /K} y)_K
  $$
\item If $\mathscr{U} \in \vartheta_K$,
  $$
  Nm_{L /K}\mathscr{U}= \mathscr{U}^{[L : K]}
  $$
\item If $L_1 \supset L \supset K$ is a tower of extensions of
  algebraic function  fields, and $\mathscr{U} \in   v_{L_1}$, 
  $$
  Nm_{L_1 /K}\mathscr{U} =Nm_{L /K}(Nm_{L_1 / L}\mathscr{U})
  $$
\end{enumerate}

\begin{proof}
  (3) and (4) are immediate consequences of the definition of the
  norm. It is also clear that the norm defines a homomorphism of
  $\vartheta_L$ into itself. We have only to prove that the image is
  contained in $\vartheta_K$ to complete the demonstration of
  (1). But this will follow if we can prove (2). 
\end{proof}

Again, it is enough to prove $(5)$ for prime divisors $\mathscr{K}_1$
of $L_1$ because of $(1)$. But using $(2), (5)$ reduces to the already
proved equality 
$$
d_{L_1 /K}(\mathscr{K}_1) = d_{L _1/ L}(\mathscr{K}_1) d_{L/ K}(\mathscr{K})
$$
for a prime divisor $\mathscr{K}_1$ of $L_1$ which lies over
$\mathscr{K}$ of $L$. 

It only remains to prove $(2)$. Let again $G$ be the Galois group of
the smallest normal extension $L_1$ of $K$ containing $L$. 

Then\pageoriginale $L_1$ is also normal over $L$ and its Galois group over $L$ is
the subgroup $H$ of $G$ of all automorphisms which fix every element
of $L$. Let $\mathscr{K}_1$ be any prime divisor of $L_1$ lying over
the prime divisor $\mathscr{K}$ of $L$. We shall denote by $Z_K$ and
$Z_L$ the decomposition group of $\mathscr{K}_1$ over $K$ and $L$
respectively. The, since $L_1$ is normal over $L$, we have  
$$
  \mathscr{K}= \left\{ \prod_{\bar{\sigma }\in H/ Z_L} \sigma
  \mathscr{K}_1\right\}^{e_{L_1 /L}(\mathscr{K}_1)},
$$
and therefore
\begin{align*} 
  & (N_{L /K}\mathscr{K})^{[Z_L : (e)]}= \left[ \prod_{\bar{\tau} \in G/
      H} \tau \left\{ \prod_{\bar{\sigma} \in  H/Z_L}\sigma
    \mathscr{K}_1\right\}^{[Z_L : (e)]}\right]^{e_{L_1/L} (\mathscr{K_1}) [L
      : K]_i} \\
  &= \left[ \prod_{\tau \in G/ H} \tau \left\{\prod_{\sigma
      \in H} \sigma \mathscr{K}_1 \right \} \right] ^{e_{L_1/
      _L}(\mathscr{K}_1) [L : K]_i} = \left[ \prod_{\tau \in G}
    \tau \mathscr{K}_1 \right]^{{e_{L_1/ _L}(\mathscr{K}_1) [L :
        K]_i}}\\ 
  &= \left\{ \prod_{ \bar{\tau}\in G/ Z_K} \tau
  \mathscr{K}_1\right\}^{{e_{L_1/ _L}(\mathscr{K}_1) [L :
        K]_i[Z_K:(e)]}}\\ 
  & = \mathscr{Y}^{\frac{{{e_{L_1/
          _L}(\mathscr{K}_1) [L : K]_i[Z_K:(e)]}}}{e_{L_1
      /K}(\mathscr{K}_1)}} =\mathscr{Y}^{[Z_L :(e)] d_{L/
      K}(\mathscr{K})}, 
\end{align*}
since
\begin{gather*}
\frac{{{e_{L_1/ _L}(\mathscr{K}_1) [L : K]_i[Z_K:(e)]}}}{e_{L_1
    /K}(\mathscr{K}_1)} = \frac{e_{L / L}(\mathscr{K}_1)}{[L_1, :
    L]_i}. \frac{[L_1 : K]_i [Z_K : (e)]}{e_{L_1 /K}(\mathscr{K}_1)} \\
    =\dfrac{[Z_L :(e)]}{d_{L_1 /L(\mathscr{K}_1)}}. d_{L_1
     /K}(\mathscr{K}_1) = [Z_L : (e)] d_{L / K}(\mathscr{K}) 
\end{gather*}\pageoriginale\

Since the group of divisors is free, it is  also torsion  free and our
formula follows. 

Finally, for any divisor $\mathscr{U}$ of $L$, we have 
$$
d_K (N_{L /_K} \mathscr{U}) = [l : k] d_L(\mathscr{U}).
$$

It is enough to prove this for a prime divisor $\mathscr{K}$. But by
the above result, we have 
$$
d_K (N_{L /_K}\mathscr{K})= d_K (\mathscr{Y}^{d_{L / _K
    (\mathscr{K})}})= d_{L /K}(\mathscr{K}) d_K(\mathscr{Y}) = d_L
(\mathscr{K}) [l : k] 
$$
which is the result we want.

\section{Ramification}\label{chap20:sec35} % \sec 35

We wish to prove two theorems on ramification. The first one is easy.

\begin{theorem*}%Thm
  If $L /l$ is an algebraic extension of $K /k$ such that $L$ is
  purely inseparable over $K$, there is exactly one prime divisor
  $\mathscr{K}$ of $L$ lying over a given prime divisor $\mathscr{Y}$
  of $K$ and $\mathscr{Y}=\mathscr{Y}^{pt}$ where $p$ is the
  characteristic of $K$ and $t$ a non-negative integer. 
\end{theorem*}

\begin{proof}
  Let\pageoriginale $Y$ be any element of $L$. Since $L$ is purely inseparable over
  $K$, there exists an integer $n$ such that $Y_0= Y^{p^n} \in
  K$. Then, if $\mathscr{K}$ be any prime divisor lying over
  $\mathscr{Y}$, we have 
  $$
  p^n v_{\mathscr{K}}(Y) = v_{\mathscr{K}}(Y_o) = e_{L
    /K}(\mathscr{K}) v_{\mathscr{Y}}(Y_o) 
  $$
  and therefore the value of $v_{\mathscr{K}}(Y)$ is uniquely
  determined by $v_\mathscr{Y}(Y_0)$.  
  Hence $\mathscr{K}$ is unique.
\end{proof}

If we choose $Y$ such that $v_\mathscr{K}(Y)=1$, we deduce that $e_{L
  /K}(\mathscr{K})$ divides $p^n$, and our theorem is proved. 

We say that a prime divisor $\mathscr{K}$ of an extension $L$ of an
algebraic function field $K$ is \textit{ramified} if $e_{L /
  K}(\mathscr{K}) > 1$. We have the following 

\begin{theorem*}%Thm
  If $L$ is separably algebraic over $K$, there are at most a finite
  number of prime divisors of $L$ which are either ramified or
  inseparable over $K$. 
\end{theorem*}

\begin{proof}
  We give the proof in three steps. We first prove the theorem for
  finite normal extension, then for finite separable extension, and
  finally in the general case. 
\end{proof}

First assume that $L$ is finite and normal over $K$. A prime divisor
$\mathscr{K}$ of $L$ is either ramified or inseparable only if
$[T(\mathscr{K}):(e)]= e_{L/K}(\mathscr{K}) d_{L /K}(\mathscr{K})_i
>1$. 

This implies that there is an automorphism $\sigma$ of $L$ in the
group $T(\mathscr{K})$ which is not the identity automorphism. Since
$L$ is finite and separable over $K$ it is a simple extension $K(Z)$
of $K$. Hence $\sigma Z \neq Z$. 

Al\pageoriginale least one of the elements $Z, \dfrac{1}{Z}$ lies in
$\mathscr{O}_\mathscr{K}$, and since $\sigma \in T(\mathscr{K})$ we
deduce that one of the two inequalities 
$$
v_\mathscr{K}(Z- \sigma Z) > 0 \text{ or } v_\mathscr{K} \left(\frac{1}{Z}-
\frac{1}{\sigma Z}\right) > 0 
$$
should hold. Since there are only finitely many automorphisms $\sigma$
of $L$  over $K$ and only finitely many prime divisors $\mathscr{K}$
for which one of the two inequalities above can be valid, the theorem
is proved in this case. 

If $L/K$ were finite and separable but not normal, let $L_1$  be the
smallest normal extension of $K$ containing $L$. If a prime divisor
$\mathscr{K}$ of  $L$ is ramified or inseparable over $K$, the same
property should also hold for any prime divisor of $L_1$ lying over
$\mathscr{K}$. Hence the theorem in this case follows from the first
part. 

Finally, suppose $L/K$ is any separably algebraic extension It $l$
is the constant field of $L$, $L$ is evidently a finite separable
extension of the composite extension $Kl$. Also, a prime divisor of
$L$ in ramified (inseparable) over $K$ if and only  if it is either
ramified (inseparable ) over $Kl$ or the prime divisor of $Kl$ over
which it lies is ramified (inseparable) over $K$. Our theorem follows
from what we have proved above and the following lemma. 

\begin{lemma*}%Lem
  If $L/l$ is an algebraic function field which is separably
  algebraic over $K/k$ and the that $L =Kl$, there are no prime
  divisors of $L$ ramified or separable over $K$. 
\end{lemma*}

\begin{proof}
  If $\mathscr{K}$ is a prime divisor of $L$ which is ramified
  (inseparable) over $K$, find an element $Y \in Kl$ such that $V_
  \mathscr{K}(Y)=1(\bar{Y} \in L_{\mathscr{K}}$ is inseparable\pageoriginale over
  $K_\mathscr{Y})$. Since $Y$ is a rational combination of a finite
  number of elements of $K$ and $l$, $Y$ lies in a finite extension
  $K(\alpha_1, \ldots \alpha_n)$ of $K$, where $\alpha_i$ are elements
  of $l$. We may also assume that $L_1 = K (\alpha_1, \ldots
  \alpha_n)$ is a normal separable extension of $K$; for it is already
  separable, and we have only to adjoin to it the conjugates of the
  $\alpha_i$ (which are finite in number) to make it normal. Also by
  our choice of $Y$, we see that the prime divisor $\mathscr{K}_1$ of
  $L_1$ over which $\mathscr{K}$ lies is ramified (inseparable) over
  $K$. 
\end{proof}

Let $T(\mathscr{K}_1)$ be the inertia group of $\mathscr{K}_1$ in
$L_1$ over $K$. Then, we should have 
$$
[ T(\mathscr{K}_1 ) : (e) ] = e_{L_1/_K} (\mathscr{K}_1) d_{L_1 /_K}
(\mathscr{K}_1)_i > 1, 
$$
and  there exists an element $\sigma \in T (\mathscr{K}_1)$ which is
not the identity. But since $\sigma \in T (\mathscr{K}_1)$, 
$$
v_{\mathscr{K}_1} (\alpha_\nu - \sigma \alpha_\nu) > 0 \quad (\nu = 1,
2, \ldots n) 
$$
and the $\alpha_\nu$ being constants, we should have
$$
\alpha_\nu = \sigma \alpha_\nu
$$

Hence $\sigma$ is the identity automorphism when restricted to $K$ and
fixes each one of elements  $\alpha_1, \ldots, \alpha_n$. Hence
$\sigma$ should 
be the identity automorphism of $L_1 = K (\alpha_1, \ldots
\alpha_n)$. This is a contradiction and our theorem is proved. 
