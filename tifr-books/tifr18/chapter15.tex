\chapter{Lecture 15}\label{chap15}%cha 15

\setcounter{section}{27}
\section{Classes Modulo \texorpdfstring{$\mathcal{F}$}{F}}\label{chap14:sec28}%sec 28

Let\pageoriginale $K$ be an algebraic  function field over an arbitrary constant
field $k$ and $ \mathcal{F} $ an integral divisor of $ K$. We shall
denote the distinct prime divisors of $ \mathcal{F} $ by $
\mathscr{U}_1, \ldots , \mathscr{U}_r $. We now define some groups
associated to the integral divisor $ \mathcal{F} $. In all the cases,
it is an easy matter to verify that the sets defined  are closed under
taking products or inverses. 

$ \vartheta^{\mathcal{F}} $ will be the group of divisors coprime  to
$ \mathcal{F} $, and $ \vartheta^{\mathcal{F}}_0 $ the subgroup of all
elements of $ \vartheta^\mathcal{F} $ whose degree is zero. $ K^{*
  \mathcal{F}} $ is  the multiplicative group of elements of $K^*$
which are coprime to $ \mathcal{F} $.  $ E^\mathcal{F} $ is  the
group of principal divisors coprime to $ \mathcal{F} $. 
 
 Clearly, $ k^* \subset K^{* \mathcal{F}} $, and since two elements of
 $ K^{* \mathcal{F}} $ define the same divisors  if and only if their
 ratio is a constant, we have the isomorphism  
 $$
 E^\mathcal{F} \simeq ~ K^{* \mathcal{F}} /_{k^*}
 $$
 
 Moreover, since every class contains a divisor coprime to $
 \mathcal{F} $, the saturation of  $ \vartheta^\mathcal{F} $ by $E$ is
 the whole of $ \vartheta $, and  consequently  
 $$
 \vartheta^\mathcal{F}/_{E^\mathcal{F}} =
 \frac{\vartheta^\mathcal{F}}{E \cap \vartheta^\mathcal{F}} \simeq
 \frac{\vartheta^\mathcal{F} E}{E} = \frac{\vartheta}{E} = \mathfrak{R}, 
 $$
 and similarly $ \vartheta^\mathcal{F}_o /_{ E^\mathcal{F}} \simeq \vartheta_0 $.
 
 We\pageoriginale shall say that  $ X \equiv Y ( \mod^+ \mathcal{F} ) $ if $
 v_{\mathscr{U}_{i}} (X -Y ) \ge v_{\mathscr{U}_{i}} (\mathcal{F})~ (
 i= 1,~ r) $. The condition is evidently equivalent to saying that $ X
 - Y \in \Gamma ( \mathcal{F}/\mathscr{U}_{1,} \mathscr{U}_{r} )
 $. Let $ K^*_\mathcal{F} $ be the set of elements $X$ of $K^*$ which
 satisfy $ X \equiv 1 ( \mod^+ \mathcal{F} ) $. If $ X \in
 K^*_\mathcal{F} $, it follows that $ v_{\mathscr{U}_{i}} (X) = \min (
 v_{\mathscr{U}_{i}} (X-1) $, $ v_{\mathscr{U}_{i}} ( 1 )) = 0
 $. Hence, if $ X $, $ Y \in K^* $, we have $ XY \equiv Y \equiv 1 (
 \mod^+ \mathcal{F} ) $ and  therefore   $ XY \in K^*_\mathcal{F}
 $. Similarly, $ \dfrac{1}{X} $ is also in $ K^*_\mathcal{F} $. Since $
 X \in K^*_\mathcal{F} \Rightarrow_{\mathscr{U} i} (X) = 0 $, we
 deduce that $ K^*_\mathcal{F} $. $ E_\mathcal{F} $ shall denote the
 group of principal divisors of $K^*_{\mathcal{F}} \subset
 K^{*\mathcal{F}}$. $E_{\mathcal{F}}$ elements of $ K^*_\mathcal{F} $. It
 follows from the above inclusion that  $ E_\mathcal{F} \subset
 E^\mathcal{F} $. $ E_\mathcal{F} $ is  called \textit{ the ray modulo
   $ \mathcal{F} $.} 

If $ \mathcal{F} \neq N $, it is easy to see that  $ k^* \cap
K^*_\mathcal{F} = \big\{ 1 \big \} $. Hence, we deduce, in this case,
that $ E_\mathcal{F} = \dfrac{K^*_{\mathcal{F}} k^*}{k*} \simeq
\dfrac{K*}{k^*_\mathcal{F} \cap k^*} \simeq K^*_\mathcal{F} $. 

The \textit{class group modulo} $ \mathcal{F} $ is by definition the
quotient group $ \mathfrak{R}_{\mathcal{F}} = \vartheta^{\mathcal{F}}
  /_{E_{\mathcal{F}}} $  and its elements are called  \textit{
  classes modulo $\mathcal{F}$.} The subgroup $
\mathfrak{R}_{0_{\mathcal{F}}} =
\vartheta^\mathcal{F}_{0}/_{E_{\mathcal{F}}} $ of $
\mathfrak{R}_\mathcal{F} $ is called the \textit{group of classes
  modulo $\mathcal{F}$ of degree zero} and its order $h_\mathcal{F}$
is called the \textit{ class number modulo $ \mathcal{F} $.} In the
case of a finite field $ k,h_\mathcal{F} $ can be expressed  in terms
of the class number $h$. We have the following  theorem. 

\begin{theorem*}
  Let the constant field $k$ be finite with $q$ elements and let $
  \mathcal{F} $ be an integral divisor different from $N$. Then, 
  $$
  h_\mathcal{F} = \frac{h N \mathcal{F}}{q-1} \prod^{r}_{\nu = 1} \left( 1
  - \frac{1}{N\mathscr{U}_\nu}\right)  
  $$
\end{theorem*} 
 
\begin{proof}
  From the isomorphisms 
  $$
  \mathfrak{R} 0 \simeq \vartheta^\mathcal{F}_0 / E^\mathcal{F} \simeq
  \frac{\vartheta^\mathcal{F}_0 / E_\mathcal{F}}{E^\mathcal{F}/
    E_\mathcal{F}}, 
  $$
  it\pageoriginale follows that  
  $$
  h = \frac{h_\mathcal{F}}{\left(\frac{E^\mathcal{F}}{E_\mathcal{F}}:1 \right)}
  $$
  Again, 
  $$
  \displaylines{\hfill 
  \frac{E^\mathcal{F}}{E_\mathcal{F}} \simeq \frac{K^{*
      \mathcal{F}}/k^*}{K^*_\mathcal{F} k^* / k^*} \simeq \frac{K^{*
      \mathcal{F}}}{K^*_\mathcal{F} k^*} \simeq \frac{K^{* \mathcal{F}}
      /K^*_{\mathcal{F}}}{K^*_\mathcal{F} k^*/K^*_\mathcal{F}}, \hfill\cr
  \text{and}\hfill 
  \frac{K^*_{\mathcal{F}} k^*}{k^*_{\mathcal{F}}} \simeq \frac{k^*}{k^* \cap
    K^*_\mathcal{F}}\simeq k^*,\hfill }
  $$
  since $ \mathcal{F} \neq N $ and therefore $ k^*  \cap
  K^*_\mathcal{F} = \{ 1 \} $. Hence, we obtain  
  $$
  h_{\mathcal{F}} = \frac{h}{q-1} \left[ K^{* \mathcal{F}} :
    K^*{_{\mathcal{F}}} \right].
  $$
\end{proof} 
 
Now , $ K^{* \mathcal{F}} $ is  precisely the set of elements of $
\Gamma ( N / \mathscr{U}_1 , \ldots \mathscr{U}_r ) $ which do not lie
in any of the spaces $ \Gamma ( \mathscr{U}_i / \mathscr{U}_1 , \ldots
\mathscr{U}_r ) $. Also, if $ X, Y \in K^{* \mathcal{F}} $, they
belong to the same coset modulo $ K^* _\mathcal{F} $ if and only if $
XY^{-1} \equiv 1$ $\pmod {^+ \mathcal{F}} \Longleftrightarrow $ $ X \equiv
Y \pmod {^+ \mathcal{F}}$ ( since $Y$ is coprime to $ \mathcal{F} )$ $
\Longleftrightarrow  X- Y \in  \Gamma ( \mathcal{F} / \mathscr{U}_1
\cdots \mathscr{U}_r ) $. We deduce from these facts that  (with  $ S
= ( \mathscr{U}_1, \ldots \mathscr{U}_r ))$ 
\begin{align*}
  \left[ K^{* \mathcal{F}} : K^*_\mathcal{F} \right] & = q^{\dim 
  \frac{\Gamma (n/s)}{\Gamma( \mathcal{F} /s)}} - \sum_{i} q^{\dim
  \frac{\Gamma( \mathscr{U}_i /s}{\Gamma( \mathcal{F}/s)}} + \sum_{i
    \neq j} q^{\dim \frac{ \Gamma ( \mathscr{U}_i \mathscr{U}_j /s
    )}{\Gamma ( \mathcal{F} /s )} \cdots}\\ 
  & = q^{d (\mathcal{F})} - \sum_{i} q^{d (\mathcal{F}) -d
    (\mathscr{U}_i)} + \sum_{i \neq j} q^{d (\mathcal{F} ) -d
    (\mathscr{U}_i) -d (\mathscr{U}_j)}\cdots \\ 
  & = N ( \mathcal{F}) \prod^{r}_{\nu = 1} \left( 1 - \frac{1}{N
    \mathscr{U}_\nu}\right) 
\end{align*}

 Substituting\pageoriginale this expression in the value of $ h_\mathcal{F} $, we
 get the required result. 
 
 Note that the theorem is not valid when $ \mathcal{F} = N$. In fact,
 we have $ \mathfrak{R}_{0_N} = \mathfrak{R}_0$ and $h_N = h $.  
 
 We shall end this lecture with a simple lemma asserting the existence
 of sufficiently many  divisors in any class modulo $ \mathcal{F} $. 
 
\begin{lemma*}
  If $ \mathscr{U} $ is any given divisor, any class $ C_\mathcal{F} $
  modulo $ \mathcal{F}$ contains a divisor $ \delta$ prime to $
  \mathscr{U} $. 
\end{lemma*}  

\begin{proof}
  Let $\mathscr{U}_0 $ be any divisor of  $ C_\mathcal{F} $. Find a $
  Y \in K $ such that  
  $$
  \displaylines{\hfill 
  v_{\mathscr{U}_{\nu}} ( Y - 1 ) \ge v_{\mathscr{U}_{\nu}} (
  \mathcal{F}) ~( \nu = 1, \ldots ,r ) \hfill \cr
  \text{and}\hfill  
  v_{\mathscr{U}} (Y)  = -v_\mathscr{U} (\mathscr{Y}_0 ) \text{ if
  }v_\mathscr{Y} (\mathscr{U}) \neq 0 \text{ and } \mathscr{Y} \neq
  \mathscr{U}_\nu  \hfill }
  $$
  Then it is easy to verify that $\delta = \mathscr{U}_0
  (Y) $ satisfies the conditions of the lemma. 
\end{proof}
