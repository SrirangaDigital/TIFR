\thispagestyle{empty}

\begin{center}
{\Large\bf  Lectures on}\\[5pt]
{\Large\bf Mean Values of The Riemann Zeta Function}\\[50pt]
\vfill

{\bf By}\\[10pt]

{\large\bf A.Ivic}
\vfill

{\bf Tata Institute of Fundamental Research}

{\bf Bombay}

{\bf 1991}
\end{center}

\eject

\thispagestyle{empty}
\begin{center}
{\Large\bf  Lectures on}\\[5pt]
{\Large\bf Mean Values of The Riemann Zeta Function}\\[50pt]
\vfill

{\bf By}\\[10pt]

{\large\bf A. Ivic}
\vfill



Published for the 

{\bf Tata Institute of Fundamental Research}

SPRINGER-VERLAG

Berlin Heidelberg New York Tokyo
\end{center}

\eject

\thispagestyle{empty}

\begin{center}
Author\\
{\large\bf A. Ivic}\\
S.Kovacevica, 40 Stan 41\\
Yu-11000, Beograd\\
YUGOSLAVIA
\vfill

{\bf \copyright \quad Tata Institute of Fundamental Research, 1991}\\[10pt]

ISBN 3-350-54748-7-Springer-Verlag, Berlin. Heidelberg.\\ 
New York. Tokyo
\smallskip

ISBN 0-387-54748-7-Springer-Verlag, New York, Heidelberg.\\ 
Berlin. Tokyo
\vfill


\parbox{0.7\textwidth}{No part of this book may e reproduced in any form by print, microfilm
or any other mans without written permission from the Tata Institute
of Fundamental Research, Colaba, Bombay 400 005}
\vfill

Printed by Anamika Trading Company 

Navneet Bhavan, Bhavani Shankar Road

Dadar, Bombay 400 028\\ 

and\\ 

Published

by H.Goetze, Springer-Verlag,

Heidelberg, Germany

PRINTED IN INDIA
\end{center}


\chapter{Preface}

These lectures were given at the Tata Institute of Fundamental Research
in the summer of 1990. The specialized topic of mean values of the
Riemann zeta-function permitted me to go into considerable depth. The
central theme were the second and the fourth moment on the critical
line, and recent results concerning these topic are extensively
treated. In a sense this work is a continuation of my monograph
\cite{Anderson1}, since except for the introductory Chapter \ref{c1}, it starts where
\cite{Anderson1} ends. Most of the results in this text are unconditional, that
is, they do not depend on unproved hypothesis like Riemann's (all
complex zeros of $\zeta(s)$ have real parts equal to $\frac{1}{2}$) or
Lindel\"of's $(\zeta (\frac{1}{2} + \text{it})~ \ll t^\epsilon)$. On
the other hand, in many problems concerning mean values one does not
obtain the conjectured results even if one assumes some unproved
hypothesis. For example, it does not seem possible to prove $E(T) \ll
T^{\frac{1}{4}+\epsilon}$ even if one assumes the Riemann
hypothesis. Incidentally, at the moment of writing of this text, it is
not yet known whether the Riemann or Lindelof hypothesis is true or
not.

Each chapter is followed by Notes, where some results not treated in
the body of the text are mentioned, appropriate references: are given
etc. Whenever possible, standard notation (explained at the beginning
of the text) is used. 

I've had the pleasure to have in my audience well-known specialists
from analytic number theory such as R. Balasubramanian,
K. Ramachandra, A. Sankaranarayanan, T.N. Shorey and S. Srinivasan. I
am grateful to all of them for their interest, remarks and stimulating
discussions. The pleasant surroundings of the Institute and the
hospitality of the Tata people made my stay a happy one, inspite of
heavy monsoon rains.

I wish to thank also M.N. Huxley, M. Jutila, K. Matsumoto and
T. Meurman for valuable remarks concerning the present lecture notes.

Special thanks are due to K. Ramachandra, Who kindly invited me to the
Tata Institute, and with whom I've been in close contact for many
years.

Finally, most of all I wish to thank Y. Motohashi, whose great work on
the fourth moment influenced me a lot, and who kindly permitted me to
include his unpublished material in my text.

Belgrade, March 1991.

\chapter{Notation}

Owing to the nature of this text, absolute consistency in notation
could not be attained, although whenever possible standard notation is
used. The following notation will occur repeatedly in the sequel.

\medskip
\noindent
\begin{longtable}{p{3cm}p{7cm}}
$k$, $l$, $m$, $n$ & Natural numbers (positive integers).\\
$A, B, C, C_1, \ldots$ & Absolute positive constants (not necessarily
  the same at each occurrence).\\
$s, z, w$ & Complex variables $\re s$ and $\im s$ denote the real
  imaginary part of $s$, respectively; common notation is
  $\sigma=\res$ and $t= \ims$.\\
$t, x, y$ & Real variables.\\
$\displaystyle{\mathop{\res}_{s= s_\circ}}~ F(s)$ & The residue of
  $F(s)$ at the point $s= s_\circ$.\\
$\zeta (s)$ & Riemann's zeta-function defined by $\zeta(s)=
  \displaystyle{\sum^\infty_{n=1}} n^{-s}$ for $\res > 1$ and for other
  values of $s$ by analytic continuation.\\
  $\gamma (s)$ & The gamma-function, defined for $\res > 0$ by
  $\Gamma(s) = \displaystyle{\int^\infty_0} t^{s-1} e^{-t} dt$,
  otherwise by analytic continuation.\\
  $\gamma $ & Euler's constant, defined by $\displaystyle{r =
    \int^\infty_0 e^{-x} \log x\, dx = 0.5772156649}\ldots$. \\
  $\chi (s)$ & $= \zeta(s) /\zeta(1-s)= (2 \pi)^s/(2 \Gamma (s)\cos
  \left(\frac{\pi s}{2} \right))$.\\
  $\mu(\sigma)$ & $\displaystyle{= \mathop{\lim\sup}_{t \to \infty}
    \frac{\log \big | \zeta (\sigma + \text{it})\big|}{\log t}}
    (\sigma$ real).\\
  $\exp z$ & $= e^z$.\\
  $e (z)$ & $= e^{2 \pi iz}$.\\
  $\log x$ & $= \text{Log}_e x (= \text{ln} x)$.\\
  $[x]$ & Greatest integer not exceeding $x$.\\
  $d_k (n)$ & Number of ways $n$ can be written as a product of $k$
  fixed natural numbers.\\
  $d(n)$ & $= d_2 (n) =$ number of divisors of $n$.\\
  $\displaystyle{\sum_{d \big| n}}$ & A sum over all positive divisors
  of $n$.\\
  $\sigma_a (n)$ & $= \displaystyle{\sum_{d \big | n} d^a}$.\\
  $\displaystyle{\sum_{n \leq k} f(n)}$ & A sum taken over all natural
  numbers not exceeding $x$; the empty sum is defined to be zero.\\
  $\displaystyle{\sideset{}{'}\sum_{n \leq x}} f(n)$ & Same as above,
  only $'$ denotes that the last summand is halved if $x^*$ is an
  integer.\\
  $\displaystyle{\prod_{j}}$ & A product taken over all possible
  values of the index $j$; the empty sum is defined to be unity.\\
  $I_k(T)$ & $= \displaystyle{\int^T_0 \Big| \zeta (\frac{1}{2}+
    \text{it})\Big|^{2k} dt} (k \geq 0)$.\\
  $E_k (T)$ & The error term in the asymptotic formula when $k \geq 1$
  is an integer.\\
  $E(T) (= E_1(T))$ & $\displaystyle{\int^T_0 \Big| \zeta(\frac{1}{2}
    + \text{it})\Big|^2} dt - T \log \left( \frac{T}{2 \pi}\right)- (2
  \gamma -1) T$.\\
  $E_\sigma (T)$ & $\displaystyle{\int^T_{0} \Big| \zeta(\sigma +
    it)\Big|^2 dt- \zeta (2 \sigma) T}$\\
  & $- \frac{\zeta (2 \sigma -1)\Gamma (2 \gamma -1)}{1- \sigma} \sin
  (\pi \sigma) T^{2- 2 \sigma} \left(\frac{1}{2} < \sigma < 1
  \right)$.\\
  ar $ \sinh z$ & $= \log (z+ (z^2 + 1)^{\frac{1}{2}})$.\\
  $\triangle (x)$ & $= \displaystyle{\sideset{}{'}\sum_{n \leq x}}
  d(n) - x(\log x + 2 \gamma -1)- \frac{1}{4}$.\\
  $\triangle_k (x)$ & Error term in the asymptotic formula for
  $\displaystyle{\sum_{n \geq x} d_k (n)}$; $\triangle_2 (x) =
  \triangle (x)$.\\
  $J_p (z)$, $K_p (z)$, $Y_p (z)$ & Bessel functions of index $p$.\\
  $G(T)$ & $ = \displaystyle{\int^T_{2}} (E(T)- \pi dt)$.\\
  $G_\circ (T)$ & $ = \displaystyle{\int^T_{2} (E_\sigma (t)- b (\sigma))
  dt \;  \left( \frac{1}{2} < \sigma < \frac{3}{4}\right)}$.\\
  $B(\sigma)$ & The constant defined by \eqref{c3:eq3.3}.\\
  $S(m, n; c)$ & Kloosterman sum, defined as   $\displaystyle{\sum_{1\leq d \leq c , (d,c)=1, dd'}} \equiv \pmod{c} e\left(\frac{md
      + nd'}{c}\right)$.\\
  $c_r (n)$ & Ramanujan sum, defined as $c_r (n)=
  \displaystyle{\sum_{1 \leq h \leq r, (h, r)=1}} e \left(\frac{h}{r}
  n \right)$. \\
  $\alpha_j$, $x_j$, $H_j \left( \frac{1}{2}\right)$ & Quantities
  appearing in the spectral theory of the non-Euclidean Laplace
  operator, defined in Section \ref{c5:sec5.3}.\\
  $f(x) \sim g(x)$ as\break  $x \to x_0$ & Means
  $\displaystyle{\lim\limits_{x \to x_0} \frac{f(x)}{g(x)}=1}$, with
    $x_0$ possibly infinite.\\
  $f(x)= O (g(x))$ & Means $| f(x)| \leq Cg (x)$ for $g(x)> 0$, $x
    \geq x_0$ and some absolute constant $C > 0$.\\
  $f(x) \ll g(x)$ & Means the same as $f(x)= 0(g(x))$.\\
    $f(x) \gg g(x)$ & Means the same as $g(x)= 0(f(x))$.\\  
    $f(x) \asymp g(x)$ & Means that both $f(x) \ll (g(x))$ and $g(x)
    \ll f(x)$ hold.\\
    $(a, b)$ & Means the interval $a < x < b$.\\
    $[a, b]$ & Means the integral $a \leq x \leq b$.\\
    $\delta $, $\epsilon$ & An arbitrarily small number, not
    necessarily the same at each occurrence in the proof of a theorem
    or lemma.\\
    $C^r [a, b]$ & The class of functions having a continuous $r$-th
    derivative in $[a, b]$.\\
    $f(x) = O (g (x))$ as $x \to x_0$ & Means that
    $\displaystyle{\lim\limits_{x \to x_0}}\frac{f(x)}{g(x)}=0$, with
    $x_0$ possibly infinite.\\
    $f(x) = \Omega_+ (g(x))$ & Means that there exists a suitable
    constant $C> 0$ and a sequence $x_n$ tending to $\infty$ such that
    $f(x) > C g(x)$ holds for $x= x_n$.\\
    $f(x)= \Omega_- (g(x))$ & Means that there exists a suitable
    constant $C> 0$ and a sequence $x_n$ tending to $\infty$ such that
    $f(x) < - Cg (x)$ holds for $x= x_n$.\\
    $f(x) = \Omega_{\pm}(g(x))$ & Means that both $f(x) = \Omega_+
    (g(x))$ and $f(x) = \Omega_- (g(x))$ holds.\\
    $f(x) = \Omega (g(x))$ & Means that $|f(x)|= \Omega_+ (g(x))$.
\end{longtable}
