
\chapter{$\mathscr{P}$-adic group}\label{chap4}


\section{Statement of results}\label{chap4:sec4.1}
\pageoriginale

In this chapter by a $\mathscr{P}$-adic field we mean a discrete
valuated complete field of characteristic zero with finite residue
class field. We shall first state two theorems 

\setcounter{thm}{0}
\begin{thm}\label{chap4:thm1}
Let $G$ be a linear algebraic group defined over a $\mathscr{P}$-adic
field $K$ and assume $G$ to be semi - simple and simply connected;
then $H^1(K,G) = \{ 1 \}$   
\end{thm}

\begin{thm}\label{chap4:thm2}
Let $G$ be a semisimple and connected algebraic group defined over $K$;
 let $\tilde G \rightarrow G$ be the simply connected covering with
kernel $F$. Then the mapping $\delta : H^1 (K, G) \rightarrow
H^2(K,F)$ obtained from the exact sequence of algebraic groups $1
\rightarrow F \rightarrow \tilde{G} \rightarrow G \rightarrow 1$ is
surjective.  
\end{thm}

Putting these together we can conclude that $\delta$ of theorem \ref{chap4:thm2} is
actually bijective. For $H^1 (K, \tilde{G}) = \{ 1 \}$ by theorem \ref{chap4:thm1},
so that only the distinguished element of $H^1 (K,G)$ gets mapped into
the distinguished element of $H^2 (K,F)$ by $\delta$ . Now suppose two
elements $(a_s)$, $(b_s)$ of $H^1 (K,G)$ get mapped into the element
of $H^2(K,F)$ by $\delta$.  Twist $G$ by the cocycle $(a_s)$ and call
the twisted group ${}_aG$. Since $G$ operates on $\tilde {G}$ by inner
automorphisms $\tilde {G}$ can be twisted by $(a_s)$; let ${}_a
\tilde{G}$ be the twisted group. The resulting sequence $1 \rightarrow 
F \rightarrow {}_a \tilde{G} \rightarrow {}_a G \rightarrow 1$ is again
exact; moreover since ${}_a \tilde{G}$ is the simply connected
covering of the semisimple group ${}_aG$ by the first part of the
argument the images of the cohomology classes $(a_s)$
and\pageoriginale $(b_s)$ 
under the bijective mapping $H^1(K,G) \rightarrow H^1 (K,{}_a G)$ are
both the distinguished element of $H^1 (K,{}_a G)$ and so $(a_s)$ and
$(b_s)$ are cohomologous. Hence $\delta$ is bijective. This shows that
a knowledge of the cohomology of the finite abelian group $F$ will
enable us to determine the cohomology of $G$. 

We shall prove these theorems only for the classical groups, following
the first part of \cite{keyK2}; the second part of that paper contains a
case by case proof for exceptional groups. More satisfactory is the
general theory of semisimple groups over $\mathscr{P}$-adic fields by
$F$. Bruhat and $J$. Tits \cite{keyB-T} of which theorem \ref{chap4:thm1} is a
consequence. 

We start by classifying the $\mathscr{P}$-adic classical groups. By
the results of chapter \ref{chap2}, this reduces to the classification of
simple algebras $A$ over $K$ with involution $I$. Let $(A,I)$ be a
central simple algebra over $K$ with involution $I$. If $A^o$ is the
opposite algebra $I$ gives an isomorphism of $A$ onto $A^o$ so that
$inv_K A = inv_K A^o $, even if $I$ is of the second kind, and so the
isomorphism between $A$ and $A^o$ is not a $K$-isomorphism; but since
$A^o$ is the inverse of $A$ in the Brauer group $B_K$ by \S
\ref{chap3:sec3.2}, 
theorem \ref{chap3:thm2} we gave $inv_K A = -inv_K A^o $ so that
$inv_K A =0 $ or 
$\dfrac{1}{2}$; in the first case $A$ is a matrix algebra over $K$. In
the second case let $A = D \otimes_K M_\Gamma (K)$ where $D$ is a division
algebra over $K$. Since $inv_K M_\Gamma(K) = 0 $ and $inv_K A = inv_K D +
inv_K M(K)  \mod 1$ we have $inv_K D = \dfrac{1}{2}$, i.e. $D$ is a
quaternion division algebra so that $A$ is a matrix ring over the
quaternion division algebra $D$. This gives the following 

\setcounter{proposition}{0}
\begin{proposition}\label{chap4:prop1}%pro 1
Any central\pageoriginale simple $K$-algebra with involution is either
a matrix algebra over $K$ or a matrix ring over quaternion division 
algebra. Using these results we shall prove the following
classification theorem. 
\end{proposition}

\begin{theorem*}
The only simply connected absolutely almost simple classical groups
over $K$ are the following : 
\begin{itemize}
\item[${}^1 A_n$]-Norm-one group of simple $K$-algebras 

\item[${}^2 A_n$]-Special unitary groups of hermitian forms over
  quadratic extensions of $K$ 
\end{itemize}

$C_n$-Sympletitic groups and unitary groups of hermitian forms over
quaternion division algebras. 

$B_n$, $D_n$-Spin groups of quadratic forms, spin groups of skew
hermitian forms over quaternion algebras  
\end{theorem*}

\begin{proof}
For type ${}^1 A_n$ there is nothing to prove anew. For subtype ${}^2 A_n$
we know the corresponding groups are $\left\{ x \in A/ xx^I = 1; Nx =1
\right\}$ 
where $A$ is a simple $K$-algebra with involution $I$ of the second kind
and $N$ stands for the reduced norm. We know by \S
\ref{chap2:sec2.5} that $A \cong 
M_r (L)$ where $L$ is a quadratic extension of $K$. If $x = (x_{ij})
\in M_r (L)$ let $x^* = (x^I_{ji})$; then $x \to x^*$ is an involution
of the second kind on $M_r (L)$. Since the restrictions of $I$ and $*$
to $L$ are the same there exists an invertible element $a \in A$ such
that $x^I = ax^*a^{-1}$ for every $x \in A$. We saw in \S
\ref{chap2:sec2.5} that $a$ 
can be chosen to satisfy $a^* = a$, i.e. hermitian. The condition $xx^I
= 1$ means that $xax^* = a$, i.e $x$ is an element of the unitary group
of $a$; the condition $Nx = 1$ means $\det x = 1$\pageoriginale under the
identification $A M_r(L)$. Hence $x$ is actually an element of the
special unitary group of the hermitian form defined by $a$. Conversely
groups defined in this way belong to subtype ${}^2 A_n$. The other types
are classified in \S \ref{chap2:sec2.6}. 
\end{proof}

\section{Proof of theorem 2}\label{chap4:sec4.2}
We shall now prove theorem \ref{chap4:thm2}. We shall construct a commutative
subgroup $ \tilde{T} \subset \tilde{G}$ such that $\tilde{T} \supset
F$ and such that $H^2 (K, \tilde{T}) = \{ 1 \}$. Then the diagram
below is commutative with exact rows.  
\[
\xymatrix{
1 \ar[r] & F \ar[r] & \tilde{G} \ar[r] & G \ar[r] & 1 \\
1 \ar[r] & F \ar[r] \ar[u] & \tilde{T} \ar[r] \ar[u] & T \ar[r] \ar[u]
& 1
}
\]
Here $T$ is the quotient of $\tilde{T}$ by $F$ and the vertical maps
are inclusions. Using $H^2 (K, \tilde{T}) = \{ 1 \}$  we then get a
commutative diagram
\[
\xymatrix{
H^1 (K,G) \ar[r]^{\delta} & H^2 (K,F) & \\
H^1 (K,T) \ar[r]^{\delta} \ar[u] & H^2 (K,F) \ar[r] \ar[u] & 1
}
\]
which shows that $\delta : H^1(K,G) \to H^2(K,F)$ is surjective, hence
the theorem will be proved if we can construct such a $\tilde{T}$. For
this we have the following  

\setcounter{lem}{0}
\begin{lem}\label{chap4:lem1}
$\tilde{G}$ contains an algebraic torus $\tilde{T}$ defined over $K$
  and containing $F$ such that $H^2 (K, \tilde{T}) = \{ 1 \}$. 

By the results of \S \ref{chap2:sec2.2}, it suffices to prove lemma
\ref{chap4:lem1} for 
$\tilde{G}$ absolutely almost simple. We know by \S
\ref{chap3:sec3.2}, theorem \ref{chap3:thm5} 
that if $\tilde{T}$ is a\pageoriginale torus of $\tilde{G}$ over $K$
split by the Galois extension $L$ of finite degree over $K$ then  
$$
H^2(g_{L/K}\tilde{T}) \cong \hat{H}^o (g_{L/K} X_{\tilde{T}}) \text{
  where } X_{\tilde{T}}= \Hom (G_m, \tilde{T}) 
$$
is the set of homomorphisms of $G_m$ into $\tilde{T}$ defined over
$L$. Since center $\tilde{G} \supset F$, if we can construct a torus
$\tilde{T}$ such that $i)$ $\tilde{T} \supset$ centre of $\tilde{G}$
and $ii)$ there exists no nontrivial rational homomorphism of $G_m$
into $\tilde{T}$ defined over $K$, then it will follow that
$\hat{H}^o (g_{L/K} X_{\tilde{T}}) = \{ 1 \}$ and by what precedes
lemma \ref{chap4:lem1} will be proved after passing to the inductive limit. Hence
lemma \ref{chap4:lem1} is a consequence of the following  
\end{lem}

\begin{lem}\label{chap4:lem2}
If $G$ is a simply connected absolutely almost simple classical
group over $K$ there exists a torus $T / K$ containing the centre such
that there exists no non-trivial homomorphism of $G_m$ into $T$ over
$K$.  
\end{lem}

\medskip
\noindent{\bf Type} ${}^1A_n$: here $G = \{ x \in A/Nx = 1 \}$ where $A$ is a
simple $K$-algebra. Let $L$ be a maximal commutative subfield of $A$ and
define $T = \{ x \in L/Nx = 1 \}$; we saw in \S \ref{chap3:sec3.2}
Example 2 that 
$T$ is a torus; clearly $T \subset G$ and contains the center of $G$. 

We claim that $T$ satisfies our requirements: let $L^*$ be the torus
defined in \S \ref{chap3:sec3.1} Example 1 let $f : L^* \to G^{n+1}_m$ be the
isomorphism defined over $\bar{K}$. Let for each $i$ such that $1 \leq
i \leq n+1$, $u_i$ be the homomorphism $u_i : G_m \to G^{n+1}_m $
defined by $ x \to (1, \ldots, x \cdots 1)$. These $u_i$'s generate
$\Hom_{\bar{K}}( G_m, G^{n+1}_m)$ this being the set of rational
homomorphisms of $G_m$\pageoriginale into $G^{n+1}_m$, defined over
$\bar{K}$. Hence 
$\bar{f}^{-1} \cdot u_i$ generate  $\Hom_{\bar{K}}(G_m,T)$; moreover these
$\bar{f}' o u_i' s$ are free generators of the free abelian group
$\Hom_{\bar{K}}(G_m,T)$. The isomorphism $f: L^*_{\bar{K}} \to
(\bar{K}^*)^{n+1}$ is defined in the following way; let $s_1,
\ldots,s_{n+1}$ be the distinct $K$-isomorphisms of $L$ into $\bar{K}$;
then $f$ is induced by the map $L \otimes_K \bar{K} \to \bar{K} \oplus \cdots
\oplus \bar{K}$ given by $x \otimes y \to ({}^{s_1}x.y,
{}^{s_2} x.y,\ldots, {}^{s_{n+1}} x.y)$. Using this it follows
that if $F$ is a finite Galois extension of $K$ which splits $T$ then
$g_{F/K}$ acts transitively on the set of $f^{-1} o u_i's$ and that its
action is simply to permute them. This implies if $\prod ( f^{-1} o
u_i)^{r_i} \in \Hom_K(G_m,L^*)$ then $r_i = r$ for all $i$. The image of
$\prod ( f^{-1} o u_i)^{r_i}$ will be in $T$ if and only $\sum r_i =
0$; these two conditions can be satisfied only when all the $r_i$ are
zero.
  Hence there exists no non-trivial homomorphism of $G_m$ into $T$
defined over $\bar{K}$. This proves the lemma for the subtype ${}^1
A_n$.  

Next consider the subtype ${}^2 A_n$; these are special unitary groups
of hermitian forms over quadratic extensions $L$ of $K$. Let $G =
SU(h)$ accordingly; choose an orthogonal basis $e_1, e_2, \ldots,
e_{n+1}$ of the corresponding vector space. Define  $T = \{ x \in G
/xe_i = \lambda_i e_i \}$ where the $\lambda_i's$ are scalars; the
$\lambda_i's$ satisfy the conditions $\prod \lambda_i = 1$ and
$N_{N/K}\lambda_i =1$; this shows that $T$ is isomorphic to a product
of groups of the previous type considered and we can repeat the
argument for each of the components of this product. 

Next consider the type $C_n$: Here we have to consider  $i)$ $G =
Sp_{2n}$ $ii)$ $G$=unitary group of hermitian form over quaternion
division algebra. 

\begin{description}
\item[{\rm Case (i).}]
We have\pageoriginale $Sp_{2n} \supset Sp_2 \times \cdots  \times Sp_2
\cong SL_2 \times  \cdots \times  SL_2$; since $SL_2$  
$$
n \text{ factors } \qquad n \text{ factors }
$$
is of type ${}^1 A_n$ we can construct a torus $T_1$ of dimension one in
$SL_2$ containing the centre of $SL_2$ for which $\Hom_K (G_m, T_1) =
\{ 1 \}$.  

\noindent
Then $T = T_1 \times  \cdots  \times  T_1$ will be a torus of the
desired kind; for  
$$
n \text{ factors}
$$
$\Hom_K (G_m, T) \cong \{ \Hom_K (G_m, T_1)\}^n = \{ 1 \}$. The centre
of $Sp_{2n}$ is $\{ \pm 1 \}$ which is contained in the centre of
$Sp^n_2$; hence centre $Sp_{2n} \subset$ centre $SL^n_2$ but $T^n_1
\supset$ centre of $SL^n_2$; hence $T \supset $ centre $Sp_{2n}$: this
proves the result in this case. 

\item[{\rm Case (ii).}]
The argument is similar; let $G =U_n (C/K,h)$ where $C$ is a
quaternion division algebra over $K$ and $h$ is a non-degenerate
hermitian form over $C$. Taking an orthogonal basis for the corres
ponding vector space we see that $U_n (C/K,h)$ contains the product of
$n$ one-dimensional unitary groups operating on $C$ as a left vector
space over itself with $h(x,y) = xay^I$; here $I$ is the standard
involution on $C$ and $a \in K$. Any $C$-linear automorphism of $C$ is
of the form $x \to x \lambda$ with $\lambda \in C^*. h(x \lambda , y
\lambda) = h(x,y) \Longleftrightarrow \lambda \lambda^I = 1$. 

Hence $U_1 (C/K,h)$ is isomorphic to the group of elements of $C$ with
norm 1 and so $U_1$ is a group of type ${}^1A_n$. Hence by what we
have proved there exists a torus $T_1$ of dimension one defined over
$K$ containing the centre of $U_1 (C/K,h)$ and such that $\Hom_K(G_m,\break
T_1) = \{ 1 \}$. Define $T= T^n_1$; as in the last case considered $T$
will be a torus having the requisite properties. 
\end{description}

Next\pageoriginale consider types $B_n$, $D_n$: 

\noindent
i) $G= SO_{2n}(n \geq 2)$; we need the following 

\begin{lem}\label{chap4:lem3}
Let $V$ be a non-degenerate quadratic space over $K$ of dimension $2n$
with $n \geq 2$. Then there exists an orthogonal splitting $V = V_1
\perp V_2 \perp \cdots \perp V_n$ where the $V'$s are two dimensional
anisotropic subspaces.  
 \end{lem} 
 
 We shall assume this lemma for the moment. Let $T_i$ be the special
 orthogonal group of the quadratic space $V_i$; then $SO_{2n} \supset
 T_1 \times  \cdots \times  T_n$; we have shown in \S \ref{chap3:sec3.1} Example
\ref{chap3:exam3} that the $T_i'$s are algebraic tori; define $T= T_1 \times \cdots
 \times T_n$; we claim that 
 this torus satisfies our requirements. The proof of this is exactly
 the same as in the case of $Sp_{2n}$; we have only to use the fact
 that the centre of $SO_{2n}$ is $\pm 1$. Consider the spin group now;
 let $p : \Spin_{2n} \to SO_{2n}$ be the covering homomorphism. If $T$
 is the torus of $SO_{2n}$ constructed as above then we claim that
 $\tilde{T} = p^{-1}(T)$ has the required properties; firstly since
 $p| \tilde{T}: \tilde{T} \to T$ is a surjective morphism with finite
 kernel if we prove that $\tilde{T}$ is connected it will follow from
 \S \ref{chap3:sec3.1}. Remark 2 that $\tilde{T}$ is
 an algebraic torus.  
 
 Let $V_1$ be a two dimensional anisotropic quadratic space. Then the
 Clifford algebra $C(V_1)$ is of dimension 4 and $C^+(V_1)$ is just
 $K ( \sqrt{c})$ where $c$ is the discriminant of $V_1$ and the $*$
 automorphism is the non-trivial $K$-automorphism of $K (
 \sqrt{c})$. Hence we have $\Spin_2(V_1) = \{ x \in C^+ / xx^* = 1 \}$
 is isomorphic to the group of elements of  $K ( \sqrt{c})/K$ with
 norm 1 which is a torus. In particular $\Spin_2(V_1)$ is
 connected.\pageoriginale Moreover if $T'_i = \Spin (V_i)$ is
 considered as a 
 subgroup of $\Spin (V)$ then $p^{-1}(T) = p^{-1} (T_1 \times \cdots
 \times  T_n)
 =T'_1 \times  \cdots \times  T'_n $ is connected. Hence as said before
 $\tilde{T}$ is a torus. Evidently $\tilde{T} \supset$ center of
 $\Spin_{2n}$ since $T \supset$ center of $SO_{2n}$. Next if $f : G_m \to
 \tilde{T} $ is a non-trivial homomorphism defined over $K$, $f(G_m)$
 is one-dimensional; since the kernel of $p| \tilde{T}: \tilde{T} \to
 T$ is zero dimensional the composite $p| \tilde{T} o f$ will be a
 non-trivial homomorphism of $G_m$ into $T$ defined over $K$ which
 contradicts the property of $T$; hence $\Hom_K (G_m,\tilde{T})= \{ 1
 \}$ and so $\tilde{T}$ has the required properties. For type $D_n$ we
 need the following  

 \begin{lem}\label{chap4:lem4}
If $h$ is a skew-hermitian form on the quaternion division algebra $C$
over $K$ then $SU_1(C,h)$ is a torus $T$ without non-trivial
homomorphism of $G$ into $T$ over $K$. 
 \end{lem}

\begin{proof}
If $h(x,y) = xay^I$ with $a^I = -a$, then $SU_1(C,h) = \{ x \in C/ x a
x^I = a, Nx = 1 \} = \{ x \in C/ xa = ax, Nx = 1 \}$, i.e. the
elements of norm 1 in the quadratic extension $K(a)$. 
\end{proof} 

Since our groups of type $D_n$ are isomorphic to the unitary group of
a skew-hermitian form over a quaternion division algebra say
$U_n(C/K,h)$ we can carry out the procedure adopted for $SO_{2n} $ to
construct a torus with the requisite properties for $U_n(C/K;h)$. 

\setcounter{proofoflem}{2}
\begin{proofoflem} %%% 3
To  start with let the quadratic space $V$ be of dimension four. There
certainly exists a decomposition $V= V_1 \perp V_2$ with $V_1$, $V_2$
two dimensional and $V_2$ containing an anisotropic vector $e_2$; let
$\mathscr{G}$ be the quadratic form associated with $V$. If both $V_1$
and $V_2$\pageoriginale are anisotropic there is nothing to prove. So
suppose $V_1$ 
is isotropic. Using the fact that the quadratic form of an isotropic
space represents any non-zero element of $K$ we can choose $e_1 \in
V_1$ so that $ \mathscr{G}(e_1,e_1) =
\dfrac{-a}{\mathscr{G}(e_2,e_2)}$ where $a$ is some element of $K^*$
which is not a square and not equal to the discriminant of $V$ modulo
squares, the choice of such an a being always possible in the
$\mathscr{P}$-adic field. Then the two dimensional subspace $V'_1=
Ke_1 \oplus Ke_2 $ of $V$ has discriminant equal to $-a$ modulo
squares; by the choice of $a$, $V'_1$ is anisotropic; if $V'_2$ it the
orthogonal complement of $V'_1$ it is also anisotropic since by choice
$ a \not\equiv -d(V) \mod$ squares. 
\end{proofoflem}

\noindent
Hence $V = V'_1 \perp V'_2$ is the required decomposition. The general
case is proved by induction on the integer $2n$. If $n=2$ we have just
seen that the lemma is true. So we can assume that $n \geq 3$ and that
the lemma is true for all subspaces of $V$ of dimension $2m$ with $2
\leq m < n$. Let $U$ be a four dimensional non-degenerate subspace of
$V$. Let $V = U \perp W$  be an orthogonal splitting. For $U$ we have
a decomposition $U= V_1 \perp V_2 $ with $V_1$ and $V_2$ both
anisotropic. The subspace $V_2 \perp W$  is of dimension $2(n-1) \geq
4$ so induction assumption can be applied. 

\noindent
Hence there exists an orthogonal splitting $V_2 \perp W = V'_2 \perp
V'_3 \perp \cdots \perp V'_n$ where $V'_2, V'_3, \ldots V'_n$ are
anisotropic subspaces of dimension two. 
Hence $V = V_1 \perp V'_2 \perp V'_3 \perp \cdots \perp V'_n$ is a
splitting of the required kind for $V$. This proves the lemma. 

Now consider $SO_{2n+1}$ for odd dimension. In this case we can prove
that the corresponding vector space has a decomposition  $V_1 \perp
\cdots \perp V_n \perp W$\pageoriginale with $V_i'$s two dimensional
anisotropic and $W$ one dimensional. As before we can prove that $p:
\Spin_{2n+1} \to SO_{2n +1}$ is the covering homomorphism and $T =
SO(V_1) \times \cdots \times SO(V_n)$ then $p^{-1}(T)$ will be the
torus with the required properties. This completes the proof of
theorem \ref{chap4:thm2}.  

\section{Proof of theorem 1}\label{chap4:sec4.3}

\medskip\noindent
{\bf Type {\boldmath${}^1 A_n $:}} The classical group of this type is $G = \left\{ x
\in A/Nx = 1 \right\}$ where $A$ is a simple $K$-algebra. From the exact
sequence $1 \to G \to A^* \to G_m \to 1$ and from the fact $H^1
(K,A^*) = \{ 1 \}$ proved in 1.7. Example 1 we get the
following exact sequence of cohomology sets:  
$$
A^*_K \overset{N}{\to} K^* \to H^1(K,G) \to 1.
$$

To show that $H^1 (K,A^*) = \{ 1 \}$ we have only to prove the  




\setcounter{lem}{0}
\begin{lem}%% 1
$A^*_K \overset{N}{\to} K^*$ is surjective.
\end{lem}

\begin{proof}
Let $t \in K^*$ be a prime element i.e. an element of order 1 in the
discrete valuation. The polynomial $f (x) = x^{n+1}+ (-1)^{n+1}t$ is
an Eisenstein's polynomial over $K$ and consequently irreducible. 
Hence $\dfrac {K[X]}{(f)}$ is a field extension of $K$ of degree
$(n+1)$ so that by theorem \ref{chap3:thm3}, \S
\ref{chap3:sec3.2}. A is split by the extension 
$\dfrac {K[x]}{(f)}$. Hence there exists a $K$-isomorphism of $\dfrac
{K[x]}{(f)}$ onto a subfield  $L$ of $A$; the reduced norm of an
element of $L$ is the usual norm $N_{L/K}$; since the residue class of
$x$ in $\dfrac {K[x]}{(f)}$ has norm over $K$ equal to $t$ we see that
$t$ is the reduced norm of an element of $A^*$. Next if $\varepsilon $ be any
unit of $K^*$ both $t$ and $t \varepsilon $ are prime elements and so they are
reduced norms of elements of $A^*$; consequently $\varepsilon $ is
the\pageoriginale reduced norm of element of $A^*$. Since any element
of $K^*$ can be written as $t^r \varepsilon$ where $r \in Z$ and
$\varepsilon$ is a unit the lemma follows. 
\end{proof}

\medskip
\noindent
{\bf Type {\boldmath${}^2A_n$}.} Here $G= SU_{n+1}(L/K,h)$ where
$[L:K] =2$ and $h$ 
is a hermitian form over $L$. Since we have seen from the Dynkin
diagram that there is no subtype ${}^2 A_1$, $SU_2(L/K,h)$ is
isomorphic to a group belonging to type ${}^1 A_n\cdot A$ simple
direct proof will be as follows. 

\begin{lem}%%% 2
Let $A$ be a quaternion $K$-algebra with involution $I$ of the second
kind; let $L$ be the centre of $A$ so that $[L : K]= 2$. Then the
group $G_1 = \left\{ x \in A/xx^I = 1, Nx =1 \right\}$ is isomorphic
to the group $G_2 = \left\{ x \in B/Nx = 1 \right\}$ where $B$ is a
quaternion algebra of centre $K$. 
 \end{lem} 

 \begin{proof}
We proved in \S \ref{chap2:sec2.5} proposition \ref{chap2:prop1}
that there exists a quaternion 
algebra $B$ of centre $K$ such that $A \cong B \otimes_K L$, that $I$
induces the standard involution on $B$ and on $L$ the action of $I$ is
the non-trivial $K$-automorphism. Let $L= K (\sqrt{d})$ Then any $x \in
A$ can be written as $x = x_1 \otimes 1 + x_2 \otimes \sqrt{d}$. Then  
\end{proof} 

\noindent
$xx^I = (x_1 \otimes 1 + x_2 \otimes \sqrt{d}) (x^I_1 \otimes 1 -
 x^I_2 \otimes \sqrt{d}) = ( x_1 x^I_1 - dx_2 x^I_2) \otimes 1 + (x_2
 x^I_1 - x_1 x^I_2) \otimes \sqrt{d}$; the condition $xx^I=1 $ implies
 $(x^{-1}_1 x_2)^I = x^{-1}_1 x_2$, hence $x_2 =tx_1$ with $t \in
 K$. Then $x = x_1 \otimes 1 + tx_1 \otimes \sqrt{d} = x_1 \otimes (1
 + t \sqrt{d}) = x_1 \otimes z$, say where $z \in L$. Now $xx^I = 1$
 is equivalent to $x_1 x^I_1 \otimes zz^I= 1$; the condition $Nx = 1$ is
 equivalent to $x_1 x^I_1 \otimes z^2 = 1$ (because in $B$, $x_1 x^I_1$
 is the reduced norm; the term $z^2$ accounts for the fact that in $A$
 the reduced norm is taken with respect to $L$). The last two
 conditions give $z = z^I$; i.e. $z \in K$; hence if $x \in G_1$ we have
 proved that\pageoriginale $x = x_1 z \otimes 1$ i.e. $x \in B$; but
 then $B$, $xx^I =1$ and $Nx= 1$ are equivalent; hence the lemma.  

The lemma implies by case ${}^1 A_n$ that $H^1(K,SU_2(h))=\{1\}$. 
Let $V$ be the vector space of $(n+1)$ dimension over $L$
corresponding to this hermitian from $h$; choose a vector $a \in V$
such that $h(a,a)\neq 0$. Let $H$ be the subgroup of $G$ consisting of
those elements which fix the vector $a$. Then $H$ is the unitary group
$SU_n$ of dimension $n$ corresponding to the orthogonal complement of
$a$ in $V$. Applying induction on $n$ we have only to prove the
following. 
 
 \begin{lem}%lemm 3
The map $i : H^1(K,SU_n) \to H^1(K,SU_{n+1})$ induced by the injection
$SU_n \to SU_{n+1}$ is surjective for $n \ge 2$. 
 \end{lem} 
 
 \begin{proof}
Let $a=(a_s) \in H^1(K,SU_{n+1})$; (a) is in the image of $i$ if and
only if the twisted homogeneous space ${}_a(SU_{n+1}/ SU_n)$ has a
$K$-rational point by \S \ref{chap1:sec1.5}. Proposition
\ref{chap1:prop1}. Now $SU_{n+1}/ SU_n 
\cong U_{n+1}/U_n$. Let $T= \bigg\{ x \in V/h(x,x)=c \bigg\} $ where
$c=h(a,a)$, $U_{n+1}$ acts transitively on $T$ and the subgroup of
$U_{n+1}$ fixing $T$ is $U_n$; hence $U_{n+1}/U_n \cong T$ so that
${}_a(SU_{n+1}/SU_n)\cong a^T$, 
\end{proof} 

Since $G$ is a group of $L$-automorphisms of $V$ and that the 
hermitian from $h$ is fixed by all these automorphisms we can twist
both $V$ and $h$ by the cocycle $(a_s)$ to get a vector space $V'$ and
a hermitian form $h'$ and an isomorphism $f: V \otimes_K \bar{K} \to
V' \otimes_K \bar{K}$ such that $f^{-1}o^s f=a_s$. Let $T'=\bigg\{ x
\in V'/h'(x,x)=c \bigg\}$; then $f(T)=T'$ holds; and $f^{-1}o^s
f=a_s$ show that $T' \cong_a T$. We have only to shows
that\pageoriginale $T'$ has 
a $K$-rational point. But this follows from the fact that any
non-degenerate hermitian form of $\dim \ge 2$ is a quadratic form of
dimension $\ge 4$, and so over a local field represents any non-zero
element of the field. Hence the lemma is proved and consequently
theorem $I$ for type ${}^2 A_n$. 

\noindent
{\bf Type $C_n$} i) $G=Sp_{2n}$; we proved in \S \ref{chap1:sec1.7}. Example
\ref{chap1:exam3} that $H^1(K,Sp_{2n})=1$. ii) $G=U_n(C/K,h)$ where $C/K$ is a
quaternion algebra of centre $K$ and $h$ is a non-degenerate hermitian
form over $C$; let $V$ be the corresponding $n$-dimensional vector
space. For $n=1 U_1(C/K,h)$ is isomorphic to a group of type ${}^1A_n$
so that $H^1(K,U_1(C/K,h))=\{1\}$ by theorem 
\ref{chap4:thm1} for type ${}^1A_n$. Let $U_{n-1}(C/K,h)$ be the subgroup of
$U_n(C/K,h)$ fixing an anisotropic vector of $V$; then just as in the
discussion of type ${}^2A_n$, the map $H^1(K,U_{n-1}(C/K,h)) \to H^1
(K,U_n(C/K,h))$ is surjective. Here we have only to use the fact that
any hermitian form $h$ with respect to the standard involution on $C$
represents any non-zero element of the local field $K$. Applying
induction on $n$ we see the truth of theorem \ref{chap4:thm1} in this case. 
  
 \noindent
 {\bf Types $B_n$ and $D_n$}. First consider $SO_n$ and its
 universal covering Spin $n$; for $n \le 6$ the theorem follows by the
 isomorphism of $\Spin_n$ with group of the previous types considered;
 for example $\Spin_3$ is the norm-one group of the second Clifford
 algebra which is a quaternion algebra (\S \ref{chap2:sec2.3}, \cite{keyDi},
 \cite{keyE3}) and so 
 $H^1(K,\Spin_3)=\{1\}$; we can assume $n \ge 4$ and apply induction;
 let $0_{n-1}$ be the subgroup of $0_n$\pageoriginale fixing an
 anisotropic vector; 
 then $\Spin_n / \Spin_{n-1}=SO_n/SO_{n-1} \cong O_n /O_{n-1}$; by
 Witt's theorem $O_{n/O_{n-1}}=\bigg \{ x \in V/
 \mathscr{G}(x,x)=\mathscr{G}(a,a)\bigg\}$ have $V$ is the vector
 space corresponding to the quadratic form $\mathscr{G}$ and $a$ is the
 anisotropic vector chosen. Applying the twisting argument and the
 fact that any quadratic form over a $\mathscr{P}$-adic field $K$ in
 at least four variables represents any non-zero element of $K$ it will
 follow that the map $H^1(K, \Spin_{n-1}) \to H^1(K, \Spin_n)$ is
 surjective for $n \ge 4$. Induction now works for the proof of
 theorem \ref{chap4:thm1}. 
 
 Finally consider the remaining classical groups of type $D_n$; they
 are $G= \Spin_n(C/K,h)$ where $C$ is a quaternion algebra over $K$
 and $h$ is a skew hermitian form over $C$  with respect to the
 standard involution on $C$. Let $V$ be the corresponding vector space
 over $C$. Then $G$ is the simply connected covering of $SU_n(C/K,h)$
 the special unitary group. Now for $n=3$ this group is isomorphic
 over $K$ to a group of type $A_3$ so that its Galois cohomology is
 trivial. This can be used to apply induction on $n$. If $U_{n-1}$ is
 the subgroup of $U_n$ fixing an anisotropic vector of $V$ then
 $\Spin_{n} /_{\Spin_{n-1}} \cong SU_n/_{SU_{n-1}}\cong
 U_n/_{U_{n-1}}$; in this case also $U_n/_{U_{n-1}}$ is a sphere; by
 the methods previously employed it is now evident that the truth of
 theorem \ref{chap4:thm1} will be guaranteed by the following 
 
 \begin{lem} %lemm 4
Any skew-hermitian form $h'(x,y)$ over $C$ of dimension at least three
represents any non-zero skew-quaternion of $C$. For proofs see \cite{keyJ},
\cite{keyK2}, \cite{keyTs}. We shall give here one more proof,
using\pageoriginale 
the isomorphisms of $SU$ with groups of type $A_1 \times A_1$, $A_3$.   
 \end{lem} 

\begin{proof}
Let $c \in C$ be the skew quaternion of the theorem. It is sufficient
to consider vector spaces of dimension 3. Let $V'$ be the vector
space corresponding to $h'(x,y)$; we have to shows that the sphere $
S'=\bigg\{ x '\in V'/h'(x,x)=c \bigg\}$ has a $K$-rational
point. Construct a vector space $V$ of dimension 3 over $C$ with a
skew-hermitian form $h$ which represents $c$ $K$-rationally. $V$ can
also be constructed to have discriminant equal to that of $V$. This is
assured by the following 
 \end{proof} 
 
\medskip
 \noindent{\bf sublemma.}
Any non-zero element of a quaternion algebra $C$ over any field $K$ 
can be written as the product of two skew symmetric elements of $C$
with respect to the standard involution. 
 
 \begin{proof} 
Let $c \in C$, $c \neq 0$ be given; let $\bar{C}$ denote the space of
skew symmetric elements of $C$; the intersection $c \bar{C} \cap
\bar{C}$ is non-zero since both $C^-$ and $c C^-$ are of
dimension three whereas $C$ is of dimension 4; hence there exists two
non-zero elements $c_1,c_2 \in C^-$ such that $cc_1=c_2$; i.e,
$c=c^{-1}_1 c_2$; here both $c^{-1}_1$ and $c_2$ are skew
symmetric. This proves the sublemma. 
 \end{proof} 
 
 We know that any non-zero element of $K$ is the reduced norm of an
 element of $C$; choose $d \in C$ so that $Nd = \dfrac{d(V')}{Nc}$;
 write $d=d_1d_2$ where $d_1$ and $d_2$ are skew symmetric; the matrix  
$
 \begin{pmatrix}
c & 0 & 0\\
0 & d_1 & 0 \\
0 & 0 & d_2\\
 \end{pmatrix} 
 $
 is skew hermitian and if the vector space $V$ is provided with the
 hermitian form $h$ corresponding to this matrix then $h$ represents
 $c$ and\pageoriginale that discriminant of $V'$. 
  Now $(v,h)$, and $(V',h')$ are isomorphic over $\bar{K}$; let
 $f:V_{\bar{K}} \to V'_{\bar{K}}$ be this isomorphism, then $f^{-1}o
 {}^sf=a_s \in U_3(V,h)$ where $U$ denotes the unitary group of $h$;
 since the discriminants of $V$ and $V'$ are equal we have by \S
 \ref{chap2:sec2.6}  lemma \ref{chap2:lem3} that $a_s \in
 SU_3(V,h)$; let $U_2$ be the unitary group of 
 the orthogonal complement of a chosen vector representing $c$; let
 $S$ be the homogeneous space $SU_3 /SU_2$ which is the sphere
 $\bigg\{x \in V/h(x,x)=c \bigg\}$; then the sphere $S$ is just the
 twisted sphere ${}_aS$. The proof of the theorem will be achieved once
 we prove that the cocycle $(a_s)$ is in the image of the map
 $H^1(K,SU_2) \to H^1(K,SU_3)$; we shall actually prove that this map
 is bijective. The commutative diagram 
\[
\xymatrix{
1 \ar[r] & Z_2 \ar[r] & \Spin_3 \ar[r] & SU_3 \ar[r] & 1\\
1 \ar[r] & Z_2 \ar[r] \ar[u] & \Spin_2 \ar[r] \ar[u] & SU_2 \ar[r]
\ar[u] & 1
}
\]
 with exact rows and whose vertical maps are the natural ones, gives
 rise to the commutative diagram 
\[
\xymatrix{
H^1 (K,SU_3) \ar[r]^{\delta} & H^2 (K,Z_2) \\
H^1 (K,SU_2) \ar[r]^{\delta} \ar[u] & H^2 (K,Z_2) \ar[u]_{\rm
  identity}
}
\] 
 By theorems (1) and (2) the rows are isomorphisms; since the
 righthand column is an isomorphism the left hand column is also an
 isomorphism. This proves the result. Hence theorem \ref{chap4:thm1} is proved
 completely. 
