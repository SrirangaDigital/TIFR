
\chapter{The symplectic  metric}%sec 7

The\pageoriginale paragraph deals with the symplectic metric in $\mathscr{Y}$
defined by a positive quadratic differential form which is invariant
under all symplectic substitutions. Let $Z_1$, $Z_2 \in \mathscr{Y}$
and $M = \begin{pmatrix} A & B \\ C & D \end{pmatrix}\in S_n$. Let
$Z^*_\nu = M <Z_\nu>$, $\nu = 1, 2$. A simple computation yields by means
of the relation 
$$
Z^*_\nu = (A Z_\nu + B)(CZ_\nu +D)^{-1} = (Z_\nu C' + D')^{-1} (Z_\nu
A' + B') 
$$ 
and the typical relations for a symplectic  matrix established in $\S
1$, the formulae 
\begin{align*}
Z^*_2 - Z^*_1 & = (Z_1 C' + D')^{-1} (Z_2 - Z_1) (CZ_z + D)^{-1} \\ 
Z^*_2 - \bar{Z}^*_1 & = (\bar{Z}_1C' + D')^{-1} (Z_2 - \bar{Z}_1) (CZ_2 +
D)^{-1} \\ 
\bar{Z}^*_2 - \bar{Z}^*_1 & = (\bar{Z}_1C' + D')^{-1} (\bar{Z}_2 -
\bar{Z}_1) (C \bar{Z}_2+ D)^{-1} \tag{122}\label{eq122} \\ 
\bar{Z}^*_2 - \bar{Z}^*_1 & = (Z_1C' + D')^{-1} (\bar{Z}_2 - {Z}_1)
(C \bar{Z}_2+ D)^{-1} 
\end{align*}

Now for any points $Z_1, Z_2 \in \mathscr{Y}$ it is easily seen that
$Z_2 - \bar{Z}_1$ and $\bar{Z}_2 - Z_1$ also belong to $\mathscr{Y}$
and by (\ref{eq25}), every $Z \in \mathscr{Y}$ is nonsingular. Thus the
inverses $(Z^*_2 - \bar{Z}^*_1)^{-1} (\bar{Z}^\ast_2 - Z^\ast_1)^{-1}$
exist. From (\ref{eq122}) we now obtain that  
\begin{gather*}
(Z^*_2 - Z^*_1) (Z^*_2 - \bar{Z}^*_1)^{-1} (\bar{Z}^\ast_2 -
  \bar{Z}^\ast_1)(\bar{Z}^*_2 - 
  \bar{Z}^*_1)^{-1} \tag{123}\label{eq123}\\ 
= (Z_1 C' + D')^{-1} \varrho (Z_1 , Z_2) (Z_1 C' + D')
\end{gather*}
where
\begin{equation*}
\varrho (Z_1 , Z_2) = (Z_2 - Z_1) (Z_2 - \bar{Z}_1)^{-1} (\bar{Z}_2 -
\bar{Z}_1) (\bar{Z}_2 - Z_1)^{-1} \tag{124}\label{eq124} 
\end{equation*}

It is\pageoriginale immediate from (\ref{eq123}) and (\ref{eq124}) that the
characteristic roots 
of $\varrho (Z_1 , Z_2)$ for any two $Z_1$, $Z_2 \in \mathscr{Y}$ are
invariant under the symplectic mapping $Z_\nu \rightarrow M < Z_\nu> ,
M \in S_n$. In particular it follows that $\sigma (\varrho(Z_1 ,
Z_2))$ is an invariant function of $Z_1$, $Z_2$ meaning 
\begin{equation*}
\sigma (\varrho(Z_1 , Z_2)) = \sigma  (\varrho (M<Z_1>, M <Z_2>)) , M
\in S_n \tag{125}\label{eq125} 
\end{equation*}

For any matrix $Z$, we define $d Z$ the matrix of the differentials as
$d Z = (d Z_{\mu \nu})$ where $Z = (Z_{\mu \nu})$. For two matrices
$Z_1, Z_2$ the following relations are easily verified. 
\begin{align*}
1) \quad d(Z_1 + Z_2) & = d Z_1 + d Z_2 \tag{126}\label{eq126} \\
2) \quad d(Z_1 + Z_2) & = d Z_1. Z_2 + Z_1 dZ_2
\end{align*}

As a consequence of (\ref{eq2}) above we have, if $|Z| \neq 0$, 
\begin{align*}
0 = d (E) & = d (ZZ^{-1}) = dZ Z^{-1} + Z \cdot dZ^{-1} \text{ so that }\\ 
d Z^{-1} & = Z^{-1} dZ \cdot Z^{-1} \tag*{$(126)'$}\label{eq126'} 
\end{align*}

With these preliminaries about differentials, we once again take up
the main thread. In (\ref{eq124}), we specialize $Z_1$, $Z_2$ as $Z_1 = Z$,
$Z_2 = Z + dZ$ and obtain 
\begin{equation*}
\rho (Z , Z + d Z) = \frac{1}{4} dZ Y^{-1} d \bar{Z} y^{-1} , Z = X +i
Y. \tag{127}\label{eq127} 
\end{equation*}

From (\ref{eq125}) it now follows that 
\begin{equation*}
d \mathscr{S}^2 = \sigma (dZ . Y^{-1} \bar{dZ} Y^{-1}) \tag{128}\label{eq128}
\end{equation*}
is an invariant quadratic differential form in the elements $d X_{\mu
  \nu}$, $d Y_{\mu \nu}$ of $dX$, $dY$\pageoriginale respectively. We
may also consider 
$d's^2$ as a hermitian form in the elements $d Z_{\mu \nu}$
of $d Z$. Further, if $R$ be a real matrix such that $Y^{-1} = RR'$
and if $\Omega = (\omega_{\mu \nu}) = R' d Z R$ then $\Omega =
\Omega'$ and $d \mathscr{S}^2 = \sigma(R' dZ RR' \bar{d}Z R) = \sigma
(\Omega ' \bar{\Omega}) = \sum_{\mu ,\nu} \omega_{\mu \nu}
\bar{\Omega}_{\mu \nu} \ge 0$. If now $\sum_{\mu ,\nu} \omega_{\mu
  \nu} \bar{\omega}_{\mu \nu} = 0$ then $\Omega = 0$ and this will
require that $d Z = 0 $ as $R \neq 0$. Thus $d s^2 > 0$ if
$d Z \neq 0$ and what is the same, $d s^2>0$ represents a
positive quadratic form. Now we can consider $\mathscr{Y}$ as a
Riemannian space with the fundamental metric given by $d
s^2$, called the \textit{symplectic metric}. The
decomposition $d Z = dX - dY$ yields  
\begin{align*}
ds^2 & = \sigma \{ (d X + d Y) Y^{-1} (dX - dY)  Y^{-1}
\}\\ 
& = \sigma \{ (d X   Y^{-1}) dX  Y^{-1}  + (dY - Y^{-1} dY Y^{-1} \}
\end{align*}
taking only the real part, since the imaginary part has got to vanish
as $d s^2 > 0$. Thus we get 
\begin{equation*}
d s^2 = \sigma (Y^{-1} dX)^2 + \sigma (Y^{-1} dY)^2 \tag{129}\label{eq129}
\end{equation*}
as expression which is well known in the case $n = 1$. 

We now carry over these results to the generalized unit circle, which
we may recall is the set $ \mathfrak{K}$ of $W's$ satisfying the
condition 
\begin{equation*}
W = W ' , E - W ' \bar{W} > 0 \tag*{$(2b)'$}\label{eq2b'} 
\end{equation*}
we know that we can map $\mathscr{Y}$ into $\mathfrak{K}$ by means of
the mapping 
$$
W = (Z - iE)(Z + iE)^{-1} = (Z + iE)^{-1}(Z - iE),
$$

If $W_1$, $W_2$ correspond to $Z_1$, $Z_2$ we have 
\begin{align*}
W_2 - W_1 & = 2i  (Z_1 + iE)^{-1} (Z_2 - Z_1)(Z_2 + iE)^{-1} \\ 
E - \bar{W}_1  W_2 & = -2i  (\bar{Z}_1 - iE)^{-1} (Z_2 -
\bar{Z}_1)(Z_2 - iE)^{-1} \\ 
\bar{W}_2 - \bar{W}_1 & = -2i  (\bar{Z}_1 - iE)^{-1} (\bar{Z}_2 -
\bar{Z}_1)(\bar{Z}_2 - iE)^{-1} \tag{130}\label{eq130}\\ 
E - {W}_1  \bar{W}_2 & = 2i  ({Z}_1 + iE)^{-1} (\bar{Z}_2 -
{Z}_1)(\bar{Z}_2 - iE)^{-1}  
\end{align*}\pageoriginale
and then 
{\fontsize{10pt}{12pt}\selectfont
$$ 
 (W_2 - W_1)(E - \bar{W}_1 W_2)^{-1} (\bar{W}_2 - \bar{W}_1) (E - W_1
\bar{W}_2)^{-1} = (Z_1 + iE)^{-1} \varrho (Z_1, Z_2) (Z_1 + iE) 
$$}\relax

In particular therefore, the matrix represented by the left side has
the same characteristic roots as $\varrho (Z_1 , Z_2)$ and hence also
the same trace. Setting $Z_1 = dZ$, $Z_2 = Z + Z$ we have $W_1 = W$, $W_2 = W
+ dW$ and then from the above, and from (\ref{eq127}), (\ref{eq128}) we have  
\begin{align*}
d s^2 & = 4 \sigma ( \varrho (Z, Z + dZ)) \\
& = 4 \sigma (dW( E - \bar{W}W)^{-1} d \bar{W} (E - W \bar{W})^{-1})
\tag{131} \label{eq131}
\end{align*}

This provides another useful expression for $d s^2$

We observe at this stage that given two points $Z_1$, $Z_2 \in
\mathscr{Y}$ we can always find a symplectic substitution which brings
$Z_1$, $Z_2$ into the special position  
$$
Z_1 = iE ,Z_2 = iD = i( \delta_{\mu \nu \delta_{\nu}})\quad 1 \le d_1 \le 
d_2 \le \cdots \le d_n 
$$

For, in view of the $1-1$ correspondence between $\mathscr{Y}$ and
$\mathfrak{K}$ it suffices to determine a permissible substitution
which takes two assigned points $W_1$, $W_2$ into $0$ and a special
points $D_1 = (\delta_{\mu\nu}, \dfrac{d_\nu- 1}{d_\nu +1})$ and in view
\pageoriginale of the homogenity of the space $\mathfrak{K}$ it
suffices to determine 
a mapping which leaves the origin fixed and takes a given point $W$
into a point of the type $D_1$ viz, a diagonal matrix with positive
diagonal elements in the increasing order. This we know is possible by
Lemma \ref{chap2:lem2} by a mapping of the kind 
$$
W \rightarrow u' W u , u -\text{unitary},
$$
and this settles our claim,

We now introduce a parametric representation for positive matrices $y
> 0$. We choose matrices $ F = F^{(r)}$, $G = G^{(r)} $ and $H = H^{(r , n -r)}$
where $r$ is a fixed integer in $1 \le r < n$ to satisfy 
\begin{equation*}
y = \begin{pmatrix} F & 0 \\ 0 & G\end{pmatrix} \bigg [ \begin{pmatrix}
      E & H_{n r} \\ 0 & E\end{pmatrix} \bigg] = \begin{pmatrix} F & FH
    \\ H'E & G + F[H]\end{pmatrix} \tag{132}\label{eq132} 
\end{equation*}

Such a choice is always possible as a comparison of the two extremes
shows. Also from (\ref{eq132}) it is clear that $Y > 0$ is equivalent with
$F > 0$, $G > 0$. By means of (\ref{eq132}) we also have  
$$
y^{-1} = \begin{pmatrix} F^{-1 } + G^{-1} [H'] &  HG^{-1} \\ -G^{-1}H'
&  G^{-2} \end{pmatrix} 
$$
and obtain by a simple computation, using properties (\ref{eq126}),
(\ref{eq126'}), that 
\begin{equation*}
\sigma (y^{-1} dy)^2 = \sigma (F^{-1} dF)^2 + \sigma (G^{-1} dG)^2 + 2
\sigma (G^{-1} dH'FdH) \tag{133}\label{eq133} 
\end{equation*}

 We wish\pageoriginale  to remark at this stage that all terms in
 (\ref{eq133}) are 
 positive quadratic forms in the appropriate elements; a possible
 doubt can only be about the last term but by specialising $dF, dG$ it is
 easily seen that the last term represents a positive quadratic form
 in the elements of $d H$. 

We proved to consider the existence and uniqueness of geodesic lines
in $\mathscr{Y}$ and we prove the following 

\setcounter{thm}{7}
\begin{thm}\label{chap7:thm8}%theo 8 
 Given any two points  $Z_1$, $Z_2 \in \mathscr{Y}$
  there exists a uniquely determined geodesic line joining them. The
  length $s(Z_1 , Z_2)$ of this geodesic line is given by  
\begin{equation*}
s(Z_1 , Z_2) = \sqrt{\sum^n_{\nu =1} (\log \frac{1+
    \lambda_{\nu}}{1 - \lambda_\nu})^2} \tag{134}\label{eq134} 
\end{equation*}
where $\lambda^2_\nu, \nu = 1, 2 , \ldots ,n$ are the characteristic roots
of   
$$
\rho (Z_1 ,Z_2 )=(Z_2 - Z_1) (Z_2 - \bar{Z}_1)^{-1} (\bar{Z}_2 -
\bar{Z}_1) (\bar{Z}_2 - Z_1)^{-1}.  
$$

In the special case $Z_1 = iE$, $Z_2 = i D = l (\delta_{\mu
  \nu})$ a parametric representation of the geodesic line
  given by  
$$
Z = Z(t) = (\delta_{\mu \nu} d^t_\nu ; 0 \le \varepsilon \le 1
$$
\end{thm}

\begin{proof}%proof
Since we know that  there always exists a symplectic substitution which
takes $Z_1$, $Z_2$ into $iE$, $i D$ and that the characteristic roots
$\lambda^2_\nu$ of $\rho (Z_1 , Z_2)$ are invariant under such a
substitution, it clearly suffices to prove the theorem for these
special values of $Z_1$, $Z_2$ Assume for a moment that a geodesic line
joining $Z_1 = iE$ to $Z_2 = iD$ exists with a parametric
representation $Z = Z(t), 0\le t \le 1 $ such  that the elements $Z_{\mu
  \nu} = Z_{\mu \nu}(t)$ have continuous derivatives with respect to 
$t$.\pageoriginale  We shall show that such a line is unique. Also, in
the course 
of the proof we explicitly determine what $Z(t)$ is. Then the
existence of a geodesic line with the desired properties is immediate
- in fact, the curve represented by $Z(t)$ which we have explicitly
determined is the desired geodesic line. 
\end{proof}

We shall denote the differential coefficient with respect to $t$ as\break
$\dfrac{d}{dt} (*)  = (\dot{\ast})$ By means of (\ref{eq128}) and
(\ref{eq133}) we have   
\begin{align*}
s(Z_1 ,Z_2) & = \int^{Z_2}_{Z_1} (d s^2 )^{1/2} =
\int^1_0 \big \{ \sigma (Z)^{-1} \bar{Z} Y^{-1} \big\}^{1/2} dt \\ 
& =  \int^1_0 \big \{ \sigma (Y)^{-1} {Z} Y^{-1} \bar{Z}\big\}^{1/2}
dt \\ 
& = \int^1_0 \big \{ \sigma (Y^{-1}x)^2 + \sigma (Y^{-1} y)^2
\big\}^{1/2} dt\\ 
& = \int^1_0 \big \{ \sigma (Y^{-1}x)^2 + \sigma (F^{-1}\dot{F})^2 +\sigma
(G^{-1} \dot{G})^2 \\
& \qquad \qquad + 2 \sigma(G^{-1}) \dot{H}^1 F \dot{H} ) \big\}^{1/2} dt\\ 
& \ge  \int^1_0 \big \{ \sigma (F^{-1} \dot{F})^2 + \sigma (G^{-1}
\dot{G})^2 \big\} dt	 
\end{align*}

We claim that the last inequality is actually a equality. For
otherwise, in the equation $Z_{(t)} = \begin{pmatrix}F(t) & C \\ 0 &
  G(t) \end{pmatrix}$ we will have a curve joining $Z_1 $ and $Z_2$
whose length will be actually 	 
$$
\int^1_0 \{ \sigma (F^{-1} \dot{F})^2 + \sigma (G^{-1} \dot{G})^2  \} dt <
s(Z_1, Z_2) 
$$
contradicting our assumption. But then, the equality can hold in the
above when and only when  
\begin{equation*}
\dot{X} = 0 = \dot{H}\tag{135}\label{eq135}
\end{equation*}
 
 We therefore conclude that (\ref{eq135}) holds identically in $t$,
 which means  that\pageoriginale  $X \equiv 0 \equiv H$ as $ X(0) = 0 =
 H(0)$ then the parametric representation of our geodesic line is
 given by 
 \begin{equation*}
Z(t) =  i \begin{pmatrix} F(t) &  0 \\  0 &  G(t) \end{pmatrix}
\tag{136}\label{eq136} 
 \end{equation*} 
 where $F=F^{(r)}$, $G^{(n -r)}$ and $r$ is arbitrary with $1 \le r <
 n$. According to the permissible cases $r = 1, 2,\ldots, n-1$ the same
 geodesic line will admit of ($n-1$) formally different
 representations so that it follows from (\ref{eq136}) that $Z(t)$ is
 necessarily a diagonal matrix $Z(t) = i (\delta_{\mu\nu} y_{\mu\nu}
 (t))$ Consequently we obtain  
 $$
 s(Z_1 , Z_2) = \int^1_0  \sqrt{\sum^k_{\nu = 1}
   \mathscr{Y}^2_{\nu \nu}}\mathscr{Y}^{-2}_{\nu \nu} dt = \int^1_0
 \sqrt{\sum^n_{\nu = 1} \mathscr{Y}^{2}_{\nu \nu}} dt 
 $$
 putting $n_{\mu \nu} = \log \mathscr{Y}_{\nu \nu}$. In the $n$
 dimension Euclidean space with $n_{\nu \nu}, \nu = 1,2, \ldots n$
 as rectangular cartesian coordinates, $s(Z_1 , Z_2)$ then
 represent the Euclidean length of our curve. But in this case we know
 that the curve of shortest length is the straight line segment
 joining the points corresponding to $Z_1$, $Z_2$. Hence we conclude
 that  
 $$
 \log \mathscr{Y}_{\nu \nu}(t) = \eta_{\nu \nu} (t) = t \log d_2 \; \;
 (\nu = 1,2, n) 0 \le t \le 1. 
 $$
 
 In other words we have the parametric representation of the geodesic
 line joining $Z_1 = i E$, $Z_2 = iD = i(\delta_{\mu \nu}
 d_\nu)$  
  as
 $$
 Z(t) = i (\delta_{\mu \nu} \; d^t_\nu) , 0\le t \le ) 
 $$
 
 Now we compute the characteristic roots $\lambda^2_\nu$ of $(iE, iD)$ 
 \begin{align*}
\varrho (i E , iD)  & =(D - E)(D + E)^{-1} (D - E)(D +
E)^{-1 } \\ 
& = \big(\delta_{\mu \nu} (\frac{d_\nu -1}{d_\nu +1})^2 \big)
 \end{align*} 
 so that\pageoriginale  $\lambda^2_\nu = (\dfrac{d_\nu -1}{d_\nu
   +1})^2$. Consequently 
 $\pm \lambda_\nu = \dfrac{d_\nu -1}{d_\nu +1}$ which means that
 $\big( \log \dfrac{1 +\lambda \nu}{1-\lambda \nu})^2 \big) = (\log d_\nu)^2$. 
 
 We know that $s(Z_1 ,Z_2)$ is the Euclidean length of the
 segment joining $(0,0,\ldots 0)$ and $(\log d_1, \ldots \log d_n)$. 
 \begin{align*}
\text{ Hence } s (Z_1 , Z_2) & =\bigg ( \sum^n_{\nu =
  1}(\log d_\nu)^2 \bigg) ^{1/2}\\ 
& = \bigg\{ \sum^n_{\nu = 1}(\log  \frac{1 + \lambda_\nu}{1 -
  \lambda_\nu} )^2 \bigg\} ^{1/2} 
 \end{align*} 
 and the proof is complete.
 
 Having thus obtained the 'shortest distance' $s(Z_1 ,Z_2)$
 between any two points $Z_1 , Z_2$ we wish to remark that the
 \textit{symplectic sphere} defined as the set of points $Z$ with $Z
 \in \mathscr{Y} , s(Z , Z_0) \le r -Z_0$ being a fixed point
 in $\mathscr{Y}-$ is a compact set. Certainly we can assume that $Z_0
 = iE$. 
 
 Then
 \begin{equation*}
s(Z , iE) = \bigg\{ \sum^n_{\nu = 1} (\log \frac{1 +
  \lambda_\nu}{1 - \lambda_\nu})^2 \bigg\}^{1/2} \le r \tag{137}\label{eq137} 
\end{equation*}
where $\lambda^2_\nu$ are the characteristic roots of $\rho(Z, i E)$. But 
$$
\varrho (Z , iE) =(Z - iE) (Z + iE)^{-1} (\bar{Z} + iE) (\bar{Z} -
iE)^{-1} = W \bar{W} 
$$
where $W$ is the point the in the Generalised unit circle which
corresponds to $Z$ under the usual mapping and we know from  (\ref{eq26})
that $W \bar{W} < E$. We therefore\pageoriginale  conclude that
$\lambda^2_\nu < 1, 
\nu = 1, \ldots r$. From (\ref{eq137}) it is clear that $\lambda_\nu$ cannot be
arbitrarily close to 1 so that we should have $\lambda^2_\nu \le  1 -
\delta < 1$ for some $\delta > 0$. Then $W \bar{W} \le (1 - \delta) E$
and the set of $W's$ consistent with this inequality is
clearly compact. Therefore the same is true of the corresponding $Z$
set too and hence also of our symplectic sphere. 

We add here one more result for future reference, viz.

\setcounter{lem}{10}
\begin{lem}\label{chap7:lem11}%lemm 11
 Given a point $Z_0 \in \mathscr{Y}$ there are at most a
  finite number of modular substitution which have $Z_o$  as a
  fixed point. 
\end{lem}

\begin{proof}%proo 11
We first observe that the modular group is discrete and hence also
countable. Let $M \neq \pm E$ be any modular substitution. Then the
equation $M < Z > = Z$ define for a given $M$ a complex analytic
manifold of dimension less than $\dfrac{n (n+1)}{2}$. Consider now all
analytic manifolds of this kind, viz those formed by the fixed points
of a given modular substitution. In view of the above fact, given any
point $Z_o \in \mathscr{Y}$ there always exist points $Z$ in any
neighbourhood of $Z_o$ with $M <Z> \neq Z$ for any $M \in M_n$, $M \neq
\pm E$. If now $M_k <Z_0> = Z_0$ for an infinity of $\mathfrak{K}'s$,
 say $\mathfrak{K} = 1,2 \ldots , and M_{\mathfrak{K}} \neq
\pm M_l \mathfrak{K} \neq \ell$ then choosing $Z$ with $M <Z> \neq Z$
for any $M$ different from $\pm E $ we have $M^{-1}_\mathfrak{K} <Z>
\neq Z$ for $\ell \neq \mathfrak{K}$ i.e. $M_\mathfrak{K}<Z> \neq M_\ell
<Z>$ for $ \neq \mathfrak{K}$. On the other hand,
$\mathscr{S}(M_\mathfrak{K} <Z>, Z_0) = \mathscr{S}(M_\mathfrak{K}<Z>,
M_\mathfrak{K}<Z_\nu>) = \mathscr{S}(Z, Z_0)$. Consequently the points
$M_\mathfrak{K} <Z>$, $\mathfrak{K} = 1,2$, are all distinct and lie on a
compact subset of $\mathscr{Y}$ so that they have at least one limit
point contradicting Lemma \ref{chap4:lem8}. We therefore conclude that
$M_\mathfrak{K}$ are identical after a certain stage and the Lemma is
proved. 
\end{proof}

We\pageoriginale  are now on the look out for the invariant volume
element of the 
symplectic geometry. We make a few preliminary remarks. Let $R =
(r_{\mu\nu})$ denote a variable $n -$ rowed symmetric matrix and
$\Omega = (\omega_{\mu \nu})$ a $n \times n$ square matrix. Let $S =
(s_{\mu \nu}) = R [\Omega]$ and let
$\dfrac{\partial(S)}{\partial(R)}$ denote the functional determinant
of the $n(n + 1)/2$ independent linear functions $\mathscr{S}_{\mu
  \nu} (p \le \nu)$ with respect to the independent variables $r _{\mu
  \nu(\mu \le \nu)}$. Then $\dfrac{\partial(S)}{\partial(R)} \neq C$
if and only if the mapping $R \rightarrow S$ is $1-1$. But this
mapping is $1-1$ when and only when $|\Omega| \neq 0$. For, while the
one way implication is trivial, viz. when $|\Omega| \neq 0$ the mapping
$R \rightarrow S$ is actually invertible as $R = S[\Omega^{-1}]$,
to realize the converse, we argue that if $|\Omega| = 0$ there exists
a row $XXXXX$ with $W \Omega = 0$. The choice of has $R = \nu \nu$
leads to $R [Q] = \Omega ' R \Omega = 0$ without $R$ being zero,
verifying that the mapping $R \rightarrow S = R[\Omega]$ is not
$1-1$. It therefore follows that $\dfrac{\partial (S)}{\partial(R)} = 0$
when and only when $|\Omega| = 0$. In other words, considered as
polynomials in the $n^2$ variables $\Omega_{\mu \nu},
\dfrac{\partial(S)}{\partial(R)}$  and $|\Omega|$ have the same
zeros. Since $|\Omega|$ is an irreducible polynomial in these $n^2$
variable, we conclude by means of a well known algebraic result that
$\dfrac{\partial(S)}{\partial(R)} = C|\Omega|^{n+1}$ with a constant
$C \neq 0$. The index $(n +1)$ in the right side is suggested by a
comparison of the degrees of both sides in $\omega_{\mu \nu }$. The
special choice $\Omega = E$ yields to the determination $C =1$. Thus
we have  
\begin{equation*}
\frac{\partial(S)}{\partial(R)} = |\Omega|^{n+1} \text{ for } S =
R[\Omega]. \tag{138}\label{eq138} 
\end{equation*}

Let now $M = \begin{pmatrix} A & B \\ C & D \end{pmatrix}  \in Z
\mathscr{Y}$ and consider the symplectic substitution $Z \rightarrow
Z^* = M < Z >$. One of the relations $122$ gives  
$$
Z^*_2 - Z^*_1 = (Z_1 C' + D' )^{-1} (Z_2 - Z_1)(CZ_2 + D)^{-1}
$$ \pageoriginale 
and it then follows that 
$$
dZ^* = (ZC' + D)^{-1} dZ (C Z + D)^{-1})
$$

The computation of the functional determinant $\dfrac{\partial
  (Z^*)}{\partial(Z)}$ requires only the knowledge of the linear
relation between $dZ$ and $dZ^*$ given by the last formula. By means
of (\ref{eq138}) we then get 
\begin{equation*}
\frac{\partial (Z^*)}{\partial(Z)} = (C + D)^{-n-1} \tag{139}\label{eq139}
\end{equation*}

Decomposing $Z$, $Z^*$, $\bar{Z}$, ${\bar{Z}}^*$ into their real and
imaginary parts we get 
\begin{align*}
Z &= X + iY  , Z^* = X^* + iY^* \\
\bar{Z}& = X + iY , {\bar{Z}}^* = X^* + iY^* 
\end{align*}
and it is immediate that 
$$
\frac{ \partial (Z ,\bar{Z})} {\partial (X ,Y)} = \frac{ \partial (Z^* 
  ,{{\bar{Z}}^*})} {\partial (X^* ,Y^*)} 
$$

From $Z^* = M <Z>$ and ${\bar{Z}}^\ast = M <\bar{Z}>$ we then obtain
using (\ref{eq139}) and (\ref{eq72}) that  
\begin{align*}
\frac{ \partial (X^* ,Y^*)} {\partial (X ,Y)} & = \frac{ \partial (X^*
  ,Y^*)} {\partial (Z^* ,Z^*)} \quad \frac{ \partial (Z^* ,{\bar{Z}}^*)}
     {\partial (Z ,\bar{Z})} \quad \frac{ \partial (Z ,\bar{Z})}
     {\partial (X ,Y)} \\ 
& = \frac{ \partial (Z^* ,\bar{Z})} {\partial (Z ,\bar{Z})} = \frac{
       \partial (Z ,\bar{Z^*})} {\partial (Z)}^* \\ 
& = || C Z + \nu ||^{2 r 2} = (\frac{Z^*}{Z})
\end{align*}

It\pageoriginale  now follows that the volume element
\begin{align*}
d v  = |y|^{n-1} [dx][dy] \tag{140}\label{eq140} \\
\text{with} \begin{cases} [dx] = \pi_{\mu \le \nu} dr_{\mu \nu} \\
 [dy] =  \pi_{\mu \le \nu} dy_{\mu \nu}\end{cases} 
\end{align*}
is invariant relative to symplectic substitutions.

We supplement our above results with the following two lemmas.

\begin{lem}\label{chap7:lem12}%lemma 12
If $Z$, $Z^* \in \mathscr{Y}_n$ and $Z_1, Z^*_1 \in \mathscr{Y}_r$
denote the matrices which arise from $Z$, $Z^*$ by deleting their last
$(n-2)$ rows and columns, where $r \le n$ then  
\begin{equation*}
s(Z_1 , Z^*_1) \le s(Z , Z^*) \tag{141}\label{eq141}
\end{equation*}
\end{lem}

\begin{proof}%proof 0
Since we know that the geodesic lines are uniquely determined, it
clearly suffices to prove the corresponding inequality for $ds$, viz.
$ds_1 \le d s$ in an obvious notation. In other
words we need only show that 
$$
\sigma (y^{-1}_1 dy_1)^2 + \sigma (x_1^2  d x_1)^2 \le \sigma (y^{-1}
dy)^2 + \sigma(y^{-1} dx)^2 
$$
where $Z = Z^{(n)} = x + iy$ and $Z_1= Z^{(r)}_1 = x_1 + iy_1$. We use
the parametric representation we had for positive matrices to write $
y = \begin{pmatrix}y_1^*
  \\ x^* \end{pmatrix}\begin{bmatrix} \begin{pmatrix} E & H \\ 0 
   & E \end{pmatrix} \end{bmatrix}$ and appeal to (\ref{eq133}) to infer that
$\sigma (y^{-1} dy)^2 \ge \sigma(y^{-1_1} dx_1)^2$ 
\end{proof}

Since $X = \begin{pmatrix}x_1^* \\ x^* \end{pmatrix}$ we also have, as
in the above case, $\sigma (y^{-1} dx)^2 \ge \sigma(y^{-1}_1 dx_1 )^2$  

\setcounter{pageoriginal}{100}
The\pageoriginale  desired result is now immediate.

\begin{lem}\label{chap7:lem13}%lemma 13
If $Z , Z^* \in y_n$ and $Z_1 = X_1 + y_1$, $Z^*_1 = X^*_1 +
LY^*_1$ be as in the previous Lemma, and if $\mathscr{S}(Z ,
Z^*) \le \varrho$ then there exist positive constants $M_\nu =
M_\nu (\varrho \nu ) \nu = 1,2$ with  $\dfrac{1}{M_1} \le
\dfrac{\sigma (Y^*_1)}{\sigma (Y_1)} \le M_1$, $\dfrac{1}{M_2} \le
\dfrac{|Y^*_1|}{|Y_1|} \le M_2$. 
\end{lem}

\begin{proof}
Since the condition $s(Z ,Z^*) \le \varrho$ implies by Lemma
\ref{chap7:lem12} that $s(Z_1 , Z^*_1) \le \varrho$ we need only consider
the case $r = n $. In view of the interchangeability of $Y$ and $Y^*$
it clearly suffices to prove one part of each of the two sets of
inequalities. Let then $W$ be an orthogonal matrix $(W W' = E)$ so
determined that $y^* [W] = D = (\delta_{\mu \nu}, d_\nu), d_\nu > c \nu
= 1,2, \ldots n$. Let $\tilde{Z} = (Z - X^*) [W D^{-1}] = \tilde{X} +
$ and $\tilde{Z}^* = (Z^* - X^*) [W D^{-1}] = i y^* [WD^{-1}] =
iE$. Since $s$ is invariant under symplectic substitutions,
we have $s(\tilde{Z} , iE) = s(\tilde{Z},
\tilde{Z}^*) = s(Z,Z^*) \le \varrho$ 

In other words $\tilde{Z}$ belongs to a symplectic sphere which we know
is a compact subset of $\mathscr{Y}_n$ and consequently 
\begin{equation*}
\sigma(\tilde{Y}) \le m_1 (n , \varrho), \frac{1}{m_2} \le (\tilde{y}) 
\le m_2 (n , \varrho) \tag{142}\label{eq142} 
\end{equation*}
with suitable constants $M_\nu \; \nu = 1 ,2$. Also $|\tilde{Y}| = |y| W 
d^{-1}| = |y| D^{-1,2} = |y| Y^*$. Then (\ref{eq142}) implies that  
$$
\frac{1}{M_2} \le |y| |y^*|^{-1} \le M_2  
$$

Further, if $y [W] = H= (h_{\mu \nu})$ then 
\begin{align*}
\sigma (\tilde{Y}) & = \sigma(H [D^{-1}]) = \sum^n_{\nu =1}
\frac{h_{\nu \nu}}{d^2_\nu} \\ 
& \ge \frac{\sum^n_{\nu = 1} h_{\nu \nu }}{\sum^n_{\nu =1} d^2_\nu} =
\frac{\sigma(H)}{\sigma(Y^*)} = \frac {\sigma (y)}{\sigma(y^*)}. 
\end{align*}\pageoriginale  
(\ref{eq142}) again implies now that $\sigma(\gamma) \sigma (y^*)^{-1} \le m_1
(n, \varrho)$ and as remarked earlier, the interchange of $y$ and
$y^*$ leads to the other half of our inequality. The lemma is thus
established. 
\end{proof}

