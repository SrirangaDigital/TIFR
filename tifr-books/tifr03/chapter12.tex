 
\chapter{The field of modular Functions}%Chap 12

We shall\pageoriginale need the following generalization of Lemma
\ref{chap9:lem16}, viz. 

\setcounter{lem}{16}
\begin{lem}\label{chap12:lem17}%Lem 17
Let $Z=X-iy \in \mathscr{Y}_n, \sigma(xx') \leq m_1,\sigma(y^{-1})\leq
m_2$. Then 
\begin{equation*}
||CZ + D|| \geq \in_0 || Ci + D||\tag{193}\label{eq193}
\end{equation*}
with a certain positive constant $\in_n = \in_0(n, m, m_2)$,
where $(C \; D)$ is the second matrix row of an arbitrary
  symplectic is matrix 
\end{lem}

\begin{proof}
We prove the lemma in stages.
\end{proof}

First we show that if $\lambda_1, \lambda_2,\ldots, \lambda_n$ is a
set of real numbers with $0 < \lambda_1 \leq \lambda_2 \leq \cdots
\leq \lambda_n$ and $R^{(n, m)}=(r_{\mu, \nu}=(\mathscr{W}_1
\mathscr{W}_2 \cdots \mathscr{W}_m))$, $S^{(n, m)}= (\mathscr{S}_{\mu
  \nu})=(\sigma_1 \sigma_2 \cdots \sigma_m)$ with $m \leq n$ be real
matrices such that $s_{\mu\nu}=\lambda_{\mu} r_{\mu \nu}$ then 
\begin{align*}
& (\lambda_{n-m+1}\lambda_{n-m+2}\cdots \lambda_n)^{-2}|\mathscr{P}'_\mu
  \mathscr{P}_\nu | \leq | \mathscr{W}'_\mu
  \mathscr{W}_\nu|\\ 
& \qquad \qquad \qquad \leq (\lambda_1,
  \lambda_2,\ldots,\lambda_n)^{-2}| \mathscr{P}'_\mu \mathscr{P}_\nu  |, \mu,
  \nu=1,2,\ldots,m   \tag{194}\label{eq194}  
\end{align*}

By a well known development of a determinant, we have
\begin{align*}
|\lambda_\mu \mathscr{W}_\nu | &= \sum_{1 \leq \rho_1< \rho_2 < \cdots
  < \rho_m < n}\begin{vmatrix} r_{s, i} \cdots r_{s_2, m} \\ \cdots
  \cdots \cdots \\ r_{s,r}\cdots r_{s_m,n}\end{vmatrix}^2\\ 
 &=\sum_{1 \leq \rho_1< \rho_2 < \cdots < \rho_m <
  r}(\lambda_{\rho_1}\lambda_{\rho_2},\ldots
\lambda_{\rho_n})\begin{vmatrix} \mathscr{S}_{s, i} \cdots
  \mathscr{S}_{s_2, m} \\ \cdots \cdots \cdots
  \\ \mathscr{S}_{s,r}\cdots \mathscr{S}_{s_m,n}\end{vmatrix}^2\\ 
\end{align*}

 Also
\begin{align*}
 |\mathscr{P}'_\mu \mathscr{P}_\nu|&= \sum_{1 \leq \rho_1< \rho_2 <
   \cdots < \rho_m < 
   n}\begin{vmatrix} \mathscr{S}_{s, i} \cdots \mathscr{S}_{s_2, m}
   \\ \cdots \cdots \cdots \\ \mathscr{S}_{s,r}\cdots
   \mathscr{S}_{s_m,n}\end{vmatrix}^2 
\end{align*}
(\ref{eq194}) is now immediate.

We next\pageoriginale prove that if $y=y^{(n)}>0, T=T^{(n)}=T'$  real,
and $\sigma (y^{-1}) \leq m_2$, then  
\begin{equation*}
||T+iy || \geq \varepsilon_1 ||T+LE ||, \in_1(n,m_\nu)>0
\tag{195}\label{eq195} 
\end{equation*}

Determine an orthogonal matrix $W$ such that $|y[W]=D^2$ with

 $D=(\delta_{\mu \nu}\lambda_\nu),0 < \lambda_1 \leq \lambda_2 \leq
\lambda_n$ and set 
 
 \noindent
$ T[W]=S=(\mathscr{S}_{\mu, \nu}), Q=S[D^{-1}]=(q_{\mu, \nu})$ Then

\noindent
$|Y|=|D|^2=(\lambda_1\lambda_2 \cdots \lambda_n)^2$ Also
$\mathscr{S}_{\mu, \nu}= \lambda_\mu \lambda_\nu q_{\mu \nu}$ 

\noindent
and
\begin{align*}
||T+iy||^2 &=||T[W]+iy[W]||^2 \qquad=|| S+iD^2 ||^2\\
&=||Q+iE||^2|D|^4\qquad \quad=||Q+iE||^2 |Y|^2\\
&=|Q+ iE ||Q-iE||Y|^2  = |Q'Q+E| |Y|^2\tag{196}\label{eq196}
\end{align*}

Let $Q=(\mathscr{U}_1\mathscr{U}_2\cdots \mathscr{U}_n)S=(*************)$ and
introduce $R=(\mathscr{W}_1 \mathscr{W}_2\cdots \mathscr{W}_n)=(r_{\mu
  \nu})$ where $\mathscr{U}_2=\dfrac{1}{\lambda_\nu}\mathscr{W}_\nu,
\nu=1,2,\ldots n$ Then we have $\mathscr{S}_{\mu, \nu}=\lambda_\mu
\lambda_\nu q_{\mu \nu}=\lambda_{\mu}r_{\mu \nu}$ so that the result
(\ref{eq194}) is now applicable (with the obvious changes). Then 
\begin{align*}
1Q'Q+C1 &=1+\sum_{\nu=1}^{n} \sum_{\mu_1<\mu_2<  \mu_2}
|M'_{\mu_\mathscr{S}}n_{\mu_\mathfrak{K}} |\\ 
&=1+\sum_{\nu=1}^{n}\sum_{\mu_1<\cdots
  <\mu_2}(\lambda_{\mu_1}\lambda_{\mu_2}
\lambda_{\mu_2})^{-2}1\mathscr{W}'_{\mu_2}\mathscr{W}'_{\mu_\mathfrak{K}}\\ 
&\geq 1+\sum_{\nu=1}^{n}(\lambda_{n-\nu+1}\cdots \lambda_n)^{-2}
\sum_{\mu < \cdots \mu_[\nu]}|1 \mu_{\mu_g}\mathscr{W}_{\mu_\mathfrak{K}}|\\ 
&\geq 1+ \sum_{\nu=1}^{n}(\lambda_{n-\nu+1 }\cdots \lambda_n)^{-4}
\sum_{\mu < \cdots \mu_2}| \mathscr{P}_{\mu_g}
\mathscr{P}_{\mu_\mathfrak{K}}|\tag{197} \label{eq197} 
\end{align*}

Since by assumption $\sigma(y^{-1}) \leq m_2$ all characteristic roots
$\lambda_nu^2$ of $y$ have positive lower bound which depends 
only\pageoriginale on $m_2$ so that for a suitable constant
$\varepsilon_1 = \varepsilon _1 (n_1 m_2)$ we have    
\begin{equation*}
(\lambda_1 \lambda_2 \ldots \lambda_m \mu)^4 > \varepsilon_1^2, \mu =
  0= 1,  2,, n  \tag{198}\label{eq198} 
\end{equation*}
The in view of (\ref{eq196}) - (\ref{eq198}) we have 
\begin{align*}
\| T + i y \| ^2 & = |Q'Q + i E | |Y| ^2 \\
& = | Q'Q + E | ( \lambda_1 \lambda_2 \ldots \lambda_n )^r \\
& \qquad \geq  \varepsilon_1^2 (1 + \sum\limits_{\mu=2}^{n} \sum
\limits_{m_1>, <m_\lambda} | y _{m_y}, y _{m_\mathfrak{K}}|)\\
&= \varepsilon_1^2 | s's+ E | = \varepsilon_1^2 | s+ i E || s - i E |\\
&= \varepsilon_1^2 |T + ; E | |\ T - i E | \varepsilon_1 ^2 || = T + i
E ||^2
\end{align*}

Hence $|| T + i Y || > \varepsilon_4 i E ||$  as desired. 
We now contend that under the assumption $\sigma (xx') \le m_1$, where 
$x = x^{(n)} = x'$ and $s = s^(n) = s'$ real, we have 
\begin{equation*}
|| x + s + i E || \ge  \sum_2 || s + i E || \tag{199}\label{eq199}
\end{equation*}
with a certain positive constant $\sum_2 = \sum_2(n_1 m_2) $.
This only means that the quotient $|| x+ s + i E || /  || s + i E ||$
has a positive lower bound and this is certainly ensured if we show
that $L = \log || \times + s + CE||$ - $\log || S + \ell E ||$ is
bounded. By the mean value theorem of the differential calculus we
have 
$$
L = \sum_{1 \leq \mu \leq \nu \leq n } \xi_{\mu \nu } \frac{\partial 
  \log || S^* + i E ||} { \partial \mathscr{S}_{\mu \nu }^* }
\bigg/_{S^* = S + \mu x } 
$$
with $\times = (x_{\mu \nu })$, $S^* = (\mathscr{S}_{\mu \nu }^* )$ and
$0 <\vartheta < 1 $. Also\pageoriginale the assumption $\sigma (x x')
\le  m_2$ entitles us to conclude  that $\pm \chi_{\mu \nu} <
\mathscr{C}, \mu, \nu = 1,2 \cdots n $. Hence   
	\begin{align*}
=|L|  \le \mathscr{C} \sum_{1 \le \nu \le \nu \le n} \bigg |
\frac{\partial \log \|  S^* +  i E \|}{\partial s^*_{\mu \nu}} \bigg
|_{S^* = S+ \vartheta \chi}\\ 
= \mathscr{C} \sum_{1 \le \nu \le \nu \le n} \bigg | \frac{\partial
  \log |  S^* +  i E |}{\partial s^*_{\mu \nu}} \bigg |_{S^* = S+
  \vartheta \chi}\\ 
	= \mathscr{C} \sum_{1 \le \nu \le \nu \le n} \bigg | Re
        \bigg\{|S^* +iE^{-1}\frac{\partial |  S^* +  i E |} {\partial
          s^*_{\mu \nu}}\bigg\} \bigg |_{S^* = S+ \vartheta \chi}\\ 
=\le \mathscr{C} \sum_{1 \le \nu \le \nu \le n} \bigg ||S^* +iE1^{-1}
\frac{\partial |  S^* +  i E |}{\partial s^*_{\mu \nu}} \bigg |_{S^* =
  S+ \vartheta \chi} \tag{200}\label{eq200} 
	\end{align*}

	Let $W = (\omega_{\mu \nu}$ be an orthogonal matrix such that  
	$$
	W' S^* w=  D = (\delta_{\mu \nu }d \nu), d_\nu) \ge o.  
	$$

	Then $W' (S^* +i E) w= D +i E$, or equivalently
	\begin{equation*}
	(S^* i E)^{-1} = W (D + i E)^{-1} W' \tag{201}\label{eq201} 
	\end{equation*}

	Let $(S^*+ iE)^{-1}= (\omega_{\mu \nu})$ and observe that the
        diagonal elements of $D+i E$  have the lower bound 1 as they
        are of the form  $1/d_{\nu +i}$, $d_\nu \ge o$. Then
        $\omega_{\mu \nu}= \sum\limits_{\varrho} \omega_{\mu S}
        \bar{d_3 +i} \omega_{\varrho \nu}$ from (\ref{eq201}) so that,
        considering absolute values, $(\omega_{\mu\nu}
        \bar{\omega}_{\mu\nu})^{1/2} \leq \sum^n_{\rho=1}
        |\omega_{\mu\rho} \omega_{\mu \rho}|$ 
and consequently $\pm
        \omega_{\mu \nu} \le 1$. From (\ref{eq200}) we have then 
	$$
	|L| \le \mathscr{C} \sum_{\mu \nu=1}^{n} (\omega_{\mu \nu} 
        \bar{\omega}_{\mu \nu})^{1/2} \le \mathscr{C} n^3 
	$$

	In other words $L$ is bounded and this is precisely what we
        wanted to show. 

We are\pageoriginale now in position to face the main Lemma. We first
prove it in 
the case when $|c| \neq o$. What we want to show is then that $\| CZ
+D \| \ge \varepsilon_o \| C i +D\|$ or equivalently $\| Z+C^{-1}D\|
\ge \varepsilon_o \| i E +  C^{-1} D \|$. With $Z=X+i Y$, this is the
same as $\| \chi +  C^{-1} D+ i Y \|  \ge \varepsilon_o \| C^{-1}D+ i
E \|$. In other words $\| T +  i Y \| \ge \varepsilon_o \|  S + i E
\|$ with  $T = \chi +  C^{-1} D$  and $S= C^{-1}D$. An appeal to
(\ref{eq195}) which is by way legitimate, yields $\| T +  i Y \| \ge
\varepsilon_i \| T+ i E \|$ and then in view of (\ref{eq199}), $\| T +  i Y
\|  \varepsilon_1 \|  T+ i E \| \ge \varepsilon_1 \varepsilon_2 \|S +
i E \|$. Setting $\varepsilon_1 \varepsilon_2 = \varepsilon_o$ we have
the desired result. 

The general case can be reduced to the above case as follows. If $(C
D)$ is the second matrix row $\lambda$ the symplectic matrix $M$, then
$(D,C)$ is the second matrix row of $M_1 = M \begin{pmatrix} o & -E
  \\ E & o \end{pmatrix}$ and $M_1$ is again  symplectic matrix. Hence
in particular $| DZ-C1 \neq o$ an consequently $| DS +C |$ does not
vanish identically in $S= S'$. We can therefore  determine a sequence
of real  symmetric matrices $S_{\mathfrak{K}}$ such that
$\lim\limits_{\mathfrak{K} \to \infty} S_{\mathfrak{K}}= (o)$ while
$|DS_{\mathfrak{K}+C| \neq o}$. Then, in view of the truth of our Lemma
for a  special  case we established above, 

	$\| (DS_{\mathfrak{K}}+  C) Z+D \| \ge \varepsilon_o \|
(DS_{\mathfrak{K}}+C) i+ D \|$. 

	Proceeding to the limit as $\mathfrak{K} \to \infty$ we get the
        desired result. This completes the proof  of Lemma \ref{chap12:lem17}. 

	With these preliminaries we proved to define the field of
        modular functions. It looks  quite  plausible and natural 
        for one to define a modular function  as a meromorphic
        function in $\mathscr{Y}_n$ which is invariant under the group
        of modular substitutions and which behave a specified manner as
        we approach the boundary of $\rho_n$. 

	It is\pageoriginale also true that with this definition, the
        modular 
        functions constitute a field. But unfortunately, it has not
        been found possible to determine  the structure of this
        function field expect in the classical case $n=1$. With  a
        view to getting the  appropriate theorem on their algebraic
        dependence we are forced  to give a (possibly more
        restrictive) definition as follows:- A \textit{modular
          function of degree n} is a quotient of two modular forms of
        degree $n$ both of which have the same weight. It is obvious
        that such a function  is meromorphic in $\mathscr{Y}_n$ and is
        invariant under modular  substitutions. However, it is not
        known  whether  the converse is also true. We shall be
        subsequently concerned with  determining the structure  of the
        field of all  modular function of degree $n$. Specifically we
        shall prove the  existence  of  $\dfrac{n(n+1)}{2}$ modular
        functions of degree $n$ which are algebraically independent. 

	Let $f_{\mathfrak{K}}$ be a modular form degree $n$ and weight
        $\mathfrak{K}$ not vanishing identically. Such a  form exists
        for  $\mathfrak{K} 
        \equiv 0(\mod 2)$, $\mathfrak{K} > n+1$, as for instance $f_\mu (z) =
        g(z.o) =g _\mathfrak{K}(Z,o)$ is one such. We introduce the
        series 
	\begin{equation*}
	M(\lambda )= \sum_{\{ C,D\}} ( \lambda- f_\mathfrak{K}o) |C
        Z+D|^{\mathfrak{K}})^{-1} \tag{202}\label{eq202}  
	\end{equation*}
	where $\lambda$ denotes a complex variable, and $(C,D)$ runs
        over a complete set of non-associated coprime symmetric
        pairs. Since a Poincare' series converges uniformly in every
        compact  subset of $\mathscr{Y}$, $M (\lambda)$ represents  a
        meromorphic function of $\lambda$ with  all poles,
        simple. Assume $f_\mathfrak{K} \rho_{\mathfrak{K}} \neq 0$ in
        a neighbourhood of 0.  

Then\pageoriginale in such that neighbourhood we have
\begin{align*}
-M (\lambda) &= \sum_{\{ C,D\}} \mathfrak{f} (z)^{-1} |cz + D|^{-\mathfrak{K}}
(1- \frac{\lambda} {\mathfrak{f}_\mathfrak{K} (z) |2+ D|^\mathfrak{K}})^{-1}\\ 
& =  \sum_{\{ C,D\}}  \sum_{m=1}^{\infty} \lambda^{m-1}
(\mathfrak{f}_\mathfrak{K} (z) |cz + D |^\mathfrak{K})^{-n}\\ 
&= - \sum_{m=1}^{\infty} (\mathfrak{f}_\mathfrak{K} (2))^{-m} \{  \sum_{\{ C,D\}}
\frac{1}{|cz + D |^\mathfrak{K}}\}\lambda^{m-1}\\ 
& = \sum_{m=1}^{\infty} \varphi_m (z) \lambda^{m-1}
\end{align*}
with $\varphi _m (z) = g_{m \mathfrak{K}} (z , e) / \mathfrak{f} (2)^m$.

$\varphi _m (z)$ is clearly invariant under the modular substitutions
so that two points $Z$ and $Z_1$ which are equivalent with respect to
$M_n$ define the same function $M(\lambda)$. A partial converse is also
true, viz :- 

\begin{lem}\label{chap12:lem18}%lemma 18
 Provided $Z$ does not lie on certain algebraic surfaces which
  have the property that any compact subset of $\mathscr{Y}$ is
  intersected by only a finite number of them, the equations $\varphi
  _m (z) = \varphi _m (z_1)$ for $m = 1,2, \ldots $ imply the
  equivalence of $Z$ and $Z_1$ with respect to $M$. 
\end{lem}

\begin{proof} 
Due to the invariance property of $\varphi_m (z)$, we may assume that
$z, z, \varepsilon 5$, and in view of the proviso in our lemma, we can
further assume that $z \in$ interior $\mathfrak{F}$ as the boundary of
$\mathfrak{F}$ is composed of a finite number of algebraic
surfaces. We then have   
\begin{equation*}
|CZ + D | > L , \qquad |CZ + D | \geq 1 \tag{203}\label{eq203} 
\end{equation*}
for $\{ C,D\} \neq \{ 0 , E\}$ . The poles of $M(\lambda)$ are clearly
at the\pageoriginale points $\mathfrak{f} (\pi) |CZ + D|^\mathfrak{K}$
and then the first of 
the inequalities (\ref{eq203}) implies that among these is a point nearest
to the origin, viz. $\mathfrak{f}_\mathfrak{K} (Z)$ corresponding to the choice
$\{ C,D \} = \{ 0 , E \}$ and the residue of $M(\lambda)$ at this  
point is 
\begin{enumerate}
\item Due to one of our assumptions, $M(\lambda)$ is an invariant
  function of $\lambda$ for change of $Z$ to $Z_1$ so that the poles
  of the same $M(\lambda)$ (corresponding to $Z$) are also given
  by. $\mathfrak{f} (I_1) |CZ _1 + D |^\mathfrak{K}$. A consideration of the 
  existence of a pole of $M(\lambda)$ nearest to the origin at
  $\mathfrak{f}_\mathfrak{K} (Z)$ with residue $1$ implies now by means of
  (\ref{eq203}) that 
$$
|CZ_1 + D | > 1 \text{ for } \{C,D \} \neq \{0, E \} \text{ and that }
\mathfrak{f}_\mathfrak{K} (Z_1) = \mathfrak{f}_\mathfrak{K} (z). 
$$
\end{enumerate}


As the poles permute among themselves by a change of $Z$ to $Z_1$ it
follows that 
$$
\mathfrak{f}_\mathfrak{K} (Z) |CZ + D|^\mathfrak{K} =
\mathfrak{f}_\mathfrak{K} (Z_1) | C_1 Z_1 + D| ^\mathfrak{K} 
$$
and consequently 
\begin{equation*}
|CZ + D | = \in |C_1 Z_1 + D_1 |. \tag{204}\label{eq204} 
\end{equation*}
for a certain permutation $\{ C_1 , D_1 \}$ of all classes $\{ C, D \} 
\neq \{ 0 , E \}$ where $\in$ denotes a $\mathfrak{K}^{\rm th}$
root of unity which may depend on $C$, $D$, $Z$, $Z_1$. 
\end{proof}

If $\mathscr{L}$ is any compact subset of $\mathscr{F}$ we can show
that, $Z$ being arbitrary in $\mathscr{L}$ and $Z_1$ arbitrary in 
$\mathscr{F}$, there exists only a finite number of classes $\{ C_1 ,
D_1 \}$ consistent with (\ref{eq204}) for any given class $\{ C, D \}$. 

The proof is as follows:

Let $\mathfrak{K}$ be so determined that $|CZ + D | \leq \mathfrak{K}$
for $Z \in XXXX $ 
fixed. With appropriate choices of the constants $m_1$, $m_2$, a domain
of the type $\sigma (x x') \leq \pi m_1 \mathscr{F} (y^{-1}) \leq 1_2$
contains (in an\pageoriginale obvious notation) so that $Z_1$ can be
assumed to lie 
in one of these domains. Then by Lemma \ref{chap12:lem17},  
$$
\mathfrak{K} \geq || CZ + D || = \varepsilon || c, z_i + D_1 || \geq
\varepsilon \varepsilon_o || C_1 i + D_1 ||  
$$
where $\varepsilon = \varepsilon_o (n , m _1 m_2) > o$. In other
words, the determinant $|C_1 i + D_1|$ is bounded (by $\mathfrak{K} /
\varepsilon_o$) so that, by an earlier argument (p. 44), the number
of possible choices for $\{ C_1 , D_1\}$ is finite. This settles our
claim. 

Assume now that rank $C= 1$. Then $|CZ + D | = CZ [g] +d$ where $g$ is
a primitive column and $(c,d)$ is a pair of coprime integers. Also from
Lemma \ref{chap1:lem1}, we shall have $| C, Z_1 + D_1 | = |C_2 Z_2 +
D_2 |$ with 
$Z_z = Z_1 [\theta _z] \theta_z = \theta_z^{n,m}$, primitive and $C_2
= C^n_z | C_2 | \neq 0$, $r$ denoting the rank of $C_1$. From (\ref{eq204})
we then have 
\begin{equation*}
CZ [\eta] + d = \in |C_2 Z_z + D_2| \tag{205}\label{eq205}  
\end{equation*}
where to repeat our assumptions, $\eta$ denotes an arbitrary primitive
column, $(c,d)$ denotes a pair of coprime integers, $(C_2^r , D_2^r)$
denotes a coprime symmetric pair of matrices with $|C_2| \neq 0$ and
finally $Z_2$ stands for $Z_1$ transformed with a  certain primitive
matrix $\theta^{(n,m)}_{z}$. Keeping $\eta$ fixed, we choose $(c,d ) =
(1,1)$ and $(0,1)$ successively and obtain from (\ref{eq205}), the relations 
\begin{align*}
Z [\eta ] + 1 & = \in_2|c_2 z_2 + D_2 |, z_2 = z_1
[\theta^{(n,m)}_{2}]\\ 
 Z [\eta ] + 1& = \in_3 |c_3 z_3 + D_3 |, z_3 = z_1
 [\theta^{(n,m)}_{3}] \tag*{$(205)'$}\label{eq205'}  
\end{align*}
which by subtraction yields
\begin{equation}
\in_2 |C_2 Z_2 + D_2 |- \in_3 C_3 Z_3 + D_3 =1
\tag{206}\label{eq206}  
\end{equation}
where\pageoriginale $C_3 = C_3^{(s)}$, $D_3 = D_3 ^{(s)}$ form a pair
of coprime symmetric matrices with $|C_3| \neq 0$, $Q_3 ^{(n,s)}$
denotes a primitive matrix and $\in_2$, $\in_3$ stand for $k^{\rm th}$
roots of unity.  

Let $n_{\mu , \nu}$ be the column having 1 in the $\mu^{th}$ and
$\nu^{th}$ places and $0$ elsewhere. Then 
\begin{equation*}
Z [\eta_{\mu , \nu}] = 
\begin{cases}
Z _{\mu \mu} \qquad \text{ if } \mu = \nu\\
Z_{\mu , \mu} + Z_{\nu \nu}+ 2Z_{\mu \mu} \text{ if } \mu \neq \nu
\tag{207}\label{eq207}  
\end{cases}
\end{equation*}

Choosing\label{p165} $\eta_{\mu, \nu}$ in the place of $\eta$ in
\ref{eq205'}, we find 
from the relation $Z [\eta] = \varepsilon_3 |C_3 Z_3 + D_3 |$ that all
the elements of $Z$ can be represented as polynomials in the elements
of $Z_1$ and the coefficients of these polynomials belong to a finite
set  of numbers. To realize the last part, we need only observe that
the columns $\eta_{\mu \nu}$ are finite in number and so are the pairs
$(C_3 , D_3)$ as the pairs $(C_1 , D_1)$ are so for a given
$(C,D)$. Consider now the equation (\ref{eq206}), viz. 
$$
\in_2 |C_2 Z_2 + D_2 | - \in _3 |C_3 Z_3 + D_3 | =1 
$$ 

This is an equation in $Z_1$ and all the coefficients belong to a
finite set of numbers which means that the number of possible such
that equations is finite. We can then assume that above holds
identically in $Z_1$ as in the alternative case, the finite number of
(non identical) relations in $Z$ imply a finite number of relations
among the $Z _{\mu \nu }'s$ which are representable as polynomials in
the $\dfrac{n (n+1)}{2}$ independent elements of $Z_1$ which in its
turn  means that $Z$ lies on a certain finite set of algebraic
surfaces- a proviso already assumed\pageoriginale in the statement of
Lemma. Rewriting the above equation then, in the form 
\begin{equation*}
|Z_2 + P_2 | - \alpha |Z_3 + P_3| = \beta \tag{208}\label{eq208} 
 \end{equation*}
 with $\alpha \beta \neq 0$, $P_2 = C_2 ^{-1} D_2 $ and $P_3 =
 C_3^{-1} D_3$, and comparing the terms of the highest degree in this
 identity in $Z_1$ we get $r = s$ and $|Z_2| = \alpha |Z_3|$. The
 last is again an identity in $Z_1$. Replacing $Z_1$ by $Z_1 [\mu_1]$
 where $\mu_1$ is unimodular, and then transforming $Z_2$. $Z_3$ with
 a unimodular matrix $\mu_2$ it results that 
 \begin{equation*}
|Z _1 [\mu_1 Q_2 \mu_2 ]| = \alpha |Z_1 [\mu _1 Q_3 \mu_2] |
\tag{209}\label{eq209}  
 \end{equation*} 
 
 We can assume $\mathcal{U}_1 , \mathcal{U}_2$ to be so chosen that
 $\mathcal{U}_1 Q_2 \mathcal{U}_ 
 2 = \begin{pmatrix} E^{(r)}\\ 0 \end{pmatrix}$
 
 Writing $\mathcal{U}_1 Q_3  U_2$ analogously as $U_1 Q_3 U_2
 = \begin{pmatrix}  R \\ S \end{pmatrix}$ with a square matrix $R = R
 ^{(r)}$, and choosing $Z_1 = \begin{pmatrix} E^{(r)} 0 \\ 0
   0 \end{pmatrix}$ we obtain from (\ref{eq208}) that $\alpha |R|^2 =1$. In
 particular this means that $|R| \neq 0$ and then we can determine a
 non-singular $\nu$ such that 
 $$
 R'^{-1}S' S R^{-1} = \nu' H \nu , \nu = \nu^{(r)}, H = (\delta_{\mu ,
   \nu} h_{\nu}). 
 $$
 
 The choice $Z_1 = \begin{pmatrix} \lambda \nu' \nu  0 \\ 0
   E \end{pmatrix}$ with a variable $\lambda$ leads by means of
(\ref{eq209}) to the relations 
 \begin{align*}
\lambda^r |\nu |^2 & = | Z_1 | u_1 Q_2 u_2 | = \lambda |Z_1 [ u_1 Q_3 u_2] |\\
& = \mathcal{L}|\lambda R' \nu' \nu R + S S | = |\lambda \nu' \nu +
R^{-1} S S R^{-1}|\\ 
&= |\lambda \nu' \nu + \nu' H \nu | = |\lambda E + H || \nu|^2 =
\prod_{\nu=1}^{r} (\lambda + h_{\nu}) |\nu|^2  
 \end{align*} 
 
 Comparing\pageoriginale the extremes which provide an identity in
 $\lambda$, we deduce that $H = 0$ and therefore that $S= 0$.   
 
 With $Z^* = Z_1 [\mathcal{U}_1 Q_2 \mathcal{U}_2]$, (\ref{eq208})  yields 
 $$
 |Z' + P_2 [u_2] | - |Z' + P_3 [u_2 R^{-1}] | =  \beta |u_2|^2 =
 \beta_1 \neq 0 
 $$
 identically in $Z^*$. Replacing $Z^*$ by $Z* + P_2 [u_2]$ we obtain 
 \begin{equation*}
|Z* | - |Z* - P_1| = \beta_1 \tag{210}\label{eq210} 
 \end{equation*} 
 identically in $Z^*$ with a certain symmetric matrix $P$. 
 
 Setting $P = W' \begin{pmatrix} E^{(t)} & 0 \\ 0 & 0 \end{pmatrix}
 \mathfrak{K}$  and $Z^* = \lambda W' W$ with $|W| \neq 0$ and a
 variable $\lambda$, (\ref{eq210}) yields 
 $$
 \lambda^r |w|^2 - (\lambda + 1)^t \lambda^{r-t} |w|^2 = \beta_1 
 $$
 identically $\lambda$. 0f necessity then, $ r = t = 1$ and
 consequently $R^{(r)} = \pm 1$. Since we know that $\alpha |R^2| = L
 $, this means that $\alpha = 1$. Also $C_2 = C_2 ^{(r)}$ and $D_2=
 D_2^{(r)}$ reduce to pure numbers and from (\ref{eq206}). 
 $$
 \in_2 |C_2 Z_2 + D_2 | \in_3 |C_3 Z_3 + D_3 | = 1.
 $$
 
 Considering the terms of the highest degree in the elements of $Z_1$
 we get that 
 $$
 \in _2 C_2 Z_2 |\theta_2| - \in_3 C_3 Z_1 [\theta_3]
 = 0 
 $$
 
 Also 
 $$
 u_1 \theta_3 u_3 = \begin{pmatrix} R \\ S \end{pmatrix}
 = \begin{pmatrix} R \\ 0 \end{pmatrix} = \begin{pmatrix} E
   \\ 0 \end{pmatrix} R = \pm \begin{pmatrix} E \\ 0 \end{pmatrix} =
 \pm u_1 Q_2 u_2 
 $$
 so that $Q_3 = \pm Q_2$. The last relation now gives 
 \begin{equation*}
\varepsilon_2 C_2 - \varepsilon_3 C_3  \tag{211}\label{eq211} 
 \end{equation*} 
 
 Choosing\pageoriginale $Z_1 = 0$ in (\ref{eq206}), (\ref{eq210})
 implies that 
 \begin{equation*}
\varepsilon_2 D_2 - \varepsilon_3 D_3 = 1 \tag{212}\label{eq212} 
 \end{equation*} 
 
 Since $C_\nu$, $D_\nu$ are integers $(\mathcal{U}= 2,3)$ it follows
 from (\ref{eq211}) - (\ref{eq212}) that $\in_2$, $\in_3$ are
 rational numbers and being roots of unity are therefore equal to $\pm
 1$. We can assume without loss of generality that $\in_2$,
 $\in_3 = 1$. It is now to be inferred from (\ref{eq206})- (\ref{eq207}) that
 the elements $Z_{\mu \nu}$ of $Z$ are linear functions of the
 elements $S_{\mu \nu} $ of $Z_1$ with coefficients all belonging to a
 finite set of numbers, say 
 \begin{equation*}
Z_{\mu \nu} = a_{\mu \nu} + \sum_{\varrho , \sigma} a _{\mu \nu ,
  \varrho \sigma} \zeta_{\varrho \sigma} \tag{213}\label{eq213}  
 \end{equation*}
 
 Since $Z = Z'$ and $Z_1 = Z'_1$ we may assume in the above that
 $a_{\mu \nu} = a _{\nu \mu}$ and $a_{\mu \nu \sigma \varrho} = a_{\mu
   \nu, \varrho \sigma}$. Choosing in (\ref{eq205}), $c=1$, $d=0$ and $\eta =
 (q_1 q_2 \cdots q_n)'$ from among a given finite set of primitive
 columns, and setting $(C_2 , D_2) = (C_1, d_1)$ a pair of coprime
 integers (we have already shown that $C_2$, $D_2$ are pure numbers)
 and $Z_2 = Z_3 [\mu_2]$ with $Q_2 = (p_1 p_2 , p_n)$,  the $p_i' s$
 being coprime we conclude from (\ref{eq213}) that 
 \begin{equation*}
\sum_{\mu \nu, \varrho \sigma} a_{\mu \nu, \varrho \sigma}
\zeta_{\zeta \sigma} q_{\mu} q_{\nu}  + \sum_{\mu, \nu} a_{\mu, \nu}
q_{\mu} q_{\nu} = C_1 \sum_{\varrho \sigma} \zeta_{\varrho ,
  \sigma}p_{\varrho} p_{\sigma} + d_1 \tag{214}\label{eq214}  
 \end{equation*} 
 
 The possible choices for $c_1, c_2, p_\nu (\nu = 1, 2, \ldots n)$
 are finite in number so that, by 
 an earlier argument (p. \pageref{p165}) we can assume, in view of the proviso
 in our Lemma that (\ref{eq214}) is an identity in the $\zeta's$. Then
 (\ref{eq214}) implies that 
 \begin{equation*}
\sum_{\mu} \sum_{\mu, \nu} a_{\mu \nu, \varrho \sigma} q_{\mu} q_{\nu}
= C_1 p_{\varrho} p_{\sigma} \tag{215}\label{eq215}  
 \end{equation*}
 and,\pageoriginale $C_1 \neq 0$ as we know that rank $C_2 =1$. This
 shows that the 
 symmetric matrix $(Q_{\varrho \sigma}) = ( \sum _{\mu , \nu} a_{\mu
   \nu, \varrho \sigma} q_{\mu} q_{\nu})$ has the rank 1 for each
 primitive column $g$. Taking a sufficiently large number of $g$, $s$,
 the truth of our above statement for all these will imply that
 $(Q_{\varrho \sigma})$ has rank $1$ identically in $g$ and
 consequently all sub-determinants of two rows and two columns
 vanish. In particular, for all $\varrho$, $\sigma$ 
 \begin{equation*}
Q_{\varrho \varrho} Q_{\sigma \sigma} = 0 \tag{216}\label{eq216} 
  \end{equation*}  
  and all  these are algebraic conditions.
  
  We shall say polynomials $F$, $G$ to be equivalent, written $ F \sim
  G$, if they differ only by a constant factor. Let  us  assume that
  $Q_{11} \neq 0$ and $Q_{\mathfrak{K} \lambda} \neq 0$. Then from (\ref{eq216}),
  $Q_{\mathfrak{K}k} a_{\lambda \lambda} \neq 0$, $Q_{1 \mathfrak{K}}
  \neq 0$ and $Q_{1 
    \lambda} \neq 0$. Thus all $Q_{\mathscr{H}\mathfrak{K}}$, $a_{\lambda \lambda}$,
  $a_{1 \mathfrak{K}}$, $a_{1 \lambda}$ differ from 0. 
  
  Two  cases arise now.
  
\medskip
\noindent{\textbf{Case. i:}}
Let $Q_{11}$ have no square divisor. Then from (\ref{eq216}) $Q_{11} Q_{H
  x}- \varrho^2 _{1\mathfrak{K}}=0$ and it immediate that 
$$
Q_{11} \sim Q_{1\mathfrak{K}} \sim Q_{H X} \sim Q_{\lambda \lambda} \sim Q_{H
  \lambda}. 
$$

This means that $Q_{\varrho \sigma} = C_{\varrho} C_{\sigma} Q$ with
$Q = \sum_{\mu , \nu} q_\mu q_{\nu}$ and then, from (\ref{eq213}), $Q_{\mu
  \nu, \varrho \sigma} = C_{\varrho} C_{\sigma}$ and $Z_{\mu \nu} =
a_{\mu \nu} + C_{\mu, \nu} \xi$ with $\xi = \sum_{\varrho \sigma}
C_{\varrho} C_{\sigma} \zeta_{\varrho \sigma}$ Since $a_{\mu \nu},
C_{\mu \nu}$ belong to a finite set of numbers, these equations
represent a finite number of algebraic surfaces, and the assumptions
of our theorem permit us to exclude this case 

\medskip
\noindent{\textbf{Case. ii:}}
Let us assume then that $Q_{11}$ assume then that $Q_{11}$ is not
square free. Then $Q_{11} \sim L^2_1$,\pageoriginale where $L$,
denotes a linear from. Then using (\ref{eq216}) $Q_{n n}\sim
L^2_\mathfrak{K}$, $Q_{\lambda \lambda} \sim  
L_{\lambda}^{2}$ and $Q_{n \lambda} \sim L_{n} L_{\lambda}$ where the
$L's$ are again linear forms. By suitably normalizing these forms we
can assume that 
$$
a_{n \lambda} = L_{n}L_{\lambda}.
$$

Let $L_\mathfrak{K} = \sum\limits_{\mu} \ell_{\mu n}q_{\mu}$ (The $b'$ s may be
complex constants). 


Then
$$
Q_{n \lambda} = \sum_{\mu \nu} \ell_{\mu n} \ell _{\nu
  \lambda}q_{\mu}q_{\nu} = \sum_{\mu \nu} a_{\mu \nu, n
  \lambda}q_{\mu} q_{\nu} 
$$
and being an identity in the $q'$s, this gives
\begin{equation*}
a_{\mu \nu , n \lambda} = \ell _{\mu \nu } \ell _{\nu \lambda}
\tag{217}\label{eq217}  
\end{equation*}

In other words
\begin{equation*}
Z = Z_1 [B] + A \tag{218}\label{eq218} 
\end{equation*}
with certain matrices $A = (Q_{\mu \nu})$, $B = (\ell _{\mu \nu})$ which
belong to a finite set. It remains only to show that $B$ is unimodular
and $A$ is integral and then the Lemma would have been established.  

Let us choose $y = y_{\mu \nu}$ in (\ref{eq215}) and then we obtain 
$$
a_{\mu \nu, \varrho \sigma} + a_{\mu \nu, \varrho \sigma} + Z a_{\mu
  \nu, \varrho \sigma} = C_{\mu \nu} p_{\mu \nu, \varrho }p_{\mu \nu,
  \sigma} (\mu \neq \nu) 
$$
and $a_{\mu \nu, \varrho \sigma} = C_{\mu} p_{\mu} p_{\mu \sigma}$
with integers $C_{\mu \nu}$, $p_{\mu \nu \varrho}$, $C_{\mu}$, $p_{\mu
  \sigma}$ where $C_{\mu \nu } C_{\mu} \neq 0$. For $\sigma = \varrho$
these yield in particular, by means of (\ref{eq217}), 
$$
(\ell _{\mu \sigma} + \ell _{\nu \sigma})^2 = C_{\mu \nu}p^2 _{\mu \nu
  \sigma}(\mu \neq \nu) 
$$
and $\ell ^2 _{\mu \sigma} = C_{\mu} p^2_{\mu \sigma}$.

Hence\pageoriginale it follows that
\begin{equation*}
\sqrt{C_{\mu \nu}} p_{\mu \nu \sigma} =  \sqrt{C_{\mu}}p_{\mu \sigma}
+ \sqrt{C_{\nu}}p_{\nu \sigma} \mu \neq \nu  \tag{219}\label{eq219}  
\end{equation*}
and squaring,
\begin{equation*}
C_{\mu \nu} p^2_{\mu \nu \sigma} = C_{\mu} p^2_{\mu \sigma} + C_{\nu}
p^2_{\nu \sigma} + 2 \sqrt{C_{\mu}C_{\nu}} p_{\nu \sigma}, \mu \neq
\nu . \tag{220}\label{eq220}  
\end{equation*}

If at least  one product $p_{\mu \sigma} p_{\nu \sigma} \neq 0 (\mu
\neq \nu)$ then $C_{\mu} C_{\nu}$ is clearly the square of a rational
number. This is also true if $p_{\mu \sigma} p_{\nu \sigma} = 0$ for
all $\mu$, $\nu (\mu \neq \nu)$ as then $C_{\mu \nu} C_[\mu]$ and
$C_{\mu \nu} C_[\nu]$ are square of rational numbers as is seen from
(\ref{eq220}). In particular the numbers $c_1 , c_2, \ldots c_n$ have all
the same sign. Since $Z \in \mathscr{Y}$ and $A$ is a real symmetric
matrix, $|Z - A| \neq 0$ and consequently from (\ref{eq218}), $|B| \neq
0$. It turns by means of (\ref{eq218}) that $Z_1$ also belongs to a compact
subset of $\mathfrak{f}$ as $Z$ does by assumption. This means that
all that we have proved above for $Z$ hold also for $Z_1$ Since we can
assume $Z_1 = Z [B'] + A^+$ where $B* = B^{-1}$ in view of the proviso
in our lemma, this in particular means that a representation of the
kind $B = (\pi_{\mu} p_{\mu \nu})$ can be found for $B^{-1}$ too, say
$B^{-1} = \sqrt{C_{\mu}} p'_{\mu \nu}$. Since $B B^{-1} = E$, from the
above representation for $B, B^{-1}$ we have 
$$
\sqrt{C_1 C_2 C_n C'_1 C'_2  C_n} |p_{\mu \nu}| | p'_{\mu \nu}| = 1 
$$
and as all the $c'$s and $p'$s are integers this gives
$$
C_1 C_2 \cdots C_n C"_1 C'_2 \cdots C'_n = 1 , |p_{\mu \nu}|= \pm 1.
$$

We know\pageoriginale that all $c'$s are of the same sign and
therefore $c_1 = c_2 = \cdots \cdots = c_n = 1$.  

It is now immediate that $B$ is unimodular.

We now specialize $(C, D)$ in (\ref{eq204}) to be $E , 0$ respectively and
obtain from (\ref{eq218}) that 
$$
\varepsilon |C_1 Z_1 + D_1| = |Z| = |Z _1 [B] + A | = |Z_1 + A
            [B^{-1}] | 
$$

This again can be assumed to be an identity in $Z_1$ and we conclude
that $\varepsilon |C_1| = 1$, so that $|C_1| = \pm 1$. Then 
$$
|Z_1 + A [B^{-1}] | = \varepsilon |C_1 Z_1 + D_1 | = Z_1 + C_1^{-1} D_1 |
$$

Replacing $Z_1 + C_1^{-1} D_1$ in the above by $Z_1$ we get  
$$
|Z_1| = |Z_1 + C_1^{-1} D_1 - A[B^{-1}] |
$$
identically in $Z_1$ and as a consequence, $A[B^{-1}] = C_1^{-1} D_1$
or\break $A = C_1^{-1}D_1 [B_1] C_1$ is a unimodular matrix as we have shown
earlier and this means that $C_1^{-1} D_1$ and consequently $A$ are
integral. Thus, from (\ref{eq218}), $Z = Z_1 [u] + A$ with a unimodular $u$ and
integral $A$, in other words $Z$ and $Z_1$, are equivalent with regard
to $M$. By assumption $Z$ is an interior point of $\mathfrak{f}$ and
$Z_1 \in \xi$ and therefore $Z$ and $Z_1$ coincide. This proves Lemma
\ref{chap12:lem18}. 

\medskip
We now prove that any maximal set of algebraically independent
functions in the sequence $\{ \varphi _{\mu} (Z) \}, \nu = 1,2, \ldots
$, consists of $\dfrac{n(n+1)}{2}$ elements. Let such that a set
comprise of the functions $\lambda_q (z) = \varphi_{\nu_{a}} (z) a, =
1,2, \ldots$. We first show that the $\lambda_a ' s$ are at least $n
(n+1)/2$ in number. For if their number is a $q < n (n+1)/2$, then
$\lambda \lambda \cdots \lambda_q \varphi_m$ are algebraically
dependent for each m. In other words, there\pageoriginale exists a
polynomial $F_m 
(t)$ - which we may assume to be irreducible of one variable over the
integrity domain generated by $\lambda_a (z) , a = 1,2, \ldots q$,
such that $F_m (\varphi_m) =0$, $m = 1,2, \ldots$ identically in
$Z$. These polynomials being separable, their discriminants $D_m (m =
1,2, \ldots)$ are all polynomials in $\lambda_1, \lambda_2, \ldots
\lambda_q$ not vanishing identically. By Lemma \ref{chap12:lem18}, it
is possible 
to find  a bounded domain $\mathscr{G} \subset \mathfrak{f}$ viz. an open
connected non-empty subset of $\mathfrak{f}$ such that the condition
$\varphi_m (z) = \varphi_m (z_1)$, $m \geq 1, Z \in \mathscr{G} , z ,\in
\mathfrak{f} $ will automatically imply that $Z = Z_1$. We now
construct a sequence of sub-domains of $\mathscr{G}$ as follows. Let
$\mathscr{G}_1 
\subset \mathscr{G}$ be so determined that $D_1 \neq 0$ for any $Z \in
\mathscr{G}_1$. Then let $\mathscr{G}_2 \subset \mathscr{G}_1$ be such
that $D_2 \neq 0$ for any $z 
\in \mathscr{G}_2$. This may be continued and there exists at least
one point $Z_o$ which belongs to every $\mathscr{G}_m$ so that
$D_\mathfrak{K} (z_o) \neq 0$ for any  $\mathfrak{K}$. 

Consider now the equations
\begin{equation*}
\lambda_o (z) = \lambda_o (Z_o), a = 1,2, \ldots q, \tag{221}\label{eq221} 
\end{equation*}

These define an analytic manifold of complex dimension at least $n(n +
1)/ 2 - q > 0$. Thus in $\mathscr{G}$ there exists an analytic curve through
$Z_o$ on which (\ref{eq221}) is satisfied at  every point. It is then
immediate that functions $\lambda_a (z)$ are all constants on this
curve. This means as a consequence that the functions $\varphi_m (z)$
are also constants on this curve, For the polynomials $Z_m (t)$ have
all their coefficients on this curve and being separable their zeros
are all distinct. In a sufficiently small neighbourhood of $Z_o$ on
this curve we should then have $\varphi _m (z) = \varphi_m (z_o)$ for
every $m$ and by the choice of $\mathscr{G}$ which\pageoriginale
contains this curve we have $Z =Z_1$.In other words the curve reduces
to a point- an obvious contradiction to the assumption that the
dimension of the manifold defined by (\ref{eq221}) is
positive. Consequently the supposition $q < n (n + 1)/2$ is  not
tenable. We now show that the number of $\lambda 's$ cannot exceed $n
(n+1)/2$. For if it does, let $q > n (n+1)/2$ and consider the modular
forms $\mathfrak{f}_k$, $g_{\nu_a \mathfrak{K}} =
\mathfrak{f}^{\nu_a}_\mathfrak{K} \lambda_a$, $a = 1,2, \ldots
q$. These are at least $n (n+1)/2+2$ in number so that, by theorem
\ref{chap6:thm7}, these satisfy a  non trivial isobaric algebraic
relation   
$$
\sum C_{\mu_1 \mu_2 \cdots \mu_{q+1}} g^{\mu_{1} }_{\nu_{1} \mathfrak{K}}
g_{\nu_2 \mathfrak{K}}^{\mu_{2}} \cdots g^{\mu_q}_{\nu_{q} \mathfrak{K}}
\mathfrak{f}^{\mu_{q +1}}_{\mathfrak{K}} =0  
$$
with $\mu_i \geq 0$ and $\sum \mu_1 \nu_1 + \mu_2 \nu_2 + \mu_{q+1} =
\lambda$ ($\lambda$ being a fixed number). From the above relation it
is immediate that $\sum C_{\mu_1 \mu_2 \cdots \mu_{q+1}}
\lambda_1^{\mu_1} \lambda_2^{\mu_2}\cdots \lambda_q ^{\mu_q} =$ 
\begin{align*}
& = \sum C_{\mu_1 \mu_2 \cdots \mu_{q+1}} \bigg ( \frac{g_{\ni_1
      \mathfrak{K}}}{\mathfrak{f}^{\nu_1}_\mathfrak{K}} \bigg )^{\mu_1} \bigg (
  \frac{g_{\ni_2 \mathfrak{K}}}{\mathfrak{f}^{\nu_2}_\mathfrak{K}} \bigg )^{\mu_2} \cdots
  \bigg ( \frac{g_{\ni_q \mathfrak{K}}}{\mathfrak{f}^{\nu_q}_\mathfrak{K}} \bigg
  )^{\mu_q}\\ 
& = \frac{1}{\mathfrak{f}_{\mathfrak{K}}^{\lambda}} \bigg ( \sum C_{\mu_1 \mu_2
    \cdots \mu_{q+1}}  g_{\nu_{1} \mathfrak{K}}^{\mu_{1}} g_{\nu_2 \mathfrak{K}}^{\mu_2}
  g_{\nu_q }^{\mu_q} \mathfrak{f}_\mathfrak{K} ^{\mu_{q +1}}\bigg ) =0  
\end{align*}

In other words we have established a non trivial relation among the
$\lambda 's$ so that cannot be independent - thus providing a
contradiction. Thus $q \not \langle \dfrac{n(n+1)}{2}$ we conclude
that 
$$
q = \frac{n (n +1)}{2}
$$

We are now equipped with the relevant preliminaries needed to
establish the main result of this section viz. 

\setcounter{thm}{15}
\begin{thm}\label{chap12:thm16}%the 16
The\pageoriginale field of modular functions of degree $n$ is
isomorphic to 
  an algebraic function field of degree of transcendence
  $\dfrac{n(n+1)}{2}$. That is to say, every modular function of
  degree $n$ is a rational function of $\dfrac{n(n+1)}{2}+1$
  special modular functions. These functions are algebraically
  dependent but every $\dfrac{n(n+1)}{2}$ of them are independent. 
\end{thm}

\begin{proof}
Let $q = \dfrac{n(n+2)}{2}$ and let
$$
\mathfrak{K}_a (z) = \varphi_{\nu_a}(z) = g_{\nu_a \mathfrak{K}}(z)
(\mathfrak{f}_\mathfrak{K}(z))^{-\nu_a}, a= 1,2, \ldots ,q 
$$
with $\mathfrak{f}_\mathfrak{K} (z) = g_\mathfrak{K} (z,o)$ be a set
of algebraically 
independent modular functions. The existence of such a set has already
been shown. We first prove that the $(q +1)$ Eisenstein series $g_\mathfrak{K}
(z,o), g_{\nu_a \mathfrak{K}} (z,o)$ are algebraically independent. For, let 
\begin{equation*}
\sum^{m}_{\mu =0} L_{\mu}(z) =0 \tag{222}\label{eq222}
\end{equation*}
with 
$$
L_\mu (z) = \sum _{\mu _o \cdots \mu_q} C_{\mu_o \mu_1 \cdots
  \mu_q}^{(\mu)} g_\mathfrak{K}^{\mu_o} g_{\nu_q \mathfrak{K}}^{\mu_1} \cdots g_{\nu_q
  \mathfrak{K}}^{\mu_q}, \mu_c \geq 0,  
$$
$\mu + \mu_1 \nu_1 + \cdots \mu_q \nu_q = \mu$, be any algebraic
relation among the $g' s$. We wish to show that such a relation cannot
be non trivial. First we  shall show that $L_{\mu}(z) = 0$ for each
$\mu$. Applying the modular substitution $Z \to (AZ + B) (CZ +
D)^{-1}$ to (\ref{eq222}) we obtain  
\begin{equation*}
\sum_{\mu =0}^{m} |CZ + D |^{\mu | Z} L_{\mu} (z) =0 \tag{223}\label{eq223} 
\end{equation*}
identically in $Z$ for all symmetric coprime pairs $\{C,D\}$. We can
choose $C_\nu, D_{\nu_1} \nu = 1,2, \ldots m$ such that $|C_{\nu} Z +
D_nu |$ are all different. 
\end{proof}

Then\pageoriginale we shall have from (\ref{eq223}) a system of linear
equations in $L_\mu (Z)$ whose determinant is nonzero. This requires
that $L_\mu (Z) = 0$ for $\mu = 0 , 1,2, \ldots m$. Thus   
$$
\sum_{\mu_o, \mu_1 , \ldots \mu_q } C_{\mu_o, \mu_1 , \ldots \mu_q }
g_{\mathfrak{K}}^{\mu_o} g_{\nu_1 \mathfrak{K}}^{\mu_1} \cdots
g^{\mu_q}_{\nu_q \mathfrak{K}} =0 , \mu = 1,2, \ldots m 
$$
where 
$$
\mu_o + \sum^{q}_{q=1}\mu_c \nu_c = \mu 
$$

Dividing $L_{\mu}(Z) $ by $g_{\mathfrak{K}}^{\mu} =
g_{\mathfrak{K}}^{\mu_o }+ \sum_{i = 1 
}^{q} \mu_1 \nu_1$ the above yields  that\break $\sum C_{\mu_o, \mu_1 ,
  \ldots \mu_q }^{(\mathfrak{K})} \lambda_1^{\mu_1} \lambda_{2}^{\mu_2} \cdots
\lambda_q ^{\mu_q} =0$ and consequently $C_{\mu_o, \mu_1 , \ldots
  \mu_q }^{(\mu)} = 0 $ for all indices. Thus we have shown that $g_\mathfrak{K}
, g_{\nu_a \mathfrak{K}}, a =1 \cdots q$ are all independent.  

\medskip
Let now $\lambda (Z)$ be an arbitrary modular function and $\lambda
(Z) = \dfrac{\mathfrak{f}(Z)}{g (Z)}$ a representation of $\lambda$ as
the quotient two modular forms $f$, $g$ of the same weight
$\ell$. Since any $\dfrac{n (n+1)}{2} +2$ modular forms are
algebraically dependent by theorem \ref{chap6:thm7}, this is true in
particular of 
the two sets \break $\mathfrak{f}, g_\mathfrak{K} , g_{\nu_1
  \mathfrak{K}} \cdots g_{\nu_q \mathfrak{K}}$ and 
$g , g_\mathfrak{K}, g_{\nu_1 \mathfrak{K}}, \ldots g_{\nu_q
  \mathfrak{K}}$ and there exist nontrivial isobaric relations 
\begin{align*}
\varphi (\mathfrak{f}) & = \sum C_{\mu \nu \varrho_1 \cdots \varrho_q
} \mathfrak{f}^\mu g_{\mathfrak{K}}^{\mu} g^{\mu_1}_{\nu _q \mathfrak{K}}= 0 \\ 
\varphi (g) & = \sum d_{\mu \nu \varrho_1 \cdots \varrho_q }  g^{\mu}
g_{\mathfrak{K}}^{\nu}g_{\nu_q \mathfrak{K}}^{\mu_q} = 0  
\end{align*}
with all exponents non negative satisfying the condition 
$$
\mu \ell + (\nu + \rho_1 \nu_1 + \cdots + \rho_\nu \nu_\nu ) \mathfrak{K} = m
\ell \nu_1 \nu_2 \cdots \nu_\nu \mathfrak{K}^{q + l} 
$$
a certain\pageoriginale integer $m = m (n)$. It is immediate that  
$$
\mu \le m \nu_1 \nu_2 \cdots \nu_\nu \mathfrak{K}^{q + 1} 
$$
so that the polynomials $\varphi (t)$ and $\psi(t)$ are of bounded
degree. We can assume without loss of generality that $\psi(0) \neq 0$,
$\psi (0) \neq 0$. In $\varphi(t)$ we replace $t$ by $t  g$ and let
$\psi_1(8) = \varphi (t g)$. The resultant of $\varphi (g)$ and
$\psi_1(g)$ (as polynomials in 8) is a polynomial in $t$ denoted by
$\sigma (t)$ say. Since $\varphi (0) \neq 0$, $\psi (\varphi \neq 0)$ it
follows that $\sigma (0) \neq 0$, in other words, $\sigma$ does not
vanish identically. Further $\varphi (g) = 0$ and $\{\psi_1 (g)\}_{t =
  \lambda} = \varphi(f) = 0$ so that $\psi$ and $\psi_1$ have a common
zero at the point $\lambda$. Hence $\sigma$ has a zero too at this
point. The degree of $\sigma$ is clearly bounded (for varying
$\lambda$ ) as the degrees of $\varphi$ and $\psi$ are bounded. Also,
the coefficients of $\sigma$ are all isobaric polynomials of $g_\mathfrak{K},
g_{\nu_q rk}$, in other words, all of them are modular forms of the
same weight. It is clear then as in earlier contexts that, for a
suitable $\lambda, g^{-\lambda}_\mathfrak{K} \sigma (t) = \sigma_1$ has
coefficients all of which can be written as polynomials in $\chi_1$,
$\lambda_q$ and then $\sigma_1$ satisfies the following conditions:  
\begin{enumerate}[i)] 
\item the degree of $\sigma_1$ is bounded with respect to $\lambda$ 
\item $\sigma_1 \not\equiv 0$, and specifically, $\sigma_1 (0) \neq
  0$ 
\item $\sigma_1 (\lambda)= 0$. 
\end{enumerate}



Let now $\lambda = \lambda_o$ be so chosen that it is a zero of an
irreducible polynomial $F(t)$ which has all the above properties of
$\sigma1_1$ and whose degree in $t$ is maximal (with regard to
$\lambda$). The existence of $\lambda_o$ is easily proved. Then the $q
+ 1$ modular functions $\lambda_o$, $\lambda_1$, $\lambda_q$ generate the
field\pageoriginale of all modular functions. For if $K = R[\lambda_1, \ldots,
  \lambda_q]$ is the field generated by $\lambda_1$, $\ldots$, $\lambda_q$
and $K[\lambda_o]$ that generated by $\lambda_o, \lambda_1, \ldots
\lambda_q$ and if $\lambda$ is an arbitrary modular function then  
$$
K \subset K [\lambda_o] \subset K [\lambda_o, \lambda].
$$

All these are finite extensions over $K$ and $\lambda$ being algebraic
over $K$, the last is clearly a simple extension over $K$, say,
$K[\lambda_1]$. Then, unless $K[\lambda_o] = K[\lambda^*_1]$, the
irreducible equation over $K$ satisfied by $\lambda_1$ will have a
degree higher than that of the irreducible equation that $\lambda_o$
satisfies-a contradiction to the choice of $\lambda_o$. Hence $K [
  \lambda_o] = K[\lambda_o, \lambda^*]$ and consequently $\lambda^*$
is a rational function of $\lambda_o, \lambda, \ldots, \lambda_q$.
The proof of Theorem \ref{chap12:thm16} is now complete. 

