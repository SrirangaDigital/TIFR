\chapter{The Fundamental Domain of the Modular Group of Degree~n}%chap.4
 
The tools\pageoriginale now at our disposal enable us to construct a
fundamental 
domain in the generalised upper half plane for the  modular group
acting on it. We need a few  preliminaries.  
 
 We first prove that given any point $ Z \in \mathscr{Y} $ there
 exists only a finite number of classes $ \{ C, D \} $ of  coprime
 symmetric pairs  $C$, $D$ for  which the absolute value of $ \mid CZ
 + D \mid$, written as $\mid\mid CZ + D \mid\mid$, has a given upper
 bound $\mathfrak{K}$, i.e. such that $\mid\mid CZ + D\mid\mid \leq
 \mathfrak{K}$. Let $r$ denote the rank of $C$. We employ the
 parametric representation of $(C, D)$ given by Lemma
 \ref{chap1:lem1}. Then    
 \begin{equation*}
C = \mathcal{U}_1 
\begin{pmatrix} 
C_1 & 0 \\
 0 & 0
 \end{pmatrix}
\mathcal{U}'_2,  D = \mathcal{U}_1  
\begin{pmatrix}
 0_1 & 0 \\ 
0 &  E
 \end{pmatrix}
 \mathcal{U}^{-1}_{2} \tag{68}\label{eq68} 
 \end{equation*} 
where $ \mathcal{U}_2 = ( Q R )$, $Q = Q^{(n,r)} $

We have then 
\begin{align*} 
 &CZ + D = \mathcal{U}_1 \bigg\{
 \begin{pmatrix}
 e_1 & 0 \\ 
0 & 0 
\end{pmatrix} 
\mathcal{U}'_2 Z \mathcal{U}_2 + 
\begin{pmatrix}
   D_1 & 0 \\
 0 & E 
\end{pmatrix} \bigg\} 
\mathcal{U}^{-1}_2   &\text{ and } \\ 
 &\mathcal{U}'_2 Z \mathcal{U}_2 =
  \begin{pmatrix}
Q'  \\
 R'
\end{pmatrix}
 Z ( Q R ) =  
\begin{pmatrix}
 Z[ Q ] & Q' Z R\\
 R'Z Q & Z[ R ]
 \end{pmatrix}  
\end{align*}

Thus $ CZ + D  = \mathcal{U}_1
 \begin{pmatrix}
 C_1 Z [ Q ] +D_1 & * \\
 0  & E 
 \end{pmatrix} u^{-1}_2  $
 \hfill{ and  } 
\begin{align*}
\mid CZ + D \mid &= \mid \mathcal{U}_1 \mid  \mid \mathcal{U}_2
\mid^{-1} \mid C_1 Z[ Q ] + D_1 \mid   = \pm \mid C_1 Z [ Q ] + D_1
\mid \\ 
&= \pm \mid C_1 \mid  \mid Z[ Q ] + P \mid          \text{ where }  P
= C_1^{-1} D_1 
\end{align*}

We shall subsequently need this relation 
\begin{equation*}
 \mid C Z + D \mid = \pm \mid C_1 \mid \big| Z [ Q ] + P \mid
 \tag{69}\label{eq69} 
\end{equation*}

We may observe that $P$ is a rational symmetric matrix which uni\-quely
determines and is uniquely determined by the class $ \{ C_1, D_1\}$. 
For  if $ C^{-1}_0 D_0 = C^{-1}_1 D_1 = P $, then $ D_0 C'_1 = C_0
D'_1 $ which means that  
$\{ C_0, D_0 \}\break = \{C_1, D_1\}$,\pageoriginale while if $A$, $B$ is
any pair in the 
class $\{C_1, D_1\}$, then $A = \mathcal{U} C_1$, $B = \mathcal{U}D_1$,
for some unimodular $\mathcal{U}$, and $A^{-1}B = C^{-1}_1
D_1=P$. What is more, while $P = C^{-1}_1 D_1$ is rational symmetric
for any coprime symmetric pair $(C_1, D_1)$, conversely too, any
rational symmetric matrix $P$ is of the above form. We need to only
observe that in this case, we can choose unimodular matrices
$\mathcal{U}_3$, $\mathcal{U}_4$ such that 

$\mathcal{U}_3 P \mathcal{U}_4 =\bigg(\delta_{\mu
  \nu}\dfrac{a_\nu}{b_\nu}\bigg) $, viz. a diagonal matrix with the 
diagonal elements $a_{\nu / b_\nu}, \nu =1, 2, \ldots n$ where
$a_\nu$, $b_\nu$ are coprime integers with $\ell_\nu > O$ and then,
defining $C_1 = (\delta_{\mu \nu}b_\nu) \mathcal{U}^{-1}_{3}$, $D_1 =
(\delta_{\mu \nu}a_\nu) \mathcal{U}_4$, we have $P = C^{-1}D_1$. 

We now return to the representation (\ref{eq68}) of $C, D$. Let $X, Y$
denote the real and imaginary parts of $Z$. Since $Q$ in the above
representation can always be replaced by $Q \mathcal{U}_3$,
$\mathcal{U}_3$-unimodular, while preserving the form of (\ref{eq68}), we
can assume that $Y[Q]$ is reduced. Let then 
\begin{equation*}
\left.
\tag{70}\label{eq70} 
\begin{aligned}
S & = X[Q]+P\\
T & = Y[Q]
\end{aligned}
\right \}
\end{equation*}
and lat $F^{(r)}$ a real matrix with $T[F] = E, S[F]=H=(\delta_{\mu
  \nu}b_\nu)$. Then  $(S+iT) [F]=H+iE$ and $(S+iT)=(H+iE)[F^{-1}]$  
\begin{align*}
\text{ Now }|Z[Q]+P| & =|S+iT| = |T| \prod \limits ^{r}_{\nu=1}
(i+h_\nu)  \qquad \text{ and }\\ 
||Z[Q]+P||^2 & =|T|^2 \prod \limits^{r}_{\nu =1}(l+h^2_\nu). \qquad
\text{ Hence it follows that }\\ 
||  CZ +D ||^2 & =|C_1|^2 |T|^2 \prod \limits ^{r}_{\nu =1}(1+h^2_\nu)
\tag{71}\label{eq71}  
\end{align*}

By assumption $|| CZ +D||\leq \mathfrak{K}\qquad $; also, $C_1$ being
integral, $|C_1| >1$, and trivially $1+h^2_\nu > 1$. (\ref{eq71}) now
implies that $|T|$ is bounded. But $ T$ is reduced so that one of the
inequalities (\ref{eq47}-\ref{eq49}) will imply that 

$\prod\limits^{r}_{\nu =1}Y[\mathscr{Y}] < Q_1 | T | $ where we
denote $Q = \mathscr{Y}_1 \mathscr{Y}_2 \mathscr{Y}_n)$ 
Since each of the factor $Y [\mathscr{Y}_\nu]$ has a positive lower
bound, vis $Y[\mathscr{Y}\nu] \geq \lambda \mathscr{Y}'_\nu
\mathscr{Y}_\nu \geq \lambda > o$\pageoriginale where $\lambda $ denotes the
smallest characteristic root of $Y$, we conclude from the above that
each factor $Y[\mathscr{Y}_\nu]$  is bounded. As a consequence, the
$\mathscr{Y}_\nu'\mathscr{S}$ belong to a finite set of primitive
columns and there are only a finite number of possible choices for
$Q$. In particular, therefore, the elements of $T=Y[Q]$ are
bounded and since $T=F'^{-1}F^{-1}$ the same is true of the elements of
$F$ too. Also, the number of distinct $T's$ being finite, $|T|$ has a
positive lower bound and being an integer, $|C_1|^2 \geq 1$ so that we
infer from (\ref{eq71}) that the $h_\nu's$ and what is the same,
the matrix $H$ are bounded. But 


$S+ i T=(H +  i E) [\Gamma^{-1}]$ and we already know that $F^{-1}$ is
bounded. Hence $S+iT=Z[Q]+P$ is bounded and so also $P=S-\times[Q]$ Since
$|C_1|P$ is integral and $|C_1|$ is bounded as is seen from
the relation 
$$
|C_1|^2 |T|^2 \prod^r_{\nu=1}(1+h^2_\nu) < \mathfrak{K}
$$
it follows that the number of distinct $P's$ occurring here is
finite. We know however that distinct classes $\{C_1,D_1\}$ correspond
to distinct $P's$ so that the number of distinct classes $\{C_1,D_1\}$
occurring are finite. As we have shown already that the number of
classes $\{Q\}$ is also finite we conclude that the number of classes
$\{C,D\}$ in our discussion, viz. those which satisfy $|| CZ + D || <
\mathfrak{K}$ for a given $\mathfrak{K} > O$ is finite and this was what
we were after. 

Let $Y$, $Y^*$ be the imaginary parts of two points $Z$, $Z^* \in 
\mathscr{Y}$ which are equivalent with regard to $\mathcal{M}$. We
call $Z^*$ \textit{higher than} $Z$ if we have $|Y^*| > |Y|$. If
$Z^*=(AZ+B)(CZ+D)^{-1}$, then we have from (\ref{eq24}) that
$Y^*=(ZC'+D')^{-1}Y(C\bar{Z}+D)^{-1}$ so that  
\begin{equation*}
|Y^*|=|Y| \, \big|CZ+D \big|^{-2} \tag{72}\label{eq72}  
\end{equation*}


Thus\pageoriginale $Z^*$ is higher than $Z$ if and only if $||CZ + D||
\geq 1$ where 
we assume $Z^* = (AZ + B)(CZ + D)^{-1}$ As a consequence we have that
in each class of equivalent points there exist at least one highest
point. For, in the alternative case, there would exist a sequence of
equivalent points $Z_{\mathfrak{K}}$, 
$$
Z_\mathfrak{K} = X_\mathfrak{K} + i Y_\mathfrak{K}=(A_\mathfrak{K} Z_1 +
B_\mathfrak{K}) (C_\mathfrak{K} Z_1 + D_\mathfrak{K})^{-1} 
\begin{pmatrix}
  A_\mathfrak{K} B_\mathfrak{K} \\ 
C_\mathfrak{K} D_\mathfrak{K} \end{pmatrix} 
\in \mathcal{M}
$$
 such that $|Y_1 | <
|Y_2 | < \ldots ad \inf$. and this will imply in its turn by (\ref{eq72})
that 

$1 > || C_Z Z_1 +D_2 || > ||C_3 Z_1 +D_3 ||> \ldots ad \inf$. In other
words we end up with the conclusion that an infinite number of classes
$\{C_\mathfrak{K}, D_\mathfrak{K}\}$ have the property that $||
C_\mathfrak{K} Z_1 +D_\mathfrak{K} || < 1$ a contradiction to an earlier 
result. Now we state 

\setcounter{thm}{2}
\begin{thm}\label{chap4:thm3}%them 3
 The domain $\mathfrak{f}$ defined by the following inequalities
  represents a fundamental domain in $\mathscr{Y}$ with regard to
  $\mathcal{M}$. We denote $Z = x + i y$, $X = (x_{\mu
  \nu})$, $Y=(y_{\mu \nu})$ and then $Z \in \mathcal{F}$ is
defined by  
\begin{enumerate}[(i)]
\item $|| CZ +D || \geq 1$ for all coprime symmetric pairs $C$, $D$, 

\item $\left.
\begin{aligned} 
 Y [y_\mathfrak{K}] & \geq y_{\mathfrak{K} \mathfrak{K}},\\
y_{\mathfrak{K} \mathfrak{K}+1} & \geq 0,\\
& \mathfrak{K} =1,2, \ldots
\end{aligned}
\right)
\text{(Minkowski's reduction conditions)}
$
where $y_\mathfrak{K}$\break is an arbitrary integral column
  with its last $n-k+1$ elements coprime. 

\item $- \dfrac{1}{2} \leq x_{k \ell } \leq \dfrac{1}{2},
  \mathfrak{K},\ell =1,2, \ldots n$ 

Also $\mathcal{F}$ is a connected and closed set which is 
  bounded by a finite number of algebraic surfaces. 
\end{enumerate}
\end{thm}

\begin{proof}
We first\pageoriginale show that given any point $Z \in \mathscr{Y}$,
$\mathcal{F}$ 
contains an equivalent point $Z_1=M < Z >$. In fact, since we know
that any class of equivalent points contains a highest point, we can
assume that $Z$ is a highest point and then $|| CZ+D|| \geq 1$ for all
coprime symmetric pairs $C$, $D$. If now $M= \begin{pmatrix}\mathcal{U'}
  & S \mathcal{U}^{-1}\\0 & \mathcal{U}^{-1} \end{pmatrix}$ where
$\mathcal{U}$ is unimodular and $S$ an integral symmetric matrix to be
specified presently, we shall have  

$Z_1=Z[\mathcal{U}] + S, Y_1 = Y[\mathcal{U}] = \mathcal{U'}Y
\mathcal{U}$ and $X_1 = X [\mathcal{U}] + S$ where we assume $Z_1=X_1
+ i Y_1$. We can determine $\mathcal{U}$ such that
$Y_1=Y[\mathcal{U}]$ is reduced and then $S$ can be chosen such that
$X_1$ satisfies the last of the three conditions in the theorem. The
first condition is then automatically ensured as $| Y_1| = |Y|$ so
that $Z_1$ is a highest point as $Z$ is. The $Z_1$ thus determined
clearly serves. 
\end{proof} 	 

\medskip
Further, no two distinct interior points of $\mathcal{F}$ can be
equivalent. For let $Z, Z_1 \in Int. \mathcal{F}$ and $Z_1 = M < Z >,
M = \begin{pmatrix} A & B \\ C & D \end{pmatrix}$. Then, in the
conditions stipulated in theorem \ref{chap4:thm3} for $Z$, we have strict
inequality throughout except in the cases where these reduce to
identities in $Z$, viz. when $\mathscr{Y}_\mathfrak{K}= \pm
n_\mathfrak{K}$ and $(C,D) = (0, \mathcal{U}^{-1})$, $\mathcal{U}$ being
unimodular. Now $Z_1 = ( AZ + B) (CZ + D)^{-1}$ so that  
\begin{equation*}
(-C' Z_1 + A') (CZ + D) = E\tag{73}\label{eq73}  
\end{equation*}
 
Since $Z \in \mathcal{F}$ and $(C, D), (-C', A')$ are coprime
symmetric, we have $|| CZ + D || \geq 1$ and $||-C'Z_1+A' || \geq
1$. Since their product is equal to 1 by (\ref{eq73}), we conclude that
the equality holds in both cases. It therefore follows that $(C, D)$ is
one among the exceptional\pageoriginale pairs singled out earlier,
viz. $C = o$, $D = \mathcal{U}^{-1}$ for some unimodular matrix
$\mathcal{U}$. Then from (\ref{eq73}) we have   
\begin{equation*}
\left.
\begin{aligned}
Z_1 = Z[\mathcal{U}] + S \\
A =D'^{-1} = \mathcal{U}'
\end{aligned}
\right \} \tag{74}\label{eq74}  
\end{equation*}
where we write $S = B \mathcal{U}$. In particular, writing $Z_1 = X_1
+  i Y_1$ we have $X_1 = X[\mathcal{U}] + S$, $Y_1 =
Y[\mathcal{U}]$. Since $Y$ and $Y_1$ are both reduced matrices and $Y$
is an interior point of $\mathscr{R}$, theorem \ref{chap3:thm2} states that
$\mathcal{U} = \pm E$. Then $X_1 = X + S$ and condition (iii) of
theorem \ref{chap4:thm3} will require that $S = O$. Thus $Y_1=Y$ and $X_1=X$ and
consequently $Z_1 = Z$. We have therefore shown that two points of
$\mathcal{F}$ can be equivalent only if both are boundary points of
$\mathcal{F}$, a property typical of a fundamental domain and this
settles the first part of theorem \ref{chap4:thm3}. 

We now show that $\mathcal{F}$ is a closed set. Let $Z_\mathfrak{K} \in
\mathcal{F}, \mathfrak{K}=1, 2, \ldots ,$ and $Z_\mathfrak{K} \to
Z$. Clearly the conditions $(i)$ and (iii) are satisfied by $Z$ as
they are true of each $Z_\mathfrak{K}$. Condition (ii) also would be
fulfilled by $Z$ except that we do not know from the fact
$Y_\mathfrak{K} > o$ for each $\mathscr{R}$, that $Y > o$. We shall
show that this is so. More generally we show that $|Y|$ has a positive
lower bound for all $Y$ such that $Z=X + i Y \in \mathcal{F}$. If
$(C^{(r)}_1$, $D^{(r)}_1)$ is any coprime pair with $|C_1 | \neq o$ and
$Q^{(n, r)}$, a primitive matrix, then the pair $(C, D)$ given by
(\ref{eq68}) is always symmetric and coprime so that for (\ref{eq69}) we have
$||CZ  + D|| = ||C_1Z[Q] + D_1|| \geq 1$. Choosing in particular $C_1
= E^{(r)},D_1 = o$ and $Q = \begin{pmatrix} E^{(r)} \\ 0 \end{pmatrix}$,
we have $|| CZ + D || = || Z_r ||, Z_r$ being the matrix which results
when $Z$ is deprived of its last $(n - r)$ rows and columns and this
is true for\pageoriginale each $r$. If now $Z \epsilon \mathcal{F}$,
then $|| CZ+D|| 
\geq 1$ by one of the condition of $\mathcal{F}$ so that we can
conclude that $|| Z_r || \geq 1$ for each $r$ and every $Z \epsilon
\mathcal{F}$. In particular, setting $r=1$, this means that $|| Z_1 ||
= |Z_{11}|\geq 1$ which in its turn implies by reason of $|
X_{11}|$. being less than or equal to $\dfrac{1}{2}$ that  
\begin{equation*}
\text{\fbox{$y_{11} \geq \dfrac{1}{2}\sqrt{3}$}}\tag{75}\label{eq75}   
\end{equation*}
where $x_{11} + i y_{11} = Z_{11}$

In view of the reduction conditions (\ref{eq47} - \ref{eq49}) we have

$y^n_{11} \leq \prod\limits^{n}_{r=1}y_{rr} < C_1 |Y| $ and the above
inequality now implies the desired result. Specifically we have  
\begin{equation*}
|Y| > C^{-1}_1 (\frac{1}{2}\sqrt{3})^n > o \tag{76}\label{eq76}  
\end{equation*}
for every $Z = X + i y \in \mathcal{F}$

The proof of the connectedness of $\mathcal{F}$ is a bit more involved.

let $(C, D)$ be a given coprime symmetric pair and use the
representation (\ref{eq69}) for $C, D$. We recall the following notation. 
$$
P = C^{-1}_1D_1, S = X[Q] + P, T = Y[Q], T [F] = E, S[F] = R =
(\delta_{\mu \nu}h_\nu)
$$ 

We further introduce $Z_1 = X + i \lambda Y$ with $\lambda \geq
1$. Then we have, as in deriving (\ref{eq71}), 

$Z[Q]+C^{-1}_1 D_1=(H +iE)[F^{-1}],Z_1[Q]+C^{-1}_1D_1 = (H+i
\lambda E)[F^{-1}]$,\break $||CZ + D||^2=|C_1|^2 |T|^2 \prod \limits^{r}_{\nu
  =1}( 1+h_\nu)^2 $ and 
\begin{equation*}
|| CZ_1 + D ||^2 = | C_1 |^2 | T |^2 \prod \limits^{r}_{\nu
  =1}(\lambda^2 + h^2_\nu ) \tag*{$(71)'$}\label{eq71'}   
\end{equation*}\pageoriginale

It is immediate that $|| CZ_1+D|| \geq || CZ+D ||$ and that $Z \in
\mathcal{F}$ implies that $Z_1 \in \mathcal{F}$ for $\lambda \geq 1$,
$Z_1 = X + i \lambda Y$. 
 
We now ask how to choose $\lambda$ such that $Z_1 = X + i \lambda Y
\in \mathcal{F}$ for two matrices $X, Y$ which satisfy the condition
(iii) and condition (ii) respectively of theorem \ref{chap4:thm3}, and
we show that this will be fulfilled if $\lambda$ is chosen sufficiently
large-specifically if  
\begin{equation*}
\lambda \geq \frac{C_1}{y_{11}}, C_1 = C_1 (n) \tag{77}\label{eq77}  
\end{equation*}

Let $\lambda$ satisfy the above condition (\ref{eq77}). Then from
(\ref{eq71'}) we 
have $|| CZ_1 + D||^2 \geq \lambda^{2r}|T|^2$ so that 
$$
|| CZ_1 + D || > \lambda^r | T | = \lambda^r | Y[Q] | 
$$

As in earlier contexts we can assume that $T=Y[Q]$ is reduced by an
appropriate choice of $Q$. Then, in view of (\ref{eq47} - \ref{eq49}),
$| T | \geq 
C^{-1}_1 \prod\limits^{r}_{\nu =1} Y[y_\nu] \geq
C^{-1}_1 y^r_{11}\geq \lambda^{-r} $ where we denote $Q =
(y_1 y_2 \ldots y_r)$ and assume without
loss of generality that $C_1 > 1$. Hence $||CZ_1+D||\geq \lambda^r |T|
\geq 1 $ for all symmetric coprime pairs $(C, D)$ and this is
precisely what ensures us that $Z_1 \in \mathcal{F}$ 

Let now $Z_\nu \in \mathscr{F}$, $Z_\nu = X_\nu + i Y_\nu$, $\nu
=1, 2$, and let $\lambda_o \geq 2C_{1 / \sqrt{3}}$ join $Z_1$ and
$Z_2$ by the polygon consisting of 
\begin{align*}
 Z & = X_1 + i \lambda Y_1, 1  \leq \lambda \leq \lambda_o ,
 \tag{78}\label{eq78}   \\
 Z & = (1-\lambda ) (X_1 + i \lambda_0 Y_1) + \lambda (X_2 + i \lambda_o
 Y_2), o \leq \lambda \leq 1,\tag{79}\label{eq79}   \\
Z & = X_2 + i	 \lambda_{Y_2}, i \leq \lambda \leq \lambda_o
\tag{80}\label{eq80}   
\end{align*}\pageoriginale
 
 We prove that this polygon lies completely in $\mathcal{F}$. The
 result we proved above and (\ref{eq75}) together imply that the lines
 defined by (\ref{eq78}) and (\ref{eq80}) both lie in $\mathcal{F}$. It remains
 then to consider only the line determined by (\ref{eq79}). Since $y_1, y_2$
 are reduced, and $\mathscr{R}$ is convex, $(1-\lambda )Y_1 + \lambda
 Y_2 \in \mathscr{R}$ for $0 \leq \lambda \leq 1$. Also since $x_1,
 x_2$ satisfy the condition (iii) of theorem \ref{chap4:thm3}, so does
 $(1-\lambda ) x_1 + \lambda x_2, o \leq \lambda \leq 1 $. Hence, in
 view of our earlier result, $Z = \{ (1-\lambda ) x_1 + \lambda x_2 \}
 + i \lambda_o \{(1-\lambda) y_1 +\lambda y_2 \}$ belongs to
 $\mathcal{F}$ provided $\lambda_o$ satisfies (\ref{eq77}). We shall show
 that it does. This is in fact immediate since 
\begin{align*}
\lambda_o \geq \frac{2C_1}{\sqrt{3}} &= \frac{C_1}{(1-\lambda
  )\frac{\sqrt{3}}{2} + \lambda \frac{\sqrt{3}}{2}}\\ 
\geq & C_I / (1- \lambda )y'_{11} + \lambda y''_{11}
\end{align*}

It follows therefore that the points $Z$ in (\ref{eq79}) all belong to
$\mathcal{F}$. We can now conclude that $\mathcal{F}$ is a connected
set.  

Our assertion concerning the boundary of $\mathcal{F}$ still remains
to be settled. Specifically, we have got to show that the boundary of
$\mathcal{F}$ consists of a finite number of algebraic surfaces. First
we note that every positive matrix $\mathcal{Y}$ satisfies the
inequality 
\begin{equation*}
| y | \leq  y_{11} y_{22} . . y_{nn} \tag{81}\label{eq81}   
\end{equation*}\pageoriginale
where we assume $y = (y_{\mu \nu})$

For we can write $y = \mathcal{R'}\mathfrak{K}$ with a non-singular
real matrix $\mathfrak{K} = ( \mathscr{W}_1 \mathscr{W}_2 \cdots
\mathscr{W}_n)$ Then it is known that  
$$
| \mathfrak{K}|^2 \leq \prod^n_{\nu =1}\mathscr{W}'_\nu \mathscr{W}_\nu
= \prod^n_{\nu =1} y_{\nu \nu} 
$$

But $|\mathfrak{K}|^2 = | \mathcal{Y}|$ and this proves what is
desired. We proceed to determine a lower bound for the smallest
characteristic root $\lambda$ of a positive reduced matrix
$Y$. Let $\lambda_1, \lambda_2, \cdots, \lambda_n$ denote the
characteristic roots of $\mathcal{Y}$ and let $\mathcal{Y}_\nu$ denote
the matrix which arises from $\mathcal{Y}$ by deleting its $\nu^{th}$
row and column. It is known then that the characteristic roots of
$\mathcal{Y}^{-1}$ are $\lambda^{-1}_\nu, \nu =1, 2, \ldots ,n$.
Denoting by $\sigma (A)$, the trace of a square matrix $A$, we have
from (\ref{eq49}), (\ref{eq81}) and (\ref{eq47}) 
\begin{align*}
\frac{1}{\lambda} & =\max_\nu \frac{1}{\lambda_\nu} \leq
\frac{1}{\lambda_1} + \frac{1}{\lambda_2} + \cdots
\frac{1}{\lambda_\nu}= \sigma (Y^{-1})\\ 
& =\sum^n_{\nu =1} \frac{|y_\nu |}{|y|} \leq C_I
\sum^n_{\nu = I}\frac{y_{11}y_{22}\ldots
  y_{\nu-1}y_{\nu-1} \cdot y_{\nu+1} y_{\nu+1} \ldots y_{nn}}{y_{11}
  \cdot y_{22} \ldots y_{nn}}\\ 
 & =C_1 \sum^n_{\nu =1} \frac{1}{y_{\nu \nu }}\leq \frac{n
  C_1}{y_{11}} \tag{82}\label{eq82}    
\end{align*}

Thus if $y > o$ is reduced and $ y = (y_{\mu
  \nu})$ and if $\lambda$ be the minimum of the characteristic roots
of $y$, then 
\begin{equation*}
\lambda \geq y_{11}/n C_1 \tag{82}\label{addeq82}   
\end{equation*}

If now $Z = X + i Y \in \mathcal{F}$ then $y_{11} \geq
\dfrac{1}{2}\sqrt{3}$ from (\ref{eq75}) so that in this case 
\begin{equation*}
\lambda \ge \sqrt{3}/ 2n C_1, C_1 = C_1(n)\tag{83}\label{eq83}   
\end{equation*}

Let\pageoriginale $Z \in Bd \cdot  \mathcal{F}$ and let $Z_\mathfrak{K} \to
Z,Z_\mathfrak{K} \notin \mathcal{F}$. To each $Z_\mathfrak{K}$ we can
determine an equivalent point $\mathcal{W}_\mathfrak{K} \epsilon
\mathcal{F}$ say, $\mathcal{W}_\mathfrak{K} = (A_\mathfrak{K}
Z_\mathfrak{K}+B_\mathfrak{K})(C_\mathfrak{K}
Z_\mathfrak{K}+D^{-1}_\mathfrak{K})$. Then   
\begin{equation*}
\begin{pmatrix}
A_\mathfrak{K} &  B_\mathfrak{K}\\ C_\mathfrak{K} & D_\mathfrak{K}
\end{pmatrix}
\neq \pm 
\begin{pmatrix}
E & o \\ o & E
\end{pmatrix}
\end{equation*}

\medskip
\noindent{\textbf{Case. i:}}
Let $C_\mathfrak{K}\neq O$ for an infinity of indices. By passing to a
subsequence, we may assume this to be true for all
$C_\mathfrak{K}'s$. Also, since their ranks belong to the finite set
$1, 2, \ldots n$, we can assume all $C_\mathfrak{K}, s$ to be of the
same rank $r > o$. To each class $\{C_\mathfrak{K}, D_\mathfrak{K} \}$,
we determine the corresponding classes $\{C^{(r)}_o, D^{(r)}_o\}$ and
$\{Q^{(n, r)}\}$ with $|C_o | \neq 0$, given by Lemma \ref{chap1:lem1}. Both
classes $\{C_o, D_o\}$ and $\{Q \}$ depend on $\mathfrak{K}$ though the
notation is not suggestive. Then from (\ref{eq70}) and (\ref{eq71}) we have 
\begin{align*}
||C_\mathfrak{K}Z_\mathfrak{K} + D_\mathfrak{K} ||^2 &= ||C_o
Z_\mathfrak{K}[Q] + D_o ||^2 \\ 
&= | C_o |^2 | T |^2 \prod^r_{\nu =1}(1 + h^2_\nu ) \tag{84}\label{eq84}   
\end{align*}
where $T = y_\mathfrak{K}[Q]$, $S = x_\mathfrak{K} [Q] + C^{-1}_o D_o$,
$$
x_\mathfrak{K} + i y_\mathfrak{K} = Z_\mathfrak{K}, T[F] = E
$$
and $S[F] = H = (\delta_{\mu \nu} h_\nu)$

As usual we assume that $T$ is reduced. Let
$y_1, y_2,\ldots y_r$ be the colu\-mns of
$C$. Then $T=(\mathscr{U}'_\nu \mathcal{Y}_\mathfrak{K}
\mathscr{U}_\nu)$ so that by (\ref{eq49}), 
\begin{equation*}
\prod^r_{\nu =1} \mathcal{Y}_\mathfrak{K}[\mathscr{U}_\nu] < C_1 | \top
| \tag{85}\label{eq85}    
\end{equation*}
 
As is well known, the smallest characteristic root
$\lambda^{(\mathfrak{K})}$ of $y_\mathfrak{K}$ is given\pageoriginale
by  
$$
\lambda^{(\mathfrak{K})} = \min_{\mathcal{E'} \mathcal{E} =
  1}\mathcal{Y}_\mathfrak{K} [\mathcal{E}] \qquad (\mathcal{E}- \text{
  real column }). 
$$


Since $\lambda^{(\mathfrak{K})}$ is a continuous function of
$\mathcal{Y}_\mathfrak{K}$ and $\mathcal{Y}_\mathfrak{K} \to
\mathcal{Y}$ we have $\lim\limits_\mathfrak{K}
\lambda^{(\mathfrak{K})}=\lambda$ the smallest characteristic root of
$Y$, and further $\lambda \geq \sqrt{3}/2nC_1$, from (\ref{eq83}), so that 
\begin{equation*}
\lambda^{(\mathfrak{K})} \geq \sqrt{3}/4 nc_1 \tag{86}\label{eq86}   
\end{equation*}
for sufficiently large $k$. We shall be concerned only with these
$k's$ in the rest of the proof. Then we have 
$$
\mathcal{Y}_\mathfrak{K}[\mathscr{U}_\nu] \geq \lambda^{(\mathfrak{K})}
\mathscr{U'}_\nu \mathscr{U}_\nu \geq \lambda^{(\mathfrak{K})} \geq
\sqrt{3}/4 nC_1 
$$
and therefore from (\ref{eq85}),
\begin{equation*}
| \top | > \bigg(\frac{\sqrt{3}}{4^n}\bigg)^r C_1^{- r - 1} > o
\tag{87}\label{eq87}    
\end{equation*}

Clearly $\mathcal{W}_\mathfrak{K}$ satisfies the relation
$$
(-C'_\mathfrak{K} \mathcal{W}_\mathfrak{K} +A'_\mathfrak{K})
(C_\mathfrak{K} Z_\mathfrak{K} + D_\mathfrak{K}) = E 
$$
and since $\mathcal{W}_\mathfrak{K} \in \mathcal{F}$ we have $|| -
C'_\mathfrak{K} \mathcal{W}_\mathfrak{K} + A'_\mathfrak{K} || \geq 1$ 

We therefore conclude that $|| C_\mathfrak{K} Z_\mathfrak{K} +
D_\mathfrak{K}|| \leq 1$ Hence  
$$ 
1 \geq || C_\mathfrak{K} Z_\mathfrak{K} + D_\mathfrak{K}||^2 =  | C_o |^2
| \top |^2 \pi(1-h^2_\nu) 
$$
which implies that $| C_o |$ is bounded as $| \top |$ and
$(1+h^2_\nu)$ have positive lower bounds as a result of (\ref{eq87}). By
appealing to (\ref{eq84}) and (\ref{eq87}) we then conclude that
$|C_o|, | \top |, h_1, h_2, \ldots h_r$ all lie between bounds which
do not depend upon $\mathfrak{K}$ and $Z$. Since  
\begin{align*}
\frac{\sqrt{3}}{4nc_1}\mathscr{U'}_\nu \mathscr{U}_\nu & \leq
\lambda^{(\mathfrak{K})} \mathscr{U'}_\nu \mathscr{U}_\nu \leq
\mathcal{Y}[\mathscr{U}_\nu ]\\ 
& \leq \frac{C_1| \top |}{\prod_{\mu \neq
    \nu}\mathcal{Y}_\mathfrak{K}[\mathscr{U}_\mu ]}\leq
(\frac{4n}{\sqrt{3}})^{r-1}C^r_1 |\top | 
\end{align*}
the vector\pageoriginale $\mathscr{U}_{\nu}$ is bounded. This proves
that the matrix 
$Q$ belongs to a finite set of matrices which does not depend on
$Z$. Since the diagonal elements $\mathscr{U}_{\nu}
Y_{\mathscr{R}}\mathscr{U}_{\nu}$ of $T$ are bounded as a consequence
of (\ref{eq84}) and (\ref{eq87}) and $T > 0$, it follows that all the
elements of 
$T$ are bounded and this in its turn, in view of the relations $T = E
[ F^{-1}]; S = H [F^{-1}]$ implies that the elements of $F^{-1}$ and
$S$ are bounded. Now we observe that $| x_{\mathscr{R}} |
x^{(\mathscr{R})}_{\mu \nu}| \leq \frac{1}{2}$ and $x_{\mathscr{R}}
\to x = ( x_{\mu \nu})$ so that $|x_{\mu \nu} | \leq \dfrac{1}{2}$ and
consequently  $X$ is bounded. But $S = x_{\mathscr{R}} [Q] + C^{-1}_o
D_o i.e. C_o^{-1} = \delta -x_{\mathscr{R}}[Q]$. The above then shows
that $P =  C^{-1}_o D_o$ is bounded; in their words only a finite
number of choices exist for $P$. But we know that $P$ determines the
class $\{ C_o, D_o \}$ uniquely. Hence the number of choices for $\{
C_o, D_o \}$ is also finite. As we have already shown that the number
of classes $\{Q \}$ is finite, it follows that the number of distinct
classes $\{ C_{\mathscr{R}}, D_{\mathscr{R}} \}$ is finite; in other
words the classes $\{ C_{\mathscr{R}}, D_{\mathscr{R}} \}$ are equal
for an infinity of $k' s$.We shall denote this common class by $\{ C,
D \}$. Then, taking the appropriate sequence of $k' s$, 
$$
|| CZ + D || = \lim_{\mathscr{R}}|| C_{\mathscr{R}} Z_{\mathscr{R}} +
D_{\mathscr{R}}|| \leq 1, \text{ as } || C_{\mathscr{R}}
Z_{\mathscr{R}} + D_{\mathscr{R}}|| \leq 1 
$$
for each $k$.

\setcounter{pageoriginal}{55}
Since $Z \in \mathfrak{f}$, we have the reverse inequality and the
equality results. Hence in this case $Z \in Bd. \mathfrak{f}$ satisfies
the equation 
\begin{equation*}
|| CZ + D || = 1 \tag{88}\label{eq88}   
\end{equation*}
for\pageoriginale some out of a finite number of paris $(C, D)$ which
are not mutually equivalent.  

\medskip
\noindent{\textbf{Case ii:}}
Suppose $C_{\mathfrak{K}} = O$ for all sufficiently large $k'
s$. Considering only these $k' s$, we have  
$$
W_{\mathfrak{K}} = Z_{\mathfrak{K}} [\mathcal{U}_{\mathfrak{K}}] +
S_{\mathfrak{K}} 
$$
with a unimodular matrix $\mathcal{U}_{\mathfrak{K}}$ and a symmetric
integral matrix $S_{\mathfrak{K}}$ as in (\ref{eq74}). If $V_{\mathfrak{K}}$
be the imaginary part of $W_{\mathfrak{K}}$, then
$Y_{\mathfrak{K}}[\mathcal{U}_{\mathfrak{K}}] = V_{\mathfrak{K}} \in
\mathscr{R}$. We wish to show that the number of
$[u_{\mathfrak{K}}]' s$ is finite. Since the imaginary part
$Y$ of $Z$ belongs to $\mathscr{R}$, representing $Y$ in the form $Y =
D [B]$, $D = ( \delta_{\mu \nu}d_{\nu})$, and $B = (b_{\mu \nu})$, we
have from (\ref{eq59}) that $\dfrac{d_{\nu}}{d_{\nu+1}}$, $b_{\mu \nu}$ are
all bounded by $C_{11} = C_{11}(n)$. But $\mathcal{Y}_{\mathfrak{K}}
\to \mathcal{Y}$ so that representing $\mathcal{Y}_{\mathfrak{K}}$
analogous to $Y$ as $y_{\mathfrak{K}} = D_{\mathfrak{K}}
[B_{\mathfrak{K}}]$, $D_{\mathfrak{K}} = (\delta_{\mu \nu}
d^{(\mathfrak{K})}_{\nu}$, and, $B_{\mathfrak{K}} =
(\ell^{(\mathfrak{K})}_{\mu \nu})$. we have $D_{\mathfrak{K}} \to D$ and
$B_{\mathfrak{K}} \to B$. We therefore deduce that the ratios
$d^{(\mathfrak{K})}_{\nu}/d^{(\mathfrak{K})}_{\nu +1}$
and$(\ell^{(\mathscr{R})}_{\mu \nu})$. are bounded, say, by $2
C_{11}$. Lemma \ref{chap3:lem6} now implies that there can be only a
finite number 
of $\mathscr{U}_{\mathscr{R}}'s$ with
$y_{\mathscr{R}}[\mathscr{U}_{\mathscr{R}}] =
D_{\mathscr{R}}[B_{\mathscr{R}} \mathscr{U}_{\mathscr{R}}]$ reduced,
in other words, for an infinity of $k's$, the
$\mathscr{U}_{\mathscr{R}}'\mathscr{S}$ are the same, say
$\mathscr{U}_{\mathscr{R}} = \mathscr{U}$. As $\mathscr{R} \to \infty$
through the sequence of these values, we have $y[\mathscr{U}] =
\lim_{\mathscr{R}} y_{\mathscr{R}}[\mathscr{U}]$ and since each
$y_{\mathscr{R}}[\mathscr{U}] \in \mathscr{R}$ and $Y > 0$, it follows
that $y[\mathscr{U}] \in \mathscr{R}$. But we already know that $y \in
\mathscr{R}$ and this, by theorem \ref{chap3:thm2}, is possible only
if both $Y$ and 
$Y[\mathscr{U}]$ belong to the boundary of $\mathscr{R}$. Hence  
\begin{equation*}
y[\mathcal{U}_{\mathscr{R}}] = \mathscr{Y}_{\mathscr{R}} \tag{89}\label{eq89}   
\end{equation*}
where\pageoriginale $\mathscr{U}_{\mathscr{R}}$ is the first column of
$\mathscr{U}$ 
such that $\mathcal{U}_{\mathscr{R}}  \neq \pm \mathscr{H}
\mathscr{R}$ provided such a $\mathscr{U}_{\mathscr{R}}$ exists. This
is therefore true if $\mathscr{U}$ is not a diagonal matrix. 

Let now $\mathcal{U}$ be a diagonal matrix. Two cases arise,
viz. either $\mathcal{U} \neq \pm E$ or $\mathcal{U} = \pm E$. In
the former case it follows, as in (\ref{eq68}) that  
 \begin{equation*}
\mathscr{Y}_{\mathscr{R} \mathscr{R} + 1} = 0 \tag{90}\label{eq90}   
 \end{equation*} 
for some $k$. In the latter case, taking the k's for which
$\mathscr{U}_{\mathscr{R}}= \mathscr{U} = \pm E$ we have
$W_{\mathscr{R}} = Z_{\mathscr{R}}+ S_{\mathscr{R}}$ with
$S_{\mathscr{R}} \neq O$ as $W_{\mathscr{R}} \in \mathfrak{f}$ while
$Z_{\mathscr{R}} \notin \mathfrak{f}$. Then the condition (i) of
theorem \ref{chap4:thm3} for the elements of $\mathfrak{f}$ implies that  
\begin{equation*}
x_{\mu \nu} = \pm \frac{1}{2} \tag{91}\label{eq91}   
\end{equation*}
for at least one pair $(\mu, \nu)$.

Thus any point $Z \in Bd ,\mathfrak{f}$ satisfies one or the other of
the system of equations (\ref{eq88}) - (\ref{eq91}), in other words,
$Z$ satisfies 
one of a finite set of equalities, and each one of them defines an
algebraic surface. It therefore follows that a finite number of
algebraic surfaces bound $\mathfrak{f}$ and this completes the proof of
theorem \ref{chap4:thm3}. 

We insert here the following remarks for future reference. A group of
symplectic matrices shall be called \textit{discrete} if every
infinite sequence of different elements of the group diverges,
or equivalently any compact subset of it contains only finitely many
distinct elements. The modular group of degree $n$ is then clearly
discrete. 

We\pageoriginale shall call a group of symplectic substitutions
\textit{discontinuous} in $\mathscr{Y}$, if for every $z \in
\mathscr{Y}$, the set of images of $z$ relative to the given group has
no limit point in $\mathscr{Y}$. Concerning the modular group then, we
have  

\setcounter{lem}{7}
\begin{lem}\label{chap4:lem8}%lemma 8
 The modular group of degree $n$ is discontinuous in
  $\mathscr{Y}_n$ 
\end{lem}

The proof is by contradiction. Let $z_1, z_2 \cdots$ be a sequence of
equivalent points, all different, converging to a point $z \in
\mathscr{Y}$. Then $Z_{\mathscr{R}} = M_{\mathscr{R}} < Z_1>$ for some
$M_{\mathscr{R}} = \begin{pmatrix} A_{\mathscr{R}} & B_{\mathscr{R}}
  \\ C_{\mathscr{R}} & D_{\mathscr{R}} \end{pmatrix} \in \mathcal{M}_n$ 

Let $z_{\mathscr{R}} = x_{\mathscr{R}} + Y_{\mathscr{R}}$ and $Z =
X + iY$. 

Then $Y_{\mathscr{R}} = (\bar{Z}_1' C_{\mathscr{R}}'
D_{\mathscr{R}}^{-1} y_1 (C_{\mathscr{R}}z_1 + D_{\mathscr{R}})^2$ by
(\ref{eq24}). If $Y_1 = R'R$ this gives $Y_{\mathscr{R}} =
\bar{\Omega}'\Omega$ with $\Omega = R (C_{\mathscr{R}} Z_1 +
D_{\mathscr{R}})^{-1}$. Also $y_{\mathscr{R}}'s$ are bounded as
$Y_{\mathscr{R}} \to Y$, and it follows that $\Omega$ is bounded. This
in its turn implies that $(C_{\mathscr{R}} Z_1 +
D_{\mathscr{R}})^{-1}$ is bounded. Also $||C_{\mathscr{R}} z_1
+D_{\mathscr{R}}||$ is bounded as $||C_{\mathscr{R}} Z_1 +
D_{\mathscr{R}}||^2 = |Y_1|/|Y_{\mathscr{R}}|$ from (\ref{eq72}). We
therefore infer that $C_{\mathscr{R}} Z_1 + D_{\mathscr{R}}$ is
bounded. Considering the real and imaginary part separately this
implies that $C_{\mathscr{R}}$and hence also $D_{\mathscr{R}}$ are
bounded and consequently also $A_{\mathscr{R}}$ and $B_{\mathscr{R}}$
as the relation $Z_{\mathscr{R}}(C_{\mathscr{R}} Z_1 +
D_{\mathscr{R}}) = A_{\mathscr{R}} Z_1 + B_{\mathscr{R}}$ shows. Thus
the $M_{\mathscr{R}}' s$ are all bounded for $\mathfrak{K} =
1, 2, \ldots$ and therefore for an infinity of $k's$ the
$M_{\mathscr{R}}'s$  are identical, and for these $k's$ the
$Z_{\mathscr{R}}'s$ are identical, contradicting our assumption. This
proves the Lemma. 

