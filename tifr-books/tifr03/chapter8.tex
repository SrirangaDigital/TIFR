
\chapter{Lemmas concerning special integrals}%sect 8

We need\pageoriginale  to settle some special integrals to be applied
subsequently in 
our consideration of the Poincare$'$-theta series. First we generalize
Euler's well known $\Gamma$-integral  
$$
\int\limits_o^\infty e^{-ty} y^{-s -1} = \Gamma (s) t^{-s}, t > 
o, Re. s > o 
$$
to the case of a matrix variable.

\setcounter{lem}{13}
\begin{lem}\label{chap8:lem14}%lemma 14.
If $T = T^{(n)} > 0$  and $ Re. s > \dfrac{n -1}{2}$, we have
\end{lem}
\begin{equation*}
\int\limits_{\gamma = \gamma^{(n)} > o} e^{- \sigma (T \gamma)} | y
|^{s-\frac{n+1}{2}} [dy] = \pi^{\frac{n(n-1)}{4}} 
\prod\limits^{n-1}_{\nu = o} \Gamma (s - \nu_{/2}) | T |^{-s}
\tag{143}\label{eq143}  
\end{equation*}

The proof is by induction on $n$. We first observe that it suffices to
consider only the case $T = E$, as in the general case, we can write
$T = R R'$ and changing the variable $y$ in the above integral to
$y^* = y [R]$ we shall have $\dfrac{\partial (y^*
  )}{\partial (y)} = | R |^{n+\ell}$ by (\ref{eq138}) so that  
\begin{align*}
| y^* |^{\frac{n+1}{2}} [ d y^* ] = | [R] |^{-n+\ell}{2} [ d
  y ] & = | y |^{-n + \ell}{2} | R |^{-n-1} \frac{\partial
  (y^*)}{\partial (y)} [ d y ] \\ 
&= | y |^{-n + \ell}{2} [ d y ]
\end{align*}
and $| y |^s = | y^* |^s  | T |^{-s}$ as $y = y^* [
  R^{-1} ]$. 

Then the integral we had, transforms into 
$$
| T |^{-s} \int\limits_{y^* > o} e^{- \sigma ({y^*})} | y^*
|^{s - \frac{n+ \ell}{2}} [ d y^* ] 
$$
and what we claimed is now immediate. We shall assume then that $T =
E$. Let  
$$
Y = 
\begin{pmatrix}
y_1 & \mathscr{Y}\\ 
\mathscr{Y}' & y_{n n}
\end{pmatrix}
-
\begin{pmatrix}
y_1 & o \\
o & y
\end{pmatrix}
\begin{bmatrix}
\begin{pmatrix}
E & w\\
o & 1
\end{pmatrix}
\end{bmatrix}.
$$

The\pageoriginale  above is a special case of our representation
(\ref{eq132}) where we 
have set $r = n - 1$. The following relations are immediate. 
  \begin{equation}
  \left. 
  \begin{aligned}
  y_{n n} &= y + y_1 [w]\\
  \mathscr{Y}& = y_1 [ w ]
  \end{aligned}
  \right \} \tag{144}\label{eq144}  
  \end{equation}

Clearly then, the functional determinant

$\dfrac{\partial (y)}{\partial (y_1, y , w)} = \dfrac{\partial (y_1 ,
  y_{n n})}{\partial (y_1, y, w)}$ is of the
type \begin{tabular}{|c|c|c|} 
E & 0 &0 \\
\hline
* & 1 & *\\ 
\hline
* & 0 & $y_1$
\end{tabular}
 so that 
\begin{equation*}
\frac{\partial (y)}{\partial (y_1, \mathcal{Y},
  \mathscr{W})} = | y_1 | \tag{145}\label{eq145}   
\end{equation*}

Also $| y | = | y_1 | \mathcal{Y}, \sigma (y) = \sigma
(y_1) + \mathcal{Y} + y_1 [\mathscr{W}]$, and $y > o$
when and only when $y_1 > o$ and $\mathcal{Y} > o$ all these
being immediate consequences of (\ref{eq144}). Hence we have, writing $d
[\mathscr{W}] = \pi_{\nu = 1}^{n - 1} d \vartheta_\nu$ there
$\mathscr{W}' = (\vartheta_1. \vartheta_2, \ldots \vartheta_{n-1})$, 
\begin{align*}
\int\limits_{y > 0} e^{-\sigma (y)} | y |^{s- \frac{n + 1}{2}} [ d
  y ] & = \int\limits_{y_1 > o} \int\limits_{y>o}
\int\limits_{\mathscr{W}} e^{- \sigma (y_1) - y - y_1 [\mathscr{W}]}\\
& \qquad \qquad  | y_1
|^{s- \frac{n-1}{2}} \mathscr{Y}^{s-\frac{n+1}{2}} [ d \mathscr{W} ] d
\mathscr{Y} [ n y_1 ] 
\end{align*}
\begin{align*}
& = y (s- \frac{n+1}{2}) \int\limits_{y_1 > o} e^{-\sigma (y_1)}
  | y_1 |^{s - \frac{n-1}{2}}\bigg(_{\mathscr{W}} e^{-y_1
    [\mathscr{W}]} [ d \mathscr{W} ] \bigg) [ d y_1] \\ 
& = \pi^{\frac{n-1}{2}} y (s- \frac{n-1}{2}) \int\limits_{y_1 >
    o} e^{-\sigma (y_1)} | y_1 |^{s- \frac{(n - 2) + 1 }{2} [ d y_1 ]} 
\end{align*}
\hfill{observing}

that, if we write $y = R' R$ on $\mathscr{Y} = ( w_1, w_2, \ldots
w_{n-1}) = (R \mathscr{W})$ 
then\pageoriginale  the integral within the parenthesis, upto a
factor, reduces to   
$$
\int e^{w_1^2 - w_{n - 1}^2} d w_1 \ldots d w_{n-1} = \bigg(
\int\limits_\sigma^\alpha e^{- w^2} d w  \bigg)^{n -1} = \bigg(
\pi^{1/2}\bigg)^{n-l}. 
$$

By the induction assumption, the last integral, viz.
$$
\int\limits_{y_1 > o} e^{- \sigma (y_1)} | d y_1| = \pi^{\frac{n-1}{2}
  , \frac{n-2}{2}} \prod_{\nu = 0}  ^{n-2} y ( \mathcal{S} 0 \nu
/ 2) .  
$$

Substituting this in the above we get 
$$
\int\limits_{ y > 0} e^{- \sigma (y)} | y |^{\mathcal{S} - \frac{n +
    1}{2}} [ d y ] = \Pi ^{ \frac{n}{2}. \frac{n - 1}{2}}
\prod^{n-1}_{\nu = 0} y (\mathcal{S} - \nu / 2) 
$$
and the proof is complete. 

With $\mathscr{R}_n$ denoting space of all reduced matrices $ y =
y ^{n} > 0$ we state  

\begin{lem}\label{chap8:lem15} %lemma 15
Let $f(y)$ be a real and continuous functions on the ray $y \geq o
  $ and let $ s + \dfrac{n -2}{2} > 0$ . Then  
\begin{equation*}
\int\limits_{\substack{y \in \mathfrak{K}_n \\ |y| \geq t }}
f (| y |) | y|^s [ d y ] =  \frac{n + 1}{2} \vartheta_n
, \int\limits_0^1 f(y) y^{s + \frac{n - 1}{z}} dy \tag{146}\label{eq146}   
\end{equation*}
where $\vartheta_n$ is a certain positive constant, depending only on
$n$.  
\end{lem}


\begin{proof}%proof 
We first prove the lemma in the special case when $f (y ) \equiv 1$
and $s = o$. Then (\ref{eq146}) reduces to  
\begin{equation*}
\int\limits_{\substack{y \in \mathfrak{K}_n\\ ||y | \leq t|}
} [d y] = \vartheta _n t\frac{n + 1}{2} \tag*{$(146)'$}\label{eq146'}   
\end{equation*}

If only we know that the integral on the left of (\ref{eq146'}), exists then
in view of its homogenity in the variables $y$, (\ref{eq146'}) certainly
holds with $\vartheta_n = \int\limits_{\substack{y \in
    \mathfrak{K}_n \\ |y| \leq L}} [dy]$.  
\end{proof}

We\pageoriginale  shall now show that the integral 
$$
y_n (t) = \int\limits_{\substack{ y \in \mathfrak{K}
    \\ |y|\leq \in }}[d y ] 
\text { exists. }
$$

Towards this end, we approximate $y _n (t)$ by the following
integral $y _n (\delta , t) = \int\limits_{\substack{ y \in
    \mathfrak{K}_n \\ y _ll \geq \delta }} |y \leq \in [d y]
$ where $\delta$ denotes a small positive number. The above requirements in
$y$ will imply, in view of the reduction conditions that all $ y
_{\nu\nu} ' s$ are bounded as  
$$
\delta \leq y_{i1} \leq \ldots \leq y_{nn}. \Pi^n _{\nu = 1}
y_{\nu\nu} \leq c , |\lambda| \leq C_i t 
$$
and as a consequence, all the $y _{ N n }' s $ are bounded. In other
words the domain of integration for $J_n (s , \in)$ is finite
(i.e. $J_n (\delta , \in)$) is a proper integral ) so that $J_n
(\delta , \in)$ certainly exists. We need then only show that $J_n
(\delta , \in)$ has a limit as $\int \to o$ and then this limit is
obviously $J_n (\delta, \in)$. Specifically we show that if $0 <
\delta < \delta '$ , then the difference $J_n (\delta , \in), J_n
(\delta , \in)$ (which is positive ) can be made arbitrarily small by
taking $\delta$ sufficiently small. This difference is representable
as the integral.  
\begin{equation*}
\int\limits_{ y \in \mathfrak{K}_n, |y| \leq \in , \delta \leq y_{11}
  \leq \delta} [dy]\tag{147}\label{eq147}   
\end{equation*}

If we write $\lambda = \left (\begin{smallmatrix} y_{11} & x \\ x &
  y_i \end{smallmatrix} \right ) , y_i = y_i^{(n-1)}$ ,
then the domain of integration on (\ref{eq147}) is contained in the
following domain  
\begin{gather*}'
\delta \leq y_{ii} \leq \delta \\
| y_{2 v}| \leq \frac{1}{2} y_{i1}\nu = 2, 3 , \ldots ,,.\\
|y_i > 0|
\end{gather*}
as the\pageoriginale  inequalities $y_{i1} |y_i| \leq y_{i1} y_{22}
\cdots \leq 
c_i |\lambda| \leq t$ which are consequences of (\ref{eq81}) and (\ref{eq49}),
show. Denoting the last domain as $\mathscr{L}$, we have, in view of
the induction hypothesis ,  
\begin{align*}
\int\limits_{ \mathscr{L}} [dy] & = \int\limits_{ \delta \leq y _{11}
  \leq \delta} \int\limits_{\substack{ | y_i \nu \leq ^1 \backslash 2|
    y_{11}\\ \nu = 2. 3 ,  n}} \int\limits_{\substack{ y_1 \in
    \mathfrak{K}_{n-1}\\ |r_i| \leq C, t \backslash y{11}}} [dy,] dy ,z
.. dy_in dy_ii\\ 
& = \vartheta_{n-1} \int\limits_{y < y_{11} \delta }
\int\limits_{|y_11| \leq ^1 / _2  y _{11}|} \bigg ( \frac{C_1
  t}{y_11}^{n/2}  \bigg)  dy_{12} \ldots dy_{1n} dy_{ii}\\ 
&= \vartheta _{n - 1} \int\limits_{\delta \leq y_{11} \leq \delta}
\bigg( \frac{C_1 t}{y_11}  \bigg)^{^{n/2}}  dy_{12} \ldots , dy_{1n}
dy_{11}\\ 
& = (c_i)^{n\backslash 2} \vartheta_{n-1} \int\limits_{ \delta ' \leq
  y_{11} \leq \delta } y_{11}  ^{n/ {2}-1} d y_{11} < \in  
'\end{align*} 
if $\delta$ is sufficiently small,  (\ref{eq146'}) now follows. 

It is also clear that the existence of $\int\limits_{\substack{ y
    \in \mathfrak{K} n \\ |y| \leq t}} [dy]$ which we proved above
already implies the existence of the integral on the left side of
(\ref{eq146}), taking into consideration the continuity on $f(y)$ in $y \geq
0$, so that it remains only to establish the equality of the two
side of (\ref{eq146}). Let us introduce the following truncated integral  
$$
I (a, t ) = \int\limits_{\substack{ y \in \mathfrak{K}_n \\ a \leq
    |y| \leq t}} f (|y|)| y |^s [dy] 
$$ 
and observe that 
$$
I (a, t + h) - I (a, t) = I (t, t + h) = \int\limits_{\substack{
    y \in \mathfrak{K}_n \\ t \leq | y | \leq t + \ell}} f (
|y|) | y |^A [dy] 
$$

By a\pageoriginale  mean theorem, in view of the continuity of the
function $f(y). 
y^2$ in the closed interval $[t, t, \mathfrak{K}]$, the last integral
is equal to  
$$
f ( t + \vartheta h ) (t + \vartheta + h )^s \int\limits_{ y \in 
  \mathfrak{K} n , t \leq |y| \leq t, h}[dy]  
$$
with a certain $\vartheta$ in the interval [0,1]

Thus we have from (\ref{eq146'}), 
\begin{align*}
I( a, t + h ) & - I(a, t)  = f (t + \vartheta |)( t + \vartheta)^s
\Big\{ J (t + h) - 7 (t)\Big\} \vartheta\\ 
& = \mathfrak{f} (t + \vartheta h) (t - \vartheta h)^s \{(t -
h)^{\frac{n + 1}{2}} - t \frac{n + 1}{2} \} \vartheta_n 
\end{align*}
and therefore 
\begin{align*}
\lim_{h \to 0} \frac{I (a, t + R) - I (a , t)}{h} & = f (t) t ^s
\vartheta _n \lim_h \to o \frac{(t + h \frac{n + 1}{2}) t \frac {n + 1
  }{2}}{h}\\ 
& = \frac{n + 1}{ 2} \vartheta _n f(t) t^s + \frac{n -1}{2} 
\end{align*}

The limit in the left side is clearly equal to $\dfrac{ \partial (t (
  0 . t))}{\partial t}$. It now follows that  
$$
I (a, t)= \int\limits_a^t \frac{\partial I (a, t)}{\partial t} d t = \frac{n
  + 1}{2} \vartheta _n \int^{t}_o f(t) t ^s + \frac{n + 1}{2} dt 
$$
and since $s + \dfrac{ n + 1}{2} > 0$ by assumption, letting $a \to
0$, and since $s + \dfrac{n + 1}{2} > 0$  by assumption, letting $a
\to 0$, what results is precisely what we wanted. We single out
important special cases of (\ref{eq146}), viz.  

$1)$ Choosing $ f (t) \equiv 1$ in the above, (\ref{eq146}) yields that 
\begin{align*}
\int\limits_{ \substack{y \in \mathfrak{K}_n \\ | y |\leq
    t}} |y|^s [d y] & = \frac{n +1}{2} \vartheta_n \int^t_o
y^{s + \frac{n-1}{2}} x y = \frac{ n + 1}{ 2s +n + 1}\\
& \qquad  \vartheta_n \tau
_n t ^{s + \frac{n +1}{s}}  \text{ provided } {s + \frac{n + 1}{2}} >
0. \tag{148}\label{eq148}  
\end{align*}

$2)$ Setting $f (y) = e^{ - a y^d } ( a , 0, \partial > 0)$ in the
above, and letting $ t \to \infty$ in (\ref{eq146}), that the limit exists is
seen by considering\pageoriginale  the right side and we get  
$$
\int\limits_{y \in \mathfrak{K}n} { e^{- a | y |^d}}|y|^s [dy] =
\frac{2 + 1}{2}2_n \int\limits_{0}^{y} e^{- ay^\partial} y ^{ s +
  \frac{n - 1}{2}} dy 
$$
\begin{equation*}
= \frac{ n + 1}{2 d} \vartheta_n |(\frac{s}{2} + \frac{n +1}{2d}) d^{
  -s \partial } - ^{ n + 1} _{ 2d} \tag{149}\label{eq149}   
\end{equation*}
provided $a > 0, \partial > 0$ and $s+\dfrac{ n +1}{2} > 0 C $. 

It is remarkable that all integrals which we use and need for a
consideration of the Poincare' series are easily computed as seen from
the above.  
