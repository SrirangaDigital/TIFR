\chapter{The Main Theorem}\label{chap2}

IN\pageoriginale THIS CHAPTER,  we state and prove the Main Theorem. \textit
{Throughout this chapter $K$ denotes an algebraically closed field and
  $\mathbb{P}^n_K$ the projective $n$-space over $K$}. 

\section{The Statement of the Main Theorem}\label{chap2:sec1}%sec A

\subsection{Main theorem}\label{chap2:sec1:subsec2.1}%mai 2.1

Let $V_1 = V(I_1) $ and $V_2 = V(I_2)$ be two pure dimensional
  subvarieties in $\mathbb{P}^n_K$ defined by homogeneous  ideals
  $I_1$ and $I_2$ in $K[X_0, \ldots,  X_n]$. There exists a collection
  $\{C_i\}$ of irreducible subvarieties of $V_1 \cap V_2$ (one of which
  may be $\phi$) such that 
\begin{enumerate}[(i)]
\item  For every $C_i \in \{ C_i \}$ there are intersection
  numbers, say $j(V_1,\break V_2; C_i)$ $\ge 1$ of $V_1$ and $V_2$ along $C_i$
  given by the lengths of certain well-defined primary ideals such
  that
  $$
  \deg (V_1)\cdot \deg (V_2) = \sum_{C_i \in \{ C_i ]\}} j(V_1,
  V_2; C_i)\cdot \deg (C_i), 
  $$
  where we put $\deg (\phi) = 1$.
\item  If $C \subset V_1 \cap V_2$ is an irreducible component of $V_1
  \cap V_2$ then $C_i \in  \{ C_i\}$. 
\item For every $C_i \in  \{ C_i \}$
  $$
  \dim (C_i) \ge \dim (V_1) + \dim (V_2) - n.
  $$
\end{enumerate}

In\pageoriginale order to prove the main theorem
\ref{chap2:sec1:subsec2.1} we need some preliminary results. 

\section{The Join-procedure}\label{chap2:sec2}%sec B

The following notation will be used in the sequel.  

\setcounter{subsection}{1}
\subsection{}\label{chap2:sec2:subsec2.2}

Let $V_1 = V(I_1)$ and $V_2 = V(I_2)$ be two pure dimensional
subvarieties in $\mathbb{P}^n_K$ defined by homogeneous ideals $I_1$
and $I_2 \subset R_0:  = K[X_0, \ldots, ]$. 

We introduce two copies $R_i:  = K[X_{i0}, \ldots,  X_{in}]$, $i = 1,
2$ of $R_0$ and denote $I'_i$ the homogeneous ideal in $R_i$
corresponding to $I_i, i = 1, 2$. 

Put $N:  = 2(n+1)-1, R:  = K[X_{ij}| i=1, 2; 0 \le n]$ and $\tau:  =$
the diagonal ideal in $R$ generated by $\{ X_{ij}-X_{2j} | 0 \le j \le
n\}$.

We introduce new independent variables $U_{kj}$ over $K$, $0 \le j$,
$k \leq n$. 
Let $\bar{K}$ be the algebraic closure of $K(U_{kj} | 0 \le j, k \le
n$). Put $\bar{R} := \bar{K}[X_{ij}|i = 1, 2; 0 \le j \le n]$. Then we introduce
so called \textit{ generic linear forms} $\ell_o, \ldots,  \ell_n:$ 
$$
\ell_k:  = \sum^{n}_{j=0} U_{kj} (X_{1j}-X_{2j}), \text { for }  0 \le
k \le n \text{ in } \bar{R}. 
$$

Note that since $\tau \bar{R}$ is generated by $(n+1)$ - elements and
$\ell_0, \ldots,  \ell_n \in  \bar{R}$, it is clear that $\tau
\bar{R} = (\ell_0, \ldots,  \ell_n) \bar{R}$. 

Let $J(V_1, V_2)$ be the \textit{ join-variety } defined by
$(I'_1+I_2') \bar{R}$ in $\mathbb{P}^N_{\bar{K}}$. 

\setcounter{lemma}{2}
\begin{lemma}\label{chap2:sec2:lem2.3}
  \begin{enumerate}[\rm 1.]
  \item The\pageoriginale ideal $(I'_1 + I'_2) \bar{R}$ is unmixed and
    hence $U ((I'_1  + I'_2)\bar{R} = (I'_1 + I'_2)\bar{R}$. 
  \item ~
    \vskip -1.4cm
    \begin{align*}
      K-\dim & (J(V_1, V_2)) = K-\dim (I'_1 + I'_2) \bar{R}\\ 
      &= K-\dim (I_1) + K-\dim (I_2) = \dim (V_1) + \dim (V_2)+2.\\ 
      K-\dim & ((I'_1+I'_2) \bar{R}+\tau \bar{R}) = K-\dim (I_1+I_2)  = \dim
      V_1 \cap V_2 +1. 
    \end{align*}
  \item There is a one-one correspondence between the isolated prime
    ideals of $(I_1+I_2)$ in $R_0$ and the isolated prime ideals of
    $(I'_1 + I'_2) \bar{R}+\tau \bar{R}$ in $\bar{R}$ and that this
    correspondence preserves dimensions and degrees. 
  \item For every irreducible component $C$ of $V_1 \cap V_2$
    $$
    \dim C \ge \dim V_1 + \dim V_2 - n.
    $$
  \item ~
    \vskip -1.45cm
    \begin{align*}
      \deg (V_1). \deg (V_2) & = h_0(I_1). h_0(I_2)=h_0 ((I'_1 + I'_2)
      \bar{R})\hspace{2cm}\\ 
      & = h_0 ((I'_1 + I'_2) \bar{R}. h_0 (\tau R).
    \end{align*}
  \end{enumerate}
\end{lemma}

\begin{proof}%pro
~
  \begin{enumerate}
  \item Follows from (\ref{chap1:sec4:prop1.47}).
  \item Follows from (\ref{chap1:sec4:prop1.47}).
  \item We have a ring homomorphism
    $$
    \varphi:  \bar{R} \to \bar{R}_0:  \bar{K} [X_0, \ldots,  X_n]
    $$
    given by $X_{ij} \to X_j$ for $i=1,2$ and every $0 \le j \le n$. It is
    easy to see that Ker $\varphi = \tau \bar{R}, \varphi^{-1}
    (\mathscr{Y}) = (\mathscr{Y}' + \tau) \bar{R}$. where $\mathscr{Y}'$
    is\pageoriginale the prime ideal in $\bar{R}_1$ corresponding to the prime ideal
    $\mathscr{Y}$ of $\bar{R}_\circ$ and $ \varphi^{-1} ((I_1 + I_2)
    \bar{R}_\circ) =
    (I'_1 + I'_2) \bar{R} + \tau \bar{R}$. Therefore $\bar{R}/(I'_1 +
    I'_2) \bar{R} + \tau \bar{R} \overset{\sim}{\to} =\bar{R}_0/(I_1 + I_2)
    \bar{R}_0$ and $\mathscr{Y} \leftrightarrow \mathscr{Y}' + \tau
    \bar{R}$ gives $1-1$ correspondence between the isolated prime ideals
    of $(I_1 + I_2) \bar{R}_0$ and the isolated prime ideals of $(I'_1 +
    I'_2) \bar{R} + \tau \bar{R}$ in $\bar{R}$. It is clear that this
    correspondence preserves the dimension and degree. Now (iii) follows
    from (\ref{chap1:sec4:prop1.47}). 
\item This follows from (iii) and the fact that, every isolated
  prime ideal of $(I'_1 + I'_2) \bar{R}$ has Krull dimension $K-\dim
  (I'_1) + K-\dim (I'_2) = \dim V_1 + \dim V_2 + 2$ (see (i) and
  (ii)).  
  
  Therefore\pageoriginale it follows that every isolated prime ideals of  $(I'_1 +
  I'_2) \bar{R}+\tau \bar{R}$ has Krull dimension $\ge \dim (V_1) +
  \dim (V_2)+2- (n+1) = \dim V_1 + \dim V_2 - n + 1$. 
\item We have $h_0 (\tau \bar{R})=1$ by
  (\ref{chap1:sec3:rem1.34}). Therefore we only 
  have to prove $h_0(I_1). h_0 (I_2) = h_0 ((I'_1+ I'_2) \bar{R}$). We
  have $\bar{R}/(I'_1 + I'_2) \bar{R} \overset{\sim}{\to}
  \bar{R}_1/I'_1 \underset{K}{\otimes} \bar{R}_2/I'_2 \simeq
  \bar{R}_0/I_1 \underset{K}{\otimes} \bar{R}_0/I_2$. Therefore
  $H((I'_1 + I'_2) \bar{R}, t) = \sum\limits_{i + j = t} H(I_1
  \bar{R}_0, i)$. $H(I_2 \bar{R}_0, j)$ for all $t \ge 0$. Choose an
  integer $r$ such that the Hilbert functions $H (I_1 \bar{R}_0, i) =:
  H_i$ and $H(I_2 \bar{R}_0, i) =:  H'_i$ are given by polynomials
  $h_i$ and $h'_1$, respectively for $i > r$. Then 
  $$
  \sum^{i = 0}_{t} H_i\cdot H'_{t-i}= \sum^{t}_{i=0} h_i\cdot h'_{t-i} +
  \sum^{r}_{i = 0} (H_i - h_i)\cdot h'_{t-i} + \sum^{t}_{i=t-r} h_i
  (H'_{t-i} -h'_{t-i}) 
  $$ 
  for $n >> 0 (n > 2r)$.
\end{enumerate}
\end{proof}
Therefore it follows from \ref{chap1:sec3:subsec1.32} that

\begin{align*}
  \sum^{t}_{i=0} H_i\cdot H'_{t-i} & = h_0 (I_1
  \bar{R}_0)\cdot h_0 (I_2 \bar{R}_0) \cdot [\sum^{t}_{i=0}
    \binom{i}{d_1} \binom{t-i}{d_2}] + ~\text {
    (other terms)}\\ 
  & = h_0 (I_1 \bar{R}_0)\cdot h_0(I_2 \bar{R}_0) \binom{t}{d_1+d_2+1}\\ 
  & \qquad \qquad + \text{terms with degree (in t)} \le d_1 + d_2.  
\end{align*}

Therefore we get
\begin{align*}
  h_0 ((I'_1+I'_2) R) & = h_0 (I_1 \bar{R}_0). h_0 (I_2 \bar{R}_0)\\
  & = h_0 (I_1). h_0 (I_2) 
\end{align*}
by \ref{chap1:sec4:prop1.46}.

It is clear that Lemma (\ref{chap2:sec2:lem2.3}), the Join-Procedure in
$\mathbb{P}^N_{\bar{K}}$, reduces our considerations to the case that
one variety is a complete intersection of degree 1. 

To calculate $h_0((I'_1 + I'_2) \bar{R})$, we will study the sum ideal
$(I'_1 + I'_2) \bar{R}+\tau \bar{R}$ and the radical (denoted by $\rad 
(\cdots)$) of this ideal. 

\setcounter{notation}{3}
\begin{notation}\label{chap2:sec2:not2.4}
  The following notation will be used in the sequel: 
  \begin{alignat*}{3}
    \delta:  &= K-\dim (( I'_1 + I'_2) \bar{R})  &&=  \dim V_1 + \dim V_2 + 2\\
    d:  &= K-\dim ((I'_1 + I'_2) \bar{R} + \tau \bar{R} && =  K-\dim ((I_1 +
    I_2) R_0)\\ 
    && & = \dim (V_1 \cap V_2) + 1.
  \end{alignat*}
\end{notation}

Let $\mathscr{Y}_{i, j}$ be the minimal prime ideals of $(I'_1 + I'_2)
\bar{R} + \tau \bar{R}$ of Krull dimension $j, 0 \le t \le j \le d \le
\delta$. We thus put:  
\begin{equation}
   \rad  ((I'_1+I'_2) \bar{R} + \tau \bar{R}) = \mathscr{Y}_{1,
    d} \cap \ldots \cap \mathscr{Y}_{m_d,d} \cap \ldots \cap
  \mathscr{Y}_{1, t} \cap \ldots \cap \mathscr{Y}_{m_t,t}, \tag{*} 
\end{equation}
where\pageoriginale $m_d \ge 1, m_{d-1}, \ldots,  m_t \ge 0$ are
integers, and where 
we set $m_j = 0$ for an integer $t \le j \le d-1$ if ($*$) has no
prime ideal of Krull dimension $j$. 

\setcounter{remark}{4}
\begin{remark}\label{chap2:sec2:rem2.5}
  From \ref{chap2:sec2:lem2.3} (iii) it follows that the prime ideals
  $\mathscr{Y}_{i, j} 
  $ in ($*$) of \ref{chap2:sec2:not2.4} are in $1-1$ correspondence
  with the irreducible 
  components of $V_1 \cap V_2$ and that this correspondence preserves
  the dimension and the degree. 
\end{remark}

\setcounter{lemma}{5}
\begin{lemma}\label{chap2:sec2:lem2.6}
  Let $C$ be an irreducible component of $V_1 \cap
  V_2$ and $\mathscr{Y}_{i, j}$ be the prime ideal
  corresponding to $C$ in ($*$)  of \ref{chap2:sec2:not2.4}. Put $\bar{A} =
  (\bar{R}/(I'_1+I'_2) \bar{R})_{\mathscr{Y}_{i, j}}$, the local ring
  of the join-variety $J(V_1, V_2)$ at $\mathscr{Y}_{i, j}$. Then 
  \begin{enumerate}[(i)]
  \item $K-\dim (\bar{A}) = K-\dim (\bar{R}/ (I'_1+I'_2) \bar{R}) -
    K-\dim (\mathscr{Y}_{i, j}) = \delta - j$. 
  \item Let $\mathscr{Y} \subset \bar{R}$ be a prime ideal. Then
  \end{enumerate}
  $\mathscr{Y} \in  \text { Ass} (\bar{R}/(I'_1 + I'_2) \bar{R}+
  (\ell_0, \ldots,  \ell_k) \bar{R})$ with $\mathscr{Y} \subset
  \mathscr{Y}_{ij}$ and 
  $K-\dim (\mathscr{Y})\break = \ell $ if and only if $\mathscr{Y} \bar{A}
  \in  \text {Ass} (\bar{A}/(\ell_0, \ldots,  \ell_k) \bar{A}) $
  and $K-\dim (\mathscr{Y}\bar{A}) = \ell - j$. 
\end{lemma}

\begin{proof}%pro
  Follows from \ref{chap2:sec2:lem2.3} (i) and
  \ref{chap1:sec4:lem1.45}. 
\end{proof}

\setcounter{proposition}{6}
\begin{proposition}\label{chap2:sec2:prop2.7}
  \begin{enumerate}[(i)]
  \item For any $\delta - d$ generic linear forms, say $\ell_0,
    \ldots,\break \ell_{\delta-d-1}$ we have
    $$
    K-\dim ((I'_1+I'_2) \bar{R}+ (\ell_0, \ldots,  \ell_{\delta-d-1}) \bar{R} = d
    $$
    (if $\delta = d$ then we set $(\ell_0,  \ldots,  \ell_{\delta-d-1})
    \bar{R} = (0))$. 
  \item $\delta-t-1 \le n$\pageoriginale and equality holds if and only if $t = \dim
    (V_1) + \dim (V_2) - n + 1$. 
  \end{enumerate}
\end{proposition}

\begin{proof}%pro
  \begin{enumerate}[(i)]
  \item Assume that there exists $k$ such that $0 \le k \le \delta-d-1$
    and $\ell_k \in  \mathscr{Y}$ for some $\mathscr{Y}
    \in $ Ass $(\bar{R}/(I'_1+I'_2) \bar{R} +
    (ell_0, \ldots,  \ell_{k-1}) \bar{R})$ with $K-\dim (\mathscr{Y}=
    \delta -K)$. Let $k+1 \leq m \leq n$. Let $\varphi_m$ be te
    automorphism of $\bar{K}$ over $K$ given
    by $\varphi_m(U_{k\ell}) = U_{m \ell}, \varphi_{m} (U_{m \ell}) =
    U_{k \ell}$ and $\varphi_m (U_{p \ell}) = U_{p \ell}$ for all $0 \le
    p(\neq k, m) \le n $ and $0 \le \ell \le n$. Now, since
    $\mathscr{Y}$ is defined over $K_1 = \overline{K(U_{pj} | 0 \le p
      \le k-1, o \le j \le n)}$ and $\varphi_m (K_1) \subset K_1$, we
    get $\varphi_m (\ell_k) = \ell_m \in  \mathscr{Y}$ and
    therefore $(I'_1+I'_2) \bar{R}+ (\ell_0, \ldots,  \ell_n) \bar{R}
    \subset \mathscr{Y}$. Therefore we get $d=K-\dim ((I'_1+I'_2) \bar{R}
    + \tau \bar{R}) \ge K-\dim (\mathscr{Y}) = \delta-k$, that is,
    $\delta-d-1 \ge k \ge \delta - d$ which is absurd. This proves
    (i). 
  \item From \ref{chap2:sec2:lem2.3} (i) and \ref{chap2:sec2:rem2.5}
    we get $t \ge \dim (V_1) + \dim (V_2)-n + 1 \ge \delta-n-1$. 
  \end{enumerate}
  
  Therefore $\delta-t-1 \le n$ and equality holds if and only if
  $$
  t = \dim (V_1) + \dim (V_2) - n + 1.
  $$
\end{proof}

\begin{proposition}\label{chap2:sec2:prop2.8}
  Let $C$ be an irreducible component of $V_1 \cap V_2$ and
  $\mathscr{Y}_{i, j} $ be the prime ideal corresponding to $C$ in ($*$)
  of \ref{chap2:sec2:not2.4}. Let $\bar{A}= (\bar{R}/(I'_1+I'_2)
  \bar{R})_{\mathscr{Y}_{i, j}}$ be the local ring of the join-variety
  $J(V_1, V_2)$ at $\mathscr{Y}_{i, j}$. Then $\{ \ell_0, \ldots,
  \ell_{\delta-j-1} \}$  is a reducing system of parameters for
  $\bar{A}$. 
\end{proposition}

\begin{proof}
  In\pageoriginale view of \ref{chap2:sec2:lem2.6} (ii) it is enough
  to prove:  
\end{proof}

\begin{enumerate}[(i)]
\item For every $1 \leq k \leq \delta -j-1$,

  $\ell_{k-1}\neq \mathscr{Y}$ for all $\mathscr{Y} \in$ Ass
  $(\bar{R}/(I'_1+I'_2)\bar{R}+(\ell_0,\ldots, \ell_{k-2})\bar{R})$
  with $\mathscr{Y}\subset \mathscr{Y}_{i,j}$ and $k-\dim
  (\mathscr{Y})\geq \delta -k$. 

\item $\ell_{\delta -j-1} \notin \mathscr{Y}$ for all $\mathscr{Y}\in
  Ass
  (\bar{R}/(I'_1+I'_2)\bar{R}+(\ell_0,\ldots,\ell_{\delta-j-2})\bar{R})$
  with $\mathscr{Y}\subset \mathscr{Y}_{i,j}$ and $k-\dim
  (\mathscr{Y})=K-\dim ((I'_1+I'_2)\bar{R}(\ell_0,\ldots, \ell_
  {\delta-j-2})\bar{R})$ 
\end{enumerate}

\noindent \textbf{Proof of (i) :}
  Suppose for some $1\leq k \leq \delta-j-1,\ell_{k-1} \in
  \mathscr{Y}$ for some $\mathscr{Y} \in$ Ass
  $(\bar{R}/(I'_1+I'_2)\bar{R})$ with $\mathscr{Y} \subset
  \mathscr{Y}_{i,j}$ and $K-dim(\mathscr{Y})\geq \delta-k$. Then from
  the proof of \ref{chap2:sec2:prop2.7}(i) we get 
$$
(I'_1+I'_2)\bar{R}+ \tau \subset \bar{R} \subset \mathscr{Y}\subset
\mathscr{Y}_{i, j}. 
$$

Therefore $\mathscr{Y}=\mathscr{Y}_{i,j}$ and $K-\dim(\mathscr{Y})=j
\geq \delta -k$. This shows that $j \geq \delta -k \geq \delta-\delta
+j+1 = j+1$ which is absurd. 

\noindent \textbf{Proof of (ii) :}
  From (i) we get

$K-dim(I'_1+I'_2)\bar{R}+(\ell_0,\ldots,
\ell_{\delta-j-2}\bar{R})=\delta-(\delta-j-1)=j+1$. If
$l_{\delta-j-1}\in \mathscr{Y}$ for same $\mathscr{Y} \in
Ass(\bar{R}/(I'_1+I'_2)\bar{R}+(\ell_o, \ldots,
\ell_{\delta-j-2})\bar{R})$ with $\mathscr{Y}\subset \mathscr{Y}_{i,
  j}$. Then by the same argument in (i) above we get $\mathscr{Y} =
\mathscr{Y}_{i, j}$. Therefore $K-dim(\mathscr{Y}) = j$. This proves
(ii). 

\section{Step I of the Proof}\label{chap2:sec3}

\begin{step}\label{chap2:sec3:step1}
%\noindent \textbf{Step I:}
  In this step, we define the intersection numbers $j (V_1, V_2 ; C)$
  of $V_1$ and $V_2$ along $C$, where $C$ is an irreducible component
  of $V_1 \bigcap V_2$ with $\dim(C)=dim (V_1 \bigcap V_2)$. 
\end{step}

\textbf{The\pageoriginale following notation will be used in the sequel.}
\begin{align*}
    [(I'_1+I'_2)\bar{R}]_{-1} &:= (I'_1+I'_2)\bar{R}\\
    [(I'_1+I'_2)\bar{R}]_{k} &: = \cup ([(I'_1+I'_2)\bar{R}]_{k-1})+\ell_k \bar{R}
  \end{align*}
  for any $0 \leq k \leq \delta-d-1$.

\setcounter{remarks}{8}
\begin{remarks}\label{chap2:sec3:rem2.9}
  \begin{enumerate}[(i)]
  \item $(I'_1+I'_2)\bar{R}\subset (I'_1+I'_2)\bar{R}+(\ell_0, \ldots,
    \ell _k)\bar{R}\subset [(I'_1+I'_2)\bar{R}]_k$ 
    for every $0 \leq k \leq \delta-d-1$.
    
  \item It follows from the lemma \ref{chap2:sec2:prop2.7} (i) and the repeated
    application of \ref{chap1:sec4:prop1.46}(ii) that 
    \begin{align*}
      \rad (\cup([(I'_1+I'_2)_{k-1}) &= \rad
        (\cup(\cup([I'_1+I'_2])+l_{k-2} \bar{R})\\ 
        &=\rad (\cup([I'_1+I'_2]_{k-2})+l_{k-1}\bar{R})\\
        &= \ldots\\
        &=\rad (I'_1+I'_2)\bar{R}+(\ell_0,\ldots, \ell_{k-1})\bar{R}
    \end{align*}
    for every $0 \leq k \leq \delta-d$.    
  \item From (ii), we get
  \end{enumerate}
  $$
  (\cup ([(I'_1+I'_2)\bar{R}]_{k-l}):\ell_k)=\cup ([(I'_1+I'_2)\bar{R}]_{k-l})
  $$
  for every $0 \leq k \leq \delta-d-1$.
\end{remarks}

Now we study the primary decomposition of $\cup ([(I'_1 +
  I'_2)\bar{R}]_{\delta-d-l})$ in the following lemma. 

\setcounter{lemma}{9}
\begin{lemma}\label{chap2:sec3:lem2.10}
\begin{enumerate}[\rm (i)]
\item The primary decomposition of 
$$
U\cup
  ([(I'_1+I'_2)\bar{R}]_{\delta-d-1})
$$ 
is given by\pageoriginale 
$$
  \cup ([(I'_1+I'_2)\bar{R}]_{\delta-d-l})=q_{l,d} \bigcap \ldots
  q_{m_d}, d \bigcap \mathscr{O}_1 
  $$
  where $q_{i, d}$ are primary ideals belonging to the prime ideals
  $\mathscr{Y}_{i,d}$ in $(*)$ of  (\ref{chap2:sec2:not2.4}) $1\leq i
  \leq m_d$ and 
  $\mathscr{O_1}$ is the intersection of all other primary component
  of $\cup ([I'_1+I'_2)\bar{R}]_{\delta-d-1} $). 
\item ~
\vskip -1.45cm
  \begin{align*}
    \deg (V_1)\cdot \deg (V_2) & = h_o ((I'_1 + I'_2)\bar{R})=h_o (\cup
    ([I'_1 + I'_2)\bar{R}]_{\delta-d-1}))\\ 
    & = \sum^{m_d}_{i=1} (\text {length of}q_{i,d})\cdot h_o
    (\mathscr{Y}_{i,d}) + h_o (\mathscr{O}) 
  \end{align*}
\item Every prime ideals $\mathscr{Y}_{i, j}$ in $(*)$ of
   (\ref{chap2:sec2:not2.4}) with $t \leq j \leq d-1$ contains $\mathscr{O}_1$. In
   particular, if $V_1\cap V_2$ has an irreducible component of Krull
   dimension $\leq d-1$ then $\mathscr{O}_1 \neq \bar{R}$. 
\item Every associated prime ideal $\mathscr{Y}$ of
  $\mathscr{O}_1$ has Krull dimension $d$. 
\item The diagonal ideal $\tau \bar{R}$ is not contained in
  any associated prime of $\mathscr{O}_1$. 
\end{enumerate}
\end{lemma}
 
 \begin{proof}
\begin{enumerate}[(i)]
 \item From (\ref{chap2:sec2:prop2.7})(i) and
   (\ref{chap2:sec3:rem2.9})(ii), we have  
   \begin{multline*}
   K-\dim (\cup
   ([I'_1+I'_2)\bar{R}]_{\delta-d-1}))=\\
   K-\dim
   (I'_1+I'_2)\bar{R}+(l_0,\cdots, l_{\delta -d-1})\bar{R})=d
   \end{multline*}
   and   
   \begin{multline*}
   \rad ~ (\cup ([I'_1+I'_2)\bar{R}]_{\delta-d-1}))=\\
   \rad ~
   ((I'_1+I'_2)\bar{R}+(\ell_0,\ldots, \ell_{\delta
   -d-1})\bar{R})\subset \mathscr{Y}_{i,d}
   \end{multline*}
   for every $1 \leq i \leq
   m_d$. Therefore $\mathscr{Y}_{i,d}$ is associated to $\cup
   ([I'_1+I'_2)\bar{R}]_{\delta-d-1})$ for every $1 \leq i \leq m_d$. 
\item From\pageoriginale \ref{chap1:sec3:subsec1.40} and
  (\ref{chap2:sec3:rem2.9})(iii), \ref{chap1:sec3:subsec1.36}(iii), we get 
  \begin{align*}
    h_o (\cup ([I'_1+I'_2)\bar{R}]_{\delta-d-1}))& =h_o
  ([I'_1+I'_2)\bar{R}]_{\delta-d-2})+\ell_{\delta-d-1}\bar{R}))\\
    & =h_0 (U([I'_1+I'_2)\bar{R}]_{\delta-d-2}))\\
    & = \ldots=h_0
  (I'_1+I'_2)\bar{R})=\deg (V_1)\cdot\deg (V_2) 
  \end{align*}
\item We have from the proof of (i) that
  $$
  \rad ~ (\cup([I'_1+I'_2)\bar{R}]_{\delta-d-1}))=\mathscr{Y}_{1,d}\cap
  \cdots \cap \mathscr{Y}_{m_d,d} \cap \rad  (\mathscr{O})\subset
  \mathscr{Y}_{i, j}
  $$ 
  for every $t \leq j \leq d-1$ and for all
  $i$. Therefore we get $\mathscr{O}_1 \subset \rad 
  (\mathscr{O}_1)\subset \mathscr{Y}_{i,j}$ for every $t \leq j \leq
  d-1$ and for all $i$. 
\item Clear.
\item If $\tau \bar{R} \subset \mathscr{Y}$ for some
  $\mathscr{Y}\in  Ass (\bar{R}/\mathscr{O}_1)$ then
  $(I'_1+I'_2)\bar{R}+\tau \bar{R}\subset \rad 
  ((I'_1+I'_2)\bar{R})\bar{R}+\tau \bar{R})\subset
  \mathscr{Y}$. Therefore $\mathscr{Y}=\mathscr{Y}'_{i,d}$ because
  $K-\dim (\mathscr{Y})=d$ (from~(iv)). This is contradiction!  
\end{enumerate}
\end{proof}

\setcounter{definition}{10}
\begin{definition}\label{chap2:sec3:def2.11}
  Let $C \subset V_1 \cap V_2$ be an irreducible component with
  $\dim C=\dim (V_1 \cap V_2)=d-1$. Let $\mathscr{Y}_{i, d}$ be the
  prime ideal in $(*)$ of (\ref{chap2:sec2:not2.4}) corresponding $C$
  (see (\ref{chap2:sec2:rem2.5})). We 
  define the \textit{intersection number $j(V_1,V_2;C)$ of $V_1$ and
    $V_2$ along
  } $C$ to be the length of the corresponding $\mathscr{Y}_{i,
    d}$-primary component $q_{i, d-1}$ of $
  \cup([(I'_1 + I'_2)\bar{R}]_{\delta -d -l})$. 
\end{definition}

From (\ref{chap2:sec3:lem2.10})(i) it is clear that, for every
irreducible component $C$ 
of $V_1 \cap V_2$ with $\dim (C)=\dim (V_1 \cap V_2)$ the intersection
number $j(V_1,V_2;C)$ of $V_1$  and $V_2$ along $C$ is defined and
$j(V_1, V_2; C) \geq l$.
  
\setcounter{remarks}{11}
\begin{remarks}\label{chap2:sec3:lem2.12}
  \begin{enumerate}[(i)]
  \item It follows (\ref{chap2:sec3:lem2.10})(ii) and the definition
    (\ref{chap2:sec3:def2.11})\pageoriginale that 
    $$
    \deg (V_1)\cdot\deg (V_2)= \sum\limits_C j(V_1,V_2;C)\cdot\deg
    (C)+h_0(\mathscr{O}_1),
    $$ 
    where $C$ runs through all irreducible
    components of $V_1 \cap V_2$ with $\dim (C) =\dim (V_1 \cap V_2)$. 
  \item If $t=d$, then our algorithm stops.
    
    Assume that $t < d$, that is, $V_1 \bigcap V_2$ has irreducible
    components of Krull dimension $\leq d-1$. Therefore by
    (\ref{chap2:sec3:lem2.10}) (iii)
    $\mathscr{O}\not\neq \bar{R}$. In the next step we apply modified
    procedure to study $h_0(\mathscr{O}_1)$.  
  \end{enumerate}
\end{remarks}

\section{Step II of the Proof}\label{chap2:sec4}

\begin{step}\label{chap2:sec4:step2}
%\noindent \textbf{Step II: }
 In the step, we define the intersection numbers $j(V_1, V_2; C)$ of
  $V_1$ and $V_2$ along $C$ in the following two cases:  
 
 \begin{enumerate}[(i)]
\item If $C$ is an irreducible component of $V_1 \cap V_2$ of Krull
  dimension $\leq d-1$. 
\item Certain imbedded irreducible subvarieties $C$ of $V_1 \cap V_2$
  with $t \leq K-\dim (C) \leq d-1$. 
 \end{enumerate}
\end{step}

\setcounter{subsection}{12}
\subsection{}\label{chap2:sec4:subsec2.13}
  From (\ref{chap2:sec3:lem2.10})(ii) we have $(l_{\delta-d}, \ldots,
  l_n)\bar{R}\not\subset \mathscr{Y}$ for every prime ideal
  $\mathscr{Y}\in Ass (\bar{R}/\mathscr{O}_1)$. It follows from the
  proof of the proposition (\ref{chap2:sec2:prop2.7})(i) that $\ell_r
  \notin \mathscr{Y}$ 
  for every $\mathscr{Y}\in Ass (\bar{R}/\mathscr{O}_1)$ with $n \geq r
  \geq \delta -d$. Consider $l_{\delta-d}$, we have 

  Therefore $K-\dim (\mathscr{O}_1+l_{\delta-d}\bar{R})=K-\dim
  (\mathscr{O}_1)-1=d-1$ and $(\mathscr{O}_1:
  l_{\delta-d'})=\mathscr{O}_1$\pageoriginale
  
  Now we study the primary decomposition of the ideal
  $U(\mathscr{O}_1+l_{\delta-d'} \bar{R})$. Every primary component
  $q$ of $U(\mathscr{O}_\ell+\ell_{\delta-d'} \bar{R})$ belongs to one
  of the following three cases: 
  \begin{case}\label{chap2:sec4:case1}
    $q$ is $\mathscr{Y}$- primary such that there is a prime ideal
    $\mathscr{Y}_{i,j}$ in $(*)$ of (\ref{chap2:sec2:not2.4}), $t \leq
    j \leq d-1$ with $\mathscr{Y}=\mathscr{Y}_{i,j}$. 
  \end{case}

  \begin{case}\label{chap2:sec4:case2}
    $q$ is $\mathscr{Y}$-primary such that there is a prime ideal
    $\mathscr{Y}_{ij}$ in $(*)$ of (\ref{chap2:sec2:not2.4}) with
    $\mathscr{Y}_{i,j}\subset \mathscr{Y}$. 
  \end{case} 

  \begin{case}\label{chap2:sec4:case3}
    $q$ is $\mathscr{Y}$-primary such that
    $\mathscr{Y}_{i,j} \not\subset \mathscr{Y} $ for all
    prime ideals $\mathscr{Y}_{i,j}$ in $(*)$ of (\ref{chap2:sec2:not2.4}) 
  \end{case}

Let $\cup(\mathscr{O}_\ell+l_{\delta-d'} \bar{R})=\bigcap q_1 \cap
\bigcap q_2 \cap \bigcap q_3$ be the primary decomposition of
$U(\mathscr{O}_1+\ell_{\delta-d'} \bar{R})$, where $q_1, q_2$ and
$q_3$ run through the primary components of
$U(\mathscr{O}_1+\ell_{\delta-d'} \bar{R})$ which appear in case
(\ref{chap2:sec4:case1}), case (\ref{chap2:sec4:case2}) and case (\ref{chap2:sec4:case3}), respectively. If there is no primary
component in case (\ref{chap2:sec4:case1}), case
(\ref{chap2:sec4:case2}) or case (\ref{chap2:sec4:case3}) then we set $\bigcap 
q_i=\bar{R}$ for $i=1,2$ or $3$. We put $\mathscr{O}_2: =  \bigcap
q_3$. We then have 

\setcounter{lemma}{13}
\begin{lemma}\label{chap2:sec4:lem2.14}
  \begin{enumerate}[(i)]
  \item If $V_1 \cap V_2$ has irreducible components of Krull
    dimension $d-1$, then $\mathscr{Y}_1$ runs through prime ideals
    $\mathscr{Y}_{i,d-1} ~1 \leq i \leq m_{d-1}$ in $(*)$ of
    (\ref{chap2:sec2:not2.4}).  
  \item $h_0 (\mathscr{O}_1)=\sum \limits_{q_1} (\text{length of}~
    q_1)\cdot h_0 (\mathscr{Y}_1)+\sum \limits_{q_2}(\text{length of}~
    q_2)h_0 (\mathscr{Y}_2)+h_o(\mathscr{O}_2)$. 
  \item Every\pageoriginale prime ideal $\mathscr{Y}_{i, j}$ in $(*)$
    of (\ref{chap2:sec2:not2.4}) 
    with $t \leq j \leq d-2$ contains $\mathscr{O}_2$. In particular,
    if $V_1 \cap V_2$ has an irreducible component of Krull dimension
    $\leq d-2$ then $\mathscr{O}_2 \neq \bar{R}$. 
  \item Every associated prime ideal $\mathscr{Y}$ of
    $\mathscr{O}_2$ has Krull dimension $d-1$. 
  \item The diagonal ideal $\tau \bar{R}$ is not contained in any
    associated prime ideal of $\mathscr{O}_2$. 
\end{enumerate}
\end{lemma}

\begin{proof}
 \begin{enumerate}[(i)]
  \item From (\ref{chap2:sec3:lem2.10})(iv) and
    (\ref{chap2:sec3:lem2.10})(iii), we have $\mathscr{O}_1 
    \subset \mathscr{Y}_{i,d-1}$ for every $1\leq i \leq
    m_{d-1}$. Therefore\pageoriginale $(\mathscr{O}_1+\ell_{\delta
      -d}\bar{R})\subset \mathscr{Y}_{i,d-1}$ for every $1\leq i \leq
    m_{d-1}$. Since $K-\dim
    (\mathscr{O}_1+l_{\delta-d}\bar{R})=d-1=K-\dim
    (\mathscr{Y}_{i,d-1})$ it follows that $\mathscr{Y}_{i,d-1}$ is
    associated to $(\mathscr{O}_1+l_{\delta-d}\bar{R})$ for every $1
    \leq i \leq m_{d-1}$.  
  \item From \ref{chap1:sec3:subsec1.36} (iii) and
    \ref{chap1:sec3:subsec1.40}, we get 
    \begin{multline*}
      h_0(\mathscr{O}_1)=h_0 (\mathscr{O}_1+l_{\delta-d}\bar{R}) =h_0
      (U(\mathscr{O}_1+l_{\delta-d}\bar{R}))\\ 
      =\sum _{q_1}(\text{length of } q_1)
      h_0(\mathscr{Y}_1)+\sum_{q_2} (\text {length of } q_2)\cdot h_0
      (\mathscr{Y}_2)+h_0(\mathscr{O}_2) 
    \end{multline*}
  \item From (\ref{chap2:sec3:lem2.10}) (iii), we have $(\mathscr{O}_1)\subset
    \mathscr{Y}_{i,j}$ for every $t \leq j \leq d-2$. Therefore
    $(\mathscr{O}_1 + \ell_{\delta - d} \bar{R}) \subset
    \mathscr{Y}_{i, j}$  for every $t\leq j \leq d-2$. Now it follows
    from (\ref{chap1:sec4:lem1.45}) (ii) that 
    \begin{multline*}
      \rad 
      (\cup(\mathscr{O}_1+l_{\delta-d}\bar{R}))\\
      =\mathscr{Y}_{i,d-1}\bigcap
      \cdots \bigcap \mathscr{Y}_{m_{d-l},d-l} \bigcap \cap
      \mathscr{Y}_2 \bigcap \rad  (\mathscr{O}_2)\\ 
      =\rad  (\mathscr{O}_1+l_{\delta-d}\bar{R}) \subset
      \mathscr{Y}_{i,j} ~\text{for every}~ t \leq j \leq d-2 
    \end{multline*}
    Therefore $\mathscr{O}_2 \subset \rad (\mathscr{O}_2)\subset
    \mathscr{Y}_{i,j}$ for every $t \leq j \leq d-2$. 
  \item Clear.
  \item If $\tau \bar{R} \subset \mathscr{Y}$ for some $\mathscr{Y}\in
    Ass (\bar{R}/\mathscr{O}_2)$ then $(I'_1+I'_2)+ \tau \subset
    \mathscr{Y}$. Therefore $\mathscr{Y}_{i,j} \subset \mathscr{Y}$
    for some $t \leq j \leq d$ and some $i$. This is a contradiction
    (see (i)). 
\end{enumerate}
\end{proof} 
 
\setcounter{definition}{14}
\begin{definition}\label{chap2:sec4:def2.15}
  \begin{enumerate}[\rm (a)]
  \item Let $C \subset V_1 \cap V_2$ be an irreducible component with
    $K-\dim (C)=d-1$. Let $\mathscr{Y}_{i,d-1}$ be the prime ideal in
    $(*)$ of (\ref{chap2:sec2:not2.4}) corresponding to $C$ (see
    (\ref{chap2:sec2:rem2.5})). We define \textit
    {the intersection number $J(V_1,V_2;C)$ of\pageoriginale $V_1$
      and $V_2$ along 
      $C$} to be the length of the corresponding
    $\mathscr{Y}_{i,d-1}$-primary component $q_{i,d-1}$ of
    $\cup(\mathscr{O}_1+l_{\lambda-d}\bar{R})$ (see
    (\ref{chap2:sec4:lem2.14})(i)).  
    
    From (\ref{chap2:sec4:lem2.14})(i), it is clear that, for every
    irreducible component 
    $C$ of $V_1 \bigcap V_2$ with $K-\dim (C)=d-1$ the intersection
    number $j(V_1,V_2;C)$ of $V_1$ and $V_2$ along $C$ is defined and
    $j(V_1,V_2;C) \geq 1$. 

    From (\ref{chap2:sec2:lem2.3})(iii), it follows that the prime
    ideals $\mathscr{Y}_2$ 
    which appear in case (\ref{chap2:sec4:case2}) of \ref{chap2:sec4:subsec2.13}
    corresponds to certain imbedded 
    irreducible subvariety of $V_1 \cap V_2$. 
  \item Let $C \subset V_1 \cap V_2$ be an irreducible subvarieties of
    $V_1 \cap V_2$ corresponding to the prime ideal $\mathscr{Y}_2$
    which appear in case (\ref{chap2:sec4:case2}) of
    \ref{chap2:sec4:subsec2.13}. We define \em{the
      intersection number $j(V_1,V_2;C)$ of $V_1$ and $V_2$ along $C$}
    to be the length of the corresponding $\mathscr{Y}_2$ -primary
    component $q_2$ of $U(\mathscr{O}_1 +l_{\delta -d}\bar{R})$ (see
    \ref{chap2:sec4:subsec2.13}). It is clear that $j(V_1,V_2;C)\geq 1$. 
  \end{enumerate}
\end{definition}

\setcounter{remarks}{15}
\begin{remarks}\label{chap2:sec4:rem2.16}
  \begin{enumerate}[(i)]
  \item Put $c_1:=\sum \limits_{q_2}(\text {length of }q_2)h_0
    (\mathscr{Y}_2)$, where $q_2$ runs\break  through all primary components
    of $U(\mathscr{O}_1 +l_{\delta -d}\bar{R})$ which appears in case
    (\ref{chap2:sec4:case2}) of \ref{chap2:sec4:subsec2.13}. Then it
    follows from (\ref{chap2:sec3:lem2.12})(i) 
    and (\ref{chap2:sec4:lem2.14})(ii) that  

    $\deg (V_1). \deg (V_2)=\sum \limits_{C} j(V_1,V_2;C). \deg
    (C)+c_1+h_0(\mathscr{O}_2)$. 

    where $C$ runs through all irreducible components of $V_1 \bigcap
    V_2$ with $d-1 \leq K-\dim (C)\leq d$.  
  \item If $t=d-1$, then our algorithm of step \ref{chap2:sec4:step2}
    stops.  

\setcounter{subsection}{16}
\subsection{}\label{chap2:sec4:subsec2.17} Assume
    that $t < d-1$, that is $V_1 \bigcap V_2$ has irreducible component
    of Krull dimension $\leq d-2$. Therefore by (\ref{chap2:sec4:lem2.14})(iii),
    $\mathscr{O}_2 \neq \bar{R}$. Then we again apply the above procedure
    to the ideal $\mathscr{O}$. In general, the application of our
    algorithm to the ideal $\mathscr{O}_s \leq s \leq d-t$ is given by the
    following considerations.  
    
    Suppose the ideals $\mathscr{O}_2, \cdots, \mathscr{O}_s, 2 \leq s
    \leq d-t$ are already defined. Consider the ideal
    $(\mathscr{O}_sl_{\delta-d+s-1}\bar{R})$. Then we have  
  \end{enumerate}
\end{remarks}

\begin{enumerate}[(i)]
\item $K-\dim (\mathscr{O}_s+\ell_{\delta-d+s-1}\bar{R})=K-\dim
  (\mathscr{O}_s+l_{\delta -d+s-1}\bar{R})=d-s$ \textit{and}
  $(\mathscr{O}_s:\ell_{\delta -d+s-1})=\mathscr{O}_s$ 

  Let $(\mathscr{O}_s+\ell_{\delta -d+s-1}\bar{R})=\bigcap q_1\cap
  \bigcap q_2 \cap \bigcap q_3$ be the primary decomposition of\pageoriginale
  $U(\mathscr{O}_s+\ell_{\delta -d+s-1}\bar{R})$, which appear in case
  (\ref{chap2:sec4:case1}), case (\ref{chap2:sec4:case2}) and case
  (\ref{chap2:sec4:case3}) of \ref{chap2:sec4:subsec2.13},
  respectively. We put 
  $\mathscr{O}_{s+1}:=\bigcap q_3$. Then we have 
\item If $V_1 \bigcap V_2$ has irreducible components of $K$-dimension
  $d-s$ then $\mathscr{Y}_1$ runs through the prime ideals
  $\mathscr{Y}_{i, d-s}$ in $(*)$ of (\ref{chap2:sec2:not2.4}). 
\item ~
\vskip -1.5cm
  \begin{align*}
    h_0 (\mathscr{O}_s) & = h_0(\mathscr{O}_s+l_{\delta
      -d+s-1}\bar{R})=h_0(U(\mathscr{O}_s+l_{\delta
      -d+s-1}\bar{R}))\\ 
    & = \sum _{q_1}(\text{length of }(q_1)\cdot h_0(\mathscr{Y}_1)\\
    & + \sum _{q_2}(\text{length of }(q_2).h_0
    (\mathscr{Y}_2)+h_0(\mathscr{O}_{s+1}). 
  \end{align*} 
  We put $c_s=\sum \limits_{q_2} (\text{ length of }(q_2).h_0
  (\mathscr{Y}_2)$, where $q_2$ runs through all primary
    components of $U(\mathscr{O}_s+\ell_{\delta -d+s-1}\bar{R})$ which
    appear in case (\ref{chap2:sec4:case2}) of \ref{chap2:sec4:subsec2.13}. 
  \item Every prime ideal $\mathscr{u}_{i, j}$ in $(*)$ of
    (\ref{chap2:sec2:not2.4}) with $t \leq j \leq d-s-1$ contains $\mathscr{O}_{s+1}$. In
    particular, if $V_1 \bigcap V_2$ has an irreducible of Krull
    dimension $\leq d-s-1$ then $\mathscr{O}_{s+1}\neq \bar{R}$. 
  \item Every associated prime $\mathscr{Y}$ of
    $\mathscr{O}_{s+1}$ has Krull dimension $d-s$. 
  \item The diagonal ideal $\tau \bar{R}$ is not contained in
    any associated prime ideal of $\mathscr{O}_{s+1}$ 
 
    In any case, our algorithm of Step \ref{chap2:sec4:step2} stops if we have
    constructed the\pageoriginale ideal $\mathscr{O}_{d-t+1}$. We
    obtain this ideal 
    by studying the primary decomposition of
    $U(\mathscr{O}_{d-t}+\ell_{\delta-t-1}\bar{R})$. Therefore the
    last step yields the following result:  
    \begin{align*}
      h_0(\mathscr{O}_{d-t}) & = h_0(\mathscr{O}_{d-t}+\ell_{\delta-t-1}
      \bar{R}) = h_0(\cup(\mathscr{O}_{d-t}+\ell_{\delta-t-1} \bar{R}))\\ 
      & = \sum_{i=0}^{m_t} ~\text{ (length of }q_{i,t}) h_0 (\mathscr{Y}_{i,t})
      + c_{d-t}+h_0 (\mathscr{O}_{d-t+1}) 
    \end{align*} 
    where $q_{i,t}$ is the $\mathscr{Y}_{i,t}$-primary component of
    $U(\mathscr{O}_{d-t}+\ell_{\delta-t-1} \bar{R})$ for all $1 \leq i
    \leq m_t$ and $c_{d-t}=\sum \limits_{q_2}$ (length of $q_2$)
    $h_0(\mathscr{Y}_2)$, where $q_2$ runs through all primary
    components of $U(\mathscr{O}_{d-t}+\ell_{\delta-t-1} \bar{R})$
    which appear in case (\ref{chap2:sec4:case2}) of
    \ref{chap2:sec4:subsec2.13}. 
\end{enumerate}

Summarizing all these we have: 
\setcounter{subsection}{17}
\subsection{}\label{chap2:sec4:subsec2.18}
\begin{enumerate}[(i)]
\item For every irreducible component $C $ of $V_1 \bigcap V_2$ we
  have defined the intersection number $j(V_1,V_2;C)$ of $V_1$ and
  $V_2$ along $C$. Moreover, $j (V_1, V_2; C) \geq 1$ and
  $j(V_1, V_2; C)$ is the length of the corresponding $\mathscr{Y}_{i,
    j}$-primary component $q_{i, j}$ of
  $\cup(\mathscr{O}_{d-t}+\ell_{\delta-t-1} \bar{R}),t \leq j\leq d$ (see
  \ref{chap2:sec4:subsec2.17} (ii)). 
\item We have collected certain imbedded irreducible subvarieties of
  $V_1 \bigcap V_2$ corresponding to the primary components of
  $\cup(\mathscr{O}_{d-t}+\ell_{\delta-t-1} \bar{R})$, $1 \leq s\leq d-t$
  which appear in case (\ref{chap2:sec4:case2}) of
  \ref{chap2:sec4:subsec2.13}. For every imbedded irreducible 
  subvariety $C$ of $V_1 \bigcap V_2$ in this collection we have
  defined the intersection number $j(V_1, V_2;C)$ of $V_1$ and $V_2$
  along $C$. Moreover, $j(V_1,V_2;C)\geq 1$ and $j(V_2,V_2;C)$ is the
  length of the corresponding $\mathscr{Y}_2$-primary\pageoriginale component$q_2$ of
  $U(\mathscr{O}_{d-t}+l_{\delta-t-1} \bar{R}),1\leq s \leq d-t$ which
  appear in case (\ref{chap2:sec4:case2}) of \ref{chap2:sec4:subsec2.13}. 
\item It follows from (\ref{chap2:sec4:rem2.16})(i) and
  \ref{chap2:sec4:subsec2.17} (iii),(vii) that
  \begin{multline*}
  \deg (V_1)\cdot \deg (V_2)\\
  = \sum \limits_{C} j(V_1,V_2;C)\deg
  (C)+c_1+c_2+ \cdots c_{d-t}+h_0 (\mathscr{O}_{d-t+1}), 
  \end{multline*}
  where $C$
  runs through all irreducible components of $V_1 \bigcap V_2$. We put
  $c(V_1,V_2):=c_1+c_2+ \cdots c_{d-t}+h_0(\mathscr{O}_{d-t+1})$. This
  $c(V_1,V_2)$ is called the \textit{correction term.} 

\item If $\delta-t-1=n$ then $\mathscr{O}_{d-t+1}=\bar{R}$.
\end{enumerate}  

\begin{proof}
  If $\delta-t-1=n$ then $(I'_1+I'_2)\bar{R}+ \tau \bar{R}\subset
  \mathscr{O}_{d-t+1}$. Therefore, if $\mathscr{O}_{d-t+1}\neq
  \bar{R}$ then for every associated some prime ideal $\mathscr{Y}$ of
  $\mathscr{O}_{d-t+l}$ contains some prime ideal 
  $\mathscr{Y}_{i, j}$ in $(*)$ of (\ref{chap2:sec2:not2.4}). This is
  a contradiction (see \ref{chap2:sec4:subsec2.17}).  
\end{proof}  
 
 We note the following important observation from Step
 \ref{chap2:sec4:step2}. 
  
\setcounter{lemma}{18}
\begin{lemma}\label{chap2:sec4:lem2.19}
  Let $C$ be an irreducible component of $V_1 \bigcap V_2$. Let
  $\mathscr{Y}_{i, j}$ be the prime ideal corresponding to $C$ in
  $(*)$ of (\ref{chap2:sec2:not2.4}). Let
  $\bar{A}=(\bar{R}/(I'_1,I'_2)\bar{R})_{\mathscr{Y}_{i,j}}$ be the
  local ring of the join-variety $J(V_1,V_2)$ at
  $\mathscr{Y}_{i,j}$. Then we have 
  \begin{enumerate}[\rm (i)]
  \item
    $\mathscr{O}_{k+1}\bar{A}=U(\mathscr{O}_k+\ell_{\delta-d+k-1}\bar{A})$
    for every $0\leq k \leq d-j-1$, where
      $\mathscr{O}:=U([(I'_1,I'_2)]_{\delta-d-1})$. 
  \item
    $(\mathscr{O}_{d-j}+l_{\delta-j-1})\bar{A}=U(\mathscr{O}_{d-j}+l_{\delta-j-1})
    \bar{A}=q_{i, j}\bar{A}$ \textit{where} $q_{i,j}$ is the
    $\mathscr{Y}_{i, j}$-primary component of
    $U(\mathscr{O}_{d-j}+l_{\delta-j-1}\bar{R}) $ 
  \item  $\mathscr{U}_{d-j+1}\bar{A} = \bar{A}$.
 \end{enumerate} 
\end{lemma}

\begin{proof}%proof
\begin{enumerate}[(i)]
\item From\pageoriginale \ref{chap2:sec4:subsec2.17}(i), we have 
  \begin{multline*}
  K - \dim
  U(\mathscr{O}_k+\ell_{\delta-d+k-1}\bar{R}) = K- \dim
  (\mathscr{O}_{k+1}) = d-k ~\text{and}~\\
  \mathscr{Y}\in \ass(\mathscr{O}_{k+1} \Longleftrightarrow
  \mathscr{Y}\in  \ass
  (U(\mathscr{O}_{k}+\ell_{\delta-d+k-1}\bar{R})) ~\text{and}~
  \mathscr{Y}_{p,\ell} \nsubseteq\mathscr{Y}
  \end{multline*}
  for all $p$ {and}  $t\leq  \ell \leq d$. 

  Therefore, $\mathscr{Y}\in \ass(\mathscr{O}_{k+1})$ and
  $\mathscr{Y}\subset\mathscr{Y}_{i,j}
  \Longleftrightarrow\mathscr{Y}\in  \ass
  (U(\mathscr{O}_{k}+\ell_{\delta-d+k-1}\bar{R})) $ and $
  \mathscr{Y}\underset{+}{\subset}\mathscr{Y}_{i,j}$. This shows that
  $\mathscr{O}_{k+1}\bar{A} =
  U(\mathscr{O}_{k}+\ell_{\delta-d+k-1})\bar{A}$ for every $0\leq k
  \leq d-j-1$. 
\item It follows from \ref{chap2:sec4:subsec2.17})(i) and
  \ref{chap2:sec4:subsec2.17}(ii) that $K-\dim
  (\mathscr{O}_{d-j}+ \ell_{\delta-j-1}\bar{R}))=j$ and
  $\mathscr{Y}_{i,j} \in  \ass(U(\mathscr{O}_{d-j}+\ell_{\delta
  -j-1}\bar{R})) \subset \ass
  (\mathscr{o}_{d-j}+\ell_{\delta-j-1}\bar{R}$. Therefore
  $\mathscr{O}_{d-j}+\ell_{\delta-j-1})\bar{A} =
  U(\mathscr{O}_{d-j}+\ell_{\delta-j-1})\bar{A}= q_{i,j}\bar{A}$,
  where $q_{i,j}$ is the $\mathscr{Y}_{i,j}$-primary component of
  $U(\mathscr{O}_{d-j}+\ell_{\delta-j-1}\bar{R})$. 
\item Since $K-\dim (\mathscr{O}_{d-j+1}) = j = K - \dim
  (\mathscr{Y}_{i,j})$, it follows from the proof of (i) that
  $\ass(\mathscr{O}_{d-j+1}\bar{A} = \phi$. Therefore
  $\mathscr{O}_{d-j+1}\bar{A} = \bar{A}$. 
\end{enumerate}
\end{proof}

\setcounter{corollary}{19}
\begin{corollary}\label{chap2:sec4:coro2.20}
  $$
  e_0((\ell_0,\ldots,\ell_{\delta-j-1}) \bar{A};\bar{A} = \ell(\bar{A}/
  (\mathscr{O}_{d-j}+\ell_{\delta-j-1})\bar{A}) = j(V_1,V_2;C) 
  $$
\end{corollary}

\begin{proof}%proof
  This follows from (\ref{chap2:sec3:rem2.9}),
  (\ref{chap2:sec4:lem2.19})(i) and (ii), (\ref{chap1:sec2:prop1.25})
  and \ref{chap2:sec4:subsec2.18}(i). 
\end{proof}

\section{Step III of the Proof}\label{chap2:sec5}%sec E

\begin{step} \label{chap2:sec5:step3}
%\noindent \textbf{Step III.} 
  In this step we collect certain imbedded irreducible
  subvarieties\pageoriginale of 
  $V_1 \cap V_2$ with $t-s \leq K - \dim (C) < t$, where $s=
  n-\delta+t+1\geq 0$ (see (\ref{chap2:sec2:prop2.7})(ii)). 
\end{step}

\setcounter{subsection}{20}
\subsection{} \label{chap2:sec5:subsec2.21}
From (\ref{chap2:sec2:prop2.7})(ii), we have $\delta-t-1\leq n$. If
$\delta-t-1 = n$ then 
$\mathscr{O}_{d-t+1} = \bar{R}$ (see \ref{chap2:sec4:subsec2.18}(iv))
and our algorithm stops. 

Assume that $\delta -t-1<n$ and $\mathscr{O}_{d-t+1} \neq
\bar{R}$. Put $\delta -t-1+s = n$ for some $s>0$. To calculate
$h_0(\mathscr{O}_{d-t+1})$, we study the primary decomposition of the
ideal $U(\mathscr{O}_{d-t+1}+\ell_{\delta-t}\bar{R})$. Every primary
component $q$ of $U(\mathscr{O}_{d-t+1}+\ell_{\delta-t}\bar{R}$
belongs to one of the following two cases: 

\setcounter{case}{0}
\begin{case}\label{chap2:sec5:case1}%case 1
  $q$ is $\mathscr{Y}$-primary such that there is a prime ideal
  $\mathscr{Y}_{i,j}$ in $(*)$ of (\ref{chap2:sec2:not2.4}) such that
  $\mathscr{Y}_{i,j}\underset{+}{\subset} \mathscr{Y}$. 
\end{case}

\begin{case}\label{chap2:sec5:case2}%case 2
$q$ is $\mathscr{Y}$-primary such that the
  $\mathscr{Y}_{i,j}\not\subset\mathscr{Y}$ for all prime ideals
  $\mathscr{Y}_{i,j}$ in (*) of (\ref{chap2:sec2:not2.4}). 
\end{case}

Let $U(\mathscr{O}_{d-t+1}+\ell_{\delta-t}\bar{R})= \cap q_1 \cap
\bigcap q_2$  be the primary decomposition of
$U(\mathscr{O}_{d-t+1}+\ell_{\delta-t}\bar{R})$, where $q_1$ and $q_2$
run through the primary components of
$U(\mathscr{O}_{d-t+1}+\ell_{\delta-t}\bar{R}$ which appear in case
(\ref{chap2:sec5:case1}) and case (\ref{chap2:sec5:case2}),
respectively. We put $\mathscr{O}_{d-t+2} \coloneqq \cap q_2$. Then we have 
\begin{align*}
  h_0(\mathscr{O}_{d-t+1}) & = h_0(\mathscr{O}_{d-t+1}+\ell_{\delta-t}\bar{R})\\
  & = \sum_{q_1} (\text{ length of } q_1) h_0(\mathscr{Y}_{1}) +
  h_0(\mathscr{O}_{d-t+2}). 
\end{align*}
where $q_1$ runs through the primary components of
$U(\mathscr{O}_{d-t+1}+\ell_{\delta-t}\bar{R})$ which appear in case
(\ref{chap2:sec5:case1}). 

From\pageoriginale (\ref{chap2:sec2:lem2.3})(iii), it follows that the
prime ideals $\mathscr{Y}_1$ 
which appear in case (\ref{chap2:sec4:case2}) corresponds to certain
imbedded irreducible subvarieties of $V_1 \cap V_2$. 

\setcounter{definition}{21}
\begin{definition}\label{chap2:sec5:def2.22}
  Let $C\subset V_1 \cap V_2$ be an irreducible subvariety of $V_1
  \cap V_2$ corresponding to the prime ideal $\mathscr{Y}_1$ which
  appear in case (\ref{chap2:sec4:case1}) of
  (\ref{chap2:sec5:subsec2.21}). We define \textit{ the intersection 
    number } $j(V_1,V_2;C)$ \textit{ of } $V_1$ \textit{ and } $V_2$
  \textit{ along } $C$ to be the length of the corresponding
  $\mathscr{Y}_1$-primary component $q_1$ of
  $U(\mathscr{O}_{d-t+1}+\ell_{\delta-t}\bar{R})$ (see
  \ref{chap2:sec5:subsec2.21}). It is
  clear that $j(V_1,V_2;C)\geq 1$. Assume $\mathscr{O}_{d-t+2} =
  \bar{R}$; then our algorithm of step \ref{chap2:sec5:step3} stops. 
\end{definition}

\setcounter{subsection}{22}
\subsection{}\label{chap2:sec5:subsec2.23}
If $\mathscr{O}_{d-t+2}\neq\bar{R}$, then we repeat the above
procedure to the ideal $\mathscr{O}_{d-t+2}$ by using $\ell_{\delta
  -t+1}$. 

In general, the application of our algorithm to the ideal
$\mathscr{O}_{d-t+k},1\leq k \leq s = n-\delta +t+1$ is given by the
following considerations: 

Suppose the ideals $\mathscr{O}_{d-t+2},\ldots,\mathscr{O}_{d-t+k}$
are already defined for $2\leq k \leq s$. Then consider the ideal
$(\mathscr{O}_{d-t+k}+\ell_{\delta -t+k-1}\bar{R})$. Then we have 
$$
K-\dim (\mathscr{O}_{d-t+k}+\ell_{\delta-t+k-1}\bar{R})= K -\dim
U(\mathscr{O}_{d-t+k}+\ell_{\delta-t+k-1})= t-k
$$
and $(\mathscr{O}_{d-t+k}:  \ell_{\delta-t+k-1}) = \mathscr{O}_{d-t+k}$.

Let $U(\mathscr{O}_{d-t+k}+\ell_{\delta-t+k-1}\bar{R}) = \cap q_1 \cap
\cap q_2$ be the primary decomposition of
$U(\mathscr{O}_{d-t+k}+\ell_{\delta-t+k-1}\bar{R})$, where $q_1$ and
$q_2$ are the primary components of
$U(\mathscr{O}_{d-t+k}+\ell_{\delta-t+k-1}\bar{R})$ which appear\pageoriginale in
case (\ref{chap2:sec5:case1}) and case (\ref{chap2:sec5:case2}) of
\ref{chap2:sec5:subsec2.21}, respectively. We put
$\mathscr{O}_{d-t+k+1} = \cap q_2$. Then we have 
\begin{multline*}
h_0(\mathscr{O}_{d-t+k} =
h_0(\mathscr{O}_{d-t+k}+\ell_{\delta-t+k-1}\bar{R})=
h_0(U(\mathscr{O}_{d-t+k}+\ell_{\delta-t+k-1}\bar{R}))\\   
= \sum_{q_1} \text{ length of } q_1)~ h_0(\mathscr{Y}_1) +
h_0(\mathscr{O}_{d-t+k+1}) \text{ for } 2 \leq k \leq s. 
\end{multline*}

In any case, our algorithm stops if we have used all generic linear
forms $\ell_0,\ldots,\ell_n$. Therefore $\mathscr{O}_{d-t+s+1} =
\bar{R}$, where $s = n - \delta+ t+1$. Therefore the last step yields: 
\begin{align*}
h_0(\mathscr{O}_{d-t+s}) &= h_0(\mathscr{O}_{d-t+s} + \ell_n \bar{R} =
h_0(U(\mathscr{O}_{d-t+s}+\ell_n\bar{R}))\\ 
&= \sum_{q_1} (\text{ length of } q_1).h_0(\mathscr{Y}_1)
\end{align*}
where $q_1$ runs through all $\mathscr{Y}_1$-primary components of
$U(\mathscr{O}_{d-t+s}+\ell_n\bar{R}$ (Note that all primary
components of $U(\mathscr{O}_{d-t+s}+\ell_n\bar{R})$ appear in case
(\ref{chap2:sec4:case1}) of \ref{chap2:sec5:subsec2.21}). 

\setcounter{remark}{23}
\begin{remark}\label{chap2:sec5:rem2.24}
  Note that $K-\dim \mathscr{O}_{d-t+k} = t-k$ for every $1\leq k \leq
  s$. Therefore, in this step, we have collected certain imbedded
  irreducible subvarieties $C$ of $V_1\cap V_2$ corresponding to the
  primary components of
  $U(\mathscr{O}_{d-t+k}+\ell_{\delta-t+k-1}\bar{R}),1\leq k \leq s$,
  which appear in case (\ref{chap2:sec4:case1}) of
  \ref{chap2:sec5:subsec2.21}. For every imbedded irreducible 
  subvariety $C$ of $V_1 \cap V_2$ in this collection, we have defined
  the intersection number $j(V_1,V_2;C)$ of $V_1$ and $V_2$ along
  $C$. Moreover, $j(V_1,V_2;C)\geq 1,j(V_1,V_2;C)$ is the length of
  the corresponding $\mathscr{Y}_1$-primary component $q_1$ of
  $U(\mathscr{O}_{d-t+k}+\ell_{\delta-t+k-1}\bar{R})$, $1\leq k \leq s
  = n-\delta +t-1$ which appear in case (\ref{chap2:sec4:case1}) of
  \ref{chap2:sec5:subsec2.21}.  
\end{remark}

\noindent \textbf{Proof of the Main Theorem(2.1).}

\begin{enumerate}[(i)] 
\item Let $\{ C_i\}$ be the collection of irreducible
  subvarieties\pageoriginale of 
  $V_1 \cap V_2$ consisting of irreducible subvarieties of $V_1 \cap
  V_2$ which are collected in
  \ref{chap2:sec4:subsec2.18}(i), \ref{chap2:sec4:subsec2.18} (ii) and
  \ref{chap2:sec5:subsec2.23}. Then  
  the intersection numbers $j(V_1,V_2;C_i) \geq 1$ of $V_1$ and $V_2$
  along $C_i$ are defined and $j(V_1,V_2;C_i)$ are the lengths of
  certain well-defined primary ideals. It follows from
  \ref{chap2:sec4:subsec2.18} (iii) and \ref{chap2:sec5:subsec2.23} that 
  $$
  \deg (V_1).\deg (V_2) = \sum_{C_i} j(V_1,V_2;C_i)\deg (C_i)
  $$
\item It is clear from \ref{chap2:sec4:subsec2.18} (i) that every
  irreducible component of $V_1\cap V_2$ belongs to our collection $\{
  C_i \}$.  
\item Let $C_i \in  \{C_i\}$. Then it follows
  \ref{chap2:sec4:subsec2.18} (ii) and \ref{chap2:sec5:subsec2.23} that  
  $$ 
  \displaylines{\hfill
  K-\dim (C_i) \geq t-s = t -(n-\delta+t+1) = \delta - n -1\hfill \cr
 \text{that is},\hfill
  \dim (C_i) \geq \dim (V_1) + \dim (V_2) - n.\hfill }
  $$
\end{enumerate}
  This completes the proof of the main theorem \ref{chap2:sec1:subsec2.1}.

  We have the following generalization of the main theorem
  \ref{chap2:sec1:subsec2.1}. 

\setcounter{subsection}{24}
\subsection{The General Main Theorem}\label{chap2:sec5:subsec2.25}

Let $V_1 = V(I_1),\ldots,V_r = V(I_r),r\geq 2$  be pure
  dimensional projective varieties in  $\mathbb{P}^n_K$
  defined by homogeneous ideals $I_1,\ldots,I_r \subset
K[X_0,\ldots,X_n]$. There exists a collection $\{ C_i\}$
of irreducible subvarieties of  $V_1 \cap \cdots \cap V_r$
(one of which may be $\phi$) such that 
\begin{enumerate}[(i)] 
\item For\pageoriginale every $C_i \in  \{ C_i\}$ there are
  intersection number, say $j(V_1,\ldots,V_r$; $C_i)\geq1$, of
  $V_1,\ldots,V_r$ along $C_i$ given by the lengths
  of certain well-defined primary ideals such that 
  $$
  \prod_{i=1}^r \deg(V_i) = \sum_{C_i \in  \{ C_i\}}
  j(V_1,\ldots,V_r;C_i).\deg (C_i), 
  $$
  where we put  $\deg ~(\phi) = 1$.
\item If $C\subset V_1 \cap \cdots \cap V_r$ is an irreducible
  component of  $V_1 \cap \cdots \cap V_r$ then $C_i
  \in  \{ C_i\}$. 
\item For every $C_i \in  \{ C_i\}$,
  $$
  \dim (C_i) \geq \sum_{i=1}^r \dim (V_i) - (r-1).n.
  $$
\end{enumerate}

\begin{proof}%proof
  Proof of this theorem is very similar to the proof of that in case
  $r=2$ (see \ref{chap2:sec1:subsec2.1}). Therefore we omit the proof. (For details, see
  Patil-Vogel [56, Main theorem \ref{chap1:sec1:subsec1.2}]). 
\end{proof}

\section{Consequences}\label{chap2:sec6}%section F

In the following, we list some immediate consequences of the main
theorem some of which are already known. 

A typical classical result in this direction says that if
$V_1,\ldots,V_r, r\geq 2$ are pure dimensional subvarieties in
$\mathbb{P}^n_{K}$, and $\sum\limits_{i=1}^{r}\dim (V_i)= (r-1).n$,
and $\bigcap\limits_{i=1}^r V_i$ is finite set of isolated points,
then $\bigcap\limits_{i=1}^r V_i$ contains atmost
$\prod\limits_{i=1}^r \deg (V_i)$ points. The following corollary
\ref{chap2:sec6:subsec2.26} strengthens this to allow arbitrary intersections. 

\setcounter{subsection}{25}
\subsection{Corollary (Refined Bezout's Theorem)}\label{chap2:sec6:subsec2.26}
%subsec (2.26) 

Let\pageoriginale $V_1,\ldots$, $V_r \subset \mathbb{P}^n_K, r\geq 2$ be pure
  dimensional projective varieties in $\mathbb{P}^n_K$. Let
$Z_1,\ldots,Z_m$ be the irreducible components of
$\bigcap\limits_{i=1}^r V_r$. Then 
$$
\prod_{i=1}^{r} \deg (V_i)\geq \sum_{i=1}^m \deg (Z_i) \geq m.
$$

This refined Bezout's theorem was developed by W.Fulton and\break 
R.MacPherson (see \cite{19}, \cite{18}) to give an affirmative answer the
following question asked by S.Kleiman in 1979. 

\setcounter{subsection}{26}
\subsection{Corollary (Kleiman's Question)}
\label{chap2:sec6:subsec2.27}%corollary (2.27) 

Let $V_1,\ldots,V_r \subset \mathbb{P}^n_K$, $r \geq 2$ be pure
  dimensional projective varieties in  $\mathbb{P}^n_K$. Then the
  number of irreducible components of  $\bigcap\limits_{i=1}^r V_i$
is bounded by the Bezout's number $\prod\limits_{i=1}^r \deg
(V_i)$. 

The first proof is given in [17, \S 7.6] (see also \cite{18}). A second
proof (see \cite{17}) was suggested by a construction of Deligne used to
reduce another intersection question in projective space to an
intersection with a linear factor (see also the method used in [98
  Lemma on p.127] and (\ref{chap2:sec2:lem2.3}) (v)). A new
interpretation of the refined Bezout's theorem was given by
R.Lazarsfeld \cite{45}.  

The following Corollary (\ref{chap2:sec6:coro2.28}) strengthens the
refined Bezout's theorem \ref{chap2:sec6:subsec2.26}. 


\setcounter{corollary}{27}
\begin{corollary}\label{chap2:sec6:coro2.28}% corollary (2.28)
  Let $V_1,\ldots, V_r \subset \mathbb{P}^n_K, r \geq 2$ be pure
  dimensional projective\pageoriginale varieties in 
  $\mathbb{P}^n_K$. Then  
  $$
  \prod_{i=1}^r \deg (V_i) \geq \sum_C j(V_1,\ldots,V_r;C) \deg (C)
  $$
  where $C$ runs through all irreducible components of $V_1\cap\cdots
  \cap V_r$. 
\end{corollary}

The following corollary (\ref{chap2:sec6:coro2.29}) gives a generalization of
$C.G$.Jacobi's observation (see the Historical Introduction,
\cite{36}, \cite{16} and \cite{60}). 

\setcounter{corollary}{28}
\begin{corollary}\label{chap2:sec6:coro2.29} % corollary (2.29)
  Let $F_1,\ldots,F_n$ be any hypersurfaces in $\mathbb{P}^n_K$
  of degrees $d_1,\ldots,d_n$, respectively. Assume
  that $\bigcap\limits_{i=1}^n F_i$ contains a finite set of
  isolated points,say $P_1,\ldots,P_s$. Then we get 
  $$
  \prod_{i=1}^n d_i - \sum_C \deg (C) \geq \prod_{i=1}^n d_i -\sum_C
  j(F_1,\ldots,F_n;C).\deg (C)\geq s. 
  $$
  where $C$ runs through all irreducible components of $V \coloneqq
  \bigcap\limits_{i=1}^n F_i$ with $\dim (C) \geq 1$. 
\end{corollary}

\setcounter{subsection}{29}
\subsection{}\label{chap2:sec6:subsec2.30}
Analyzing these results and their proofs, one might be tempted to ask
the following question: 

Let $V_1 = V(I_1)$ and $V_2 = V(I_2)$ be pure dimensional projective
varieties in $\mathbb{P}^n_K$ defined by homogeneous ideals $I_1$, and
$I_2 \subset K[X_n,\ldots,X_n]$. We consider a primary decomposition
of  
$$
I_1+I_2 = q_1 \cap \cdots \cap q_m \cap q_{m+1}\cap \cdots \cap q_\ell 
$$
where $q_i$ is $\mathscr{Y}_i$-primary and
$\mathscr{Y}_1,\ldots,\mathscr{Y}_m$ are the minimal prime ideals of
$I_1+I_2$. Then 

\begin{question}\label{chap2:sec6:qus1}% question 1
  If\pageoriginale $\dim (V_1\cap V_2) = \dim (V_1) + \dim (V_2) -n$
  is then $\deg  (V_1)$. $\deg (V_2) \geq \ell -1?$  
\end{question}

\begin{question}\label{chap2:sec6:qus2}% question 2
  If $\dim (V_1\cap V_2) > \dim (V_1) + \dim (V_2) - n$ is then $\deg
  (V_1)$. $\deg (V_2) \geq \ell ?$ 
\end{question}

\begin{question}\label{chap2:sec6:qus3}% question 3
  If $\deg (V_1) \deg{(V_2) > m}$, is then $\deg (V_1) \deg (V_2) \geq \ell?$
\end{question}
 
However, these questions have negative answers, as we will show by
examples (\ref{chap3:sec1:exp3.14}) in the next chapter. 
