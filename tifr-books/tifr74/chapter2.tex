\chapter{Preliminary Results}\label{chap1}

\section{Preliminary Definitions and Remarks}\label{chap1:sec1}

\subsection{}\label{chap1:sec1:subsection1.1}
Let\pageoriginale $R$ be a noetherian ring and $I$ be an ideal in $R$. The
Krull-dimension, $K-\dim(I)$ of $I$ is the Krull-dimension of the ring
$R/I$. Suppose that $I=q_1\cap\cdots \cap q_r$ is a primary
decomposition of $I$, where $q_i$ is $\mathscr{Y}_i$-primary,
$\mathscr{Y}_i \in  
\spec (R)$ for $1\leq i \leq r$. We say that $q_i$ is a
$\mathscr{Y}_i$\textit{-primary component } of $I$ any $\mathscr{Y}_i$
is \textit{ an associated prime of } $R/I$. We write $Ass
(R/I)={\mathscr{Y}_1,\ldots, \mathscr{Y}_r}$. Suppose that $K-\dim (I)
= K -\dim (q_i)$ for $1\leq i \leq s \leq r$. We set $U(I):=
\bigcap\limits^s_{i=1} q_i$. This ideal is well defined and is called
the \textit{ unmixed part of } $I$. It is clear that $I\subset U(I)$
and $k \dim U(I)$. An ideal $I \subset R$ is called \textit{unmixed} if
and only if $I= U(I)$. A ring $R$ is called \textit{unmixed} if 
the zero ideal $(0)$ in $R$ is unmixed.  

Let $\mathscr{Y}\in  \spec (R)$ and $q$ be a
$\mathscr{Y}$-primary ideal. The length of the Artinian local ring
$(R/q)_{\mathscr{Y}}$ is called \textit{ the length of } $q$ and we
will denote it by $\ell_R(q)$. It is easy to see that the length of
$q$ is the number of terms in a composition series, $q=q_1\subset
q_2\subset \cdots \subset q_\ell = \mathscr{Y}$ for $q$, where
$q_1,\cdots, q_r$ are $\mathscr{Y}$-primary ideals.  

\begin{remark*}
  (see\pageoriginale [\cite{106}, Corollary 2 on $p$. 237, vol. 1]) Let
  $\mathfrak{M} \subset R$ be a maximal ideal of $R$ and $q \subset R$
  be a $\mathfrak{M}$ primary ideal. If $q=q_1\subset q_2\subset \cdots
  \subset q_\ell = \mathfrak{M}$  is a composition series for $q$,
  where $q_1,\ldots, q_\ell$ are  $\mathfrak{M}$-primary ideals. Then
  there exist $a_i \in q_i, 2\leq i \leq \ell$, such that  
  \begin{enumerate}[(i)]
  \item $a_i \not\notin~q_{i-1}$
  \item $q_i=(q_{i-1}, a_i)$
  \item $\mathfrak{M}q_i\subset q_{i-1}$  for all $2 \leq i \leq \ell $.
  \end{enumerate}
\end{remark*}

\begin{proof}
(i) and (ii) are easy to prove.

(iii) Replacing $R$ by $R/q$ we may assume that $q=0$ and $R$ is
Artinian local. Suppose $\mathfrak{M} q_i \not \subset q_{i-1}$ for
some $2 \leq i \leq \ell$. Then we get
$q_{i-1}\subsetneq(q_{i-1}+\mathfrak{M}q_i)=q_i=(q_{i-1},
a_i)$. Therefore we can write $a_i =q+ma_i$ for some $q \in
q_{i-1}$ and $m \in \mathfrak{M}$. Then
$a_i=\dfrac{1}{(1-m)}\cdot q\in q_{i-1}$ which is a
contradiction to (i).
\end{proof}

Let $R$ be a semi-local noetherian ring and rad$(R)$ be the Jacobson
radical of $R$. An ideal $q\subset R$ is called an \textit{ ideal of
  definition} if $(rad (R))^n \subset q \subset rad (R)$ for some $n
\in \mathbb{N}$.  

\subsection{The Hilbert-samuel
  Function}\label{chap1:sec1:subsec1.2} %%%1.2

Let $R$ be a semilocal noetherian ring and $q \subset R$ be an ideal
of definition. Let $M$ be any finitely generated $R$-module. The
numerical function $H^1_M(q,-): \mathbb{Z^+\to Z^+ }$ given by
$H^1_M(q,n) = \ell(M/q^{n+1} M)<\infty$ is called \textit{ the
  Hilbert-Samuel function of } $q$ on $M$. If $M=R$ we say that
$H^1(q,-):=H^1_R(q,-)$ is the\pageoriginale Hilbert-Samuel function of $q$. If
$(R,\mathscr{M})$ is a local ring then $H^1_R(-):=H^1(\mathcal{M},-)$
is called \textit{the Hilbert-Samuel function of }$R$.  

The following theorem is well known(for proof, see \cite{72} or \cite{106}). 

\subsection*{Hilbert-samuel Theorem.} 
Let $R$ be a semilocal
  noetherian ring and $q \subset R$ be an ideal of definition. Let $M$
  be any finitely generated $R$-module. Then $H^1_M(q,-)$ is, for $n>>
  1$, a  polynomial  $P_M(q,-)$ in $n$, with
coefficients in $Q$. The degree of  $P_M(q,-) $  is
$\delta$ where $\delta$ = Krull dimension of $M (:=k-\dim
R/ann_R)(m))$.  

We will write this polynomial in the following form: 
$$
P_M(q,n)=e_0 \binom{n+d}{d}+e_1\binom{n+d-1}{d-1}+\cdots+e_d 
$$
where $e_0(\geq 0). e_1,\ldots,e_d$ are integers and $d=K-\dim
(r)$. The  multiplicity of $q$  on $M, e_0 (q;M)$,
is defined by $ e_0 (q;M):=e_0$. Note that $ e_0 (q;M):=0$ if and only
if $K-\dim(M) < K-\dim (R)$. The positive integer $ e_0 (q; R)$ is
called the multiplicity of $q$. If $R$ is local and $q=\mathscr{M}$ is
the maximal ideal of $R$ then $ e_0(R):= e_0 (\mathscr{M}; R)$ is
called the multiplicity of $R$.  

\begin{remark*}
~

  (i)~ Let $(A, \mathscr{M})\to (B, \mathscr{N})$ be a flat local
  homomorphism of local rings. Assume that
  $\mathscr{M}B=\mathscr{N}$. Then for every $\mathscr{M}$-primary
  ideal $q$ of $A$ we have  
  $$
  e_0(q;A)=e_0(qB;B)
  $$
\end{remark*}


\begin{proof}
It\pageoriginale is easy to see (see, e.g. [\cite{34}, (1.28)] that, $H^1_B(qB,t)=
  \ell_B\break (B/q^tB)=\ell _B(A/q^t\otimes_A B)=\ell_A(A/q^t) \ell_B
  (B/\mathscr{M}_B)=\ell_A(A/q^t)=H^1_A(q,t)$ for all $t \geq
  0$. Therefore $e_0(qB;B)=e_0(q;A)$.  
\end{proof}

Let $A=\underset{n\geq 0}{\oplus} A_n $ be a graded ring such that
$A_0$ is artinian and $A$ is generated as an $A_0$-algebra by $r$
elements $\bar{x},\ldots,\bar{x}_r$. of $A_1$. Let $N=\underset{n\geq
  0}{\oplus} N_n $ be a finitely generated graded $A$-module. The
numerical function $H^1_A(N,-):\mathbb{Z}^+\to \mathbb{Z}^+$ defined
by $H^1_A(N,n)=\ell_{A_0}(N/N_{n+1})$ is called \textit{the Hilbert
  function of} $N$. The following is a well-known theorem (for proof,
see \cite{55} or \cite{72}).  

\begin{theorem*}[HILBERT] 
  The function  $H^1_A (N,-) $ is, for $n
  >> 1$, a polynomial $ P_A (N,-)$ in $n$
  with coefficients in  $Q$. The degree of 
  $P_A(N,-)$ is $\leq r$. 
  
  We will write this polynomial in the following form: 
  $$
  P_A(N,n)= h_0 \binom{n+r}{r}+ h_1\binom{n+r-1}{r-1}+\cdots+h_r, 
  $$ 
  where $h_0(\geq 0), h_1,\ldots,h_r$ are integers. 
\end{theorem*}

Note that 
$$
H^0_A(N,n)=\ell_{A_0}(N_n)= h_0 \binom{n+r-1}{r-1} + h'_1 
\binom{n+r-2}{r-2}+\cdots+h'_{r-l},
$$
where $h'_1, \ldots, h'_r$ are integers. 

\begin{remark*}
~
\begin{enumerate} [(ii)]
\item From\pageoriginale the exact sequence
  $$
  0 \to \left(0 \underset{N}{:}\bar{X}_1\right) \to N
  \xrightarrow{\bar{X}} N \to  N/\bar{X}_1 \to 0 
  $$
  of graded modules, it follows that
  \begin{align*}
  & H^1_{A/\bar{X}_1} (M/\bar{X}_1 N, n) - H^1_{A/\bar{X}_1} \left(\left( 0
  \underset{N}{:} \bar{X}_1\right), n-1\right)\\
  &\qquad  = H^0_A (N, n) ~\text {
    for all }~ n \geq 0.  
  \end{align*}
  
  Therefore we have $h_0 (N/\bar{X}_1 N) - h_0 \left((0 \underset{N}{:}
  \bar{X}^1)\right) = h_0 (N)$. 
\item Let $R$ be a semi-local ring and $q = (x_1, \ldots, x_d) \subset
  R$ be an ideal of definition generated by a system of parameters
  $x_1, \ldots,  x_d$ for $R$. Let $M$ be any finitely generated
  $R$-module.  
  
  Then $H^1_M (q, n) = H^1_{gr_q}(gr_q (M),n)$ for all $n$. Therefore
  $P_M (q, n)\break = P_{gr_q(R)} (gr_q (M), n)$ for all n and $e_0 (q; M) =
  h_0 (gr_q (m))$, where $gr_q (R) = \underset{n \geq 0}{\oplus}
  q^n/q^{n+1} $ and $gr_q (M) = \underset{n \geq 0}{\oplus} q^n
  M/q^{n+1}M$. 
\end{enumerate}
\end{remark*}

\section{The General Multiplicity Symbol}\label{chap1:sec2}

Let $R$ be a noetherian ring and $M$ be any finitely generated
$R$-module. Let $x_1, \ldots, x_d$ be a system of parameters for $R$. We
shall now define the general multiplicity symbol, $e_R(x_1, \ldots,
x_d | M)$, of $x_1, \ldots, x_d $ on $M$. 

\setcounter{definition}{2}
\begin{definition}\label{chap1:sec2:def1.3}
  Let $R$ be a noetherian ring and $M$ be any finitely generated
  $R$-module. Let $x_1, \ldots, x_d$ be a system of parameters for $R$. We
  shall define $e_R (x_1, \ldots, x_d | M)$, by induction on\pageoriginale $d$. If
  $d=0$, then define $e_R(\cdot|M)=\ell_R(M)< \infty$. Assume that $d
  \geq 1$ and the multiplicity symbol has been defined for $s \leq d-1$
  elements and all modules. Define $e_R(x_1,\ldots,x_d |M) = e_{R/x_1}
  (x_2,\ldots,x_d |M/x_1M)-e_{R/x_1} (x_2,\ldots,x_d | (0
  \underset{M}{:} x_1 ))$. It is clear that $e_R (x_1,\ldots,x_d | M)$
  is an integer (in fact, non-negative, see \ref{chap1:sec2:subsec1.9}).
\end{definition}

\setcounter{remarks}{3}
\begin{remarks}\label{chap1:sec2:rem1.4}
\begin{enumerate}[(i)]
\item By induction on $d$, it follows that 
  $e_R (x_1,\ldots,\break x_d |M)) = \ell (M/qM)- \ell ((q_{d-1} M
  \underset{M}{:} x_d)/q_{d-1}M)- \sum\limits^{d-1}_{k=1} e_{R/q_k}
  (x_{k+1}, \ldots,\break  x_d | (q_{k-1} M \underset{M}{:} x_k)/q_{k-1}M)$ 
  where $q_k = (x_1, \cdots, x_K) R, 0 \le k \le d-1, q = (x_1,\cdots,x_d)$.
\item Assume that $d \geq 2$ and $1 \leq m < d$. Then
  $$
  e_R(x_1,\ldots,x_d |M) = \sum_\nu \varepsilon_\nu e_{R/q_{m-1}}
  (x_m,\ldots,x_d |M_\nu) 
  $$
  where $\in_\nu = \pm 1 $ and $M_\nu$ are uniquely determined by $M$
  and $x_1,\ldots,\break x_{m-1}$. 
\end{enumerate}
\end{remarks}

\noindent \textbf{\large Some Properties of the General Multiplicity Symbol}

\setcounter{subsection}{4}
\subsection{The additive property}\label{chap1:sec2:subsec1.5} 

\textit{Let $0 \to M' \to M'' \to 0$ be an exact sequence of finitely
  generated $R$-modules and $x_1, \ldots, x_d$ be a system of parameters
  for $R.$ Then} 
$$
e_R(x_1, \ldots, x_d|M) = e_R (x_1, \ldots, x_d|M') + e_R (x_1,
\ldots, x_d|M''). 
$$

\setcounter{corollary}{5}
\begin{corollary}\label{chap1:sec2:coro1.6}
  Let\pageoriginale $o \to M_p \to M_{p-1} \to \cdots \to M_1 \to M_0 \to 0$ be an
  exact sequence of finitely generated $R$-modules and $x_1, \cdots, x_d$
  be a system of parameters for $R$. Then 
  $$
  \sum^{p}_{i = 0} (-1)^i e_R (x_1, \ldots, x_d|M_i) = 0.
  $$
\end{corollary}

\begin{proof}
  It is convenient to prove \ref{chap1:sec2:subsec1.5} and
  (\ref{chap1:sec2:coro1.6}) simultaneously. Proof by 
  induction on $d$. If $d=0$ then 
  $$
  \sum_{i=0}^{p} (-1)^i e_R (\cdot |M_i) = \sum^{p}_{i=0} (-1)^i \ell (M_i) = 0.
  $$
  
  Now suppose that $d = s + 1, s \geq 0$ and that
  \ref{chap1:sec2:subsec1.5} and (\ref{chap1:sec2:coro1.6}) 
  holds for $d=s$. We have an exact sequence 
  \begin{multline*}
    0 \to (0 \underset{M'}{:} x_1) \to (0 \underset{M}{:}x_1) \to (0
    \underset{M''}{:}x_1) \to M'/x_1M'\\
    \to M/x_1 M \to M'' /x_1 M'' \to 0 
  \end{multline*}
  Therefore by induction hypothesis, we have
{\fontsize{10pt}{12pt}\selectfont
  \begin{multline*}
    e_{R/x_1} (x_2, \ldots,  x_d |M/x_1 M) - e_{R/x_1} (x_2, \ldots,
    x_d |(0\underset{M'}{:} x_1))\\ 
    = e_{R/x_1} (x_2, \ldots, x_d | M'/x_1 M') -e_{R/x_1} (x_2,
    \ldots, x_d | (0\underset{M'}{:} x_1))\\ 
    \hspace{2cm}+ e_{R/x_1} (x_2, \ldots,  x_d |M''/x_1 M'') -e_{R/x_1}
    (x_2, \ldots,  x_d) | (0\underset{M''}{:} x_1)) 
  \end{multline*}}\relax
  Hence $e_R (x_1, \cdots,  x_d |M) = e_R (x_1, \cdots,  x_d |M') +
  e_R (x_1, \cdots,  x_d |M")$. 
\end{proof}

\setcounter{subsection}{6}
\subsection{The Exchange Property}\label{chap1:sec2:subsec1.7}

Let $M$ be any finitely generated $R$-module and $x_1, \ldots,
  x_d$ be a system of parameters for $R$. Then 
$$
e_R (x_1, \ldots,  x_d |M) = e_R (x_{i_1}, \ldots,  x_{i_d} |M)
$$
for\pageoriginale every permutation $(i_1, \ldots, i_d)$ of $(1, \ldots, d)$.

\begin{proof}
  By remark (\ref{chap1:sec2:rem1.4}) (ii), it is enough to prove that,
  $$
  e_R (x_1, \ldots,  x_d |M) = e_R (x_2, x_1, \ldots, x_d | M).
  $$

  Let $K$ be any finitely generated $R/(x_1, x_2)$ -module. Then we
  denote $e_{R/(x_1, x_2)} (x_3, \ldots, x_d | K) $ by $[K]$. 
\end{proof}

Now, we have 
\begin{align*}
  e_R (x_1, \ldots,  x_d |M)& = [M/(x_1, x_2)M]- [(0
    \underset{M/x_1 M}{:} x_2)]\\
-[(0 \underset{M}{:}x_1)/x_2 (0 \underset{M}{:} x_1)]+ [(0 \underset{M}{:}x_2)]
& = [a] - [b] - [c] +[d],
\end{align*}
where 
$$
\displaylines{\hfill 
[a] = [M/(x_1, x_2) M], [b] = [(0 \underset{M/x_1 M}{:} x_2)], [c] =
[(0 \underset{M}{:} x_1)/x_2 (0 \underset{M}{:} x_1)]\hfill \cr 
\text{and}\hfill [d] = [(0 \underset{M_1}{:}x_2)]$, with $M_i:= (0
\underset{M}{:} x_i) ~\text{for}~ i = 1, 2.\hfill }
$$ 

Now,  
$$
(0 \underset{M/x_1 M}{:} x_2) \xrightarrow{\sim} (x_1 M: x_2)/x_1 M,
(0 \underset{M_1}{:}x_2)= (0 \underset{M}{:} x_1) \cap (0
\underset{M}{:} x_2)
$$ 

Therefore, $[a]$ and $[d]$ are symmetric in $x_1$ and $x_2$. Thus it
is enough to prove that $[b]+[c]$ is also symmetric in $x_1$ and
$x_2$. Since $x_1 M \subset x_1 M + (0 \underset{M}{:} x_2) \cap (x_1
M \underset{M}{:} x_2)$ we get by \ref{chap1:sec2:subsec1.5}, 
\begin{align*}
  [b] &= [x_1 M + (0 \underset{M}{:}M x_2)/x_1 M] + [(x_1 M
    \underset{M}{:} x_2)/x_1 M+ (0 \underset{M}{:} x_2)]\\ 
  &= [(0 \underset{M}{:} x_2)/x_1 M \cap (0 \underset{M}{:} x_2)] +
  [x_1 M \cap x_2 M/x_1x_2M] = [e] + [f] 
\end{align*}
where\pageoriginale $[e]= [\underset{M}{(0:  x_2)}/x_1 M \cap
  \underset{M}{(0:x_2)}]$ and $[f]= [x_1 M \cap x_2 M/x_1 x_2
  M]$. Clearly $[f]$ is symmetric in $x_1$ and $x_2$. Now consider
$[c]+[e]$. Since $x_1 M \cap \underset{M}{(0:x_2)}= x_1
\underset{M}{(0:x_1x_2)}$ and $x_2 \underset{M}{(0:x_2)} \subset x_2
\underset{M}{(0:x_1x_2)} \subset x_1 \underset{M}{(0:x_2)}$ we get by
\ref{chap1:sec2:subsec1.5}, $[e]+[c]$ 
\begin{align*}
  &=[x_2 (0\underset{M}{:} x_1 x_2)/ x_2 (0\underset{:}x_1)] +
  [(0 \underset{M}{:}x_1)/x_2 (0\underset{M}{:}x_1x_2)]\\ 
  & \hspace{5cm}+
  [(0\underset{M}{:}x_2)/x_1 (0\underset{M}{:}x_1x_2)]\\ 
  &= [g]+[h]
\intertext{where}
  [g] &= [x_2[\underset{M}{(0:x_1x_2)}/x_2\underset{M}{(0:x_1)}] \text{ and }\\
    [h]&= [\underset{M}{(0:x_1)}/x_2\underset{M}{(0:x_1x_2)}]+
    [\underset{M}{(0:x_2)}/x_1\underset{M}{(0:x_1x_2)}]. 
\end{align*}

Clearly $[h]$ is symmetric in $x_1$ and $x_2$ and since
$\underset{M}{(0:x_1x_2)}/
\underset{M}{(0:x_1)}+\underset{M}{(0:x_2)} \xrightarrow{\underset{\sim}{x_2}}
x_2 \underset{M}{(0:x_1x_2)}/x_2 \underset{M}{(0:x_1)}, [g]=
[\underset{M}{(0:x_1x_2)}/x_2 \underset{M}{(0:x_1)}\break +
  \underset{M}{(0:x_2)}]$ 
is also symmetric in $x_1$ and $x_2$. Therefore $e_R(x_1, \ldots,
x_d|M)= [a] + [d] - [f] - [g] - [h] $ is symmetric in $x_1$ and
$x_2$. This completes the proof. 

\setcounter{subsection}{7}
\subsection{}\label{chap1:sec2:subsec1.8}

 Let $M$ be any finitely generated $R$-module and $x_1, \ldots,
   x_d$ be a system of parameters for $R$. Suppose $x^m_i M=0$ for
   some $1 \leq i \leq d$ and $m \in \mathds{N}$. Then $e_R(x_1,
   \ldots,  x_d |M)=0$. 

\begin{proof}
  By \ref{chap1:sec2:subsec1.7} we may assume the $i=1$. Proof by induction on $m$. If
  $m=1$ then $M=M/x_1 M$ and $\underset{M}{(0:x_1)}=M$ and hence
  $e_R(x_1, \ldots,  x_d |M)= e_R/_{x_1} (x_2,  \ldots,  x_d |M)- e_R/
  _{x_1}(x_2, \ldots, x_d |M)-e_R/_{x_1} (x_2, \ldots,  x_d |M)=0$.  
\end{proof}

Now suppose that $d=s+1, s \geq 0$ and the result holds for $d=s$. We
have by \ref{chap1:sec2:subsec1.5}, 
$$
e_R(x_1, \ldots,  x_d|M)= e_R(x_1, \ldots,  x_d |x_1M) +e_R(x_1,
\ldots, x_d|M/ x_1 M). 
$$
Since\pageoriginale $x_1^{m-1}(x_1M)= x_1(M/x_1 M)=0$, by induction
the result follows. 

\setcounter{subsection}{8}
\subsection{}\label{chap1:sec2:subsec1.9}

 Let $M$ be any finitely generated $R$-module and $x_1, \ldots,
   x_d$ be a system of parameters for $R$. Then 
 $$
 0 \leq e_R(x_1, \ldots,  x_d|M) \leq \ell (M/ (x_1, \ldots,  x_d )M) < \infty.
 $$
\begin{proof}
  First, by induction on $d$, we show $e_R(x_1, \ldots,  x_d|M) \leq
  0$. If $d=0$ then $e_R(\cdot|M)= \ell_R(M) \geq 0$. Now suppose that
  $d=s+1, s > 0$ and the result holds for $d=s$. 
\end{proof}

Put $N=M/\underset{M}{(0: x_1^m)}$. If $\mathscr{M} >> 1$, then it is easy to
see that $\underset{N}{(0: x_1^m)}=0$. From \ref{chap1:sec2:subsec1.5}
and \ref{chap1:sec2:subsec1.8}, we get
$e_R (x_1, \ldots,  x_d |M)=e_R(X_1, \ldots,  x_d |N)= e_R (x_2,
\ldots,  x_d|N/ x_1 N)$ and hence, by induction, it follows that
$e_R(x_1, \ldots$,  $x_d /M) \geq 0$. The second inequality follows from
(\ref{chap1:sec2:rem1.4}) (i). 

\setcounter{corollary}{9}
\begin{corollary}\label{chap1:sec2:coro1.10}
  If $(x_1, \ldots, x_d) M=M$, then $e_R (x_1, \ldots,  x_d |M)=0$.
\end{corollary}

\setcounter{proposition}{10}
\begin{proposition}\label{chap1:sec2:prop1.11}
  Let $M$ be any finitely generated $R$-module. Let $x_1, \ldots,
  x_{d-1}, x$ and $x_1, \ldots,  x_{d-1},y$ be two systems of
  parameters for $R$. Then we have  $ e_R(x_1, \ldots, x_{d-1},
  xy|M)= e_R(x_1, \ldots,  x|M)+e_R(x_1, \dots$, $x_{d-1}, y|M)$.  
\end{proposition}

 \begin{proof}
   By induction on $d$. Suppose $d=1$. Then we have exact sequences
   \begin{align*}
   0 \rightarrow \underset{M}{(xyM:y)}/xM \rightarrow M/xM
   \xrightarrow{y} M/xyM \rightarrow M/yM \rightarrow 0 \\
   0 \rightarrow \underset{M}{(0:x)} \rightarrow \underset{M}{(0:xy)}
   \xrightarrow{x} \underset{M}{(0:x)} \rightarrow
   \underset{M}{(0:x)}/ \underset{M}{(0:xy)} \rightarrow 0 
   \end{align*}\pageoriginale

   Therefore we get 
   $$
   \displaylines{\hfill
   \ell (M/yM)+ \ell (M/xM) = \ell (M/xyM) + \ell
   \underset{M}{((xyM:y)/xM)},\hfill \cr 
   \text{ and} \hfill 
   \ell \underset{M}{((0:y))} + \ell \underset{M}{((0:x))} = \ell
   \underset{M}{((0:xy))} + \ell \underset{M}{((0:y))} 
   \underset{M}{((0:xy))}.\hfill } 
   $$
 \end{proof}
 Now, it is easy to see that $\underset{M}{(0:y)}/
 \underset{M}{(0:xy)} \xrightarrow{\sim} (xyM \underset{M}{:}y)/ xM$ is an
 isomorphism. Therefore we get that  
 \begin{align*}
   e_R(x|M)+e_R(y|M) &= \ell (M/xM)- \ell \underset{M}((0:x)) + \ell
   (M/yM) -\ell \underset{M}(( 0: y))\\ 
   &= \ell (M/ xyM) -\ell \underset{M}((0: xy))=e_R(xy|M).
 \end{align*} 
 
 Now suppose that $d=s+1, s \geq 1$ and the result holds for
 $d=s$. Let $q=(x_1, \ldots, x_{d-1})$. Then, by induction, we have 
 \begin{align*}
 e_R(x_1, \ldots,  x_{d-1}, xy|M) & = e_R/_{x_1}(x_2, \ldots,  x_{d-1},
 xy|M/ x_1 M)\\ 
 & \qquad -e_{R/x_1}(x_2, \ldots,  x_d, xy| \underset{M}{(0:x_1)}\\
   e_R/_{x_1}(x_2, \ldots, x_{d-1}, xy|M)& = e_{R/x_1}(x_2, \ldots,
   x_{d-1}, xy|M/ x_1M)\\ 
   & \qquad -e_R/_{x_1}(x_2, \ldots,  x_{d-l}, x|
   \underset{M}{(0:x_1)}\\
   & \qquad -e_{R/ x_1} (x_2, \ldots,  x_{d-l}, y| \underset{M}{(0:x_1)}\\ 
   =& e_R(x_1, \ldots,  x_{d-l}, x|M) + e_R(x_1, \ldots, x_{d-1}, y|M).
 \end{align*}

\setcounter{corollary}{11}
\begin{corollary}\label{chap1:sec2:coro1.12}
  For any positive integers $n_1, \ldots,  n_d$, we have
  \begin{enumerate}[(i)]
  \item $e_R(x_1^{n_1},  \ldots,  x_d^{n_d} M) = n_1 \ldots n_d e_R(x_1,
    \ldots,  x_d|M)$ \pageoriginale 
  \item $0 \leq e_R(x_1, \ldots, x_d|M) \leq \dfrac{\ell (M/(x_1^{n_1},
    \ldots,  x_d^{n_d})M)}{n_1 \ldots n_d}$ 
  \end{enumerate}
\end{corollary}

 \begin{proof}
   (i) follows from (\ref{chap1:sec2:subsec1.7})
   and (\ref{chap1:sec2:prop1.11}). (ii) follows from (i)
   and \ref{chap1:sec2:subsec1.9}.
 \end{proof} 
 
\setcounter{corollary}{12}
\begin{corollary}\label{chap1:sec2:coro1.13}
  If $x_i^{m_{M}} \subset (x_1, \ldots, x_{i-1}, x_{i+1},  \ldots,
  x_d)M$ for some $i \leq i \leq d$ and $m \in \mathds{N}$,
  then $ e_R(x_1, \ldots,  x_d |M)=0 $. 
\end{corollary}

\begin{proof}
  By \ref{chap1:sec2:subsec1.7}, we may assume that $i=1$. If $n>m$,
  then $(x_1^n, x_2, 
  \ldots,\break  x_d)$ $M=(x_2, \ldots,  x_d)M$ and so, by
  (\ref{chap1:sec2:coro1.12}), we get,   
  $ 0 \leq e_R(x_1, \ldots, x_d|M) \leq$ $\frac{\ell(M/(x_2, \ldots,
    x_d)M)}{n} \rightarrow 0 \text{ as } n \rightarrow \infty $. Hence
  $e_R(x_1, \ldots,  x_d|M)=0$.
 \end{proof} 
 
\setcounter{proposition}{13}
\begin{proposition}\label{chap1:sec2:prop1.14}
  Let $M$ be any finitely generated $R$-module and $x_1, \ldots,
  x_d$ be a system of parameters for $R$ contained in rad $(R)$. Then
  $e_r(x_1, \ldots, x_d|M)= \ell (M/(x_1, \ldots,  x_d)M)$ if and only
  if $x_1, \ldots,  x_d$ is an $M$-sequence, that is, $((x_1, \ldots,
  x_{i-1}) \underset{M}{M:x_i}) = (x_1, \ldots,  x_{i-1})M$ for $1
  \leq i \leq d$. 
\end{proposition}

 \begin{proof}
   $(< =)$ This implication follows from \ref{chap1:sec2:prop1.14}
   (i)~ $(=>)$ Proof by 
   induction on $d$. Suppose $d=1$. Then we have $\ell(M/x_1M) =
   e_R(x_1|M)= \ell_R(M/x_1 M)-\ell_R \underset{M}{((0:x_1))}$.
   Therefore, we get $\ell_R \underset{M}{((0:x_1))}=0$, that is,
   $\underset{M}{(0 : x_1)}=0$  
 \end{proof} 

Now suppose that $d=s+1$ and the result holds for $d=s$.
 
Let\pageoriginale $n_1, \ldots,  n_d$ be arbitrary positive integers.  Then by
\ref{chap1:sec2:subsec1.9} and (\ref{chap1:sec2:coro1.12}) (i) we have 
{\fontsize{10pt}{12pt}\selectfont
\begin{multline*}
  e_R(x^{n_1}_1, \ldots,  x^{n_d}_d)
  \leq l (M/(x^{n_1}_1, \ldots,  x^{n_d}_d)M)\leq n_1 \ldots \ldots n_d
  l\\ 
  (M/(x_1, \ldots, x_d)M) 
  = n_1 \ldots\ldots n_d e_R(x_1, \ldots,
  x_d|M)=e_R(x^{n_1}_1, \ldots, x^{n_d}_d)|M).  
\end{multline*}}\relax

Put $N=M/( \underset {M}{0: x_1})$. Then by (1. 8) we have 
{\fontsize{10pt}{12pt}\selectfont
\begin{multline*}
  \ell(M/(x^{n_1}_1, \ldots,  x^{n_d}_d)M = e_R(x^{n_1}_1, \ldots,
  x^{n_d}_d)|M) =e_R(x^{n_1}_1, \ldots,  x^{n_d}_d)|N)\\
  \leq l
  (N/(x^{n_1}_1, \ldots,  x^{n_d}_d)N)= l(M/( \underset {M}{0: x_1}
  )+(x^{n_1}_1, \ldots,  x^{n_d}_d)M)
\end{multline*}}\relax
and hence $( \underset {M}{0: x_1} )\subset (x^{n_1}_1, \ldots,
x^{n_d}_d)M$ for arbitrary positive integers\break $n_1, \ldots,  n_d$. Then
we get that $(0 \underset {M}: x_1 )\subset \bigcap \limits_{n \geq
  0}(x^{n_1}_1, \ldots,  x^{n_d})M \subset \bigcap \limits_{n \geq 0}
q^n M=0$ by Krull's Intersection Theorem, where $q=(x_1,  \ldots,
x_d)$.  

Now, 
\begin{multline*}
  e_{R/x_1} (x_2, \ldots,  x_d|M/x_1 M)=e_R(x_1, \ldots, x_d|M)=l\\
  (M/(x_1, \ldots,  x_d)M)=l(M/x_1 M/(x_2, \ldots, x_d)M/x_1M).  
\end{multline*}


Therefore, by induction, we get that $\{x_2, \ldots,  x_d\}$ is $M/x_1
M$-seque\-nce.  

This completes the proof. 

\setcounter{corollary}{14}
\begin{corollary}\label{chap1:sec2:coro1.15}
  \begin{enumerate}[(i)]
  \item \textit{Let $(R, \mathscr{M})$ be a noetherian local ring. Then
    $R$ is a Cohen-Macaulay ring if and only if there exists a system of
    parameters\pageoriginale $\{x_1, \ldots,  x_d\}$ for $R$ such that
    $e_R (x_1, \ldots,  x_d |R)=$ \break $l(R/(x_1, \ldots,  x_d))$.} 
  \item \textit {Let $(R, \mathscr{M})$ be a noetherian local ring. Then
    $R$ is a Cohen-Macau\-lay ring if and only if for every system of
    parameters $x_1, \ldots,  x_d$ for $R$, we have $e_R (x_1, \ldots,
    x_d |R)=l(R/(x_1, \ldots,  x_d))$. } 
  \end{enumerate}
\end{corollary}

\begin{proof}
  (i) Clear. (ii) Follows from [71, Theorem 2, VI-20] and
  (\ref{chap1:sec2:prop1.14}).  
\end{proof}

\setcounter{subsection}{15}
\subsection{The limit formula of Lech:}\label{chap1:sec2:subsec1.16} 
Let $M$ be any finitely generated $R$-module and $x_1, \ldots,
  x_d$ be a system of parameters for $R$. Let $n_1, \ldots,  n_d$ be
  positive integers. Then 
$$
\min \overset{\lim}{(n_i)}\to\infty \frac{l (M/(x^{n_1}_1, \ldots,
  x^{n_d}_d)M)}{n_1\cdots\cdot\cdot n_d} =e_R (x_1 \ldots x_d|M). 
$$ 

\begin{proof}
  Proof by induction on $d$. Suppose $d=1$. Then, for any $n>0$, we have
  $e_R(x^n|M)=l(M/x^n M)-l((0 \underset {M}: x^n))$. Choose an integer
  $m>0$ such that $(0 \underset {M}: x^n)=(0 \underset {M}: x^m)$ for
  all $n \ge m$. Therefore, by (\ref{chap1:sec2:coro1.12})(i), we have
  $e_R(x^n|M)=ne_R(x|M)=\ell(M|x^nM)-l((0 \underset {M}: x^m))$ for $n
  \geq m$. Thus $e_R(x|M) = \frac{\ell(M/x^n M)}{n} + C/n$, where $C$
  is independent of $n$. In particular, we get  
  $$
  \lim_{n \to \infty} \frac{l(M/x^n M)}{n}=e_R(x|M). 
  $$
  
  Now suppose that $d=s+1$, $s \geq 1$ and the result holds for $d=s$. 

  Using\pageoriginale \ref{chap1:sec2:subsec1.5} and
  \ref{chap1:sec2:subsec1.8} and replacing $M$ by $N:=M/(0 \underset
      {M}: 
x^m_1), m>>1$, we may assume that $(0 \underset {M}: x_1)=0$. Note that
$e_R(x_1, \ldots,  x_d|M)=e_R(x_1, \ldots,  x_d|N)$ and 
\begin{multline*}
  0 \leq
  \ell_R(M/(x^{n_1}_1, \ldots,  x^{n_d}_d)M)-\ell_R(N/(x^{n_1}_1, \ldots
  , x^{n_d}_d)N)\\ 
  =l((0:x^m_1)+(x^{n_1}_\ell, \ldots, 
  x^{n_d}_d)M/(x^{n_1}_1, \ldots,  x^{n_d}_d)M=\ell((0 \underset {M}:
  x^m_1)/(0:  x^m_1)\\
  \cap(x^{n_1}_1, \ldots,  x^{n_d}_d)M)\leq l
  ((\underset {M}{0:x^m_1})/(x^{n_2}_2, \ldots,  x^{n_d}_d)(\underset
       {M}{0:x^m_1}))\\ 
       \leq n_2 \cdots n_d l((( \underset {M}{0:x^m_1})/(x_2,
       \ldots,  x_d). (0 \underset {M}:x^m_1))
       =n_2 \ldots n_d C,
\end{multline*}
 where $C$
is a positive integer which is independent of $n_1, \ldots,  n_d$.  

Thus we get
$$
\displaylines{\hfill
0 \leq \frac{l(M/(x^{n_1}_1, \ldots,  x^{n_d}_d)M)-l(N/(x^{n_1}_1,
  \ldots,  x^{n_d}_d)N)}{n_1 n_2 \cdots n_d} \leq C/n_1 \hfill \cr
\text{i.e.,}\hfill \lim\limits_{\min (n_i)\to \infty}  l(M/(x^{n_1}_1,
\ldots,  x^{n_d}_d)M)=\lim\limits_{\min (n_{i})\to \infty}
l(N/(x^{n_1}_1, \ldots,  x^{n_d}_d)N).\hfill}  
$$

This shows that we may assume $(0 \underset {M}:x_1)=0$. Now by
(\ref{chap1:sec2:coro1.12}),  
\begin{align*}
  0 \leq n_1 \cdots n_d e_R(x_1, \ldots,  x_d|M) & \le l(M/(x^{n_1}_1,
  \ldots,  x^{n_d}_d)M)\\ 
  & \leq n_1 l(M/(x_1, x^{n_2}_2, \ldots,  x^{n_d}_d)M)\\
  &=n_1 l(\bar M/x^{n_2}_2, \ldots,  x^{n_d}_d)\bar M)
\end{align*}
where $\bar M = M/x_1 M$. Therefore, by induction, it follows that 
\begin{alignat*}{3}
  &e_R(x_1, \ldots, x_d|M)\\ 
  & \qquad\leq \lim_{\min (n_i)\to \infty}
  \frac{\ell(M/(x_1^{n_1}, \ldots, x_d^{n_d} )M )}{n_1 n_2
    \cdots\cdot\cdot n_d}\\ 
  & \qquad \leq \lim_{\min (n_i)\to \infty}  \frac{\ell(\bar {M}/(x_2^{n_2},
    \ldots, x^{n_d}_d)M)}{n_2 \cdots\cdot\cdot n_d}& &=e_{R/x_1}(x_2,
  \ldots, x_d|\bar{M})\\ 
  &\qquad  &&= e_R(x_1, \ldots,  x_d| M), \text{ since }
\end{alignat*}\pageoriginale
$(\underset {M}{0:x_1)}=0$. Thus we get 
$$
\lim_{\min (n_i)\to \infty}  \frac{\ell(M/(x^{n_1}_1, \ldots,
  x^{n_d}_d)M)}{n_1\cdots \cdot\cdot n_d}=e_R(x_1, \ldots, x_d| M). 
$$ 

In the next proposition, we will prove that the general multiplicity
symbol is nothing but the multiplicity defined in
\ref{chap1:sec1:subsec1.2}. \textit{Now 
  onwards, we assume that $R$ is semilocal noetherian}.  
\end{proof}

\subsection{The Limit Formula of Samuel}\label{chap1:sec2:subsec1.17} 
Let $M$ be any finitely generated $R$-module and $x_1, \ldots,
  x_d$ be a system of parameters for $R$.  

Assume that $q=(x_1, \ldots, x_d)$ is an ideal of definition
  in $R$. Then $e_0(q;M)=\lim\limits_{n\to \infty}
\dfrac{l(M|q^nM)}{n^d/d!}=e_R(x_1, \ldots, x_d |M)$.  

For the proof of this formula, we need the following lemma. 

\setcounter{lemma}{17}
\begin{lemma}\label{chap1:sec2:lem1.18}
  Let $M$ be any finitely generated $R$-module and $q=(x_1,
  \ldots, x_d$ be an ideal of definition generated by a system of parameters
  $\{ x_1, \ldots, x_d\}$ for $R$. Then 
  $$
  e_0(q;M)=e_{gr_q}(R)(\bar{x}_1, \ldots, \bar{x}_d|gr_q(M)), 
  $$
  where\pageoriginale $ \bar{x}_1, \ldots, \bar{x}_d$ are images of
  $x_1, \ldots,  x_d$ in $q/q^2$. 
\end{lemma}

\begin{proof}
  We put $A:=gr_q(R), N:=gr_q(M)$. By induction on $d$ we shall prove
  that $h_0(N)=e_A(\bar{x}_1, \ldots, \bar{x}_d|N)$. Suppose $d=0$,
 then $q=0$, $A=R$,  $N=M$ and $h_0(N)=l_R(M)=\ell_A(N)=e_A(\cdot|N)$. Now suppose that
  $d=s+1$, $s \geq 0$ and the result holds for $d=s$. Then we have by
  induction 
  \begin{align*}
    e_A(\bar{x}_1, \ldots,
    \bar{x}_d|N)& =e_{A/\bar{x}_1}(\bar{x}_2, \ldots, \bar{x}_d|N/\bar{x}_1
    N)-e_{A/\bar{x}_1}(\bar{x}_2, \ldots, \bar{x}_d|(0 \underset{N}:
    \bar{x}_1))\\  
    & =h_0(N/\bar{x}_1N)-h_0((0\underset{N}:
    \bar{x}_1))=h_0(N), 
  \end{align*}
  see remark (ii) in \ref{chap1:sec1:subsec1.2}. Also, it follows from
  the same remark (iii) that
  $e_0(q;M)=h_0(gr_g(M)))=e_{gr_q(R)}(\bar{x}_1, \ldots,
  \bar{x}_d|gr_q(M))$.  
\end{proof}

\noindent \textbf{Proof of (1.17)}
  First, we prove that
  $$
  \lim_{n \to \infty}\frac{l(M|q^n M)}{n^d/d!}\leq e_R(x_1, \ldots, x_d|M). 
  $$

  If $d=0$ then $q=0$ and $\lim\limits_{n \to \infty}\dfrac{\ell
    (M|q^nM)}{n^d/d!}=\ell (M)= e_R(x_1, \ldots, x_d|M)$.  

Now suppose that $d \geq 1$ and put $\bar{M}=M/x_1M, \bar{R}=R/x_1,
\bar{q}=q/x_1$.  

The we have $\bar{M}/\bar{q}^n \bar{M}= M/(x_1M+q^n M)$
\begin{align*}
  \ell_{\bar{R}}(\bar{M}/\bar{q}^n M) &=\ell_{\bar{R}} (M/q^nM)-\ell
  (x_1M+q^nM/q^nM)\\ 
  & =\ell (M/q^nM)-\ell (x_1 M/x_1M \cap q^nM) 
\end{align*}

Now, it is easy to see that $x_1M/x_1M \cap q^nM=x_1M/x_1(q^n M
\underset{M}: x_1) \xleftarrow[\approx]{x_1} M/(q^n M \underset{M}:
x_1)$ is an isomorphism. Therefore we get  
{\fontsize{10pt}{12pt}\selectfont
\begin{align*}
  \ell	_{\bar{R}}(\bar{M}/\bar{q}^n \bar{M})& =
  l(M/q^nM)-l(M/(q^n\underset{M}{ M: x_1})) \geq l (M/q^{n}M)- l
  (M/q^{n-1}M)\\
  &= H^0_{gr_q(R)}(gr_q(M), n-1)\text{ for all n}. 
\end{align*}}\relax\pageoriginale

Thus
$$
e_0(\bar{q};\bar{M})=\lim_{n \to \infty} \frac{\ell (\bar{M}/\bar{q}^n
  M)}{n^{d-1}/(d-1)!}\geq e_0(q;M)  
$$

If $d \geq 2$, replacing $M$ by $\bar{M}$, $R$ by $\bar{R}$, $q$ by
$\bar{q}$, we get 
\begin{multline*}
e_0(q;M)\leq e_0(q/x_1 ; M/x_1 M)\leq e_0(q/(x_1,
x_2), M/(x_1, x_2)M)\leq \cdots\\ 
\leq e_0((0);M/qM)=\ell(M/qM).  
\end{multline*}
$$
\displaylines{\text{i.e.,}\hfill \lim\limits_{n \to \infty}
\dfrac{\ell(M/q^n M)}{n^d/d!} \leq \ell (M/(x_1, \ldots, x_d)M).\hfill}
$$  

Now, replacing $x_1, \ldots, x_d$ by $x^p_1, \ldots, x^p_d$, we get 
\begin{align*}
  \lim_{ n \to \infty} \frac{\ell(M/q^{np}M)}{(np)^d/d!} & \leq \lim_{ n
    \to \infty} \frac{\ell(M/(x^p_1, \ldots, x^p_d)^n M)}{(np)^d/d!} \\ 
  & \leq \frac{l(M/(x^p_1, \ldots, x^p_d)^n M)}{p^d} \text{ for all } p
  \ge 0.  
\end{align*}

Hence $e_0(q;M)\leq \lim\limits_{ p \to \infty}\dfrac{\ell (M/(x^p_1,
  \ldots, x^p_d)M)}{p^d} =e_R(x_1, \ldots, x_d|M)$ by
\ref{chap1:sec2:subsec1.16}. It
remains to prove the reverse inequality. Let $n_1, \ldots, n_d$ be
positive integers. Put $A=gr_q(R), N=gr_q(M)$ and $\bar{x}_1, \ldots,
\bar{x}_d$ be the image of $x_1, \ldots, x_d$ in $q/q^2$. Set $F:
=(x^{n_1}_1, \ldots, x^{n_d}_d) M, K:= \underset {n \geq 0}{\otimes}
q^nM\cap (q^{n+1}M+F)/q^{n+1}M$ and $L=(\bar{x}^{n_1}_1, \ldots,
\bar{x}^{n_d}_d)N$. Then it is clear that $K, L$ are graded
$A$-submodules\pageoriginale of $N, L \subset K$ and  
$$
N/K =\underset{n \geq 0}\oplus q^nM/q^nM \cap(q^{n+1}M+F)=\underset{n
  \geq 0}{\oplus} q^n M+F/q^{n+1}M+F.  
$$

For $n \geq n_1+\cdots +n_d$, we have $q^n M \subset F$. Therefore we
get 
\begin{multline*}
\ell_R(M(x^{n_1}_1, \ldots, x^{n_d}_d) M)=\sum \limits_{n \geq 0}
\ell_R(q^nM+F/q^{n+1}M+F)\\
=\ell_R(N/K)\leq \ell_R(N/L)=	\ell_R
(N/(\bar{x}^{n_1}_1, \ldots, \bar{x}^{n_d}_d)N).
\end{multline*}

If $\ell_R (N/(\bar{x}^{n_1}_1, \ldots,
\bar{x}^{n_d}_d)N)=\ell_A(N/\bar{x}^{n_1}_1, \ldots,
\bar{x}^{n_d}_d). N)$, then we get $\ell_R$ $( M/(x^{n_1}_1, \ldots,
x^{n_d}_d) M)\leq \ell_A(N/(\bar{x}^{n_1}_1, \ldots, \bar{x}^{n_d}_d)
N)$ for arbitrary positive integers $n_1, \ldots, n_d$. Therefore, by
\ref{chap1:sec2:subsec1.16}, we get  
\begin{align*}
  e_R(x_1, \ldots, x_d|M) &= \lim_{\min (n_i)\to \infty} \frac{l_R
    (M/(x^{n_1}_1, \ldots, x^{n_d}_d)M)}{n_1\cdots\cdot \cdot n_d}\\ 
  & \leq \lim \limits_{\min (n_i)\to \infty} \frac{\ell_A (N/(\bar{x}^{n_1}_1,
  \ldots, \bar{x}^{n_d}_d)N)}{n_1\cdots\cdot\cdot n_d}\\ 
  & =e_A(\bar{x}_1, \ldots, \bar{x}_d|N)=e_0(q;M)~
\end{align*}
\text{by}\ref{chap1:sec2:lem1.18}.  

Thus it is enough to prove that
$$
\ell_R(N/\bar{x}^{n_1}_1, \ldots,
\bar{x}^{n_d}_d)N)=\ell_A(N/\bar{x}^{n_1}_1, \ldots,
\bar{x}^{n_d}_d)M).  
$$

Put $I=(\bar{x}_1, \ldots, \bar{x}_d)\cdot A$. Since every $A$-module
is also $R$-module, it follows that $\ell_A(N/\bar{x}^{n_1}_1, \ldots,
\bar{x}^{n_d}_d)N)\le l_R (N/\bar{x}^{n_1}_1, \ldots,
\bar{x}^{n_d}_d)N)$ and \break $l_A$  $((N/\bar{x}^{n_1}_1,$ $\ldots,
\bar{x}^{n_d}_d)N/I^nN)$ $\le l_R((\bar{x}^{n_1}_1, \ldots,
\bar{x}^{n_d}_d)N/I^nN)$, where $n \geq n_1+ \cdots n_d$.  

Therefore\pageoriginale it is enough to prove that
$$
\ell_R(N /I^n N)=\ell_A(N/I^n N). 
$$ 


Since $\bar{x}_1, \ldots, \bar{x}_d$ annihilates $I^i N/I^{i+1}N$, it
follows that $l_R(I^i N/I^{i+1}N)\break =l_A(I^i N/I^{i+1}N)$ for all $i \ge
0$. Therefore we get $l_R(N/I^nN)=\sum \limits^{n-1}_{i=0}l_R$ $(I^i
N/I^{i+1}N)=\sum \limits^{n-1}_{i=0}
\ell_A(I^iN/I^{i+1}N)=\ell_A(N/I^nN)$.  

This completes the proof of \ref{chap1:sec2:subsec1.17}. 

\setcounter{corollary}{18}
\begin{corollary}\label{chap1:sec2:coro1.19} %% 1.19
  Let $M$ be any finitely generated $R$-module and\- \hbox{$(x_1, \ldots,
    x_d) =q \subset R$} be an ideal of definition generated by a system
  of parameters for $R$. Then
  $$
  e_R(x_1, \ldots, x_d|M)=0 \Longleftrightarrow K-\dim (M) < \dim (R)=d.  
  $$
\end{corollary}

In particular, $e_R(x_1, \ldots, x_d|R)>0 $. 

\begin{corollary}\label{chap1:sec2:coro1.20}
  Let $M$ be any finitely generated $R$-module and $q=(x_1,
  \ldots, x_d), q'=(x'_1, \ldots, x'_d)$ be ideals of definitions
  generated by systems of parameters for $R$. Then 
  \begin{enumerate}[(i)]
  \item If $q' \subset q$, then $e_R(x_1, \ldots, x_d|M)\leq e_R(x'_1,
    \ldots, x'_d|M)$ and 
  \item If $q'=q$, then $e_R(x_1, \ldots, x_d|M)= e_R(x'_1, \ldots,
    x'_d|M)$.  
  \end{enumerate}
\end{corollary}

\begin{corollary}\label{chap1:sec2:coro1.21}
  Let $M$ be any finitely generated $R$-module and $q=(x_1,
  \ldots, x_d)$ be an ideal of definition generated by a system of
  parameters for $R$. Then 
  \begin{multline*}
    e_0(q;M)=l(M/qM)-\ell ((q_{d-1}M \underset{M}:x_d)/q_{d-1}M)\\
    -\sum^{d-1}_{k=l}e_0(q/q_k, (q_{k-1}0\underset{M}{:}x_k)/q_{k-l}M)   
  \end{multline*}
  where\pageoriginale \qquad $q_k=(x_1, \ldots, x_k), 0 \leq k \leq d-1$.
\end{corollary}

\begin{proof}
  This follows from remark (\ref{chap1:sec2:rem1.4})(i) and
  (\ref{chap1:sec2:subsec1.17}). 
\end{proof}

%\setcounter{corollary}{21}
\begin{corollary}\label{chap1:sec2:coro1.22}
 Let $M$ be any finitely generated $R$-module and $q=(x_1,
  \ldots, x_d) \subset R$ be any ideal of definition generated by a
  system of parameters $\{x_1, \ldots, x_d\}$ for $R$. Then
  $e_0(q;M)=\ell(M/qM)-\ell((q_{d-1}\underset{M}{:x_d)}q_{d-1} M)$ if
  and only if $x_k$ is not in any prime ideal $\mathscr{Y}$ belonging
  to Ass $(M/q_{k-1}M)$ such that $K-\dim R/\mathscr{Y} \geq d-k$,
  where $q_k =(x_1, \ldots, x_k), 0 \leq k \leq d-1$. 
\end{corollary}

\begin{proof}
  It is easy to see that 
  $$
  \text{ Ass } ((q_{k-1}\underset{M}{:x_k)}/q_{k-1}M)=\text{ Ass
  }(M/q_{k-1}M)\cap V((x_k)).  
  $$
  
  Therefore we get
  $$
  K-\dim(q_{k-1}M\underset{M}{:}x_k)/q_{k-1}M)= \underset {\mathscr{Y}_i
    \in Ass (M/q_{k-1}M)\cap V((x_k))}{\Sup K-\dim R/\mathscr{Y}_i} <d-k 
  $$
  if and only if $x_k \notin \mathscr{Y}$ for all $\mathscr{Y} \in Ass
  (M/q_{k-1}M)$ with $K-\dim R/\mathscr{Y} \geq d-k$.  
\end{proof}

Therefore by (\ref{chap1:sec2:coro1.19}), we get
$e_0(q/q_k;(q_{k-1}M:x_k)/q_{k-1}M)=0$ if and only if $x_k \notin
\mathscr{Y}$ for all $\mathscr{Y} \in Ass (M/q_{k-1}M)$ with $K-\dim
R/\mathscr{Y}\geq d-k$. Now (\ref{chap1:sec2:coro1.22}) follows from
(\ref{chap1:sec2:coro1.21}).   

\setcounter{definition}{22}
\begin{definition}\label{chap1:sec2:def1.23}
  (see \cite{2}). Let $M$ be a finitely generated $R$-module. A set of
  elements $x_1, \ldots, x_d \in \rad (R)$ is said to be a 
  reducing system of parameters with respect to $M$ if  
  \begin{enumerate}[(a)]
  \item $\{ x_1, \ldots, x_d\}$ is system of parameters for $R$
  \item $e_0 (q; M) = \ell (M / q M ) -\ell ((q_{d-1}M:_M x_d)
    /q_{d-1}M)$ where $q = (x_1, \ldots$, $x_d)$ and $q_{d-1} = (x_1,
    \ldots$, $x_{d-1})$.  
  \end{enumerate}\pageoriginale
\end{definition}

The following propositions are useful for the computation of the
multiplicity  

\setcounter{proposition}{23}
\begin{proposition}\label{chap1:sec2:prop1.24} %%% 1.24
  Let $M$ be any finitely generated $R$-module and $(x_1, \ldots,
  x_d) = q \subset R$ be an ideal generated by a system of parameters
  $x_1, \ldots, x_d$ for $R$. Then $q$ can be generated by a reducing
  system of parameters with respect to $M$. 
\end{proposition}

\begin{proof}
  By (\ref{chap1:sec2:coro1.22}), it is enough to prove that $x_k
  \notin \mathscr{Y}$ for 
  all $\mathscr{Y} \in $ Ass $(M/ q_{k-1}M)$ such that $K-\dim
  R/\mathscr{Y} \geq d-k$, where $q_k = (x_1, \ldots, x_k), 0 \leq k
  \leq d-1$.  
\end{proof}

Let $i$ be an integer with $1 \leq i \leq d$. Suppose that there exist
elements $y_1, \ldots, y_{i-l}$ such that $q=(y_1, \ldots, y_{i-l},
x_i, \ldots, x_d) $ and $y_j \notin \mathscr{Y}$ for all $\mathscr{Y}
\in $ Ass $(M/ (y_1, \ldots, y_{j-l})M)$ with $K- \dim R/\mathscr{Y}
\geq d-j$, for any $j= 1, \ldots, i - 1$.  

We set $q=(y_1, \ldots,  y_{i-l}, x_{i+1}, \ldots, x_d) $. It is clear
that $q \subset m q+ q_i$, where $m$= rad $(R)$. Hence there is an
element $y_i \in q$ such that $ y_i \notin m q+q_i$ and $ y_i \notin
\mathscr{Y}$ for any $\mathscr{Y} \in $ Ass $(M/ (y_1, \ldots,
y_{i-1})M)$ with $R/ \mathscr{Y} \geq d -i$ Since $y_1, \ldots,
y_{i-1}, x_{i+1}, \ldots,  x_d$ are linearly independent mod $m, q$,
Nakayama's lemma implies $q=(y_1, \ldots,  y_{i}, x_{i+1}, \ldots,
x_d)$.  

\begin{proposition}\label{chap1:sec2:prop1.25}
Let $(R, \mathscr{M})$ be a noetherian local ring and $q=
  (x_{1}, \ldots, x_d) \subset R$\pageoriginale be an ideal generated
by a system 
  of parameters $(x_{1},$ $\ldots, x_d) $ for $R$. Then we put
  $\mathscr{O}_0:  = (0)$ and $\mathscr{O}_k:  = U( \mathscr{O}_{k-1})+
  (x_k)$ for any $0 < k< d$. Then $e_0 (q;R) = l (R/
  \mathscr{O}_d)$. 
\end{proposition}

\begin{proof}
  From the proof of (\ref{chap1:sec2:subsec1.9}), we have 
  $$
  e_0(q;R) = e_0 (q/ x_1 ; R/ ((x_1) + (0:  x^n_1)))
  $$
  for\pageoriginale large $n$. Proof by induction on $d$. Let
  $d=1$. Then it is clear 
  that $(0:  x^n_1) =U(0)$ for large $n$ and $e_0((x_1);R) = \ell(R/((x_1)
  + (0:  x^n_1))) = \ell (R/((x_1) +U (0 ))) = \ell (R/ \mathscr{U})$.  
  
  Now suppose that $d = s+ 1, s \geq 1$ and the result holds for
  $d=s$. First we shall show that $U((x_1) + (0:  x^n_1)) = U((x_1) +
  U(0))$ for large $n$. Let $\mathscr{Y} \in V((x_1))$ be such that $K-
  \dim R/ \mathscr{Y} = d-1$. Then it is easy to see that, for large
  $n$  
  $$
  \mathscr{Y} \in \text{ Ass }(R/ ((x_1) + (0:  x^n_1)))
  \Longleftrightarrow \mathscr{Y} \in \text{ Ass }(R/ ((x_1) + U( 0))) 
  $$

  Moreover, $((x_1) + (0:  x^n_1))_{ \mathscr{Y}} =((x_1) + U(
  0))_{\mathscr{Y}} $ for any $\mathscr{Y} \in \text{ Ass }\break (R/ ((x_1) +
  (0:  x^n_1))) = \text{ Ass } (R/ ((x_1) + U( 0)))$ with $K-\dim R/
  \mathscr{Y} = d-1$.  
\end{proof}

Therefore $U((x_1) + (0:  x^n_1)) =U((x_1) +U(0)) $ for larger $n$

Put $R':  = R/(x_1) + (0:  x^n_1)$ for large $n, \mathscr{O}1= (0)$
and $\mathscr{O}_k = (x_k) + U (\mathscr{O}_{k-1}$ for any $1 < k \leq
d$. Then by induction we get $e_0 (q;R) = e_0(q';R') = \ell( R' /
\mathcal{U})$. Now, since $\mathscr{O}= U(0) + (x_1)$, it follows that  

$U(\mathscr{O})=U(U(0)+U((x_1)+(0:x^n_1)$ and
$\mathscr{O}'_d=\mathscr{O}_d/((x_1)+(0:x^n_1))$ for large
$n$. Therefore
$e_0(q;R)=e_0(q';R')=\ell(R'/\mathscr{O}'d)=\ell(R/_{\mathscr{O}d})$.  

\setcounter{example}{25}
\begin{example}\label{chap1:sec2:exp1.26}
  Take the classical example from [\cite{90}, \S 11] ( see also [\cite{26},
    p. 180] and [\cite{50}, p. 126]).  
  
  Let $V_1, V_2$ and $C$ be the subvarieties of $\mathds{P}^4_k$ with
  defining prime ideals: 
  \begin{align*}
    \mathscr{Y}_{v_1} & =(X_1 X_4-X_2 X_3, X^2_1 X_3-X^3_2,
    X_1X^2_3-X^2_2 X_4, X_2X^2_4 -X^3_3), \\ 
    \mathscr{Y}_{v_1} & =(X_1, X_4) \text{ and } \mathscr{Y_C}=(X_1,
    X_2, X_3, X_4).  
  \end{align*}
\end{example}

We put $A(V_1;C):=A:=(K[(X_0, X_1, X_2, X_3,
  X_4]/\mathscr{Y}_{v_1})\mathscr{Y}_c$.  

Then $\mathscr{Y}_{v_2}. A=(X_1, X_4)A$ is generated by a system of
parameters $X_1, X_4$ for $A$ and  
$$
\mathscr{Y}_{v_2}+(X_1)=(X_1, X_2 X_3, X^2_2, X_2 X^2_4-X^3_3)\cap(X_1,
X^3_2, X_3, X_4) 
$$
is a primary decomposition of $\mathscr{Y}_{v_1}+(X_1)$ in
$A$. Therefore $U(\mathscr{Y}_{v_2}+(X_1))=(X_1, X_2 X_3, X^2_2, X_2
X^2_4-X^3_3)$. It follows from (\ref{chap1:sec2:prop1.25}) that 
\begin{align*}
  e_0(\mathscr{Y}_{v_2} A;A) & =\ell (A/(x_4) + U(\mathscr{Y}_{v_1}+(X_1)))\\
  & = \ell (A/(X_1, X_4, X^2_2, X_2 X_3, X^2_3)A) = 4. 
\end{align*}

Also, $\ell(A/\mathscr{Y}_{v_2})=\ell(A(X_1, X_4, X_2X_3, X^3_2,
X^3_3)A)=5$. Therefore in this example the inequality in
\ref{chap1:sec2:subsec1.9} is a
strict inequality, i.e., $e_0(A/\mathscr{Y}_{v_2}A;A)\break <
l(A/(\mathscr{Y}_{v_2}$). 

\section{The HILBERT Function and the Degree}\label{chap1:sec3}

\setcounter{notation}{26}
\begin{notation}\label{chap1:sec3:not1.27}
  The\pageoriginale following notation will be used in sequel.
\end{notation}

Let $K$ be a field and $ R: = K[ X_O, \ldots, X_n]$ be the polynomial
ring in $(n+1)$-variables over $K$. Let $V(n+1, t)$ denote the
$K$-vector space consisting all forms of degree $t$ in $ X_O, \ldots,
X_n$. It is easy to see that $\dim_K V(n+1, t)= (\overset{t+n}n)$, for
all $t \geq 0, n \geq 0$. 

Let $I \subset R$ be a homogeneous ideal. Let $V(I,t)$ be the
$K$-vector space consisting of all forms in $V(n+1, t)$ which are
contained in $I$.  

\setcounter{definition}{27}
\begin{definition}\label{chap1:sec3:def1.28}
  The numerical function $H(I,-): \mathbb{Z}^+ \to \mathbb{Z}^+$ defined
  by $H(I, t) = \dim_K V(n+ 1,t) - \dim_K V(I,t)$ is called the
  \textit{Hilbert function of} I. 
\end{definition}

\section*{General properties of the HILBERT function}

Let $I, J \subset R $ be two homogeneous ideals.

\setcounter{subsection}{28}
\subsection{}\label{chap1:sec3:subsec1.29}
\begin{enumerate}[(i)]
\item  If  I $\subset J$ then $H(I, t) \geq H(J,t)$  for all 
  $t \geq 0 $. 
\item $H(I + J, t) = H(I,t) + H(J, t) -H (I \cap J,t)$  for all
 $t \geq 0 $. 
\end{enumerate}

\begin{proof}
  Since $V(I,t) \leq V(J,t)$ if $I \subset J$ and
  $V(I+J,t)=V(I,t)+V(J,t)-V( I \cap J,t)$ for all $t \geq 0$, (i) and
  (ii) are clear. 
\end{proof}

Let $\varphi \varepsilon R$ be a form of degree $r$. Then

\setcounter{subsection}{29}
\subsection{} \label{chap1:sec3:subsec1.30}
\begin{enumerate}[(i)]
\item ~
  \vskip -1.6cm
  \begin{align*}
    H(\varphi R,t) & = H((0), t) - H((0),t-r)\\
    & = ( ^{t+n}_{n}) - (^{t-r+n}_n) \text{ for } t \geq r-n \\
    & = (^{t+n}_n) \text{ for } 0 \leq t \leq r-n
  \end{align*}\pageoriginale
\item ~
  \vskip -1.6cm
  \begin{align*}
    H(I \cap \varphi R,t) & =H(\varphi R,t)+ H(I: \varphi), t-r).\\
    & = H(\varphi R,t)+ H(I, t-r) \text{ if } (I:\varphi) = I. 
  \end{align*}
\item ~
  \vskip-1.6cm
  \begin{align*}
    H(I+\varphi R,t) &= H(I,t) - H(I:  \varphi), t-r).\\
    & = H(I,t) - H(I:  t-r) \text{ if } (I: \varphi) = I.
  \end{align*}
\end{enumerate}

In particular, 
\begin{tabular}[t]{l}
  $H((0),t) = \binom{t+n}{n}$\\
  $H((1),t) = 0$ for all  $t \geq 0$.
\end{tabular}

\begin{proof}
  It is easy to see that $I \cap \varphi R = (I: \varphi)\cdot \varphi
  R$. Therefore we get  
  $$
  \dim_K V(I \cap \varphi R, t) = \dim_K V(I: \varphi,  t-r)
  $$
  In particular (take $I= R),\dim_K V(I \varphi R, t) = \dim_K( R,
  t-r)$. From this and \ref{chap1:sec3:subsec1.29} all (i),(ii), (iii)
  are clear.  
\end{proof}

\setcounter{subsection}{30}
\subsection {}\label{chap1:sec3:subsec1.31}
Let $\mathscr{Y} \subset R$ be a homogeneous prime ideal
  with $K-\dim R/ \mathscr{Y} = 1$ If $K$ is algebraically closed,
  then $\mathscr{Y}$ is generated by $n$ linear forms and  
  \begin{enumerate}[(i)]
  \item $ H(\mathscr{Y},t) = 1~\textit{ for all } t \geq 0$
  \item $\textit{ For any } r >0,  H (\mathscr{Y}^r, t) = 1 +\binom n l+ 
    \binom{n+1}{2}+ \cdots + \binom{n+r-2}{r-1} = \binom{n+r-2}{r-1}$ 
  \end{enumerate}
  
\begin{proof}
  We\pageoriginale may assume that $X_0 \notin \mathscr{Y}$. Consider the ideal
  $\mathscr{Y}_* = \{ f_*|f(1,$ $X_1 / X_0, \ldots, X_n / X_0 ), f \in
  \mathscr{Y}\} \subset K[ X_1 / X_0, \ldots, X_n / X_0]$. It is easy to
  see that this is a maximal ideal in $K[ X_1 / X_0, \ldots, X_n /
    X_0]$. Therefore, by Hilbert's Nullstellensatz, there exist $a_1,
  \ldots, a_n \in K$ such that $\mathscr{Y}_*=(X_1 / X_0-a_1, \ldots, X_n
  / X_0-a_n)$. Now it is easy to see that $\mathscr{Y}=(X_1 -a_1 X_0,
  \ldots, X_n- a_n X_0)$. To calculate $H(\mathscr{Y},t)$ and
  $H(\mathscr{Y}^r,t)$, we may assume that $\mathscr{Y}= (X_1,
  \ldots, X_n)$ Then it is clear that $H(\mathscr{Y},t)=1$ for all $t
  \geq 0$ and since $\mathscr{Y}^r$ is generated by forms of degree $r$
  in $X_1, \ldots, X_n$ it follows that 
\end{proof}

\begin{align*}
  H(\mathscr {Y}^r,t) & = \sum^{r-1}_{k=0} \text{ (forms of degree}~ k
  ~\text{in}~   x_1, \ldots, x_n )\\ 
  & =\sum^{r-1}_{k=0} \binom{n+k-1}{k}= \binom{n+r-1}{r-1}
\end{align*}

The following is a well-known theorem (for proof see \cite{26},
  \cite{55} or \cite{72}). 

\subsection{HILBERT-SAMUEL Theorem}\label{chap1:sec3:subsec1.32} 
Let $ I \subset R$ be a homogeneous ideal. The Hilbert function
  $H(I, t)$, for large $t$ is a polynomial $P(I, t)$ in $t$ with
coefficients in $Q$. The degree $d(0
\leq d \leq n)$ of this polynomial $P(I,t)$ is called the 
  projective dimension or dimension of $I$ and we will denote it by
$\dim(I)$. It is well-known that $\dim (I) = K- \dim (I)-1$. We will
write the polynomial $P(I,t)$ in the following form:  
$$
P(I,t)= h_0(I) (^t_d) + h_1(^t_{d-1}) + \cdots + h_d,
$$
where\pageoriginale $h_0(I) > 0, h_1, \ldots, h_d$ are integers.

\setcounter{definition}{32}
\begin{definition}\label{chap1:sec3:def1.33}
  \begin{enumerate}[(a)]
  \item Let $I \subset R$ be a homogeneous ideal. The positive integer
    $h_0 (I)$ is called \textit{the degree} of $I$. 
  \item Let $V = V(I) \subset \mathbb{P}^n_K$ be a projective variety in
    $\mathbb{P}^n_K$ defined by a homogeneous ideal $I \subset R$. Then
    $K-\dim (I) (\resp  \dim(I)$, degree of I) is called the \textit{
      Krull-dimension of  V (\resp  The dimension of $V$, the degree of
     $V)$} and we denote it by $K- \dim (V)(\resp  \dim(V), \deg (V))$. $V$
    is called \textit{ pure dimensional or unmixed} if $I$ is unmixed. 
  \end{enumerate}
\end{definition}

\setcounter{remark}{33}
\begin{remark}\label{chap1:sec3:rem1.34}
 In general, the degree of $V$ is to be the number of points in which
 almost all linear subspaces $L^{n-\dim(V)}\subset \mathbb{P}^n_K$
 meet $V$. By combining this geometric definition with a variant of
 the Hilbert polynomial, we can give our purely algebraic definition
 of $\deg (V)$ and open the way to the deeper study of this properties
 (see [50, Theorem (6.25) on p. 112]).  
\end{remark}

\section*{Some properties of the degree.}

\setcounter{subsection}{34}
\subsection{}\label{chap1:sec3:subsec1.35}
 
Let $\varphi_1,  \ldots, \varphi_s \in R$ be
  forms of degrees $r_1,  \ldots, r_s$, respectively. If\break $((\varphi_1,
  \ldots, \varphi_{i-1}):\varphi_i)=(\varphi_1,  \ldots,
  \varphi_{i-1})$ for any $1 \leq i \leq s$ then 
  $$
  h_0((\varphi_1,  \ldots, \varphi_s))= r_1 \ldots.. r_s.
  $$
\begin{proof}
  Proof by induction on $s$. Suppose $s=1$. Then by
  \ref{chap1:sec3:subsec1.30} (i) we have $
  H((\varphi_1),t)= \binom{t+n}{n} - \binom{n+r-n}{n^1}=r_1 \binom{t}{n-1})+
  \cdots$ for all $t \geq r_1 -n$. 
\end{proof}

Therefore\pageoriginale $h_0((\varphi_1))=r_1$. Now suppose $s = p+1,
p \geq 1$ and 
result holds for $r=p$. Since $((\varphi_1,  \ldots,
\varphi_{s-1}):\varphi_s)=(\varphi_1,  \ldots, \varphi_{s-1})$ by
\ref{chap1:sec3:subsec1.30} (iii) we have  
\begin{multline*}
  H((\varphi_1,  \ldots, \varphi_s),t)  = H((\varphi_1,  \ldots,
  \varphi_{s-1}),t)-H((\varphi_1,  \ldots, \varphi_{s-1}),t-r_s)\\ 
   = h_0((\varphi_1,  \ldots, \varphi_{s-1}))\binom{t}{n-s+1} +
  \cdots -h_0((\varphi_1,  \ldots, \varphi_{s-1}))
  \binom{t-r_s}{n-s+1} \cdots \\ 
  = r_s-h_0((\varphi_1,  \ldots, \varphi_{s-1}))(\underset{n-s}{t})+
  \cdots \text{ for all } t \geq r_s -n. 
\end{multline*}
Therefore, by induction, we get $h_0((\varphi_1,  \ldots,
\varphi_{s}))=r_s\cdot h_0 (\varphi_1,  \ldots, \varphi_{s-1}) = r_1,
\ldots r_s$. 

\setcounter{subsection}{35}
\subsection{}\label{chap1:sec3:subsec1.36}
Let $I \subset R$ be a homogeneous ideal and $\varphi \in R$
  be a form of degree $r$. Then 
\begin{enumerate}[(i)]
\item If $\dim (I,\varphi) = \dim (I) =\dim(I:  \varphi)$ then
  $ h_0( I, \varphi) = h_0(I) - h_0(I: \varphi))$. 
\item If $\dim (I,\varphi) = \dim (I)  > \dim(I:\varphi) $
  then $ h_0( I, \varphi) = h_0(I)$. 
\item If $(I:\varphi) = I$, then $h_0 (I, \varphi) = r. h_0(I)$. 
\end{enumerate}

\begin{proof}
This follows from \ref{chap1:sec3:subsec1.30}
\end{proof}

\subsection{}\label{chap1:sec3:subsec1.37}

Let $I \subset R$ be a homogeneous ideal. Then 
$$
h_0(I) = h_0(U(I))
$$
\begin{proof}
  Suppose $\dim (I) = d$. We may assume that $I\not\subseteq \subset U
  (I)$. Then we have $I = U(I) \cap J$ where $J \subset R$ is a
  homogeneous ideal with $\dim (J) < \dim (U)= \dim (I) = d$. Therefore from
  \ref{chap1:sec3:subsec1.29} (ii), we get $h_0(I) = h_0(U(I))$.  
\end{proof}

\subsection{}\label{chap1:sec3:subsec1.38} 
Let $\mathscr{Y} \subset R$ be a homogeneous prime ideal
  and $q \subset R$ be a homo\-geneous\pageoriginale $\mathscr{Y}$-primary ideal.  Then  
$$
h_0(q) = l (q). h_0(\mathscr{Y}).
$$
\begin{proof}
  Let $q= q_1 \subset q_2 \subset \cdots \subset q_\ell	 = \mathscr{Y}
  $ be a composition series for $q$. It is enough to prove that  
$$
h_0(q_i) = h_0(q_{i+1})+ h_0(\mathscr{Y}) \text{ for any } 1 \leq i \leq \ell-2.
$$ 
\end{proof}

We assume $i = 1$. There exist forms $\varphi_1,  \ldots, \varphi_s$
such that $q_2 = (q_1, \varphi_1,  \ldots$, $\varphi_s)$.  By using
remark in (1.1) to the $\mathscr{Y} R_{\mathscr{Y}}$- primary ideal $q
R_{\mathscr{Y}} \subset R_{\mathscr{Y}}$, it follows the $\mathscr{Y}
\varphi_i \subset q_1$ for all i= $1,  \ldots, s$ and there exist
forms $\alpha_i$ and $\beta_i, 2 \leq i \leq r$ such that  
\begin{enumerate}[(i)]
\item $\beta_i \notin \mathscr{Y}$ for all $2 \leq i \leq s$.
\item $\alpha_i \varphi_i - \beta_i \varphi_1 \in q_1 $ for all $2
  \leq i \leq s$ 
\end{enumerate} 

Therefore $(q_1:  \varphi_1) = \mathscr{Y}$ and since $\mathscr{Y}
 \underset{+}{\subset} (( q_1 \varphi_1,  \ldots, \varphi_i):
 \varphi_{i+1})$ the homogeneous ideals $(( q_1 \varphi_1,  \ldots,
 \varphi_i):  \varphi_{i+1})$ have dimension $< d$, for any $1 \leq i
 \leq s-1$. Therefore from
 \ref{chap1:sec3:subsec1.36}(i), \ref{chap1:sec3:subsec1.36} (ii), we
 get $h_0(q_2) = h_0 
 (( q_1 \varphi_1,  \ldots, \varphi_{s-1})= h_0 (( q_1 \varphi_1,
 \ldots, \varphi_{s-2})= \cdots = h_0 (q_1) - h_0(\mathscr{Y})$. 

 \subsection{}\label{chap1:sec3:subsec1.39}
  Let $\mathscr{Y}_1 \neq \mathscr{Y}_2$ be two homogeneous ideals in
   $R$ and let $q_i$ be two homogeneous $\mathscr{Y}_i$- primary
   ideals for $i=1,2$ If $\dim q_1 = \dim q_2 $ then $h_0 ( q_1 \cap
   q_2) = h_0(q_1) + h_0(q_2)$. 

 \begin{proof}
   Since $\mathscr{Y}_1 \neq \mathscr{Y}_2$,  it follows that $\dim (q_1
   + q_2) < \dim q_1 =\dim q_2$. Therefore,
   from \ref{chap1:sec3:subsec1.29} (ii), we have 
   $$
   h_0 ( q_1 \cap q_2) = h_0(q_1) + h_0(q_2).
   $$
 \end{proof}

\subsection{ }\label{chap1:sec3:subsec1.40}
Let\pageoriginale $ I \subset R$ be a homogeneous ideal. Then 
$$
h_0(I) = h_0(U(I)) = \sum l(q). h_0 (\mathscr{Y}).
$$
where $q$ runs through all $\mathscr{Y}$-primary components of
  $I$ with $\dim (q) = \dim (I)$. 
\begin{proof}
  This follows from
  \ref{chap1:sec3:subsec1.37}. \ref{chap1:sec3:subsec1.39}
  and \ref{chap1:sec3:subsec1.38}. 
\end{proof}

\subsection{}\label{chap1:sec3:subsec1.41}
Let $\bar{I} \subset K[ X_0, \ldots, X_{n-1}]$ be a homogeneous ideal
of dimension $d$ with the Hilbert function $H(\bar{I},t) = h_0(^t_d) +
h_1 (^t_{d-1})+ \cdots h_d$ for $t>>1$. Let $I^{*} \subset K[ X_0, \ldots,
  X_{n}]$ be the homogeneous ideal generated by $\bar{I}$ Then $\dim
(I^{*})= \dim (I) + 1 = d+1$ and the Hilbert function of $I^*$
is given by  

$H(I^*,t) = h_0 (^t_{d+1}) + (h_0 + h_1) (^t_d) + \cdots + (h_d +h_d
+1)$ for $t >> 1$. 
\begin{proof}
  Every form $\varphi \in I^*$ of degree $t$ can be written uniquely
  in the form  
  $$
  \varphi = \varphi_t + \varphi_{t-1}X_n + \cdots + X^t_n
  $$
  where $\varphi_t , \varphi_{t-1}, \ldots$ are forms of degrees $t,t-1,
  \ldots$ in $\bar{I}$. 
  
  Therefore $V(I^*,t) = \sum \limits^{t}_{k = 0} V(\bar{I},k)$ and hence 
\end{proof}
\begin{align*}
  H(I^*,t) & = \binom{t+n}{n} -\sum^{t}_{k = 0} [
    \binom{t+n}{n-1})-H(\bar{I},k) ]. \\ 
  & = \sum ^{t}_{k = 0}H(\bar{I},k), \text{ since } \sum^{t}_{k =
    0}\binom{t+n}{n-1})= \binom{t+n}{n}\\ 
  & = h_0 \sum^{t}_{k = 0} \binom k d + h_1 \sum^{t}_{k = 0} \binom{k}{d-1} +
  \cdots + h_d \sum^{t}_{k = 0} \binom k 0\\ 
  & = h_0 \binom{t+l}{d+l}+ h_1 \binom{t+1}{d} + \cdots + h_d \binom{t+1}{l}\\
  & = h_0 \binom{t}{d+1}+ (h_0+ h_1) \binom t d + \cdots + (h_{d-1}+
  h_d)\binom t l+ (h_{d}+ h_{d+1}) 
\end{align*}

\subsection{}\label{chap1:sec3:subsec1.42} 
Let\pageoriginale $I \subset K[ X_0, \ldots, X_{n}]$ be a homogeneous ideal of
dimension $d(0 \leq d \leq n-1)$. Put $\bar{I} =I \cap K[ X_0, \ldots,
  X_{n-1}] $ 

$I_1 = \{ \varphi \in K[ X_0, \ldots, X_{n-1}]| \varphi_i$
 is a form such that $\varphi_0 + \varphi_1 X_n \in I$
 for some form  $\varphi \in K[ X_0, \ldots, X_{n-1}] \}$

$I_1 = \{ \varphi_i \in K[ X_0, \ldots, X_{n-1}]| \varphi_i$ 
is a form and $\varphi_0 + \varphi_1 X_n + \cdots + \varphi_i X^i_n
\in I $ for some forms $\varphi_0,  \ldots, \varphi_{i-1}  \in
K[ X_0, \ldots, X_{n-1}] \}$ for $i \geq 1$.  

Then it is clear that 
$$
\bar{I} \subset I_1 \subset I_2 \subset \cdots \subset I_r = I_{r+1} =
\ldots \text{ for some } r \geq 1. 
$$
Therefore, we get 
$$
\dim V (I,t) = \dim V (I,t) + \sum^t_{k=1} \dim V (I_k, t-k) \text{
  for all } t \geq 0 
$$
and hence 
\begin{align*}
  H(I,T) &= \binom{t+n}{n}- \binom{t+n-1}{n-1} + H(\bar{I}, t) -\sum^t_{k=1}
  [(^{t+n-1}_{n-1}) -H (I_k, t-k)] \\ 
  &= H(\bar{I}, t) -\sum^t_{k=1}H (I_k, t-k) \text{ for all } t \geq 0.
\end{align*}\pageoriginale

\setcounter{example}{42}
\begin{example}\label{chap1:sec3:exp1.43}
\begin{enumerate}[(i)]
 \item Let $\mathscr{Y}$ be the prime ideal 

   $(X_0 X_2 -X^2_1, X_1 X_2-X_0 X_3, X^2_2-X_1 X_3) \subset K[X_0,
   X_1, X_2, X_3]$. 

   Following the notation of \ref{chap1:sec3:subsec1.42}, it is easy
   to see that  
   \begin{align*}
     \bar{\mathscr{Y}} &= (X_0 X_2 - X^2_1)\\
     \mathscr{Y}_1 & =\mathscr{Y}_2 = \ldots (X_0,X_1).
   \end{align*}

   Therefore, by \ref{chap1:sec3:subsec1.35}, we get 
   
   $H(\bar{\mathscr{Y}},t) = 2t+1$ and
   $H(\mathscr{Y}_1,t)=H(\mathscr{Y}_2,t)= \cdots = 1$ for all $t \geq
   0$. 
   
   Hence by \ref{chap1:sec3:subsec1.42}
   $$
   H(\mathscr{Y},t)=H(\bar{\mathscr{Y}},t)+ \sum ^t_{k=0}
   H(\mathscr{Y}_k,t-k)= 3t+1. 
   $$
   Therefore $h_0(\mathscr{Y}) = 3$


 \item Let $\mathscr{Y}$ be the prime ideal 
   $(X_0 X_2 -X^2_1, X^2_2-X_0 X_3) \subset K[X_0,
   X_1,\break X_2, X_3]$. Then $\bar{\mathscr{Y}}=(X_0 X_2 -
   X^2_1),H(\bar{\mathscr{Y}},t) = 2t +1$, for all $t \geq
   0$. $\mathscr{Y}_1=\mathscr{Y}_2= \ldots = (X_0,X^2_1)$, 
   \begin{equation*}
     H(\mathscr{Y}_1,t) = H(\mathscr{Y}_2,  t) = 
     \begin{cases}
       1 \text{ for } t = 0 \\
       2 \text{ for all } t \geq 1
     \end{cases}
   \end{equation*}
   Therefore, by \ref{chap1:sec3:subsec1.42}, we get
   \begin{equation*}
     H(\mathscr{Y},t) = H(\bar{\mathscr{Y}},  t) + \sum\limits_{k=0}^t
     H(\mathscr{Y}_k,t-k) =
     \begin{cases}
       1 \text{ for } t = 0 \\
       4t \text{ for all } t \geq 1.
     \end{cases}
   \end{equation*}
   Hence\pageoriginale $h_0 (\mathscr{Y}) = 4$.
 \item Let $\mathscr{Y}$ be the prime ideal 
   $(X^2_0 X_2 -X^3_1, X_0 X_3-X_1 X_2, X_0 X^2_2 X^2_1 X_3,\break X_1 X^3_2
   - X^3_2) \subset K[X_0, X_1,X_2, X_3]$ 
   
   Then 
   \begin{align*}
     \bar{\mathscr{Y}}& =(X^2_0 X_2 -X^3_1),  H(\bar{\mathscr{Y}})=
     \begin{cases}
       1 \text{ for } t = 0 \\
       3t \text{ for all } t \geq 1.
     \end{cases}\\
     \mathscr{Y}_1 &=(X_0 X^2_1 ),  H(\mathscr{Y}_1,t)=
     \begin{cases}
       1 \text{ for } t = 0 \\
       2 \text{ for all } t \geq 1.
     \end{cases}
   \end{align*}
   $\mathscr{Y}_2 =\mathscr{Y}_3 = \cdots = (X_0,X_1), H(\mathscr{Y}_2,t)
   = H(\mathscr{Y}_3,t)= \ldots = 1 $ for all $t \geq 0$. Therefore, by
   \ref{chap1:sec3:subsec1.42}, we get  
   \begin{equation*}
     H(\mathscr{Y},t) = H(\bar{\mathscr{Y}},t)+ \sum ^t_{k=0}
     H(\mathscr{Y}_k,t-k)= 
     \begin{cases}
       1    &\text{ for } t = 0 \\
       4    &\text{ for } t \geq 1.\\
       4t+t &\text{ for } t = \geq 2.
     \end{cases}
   \end{equation*}
   Hence $H_0(\mathscr{Y}) = 4$. 
 \item Let $\mathscr{Y} \subset K[X_0, X_1,X_2, X_3] = R $ be the prime
   ideal in example (iii) above and $\mathscr{Y} =(X_1,X_4) \subset
   K[X_0, X_1,X_2, X_3]$. 
\end{enumerate}
\end{example}

Then $q:  = (\mathscr{Y}+\mathscr{Y'}) = (X_0, X_3,X_1, X_2,X^3_1,X^3_2)$
is $(X_0, X_1,X_2, X_3)$- primary ideal and it is easy to see that $l
(R/ q) = 5$. Therefore, by \ref{chap1:sec3:subsec1.38}, we have $h_0 (q) = \ell (R/ q)\cdot
h_0 ((X_0, X_1,X_2, X_3))=5$ and from example (iii)
$h_0(\mathscr{Y})=4$. This shows that 
$$ 
5 = h_0(q) \neq h_0(\mathscr{Y})\cdot h_0(\mathscr{Y}') = 4.
$$

\section{Miscellaneous Results}\label{chap1:sec4} %%% D

Now\pageoriginale we collect some results which will be used in the next
sections. Let $K$ be a field. 

\setcounter{proposition}{43}
\begin{proposition}\label{chap1:sec4:prop1.44}
  Let $A$ be a finitely generated $K$-algebra. Then $U = \{
  \mathscr{Y} \in \spec  (A) |A_{\mathscr{Y}}$ is
  Cohen-Macaulay $\}$ is a non-empty Zariski-open subset of  $\spec
  (A)$. 
\end{proposition}

For the proof of this proposition, we need the following lemma. 

\setcounter{lemma}{44}
\begin{lemma}\label{chap1:sec4:lem1.45}
  Let $A$ be a finitely generated $K$-alegbra and $\mathscr{Y} \in
  \spec (A)$ If $A_{\mathscr{Y}}$ is Chone-Macaulay, then there
  exists a maximal ideal $m$ of $A$ containing $\mathscr{Y}$ such that
  $A_\mathscr{M}$ is Chone-Macaulay. 
\end{lemma}

\begin{proof}
  Proof by induction on $d:  = \dim (A)$.
\end{proof}

\noindent \textbf{Case(i):} ht $\mathscr{Y} = 0$. In this case, $\mathscr{Y}$ is a
  minimal prime ideal of $A$. If $d = K- \dim A = o$, then there is
  nothing to prove. Now suppose that $d= s+1, s \geq 0$ and the result
  holds for $d = s$. Replacing $A$ by $A_f$ for some $f \notin
  \mathscr{Y}$, we may assume that Ass $(A) = \{ \mathscr{Y} \}$ and
  $K-\dim A> O$. Then depth $(A) > 0$. Let $x \in A$ be a non-zero-
  divisor in $A$ and $q$ be a minimal prime ideal of $x\, A$. Then by
  Krull's $PID$ ht $q = 1$ and hence $\mathscr{Y} \subset q$. 

  Put $A' = A/$(x) and $q' = qA$. Then $A_p'$ is Cohen Macaulay and
  $h\,t\, q' = 0$. Therefore,by induction, there exists a maximal ideal
  $\mathscr{M'}$ of $A'$ with $q' \subset \mathscr{M'}$ and $A'm$ is
  Cohen-Macaulay.\pageoriginale Then $\mathscr{M}=\mathscr{M'} \cap A$ is a maximal
  ideal of $A$ containing $q \supset \mathscr{Y}$ and $A_m$ is
  Cohen-Macaulay. 

\noindent \textbf{Case (ii):} ht $\mathscr{Y} = r > 0$.


Since $A_{\mathscr{Y}}$ is Cohen-Macaulay of dimension $r$ there exist
$x_1,  \ldots x_r$ in $\mathscr{Y}$ such that $\{ x_1,  \ldots x_r \}
$is an $A_{\mathscr{Y}}$- sequence. 

By replacing $A$ by $A_f$ for some $f \notin\mathscr{Y}$, we may
assume that\break $\{ x_1,  \ldots x_r \}$ is an A-sequence and
$\mathscr{Y}$ is a minimal prime ideal of\break $(x_1,  \ldots x_r)$. Put
$A' = A/ (x_1,  \ldots x_r)$ and $\mathscr{Y}' = \mathscr{Y} A'$. Then
ht $\mathscr{Y}' =0$ and $A'_{\mathscr{Y}'}$ is Cohen-Macaulay; therefore,
by case(i), there exists a maximal ideal $m'$ of $A'$ such that $m'
\supset \mathscr{Y}'$ and $A_{m'}$ is Cohen-Macaulay. Then $m = m'
\cap A$ is a maximal ideal of $A$ with $m \supset \mathscr{Y}$ and
since $\{ x_1,  \ldots x_r \} $ is an A-sequence, it follows that
$A_\mathscr{M}$ is Cohen-Macaulay. 

\medskip
\noindent \textbf{Proof of Proposition (1.44). } 
Clearly $U \neq \phi $. Let $\mathscr{Y} \in U$ shall show that there
exists $ f \notin \mathscr{Y} $such that  
$D(f) = \{ q \in \text{ Spec } (A) | f \notin q \} \subset U$, that
is, $A_f$ is Cohen-Macaulay for some $f \notin \mathscr{Y} $. By
(\ref{chap1:sec4:lem1.45}), we may assume that $\mathscr{Y} = \mathscr{M}$ is a maximal
ideal of $A$. Replacing $A$ by $A_f$ for some $f \notin \mathscr{M}$
we may assume that Ass $(A) = \mathscr{Y}_1,  \ldots,  \mathscr{Y}_r $
with $\mathscr{Y}_i \subset m,  i \leq i \leq r$. Since $A_m$ is
Cohen- Macaulay, we have $d:  = $ht $m = \dim A_m = \dim (A/
\mathscr{Y}_i)_m$ for all $1 \leq i \leq r$. Therefore  
$$
\dim A = \sup_{1 \leq i \leq r} \dim A / \mathscr{Y}_i = \sup_{1 \leq
  i \leq r}  \dim (A/\mathscr{Y}_i)_\mathscr{M} = d 
$$
and\pageoriginale there exist $x_1, \ldots,  x_d \in m$ such that $\{ x_1, \ldots,
x_d\}$ is an $A_m $-sequence. Further, replacing $A$ by $A_f$ for some
$f \not\in m$,  we may assume that $\{ x_1, \ldots$, $x_d\}$ is an
$A$-sequence. This shows that $A$ is Cohen-Macaulay. 

\setcounter{proposition}{45}
\begin{proposition}\label{chap1:sec4:prop1.46}~

  \begin{enumerate}[\rm 1.]
  \item Let $L|K$ be a field extension. Let $I \subset K_0
    [X_0, \ldots,  X_n]$ $=:R$ be a homogeneous ideal. Put
    $\bar{R}=L [X_0, \ldots,  X_n]$. Then $h_0 (I) = h_0 (I\bar{R})$. 
  \item Let $A$ be a finitely generated $K$-algebra and 
    $I \subset A$ be an unmixed ideal. Let $x \in A$
    be such that $K-\dim (A/ (I, x)) = K-\dim (A/I)-1$. Then 
    $$
    \rad  (U((I, x))) = \rad  (I, x).
    $$
  \item Let $V= V(I) \subset \mathbb{P}^n_K$ be a projective
    variety defined by the homogeneous ideal $I \subset K [X_0, \ldots,
      X_n] =:  R$. Let $C$ be an irreducible subvariety of $V$ with the
    defining prime ideal $\mathscr{Y}$. Let $A=(R/I)_\mathscr{Y}$ be the
    local ring of $V$ at $C$. If $V$ is pure dimensional, then 
    $$
    K-\dim (A) = K-\dim (V)-K-\dim(C).
    $$
  \end{enumerate}
\end{proposition}

\begin{proof}%pro
\begin{enumerate}
\item Clear.
\item Put $d:  = K-\dim (A/I)$. It is enough to prove that, for every
  minimal prime ideal $q$ of ($I, x$) 
  $$
  K-\dim (A/q) = d-1.
  $$

  Since\pageoriginale $I$ is unmixed $d=K-\dim$($A/I$) = $K-\dim $ ($A/\mathscr{Y}$)
  for every $\mathscr{Y} \in $ Ass ($A/I$). Let ($I, x$) $
  \subset q \subset A$ be a minimal prime ideal of $(I,x)$. Then there
  exists a minimal prime ideal $\mathscr{Y}$ of $I$ such that
  $\mathscr{Y} \subsetneq q $ and by Krull's Principal Ideal Theorem,
  we have ht $q/ \mathscr{Y}=1$. Therefore; since $A$ is a finitely
  generated $K$-algebra, we get 
  $$
  K-\dim A/q = K-\dim A/\mathscr{Y} - ht q/\mathscr{Y} = d - 1.
  $$
\item Let $I \subset q \subset R$ be a minimal prime ideal of $I$ such
  that $K-\dim$ ($R/q$)$_\mathscr{Y}$ = $K-\dim$ ($A$). Then,  since
  $I$ is unmixed and $R/q$ is a finitely generated $K$-algebra, we get 
  \begin{multline*}
    K-\dim (V) = K-\dim R/I=K-\dim R/q=K-\dim (R/\mathscr{Y})\\
    +K-\dim  (R/q) \mathscr{Y} = K-\dim (C)+K-\dim (A). 
  \end{multline*}
\end{enumerate}
\end{proof}

\setcounter{proposition}{46}
\begin{proposition}\label{chap1:sec4:prop1.47}
  Assume that $K$ is algebraically closed. Let $L|K$ and $L'|K$
  be field extensions and $A,B$ be finitely generated
  K-algebras. Then 
  \begin{enumerate}[(i)]
  \item
    \begin{enumerate}[(a)]
    \item $L \underset{K}{\otimes} L'$ is an integral domain.
    \item  $K-\dim$ ($A \underset{K}{\otimes} B$) = K-dim ($A$) $+$ K-dim
      ($B$) and if $A$ and $B$ are integral domain then $A
      \underset{K}{\otimes} B$ is an integral domain.
    \item Put $A_L:  = L \underset{K}{\otimes} A$. Then K-dim $A_L =
      $K-dim $A$ and if $A$ is an integral domain then $A_L$ is an
      integral domain. 
    \end{enumerate}
  \item There is a one-one correspondence between the isolated
    prime ideals of $A$ and the isolated prime ideals of  $A_L = L
    \underset{K}{\otimes} A$\pageoriginale which preserves $K$-dimensions. 
  \item 
    \begin{enumerate}[(a)]
    \item If $A$ is unmixed then $A_L = L \underset{K}{\otimes}
      A$ is unmixed. 
    \item If $A$ and $B$ are unmixed then $A
      \underset{K}{\otimes} B$ is unmixed. 
    \end{enumerate}
  \item 
    \begin{enumerate}[(a)]
    \item If $A$ and $B$ are Cohen-Macaulay then $A
      \underset{K}{\otimes} B$ is Cohen-Macaulay. 
    \item Let $\mathscr{Y} \in \spec (A)$ and $q \in 
      \spec (B)$. If $A_{\mathscr{Y}}$ and $B_q$ are
      Cohen-Macaulay then $A_{\mathscr{Y}} \underset{K}{\otimes} B_q$ is
      Cohen-Macaulay.  
    \end{enumerate}
  \end{enumerate}
\end{proposition}

\begin{proof}%pro
\begin{enumerate}[(i)]
\item
  (a)~  We may assume that $L$ is finitely generated over $K$. Let $\{
    x_1, \ldots, x_n \} \subset L$ be a separating transcendence basis of
    $L|K$ (since $K$ is algebraically closed it exists). Put $L_1:  =
    K(x_1, \ldots, x_n)$. Then $L|L_1$ is separable and hence $L = L_1
    (\alpha) = L_1 [x]/(f(x)$), where $f(x)$ is the irreducible
    polynomial of over $L_1$. Sine $L_1 \underset{K}{\otimes} L' \simeq
    S^{-1 }$ ($L'$[$x_1, \ldots, x_n$]), where $S=K$ [$x_1, \ldots,
      x_n$] - $0, L_1 \underset{K}{\otimes} L'$ is an integral domain
    with quotient field $E = L' (x_1, \ldots, x_n)$. Now, note that
    since $K$ is algebraically closed in $L'$, it is easy to see that
    $L_1 = E(x_1, \ldots, x_n)$ is algebraically closed in $E$. Then $L
    \underset{K}{\otimes} L' = L \underset{L_1}{\otimes} L_1 L_1
    \underset{L_1}{\otimes} L'\supset L \underset{L_1}{\otimes} E =
    L_1 [x]/(f(x))
    \underset{L_1}{\otimes} E = E[x] /(f(x))$ is an integral domain. 

  (b)~  By\pageoriginale Normalization Lemma, we have
    $$
    K-\dim (A \underset{K}{\otimes} B) = K-\dim (A) + K-\dim (B).
    $$

    Let $L$ (\resp  $L'$) be the quotient field of $A$ (\resp  $B$). 

    Then $A \underset{K}{\otimes} B \subset L \underset{K}{\otimes}
    L'$ which is an integral domain by ($a$). 

  (c)~ Similar to ($b$).
\item Let $\mathscr{Y} \in \spec (A)$. Then by ($i$) ($c$)
  $\mathscr{Y} A_L \in \spec (A_L)$ and $K-\dim (\mathscr{Y})
  = K-\dim (\mathscr{Y} A_L$). It is easy to see that $\mathscr{Y}$ is
  isolated if and only if $\mathscr{Y} A_L$ is isolated. Therefore
  $\mathscr{Y} \leftrightarrow \mathscr{Y} A_L$  is, as required $a
  1-1$ correspondence. 
\item 
  (a)~ Let $\mathscr{Y} \in$ Ass ($A$). Then by $(i) (c)
    K-\dim \mathscr{Y} A_L = K-\dim \mathscr{Y}= K-\dim A= K-\dim 
    A_L$. Therefore it is enough to prove that Ass $(A_L) = \{
    \mathscr{Y} A_L | \mathscr{Y} \in \text { Ass } (A) \}$,
    which follows from ($ii$). 

  (b)~ Let $\mathscr{Y} \in$ Ass ($A$) and $q \in $
    Ass ($B$). Then by ($ii$) ($b$) ($\mathscr{Y}, q$) is a prime
    ideal in  $A \underset{K}{\otimes} B =:  C$ and  $K-\dim
    (\mathscr{Y}, q) C=K-\dim \mathscr{Y}+K-\dim q=K-\dim A+K-\dim
    B=K-\dim (A \underset{K}{\otimes} B)$. 

  Therefore it is enough to prove that
  $$
  \text {Ass} (C) = \{ (\mathscr{Y}, q). C | \mathscr{Y} \in
  \text { Ass} (A), q \in (B) \}. 
  $$
  
  Let $P \in$ Ass ($C$). Then since $C$ is flat over $A$ and $B$
  it follows that $P \cap A = \mathscr{Y} \in$ Ass ($A$) and $P
  \cap B = q \in$ Ass ($B$). 
  
  By replacing $A$ by $A _\mathscr{Y}$ we may assume that $A$ is local\pageoriginale
  with maximal ideal $\mathscr{Y} \in$ Ass ($A$). Since $A$ is
  unmixed $A_{\mathscr{Y}}$ is unmixed and therefore $A_{\mathscr{Y}}$
  is artinian. 
  
  Now there exists a coefficient field $L$ of $A$ containing $K$ and $L
  \underset{K}{\otimes} B \to A \underset{K}{\otimes} B$ is an
  integral extension. It follows from ($a$) that $B_L:  = L
  \underset{K}{\otimes} B $ is unmixed and by ($ii$) $qB_L \varepsilon$
  Ass ($B_L$). If $(\mathscr{Y}, q)\,  C \subsetneq P$) then  $qB_L \subsetneq
  P \cap B_L$ because $B_L \to A \underset{K}{\otimes} B$ is an
  integral extension. 
  
  Since $A \underset {K}{\otimes} B$ is a free $B_L$-module it follows
  that $P \cap B_L \in$ Ass ($B_L$). 

  This contradicts the fact that $B_L$ is unmixed. Therefore $P =
  (\mathscr{Y}, q)\break \cdot C$. 
\item
  (a)~ Let $K$-dim $A = r$ and K-dim $B=s$. Then we have K-dim $A
    \underset{K}{\otimes} B =$ K-dim $A+K-\dim B=r + s$. Let $\{ a_1,
    \ldots,  a_r \}$ (\resp  $\{ b_1, \ldots,  b_s \}$) be an
    $A$-sequence (\resp  $B$-sequence). Then, since $K$ is a field, it is
    easy to see that $\{ a_1 \otimes 1, \ldots,  a_r \otimes 1,
    \otimes b_1, \ldots$,  $1 \otimes b_s\}$ is an ($A
    \underset{K}{\otimes} B$)-sequence of length $r+s$. Therefore   
    $A\underset{K}{\otimes} B$ is Cohen-Macaulay. 

(b)~It is easy to see that $A_{\mathscr{Y}} \underset{K}{\otimes}
  B_q  \xrightarrow{\sim} S^{-1} (A \underset{K}{\otimes} B)$, where
  $S$ is the multiplicative set $(A-\mathscr{Y}) \underset{K}{\otimes}
  (B-q)$ in $A \underset{K}{\otimes} B$. By
  (\ref{chap1:sec4:prop1.44}) there exist $f 
  \in A-\mathscr{Y}$ and $ g \in B-q$ such that $A_f$
  and $B_f$ are Cohen-Macaulay. Therefore by ($a$) $A_f
  \underset{K}{\otimes} B_g$ is Cohen-Macaulay. Since $A_{\mathscr{Y}}
  \underset{K}{\otimes} B_g \xrightarrow{\sim} S^{-1}
  \underset{K}{\otimes} B$) is a localization of $A_f
  \underset{K}{\otimes} B_g$ it follows that $A_{\mathscr{Y}}
  \underset{K}{\otimes} B_g$ is Cohen-Macaulay. 
  \end{enumerate}
\end{proof}
