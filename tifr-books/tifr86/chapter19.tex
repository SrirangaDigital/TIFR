\chapter{Berger spheres}\label{c19}

The\pageoriginale proof of Theorem \ref{c18:thm18.1} is completed in
this lecture by constructing a negatively curved Riemannian metric on
$M\# \Sigma$ satisfying property 2 of \ref{c18:thm18.1} when the
injectivity radius of $M$ is large enough; how large depends on how
small the given real number $\epsilon$ is. This construction depends
on the following result that puts a negatively curved Riemannian
metric on $\mathbb{R}^{2m}$ which agrees with $\mathbb{CH}^{m}$ near
$\infty$, with $\mathbb{H}^{2m}$ near $0$, and whose sectional
curvatures are $\epsilon$-pinched close to $[-4,-1]$. Here is the
precise statement.

\begin{lemma}\label{c19:lem19.1}
Given a positive integer $m$, there exists a family $b_{\gamma}(,)$ of
complete Riemannian metrics on $\mathbb{R}^{2m}$ which is
parameterized by the real numbers $\gamma\geq e$ and has the following
three properties.
\begin{enumerate}
\item The sectional curvatures of $b_{\gamma}(,)$ are all contained in
$$
[-4-\epsilon(\gamma),-1+\epsilon(\gamma)]
$$ 
where $\epsilon(\gamma)$
  is a $\mathbb{R}^{+}$ valued function such that
  $\lim\limits_{\gamma\to +\infty}\epsilon(\gamma)=0$.

\item The ball of radius $\gamma$ about $0$ in
  $(\mathbb{R}^{2m},b_{\gamma})$ is isometric to a ball of radius
  $\gamma$ in $\mathbb{H}^{2m}$.

\item There is a diffeomorphism $f$ from
  $(\mathbb{R}^{2m},b_{\gamma})$ to $\mathbb{CH}^{m}$ which maps the
  complement of the ball of radius $\gamma^{2}$ centered at $0$
  isometrically to the complement of the ball of radius $\gamma^{2}$
  centered at $f(0)$.
\end{enumerate}
\end{lemma}

Before constructing these metric $b_{\gamma}(,)$, let us use them to
complete the proof of \ref{c18:thm18.1}. Start by using the last
paragraph of Lecture \ref{c18} to select an exotic $8n+2$ dimensional
sphere $\Sigma$ such that $f^{*}_{\mathbb{C}P^{4n+1}}([\Sigma])\neq
0$. Then fix a positive real number $\gamma$ such that
$\epsilon(\gamma)<\epsilon_{0}$ where
$\epsilon_{0}=\min(\epsilon,\frac{1}{2})$ and let $M$ be a closed
complex $4n+1$ dimensional hyperbolic manifold such that $M\# \Sigma$
is not diffeomorphic to $M$ and so that the injectivity radius of $M$
is bigger than $\gamma^{2}+1$. Such a manifold $M$ exists because of
\ref{c17:thm17.2}, \ref{c17:thm17.3}, \ref{c18:coro18.3} and
\ref{c16:coro16.2}. Let $\mathbb{D}_{1}$ be a ball of radius
$\gamma^{2}+1$ in $\mathbb{CH}^{4n+1}$ centered at $f(0)$ where $f$ is
the diffeomorphism given by property 3 of
\ref{c19:lem19.1}. Isometrically identify $\mathbb{D}_{1}$ with a
codimension-0 submanifold of $M$. Now change the metric on
$\mathbb{D}_{1}$ to $b_{\gamma}(,)$ using $f$. Let $\mathbb{D}_{2}$ be
the ball in $(\mathbb{R}^{2n+1},b_{\gamma})$ with center\pageoriginale
0 and radius $\gamma$. Perform the connected summing of $M$ with
$\Sigma$ inside of $f(\mathbb{D}_{2})$. And additionally require that
$\gamma>\alpha+1$ where $\alpha$ is the real number given in Lemma
\ref{c17:lem17.4} which depends on $8n+2$ and $\epsilon_{0}$. Then the
argument proving \ref{c17:lem17.4} also shows how to put a Riemannian
metric on $M\# \Sigma$, keeping the already constructed Riemannian
metric on $M-f(\mathbb{D}_{2})$, so that property 2 of
\ref{c18:thm18.1} is satisfied. Setting $N=M\# \Sigma$, Theorem
\ref{c18:thm18.1} is proven.

The remainder of this lecture is devoted to constructing the
Riemannian metrics $b_{\gamma}(,)$ posited in \ref{c19:lem19.1}. This
is done by finding a ``nice'' family of Riemannian metrics $c_{t}(,)$
on $S^{2m-1}$ parametrized by $t\in (0,+\infty)$ such that
$(S^{2m-1},c_{t})$ is isometric to the boundary of the ball of radius
$t$ in $\mathbb{H}^{2m}$ for $t\leq \gamma$ and, respectively, in
$\mathbb{CH}^{m}$ for $t\geq \gamma^{2}$. Use vector space scalar
multiplication to identify $S^{2m-1}\times(0,+\infty)$ with
$\mathbb{R}^{2m}$. Then $b_{\gamma}(,)$ is defined so that the induced
metric on $S^{2m-1}\times t$ is $c_{t}(,)$, the induced metric on
$x\times (0,+\infty)$ comes from the canonical one on $\mathbb{R}$,
and $S^{2m-1}\times t$ is perpendicular to $x\times (0,+\infty)$ for
each $(x,t)\in S^{2m-1}\times (0,+\infty)$. This outlines the
construction. We now furnish details.

Let $S^{2m-1}$ be the sphere of unit radius in $\mathbb{C}^{m}$
relative to the standard positive definite Hermitian form
$$
u\cdot
v=u_{1}\overline{v}_{1}+u_{2}\overline{v}_{2}+\cdots+u_{m}\overline{v}_{m}.
$$

There is a natural free action of the circle
$$
S^{1}=\{z\in \mathbb{C}|~|z|=1\}
$$
on $S^{2m-1}$ whose orbits fiber $S^{2m-1}$ over
$\mathbb{C}P^{m-1}$. This equips $S^{2m-1}$ with complementary
distributions $\eta_{1}$, $\eta_{2}$ where the Whitney sum
$\eta_{1}\oplus \eta_{2}$ equals the tangent bundle $TS^{2m-1}$ of
$S^{2m-1}$. The 1-dimensional distribution $\eta_{1}$ is tangent to
the orbits of the $S^{1}$-action while the $(2m-2)$-dimensional
distribution $\eta_{2}$ consists of all tangent vectors perpendicular
to $\eta_{1}$ relative to the inner product real $(u\cdot v)$; i.e.,
the real part of the Hermitian form $u\cdot v$. Note that $\eta_{2}$
is a $\mathbb{C}$-vector bundle.

Fix\pageoriginale an ordered pair of positive real numbers $a$, $b$
and define a new Riemannian metric $\langle,\rangle$ on $S^{2m-1}$ by
requiring the following equations be valid when $\nu\in \eta_{1}$ and
$u\in \eta_{2}$;
\begin{align*}
& \langle u,u\rangle = a^{2}u\cdot u,\\
& \langle u,v\rangle =b^{2}v\cdot v,\quad\text{and}\\
& \langle u,v\rangle =0.
\end{align*}

The Riemannian manifold $S^{2m-1}$ equipped with $\langle,\rangle$ is
called a Berger sphere and is denoted by $S^{2m-1}_{a,b}$.

Berger spheres are relevant to constructing the family of Riemannian
manifolds $(S^{2m-1},c_{t})$ since the distance sphere of radius $t$
in $\mathbb{CH}^{m}$ and $\mathbb{H}^{2m}$, respectively, is $S_{a,b}$
where $a=\sinh(t)$, $b=\sinh(t)\cosh (t)$ in the case of
$\mathbb{CH}^{m}$ and $a=b=\sinh(t)$ in the case of
$\mathbb{H}^{2m}$. It is also interesting to note that the distance
spheres of radius $t$ in the positively curved dual symmetric spaces
to $\mathbb{CH}^{m}$ and $\mathbb{H}^{2m}$; namely, in
$\mathbb{C}P^{m}$ and $S^{2m}$, are also the Berger spheres $S_{a,b}$
where $a=\sin(t)$, $b=\sin(t)\cos(t)$ in the case of
$\mathbb{C}P^{m}$, $t\in (0,\frac{\pi}{2})$, and where $a=b=\sin(t)$
in the case of $S^{2m}$, $t\in (0,\pi)$.

To prove that the metrics $b_{\gamma}(,)$ constructed by the above
outline using Berger spheres for $(S^{2m-1},c_{t})$ satisfy property 1
of \ref{c19:lem19.1}, it is necessary to know the sectional curvatures
of Berger spheres. Fortunately, these are well known; cf.\@ \cite{11}
and \cite{19}. One method for calculating them is the following. Let
$U(m)$ denote the unitary group; i.e., the isometry group of the
Hermitian form $u\cdot v$. It acts transitively on $S^{2m-1}$. On the
other hand, it is also contained in $\text{Iso}(S^{2m-1}_{a,b})$;
i.e., $\langle,\rangle$ is a $U(m)$-invariant Riemannian metric on
$S^{2m-1}$. Therefore O'Neill's Riemannian submersion formula
\cite{81} can be used to give the following calculation of the
sectional curvature $K(P)$ of a plane $P$ tangent to $S^{2m-1}_{a,b}$
(See \cite[\S 2]{46} for the details of how this calculation is done.)
Pick an orthonormal basis $\{u,\cos\theta v+\sin \theta w\}$ for $P$
where $u$, $v\in\eta_{2}$ and $w\in \eta_{1}$. (All measurements
relative to the calculation of $K(P)$ are with respect to
$\langle,\rangle_{ab}$.) Note that $\theta$ is the angle between
$\eta_{2}$ and $P$. And let $\omega$ denote the angle between $v$ and
$iu$, then
$$
K(P)=\frac{b^{2}}{a^{4}}\sin^{2}\theta+\left(\frac{1}{a^{2}}+\frac{3(a^{2}-b^{2})}{a^{4}}\cos^{2}\omega\right)\cos^{2}\theta. 
$$

Berger's\pageoriginale interest in the Riemannian manifold $S_{a,b}$
can be explained from this formula by setting $a=1$ so that it
specializes to
$$
K(P)=b^{2}\sin^{2}\theta+(1+3(1-b^{2})\cos^{2}\omega)\cos^{2}\theta.
$$

It is then immediately seen that $\{S_{1,b}\mid b<1\}$ is a set of
positively curved, simply connected, Riemannian manifolds whose
sectional curvatures are all bounded above by 5 but which contains
manifolds of arbitrarily small injectivity radius. In fact, this
family of Riemannian manifolds more and more resembles
$\mathbb{C}P^{m-1}$ as $b\to 0$. Berger called attention to this
interesting phenomenon when $m=2$. It is also interesting to note that
these examples of Berger are essentially the distance spheres of
radius $t$ in $\mathbb{C}P^{m}$ as $t\to \frac{\pi}{2}$; more
precisely, these distance spheres are $S_{a,b}$ where $a=\sin t$,
$b=\sin t\cos t$.

On the other hand, we are interested in the distance sphere of radius
$t$ in $\mathbb{CH}^{m}$ which is $S_{a,b}$ where $a=\sinh(t)$,
$b=\sinh(t)\cosh (t)$, and that of radius $t$ in $\mathbb{H}^{2m}$
which is $S_{a,a}$. We also need to interpolate between these two;
i.e., to consider $S_{a,b}$ where $a=\sinh(t)$ and $a\leq b\leq
\sinh(t)\cosh(t)$. Now the above curvature formula in this situation
yields the following fact.

\begin{prop}\label{c19:prop19.2}
Assume that $a=\sinh(t)$ and $a\leq b\leq \sinh(t)\cosh(t)$. Then all
of the sectional curvatures of $S_{a,b}$ lie in the interval
$$
[-3,coth^{2}(t)].
$$ 
\end{prop}

I now construct the Riemannian manifolds $(S^{2m-1},c_{t}(,))$ used to
define $b_{\gamma}(,)$. To do this, fix a smooth function
$\psi:\mathbb{R}\to [0,1]$ which has the following properties.
\begin{enumerate}
\item $\psi(t)\geq 0$ for all $t\in \mathbb{R}$;

\item $\psi^{-1}(0)=(-\infty,1]$;

\item $\psi^{-1}(1)=[2,+\infty)$.
\end{enumerate}

Then use $\psi$ to define a family of smooth functions $\phi_{\gamma}$
parameterized by all real numbers $\gamma\geq e$ where
$\phi_{\gamma}:(0,+\infty)\to [0,+\infty)$. These functions are
  defined by the following formula:
$$
\phi_{\gamma}(t)=\psi\left(\frac{\text{ln\,}t}{\text{ln\,}\gamma}\right)t\quad\text{where}\quad
t\in (0,+\infty).
$$

Then\pageoriginale $(S^{2m-1},c_{t}(,))$ is the Berger sphere
$S_{a,b}$ where $a=\sinh(t)$ and
$b=\sinh(t)\cosh(\phi_{\gamma}(t))$. It is easy to see from this
definition that $b_{\gamma}(,)$ satisfies properties 2 and 3 of
\ref{c19:lem19.1}. However, the curvature calculations needed to
verify property 1 are quite complicated. These calculations use
\ref{c19:prop19.2}; but, more importantly, they use the method for
obtaining \ref{c19:prop19.2} via O'Neill's Riemannian submersion
formula. This method is used in a more elaborate way in varifying
property 1. See \cite[\S 2]{46} for details.
