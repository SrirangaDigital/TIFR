\chapter{Thin $H$-Cobordism Theorem}\label{c11}

Recall\pageoriginale that an $h$-cobordism $W^{m+1}$ with base a closed manifold
$M^m$ is a compact manifold with boundary such that $\partial W = M
\amalg N$ (disjoint union) and both $M$ and $N$ are deformation
retracts of $W$. If $r_t : W \to W$, where $t \in [0, 1]$, is a
deformation retraction onto $M$, then the \textit{tracks} of $r_t$ are
the following family of curves $\{ \alpha_x\mid x \in W\}$ in $M$
defined by 
$$
\alpha_x (t) = r_1 (r_t (x)), t \in [0, 1].
$$

The $h$- cobordism $W$ is said to be $\epsilon$-controlled if there
exists a deformation retraction $r_t$ onto $M$ such that each of its
tracks has diameter $\leq \epsilon$. (Here, some metric $d(\,,\,)$ on $M$
is fixed.) The next result is called the Controlled (or Thin)
$h$-Cobordism Theorem.

\begin{thm}[Ferry \cite{51}]\label{c11:thm11.1}
  Let $M^m$ be a closed, connected, smooth manifold with $m \geq 5$
  and equipped with a fixed metric $d(\,,\,)$. Then, there exists a number
  $\epsilon > 0$ such that every $\epsilon$-controlled $h$-cobordism
  with base $M$ is a cylinder; i.e., is homeomorphic to $M \times [0,1]$.
\end{thm}

\begin{prop}\label{c11:prop11.2}
  The Controlled $h$-Cobordism Theorem is implied by the
  Connell-Hollingsworth Conjecture.
\end{prop}

The remainder of this lecture is devoted to proving Proposition
\ref{c11:prop11.2}. The proof given here is implicit in the
Connell-Hollingsworth paper \cite{20}.

Assume Conjecture \ref{c10:conj10.3} is true and let $W$ be an
$h$-cobordism with base $M$. Smoothing theory, as developed in
\cite{67}, shows that $W$ has a smooth manifold structure inducing the
given structure on $M$. Hence, by the $s$-Cobordism Theorem, $W$ is a
cylinder if its Whitehead torsion $\tau (W, M)$ is 0 in $Wh\Gamma$,
where $\Gamma = \pi_1 M$. Because of Lemma \ref{c10:lem10.12}, there
is a number $\delta > 0$ such that $\hat{f} =0$ in $Wh\Gamma$ for any
$\delta$-isomorphism $f: G_1 \to G_2$ between geometric groups on
$M$. We next show that, if we set $\epsilon = \delta/(16m + 32)$ in
the statement of Theorem \ref{c11:thm11.1}, then there exists a
$\delta$-isomorphism $f: G_{\text{odd}} \to G_{\text{even}}$ such that
$\hat{f} = \tau (W, M)$\pageoriginale as elements in
$Wh\Gamma$. (Here, $f$ depends on $W$.) This proves Proposition
\ref{c11:prop11.2}, once $f$ is constructed.

Let $K$ be a triangulation of the pair $(W, M)$ by small simplies;
i.e., we require
\begin{center}
  diameter $r_1 (\Delta) \leq \epsilon$
\end{center}
for each simplex $\Delta$ in $K$. Let $\sigma_1, \sigma_2, \ldots
\sigma_n$ and $\tau_1, \tau_2, \ldots, \tau_n$ be, respectively,
orderings of the odd and even dimensional open simplies of $K$ in
$W-M$. (Since $W$ is an $h$-cobordism, the number of odd and even
dimensional simplies is same.) Pick points $x_1, x_2, \ldots, x_n$ and
$y_1, y_2, \ldots, y_n$ in $M$ such that $x_i \in r_1 (\sigma_i)$ and
$y_j \in r_1 (\tau_j)$. Then, $G_{\text{odd}}$ and $G_{\text{even}}$
are the geometric groups with bases $x_1, x_2, \ldots, x_n$ and $y_1,
y_2, \ldots , y_n$, respectively.

Let $(C_i, d_i)$ be the integral simplicial chain complex for $(W,
M)$.\break There is an obvious identification
\begin{equation*}
  G_{\text{even}} = \bigoplus_{i \geq 0} C_{2i}~\text{and}~
  G_{\text{odd}}= \bigoplus_{i \geq 0} C_{2i +1}.\tag{0}\label{c11:eq0}
\end{equation*}

Under this identification, the differentials $d_i$ determine a pair of
homomorphisms 
\begin{alignat*}{3}
  &D_{\text{odd}}  : G_{\text{odd}} \to G_{\text{even}}, 
  \quad & D_{\text{odd}} = \bigoplus_{i \geq} d_{2i+1};\\
  & D_{\text{even}}   : G_{\text{even}} \to G_{\text{odd}}, &
  D_{\text{even}}= \bigoplus_{i \geq 1} d_{2i}.
\end{alignat*}

Using the fact that the simplices of $K$ are small, it is seen that
both $D_{\text{odd}}$ and $D_{\text{even}}$ are
$\epsilon$-homomorphisms. 

The deformation retraction $r_t$ determines a chain contraction $c_i:
C_i \to C_{i+1}$. Under the above identification, This contraction
defines a pair of homomorphisms
\begin{align*}
  \Sigma_{\text{odd}} & : G_{\text{odd}} \to G_{\text{even}},\quad
  \Sigma_{\text{odd}} = \bigoplus_{i \geq 0} c_{2i+1};\\
  \Sigma_{\text{even}} & : G_{\text{even}} \to G_{\text{odd}},\quad
  \Sigma_{\text{even}}= \bigoplus_{i \geq 0} c_{2i}.
\end{align*}

Using that the tracks of $r_t$ and the simplices of $K$ both have
diameter $\leq \epsilon$, it is seen that both $\Sigma_{\text{odd}}$
and $\Sigma_{\text{even}}$ are $3 \epsilon$-homomorphisms.

Let\pageoriginale $f : G_{\text{odd}} \to G_{\text{even}}$ be the sum
$D_{\text{odd}}+ \Sigma_{\text{odd}}$. Then, $f$ is a $4
\epsilon$-homomorphism. Likewise, $g: G_{\text{even}}\to G_{odd}$ is
also a $4 \epsilon$-homomor\-phism, where $g= D_{\text{even}} +
\Sigma_{\text{even}}$. The following calculation is a consequence of
the fact that $(C_i, c_i)$ is a chain contraction for $(C_i, d_i)$.
\begin{equation*}
  \begin{aligned}
    f \circ g & = (D_{\text{odd}}+ \Sigma_{\text{odd}}) \circ
    (D_{\text{even}}+\Sigma_{\text{even}})\\
    & = D_{\text{odd}} D_{\text{even}} + \Sigma_{\text{odd}}
    \Sigma_{\text{even}} + \Sigma_{\text{odd}} D_{\text{even}} +
    D_{\text{odd}} \Sigma_{\text{even}}\\
    & = 0+ \Sigma_{\text{odd}} \Sigma_{\text{even}}+ id_{\text{even}}.
  \end{aligned}\tag{1}\label{c11:11.2-1}
\end{equation*}

Since $\sum_{\text{odd}} \sum_{\text{even}}$ raises degree by 2, it is
nilpotent. Consequently,
\begin{equation*}
  f^{-1} = g\circ (1+ \alpha + \alpha^2+ \cdots +
  \alpha^s)\tag{2}\label{c11:11.2-2} 
\end{equation*}
where $s= [(m+1)/2]$ and $\alpha=- \Sigma_{\odd}
\Sigma_{\even}$. (Recall that $m+1= \dim W$.) Using the Remark after
Lemma \ref{c9:lem9.3} together with equation \eqref{c11:11.2-2}, we
conclude that $f^{-1}$ is a $16 (m+2) \epsilon$-homomorphism; i.e.,
it is a $\delta$-homomorphism. Consequently, $f$ is a
$\delta$-isomorphism and hence $\hat{f}$ represents 0 in $Wh\Gamma$,
because of Lemma \ref{c10:lem10.12} and the assumption that Conjecture
\ref{c10:conj10.3} is true.

It is a pleasant exercise, using \ref{c11:sec11.3} below, to show that
the matrices $\hat{d}_i$ represent the differentials in the simplicial
chain complex $H_i (\tilde{W} , \tilde{M})$ in terms of a natural
$\mathbb{Z}\Gamma$ basis given by lifts of the simplices in
$K$. (Here, $\tilde{W}$ denotes the universal cover of $W$. Also, $d_i
: C_i \to C_{i-1}$ is regarded as a $\delta$-homomorphism between
geometric submodules of $G_{\even}$ and $G_{\odd}$ using the
identifications given in \eqref{c11:eq0}.) Likewise, consider
$\sigma_i : C_i \to C_{i+1}$. Using Lemma \ref{c10:lem10.7} on the
equation
$$
d_{i+1} \sigma_i + \sigma_{i-1} d_i = id,
$$    
we obtain the matrix equation
$$
\hat{d}_{i+1} \hat{\sigma}_i + \hat{\sigma}_{i-1} \hat{d}_i=I.
$$

Hence, the matrices $\hat{\sigma}_i$ give a chain contraction of the
$\mathbb{Z} \Gamma$-chain complex $(H_i (\tilde{W}, \tilde{M}),
\hat{d}_i)$. Therefore, the Whitehead torsion $\tau (W, M)$ of this
chain complex is represented by 
$$
\hat{D}_{\odd} + \hat{\Sigma}_{\odd}= \hat{f}.
$$

The\pageoriginale $s$-Cobordism Theorem now implies that $W$ is a cylinder since
$\hat{f}$ represents 0 in $Wh \Gamma$. \hfill Q.E.D.

\setcounter{section}{2}
\section{Second Method of Constructing $\hat{f}$.}\label{c11:sec11.3}

We end this lecture by giving an alternative method for constructing
$\hat{f}$, where $\hat{f} : G_1 \to G_2$ is a $\delta$-homomorphism of
geometric groups on a compact Riemannian manifold $X$. Let $p:
\tilde{X} \to X$ denote the universal covering space of $X$, and
identify $\Gamma= \pi_1 (X, *)$ with the group of all its deck
transformations by picking a base point $\tilde{*} \in \tilde{X}$ with
$p (\tilde{*})= *$. Each $G_i$ induces a \textit{geometric $\mathbb{Z}
  \Gamma$-module} $p^* G_i$ on $\tilde{X}$ as follows. Let $x_1, x_2,
\ldots, x_n$ and $y_1, y_2, \ldots y_m$ be respectively the bases of
$G_1$ and $G_2$, and $\alpha_i$, $\beta_j$ be a choice of paths
connecting $*$ to $x_i$, $y_j$. Let $\tilde{\alpha}_i$,
$\tilde{\beta}_j$ be the lifts of $\alpha_i$, $\beta_j$ to $\tilde{X}$
starting at $*$, and let $\tilde{x}_i$, $\tilde{y}_j$ be the endpoints
of $\tilde{\alpha}_i$, $\tilde{\beta}_j$, respectively. Then, $p^*
G_1$ and $p^* G_2$ are the free $\mathbb{Z}\Gamma$-modules 
with bases $\tilde{x}_{1},\tilde{x}_{2},\ldots,\tilde{x}_{n}$ and
$\tilde{y}_{1},\tilde{y}_{2},\ldots,\tilde{y}_{m}$, respectively. Note
that $p^{*}G_{1}$ is a, perhaps 
infinitely
generated, ``geometric group'' on $\tilde{X}$ with basis $\{\gamma
\tilde{x}_i \mid 1 \leq i \leq n, \gamma \in \pi_1 (X)\}$. A similar
remark holds for $p^* G_2$.

Observe the following consequences of the fact that $p$ is a local
isometry when $\delta$ is picked to be smaller than the injectivity
radius of $X$.

\noindent\textbf{Observation. } Let $\tilde{x} \in \tilde{X}$, $y \in
X$ be any pair of points such that $d(p(\tilde{x}), y)\leq
\delta$. Then, there is a unique point $\hat{y} \in \tilde{X}$, with
$d(\tilde{x}, \hat{y}) \leq \delta$, solving the equation $p
(\hat{y})=y$.

Recall that the $\delta$-homomorphism $f$ defines an integral matrix
$(a_{ij})$ via the system of equations 
$$
f(x_j) = \sum_{i=1}^n a_{ij}y_i.
$$

Since $a_{ij} \neq 0$ implies $d(x_j, y_i)\leq \delta$, we can apply
the observation to the pair $\tilde{x}_j$, $y_i$. This yields a point
$\hat{y}_{ij}$ with $p(\hat{y}_{ij})= y_i$. And, there is a unique
group element $\gamma_{ij} \in \Gamma$ such that $\gamma_{ij}
\tilde{y}_{i}=\hat{y}_{ij}$, since also $p(\tilde{y}_i)= y_i$. It can be seen, in
this way, that $\hat{f}$ is the $\mathbb{Z}\Gamma$-matrix whose
entries are $\hat{f}_{ij}= a_{ij}\gamma_{ij}$.
