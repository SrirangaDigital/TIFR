\chapter{Complex hyperbolic manifolds}\label{c18}

This\pageoriginale lecture starts the discussion of the
counterexamples constructed by Farrell and Jones in \cite{46} to the
smooth rigidity problem when $M^{m}$ is a complex hyperbolic
manifold. (The discussion is completed in the following lecture.) I
start by identifying complex hyperbolic $n$-space $\mathbb{CH}^{n}$
with an open subset of complex projective $n$-space
$\mathbb{C}P^{n}$. Recall that $\mathbb{C}P^{n}$ is the space whose
points consist of all complex lines containing $0$ in
$\mathbb{C}^{n+1}$. Fix the following non-degenerate indefinite
Hermition form $b(,)$ on $\mathbb{C}^{n+1}$ defined by
$$
b(x,y)=x_{1}\overline{y}_{1}+\cdots+x_{n}\overline{y}_{n}-x_{n+1}\overline{y}_{n+1}
$$
where the subscript denotes the co-ordinate of the vector referred
to. Then
$$
\mathbb{CH}^{n}=\{L\in \mathbb{C}P^{n}|\text{~the restriction $b|_{L}$
  is negative definite}\}.
$$

It is relatively easy to see that the thus defined $\mathbb{CH}^{n}$
is an open subset of $\mathbb{C}P^{n}$ and is biholomorphically
equivalent to $\mathbb{C}^{n}$. The group of all isometries of $b(,)$
with determinant equal to $1$ is $SU(n,1)$. This Lie group acts
transitively on $\mathbb{CH}^{n}$ and its isotropy subgroup at the
complex line spanned by $(0,\ldots,0,1)$ is $S(U(n)\times U(1))$ which
is a maximal compact subgroup of $SU(n,1)$. There is a Riemannian
metric $\langle,\rangle$ on $\mathbb{CH}^{n}$ such that $SU(n,1)$ acts
via isometries of it. And $\langle,\rangle$ is unique if we require,
as we now do, that its maximal sectional curvature is $-1$. This is
the canonical Riemannian metric on $\mathbb{CH}^{n}$. Let
$A(\mathbb{CH}^{n})$ denote the subgroup of
$\text{Iso}(\mathbb{CH}^{n})$ consisting of all holomorphic
isometries; it has index 2 in $\text{Iso}(\mathbb{CH}^{n})$. The
homomorphism $SU(n,1)\to \text{Iso}(\mathbb{CH}^{n})$ has image
$A(\mathbb{CH}^{n})$ and its kernel is the center of $SU(n,1)$ which
is a finite group and is identified as 
$$
\{\omega I\mid \omega\in \mathbb{C}\text{~ and~ } \omega^{n+1}=1\}
$$
where $I$ denotes the identity matrix.

We now recall the basic curvature properties of $\mathbb{CH}^{n}$
starting with the fact that all of its sectional curvatures lie in the
closd interval $[-4,-1]$. Since $\mathbb{CH}^{n}$ is a complex
manifold, given\pageoriginale a tangent vector $v$, we can form
$iv$. Then the sectional curvature in the direction of the
$\mathbb{R}$-plane $P$ spanned by $\{v,iv\}$ is $-4$. On the other
hand if $u\perp P$, then the sectional curvature in the direction of
the $\mathbb{R}$-plane $Q$ spanned by $\{u,v\}$ is $-1$. This is dual
in some specific sense to the situation for $\mathbb{C}P^{n}$; in
particular, $\mathbb{C}P^{n}$ has a canonical Riemannian metric whose
sectional curvatures all lie in \cite{1,4}. And, if $P$ and $Q$ are as
above, then the sectional curvature of $P$ is $4$ while that for $Q$
is $1$; cf.\@ \cite[pp. 321-329]{82}.

A {\em complex $n$-dimensional hyperbolic manifold} is a orbit space\break
$\mathbb{CH}^{n}/\Gamma$ where $\Gamma$ is a discrete, torsion-free
subgroup of $A(\mathbb{CH}^{n})$. (The real dimension of
$\mathbb{CH}^{n}/\Gamma$ is, of course, $2n$.) Such a group $\Gamma$
is said to be {\em regular} provided it splits back to $SU(n,1)$;
i.e., if there exists a subgroup $\overline{\Gamma}$ of $SU(n,1)$
mapping isomorphically onto $\Gamma$ under $SU(n,1)\to
A(\mathbb{CH}^{n})$. It is an easy consequence of Theorem
\ref{c17:thm17.3} that $\Gamma$ contains a regular subgroup of finite
index when $\mathbb{CH}^{n}/\Gamma$ is compact.

Since complex hyperbolic manifolds are negatively curved locally
symmetric spaces, Mostow's Strong Rigidity theorem yields isometric
rigidity in the special case of Conjecture \ref{c1:conj1.3} where both
$M$ and $N$ are complex hyperbolic of $\mathbb{C}-\dim\neq
1$. Hern\'andez \cite{58} and Yau and Zheng \cite{100} independently
extended this result to the situation where $N$ is assumed only to be
a Riemannian manifold whose sectional curvatures are all contained in
$[-4,-1]$. (But $M$ is still assumed to be complex hyperbolic of
$\mathbb{C}-\dim\neq 1$.) I now state precisely the nature of the
counterexamples to smooth rigidity constructed in \cite{46} in the
case where $M$ is a complex hyperbolic manifold.

\begin{thm}\label{c18:thm18.1}
Given any positive numbers $n\in \mathbb{Z}$ and $\epsilon\in
\mathbb{R}$, there exists a pair of closed negatively curved
Riemannian manifolds $M$ and $N$ having the following properties:
\begin{enumerate}
\item $M$ is a complex $4n+1$ dimensional hyperbolic manifold.

\item The sectional curvatures of $N$ are all in the interval
  $[-4-\epsilon,-1+\epsilon]$. 

\item The manifolds $M$ and $N$ are homeomorphic but not diffeomorphic.
\end{enumerate}
\end{thm}

The manifold $N$ in \ref{c18:thm18.1} is $M\# \Sigma$ where $M$ and
$\Sigma$ are, respectively, an appropriately chosen complex $4n+1$
dimensional hyperbolic manifold and a $8n+2$ dimensional exotic
sphere.\pageoriginale (Recall that a complex manifold is canonically
oriented.) The choices must be made so that properties 2 and 3 of
\ref{c18:thm18.1} hold. I show in this lecture how to choose $M$ and
$\Sigma$ so that property 3 holds. And the extra conditions necessary
to guarantee that property 2 also holds will be discussed in the next
lecture. Recall that $M$ and $M\# \Sigma$ are always homeomorphic
since $\dim \Sigma>4$. Hence we need only choose $M$ and $\Sigma$ so
that $M$ and $M\# \Sigma$ are not diffeomorphic in order to satisfy
property 3. Letting $[M]$ denote the concordance class of
$(M,\id_{M})$, Corollary \ref{c16:coro16.2} shows that it is
sufficient to choose $M$ and $\Sigma$ so that $[M\# \Sigma]\neq [M]$
in $\mathcal{C}(M)$; i.e., so that $f^{\ast}_{M}([\Sigma])\neq 0$. It
would be convenient at this point to be able to use Lemma
\ref{c16:lem16.3}; but unfortunately this can't be done since a closed
complex $m$-dimensional manifold $M$ is {\em never} stably
parallelizable when $m>1$; in fact, its first Pontryagin class is
never zero.

This last fact is a result of the close relationship between the
tangent bundle $TM$ of $M$ and that of its positively curved dual
symmetric space $\mathbb{C}P^{m}$. In fact, the following result was
proven in \cite[\S 3]{46}. 

\begin{lemma}\label{c18:lem18.2}
Let $\mathcal{M}$ be any closed complex $m$-dimensional hyperbolic
manifold. Then there exists a finite sheeted cover $M$ of
$\mathcal{M}$ and a map $f:M\to \mathbb{C}P^{m}$ such that the
pullback bundle $f^{*}(T\mathbb{C}P^{m})$ and $TM$ are stably
equivalent complex vector bundles.
\end{lemma}

\begin{remark*}
This result has recently been generalized by Boris Okun in
\cite{80}. He obtains a similar relationship between the tangent
bundles of any finite volume locally symmetric space of non-compact
type and that of its dual symmetric space of compact type. Both
\ref{c18:lem18.2} and \cite{80} depend on a deep result about flat
complex vector bundles due to Deligne and Sullivan \cite{25}. Their
result was also used by Sullivan in \cite{94} to prove Theorem
\ref{c17:thm17.1}. The observation made above that $\mathcal{M}$ has
finite sheeted cover $\mathbb{CH}^{m}/\Gamma$ where $\Gamma$ is
regular is needed to apply \cite{25} in proving \ref{c18:lem18.2}.
\end{remark*}

Now Lemma \ref{c18:lem18.2} can be used to prove the following useful
analogue of \ref{c16:lem16.3}.

\begin{coro}\label{c18:coro18.3}
Let\pageoriginale $\mathcal{M}$ be any closed complex $m$-dimensional
hyperbolic manifold. Then there exists a finite sheeted cover
$\mathcal{M}_{0}$ of $\mathcal{M}$ such that the following is true for
every finite sheeted cover $M$ of $\mathcal{M}_{0}$. The group
homomorphism $f^{*}_{\mathbb{C}P^{m}}:\theta_{2m}\to
\mathcal{C}(\mathbb{C}P^{m})$ factors through
$f^{*}_{M}:\theta_{2m}\to \mathcal{C}(M)$.
\end{coro}

The posited factor homomorphism $\eta:\mathcal{C}(M)\to
\mathcal{C}(\mathbb{C}P^{m})$ is constructed geometrically as
follows. Note first that \ref{c18:lem18.2} implies $M\times
\mathbb{D}^{2m+1}$ embeds as a codimension-0 submanifold in
$\Int(\mathbb{C}P^{m}\times \mathbb{D}^{2m+1})$. This embedding
determines a ``dual map''
$$
\phi:\Sigma^{2m+1}\mathbb{C}P^{m}\to \Sigma^{2m+1}M
$$
satisfying $\Sigma^{2m+1}(f_{\mathbb{C}P^{m}})$ and the composite
$\Sigma^{2m+1}(f_{M})\circ \phi$ are homotopic. The homomorphism
$\eta$ is then induced by $\phi$ via the construction given at the
end of the proof of \ref{c16:lem16.3}. See \cite[p. 70]{46} for details.

Recall that \ref{c17:thm17.2} yields a closed complex $m$-dimensional
hyperbolic manifold $\mathcal{M}$ in every $\mathbb{C}$-dimension
$m$. Hence \ref{c18:coro18.3} reduces the problem of satisfying
property 3 of \ref{c18:thm18.1} to that of showing the homomorphism
$$
f^{*}_{\mathbb{C}P^{4n+1}}:\theta_{8n+2}\to
\mathcal{C}(\mathbb{C}P^{4n+1})
$$
is non-zero for every $n\in \mathbb{Z}^{+}$. This is a classical
(albeit hard) type of algebraic topology problem. Results of Adams
\cite{1}, Adams and Walker \cite{2}, and Brumfiel \cite{15} are used
in \cite{46} to give a positive solution to it; i.e.,
$f^{*}_{\mathbb{C}P^{4n+1}}$ is {\em never} the zero homomorphism. See
again \cite{46} for details.


