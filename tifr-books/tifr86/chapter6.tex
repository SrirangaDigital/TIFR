\chapter{Condition $(\ast)$}\label{c6}

Recall\pageoriginale Alexander's Trick; namely, Theorem \ref{c1:thm1.6} from
Lecture \ref{c1}. This fundamental result is quite elementary. It
has in fact the following one line proof.
 
\begin{proof}
  Set $\ob{h} (tx)= th (x)$ where $x \in S^n$ and $t \in [0, 1]$.
\end{proof}

\begin{add}\label{c6:add6.1}
  Even when $h:S^n \to S^n$ is not a homeomorphism, but only a
  continuous map, the above map $\ob{h} :
  \mathbb{D}^{n+1}\to\mathbb{D}^{n+1}$ is a continuous extension of $h$.
\end{add}

\begin{remark*}
  Note that $\ob{h}$ is rarely differentiable at 0 even if $h$
  is. Precisely stated, $\ob{h}$ is differentiable if and only if $h
  \in O (n+1)$; i.e., $h$ is the restriction of an orthogonal linear
  transformation. There can, in fact, be no smooth analogue of
  Alexander's Trick as Milnor \cite{72} proved by constructing
  examples of exotic 7-dimensional spheres.
\end{remark*}

Let us now use Alexander's Trick together with the topological version
of Smale's $h$-cobordism theorem \cite{90}, due to Kirby-Siebenmann
\cite{67}, to calculate $S(\mathbb{D}^n, \partial)$ when $n \geq 5$. 

\begin{thm}\label{c6:thm6.2}
  When $n \geq 5$, $|\mathcal{S}(\mathbb{D}^n, \partial)|=1$.
\end{thm}

\begin{proof}
  Let $f: (N, \partial N) \to (\mathbb{D}^n, S^{n-1})$ represent an
  element in $\mathcal{S}(\mathbb{D}^n, \partial)$. Then, by definition,
  $f|_{\partial N} : \partial N \to S^{n-1}|$ is a homeomorphism. We
  must show that $f$ is homotopic rel $\partial$ to a
  momeomorphism. Let $\mathbb{B}^n$ be a (locally flatly) embedded
  closed $n$-ball in the interior of $N$. Then, $N-Int \mathbb{B}^n$
  is an $h$-cobordism between $\partial \mathbb{B}^n$ and $\partial
  N$, and hence $N- Int \mathbb{B}^n$ is homeomorphic to
  $S^{n-1}\times [0, 1]$ by the $h$-cobordism theorem. Using
  Alexander's Trick, we see that $N = (N- Int \mathbb{B}^n) \cup
  \mathbb{B}^n$ is homeomorphic to $\mathbb{D}^n$; i.e., $(N, \partial
  N)= (\mathbb{D}^n, S^{n-1})$. Let $\phi = f|_{S^{n-1}}$ and
  $\ob{\phi}: \mathbb{D}^n \to \mathbb{D}^n$ be the homeomorphism
  extending $\phi$ given by Alexander's Trick. Let $\psi = f \cup
  \ob{\phi}: S^n \to S^n$ denote the continuous map which is $f$ on
  the northern hemisphere and $\ob{\phi}$ on the southern
  hemisphere. Then its continuous extension $\ob{\psi}$, given by
  Addendum \ref{c6:add6.1}, can be used to construct the desired
  homotopy rel $\partial$ between $f$ and $\ob{\phi}$.
\end{proof}

\setcounter{remark}{0}
\begin{remark}
  We will use Theorem \ref{c6:thm6.2}, the surgery exact sequence and
  the Cartan-Hadamard theorem to show, in our next lecture, that the
  assembly map $\sigma$ in the surgery exact\pageoriginale sequence is a split
  monomorphism when $M$ is a closed non-positively curved Riemannian
  manifold and $n \geq 1$.
\end{remark}

\begin{remark}
  Set $M= \mathbb{D}^5$ in the surgery exact sequence and observe that
  $M \times \mathbb{D}^n = \mathbb{D}^{n+ 5}$. Since $Wh (\pi_1 M)= Wh
  (1) =0$, $\ob{S}(M \times \mathbb{D}^n, \partial) = S (M \times
  \mathbb{D}^n, \partial)$. These observations combined with Theorem
  \ref{c6:thm6.2} yield that the surgery map
  $$
  \sigma : \pi_{n + 5} (G/\ttop) \to L_{n+5}(1)
  $$
  is an isomorphism for all $n > 0$. Now Kervaire-Milnor \cite{65}
  calculated $L_m (1)$; namely,
  $$
  L_m (1)= \begin{cases}
    \mathbb{Z} & \text{if}~ m \equiv 0 \mod 4\\
    \mathbb{Z}_2 & \text{if}~ m \equiv 2  \mod 4\\
    0 & \text{if} ~ m ~\text{is odd}.
  \end{cases}
  $$
\end{remark}

This yields the following calculation of $\pi_m (G/\ttop)$ valid
for $m \geq 5$:
$$
\pi_m (G/ \ttop)= 
\begin{cases}
    \mathbb{Z} & \text{if}~ m \equiv 0 \mod 4\\
    \mathbb{Z}_2 & \text{if}~ m \equiv 2  \mod 4\\
    0 & \text{if} ~ m ~\text{is odd}.  
\end{cases}
$$

The same result is shown to hold for $m < 5$ by special
arguments. This is the method used by Sullivan in \cite{92} to
calculate $\pi_m (G/PL)$ and was the first step in his later
determination of $G/PL$ in \cite{93}; cf. Historical Remarks in
Lecture \ref{c5}.

I now formulate a useful abstract property which is possessed by many
aspherical manifolds M. This property was introduced in \cite{30}. It
will be shown in the next lecture how this property relates to the
question of the split injectivity of the assembly map in the surgery
exact sequence for $M$.

\begin{defi}\label{c6:defi6.3}
  A closed manifold $M^{m}$ satisfies \textit{condition} $(*)$
  provided there exists an action of $\pi_1 M^m$ on $\mathbb{D}^m$
  with the following two porperties.
  \begin{enumerate}
    \item The\pageoriginale restriction of this action to Int $(\mathbb{D}^m)$ is
      equivalent via a $(\pi_1 M)$-equivariant homeomorphism to the
      action of $\pi_1 M$ by deck transformations on the universal
      cover $\tilde{M}$ of $M^m$.
      \item Given any compact subset $K$ of Int$(\mathbb{D}^m)$ and
        any $\epsilon > 0$, there exists a real number $\delta> 0$
        such that the following is true for every $\gamma \in \pi_1
        M$. If the distance between $\gamma K$ and
        $S^{m-1}=\mathbb{D}^m$ is less than $\delta$, then the
        diameter of $\gamma K$ is less than $\epsilon$.
  \end{enumerate}
  Note that any manifold satisfying condition $(*)$ is obviously aspherical.
\end{defi}

\begin{remark*}
  Every closed (connected) non-positively curved Riemannian manifold
  $M$ satisfies condition $(*)$. This was shown in \cite{30} by
  considering the geodesic ray compactification $\ob{M}$ of
  $\tilde{M}$ defined by Eberlein and O'Neill \cite{26}. The
  compactification $\ob{M}$ of $\tilde{M}$ is homeomorphic to
  $\mathbb{D}^m$. The verification of property 2 in condition $(*)$
  uses the well known fact that $exp_x : T_x \tilde{M} \to \tilde{M}$
  is weakly expanding; cf. \cite[p. 172, Lemma 1]{59}.
\end{remark*}

