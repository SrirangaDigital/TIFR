\chapter{Generalized Borel Conjecture}\label{c2}

\section{Structure Sets}\label{c2:sec2.1}

Let\pageoriginale $M$ be a closed manifold. We define $S(M)$ to be the set of
equivalence classes of pairs $(N, F)$ where $N$ is a closed manifold
and $f:N \to M$ is a homotopy equivalence. And the equivalence
relation is defined as follows:
\begin{center}
  $(N_1 , f_1) \sim (N_2, f_2)$ if there is a homeomorphism $h: N_1
  \to N_2$
\end{center}
such that $f_2 \circ h$ is homotopic to $f_1$.

The above definition is motivated by the definition of Teichm\"uller
space $T_g$. In this classical situation, $M = M_g$ is a closed
Riemann surface of genus $g \geq 2$. And $T_g$ is the set of
equivalence clases of pairs $(N, f)$ whre $N$ is a Riemann surface of
constant curvature $-1$ and $f:N \to M$ is homotopy equivalence. The
equivalence relation is the same as in $S(M)$ except that
homeomorphism is replaced by isometry.

For an aspherical finite polyhedron $X$, Hurewicz identified Out
$(\pi_1 X)$ with $\pi_0 (\epsilon (X))$, where Out $(\pi_1, X)$ is the
group of outer automorphisms of $\pi_1 X$ and $\epsilon (X)$ is the
space of self homotopy equivalences of $X$ given the compact open
topology. Using this identification, we get a natural action of Out
$(\pi_1 M_g)$ on $T_g$ in the following way. If $\alpha \in Out (\pi_1
M_g)$ and $[N, f]\in T_g$, then $\alpha[N, f]= [N, \tilde{\alpha}
  \circ f]$ where $\tilde{\alpha} : M_g \to M_g$ is any self homotopy
quivalence inducing $\alpha$. Moreover, Out $(\pi_1 M_g)$ contains a
subgroup $\Gamma$ of finite index such that $T_g/\Gamma$ is an
aspherical manifold. Recall that $T_g$ is diffeomorphic to
$\mathbb{R}^{6 g-6}$. The manifolds $T_g/\Gamma$ are not compact; but
can be compactified, cf. \cite{56}, \cite{55}. These manifolds form
another interesting class of aspherical manifolds apparently different
from the other classes of examples mentioned in Lecture 1,
cf. \cite{55}.

\begin{remark*}
  When $M$ is aspherical, Out $(\pi_1 M)$ also acts on $S(M)$. The
  statement that $|S(M)|=1$ is equivalent to Borel's conjecture. While
  the statement that Out $(\pi_1 M)$ acts transitively on $S(M)$ is
  equivalent to the weaker statement that any closed aspherical
  manifold $N$ with $\pi_1 N \simeq\pi_1 M$ is homeomorphic to $M$.
\end{remark*}

\section{Variants of Structure Sets}\label{c2:sec2.2}

\begin{enumerate}[(1)]
\item The\pageoriginale smooth structure set is denoted by $S^s(M)$. It consists of
  equivalence classes of objects $(N, f)$ where now $N$ is a smooth
  manifold, $f: N\to M$ is a homotopy equivalence and $h: N_1 \to N_2$
  is required to be a diffeomorphism. Note that we do not require that
  $M$ is a smooth manifold, hence $S^s(M)$ could be empty. Likewise,
  the definitions of $S(M)$ and $S^s (M)$ make sense even if $M$ is
  not a manifold. But for them to be possibly non-empty $M$ must have
  the algebraic properties of a manifold; e.g., it must satisfy
  Poincare duality with arbitrary local coefficients. If this is so,
  then $M$ is called a \textit{Poincare duality space} and an
  interesting question is whether $S(M)$ or $S^s(M)$ is non-empty;
  i.e., does there exist a topological or smooth manifold $N$ which is
  homotopically  equivalent to $M$.

  \item It is also useful to define $S(M, \partial M)$ where $M$ is a
    compact manifold with boundary $\partial M$. Here, an object is
    again a pair $(N, f)$ where $N$ is a compact manifold with
    boundary $\partial N$ and $f: (N, \partial N) \to (M, \partial M)$
    is a homotopy equivalence such that the restricted map
    $f|_{\partial N}: \partial N \to \partial M$ is a
    homeomorphism. And $(N_1, f_1)\sim (N_2, f_2)$ if there is a
    homeomorphism $h: (N_1, \partial N_1) \to (N_2, \partial N_2)$
    such that $f_2\circ H$  is homotopic to $f_1$ rel $\partial N_1$;
    i.e.,  the homotopy between $f_2 \circ h$ and $f_1$ is constant on
    $\partial N_1$.

  \item The smooth structure set $S^s (M, \partial M)$ is defined
    similarly. 
\end{enumerate}

Borel's conecture can be generalized to the following statement.

\section{Generalized Borel Conjecture.}\label{c2:sec2.3}

If $M$ is a compact aspherical manifold with perhaps non-empty
boundary, then $|S (M, \partial M)|=1$.

\section{Examples of compact aspherical manifolds with
  boundary.}\label{c2:sec2.4} 

\begin{enumerate}[(1)]
\item Let $G$ be a semi-simple Lie group and $K$ a maximal compact
  subgroup. Suppose that $\Gamma$ is a torsion free arithmetic
  subgroup of $G$ for some algebraic group structure on $G$ defined
  over $\mathbb{Q}$. Then M.S. Raghunathan \cite{85} showed that the
  double coset space $M= \Gamma\backslash G/K$ is the interior of a
  compact manifold with boundary. Note that this shows that $\Gamma$
  is finitely generated and, infact, of the type $FL$ in the sense of
  Serre \cite{88}. Later Borel and Serre \cite{10} gave a second
  compactification $\ob{M}$ of $M$. In their compactification the\pageoriginale
  boundary of the universal cover of $\ob{M}$ is homotopically
  equivalent to a wedge of spheres $S^{m}$ where $m$ equals
  $\mathbb{Q}$ rank $(G)-1$. This has some useful consequences about
  the group cohomology of $\Gamma$. Namely, they deduce from this that
  the cohomological dimension of $\Gamma$ is $\dim (G/K)-\mathbb{Q}$
  rank $(G)$ and that $\Gamma$ is a duality group in the sense of
  Bieri and Eckmann \cite{7}. It is a consequence of the vanishing
  results on Whitehead torison (to be discussed later) that the two
  compactifications are diffeomorphic provided $\dim (G/K) \neq 3. 4,
  5$.
\item The compactification of $T_g/\Gamma$ due to Harvey \cite{56},
  mentioned earlier, is another example. Again, the boundary of its
  universal cover was shown by Harer \cite{55} to be homotopically
  equivalent to a wedge of spheres. Harer showed, in this way, that
  $\Gamma$ is a duality group and he also calculated its cohomological
  dimension. 
\end{enumerate}

A fundamental problem in topology is to calculate $|S(M, \partial
M)|$; i.e., the cardinality of the set $S(M, \partial M)$. Surgery
theory was developed to solve this problem. It essentially reduced the
problem to calculating certain algebraically defined obstruction
groups which are functors depending only on $\pi_1 M$ (when $M$ is
orientable). In particular, showing that
$$
|S (M \times \mathbb{D}^n, \partial (M \times \mathbb{D}^n))|=1
$$
for all sufficiently large integers $n$, yields a calculation of the
obstruction groups for $\pi_1 M$ when $M$ is aspherical. Hence the
verification of the generalized Borel conjecture would make surgery
theory and effective method for calculating $|S (N, \partial N)|$ for
any compact connected manifold (not necessary aspherical) with $\pi_1
N$ isomorphic to $\pi_1 M$, provided $\dim N \geq 5$.

We will refer to the generalized Borel conjecture as Borel's
conjecture because of a corollary to the following result of M. Davis
\cite{22}.

\setcounter{lemma}{4}
\begin{thm}[Davis]\label{c2:thm2.5}
  If $K$ is a finite aspherical polyhedron, then there exists a closed
  aspherical manifold $M$ such that $K$ is a retract of $M$.
\end{thm}

\begin{coro}\label{c2:coro2.6}
  If\pageoriginale $|S(M)|=1$ for every closed aspherical manifold
  $M$, then $|S (N, \partial N)|=1$ for every compact aspherical
  manifold $N$ with boundary $\partial N$.
\end{coro}

\begin{remark*}
  This theorem is implicit in Davis' paper \cite{22} and is made
  explicit by Bizhong Hu in \cite{63}. Hu shows that if $K$ is
  non-positively curved in the sense of Alexandroff, then the manifold
  $M$ of Davis' theorem can be constructed to also be non-positively
  curved in the sense of Alexandroff. If we consider a functor from
  topological spaces to groups, for instance the Whitehead group
  functor $X \to Wh (\pi_1 X)$, then the followng result is an
  immediate consequence of Davis' theorem. If such a functor vanishes
  on all closed aspherical manifolds, then it must also vanish on all
  finite aspherical complexes. (Hu uses this fact in his work on
  Whitehead groups.) An elaboration of this idea can be used to verify
  the above Corollary \ref{c2:coro2.6}. We will discuss this in a
  later lecture.
\end{remark*}
