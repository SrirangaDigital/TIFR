
\chapter{Arithmetic of Quaternion Algebras}\label{chap1}

\pagenumbering{arabic}
\setcounter{page}{1}

\section{Algebraic Background}\label{chap1:sec1}

\textbf{1.}~ Let\pageoriginale $k$ be a field of any characteristic (not necessarily
zero). Let $Q$ be an algebra over $k$, generated by the elements $[1,
  \omega, \Omega, \omega \Omega]$ with the following multiplication
table: 
$$
\omega^2 = p \in k, \Omega^2 = q \in k, \omega \Omega + \Omega  \omega =0. 
$$

$Q$ is called a \textit{ quaternion algebra over $k$.} Then, any
element $\alpha \in \Omega$ can be expressed uniquely as  
$$
\alpha = a_o + a_1 \omega + a_2 \Omega + a_3 \omega \Omega
$$
where $a_i \in k$. We define the \textit{trace } and \textit{norm} of
$\alpha$ as  
\begin{align*}
  \text{ trace } (\alpha)= & t(\alpha)= 2 a_\circ = \alpha +
  \bar{\alpha} \in k\\ 
  \text{ norm }(\alpha)& = n(\alpha) = a^2_\circ - p a^2_1 - q a^2_2
  + pq a^3_2\\ 
  & = \alpha \bar{\alpha} = \bar{\alpha}\alpha \in k,
\end{align*}
where $\bar{\alpha} =a_o -a_1 \omega -a_2 \Omega - a_3 \omega \Omega$
is called the \textit{conjugate of } $\alpha$. If $\alpha \in Q$, then
$\alpha^2 - (\alpha + \bar{\alpha}) \alpha +
\bar{\alpha}\alpha=0,i.e., \alpha^2 - t(\alpha)$.  $\alpha+n(\alpha)
=0$; in other words, each element $\alpha$ of $Q$ satisfies a
quadratic equation over $k$.
 
\begin{theorem}\label{chap1:sec1:thm1} % them 1
  {\em $k$ is the centre of $Q$}.
\end{theorem}

\begin{proof}
  If $C$ is the center of $Q$ by definition $k \subset C$. Conversely
  if $\alpha \in C$, then $\alpha \beta = \beta \alpha$ for all $\beta
  \in Q$. 
\end{proof}
\begin{tabbing}
  \qquad Let \qquad \= $\alpha$ \= $= a_o + a_1 \omega + a_2 \Omega + a_3 \omega
  \Omega$ \hspace{1cm}and\\  
  \>$\beta$ \> $ = b_o + b_1 \omega + b_2 \Omega + b_3 \omega \Omega$.
\end{tabbing}

Then\pageoriginale
\begin{multline*}
  \alpha \beta - \beta \alpha = 0 \Longrightarrow (2q(a_3b_2-a_2 b_3)
  + \Omega (2p(a_1b_3-a_3b_1))\\  
  \omega \Omega (2(a_1 b_2 -a_2 b_1))=0, 
\end{multline*} 

i.e., \quad $a_3 b_2 -a_2 b_1  = 0, a_1b_3 -a_3 b_1 =0, a_1 b_2  = 0$, 

i.e., \quad $a_1 : a_2 : a_3 = b_1 : b_2 : b_3$.

The left hand side being fixed and the right hand side being
arbitrary, this implies that $a_1 = a_2 = a_3 = 0, $i.e. $\alpha = a_o
\in k$, i.e., $C \subset k$. 

Using this, we shall show incidentally that our definition of a
quaternion algebra is fairly general. More explicitly, any element
$\omega' ( \in Q,  \notin k)$ of trace zero and non-zero norm may be
taken as the first element of a basis, $[1, \omega', \Omega', \omega'
  \Omega']$ which we shall  construct as follows: 

Now,
$$
\omega'^2 = -n (\omega') = p' \in k.
$$

Let, further more $\omega''$ be an element linearly independent of $1,
\omega'$ and which does not commute with $\omega'$. Such $\omega''$
always exists, or else every element of $Q$ would commute with
$\omega'$ so that $\omega'$ would belong to the centre of $Q$,i.e.,
$\omega' \in k$(by Theorem \ref{chap1:sec1:thm1}) which is contradictory to assumption. 

Define $\Omega'=\omega' \omega''- \omega'' \omega'( \neq 0$, by the
choice of $\omega'')$. 

Then 
\begin{align*}
  \omega' \Omega' & = \omega'(\omega' \omega'' - \omega'' \omega')\\
  & = \omega'' \omega'^2 - \omega' \omega'' \omega' \\
  & = (\omega'' \omega'-\omega' \omega'')\omega' \qquad \text{ since }
  \omega'^2 \in k\\ 
  & =- \Omega' \omega'  \qquad i.e. \omega' \Omega' + \Omega' \omega' = 0.
\end{align*}

Further\pageoriginale $t (\Omega') = 0$; otherwise $\Omega'^2 = t(\Omega')$. $\Omega'
- n(\Omega')$ implies that $\omega'$ commutes with $\Omega'$ which is
not true. 

Therefore $t(\Omega')=0$, i.e., $\Omega'^2 = -n (\Omega') =q' \in k$. 

Summing up, we have obtained a set $[ 1, \omega', \Omega', \omega'
  \Omega']$ such that  

(i) $\omega'^2 = p' \in k$,  \quad (ii) $\Omega'^2 = q'
\in k$,  \quad (iii) $\omega' \Omega' + \Omega' \omega' =0$. 
In other words, $[ 1, \omega', \Omega', \omega' \Omega']$ is a basis
for $Q$ over $k$. 
\begin{theorem}\label{chap1:sec1:thm2} % them 2
  If $Q$ has divisors of zero, then $Q \cong \mathfrak{M}_2 (k)$ (total matrix
  algebra of order $2$ over $k$). 
\end{theorem}

\begin{proof}
  We will find four elements $\varepsilon_1, \varepsilon_2,
  \eta_1,\eta_2 \in Q$ satisfying the following condition:  
  \begin{enumerate}[i)]
  \item $\varepsilon^2_1 = \varepsilon_1, \varepsilon^2_2= \varepsilon_2,
    \varepsilon_1
    \varepsilon_2= \varepsilon_2 \varepsilon_1, \varepsilon_2 =
    1-\varepsilon_1$ 
  \item $\eta_1 \eta_2 = \varepsilon_1,\eta_2 \eta_1 = \varepsilon_2 ;
    \varepsilon_1 \eta_1 =\eta_1 =\eta_1 \varepsilon_2, \eta^2_1=0$. 
    $$
    \varepsilon_2 \eta_2= \eta_2 =\eta_2 \varepsilon_1 ; \eta^2_2 = 0.
    $$
  \end{enumerate}

  Then we can define the mapping 
  $$
  \displaylines{\hfill \sigma : \mathfrak{M}_2 (k) \to Q\hfill \cr
    \text{as}\hfill  
    \sigma (e_i) = \varepsilon_i ; \sigma(f_i) = \eta_i; \sigma(e) =
    1,\hfill \cr
    \text{where} \hfill\cr 
    e_1= 
    \begin{pmatrix}1&0\\ 0&0\end{pmatrix} , ~e_2 = 
      \begin{pmatrix}0&0\\ 0&1\end{pmatrix},~f_1=
        \begin{pmatrix}0&1\\ 0&0\end{pmatrix} , ~f_2=
          \begin{pmatrix}0&0\\ 1&0\end{pmatrix} ~\text{ and }~ e= 
            \begin{pmatrix}1&0\\ 0&1\end{pmatrix}. \hfill } 
    $$
  $\sigma$\pageoriginale is easily seen to be an isomorphism onto.
\end{proof}

If $\alpha (\neq 0) \in Q$ is a zero divisor, $\alpha y$ is a zero
divisor for all $y \in Q$, so that $n(\alpha y) =0$, for all $y \in
Q$. Then there exists at least one $y \in Q$ for which $t( \alpha y)
\neq 0$, for otherwise, $t(\alpha y) =0$ for all $y \in Q$ and hence
for $y =1, \omega, \Omega, \omega \Omega$, would imply that $\alpha
=0$. 

Let $\alpha y = \alpha'$ then $t(\alpha'')=1$ where $\alpha''
\dfrac{\alpha'}{t(\alpha')}(t (\alpha') \neq 0)$. Putting $\alpha'' =
\varepsilon_1$ and $1-\varepsilon_1 = \varepsilon_2$, we have the
equations $(i)$. Consider $\omega' = \varepsilon_1 - \varepsilon_2 (
\in Q)$. Then $\omega'^2 = \varepsilon_1 + \varepsilon_2=1$, so that
$t(\omega')=0$ and $n(\omega') \neq 0$. Hence, as seen before, we have
a basis $[1, \omega', \Omega', \omega' \Omega']$ for $Q$ over
$k$. Then $t(\Omega')=0$ while $n(\Omega')\neq 0$, so that $\Omega'$
has an inverse. 

Define $\eta_1 = \dfrac{\varepsilon_1 \Omega'}{\Omega'^2}$ and $
\eta_2 =\varepsilon_2 \Omega'= \Omega'. \varepsilon_1$ (since $\Omega'
\omega' + \omega' \Omega' =0$). We can easily verify the set of
equations $(ii)$, for example, 
 $$
 \eta_1 \eta_2 = \frac{\varepsilon_1 \Omega'}{\Omega'^2} \Omega'
 \varepsilon_1 = \frac{\varepsilon_1 \Omega'^2 \varepsilon_1}{\Omega'^2}
 = \varepsilon^2_1 = \varepsilon_1, 
 $$
 and similarly others.

 \textbf{2.}~ $Q$\pageoriginale is said to split over $K$ or $K$ is said to be a \textit{
   splitting field } of $Q$ if $QK / K$ is isomorphic to $\mathfrak{M}_2(K). (QK$
 denoting the tensor product of $Q$ and $K$). 

\begin{theorem}\label{chap1:sec1:thm3} % them 3
  Let $Q$ be a quaternion algebra over a field $k$ of characteristic
  zero and let $K/k$ be a quadratic extension of $k$. Then $Q K/K$
  splits if and only if $K \cong \bar{K} \subset Q (K \neq \bar{K})$. 
\end{theorem}

\begin{proof}
  We may, without loss of generality, assume that $Q$ is s-field (skew
  field) since otherwise $k$ is itself a splitting field. 
\end{proof}

\begin{enumerate}[1)]
\item Let $K = k (a) \cong k(\alpha) = \bar{K} \subset Q(\alpha =
  \sigma a \in Q) $ (say)). 

  We will now show that $QK$ contains divisors and hence that $K$ splits $QK$.
  
  Since $\alpha^2 -t(\alpha).  \alpha + n(\alpha) =0, a^2-t(\alpha).a+
  n(\alpha) =0$. In other words, $X^2 - -t(\alpha). X+n(\alpha) = (X-a)
  (X-\bar{a})$ where $\bar{a} = \sigma \bar{\alpha}$. Therefore $0=
  \alpha^2 -t(\alpha).  \alpha + n(\alpha) = (\alpha -a) (\alpha -
  \bar{a})$, the factorization holding in $QK$. Both the factors cannot
  vanish, for otherwise, $\alpha \in K$, i.e., $\bar{K}= K$ which not
  so. Hence $QK$ contains divisors of zero and by
  Theorem \ref{chap1:sec1:thm2}, $QK/ K
  \cong \mathfrak{M}_2(K)$. 
\item Let $QK/K$ split, where $K= k(a),  a = \sqrt{d}$. We shall prove
  that there exists $\delta \in Q$ such that $\delta^2 =d$. Then
  $k(\delta) = \bar{K} \cong k (\sqrt{d})=K$. By hypothesis, there an
  $\varepsilon \in Q K$ such that $\varepsilon$ corresponds to
  $ \binom{0~ 1}{0~ 0}$ (in $m_2 (K)$). Hence
  $t(\varepsilon)=0$ and $n(\varepsilon) = 0$. If $Q=k[1, \omega,
    \Omega,  \omega \Omega]$, then $Q K= K[1, \omega, \Omega,  \omega
    \Omega]$, so that $\varepsilon = (a_o + \sqrt{d} b_o) + (a_1 +
  \sqrt{d} b_1) \omega + (a_2 + \sqrt{d} b_2) \Omega + (a_3 + \sqrt{d}
  b_3) \omega \Omega$ with\pageoriginale $a_i,b_i \in k, i.e., \varepsilon = \alpha
  + \sqrt{d} \beta ; \alpha, \beta \in Q$.  $(\alpha, \beta \neq
  0)$. Now $t(\varepsilon) = t(\alpha) + \sqrt{d} t(\beta)
  =0. t(\alpha),   t(\beta) \in k$. Hence $t(\alpha)=0= t(\beta)$,
  since $1$, $\sqrt{d}$ are linearly independent over $k$.  
  
  Further
  \begin{align*}
    n(\varepsilon) & = \left(\alpha + \sqrt{d} \beta\right) \left(\bar{\alpha}+
    \sqrt{d} \bar{\beta}\right)\\ 
    &= (n (\alpha) + dn(\beta)) + \sqrt{d} \left(\beta \bar{\alpha} +
    \alpha \bar{\beta}\right)\\ 
    & = 0.
  \end{align*}

  As before $n(\alpha) + d n(\beta) = 0, \beta \bar{\alpha} + \alpha
  \bar{\beta} =0$. $n(\alpha), n(\beta) \neq 0$, since $\alpha$ and
  $\beta$ are not zero divisors). Putting $\delta = \alpha
  \beta^{-1}$, we have  
  \begin{align*}
    \delta^2 & = (\alpha \beta^{-1})^2 + \alpha \beta^{-1}. \alpha \beta^{-1}\\
    & = \frac{\alpha \cdot \bar{\beta}}{\eta (\beta)}\cdot
    \frac{\alpha \cdot \bar{\beta}}{\eta (\beta)}\\ 
    & = \frac{- \beta \bar{\alpha}}{\eta(\beta)}\cdot \frac{\alpha
      \bar{\beta}}{\eta(\beta)}\\ 
    & = \frac{-\eta(\alpha)}{\eta(\beta)} = d.
  \end{align*}
  Hence there exists a $\delta \in Q$ such that $\delta^2
  =d$. Theorem \ref{chap1:sec1:thm3} is thus completely proved. 
\end{enumerate}

\textbf{3.}~ We now state four theorems which we will need in the
  sequel. For the same, we shall introduce some notations. $k^o$
  denotes the rational number field; $k$ an algebraic number field and
  $\bar{k}_{\mathscr{Y}}$ its completion with respect to a
  $\mathscr{Y}$- adic valuation. (We include the case of extensions of
  archimedian valuations also). If $Q/k$ is a quaternion algebra,
  $Q_{\mathscr{Y}}= Q.\bar{k}_{\mathscr{Y}}$, the tensor product of
  $Q$ and $\bar{k}_{\mathscr{Y}}$ and similarly $K_{\mathscr{Y}} = K
  \bar{k}_{\mathscr{Y}}$ where $K$ is a quadratic extension of $k$. 

\begin{theorem}\label{chap1:sec1:thm4} % them 4
  \begin{enumerate}[\rm a)]
  \item  $Q/ \cong \mathfrak{M}_2 (k) \Longleftrightarrow Q_{\mathscr{Y}}
    \sqrt{k_{\mathscr{Y}}} \cong \mathfrak{M}_2
    (\bar{k_{\mathscr{Y}}})$\pageoriginale for every $\mathscr{Y}$. 
  \item For every $Q /k$, there exist only a finite number of
    primes $\mathscr{Y}$ such that $Q\not{\cong} m_2
    (\bar{k}_{\mathscr{Y}})$.  

    (These exceptional primes are called ``characteristic primes").
  \item If $L_{\mathscr{Y}}$ is any quadratic extension of
    $\bar{k}_{\mathscr{Y}}$ then $L_{\mathscr{Y}}$ is a splitting field
    for $Q_{\mathscr{Y}}$. 
  \item $K$ is a splitting field for $Q \Longleftrightarrow
    K_{\mathscr{Y}}$ a splitting field for $Q_{\mathscr{Y}}$ for all
    characteristic primes. 
  \end{enumerate}
\end{theorem}

We shall not prove theorems \ref{chap1:sec1:thm4}a, \ref{chap1:sec1:thm4}b, and \ref{chap1:sec1:thm4}c, but we will show how
Theorem  \ref{chap1:sec1:thm4}d follows as a simple consequence of
Theorem \ref{chap1:sec1:thm4}a.  (For 
proofs of Theorem \ref{chap1:sec1:thm4}a, Theorem
\ref{chap1:sec1:thm4}c, refer[Deuring, \textit{ 
``Algebren''} page 117, page 113]) 

\begin{enumerate}[(i)]
\item Assume that $K$ is a splitting field for $Q$.

  Then, by theorem \ref{chap1:sec1:thm4}a, the completion $\bar{K}_{\mathscr{Y}}$ is a
  splitting field for$Q_{\mathscr{Y}}$ for every $\mathscr{Y}$. 

  $\alpha)$ ~ If $K_{\mathscr{Y}}$ is a field, then the extension
  $\mathscr{Y}$ of the valuation $\mathscr{Y}$ from $k$ to $K$ is
  unique, so that $\bar{K}_{\mathscr{Y}} =K_{\mathscr{Y}}$ and hence
  $K_{\mathscr{Y}}$ is a splitting field for $Q_{\mathscr{Y} }= Q\cdot \bar{K}_{\mathscr{Y}}=
  Q\cdot K_{\mathscr{Y}} = Q_{\mathscr{Y}}$. 

  $\beta)$ If $K_{\mathscr{Y}}$ is not a field, then $K_{\mathscr{Y}}
  \bar{K}_{\mathscr{Y}_1}+ \bar{K}_{\mathscr{Y}_2}$ where
  $\bar{K}_{\mathscr{Y}_1}$ and $\bar{K}_{\mathscr{Y}_2}$ are
  completions of $K$ with respect to the extended valuations
  $\mathscr{Y}_1$ and $\mathscr{Y}_2$ of $\mathscr{Y}$ from $k$ to
  $K$. Then  $K_{\mathscr{Y}}$ being of rank $2$ over
  $\bar{k}_{\mathscr{Y}}, \bar{K}_{\mathscr{Y}_1}$ and
  $\bar{K}_{\mathscr{Y}_2}$ are of rank $1$ each so that
  $\bar{K}_{\mathscr{Y}_1} \cong k_{\mathscr{Y}} \cong
  \bar{K}_{\mathscr{Y}_2}$. 

  Hence\pageoriginale $Q_{\mathscr{Y}_1} \cong Q_{\mathscr{Y}} \cong
  Q_{\mathscr{Y}_2}$. But we know that$\bar{K}_{\mathscr{Y}^1}$ and
  $\bar{K}_{\mathscr{Y}_2}$ are splitting fields of
  $Q_{\mathscr{Y}_1}$ and $Q_{\mathscr{Y}_2}$ respectively, i.e.,
  $k_{\mathscr{Y}}$ is a splitting field of $Q_{\mathscr{Y}}$,
  i.e. $\mathscr{Y}$ is not a characteristic prime. 

  Hence we have the fact that if $\mathscr{Y}$ is a characteristic
  prime $K_{\mathscr{Y}}$ is a field and by $(\alpha)$, a splitting
  field for $Q$. 
\item Assume that $K_{\mathscr{Y}}$ is a splitting field for
  $Q_{\mathscr{Y}}$ for all characteristic primes $\mathscr{Y}$ 
  
  $\alpha)$ ~If $\mathscr{Y}$ a characteristic prime, then
  $K_{\mathscr{Y}}$ is a splitting field for $Q_{\mathscr{Y}}$ and
  hence a priori, a field. Therefore the extension $\mathscr{Y}$ of
  $\mathscr{Y}$ from $k$ to $K$ is unique and $\bar{K}_{\mathscr{Y}} =
  K_{\mathscr{Y}}$ so that $Q_{\mathscr{Y}} = Q_{\mathscr{Y}}$ splits
  over $\bar{K}_{\mathscr{Y}}$. 
  
  $\beta)$ If $\mathscr{Y}$ is not a characteristic prime, then
  $\bar{k}_{\mathscr{Y}}$ is a splitting field for $Q_{\mathscr{Y}}$,
  so that (i) if $K_{\mathscr{Y}}$ is a field, $K_{\mathscr{Y}}$ is
  also a splitting field for $Q_{\mathscr{Y}}$, i.e.,
  $Q_{\mathscr{Y}}$ splits over $\bar{K}_{\mathscr{Y}}$ and (ii) if
  $K_{\mathscr{Y}}$ is not a field, $K_{\mathscr{Y}} =
  \bar{K}_{\mathscr{Y_1}} + \bar{K}_{\mathscr{Y}_2}$ where
  $\bar{K}_{\mathscr{Y}_1} \cong K_{\mathscr{Y}} \cong
  \bar{K}_{\mathscr{Y}_2}$, so that $Q_{\mathscr{Y}_1}$ and
  $Q_{\mathscr{Y}_2}$ split over$\bar{K}_{\mathscr{Y}_1}$and
  $\bar{K}_{\mathscr{Y}_2}$ respectively. 

  Therefore in all cases $Q_{\mathscr{Y}}$ splits over
  $\bar{K}_{\mathscr{Y}}$ for every $\mathscr{Y}$ and hence by
  Theorem \ref{chap1:sec1:thm4}a, $K$ is a splitting field over $Q$.  

  We shall now give a sketch of the proof of Theorem
  \ref{chap1:sec1:thm4}b. $(k
  = k^\circ)$. Now, $\alpha \in Q( \alpha \neq 0)$ is a zero
  divisor $\Longleftrightarrow n(\alpha)=0$. Therefore, for proving
  that $Q$ splits over $k,i.e$. for proving the existence of zero
  divisors in $Q$ we merely need to find non-trivial solutions of the
  equation 
  $$
  n(\alpha) = a^2_o -\rho a^2_1 - qa^2_2 + p q a^2_3 = 0.
  $$
\end{enumerate}

If\pageoriginale $\alpha = a_o +a_1 \omega +a_2 \Omega + a_3 \omega \Omega, a_i \in k$.
\begin{enumerate}[i)]
\item If $\rho$ is a square, choosing $a_o =0 =a_1, -a^2_2 + p a^2_3
  =0$ can be solved for $a_2, a_3 \in k$. 
\item If $\rho$ is not a square, let $K =k (\sqrt{\rho})
  \overset{\sigma}{ \cong }k (\omega)$ (since $\omega^2 = \rho)$. 
\end{enumerate}

  Then $n(\alpha) = 0 \Longrightarrow n (\sigma \zeta) -q =0 $ where
  $\zeta = \dfrac{a_o + \sqrt{\rho}a_1}{a_2 + \sqrt{\rho}a_3} \in K$. 
  
  Hence we have the existence of a solution $\zeta \in K$ such that
  $n(\sigma \zeta) -q =0$ implies and is implied by the existence of a
  zero divisor in $Q$. (The necessary part follows from the fact that
  if  
  $$
  \zeta = x_\circ + \sqrt{p}\cdot x_1, x_o + \omega x_1 + \Omega \text{ is
    a zero divisor)}. 
  $$

  We shall now take up the proof of Theorem \ref{chap1:sec1:thm4}b. Let $r$ be a prime
  $\neq 2$ and such that $p,q$ are r-adic units. This is the case for
  almost all $r$. If $\zeta = x_1 + \sqrt{p} \cdot x_2, n(\sigma \zeta)-q =0
  \Longrightarrow x^2_1 - px^2_2 =q$. We have to find $x_1, x_2 $ in
  $\bar{k}^o_r$ satisfying this equation 

\begin{enumerate}[i)]
\item If $p =p^2_1, p_1 \in \bar{k}^o_r, x_1 + p_1 x_2 = 1,  x_1 - p_1
  x_2 =q$ can be solved non-trivially for 
  $\begin{vmatrix} 1 & -p_1 \\ 1 & p_1 \end{vmatrix}  \neq 0$. 
\item If $p$ is not an r-adic square, $q$ can be written as 
  $$
  q =y^2 \text{ or } p y^2,  y \in \bar{k}^o_r.
  $$ 
\end{enumerate}

Then
\begin{enumerate}[a)]
\item $x^2_1 - px^2_2 = y^2$ in which case $x_1 = y, x_2 = 0$ are solutions.
\item $x^2_1 -p x^2_2 = p y^2$; i.e., $1 + \xi^2_2 = p \xi^2_1, \xi_1 =
  \dfrac{x_1}{p_y}, \xi_2 = \dfrac{x_2}{y}$ 
\end{enumerate}

Choosing\pageoriginale $\xi_2 \in k^o$ such that $1 + \xi^2_2$ is a quadratic
non-residue mod $r$ and is an $r$-adic unit (such $\xi_2$ always exists,
or else it would mean $n+ \xi^2_2$ is a residue for all $n$) then
there exists a $\xi_1 \in \bar{k}^{\circ}_r$ such that $1 +
\xi^{2}_2 = p \xi^2_1$.  
Since $r$ runs through almost all primes,
$Q_r$ splits over$\bar{k}^{\circ}_r$ for almost all $r$. 

\textbf{4.}~ We now give two examples of quaternion algebras over the rational
number field $k$, and calculate their characteristic primes. 

1) Let $Q/k$ be the quaternion algebra with basis $(1, \omega,
  \Omega, \omega \Omega)$ such that $\omega^2 = -1, \Omega^2 = -1,
  (\omega \Omega)^2 = -1$. 
 
 To find the characteristic primes $p$, we have only to find those
 primes $p$ for which the equation $n(\xi) = 0, \xi \in Q$ has no
 non-trivial solutions in $\bar{k}_p$. 
 \begin{enumerate}[a)]
 \item $p = \infty $(in the usual notation). Then $\bar{k}_{\infty}$
   is the field of real numbers. Therefore, if  
   $$
   \xi = x_\circ + x_1 \omega + x_2 \Omega + x_3 \omega \Omega, n(\xi) =
   x^2_\circ + x^2_1 + x^2_2 + x^2_3=0 
   $$
   has obviously no non-trivial solutions in real numbers, so that
   $\infty$ is a characteristic prime for $Q$. 
 \item $p = 2$. $\bar{k}_p$ is the field of $2$-adic numbers. Then each
   $x_i, 0 \leq i \leq 3$, has an expansion of the form  
   $$
   x_i = x_{i\circ} 2^{-r} + x_{i1} 2^{-r +1}+ \cdots \cdots
   $$
   so that, multiplying each $x_i$ by a suitable power of $2$, we may
   assume that the new numbers $x'_i$ are all $2-$adic integers of
   which at\pageoriginale least one, (say) $x'_o$ is a $2-$adic unit. 
 \end{enumerate} 

Then
$$
\displaylines{\hfill 
n(\xi) = 0 \Longrightarrow x'^2_\circ + x'^2_1 + x'^2_2 + x'^2_3
=0\hfill \cr
\text{and}\hfill  
x'_\circ = x'_{\circ \circ}+ 2. x_{\circ1} + \ldots,  x'_{\circ\circ}
\neq 0.\hfill }
$$

Let $x''_i = x_{i\circ}+ 2 x_{i1} + x^2 x_{i2}$. Then $n(\xi) = 0$ implies that 
$$
x''^2_\circ + x''^2_1 + x''^2_2 + x''^2_3 \equiv 0 \pmod 8,
$$
where $(x''_\circ,2) =1$ so that $x''^2_\circ \equiv 1 \pmod 8$ and the
other squares $x''^2_1,  x''^2_2,x''^2_3$ can be congruent to $0,1$, or
$4$ mod $8$.  

Hence 
\begin{align*}
  x''^2_o + x''^2_1 + x''^2_2 + x''^2_3 & \equiv 1+ \cdots + \cdots+
  \cdots \pmod8\\ 
  & \not{\equiv} 0 \pmod  8 
\end{align*}  
under the above conditions, so that $n(\xi)=0$ cannot be solvable in
terms of $2$-adic numbers. In other words, $p=2$ is a
charac-teristic prime  

For the proof of Theorem \ref{chap1:sec1:thm4}b, we see that only
$\infty$ and $2$ are characteristic primes for $Q$ over $k$. 


ii)~ Consider the quaternion algebra $Q$ over $k$ (rational number field)
with basis $(1, \omega, \Omega, \omega \Omega) $ such that $\omega^2 =
2, \Omega^2 = -3$ and hence $(\omega \Omega)^2 = 6$. 

Then, if 
\begin{align*}
  \xi & = x_\circ + x_1 \omega + x_2 \Omega + x_3 \omega \Omega\\
  n (\xi) &=  x^2_\circ- 2x^2_1 + 3 x^2_2 - 6 x^2_3.
\end{align*}

\begin{enumerate}[i)]
\item $ p = \infty$. $n(\xi)$\pageoriginale being an indefinite form in the $x_i - s,
  n(\xi) =0$ has always a non-trivial solution in real numbers, so
  that $p =\infty$ is not a characteristic prime. 
\item $p = 2$. The equation $n(\xi)=0$ can be rewritten as 
  \begin{align*}
    n\left( \frac{x_o + \sqrt{-3}x_2}{x_1 + \sqrt{-3}x_3} \right)= 2 &
    = n\left(\xi_1 +  \sqrt{-3} \xi_2\right) \text{ (say) },\\ 
    &= n(\mu)
  \end{align*}
  where \hspace{1cm}$\mu = \xi_1 +  \sqrt{-3} \xi_2 \in k ( \sqrt{-3}) =K$.
\end{enumerate}

Since $-3$ is not a $2$-adic square $(\bar{k}_2( \sqrt{-3}) :
\bar{k}_2)=2$ and $n(\mu) = \mu \bar{\mu}= 2 \Longrightarrow \mu
\bar{\mu} \in (2)$(extended ideal in $(\bar{k}_2( \sqrt{-3}))$. $(2)$
being a prime ideal in $(\bar{k}_2( \sqrt{-3})$, either $\mu$ or
$\bar{\mu}$ is in $(2)$, say $\mu$. Then $\bar{\mu} \in \bar{(2)}= (2)$,
since $2 \in k$.Therefore $\mu \bar{\mu} \in (2)^2$ which is
impossible, since $2 \notin (2)^2$, so that $n (\mu) = 2$ is not
solvable. In other words, $n(\xi)=0$ is not solvable with $ \xi \in
(\bar{k}_2( \sqrt{-3})$ or $p=2$ is a characteristic prime. Similarly
it can be proved that $p=3$ is a characteristic prime and again from
the proof of Theorem \ref{chap1:sec1:thm4}b, it follows that the only characteristic
primes for this $Q$ are $p =2$ and $3$. 

\textbf{5.}~ We shall state and prove
\begin{theorem} % them 5
  {\em Wedderburn's Theorem: Let $Q$ be a quaternion algebra over the
    rational number field $k$ and $\alpha,  \beta \in Q$. Then
    $\alpha$ and $\beta$ satisfy the same quadratic irreducible
    equation over $k$ (i.e.,  $t(\alpha) = t(\beta),  n(\alpha)
    = n(\beta)$), if and only if there exists an element $\rho \in Q$,
    having an inverse $\rho^{-1}$, such that $\beta = \rho^{-1} \alpha
    \rho$.} 
\end{theorem}

\begin{proof}
  We\pageoriginale first prove the converse part.
\begin{enumerate}[i)]
\item Let $\rho \in Q$ be such that $\beta = \rho^{-1} \alpha
  \rho$. Now $\alpha$ satisfies the equation 
  $$
  \alpha^2 - t(\alpha). \alpha + n(\alpha) = 0,  \text{  but  }
  \beta^2 = (\rho^{-1} \alpha \rho).  (\rho^{-1} \alpha \rho) =
  \rho^{-1} \alpha^2 \rho,  
  $$
  and $\rho^{-1} (\alpha^2 - t (\alpha). \alpha + n(\alpha)) \rho = 0$
  imply that 
  $$
  \beta^2 - t(\alpha). \beta + n(\alpha) = 0, 
  $$
  i.e., the irreducible equation satisfied by $\beta$ is the same as
  that satisfied by $\alpha$. In other words, 
  $$
  t(\alpha) = t(\beta) ; n(\alpha) = n(\beta).
  $$
\item For the direct part, we distinguish two cases:
  $$
  (a) Q \overset{\sigma}{\cong} \mathfrak{M}_2 (k) ; \quad (b) Q
  \ncong \mathfrak{M}_2 (k). 
  $$
\end{enumerate}

\begin{enumerate}[(a)]
\item Denoting the image of $\alpha$ by $\sigma$ as $\alpha^\sigma$,
  let $\alpha^\sigma = \binom{a~ b}{c ~ d}~  a,
  b, c, d, \in k$. Then there exists a matrix $\rho_1$ with elements
  in a suitable extension of $k$, such that 
  $$
  \rho^{-1}_1 \alpha^\sigma \rho_1 = 
  \begin{pmatrix}\alpha_1 & 0\\0 &\alpha_2 \end{pmatrix}
  \alpha_1. ~\text{ and }~ \alpha_2  
  $$
  being the two distinct solutions of the equation $x^2 - t(\alpha). x
  + n (\alpha) = 0$. Further, since $\beta$ satisfies the same
  equation, we have a matrix $\rho_2$ such that $\rho^{-1}_2
  \beta^\sigma \rho_2 = \binom{\alpha_1 ~ 0}{0 ~\alpha_2}$. 
  $$
  \text { Hence  } \rho^{-1}_1 \alpha^\sigma \rho_1 = \rho^{-1}_2
  \beta^\sigma \rho_2. 
  $$
  i.e., $\rho^{-1} \alpha^\sigma \rho = \beta^\sigma$ where $\rho =
  \rho_1 \rho^{-1}_2$. Of course, the elements of\pageoriginale $\rho$ lie in some
  extension of $k$. 
\item If $Q \ncong \mathfrak{M}_2 (k)$, then let $K$ be a quadratic
  extension such that $Q K/K \overset{\sigma_1}{\cong} \mathfrak{M}_2
  (K)$. (For example, we can take $K \cong k(\omega)$). Then, as
  above, there exists a matrix $\rho$ with elements in an extension of
  $K$, such that $\rho^{-1} \alpha^{\sigma_1} \rho =
  \beta^{\sigma_1}$. 
\end{enumerate}
\end{proof}

In either case, we have obtained a matrix $\rho$ whose elements lie in
a finite extension of $k$, say $L\, (L= k(1, \lambda_1,  \lambda_2,
\ldots, ))$, satisfying the condition $\rho^{-1} \alpha^{\sigma '}
\rho = \beta^{\sigma '}$ (since $QL/L \overset{\sigma '}{\cong}
\mathfrak{M}_2 (L)$).  Let $\rho ' (\in Q L) = \sigma'^{-1}
(\rho)$. Then 
$$
\rho' = \rho_\circ + \lambda_1 \rho_1 + \lambda_2 \rho_2 + \cdots,
\rho_\circ,  \rho_1,  \ldots \in Q.  
$$

Substituting in $\alpha \rho' = \rho' \beta$ we obtain
$$
\alpha \rho_\circ + \lambda_1 \alpha \rho_1 + \cdots = \rho_\circ
\beta + \lambda_1 \rho_1 \beta + \cdots .  
$$

Since $\alpha \rho_i \in Q$, expanding each of these in terms of the
basis elements $(1, \omega, \Omega, \omega \Omega)$ and using the fact
that $1, \lambda_1,  \lambda_2, \ldots$ equations $\alpha \rho_0 =
\rho_0 \beta,  \alpha \rho_1 = \rho_1 \beta,  \ldots$. 

In case (b) for at least one $i, \rho_i \ne 0$, so that $\rho^{-1}_i$
exists and hence $\beta = \rho^{-1}_i \alpha \rho_i ; \rho_i \in Q$. 

In case ($a$) we use the fact that $n(\rho ') \neq 0$ (since
$\rho^{-1}$ exists). Now, $n(\rho') = n(\rho_0) + \lambda_1 (\rho_1
\bar{\rho_0} + \rho_0 \bar{\rho_1}) + \cdots + \lambda^2_1 n(\rho_1) +
\cdots + \lambda^2_2 n(\rho_2) + \cdots = 0$. 

In\pageoriginale this quadratic expression in $\lambda_1,  \lambda_2 \ldots$ we
replace $\lambda_1,  \lambda_2 \cdots$ by indeterminates $x_1,  x_2
\ldots$ obtaining a quadratic polynomial  $f(x_1,  x_2,  \ldots)$ with
coefficients in $k$. $f$ does not vanish identically since
$f(\lambda_1, \lambda_2,  \ldots) = n(\rho') \neq 0$. Now, $k$ being
an infinite field, we can always find a set of elements
$\bar{\lambda_1}, \bar{\lambda_2},  \ldots \in k$ such that
$f(\bar{\lambda_1}, \bar{\lambda_2}, \ldots) \neq 0$. 

Let $\zeta = \rho_0 + \bar{\lambda_1} \rho_1 + \cdots ; \zeta \in Q$
and $n(\zeta) = f(\bar{\lambda_1 },  \ldots) \neq 0$ so that
$\zeta^{-1}$ exists. 
\begin{align*}
  \alpha \zeta & = \alpha \rho_0 + \bar{\lambda_1} \alpha \rho_1 + \cdots\\
  & = \rho_0 \beta + \bar{\lambda_1} \rho_1 \beta + \cdots = \zeta
  \beta, i.e., \zeta^{-1} \alpha \zeta = \beta 
\end{align*}  
  
Thus, the proof of our theorem is complete.
  
\section{Orders and Ideals}\label{chap1:sec2} % sec 2
  
 \textbf{5.}~ Let $k$ denote the rational number field; $\mathscr{O}$, the ring
 of rational integers; $\bar{k_p}$, the p-adic completion of $k$ and
 $\mathscr{O}_P$, the ring of p-adic integers. 

 \setcounter{theorem}{0}
 \begin{theorem}\label{chap1:sec2:thm1}%thm 1
   {\em Let $Q/k$ be a quaternion algebra. Then, four elements $\mu_1,
     \mu_2, \mu_3, \mu_4\in Q$ are linearly independent over $k$ if
     and only if the discriminant} 
   $$
   | t(\mu_i \mu_k) | = D(\mu_1, \mu_2, \mu_3,  \mu_4) \neq 0. 
   $$
 \end{theorem}  

\begin{proof}
  Let $Q =k [1, \omega, \Omega, \omega \Omega] = k[\nu_1, \nu_2,
    \nu_3, \nu_4]$ (say). Then
  $$
  D (\nu_1, \nu_2,  \nu_3,  \nu_4) = \begin{vmatrix} 2 & 0 & 0 & 0\\ 0
    & 2p & 0 & 0\\ 0 & 0 & 2q & 0\\ 0 & 0 & 0 &
    -2pq \end{vmatrix}^{=-16p^2 q^2 \neq 0.} 
  $$

  Now,\pageoriginale we have $\mu_i = \sum\limits^{4}_{\ell=1} m_{il} \nu_{l} ; m_{il}
  \in k, i = 1$  to  $4$ so that 
  \begin{align*}
    \mu_i \mu_k & = \bigg(\sum_l m_{il} \nu_l\bigg) \bigg(\sum_j m_{kj}
    \nu_j\bigg) = \sum_{j, l}  m_{il} \nu_{l} l_{j} m_{kj}\\ 
    \text{i.e.,}\hspace{2cm} t(\mu_i \mu_k) & = \sum_{j, l} m_{il}
    t(\gamma_{l} \gamma_j) m_{kj}. \hspace{3cm}
  \end{align*}
  
  Denoting by $(m_{kj})^T$, the transpose of $(m_{kj})$ we have
  $$
  (t(\mu_i \mu_k) = (m_{il})(t (\nu_l \nu_j)) (m_{kj})^T.
  $$
  
  Hence 
  $$
  \displaylines{\hfill 
  |t(\mu_i \mu_k) | = |t(\nu_l \nu_j)| |m_{ik}|^2, \hfill \cr
  \text{i.e.,}\hfill D(\mu_1,  \ldots,  \mu_4) = -16p^2 q^2 |m_{ik}|^2.\hfill}
  $$
  But $\nu_1 \cdots \nu_4$ being linearly independent over $k, |m_{ik}|
  \neq 0$ implies that $\mu_1 \cdots \mu_4$ are also linearly
  independent over $k$ and conversely $\mu_1 \cdots \mu_4$ linearly
  independent implies that $|m_{ik}| \neq 0$, i.e., $D(\mu_1 \cdots
  \mu_4) \neq 0$. 
\end{proof}  
    
\textbf{6.}~ Let $Q$ be a quaternion algebra over the rational number field
$k$. We now define an order in $Q$. 

\begin{defi*}%def
  An { \em order} $\mathcal{J} $ in $Q$ is a ring of elements of $Q$
  with the following properties: 
  \begin{enumerate}[\rm i)]
  \item $1 \in \mathcal{J}$,
  \item $\alpha \in \mathcal{J} \Rightarrow t(\alpha)$ and $n(\alpha)$
    are integers, 
  \item $\mathcal{J}$ has $4$ linearly independent generators over $k$.
    
    We can also define an order alternatively as follows:
    
    2) $\mathcal{J}$ is an { \em order} if it is a ring of elements of
    $Q$ such that 
    i)\pageoriginale $1 \in \mathcal{J}$, ii) $\mathcal{J}$ is a finite
    $\mathscr{O}$-module, iii) $\mathcal{J}$ has 4 linearly
    independent generators over $k$. 
  \end{enumerate}

  For the equivalence of these two definitions, we shall prove in
  Theorem \ref{chap1:sec2:thm2} that $\mathcal{J}$ an order as in $(1)$ is a finite
  $\mathscr{O}$-module. But for the converse, namely if $\mathcal{J}$
  is a finite $\mathscr{O}$-module, then $\alpha \in \mathcal{J}
  \Rightarrow n(\alpha)$ and $t(\alpha)$ integral follows from the
  fact $\alpha \in \mathcal{J} \Rightarrow \alpha \mathcal{J} \subset
  \mathcal{J} $ (for $\mathcal{J}$ is a ring), i.e., $\alpha$ is an
  ``integer'', i.e., $\alpha$ satisfies a monic polynomial with
  integral coefficients. 
\end{defi*}  
  
We shall give some examples of orders in $Q$. Let $Q$ have the basis
$[1, \omega, \Omega, \omega \Omega]$ over $k$ where $\omega^2$ and
$\Omega^2$ are integers. Then the finite $\mathscr{O}$-module $[1,
  \omega, \Omega, \omega \Omega]$ can easily be seen to be an
order. If $\omega^2$ and $\Omega^2$ were not integers, (say) $\omega^2
= \dfrac{p'}{q'}, \Omega^2 = \dfrac{p''}{q''}$, then the
$\mathscr{O}$-module  $[1, q' \omega, q'' \Omega, q' q''\omega
  \Omega]$ is an order in $Q$. 

The definition of an order in a quaternion algebra $Q_p$ over the
$p$-adic number field $\bar{k_P}$ is given exactly as above, except
that the ring of integers $\mathscr{O}$ is now replaced by the ring
$\mathscr{O}_p$ of $p$-adic integers. 
\begin{theorem}\label{chap1:sec2:thm2}%thm 2
  {\em An order $\mathcal{J}$ is a finite $\mathscr{O}$-module and has
    $4$ linearly independent generators over $\mathscr{O}$}. 
\end{theorem}

\begin{proof}
  Let $[\mu_1, \ldots,  \mu_4]$ be a basis of $\mathcal{J}$ over $k$.

  Then $\mu_i \mu_k = \sum \limits^{4}_{j=1} m_{ik}^j \mu_j,   m_{ik}^j
  \in k$. Let $N$ be the common denominator of all these $m_{ik}^{j}$
  and put $N \mu_i = \mu'_i$. Now $\mu'_i \mu'_K = \sum
  \limits^{4}_{j=1} N m_{ik}^{j} \mu'_j$. Then the $\mathscr{O}$-module
  $\mathcal{J}' = [1, \mu'_1  \ldots,  \mu'_4]$ is\pageoriginale actually an order;
  for by choice $\mu'_i \mu'_k \in \mathcal{J}'$ and hence
  $\mathcal{J}'$ is a ring $\ni 1$ and $\mathcal{J}' \subset
  \mathcal{J}$, so that every element has integral trace and
  norm. Further $\mathcal{J}'$ is of rank $4$ over $k$. 
\end{proof}

If $\mathcal{J}' \neq \mathcal{J}$, there exists an element $\alpha
\in \mathcal{J}$ such that $\alpha \notin \mathcal{J}'$. Consider the
$\mathscr{O}$-module $\mathcal{J}'' = [\mu'_1, \ldots,  \mu'_4,
  \alpha,  \alpha \mu'_1, \ldots \alpha \mu'_4,  \mu'_1 \alpha \cdots
  \mu'_4 \alpha]$.  Then $\mathcal{J}''$ is an order. For, $i) 1 \in
\mathcal{J}'', ii) \mathcal{J}'' \subset \mathcal{J}$ so that every
element of $\mathcal{J}''$ has integral trace and norm, $iii)
\mathcal{J}''$ is of rank $4$. Therefore it is enough to prove that
$\mathcal{J}''$ is a ring. 
\begin{align*}
  \alpha \mu'_i .  \alpha \mu'_k  & = (t(\alpha \mu'_i) -
  (\overline{\alpha \mu'_i})). \alpha \mu'_k\\ 
  & = t(\alpha \mu'_i).  \alpha \mu'_k - \bar{\mu_i}' \bar{\alpha}
  \alpha.  \mu'_k \in \mathcal{J}'' 
\end{align*}
since $t(\alpha \mu_i) \in \mathscr{O} \subset \mathcal{J}'',
n(\alpha)$ is integral and $\bar{\mu'_i} = t(\mu'_i)- \mu'_i \in
\mathcal{J}'$ so that $n(\alpha). \bar{\mu'_i} \mu'_k \in \mathcal{J}'
\subset \mathcal{J}''. \mathscr{O}$ being a principal ideal domain, we
know that the finite $\mathscr{O}$-module $\mathcal{J}''$ has always a
linearly independent $\mathscr{O}$-module basis (say) $[\nu_1 \cdots
  \nu_4]$ similarly $\mathcal{J}' = [\mu''_1 \cdots \mu''_4]$. 

Now, $[\nu_1 \cdots  \nu_4] \supset [\mu''_1  \cdots \mu''_i]
\Rightarrow \mu''_i = \sum \limits_{k=1}^{4} m_{ik} \nu_k,  m_{ik} \in
\mathscr{O}$ i.e., $D[\mu''_1 \cdots \mu''_4] = D [\nu_1 \cdots
  \nu_4]\cdot |m_{ik}|^2$. In other words, we have $D(\mathcal{J}') =
(\mathcal{J}''). |m_{ik}|^2. D (\mathcal{J}')$ and $D(\mathcal{J}'')$
being integers, $D(\mathcal{J}'') |D(\mathcal{J}')$. 

Continuing the above construction further we obtain a system of finite
$\mathscr{O}$-modules $\mathcal{J}', \mathcal{J}'', \mathcal{J}''',
\ldots$ which are also orders, such that $\mathcal{J} \supset \cdots
\supset \mathcal{J}''' \supset \mathcal{J}'' \supset \mathcal{J}'
\supset \cdots$ and $D(\mathcal{J})|\cdots D(\mathcal{J}''')|
D(\mathcal{J}'')| D(\mathcal{J}')$. This sequence must end after a
finite stage, so that $\mathcal{J}^{(n)} = \mathcal{J}$ for some
$n$. In other words, we have proved that $\mathcal{J}$ itself is a
finite $\mathscr{O}$-module and consequently has $4$ linearly
independent $\mathscr{O}_p$-module basis elements. 

An\pageoriginale order $\mathcal{J}$ is defined to be \textit{maximal} if it is not
properly contained in any other order. The above divisibility property
of the discriminants incidentally shows that every order is contained
in some maximal order. 

Since the above arguments can be carried over to $p$-adic integers
also, any order $\mathcal{J}_p$ in $Q_p/ \bar{k_p}$ is a finite
$\mathscr{O}$-module and any order is contained in a maximal order. 

Let now $\mathcal{J} = [\mu_1 \cdots \mu_4]$ (an integral basis) be an
order in $Q/k$. Then if $\xi \in \mathcal{J}, \xi = m_1 \mu_1 + \cdots
+ m_4 \mu_4, \mu_i \in \mathscr{O}$. We associate to $\mathcal{J}$, an
order $\mathcal{J}_p$ in $Q_p/\bar{k_p}$ such that if $\xi \in
\mathcal{J}_p, \xi = m_1' \mu'_1 + \cdots + m'_4 \mu_4,  m'_i \in
\mathscr{O}_p$, i.e., $\mathcal{J}_p = [ \mu_1 \cdots \mu_4]$ (an
$\mathscr{O}_p$-module). We now establish a connection between
$\mathcal{J}$ and $\mathcal{J}_p - s$. 

$\mathcal{J} = Q \cap
\mathcal{J}_2 \cap \mathcal{J}_3 \cap \mathcal{J}_5 \cap \cdots \cap
\mathcal{J}_p \cap \cdots$ where $p$ runs through all primes and
$\mathcal{J}_p$ is the $p$-adic extension of $\mathcal{J}$. For, if
$\xi \in \mathcal{J},  \xi = \sum \limits^{4}_{i=1} m_i \mu_i,  m_i
\in \mathscr{O}$. Therefore $m_i \in k$, and $\mu_i\in \mathscr{O}_p$ for
all $p$, i.e., $\xi \in Q \bigcap \limits_p \mathcal{J}_p$. 

Conversely if $\xi \in Q \bigcap_p \mathcal{J}_p,  \xi = \sum
\limits^{4}_{i=1} m'_i \mu_i,  m'_i$ are rational $p$-adic integers
for all $p$ and hence are rational integers, i.e., $\xi \in
\mathcal{J}$. Therefore $\mathcal{J} = Q \cap \mathcal{J}_2 \cap
\mathcal{J}_3 \cap \cdots \cap\mathcal{J}_p \cap \cdots$ 
\begin{theorem}\label{chap1:sec2:thm3}%thm 3
  {\em $\mathcal{J}$ is a maximal order in $Q/k \Leftrightarrow
    \mathcal{J}_p$ is maximal order in $Q_p/\bar{k_p}$ for every $p$}. 
  
  For the same, we shall prove
  
  $\mathcal{J}$ is not maximal $\Leftrightarrow \mathcal{J}_P$ is not
  maximal, for some $p$. 
\end{theorem}

\begin{proof}
  i)\pageoriginale If $\mathcal{J}$ is not maximal, $\mathcal{J} \subset
  \mathcal{J}'$ where $\mathcal{J}'$ is maximal, then there exists
  $\xi \in \mathcal{J}'$, which is $\notin \mathcal{J}$. If
  $\mathcal{J} = [\mu_1 \cdots \mu_4]$, then $Q$ is generated by
  $(\mu_1,  \ldots,  \mu_4)$ over $k$, so that $\xi = x_1 \mu_1 +
  \cdots + x_4 \mu_4$ where at least one $x_i$ (say) $x_1$, is not
  integral. Let $p$ be a prime dividing the denominator of
  $x_1$. Since $\xi \in \mathcal{J}' \subset \mathcal{J}'_p$, the
  $\mathscr{O}_p$-module $\mathcal{J}''_p = [\mathcal{J}_p, \xi
    \mathcal{J}_p,  \mathcal{J}_p \xi]$ is an order containing
  $\mathcal{J}_p$ properly, so that $\mathcal{J}_p$ is not maximal. 
  
  ii) Conversely, if $\mathcal{J}_P$ is not maximal for some $p$ then
  there exists $\xi \in$ some maximal order, which is $\notin
  \mathcal{J}_p$. If $\mathcal{J}_p = [\mu_1 \cdots \mu_4]$ then $Q_p$
  is generated by $[\mu_1 \cdots \mu_4]$ over $\bar{k}_p$ so that $:
  \xi = x_1\mu_1 + \cdots + x_4 \mu_4$ where not all $x_i$ are $p$-adic
  integers. 
\end{proof}

Let $x_i = x'_i + u'_i \,(u'_i$-integral)\, $(i = 1 \text{ to }  4)$,
where some $x'_i$ may vanish; $x'_i$ are rational numbers. 

Defining $\xi' = x'_1 \mu_1 + \cdots + x'_4 \mu_4,  \xi = \xi'
+\xi''$
where $\xi'' \in \mathcal{J}_p$. Further $\xi' = \xi - \xi'' \in
$ maximal order containing $\mathcal{J}_p$. We shall prove that the
$\mathscr{O}$-module $\mathfrak{M} '  = [ \mathcal{J}, \xi'
  \mathcal{J}, \mathcal{J} \xi' ] = [\mu'_1 \cdots \mu'_4]$ (say) is
actually an order containing $\mathcal{J}$ properly. 

$\mu'_i \mu'_k = \sum \limits^{4}_{j=1} r^j_{ik} \mu'_j$ where
$r^j_{ik}$ are $p$-adic integers, since the $\mathscr{O}_p$-module
$[\mu'_1 \cdots \mu'_4] = [\mathcal{J}_p, \xi' \mathcal{J}_p,
  \mathcal{J}_p \xi']$ is an order. Further they are rational numbers
since $Q$ is generated by $(\mu'_1 \cdots \mu'_4)$ over $k$. Therefore
$r^j_{ik}$ are rational $p$-adic integers. But, their denominators if
any, can contain only powers of $p$ by virtue of $\xi '$ so that
$r^j_{ik}$ are all rational integers, i.e., $\mu'_i \mu'_k \in
\mathfrak{M} '$. In other words, $\mathfrak{M}'$ is a ring. Hence
$\mathfrak{M}'$ is a finite $\mathscr{O}$-module of\pageoriginale rank $4$ over $k$
and also a ring containing $1$, so that by our second definition of an
order, $\mathfrak{M}'$ is an order. Since $\xi \in \mathfrak{M}'$ and
$\notin \mathcal{J}, \mathfrak{M}'$ contains $\mathcal{J}$ properly,
i.e., $\mathcal{J}$ is not maximal. 

\textbf{7.}~ We shall now study the maximal orders of $Q_p$ in both the cases
$i) Q_p$ is a division algebra over $\bar{k}_p$ and $ii) Q_p/\bar{k}_p
\cong \mathfrak{M}_2 (\bar{k}_p)$. In case $i)$ we have a uniqueness
theorem of orders, namely, 
\begin{theorem}\label{chap1:sec2:thm4}%thm 4
  {\em If $Q_p/\bar{k}_P$ is a division algebra, then there is in
    $Q_p$, only one maximal order $\mathcal{J}_p$ and in fact
    $\mathcal{J}_p = \{\nu \in Q_p : n(\nu) \in \mathscr{O}_p\}$ }. 
\end{theorem} 
 
We shall prove the following two lemmas from which we deduce the theorem.
\begin{lemma}\label{chap1:sec2:lem1}
$\nu \in Q_p$ and $n(\nu) \in \mathscr{O}_p \Rightarrow t(\nu)
    \in \mathscr{O}_p$ 
\end{lemma}
\begin{lemma}\label{chap1:sec2:lem2}
   $\mathcal{J}_p = \{ \nu : n(\nu) \in \mathscr{O}_p\}$ is a ring.
\end{lemma}

  \begin{enumerate}[\rm 1)]
  \item
    \begin{enumerate}[\rm a)]
    \item If $\nu \in \bar{k}_p, t (\nu) = 2 \nu$ and $n(\nu) =
      \nu^2. n (\nu) \in \mathscr{O}_p \Rightarrow \nu^2 \in
      \mathscr{O}_p$, which means that $\nu \in \mathscr{O}_p$, i.e.,
      $t(\nu) = 2 \nu \in \mathscr{O}_p$. 
    \item If $\nu \notin \bar{k}_p, \nu$ satisfies the irreducible
      equation $x^2 - t (\nu). x + n(\nu) = 0$. Given that $n(\nu) \in
      \mathscr{O}_p$, if $t(\nu) \notin \mathscr{O}_p$, let $t(\nu) =
      \dfrac{\mu}{p^r}, \mu$, a $p$-adic unit and $r > 1$. Then $\nu'
      = p^r.  \nu$ satisfies $x'^2 - \mu.  x' + n(\nu)p^{2r} \equiv
      (x' - \mu) x' \pmod p$, (replacing $x ~ by ~
      \dfrac{x'}{p^r}$). $\mu$ being a $p$-adic unit, $x'$ and $x' -
      \mu$ are coprime mod $p$ and hence by Hensel's lemma, the above
      equation $x'^2 - \mu.  x' + n(\nu).  p^{2r}$ is reducible in
      $\bar{k}_p$, which is a contradiction to the fact that it is
      irreducible. Therefore $t(\nu) \in \mathscr{O}_p$. 
    \end{enumerate}
  \item  Let\pageoriginale $\nu_1,  \nu_2 \in Q_p$ such that $n(\nu_1)$ and
    $n(\nu_2)$ are in $\mathscr{O}_p$. Then $n(\nu_1) | n(\nu_2)$ or
    $n(\nu_2)|n(\nu_1)$. We may assume that $n(\nu_1) | n(\nu_2)$,
    i.e., $n(\nu^{-1}_1 \nu_2) \in \mathscr{O}_p$. Then by
    Lemma (\ref{chap1:sec2:lem1}),
    $t(\nu^{-1}_{1} \nu_2)$ is also an integer. 
    $$
    n(\nu_1 + \nu_2) = n(\nu_1(1 + \nu^{-1}_1 \nu_2)) = n(\nu_1). (1 +
    \nu^{-1}_1 \nu_2). (1 + \overline{\nu^{-1}_1 \nu_2}).
    $$ 
  \end{enumerate}

    Hence 
    $n(\nu_1 + \nu_2) = n(\nu_1) (1 + t(\nu^{-1}_1 \nu_2) + n(\nu^{-1}_1
    \nu_2) ) \in \mathscr{O}_p$,  (since  $n(\nu_1), t(\nu^{-1}_1
    \nu_2)$, $n(\nu^{-1}_1\nu_2) \in \mathscr{O}_p$). 
    
    \noindent or $\nu_1,  \nu_2 \in \mathcal{J}_p \Rightarrow \nu_1 + \nu_2 \in
    \mathcal{J}_p.  \quad \nu_1 \nu_2 \in \mathcal{J}_p$ follows at once,
    i.e., $\mathcal{J}_p$ is a ring. 

 Now, $\mathcal{J}_p$ cannot be of rank $>  4$ since it contains
 certain rational multiples of every element of $Q_p$. It cannot be of
 rank $< 4$ also, since it contains all orders. Hence $\mathcal{J}_p$
 is of rank $4$. Therefore, by our first definition of an order,
 $\mathcal{J}_p$ is an order and it is obviously the only maximal
 order in $Q_p$.
 
 Let $Q/k\overset{\sigma}{\cong} \mathfrak{M}_2 (k)$. We shall give an
 example of a maximal order $\mathcal{J}$ and then prove that all
 maximal orders can be written in the form $\mu^{-1} \mathcal{J} \mu,
 \quad \mu \in Q$ such that $n(\mu) \neq 0 $. Define $\mathcal{J} =
       [\mu_1, \ldots,  \mu_4]$ the finite $\mathscr{O}$-module where 
 $$
 \mu^\sigma_1 = \binom{1 ~ 0}{0 ~ 0},
 \mu^\sigma_2 = \binom{0 ~ 1}{0 ~ 0},
 \mu^\sigma_3 = \binom{0 ~ 0}{1 ~ 0},
 \mu^\sigma_4 = \binom{0 ~ 0}{0 ~ 1}
 $$
 Then
 $$
 D(\mathcal{J}) = D[\mu_1 \cdots \mu_4] = 
 \begin{vmatrix} 
   1 & 0 & 0 & 0\\ 
   0 & 0 & 1 & 0\\ 
   0 & 1 & 0 & 0\\ 
   0 & 0 & 0 & 1 
 \end{vmatrix} = -1. 
 $$
 
 By\pageoriginale our second definition of an order, $\mathcal{J}$ is an order,
 since $\mathcal{J}$ is a ring containing $1$ and a finite
 $\mathscr{O}$-module of rank $4$. But $D(\mathcal{J})$ being an unit,
 $\mathcal{J}$ is a maximal order. 
 
 Let $\mu \in Q$ such that $n(\mu) \neq 0$. Consider $\mathcal{J}' =
 \mu^{-1} \mathcal{J} \mu = [\mu^{-1} \mu_1, \mu$, $\ldots,  \mu^{-1} \mu_4
   \mu]$. By means of the isomorphism one sees that $\mathcal{J} '$ is
 again a ring containing $1$, a finite $\mathscr{O}$-module of rank
 $4$ and hence an order. Further $\mathcal{J}'$ is maximal since
 $\mathcal{J}$ is maximal. 
 
 [Analogously, if $Q_P / \bar{k}_p \overset{\sigma}{\cong}
   \mathfrak{M}_2 (\bar{k}_p)$, we have the order $\mathcal{J}_p$
   corresponding to $\mathcal{J}$, which is again maximal, since $-1$
   is a $p$-adic unit. Further if $\mu \in Q_p$ such that $n(\mu) \neq
   0, \mu^{-1} \mathcal{J}_p \mu$ is an order and also maximal.] 
 \begin{theorem}\label{chap1:sec2:thm5} %thm 5
   {\em Let $Q/k \overset{\sigma}{\cong} \mathfrak{M}_2 (k)$ and let
     $\mathcal{J}$ be the maximal order defined as before and
     $\mathcal{J}'$, any other maximal order. Then there exists a $\mu
     \in Q$ such that $n(\mu) \neq 0$ and $\mathcal{J}' =
     \mu^{-1}\mathcal{J}\mu$.} 
 \end{theorem} 

\begin{proof}
  Define $\mathfrak{M}$ to be the $\mathscr{O}$-module $[l_j  l'_k,  j
    = 1 \text{ to }4, k =1 ~ to ~ 4]$ if $\mathcal{J} = [l_j]$ and
  $\mathcal{J}'=[l'_k]$. Then we shall prove 
  \begin{enumerate}[i)]
  \item $\mathfrak{M} = \mathcal{J} \mu$ for a suitable $\mu \in \mathfrak{M}$.
  \item If $\mathfrak{K}=\{ \xi : \mathfrak{M} \xi \subset
    \mathfrak{M}\}$ then $\mathfrak{K} = \mathcal{J} ' = \mu^{-1}
    \mathcal{J} \mu$. 
  \end{enumerate}
  i) We shall first prove that there exists a $\mu \in \mathfrak{M}$
  such that $n(\mu) \neq 0$ and with the additional property that
  $n(\mu) | n(\nu)$ for all $\nu \in \mathfrak{M}$. 
\end{proof} 
 
 Let $\mu_1,  \mu_2$ be any two elements of $\mathfrak{M}$. Let $N$ be
 chosen sufficiently large so that 
 $$
 \mu'_1 = N \mu_1 \sim 
 \begin{pmatrix} 
   m_{11} & m_{12} \\ 
   m_{21} & m_{22} 
 \end{pmatrix} ~ \text{and} ~ \mu'_2 = N \mu_2 \sim 
 \begin{pmatrix}
   n_{11} & n_{12}\\ 
   n_{21} & n_{22} 
 \end{pmatrix} 
 $$
where\pageoriginale $m' s$ and $n' s$ are all integers. By applying suitable
elementary transformation on the left side, $\mu'_1$ goes into  
$$
{\varepsilon_1}
_{\begin{pmatrix} 
    m_{11} & m_{12} \\ 
    m_{21} & m_{22} 
  \end{pmatrix} ~=~ 
  \begin{pmatrix} 
    m'_{11} & m'_{12} \\
    0 &  m'_{22} 
\end{pmatrix}} 
$$
and $\mu'_2$ goes into
$$
{\varepsilon_2}
_{\begin{pmatrix} 
    n_{11} & n_{12} \\ 
    n_{21} & n_{22} 
  \end{pmatrix} ~=~ 
  \begin{pmatrix} 
    n'_{11} & n'_{12} \\
    0 & n'_{22} 
\end{pmatrix}} 
$$

But $n(\mu'_1). n(\varepsilon_1) = m'_{11} m'_{22} \Rightarrow
n(\mu'_1) = m'_{11} m'_{22}$ since $\varepsilon_1$ is
unimodular. Similarly $n(\mu'_2) = n'_{11} n'_{22}$. 

Let $\gamma_{11} = (m'_{11}, n'_{11})$ and $\gamma_{22} = (m'_{22},
n'_{22})$. Then we can find integers $a_1, b_2, a_2, b_2$ such that 
$$
a_1 m'_{11} + b_1 n'_{11} = \gamma_{11}, \quad a_2 m'_{22} + b_2
n'_{22} = \gamma_{22}. 
$$ 

Now, define
$$
\displaylines{\hfill 
  \mu' = \alpha_1 \mu'_1 + \alpha_2 \mu'_2\hfill \cr 
  \text{where}\hfill  \alpha^\sigma_1 = 
  \begin{pmatrix} a_1& 0 \\ 0 & a_2 \end{pmatrix} 
  \varepsilon_1, \alpha^\sigma_2 = 
  \begin{pmatrix} b_1& 0 \\0 & b_2 \end{pmatrix}
  \varepsilon_2 ; (\alpha_1, \alpha_2 \in \mathcal{J}) \hfill \cr
  \text{so that} \hfill 
  \mu'^\sigma = 
  \begin{pmatrix} a_1& 0 \\0 & a_2 \end{pmatrix} 
  \begin{pmatrix} m'_{11}& 0 \\0 & m'_{22} \end{pmatrix}+ 
  \begin{pmatrix} b_1& 0 \\0 & b_2 \end{pmatrix} 
  \begin{pmatrix} n'_{11}& n'_{12}\\0 & n'_{22} \end{pmatrix} = 
  \begin{pmatrix} \gamma_{11}& * \\0 &\gamma_{22} \end{pmatrix} \hfill }
$$

Hence $n(\mu') = \gamma_{11}. \gamma_{22}$ which divides $n(\mu'_1)$ and
$n(\mu'_2)$. Since $\mu' = \alpha_1 N \mu_1 + \alpha_2 N \mu_2 \in
\mathcal{J}. N \mathfrak{M} \subset N \mathfrak{M}, \mu = \dfrac{\mu
  '}{N} \in \mathfrak{M}$. From the above, it implies that $n(\mu)$
divides $n(\mu_1)$ and $n(\mu_2)$. In other words $n(\mu^{-1} \mu_1)$
and $n(\mu^{-1} \mu_2)$ are integers. 

If\pageoriginale $\mu_3$  is a third element of $\mathfrak{M}$, for $\mu_3$ and
$\mu$ we can construct a $\nu \in \mathfrak{M}$ such that
$n(. \nu^{-1} \mu)$ and $n(\nu^{-1} \mu_3)$ are integers. By virtue
of $\mu, n (\nu^{-1} \mu_1)$ and $n(\nu^{-1} \mu_2)$ are also
integers. Similarly, given $n$ elements $\mu_i \in \mathfrak{M} (i = 1
~ to ~ n)$, we can find an element $\mu \in \mathfrak{M}$ such that
$n(\mu^{-1} \mu_i)$ are all integers. 

Let $\mathfrak{M} = [\nu_1 \cdots \nu_4]$. If $\xi = \sum
\limits^{4}_{i=1} x_i  \nu_i,  x_i \in \mathscr{O}$ then we have 
$$
n(\xi) = \sum^4_{i=1} n(\nu_i) x^2_i + \sum^4_{i, j=1} t(\nu_i
\bar{\nu_j}) x_i x_j = \frac{\nu}{s} 
$$ 
(say) where ${s}$ is the common denominator of $n(\nu_i),
t(\nu_i \bar{\nu}_j)$ and is fixed for all $\xi \in
\mathfrak{M}$. Hence we have a g.c.d of all $n(\xi), (\xi \in
\mathfrak{M})$ (say) $m$. Then, if $d =$ g.c.d of $n(\nu_i), t (\nu_i
\bar{\nu_j})$, we assert that $d = m$. For $m$ is evidently a multiple
of $d$. Conversely $n(\mu_i), n(\mu_i + \mu_j)\, (= n (\mu_i) + n(\mu_j)
+ t(\mu_i \,\bar{\mu}_j))$ are multiples of $m$ implies that $t(\mu_i
\,\bar{\mu_j})$ are multiples of $m$, i.e., $d$ is a multiple of $m$ so
that $d = m$. Now, choose an element $\mu \in \mathfrak{M}$ such that
$n(\mu)| n(\mu_1)$ and $n(\mu)| n(\mu_i + \mu_j) \,(i, j = 1 ~ to ~ 4, i
=\neq j)$. Then $n(\mu) | d(=m)$. But $m|n(\mu)$, by definition of
$m$. Hence $n(\mu) = m$. i.e., $n(\mu)| n(\mu')$ for all $\mu' \in
\mathfrak{M}$. 

Having obtained $\mu \in \mathfrak{M}$ with the property that $n(\mu)|
n(\mu ')$ for all $\mu ' \in \mathfrak{M}$, we assert that there
exists a basis $[\mu, \mu_1, \mu_2, \mu_3]$ for $\mathfrak{M}$. For,
if $[\nu_1 \cdots \nu_4]$ were some integral base for $\mathfrak{M}$, 
$$
\mu = \sum^4_{i=1} a_i \nu_i = t \sum^4_{i=1} a'_i \nu_i
$$
(if all $a_i$ are not coprime, $t$, an integer $> 1$.)

Now,\pageoriginale $(a'_1,  \ldots,  a'_4)$ being a coprime row, it can be completed
to a unimodular matrix $\mathcal{U}$ (say) which, on applying to the
basis $[\nu_1 \cdots \nu_4]$ gives an integral basis $\left[\dfrac{\mu}{t},
  \mu_1, \mu_2, \mu_3\right]$ for $\mathfrak{M}$. But $n(\mu)|
n\left(\dfrac{\mu}{t}\right) \Rightarrow n(\mu). \lambda = \dfrac{n(\mu)}{t^2},
\lambda$ being an integer; 
$$
\displaylines{\text{i.e.,} \hfill t^2.  \lambda = 1, ~i.e., t= \pm
  1. \hfill }
$$
Therefore we have an integral basis $[\mu, \mu_1, \mu_2,  \mu_3]$.
\begin{align*}
\text{Consider the module }\quad 
  W = \mathfrak{M} \mu^{-1} &= \left[ 1, \mu,  \mu^{-1}, \mu_2 \mu^{-1},
    \mu_3 \mu^{-1}\right]\hspace{1cm}\\ 
  & = [1, \rho_1, \rho_2, \rho_3] ~(say)
\end{align*}
Then we prove that $W = \mathcal{J}$.
  
$\mathcal{J} \mathfrak{M} \subseteq \mathfrak{M} \Rightarrow
\mathcal{J} \mathcal{W} \subseteq \mathcal{W} \Rightarrow \mathcal{J}
\subseteq \mathcal{W}$. It is enough to show $\mathcal{W} \subseteq
\mathcal{J}$. Let $\rho \in \mathcal{W},  \rho = \gamma_0 + \gamma_1
\rho_1 + \gamma_2 \rho_2 + \gamma_3 \rho_3; \gamma_i \in
\mathscr{O}$. 
  
Then 
  \begin{align*}
    n(\rho) & = (\gamma_0 + \gamma_1 \rho_1 + \gamma_2 \rho_2 + \gamma_3
    \rho_3) ( \gamma_0 + \gamma_1 \bar{\rho}_1 + \gamma_2 \bar{\rho}_2
    + \gamma_3 \bar{\rho}_3)\\ 
    & = \gamma^2_0 + \gamma^2_1 n(\rho_1) + \cdots + \gamma_0 \gamma_1
    t(\rho_1) + \cdots + \gamma_1 \gamma_2 t(\rho_1 \bar{\rho}_2)+
    \cdots 
  \end{align*}
  By the choice of $\mu,  n(\rho)$ is an integer for every $\rho \in
  \mathcal{W}$ and since in the above, all the coefficients are
  integers, it follows that $t(\rho_i)$ and $t(\rho_i \bar{\rho_K})$
  are integers. $(i, k = 1, 2, 3)$. 
  $$
  \rho_i \rho_k = \rho_i t(\rho_k) - \rho_i \bar{\rho}_k \Rightarrow
  t(\rho_i \rho_k) = t(\rho_i) t(\rho_k) - t(\rho_i \bar{\rho}_k). 
  $$
  i.e., $t(\rho_i \rho_k)$ is an integer.
  
  i.e., $D(\mathcal{W}) = |t(\rho_i \rho_k)|$ is an integer and we
  have the relation $D(\mathcal{J}) = |M|^2.  D (\mathcal{W})$, where
  $\mu_i = \sum \limits^3_{j=0} m_{ij} \rho_j (\rho_\circ) = 1) (m_{ij} \in
  \mathscr{O}$, for $\mathcal{J} \subseteq \mathcal{W})$, and if
  $\mathcal{J}= [ \mu_1 \cdots \mu_4]$. 
  
  But\pageoriginale $D(\mathcal{J}) = -1 \Rightarrow |M| = \pm 1$, since
  $D(\mathcal{W})$ is an integer and consequently $M^{-1}$ is
  integral, i.e., $\rho_j = \sum \limits^4_{i=1} \lambda_{ij} \mu_i~
  (j=0 to 3, \rho_0 =1), \lambda_{ij} \in \mathscr{O}$. In other
  words, 
  $$
  \mathcal{W} \subset \mathcal{J},  ~ i.e., ~ \mathcal{J} = \mathcal{W}.
  $$  
  
  Therefore $\mathcal{J} = \mathfrak{M} \mu^{-1}$ or $\mathfrak{M} =
  \mathcal{J}$. $\mu$.  

 ii) Let $\mathcal{J}'' = \mu^{-1} \mathcal{J} \mu =
 \mu^{-1} \mathfrak{M}$ Then $\mathfrak{M} \mathcal{J}'' = \mathfrak{M}
 \mu^{-1} \mathcal{J} \mu = \mathcal{J}. \mathfrak{M} = \mathcal{J}
 \mathcal{J} \mathcal{J}'= \mathcal{J} \mathcal{J}' = \mathfrak{M}$ so
 that $\mathcal{J}'' \subset \mathfrak{K}$. But $\mathfrak{M}
 \mathfrak{K} = \mathfrak{M} \Rightarrow \mu^{-1} \mathfrak{M} \subset
 \mathfrak{K} = \mu^{-1}  \mathfrak{M} \Rightarrow \mathcal{J}'
 \mathfrak{K} = \mathcal{J}''$. Since $1 \in \mathcal{J}'' \mathfrak{K}
 = \mathcal{J}'' \Rightarrow \mathfrak{K} \subset\mathcal{J}''$, i.e.,
 $\mathcal{J}'' = \mathfrak{K}$. Further $\mathfrak{M} \mathcal{J}' =
 \mathfrak{M} \Rightarrow \mathcal{J}'\subset \mathfrak{K} =
 \mathcal{J}'' \Rightarrow \mathcal{J}'=\mathcal{J}''$ since
 $\mathcal{J}''$ is an order and $\mathcal{J}'$ is a maximal
 order. Thus we have proved our theorem that any maximal order
 $\mathcal{J}' = \mu^{-1} \mathcal{J} \mu$, for some $\mu \in Q$. 
  
 The above theorem holds for $Q_p$ over $\bar{k}_p$ also, i.e., Any
 maximal order $\mathcal{J}_p$ in $Q_p/\bar{k}_p$ is of the form
 $\mu^{-1}\mathcal{J}_p \mu$ for some $\mu \in Q_p$, where
 $\mathcal{J}_p$ is the order in $Q_p$, corresponding to $\mathcal{J}$
 in $Q$. 
  
\textbf{8.}~ In the following, we shall introduce ideals for arbitrary orders
and study their multiplicative behaviour. In the first step, we shall
deal with the local case ($p$-adic case). The global ideals will be
defined as the intersection of $p$-adic ideals. This procedure turns out
to be convenient for our purpose, though from a formal algebraic point
of view, a direct definition (Deuring, Algebraic, p.69) of global
ideals mess to be preferable. For maximal orders, both definitions can
be shown to be equivalent. For non-maximal orders, our definition may
be more narrow. But just those ideals defined in our way will interest
us, for example in the application to modular functions. 
  
\begin{defi*}
  A\pageoriginale {\em left ideal} with respect to an order $\mathcal{J}$ in $Q_p
  / \bar{k}_p$ is an $\mathscr{O}_p$ module $\mathfrak{M}$ such that
  $\mathfrak{M} = \mathcal{J}$. $\mu, \mu \in Q_p$ such that $n(\mu) \neq
  0$. $\mathfrak{M}$ can also be written as $\mathfrak{M} =
  \mu. \mathcal{J}'$ where $\mathcal{J}' = \mu^{-1} \mathcal{J}
  \mu$. Then $\mathcal{J}$ is called the {\em left order} of
  $\mathfrak{M}$ and $\mathcal{J}'$, the {\em right order} of
  $\mathfrak{M}$. 
\end{defi*}

Similarly \textit{right ideals} $\nu \mathcal{J}$ can also be defined. 

Any left ideal $\mathcal{J}. \mu = \mu \mathcal{J}',  \mathcal{J}' =
\mu^{-1} \mathcal{J} \mu$ is consequently a right ideal for the order
$\mathcal{J}'$  

\noindent
\textbf{Product of two ideals}. If  $\mathfrak{M} = \mu$.
$\mathcal{J}$ is a right ideal for $\mathcal{J}$ and $\mathfrak{M} =
\mathcal{J}$. $\nu$ left ideal for the same order $\mathcal{J}$, then
the product $\mathfrak{M}.  \mathfrak{N}$ is defined and is equal to
$\mu$. $\nu$. $\mathcal{J}' = \mathcal{J}''$. $\mu \nu$ where
$\mathcal{J}' = \nu ^{-1} \mathcal{J} \nu$, the right order of
$\mathfrak{M}$ and $\mathcal{J}'' = \mu$.  $\mathcal{J}$. $\mu^{-1}$
the left order of $\mathfrak{M}$. 

When multiplication is defined for more than three ideals it is
associative, as a consequence of the associativity of $Q$. Every left
(or right) ideal has an inverse (in the following sense). 

If $\mathfrak{M} = \mathcal{J} \cdot \mu = \mu$. $\mathcal{J}'$ define
$\mathfrak{M}^{-1} = \mu^{-1} \mathcal{J} =
\mathcal{J}'. \mu^{-1}$. Then the product
$\mathfrak{M}$. $\mathfrak{M} ^{-1}$ is defined and $\mathfrak{M},
\mathfrak{M} ^{-1} = \mu \cdot \mu^{-1}$. $\mathcal{J} =
\mathfrak{J}$. Further, $\mathfrak{M} ^{-1}$. $\mathfrak{M} $ is also
defined and $\mathfrak{M}^{-1} \mathfrak{M} = \mu^{-1} \mu
\mathcal{J}' = \mathcal{J}' $. 

\begin{defi*}
  A {\em grouping} is a set $G = \{ A, B, \ldots \}$ with a given
  subset $I(G) = I$ of elements called unit elements of $G$ and two
  mappings $i_\ell$ and $i_r$ of $G$ into $I$ such that 
  \begin{enumerate}[\rm 1.]
  \item $A. B$\pageoriginale is defined if and only if $i_r(A) = i_\ell(B)$.
  \item If $A.  B$ and $B.C$ are defined, then $A(BC)$ and $(AB)C$ are
    defined and are equal. 
  \item $(i_\ell(A)). A$ and $A. (i_r(A))$ are defined and are equal to $A$.
  \item For every $A \in G$, there exists $A^{-1}$ in $G$ such that
    $AA^{-1}$ and $A^{-1}A$ are defined and equal respectively to
    $i_\ell(A)$ and $i_r(A)$. 
  \item If $I_1$ and $I_2$ are elements of $I$, there exists at least
    one element $A \in G$ such that  
    $$
    i_\ell(A) = I_1,  i_r(A) = I_2.
    $$
  \end{enumerate}
\end{defi*}

\noindent
\textbf{Example of a groupoid}

Let $G$ be the set of left and right ideals with respect to the set of
all maximal orders $\{ \mathcal{J}_j \} = I \, (in Q_p / \bar{k}_p)$
which we take to be the set of unit elements, together with the
mappings $i_\ell, i_r$ given by  
$$
\displaylines{\hfill 
  i_\ell(A)  = \mathcal{J}_\ell,  ~\text{where}~ A =
  \mathcal{J}_\ell. \mu \hfill \cr 
  \text{and}\hfill 
  i_r(A)  = \mathcal{J}_r,  ~\text{where}~ A = \mu
  \mathcal{J}_r.\quad \hfill }
$$
We now verify that the axioms for a groupoid are satisfied.
\begin{enumerate}
\item follows from the definition of multiplication.
\item follows from the associativity of multiplication in $Q_p$.
\item $i_\ell(A)A =  \mathcal{J}_\ell \cdot  \mathcal{J}_\ell \mu =
  \mathcal{J}_\ell \cdot \mu = A$.

  $ A(i_r A) = \mu. \mathcal{J}_r\cdot \mathcal{J} = \mu. \mathcal{J}_r = A$.
\item follows from the definition of the inverse.
\item Let $\mathcal{J}_1, \mathcal{J}_2$ be two maximal orders, say
  $$
  \displaylines{\hfill 
  \mathcal{J}_1 =\mu^{-1}_1  \mathcal{J} \mu_1,  \mathcal{J}_2 =
  \mu^{-1}_2 \mathcal{J} \mu_2\hfill \cr 
  \text{where}\hfill  \mathcal{J} = \left\{ \alpha :
  \alpha \cong \begin{pmatrix}a & b \\ c & d \end{pmatrix},  a,
  b, \ldots \in \mathscr{O}_p \right\}\hfill }
  $$ 
\end{enumerate}
Take\pageoriginale 
  $$
  \displaylines{\hfill 
  A = \mathcal{J}_1\cdot  \mu^{-1}_1 \mu_2 = \mu^{-1}_1 \mu_2\cdot
  \mathcal{J}^1\hfill \cr 
  \text{where}\hfill \mathcal{J}^1 = (\mu^{-1}_1
  \mu_2)^{-1} \cdot \mathcal{J}_1 \mu^{-1}_1 \mu_2 =
  \mathcal{J}_2\qquad  \hfill } 
  $$
Then $i_\ell(A) = \mathcal{J}_1$, $i_r (A) = \mathcal{J}_2$.

If the ideal $A$ is such that $A = \mathcal{J}_\ell, \mu = \mu
\mathcal{J}_\ell$, i.e., $\mathcal{J}_\ell = \mathcal{J}_r$ then we say that
$A$ is an \textit{ambiguous} ideal or \textit{two-sided}. 

If $Q_p /\bar{K}_p$ is a matrix algebra, then since there exists
only one maximal order $\mathcal{J}_p$, all ideals are two-sided and
they form a group with the unite element $\mathcal{J}_p$. 
\begin{defi*}
  An ideal $\mathfrak{M}$ is called an {\em integral} ideal if
  $\mathfrak{M} \subset \mathcal{J}_1$. 
\end{defi*}
$$
\mathfrak{M} \subset \mathcal{J}_\ell \Leftrightarrow \mathfrak{M}
\subset \mathcal{J}_r 
$$
\begin{enumerate}[(i)]
\item Let $\mathfrak{M} \subset \mathcal{J}_\ell$, then since
  $\mathcal{J}_\ell \mathfrak{M} = \mathfrak{M},  \mathfrak{M}
  \mathfrak{M} \subset \mathfrak{M}$. But $\mathfrak{M}\cdot
  \mathcal{J}_r = \mathfrak{M} 
  \Rightarrow \mathfrak{M} \subset \mathcal{J}_r$ since if $\mathscr{K}
  = \{\xi:\mathfrak{M} \xi \subset \mathfrak{M}, \xi \in Q_p\}$,
  then $\mathfrak{M} \mathscr{R} = \mathfrak{M} \Rightarrow
  \mu^{-1}. \mathfrak{M} \mathscr{R} = \mu^{-1} \mathfrak{M}$ 
  
  i.e., \quad $\mathcal{J}_r \mathfrak{K} = \mathcal{J}_r$ which implies that
  $\mathfrak{K} \subset \mathcal{J}_r$, or $\mathfrak{M} \subset
  \mathscr{R} \subset \mathcal{J}_r$. 
\item If $\mathfrak{M} \subset \mathcal{J}_r $ then $\mathfrak{M}
  \subset \mathcal{J}_\ell$ is similarly proved. 
\end{enumerate}

\begin{defi*}
  Let $\mathfrak{M} = \mathcal{J} \mu = \mu \mathcal{J}'$ be an
  ideal. We define the norm of $\mathfrak{M}$ or be $n(\mathfrak{M}) =
  (n(\mu))$, the principal ideal generated by $n(\mu)$ over $\mathscr{O}_p$. 
\end{defi*}

If $\mathfrak{M}_1$.  $\mathfrak{M}_2$ is defined, then $n
(\mathfrak{M}_1 \mathfrak{M}_2) = n (\mathfrak{M}_1)$. $
(\mathfrak{M}_2)$, for, if $\mathfrak{M}_1 = \mu_1 \mathcal{J}_1$,
$\mathfrak{M}_2 = \mathcal{J}_1 \mu_2$, then since $\mathfrak{M}_1
\mathfrak{M}_2 = \mu_1 \mu_2 \mathcal{J}'_1$ we have $n(\mathfrak{M}_1
\mathfrak{M}_2) = (n(\mu_1.\mu_2)) = (n(\mu_1)$. $n(\mu_2)) =
(n(\mu_1)). n(\mu_2)) = n(\mathfrak{M}_1)$. $n(\mathfrak{M}_2))$ 

We\pageoriginale will now find all integral ideals of a given maximal order, and
having norm $(p^n)$. 

To do this, we consider two cases:
\begin{enumerate}[(1)]
\item  Let $Q_p /\bar{k}_p$ be a division algebra. Then it has a
  unique maximal order $\mathcal{J}_p$ and there exists $\pi \in Q_p$
  such that $\pi^2 = p$ (since $\bar{k}_p (\sqrt{p})$ being a
  quadratic extension of $\bar{k}_p$, is a splitting field of $Q_p$
  and by Theorem \ref{chap1:sec2:thm3}, \S \ref{chap1:sec1}). Further $\pi \in \mathcal{J}_p$ since
  $n(\pi) = -p$. 
\end{enumerate}

We now have the
\begin{theorem}\label{chap1:sec2:thm6} % them 6
  \begin{enumerate}[\rm (i)]
  \item $\mathscr{P} = \mathcal{J}_p \pi$ {\em is the only integral
    ideal with }$n (\mathscr{P}) = (p)$, and  
  \item $\mathscr{P}^n = \mathcal{J} \cdot \pi^n$ {\em is the only
    integral ideal for which} $n(\mathscr{P}^n) = (p^n)$. 
  \end{enumerate}
\end{theorem}

\begin{proof}
\begin{enumerate}[(i)]
\item Let $\mathscr{P}^1 = \mathcal{J}_p.  \pi'$ be another integral
  ideal such that $n(\pi^1) = p.u_1,  u_1$ a unit, i.e.,
  $n(\mathscr{P^1}) = (p)$. Then $\mathscr{P}^1,  \mathscr{P}^{-1} =
  \mathcal{J}_p \pi' \pi^{-1}$. Let $\pi' \pi^{-1} = \mu$, then
  $n(\mu) = \dfrac{n(\pi^1)}{n(\pi)} = \dfrac{p.u_1}{-p} = u_2$, a
  unit. This means that $\mu \in \mathcal{J}_p$ and also $\mu^{-1} \in
  \mathcal{J}_p	$ since $\mu^{-1} = \dfrac{\bar{\mu}}{n(\mu)} =
  \bar{\mu}$. unit, and $\bar{\mu} \in \mathcal{J}_p$. Now  
  $$
  \mathcal{J}_p \mu \subseteq \mathcal{J}_p \Rightarrow \mathcal{J}_p \subseteq
  \mathcal{J}_p.  \mu^{-1} \subseteq \mathcal{J}_p 
  $$
  because $\mu^{-1} \in \mathcal{J}_p$ i.e., $\mathcal{J}_p =
  \mathcal{J}_p$. $\mu^{-1} ~\text{or}~ \mathcal{J}_p \mu = \mathcal{J}_p$. So
  $\mathcal{J}_p.  \mu = \mathcal{J}_p = \mathscr{P}^1
  \mathscr{P}^{-1}$, i.e., $\mathscr{P}^1 = \mathscr{P}$. 
\item Let $\mathscr{P}^1 = \mathcal{J}_p.  \pi'$ be an integral ideal
  such that $n(\mathscr{P}^1) = (p^n)$. Then $\mathscr{P}^1
  (\mathscr{P}^n)^{-1} = \mathcal{J}_p \pi'$. $(\pi^n)^{-1}$. If $\mu
  = \pi' (\pi^n)^{-1}$ then $n(\mu) = unit \Rightarrow \mu,  \mu^{-1}
  \in \mathcal{J}_p$ as before. 
  $$
  \mathcal{J}_p \mu = \mathcal{J}_p = \mathscr{P}^1
  (\mathscr{P}^n)^{-1},  i.e., \mathscr{P}^1 = \mathscr{P}^n. 
  $$
\end{enumerate}

(2)\pageoriginale \quad Let $Q_p /\bar{k}_p \cong \mathfrak{M}_2 (\bar{k}_p);
  \mathcal{J}_p = \{ \alpha : \alpha \cong \begin{pmatrix}a & b \\ c& 
    d \end{pmatrix}, a, b, \ldots,  \in \mathscr{O}_p \}$. We will now
  find all the 
  integral ideals of $\mathcal{J}_p$ which have the norm $(p^n)$. 
\end{proof}

Let $\mathfrak{M}_p = \mathcal{J}_p. \mu_p$ be an integral ideal,
i.e., $\mu_p \in \mathcal{J}_p$ and such that $n(\mathfrak{M}_p) =
(n(\mu_p)) = (p^n)$,  i.e., $n(\mu_p) = p^n$.$e$, $e$ unit. If
$\varepsilon_p$ corresponds to an integral matrix and such  that
$n(\varepsilon_p)$ is a unit, then $\varepsilon^{-1}_p$ also
corresponds to an integral matrix, i.e.,
$\underline{\varepsilon^{-1}_p \in \mathcal{J}_p}$ or $\mathcal{J}_p
\varepsilon_p = \mathcal{J}_p$ (because $\varepsilon_p \in
\mathcal{J}_p$), hence $\mathcal{J}_p \varepsilon_p \mu_p =
\mathcal{J}_p \mu_p$. Now choose $\varepsilon_p$ such that 
\begin{equation*}
  \varepsilon_p \mu_p \cong
  \begin{pmatrix}
    m_{11} & m_{12}\\ 
    0 & m_{22}
  \end{pmatrix}
\end{equation*}
then $n(\varepsilon_p \mu_p) = m_{11} m_{12} = n(\mu_p) = e.p^n$, e
unit. Let $m_{22} = e_{11}. p^{n_1}$,  $m_{22} = e_{22} p^{n_2}$, where
$n_1 + n_2 = n$, and $e_{11}, e_{22}$ are units.

Now
$$
\begin{pmatrix} e^{-1}_{11} & 0 \\  0 & e^{-1}_{22} \end{pmatrix} 
\begin{pmatrix} e_{11} p^{n_1} & m_{12}\\ 0 &  e_{22}p^{n_2}\end{pmatrix} = 
\begin{pmatrix} p^{n_1}& e^{-1}_{11} m_{12} \\ 0 &  p^{n_2} \end{pmatrix} = 
\varepsilon'_p \mu_p, \quad \text{say}. 
$$
In the product 
$$
\begin{pmatrix} 1 & t \\ 0 & 1\end{pmatrix} 
  \begin{pmatrix} p^{n_1} & e^{-1}_{11} m_{12} \\ 0&  p^{n_2} \end{pmatrix}
  = \begin{pmatrix}p^{n_1} & e^{-1}_{11}m_{12}+ p^{n_2}.t\\ 0& p^{n_2} \end{pmatrix} 
$$
we choose $t$ such that $0 \le m'_{12} < p^{n_2} = e^{-1}_{11} m_{12} + tp^{n_2}$.

Then by the matrix $\begin{pmatrix}1&  t \\ 0 & 1 \end{pmatrix},
\varepsilon'_p \mu_p \rightarrow \varepsilon''_p \mu_p
\cong \begin{pmatrix} p^{n_1} &  m'_{12} \\ 0 &
  p^{n_2} \end{pmatrix}$. 

Now\pageoriginale $\mathcal{J}_p \mu_p = \mathcal{J}_p \varepsilon'_p \mu_p =
\mathcal{J}_p.  \varepsilon''_p \mu_p$. 

Therefore the set of all integral ideals with norm $(p^n)$ is the set
all $\mathcal{J}_p. \mu : \mu \cong \begin{pmatrix} p^{n_1} & m'_{12}
  \\ 0 & p^{n_1} \end{pmatrix}, n_1 + n_2 = n, 0 \le m'_{12} <
p^{n_2}$. Further $\mu_1 \neq \mu_2 \Rightarrow \mathcal{J}_p.  \mu_1
\neq \mathcal{J}_p \mu_2$; for,  if $\mathcal{J}_p \mu_1 =
\mathcal{J}_p \mu_2$, then $\mathcal{J}_p  \mu_1 \mu^{-1}_2 =
\mathcal{J}_p$, i.e., $\mathcal{J}_p \mu = \mathcal{J}_p, $ where $\mu
= \mu_1 \mu^{-1}_2$. From this we obtain $\mathcal{J}_p =
\mathcal{J}_p. \mu^{-1}$, or $\mu^{-1} \in \mathcal{J}_p$ since $1 \in
\mathcal{J}_p, i.e., n(\mu) = $ unit so that  
$$
\mu \sim \begin{pmatrix} e_{11} & e_{12} \\ e_{21}&
  e_{22} \end{pmatrix} \qquad \text{(say)}. 
$$ 
Now $\mu \mu_2 = \mu_1$ implies that
$$
\displaylines{\hfill 
\begin{pmatrix} e_{11} & e_{12} \\ e_{21} & e_{22} \end{pmatrix} 
\begin{pmatrix} p^{n_1} & m_{12} \\0 & p^{n_2} \end{pmatrix} = 
\begin{pmatrix}p^{n'_1} & m'_{12} \\ 0 & p^{n'_2} \end{pmatrix} \hfill \cr
\text{if}\hfill 
\mu_2 \sim 
\begin{pmatrix}p^{n_1} & m_{12} \\ 0 &  p^{n_2} \end{pmatrix} 
\quad \text{and}\quad  \mu_1 \sim 
\begin{pmatrix}p^{n'_1} & m'_{12} \\ 0&  p^{n'_2} \end{pmatrix} \hfill }
$$
Therefore $e_{21} p^{n_1} = 0$ or $e_{21} = 0$ and $e_{11} p^{n_1} =
p^{n'_1}$, $e_{22}$. $p^{n_2} = p^{n'_2}$ imply that $e_{11} = e_{22} =
1, i.e$, 
$$
\mu_1 \sim \begin{pmatrix} 1 & e_{12} \\ 0  &
  1\end{pmatrix}\begin{pmatrix} p^{n_1 }& m_{12} \\ 0 &
    p^{n_2}\end{pmatrix}= \begin{pmatrix} p^{n_1 } & m_{12} + e_{12}
    p^{n_2} \\ 0 & p^{n_2} \end{pmatrix} 
$$
So, $m'_{12} = m_{12}+ e_{12} p^{n_2}$ or $m_{12} \equiv m'_{12} \pmod
{p^{n_2}}$. But, 
since  $0 \le m_{12}, m'_{12} < p^{n_2}$ we must have $m_{12} =
m'_{12}$. 

Hence,\pageoriginale the number of integral ideals with norm $(p^n)$ is equal to $1
+ p + \cdots + p^n$, the number $1$ corresponding to the values $n_1 =
n, n_2 = 0$, the number $p$ corresponding to the values $n_1 = n-1,
n_2 = 1$ and so on. 

\textbf{9.}~ We will now extend our results from the local case to the global
case. Let now $Q$ be a quaternion algebra over the rational number field
$k$, and let $\mathcal{J}$ be any order in $Q$. We have already seen
that $\mathcal{J} = Q \cap \mathcal{J}_2 \cap \mathcal{J}_3 \cdots
\cap \mathcal{J}_p \cap \cdots$Analogously, we define left ideals
$\mathfrak{M}$ with respect to the order $\mathcal{J}$ as
$\mathfrak{M} = Q \cap \mathcal{J}_2 \mu_2 \cap \cdots \cap \mathcal{J}_p
\mu_p \cap \cdots$ where $\mathcal{J}_p \mu_p = \mathcal{J}_p$ for
almost all $p$. We then have the following  

\begin{theorem}\label{chap1:sec2:thm7} % them 7
  \begin{enumerate}[\rm (i)]
  \item $\mathfrak{M}$ {\em is a finite} $\mathscr{O}-$module.
  \item $\mathfrak{M}_p = \mathcal{J}_p$. $\mu_p$ {\em where}
    $\mathfrak{M}_p$ {\em is the finite} $\mathscr{O}_p-${\em  module},
    {\em with the same basis elements as }$\mathfrak{M}$ {\em over}
    $\mathscr{O}$  
  \end{enumerate}
\end{theorem}

\begin{proof}
  $(i)$ We may, without loss of generality, assume that 
  \begin{equation}
    \mathcal{J}_p \mu_p \supseteq \mathcal{J}_p \tag{A}
  \end{equation}
  for all $p$. This is explained as follows:

  Now, $\mathcal{J}_p \mu_p = \mathcal{J}_p$ for all except a finite
  number of primes $p$, (say) $p_1, \ldots,  p_r$. So, let
  $\mathcal{J}_p \mu_p \nsupseteq \mathcal{J}_p, p = p_1, \ldots,  p_r$ and let
  $\mathcal{J}_p \mu_p = [\nu_1 \cdots \nu_4]$ and $\mathcal{J}_p =
  [L_1 \,L_2 \,L_3 \,L_4]$. 
\end{proof}

We can then write $L_i= \sum\limits^4_{k=1} m_{ik} \nu_k, m_{ik}
\in \bar{k}_p$. Choosing $n$ sufficiently large so that $m_{ik}. p^n$
are $p-$adic integers, $p^n. L_i \in \mathcal{J}_p \mu_p \Rightarrow
\mathcal{J}_p \mu_p \supseteq \mathcal{J}_p. p^n$, i.e., $\mathcal{J}_p
\mu'_p \supseteq \mathcal{J}_p$, $\mu'_p = \dfrac{\mu_p}{p^n}$.  

So we have $\mathcal{J}_{p_i} \mu_{p_i} \supseteq \mathcal{J}_{p_i}$ where
$\mu'_{p_i} = \dfrac{\mu_{p}}{{}_{p_i} n_i}$. 

Let\pageoriginale $m = \prod \limits^{r}_{i=1} pi^{n_i} = pi^{n_i} u_{pi}$(say)
$u_{pi}$, a $p_i$-adic unit. 

Then
$$
\mathcal{J}_{p_i} \frac{\mu_{p_i}}{m} =
\mathcal{J}_{p_i}\cdot \frac{\mu_{p_i}}{{}_{p_i}{n_i}\cdot u_{p_i}} =
\mathcal{J}_{p_i}\cdot \mu'_{p_i}(u_{p_i})^{-1} \supseteq \mathcal{J}_{p_i} 
$$
for all $i$, since $\mathcal{J}_{p_i} \mu'_{p_i} \supseteq \mathcal{J}_{p_i}$
and $u_{p_i} \in \mathcal{J}_{p_i}$. 

If $\mathfrak{M}' = Q \cap \cdots \cap \mathcal{J}_{p_i}
\dfrac{\mu_{p_i}}{m} \cap \cdots,  $ then $\mathfrak{M}'$ satisfies the
condition $(A)$ and if we prove that $\mathfrak{M}'$ is a finite
$\mathscr{O}-$ module, then since $\mathfrak{M} = m \mathfrak{M}'$, it
will follow that $\mathfrak{M}$ itself is a finite $\mathscr{O}-$
module. 

Now, we assume that $\mathfrak{M}$ itself satisfies condition
$(A)$. Then $\mathfrak{M} \supseteq \mathcal{J}$. If $\mathfrak{M} =
\mathcal{J}$, there is nothing to prove, so let $\mathfrak{M} \supset
\mathcal{J}$ properly, i.e., for at least one $p, \mathcal{J}_{p}
\mu_p \supset \mathcal{J}_{p}$, properly. We shall now show that
$\mathfrak{M}$ can be obtained from $\mathcal{J}$ by a finite number
of adjunctions and hence $\mathfrak{M}$ is a finite $\mathscr{O}-$
module. 

Let $\nu_p \in \mathcal{J}_p \mu_p$ and $\notin \mathcal{J}_{p} = [L_1\,
  L_2 \,L_3 \,L_4]$ over $\mathscr{O}_p$ if $\mathcal{J} = [L_1 \cdots
  L_4]$ over $\mathscr{O}$. Then $\nu_p = L_1 n_1 + \cdots + L_4 n_4,
n_i \in \bar{k}_p$ and at least one $n_i$(say) $n_1$ is not a p adic
integer. 

$n_i = n'_i + n''_i,  n''_i \in \mathscr{O}_p$(say). Then $n'_1 \neq 0;
n'_i$ are rational, $\nu_p = \left(\sum \limits^4_{j=1} L_j
n'_j\right) +\left(\sum 
\limits^4_{j=1} L_j n''_j \right) = \nu''_p + \nu''_p$ so that $\nu''_p \in
\mathcal{J}_p$ and $\nu'_p \notin \mathcal{J}_p$. If
$\mathcal{J}' = [\mathcal{J}, \nu'_p]$ then $\mathcal{J}'_p \subset
\mathcal{J}_p \subset \mathcal{J}_p \mu_p$. Again if $\mathcal{J}_p
\neq \mathcal{J}_p. \mu_p$, as before we adjoin $\nu^{(2)'}_p$ and so
on. But $\mathcal{J}_p \mu_p$ being a finite $\mathscr{O}-$ module,
there can only be a finite number of $\mathcal{J}'_p$ between
$\mathcal{J}_p$ and $\mathcal{J}_p \mu_p$ so\pageoriginale that we reach
$\mathfrak{M}^{(1)} = [\mathcal{J},  \nu^{(1)'}_{p_1}$,
  $\nu^{(2)'}_{p_1} \cdots \nu^{(k)'}_{p_1}]$ whence
$\mathfrak{M}^{(1)}_{p_1} = \mathcal{J}_{p_1} \mu_{p_1}$ (putting $p
=p_1$, one of the primes for which the inclusion is proper). Now, from
the decomposition, $\mathcal{J} = Q \cap_p \mathcal{J}_p$ we obtain on
adjunction of these elements to each component, 
$$
\mathfrak{M}^{(1)} = Q \bigcap_{p \neq p_1} \mathcal{J}_p \cap
\mathcal{J}_{p_1} ~\text{since}~ \nu^{(i)'}_{p_1} \in Q \bigcap_{p \neq p_1}
\mathcal{J}_p. 
$$
Doing the same for $\mathfrak{M}^{(1)}$ as we did for $\mathcal{J}$,
with respect to prime $p_2$, we obtain the module 
$$
\mathfrak{M}^{(2)} = \bigcap_{p \neq p_1, p_2} \mathcal{J}_p \cap
\mathcal{J}_{p_1} \mu_{p1} \cap \mathcal{J}_{p_2} \mu_{p_2} 
$$
i.e., if $\mathcal{J}_{p_2} \mu_{p_2} \supset \mathcal{J}_{p_2}$
properly; consider 

$\mathfrak{M}^{(2)} = [\mathfrak{M}^{(1)}, \nu^{(1)'}_{p_2} \cdots
\nu^{(s)'}_{p_2}]$ so that as before, the adjunction of these
elements keep $Q,  \mathcal{J}_p (p \neq p_1, p_2)$ and
$\mathcal{J}_{p_1} \mu_{p_1}$ fixed and consequently
$\mathfrak{M}^{(2)}$ has the above form. Further
$\mathfrak{M}^{(2)}_{p_2} = \mathcal{J}_{p_n} \mu_{p_2}$. Proceeding
in this manner, we obtain finally $\mathfrak{M}^{(r)} = Q \cap
\mathcal{J}_{p_1} \mu_{p_1} \cap \cdots \cap \mathcal{J}_{p_r}
\mu_{p_r} \bigcap \limits_{p \neq p_i \cdots p_r} \mathcal{J}_p =
\mathfrak{M}$, by definition. 

Thus we have proved that $\mathfrak{M}$ is a finite $\mathscr{O}-$ module.

\begin{enumerate}
\item[ii]
  \begin{enumerate} [(a)]
  \item  For $p = p_i,  i=1 $ to $r, \mathfrak{M}_{p_i}=
    \mathfrak{M}^i_{p_i} = \mathcal{J}_{pi}$. $\mu_{p_i}$ by
    constructions of $\mathfrak{M}^{(i)}$  
  \item For $p \neq p_i, i = 1$ to $r, \mathcal{J}_{p} \mu_{p} =
    \mathcal{J}_{p}$ so that $\mathfrak{M} \subset \mathcal{J}_{p}
    \mu_p = \mathcal{J}_{p}$, i.e., $\mathfrak{M}_p \subset
    \mathcal{J}_{p}$. 
  \end{enumerate}
\end{enumerate}

Further\pageoriginale $\mathfrak{M} \supset \mathcal{J} m$ (for some integer $m$
which is a p- adic unit) so that $\mathfrak{M}_p \supset
\mathcal{J}_{p}$. In other words $\mathfrak{M}_p = \mathcal{J}_{p}=
\mathcal{J}_{p} \mu_p$. 
\begin{note}
  From the above theorem we may deduce that $\mathfrak{M} = Q \bigcap
  \limits_{p} \mathfrak{M}_p$, which is the analogue of the expression
  for an order $\mathcal{J}$ in $p. 19$. 
\end{note}

\noindent
\textbf{Product of ideals}

If $\mathfrak{M} = Q \bigcap \limits_{p} \mathcal{J}_{p} \mu_{p}$ and
$\mathfrak{N} = Q \bigcap \limits_{p} \mathcal{J}'_{p} \nu_p$ are two
ideals where $\mathcal{J}'_{p} = \mu^{-1}_{p} \mathcal{J}_{p} \mu_p$,
then the product $\mathfrak{M} \mathfrak{N}$ is defined and is equal
to the ideal $Q \bigcap \limits_{p} \mathcal{J}_{p}
\mu_{p}\mathcal{J}'_{p} \nu_p = Q \bigcap \limits_{p} \mathcal{J}_{p}
\mu_{p} \nu_p$. 
\begin{theorem}\label{chap1:sec2:thm8} % them 8
  If $\mathfrak{M} = [\mu_1,  \ldots,  \mu_4], \mathfrak{N} = [ \nu_1
    \cdots \nu_4]$ are two {\em ideals and the product} $\mathfrak{M}
  \mathfrak{N}$ {\em is defined,  then } $\mathfrak{M} \mathfrak{N} =
           [\cdots i k, \ldots]$ ({\em the product module}). 
\end{theorem}

\begin{proof}
  Let $\mathfrak{M}  = Q \bigcap \limits_{p} \mathcal{J}_p \mu_p$ and
  $\mathfrak{M} = Q \bigcap \limits_{p} \mathcal{J}'_p \nu_p$. Then the
  ideal product $\mathscr{R}= \mathfrak{M} \mathfrak{N} = \bigcap
  \limits_{p} \mathcal{J}_p \mu_p \nu_p$. Let $\mathscr{R}= [\rho_1,
    \rho_2 $, $\rho_3$, $\rho_4]$. Then $\mathscr{R}_p = (\mathfrak{M}
  \mathfrak{N})_p = \mathcal{J}_p \mu \nu_p = \mathcal{J}_p \mu
  \mathcal{J}'_p \nu_p = \mathfrak{M}_p \mathfrak{N}_p$ i.e., $[\rho_k]_p
  = [\mu_i \nu_j]_p \Rightarrow \mu_i \nu_j = \sum \limits^4_{k=1}
  m^{(k)}_{ij} \rho_k, m^k_{ij} \in \mathscr{O}_p$ for all $p$.  
\end{proof}

We know already that $m^{(k)}_{ij} \in k$. Combining these two, we see
that $m^k_{ij} \in \mathscr{O}$. In other words $\mathscr{R}= [\mu_i\,
  \nu_j]$.  

We shall consider some special cases of the product of two ideals:
\begin{enumerate}[i)]
\item $\mathfrak{M} = \mathcal{J}, \Rightarrow \mathfrak{M}\mathfrak{N} =
  \mathfrak{N}$, i.e., $\mathcal{J} \mathfrak{N} = \mathfrak{N};
  \mathcal{J}$ is called the left order of $\mathfrak{N}$. 
\item Defining $\mathcal{J}' = Q \bigcup \limits_{p} \mu^{-1}_p
  \mathcal{J}_p \mu_p$, where $\mu^{-1}_p \mathcal{J}_p \mu_p =
  \mathcal{J}_p$ for almost all $p$, it can be proved as for
  $\mathfrak{M}$, that $\mathcal{J}'$ is a finite
  $\mathscr{O}$-module.\pageoriginale By definition, $\mathcal{J}'$ is a ring
  containing $1$ and $\mathcal{J}' \supset \mathfrak{M} \mathcal{J}$
  for some integer m so that $\mathcal{J}'$ is of rank $4$. Therefore,
  by our second definition of an order, $\mathcal{J}'$ is an order. 
  
  Now, if $= Q \bigcap \limits_{p} \mathcal{J}_p \mu_p$, then
  $\mathfrak{M} \mathcal{J}'$ is defined and $\mathfrak{M} \mathcal{J}'
  = \mathfrak{M}; \mathcal{J}'$ is called the right order for
  $\mathfrak{M}$. 
\item If $\mathfrak{M} = Q \bigcap \limits_{p} \mathcal{J}_p \mu_p$,
  we defined its inverse $\mathfrak{M}^{-1} = Q \mu^{-1}_p
  \mathcal{J}_p$ (a right ideal for $\mathcal{J}) \mathfrak{M}^{-1}$
  can be rewritten $Q \bigcap \limits_{p} \mathcal{J}'_p \mu^{-1}_p $
  (a left ideal for $\mathcal{J}'$). Then $\mathfrak{M}
  \mathfrak{M}^{-1}$ is defined and $= Q \bigcap \limits_{p}
  \mathcal{J}_p = \mathcal{J}$. 

  Similarly $\mathfrak{M}^{-1} \mathfrak{M}$ is defined and $= \mathcal{J}'$.
\item When the product of more than two ideals is defined, this is
  easily seen to be associative, for let $\mathfrak{M}= Q \bigcap \limits_{p}
  \mathcal{J}_p \mu_p,  \mathfrak{N} = Q \bigcap \limits_{p}
  \mathcal{J}'_p \nu_p$ and $\vartheta = Q \bigcap \limits_{p}
  \mathcal{J}''_p \lambda_p$. If $(\mathfrak{N})$ and $(\mathfrak{M}
  \mathfrak{N}) \vartheta$ are to be defined, then $\mathcal{J}'_p =
  \mu^{-1}_p \mathcal{J}_p \mu_p$ and $\mathcal{J}''_p = (\mu_p
  \nu_p)^{-1} \mathcal{J}_p (\mu_p \nu_p)$, so that $(\mathfrak{M}\mathfrak{N})
  \vartheta = Q \bigcap \limits_{p} \mathcal{J}_p$ $\mu_p \nu_p
  \lambda_p$. 
\end{enumerate} 

$\mathcal{J}''_p = \nu^{-1}_p \mu^{-1}_p \mathcal{J}_p \mu_p \nu_p =
\nu^{-1}_p \mathcal{J}'_p \nu_p$ implies that $\mathfrak{N} \vartheta$
is defined and $=Q \bigcap \limits_{p} \mathcal{J}_p  \nu_p
\lambda_p$. Consequently $\mathfrak{M} (\mathfrak{N} \vartheta)$ is
also defined and $Q \bigcap \limits_{p} \mathcal{J}_p \mu_p \nu_p
\lambda_p$ i.e., $(\mathfrak{M} \mathfrak{N})\vartheta = \mathfrak{M}
(\mathfrak{N} \vartheta)$. 

Form the above considerations, it follows at once that the set of
ideals defined above with the set of all orders $\mathcal{J}' = Q
\bigcap \limits_{p}\mu^{-1}_p \mathcal{J}_p \mu_p$ ($\mu^{-1}_p
\mathcal{J}_p \mu_p$ =$\mathcal{J}_p $ for almost all $p$),  as the
class of unit elements forms a groupoid. This particular choice of
orders becomes necessary for the fifth axiom of the groupoid. 

\noindent
\textbf{Norm of an ideal :}\pageoriginale Let $\mathfrak{M}$ be an ideal,
$\mathfrak{M} =  Q \bigcap \limits_{p}\mathcal{J}_p \mu_p,
\mathcal{J}_p \mu_p = \mathcal{J}_p $ for almost all $p$. Then we
define the norm $n(\mathfrak{M})$ of $\mathfrak{M}$, to be the
principal ideal $\prod \limits_{p} (n (\mu_p))$ where by this we mean
the ideal $(\prod \limits^r_{i=1} p_i^{n_i})$ generated over
$\mathscr{O}. n(\mu_{p_i}) = p_i^{n_i}$. $u_i,)$ a $p_i$ adic unit, and
$p_1, \ldots p_r$ being the primes for which $\mathcal{J}_p \mu_p \neq
\mathcal{J}_p $. For primes other than $p_i,  n(\mu_p)$ is a unit. We
may also define $n(\mathfrak{M}) = (m)$ generated over $\mathscr{O}$,
where m is the g.c.d of all $n(\mu), \mu \in \mathfrak{M}$. 

But if $\mathfrak{M} = [\nu_1 \cdots \nu_4]$, then $m = $g.c.d. of the
coefficients of the quadratic from 
$$
n(\mu) = \sum^4_{i=1} n(\nu_i) x^2_i + \sum^4_{i=1} t(\nu_i \bar{\nu}_j) x_i x_j.
$$
For $m = \prod \limits_{p}$ g.c.p adic divisor of the same
coefficients $= \prod \limits_{p} n(\mu_p)$, since $n(\mu_p) =$
g.c.p-adic divisor of the coefficients of the above quadratic form
with $x'_i$ s p - adic integers instead of being rational integers,
for $\mathcal{J}_p \mu_p = \mathfrak{M}_p = [\nu_1 \cdots \nu_4]_p$. 

Hence, both the definitions are equivalent.

\noindent
\textbf{Integral ideals} A left ideal $\mathfrak{M}$ for an order
$\mathcal{J}$ is said to be integral if $\mathfrak{M}  \subseteq
\mathcal{J}$. This is equivalent to saying that $\mathfrak{M} \subseteq
\mathcal{J}'$ (right order of $\mathfrak{M}$), for $\mathfrak{M} \subseteq
\mathcal{J} \Rightarrow \mathfrak{M}_p \subseteq \mathcal{J}_p$ for all p,
i.e., $\mathfrak{M}_p \subseteq \mathcal{J}'_p$, since $\mathcal{J}'_p$ is a
right order for $\mathfrak{M}_p$. In other words, $\mathfrak{M} \subseteq
\mathcal{J}'$. 

$\mathfrak{M}$ integral $\Rightarrow (n(\mathfrak{M}))$ an integral
ideal, since each $n(\mu), \mu \in \mathfrak{M} \subset \mathcal{J}$
is an integer, the g.c.d. is also an integer. Let\pageoriginale $\mathcal{J}$ be
maximal order in $Q/k$, then  
$$
 \mathcal{J}' = Q \cap \cdots \cap \mu^{-1}_p  \mathcal{J}_p \mu_p \cap \cdots
$$
(where $\mu^{-1}_p  \mathcal{J}_p \mu_p = \mathcal{J}_p$ for almost al
 $p$), is also maximal; for $\mathcal{J}$ maximal $\Leftrightarrow
 \mathcal{J}_p$ maximal for every $p$, i.e., $\mathcal{J}'_p = \mu^{-1}_p
 \mathcal{J}_p \mu_p$ is maximal for every $p$, or $\mathcal{J}'_p$ is
 maximal. Conversely, we have the  

\begin{theorem}\label{chap1:sec2:thm9} % them 9
  If $\mathcal{J}''$ is any maximal order, then
  $$
  \mathcal{J}'' = Q \cap \cdots \cap \mathcal{J}''_p \cap \cdots
  $$
  where $\mathcal{J}''_p = \mathcal{J}_p$ for almost all $p$.
\end{theorem}

\begin{proof}
  We have $\mathcal{J}'' = Q \cap \cdots \cap \mathcal{J}''_p \cap
  \cdots$. Since $\mathcal{J}''$ is maximal, $\mathcal{J}''_p$ is
  maximal for every $p$, and hence there exists a $\mu''_{p} \in Q_p$,
  such that $n(\mu''_p) \neq 0$ and such that $\mathcal{J}''_p =
  \mu^{''-1}_p \mathcal{J}_p \mu'_{p}$. We have now only to show that
  $\mathcal{J}''_p = \mu^{''-1}_p \mathcal{J}_p \mu''_p = \mathcal{J}_p$
  for almost all $p$. 
\end{proof}

Let $\mathcal{J} = [L _1,  \ldots,  L_4], \mathcal{J}'' =
[L''_1,  \ldots,  L''_4]$. We can write 
$$
L''_i = \sum^4_{k=1} c_{ik} L_k,  c_{ik} \in k.
$$

Let $p$ be a prime which does not divide the denominator of any
$c_{ik}$. Then $c_{ik}$ are all p- adic integers, $i.e.,
\mathcal{J}''_p \subset \mathcal{J}_p$. But $\mathcal{J}''_p$ is
maximal, so that $\mathcal{J}''_p = \mathcal{J}_p$. 

Since almost all primes $p$ satisfy the above condition, the proof is
complete. 

\noindent \textbf{10. Zeta Function of an Order $\mathcal{J}$}

We define the zeta function $\zeta(s)$, of an order
$\mathcal{J}$(where $s$ is a complex number for which $\mathscr{R}(s)
> 1)$ as  
 $$
\zeta(s) = \sum_{\mathfrak{M}} \frac{1}{(n(\mathfrak{M}))^{2s}}
$$
where\pageoriginale $\mathfrak{M}$ runs though all the integral left ideals of the
order $\mathcal{J}$. Further we can write 
$$
\zeta(s) = \sum^\infty_{n=1} \frac{a_n}{n^{2s}}
$$
where $a_n $ is the number of integral left ideals for $\mathcal{J}$
with norm $n$; (The finiteness of $a_n$ will be proved in
\S \ref{chap1:sec3}, Lemma \ref{chap1:sec3:thm1:lem1}) 

If
$$
 n= p^{r_1}_1 -- p^{r_k}_{k}, ~\text{then}~ a_n =
 a^{\substack{(p)\\{r_1}}}_{p_1} \cdots a^{\substack{(p_k)\\{r_k}}}_{p_k} 
$$	
where $a^{\substack{(p)\\{r_1}}}_{p_1}$ is the number of $p_i$ adic integral
left ideals for $\mathcal{J}_{p_i}$, with the norm $p_i^{r_i}$. This
follows from  $\mathfrak{M} = Q \bigcap \limits_{p} \mathfrak{M}_p$
established in Theorem \ref{chap1:sec2:thm7}. Formally we may write 
$$
\zeta (s) = \prod_p \bigg( \sum^{\infty}_{r=o}
\frac{a^{(p)}_{p^r}}{(p^r)^{2s}}  \bigg) 
$$
the product being extended over all rational primes $p$. This result
is a simple consequence of the equation for $a_n$. Actually, one can
prove the convergence of this infinite product in the domain of
convergence $\zeta(s)$.

We will now restrict ourselves to maximal orders. So let $\mathcal{J}$
be a maximal order. Then 
\begin{equation*}
  a^{(p)}_{p^r} = 
  \begin{cases}
    1 &\text{if}~ Q_p \text{ is a division algebra }, \\ 
    i+ p+ \cdots p^r = \frac{1-p^{r+1}}{1-p}, & \text{if}~ Q_p \cong
    \mathfrak{M}_2 (\bar{k}_p)  
  \end{cases}
\end{equation*}
Let $p_1, \ldots,  p_t$ denote the characteristic primes, which we
know to be finite in number. Then 
\begin{align*}
  \zeta(s) &= \prod^t_{i=1} \left( \sum^\infty_{r=0}
  \frac{1}{(p_i^r)^{2s}} \right). \prod_{\substack{p \neq p_i \\ i=1,
      \ldots,  r}} \left( \sum^\infty_{r=0} \frac{1-
    p^{r+1}}{\frac{1-p}{(p^r)^{2s}}}\right) \\ 
  &=\prod^t_{i=1} \left( \frac {1}{1-p_i^{-2s}}\right) \prod_{p \neq
    p_i} \left( \frac{1}{(1-p^{-2s}) (1-p^{1-2s})} \right)  
\end{align*}
i.e.,\pageoriginale 
$$
\overline{\underline{\zeta (s) = \zeta_o (2s) \zeta_o(2s-1)
    \prod^t_{i=1} (1 - p^{1-2s}_{i})}} 
$$
where  
$$
\zeta_o(s) = \prod_p (1 - p^{-s})^{-1}
$$
$\mathscr{R}(s) > 1$, is the Riemann zeta function.

In particular, when $t = 0$, i.e.,  when there do not exist any
characteristic primes, we have 
$$
\zeta(s) = \zeta_o(2s). \zeta_o(2s-1). 
$$

Extending $\zeta(s)$ to the whole plane (this is possible since
$\zeta_o(s)$ can be extended), it follows that $\zeta(s)$ has a simple
pole at $s=1$ with the residue 
\begin{align*} 
  = & (\zeta_o (2 s))_{s=1} \prod^t_{i=1} \left(1 -\frac{1}{p_i}\right)
  (res. \zeta_o (2s - 1)_{s=1})
\end{align*}
  $=  \dfrac{\pi^2}{6}. \dfrac{1}{2} \prod^t_{i=1} \left(1
-\dfrac{1}{p_i}\right)$ since  $\zeta_o (2s - 1)$ has the expansion  
$\dfrac{1}{(2 s-1)-1} + \cdots = \dfrac{1}{2( s-1)} + \cdots$ at the
point $2s-1 = 1$, i.e., at $s= 1$ 

\begin{note}
  In the special case of orders $\mathcal{J}$ not necessarily maximal,
  but satisfying the following conditions 
\end{note}

\begin{enumerate}
\item $\mathcal{J}_p$ is maximal for all characteristic primes.	
\item $\mathcal{J}_p \cong \begin{pmatrix} \mathscr{O}_p & \mathscr{O}_p \\
  p \mathscr{O}_p & \mathscr{O}_p \end{pmatrix}$\pageoriginale for a finite number of primes.
\item $\mathcal{J}_p$ is maximal  (say) = $\begin{pmatrix}
  \mathscr{O}_p & \mathscr{O}_p \\ \mathscr{O}_p &
  \mathscr{O}_p \end{pmatrix}$, for the rest, 
\end{enumerate}

We can proved that zeta function for this order is of form
$$
\zeta (s)= \zeta_o (2s)\, \zeta_o (2 s-1) \prod_p (1+ p^{1-2s})
\prod_p (1-p^{1-2s}) 
$$
where the second product is taken over all characteristic primes, and
the first over the prime for which $\mathcal{J}_p$ is of type
$(2)$. This is a consequence of the fact that $a^{(p)}_{p^r} =
2$. $\dfrac{1-p^{r+1}}{1-p} -1$, for primes of the type (2) (M.Eichler,
Zur zahlentheorie der Quat.Alg.Uselle's Journal, $1956,P.132$ ) 

Another application of the $p$-adic theory we shall see later in the
relations between the ideals of quaternion algebras and there
quadratic subfields. 

\section{Class of Ideals}\label{chap1:sec3} % 3

\textbf{11.}~ Let $\mathcal{J}$ be a given, and let $\mathfrak{M}$ and
$\mathfrak{N}$ be two left ideals for $\mathcal{J}$. We say that
$\mathfrak{M}$ is left equivalent to $\mathfrak{N}$, (we write
$\mathfrak{M} \sim \mathfrak{N}$) is there exist a $\mu$-such that
$n(\mu) \neq 0$ and $\mathfrak{M}=\mathfrak{N} \mu$, i.e., if
$\mathfrak{N}^{-1} \mathfrak{M}$ is a  principal ideal. The above defined
relation is evidently an equivalence relation, and we obtain thus left
equivalence classes of left ideals with respect  to the order
$\mathcal{J}$. We can similarly define right equivalence for right
ideals with respect to the order $\mathcal{J}$ and obtain right
equivalence classes. We will now prove the following  

\setcounter{theorem}{0}
\begin{theorem}\label{chap1:sec3:thm1} % them 1
  {\em The\pageoriginale number of left classes with respect to an order}
  $\mathcal{J}$ {\em is finite and is equal to the number of right
    classes for} $\mathcal{J}$. {\em Further, this (say,  $h$ which is
    called the class number) is independent of the order
    $\mathcal{J}$; (in the class of unit elements of the groupoid of
    ideals)}. 
\end{theorem}

\begin{proof}
  Assuming that the number of left classes in finite, by means of the
  mapping $\mathfrak{M} \leftrightarrow \mathfrak{M}^{-1}$, the left
  ideal classes for $\mathcal{J}$ correspond in a $(1,1)$  manner to
  the right ideal classes for $\mathcal{J}$, and hence the number of
  right classes being equal to the number of left classes,  is
  finite. Let now $\mathfrak{M}_1, \ldots \mathfrak{M}_h$ be a system
  of representatives for  the left classes, then by
  axiom $5$ for a groupoid, there  exists an ideal $\mathfrak{N}$
  which has $\mathcal{J}$ as a right order, and $\mathcal{J}'$ as a
  left order, where $\mathcal{J}' = Q \bigcap\limits_{p} \mu^{-1}_p
  \mathcal{J} \mu_p, \mu^{-1}_P \mathcal{J}_P \mu_P =\mathcal{J}_P$
  for almost all $p$. The products $\mathfrak{N} \mathfrak{M}_1, \ldots
  ,  \mathfrak{N} \mathfrak{M}_h$ are then defined and are left ideals
  for the order $\mathcal{J}'$. No two of these can be left
  equivalent, for if $\mathfrak{N} \mathfrak{M}_i =
  \mathfrak{N}. \mathfrak{M}_j \varrho$, where $n (\varrho) \neq 0$
  then we would have $\mathfrak{M}_i = \mathfrak{M}_j \varrho$ which
  is a contradiction. If $h'$ denotes the class member for
  $\mathcal{J}'$ this means that $h'\ge h$. Similarly $h \ge h'$,
  i.e., $h=h'$.    
\end{proof}

To prove that the number of left classes is finite, we require two lemmas.

\setcounter{Lemma}{0}
\begin{Lemma}\label{chap1:sec3:thm1:lem1}
  For any order $\mathcal{J}$, there are only  a finite numbers
    of integral left ideals with a given norm $n$. 
\end{Lemma}

\begin{Lemma}\label{chap1:sec3:thm1:lem2}
   Let $\mathfrak{M}$ be a right ideal for a given order
    $\mathcal{J}$. Then there exist a $\mu \in \mathfrak{M}$ such
    that $0 <  | n(\mu) | < C_{\mathcal{J}}. | n (\mathfrak{M}) |$
    where $C_{\mathcal{J}}$ is a constant depending only on
    $C_{\mathcal{J}}$. 
\end{Lemma}

We\pageoriginale will first prove the theorem assuming the lemmas to the true, and
then prove the lemmas. 

\medskip
\noindent \textbf{Proof of the theorem.}
  Let $\mathfrak{M}$ be any left ideal for $\mathcal{J}$, consequently
  $\mathfrak{M}^{-1}$ is a right ideal for $\mathcal{J}$ and
  $\mathfrak{M} \mathfrak{M}^{-1} = \mathcal{J}$. Applying
  lemma \ref{chap1:sec3:thm1:lem2} 
  to $\mathfrak{M}^{-1}$ there exists a $\mu \in \mathfrak{M}^{-1}$,
  such that  
  $$
  0 <  |n (\mu) |  < C_{\mathcal{J}}. n (\mathfrak{M}^{-1}) -
    c_{\mathcal{J}} | n(\mathfrak{M}) |^{-1} 
  $$

Consider the left ideal $\mathfrak{N}=\mathfrak{M} \mu,  \mathfrak{N}
\subseteq \mathcal{J}$ since $\mu \in \mathfrak{M}^{-1}$. This means that
$\mathfrak{N}$ is an integral ideal in the left class of
$\mathfrak{M}$ and $|n(\mathfrak{N})|= | n(\mathfrak{M}). n (\mu) | <
C_\mathcal{J}$. Since there are only a finite number of integers in
the interval $[-C_\mathcal{J},C_\mathcal{J}]$ and since by
lemma \ref{chap1:sec3:thm1:lem1},
there exists only a number of integral left ideals with a given norm,
the number of $\mathfrak{N}_s$ is finite, it follows that the number
of left classes is finite. 

\begin{proofoflemma}\label{chap1:sec3:polem1}
  \begin{enumerate}[\rm (a)]
  \item If $\mathcal{J}$ is maximal, Lemma \ref{chap1:sec3:thm1:lem1} has
    already been proved 
    to be true. ( \S \ref{chap1:sec2},  Zeta function of an order). 
  \item So, one let $\mathcal{J}$ be any order. Then there exists a
    maximal order $\bar{\mathcal{J}}$ for which $\mathcal{J} \subset
    \bar{\mathcal{J}}$, i.e., $\mathcal{J}_p \subset
    \bar{\mathcal{J}}_p$ for all $p$, $\bar{\mathcal{J}}_p$ being
    maximal for all $p$ because $\bar{\mathcal{J}}$ is so.  
  \end{enumerate}
\end{proofoflemma}

To the ideal $\mathfrak{M}_p = \mathcal{J} \mu_p$ we make correspond
the ideal $\bar{\mathfrak{M}}_p= \bar{\mathcal{J}}_p \mu_p$ which is
again integral since $\mu_p \in \mathcal{J} _p \subset
\bar{\mathcal{J}}_p$. Further $n(\bar{\mathfrak{M}}_p)= n
(\mathfrak{M}_p)$. Therefore for proving that there are only a finite
number of $\mathfrak{M}_p$ with a given norm it suffices to show that
only a finite number of $\mathfrak{M}_p$ correspond to the same
$\bar{\mathfrak{M}}_p$. 
 
Let\pageoriginale $\bar{\mathcal{J}}_p \mu_i^{(i)}= \bar{\mathcal{J}}_p \mu_p$ where
$\mathcal{J}_p \mu^{(i)}_p$ are the ideals associated with
$\bar{\mathcal{J}}_p \mu_p$. Then $\mu^{(i)}_p
\mu^{-1}_p=\epsilon^{-(i)}_p$ , a unit in $\bar{\mathcal{J}}_p$. 
Denote by $\bar{\mathscr{O}}_p$ and $\mathscr{O}_p$ respectively the
unit groups of $\bar{\mathcal{J}}_p$ and $\mathcal{J}_p$. Then
$\bar{\mathscr{O}}_p \supset \mathscr{O}_p$ and we shall prove that
$\mathscr{O}_p / \mathscr{O}_p$ is finite.  
 
 Choose $n$ sufficiently large so that $\mathcal{P}^n.
 \bar{\mathcal{J}}_p \subset \mathcal{J}_p$. (The subsequent arguments
 hold good for the global orders $\mathcal{J} \subset
 \bar{\mathcal{J}} \subset Q/k$ also expect that we have to
 choose an integer $m$ sufficiently large such that
 $m$. $\bar{\mathcal{J}} \subset \mathcal{J}$). This implies that the
 ring generated by $1$, $p^n \bar{\mathcal{J}}_p$ (say) $[1, p^n
   \bar{\mathcal{J}}_p] \subseteq \mathcal{J}_p$ for $1 \in
 \mathcal{J}_p$. Let $2 \mathfrak{Y}_p$ be the group of units $\{ \varepsilon
 \}$ of the ring $[1, p^n. \bar{\mathcal{J}}_p]$ which are of the type
 $\varepsilon =1 (p^n. \bar{\mathcal{J}}_p )$ Then
 $\bar{\mathscr{O}}_p \supseteq \mathscr{O}_p \supseteq 2
 \mathfrak{Y}_p$. Now, the mapping 
 $$
 \mathfrak{Y}_p \alpha \to  p^n \bar{\mathcal{J}}_p + \alpha, ( \alpha
 \in \bar{\mathscr{O}}_p) 
 $$
 gives $a(1,1)$ image of $\bar{\mathscr{O}}_p /  \mathfrak{Y}_p$ in
 the system of residue classes $\{p^n. \bar{\mathcal{J}}_p  + \alpha
 \}$; so that $\bar{\mathscr{O}}_p / \mathfrak{Y}_p$ is finite which
 in its turn implies that $\bar{\mathscr{O}}_p / \mathscr{O}_p$ is
 finite  (say) of order $r$. We have then the cost decomposition,
 $\bar{\mathscr{O}}_p= \bigcup\limits^r_{\nu =1} \mathscr{O}_p
 \eta_\nu$. Hence $\bar{\varepsilon}^{(i)}_p \in \mathscr{O}_p$
 implies that $\bar{\varepsilon}^{(i)}_p \in \mathscr{O}_p \eta_{\nu}$
 (say), i.e., $\varepsilon^i_p= \varepsilon^{(i)}_p
 \eta_{\nu_i}, \varepsilon^{(i)}_p \in \mathscr{O}_p$ i.e.,
 $\mu^{(i)}_p = \varepsilon^{(i)}_p \eta_{\nu_i} \mu_p$ or
 $\mathcal{J}_p \mu^{(i)}_p =\mathcal{J}_p. ( \eta_{\nu_i} \mu_p)$
 since, we deduce that $\mathcal{J}_p \mu^{(i)}_p$ are finite in
 number.  
 
 Proceeding to the global case the number of integral left ideals for
 the order $\mathcal{J}$ with norm $n$  is given by
 $\prod\limits_{i=1}^s a^{(i)}_{p_i^{r_i}}$ if\pageoriginale $n=p^{r_1}_1 \cdots
 p^{r_s}_s$ and $a_{p_i^{r_i}}^{(i)}$ denotes the number of integral
 left ideals for the order $\mathcal{J}_{p_i}$ with norm $p_i^{r_i}$,
 which has been proved to be finite in the previous paragraph.  
 
 Thus our contention in completely established. 
 
\begin{proofoflemma}\label{chap1:sec3:polem2}
  Let $\mathfrak{M}= Q \bigcap\limits_{p} \mu_p$. $\mathcal{J}_p =
  [\nu_1, \nu_2, \nu_3,  \nu_4]$ and $\mathcal{J}=
  [L_1, L_2, L_3, L_4]$. Then, if $\nu_i = \sum^4_{k=1}m_{ik} L_k, m_{ik}
  \in k$, we will prove that absolute value of
  $|m_{ik}|=n(\mathfrak{M})^2$. We shall above first that
  $(|m_{ik}|)_p= (n(\mu_p)^2)_p$. Then  it would follow that  
  $$
  (n(\mathfrak{M})^2) = \prod_p (n\mu_p))^2 = \prod_p (|m_{ik}|)_p = (|m_{ik}|).
  $$
  i.e., $n(\mathfrak{M})^2 =|m_{ik}|$. rational unit = absolute value
  of $|m_{ik}|$. Since $\bar{l}_K \in \mathcal{J}_p$, we can write
  $\bar{L}_k = \sum\limits_{l=1}^4 L_1 c_{lk}$, $c_{lk} \in
  \mathscr{O}_p$. Then $(t(L_i \bar{L}_k))=(t( L_i L_k))(c_{ik})$ and
  since $L_1, \ldots,  \bar{L}_4$ also form a basis for
  $\mathcal{J}_p$ over $\mathscr{O}_p, (c_{ik})$ is $p$-unimodular,
  i.e., $|c_{ik}|$ is a $p$-adic unit. Now $D (\mu_p \mathcal{J}_p)=
  D(\mathfrak{M}_p) = |m_{ik}|^2. D(\mathcal{J}_p)$ from  the basis
  representation of the $\nu_i$-s. Since $[\mu_p L_1, \ldots, \mu_p
    L_4]$ form a basis for $\mathfrak{M}_p$, and since $t(\mu_p
  L_i. \overline{\mu_p L_j})= n(\mu_p)$. $t(L_i \bar{L}_j)$, we have
  $D(\mathfrak{M}_p) = n(\mu_p)^4$. $| t (L_i \bar{L}_j) | = n
  (u_p)^4$. $D(\mathcal{J}_p) \mu_p$. $u_p$, a $p$-adic unit, i.e.,
  $|m_{ik}| D(\mathcal{J}_p)=
  n(\mu_p)^4$. $D(\mathcal{J}_p)$. $u_p$, and $D(\mathcal{J}_p) \neq
  0$ so that $(|m_{ik}|)_p= (n(\mu_p)^2)_p$. 
\end{proofoflemma} 
 
Our object now is to find $\mu \in \mathfrak{M}$, $\mu = \nu_1 t_1
\cdots  + \nu_4 t_4 \,t_i \in \mathscr{O}$ such that $0 < |n(\mu) |<
C_\mathcal{J}| n(\mathfrak{M}) |$. Let $\nu= \nu_1 X_1 + \cdots +
\nu_4 X_4 \,X_i \in \mathscr{O}$, be any element of
$\mathfrak{M}$. Substituting $\nu_i= \sum\limits_{k=1}^{4}m_{ik} L_k$
in\pageoriginale this expression, we obtain $\nu= L_1 L_1 + \cdots + L_4 L_4$ where  
 $$
 L_j = \sum_{i=1}^4 X_i m_{ij} j=1, \ldots,  4
 $$
 are linear forms in $X_1, \ldots, X_4$ with rational coefficients. By
 Minkowski's Theorem on linear forms, since absolute value of
 $|m_{ik}|= n(\mathfrak{M})^2=(\sqrt{|n(\mathfrak{M})|})^4$, there
 exist integers $t_1, \ldots, t_4$ such that  
 $$
 |L_j| =\bigg | \sum_{i=1}^4 t_i m_{ij} \bigg | < \sqrt{|n (\mathfrak{M})|}.
 $$
  If $\mu= \nu_1 t_1 +  \cdots + \nu_4 t_4$, then $\mu \in
  \mathfrak{M}$, and since $n(\mu)= \sum_{i=1}^4 n(L_i) L^2_i +
  \sum_{i, j=1, i \neq j}^4 t(L_i \bar{L}_j). L_i L_j$ 
 we have $0  < | n(\mu)| < C_\mathcal{J}$. $| n(\mathfrak{M})|$, where
 $C_\mathfrak{J}$ is a constant depending only on the order
 $\mathcal{J}$. 
 
\textbf{12.}~ The quaternion algebras over the rational number field $k$ can
 be divided  into two classes according as the quadratic form given by
 the norm is definite or indefinite. In the first case way that the
 algebra is definite, and in the second  case that it is indefinite. 
 
 If the algebra $Q$ is definite, the class number $h$ is in general
 greater than $1$. If $Q$ is indefinite, it can be proved that $h=1$
 for maximal orders. (For a proof see M.Eichler, \textit{Math.Zeit},
1938). Furthermore, it can be proved that $h=1$
 for even a wider class of orders, i.e., for orders of the type
 $\mathcal{J}$, where  
 \begin{enumerate}
\item $\mathcal{J}_p$\pageoriginale is maximal for almost all $p$, or without
  loss of generality, 
  $$\mathcal{J}_p \cong
  \begin{pmatrix}
    \mathscr{O}_p & \mathscr{O}_p \\
    \mathscr{O}_p & \mathscr{O}_p 
  \end{pmatrix}
  $$
\item $\mathcal{J}_p$ is maximal for all characteristic primes $p$.
\item For the remaining finite number of primes,
  $$
  \mathcal{J} p  \cong 
  \begin{pmatrix}
    \mathscr{O}_p & \mathscr{O}_p \\
    p\mathscr{O}_p & \mathscr{O}_p
  \end{pmatrix}
  $$
 \end{enumerate} 
 (For a proof of this, see M.Eichler,  \textit{Math.Zeit}, 1952.
 
 We will give a proof of the above result in a special case.
 \begin{note}
   We call such an order an order of the type $(q_1, q_2), q_1, q_2$
   being the product of primes of types $(2)$ and $(3)$ respectively. 
 \end{note} 
 
\begin{theorem}\label{chap1:sec3:thm2} %them 2
  Let $Q/k \cong \mathfrak{M}_2 (k)$ (i.e., {\em there do not exist
    any characteristic primes) and let} $\mathfrak{J}
  \cong \begin{pmatrix} \mathscr{O} & \mathscr{O} \\ m \mathscr{O} &
    \mathscr{O} \end{pmatrix}$, $m$ {\em a rational integer be an
    order of} $Q$. {\em Then all left ideals for $\mathcal{J}$ are
    principal},i.e., $h=1$. 
\end{theorem} 

We required the following two lemmas:
\setcounter{Lemma}{0}
\begin{Lemma}\label{chap1:sec3:thm2:lem1} % lem 1
  Let $\mathfrak{M}= Q \bigcap\limits_p \mathcal{J}_p \mu_p$ be any
  left ideal for $\mathcal{J}$; then there exists a $\varrho \in  Q$
  such that $\mathfrak{M}$. $\varrho= \mathfrak{N}$ is integral,
  $n(\varrho) \neq 0$ and such that $(n(\mathfrak{N}), m)=1$. 
\end{Lemma}
 
\begin{Lemma}\label{chap1:sec3:thm2:lem2} % lem 2
  An ideal $\mathfrak{N}$ whose norm is coprime to $m$ is necessarily
  principal. 
\end{Lemma} 
 
Assuming the lemmas to be true, we will establish the theorem. By
lemmas \ref{chap1:sec3:thm2:lem1} and
\ref{chap1:sec3:thm2:lem2}, every left ideal $\mathfrak{M}$ for the
order  
$\mathcal{J}$ is left equivalent\pageoriginale to a principal ideal,
and hence there is only one class, viz., the class of principal
ideals, i.e., $h=1$. 


\setcounter{proofoflemma}{0}
\begin{proofoflemma}\label{chap1:sec3:thm2:polem1}
  Let the primes $p$ for which $p|m$ be denoted by\break $p_1, \ldots,
  p_r$. We have to find a $\mu \in Q, n (\mu)\neq 0$ such that
  $\mu_p$. $\mu^{-1}$ is a unit in $\mathcal{J}_p$ for all $p|m$. If
  $n(\mu_{pi})= p_i^{n_i} u_i$ ($u_i$, a $p$-adic unit), $i=1, \ldots
  , r$, let $n$ be an integer greater than max $(1,n_1+ 1-s_1, \ldots,
  n+ 1-s_r)$ where $\mu_{p_i}= p_i^{s_i}$. $\mu'_{p_i}$,  $\mu'_{p_i}
  \in  \mathcal{J}_{p_i}$. Let $\mathcal{J}=[\nu_1$, $\nu_2$, $\nu_3$,
    $\nu_4]$, then we have  
  $$
  \mu_{p_i} = \sum_{j=1}^4 \mu^{(i)}_{p_i} \nu_j,  \mu^{(j)}_{p_i} \in
  \bar{k}_{p_i}, i=1, \dots,  r. 
  $$ 
\end{proofoflemma} 
 
Now by Ostrowski's Theorem on approximation, for each $j=1, \ldots$,
$r$ we can find a $\mu^{(j)} \in k$ such that  
$$
\mu^{(j)} \equiv \mu^{(j)}_{p_i} (p^n_i.  \mathscr{O}_{p_i}), i=1, \ldots,  r.
$$

Let $\mu= \sum\limits^{4}_{j=1} \mu^{(j)} \nu_j$, then $\mu \in Q$,
and we use the notation $\mu \equiv \mu_p \pmod {p^n \mathscr{O}_p}$,
$(p=p_1, \ldots,  p_r)$. Now $n(\mu) \neq 0$, for  otherwise we would
have $0=n(\mu) \equiv n(\mu_p) (p^n \mathscr{O}_p)$, i.e., $n(\mu_p)
=p^n$. (some $p$-adic integer),  $p=p_1, \ldots,  p_r$ which is a
contradiction to the choice of $n$. 

Now $\mu_{p_i}= p_i ^{s_i}$. $\mu'_{p_i}, \mu'_{p_i} \in
\mathcal{J}_{p_i}, n= max (1,n + 1-s_1, \ldots,  n_r + 1-s_r)$ and
$\mu= \mu_{p_i}+p^n_i \varrho_i, \varrho 
_i \in \mathcal{J}_{p_i}$. Then  
\begin{align*}
  \mu \mu^{-1}_{p_i} &=1 + p^n_i \varrho_i \frac{\bar{\mu}_{p_i}}{n(\mu_{p_i})}\\
  &= 1 + p^n_i \varrho_i \frac{p_i^{s_i}. \bar{\mu'_{pi}}}{p_i^{n_i}
    u_i}=1 + p_i^{n+s_i - n_i} \bigg (
  \frac{\bar{\mu'_{p_i}}}{u_i}\bigg ) \\ 
  &= 1+  p_i \lambda_i, \lambda_i  \in  \mathcal{J}_{p_i},
\end{align*}
because\pageoriginale $\dfrac{\bar{\mu}_{p_i}}{u_i} \in \mathcal{J}_{p_i}$, and
$n+ s_i - n_i \ge 1$ by choice of $n$. Hence $\mu \mu^{-1}_{p_i} \in
\mathcal{J}_{p_i}$ also since $n(\mu \mu^{-1}_{p_i})$ is a $p_i-$adic
unit, it follows that $\mu \mu^{-1}_{p_i}$ is a unit of
$\mathcal{J}_{p_i}$ for $i=1, \ldots,  r$. Consider now the ideal
$\mathfrak{M} \mu^{-1}=Q \bigcap\limits_p \mathcal{J}_p \mu_p
\mu^{-1}= \mathfrak{N}'$(say), then $\mathfrak{N}'_p = \mathcal{J}$
for all $p |m$, also for almost all $p$. Now $(n(\mathfrak{N}'))=
\prod_p (n(\mathfrak{N}'_p))$, and since $n(\mathfrak{N}'_p)$ is a
p-adic unit for all $p|m$, it follows that $n(\mathfrak{N}')$ is
coprime to $m$.  

Let $q_1, \ldots,  q_s$ be the primes for which $\mathfrak{N}'_p \neq
\mathcal{J}_p$, then we choose $r_i$ such that $q_i^{r_i}
\mathfrak{N}'_{q_i} \subseteq \mathcal{J}_{q_i}$. Let $n=q_1^{r_1}
\cdots q_s^{r_s}$, then $n. \mathfrak{N}'_{q_i} \subseteq
\mathcal{J}_{q_i}$, $i=1, \ldots, s$ and $n. \mathfrak{N}'_p=
n$. $\mathcal{J}_p = \mathcal{J}$ for $p \neq q_i$. since $n$ is a
p-adic unit. Therefore the ideal $\mathfrak{M}. \mu^{-1} n =
\mathfrak{N}'. n  = \mathfrak{N}$, or $\mathfrak{M} \varrho =
\mathfrak{N}$, $\varrho = \mu^{-1}$. $n$, is integral, and
$n(\mathfrak{N})= n^2$.  $n (\mathfrak{N}')$ is coprime to $m$ since
both factors are coprime to $m$. 

\begin{proofoflemma}\label{chap1:sec3:thm2:polem2}
  \begin{enumerate}[\rm (1)]
  \item If $m=1$, we know that there exists $\mu \in \mathfrak{N}$
    such that $n(\mu) | n (\nu)$ for all $\nu \in \mathfrak{N}$ and
    then we proved (in $Th/5 \S 2$) that $\mathfrak{N}=
    \mathcal{J}$. $\mu$, i.e., $\mathfrak{N}$ is principal, or $h=1$. 
  \item $m > 1$. In this case we proceed in a similar
    fashion. Consider all $\nu \in \mathfrak{N}$ such that
    $(n(\nu),m)=1$. There exist such $\nu$, since $n(\mathfrak{N}) =$
    g.c.d. $n(\nu) \nu \in \mathfrak{N}$. If there exist two such\pageoriginale
    elements $\nu_1, \nu_2$, then we can find units $\varepsilon_1$,
    $\varepsilon_2$ such that 
    $$
    \varepsilon_1 \nu_1 = 
    \begin{pmatrix} 
      *& * \\ 0 & * 
    \end{pmatrix},  \quad
    \varepsilon_2 \nu_2 =
    \begin{pmatrix}
      * & * \\ 0 & *
    \end{pmatrix}  
    $$
  \end{enumerate}
  
  We only show how to find the unit $\varepsilon_1$, and then
  $\varepsilon_2$ can be constructed in the same way. Let 
  $$
  \varepsilon_1  = 
  \begin{pmatrix}
    e_{11}& e_{12} \\ me_{21} & e_{22}
  \end{pmatrix},
  \nu_1 = 
  \begin{pmatrix}
    n_{11} & n_{12} \\ m.n_{21} & n_{22}
  \end{pmatrix}
  $$
  Then 
  $$
  \varepsilon_1 \nu_1 = 
  \begin{pmatrix}
    * & * \\ m(e_{21}n_{11}+ e_{22}n_{21}) & *
  \end{pmatrix}
  $$
  In order that $\varepsilon_1$ be a unit and $\varepsilon_1 \nu_1$ be
  of the required form, we must have 
  \begin{align*}
    e_{11} e_{22} &- m e_{21} e_{12} =1, \\
    e_{21} n_{11} &+ e_{22} n_{21} = 0. 
  \end{align*}
\end{proofoflemma}

Put $e_{21}=  \dfrac{n_{21}}{(n_{21}, n_{11})},- e_{22}=
\dfrac{n_{11}}{(n_{21}, n_{11}})$, then $(e_{21},
e_{22})=1$, further $(me_{21}, e_{22})=1$, because $(n_{11}, m.n_{21})
=1$ since $(n (\nu_1),m)=1$, so there exists integers $a$ and $b$ such
that $b e_{22}-a m e_{21}=1$, then put $b=e_{11}, a =e_{12}$, and we
easily see that the required conditions are satisfied.   

Proceeding as before, we obtain $\nu \in \mathfrak{N}$ such that
$n(\nu) |n(\nu_1)$ and $n(\nu_2)$. Continuing thus, since the
denominators of $n(\nu)$ are bounded, we obtain after a finite number
of steps, a $\nu \in \mathfrak{N}$ such that $n(\nu)$ divides the
norms of all elements in $\mathfrak{N}$ and\pageoriginale then $n(\nu) =n
(\mathfrak{N})$. 

It is enough to show that $\mathcal{J}_p. \nu=
\mathcal{J}_p.  \nu_p$ for all $p$, where $ \mathfrak{N}= Q
\bigcap\limits_p \mathcal{J}_p \nu_p$, for this would  mean that  
$$
(\mathcal{J}. \nu )_p = \mathcal{J}_p. \nu = \mathcal{J}_p \nu_p =
\mathfrak{N}_p 
$$
for all $p$, i.e., $\mathcal{J}. \nu= \mathfrak{N}$.

Now $\nu = L_p$. $\nu_p$, $L_p \in  \mathcal{J}_p$ so that $n(\nu)=
n(L_p).n (\nu_p)$. But $n(\nu)=$(p-adic unit). $n(\nu_p)$ by
definition of norm and this means that $n(L_p)=$ p-adic unit, i.e.,
$L^{-1}_p \in \mathcal{J}_p \Rightarrow L_p$ a unit in
$\mathcal{J}_p$. in other words, $\mathcal{J}_p$. $\nu =
\mathcal{J}_p$. $\nu_p$. 

\textbf{13.}~ We will prove an important lemma concerning an order
$\mathcal{J}$ of type $(q_1, q_2)$ presently. 
\begin{Lemma}\label{chap1:sec3:lem3} % lem 3
  $\underline{D(\mathcal{J})= q^2_1 q^2_2}$.
\end{Lemma}  

\begin{proof}
  We have $(D) = \prod\limits_p (D)_p$. But  for all primes of the
  type $(1)\,(D)_p =(D(\mathcal{J}_p))_p= \mathscr{O}_p$ so that our
  purpose, it is enough to consider the primes dividing $q_1$ and
  $q_2$.   
\end{proof}  
  
(i)~ In the case of characteristic primes $p|q_1, \mathcal{J}_p$ is
maximal and we construct a special basis $[1, \omega, \Omega, \omega
  \Omega]$ such that 

$D(\mathcal{J}_p) =D [1, \omega,
  \Omega, \omega \Omega]=p^2. u; u$, $a$ $p$-adic unit. 

Let $\bar{K}$ be an  unramified quadratic extension of $\bar{k}_p$ and
$[1, \omega]$ a base for the maximal order (or the ring of integers in
$\bar{K}$). Then, by theorem \ref{chap1:sec2:thm4}.c of \S
\ref{chap1:sec1}, $\bar{K}$ is a splitting 
field for $Q_p$ and hence $\bar{K} \cong K \subset Q$ by
theorem \ref{chap1:sec2:thm3}
of \S \ref{chap1:sec1} so that we may look upon $[1, \omega ]$ as an integral base
for integers in $K$. 

Then\pageoriginale we choose an element $\Omega \in Q_p$ such that $\Omega^{2}\in
\bar{k}_p$, $\Omega^{-1} \omega \Omega = \bar{\omega}$. But $\Omega^2$
cannot be the norm of an 
element of $\bar{K}$, for then it would mean that $Q_p$ is a matrix
algebra. Since every unit of $k_p$ is a norm of an element of
$\bar{K}$ and $p$ is not, $\Omega^2=p$. $\mathcal{U}$ with
$\mathcal{U},a$ unit. We shall now prove  that $[1, \omega, \Omega,
  \omega \Omega]$ is the maximal order in $Q_p$. That it is an order,
follows by  construction. Further 
$$
D [1, \omega, \Omega, \omega \Omega]=
\begin{vmatrix}
  2 & t(\omega) & 0 &0 \\
  t(\omega) & t(\omega^2) & 0 & 0 \\
  0 & 0 & 2p & pt(\omega) \\
  0 & 0 & pt(\omega) & 2p.n(\omega) 
\end{vmatrix}
$$
Now,
$$
D[1,\omega] = 
\begin{vmatrix}
  t(1.1) & t(1. \omega) \\
  t(\omega. 1) & t(\omega. \omega)
\end{vmatrix} 
=
\begin{vmatrix}
  2 & t(\omega) \\ 
  t(\omega) & t(\omega^2)
\end{vmatrix}
$$
is a $p$-adic unit, since the extension is  unramified (follows by
Dedekind's theorem).  Furthermore, 

$
\begin{vmatrix}
  t(1.1) & t(1, \bar{\omega}) \\
  t(\omega.1) & t(\omega,  \bar{\omega}
\end{vmatrix}
= D[1,\omega]. u; u$,  a  $p$-adic unit,
so that $[1, \omega, \Omega, \omega \Omega]=p^2$. $u_o$ being $a$
p-adic unit. It remains to prove now that $[1, \omega, \Omega, \omega
  \Omega]$ is maximal. For the same it is enough to show that if $\xi
= x_o + \cdots + x_3 \omega \Omega$ is an element of $Q_p$ such that
$n (\xi) \in  \mathscr{O}_p$, then $x_i$ are all in
$\mathscr{O}_p$. Now $\xi = \xi_1 +  \xi_2 \Omega$, where $\xi_1 = x_o
+x_1 \omega$ and $\xi_2 =  x_2 +x_3 \omega$;  $n(\xi)= n(\xi_1)- pn
(\xi_2)$ is a p-adic integer. 

Since\pageoriginale $\xi_1 \in K $, either $\xi_1 \in  \mathscr{O}$ (ring of
integers in $K$) or $\xi^{-1}_1 \in \mathfrak{K} =(p)$ by virtue of $K$
being unramified, i.e., $\xi_1= p^{-r_1}$.  $u_1 ; u_1$,a  unit  of
$\mathscr{O}$. Similarly, either $\xi_2 \in \mathscr{O}$ or $\xi_2
=p^{-r_2} u_2; u_2$,  a unit. Therefore $n(\xi)= p^{-2_{r_1}}.
n(u_1)-p^{-2{r_2}}$. $n(u_2)$. Since $n(u_1)$ and $n(u_2)$ are
$p$-adic units, this cannot happen. Hence $\xi_1$ and $\xi_2$ are both
in $\mathscr{O}$, i.e., $x_0, x_1, x_2,x_3$ are all p-adic integers,
since [$1, \omega$] forms a base of $\mathscr{O}$ over
$\mathscr{O}_p$. 

(ii) In the case of primes $p|q_2$,
$$
\mathcal{J}_p \cong 
\begin{pmatrix}
  \mathscr{O}_p & \mathscr{O}_p \\
  p\mathscr{O}_p & \mathscr{O}_p
\end{pmatrix}
$$
so that we have a basis
$$
\begin{bmatrix}
  \begin{pmatrix}
    1 & 0 \\0 & 0
  \end{pmatrix},
  
  \begin{pmatrix}
    0 & 1 \\0 & 0
  \end{pmatrix},
  
  \begin{pmatrix}
    0 & 0 \\p & 0 
  \end{pmatrix},
  
  \begin{pmatrix}
    0 & 0 \\ 0 & 1
  \end{pmatrix}
\end{bmatrix}
$$

Therefore
$$
D[\mathcal{J}_p] =
\begin{vmatrix}
  1 & 0 & 0 &0 \\
  0 & 0 & p &0 \\
  0 & p & 0 &0 \\
  0 & 0 &0 &1
\end{vmatrix}
=-p^2= p^2 \text{ a $p$-adic unit }.
$$
Hence our assertion is completely established.

