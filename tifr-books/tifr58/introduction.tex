\chapter{Introduction}

In recent years, the theory of torus embeddings has been finding many  
applications. The point of the theory lies in its  ability of
translating meaningful algebra-geometric and analytic phenomena into
very simple statements about the combinatorics of cones in affine
space over the reals. 

In different terminology, it was first introduced by Demazure \cite{keyD2}
and then by Mumford et al. \cite{keyTE}, Satake \cite{keyS1} and Miyake-Oda
\cite{keyMO}. There is already a good and concise account on it in
\cite{keyTE}. Nevertheless, we produce here another. For one thing, we
wanted to supply the details of the partial classification of complete
non-singular 3-dimensional torus embeddings announced in \cite{keyMO}. 

Besides, we wanted to make the theory, in its most general form,
accessible to non-algebraic geometers in view of its possible
applications in other branches of mathematics. We can state at least
the main results without using algebraic geometry, although for the
proof we cannot avoid using it. This was made possible by the
following down-to-earth description due to Ramanan of normal affine
torus embeddings over a field $k$ of finite type as the set of unitary
semigroup homomorphisms 
$$
U(\sigma) = \Hom_{\text { unit.semigr } .}(\check{\sigma} \cap M,k),  
$$
where $(i) N \cong \mathbb{Z}^r$ and $\sigma$ is a convex rational
polyhedral cone in $N_R \cong R^r$ with $\sigma \cap (-\sigma) =
\{0\},(ii)$ for the dual $\mathbb{Z}$-module $M$ of $N$, 
$$
\check{\sigma} \cap M = \{ m \in M: \langle m,y \rangle \ge 0 \text{
  for all } y \in \sigma \} 
$$
is a finitely generated additive semigroup and $(iii) ~k$ is
considered to be a semigroup under the \textit{multiplication}. 

When we have a suitable collection $\triangle$ of such cones, an
\textit{r.p.p. decomposition}, then $U(\sigma)'s$ can be canonically
patched together to form a normal and separated algebraic variety over
$k$ locally of finite type 
$$
T_N \emb (\triangle)
$$
which has an effective action of the algebraic torus 
$$
T_N = N \otimes_{\mathbb{Z}} k^* = \Hom_{gr} (M,k^*) = k^* \times
\ldots \times k^* 
$$
with a dense orbit. Such a variety is called a \textit{torus
  embedding}, since it is a partial compactification of
$T_N$. Conversely, we get all normal torus embeddings in this way 
(Theorem \ref{chap1:thm4.1}). Important Algebre-geometric phenomena can most often
be described purely in terms of $\triangle$ (Theorems
\ref{chap1:thm4.2}, \ref{chap1:thm4.3}, \ref{chap1:thm4.4} and
Corollary \ref{chap1:coro4.5}).  

We may say that $\triangle$ contains all the relevant information, in
a unified and globalized way, about the ``exponents of monomials''
necessary to describe such varieties. When an algebraic variety or a
morphism can be described solely in terms of monomials, then there is
a good chance that it can most effectively be described in terms of
torus embeddings. For instance, a normal irreducible affine algebraic
variety $V \subset \mathbb{A}_r$ defined by equation of the form  
$$
X^{a_1}_1 ~X^{a_2}_2 \ldots X^{a_r}_r  = X^{b_1}_1~X^{b_2}_2 \ldots
X^{b_r}_r 
$$
can be expressed as $U(\sigma)$ for some $\sigma$. (cf. (7.9))

We try to avoid overlaps with Demazure \cite{keyD2} and Mumford et
al. \cite{keyTE} as much as possible. For later convenience for reference,
we collect together in \S. \ref{chap1:sec4} the first main theorem in their most
general form and leave their proof to \S. \ref{chap1:sec5}. Various standard
examples are collected together in \S. \ref{chap1:sec7}. 

In \S. \ref{chap1:sec6}, we deal with torus embedding which can be embedded into
projectives spaces. The results are slightly more general than those
in \cite{keyD2} but less so than those in \cite{keyTE}. 

\S. \ref{chap1:sec8} and \S. \ref{chap1:sec9} are devoted to the
classification of complete 
nonsingular torus embeddings. We are reduced to the classification of
certain weighted circular graphs and that of weighted triangulations
of the 2-sphere. As by-products, we obtain many interesting complete
non-singular rational three folds. (cf. Prop. \ref{chap1:prop9.4}) Besides, we see
that torus embeddings provides us with a good testing ground for
important conjectures on birational geometry in higher dimension. 

There are many basic results we left out : For the cohomology of
equi variant coherent sheaves on torus embeddings as well as the
description of the automorphism groups of torus embeddings, we refer
the reader to Demazure \cite{keyD2}. 

Mumford et al. \cite{keyTE} generalizes the notion of torus embedding to
that of \textit{toroidal embeddings} and proves very important
\textit{semi-stable reduction theorem}. Torus embeddings have also
been used effectively in the compactification problem of the moduli
spaces by Satake \cite{keyS1}, Hirzebruch \cite{keyH4}, Mumford et
al.\cite{keySC}, Namikawa \cite{keyN5}, Nakamura \cite{keyN3},
Rapoport \cite{keyR1}, Oda-Seshadri \cite{keyOS} and Ishida
\cite{keyI5}.  

Here we deal with more elementary but illustrative applications in
Chapter \ref{chap2}. 

When the ground field $k$ is the field $\mathbb{C}$ of complex numbers
or one with a \textit{non-archimedean} rank one valuation, then
$U(\sigma) = \Hom_{unit.semigr.}\break(\check{\sigma} \cap M, k) $, hence
$T_N \emb (\triangle)$, has the topology induced from that of $k$ by
the valuation. Let $CT_N$ be the maximal compact torus of $T_N$  in
this topology. Then we can usually draw the picture of the quotient of
a torus embedding $T_N \emb (\triangle)$ by $CT_N$ and get better
geometric insight. The quotient was introduced by Mumford et
al. \cite{keySC}. We call it the \textit{manifold with corners} after
Borel-Serre \cite{keyBS} and denote it by 
$$
Mc(N,\triangle) = T_N \emb (\triangle)/CT_N.
$$

Using this we will be able to visualize the construction and
degenerations of complex tori, Hopf surfaces and other class $VII_0$
surfaces introduced by Inoue. (cf. \S. \ref{chap2:sec11},
\S. \ref{chap2:sec13}, \S. \ref{chap2:sec14} and 
\S. \ref{chap2:sec15}). Using Suwa's classification of hyper elliptic surfaces om
\cite{keyS4}, Tsuchihashi \cite{keyT1} were able to describe their
degenerations and the compactification of the moduli space in terms of
torus embeddings. 

There are many recent results on the actions of algebraic and analytic
groups other than algebraic tori on algebraic varieties and complex
manifolds. See, for instance, Akao \cite{keyA1}, Popov \cite{keyP2},
\cite{keyP3}, Orlik-Wagreich \cite{keyOW} and Ishida \cite{keyI4}. 

Complete non-singular $2$-dimensional torus embeddings over
$\mathbb{C}$ give rise to rational compactifications of
$\mathbb{C}^2$ and $(\mathbb{C}^*)^2$. Compactifications of
$\mathbb{C}^2$ were shown to be always rational by Kodaira \cite{keyK4} and
were classified by Morrow \cite{keyM6}. There are, however, many
non-rational, even non-algebraic, compactifications of
$(\mathbb{C}^*)^2$. They were recently classified by simha \cite{keyS2} and
Ueda \cite{keyU1}. 

These notes are based on a joint work with $K$. Miyake of  Nagoya
University and grew out of the lectures which the author gave at Tata
Institute of Fundamental Research, University of Paris-Sud, Orsay,\break
Nagoya University, Tohoku University, Instituto Jorge Juan, Harvard
University and various other places. He would like to thank the
mathematicians at these institutions for the hospitality and the
patience shown to him. The notes taken by K.Makio at Tohoku University
was very helpful.\footnote{After these notes were written, M.Ishida of
  Tohoku University obtained the following seemingly definitive result
  on the Cohen-Macaulay and Gorenstein properties in relation to torus
  embeddings.} 

Recall that for a connected locally noetherian scheme $X$, its
\textit{dualizing complex} $R_X$ is determined uniquely up to
quasi-isomorphism, dimension shift and the tensor product of
invertible $0_X$-modules. (cf. Hartshorne, Residue and duality,
Lecture Notes in Math. 20, Springer-Verlag, 1966). 

Let $T = T_N$  be an algebraic torus and consider the normal torus
embedding Temb$(\triangle)$ corresponding to an r.p.p.decomposition
$(N,\triangle)$. Let us fix an orientation for each cone $\sigma \in
\triangle$ and define, in case $\dim \tauup - \dim \sigma = 1$, the
incidence number $[\sigma : \tauup ] = 0,1$ or $-1$ in an
appropriate manner so that we have a complex 
$$
C(\triangle, \mathbb{Z}) = (\ldots \rightarrow 0 \rightarrow C^0
\overset{d}{\rightarrow} C^1 \rightarrow \ldots \rightarrow C^{\text{rank}
  ~N} \rightarrow 0 \rightarrow \ldots) 
$$
where $C^j$ is the free $\mathbb{Z}$-module generated by the set of
$j$-dimensional cones in $\triangle$ and where 
$$
d(\sigma) = \sum_{\dim \tauup -\dim \sigma = 1} [\sigma : \tauup]
(\tauup). 
$$
 
Let $X$ be a \textit{$T$-invariant closed connected reduced subscheme}
of\break Temb$(\triangle)$ and consider the complex $R_X$ of $0_X$-modules
defined by  
$$
R^j_X = \oplus~ 0_Y
$$
where the direct sum is taken over all the T-invariant closed 
\textit{irreducible} subvarieties of $X$ which is of codimension $j$
\textit{in Temb$(\triangle)$} and where $\delta : R^j_X \rightarrow
R^{j+1}_X$ is defined as follows : $X = \bigcup\limits_{\sigma \in
  \sum}\orb (\sigma)$ for a \textit{star closed} subset $\sum = \{
\sigma \in \triangle ;~ \orb (\sigma) \subset X\}$. When $Y =
\overline{\orb(\sigma)}$, then for $f_\sigma \in 0_Y$, 
$$
\delta (f_\sigma) = \sum[\sigma : \tauup] (\text{the restriction of }
f_\sigma \text{ to } \overline{\orb(\tauup)}) 
$$
with the sum taken over $\tauup \in \triangle$ with $\dim \tauup -
\dim \sigma = 1$.  

\begin{theorem}[(Ishida)] 
If $X$ is a T-invariant closed connected reduced subscheme of a normal
torus embedding Temb$(\triangle)$, then $R_X$ defined above is the
dualizing complex for $X$. 
\end{theorem}

For $\rho \in E$, let $\sum_\rho = \{ \sigma \in \sum ; \sigma <
\rho\}$ and consider the complex of $k$-modules $C(\sum_\rho,k)$ with
the coboundary map induced by $d$. 

\begin{corollary}[(Ishida)]
Let $X$ be as above. Then $X$ is Cohen-Macaulay if and only if there
exists $\ell$ such that for any $\rho \in \sum$, we have  
$$
H^j (\sum_\rho , k) =0 \quad \text{for} \quad  j \neq \ell
$$
\end{corollary}

We immediately see that Tamb($\delta$) itself is Chohen-Maculay 
 (cf. Remark after our Prop. \ref{chap1:prop6.6} and Hochster
\cite{keyH5}). On the other 
hand, if Temb$(\triangle) = \mathbb{A}_r$ is the affine space, we get
the results in Reisner (Cohen-Macaulay quotients of polynomial rings,
Advances in Math. 21(1976), 30-43) and Hochster (Cohen-Macaulay
rings, combinatorics and simplicial complexes, in \textit{Ring Theory
  II}, Lecture Notes in Pure and
App. Math. 26(1977). Dekker, pp. 171-223). 

In a special case, the complex $R_X$. already appears in \cite{keyN3}.

\begin{flushright}
Tadao Oda\\
Mathematical Institute\\
Tohoku University\\
Sendai, 980 Japan
\end{flushright}


