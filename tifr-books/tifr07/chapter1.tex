\chapter{Vector groups and linear inequalities}\label{chap1}

\section{Vector groups}\label{chap1:sec1}\pageoriginale

Let $K$ be the field of real numbers and $V$ a vector space of
dimension $\ub{n}$ over $K$. Let us denote element of $V$ by small
Greek letters and elements of $K$ by small Latin letters. The identity
element of $V$ will be denoted by $\ub{0}$ and will be called the zero
element of $V$. We shall also denote by $\ub{0}$ the zero element in
$K$.  

Let $\varepsilon_{1},\ldots,\varepsilon_{n}$ be a base of $V$ so that for
any $\xi\in V$
$$
\xi = \sum_{i} \xi_{i} \varepsilon_i x_{i},\quad x_{i}\in K.
$$
We call $x_{1},\ldots,x_{n}$ the {\em coordinates} of $\xi$. Suppose
$\varepsilon'_{1},\ldots,\xi'_{n}$ is another basis of $V$, then
$$
\varepsilon'_{i}=\sum_{j}\varepsilon_{j}a_{ji},\quad i=1,\ldots,n
$$
where $a_{ji}\in K$ and the matrix $M=(a_{ji})$ is non-singular. If in
terms of $\varepsilon'_{1},\ldots,\varepsilon'_{n}$
$$
\xi=\sum_{i}\varepsilon'_{i}y_{i},\quad y_{i}\in K
$$ 
then it is easy to see that
\begin{equation*}
\begin{pmatrix}
x_{1}\\
:\\
:\\
x_{n}
\end{pmatrix}
=M
\begin{pmatrix}
y_{1}\\
:\\
:\\
y_{n}
\end{pmatrix} \tag{1}\label{c1:eq1}
\end{equation*}

Suppose $\alpha_{1},\ldots,\alpha_{m}$ is any finite set of elements
of $V$. We denote by $L(\alpha_{1},\ldots,\alpha_{m})$ the linear
subspace generated in $V$ by\pageoriginale
$\alpha_{1},\ldots,\alpha_{m}$. This means that
$L(\alpha_{1},\ldots,\alpha_{m})$ is the set of elements of the form
$$
\alpha_{1}x_{1}+\cdots+\alpha_{m}x_{m},\quad x_{i}\in X.
$$
It is clear that $L(\alpha_{1},\ldots,\alpha_{m})$ has dimension
$\leqslant \Min (n,m)$.

Let $R_{n}$ denote the Euclidean space of $n$ dimensions, so that
every point $P$ in $R_{n}$ has coordinates $x_{1},\ldots,x_{n}$,
$x_{i}\in K$. Let $\varepsilon_{1},\ldots,\varepsilon_{n}$ be a basis of $V$
and let $x_{1},\ldots,x_{n}$ be the coordinates of $\xi$ in $V$ with
regard to this basis. Make correspond to $\xi$, the point in $R_{n}$
with coordinates $x_{1},\ldots,x_{n}$. It is then easily seen that
this correspondence is $(1,1)$. For any $\xi\in V$ define the {\em
  absolute value} $|\xi|$ by
$$
|\xi|^{2}=\sum^{n}_{i=1}x^{2}_{i}
$$
where $x_{1},\ldots,x_{n}$ are coordinates of $\xi$. Then $|\;|$
satisfies the axioms of a distance function in a metric space. We
introduce a topology in $V$ by prescribing a fundamental system of
neighbourhoods of $\xi$ to be the set of $\{S_{d}\}$ where $S_{d}$ is
the set of $\eta$ in $V$ with 
\begin{equation*}
|\xi-\eta|<d\tag{2}\label{c1:eq2}
\end{equation*}
$S_{d}$ is called a sphere of radius $d$ and center $\xi$. The
topology above makes $V$ a locally compact abelian group. The closure
$\ob{S}_{d}$ of $S_{d}$ is a compact set. From \eqref{c1:eq1}, it follows that
the topologies defined by different bases of $V$ are equivalent. 

A\pageoriginale subgroup $G$ of $V$ is called a {\em vector
  group}. The closure $\ob{G}$ of $G$ in $V$ is again a vector
group. We say that $G$ is {\em discrete} if $G$ has no limit points in
$V$. Clearly therefore a discrete vector group is closed.

Suppose $G$ is discrete, then there is a neighbourhood of zero which
has no elements of $G$ not equal to zero in it. For, if in every
neighbourhood of zero there exists an element of $G$, then zero is a
limit point of $G$ in $V$. This contradicts discreteness of $G$. Since
$G$ is a group, it follows that all elements of $G$ are isolated in
$V$. As a consequence we see that every compact subset of $V$ has only
finitely many elements of $G$ in it.

We now investigate the structure of discrete vector groups. We shall
omit the completely trivial case when the vector group $G$ consists
only of the zero element.

Let $G\neq \{0\}$ be a discrete vector group. Let $\xi\neq 0$ be an
element of $G$. Consider the intersection
$$
G_{1}=G\cap L(\xi).
$$

Let $d>0$ be a large real number and consider all the $y>0$ for which
$\xi y$ is in $G_{1}$ and $y\leqslant d$. If $d$ is large, then this
set is not empty. Because $G$ is discrete, it follows that there are
only finitely many $y$ with this property. Let $q>0$ be therefore the
smallest real number such that $\xi_{1}=\xi\cdot q\in G_{1}$. Let
$\beta=\xi x$ be any element in $G_{1}$. Put $x=hq+k$ where $h$ is an
integer and $0\leqslant k<q$. Then $\xi x$ and $\xi_{1}h$ are in
$G_{1}$ and so\pageoriginale $\xi k$ is in $G_{1}$. But from
definition of $q$, it follows that $k=0$ or
$$
\beta=\xi_{1}h,\quad h\text{ integer}.
$$
This proves that
$$
G_{1}=\{\xi_{1}\},
$$
the infinite cyclic group generated by $\xi_{1}$.

If in $G$ there are no elements other than those in $G_{1}$, then
$G=G_{1}$. Otherwise let us assume as induction hypothesis that in $G$
we have found $m(\leqslant n)$ elements $\xi_{1},\ldots,\xi_{m}$ which
are linearly independent over $K$ and such that $G\cap
L(\xi_{1},\ldots,\xi_{m})$ consists precisely of elements of the form
$\xi_{1}g_{1}+\cdots+\xi_{m}g_{m}$ where $g_{1},\ldots,g_{m}$ are
integers. This means that
$$
G_{m}=G\cap L(\xi_{1},\ldots,\xi_{m})=\{\xi_{1}\}+\cdots+\{\xi_{m}\}
$$
is the direct sum of $m$ infinite cyclic groups. If in $G$ there exist
no other elements than in $G_{m}$ then $G=G_{m}$. Otherwise let
$\beta\in G$, $\beta\not\in G_{m}$. Put
$$
G_{m+1}=G\cap L(\xi_{1},\ldots,\xi_{m},\beta).
$$

Consider the elements $\lambda$ in $G_{m+1}\subset G$ of the form
$$
\lambda=\xi_{1}x_{1}+\cdots+\xi_{m}x_{m}+\beta y,\quad x_{i}\in K.
$$
where $y\neq 0$ and $y\leqslant d$ with $d$ a large positive real
number. This set $C$ of elements $\lambda$ is not empty since it
contains $\beta$. Put now $x_{i}=g_{i}+k_{i}$ where $g_{i}$ is an
integer and $0\leqslant k_{i}<1$, $i=1,\ldots,m$. Let
$\mu=\xi_{1}g_{1}+\cdots+\xi_{m}g_{m}$, then $\mu\in G_{m}$ and so
$$
\lambda-\mu=\xi_{1}k_{1}+\cdots+\xi_{m}k_{m}+\beta y
$$
is\pageoriginale an element of $G_{m+1}$. Thus for every $\lambda\in
G_{m+1}$ there exists a $\lambda=\lambda-\mu\in G$ with the property
$$
\lambda'=\xi_{1}k_{1}+\cdots+\xi_{m}k_{m}+\beta y
$$
$0\leqslant k_{i}<1$, $y\leqslant d$. Thus all those $\lambda$'s lie
in a closed sphere of radius $(m+d^{2})^{\frac{1}{2}}$. Since $G$ is
discrete, this point set has to be finite. Thus for the $\lambda$'s in
$G$ the $y$ can take only finitely many values.

Therefore let $q>0$ be the smallest value of $y$ for which
$\xi_{m+1}=\xi_{1}t_{1}+\cdots+\xi_{m}t_{m} + \beta q$ is in $G$. Let
$$ 
\lambda=\xi_{1}x_{1}+\cdots+\xi_{m}x_{m}+\beta y
$$
be in $G_{m+1}$. Put $y=qh+k$ where $h$ is an integer and $0\leqslant
k<q$. Then
$$
\lambda-\xi_{m+1}h=\xi_{1}(x_{1}-t_{1}h)+\cdots+\xi_{m}(x_{m}-t_{m}h)+\beta k
$$
is in $G_{m+1}$. By definition of $q$, $k=0$. But in that case by
induction hypothesis $x_{i}-t_{i}h=h_{i}$ is an integer. Thus
$$
\lambda=\xi_{1}h_{1}+\cdots+\xi_{m}h_{m}+\xi_{m+1}h
$$
$h_{1},\ldots,h$ are integers. This proves that
$$
G_{m+1}=\{\xi_{1}\}+\cdots+\{\xi_{m+1}\}
$$
is a direct sum of $m+1$ infinite cyclic groups.

We can continue this process now but not indefinitely since
$\xi_{1},\ldots,\break\xi_{m+1},\ldots$ are linearly independent. Thus after
$r\leqslant n$ steps, the process ends. We have hence the 

\begin{thm}\label{chap1:thm1}
Every\pageoriginale discrete vector group $G\neq \{0\}$ in $V$ is a
direct sum of $r$ infinite cyclic groups, $0<r\leqslant n$.
\end{thm}

Conversely the direct sum of cyclic infinite groups is a discrete
vector group. We have thus obtained the structure of all discrete
vector groups.

We shall now study the structure of all closed vector groups.

Let $G$ be a closed vector group. Let $S_{d}$ be a sphere of radius
$d$ with the zero element of $G$ as centre. Let $r(d)$ be the maximum
number of elements of $G$ which are linearly independent and which lie
in $S_{d}$. Clearly $r(d)$ satisfies
$$
0\leqslant r(d)\leqslant n.
$$
Also $r(d)$ is an increasing function of $d$ and since it is
integral valued it tends to a limit when $d\to 0$. So let 
$$
r=\lim\limits_{d\to 0}r(d).
$$
This means that there exists a $d_{0}>0$ such that for $d\leqslant
d_{0}$ 
$$
r=r(d).
$$
We call $\ub{r}$ the {\em rank} of $G$.

Clearly $0\leqslant r\leqslant n$. Suppose $r=0$, then we maintain
that $G$ is discrete; for if not, there exists a sequence
$\gamma_{1},\ldots,\gamma_{n},\ldots$ of elements of $G$ with a limit
point in $V$. Then the differences $\{\gamma_{k}-\gamma_{1}\}$, $k\neq
1$ will form a set of elements of $G$ with zero as a limit point and
so in every neighbourhood of zero there will be elements of $G$ which
will mean that $r>0$.

Conversely\pageoriginale if $G$ is discrete there exists a sphere
$S_{d}$, $d>0$ which does not contain any point of $G$ not equal to
zero and containing zero. This means $r=0$. Hence
$$
r=0\Leftrightarrow G \text{ \ is discrete.}
$$

Let therefore $r>0$ so that $G$ is not discrete. Let $d$ be a real
number $0<d<d_{0}$ so that $r(d)=r$. Let $S_{d}$ be a sphere around
the zero element of $G$ and of radius $d$. Let
$\alpha_{1},\ldots,\alpha_{r}$ be elements of $G$ in $S_{d}$ which are
linearly independent. Let $t>0$ be any real number and let $d_{1}>0$
be chosen so that $d_{1}<\Min(d,\dfrac{t}{n})$. Then $r(d_{1})=r$. If
$\beta_{1}, \ldots,\beta_{r}$ be elements of $G$ which are linearly
independent and which are contained in the sphere $S_{d_{1}}$ around
the zero element of $G$, then
$L(\beta_{1},\ldots,\beta_{r}) \subset L(\alpha_{1},\ldots,\alpha_{r})$ since
$S_{d_{1}}\subset S_{d}$. But since both have dimension $r$,
$$
L(\beta_{1},\ldots,\beta_{r})=L(\alpha_{1},\ldots,\alpha_{r}).
$$
Since $\beta_{1},\ldots,\beta_{r}$ are in $S_{d_{1}}$ we have 
$$
|\beta_{i}|\leqslant d_{1}\leqslant \frac{t}{n},\quad i=1,\ldots,r.
$$

Let $\xi\in L(\alpha_{1},\ldots,\alpha_{r})$. Then by above
$$
\xi=\beta_{1}x_{1}+\cdots+\beta_{r}x_{r}.
$$
Put $x_{i}=g_{i}+k_{i}$ where $g_{i}$ is an integer and $0\leqslant
k_{i}<1$. Put $\beta=\beta_{1}g_{1}+\cdots+\beta_{r}g_{r}$. Since
$\beta_{1},\ldots,\beta_{r}\in G$, $\beta$ will also be in $G$. Now
\begin{gather*}
|\xi-\beta|=|\beta_{1}k_{1}+\cdots+\beta_{r}k_{r}|\\
\leqslant |\beta_{1}k_{1}|+\cdots+|\beta_{r}k_{r}|<\frac{t}{n}\cdot n=t.
\end{gather*}\pageoriginale

Since $t$ is arbitrary, it means that in every neighbourhood of $\xi$
there are elements of $G$. Hence $\xi\in\ob{G}$. But $G$ is closed and
$\xi$ is arbitrary in $L(\alpha_{1},\ldots,\alpha_{r})$. Thus
$$
L(\alpha_{1},\ldots,\alpha_{r})\subset G.
$$

We have now two possibilities; $r=n$ or $r<n$. If $r=n$ then
$V=L(\alpha_{1},\ldots,\alpha_{r})\subset G\subset V$, which means
$G=V$. So let $r<n$. Complete $\alpha_{1},\ldots,\alpha_{r}$ into a
basis $\alpha_{1},\ldots,\alpha_{n}$ of $V$. In terms of this basis,
any $\gamma\in G$ may be written
$$
\gamma=\alpha_{1}x_{1}+\cdots+\alpha_{n}x_{n},\quad x_{i}\in K.
$$
But $\lambda=\alpha_{1}x_{1}+\cdots+\alpha_{r}x_{r}$ is an element of
$L(\alpha_{1},\ldots,\alpha_{r})$ and so of $G$. Thus
$$
\delta=\gamma-\lambda=\alpha_{r+1}x_{r+1}+\cdots+\alpha_{n}x_{n}
$$
is in $G$. Also $\delta\in L(\alpha_{r+1},\ldots,\alpha_{n})$. It is
to be noted that $\gamma$ determines $\delta$ uniquely. The $\delta$'s
that arise in this manner clearly form a vector group contained in
$L(\alpha_{r+1},\ldots,\alpha_{n})$ and isomorphic to the factor
group $G-L(\alpha_{1},\ldots,\alpha_{r})$. We contend that this
subgroup of $\delta$'s is discrete. For, if not let
$\delta_{1},\ldots$ be a sequence of elements with a limit point in
$V$. Then in every arbitrary neighbourhood of zero there are elements
of the set $\{\delta_{k}-\delta_{l}\}$, $k\neq 1$. Since
$\delta_{k}-\delta_{l}$ is an element of
$L(\alpha_{r+1},\ldots,\alpha_{n})$, this means that the rank of $G$
is $\geqslant r+1$. This contradiction proves our contention. Using
theorem \ref{chap1:thm1}, it follows that there exist $s$ elements
$\delta_{1},\ldots,\delta_{s}$ is $G$ such that every $\xi\in G$ can
be written\pageoriginale uniquely in the form
\begin{equation*}
\xi=\alpha_{1}x_{1}+\cdots+\alpha_{r}x_{r}
+\delta_{1}g_{1}+\cdots+\delta_{s}g_{s}\tag{$\ast$} 
\end{equation*}
where $x_{i}\in K$ and $g$'s are integers. The uniqueness of the above
form implies that $\delta_{1},\ldots,\delta_{s}$ are linearly
independent. We have hence the

\begin{thm}\label{chap1:thm2}
Let $G$ be a closed vector group. There exist integers $r$ and $s$,
$0\leq r\leq r+s\leq n$ and $r+s$ independent elements
$\alpha_{1},\ldots,\alpha_{r},\delta_{1},\ldots,\delta_{s}$ in $G$
such that every element $G$ can be uniquely expressed in the form $(\ast)$.
\end{thm}

It is easy to see that if $G$ is a vector group such that $G$ consists
of all elements $\xi$ of the form $(\ast)$ then $G$ is closed. In
particular if $r=0$, we have discrete groups as a special case.

It can be seen that $L(\alpha_{1},\ldots,\alpha_{r})$ is the connected
component of the zero element in $G$.


\section{Lattices}\label{chap1:sec2}

Let $G$ be a discrete vector group. There exist $r\leq n$ elements
$\alpha_{1},\ldots,\alpha_{r}$ of $V$ such that
$$
G=\{\alpha_{1}\}+\cdots+\{\alpha_{r}\}
$$
is a direct sum of $r$ infinite cyclic groups. If
$\beta_{1},\ldots,\beta_{r+1}$ are any $r+1$ elements of $G$, then
there is a non-trivial relation.
$$
\beta_{1}h_{1}+\cdots+\beta_{r+1}h_{r+1}=0
$$
where $h_{1},\ldots,h_{r+1}$ are integers. For let
$\beta_{i}={\displaystyle{\mathop{\sum}_{j=1}^{r}}}\alpha_{j}a_{ji}$,
    $i=1,\ldots,r+1$. Then the matrix $A=(a_{ji})$ has $r$ rows and
    $r+1$ columns and is an integral matrix. There exist therefore
    rational numbers\pageoriginale $h_{1},\ldots,h_{r+1}$ not all zero
    such that
$$
A\cdot
\begin{pmatrix}
h_{1}\\
\vdots\\
h_{r+1}
\end{pmatrix}
=
\begin{pmatrix}
0\\
\vdots\\
0
\end{pmatrix}
$$
This means that there are rational numbers $h_{1},\ldots,h_{r+1}$ not
all zero such that
$\beta_{1}h_{1}+\cdots+\beta_{r+1}h_{r+1}=0$. Multiplying by a common
denominator we obtain the result stated.

Let us now make the

\begin{defi*}
A vector group $G$ is said to be a lattice if $G$ is discrete and
contains a basis of $V$.
\end{defi*}

This means that there exists a basis $\alpha_{1},\ldots,\alpha_{n}$ of
$V$ such that
\begin{equation*}
G=\{\alpha_{1}\}+\cdots+\{\alpha_{n}\}.\tag{3}\label{c1:eq3}
\end{equation*}
The quotient group $V-G$ is clearly compact. Conversely suppose $G$ is
a discrete vector group such that $V-G$ is compact. If
$\alpha_{1},\ldots,\alpha_{r}$ are independent elements of $G$
generating $G$, complete $\alpha_{1},\ldots,\alpha_{r}$ to a basis
$\alpha_{1},\ldots,\alpha_{n}$ of $V$. A set of representatives of $V
\rm{mod} \; G$ is then given by 
$$
\alpha=\alpha_{1}x_{1}+\cdots+\alpha_{n}x_{n}
$$
where $0\leq x_{i}<1$, $i=1,\ldots,r$. Since $V\rm{mod} \; G$ is compact, it
follows that $r=n$. Thus a lattice is a discrete vector group $G$ with
$V-G$ compact.

A set of elements $\alpha_{1},\ldots,\alpha_{n}$ of $G$ generating $G$
is said to be a {\em base of the lattice} $G$. If
$\beta_{1},\ldots,\beta_{n}$ is another base\pageoriginale of $G$ then
\begin{align*}
(\beta_{1},\ldots,\beta_{n}) &= (\alpha_{1},\ldots,\alpha_{n})A\tag{4}\label{c1:eq4}\\
(\alpha_{1},\ldots,\alpha_{n}) &= (\beta_{1},\ldots,\beta_{n})B
\end{align*}
where $A$ and $B$ are $n$ rowed integral matrices. Because of \eqref{c1:eq4}, it
follows that $AB=E$, $E$ being the unit matrix of order $n$. Thus
$|A|=\pm 1$, $|B|=\pm 1$. 

We call a matrix $A$ {\em unimodular} if $A$ and $A^{-1}$ are both
integral. The unimodular matrices form a group $\Gamma$. \eqref{c1:eq4} shows
that a transformation of a base of $G$ into another base is
accomplished by means of a unimodular transformation.

Conversely if $\alpha_{1},\ldots,\alpha_{n}$ is a base of $G$ and $A$
is a unimodular matrix, then $\beta_{1},\ldots,\beta_{n}$ defined by
$$
(\beta_{1},\ldots,\beta_{n})=(\alpha_{1},\ldots,\alpha_{n})A
$$
form again a base of $G$ as can be easily seen. Thus $\Gamma$ is the
group of automorphisms of a lattice.

Let $G$ be a lattice and $\alpha_{1},\ldots,\alpha_{n}$ a base of
it. Let $\beta$ be any element in $G$. Then $\beta$ can be completed
into a base of $G$ if and only if
$$
G\cap L(\beta)=\{\beta\}
$$
as is evident from section \ref{chap1:sec1}. Let
$\beta=\alpha_{1}g_{1}+\cdots+\alpha_{n}g_{n}$ where
$g_{1},\ldots,g_{n}$ are integers. If $\beta$ can be completed into a
base $\beta$, $\beta_{2},\ldots,\beta_{n}$ of $G$ then, by above, the
transformation taking $\alpha_{1},\ldots,\alpha_{n}$ to $\beta$,
$\beta_{2},\ldots,\beta_{n}$ is unimodular. This means that 
$$
(g_{1},g_{2},\ldots,g_{n})=1.
$$

Conversely\pageoriginale let
$\hat{\beta}=\alpha_{1} g_{1}+\cdots+\alpha_{n}g_{n}$ 
with $(g_{1},\ldots,g_{n})=1$. Let $\beta\in G$ and
$$
G\cap L(\rho)=\{\beta_{1}\}
$$
where $\beta_{1}=\alpha_{1}t_{1}+\cdots+\alpha_{n}t_{n}$. Since
$\beta\in L(\beta)$, it follows that $\beta\in\{\beta_{1}\}$ and
$\beta=\beta_{1}q$ for some integer $q$. Because of independence of
$\alpha_{1},\ldots,\alpha_{n}$, it follows that $q$ divides
$(g_{1},\ldots,g_{n})$. This means that $q=1$, that is
$$
G\cap L(\beta)=\{\beta\}
$$
Therefore $\beta$ can be completed to a base of $G$. Hence the

\begin{thm}\label{chap1:thm3}
Let $G$ be a lattice with a base $\alpha_{1},\ldots,\alpha_{n}$. Let
$\beta=\alpha_{1}g_{1}+\cdots+\alpha_{n}g_{n}$ be an element in
$G$. Then $\beta$ can be completed to a base of $G$ if and only if
$(g_{1},\ldots,g_{n})=1$. 
\end{thm}

From the relation between bases of $G$ and unimodular matrices, we
have

\begin{coro*}
Let $g_{1},\ldots,g_{n}$ be $n$ integers. They can be made the first
column of a unimodular matrix if and only if $(g_{1},\ldots,g_{n})=1$.
\end{coro*}

\section{Characters}\label{chap1:sec3}

Let $G$ be a vector group. A character $\chi$ of $G$ is a real valued
function on $V$ with the properties
\begin{enumerate}
\renewcommand{\labelenumi}{\theenumi)}
\item $\chi(\alpha)$ is an integer for $\alpha\in G$

\item $\chi$ is continuous on $V$

\item $\chi(\alpha+\beta)=\chi(\alpha)+\chi(\beta)$, $\alpha$,
  $\beta\in V$
\end{enumerate}
It follows trivially therefore that
$$
\chi(0)=0.
$$

Since\pageoriginale $\chi$ is a continuous function, we have
$$
\lim\limits_{n}\chi(\lambda_{n})=\chi(\lambda)
$$
where $\lambda_{1}$, $\lambda_{2},\ldots$ is a sequence of elements in
$V$ converging to $\lambda$.

If $p$ is an integer then $\chi(\omega p)=p\chi(\omega)$. If $r$ is a
rational number, say $r=\dfrac{a}{b}$, $a$, $b$ integers, then
$b\chi(\omega r)=\chi(\omega a)=a\chi(\omega)$ so that
$$
\chi(\omega r)=r\chi(\omega).
$$

By continuity it follows that if $r$ is real
$$
\chi(\omega r)=r\chi(\omega).
$$

Suppose $\chi_{1}$ and $\chi_{2}$ are two characters of $G$. Define
$\chi=\chi_{1}+\chi_{2}$ by
$$
\chi(\omega)=\chi_{1}(\omega)+\chi_{2}(\omega).
$$
It is then trivial to verify that $\chi$ is a character of $G$. It
then follows that the characters of $G$ form a group $G^{\ast}$,
called the {\em character group} or the dual of $G$.

Let $G$ be a vector group and $\ob{G}$ its closure. Then 

\begin{lemma*}
$G$ and $\ob{G}$ have the same character group.
\end{lemma*}

\begin{proof}
A character of $\ob{G}$ is already a character of $G$.
\end{proof}

Conversely let $\chi$ be a character of $G$. Then $\chi$ satisfies
properties 2) and 3). We have only to verify the property 1). Let
$\omega\in\ob{G}$. Then there is a sequence of elements
$\omega_{1},\omega_{2},\ldots$ in $G$ with $\omega$ as the
limit. Since $\chi$ is continuous
$$
\lim\limits_{n}\chi(\omega_{n})=\chi(\omega).
$$
But\pageoriginale $\chi(\omega_{n})$ are all integers. Thus
$\chi(\omega)$ is integral. Thus $\chi$ is a character of $\ob{G}$.

The interest in lemma is due to the fact that in order to study the
structure of $G^{\ast}$, it is enough to consider $G^{\ast}$ as the
dual of the closed vector group $\ob{G}$ whose structure we had
investigated earlier.

Let $\ob{G}$ be the closure of the vector group $G$ and $G^{\ast}$ its
character group. By theorem \ref{chap1:thm2} there exists a base
$\omega_{1},\ldots,\omega_{n}$ of $V$ such that
$$
\xi=\omega_{1}x_{1}+\cdots+\omega_{n}x_{n},\quad x_{i}\in K
$$
belongs to $\ob{G}$ if and only if $x_{i}$ is integral for $r<i\leq
r+s$ and $x_{i}=0$ for $i>r+s$, $r$ and $s$ being integers determined
by theorem \ref{chap1:thm2}. If $\chi\in G^{\ast}$ then for $\xi\in V$
$$
\chi(\xi)=x_{1}\chi(\omega_{1})+\cdots+x_{n}\chi(\omega_{n}).
$$
If however $\xi\in \overline{G}$ then $\chi(\xi)$ is
integral. Therefore
$$
\chi (\omega_i) =
\begin{cases}
0 & i\leq r\\
\text{integer} & r<i\leq r+s\\
\text{arbitrary real} & i>r+s.
\end{cases}
$$
Thus for $\xi\in\ob{G}$
$$
\chi(\xi)=\sum^{r+s}_{i=r+1}\chi(\omega_{i})\cdot x_{i}
$$

If $\xi\not\in\ob{G}$, then because of definition of
$\omega_1,\ldots, \omega_n$ it follows that either at least one of
$x_{r+1},\ldots, x_{r+s}$ is not an integer or at least one of
$x_{r+s+1},\ldots, x_n$ is not zero. Suppose that $\xi =
\sum\limits_{i}\omega_1 x_i$, $x_{r+1}\neq 0(\rm{mod} \; 1)$. Define the
linear function $\chi$\pageoriginale on $V$ by
$$
\chi(\omega_i) =
\begin{cases}
1 & \text{ if } i = r+1\\
0 & \text{ if } i \neq r + 1.
\end{cases}
$$

Then $\chi$ is a character of $G$ and 
$$
\chi(\xi) = \chi (\omega_{r+1})x_{r+1} = x_{r+1} \not\equiv 0(\rm{mod} \;
1)
$$
The same thing is true if $x_{r+i}\not\equiv 0 (\rm{mod} \; 1)$, $1\leq i \leq
s$. Suppose now that $\xi=\sum\limits_i \omega_i x_i$ and one of
$x_{r+s+1},\ldots, x_n$ say $x_n\neq 0$. Define $\chi$ linear on $V$
by
$$
\chi(\omega_i)=
\begin{cases}
0 & \text{ if } i \neq n\\
\dfrac{1}{2x_n} & \text{ if } i = n.
\end{cases}
$$

Then $\chi$ is a character of $G$ and $\chi(\xi)=\dfrac{1}{2}\not\equiv
0(\rm{mod} \; 1)$. Hence if $\xi\not\in\ob{G}$ there is a character of $G$
which is not integral for $\xi$. We have thus proved.


\begin{thm}\label{chap1:thm4}
Let $\xi\in V$. Then $\xi \in\ob{G}$ if and only if for every
character $\chi$ of $G$, $\chi(\xi)$ is integral.
\end{thm}

Let us fix a basis $\omega_1,\ldots,\omega_n$ of $V$ so that
$\omega_1,\ldots,\omega_{r+s}$ is a basis of $\ob{G}$. If $\chi \in
G^{\ast}$ then $\chi(\omega_i)=c_i$ where 
$$
c_{i}=
\begin{cases}
0 & i\leq r\\
\text{integer} & r<i\leq r+s\\
\text{real} & i>r+a
\end{cases}
$$
If $(c_{1},\ldots,c_{n})$ is any set of $n$ real numbers satisfying
the above conditions, then the linear function $\chi$ defined on $V$
by
$$
\chi(\xi)=\sum^{n}_{i=1}c_{i}x_{i}
$$
where\pageoriginale $\xi=\sum\limits_{i}\omega_{i}x_{i}$, is a
character of $G$. If $R_{n}$ denotes the space of real $n$-tuples
$(x_{1},\ldots,x_{n})$ then the mapping
$$
\chi\to (c_{1},\ldots,c_{n})
$$
is seen to be an isomorphism of $G^{\ast}$ into $R_{n}$. Thus
$G^{\ast}$ is a closed vector group of rank $n-r-s$.

It can be proved easily that $G^{\ast\ast}$ the character group of
$G^{\ast}$ is isomorphic to $\ob{G}$.

\section{Diophantine approximations}\label{chap1:sec4}

We shall study an application of the considerations in \S \ref{chap1:sec3} to a
problem in linear inequalities.

Let 
$$
L_{i}(h)=\sum^{m}_{j=1} a_{ij}h_{j},\quad (i=1,\ldots,n)
$$
be $n$ linear forms in $m$ variables $h_{1},\ldots,h_{m}$ with real
coefficient $a_{ij}$. Let $b_{1},\ldots,b_{n}$ be $n$ arbitrarily
given real numbers. We consider the problem of ascertaining necessary
and sufficient conditions on the $a_{ij}$'s so that given $a>0$ there
exist integers $h_{1},\ldots,h_{m}$ such that
$$
|L_{i}(h)-b_{i}|<\alpha,\quad (i=1,\ldots,n).
$$

In order to study this problem, let us introduce the vector space $V$
of all $a$ rowed real columns
$$
\alpha=
\begin{pmatrix}
a_{1}\\
\vdots\\
a_{n}
\end{pmatrix},\quad a_{i}\in K.
$$
$V$\pageoriginale has then dimension $n$ over $K$. Let
$\alpha_{1},\ldots,\alpha_{n}$ be elements of $V$ defined by
$$
\alpha_{i}=
\begin{pmatrix}
a_{1i}\\
\vdots\\
a_{ni}
\end{pmatrix},\quad
i=1,\ldots,m
$$
and let $G$ be the vector group consisting of all sums
$\sum\limits^{m}_{i=1}\alpha_{i}g_{i}$ where $g_{i}$'s are
integers. Let $\gamma$ be the vector 
$$
\gamma=
\begin{pmatrix}
b_{1}\\
\vdots\\
b_{n}
\end{pmatrix}
$$
Then our problem on linear forms is seen to be equivalent to that of
obtaining necessary and sufficient conditions that there be elements
in $G$ as close to $\gamma$ as one wishes; in other words the
condition that $\gamma$ be in $\ob{G}$. Theorem \ref{chap1:thm4} now gives
the answer, namely that
$$
\chi(\gamma)\equiv 0(\rm{mod} \; 1)
$$
for every character $\chi$ of $G$.

Let us choose a basis $\varepsilon_{1},\ldots,\varepsilon_{n}$ of $V$ where
$$
\varepsilon_{i}=
\begin{pmatrix}
0\\
\vdots\\
1\\
0\\
\vdots\\
0
\end{pmatrix}\quad 
i=1,\ldots,n
$$
with zero everywhere except at the $i$ th place. Now in terms of this
basis
$$
\alpha_{k}=\varepsilon_{1}a_{1k}+\cdots+\varepsilon_{n}a_{nk},\quad
k=1,\ldots,m
$$
Therefore\pageoriginale if $\chi$ is a character of $G$
$$
\chi(\alpha_{k})=\sum^{n}_{i=1}a_{ik}c_{i}
$$
where $\chi(\varepsilon_{i})=c_{i}$, $i=1,\ldots,n$. Also
$\chi(\varepsilon_{k})\equiv 0(\rm{mod} \; 1)$. Furthermore if
$c_{1},\ldots,c_{n}$ be any real numbers satisfying
$$
\sum^{n}_{i=1}c_{i}a_{ik}\equiv 0(\rm{mod} \; 1),\quad k=1,\ldots,m,
$$
then the linear function $\chi$ defined on $V$ by
$\chi(\varepsilon_{i})=c_{i}$ is a character of $G$. By theorem
\ref{chap1:thm4} therefore
$$
\sum^{n}_{i=1}c_{i}b_{i}\equiv 0(\rm{mod} \; 1)
$$
We have therefore the theorem due to {\em Kronecker}.

\begin{thm}\label{chap1:thm5}
A necessary and sufficient condition that for every $t>0$, there exist
integers $h_{1},\ldots,h_{m}$ satisfying
$$
|L_{i}(h)-b_{i}|<t,\quad i=1,\ldots,n,
$$
is that for every set $c_{1},\ldots,c_{n}$ of real numbers satisfying
$$
\sum^{n}_{i=1}c_{i}a_{ik}\equiv 0(\rm{mod} \; 1),\quad k=1,\ldots,m,
$$
we should have
$$
\sum^{n}_{i=1}a_{i}b_{i}\equiv 0(\rm{mod} \; 1).
$$
\end{thm}

We now consider the special case $m>n$. Let $m=n+q$, $q\geq 1$. Let
the linear forms be
$$
\sum^{q}_{j=1}a_{ij}h_{j}+g_{i},\quad i=1,\ldots,n
$$
in\pageoriginale the $m$ variables
$h_{1},\ldots,h_{q},g_{1},\ldots,g_{n}$. Then the vectors
$\alpha_{1},\ldots,\alpha_{m}$ above are such that
$$
\alpha_{q+i}=\varepsilon_{i},\quad i=1,\ldots,n.
$$
This means that if $\chi$ is a character of $G$,
$c_{i}=\chi(\varepsilon_{i})$ is an integer. Thus

\begin{cor}\label{chap1:coro1}
The necessary and sufficient condition that for every $t>0$, there
exist integers $h_{1},\ldots,h_{q}$, $g_{1},\ldots,g_{n}$ satisfying
$$
\left|\sum^{q}_{j=1}a_{ij}h_{j}+g_{i}-b_{i}\right|<t,\quad i=1,\ldots,n
$$
is that for every set $c_{1},\ldots,c_{n}$ of integers satisfying
$$
\sum_{i}c_{i}a_{ij}\equiv 0(\rm{mod} \; 1),\quad j=1,\ldots,q
$$
we have
$$
\sum_{i}c_{i}b_{i}\equiv 0(\rm{mod} \; 1).
$$
\end{cor}

We now consider another special case $q=1$. The linear forms are of
the type
$$
a_{i}h+g_{i}-b_{i}\quad i=1,\ldots,n
$$
$a_{1},\ldots,a_{n}$, $b_{1},\ldots,b_{n}$ being real numbers. Suppose
now we insist that the condition on $b_{1},\ldots,b_{n}$ be true {\em
  whatever} $b_{1},\ldots,b_{n}$ are. This will mean that from above
Corollary $c_{1}=c_{2}=\ldots=c_{n}=0$ or, in other words, that
$a_{1},\ldots,a_{n}$ have to satisfy the condition that
$$
\sum_{i}c_{i}a_{i}\equiv 0(\rm{mod} \; 1),\quad c_{i}\text{ \  integral}
$$
if\pageoriginale and only if $c_{i}=0$, $i=1,\ldots,n$. This is
equivalent to saying that the real numbers $1$, $a_{1},\ldots,a_{1}$
are linearly independent over the field of rational numbers.

Let us denote by $R_{n}$ the Euclidean space of $n$ dimensions and by
$F_{n}$ the unit cube consisting of points $(x_{1},\ldots,x_{n})$ with
$$
0\leq x_{i}<1\quad i=1,\ldots,n.
$$
For any real number $x$, let $((x))$ denote the fractional part of
$x$, i.o.\@ $((x))=x-[x]$. Then

\begin{cor}\label{chap1:coro2}
If $1$, $a_{1},\ldots,a_{n}$ are real numbers linearly independent
over the field of rational numbers, then the points
$(x_{1},\ldots,x_{n})$ where
$$
x_{i}=((ha_{i}))\quad i=1,\ldots,n
$$
are dense in the unit cube, if $h$ runs through all integers.
\end{cor}

We consider now the homogeneous problem namely of obtaining integral
solutions of the inequalities
$$
|L_{i}(h)|<t,\quad i=1,\ldots,n
$$
$t>0$ being arbitrary. Here we have to insist that
$h_{1},\ldots,h_{m}$ should not all be zero.

We study only the case $m>n$. As before introduce the vector space $V$
of $n$-tuples. Let $\alpha_{1},\ldots,\alpha_{m}$ and $G$ have the
same meaning as before. If the group $G$ is not discrete, it will mean
that the inequalities will have solutions for any $t$,
however\pageoriginale small. If however $G$ is discrete then since 
$m > n$ the elements $\alpha_{1},\ldots,\alpha_{m}$ have to be linearly
integrally dependent. Hence we have integers $h_{1},\ldots,h_{m}$ not
all zero such that
$$
\alpha_{1}h_{1}+\cdots+\alpha_{m}h_{m}=0.
$$
We have hence the

\begin{thm}\label{chap1:thm6}
If $m>n$, the linear inequalities
$$
|L_{i}(h)|<t,\quad i=1,\ldots,n
$$
have for every $t>0$, a non-trivial integral solution.
\end{thm}



\begin{thebibliography}{99}
\bibitem{c1:key1} O.\@ Perron : {\em Irrationalzahlen} Chelsea, 1948.

\bibitem{c1:key2} C. L.\@ Siegel : {\em Lectures on Geometry of Numbers},
 New York University, New York, 1946.
\end{thebibliography}
