\chapter{$B$-Module Theory}\label{chap2}

Let\pageoriginale $k$ be an algebraically closed field. Let $H$ be an
algebraic group over $k$. Let $V$ be vector space over $k$. A group
morphism $H\to GL(V)$ is called a (rational) representation of
$H$. When $G$ is reductive and connected we get a good hold on the
representation theory of $G$ by looking at the representations of its
Borel subgroup $B$. For example, Weyl's 
highest weight theory\index{A}{highest weight theory [8]} in
characteristic zero gives a description of irreducible representations
of $G$ in terms of dominant characters of $B$. In this chapter we
introduce the dual Joseph modules and relative Schubert modules. These
two classes of $B$-modules are analogues of irreducible $G$-modules in
characteristic zero.

\label{page9}
In the first section we prove the Frobenius reciprocity for our
connected reductive group $G$ and its Borel subgroup $B$. Let $\C_{G}$
denote the category of $G$-modules. The reciprocity implies that
$\C_{G}$ is a full subcategory of $\C_{B}$.

In the second section, we introduce the Joseph functor $H_{w}$ on the
category of $B$-modules associated to a Schubert variety $X_{w}\subset
G/B$.

In the third section we introduce the dual Joseph modules. For a
character $\lambda$, let $w=w_{\lambda}$ denote the minimal element of
the Weyl group $W$ such that $w^{-1}\lambda \in X(T)^{-}=\{\mu\in
X(T)\mid (\mu,\alpha)\leq 0$ for all roots $\alpha$ of $B\}$. The dual
Joseph module $P(\lambda)$ is then defined as
$H_{w}(w^{-1}\lambda)$. The relative Schubert module $Q(\lambda)$ is
defined as the kernel of the restriction map from
$H_{w}(w^{-1}\lambda)$ to the sections over the boundary $\p X_{w}$ of
$X_{w}$. 

In positive characteristic we do not have complete reducibility. In
order to ``understand'' the indecomposable $B$-modules we introduce
the concepts of excellent filtrations and relative Schubert
filtrations. Indeed we will be studying the excellent filtrations
extensively throughout these notes.

We\pageoriginale finish this chapter by giving examples of modules
with relative Schubert filtration.

\section{Frobenius Reciprocity}\label{chap2-sec2.1}

\label{page10}
Let $H$ be an algebraic group. We call an $H$-module $M$ 
{\em simple}\index{B}{simple module}
(and the corresponding representation {\em
  irreducible})\index{B}{irreducible representation} if $M\neq 0$
and if $M$ has no $H$-submodules other than $0$ and $M$. It is called
{\em indecomposable}\index{B}{indecomposable} 
if it cannot be decomposed into a direct sum of
two proper $H$-submodules and it is {\em semi-simple} if it is a
direct sum of simple $H$-submodules. For any $M$ the sum of all its
simple submodules is called the {\em socle}\index{B}{socle} of $M$ and denoted by
$\soc_{H}M$ or simply $\soc M$ if it is clear which $H$ is
considered. It is the largest semi-simple\index{B}{semi-simple} 
$H$-submodule of $M$. Each
one-dimensional $H$-module is simple. Let $\C_{B}$ and $\C_{G}$ denote
the categories of $B$-modules and $G$-modules respectively.

For a subgroup $H$ of $G$ and a $G$-module $M$ we can restrict the
action of $G$ to $H$. This functor from $\C_{G}$ to $\C_{H}$ is called
the {\em restriction functor}\index{B}{restriction functor} 
and denoted by $\res^{G}_{H}(?)$. It
takes an exact sequence of $G$-modules to an exact sequence of
$H$-modules and thus it is an exact functor.

Let $G$ be our reductive connected algebraic group. We fix once and
for all a maximal torus $T$ and a Borel subgroup $B$ of $G$ containing
$T$. Let $W$ be the Weyl group of $G$. Recall that our choice of $B$
gives us a set of preferred generators $S=\{s_{1},\ldots,s_{l}\}$ of
$W$, called simple reflections. Let $X(T)$ denote the set of
characters of $T$. Recall that the Weyl group $W$ acts naturally on
characters of $T$ and fix a $W$-invariant inner product on
$X(T)\otimes \mathbb{R}$.

Since $T\subset B$, $\res^{B}_{T}(M)$ is a $T$-module for any
$B$-module $M$. As $T$ is diagonalisable, $M$ then decomposes as a
direct sum of one-dimensional submodules. The character with which
$T$ acts on a one-dimensional submodule is called the {\em weight} of
that submodule. The direct sum of the one-dimensional submodules of
$M$ having the same weight $\lambda$ is called the weight
space\index{B}{weight space} 
$M_{\lambda}$ of $M$. Let $\C_{\leq R}$ denote the category of
$B$-modules all of whose weights are of length not more than $R$ with
respect to the chosen $W$-invariant inner product on $X(T)\otimes
\mathbb{R}$. For a $B$-module $M$, we denote by $M_{\leq R}$ the
largest $B$-submodule of $M$ which is in $\C_{\leq R}$. This defines a
left exact functor from $\C_{B}$ to $\C_{\leq R}$. For example, if
$R=0$, then $M_{\leq R}$ is nothing else than $H^{0}(B,M)$, the
subspace of $B$-fixed vectors in $M$.

\begin{exercise}\label{chap2-exer2.1.1}
Give\pageoriginale an example to show that the functor $M\mapsto
M_{\leq R}$ is not right exact.
\end{exercise}

\label{page11}
For $M\in \C_{B}$, let $\L(M)$ denote the associated $G$-vector
bundle, (possibly infinite dimensional), on $G/B$, as introduced
before. The group $G$ acts on $\L(M)$ and therefore we have a natural
$G$ action on 
$$
H^{0}(G/B,\L(M)),
$$ 
cf.\@ Jantzen \cite[I 5.11
  Remark]{key11}. We call this $G$-module $\ind^{G}_{B}(M)$. Thus we
have a functor $\C_{B}\to \C_{G}$ given by $M\mapsto
\ind^{G}_{B}(M)$. This functor is called the {\em induction
  functor}.\index{B}{induction functor} The reader should note that in
Jantzen's book the 
induction functor is defined more algebraically but for us this
equivalent definition will prove more useful.

If $M$ were a $G$-module then the associated vector bundle $\L(M)$ is
isomorphic with the trivial bundle $G/B\times M$. Further, as $G/B$ is
a complete variety we have $H^{0}(G/B,\L(M))=M$. Therefore if $M\in
\C_{G}$, then $\ind^{G}_{B}(M)=M$.

\begin{remark}\label{chap2-rem2.1.2}
If $P$ is a parabolic subgroup of $G$ then we define in a similar way
the induction functor $\ind^{P}_{B}(?)$ by assigning the $P$-module
$H^{0}(P/B,\L(M))$ to a $B$-module $M$. As before, if $M$ were a
$P$-module, we get $\ind^{P}_{B}(M)=M$.
\end{remark}

\begin{remark}\label{chap2-rem2.1.3}
The fibre over the $B$-fixed point $B/B$ of the vector bundle $\L(M)$
is canonically isomorphic with $M$. Therefore the restriction map
$H^{0}(G/B,\L(M))\to \L(M)|_{B/B}$ gives a natural $B$-equivariant
morphism $\ind^{G}_{B}(M)\to M$. This map is called the {\em
  evaluation map}.\index{B}{evaluation map}
\end{remark}

\begin{exercise}\label{chap2-exer2.1.4}
\begin{enumerate}
\renewcommand{\theenumi}{\roman{enumi}}
\renewcommand{\labelenumi}{(\theenumi)}
\item Prove that the evaluation map $\ind^{G}_{B}(M)\to M$ is an
  isomorphism if $M$ is a $G$-module.

\item Give examples to prove that this map need not be injective and
  need not be surjective.
\end{enumerate}
\end{exercise}

\begin{remark}\label{chap2-rem2.1.5}
The functor $\ind^{G}_{B}(?)$ is left exact and commutes with forming
direct sums, intersections of submodules, and direct
limits\index{B}{direct limit} over 
directed systems. (The latter property helps to understand the meaning
of $\ind^{G}_{B}(M)$ for an infinite dimensional module $M$, as $M$ is
a union of its finite dimensional submodules.) There is a 
transitivity of induction,\index{B}{transitivity of induction} 
that is, if $B\subseteq P$, then
$\ind^{G}_{B}=\ind^{G}_{P}\circ \ind^{P}_{B}$. We also have the
following tensor identity:\index{B}{tensor identity for induction}
$$
\ind^{G}_{B}(M\otimes \res^{G}_{B}(N))=(\ind^{G}_{B}(M))\otimes N
$$
for\pageoriginale any $G$-module $N$ and $B$-module $M$.
\end{remark}

\label{page12}
The Frobenius reciprocity\index{B}{Frobenius reciprocity} says that the
induction functor is right 
adjoint of the restriction functor.

\begin{proposition}[Frobenius reciprocity]\label{chap2-prop2.1.6}
For any $G$-module $N$ and $B$-module $M$ we have
$\Hom_{G}(N,\ind^{G}_{B}(M))=\Hom_{B}(\res^{G}_{B}(N),M)$. 
\end{proposition}

\begin{proof}
Composing $N\to \ind^{G}_{B}(M)$ with the evaluation map
$\ind^{G}_{B}(M)\to M$ gives us a natural map
$\Hom_{G}(N,\ind^{G}_{B}(M))\to
\Hom_{B}(\res^{G}_{B}(N),M)$. Conversely given a $B$-equivariant map
$f:N\to M$ we associate to it a $G$-equivariant map $\tilde{f}:N\to
\ind^{G}_{B}(M)$ by the formula $\tilde{f}(n)=(\overline{g}\mapsto
\overline{(g,f(g^{-1}n))})$. 
\end{proof}

\begin{corollary}\label{chap2-coro2.1.7}
One may view $\C_{G}$ as a full subcategory of $\C_{B}$.
\end{corollary}

\begin{proof}
If $M$, $N\in \C_{G}$ then
$\Hom_{G}(N,M))=\Hom_{G}(N,\ind^{G}_{B}\res^{G}_{B}M))=\Hom_{B}(\res^{G}_{B}(N),\res^{G}_{B}M)$. 
\end{proof}

\begin{remark}\label{chap2-rem2.1.8}
As we will see, many questions about $G$-modules are special cases of
questions about $B$-modules through this identification of $\C_{G}$
with a subcategory of $\C_{B}$.
\end{remark}

\begin{remark}\label{chap2-rem2.1.9}
Here $G$ need not be reductive, of course, and we will not hesitate
to use the result more generally. We will often discuss only $G$
and/or $B$, leaving it to the reader to find the scope of the
arguments. When in doubt, consult \cite{key11}.
\end{remark}

In fact the identification of $\C_{G}$ with a full subcategory of
$\C_{B}$ even works on the level of Ext groups. This is derived from
the corollary using Kempf's Vanishing Theorem A.2.7. Indeed we have

\begin{lemma}\label{chap2-lem2.1.10}
Let $P$ be a parabolic subgroup containing $B$ and let $M$, $N$ be
$P$-modules. Then $\Ext^{i}_{P}(M,N)=\Ext^{i}_{B}(M,N)$ for all $i$.
\end{lemma}

\begin{proof}
In \cite[II Corollary 4.7]{key11} this is stated for $G$ and $P$
instead of $P$ and $B$, but the argument is the same.
\end{proof}

\section{Joseph's Functors}\index{B}{Joseph's functors}\label{chap2-sec2.2}\pageoriginale 
\label{page13}
In characteristic zero, a rational representation of $G$ is completely
reducible. Further, the irreducible $G$-modules are induced up from
irreducible $B$-modules. We do not have such a nice result for
representations of $G$ in characteristic $p>0$. In this section we
define Joseph's functors. These functors will then lead us to study
dual Joseph modules and relative Schubert modules which form some kind
of building blocks for a class of representations of $B$ or $G$,
sharing good properties with the $G$-modules of characteristic $0$.

For a Schubert variety $X_{w}$ of $G/B$, we get a natural $B$ action
on $H^{0}(X_{w},\L(M))$, the sections of the vector bundle
$\L(M)|_{X_{w}}$ over $X_{w}$.

\begin{definition}\label{chap2-defi2.2.1}
The functors $H_{w}:\C_{B}\to \C_{B}$ given by the rule $M\mapsto
H^{0}(X_{w},\L(M))$ are called {\em Joseph's functors}.
\end{definition}

\begin{remark}\label{chap2-rem2.2.2}
The Joseph functors are also defined for Kac-Moody groups using
cohomological algebra. (See \cite{key18}.) They are actually dual to
the functors originally studied by Joseph in \cite{key12}, also with
cohomological algebra. The above definition gives a kind of
``representability'' of the Joseph Functors.
\end{remark}

\begin{remark}\label{chap2-rem2.2.3}
It should be noted that for the element of largest length $w_{0}$ of
$W$, the two functors $H_{w_{0}}$ and $\ind^{G}_{B}$ are the same. (Up
to $\res^{G}_{B}$, which may safely be ignored from now on, because of
Corollay \ref{chap2-coro2.1.7})
\end{remark}

\begin{remark}\label{chap2-rem2.2.4}
We denote by $P_{s}$ the minimal parabolic subgroup associated to a
simple reflection $s\in S$. The Schubert variety $X_{s}\subset G/B$ is
the image of $P_{s}$ under the projection map. It is thus isomorphic
with the complete variety $P_{s}/B$. Also, for any $B$-module $M$ the
vector bundle $\L(M)$ on $P_{s}/B$ is isomorphic with the restriction
of the vector bundle $\L(M)$ to $X_{s}\subset G/B$. We thus get that
the functor $H_{s}:\C_{B}\to \C_{B}$ is the composition of two
functors $\res^{P_{s}}_{B}\circ \ind^{P_{s}}_{B}$. That is, in this
particular case, the module $H_{s}(M)$ is a $P_{s}$-module viewed as a
$B$-module. 
\end{remark}

Recall that for an element $w\in W$, the length $l(w)$ of $w$ is the
length of any of its reduced expressions in the chosen generators. It
is independent of which reduced expression one is using and thus
defines a integer valued function\pageoriginale on $W$. For any $w\in
W$ and $s\in S$, the preferred set of generators, we have: $l(sw)\neq
l(w)$ and in fact $l(sw)$ is either $l(w)+1$ or $l(w)-1$.
\label{page14}
\begin{proposition}\label{chap2-prop2.2.5}
For $s\in S$, $w\in W$ and $M\in\C_{B}$, we have:
\begin{itemize}
\item[\rm(i)] $H_{s}H_{w}(M)=H_{w}(M)$\quad if $l(s\cdot w)=l(w)-1$.

\item[\rm(ii)] $H_{s}H_{w}(M)=H_{sw}(M)$\quad if $l(s\cdot
  w)=l(w)+1$. 
\end{itemize}
\end{proposition}

\begin{proof}
Let $P_{s}$ denote the parabolic subgroup associated to the simple
reflection $s\in W$. Consider the following diagram
\[
\xymatrix{
P_{s}\times^{B}X_{w}\ar[d]^{\pi}\ar[r]^{m} & P_{s}X_{w}\subset G/B\\
P_{S}/B
}
\]
where the morphism $m$ is the multiplication map which descends to the
fibre bundle.
\begin{itemize}
\item[(i):] $l(s\cdot w)=l(w)-1$.

In this case the image of the multiplication map $m$ is
$B_{S}BX_{w}\cup BX_{w}=X_{s\cdot w}\cup X_{w}=X_{w}$ by
\cite[28.3]{key9}. Therefore the natural left action of $P_{s}$ on
$G/B$ leaves $X_{w}$ invariant. The vector bundle $\L(M)$ on $G/B$ has
a natural $G$ (and hence $P_{s}$) action. This gives a natural $P_{s}$
action on $H^{0}(X_{w},\L(M))$. When restricted to $B$, this action
gives the $B$-module action on $H_{w}(M)$. Therefore we have
$H_{w}(M)\in \C_{P_{s}}\subset \C_{B}$. Hence
$\ind^{P_{s}}_{B}(H_{w}(M))=H_{w}(M)$. Also we have
$H_{s}(M)=\res^{P_{s}}_{B}\circ \ind^{P_{s}}_{B}(M)$. Thus by Remark
\ref{chap2-rem2.1.2} we get that $H_{s}H_{w}(M)=H_{w}(M)$. 

\item[(ii):] $l(s\cdot w)=l(w)+1$.

Now the associated fibre bundle over $P_{s}/B$ in the above diagram is
such that the multiplication morphism $m$ is birational and proper
with $P_{s}X_{w}=X_{sw}$. As $X_{sw}$ is normal (cf.\@ \cite{key25}),
this implies 
$$
m_{\ast}\O_{P_{s}\times^{B}X_{w}}=\O_{X_{sw}}
$$ 
(\cite[II Lemma 14.5]{key11}). For a $B$-module $M$, this gives
\begin{align*}
H_{sw}(M) &= H^{0}(X_{sw},\L(M))\\
&= H^{0}(X_{sw},m_{\ast}\O_{P_{s}\times^{B}X_{w}}\otimes \L(M))\\
&= H^{0}(P_{s}\times^{B}X_{w},m^{\ast}\L(M))\\
&= H^{0}(P_{s}/B,\pi_{\ast}m^{\ast}\L(M)) 
\end{align*}
But\pageoriginale we have:
$\pi_{\ast}m^{\ast}\L(M)=\L(H^{0}(X_{w},\L(M)))$. Therefore we get
that:\label{page15}
\begin{align*}
H_{sw}(M) &= H^{0}(P_{s}\times^{B}X_{w},m^{\ast}\L(M))\\
&= H^{0}(P_{s}/B,\L(H^{0}(X_{w},\L(M))))\\
&= H_{s}(H_{w}(M)).
\end{align*}
\end{itemize}

This proves the proposition.
\end{proof}

\begin{exercise}\label{chap2-exer2.2.6}
Prove that $\pi_{\ast}m^{\ast}\L(M)=\L(H^{0}(X_{w},\L(M)))$. 
\end{exercise}

\begin{corollary}\label{chap2-coro2.2.7}
Let $w\in W$ and let $w=s_{i_{1}}\ldots s_{i_{r}}$ be a 
reduced expression.\index{B}{Joseph's functor!and reduced expression} Then $H_{w}=H_{s_{i_{1}}}\circ\cdots\circ H_{s_{i_{r}}}$. 
\end{corollary}

Let $k_{\lambda}$ denote the one-dimensional $B$-module on which $B$
acts via a character $\lambda$. We denote by $\L(\lambda)$ its
associated line bundle and by $H_{w}(\lambda)$ its image under the
Joseph functor $H_{w}(?)$.

\medskip
\noindent
{\bf Extra hypothesis 2.2.8.} For all the questions we are interested
in, one may easily reduce to the case that the commutator subgroup of
our connected reductive algebraic group $G$ is simply
connected.\index{A}{simply connected [9]} This
implies that for each simple root, the corresponding homomorphism from
$SL_{2}$ into $G$ is a closed embedding. (Recall that the other
possibility would be that the image of this homomorphism is isomorphic
to $PSL_{2}$.) Let us assume simply connectedness from now on. Then
any line bundle on $G/B$ is associated to a one-dimensional
representation of $B$ (cf.\@ Corollary \ref{coro-A.4.3})  and if the associated
character $\lambda$ is anti-dominant\index{B}{anti-dominant} 
{\em i.e.} $\lambda\in X(T)^{-}$,
then $\L(\lambda)$ is base point\index{B}{base point} 
free, {\em i.e.} given any point
$x\in G/B$ there exists a global section $s\in H^{0}(G/B,\L(\lambda))$
with $s(x)\neq 0$. Conversely, if $H^{0}(G/B,\L(\lambda))\neq 0$ then
$\L(\lambda)$ is base point free (because of equivariance) and
$\lambda$ is anti-dominant. (See \cite[II 2.6]{key11}, keeping in mind
that his dominant weights are our anti-dominant ones.)

\setcounter{theorem}{8}
\begin{lemma}\label{chap2-lem2.2.9}
For any $\lambda\in X(T)^{-}$, the socle of $H_{w}(\lambda)$ is
one-dimensio\-nal and its character is $w\lambda$.
\end{lemma}

\begin{proof}
The Bruhat decomposition of $G/B$ says that the $B$-orbit of $w$ in
$X_{w}$ is open (and thus dense) in $X_{w}$. Therefore for a
$B$-module $M$ a section of $H_{w}(M)$ on which $B$ acts by a
character is determined uniquely by its image under\pageoriginale the
restriction map $H_{w}(M)\to \L(M)|_{w}$. Therefore, since the fibre
of $\L(\lambda)$ is of dimension one, we can have only one
$B$-invariant (up to scalar multiplication) section of\label{page16}
$H_{w}(\lambda)$. Further, as the restriction is
$T$-equivariant,\index{A}{equivariant [9]} $B$
acts by the character $w\lambda$ on such a section. On the other hand
$H_{w}(\lambda)\neq 0$ because the line bundle is base point free. By
the Borel Fixed Point Theorem, (Theorem \ref{chap1-thm1.1.2}) there
exists a fixed point for the $B$ action on the projective space
$\mathbb{P}(H_{w}(\lambda))$. This proves the existence (cf.\@
\cite[II 2.1]{key11}) of a $B$-invariant one-dimensional subspace of
$H_{w}(\lambda)$. Thus the result.
\end{proof}

\begin{corollary}\label{chap2-coro2.2.10}
Let $\lambda\in X(T)^{-}$. Then, $w'\lambda$ occurs as a weight in
$H_{w}(\lambda)$ for every $w'\leq w$ in the Bruhat
order.\index{A}{Bruhat order [34]}
\end{corollary}

\begin{proof}
Since the line bundle $\L(\lambda)=\L(k_{\lambda})$ is base point
free, the natural restriction map from $H_{w}(\lambda)$ to
$H_{w'}(\lambda)$ is not the zero map. The socle of the image is thus
a non-zero subspace of the socle of $H_{w'}(\lambda)$. The socle of
$H_{w'}(\lambda)$ is of dimension one and has weight
$w'\lambda$. Therefore as the restriction map is $B$-equivariant,
$w'\lambda$ occurs as a weight in $H_{w}(\lambda)$.
\end{proof}

\begin{lemma}\label{chap2-lem2.2.11}
For any two $B$-invariant closed subsets $S$, $S'$ of $G/B$ and any
line bundle without base points $\L$ on $G/B$, we have an exact
Mayer-Vietoris sequence
$$
0\to H^{0}(S\cup S',\L)\to H^{0}(S,\L)\oplus H^{0}(S',\L)\to
H^{0}(S\cap S',\L)\to 0
$$

Moreover the map $H^{0}(G/B,\L)\to H^{0}(S,\L)$ is surjective.
\end{lemma}

\begin{proof}
This Mayer-Vietoris Lemma\index{B}{Mayer-Vietoris Lemma} used Ramanathan
\cite{key31} for the 
surjectivity statements (cf.\@ Proposition \ref{prop-A.2.6}), and it used
Ramanathan once more for knowing that $S\cap S'$ is also the scheme
theoretic intersection, {\em i.e.} that its ideal 
sheaf\index{A}{ideal sheaf [7]} in $G/B$ is
the sum of the ideal sheafs of $S$ and $S'$. This then gives an exact
sequence of sheaves
$$
0\to \I_{S\cup S'}\to \I_{S}\oplus \I_{S'}\to \I_{S\cap S'}\to 0
$$
from which the result follows easily. (The ``unattentive'' reader is
alerted here that one should be worrying that the scheme theoretic
intersection might not be reduced. See the exercise below.)
\end{proof}

\begin{remark}\label{chap2-rem2.2.12}
The\pageoriginale similar\label{page17} Mayer-Vietoris exact sequence is valid on
$G/P$ for any parabolic $P$, cf.\@ Exercise \ref{exer-A.2.9}. The passage from
$G/B$ to $G/P$ is easy as the projection $G/B\to G/P$ is a $P/B$ fibration.
\end{remark}

\begin{exercise}\label{chap2-exer2.2.13}
Find an example of an affine variety $X$ and two closed subvarieties
$S$, $S'$ so that $H^{0}(S\cup S',\O_{X})$ is not the kernel of
$H^{0}(S,\O_{X})\oplus H^{0}(S',\O_{X})\to H^{0}(S\cap
S',\O_{X})$. Here unions and intersections are simply taken set
theoretically. 
\end{exercise}

\begin{definition}\label{chap2-defi2.2.14}
We say a weight occurring in an indecomposable $B$-module is {\em
  extremal} if it has the largest length.
\end{definition}

The modules $H_{w}(\lambda)$ are indecomposable as they have
one-dimensio\-nal socle. The following proposition gives a nice
description of the extremal weights of $H_{w}(\lambda)$.

\begin{proposition}\label{chap2-prop2.2.15}
Let $\lambda\in X(T)^{-}$. The extremal weights of $H_{w}(\lambda)$
are $w'\lambda$ for $w'\leq w$. Further, the weight spaces
corresponding to the extremal weights are one-dimensional.
\end{proposition}

\begin{proof}
The Corollary \ref{chap2-coro2.2.10} says that these $w'\lambda$ occur
as a weight in $H_{w}(\lambda)$. 

For $\lambda\in X(T)^{-}$ the global sections module
$H^{0}(G/B,\L(\lambda))$ is $\neq 0$. We start with showing that the
extremal weights of 
$$
H_{w_{0}}(\lambda):=H^{0}(G/B,\L(\lambda))
$$ 
are $w\lambda$ for $w\in W$.

The module $H^{0}(G/B,\L(\lambda))$ is a $G$-module. Therefore for
every $w\in W$ and every extremal weight\index{B}{extremal weight} 
$\nu$, the character $w\nu$
occurs as a weight of $H^{0}(G/B,\L(\lambda))$. Further $w\nu$ is also
extremal as the inner product on the vector space $X(T)\otimes
\mathbb{R}$ is $W$-invariant. Let $w_{\nu}\in W$ be such that the
character $\nu_{0}=w_{\nu}\nu$ is a dominant\index{B}{dominant} 
character, {\em i.e.}
such that $\nu_{0}\in X(T)^{+}=\{\mu\in X(T)|(\mu,\alpha)\geq 0$ for
all roots $\alpha$ of $B\}$. Now for any positive root $\alpha$
occurring in the Lie algebra of $B$, we consider the corresponding
copy of $SL_{2}$ in $G$ and its Borel subgroup $B_{1}$. The weight
space of $w\nu$ is $B_{1}$-invariant for otherwise (\cite[31.1]{key9})
there would be a weight space with weight $w\nu+i\alpha$, $i>0$, and
such a translate of $w\nu$ will have larger length which will be a
contradiction to the extremalness of $w\nu$. Thus the dominant
extremal weight $\nu_{0}$ occurs in the $B$-socle of
$H^{0}(G/B,\L(\lambda))$ which has weight $w_{0}\lambda$ by Lemma
\ref{chap2-lem2.2.9}. Thus $\nu$ is a $W$-translate of this weight
$w_{0}\lambda$. Also since the socle of $H^{0}(G/B,\L(\lambda))$
is\pageoriginale one-dimensional\label{page18} we see that the weight space for any
extremal weight of $H_{w_{0}}(\lambda)$ is one-dimensional.

The line bundle $\L(\lambda)$ is base point free. Therefore the
restriction map on to sections over a $T$-fixed point $w\cdot B/B$ is
surjective for every $w\in W$. The torus $T$ acts by the character
$w\lambda$ on the fibre of this fixed point. This gives a geometric
description of the one-dimensional extremal 
weight\index{B}{geometric description of extre\-mal weight} space
$H^{0}(G/B,\L(\lambda))_{w\lambda}$, namely it is spanned by ``the''
$T$-semi-invariant\index{A}{semi-invariant [9]} 
section of $\L(\lambda)$ whose restriction to the
fibre $\L(\lambda)|_{wB/B}$ is non-zero. This section vanishes at
$zB/B$ for $z\in W$ with $z\lambda\neq w\lambda$. Note that in
$H^{0}(X_{w},\Lambda(\lambda))$ the restricted section is even
$B$-semi-invariant so that its zero set is a union of the Schubert
varietires $X_{z}$ with $z\leq w$ and $z\lambda\neq w\lambda$.

To see the general case we note that that natural restriction map from
$H_{w_{0}}(\lambda)$ to $H_{w}(\lambda)$ for $w\in W$ preserves the
length of a weight and is surjective  by Ramanathan (Proposition
\ref{prop-A.2.6}). Therefore we see that the weight $w\lambda$ of
$H_{w}(\lambda)$ 
is extremal and any other extremal weight of $H_{w}(\lambda)$ is also
an extremal weight of $H^{0}(G/B,\L(\lambda))$. Now let us be given an
extremal weight $\mu$ of $H_{w}(\lambda)$ and a non-zero section $f$
of weight $\mu$ over $G/B$. Choose $w'$ minimal in the Bruhat order so
that $w'\lambda=\mu$. We claim $w'\leq w$. Otherwise the
Mayer-Vietoris Lemma \ref{chap2-lem2.2.11} gives a $g\in
H^{0}(X_{w'}\cup X_{w},\L(\lambda))_{\mu}$ with the same restriction
to $w'\cdot B/B$ as $f$, but vanishing on $X_{w}$. By Ramanathan
Proposition \ref{prop-A.2.6} this section $g$ lifts to
$H^{0}(G/B,\L(\lambda))_{\mu}$ and thus agrees with $f$, which is
absurd. Here we have been using several times that $T$ is semi-simple,
so that if $M\to N$ is a surjective $T$-module map, $M_{\mu}\to
N_{\mu}$ is surjective for every weight $\mu$ of $N$. 
\end{proof}

\begin{remark}\label{chap2-rem2.2.16}
One can also prove the above proposition by induction on the length of
$w$, using Corollary \ref{chap2-coro2.2.10} and Proposition
\ref{chap2-prop2.2.5}. 
\end{remark}

\section{Dual Joseph Modules}\label{chap2-sec2.3}

For any character $\mu\in X(T)$, there exists an element $w\in W$, the
Weyl group of $G$, such that $\mu_{1}=w\mu\in X(T)^{-}$. We define
$$
P(\mu)=H^{0}(X_{w^{-1}},\L(w\mu)).
$$ 
Thus the socle of $P(\mu)$ is of
dimension one and has weight $\mu$. 

\begin{lemma}\label{chap2-lem2.3.1}
$P(\mu)$\pageoriginale is\label{page19} independent of $w$, i.e.\@ for any $w_{1}$,
  $w_{2}\in W$ with $w_{1}\mu=w_{2}\mu\in X(T)^{-}$, we have
  $H^{0}(X_{w_{1}^{-1}},\L(w_{1}\mu))=H^{0}(X_{w_{2}^{-1}},\L(w_{2}\mu))$. 
\end{lemma}

\begin{proof}
We denote by $\lambda$ the translate of $\mu$ under $W$ such that
$\lambda\in X(T)^{-}$. Then recall (\cite[1.8, 1.10, 1.12]{key9}) that
there exist elements $w_{\min}$ and $w_{\max}$ of the Weyl group $W$
with the property that for any other $w$ with $w\mu=\lambda$, we have
$w_{\min}\leq w\leq w_{\max}$. Now consider the natural restriction
map
$$
H^{0}(X_{w^{-1}_{\max}},\L(\lambda))\to H^{0}(X_{w^{-1}},\L(\lambda)).
$$

Since this map restricts to identity on the socles of the two modules,
socles of both modules are one-dimensional and have weight $\mu$, it
is injective, and it is surjective according to Proposition
\ref{prop-A.2.6}. Thus it is an isomorphism. This proves the proposition.
\end{proof}

\begin{definition}\label{chap2-defi2.3.2}
A $B$-module $M$ is called 
{\em dual Joseph module}\index{B}{dual Joseph module} if $M$ is 
isomorphic with $P(\mu)$ for some character $\mu$.
\end{definition}

\begin{example}\label{chap2-exam2.3.3}
\begin{enumerate}
\item For $\mu\in X(T)^{-}$ we have $P(\mu)=k_{\mu}$, the
  one-dimensional $B$-module with character $\mu$.

\item For $\mu\in X(T)^{+}$ we have $P(\mu)=H^{0}(G/B,\L(w_{0}\mu))$.
\end{enumerate}
\end{example}

\begin{definition}\label{chap2-defi2.3.4}
\begin{enumerate}
\renewcommand{\theenumi}{\roman{enumi}}
\renewcommand{\labelenumi}{(\theenumi)}
\item If $S'\subset S$ are $B$-invariant closed subspaces of $G/B$ and
  $\lambda\in X(T)^{-}$, we define a {\em relative Schubert
  module}\index{B}{relative Schubert module}\break 
  $Q(S,S',\lambda)$ by:
  $$
Q(S,S',\lambda)=\ker(\text{res~:}H^{0}(S,\L(\lambda))\to
  H^{0}(S',\L(\lambda))). 
$$

\item If $X_{w}$ is a Schubert variety its 
``boundary''\index{B}{boundary of Schubert veriety} $\p X_{w}$ is
  defined as the union of all Schubert varieties that are strictly
  contained in $X_{w}$. Thus the boundary is the complement in $X_{w}$
  of the Bruhat cell $BwB/B$.

\item For any $\mu\in X(T)$, we define a {\em minimal relative
  Schubert module}, denoted by $Q(\mu)$ by:
$$
Q(\mu)=\ker(\res:H^{0}(X_{w^{-1}_{\min}},\L(\lambda))\to H^{0}(\p
X_{w^{-1}_{\min}},\L(\lambda))) 
$$
where as before, $\lambda=w_{\min}\mu\in X(T)^{-}$ and $w_{\min}$ is a
minimal such element in $W$. 
\end{enumerate}
\end{definition}

\begin{remark}\label{chap2-rem2.3.5}
Note\pageoriginale that\label{page20} $Q(\mu)\hookrightarrow P(\mu)$. Also, the
geometric description of the extremal weights of $P(\mu)$ tells us
that an extremal weight of $P(\mu)$ other than $\mu$ does not restrict
to zero on the boundary. Therefore $\mu$ is the only extremal weight
of $Q(\mu)$.
\end{remark}

\begin{definition}\label{chap2-defi2.3.6}
A $B$-module $M$ is said to have an {\em excellent
  filtration}\index{B}{excellent filtration} if and 
only if there exists a filtration $0\subset F_{0}\subset
F_{1}\subset\ldots$ by $B$-modules such that $\cup F_{i}=M$ and
$F_{i}/F_{i-1}\approx \oplus P(\lambda_{i})$ for some $\lambda_{i}\in
X(T)$. Here $\oplus$ stands for any number of copies, ranging from
zero copies to infinitely many.
\end{definition}

\begin{remark}\label{chap2-rem2.3.7}
The property of having excellent filtration is closed under extension
for finite dimensional $B$-modules. Thus for any short exact sequence
$0\to M_{1}\to M\to M_{2}\to 0$ of finite dimensional $B$-modules, $M$
has excellent filtration whenever $M_{1}$ and $M_{2}$ both have
excellent filtration.
\end{remark}

In the next chapter, using the cohomological criterion for excellent
filtrations, we will remove the finite dimensionality condition (cf.\@
Corollary \ref{chap3-coro3.2.10}). 

\begin{definition}\label{chap2-defi2.3.8}
A $B$-module $M$ is said to have a 
{\em relative Schubert filtration}\index{B}{relative Schubert filtration}
if and only if there exists a filtration $0\subset F_{0}\subset
F_{1}\subset \ldots$ by $B$-modules such that $\cup F_{i}=M$ and
$F_{i}/F_{i-1}\approx \oplus Q(\lambda_{i})$ for some $\lambda_{i}\in
X(T)$. 
\end{definition}

\begin{remark}\label{chap2-rem2.3.9}
The property of having relative Schubert filtration is also closed
under extension for finite dimensional $B$-modules.
\end{remark}

In the next chapter we use Polo's theorem to give a criterion for
$B$-modules to have an excellent filtration. Here we will now give
examples of modules with relative Schubert filtration.

\begin{lemma}\label{chap2-lem2.3.10}
The relative Schubert module $Q(S,S',\lambda)$ has relative Schubert
filtration for all $B$-invariant closed subsets $S'\subset S$ and any
antidominant character $\lambda$.
\end{lemma}

\begin{proof}
The proof is by induction on the number of Schubert varieties
contained in $S$ but not in $S'$.

First\pageoriginale assume\label{page21} there is just one such Schubert variety,
say $X_{w}$. Then $X_{w}\cap S'=\p X_{w}$ and from the Mayer-Vietoris
Lemma \ref{chap2-lem2.2.11} one gets $Q(S,S',\lambda)=Q(X_{w},\p
X_{w},\lambda)$, which is either zero or $Q(w\lambda)$.

If there are more, choose a $B$-invariant $S''$ strictly between $S$
and $S'$ and consider the following exact sequence.
$$
0\to Q(S,S'')\to Q(S,S')\to Q(S'',S')\to 0.
$$

Note that the exactness of this sequence is due to the Mayer-Vietoris
Lemma \ref{chap2-lem2.2.11}.

By the induction hypothesis both the quotient and the submodule of
$Q(S,S')$ have relative Schubert filtration. Now the Remark
\ref{chap2-rem2.3.9} proves the result.
\end{proof}

Another set of examples of modules with relative Schubert filtration
is given by the following proposition.

\begin{proposition}\label{chap2-prop2.3.11}
For any $B$-invariant closed subset $S$ of $G/B$ and $\lambda \in
X(T)^{-}$, $H^{0}(S,\L(\lambda))$ has a relative Schubert filtration
with layers $Q(w\lambda)$. Moreover $Q(w\lambda)$ occurs only when
$w\lambda$ is an extremal weight of $H^{0}(S,\L(\lambda))$, and has
multiplicity one.\index{B}{multiplicity of a weight}
\end{proposition}

\begin{proof}
The previous proof applies also for empty $S'$ and the rest should be
clear from the discussion.
\end{proof}

\begin{corollary}\label{chap2-coro2.3.12}
The modules $H_{w}(\lambda)$ has relative Schubert filtration for all
$w\in W$ and $\lambda\in X(T)$.
\end{corollary}


