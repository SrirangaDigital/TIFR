\chapter{Other Base Rings}\label{chap7}

In\pageoriginale this\label{page68} chapter we state the earlier results in their
proper generality: The base ring need not be an algebraically closed
field of characteristic $p$, but may in fact be any commutative
ring. In particular it may be the complex number field
$\mathbb{C}$. While for $G$-modules there is nothing to prove in that
case, the results for $B$-modules are also of interest over fields of
characteristic $0$.

\section[The group schemes and the Schubert varieties...]{The group
  schemes and the Schubert varieties over the 
  integers}\label{chap7-sec7.1}  

Recall that over an algebraically closed field $k$ we have been
considering a connected reductive group $G$ together with a maximal
torus $T$, a Borel group $B$ and embeddings of $SL(2,k)$ or $PSL(2,k)$
into $G$ (one for each simple root). Let us assume that $G$ is in fact
semi-simple simply connected, so that we are dealing with embeddings
$\phi_{i}:SL(2,k)\to G$. Now Chevalley and Demazure have shown that
corresponding to this data $(G,T,B,\{\phi_{i}\}_{i\in I})$ over $k$
one gets a group (affine group scheme) $G_{\mathbb{Z}}$ over
$\mathbb{Z}$ with subgroups (closed subgroup schemes)
$T_{\mathbb{Z}}$, $B_{\mathbb{Z}}$ and embeddings of
$SL(2)_{\mathbb{Z}}$ into $G_{\mathbb{Z}}$, such that the situation
over $k$ may be recovered from that over $\mathbb{Z}$ by extension of
scalars from $\mathbb{Z}$ to $k$. One says that $G_{\mathbb{Z}}$ is a
$\mathbb{Z}$-form\index{B}{Z@$\mathbb{Z}$-form} of $G$. More generally, if $S$ is some structure
over $k$, a $\mathbb{Z}$-form $S_{\mathbb{Z}}$ of $S$ is an analogous
structure over $\mathbb{Z}$ together with an isomorphism between $S$
and the structure $S_{k}$ obtained from $S_{\mathbb{Z}}$ by extension
of scalars from $\mathbb{Z}$ to $k$. The group scheme
$SL(2)_{\mathbb{Z}}$ is the affine algebraic group defined over
$\mathbb{Z}$ which represents the functor $R\mapsto SL(2,R)$. The
torus $T_{\mathbb{Z}}$ is diagonalisable.\pageoriginale (This\label{page69} means
that we are discussing ``split'' reductive group schemes.) We write
$G(R)$ for $G_{\mathbb{Z}}(R)$, the group of points rational over the
ring $R$ of the group scheme $G_{\mathbb{Z}}$. For each simple root we
get a homomorphism $\phi_{i}:SL(2,\mathbb{Z})\to G(\mathbb{Z})$. 

\begin{remark}\label{chap7-rem7.1.1}
We do not just try to descend $G$ from $k$ to $\mathbb{Z}$, but $G$
together with $B$, $T$ and the $\phi_{i}$. That is because $G$ has too
many automorphisms, so that there is no canonical ``descent'' for
it. We have ``rigidified'' by also giving the rest of the
data. (Assume the $\mathbb{Z}$-forms $T_{\mathbb{Z}}$ and
$SL(2)_{\mathbb{Z}}$ already chosen.) Thanks to the rigidification we
get a {\em canonical} map from $G(k)$ to the original $G$.
\end{remark}

\begin{remark}\label{chap7-rem7.1.2}
Just as one has a $\mathbb{Z}$-form for $G$, one also has one for
$G/B$. In fact for $(G/B)_{\mathbb{Z}}$ one simply takes
$G_{\mathbb{Z}}/B_{\mathbb{Z}}$. It is also straightforward to get
analogues over $\mathbb{Z}$ of the Demazure resolutions and one may
simply define the Schubert variety $(X_{w})_{\mathbb{Z}}$ to be the
image of
$(P_{s_{1}}\times^{B}\cdots\times^{B}P_{s_{m}}/B)_{\mathbb{Z}}\to
(G/B)_{\mathbb{Z}}$. Unions of Schubert varieties are defined by
intersecting their ideal sheafs. It is not obvious, but true, that
these constructions do indeed yield $\mathbb{Z}$-forms of Schubert
varieties and their unions respectively. In fact, if one looks in
\cite{key11}, one sees that to prove that you really get
$\mathbb{Z}$-forms of Schubert varieties, you should first try to
understand the $H^{0}((X_{w})_{\mathbb{Z}},\L^{n})$ for high powers
$\L^{n}$ of some ample line bundle $\L$ on $(G/B)_{\mathbb{Z}}$.
\end{remark}

\section{Forms of the Modules}\label{chap7-sec7.2}

Because of the technicalities indicated in \ref{chap7-rem7.1.2} it is
best to avoid the $\mathbb{Z}$-forms of Schubert varieties as much as
possible when looking for $\mathbb{Z}$-forms
$P(\lambda)_{\mathbb{Z}}$, $Q(\lambda)_{\mathbb{Z}}$ of the
$B$-modules $P(\lambda)$, $Q(\lambda)$. One can then later exploit the
understanding of the $P(\lambda)_{\mathbb{Z}}$ to get he grips with
the $(X_{w})_{\mathbb{Z}}$ and to make the passage to characteristic
zero. (Passage to characteristic $0$ uses semi-continuity and
constructibility properties, cf.\@ \cite[9.2.6.2, 9.4.2,
  12.2.4]{key6}, and generic flatness. See \cite[II Chapter 14]{key11}
and also \cite{key17}.) Fortunately there is an alternative, thanks to
the Demazure resolution. Indeed one knows--but this is also not
obvious--that $(X_{w})_{\mathbb{Z}}$ is normal, and that leads to the
alternative description of $H^{0}((X_{w})_{\mathbb{Z}},\L)$ as being
$H^{0}((Z_{i})_{\mathbb{Z}},\psi^{*}_{i}\L)$, where
$\psi_{i}:(Z_{i})_{\mathbb{Z}}\to (G/B)_{\mathbb{Z}}$\pageoriginale is
the\label{page70} Demazure resolution of $(X_{w})_{\mathbb{Z}}$. This hopefully
explains our clumsy looking constructions below.

\begin{definition}\label{chap7-defi7.2.1}
For any $\mu\in X(T)$, let $\mathbb{Z}_{\mu}$ denote the
$B_{\mathbb{Z}}$-module corresponding with the character $\mu$. As a
$\mathbb{Z}$-module it is free of rank 1. 
\end{definition}

Given $\lambda\in X(T)$ we choose simple reflections
$s_{1},\ldots,s_{m}$ and anti-dominant $\lambda_{1}$ such that
$\lambda=w\lambda_{1}$, where $w$ has reduced expression $s_{1}\ldots\break
s_{m}$. (We also take $m$ minimal.) Then we define
$$
P(\lambda)_{\mathbb{Z}}=\ind^{P_{1}}_{B}\ind^{P_{1}}_{B}\ldots
\ind^{P_{m}}_{B}\mathbb{Z}_{\lambda_{1}}, 
$$
where we have simplified notation a bit by dropping some of the
subscripts $\mathbb{Z}$. (Everything is to be done over $\mathbb{Z}$.)
We will see later that the notation is justified, by showing that
$P(\lambda)_{\mathbb{Z}}$ does not depend on the choices made here. It
only depends on $\lambda$. Similarly, we define
$Q(\lambda)_{\mathbb{Z}}$ inductively:
$$
Q(\lambda)_{\mathbb{Z}}=F_{1}F_{2}\ldots
F_{m}\mathbb{Z}_{\lambda_{1}},
$$
where
$F_{i}(M):=\mathbb{Z}_{\rho}\otimes_{\mathbb{Z}}\ind^{P_{i}}_{B}(\mathbb{Z}_{-s_{i}\rho}\otimes_{\mathbb{Z}}M)$. The
reader will be asked later to check that this is independent of the
choices made.

\begin{proposition}[Base change]\label{chap7-prop7.2.2}
For any algebraically closed field $k$ of finite characteristic,
$P(\lambda)_{k}$ is the dual Joseph module of highest weight $\lambda$
and $Q(\lambda)_{k}$ is the minimal reltive Schubert module of highest
weight $\lambda$. In other words, $P(\lambda)_{\mathbb{Z}}$ and
$Q(\lambda)_{\mathbb{Z}}$ are indeed $\mathbb{Z}$-forms of what the
notation suggests.
\end{proposition}

\begin{proof}
A universal coefficient theorem (\cite[I 4.18]{key11}) says that we
have an exact sequence
$$
0\to R^{i}\ind^{P_{\mathbb{Z}}}_{B_{\mathbb{Z}}}(N)\otimes k\to
R^{i}\ind^{P_{k}}_{B_{k}}(N_{k})\to
\Tor^{\mathbb{Z}}(R^{i+1}\ind^{P_{\mathbb{Z}}}_{B_{\mathbb{Z}}}(N),k) 
$$
for any parabolic $P$ and any flat ({\em i.e.} torsion free)
$\mathbb{Z}$-module $N$ with $B_{\mathbb{Z}}$ action. So we can pass
to formulas over $k$ whenever the higher derived functors of induction
vanish. And they vanish over $\mathbb{Z}$ if they do over all
$k$. (Observe that a finitely generated $\mathbb{Z}$-module $M$ is
zero if all $M_{k}$ vanish.) Thus, from what we know in finite
charcteristic, we may conclude that, in the notations of
\ref{chap7-defi7.2.1},
$P(\lambda)_{k}=\ind^{P_{1}}_{B}\ind^{P_{1}}_{B}\ldots\ind^{P_{m}}_{B}k_{\lambda_{1}}$. The
result for $P(\lambda)_{k}$ thus follows from Proposition
\ref{chap2-prop2.2.5}. 

For\pageoriginale $Q(\lambda)$\label{page71} we argue similarly. So we must check
over $k$ that the higher derived functors of induction vanish at the
relevant coefficients and that $k_{\rho}\otimes
\ind^{P_{s}}_{B}(k_{-s_{\rho}}\otimes Q(\mu))=Q(s\mu)$ when $s$ is a
simple reflection with $s\mu>\mu$. 

First let us consider an example. Take $\mu=-\rho$. Then
$Q(-\rho)=k_{-\rho}$ and $P(-s\rho)$ is two-dimensional with weights
$\rho$ and $-s\rho$, as the degree of the line bundle $\L(-\rho)$ is
$1$ on $P_{s}/B$. So $Q(-s\rho)=k_{-s\rho}$. In the exact sequence
$0\to Q(-s\rho)\to P(-s\rho)\to P(-\rho)\to 0$ we may interpret
$Q(-s\rho)$ as $H^{0}(P_{s}/B,\I\otimes \L(-\rho))$ where $\I$ is the
ideal sheaf of the point $B/B$. We claim that $\I$, as a
$B$-equivariant sheaf, is just $\L(-s\rho)[\rho]$. (Notations as in
\ref{chap4-defi4.3.14}.) Indeed, if one substitutes that for $\I$, one
finds $H^{0}(P_{s}/B,\I\otimes \L(-\rho))=k_{-s\rho}$. In view of the
classification of $B$-equivariant sheafs (see Lemma \ref{lem-A.4.1}), no other
equivariant line boundle gives that answer. Of course one may also
just compute the action on $\I$ in local co-ordinates.

More generally one thus wants to see that, if $s\mu>\mu$, the
evaluation map $\ind^{P_{s}}_{B}Q(\mu)\to Q(\mu)$ is surjective and
that its kernel $H^{0}(P_{s}/B,\I\otimes \L(Q(\mu)))$ equals
$Q(s\mu)$. (The surjectivity will yield the necessary vanishing of
$H^{1}(P_{s}/B,\I\otimes\L(Q(\mu)))$.) Say $\mu=z\lambda_{1}$,
$\lambda_{1}\in X(T)^{-}$, with $z$ minimal. Now if one has a section
of $Q(\mu)$, then that is a section of
$P(\mu)=H^{0}(X_{z},\L(\lambda_{1}))$, which extends by zero to $\p
X_{sz}$ by the Mayer-Vietoris Lemma \ref{chap2-lem2.2.11}. That
section in turn extends to one of $P(s\mu)$ by Ramanathan (Proposition
\ref{prop-A.2.6}), and if one views it as a section of
$H^{0}(P_{s}\times^{B}X_{z},\L(\lambda_{1}))$, cf.\@ Proposition
\ref{chap2-prop2.2.5}, then it vanishes on $H^{0}(P_{s}\times^{B}\p
X_{z},\L(\lambda_{1}))$ by construction. This shows the
surjectivity. The kernel of the map $\ind^{P_{s}}_{B}Q(\mu)\to Q(\mu)$
consists of sections of $H^{0}(P_{s}\times^{B}X_{z},\L(\lambda_{1}))$
that vanish on $B\times^{B}X_{z}\cup P_{s}\times^{B}\p X_{z}$ and that
is just the same as sections of $P(s\mu)$ that vanish on $\p X_{sz}$. 
\end{proof}

It is worthwhile to make explicit what we have just shown. One may
compare it also with Proposition \ref{chap2-prop2.2.15} and
\ref{chap2-prop2.3.11}. 

\begin{lemma}\label{chap7-lem7.2.3}
If $\mu\in X(T)$ and $s$ is a simple reflection such that $s\mu>\mu$,
then the following sequence is exact:
$$
0\to Q(s\mu)\to H_{s}(Q(\mu))\xrightarrow{\eval}Q(\mu)\to 0.
$$
\end{lemma}

\begin{exercise}\label{chap7-exer7.2.4}
Use the formula $k_{\rho}\otimes \ind^{P_{s}}_{B}(k_{-s\rho}\otimes
Q(\mu))=Q(s\mu)$, valid for $s\mu>\mu$ by the above, to derive a
``Demazure character formula''\index{B}{Demazure character formula} for $Q(\lambda)$, analogous to the one
for $P(\lambda)$ in \cite[II Proposition 14.18]{key11}.
\end{exercise}

\begin{definition}\label{chap7-defi7.2.5}
Just\pageoriginale like\label{page72} before in Definition \ref{chap2-defi2.3.6} we say that a
$B_{\mathbb{Z}}$-module has excellent filtration if it has an
exhaustive filtration whose successive filter quotients are isomorphic
to direct sums of modules $P(\lambda)_{\mathbb{Z}}$. More generally,
if $R$ is any commutative ring we say that a $B_{R}$-module has
excellent filtration if it has an exhaustive filtration whose
successive filter quotients are isomorphic to direct sums of modules
$P(\lambda)_{R}$. 
\end{definition}

\begin{theorem}\label{chap7-thm7.2.6}
Let $M_{\mathbb{Z}}$ be a $B_{\mathbb{Z}}$-module, finitely generated
and flat as a $\mathbb{Z}$-module. Assume that for any algebraically
closed field $k$ of finite characteristic the module $M_{k}$ has
excellent filtration. Then so does $M_{\mathbb{Z}}$.
\end{theorem}

\begin{proof}
First observe that the integers $m_{\lambda}$ in 
$$
\ch(M_{k})=\sum m_{\lambda}\ch(P(\lambda)_{k})
$$ 
do not depend on the characteristic of
$k$ because the $\ch(P(\lambda)_{k})$ are linearly independent and do
not depend on the characteristic. (They are given by the Demazure
character formula,\index{B}{character (formal)} see \cite[II Proposition 14.18]{key11}. Note that
$\ch(P(\lambda)_{k})=e^{\lambda}$ plus terms with weights preceding
$\lambda$ in length-height order.) Fix $\lambda$ minimal in
length-height order with $m_{\lambda}\neq 0$. Then
$\dim_{k}(\Hom_{B_{k}}(P(\lambda)_{k},M_{k}))=m_{\lambda}$ is
independent of the characteristic, so that we expect the injective map
$$
\Hom_{B_{\mathbb{Z}}}(P(\lambda)_{\mathbb{Z}},M_{\mathbb{Z}})\otimes
k\to \Hom_{B_{k}}(P(\lambda)_{k},M_{k})
$$ 
to be an isomorphism. To see
this is indeed so, recall the corresponding universal coefficient
theorem (\cite[I 4.18]{key11}) which says that we have an exact
sequence
$$
0\to H^{i}(B_{\mathbb{Z}},N)\otimes k\to H^{i}(B_{k},N_{k})\to
\Tor^{\mathbb{Z}}(H^{i+1}(B_{\mathbb{Z}},N),k)
$$
for any flat ({\em i.e.} torsion free) $\mathbb{Z}$-module $N$ with
$B_{\mathbb{Z}}$ action.

So we wish to get hold of the $\mathbb{Z}$-module
$H^{i}(B_{\mathbb{Z}},N)$, with
$$
N=\Hom_{\mathbb{Z}}(P(\lambda)_{\mathbb{Z}},M_{\mathbb{Z}}).
$$ 

It is
finitely generated by weight considerations as in \cite[II Prop.\@
  4.10]{key11}. (The weight spaces of the $U$-cohomology are finitely
generated.) Now
\begin{align*}
H^{i}(B_{k},\Hom_{\mathbb{Z}}(P(\lambda)_{\mathbb{Z}},M_{\mathbb{Z}})\otimes
k)&=H^{i}(B_{k},\Hom_{k}(P(\lambda)_{k},M_{k}))\\ 
&=\Ext^{i}_{B_{k}}(P(\lambda)_{k},M_{k})
\end{align*}
vanishes for $i>0$ by the strong form of Polo's theorem.


Next we consider the natural homomorphism
$$
\phi:P(\lambda)_{\mathbb{Z}}\otimes_{\mathbb{Z}}\Hom_{B_{\mathbb{Z}}}(P(\lambda)_{\mathbb{Z}},M_{\mathbb{Z}}))\to
M_{\mathbb{Z}}.
$$

When tensored with $k$ one always gets an isomorphism from a direct
sum of $m_{\lambda}$ copies of $P(\lambda)_{k}$ with a submodule of
$M_{k}$. By the elementary divisors theorem\pageoriginale 
this\label{page73} means the cokernel of $\phi$ is torsion free and thus is a module
as in the theorem, but with smaller rank. The theorem follows by
induction on the rank.
\end{proof}

\begin{corollary}[Uniqueness]\label{chap7-coro7.2.7}
Let $M_{\mathbb{Z}}$ be a $B_{\mathbb{Z}}$-module, finitely generated
and flat as a $\mathbb{Z}$-module. Assume that for any algebraically
closed field $k$ of finite characteristic the module $M_{k}$ is the
dual Joseph module of highest weight $\lambda$. Then $M_{\mathbb{Z}}$
is isomorphic with $P(\lambda)_{\mathbb{Z}}$.
\end{corollary}

\begin{proof}
In the excellent filtration of $M_{\mathbb{Z}}$ we must find
$P(\lambda)_{\mathbb{Z}}$, and nothing else, because of
characters. (Compare the proof of the preceding theorem.) Note that it
follows that the choices made in the construction of
$P(\lambda)_{\mathbb{Z}}$ do not make a difference.
\end{proof}

\begin{exercise}\label{chap7-exer7.2.8}
\begin{itemize}
\item[(i)] Show that $\Ext_{B}(Q(\lambda),Q(\mu))$ vanishes when
  $\lambda=\mu$ and also when $-\lambda$ precedes $-\mu$ in
  length-height order.

\item[(ii)] Now formulate and prove a similar theorem and corollary
  with relative Schubert filtrations.
\end{itemize}
\end{exercise}

\begin{theorem}[Main Theorem; Mathieu
    \cite{key20}]\index{B}{Main Theorem}\label{chap7-thm7.2.9}
Let $R$ be a commutative ring, $M$ a $B_{R}$-module with excellent
filtration. Let $\lambda\in X(T)^{-}$. Then
$\lambda_{\mathbb{Z}}\otimes_{\mathbb{Z}}M$ has excellent filtration.
\end{theorem}

\begin{proof}
As the $P(\mu)_{\mathbb{Z}}$ are flat, it suffices to take
$R=\mathbb{Z}$. By the Local-Global Theorem \ref{chap7-thm7.2.6} it
now follows from Polo's Conjecture \ref{chap6-thm6.2.1} as proved in
the previous chapter.
\end{proof}

In the same vain we get

\begin{theorem}[Restriction Theorem]\index{B}{Restriction Theorem}\label{chap7-thm7.2.10}
Let $R$ be a commutative ring, $M$ a $B_{R}$-module with excellent
filtration. Let $L_{R}$ be the Levi factor of a parabolic,
corresponding with a subset of the simple roots. Then
$\res^{B_{R}}_{L_{R}\cap B_{R}}M$ is an $L_{R}\cap B_{R}$-module with
excellent filtration.
\end{theorem}

\section{Passage to Characteristic
  0}\label{chap7-sec7.3}\pageoriginale

Many\label{page74} properties that have been proved with the help of Frobenius
splittings easily extend to characteristic $0$ by semi-continuity and
constructi\-bility properties as developed in \cite{key6}. We will
illustrate this with an example. Observe however that in
characteristic $0$ our theory says nothing interesting about
$G$-modules because of complete reducibility. On the other hand, the
main theorem certainly gives non-obvious results for $B$-modules. We
do not even know a direct proof that, for anti-dominant $\lambda$ and
dominant $\mu$, the character of $\lambda\otimes P(\mu)$ is a sum of
characters of dual Joseph modules.

We know in finite characteristic that Schubert varieties are
normal. As is well known this yields:

\begin{lemma}\label{chap7-lem7.3.1}
Over the complex numbers Schubert varieties are also normal.
\end{lemma}

\begin{proof}
Let $w\in W$. Let $(X_{w})_{\mathbb{Z}}$ be defined as the closure of
$B_{\mathbb{Z}}wB_{\mathbb{Z}}/B_{\mathbb{Z}}$ in
$G_{\mathbb{Z}}/B_{\mathbb{Z}}$. In other words, the ideal sheaf of
$(X_{w})_{\mathbb{Z}}$ consists of the functions that pull back to
zero on $B_{\mathbb{Z}}wB_{\mathbb{Z}}$. It is clear that
$(X_{w})_{\mathbb{Z}}$ is flat over $\mathbb{Z}$. (We do not really
need that much; generic flatness would have been enough.) Now
$(X_{w})_{\mathbb{C}}$ is obtained by flat extension, and one sees it
is the Schubert variety we want to study. It is reduced, connected,
irreducible of dimension $l(w)$ and it contains $BwB/B$. So by
\cite[9.2.6.2, 12.2.4]{key6} and common sense (for the containment),
there is a neighborhood of the generic point of $\Spec (\mathbb{Z})$,
such that the analogous properties hold for $(X_{w})_{k}$ whenever $k$
is a geometric point of $V$. (That is, $k$ is algebraically closed and
its image in $\Spec(\mathbb{Z})$ lies in $V$.) But then for such a
geometric point of finite characteristic, $(X_{w})_{k}$ cannot be
anything else than a Schubert variety. So it is normal. Now the same
Theorem \cite[12.2.4]{key6} finishes the job.
\end{proof}

\begin{lemma}\label{chap7-lem7.3.2}
The $B_{\mathbb{C}}$-module $P(\lambda)_{\mathbb{C}}$ is indeed
$H^{0}(X_{w},\L_{\lambda_{1}})$, with $w\in W$ and $\lambda_{1}$
anti-dominant such that $\lambda=w\lambda_{1}$.
\end{lemma}

\begin{proof}
As $\mathbb{C}$ is flat over $\mathbb{Z}$, we have
$P(\lambda)_{\mathbb{C}}=\ind^{P_{1}}_{B}\ind^{P_{2}}_{B}\ldots
\ind^{P_{m}}_{B}\mathbb{C}_{\lambda_{1}}$. So what we need is the
analogue of Proposition \ref{chap2-prop2.2.5}. But hat depended on
normality of Schubert varieties, so it goes through.
\end{proof}





