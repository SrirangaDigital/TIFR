\chapter{Premilinaries}\label{chap1}

This\pageoriginale chapter should be taken as a guideline of what
notation and terminology is used later on during the course rather
than giving a complete treatment of the structure theory of reductive
groups. An excellent reference for a detailed discussion of the
contents of the first section is the book [Humphreys: Linear Algebraic
  Groups]. Indeed, most of the material is taken from it.


\section{Reductive Algebraic Groups}\label{chap1-sec1.1}

\label{page1}
Let $k$ be an algebraically closed field. Let $G$ be a variety over
$k$ with the structure of a group on its set of points. We call $G$
an {\em algebraic group} if the maps $\mu:G\times G\to G$, where
$\mu(x,y)=xy$, and $\tau:G\to G$, where $\tau(x)=x^{-1}$, are
algebraic morphisms.

By a morphism of groups we mean an algebraic group homomorphism
between the two varieties. A morphism from an algebraic group $G$ to
$GL(n,k)$ is called a (rational) representation\index{B}{rational representation} of $G$ of dimension
$n$ with underlying vector space $k^{n}$.

The {\em additive group} $\mathbb{G}_{a}$ is the affine line
$\mathbb{A}^{1}$ with the group law $\mu(x,y)\break =x+y$. The {\em
  multiplicative group} $\mathbb{G}_{m}$ is the open affine subset
$k^{\ast}\subset \mathbb{A}^{1}$ with group law $\mu(x,y)=xy$. The set
$GL(n,k)$ of $n\times n$ invertible matrices with entries in $k$ is a
group under matrix multiplication called the {\em general linear
  group}.

A closed subgroup of an algebraic group is an algebraic group. Thus
the {\em special linear group} $SL(n,k)$ of all the matrices of
determinant $1$ in $GL(n,k)$ and\pageoriginale the subgroup $D(n,k)$
of all diagonal matrices are algebraic groups. An algebraic group is
called a torus\index{B}{torus} if it is isomorphic to $D(n,k)$ for some $n$.

\label{page2}

Let $G$ be an algebraic group, $X$ a variety. We say that $G$ {\em
  acts} (rationally) on $X$ if we are given a morphism
$\varphi:G\times X\to X$ such that for $x_{i}\in G$, $y\in X$ we have
$\varphi(x_{1},\varphi(x_{2},y))=\varphi(x_{1}x_{2},y)$ and
$\varphi(e,y)=y$. One usually writes $g\cdot v$ or $gv$ for
$\varphi(g,v)$.

Let $\varphi:G\to GL(n,k)$ be a (rational) representation of an
algebraic group $G$. Then $G$ acts on the affine $n$-space
$\mathbb{A}^{n}$ via this representation,\index{B}{weights of representation} {\em i.e.} $x\cdot
v=\varphi(x)(v)$, and thus on a $n$-dimensional vector space $V$ over
$k$. In this case we call $V$ a (rational) $G$-module. More generally,
if $G$ acts linearly on a $k$-vector space $V$, then $V$ is called a
(rational) $G$-module\index{B}{module (rational)} 
if it is the union of finite dimensional
subspaces on which $G$ acts rationally.

A {\em character}\index{B}{character} of an algebraic group $G$ is a
morphism of algebraic 
groups $\chi:G\to \mathbb{G}_{m}$. We denote the group of characters
of $G$ by $X(G)$.

Let $H$ be a diagonal subgroup (or a subgroup of $GL(n,k)$ which is
diagonalisable). Let $V$ be an $H$-module. Then $V$ decomposes as
direct sum of subspaces $V_{\alpha}$, where $\alpha$ runs over the
character group $X(H)$ of $H$ and
$$
V_{\alpha}=\{v\in V\mid x\cdot v=\alpha(x)v\}.
$$

Those $\alpha$ for which $V_{\alpha}$ is non-zero are called the {\em
  weights} of $V$ and $v\in V_{\alpha}$ is called a weight
vector\index{B}{weight vector} of
weight $\alpha$.

Every algebraic group contains a unique largest connected normal
solvable group. We call this subgroup of $G$ the {\em radical} of
$G$. It is denoted by $R(G)$. A group $G$ is called semi-simple if
$R(G)$ is trivial. The subgroup of $R(G)$ consisting of all
unipotent\index{A}{unipotent [9]}
elements is normal in $G$; we call it the {\em unipotent
  radical}\index{B}{radical} of 
$G$. We denote it by $R_{u}(G)$. We call $G$ reductive if $R_{u}(G)$
is trivial.

The group $SL(n,k)$ is semi-simple and $GL(n,k)$ is
reductive.\index{B}{reductive} Note that any semi-simple
group\index{B}{semi-simple group} is 
automatically reductive. 

From now on we will assume that our group $G$ is connected reductive.

A {\em Borel subgroup}\index{B}{Borel subgroup} of $G$ is a maximal
closed connected solvable 
subgroup of $G$. A connected solvable subgroup of largest possible
dimension in $G$ is of course a Borel subgroup and it is also true
that every Borel subgroup of $G$ has the same dimension. In fact we
have the following stronger theorem:

\begin{theorem}\label{chap1-thm1.1.1}
Let $B$ be any Borel subgroup of $G$. Then $G/B$ is a projective
variety, and all other Borel subgroups are conjugate to $B$. 
\end{theorem}

We\pageoriginale call a closed subgroup of $G$ {\em parabolic} if it
contains a Borel subgroup. The centralizer $C$ of a maximal torus $T$
of $G$ is called a {\em Cartan subgroup}\index{B}{Cartan subgroup} of
$G$. Note that we did not 
pur the condition of it being a connected subgroup of $G$ as it can be
shown that any Cartan subgroup of a connected algebraic group is
connected. For reductive groups, the Cartan subgroup $C_{G}(T)$ equals
$T$.

\label{page3}
We now state the Borel Fixed Point Theorem\index{B}{Borel Fixed Point Theorem} and some of its consequences.

\begin{theorem}[Borel Fixed Point Theorem]\label{chap1-thm1.1.2}
Let $B$ be a connected solvable algebraic group, and $X$ be a complete
variety\index{A}{complete variety [9]} on which $B$ acts. Then $B$ has a fixed point in $X$.
\end{theorem}

From this theorem one can deduce Theorem \ref{chap1-thm1.1.1} and
also:
\begin{itemize}
\item[(i)] All maximal tori, and all Borel subgroups are conjugate.

\item[(ii)] $P$ is parabolic\index{B}{parabolic} subgroup of $G$ if and
  only if $G/P$ is a complete variety.
\end{itemize}

If $S$ is any torus in $G$, we call $N_{G}(S)/C_{G}(S)$ {\em Weyl
  group\index{B}{Weyl group} of $G$ relative to $S$,} where $N_{G}(H)$
and $C_{G}(H)$ 
denote the normalizer and centralizer in $G$ of a subgroup $H$ of
$G$. Since all maximal tori are conjugate, all their Weyl groups are
isomorphic. We call this group the {\em Weyl group
  of $G$}. We denote 
it by $W$. We state here some of the important properties of this
group $W$. Recall that $G$ is a connected reductive algebraic group.
\begin{itemize}
\item[(i)] $W$ is a finite group.

\item[(ii)] $W$ is generated by elements $s_{i}(1\leq i\leq l)$, for
  some $l$, with the following defining relations between them:
  $(s_{i}s_{j})^{m(i,j)}=e$, with $m(i,i)=1$ and $2\leq m(i,j)<\infty$
  for $i\neq j$. A group generated by elements having such defining
  relations is called a {\em Coxeter group}.\index{B}{Coxeter group} 

\item[(iii)] If $\chi\in X(T)$ and $t\in T$ the formula
$$
(\omega \chi)(t)=\chi(n^{-1}tn)
$$
gives us an action of an element $\omega\in W$ on $X(T)$; here $n$
denotes a coset representative of $w$ in $N_{G}(T)$.

\item[(iv)] Since\pageoriginale the real vector space $X(T)\otimes
  \mathbb{R}$ is a $W$-module, we can put a metric on it which is
  invariant under the action of the finite group $W$, {\em i.e.} there
  is an inner product ( , )\label{page4} such that $(w\chi,w\mu)=(\chi,\mu)$ for
  every $\chi,\mu\in X(T)$.

\item[(v)] If we fix a Borel subgroup $B$ and a maximal torus
  $T\subset B$, we get a preferred set of generators of $W$. We call
  them simple reflections.\index{A}{simple reflection [9]} If they are indexed by a (finite) set $I$
  (e.g.\@ the nodes of the Dynkin diagram),\index{A}{Dynkin diagram [9]} then for each $i\in I$, we
  also have a simple root\index{A}{simple root [9]} $\alpha$ and we may choose a homomorphism
  $SL(2,k)\to G$, mapping
$$
\begin{pmatrix}
1 & t\\
0 & 1
\end{pmatrix}
\mapsto x_{\alpha}(t),\q 
\begin{pmatrix}
1 & 0\\
t & 1
\end{pmatrix}
\mapsto x_{-\alpha}(t).
$$
Here if $\beta$ is a root, $x_{\beta}:\mathbb{G}_{a}\to B$ denotes a
conveniently normalized injective homomorphism satisfying
$hx_{\beta}(t)h^{-1}=x_{\beta}(\beta(h)t)$ for $t\in k$, $h\in
T$. (Cf.\@ \cite[Chapters 9, 10]{key34}.) Our homomorphism $SL(2,k)\to
G$ has the property that it has at most $\{1,-1\}$ as kernal and hence
the image is isomorphic to either $SL(2,k)$ or to the quotient
$PSL(2,k)$ of $SL(2,k)$ by this subgroup of order $2$. We note that in
characteristic 2, the above group $\{1,-1\}$ does not differ from
$\{1\}$ and one must replace it by a ``group scheme'' of order 2.
\end{itemize}

If $\varphi:G\to GL(V)$ is a representation, the {\em weights} of $V$
are the images in $X(T)$ of the weights of $\varphi(T)$ in $V$ via the
canonical homomorphism $X(\varphi(T))\to X(T)$. We make $W$ act on
weights of $V$ via this canonical homomorphism.

Let us fix a Borel subgroup $B$ and a maximal torus $T$ of $B$. Let
$W$ denote the Weyl group of $G$ relative to $T$. As we have just
pointed out this choice of $B$ and $T$ gives us a preferred set of
generators of $W$ and for each simple reflection we either have a copy
of $SL(2,k)$ or $PSL(2,k)$ embedded in $G$. Any such subgroup together
with $B$ generateds a parabolic subgroup of $G$. We call these
subgroups {\em minimal parabolic}\index{B}{minimal parabolic} 
subgroups of $G$. If $s_{i}$ is a
simple reflection in $W$ and $P_{i}$ denotes the associated minimal
parabolic subgroup then $P_{i}$ contains a representative of $s_{i}$
in $G$. Note that since $T$ lies in $B$, the double coset $BnB$ is
independent of the choice of $n$ representing a given $w\in W$. We
thus write $BwB$ for this double coset. Its image in $G/B$ is called a
{\em Bruhat cell}\index{B}{Bruhat cell} and the closure of a Bruhat cell
is called a {\em 
  Schubert variety}.\index{B}{Schubert variety} It is a union of Bruhat cells. Any element $w\in
W$ can be expressed as the product\pageoriginale $s_{1}\ldots s_{r}$
for some sequence $\{s_{1},\ldots,s_{r}\}$ of simple reflections in
$W$. If this expression is reduced\index{A}{reduced expression [9]} 
and $P_{i}$ is the minimal
parabolic corresponding with $s_{i}$, then $BwB$ has as its closure
the set $P_{1}\ldots P_{r}$, i.e.\@ the image of $P_{1}\times
\cdots\times P_{r}$ under the multiplication map $G\times\cdots\times
G\to G$.

\label{page5}
\begin{theorem}[Bruhat decomposition]\index{B}{Bruhat decomposition}\label{chap1-thm1.1.3} 
For any reductive group $G$, we have $G=\cup_{w\in W}BwB$, with
$Bw_{1}B=Bw_{2}B$ if and only if $w_{1}=w_{2}$ in $W$.
\end{theorem}

\begin{corollary}\label{chap1-coro1.1.4}
Let $G$ be a reductive group and $B$ be a Borel subgroup of $G$. We
have $G/B=\cup_{w\in W} BwB/B$ with $Bw_{1}B/B=Bw_{2}B/B$ if and only
if $w_{1}=w_{2}$.
\end{corollary}

This decomposition gives a stratification of the smooth projective
variety $G/B$ by the Bruhat cells, the $i^{th}$ stratum being the
union of all Bruhat cells of dimension $i$. A codimension one Schubert
variety of $G/B$ is called a Schubert divisor of $G/B$.

\section{Demazure Desingularisation of $G/B$}\label{chap1-sec1.2}

The projective variety $G/B$ being homogeneous it is smooth. However,
the Schubert varieties are not all smooth subvarieties of
$G/B$. Further, two Schubert divisors\index{B}{Schubert divisor} 
need not intersect transversally
with each other. Demazure constructed a ``desingularisation'' of $G/B$
to overcome this problem. In this section we first discuss Kempf's
approach via the standard modifications. Next we reformulate the
resolution in terms of fibre bundles. It is the latter description
which will be used later.

Recall $G$ is a connected semi-simple or reductive algebraic group
over an algebraically closed field of arbitrary characteristic. We
fixed a maximal torus $T$ and a Borel subgroup $B$ containing $T$. The
unipotent radical of $B$ will be denoted $U$. If $W$ is the Weyl group
of $G$, then we have a preferres set of generators of $W$, called
simple reflections. We typically denote them by $s$ or $s_{i}$. Then
$P_{s}$ or $P_{i}$ denotes the associated (minimal) parabolic subgroup
of $G$. For any parabolic subgroup $Q\supseteq B$ of $G$, by a
Schubert variety in $G/Q$\index{B}{Schubert variety!in $G/Q$} 
we mean the closure of a $B$-orbit in
$G/Q$. We will be dealing mostly with Schubert varieties in $G/B$. The
properties for Schubert varieties in $G/Q$ can be deduced from those
of in $G/B$ by studying the fibration $G/B\to G/Q$.

We\pageoriginale have the Bruhat decomposition $G/B=\cup_{w\in
  W}BwB/B$ of $G/B$ into $B$-orbits. Note that as this is a finite
union, any $B$-invariant irreducible closed subvariety of $G/B$ is a
Schubert variety.

\label{page6}
Let $X_{w}=\overline{BwB}/B$ be a Schubert variety of dimension
$r$. Let $w=s_{1}\ldots s_{r}$ be a reduced expression for $w$. We
also complete it into a reduced expression for the element $w_{N}$ of
maximal length: $w_{N}=s_{1}\ldots s_{r}\break \ldots s_{N}$. Let
$w_{j}=s_{1}\ldots s_{j}$ and $X_{j}=X_{w_{j}}$ be the corresponding
Schubert variety of dimension $j$. Note that $X_{r}=X_{w}$. It is
known (refer to Kempf \cite{key13})
\index{B}{standard modification of Kempf} that the varieties $X_{j}$ are 
saturated for the morphism $\pi_{j}:G/B\to G/P_{j}$ and that $X_{j-1}$
maps birationally\index{A}{birational map [9]} 
onto its image $\pi_{j}(X_{j-1})=\pi_{j}(X_{j})$.

The {\em standard modification} $\varphi_{j}:M_{j}\to X_{j}$ is
defined by the Cartesian square:
\[
\xymatrix{
M_{j}\ar[d]\ar[r]^{\varphi} & X_{j}\ar[d]^{\pi_{j}}\\
X_{j-1}\ar[r]^{\pi_{j}} & \pi_{j}(X_{j-1})
}
\]

Thus $M_{j}$ is a $\mathbb{P}^{1}$-bundle over the
divisor\index{A}{divisor [7]} $X_{j-1}$
in $X_{j}$.

The {\em Demazure resolutions}\index{B}{Demazure resolutions} (or
desingularisations) 
$\psi_{j}:Z_{j}\to X_{j}$ are defined inductively. We start by taking
$Z_{0}=X_{0}$, a point, and $\psi_{0}:Z_{0}\to X_{0}$ the identity
morphism. Then $\psi_{j}:Z_{j}\to X_{j}$ is defined by the diagram
with Cartesian squares:
\[
\xymatrix{
Z_{j}\ar[d]^{f_{j}}\ar[r] & M_{j}\ar[d]\ar[r]^{\varphi} &
X_{j}\ar[d]^{\pi_{j}}\\
Z_{j-1}\ar[r]^{\psi_{j-1}} & X_{j-1}\ar[r]^{\pi_{j}} & \pi_{j}(X_{j-1})
}
\]

Note that $f_{j}:Z_{j}\to Z_{j-1}$ is a $\mathbb{P}^{1}$-bundle being
a pullback of the $\mathbb{P}^{1}$-bundle $X_{j}\to
\pi_{j}(X_{j-1})$. This implies that the $Z_{j}$ are nonsingular by
induction. We also have a section $\sigma_{j}:Z_{j-1}\to Z_{j}$ given
by the inclusion $X_{j-1}\subset X_{j}$. Further $\psi_{j}$ is
birational since $\pi_{j}$ is birational on $X_{j-1}$ and by inductive
hypothesis we can assume $\psi_{j-1}$ is birational.

Since $X_{r}=X_{w}$ we get by this process a standard modification and
Demazure resolution of $X_{w}$. Note that this resolution depends on
the reduced expression chosen for $w$. We also get a Demazure
resolution for $G/B$ by this process as $X_{w_{N}}=G/B$.

We\pageoriginale now prepare to give another description for the
varieties obtained by the desingularisation process. Recall that since
$G\to G/B$ is a principal $B$-fibration, given any $B$-space $X$ ({\em
  i.e.} any variety $X$ such that $B$ acts on it on the left) we can
associate a fibre bundle over $G/B$ with fibre being isomorphic to
$X$. We denote such associated fibre 
bundle\index{B}{associated fibre bundle} by $G\times^{B}X$. It is
defined as the quotient of $G\times X$ given by the equivalence
relation $(g,x)\sim (gb,b^{-1}x)$. Note that the natural left
multiplication action of $G$ on $G\times X$ descends to a left action
on the associated fibre bundle. This action commutes with the
projection morphism and thus the associated fibre bundle is a $G$
fibre bundle on $G/B$.

\label{page7}
\begin{exercise}\label{chap1-exer1.2.1}
\begin{enumerate}
\renewcommand{\theenumi}{\roman{enumi}}
\renewcommand{\labelenumi}{(\theenumi)}
\item This fibre bundle is locally trivial in the Zariski
  topology. (Check that for any $g\in G$ it is trivial over
  $gBw_{0}B/B$, where $w_{0}$ denotes the longest element of the Weyl
  group.)

\item Prove similar statements for $P\times^{B}X$ and for
  $G\times^{P}Y$ where $P$ is a parabolic--always containing $B$-and
  $Y$ is a $P$-space. Here it may help to the familiar with standard
  coordinates in Bruhat cells, as explained for instance in
  \cite[Chapter 10]{key34}. Observe that the fibration $G/B\to G/P$ is
  an example of an associated fibre bundle with $X=P/B$.
\end{enumerate}
\end{exercise}

\begin{remark}\label{chap1-rem1.2.2}
If $X$ is in fact a $G$-space, then the fibre bundle $G\times^{B}X$ is
globally trivial by means of the map $G\times^{B}X\to G/B\times X$
which sends the class of $(g,x)$ to $(gB,gx)$.
\end{remark}

Now each parabolic $P_{i}$ contains $B$, and hence they are
$B$-invariant under the left translation action of $B$ on $G$. The
Demazure desingularisation of $X_{w}$ is the associated fibre bundle
$Z_{r}=P_{1}\times^{B}(P_{2}\times^{B}\cdots \times^{B}P_{r}/B)$ and
the map $\psi_{r}:Z_{r}\to X_{w}$ is the multiplication map defined on
the product $P_{1}\times\cdots\times P_{r}$ which actually descends to
the associated fibre bundle. This description will be very useful for
us later on. It now enables us to say that in the Bruhat decomposition
of $G/B$, the varieties $\overline{BwB}/B$ are birational to the image
of $P_{1}\times^{B}(P_{2}\times^{B}\cdots\times^{B}P_{r}/B)$ under the
multiplication map. Thus the dimension of $X_{w}$ is just the length
of the reduced expression of $w$. More specifically, the subset
$B_{s_{1}}B\times^{B}(Bs_{2}B\times^{B}\cdots \times^{B}Bs_{r}B/B)$ of
$Z_{r}$ maps isomorphically to the Bruhat cell $BwB/B$. (Compare
\cite[Chapter 10]{key34}.) It will also be useful
to\pageoriginale consider the analogue of $Z_{r}$ for words
$s_{1}\ldots s_{r}$ that are not reduced. Then of course one will not
get a birational map.

\label{page8}
To any $B$-module $M$, we can associate a fibre bundle on $G/B$ with
the fibre being isomorphic with $M$ as before. We denote this bundle
on $G/B$ by $\mathfrak{L}(M)$. This $G$ fibre bundle is called the
associated vector bundle\index{B}{associated vector bundle} 
for the given representation. The reader will
see during the course of lectures that this construction will enable
us to use ``geometric'' results to study the representations of $G$
and $B$.







