\chapter{Preface}

\markboth{Preface}{Preface}

These notes are based on a course given at the Tata Institute of
Fundamental Research in the beginning of 1990. The aim of the course
was to describe the solution by O.\@ Mathieu of some conjectures in
the representation theory of semi-simple algebraic groups. These
conjectures concern the inner structure of dual Weyl modules and some
of their analogues.

Recall that if $G$ is a (connected, simply connected) semi-simple
complex Lie group and $B$ a Borel subgroup, the Borel--Weil--Bott
Theorem tells that one may construct the finite dimensional
irreducible $G$-modules in the following way. Take a line bundle $\L$
on the generalized flag variety $G/B$, such that $H^{0}(G/B,\L)$ does
not vanish. Then $H^{0}(G/B,\L)$ is an irreducible $G$-module, called
a dual Weyl module or an ``induced module'', and by varying $\L$ one
gets all finite dimensional irreducibles.

More generally one may, after Demazure, consider the $B$-modules
$H^{0}(\overline{BwB}/B,\L)$ with $\L$ as above.\,(So one still
requires that $H^{0}(G/B,\L)$ does not vanish.) The ``Demazure
character formula'' determines the character of
$H^{0}(\overline{BwB}/B,\L)$. It was shown by P.\@ Polo that the
$B$-module $H^{0}(\overline{BwB}/B,\L)$ has a nice homological
characterization in terms of its highest weight $\lambda$ (see
\ref{chap3-thm3.1.10}). We therefore use the notation $P(\lambda)$ for
this module. The $P(\lambda)$ are generalizations of dual Weyl
modules. Indeed recall that nothing is lost when restricting a
rational module from $G$ to $B$; inducing back up from $B$ to $G$ one
recovers the original module (see \ref{chap2-coro2.1.7}).

Now the conjectures are about filtrations of the dual Weyl modules
$H^{0}(G/B,\L)$ or their generalizatoins $P(\lambda)$, for semi-simple
algebraic groups in arbitrary characteristic. (Over the integers,
actually.) The strongest conjecture of the series is Polo's
conjecture, which says that if one twists a $P(\lambda)$ by an
anti-dominant character the resulting $B$-module can be filtered with
subsequent quotients $P(\mu_{i})$. In Polo's terminology--which we
will follow--the twisted module has an {\em excellent filtration}. (In
Mathieu's terminology the twisted module is {\em strong}.)

This conjecture, now a theorem of Mathieu, has many nice
consequences. For instance, suppose one takes a semi-simple subgroup
$L$ of $G$ corresponding with a subset of the set of simple
roots. Then if one restricts the representation $P(\lambda)$ from $B$
to $B\cap L$, that restriction has excellent filtration. For the case
of dual Weyl modules this confirms Donkin's conjecture that the
restriction to $L$ of a dual Weyl module has ``good filtration'', {\em
  i.e.} a filtration whose successive quotients are dual Weyl modules
again. (Unlike the preceding statements, this is not interesting in
the case of semi-simple Lie groups, where any finite dimensional
$L$-module has good filtration, because of complete reducibility.)

Another consequence is a solution of the well-known problem of showing
that the tensor product of two modules with good filtration has good
filtration. This problem was around at least since 1977 when J.E.\@
Humphreys was drawing attention to it. Actually Mathieu has to solve
this problem first, before settling Polo's conjecture. Mathieu's proof
was the first that did not need to exclude any cases. (And this was
achieved by not having any case distinctions to begin with.) Later a
proof has been found that uses the canonical bases of Lusztig (=
crystal bases of Kashiwara).

A different type of consequence, amply demonstrated in the works of
Donkin, is that many results can be carried over from characteristic
$0$ to characteristic $p$. This is because modules with excellent
filtration have nice cohomological properties and thus nice base
change properties. (But observe that the proofs by Mathieu start at
the other end and rely very much on characteristic $p$ methods.)

Although the subject of the course is the contribution of Mathieu, one
should of course not forget the work of Wang, Donkin, Polo,\ldots that
prepared the way. This story is not told here. To exacerbate things,
but in keeping with established practice, our choice of names of
mathematicians in terminology is quite arbitrary. We encourage the
reader to check the references for all the things that are left out.

As is already evident from the above, we place much emphasis on
$B$-modules (more than Mathieu did). Indeed we believe a good setting
for the theory is provided by the category of $B$-modules, enriched
with the tensor product operation and also with a {\em highest weight
  category structure} (in the sense of Cline, Parshall, Scott
\cite{key2}) with the $P(\lambda)$ as ``induced modules''. In the
lectures the highest weight category structure was simply disguised as
a particular total ordering of the weights, dubbed ``length-height
order''. (Weights are ordered by length according to a Weyl group
invariant inner product, and then for fixed length by height.) Indeed
no derived categories are found in the notes.

In \cite{key35} we identified another class of $B$-modules. The module
in this class with highest weight $\mu$ we call $Q(\mu)$. It is
related to the $P(\lambda)$ by the following type of duality:
$$
\dim(\Ext^{i}_{B}(Q(\mu)^{*},P(\lambda)))=
\begin{cases}
1, & \text{if~ } i=0\text{~ and~ } \lambda=-\mu;\\
0, & \text{otherwise.}
\end{cases}
$$

The interaction between the $P(\lambda)$ and the $Q(\nu)$ has much
relevance for the filtration conjectures.

Mathieu's proof of these conjectures involves an innovative way to
exploit Frobenius splittings on Bott-Samelson-Demazure-Hansen
resolutions of Schubert varieties and some of their
generalizations. It was interesting to be lecturing about Frobenius
splittings at TIFR, with the originators of that theory in the
audience.

\medskip
\noindent
{\bf Warning.} When we speak of highest weight, we are using the
ordering in which the roots of $B$ are positive. This is opposite to
the choice in much of the recent literature, but we hope the reader
agrees that in our situation--where the main concern is modules
$P(\lambda)$ with one-dimensional socles generated by a highest weight
vector of weight $\lambda$--it would be silly to reverse the
ordering. 

The lectures given in Bombay have served as a starting point for the
present notes, but is was not a straightforward job to convert the
oral story into something organized and intelligible. I am very
grateful to S.P. Inamdar who wrote the main body of the notes. He
smoothed out many rough spots and mercifully removed some of my less
fortunate variations.

Finally, it is a pleasure to thank colleagues and staff at TIFR for
providing such a friendly environment for us visitors.
\vskip 2cm

\noindent
Utrecht 1993,\hfill {\large\bf Wilberd van der Kallen}

\hfill {\em e-mail: vdkallen@math.ruu.nl}


