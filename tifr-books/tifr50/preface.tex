\thispagestyle{empty}
\begin{center}
{\Large\bf Lectures on}\\[5pt]
{\Large\bf Disintegration of Measures}
\vfill

{\bf By}
\medskip

{\large\bf L. Schwartz}
\vfill

{\bf Tata Institute of Fundamental Research}
\medskip

{\bf Bombay}
\medskip

{\bf 1976}
\end{center}
\eject

\thispagestyle{empty}

\begin{center}
{\Large\bf Lectures on}\\[5pt]
{\Large\bf Disintegration of Measures}
\vfill

{\bf By}
\medskip

{\large\bf L. Schwartz}
\vfill


\vfill

{\bf Notes by}
\medskip

{\large\bf S. Ramaswamy}
\vfill


{\bf Tata Institute of Fundamental Research}
\medskip

{\bf Bombay}
\medskip

\medskip
{\bf 1975}
\end{center}
\eject

\thispagestyle{empty}
\begin{center}
~
\phantom{a}


\vfill

{\bf \copyright \quad Tata Institute of Fundamental Research, 1975}
\vfill

\parbox{0.7\textwidth}{%
No part of this book may be reproduced
in any form by print, microfilm or any
other means without written permission
from the Tata Institute of Fundamental
Research, Colaba, Bombay 400005}

\end{center}
\eject

\chapter{Preface}


THE CONTENTS OF these Notes have been given as a one month course of
lectures in the Tata Institute of Fundamental Research in\break March
1974. The present reduction by Mr. S. Ramaswamy gives under a very
short form the main results of my paper ``Surmartingales
r\'eguli\`eres \`a valeurs mesures et d\'esint\'egrations
r\'eguli\`eres d'une measure'' which appeared in the Journal d'Analyse
Mathematique, Vol XXVI, 1973. For a first reading these Lecture Notes
are better than the complete paper which however contains more
results, in more general situations. Nothing is said here about
stopping times. On the other hand, the integral representation with
extremal elements of \S 3 in Chapter VII here, had not been published
before.
\vskip 1cm

\hfill{\large\bf L. Schwartz}

\eject


\chapter{Note}

The material in these Notes has been divided into two parts. In part
I, disintegration of a measure with respect to a single
$\sigma$-algebra has been considered rather extensively and in part
II, measure valued supermartingales and regular disintegration of a
measure with respect to an increasing right continuous family of
$\sigma$-algebras have been considered. The definition, remarks,
lemmata, propositions, theorems and corollaries have been numbered in
the same serial order. To each definition (resp. remark, lemma,
proposition, theorem, corollary) there corresponds a triplet (a, b,
c) where `$a$' stands for the chapter and `$b$' the section in which
the definition (resp. remark, lemma, proposition, theorem, corollary)
occurs and `$c$' denotes its serial number. References to the
bibliography have been indicated in square brackets. 

The inspiring lectures or Professor L. Schwartz and the many
discussions I had with him, have made the task of writing these Notes
easier.
\vskip 1cm

\hfill{\large\bf S. Ramaswamy}
