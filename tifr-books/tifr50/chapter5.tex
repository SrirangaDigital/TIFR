
\chapter[Existence and Uniqueness of...]{Existence and Uniqueness of
  regular modifications of measure 
  valued supermartingales}\label{part2:chap5}

\section{Uniqueness theorem}\label{part2:chap5:sec1}

Throughout\pageoriginale this section, let $(\Omega, \mathscr{O},
\lambda)$ be a measure space. Let $(Y, \mathscr{Y})$ be a measurable
space. Let $\nu$ be a measure valued supermartingale on $\Omega \times
\mathbb{R}$  with values in $\mathfrak{m}^+ (Y, \mathscr{Y})$ adapted
to an increasing right continuous family $(\mathscr{C}^t)_{t
  \in\mathbb{R}}$  of sub $\sigma$-algebras of
$\hat{\mathscr{O}}_\lambda$. Let $\nu$ take a point $(w,t)$ of $\Omega
\times \mathbb{R}$ to the measure $\nu^t_w$ on $\mathscr{Y}$. Let
$\forall \; t$,  $J^t = \int \nu^t_wd \lambda(w)$ and let $J =
\sup\limits_{t \in \mathbb{R}} J^T$. 

\begin{thm}[Uniqueness]\label{part2:chap5:thm76}
Let $J$ be a finite measure on $\mathscr{Y}$ and let $\mathscr{Y}$
have the $J$-countability property. If $\chi$ and $\Psi$ are two
regular modifications of $\nu$, then $\forall_\lambda w$, $\forall t$,
$\chi^t_w = \Psi^t_w$. 
\end{thm}

\begin{proof}
Since $\chi$ and $\Psi$ are regular, $\chi(1)$ and $\Psi(1)$ are
regular supermartingales. Hence $\forall_\lambda w$, $\forall t$,
$\chi^t_w(1)$ and $\Psi^t_w(1)$ are finite. Hence $\forall_\lambda w$,
$\forall \; t$, $\chi^t_w$ and $\Psi^t_w$ are finite measures on
$\mathscr{Y}$. Thus, there exists a set $N_1 \in \mathscr{O}$ with
$\lambda(N_1) =0$ such that if $w \not\in N_1$, $\chi^t_w$ and
$\Psi^t_w$ are finite measures on $\mathscr{Y}$ for all $t \in
\mathbb{R}$.

Since $\mathscr{Y}$ has the $J$-countability property, there exists a
set $\mathbb{N} \in \mathscr{Y}$ such that the $\sigma$-algebra
$\mathscr{Y}'= \mathscr{Y} \cap \complement N$ on $Y' = Y \cap
\complement N$ is countably generated. Let $\mathscr{B}$ be a
countable class generating $\mathscr{Y}'$. Without loss of generality,
we can assume that $\mathscr{B}$ is a $\pi$-system, containing $Y'$. 

Let $B \in \mathscr{B}$. The supermartingales $\chi(B)$ and $\Psi(B)$
are both regular modifications of the supermartingale $\nu(B)$. Hence,
by remark (\ref{part2:chap4}, \S\ \ref{part2:chap4:sec1},
\ref{part2:chap4:rem64}), 
$$
\forall_\lambda w, \; \forall t, \; \chi^t_w (B) = \Psi^t_w (B). 
$$
Thus, $\forall \; B \in \mathscr{B}$, $\forall_\lambda w$, $\forall \;
t$, $\chi^t_w(B) = \Psi^t_w(B)$. 


Since\pageoriginale $\mathscr{B}$ is countable,
$$
\forall_\lambda w, \forall B \in\mathscr{B}, \forall \; t, \chi^t_w
(B) = \Psi^t_w (B). 
$$
Hence there exists a set $N_2 \in \mathscr{O}$, with $\lambda(N_2)=0$
such that if $w \not\in N_2$, $\forall t \in \mathbb{R}$, $\forall B
\in \mathscr{B}$, $\chi^t_w (B) = \Psi^t_w(B)$. Let $w \in
\Omega$. Consider the class $\mathscr{C}_w$ of all sets $C \in
\mathscr{Y}'$ such that $\forall t$, $\chi^t_w(C) = \Psi^t_w(C)$. If
$w \not\in N_1 \cup N_2$, the class $\mathscr{C}_w$ is a $d$-system
containing the $\pi$-system $\mathscr{B}$. Hence, by the Monotone
class theorem, for $w \not\in N_1 \cup N_2$, $\mathscr{C}_w$ contains
the $\sigma$-algebra generated by $\mathscr{B}$ which is
$\mathscr{Y}'$. Thus, if $w \not\in N_1 \cup N_2$, $\forall A \in
\mathscr{Y}'$, $\forall t$, $\chi^t_w(A) = \Psi^t_w(A)$.

Now, since $J(N) = 0$, $\forall t$, $J^t (N) =0$. 
$$
J^t (N) = \int \chi^t_w (N) d\lambda(w). 
$$
Hence $\forall t$, $\forall_\lambda w$, $\chi^t_w(N) =0$. Therefore,
$\forall_\lambda w$, $\forall t \in\mathbb{Q}$, $\chi^t_w(N)
=0$. Since $\forall_\lambda w$, $t \to \chi^t_w(N)$ is right
continuous, it follows that,
$$
\forall_\lambda w, \; \forall t, \; \chi^t_w(N) = 0. 
$$
Similarly, $\forall_\lambda w$, $\forall t$, $\Psi^t_w(N) =0$.

Hence, there exists a set $N_3 \in \mathscr{O}$ with $\lambda(N_3) =0$
such that if $w \not\in N_3$,
$$
\forall \; t , \chi^t_w(N) = \Psi^t_w(N) = 0.
$$ 

Let $w \not\in N_1 \cup N_2 \cup N_3$. Let $A \in \mathscr{Y}$.
$$
A = A \cap Y' \cup A \cap N. 
$$
\begin{align*}
\text{If } \quad t \in \mathbb{R}, \; \chi^t_w(A) & = \chi^t_w(A \cap
Y')\\
& = \Psi^t_w(A \cap Y')\\
& = \Psi^t_w(A). 
\end{align*}

Hence, if $w \not\in N_1 \cup N_2 \cup N_3$, $\forall t \in
\mathbb{R}$, $\forall A \in \mathscr{Y}$, $\chi^t_w(A) =
\Psi^t_w(A)$. Therefore,
$$
\forall_\lambda w, \; \forall t, \chi^t_w = \Psi^t_w \text{ on }
\mathscr{Y}. 
$$
\end{proof}


\section{Existence theorem}\label{part2:chap5:sec2}

Throughout\pageoriginale this section, let us assume that $(\Omega,
\mathscr{O}, \lambda)$ is a measure space, $(\mathscr{C}^t)_{t \in
  \mathbb{R}}$ is an increasing right continuous family of
$\sigma$-algebras contained in $\hat{\mathscr{O}}_\lambda$, $Y$ is a
topological space, $\mathscr{Y}$ is its Borel $\sigma$-algebra, $\nu$
is a measure valued supermartingale on $\Omega \times \mathbb{R}$ with
values in $\mathfrak{m}^+(Y, \mathscr{Y})$, adapted to
$(\mathscr{C}^t)_{t \in \mathbb{R}}$, $J^t = \int \nu^t_w d\lambda
(w)$ and $J = \sup\limits_{t \in \mathbb{R}} J^t$.  

\begin{defn}\label{part2:chap5:def77}
We say $t \to J^t$ is {\em right continuous} if $\forall$ function $f$
on $Y$, $f \geq 0$, $f \in \mathscr{Y}$, $t \to J^t(f)$ is right
continuous. 
\end{defn}

\begin{thm}[Existence]\label{part2:chap5:thm78}
Let $J$ be a finite measure on $\mathscr{Y}$ and let $t \to J^t$ be
right continuous. Let $Y$ have the $J$-compacity metrizability
property. Then, there exists a regular modification of $\nu$. 
\end{thm}

\begin{proof}
Since $J$ is a finite measure, $\forall t$, $J^t$ is also a finite
measure. Hence $\forall t$, $\forall_\lambda w$, $\nu^t_w $ is a
finite measure. Define the measure valued function $\nu'$ on $\Omega
\times \mathbb{R}$ as 
$$ 
{\nu'}^t_w = 
\begin{cases}
\nu^t_w, & \text{ if } \nu^t_w \text{ is a finite measure}\\
0, & \text{ otherwise}.
\end{cases}
$$

Then, $\nu'$ is a modification of $\nu$ and for all $w \in \Omega$,
for all $t \in \mathbb{R}$, ${\nu'}^t_w$ is a finite measure on
$\mathscr{Y}$. $\nu'$, being a modification of $\nu$, has the same
$J^t \;\; \forall t$ and the same $J$. Hence, if we can prove the theorem
for $\nu'$ under the assumptions mentioned in the statement of the
theorem, the theorem for $\nu$ is obvious. Hence, without loss of
generality, we shall assume that $\forall t$, $\forall w$, $\nu^t_w$
is a finite measure on $\mathscr{Y}$. 

Let us divide the proof in two cases Case I and Case II. In Case I, we
shall prove the theorem assuming $Y$ to be a compact metrizable space
and in case II, we shall deal with the general case and deduce the
result from Case I.

In Case I\pageoriginale the proof is carried out in four steps, Step
I, Step II, Step III and Step IV.

In Step I, we define a measure valued function $\tilde{\nu}$ on
$\Omega \times \mathbb{R}$ with values in $\mathscr{m}^+ (Y,
\mathscr{Y}) $ such that $\forall t\in \mathbb{R}$ and $\forall w \in
\Omega$, $\tilde{\nu}^t_w$ is a Radon measure on $Y$. 

In Step II, we prove that $\forall t$, $\tilde{\nu}^t \in
\mathscr{C}^t$. 

In Step III, we prove that if $\mathscr{C}(Y)$ is the space of all
real valued continuous functions on $Y$, $\forall \varphi \in
\mathscr{C} (Y)$, $\forall_\lambda w$, the function
$\tilde{\nu}_w(\varphi)$ on $\mathbb{R}$ taking $t \in \mathbb{R}$ to
$\tilde{\nu}^t_w (\varphi)$, is regular and $\tilde{\nu}$ is a
modification of $\nu$. 

In Step IV, we prove that $\tilde{\nu}$ is a regular measure valued
supermartingale. 

\medskip
\noindent{\textbf{Case I.}} $Y$, a compact metrizable space.

Since $J, J^t$ are finite measures on $\mathscr{Y}$, they are Radon
measures on $Y$, since $Y$ is a compact metrizable space. Let
$\mathscr{C}(Y)$ be the vector space of all real valued continuous
functions on $Y$ and $\mathscr{C}_+(Y)$, the cone of all the positive
real valued continuous functions on $Y$. 

\medskip
\noindent{\textbf{Step I.}}
Let $\varphi \in \mathscr{C}_+(Y)$. Consider the real valued
supermartingale $\nu(\varphi). \forall t$, $\int\nu^t_w(\varphi)
d\lambda (w) = J^t(\varphi)$ is finite and by hypothesis, $t \to
J^t(\varphi)$ is right continuous. Hence, by the fundamental theorem,
a regular modification for $\nu(\varphi)$ exists. Hence, by Corollary
(\ref{part2:chap4}, \S\ \ref{part2:chap4:sec1},
\ref{part2:chap4:coro67}),  $\forall_\lambda w$, $\forall t$,
$\lim\limits_{n 
  \to \infty} \nu^{\tau_n(t)}_w(\varphi)$ exists and is finite, and if
$\overline{\nu(\varphi)} (w,t) = \lim\limits_{n \to \infty}
\nu^{\tau_n (t)}_w(\varphi)$  if this limit exists and is finite, and
$=0$ otherwise, then, $\u{\nu(\varphi)}$ is a regular modification of
$\nu(\varphi)$. 

Hence, if $\varphi \in \mathscr{C}(Y)$, $\forall_\lambda w$, $\forall
t$, $\lim\limits_{n \to \infty} \nu^{\tau_n(t)}_w(\varphi)$ exists and
is finite.

Thus, for $\varphi \in \mathscr{C} (Y)$, if 
$$
\Omega^\circ_{t, \varphi} = \{ w\in \Omega \mid \lim\limits_{n \to
  \infty} \nu^{\tau_n(t)}_w(\varphi) \text{ exists and is finite }\}. 
$$
and if $\Omega^\circ_\varphi = \bigcap\limits_{t \in \mathbb{R}}
\Omega^\circ_{t, \varphi}$, then $\lambda$ is carried by
$\Omega^\circ_\varphi$ and hence a priori by $\Omega^\circ_{t,
  \varphi}$, $\forall t \in \mathbb{R}$. Also, $\forall \varphi \in
\mathscr{C} (Y)$, $\forall t \in\mathbb{R}$, $\Omega^\circ_{t,
  \varphi} \in \mathscr{C}^t$. 

Let\pageoriginale $D$ be a countable dense subset of $\mathscr{C}(Y)$,
containing 1. Let $\Omega^\circ = \bigcap\limits_{\varphi \in D}
\Omega^\circ_\varphi$ and $\Omega^\circ_t = \bigcap\limits_{\varphi
  \in D} \Omega^\circ_{t, \varphi}$. Note that $\Omega^\circ =
\bigcap\limits_{t \in \mathbb{R}} \Omega^\circ_t$.

Since $D$ is countable, $\lambda$ is carried by $\Omega^\circ$,
$\Omega^\circ_t \in \mathscr{C}^t \;\; \forall t \in \mathbb{R}$ and
$\lambda$ is carried by $\Omega^\circ_t$ for every $t \in
\mathbb{R}$. 

Let $w \in \Omega^\circ_t$. Since $1 \in D$, $\lim\limits_{n \to
  \infty} \nu^{\tau_n(t)}_w(1)$ exists and is finite. Hence $\forall
t$, $\sup\limits_n \nu^{\tau_n(t)}_w(1)$ is finite. \hfill{(1)}

Also if $w \in\Omega^\circ_t$, $ \forall \varphi \in D$,
$\lim\limits_{n\to\infty} \nu^{\nu_n(t)}_w(\varphi)$ exists and is
finite. \hfill{(2)}

From (1) and (2), we can easily deduce that if $w \in \Omega^\circ_t$,
then $\forall \varphi \in \mathscr{C}(Y)$, $\lim\limits_{n \to \infty}
\nu^{\tau_n(t)}_w(\varphi)$ exists and is finite. Thus, 
$$
\Omega^\circ_t = \bigcap\limits_{\varphi \in \mathscr{C} (Y)}
\Omega^\circ_{t , \varphi}. 
$$

In the same way,
$$
\Omega^\circ = \bigcap\limits_{\varphi \in \mathscr{C}(Y')}
\Omega^\circ_{\varphi}. 
$$

Thus, if $w \in \Omega^\circ_t$, the vague limit of
$\nu^{\tau_n(t)}_w$ exists. Conversely if for a $w \in \Omega$ and for
a $t \in \mathbb{R}$, the vague limit of $\nu^{\tau_n(t)}_w$ exists,
then it is easy to see that $w \in \Omega^\circ_t$. Thus,
$$
\Omega^\circ_t = \{w \in \Omega \mid \text{ vague limit of }
\nu^{\tau_n(t)}_w \text{ exists }\}. 
$$

Define the measure valued function $\tilde{\nu}$ on $\Omega \times
\mathbb{R}$ with values in\break $\mathfrak{m}^+(Y, \mathscr{Y})$ as
follows.
$$
\tilde{\nu}^t_w = 
\begin{cases}
\text{vague limit of $\nu^{\tau_n(t)}_w$, if } w \in \Omega^\circ_t\\
0, \text{ otherwise.}
\end{cases}
$$
$\tilde{\nu}^t_w$ is thus a Radon measure on $Y$ for all $w \in \Omega$
and for all $t \in \mathbb{R}$. 


\medskip
\noindent{\textbf{Step II.}} Let show that $\forall t$, $\tilde{\nu}^t
\in \mathscr{C}^t$.

For this, we have to show that $\forall B \in \mathscr{Y}$, the
function $\tilde{\nu}^t (B) \in \mathscr{C}^t$. 

Let\pageoriginale us first show that if $\varphi \in \mathscr{C}_+
(Y)$, then $\tilde{\nu}^t (\varphi) \in \mathscr{C}^t$.
$$
\tilde{\nu}^t_w (\varphi) = 
\begin{cases}
\lim\limits_{n\to \infty} \nu^{\tau_n(t)}_w(\varphi), \text{ if } w
\in \Omega^\circ_t\\
0, \text{ otherwise}. 
\end{cases}
$$

Thus, $\tilde{\nu}^t_w(\varphi) = \chi_{\Omega^\circ_t} \;\;  \underset{n
  \to \infty}{\lim\sup} \nu^{\tau_n(t)}_w(\varphi)$. Since
$\Omega^\circ_t \in \mathscr{C}^t$ and since $w \to \underset{n \to
  \infty}{\lim\sup} \nu^{\tau_n(t)}_w(\varphi)$ belongs to
$\mathscr{C}^t$, it follows that $\tilde{\nu}^t(\varphi) \in
\mathscr{C}^t$. 

Let $U$ be an open subset of $Y$, $U \neq \emptyset$. Then, since
$Y$ is metrizable, there exists an increasing; sequence
$(\varphi_n)_{n\in\mathbb{N}}$ of continuous functions on $Y$ such
that $0 \leq \varphi_n \leq \chi_n \forall \; n \in \mathbb{N}$ and
$\varphi_n (y) \uparrow \chi_U(y)$ {\em for all $y \in Y$}. 

Hence, $\forall \; t \in \mathbb{R}$, $\forall w \in \Omega$,
$\tilde{\nu}^t_w(\chi_U) = \lim\limits_{n \to \infty} \tilde{\nu}^t_w
(\varphi_n)$. Since $\forall$ $n \in \mathbb{N}$, $\forall \; t \in
\mathbb{R}$, $\tilde{\nu}^t(\varphi_n) \in \mathscr{C}^t$, it follows
that $\forall t\in \mathbb{R}$, $\tilde{\nu}^t(\chi_U)$ also belongs
to $\mathscr{C}^t$. 

Now, the standard application of the Monotone class theorem gives that
$\forall \; B \in \mathscr{Y}$, $\forall t \in \mathbb{R}$,
$\tilde{\nu}^t(\chi_B) \in \mathscr{C}^t$. Hence $\forall \; t \in
\mathbb{R}$, $\tilde{\nu}^t \in \mathscr{C}^t$. 

\medskip
\noindent{\textbf{Step III.}} Let us shown that $\forall \; \varphi
\in \mathscr{C}(Y)$, $\forall_\lambda w$, $\tilde{\nu}_w(\varphi)$ is
regular and that $\tilde{\nu}$ is a modification of $\nu$. 

Let $w \in \Omega^\circ$. Then $\forall \; \varphi \in \mathscr{C}
(Y)$, $\forall t$, $\lim\limits_{n \to \infty}
\nu^{\nu_n(t)}_w(\varphi)$ exists and is finite and this limit is
equal to $\tilde{\nu}^t_w(\varphi)$. Therefore, if $\varphi \in
\mathscr{C}_+(Y)$ and $w \in \Omega^\circ$, 
$$
\tilde{\nu}^t_w (\varphi) = \overline{\nu(\varphi)} (w,t)
$$
for all $t$. Since $\lambda$ is carried by $\Omega^\circ$, we
therefore have
\begin{equation*}
\forall_\lambda w, \forall t, \forall \varphi \in \mathscr{C}_+ (Y),
\tilde{\nu}^t_w(\varphi)  = \overline{\nu(\varphi)}
(w,t). \tag{1}\label{part2:chap5:eq1} 
\end{equation*}

Now, $\overline{\nu(\varphi)}$ is a regular modification of
$\nu(\varphi)$ by Corollary (\ref{part2:chap4},
\S\ \ref{part2:chap4:sec1}, \ref{part2:chap4:coro67}). Hence,
$\forall_\lambda 
w$, $\overline{\nu(\varphi)_w}$ is regular. Therefore, since from
(\ref{part2:chap5:eq1}). 
$$
\forall_\lambda w, \forall \; t, \; \forall \varphi \in \mathscr{C}_+
(Y), \tilde{\nu}^t_w(\varphi) = \overline{\nu(\varphi)} (w,t),
$$
it follows\pageoriginale that, $\forall \varphi \in \mathscr{C}_+(Y)$,
$\forall_\lambda w$, $\tilde{\nu}_w(\varphi)$ is regular. Hence
$\forall \varphi \in \mathscr{C}(Y)$, $\forall_\lambda w$,
$\tilde{\nu}_w(\varphi)$ is also regular.

If $\varphi \in \mathscr{C}_+ (Y)$,  since $\overline{\nu(\varphi)}$
is a modification of $\nu(\varphi)$, we have
$$ 
\forall \varphi \in \mathscr{C}_+(Y), \quad \forall_\lambda w,
\overline{\nu(\varphi)} \; \; (w,t)   = \nu^t_w(\varphi). 
$$
Hence from this and from (\ref{part2:chap5:eq1}), we deduce that 
$$
\forall \varphi \in \mathscr{C}_+ (Y), \forall t, \quad
\forall_\lambda w, \tilde{\nu}^t_w (\varphi) = \nu^t_w(\varphi). 
$$

Hence $\forall \varphi \in \mathscr{C}(Y) $, $\forall t$,
$\forall_\lambda w$, $\tilde{\nu}^t_w(\varphi) =
\nu^t_w(\varphi)$. Therefore, since $D$ is countable, 
$$
\forall t, \quad \forall_\lambda w, \tilde{\nu}^t_w (\varphi) = \nu^t_w(\varphi)
$$
for all $\varphi \in D$. 

Since $D$ is dense and the measures $\tilde{\nu}^t_w$ and $\nu^t_w$
are finite measures for all $w \in\Omega$ and for all $t \in
\mathbb{R}$, it follows that 
$$
\forall t, \quad \forall_\lambda w, \quad \forall \varphi \in
\mathscr{C} (Y) , \tilde{\nu}^t_w(\varphi) = \nu^t_w(\varphi)
$$
and hence, $\forall t$, $\forall_\lambda w$, $\tilde{\nu}^t_w =
\nu^t_w$ as $\nu^t_w$ and $\tilde{\nu}^t_w$ are Radon measures for all
$w \in \Omega$ and for all $t \in\mathbb{R}$. 

Hence $\tilde{\nu}$ is a modification of $\nu$.

\medskip
\noindent{\textbf{Step IV.}}

We shall show that $\tilde{\nu}$ is regular. For this we have to show
that $\forall B \in \mathscr{Y}$, $\forall_\lambda w$,
$\tilde{\nu}_w(\chi_B)$ is regular.

From Step III, if $\varphi \in \mathscr{C}_+(Y)$, $\forall_\lambda w$,
$\tilde{\nu}_w(\varphi)$ is regular. In particular, $\forall_\lambda
w$, $\tilde{\nu}_w(1)$ is regular. Hence, $\forall_\lambda w$,
$\tilde{\nu}_w(1)$ is a locally bounded function on $\mathbb{R}$
i.e. $\exists$  a set $N \in\mathscr{O}$  with $\lambda(N) =0$ such
that if $w \not\in N$, $\tilde{\nu}_w(1)$ is a locally bounded
function on $\mathbb{R}$. Hence if $B \in \mathscr{Y}$ and $w \not\in
N$, $\tilde{\nu}_w(\chi_B)$ is also a locally bounded function on
$\mathbb{R}$.

Let $U$ be an open set of $Y$, $U \neq \emptyset$. Then since $Y$ is
metrizable, there exists\pageoriginale an increasing sequence
$(\varphi_n)_{n \in \mathbb{N}}$ of continuous functions on $Y$ such
that $\forall n \in \mathbb{N}$, $0 \leq \varphi_n \leq \chi_U$ and 
$$
\varphi_n (y) \uparrow \chi_U(y) \text{ for all } \u{y \in Y}. 
$$
Hence $\forall$ $w \in \Omega$, $\forall \; t \in \mathbb{R}$,
$\tilde{\nu}^t_w (\varphi_n) \uparrow \tilde{\nu}^t_w(\chi_U)$. Since
$\forall n$, $\tilde{\nu}(\varphi_n)$ is a regular supermartingale and
since $\tilde{\nu}(\varphi_n)$ is an increasing sequence of functions
on $\Omega \times \mathbb{R}$, it follows from the `Upper envelope
theorem' that $\forall_\lambda w$, $\tilde{\nu}_w(\chi_U)$ is right
continuous and has limits form the left at all points $t \in
\mathbb{R}$. The finiteness of these limits from the left at all $t
\in \mathbb{R}$, follows from the fact that $\forall_\lambda w$,
$\tilde{\nu}_w (\chi_U)$ is locally bounded. Hence, $\forall_\lambda
w$, $\tilde{\nu}_w(\chi_U)$ is regular. 

Let $\mathscr{C} = \{C \in \mathscr{Y} \mid \forall_\lambda w,
\tilde{\nu}_w(\chi_C) \text{ is regular }\}$. Then, the class
$\mathscr{C}$  is a $d$-system again by the `Upper envelope theorem'
and by the local boundedness of $\tilde{\nu}_w(\chi_B) \;\; \forall B \in
\mathscr{Y}$, for almost all $w \in \Omega$. This class $\mathscr{C}$
contains the class $\mathcal{U}$ of all open sets which is a
$\pi$-system generating $\mathscr{Y}$. Hence, $\mathscr{C}$ contains
the $\sigma$-algebra $\mathscr{Y}$ and hence is equal to
$\mathscr{Y}$. Thus, $\forall B \in \mathscr{Y}$, $\forall_\lambda w$,
$\tilde{\nu}_w(\chi_B) $ is regular. 

Since $\tilde{\nu}$ is a modification of $\nu$, $\tilde{\nu}(\chi_B)$
is a supermartingale $\forall \; B\in \mathscr{Y}$. Moreover, since
$\forall$ $B \in \mathscr{Y}$, $\forall_\lambda w$,
$\tilde{\nu}_w(\chi_B)$ is regular, $\tilde{\nu}(\chi_B)$ is a regular
supermartingale $\forall B \in \mathscr{Y}$. Hence, $\tilde{\nu}$ is a
regular supermartingale and is a regular modification of $\nu$. 

Thus, the proof in case I is complete. Note that in this case, because
of the uniqueness theorem, the regular modification of $\nu$ is
unique. 

\medskip
\noindent{\textbf{Case II.}} $Y$, a general topological space having
the $J$-compactiy metrizability property.

Since $Y$ has the $J$-compacity metrizability property, there exists a
sequence $(X_n)_{n \in \mathbb{N}}$ of compact metrizable sets and
a set $N \in \mathscr{Y}$ such that $J(N) = 0$ and $Y =
\bigcup\limits_{n\in \mathbb{N}} X_n \cup N$. 

Let\pageoriginale $\forall n \in \mathbb{N}$, $Y_n =
\bigcup\limits^n_{i=1} X_i$. Then $\forall n$, $Y_n$ is again a
compact metrizable set. This is because $\forall n$, $Y_n$ is both
compact and Suslin each $X_i$ is so. Moreover, the sequence $(Y_n)_{n
  \in \mathbb{N}}$ is increasing and $Y = \bigcup\limits_{n \in
  \mathbb{N}} Y_n \cup N$. 


Let $\forall n\in \mathbb{N}$, $\mathscr{Y}_n$ be the Borel
$\sigma$-algebra of $Y_n$. Let $\forall$ $n \in \mathbb{N}$, $\nu^n$
be the measure valued function on $\Omega\times \mathbb{R}$ with
values in $\mathfrak{m}^+ (Y_n, \mathscr{Y}_n )$ associating to each
$(w,t) \in \Omega \times \mathbb{R}$, the measure $\nu^{n,t}_w$ on
$\mathscr{Y}_n$, which is the restriction of the measure $\nu^t_w$ to
$Y_n$. Then, it is easily seen that $\forall$ $n,\nu^n$ is a measure
valued supermartingale on $\Omega \times \mathbb{R}$ with values in
$\mathfrak{m}^+ (Y_n, \mathscr{Y}_n)$. Let $J^t_n = \int\nu^{n,t}_w
d\lambda(w)$ and $J_n = \sup\limits_{t \in \mathbb{R}} J^t_n$. Then,
$\forall \; n \in \mathbb{N}$, $J^t_n$ and $J_n$ are respectively the
restriction of $J^t$ and $J$ to $Y_n$. Hence $\forall$ $n$, $\forall
t$, $J^t_n$ is finite, $J_n$ is finite and $t \to J^t_n$ is right
continuous. Hence the hypothesis of the theorem is verified for $\nu^n
\forall n \in \mathbb{N}$. Hence by Case I, $\forall n \in
\mathbb{N}$, $\exists$ a {\em unique} measure valued supermartingale
$\tilde{\nu}^n$ with values in $\mathfrak{m}^+ (Y_n, \mathscr{Y}_n)$
taking $(w,t)$ to $\tilde{\nu}^{n,t}_w$, which is a regular
modification of $\nu^n$.

Let $\forall$ $n \in \mathbb{N}$, $\forall t$, $\tilde{\nu}^{n+1,t}_w
\big| Y_n$ denote the restriction of the measure
$\tilde{\nu}^{n+1,t}_w$ to $Y_n$ and let $\tilde{\nu}^{n+1}_{Y_n}$
denote the measure valued function on $\Omega \times \mathbb{R}$ with
values in $\mathfrak{m}^+ (Y_n, \mathscr{Y}_n)$ associating $(w,t)$ to
$\tilde{\nu}^{n+1}_w \mid Y_n$. Then it is easily seen that $\forall
n$, $\tilde{\nu}^{n+1}_{Y_n}$ is also a regular modification of
$\nu^n$. Hence, because of uniqueness, 
\begin{equation*}
\forall_\lambda w, \forall t, \tilde{\nu}^{n,t}_w =
\tilde{\nu}^{n+1,t}_w \mid Y_n \tag{1}\label{part2:chap5:eq1+}
\end{equation*}
$\forall$ $n \in \mathbb{N}$, let us consider the measures
$\tilde{\nu}^{n,t}_w$ as measures on $\mathscr{Y}$ by defining it to
be zero for any set $B \in \mathscr{Y}$ for which $B \cap Y_n =
\emptyset$. 

Then from (\ref{part2:chap5:eq1+}), it follows that $\forall_\lambda w$,
$\tilde{\nu}^{n,t}_w$ is an increasing sequence of measures on
$\mathscr{Y}$ for all $t \in\mathbb{R}$. i.e. there exists a set $N_1
\in \mathscr{O}$ with $\lambda(N_1) = 0$ such that if $w \not\in N_1$,
for all $t \in\mathbb{R}$, $(\tilde{\nu}^{n,t}_w)_{n \in \mathbb{N}}$
is an increasing sequence of measures on $\mathscr{Y}$.

Now, $\forall $ $n \in \mathbb{N}$, define the measure valued function
$\mu^n$ on $\Omega \times \mathbb{R}$, with values in $\mathfrak{m}^+
(Y, \mathscr{Y})$ taking $(w,t)$ to $\mu^{n,t}_w $ as follows. 
$$
\mu^{n,t}_w =
\begin{cases}
\tilde{\nu}^{n,t}_w , & \text{ if } w \not\in N_1\\
0, & \text{ if } w \in N_1
\end{cases}
$$\pageoriginale 
Then $\forall w \in\Omega$, $\forall t \in\mathbb{R}$,
$(\mu^{n,t}_w)_{n \in \mathbb{N}}$ is an increasing sequence of
measures of $\mathscr{Y}$. $\forall n \in\mathbb{N}$, $\mu^n$ is a
regular measure valued supermartingale on $\Omega \times \mathbb{R}$
with values in $\mathfrak{m}^+ (y, \mathscr{Y})$. 

Define $\forall w \in \Omega$, and $\forall t \in\mathbb{R}$, the
measures $\mu^t_w$ as $\sup\limits_n \mu^{n,t}_w$. Then
$$
\forall \; B \in \mathscr{Y}, \;\mu^t_w(B) = \sup\limits_n
\mu^{n,t}_w(B). 
$$
Let $\mu$ be the measure valued function on $\Omega \times \mathbb{R}$
taking $(w,t)$ to $\mu^t_w$. 

Since $\forall B \in \mathscr{Y}$, $\forall n \in \mathbb{N}$,
$\mu^n(\chi_B)$ is a regular supermartingale, it follows form the
`Upper enveloppe theorem' that $\forall_\lambda w$, $\mu_w(\chi_B)$ is
right continuous and limits from the left exist at all $t \in
\mathbb{R}$. Hence $\mu(\chi_B)$ is a right continuous supermartingale
and $\forall\; t \in \mathbb{R}$, 
\begin{align*}
\int \mu^t_w(\chi_B) d\lambda(w) & = \lim\limits_{n \to \infty} \int
\mu^{n,t}_w(\chi_B) d \lambda(w)\\
& = \lim\limits_{n \to \infty} \int \tilde{\nu}^{n,t}_w(\chi_B)
d\lambda(w)\\
& = \lim\limits_{n \to \infty} \int \nu^{n,t}_w(\chi_B) d\lambda(w)\\
& = \lim\limits_{n \to \infty} \int \nu^t_w(B\cap Y_n) d\lambda(w)\\
& = \lim\limits_{n \to \infty} J^t (B \cap Y_n)\\
& = J^t (B) <+ \infty. 
\end{align*}
Hence by remark (\ref{part2:chap4}, \S\ \ref{part2:chap4:sec1},
\ref{part2:chap4:rem65}), $\mu(\chi_B)$ is a regular
supermartingale. Therefore $\mu$ is a regular measure valued
supermartingale. 

We have to show that $\mu$ is a modification of $\nu$. i.e. we have to
show that $\forall t$, $\forall_\lambda w$, $\mu^t_w = \nu^t_w$. 


Let\pageoriginale $t \in\mathbb{R}$ be fixed. Since $J(N) = 0$, $J^t
(N) =0$. Hence $\forall_\lambda w$, $\nu^t_w(N) =0$ i.e. $\exists \;\; 
N^1_t\in \mathscr{O}$ such that $\lambda(N^1_t) =0$ and if $w \not\in
N^1_t$, then $\nu^t_w(N) =0$. $\forall n \in\mathbb{N}$,
$\tilde{\nu}^n$ is a modification of $\nu^n$. Hence,
$$
\forall \; n \in \mathbb{N}, \quad \forall_\lambda w,
\tilde{\nu}_w^{n,t} = \nu^{n,t}_w.
 $$

Therefore, $\forall_\lambda w$, $\forall n \in\mathbb{N}$,
$\tilde{\nu}^{n,t}_w  = \nu^{n,t}_w$. Hence there exists a set $N^2_t
\in \mathscr{O}$ with $\lambda(N^2_t) =0$ such that if $w \not\in
N^2_t$,
$$
\tilde{\nu}^{n,t}_w = \nu^{n,t}_w
$$
for all $n \in \mathbb{N}$,

Let $M = N^1_t \cup N^2_t \cup N_1$. Then $M \in \mathscr{O}$ and
$\lambda(M) =0$. If $w \not\in M$, and if $B \in\mathscr{Y}$,
\begin{align*}
\nu^t_w(B) & = \lim\limits_{n \to \infty} \nu^t_w(Y_n \cap B)\\
& = \lim\limits_{n \to \infty} \nu^{n,t}_w(B)\\
& = \lim\limits_{n \to \infty} \tilde{\nu}^{n,t}_w(B)\\
& = \lim\limits_{n \to \infty} \mu^{n,t}_w(B)\\
& = \mu^t_w(B).
\end{align*}
Hence, if $w \not\in M$, $\nu^t_w = \mu^t_w$. Therefore
$$
\forall \; t, \quad \forall_\lambda w, \; \mu^t_w = \nu^t_w. 
$$
This proves that $\mu$ is a modification of $\nu$. 
\end{proof}

\section{The measures $\nu^t_w$ and the left limits}\label{part2:chap5:sec3}

Throughout\pageoriginale this section, let $(\Omega, \mathscr{O},
\lambda)$ be a measure space. Let $(\mathscr{C}^t)_{t \in \mathbb{R}}$
be an increasing right continuous family of $\sigma$-algebras
contained in $\hat{\mathscr{O}}_\lambda$. Let $Y$ be a compact
metrizable space and $\mathscr{Y}$ be its Borel $\sigma$-algebra. Let
$\nu$ be a regular measure valued supermartingale on $\Omega \times
\mathbb{R}$ with values in $\mathfrak{m}^+ (Y, \mathscr{Y})$,
adapted\pageoriginale to $(\mathscr{C}^t)_{t \in\mathbb{R}}$. Let
$\forall \; t \in\mathbb{R}$, $J^t = \int \nu^t_wd\lambda(w)$ and $J
=\sup\limits_{t \in \mathbb{R}} J^t$. Let us assume that $J$ is a {\em
finite} measure on $\mathscr{Y}$. 

Since $\nu$ is a regular measure valued supermartingale,
$\forall_\lambda w$, $\nu_w(1)$ is a right continuous, regulated
function on $\mathbb{R}$. Hence $\forall_\lambda w$, $\nu_w(1)$ is
locally bounded on $\mathbb{R}$. i.e. $\exists$ a set $N_1 \in
\mathscr{O}$ such that $\lambda(N_1) =0$ and if $w \not\in N_1$,
$\nu_w(1)$ is locally bounded on $\mathbb{R}$. 

$\forall \varphi \in \mathscr{C}(Y)$, the space of all real valued
continuous functions on $Y$, $\forall_\lambda w$, $\nu_w(\varphi)$ is
a right continuous regulated function on $\mathbb{R}$. Hence, $\forall
\varphi \in \mathscr{C} (Y)$, $\forall_\lambda w$, $t \to
\nu^t_w(\varphi)$ is right continuous and has finite left limits at
all points of $\mathbb{R}$. 

Let $D$ be a countable dense subset of $\mathscr{C}(Y)$. We have, 

$\forall_\lambda w$, $\forall \varphi \in D$, $t \to \nu^t_w(\varphi)$
is right continuous and has finite left limits at all points of
$\mathbb{R}$. i.e. there exists a set $N_2 \in \mathscr{O}$ such that
if $w \not\in N_2$, $\forall \varphi \in D$, $t \to \nu^t_w(\varphi)$
is right continuous and has finite left limits at all points of
$\mathbb{R}$. 

Let $w \not\in N_1 \cup N_2$. Then it is easy to see that $\forall
\varphi \in \mathscr{C}(Y)$, $t \to \nu^t_w(\varphi)$ is right
continuous and has finite limits at all points of $\mathbb{R}$. 

Hence, define $\forall \; w \in \Omega$, $\forall \; t \in
\mathbb{R}$, the linear mappings $\nu^{t-}_w$ on $\mathscr{C}(Y)$,
with values on $\mathbb{R}$ as follows, $\forall \varphi \in
\mathscr{C}(Y)$, 
$$
\nu^{t-}_w(\varphi) = 
\begin{cases}
\lim\limits_{\substack{s \to t\\s <t}} \nu^s_w(\varphi), \text{ if } w
\not\in N_1 \cup N_2\\
0, \text{ if } w \in N_1 \cup N_2. 
\end{cases}
$$

Then, $\forall \; w \in \Omega$, $\forall \; t \in \mathbb{R}$,
$\nu^{t-}_w$ is a positive linear functional on $\mathscr{C}(Y)$ and
thus defines a Radon measure on $Y$. And, by definition, 
$$
\forall_\lambda w, \forall t, \nu^{t-}_w(\varphi) =
\lim\limits_{\substack{s \to t\\s<t}} \nu^s_w(\varphi) \quad  \forall
\; \varphi \in \mathscr{C}(Y). 
$$

\begin{proposition}\label{part2:chap5:prop79}
Let $(f_n)_{n \in\mathbb{N}}$ be an increasing sequence of functions
on $Y_n$, $f_n \geq 0$, $f_n \in \mathscr{Y}$ and $J$-integrable
$\forall \; n \in \mathbb{N}$, such that $f_n$ increases everywhere to
a\pageoriginale $J$-integrable function $f$. Further, let
$\forall_\lambda w$, $\forall \; n$, $\forall t$, $\nu^{t-}_w(f_n) =
\lim\limits_{\substack{s\to t\\s<t}} \nu^s_w(f_n)$. Then,
$\forall_\lambda w$, $\forall \; t$, $f$ is $\nu^{t-}_w$ integrable
and $\forall_\lambda w$, $\forall t$, $\nu^{t-}_w(f) =
\lim\limits_{\substack{s\to t\\s<t}} \nu^s_w(f)$
\end{proposition}

\begin{proof}
By passing to a subsequence if necessary, we shall assume that
$\forall$ $n\in \mathbb{N}$,
$$
\int|f_n-f| d J \leq \frac{1}{2^{2n}}. 
$$

Let $h = \sum\limits^\infty_{n=1} 2^n |f_n -f|$. Then $h \geq 0$, $h
\in \mathscr{Y}$ and $J$-integrable.
$$
\forall w, \forall t, |\nu^t_w(f_n) - \nu^t_w(f)| \leq \frac{1}{2^n}
\nu^t_w(h). 
$$

By proposition (\ref{part2:chap4}, \S\ \ref{part2:chap4:sec3},
\ref{part2:chap4:prop72}), $\forall_\lambda w$, $\nu_w(h)$ is a 
right continuous, regulated function on $\mathbb{R}$ and hence is
locally bounded. Hence $\forall_\lambda w$, $\nu^t_w(f_n)$ converges
to $\nu^t_w(f)$ as $n \to \infty$ locally uniformly in the variable
$t$. Hence, 
$$
\forall_\lambda w, \forall \; t \in \mathbb{R}, \lim\limits_{n \to
  \infty} \lim\limits_{\substack{s \to t\\s<t}} \nu^s_w(f_n) =
\lim\limits_{\substack{s \to t\\s<t}}  \nu^s_w(f). 
$$
(Proposition (\ref{part2:chap4}, \S\ \ref{part2:chap4:sec3},
\ref{part2:chap4:prop72}) guarantees the existence of 
$\lim\limits_{\substack{s \to t\\s<t}} \nu^s_w(f_n)$ and\break
$\lim\limits_{\substack{s\to t\\s < t}} \nu^s_w(f)$, $\forall \; n$,
$\forall t$). Hence,
$$
\forall_\lambda w, \forall t \in\mathbb{R}, \lim\limits_{n \to \infty}
\nu^{t-}_w(f_n) = \lim\limits_{\substack{s \to t\\s<t}} \nu^s_w(f). 
$$ 

Now, since $f$ is $J$-integrable, by proposition (\ref{part2:chap4},
\S\ \ref{part2:chap4:sec3}, \ref{part2:chap4:prop72}), 
$\forall_\lambda w$, $\nu_w(f)$ is a right continuous regulated
function on $\mathbb{R}$ and hence
$$
\forall_\lambda w, \; \forall t,  \; \lim\limits_{\substack{s\to
    t\\s<t}} \nu^s_w(f) < + \infty. 
$$

Therefore, $\forall_\lambda w$, $\forall t \in \mathbb{R}
\lim\limits_{n \to \infty}$. But $\forall w \in \Omega$, $\forall t
\in \mathbb{R}$, $\lim\limits_{n \to \infty} \nu^{t-}_w(f_n) =
\nu^{t-}_w(f)$ since $f_n \uparrow f$ everywhere on $Y$. Hence,
$$
\forall_\lambda w, \forall t \in \mathbb{R}, \; \nu^{t-}_w(f) < +
\infty 
$$
and
$$
\forall_\lambda w, \; \forall t \in \mathbb{R}, \; \nu^{t-}_w(f) =
\lim\limits_{\substack{s\to t\\s<t}} \nu^s_w(f). 
$$
\end{proof}

\begin{thm}\label{part2:chap5:thm80}
$\forall \; B \in \mathscr{Y}$, $\forall_\lambda w$, $\forall t$
$$
\nu^{t-}_w(\chi_B) = \lim\limits_{\substack{s \to t \\s<t}}
\nu^s_w(\chi_B). 
$$
\end{thm}

\begin{proof}
Note that $\forall \; B \in \mathscr{Y}$, $\forall_\lambda w$,
$\forall t$, $\lim\limits_{\substack{s\to t \\s<t}} \nu^s_w(\chi_B)$
exists and is finite, since $\nu(\chi_B)$ is a regular
supermartingale.

Let $U$ be an open set, $U \neq \emptyset$. Then $\chi_U$ is lower
semi continuous and $J$-integrable since $J$ is a finite
measure. Since $Y$ is a metrizable space, $\exists$ an increasing
sequence $(\varphi_n)_{n \in \mathbb{N}}$ of real valued non-negative
continuous functions on $Y$ such that $\varphi_n(y) \uparrow
\chi_U(y)$ {\em for all $y \in Y$}. 
$$
\forall_\lambda w, \forall t, \forall n, \nu^{t-}_w(\varphi_n) =
\lim\limits_{\substack{s\to t\\ s<t}} \nu^s_w(\varphi_n). 
$$

Hence, by the previous proposition (\ref{part2:chap5},
\S\ \ref{part2:chap5:sec3}, \ref{part2:chap5:prop79}),
$\forall_\lambda 
w$, $\forall t$, $\nu^{t-}_w(\chi_U) = \lim\limits_{\substack{s \to
    t\\s<t}} \nu^s_w(\chi_U)$. Let 
$$
\mathscr{C} = \{C \in \mathscr{Y} \mid \nu^{t-}_w(\chi_C) =
\lim\limits_{\substack{s\to t \\ s<t}} \nu^s_w(\chi_C)\} . 
$$

This class is a $d$-system, again by the previous proposition (\ref{part2:chap5},
\S\ \ref{part2:chap5:sec3}, \ref{part2:chap5:prop79}). This $d$-system
contains the $\pi$-system $\mathcal{U}$ of 
all open subsets of $Y$. Hence, $\mathscr{C}$ contains  the
$\sigma$-algebra $\mathscr{Y}$, which is generated by
$\mathcal{U}$. Hence,
$$
\forall \; B \in \mathscr{Y}, \; \nu^{t-}_w (\chi_B ) =
\lim\limits_{\substack{s \to t \\s <t}}  \nu^s_w(\chi_B).
$$
\end{proof}

\begin{proposition}\label{part2:chap5:prop81}
Let $f$ be any $J$-integrable extended real valued function on $Y, f
\in \mathscr{Y}$. Then, $\forall_\lambda w$, $\forall t$, $f$ is
$\nu^{t-}_{w}$-integrable and 
$$
\forall_\lambda w, \forall t, \nu^{t-}_w(f) =
\lim\limits_{\substack{t'\to t\\t'<t}} \nu^{t'}_w(f). 
$$\pageoriginale
\end{proposition}

\begin{proof}
Note that the existence of $\lim\limits_{\substack{t' \to t\\t'<t}}
\nu^{t'}_w(f) \;\; \forall \; t\in \mathbb{R}$ is guaranteed by Corollary
(\ref{part2:chap4}, \S\ \ref{part2:chap4:sec3}, \ref{part2:chap4:coro73}). 

Sufficient to prove when $f$ is $\geq 0 $, $f \in \mathscr{Y}$ and
$J$-integrable. 

If $s$ is any step function on $Y$, $0 \leq s \leq f$ and $s \in
\mathscr{Y}$, then $s$ is $J$-integrable and it follows from the
previous theorem (\ref{part2:chap5}, \S\ \ref{part2:chap5:sec3},
\ref{part2:chap5:thm80}), that  
$$
\forall_\lambda w, \; \forall t \in \mathbb{R}, \; \nu^{t-}_w(s) =
\lim\limits_{\substack{t'\to t\\t'<t}} \nu^{t'}_w(s).
$$

Now, we can find as increasing sequence $(s_n)_{n \in \mathbb{N}}$ of
step functions on $Y$, $s_n \in \mathscr{Y}$ and $0 \leq s_n \leq f \;
\forall \; n \in \mathbb{N}$ such that $\forall y \in Y$, $s_n (y)
\uparrow f(y)$. 

Now, from proposition (\ref{part2:chap5}, \S\ \ref{part2:chap5:sec3},
\ref{part2:chap5:prop79}) it follows that $\forall_\lambda 
w$, $\forall t$, $f$ is $\nu^{t-}_w$-integrable and 
$$
\forall_\lambda w, \; \forall \; t, \; \nu^{t-}_w(f) =
\lim\limits_{\substack{t'\to t\\ t'<t}} \nu^{t'}_w(f). 
$$
\end{proof}

\begin{proposition}\label{part2:chap5:prop82}
Let $E$ be a Banach space over $\mathbb{R}$. Let $g$ be a step
function on $Y$, with values in $E$, $g \in \mathscr{Y}$ and
$J$-integrable. Then, $\forall_\lambda w$, $\forall t$, $g$ is
$\nu^{t-}_w$-integrable and 
$$
\forall_\lambda w, \; \forall t, \nu^{t-}_w(g) =
\lim\limits_{\substack{t'\to t\\t'<t}} \nu^{t'}_w(g). 
$$
\end{proposition}

\begin{proof}
Note that the existence of $\lim\limits_{\substack{t'\to \\t'<t}}
\nu^{t'}_w(g) \forall t$, is guaranteed by proposition
(\ref{part2:chap4}, \S\ \ref{part2:chap4:sec3}, \ref{part2:chap4:prop74}). 

Let $g$ be of the form $\sum\limits^n_{i=1} \chi_{A_i} x_i$ where
$\forall \; i = 1, \ldots , n$, $A_i \in \mathscr{Y}$, $x_i \in E$
and\pageoriginale $A_i \cap A_j = \emptyset$ if $i \neq j$.

Since $\forall w \in \Omega$, $\forall t \in \mathbb{R}$, $\nu^{t-}_w$
is a finite measure on $\mathscr{Y}$, $\forall i$, $\forall w \in
\Omega$, $\forall t \in \mathbb{R}$, $\nu^{t-}_w(A_i) < +
\infty$. Hence $\forall \; w \in \Omega$, $\forall \in \mathbb{R}$,
$g$ is $\nu^{t-}_w$-integrable, and $\forall w \in\Omega$, $\forall
t \in \mathbb{R}$, $\nu^{t-}_w(A_i)<+\infty$. Hence $\forall \;
w\in\Omega$, $\forall \; t\in\mathbb{R}$, $g$ is
$\nu^{t-}_w$-integrable, and $\forall w \in \Omega$, $\forall t \in
\mathbb{R}$, 
$$
\nu^{t-}_w(g) = \sum\limits^n_{i=1} \nu^{t-}_w(A_i) x_i. 
$$
By theorem (\ref{part2:chap5}, \S\ \ref{part2:chap5:sec3}, \ref{part2:chap5:thm80}),
$$
\forall_\lambda w, \; \forall t, \; \forall i, \; \nu^{t-}_w(A_i) =
\lim\limits_{\substack{t' \to t\\t'<t}} \nu^{t'}_w(A_i). 
$$

Hence $\forall_\lambda \; w$, $\forall \; t$,
\begin{align*}
\nu^{t-}_w(g) & = \sum\limits^{n}_{i=1} \lim\limits_{\substack{t'\to
    t\\t'<t}} \nu^{t'}_w (A_i) x_i\\
& = \lim\limits_{\substack{t'\to t\\ t' <t}} \lim\limits^n_{i=1}
\nu^{t'}_w(A_i) x_i\\
& = \lim\limits_{\substack{t'\to t\\t'<t}} \nu^{t'}_w(g)
\end{align*}
\end{proof}

\begin{thm}\label{part2:chap5:thm83}
Let $f$ be a $J$-integrable function on $Y$, with values in a Banach
space $E$ over $\mathbb{R}$, $f \in\mathscr{Y}$. Then $\forall_\lambda
w$, $\forall t$, $f$ is $\nu^{t-}_w$-integrable and 
$$
\forall_\lambda w, \; \forall t, \nu^{t-}_w(f) =
\lim\limits_{\substack{t'\to t\\t'<t}} \nu^{t'}_w(f). 
$$
\end{thm}

\begin{proof}
Note that the existence of $\lim\limits_{\substack{t' \to t\\t' <t}}
\nu^{t'}_w(f)$, $\forall_\lambda w$, for every $t$ is guaranteed by
proposition (\ref{part2:chap4}, \S\ \ref{part2:chap4:sec3},
\ref{part2:chap4:prop75}).  

There exists a sequence $(g_n)_{n \in\mathbb{N}}$ of step functions on
$Y$ with values in $E$, $g_n \in \mathscr{Y}$ $\forall n \in
\mathbb{N}$ such that 
$$
\forall \; n \in \mathbb{N}, \; \int |g_n -f| d J \leq
\frac{1}{2^{2n}}. 
$$

Let\pageoriginale $h = \sum\limits^\infty_{n=1} 2^n |g_n -f|$. Then $h
\geq 0$, $h \in \mathscr{Y}$ and is $J$-integrable.
$$
\forall n, \; |f| \leq |g_n -f| + |g_n| \leq \frac{h}{2^n} + |g_n|. 
$$

By proposition (\ref{part2:chap5}, \S\ \ref{part2:chap5:sec3},
\ref{part2:chap5:prop81}), $\forall_\lambda w$, $\forall t$, $h$ 
is $\nu^{t-}_w$-integrable, and by the previous proposition
(\ref{part2:chap5}, \S\ \ref{part2:chap5:sec3},
\ref{part2:chap5:prop82}), $\forall w$, $\forall t$, $\forall n$,
$g_n$ is 
$\nu^{t-}_w$-integrable. Hence, $\forall_\lambda w$, $\forall t$, $f$
is $\nu^{t-}_w$-integrable. Moreover, $\forall_\lambda w$, $\forall
t$, 
\begin{align*}
\mid \nu^{t-}_w(g_n) - \nu^{t-}_w (f)\mid & = \mid \nu^{t-}_w(g_n
-f)\mid\\
& \leq \frac{1}{2^n} \nu^{t-}_w(h).
\end{align*}
Since $\forall_\lambda w$, $\forall t$, $\nu^{t-}_w(h) < + \infty$, it
follows that $\lim\limits_{n \to \infty} \nu^{t-}_w(g_n)$ exists and
is equal to $\nu^{t-}_w(f)$. \hfill{(1)}
$$
\forall_\lambda w, \;\forall t, |\forall^t_w(g_n) - \nu^t_w(f)| \leq
\nu^t_w(|g_n-f|)\leq \frac{1}{2^n}  \nu^t_w(h). 
$$

By proposition (\ref{part2:chap4}, \S\ \ref{part2:chap4:sec3},
\ref{part2:chap4:prop72}), $\forall_\lambda w$, $\nu_w(h)$ is 
right continuous and regulated. Hence, $\forall_\lambda w$, $\nu_w(h)$
is a locally bounded function on $\mathbb{R}$.

Hence, $\forall_\lambda w$, $\nu^t_w(g_n)$ converges to $\nu^t_w(f)$
in $E$ as $n \to \infty$ locally uniformly in the variable $t$. Hence
$$
\lim\limits_{n \to \infty} \lim\limits_{\substack{t'\to t\\t'<t}}
\nu^{t'}_w(g_n) = \lim\limits_{\substack{t'\to \infty\\t'<t}}
\nu^{t'}_w(f). 
$$
Therefore, 
\begin{align*}
\lim\limits_{n \to \infty} \nu^{t-}_w(g_n) =
\lim\limits_{\substack{t'\to t\\ t'<t}} \nu^{t'}_w(f) \tag{2}\label{part2:chap5:eq2} 
\end{align*}

From (\ref{part2:chap5:eq1+}) and (\ref{part2:chap5:eq2}), we see that
$$
\forall_\lambda w, \; \forall t, \; \nu^{t-}_w(f) =
\lim\limits_{\substack{t' \to t\\t'<t}} \nu^{t'}_w(f). 
$$
\end{proof}
