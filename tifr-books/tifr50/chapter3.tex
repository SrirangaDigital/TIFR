
\chapter[Conditional expectations of measure valued...]{Conditional expectations of measure valued functions,
  Existence and uniqueness theorems}\label{part1:chap3}

\section{Basic definition}\label{part1:chap3:sec1}

In\pageoriginale this chapter, we shall define the notion of
conditional expectation for measure valued functions, and prove some
existence and uniqueness theorems. Our results, as we shall see, will
give immediately an important consequence of a result of M. Jirina
\cite{key1} on regular conditional probabilities. 

Let $(\Omega, \mathscr{O}, \lambda)$ be a measure space. Let
$\mathscr{C}$ be a $\sigma$-algebra contained in
$\hat{\mathscr{O}}_\lambda$. Let $\lambda$ restricted to $\mathscr{C}$
be $\sigma$-finite. Let $(Y, \mathscr{Y})$ be a measurable space. Let
$\nu$ be a measure valued function on $\Omega$ with values in
$\mathfrak{m}^+ (Y, \mathscr{Y})$, $\nu
\in\hat{\mathscr{O}}_\lambda$. 

\begin{defn}\label{part1:chap3:def22}
A measure valued function $\nu^\mathscr{C}$ on $\Omega$ with values in
$\mathfrak{m}^+ (Y,\break \mathscr{Y})$ is said to be a {\em conditional
  expectation} of $\nu$ with respect to $\mathscr{C}$ if 
\begin{itemize}
\item[{\rm (i)}] A $\nu^\mathscr{C} \in\mathscr{C}$ and 

\item[{\rm (ii)}] $\forall A \in \mathscr{C}$, $\int\limits_A
  \nu^\mathscr{C}_w d\lambda(w) = \int\limits_A \nu_w d \lambda (w)$. 
\end{itemize}
\end{defn}

Note that $\nu^\mathscr{C}$ is a conditional expectation of $\nu$ if
and only if for every function $f$ on $Y$, $f \geq 0$, $f \in
\mathscr{Y}$, the function $\nu^\mathscr{C} (f) \in \mathscr{C}$ and 
$$
\forall A \in\mathscr{C} , \int\limits_A \nu^\mathscr{C}_A (f) d
\lambda (w) = \int\limits_A \nu_w (f) f\lambda (w). 
$$

Thus, $\nu^\mathscr{C}$ is a conditional expectation of $\nu$ if and
only if for every function $f$ on $Y$, $f \geq 0$, $f \in\mathscr{Y}$,
the function $\nu^\mathscr{C}(f)$ is a conditional expectation of
$\nu(f)$. 

We note also that if $(\lambda^\mathscr{C}_w)_{w \in\Omega}$ is a
disintegration of $\lambda$ with respect to $\mathscr{C}$, then the
measure valued function $\lambda^\mathscr{C}$ on $\Omega$ taking $w
\in \Omega$ to $\lambda^\mathscr{C}_w$ is a conditional expectation
with respect to $\mathscr{C}$ of the measure valued function $\delta$
on $\Omega$ with values in $\mathfrak{m}^+ (\Omega, \mathscr{O})$,
taking $w \in \Omega$ to the measure $\delta_w$ (the Dirac measure at
$w$).\pageoriginale And conversely, every conditional expectation of
the measure valued function $\delta$ is a disintegration of
$\lambda$. Thus, we see how the existence of disintegration for
a measure is related to the existence of conditional expectation of
measure valued functions. 

\section{Preliminaries}\label{part1:chap3:sec2}

Before we proceed to prove the existence and uniqueness theorems of
conditional expectations for measure valued functions, we collect
below in the subsections \S\ \ref{part1:chap3:sec2.1},
\S\ \ref{part1:chap3:sec2.2}, \S\ \ref{part1:chap3:sec2.3} and
\S\ \ref{part1:chap3:sec2.4}, some  
important theorems, propositions and definitions which will be used
in the proofs of the main theorems of this chapter. 

\subsection{The monotone class theorem and its
  consequences}\label{part1:chap3:sec2.1}

Let $X$ be a non-void set. Let $\mathscr{S}$ be a class of subsets of
$X$. 

We say $\mathscr{S}$ is a $\pi$-{\em system} on $X$ if it is closed
with respect to finite intersections. 

We say $\mathscr{S}$ is a $d$-{\em system} on $X$ if 
\begin{itemize}
\item[{\rm (i)}] $X \in \mathscr{S}$, 

\item[{\rm (ii)}] $A$, $B \in \mathscr{S}$, $A \subset B \Rightarrow$
  $B \backslash A \in \mathscr{S}$  and 

\item[{\rm (iii)}] $\forall \; n \in \mathbb{N}$, $A_n \in
  \mathscr{S}$, $A_n \uparrow \Rightarrow \bigcup\limits_{n
    \in\mathbb{N}} A_n \in \mathscr{S}$. 
\end{itemize}

If $\mathscr{C}$ is any class of subsets of $X$, we shall denote by
$d(\mathscr{C})$ (resp. $\sigma(\mathscr{C})$) the smallest $d$-system
(resp. $\sigma$-algebra) containing $\mathscr{C}$. $d(\mathscr{C})$
(resp. $\sigma(\mathscr{C})$) will be called the $d$-system
(resp. $\sigma$-algebra) generated by $\mathscr{C}$ . 

In these Notes, the following theorem, whenever it is referred to will
always be referred to as the {\em Monotone class theorem}. 

\begin{theorem*}[(Monotone class thoerem)]
If $\mathscr{S}$ is a $\pi$-system, $d (\mathscr{S}) =\break
\sigma(\mathscr{S})$. 

For a proof of this theorem, see R.M. Blumenthal and R.K. Getoor
\cite{key1}. Chap. 0, \S\ 2, page 5, theorem (2.2). 
\end{theorem*}

\begin{proposition}\label{part1:chap3:prop23}
Let\pageoriginale $(X, \mathfrak{K})$ be a measurable space. Let
$\mathscr{S}$ be a $\pi$-system containing $X$ and generating
$\mathfrak{X}$. Let $\mu$ and $\nu$ be two positive {\em finite}
measures on $\mathfrak{X}$. If $\mu$ and $\nu$ agree on $\mathscr{S}$,
then $\mu$ and $\nu$ are equal. 
\end{proposition}

\begin{proof}
The class $\mathscr{C} = \{ A \in \mathfrak{X} \mid \mu (A) = \nu
(A)\}$ is a $d$-system containing $\mathscr{S}$. Hence $\mathscr{C}
\supset d (\mathscr{S})$. But by the Monotone class theorem,
$d(\mathscr{S})$ is the $\sigma$-algebra generated by $\mathscr{S}$
which is $\mathfrak{X}$. Hence $\mathscr{C} \supset \mathfrak{X}$ and
therefore $\mathscr{C} = \mathfrak{X}$ and thus $\mu$ and $\nu$ are
equal. 
\end{proof}

\begin{proposition}\label{part1:chap3:prop24}
Let $\mu$ and $\nu$ be two positive $\sigma$-{\em finite} measures on
a measurable space $(X, \mathfrak{X})$. Let $\mathscr{S} \subset
\mathfrak{X}$ be a $\pi$-system generating $\mathfrak{X}$ and
containing a sequence $(B_n)_{n \in \mathbb{N}}$ with
$\bigcup\limits_{n \in \mathbb{N}} B_n = X$, $\mu(B_n) = \nu (B_n) < +
\infty \forall \; n \in \mathbb{N}$. If $\mu$ and $\nu$ agree on
$\mathscr{S}$, then $\mu$ and $\nu$ are equal. 
\end{proposition}


\begin{proof}
First, let us fix a $n \in \mathbb{N}$. 

Consider the class $\mathscr{C} = \{A \in \mathfrak{X} \mid \mu (A
\cap B_n) = \nu (A \cap B_n)\}$. Then $\mathscr{C}$ is a $d$-system
containing $\mathscr{S}$. Since $\mathscr{S}$ is a $\pi$-system and
generates $\mathfrak{X}$, by the Monotone class theorem $\mathscr{C}
\subset \mathfrak{X}$. Hence, 
$$
\forall \; A \in \mathfrak{X}, \; \mu (A \cap B_n) = \nu (A \cap
B_n). 
$$

Since $n$ is arbitrary, we have 
$$
\forall \; n \in\mathbb{N}, \; \forall A \in \mathfrak{X}, \mu (A \cap
B_n) = \nu (A \cap B_n). 
$$
if $A_n = \bigcup\limits^n_{i=1}B_i $, note that
$$
\forall \; n,\; \forall A \in \mathfrak{X}, \mu (A \cap A_n) = \nu (A
\cap A_n). 
$$
Since
$$
A_n \uparrow X, \text{ we see that } \forall A \in \mathfrak{X}, \mu
(A) = \nu (A). 
$$
\end{proof}

\subsection{The lifting theorem of D. Maharam}\label{part1:chap3:sec2.2}

Let\pageoriginale $(X, \mathfrak{X}, \mu)$ be a measure space with
$\mu$ $\sigma$-finite on $\mathfrak{X}$. Let $L^\infty(X;
\mathfrak{X}; \mu)$ stand for the Banach space whose underlying vector
space is the vector space of all $\mu$-equivalence classes of extended
real valued functions on $X$, belonging to $\hat{\mathfrak{X}}_\mu$,
having a finite essential supremum and the norm is the essential
supremum. Let $\mathscr{B} (X; \mathfrak{X})$ stand for the Banach
space whose underlying vector space is the vector space of all real
valued bounded functions on $X$ belonging to $\hat{\mathfrak{X}}_\mu$
and the norm is the supremum.

In the following, we have to keep in mind the fact that when we say
$f$ is an element of $L^\infty(X; \mathfrak{X};\mu)$, we mean by $f$
not a single function on $X$, but a class of functions on $X$, any two
functions of a class differing only on a set of $\mu$-measure zero at
most. Thus if $g$ is a function on $X$ and $f \in L^\infty (X;
\mathfrak{X}; \mu)$, the meaning of `$\u{g \in f}$' is clear i.e. $g$
belongs to the class $f$. If $h$ is a bounded function on $X,
\in\hat{\mathfrak{X}}_\mu , \u{ \u{h} \in }$ will denote the unique
element of $L^\infty (X; \mathfrak{X}; \mu)$ to which $h$
belongs. Thus, if `$a$' is a real number, considered as the constant
function `$a$' on $X$, the meaning of $\u{a}$ is clear. If $f \in
L^\infty (X; \mathfrak{X}; \mu)$, we say $f$ is {\em non-negative }
and write $\u{f \geq 0}$ if there exists a function $g$ on $X$, $g \in
f$ and a set $N_g \in \hat{\mathfrak{X}}_\mu$ with $\mu (N_g) = 0$
such that $g \geq 0$ on $\complement N_g$. Note that if there exists
one function on $g \in f$ having this property viz. there exists a set
$N_g$ with $\mu(N_g) = 0$ and $g \geq 0$ on $\complement N_g$, then
every function belonging to $f$ also has this property. If  $f_1$ and
$f_2$ are two elements $\in L^\infty (X; \mathfrak{X}; \mu) $, we
say $f_1$ is {\em greater than or equal} to $f_2$ and write $f_1 \geq
f_2$, if $f_1 - f_2 \geq 0$.

The following important theorem is due to D. Maharam and whenever we
refer this theorem, we will be referring it as the {\em lifting
  theorem}. For a proof of this theorem see D. Maharam  \cite{key1}. 

\begin{theorem*}[D. Maharam]\pageoriginale
{\em The Lifting Theorem}. There exists a mapping $\rho$ from
$L^\infty(X; \mathfrak{X}; \mu)$ to $\mathscr{B}(X; \mathfrak{X})$
such that 
\begin{itemize}
\item[{\rm (i)}] $\rho$ is linear and continuous

\item[{\rm (ii)}] $\rho(f) \in f \;\; \forall \; f \in L^\infty (X;
  \mathfrak{X}; \mu)$ 

\item[{\rm (iii)}] $\rho (f) \geq 0$ if $f \geq 0$ and 

\item[{\rm (iv)}] $\rho(\u{1}) \equiv 1$
\end{itemize}

Such a mapping $\rho$ is called a {\em lifting}. 
\end{theorem*}


\subsection{Some theorems on $L^\infty(X;\mathfrak{X};\mu)$}\label{part1:chap3:sec2.3}

In this section, let $(X; \mathfrak{X}, \mu)$ be a measure space with
$\mu (X) < + \infty$. 

If $f\in L^\infty (X; \mathfrak{X};\mu)$, we define the integral of
$f$ with respect to $\mu$ as $\int g d\mu$ where $g$ is any function
belonging to $f$. Note that $\int g d \mu$ exists since $\mu$ is a
finite measure and $g$ is essentially bounded. Note also that the
integral of $f$ with respect to $\mu$ is independent of the choice of
the function $g$ chosen to belong to $f$. The integral of $f$ with
respect to $\mu$ is written as $\int f d \mu$. 

\begin{defn}\label{part1:chap3:def25}
Let $(f_i)_{i \in I}$ be a family of elements of $L^\infty (X;
\mathfrak{X}; \mu)$. An element $f \in L^\infty (X; \mathfrak{X};
\mu)$ is said to be a {\em supremum of the family $(f_i)_{i \in I}$ in
the $L^\infty$-sense} if
\begin{itemize}
\item[{\rm (i)}] $f \geq f_i \forall i \in I$ and 

\item[{\rm (ii)}] $g \in L^\infty (X; \mathfrak{X}; \mu)$, $g \geq f_i
 \;\;  \forall i \in I \Rightarrow g \geq f$. 
\end{itemize}
\end{defn}

Note that through a supremum need not always exist, it is unique if it
exists. 

If the supremum of a family $(f_i)_{i\in I}$, $f_i \in L^\infty (X;
\mathfrak{X}; \mu) \;\; \forall i$, exists in the $L^\infty$-sense, we say
$\underset{i \in I}{\SUP}{}_\mu f_i$ exists and denote the supremum by
$\underset{i\in I } {\SUP}{}_\mu f_i$. If $I$ is a finite set say
$\{1, \ldots n\}$, then we write $\SUP_\mu (f_1, \ldots ,f_n)$ instead
of $\underset{i\in \{1, \ldots, n\}}{\SUP}{}_\mu f_i$. If $f\in
L^\infty(X; \mathfrak{X}; \mu)$, we denote by $||f||_\infty$, the
essential supremum of $f$. 

\begin{proposition}\label{part1:chap3:prop26}
If $(f_n)_{n \in \mathbb{N}}$ is an increasing sequence of elements
$\in\break L^\infty(X; \mathfrak{X}; \mu)$ such that it is bounded in norm
i.e. $\sup\limits_n ||f_n||_\infty < + \infty$, then $\underset{n \in
  \mathbb{N}}{\SUP}{}_\mu f_n $\pageoriginale exists and $\int
(\SUP_\mu f_n) d\mu = \sup\limits_{n \in\mathbb{N}} \int f_n d \mu$. 
\end{proposition}


\begin{proof}
Choose $\forall \; n \in \mathbb{N}$, a function $h_n$ on $X$, $h_n
\in\hat{\mathfrak{X}}_\mu$ such that $h_n \in f_n$. Since $(f_n)_{n
  \in \mathbb{N}}$ is increasing, $\forall n \in \mathbb{N}$,
$\exists$ a set $E_n \in\hat{\mathfrak{X}}_\mu$ such that $\mu(E_n) =
0$ and if $x \not\in E_n$, $h_n(x) \leq h_{n+1} (x)$. 

Let $E = \bigcup\limits^\infty_{n =1} E_n$. Then $E \in
\hat{\mathfrak{X}}_\mu$ and $\mu(E) = 0$. If $x \not\in E$, $h_1 (x)
\leq h_2 (x)$ $\ldots \leq h_n (x) \leq h_{n+1} (x) \leq
\ldots$. Thus, if $x \not\in E$, $(h_n (x))_{n \in\mathbb{N}}$ is a
monotonic non-decreasing sequence of extended real numbers and hence,
$\forall x \not\in E$, $\lim\limits_{n \to \infty} h_n (x)$  exists. 

Define 
$$
h(x) = \begin{cases}
\lim\limits_{n \to \infty} &  h_n (x) \text{ if } x \not\in E\\
0 & \text{ if } x \in E. 
\end{cases}
$$

Then $h \in \hat{\mathfrak{X}}_\mu \u{h} \in  L^\infty(X; \mathfrak{X};
\mu)$ since $\sup\limits_n ||f_n||_\infty < + \infty$. 

It is clear that $\u{h}$ is the supremum of $(f_n)_{n \in\mathbb{N}}$
in the $L^\infty$-sense.

Since $\int h \; d \mu = \lim\limits_{n \to \infty} \int h_n d \mu$,
it follows that 
$$
\int (\underset{n \in \mathbb{N}}{\SUP}{}_\mu f_n) d \mu =
\sup\limits_{n \in \mathbb{N}} \int f_n d\mu. 
$$

\end{proof}

\begin{proposition}\label{part1:chap3:prop27}
Let $(f_i)_{i \in I}$ be a directed increasing family of elements of
$L^\infty (X; \mathfrak{X}; \mu)$ such that the family is bounded in
norm, i.e. $\sup\limits_{i \in I}||f_i||_\infty < + \infty$. Then,
$\underset{i \in I}{\SUP}{}_\mu f_i$  exists. 
\end{proposition}

\begin{proof}
Since $\mu $ is a finite measure and since
$$
\sup\limits_{i \in I}||f_i||_\infty < + \infty, \; \sup\limits_{i \in
  I} \int f_i d\mu < + \infty. 
$$

Let $\alpha = \sup\limits_{i \in I} \int f_i d\mu$. Then $\alpha \in
\mathbb{R}$. 

Since $(f_i)_{i \in I}$ is a directed increasing family, we can find
an increasing sequence $(f_n)_{n \in \mathbb{N}}$ such that $\forall n
\in \mathbb{N}, f_n$ belongs to the family $(f_i)_{i \in I}$, and
$\int f_n \; d\mu \uparrow \alpha$. From the previous proposition,
$\underset{n \in \mathbb{N}}{\SUP} {}_\mu f_n$ exists. Let $f =
\underset{n \in \mathbb{N}}{\SUP}{}_\mu f_n$. 

Let\pageoriginale us show that $f$ is the supremum of $(f_i)_{i \in
  I}$ in the $L^\infty$-sense. 

Fix an $i \in I$. 

It is easily seen that 
\begin{align*}
& \int \SUP_\mu (f_i, f_n) d \mu \uparrow \int \SUP_\mu (f_i, f) d \mu
  \text{ as  } n \to \infty. \\
& \forall\; n \in \mathbb{N}, \int \SUP_\mu (f_i , f_n) d \mu
  \leq\sup\limits_{j \in I} \int f_jd \mu, 
\end{align*}
since $(f_i)_{i \in I}$ is a directed increasing family.

Hence, 
$$
\forall \; n \in\mathbb{N}, \; \int \SUP_\mu(f_i, f_n) d \mu \leq
\alpha. 
$$

Hence, 
$$
\qquad \int \SUP_\mu(f_i, f) d \mu \leq \alpha. 
$$
But $\int\SUP_\mu (f_i, f) d \mu \geq \alpha$, since $\SUP_\mu (f_i f)
\geq f$ and $\int f d \mu = \alpha$. 

Hence, 
$$
\int\SUP_\mu (f_i f) d \mu = \alpha = \int f d \mu. 
$$

This shows that $f \geq f_i$. 

Since $i \in I$ is arbitrary, it follows that $f \geq f_i \;\; \forall i
\in I$. Now, let $g \in L^\infty (X; \mathfrak{X}; \mu)$ be such that
$g \geq f_i \;\; \forall i \in I$. Then $g \geq f_n \;\; \forall n \in
\mathbb{N}$ and hence $g \geq f$. 

This shows that $f$ is the supremum of $(f_i)_{i \in I}$ in the
$L^\infty$-sense. 
\end{proof}

\begin{proposition}\label{part1:chap3:prop28}
Let $(f_i)_{i \in I}$ be a directed increasing family of elements of
$L^\infty (X; \mathfrak{X}; \mu)$ and let $f \in L^\infty (X;
\mathfrak{X}; \mu)$. Then, the following are equivalent,
\begin{itemize}
\item[{\rm (i)}] $f = \underset{i \in I}{\SUP}{}_\mu f_i$

\item[{\rm (ii)}] $f \geq f_i \;\; \forall i \in I $ and there exists a
  sequence $(f_n)_{n \in \mathbb{N}}$, $\forall n \in\mathbb{N}$ $f_n$
  belonging to the family $(f_i)_{i \in I}$ such that $f = \underset{n
  \in \mathbb{N}}{\SUP}{}_\mu f_n$. 


\item[{\rm (iii)}] $f \geq f_i \;\; \forall i \in I$ and $\int f d\mu =
  \sup\limits_{i \in I} \int f_i d \mu$. 
\end{itemize}
\end{proposition}


\begin{proof}
\begin{description}
\item[{\rm (i) $\Longrightarrow$ (ii).}] Since $f = \underset{i \in
  I}{\SUP} {}_\mu f_i$ and $f \in L^\infty(X; \mathfrak{X}; \mu)$, it
  follows that the family $(f_i)_{i \in I}$ is bounded in norm. Let
  $\alpha = \sup\limits_{i \in I} \int f_i d \mu$. Then $\alpha
  \in\mathbb{R}$. Then, if\pageoriginale $(f_n)_{n \in\mathbb{N}}$ is
  an increasing sequence of elements, $\forall$ $n \in \mathbb{N}$,
  $f_n$ belonging to the family $(f_i)_{i \in I}$ such that $\int f_n
  d \mu \uparrow \alpha$, we can show as in the proof of the previous
  proposition (\ref{part1:chap3}, \S\ \ref{part1:chap3:sec2.3},
  \ref{part1:chap3:prop27}) that $f = \underset{n \in \mathbb{N}} 
 {\SUP} {}_\mu f_n$. 


\item[{\rm (ii) $\Longrightarrow$ (iii).}] Let $h_n = \SUP_\mu (f_1,
  f_2, \ldots, f_n)$. Then $\forall $ $n \in\mathbb{N}$, $h_n \in
  L^\infty (X;\break \mathfrak{X}; \mu)$  and $\int h_n d \mu \uparrow f d
  \mu$. 

Since $\forall \; i \in I$, $f \geq f_i$, $\int f d \mu \geq f_i d
\mu  \;\; \forall i \in I$, and  hence, $\int f \; d \mu \geq \sup\limits_{i
\in I} \int f_i \; d\mu$. 

On the other hand, $\forall \; n \in \mathbb{N}$, $\int h_n d \mu \leq
\sup\limits_{i \in I} \int f_i d \mu$, (since the family $(f_i)_{i \in
I}$ is directed increasing). 

Hence
$$
\int f \; d\mu = \sup\limits_{i \in I} \int f_i d \mu. 
$$ 

\item[{\rm (iii) $\Longrightarrow$ (i).}] Since $f \in L^\infty(X;
  \mathfrak{X}; \mu)$  and $f\geq f_i \;\; \forall i \in I$, the family
  $(f_i)_{i \in I}$ is bounded in norm.

Let $\alpha = \int f \; d\mu  = \sum\limits_{i \in I} \int f_i
d\mu$. Then $\alpha \in \mathbb{R}$. Let $(g_n)_{n \in\mathbb{N}}$ be
an increasing sequence of elements belonging to $L^\infty (X;
\mathfrak{X}; \mu)$ such that $\forall \; n \in\mathbb{N}$, $g_n$
belongs to the family $(f_i)_{i \in I}$ and $\int g_n d \mu \downarrow
\alpha$. 

Let $g = \underset{n \in\mathbb{N}}{\SUP} {}_\mu g_n$ ($\underset{n \in
  \mathbb{N}}{\SUP}{}_\mu g_n$ exists because of proposition (\ref{part1:chap3},
  \S\ \ref{part1:chap3:sec2.3}, \ref{part1:chap3:prop26})). Then $\int
  g \; d\mu = \alpha = \int f \; d 
  \mu$. Since $f \geq f_i$ $\forall i \in I$, $f \geq g_n \forall\; n
  \in \mathbb{N}$, and hence $f \geq g$. Since $\int f \; d \mu = \int
  g \; d\mu$, it follows that $f = g$. 

If $h \in L^\infty (X; \mathfrak{X}; \mu)$  is such that $h \geq f_i $
$\forall i \in I$, then $h \geq g_n \forall n \in \mathbb{N}$ and
since $g = f = \SUP_\mu g_n$, it follows that $h \geq f$. Hence $f =
\SUP_\mu f_i$. 
\end{description}
\end{proof}

\begin{corollary}\label{part1:chap3:coro29}
Let $(f_i)_{i \in I}$ be a directed increasing family of elements of
$L^\infty (X; \mathfrak{X}; \mu)$ and let $f \in L^\infty (X;
\mathfrak{X}; \mu)$. Then $f = \underset{i \in I}{\SUP}{}_\mu f_i$ if
and only if 
\begin{itemize}
\item[{\rm (i)}] $f \geq f_i$ $\forall i \in I$ and 

\item[{\rm (ii)}] there exists an increasing sequence $(f_n)_{n\in
  \mathbb{N}}$, $f_n$, $\forall \; n \in \mathbb{N}$, belonging to the
  family $(f_i)_{i \in I}$ such that $\int f_n \; d \mu \uparrow \int
  f \; d\mu$. 
\end{itemize}
\end{corollary}

\begin{proof}
This is an immediate consequence of the above proposition. 
\end{proof}

\begin{proposition}\label{part1:chap3:prop30}
Let\pageoriginale $\rho$ be a lifting from $L^\infty (X; \mathfrak{X};
\mu)$ to $\mathscr{B}(X; \mathfrak{X})$. Let $(f_i)_{i \in I}$ be a
directed increasing family of elements of $L^\infty(X; \mathfrak{X};
\mu)$ bou\-nded in norm. Let $f = \underset{i\in I}{\SUP}{}_\mu
f_i$. Then, 
$$
\rho(f) \in \underset{i \in I}{\SUP}{}_\mu \u{\rho(f_i)} \text{ and }
\forall_\mu x, \rho(f) (x) = \sup\limits_{i \in I} \rho(f_i) (x). 
$$

In particular,
$$
\sup\limits_{i \in I} \rho(f_i) \in\hat{\mathfrak{X}}_\mu . 
$$
\end{proposition}

\begin{proof}
Note that $\rho(f) \in \underset{i\in I}{\SUP} {}_\mu \u{\rho(f_i)}$
is clear since $\underset{i\in I}{\SUP}{}_\mu \rho \u{(f_i)} =
\SUP_\mu f_i$ (since $\rho(f_i) \in f_i \forall i \in I$), $\rho(f)
\in f$ and $f = \underset{i \in I}{\SUP}{}_\mu f_i$. Since $\forall$
$i \in I$, $f \geq f_i$, $\rho(f) (x) \geq \rho (f_i) (x)$ {\em for
  all} $\u{x \in X}$ and hence
$$
\rho(f) (x) \geq \sup\limits_{i \in I} \rho(f_i) (x)
$$
for all $x \in X$. 

Let $(f_n)_{n\in \mathbb{N}}$ be an increasing sequence of elements,
$\forall$ $n \in\mathbb{N}$, $f_n$ belonging to the family
$(f_i)_{i\in I}$ such that $f = \underset{n \in \mathbb{N}}{\SUP}
{}_\mu f_n$. (Such a sequence exists because of the proposition (\ref{part1:chap3},
\S\ \ref{part1:chap3:sec2.3}, \ref{part1:chap3:prop28})). 

Then $\rho(f_n) \uparrow$ and $\int f_n \; d \mu \uparrow \int f \; d
\mu$. Hence,  $\int \rho(f_n) d\mu \uparrow \int \rho(f) d\mu$. Since
$\rho(f_u) \uparrow$, by the monotone convergence theorem,
$$
\sup\limits_n \int \rho(f_n) d \mu = \int \sup\limits_n \rho(f_n) d
\mu. 
$$

Hence, $\int \rho (f) d \mu = \int \sup\limits_n \rho(f_n) d\mu$. But
$\sup\limits_n \rho(f_n) (x) \leq \rho(f) (x)$ for all $x \in X$,
since $\forall$ $n \in \mathbb{N}$, $\rho(f_n) (x) \leq \rho(f)(x)$
for all $x \in X$. Hence $\forall_\mu x$, $\rho(f)(x) = \sup\limits_n
\rho(f_n)(x)$. 

Now, for all $x \in X$. 
$$
\rho(f) (x) \geq \sup\limits_{i \in I} \rho (f_i) (x) \geq
\sup\limits_n \rho (f_n)(x). 
$$ 

Hence, $\forall_\lambda x$, $\rho(f)(x) = \sup\limits_{i \in I}
\rho(f_i) (x)$. Since $\rho(f) \in \hat{\mathfrak{X}}_\mu$ and since
$\forall_\mu x$, $\sup\limits_{i\in I} \rho(f_i) (x) $ is equal to
$\rho(f) (x)$, it follows that
$$
\sup\limits_{i \in I} \rho(f_i) \in\hat{\mathfrak{X}}_\mu. 
$$
\end{proof}

Let\pageoriginale $\mathfrak{z}$ be a sub $\sigma$-algebra of
$\hat{\mathfrak{X}}_\mu$. Let $f \in L^\infty (X; \mathfrak{X}; \mu)$. Let
$g$ be any function on $X$, $g \in f$. Since $\forall_\mu x$, $g(x)
\leq ||f||_\infty$ and $\mu$ is a finite measure, it follows that $g$
is $\mu$-integrable, and hence $g^\mathfrak{z}$, a conditional
expectation of $g$ with respect to $\mathfrak{z}$ exists. $\forall_\mu
x$, $|g^\mathfrak{z}(x)| \leq ||f||_\infty$ and hence $g^\mathfrak{z}$
is essentially bounded. Since any two conditional expectations of $g$
with respect to $\mathfrak{z}$ are equal $\mu$-almost everywhere, we
have a unique element of $L^\infty (X; \mathfrak{z}; \mu)$ to which
any conditional expectation of $g$ with respect to $\mathfrak{z}$
belongs. Note that this element of $L^\infty(X; \mathfrak{z}; \mu)$ is
independent of the function $g$, chosen to belong to $f$ and hence
depends only on $f$. We denote this element by $f^\mathfrak{z}$ and
call it the \textit{conditional expectation} of $f$ with respect to
$\mathfrak{z}$. 

\begin{proposition}\label{part1:chap3:prop31}
Let $\mathfrak{z}$ be a sub $\sigma$-algebra of
$\hat{\mathfrak{X}}_\mu$. Let $(f_i)_{i \in I}$ be a directed
increasing family of elements of $L^\infty (X; \mathfrak{X}; \mu)$
bounded in norm. Let $f= \underset{i \in I}{\SUP}{}_\mu f_i$. Then
$\underset{i \in I}{\SUP}{}_\mu f^\mathfrak{z}_i$ exists and is equal
to $f^\mathfrak{z}$.
\end{proposition}

\begin{proof}
Since $(f_i)_{i \in I}$ is a directed increasing family bounded in
norm, $(f^\mathfrak{z}_i)_{i \in I} $ is also directed increasing and
bounded in norm. Hence $\underset{i \in I}{\SUP}{}_\mu
f^\mathfrak{z}_i$ exists. 
\end{proof}

To prove that $f^\mathfrak{z} = \underset{i \in I}{\SUP}{}_\mu
f^\mathfrak{z}_i$, it is sufficient to prove, because of proposition
(\ref{part1:chap3}, \S\ \ref{part1:chap3:sec2.3},
\ref{part1:chap3:prop28}), that $f^\mathfrak{z} \geq f^\mathfrak{z}_i
\forall 
i \in I$ and $\int f^\mathfrak{z} d\mu = \sup\limits_{i \in I} \int
f^\mathfrak{z}_i d\mu$. But this follows immediately, since $f \geq
f_i \forall i \in I$, $\int f \; d\mu = \int f^\mathfrak{z} d \mu$,
$\int f_i d \mu = \int f^\mathfrak{z}_i d\mu \forall i \in I$, and
$\int f \; d \mu = \sup\limits_{i \in I} \int f_i d \mu$.  

\subsection{Radon Measures}\label{part1:chap3:sec2.4}

Let $X$ be a topological space. (By topological spaces in these
Notes, we always mean only non-void, Hausdorff topological
spaces). Let $\mathfrak{X}$ be its Borel $\sigma$-algebra i.e. the
$\sigma$-algebra generated by all the open sets of $X$. 

\begin{defn}\label{part1:chap3:def32}
A positive measure $\mu$ on $\mathfrak{X}$ is is said to be a {\em
  Radon measure on } $X$ if 
\begin{itemize}
\item[{\rm (i)}] $\mu$\pageoriginale is {\em locally finite}
  i.e. every point $x \in X$ has a neighbourhood $V_x$ such that
  $\mu(V_x) < + \infty$.  and 

\item[{\rm (ii)}] $\mu$ is {\em inner regular} in the sense that 
\begin{align*}
\forall \; B \in\mathfrak{X}, \mu (B)  = & \sup \mu(K)\\
& K \subset B\\
& K \text{ compact }
\end{align*}
\end{itemize}
\end{defn}

\begin{defn}\label{part1:chap3:def33}
Let $\mu$ be a Radon measure on a topological space $X$. Let $(K_i)_{i
\in I}$ be a family of compact sets of $X$ and $N$, a $\mu$-null
set. $\left\{(K_i)_{i \in I}, N \right\}$ is said to be a $\mu$-{\em
  concassage}  of $X$ if 
\begin{itemize}
\item[{\rm (i)}] $X = N \cup \bigcup\limits_{i \in I} K_i$

\item[{\rm (ii)}] $K_i \cap K_j = \emptyset$ if $i \neq j$,  and 

\item[{\rm (iii)}] the family $(K_i)_{i \in I}$ is {\em locally
  countable} in the sense that every point has a neighbourhood which
  has a non-void intersection with at most a countable number of
  $K_i$'s. 
\end{itemize}

The following important theorem is stated here without proof; and for
a proof see L. Schwartz \cite{key2}, page 48, theorem 13. 
\end{defn}

\begin{theorem*}
If $\mu$ is a Radon measure on topological space $X$, there exists a
$\mu$-{\em concassage} $\{(K_i)_{i \in I}, N\}$ of $X$. 
\end{theorem*}

\begin{proposition}\label{part1:chap3:prop34}
Let $\mu$ be a {\em finite} Radon measure on $X$. Then there exists a
$\mu$-concassage $\{(K_i)_{i \in I} , N\}$ of $X$ with $I$ {\em
  countable}. 
\end{proposition}

\begin{proof}
Since $\mu$ is a finite Radon measure, using the inner regularity of
$\mu$, we see that there exists a $\mu$-null set $N_1$ and a sequence
$(X_n)_{n \in\mathbb{N}}$ of compact sets of $X$ such that 
$$
X = N_1 \cup \bigcup\limits_{n \in\mathbb{N}} X_n . 
$$

Let $\{(K_j)_{j \in J}, N_2\}$ be a $\mu$-concassage of $X$. Since the
family $(K_j)_{j \in J}$ is locally\pageoriginale countable, every
compact set can have a non-void intersection only with at most a
countable number of $(K_j)_{j \in J}$. Hence, $\forall \; n \in
\mathbb{N}$, there exists a countable set $I_n \subset J$ such that
if $j \not\in I_n$, $X_n \cap K_j = \emptyset$. Let $I =
\bigcup\limits_{n \in\mathbb{N}} I_n$. Then it is clear that
$\{(K_i)_{i \in I}, N\}$ is a $\mu$-concassage of $X$ where $N = N_1
\cup N_2$. Note the $I$ is countable. 
\end{proof}

\section{Uniqueness Theorem}\label{part1:chap3:sec3}

Let $(X, \mathfrak{X})$ be a measurable space. Let $\mu$ be a positive
measure on $\mathfrak{X}$. 

\begin{defn}\label{part1:chap3:def35}
$\mathfrak{X}$ is said to have the $\mu$-{\em countability} property
  if there exists a set $N\in \mathfrak{X}$ with $\mu(N) = 0$ such
  that the $\sigma$-algebra $\mathfrak{X}\cap \complement N$ on $X' =
  X \cap \complement N$ consisting of sets of the form $A \cap
  \complement N$, $A \in \mathfrak{X}$ is countably generated. 
\end{defn}

\begin{examples*}
Suppose $\mathfrak{X}$ is countably generated as in the case when
$\mathfrak{X}$ is the Borel $\sigma$-algebra of a topological space
$X$ having the $2^{\rm nd}$ axiom of countability, then obviously
$\mathfrak{X}$ has the $\mu$-countability property for any positive
measure $\mu$ on $\mathfrak{X}$. Also, it can be easily proved that if
$\hat{\mathfrak{X}}_\mu$  is countably generated, then $\mathfrak{X}$
has the $\mu$-countability property. 
\end{examples*}


From now onwards, in this section, let $(\Omega, \mathscr{O},
\lambda)$ be a measure space and let $(Y, \mathscr{Y})$ be a
measurable space. Let $\nu$ be a measure valued function on $\Omega$
with values in $\mathfrak{m}^+ (Y, \mathscr{Y})$, $\nu \in
\hat{\mathscr{O}}_\lambda$. Let $\mathscr{C}$ be a $\sigma$-algebra
on $\Omega$, $\mathscr{C} \subset \hat{\mathscr{O}}_\lambda$. Let
$\lambda$ restricted to $\mathscr{C}$ be $\sigma$-finite. Let $J =
\int \nu_w d \lambda(w)$. 

\begin{thm}[The Uniqueness theorem]\label{part1:chap3:thm36}
Let $J$ be a $\sigma$-finite measure on $\mathscr{Y}$ and let
$\mathscr{Y}$ have the $J$-countability property. Then, if
$\nu^\mathscr{C}_1$ and $\nu^\mathscr{C}_2$ are any two conditional
expectations of $\nu$ with respect to $\mathscr{C}$, we have
$\forall_\lambda w. (\nu^\mathscr{C}_1)_w = (\nu^\mathscr{C}_2)_w$
where $(\nu^\mathscr{C}_i)_w$ for $i = 1,2$ stands for the measure
associated by $\nu^\mathscr{C}_i$ to $w$. 
\end{thm}

\begin{proof}
First note that $J = \int \nu^\mathscr{C}_1 d \lambda = \int
\nu^\mathscr{C}_2 d \lambda$. Since $J$ is $\sigma$-finite, it
therefore follows that $\forall_\lambda w$, $(\nu^\mathscr{C}_1)_w$
and $(\nu^\mathscr{C}_2)_w$ are both $\sigma$-finite i.e. $\exists$ a
set $N_1 \in\hat{\mathscr{O}}_\lambda$ with $\lambda(N_1) = 0$ such
that if $w \not\in N_1$, $(\nu^\mathscr{C}_1)_w$ and
$(\nu^\mathscr{C}_2)_w$ are $\sigma$-finite measures on
$\mathscr{Y}$.\pageoriginale  

Since $\mathscr{Y}$ has the $J$-countability property there exists a
set $N \in \mathscr{Y}$ with $J(N) = 0$ such that the $\sigma$-algebra
$\mathscr{Y}' = \mathscr{Y} \cap \complement N$ on $Y' = Y \cap
\complement N$ is countably generated. Since $J(N) = 0$, it follows
that $\forall_\lambda w$, $(\nu^\mathscr{C}_1)_w (N) =
(\nu^\mathscr{C}_2)_w(N) = 0$. i.e., there exists a set $N_2 \in
\hat{\mathscr{O}}_\lambda$ with $\lambda(N_2) = 0$ such that if $w
\not\in N_2$, $(\nu^\mathscr{C}_1)_w(N) = (\nu^\mathscr{C}_2)_w(N) =
0$. 

With our assumptions that $J$ is $\sigma$-finite and $\mathscr{Y}'$ is
countably generated, we can find a class $\mathscr{B}$ of subsets of
$Y'$ such that $\mathscr{B}$ is countable, $\mathscr{B}$ generates
$\mathscr{Y}'$ and $J(B) < + \infty \;\;  \forall B \in \mathscr{B}$. Let
$\mathscr{C}$ be the class of subsets of $Y'$ formed by the sets which
are finite intersections of sets belonging to $\mathscr{B}$. Then
$\mathscr{C}$ is countable, $\mathscr{C}$ is a $\pi$-system generating
$\mathscr{Y}'$ and $J(C) < + \infty \;\; \forall C \in \mathscr{C}$. There
exists a set $N_3 \in\hat{\mathscr{O}}_\lambda$ with $\lambda (N_3) =
0$ such that if $w \not\in N_3$, $(\nu^\mathscr{C}_1)_w(C)$ and
$(\nu^\mathscr{C}_2)_w(C)$ are both finite for all $C \in
\mathscr{C}$. 

Now $\forall B \in \mathscr{Y}$, $\forall_\lambda w$,
$(\nu^\mathscr{C}_1)_w (B) = (\nu^\mathscr{C}_2)_w (B)$ since both
$\nu^\mathscr{C}_1(B)$ and $\nu^\mathscr{C}_2(B)$ are conditional
expectations with respect to $\mathscr{C}$ of the extended real valued
function $\nu(B)$. Hence in particular, $\forall C\in \mathscr{C}$,
$\forall_\lambda w$, $(\nu^\mathscr{C}_1)_w\break(C) =
(\nu^\mathscr{C}_2)_w(C)$. 

Since $\mathscr{C}$ is countable, we therefore have, 
$$
\forall_\lambda w, \forall C \in \mathscr{C}, (\nu^\mathscr{C}_1)_w(C)
= (\nu^\mathscr{C}_2)_w(C) \text{ i.e. }
$$
there exists a set $N_4 \in\hat{\mathscr{O}}_\lambda$ with
$\lambda(N_4) =0$ such that if $w \not\in N_4$, 
$$
(\nu^\mathscr{C}_1)_w(C) = (\nu^\mathscr{C}_2)_w(C)
$$
for all $C \in \mathscr{C}$. 

Let $w \not\in N_1 \cup N_3 \cup N_4$. Then both
$(\nu^\mathscr{C}_1)_w$ and $(\nu^\mathscr{C}_2)_w$ are
$\sigma$-finite, $(\nu^\mathscr{C}_1)_w(C) = (\nu^\mathscr{C}_2)_w (C)
\;\; \forall$ $C \in \mathscr{C}$ and $(\nu^\mathscr{C}_1)_w(C) < +
\infty$, $(\nu^\mathscr{C}_2)_w (C)< + \infty$ for all $C \in
\mathscr{C}$. Since $\mathscr{C}$ is a $\pi$-system generating
$\mathscr{Y}'$ and countable, by proposition (\ref{part1:chap3},
\S\ \ref{part1:chap3:sec2.1}, \ref{part1:chap3:prop24}), we 
conclude that $(\nu^\mathscr{C}_1)_w = (\nu^\mathscr{C}_2)_w$ on
$\mathscr{Y}'$. Hence if $w \not\in N_1 \cup N_3 \cup N_4$, the
measures $(\nu^\mathscr{C}_1)_w$ and $(\nu^\mathscr{C}_2)_w$ are equal
on $\mathscr{Y}'$. 

Therefore, if $w \not\in N_1 \cup N_3 \cup N_4 \cup N_2$, the measures
$(\nu^\mathscr{C}_1)_w$  and $(\nu^\mathscr{C}_2)_w$ are\pageoriginale
equal on $\mathscr{Y}$ since if $w \not\in N_2$,
$(\nu^\mathscr{C}_1)_w(N) = (\nu^\mathscr{C}_2)_w (N) = 0$. Hence 
$$
\forall_\lambda w, (\nu^\mathscr{C}_1)_w = (\nu^\mathscr{C}_2)_2
\text{ on } \mathscr{Y}. 
$$
\end{proof}

\section{Existence theorems}\label{part1:chap3:sec4}

Let $X$ be a topological space and let $\mathfrak{X}$ be its Borel
$\sigma$-algebra. Let $\mu$ be a positive measure on $\mathfrak{X}$.

\begin{defn}\label{part1:chap3:def37}
$X$ is said to have the $\mu$-{\em compacity} (resp. {\em
    $\mu$-compacity metrizability}) property if there exists a set $N
  \in \mathfrak{X}$ with $\mu(N) = 0$ and a sequence $(K_n)_{n \in
    \mathbb{N}}$  of compact sets (resp. compact metrizable sets) such
  that $X = \bigcup\limits_{n \in \mathbb{N}} K_n \cup N$ and $\forall \;
  n \in \mathbb{N}$, $\mu(K_n) < + \infty$. 
\end{defn}

Note that if $\mu $ is  a $\sigma$-finite Radon measure on $X$, $X$
has the $\mu$ - compacity property. If further to $\mu$ being a
$\sigma$-finite Radon measure, either $X$ is metrizable or every
compact subset of $X$ is metrizable, then $X$ has the $\mu$-compacity
metrizability property. In particular, if $X$ is a Suslin space and
$\mu$ is a $\sigma$-finite Radon measure on $X$, then $X$ has the
$\mu$-compacity metrizability property. (In a Suslin space, every
compact subset is metrizable). 

We shall now give some sufficient conditions for the existence of
conditional expectations of measure valued functions. 

Throughout this section, we shall adopt the following notations. 

$(\Omega, \mathscr{O}, \lambda)$ is a measure space. $Y$ is a
topological space and $\mathscr{Y}$ is its
Borel-$\sigma$-algebra. $\nu$ is a measure valued function on $\Omega$
with values in $\mathfrak{m}^+(Y, \mathscr{Y})$, $\nu \in
\hat{\mathscr{O}}_\lambda$ and having an integral $J$ with respect to
$\lambda$. 

\begin{thm}\label{part1:chap3:thm38}
Let a sequence $(K_n)_{n \in\mathbb{N}}$ of compact sets of $Y$ and a
set $N\in \mathscr{Y}$ exist with the following properties:
\begin{itemize}
\item[{\rm (i)}] $Y = \bigcup\limits_{n \in \mathbb{N}} K_n \cup N$

\item[{\rm (ii)}] $J(N) = 0$ \pageoriginale and 


\item[{\rm (iii)}] $\forall \; n \in \mathbb{N}$, the restriction of
  $J$ to $K_n$ is a Radon measure on $K_n$. 
\end{itemize}

Let $\mathscr{C}$ be a $\sigma$-algebra contained in
$\hat{\mathscr{O}}_\lambda$ such that $\mathscr{C}$ is complete with
respect to $\lambda$ and let $\lambda$ restricted to $\mathscr{C}$ be
$\sigma$-finite. Then, a conditional expectation of $\nu$ with respect
to $\mathscr{C}$ exists. 
\end{thm}

\begin{proof}
Let us split the proof in two cases case \ref{part1:chap3:case1} and
case \ref{part1:chap3:case2}. In case \ref{part1:chap3:case1}, we 
assume that $Y$ is compact and $J$ is a Radon measure on $Y$. In this
case, the assumptions (i), (ii) and (iii) mentioned in the statement
of the theorem are trivially verified. In case
\ref{part1:chap3:case2}, we shall consider a 
general topological space $Y$,  having the properties (i), (ii) and
(iii) mentioned in the statement of the theorem. We shall deduce case
\ref{part1:chap3:case2} from case \ref{part1:chap3:case1}. 

The proof in case \ref{part1:chap3:case1} proceeds in three steps,
Step \ref{part1:chap3:step1}, Step \ref{part1:chap3:step2} and Step
\ref{part1:chap3:step3}.  

In Step \ref{part1:chap3:step1}, we define a measure valued function
$\nu^\mathscr{C}$ on 
$\Omega$ with values in $\mathfrak{m}^+ (Y, \mathscr{Y})$ such that
$\forall w \in \Omega$, $\nu^\mathscr{C}_w$ is a Radon measure on $Y$
and such that $\forall $ real valued continuous function $\mathscr{C}$
on  $Y$, $w \to \nu^\mathscr{C}_w(\varphi)$ is a conditional
expectation with respect to $\mathscr{C}$ of the function $w \to \nu_w
(\varphi)$. 

In Step \ref{part1:chap3:step2}, we prove that $\forall \; U$ open in $Y$,
$\nu^\mathscr{C}(\chi_U)$ is a conditional expectation with respect to
$\mathscr{C}$ of the function $\nu(\chi_U)$. 

In Step \ref{part1:chap3:step3}, we prove that $\forall B \in \mathscr{Y}$,
$\nu^\mathscr{C}(\chi_B)$ is a conditional expectation with respect to
$\mathscr{C}$ of the function $\nu(\chi_B)$. 

\begin{case}\label{part1:chap3:case1}
$Y$ a compact space and $J$, a Radon measure on $Y$. 
\end{case}

\begin{step}\label{part1:chap3:step1}
Since $J$ is a Radon measure on $Y$, $J(Y) < + \infty$. Hence
$\forall_\lambda w$, $\nu_w(Y) <+ \infty$. Without loss of
generality, we can assume that $\forall \; w \in \Omega$, $\nu_w(Y) <
+ \infty$. For, consider the measure valued function $\nu'$ on
$\Omega$ with values in $\mathfrak{m}^+(Y, \mathscr{Y})$ given by 
$$
\nu'_w = 
\begin{cases}
\nu_w \text{ if } \nu_w(Y) < + \infty \\
0 \text{ otherwise i.e. the zero measure if } \nu_w (Y) = + \infty. 
\end{cases}
$$

Then\pageoriginale $\nu'$ is a measure valued function on $\Omega$
with values in $\mathfrak{m}^+ (Y, \mathscr{Y})$, $\nu'\in
\hat{\mathscr{O}}_\lambda$, $\nu'$ has the same integral $J$ as $\nu$
and further $\forall \; w \in \Omega$, $\nu'_w$ is a finite
measure. Also, $\forall_\lambda w$, $\nu'_w =\nu_w$. Hence, if a
conditional expectation of $\nu'$ with respect to $\mathscr{C}$
exists, it is a conditional expectation of $\nu$ with respect to
$\mathscr{C}$, as well.
\end{step}

Hence, we may assume that $\forall\; w \in \Omega$, $\nu_w(Y) < +
\infty$. 

Let  $\varphi$ be any real valued bounded function on $Y$, $\varphi
\in \mathscr{Y}$. Since $\forall w \in \Omega$, $\nu_w$ is a finite
measure on $\mathscr{Y}$, $\forall \; w \in \Omega$, $\varphi$ is
$\nu_w$-integrable. Consider the real valued function $\nu(\varphi)$
taking $w$ to $\nu_w(\varphi)$. Since $\lambda$ restricted to
$\mathscr{C}$ is $\sigma$-finite, a conditional expectation for
$\nu(\varphi)$ with respect to $\mathscr{C}$ exists. Let $[\nu
  (\varphi)]^\mathscr{C}$  be a conditional expectation of $\nu
(\varphi)$ with respect to $\mathscr{C}$. 

Now, let us fix $[\nu(1)]^\mathscr{C}$ once and for all as
\textit{the} conditional expectation of $\nu(1)$ with respect to
$\mathscr{C}$ in such a way that $\forall $ $w \in \Omega$, $0 \leq
       [\nu(1)]^\mathscr{C} (w) < + \infty$. 

Let $||\varphi||$ be $\sup\limits_{y \in Y} |\varphi(y)|$ 

Then, $|\nu(\varphi)| \leq ||\varphi||. \nu(1)$. 

Hence, $\forall_\lambda w$, $\left| [\nu
  (\varphi)]^\mathscr{C}\right| (w) \leq |\nu (\varphi)|^\mathscr{C}
(w) \leq ||\varphi|| [\nu(1)]^\mathscr{C} (w)$. Hence, if $A = \left\{
w: [\nu(1)]^\mathscr{C} (w) = 0\right\}$, then $A\in \mathscr{C}$ and
$\forall_\lambda w$, $w \in A$, $[\nu(\varphi)]^\mathscr{C} (w) =0$. 

Define the quotient
$\dfrac{[\nu(\varphi)]^\mathscr{C}}{[\nu(1)]^\mathscr{C}}$ to be zero
on the set $A$. On $\complement A$, the quotient
$\dfrac{[\nu(\varphi)]^\mathscr{C}}{[\nu(1)]^\mathscr{C}}$  has a
meaning. 

Now, $\forall_\lambda w$, $\left|
\dfrac{[\nu(\varphi)]^\mathscr{C}}{[\nu(1)]^\mathscr{C}} (w) \right|
\leq ||\varphi||$. Hence the function
$\dfrac{[\nu(\varphi)]^\mathscr{C}}{[\nu(1)]^\mathscr{C}}$ is
essentially bounded. Since it belongs to $\mathscr{C}$, 
$$
\dfrac{[\nu(\varphi)]^\mathscr{C}}{[\nu(1)]^\mathscr{C}}  \in L^\infty
(\Omega; \mathscr{C}; \lambda). 
$$
{\em Note that this element $\dfrac{[\nu
      (\varphi)]^\mathscr{C}}{[\nu(1)]^\mathscr{C}}$ of $L^\infty
  (\Omega; \mathscr{C}; \lambda)$ is independent of the choice of the
  conditional expectation $[\nu(\varphi)]^\mathscr{C}$ of
  $\nu(\varphi)$ and hence depends only on $\varphi$. } 

Let\pageoriginale $\rho$ be a lifting from $L^\infty (\Omega;
\mathscr{C}; \lambda)$ to $\mathscr{B}(\Omega; \mathscr{C})$. The
existence of a lifting is guaranteed by the lifting theorem mentioned
in \S\ \ref{part1:chap3:sec2.2} of this chapter.

Let $\mathscr{C}(Y)$ be the space of all real valued continuous
functions on $Y$. 

If $w \in\Omega$, define the maps $\nu^\mathscr{C}_w$ on
$\mathscr{C}(Y)$ taking real value as follows. If $\psi \in
\mathscr{C}(Y)$, define $\nu^\mathscr{C}_w(\psi)$ as
$[\nu(1)]^\mathscr{C}
(w)$. $\rho\left(\dfrac{[\nu(\psi)]^\mathscr{C}}{\u{[\nu(1)]^\mathscr{C}}}
\right) 
(w)$.  

Then, by the properties of the lifting $\nu^\mathscr{C}_w$ is a
positive linear functional on $\mathscr{C}(Y)$ and hence defines a
positive Radon measure on $Y$. Let $\nu^\mathscr{C}$ be the measure
valued function on $\Omega$ with values in $\mathfrak{m}^+ (Y,
\mathscr{Y})$ taking $w$ to $\nu^\mathscr{C}_w$. 

Since $\mathscr{C}$ is complete, $\forall \; \psi \in \mathscr{C}(Y)$,
$\rho \left( \dfrac{[\nu
    (\psi)]^\mathscr{C}}{\u{[\nu(1)]^\mathscr{C}}}\right) \in
\mathscr{C}$ and hence
$\forall$ $\psi \in \mathscr{C}(Y)$, the function
$\nu^\mathscr{C}(\psi)$ belongs to $\mathscr{C}$. Moreover, if $B
\in\mathscr{C}$, 
\begin{align*}
\int\limits_B \nu^\mathscr{C}_w (\psi) d \lambda(w) & = \int\limits_B
           [\nu(1)]^\mathscr{C} (w) \cdot \rho \left(
           \frac{[\nu(\psi)]^\mathscr{C}}{[\nu(1)]^\mathscr{C}}\right)
           (w) d \lambda(w)\\
& = \int\limits_B [\nu(1)]^\mathscr{C} \cdot
           \dfrac{[\nu(\psi)]^\mathscr{C}}{[\nu(1)]^\mathscr{C}}  d
           \lambda \\
& = \int\limits_B [\nu(\psi)]^\mathscr{C} d\lambda\\
& = \int\limits_B \nu_w(\psi) d \lambda(w). 
\end{align*}

Hence $\nu^\mathscr{C}(\psi)$ is a conditional expectation with
respect to $\mathscr{C}$ of the function $\nu(\psi)$. 

\begin{step}\label{part1:chap3:step2}
Let $U$ be an open subset of $Y$, $U \neq \emptyset$. Since $U$ is
open, $\chi_U$ is a lower-semi-continuous function on $Y$ and hence
there exists a directed increasing family $(\varphi_i)_{i \in I}$ of
continuous functions on $Y$ such that $\forall \; i \in I$, $0 \leq
\varphi_i \leq \chi_U$ and $\sup\limits_{i \in I} \varphi_i =
\chi_U$. 
$$
\forall \; i \in I, \; \forall_\lambda w, \left|
\dfrac{[\nu(\varphi_i)]^\mathscr{C}}{[\nu(1)]^\mathscr{C}}  (w)\right|
\leq ||\varphi_i|| \leq 1. 
$$
\end{step}

Hence, the directed increasing family
$\dfrac{[\nu(\varphi_i)]^\mathscr{C}}{\u{[\nu(1)]^\mathscr{C}}}$ of
elements of $L^\infty (\Omega;\break \mathscr{C}; \lambda)$ is bounded in
norm and hence $\underset{i\in I}(\SUP){}_\lambda
\dfrac{[\nu(\varphi_i)]^\mathscr{C}}{\u{[\nu(1)]^\mathscr{C}}}$ exists by
proposition (\ref{part1:chap3}, \S\ \ref{part1:chap3:sec2.3}, \ref{part1:chap3:prop27}). 

We\pageoriginale claim that 
$$
\underset{i\in I}{\SUP} {}_\lambda
\frac{[\nu(\varphi_i)]^\mathscr{C}}{\u{[\nu(1)]^\mathscr{C}}} =
\frac{[\nu(\chi_U)]^\mathscr{C}}{\u{[\nu(1)]^\mathscr{C}}}
$$

To prove this, it is sufficient to prove because of Corollary (\ref{part1:chap3},
\S\ \ref{part1:chap3:sec2.3}, \ref{part1:chap3:coro29}) that 
$$
\frac{[\nu(\chi_U)]^\mathscr{C}}{\u{[\nu(1)]^\mathscr{C}}} \geq
\frac{[\nu(\varphi_i)]^\mathscr{C}}{\u{[\nu(1)]^\mathscr{C}}} \forall
i \in I, 
$$
and that there exists an increasing sequence $(\varphi_n)_{n
  \in\mathbb{N}}$ of continuous functions, $\varphi_n \;\; \forall n \in
\mathbb{N}$, belonging to the family $(\varphi_i)_{i \in I}$ such that 
$$
\lim\limits_{n \to \infty} \int
\frac{[\nu(\varphi_n)]^\mathscr{C}}{\u{[\nu(1)]^\mathscr{C}}} d
\lambda = \int\frac{[\nu(\chi_U)]^\mathscr{C}}{\u{[\nu(1)]^\mathscr{C}}} d \lambda
$$

Since $\forall \; i \in I$, $\varphi_i \leq \chi_U$, we have
$$
\forall \; i \in I, \; \nu(\varphi_i) \leq \nu'(\chi_U). 
$$

Hence $\forall \; i \in I$, $\forall_\lambda w$,
$[\nu(\varphi_i)]^\mathscr{C} (w) \leq [\nu
  (\chi_U)]^\mathscr{C}(w)$. Therefore, 
$$
\forall \; i \in I,
\frac{[\nu(\varphi_1)]^\mathscr{C}}{\u{[\nu(1)]^\mathscr{C}}}  \leq
\dfrac{[\nu (\chi_U)]^\mathscr{C}}{\u{[\nu(1)]^\mathscr{C}}}. 
$$

Since $J$ is a Radon measure, and $(\varphi_i)_{i \in I}$ is a
directed increasing family of continuous functions with $\chi_U =
\sup\limits_{i \in I} \varphi_i$, we have
$$
J(\chi_U) = \sup\limits_{i \in I} J(\varphi_i). 
$$

Hence there exists an increasing sequence $(\varphi_n)_{n \in
  \mathbb{N}}$ of continuous functions, $\forall \; n \in \mathbb{N}$,
$\varphi_n$ belonging to the family $(\varphi_i)_{i \in I}$ such that
$$
\forall_J y, \; \varphi_n (y) \uparrow \chi_U (y). 
$$

Hence,\pageoriginale 
$$
\forall_\lambda w, \; \forall_{\nu_w} y, \; \varphi_n (y) \uparrow
\chi_U(y), 
$$

Therefore, 
$$
\forall_\lambda w, \; \nu_w (\varphi_n) \uparrow \nu_w(\chi_U). 
$$

Since $w \to [\gamma (1)]^\mathscr{C}(w) \in \mathscr{C}$  and $A \in
\mathscr{C}$, by property (iv) of the conditional expectations of
extended real valued functions mentioned in \S\ \ref{part1:chap1:sec5}
of Chapter \ref{part1:chap1}
(actually the property of $v_{\mathscr{O}, \mathscr{C}}$ mentioned
there is used), we have 
\begin{align*}
\forall \; n \in \mathbb{N}, \int
\frac{[\nu(\varphi_n)]^\mathscr{C}}{[\nu(1)]^\mathscr{C}} (w) d
\lambda(w) & = \int
\frac{[\nu\varphi_n]^\mathscr{C}}{[\nu(1)]^\mathscr{C}} (w) d
\lambda(w)\\
& = \complement A \int\limits_{\complement A}
\frac{\nu_w(\varphi_n)}{[\nu(1)]^\mathscr{C}(w)} d \lambda (w)
\end{align*}

Hence $\forall \; n\in \mathbb{N}$, $\int \dfrac{[\nu
    (\varphi_n)]^\mathscr{C}}{\u{[\nu(1)]\mathscr{C}}} d\lambda = \int\limits_{\complement A}
\dfrac{\nu_w(\varphi_n)}{[\nu(1)]^\mathscr{C} (w)}d \lambda(w)$. Since
$\forall_\lambda w$, $\nu_w (\varphi_n) \uparrow \nu_w (\chi_U)$,
$$
\int\limits_{\complement A}
\frac{\nu_w(\varphi_n)}{[\nu(1)]^\mathscr{C}(w)} d \lambda (w)
\uparrow \int\limits_{\complement A}
\frac{\nu_w(\chi_U)}{[\nu(1)]^\mathscr{C}(w)} d \lambda(w). 
$$    

Again by the same property for $v_{\mathscr{O}, \mathscr{C}}$
mentioned in (iv) of the conditional expectations of extended real
valued functions mentioned in \S\ \ref{part1:chap1:sec5} of Chapter \ref{part1:chap1}, 
\begin{align*}
\int\limits_{\complement A} \frac{\nu_w
  (\chi_U)}{[\nu(1)]^\mathscr{C}(w)} d\lambda (w) & =
\int\limits_{\complement A}
\frac{[\nu(\chi_U)]^\mathscr{C}}{[\nu(1)]^\mathscr{C}} (w)
d\lambda(w)\\
& = \int \frac{[\nu (\chi_U)]^\mathscr{C}}{[\nu(1)]^\mathscr{C}} (w) d
\lambda(w)\\
& = \int\frac{[\nu(\chi_U)]^\mathscr{C}}{\u{[\nu(1)]^\mathscr{C}}} d\lambda
\end{align*}

Hence, $\lim\limits_{n\to \infty} \int \dfrac{[\nu
    (\varphi_n)]^\mathscr{C}}{\u{[\nu(1)]^\mathscr{C}}}d \lambda$
exists and is equal to $\int
\dfrac{[\nu(\chi_U)]^\mathscr{C}}{\u{[\nu(1)]^\mathscr{C}}}
d\lambda$. 
This proves that
$\dfrac{[\nu(\chi_U)]^\mathscr{C}}{[\nu(1)]^\mathscr{C}} =
\underset{i\in I}{\SUP}{}_\lambda
\dfrac{[\nu(\varphi_i)]^\mathscr{C}}{\u{[\nu(1)]^\mathscr{C}}}$  and
therefore, by proposition (\ref{part1:chap3},
\S\ \ref{part1:chap3:sec2.3}, \ref{part1:chap3:prop30}), 
$$
\forall_\lambda w, \; \rho
\left(\frac{[\nu(\chi_U)]^\mathscr{C}}{\u{[\nu(1)]^\mathscr{C}}}
\right)  (w) = \sup\limits_{i \in I} \rho
\left(\frac{[\nu(\varphi_i)]^\mathscr{C}}{\u{[\nu(1)]^\mathscr{C}}}
\right) (w).   
$$\pageoriginale 
and $\sup\limits_{i \in I} \rho \left(\dfrac{[\nu
    (\varphi_i)]^\mathscr{C}}{\u{[\nu(1)]^\mathscr{C}}} \right) \in
\mathscr{C}$   since $\mathscr{C}$ is complete. Since $\forall w
\in\Omega$, $\nu^\mathscr{C}_w$ is a Radon measure,
\begin{align*}
\nu^\mathscr{C}_w (\chi_U) & = \sup\limits_{i \in I}
\nu^{\mathscr{C}}_w(\varphi_i)\\
& = \sup\limits_{i \in I} [\nu(1)]^\mathscr{C} (w) \rho
\left(\frac{[\nu(\varphi_i)]^\mathscr{C}}{\u{[\nu (1)]^\mathscr{C}}}
\right)(w)  \\
& = [\nu(1)]^\mathscr{C} (w). \sup\limits_{i \in I} \rho \left(
\frac{[\nu(\varphi_i)]^\mathscr{C}}{\u{[\nu(1)]^\mathscr{C}}}  
\right) (w).
\end{align*}

Hence $\nu^\mathscr{C} (\chi_U) \in \mathscr{C}$, since both
$[\nu(1)]^\mathscr{C}$ and $\sup\limits_{i \in I} \rho \left(
\dfrac{[\nu
    (\varphi_i)]^\mathscr{C}}{\u{[\nu(1)]^\mathscr{C}}}\right)$ belong
to $\mathscr{C}$. 

Let $B \in \mathscr{C}$. Then, 
\begin{align*}
\int\limits_B \nu^\mathscr{C}_w(\chi_U)d \lambda (w) & = \int\limits_B
           [\nu(1)]^\mathscr{C} (w) \sup\limits_{i \in I} \rho \left(
           \frac{[\nu(\varphi_i)^\mathscr{C}]}{\u{[\nu(1)]^\mathscr{C}}}\right)
           (w) d \lambda (w)\\
& =\int\limits^B_B [\nu(1)]^\mathscr{C} (w)
           \frac{[\nu(\chi_U)]^\mathscr{C}}{[\nu(1)]^\mathscr{C}} (w)
           d\lambda (w)\\
& = \int\limits_B [\nu(\chi_U)]^\mathscr{C} (w) d \lambda(w)\\
& = \int\limits^B_B \nu_w(\chi_U) d \lambda(w). 
\end{align*}

Hence $\nu^\mathscr{C}(\chi_U)$ is a conditional expectation of
$\nu(\chi_U)$. 

\begin{step}\label{part1:chap3:step3}
Let us prove that $\forall$ \; $B \in \mathscr{Y}$, the function
$\nu^\mathscr{C}(\chi_B)$ belongs to $\mathscr{C}$ and is a
conditional expectation with respect to $\mathscr{C}$ of the function
$\nu(\chi_B)$. 
\end{step}

Consider the class $\mathscr{C}$ of all sets $B \in \mathscr{Y}$ for
which $\nu^\mathscr{C} (\chi_B) \in \mathscr{C}$ and is  a
conditional expectation with respect to $\mathscr{C}$ of the function
$\nu(B)$. It is easily seen that $\mathscr{C}$ is a $d$-system. 

From\pageoriginale Step \ref{part1:chap3:step2}, $\mathscr{C}$ contains the class
$\mathcal{U}$ of all open sets of $Y$, $\mathcal{U}$ is a $\pi$-system
generating $\mathscr{Y}$. Hence, by the Monotone class theorem,
$\mathscr{C}$ contains $\mathscr{Y}$ and hence $\mathscr{C} =
\mathscr{Y}$. 

Hence $\forall B \in\mathscr{Y}$, the function $\nu^\mathscr{C}
(\chi_B)$ belongs to $\mathscr{C}$ and is a conditional expectation of
the function  $\nu(\chi_B)$ with respect to $\mathscr{C}$. This shows
that the measure valued function $\nu^\mathscr{C}$ on $\Omega$ with
values in $\mathfrak{m}^+ (Y, \mathscr{Y})$ is a conditional
expectation of $\nu$ with respect to $\mathscr{C}$. 

Thus, Case \ref{part1:chap3:case1} is completely proved. 

\begin{case}\label{part1:chap3:case2}
Let $Y$ be an arbitrary topological space having the properties stated
in the theorem, i.e. $\exists$ a sequence $(K_n)_{n \in\mathbb{N}}$ of
compact sets  of $Y$ and a set $N \in \mathscr{Y}$ with $J(N) =0$ such
that $Y = \bigcup\limits_{n \in \mathbb{N}} K_n \cup N$ and such that
the measure $J_{K_n}$, the restriction of $J$ to $K_n$ is a Radon
measure on $K_n$. Let 
$$
X_n = K_n \backslash (K_1 \cup \ldots \cup K_{n -1}). 
$$
Then $\forall \; n \in \mathbb{N}$, $X_n$ is a Borel set of $Y$, $X_n
\cap X_m = \emptyset$ if $n \neq m$, $Y = \bigcup\limits_{n
  \in\mathbb{N}} X_n \cup N$ and the measure $J_{X_n}$, the
restriction of $J$ to $X_n$ is a finite Radon measure on $X_n$. By
proposition (\ref{part1:chap3}, \S\ \ref{part1:chap3:sec2.4},
\ref{part1:chap3:prop34}), $\forall \; n \in\mathbb{N}$, 
$\exists$ a sequence $(X^m_n)_{m \in\mathbb{N}}$ of mutually disjoint
compact sets of $X_n$ and a Borel set $N_n$ of $X_n$ with
$J_{X_n}(N_n) = 0$ such that $\left\{(X^m_n)_{m \in \mathbb{N}}, N_n
\right\}$ is a $J_{X_n}$-concassage of $X_n$. 
\end{case}

Thus, there exists a sequence $(Y_n)_{n \in \mathbb{N}}$ of compact
sets of $Y$ and a set $M \in \mathscr{Y}$ with $J(M)=0$ such that $Y
=\bigcup\limits_{n \in \mathbb{N}} Y_n \cup M$, $Y_n \cap Y_m =
\emptyset$ if $n \neq m$ and the measure $J_{Y_n}$, the restriction of
$J$ to $Y_n$ is a Radon measure on $Y_n$. Let $\forall n $,
$\mathscr{Y}_n$ be the Borel $\sigma$-algebra of $Y_n$. 

Let $\forall$ $n \in \mathbb{N}$, $\nu^n$ be the measure valued
function on $\Omega$ with values in $\mathfrak{m}^+ (Y_n
\mathscr{Y}_n)$, taking $w \in \Omega$ to the measure $\nu^n_w$ which
is the restriction of $\nu_w$ to $Y_n$. Then, $\forall$ $n\in
\mathbb{N}$, $\nu^n \in\hat{\mathscr{O}}_\lambda$. $\forall \; B \in
\mathscr{Y}_n$, 
$$
\int \nu^n_w (B) d \lambda (w) = \int \nu_w (B) d \lambda (w) = J(B) = 
J_{Y_n} (B). 
$$\pageoriginale

Hence the integral of $\nu^n$ with respect to $\lambda$ is the measure
$J_{Y_n}$ which is a Radon measure on $Y_n$. Hence, by Case \ref{part1:chap3:case1},
$\forall$ $ n \in\mathbb{N}$, a conditional expectation
$\nu^{\mathscr{C},n}$ of $\nu^n$ with respect to $\mathscr{C}$
exists. 

Define $\forall $ $n \in\mathbb{N}$, $\forall \; w \in \Omega$, the
measures ${\nu'}^{\mathscr{C}, n}_w$ on $\mathscr{Y}$ as follows: 

If $B \in \mathscr{Y}$, define ${\nu'}^{\mathscr{C}, n}_w(B)$ as equal
to $\nu^{\mathscr{C},n}_w (B \cap Y_n)$. 

Then $\forall \; n \in\mathbb{N}$, $\forall w\in \Omega$,
${\nu'}^{\mathscr{C}, n}_w$ is a positive measure on $\mathscr{Y}$ and
the me assure valued function ${\nu'}^{\mathscr{C},n}$ on $\Omega$ with
values in $\mathfrak{m}^+(Y, \mathscr{Y})$ taking $w$ to
${\nu'}^{\mathscr{C}, n}_w$ belongs to $\mathscr{C}$. 

Define $\forall \; w \in \Omega$, the measure ${\nu'}^{\mathscr{C}}_w$
on $\mathscr{Y}$ as ${\nu'}^{\mathscr{C}}_w = \sum\limits_{n
  \in\mathbb{N}} {\nu'}^{\mathscr{C}, n}_w$ i.e. $\forall B \in
\mathscr{Y}$, ${\nu'}^\mathscr{C}_w(B) = \sum\limits_{n \in
  \mathbb{N}} {\nu'}^{\mathscr{C},n}_w(B)$. Then, the measure valued
function ${\nu'}^\mathscr{C}$ on $\Omega$ with values in
$\mathfrak{m}^+ (Y, \mathscr{Y})$ taking $w$ to the measure
${\nu'}^\mathscr{C}_w$ belongs to $\mathscr{C}$. 

Let $C \in \mathscr{C}$ and $B \in \mathscr{Y}$. Then,
\begin{align*}
\int\limits_C {\nu'}^\mathscr{C}_w(B) d \lambda(w) & = \int\limits_C
(\sum\limits_{n \in\mathbb{N}} {\nu'}^{\mathscr{C}, n}_w (B))
d\lambda(w)\\
& = \sum\limits_{n \in \mathbb{N}} \int\limits_C {\nu'}^{\mathscr{C},
  n}_w (B) d \lambda (w)\\
& = \sum\limits_{n \in\mathbb{N}} \int\limits_C \nu^{\mathscr{C}, n}_w
(B \cap Y_n) d\lambda(w)\\
& = \sum\limits_{n \in \mathbb{N}} \int\limits_C \nu^n_w (B\cap Y_n) d
\lambda(w) 
\end{align*}
since $\forall \; n \in\mathbb{N}$, $\nu^{\mathscr{C},n} (B \cap Y_n)$
is a conditional expectation with respect to $\mathscr{C}$ of $\nu^n(B
\cap Y_n)$. So the left side above is 
\begin{align*}
& = \sum\limits_{n \in\mathbb{N}} \int\limits_C \nu_w (B \cap Y_n) d
  \lambda(w)\\
& = \int\limits_C (\sum\limits_{n \in\mathbb{N}} \nu_w (B \cap Y_n)) d
  \lambda(w). 
\end{align*}

Since\pageoriginale $J(M) = 0$, $\forall_\lambda w$, $\nu_w (M) =
0$. Hence, $\forall B \in \mathscr{Y}$, $\forall_\lambda w$,
$\sum\limits_{n\in \mathbb{N}} \nu_w (B \cap Y_n) = \nu_w (B)$ since
the sequence $(Y_n)_{n \in \mathbb{N}}$ is mutually disjoint and $Y =
\bigcup\limits_{n \in \mathbb{N}} Y_n \cup M$. 

Hence, 
$$
\int\limits_C \sum\limits_{n\in\mathbb{N}} \nu_w (B \cap Y_n) d
\lambda(w) = \int\limits_C \nu_w (B) d \lambda (w). 
$$

Therefore, ${\nu'}^\mathscr{C} (B)$  is a conditional expectation with
respect to $\mathscr{C}$ of the function $\nu(B) \forall B \in
\mathscr{Y}$. Hence ${\nu'}^\mathscr{C}$ is a conditional expectation
of $\nu$ with respect to $\mathscr{C}$.
 \end{proof}


\begin{rem}\label{part1:chap3:rem39}
The assumptions in the above theorem regarding $Y$ and $J$ are
fulfilled if $J$ is a $\sigma$-finite Radon measure on $Y$ and these
are stronger than the $J$-compacity property for $Y$. 
\end{rem}

\begin{lem}\label{part1:chap3:lem40}
Let $Y$ be a compact metrizable space. Let $\mathscr{C}$ be a
$\sigma$-algebra contained in $\hat{\mathscr{O}}_\lambda$ and let
$\hat{\mathscr{C}}_\lambda$ be the completion of $\mathscr{C}$ with
respect to $\lambda$. Let  $\nu^{\hat{\mathscr{C}}_\lambda}$ be a
measure valued function on $\Omega$ with values in $\mathfrak{m}^+ (Y,
\mathscr{Y})$ belonging to $\hat{\mathscr{C}}_\lambda$ such that
$\forall \; w \in \Omega$, $\nu_w^{\hat{\mathscr{C}}_\lambda}$ is a
Radon measure on $Y$ and $J'$, the integral of
$\nu^{\hat{\mathscr{C}}_\lambda}$ with respect to $\lambda$ be a
finite measure on $\mathscr{Y}$. Then, there exists a measure valued
function $\nu^\mathscr{C}$ on $\Omega$ with values in $\mathfrak{m}^+
(Y, \mathscr{Y})$ belonging to $\mathscr{C}$, such that $\forall w
\in\Omega$, $\nu^\mathscr{C}_w$ is a Radon measure on $Y$ and
$\forall_\lambda w$, $\nu^\mathscr{C}_w = \nu^{\hat{\mathscr{C}}_\lambda}_w$. 
\end{lem}

\begin{proof}
Since $Y$ is a compact metrizable space, the Banach space
$\mathscr{C}(Y)$ of all real valued continuous functions on $Y$ has a
{\em countable dense} set $D$. If $\psi \in  \mathscr{C}(Y)$, let
$||\psi||$ be $\sup\limits_{y \in Y} |\psi (y)|$. We can assume that
$D$ has the following property, namely, given any $f \in
\mathscr{C}(Y)$, $\u{f \geq 0}$ and any $\epsilon > 0$, there exists a
function $\varphi_\epsilon \in D$, $\u{\varphi_\epsilon \geq 0}$ such
that $||f - \varphi_\epsilon|| \leq \epsilon$. i.e., the positive
elements of $\mathscr{C}(Y)$ can be approximated at will by positive
elements of $D$. Let a sequence $(\varphi_n)_{n \in \mathbb{N}}$ of
continuous functions on $Y$, constitute the set $D$. 

Now, $\forall \; n \in \mathbb{N}$, the function
$\nu^{\hat{\mathscr{C}}_\lambda} (\varphi_n)$ belongs to
$\hat{\mathscr{C}}_\lambda$ and is $\lambda$-integra\-ble. Hence
$\forall \; n \in \mathbb{N}$, there exist functions $f^n_1$ and
$f^n_2$ on $\Omega$ belonging to $\mathscr{C}$ such that 
$$
\forall \; w \in \Omega, f^n_1 (w) \leq
\nu^{\hat{\mathscr{C}}_\lambda}_w (\varphi_n) \leq f^n_2 (w)
$$\pageoriginale
and the set $B_n = \left\{ w \in\Omega \mid f^n_1(w) \neq f^n_2 (w)
\right\}$ has $\lambda$-measure zero. 

Let $B = \bigcup\limits_{n \in \mathbb{N}} B_n$. Then $B \in
\mathscr{C}$ and $\lambda(B) = 0$. 

Let $\varphi \in \mathscr{C} (Y)$. There exists a sequence
$\varphi_{n_k}$ of functions belonging to $D$ such that 
$$
|| \varphi_{n_k} - \varphi|| \to 0 \text{ as } n_k \to \infty. 
$$

Since $\forall \; w$, $\nu^{\hat{\mathscr{C}}_\lambda}_w$ is a Radon
measure on $Y$ and hence $\forall w$,
$\nu^{\hat{\mathscr{C}}_\lambda}_w (Y) < + \infty $ since $Y$ is
compact, we have $\forall w \in \Omega$,
$\nu^{\hat{\mathscr{C}}_\lambda}_w (\varphi_{n_k}) \to
\nu^{\hat{\mathscr{C}}_\lambda}_w (\varphi)$ as $n_k \to \infty$.

Hence, if $w \not\in B$, $\lim\limits_{n_k \to \infty} f^{n_k}_1(w)$
and $\lim\limits_{n_k \to \infty} f^{n_k}_2 (w) $ exist and both are
equal to $\lim\limits_{n_k \to \infty}
\nu^{\hat{\mathscr{C}}_\lambda}_w (\varphi_{n_k})$ which is
$\nu^{\hat{\mathscr{C}}_\lambda}_w (\varphi) $. 

Therefore, $\forall \varphi \in \mathscr{C}(Y)$, $\forall w \not\in
B$, the $\lim\limits_{n_k \to \infty} f^{n_k}_1(w)$ is {\em
  independent} of the choice of the sequence $(\varphi_{n_k})$ chosen
to converge to $\varphi$ in $\mathscr{C}(Y)$. 

Hence, $\forall w \in \Omega$, define the map $\nu^\mathscr{C}_w$ on
$\mathscr{C}(Y)$ as, 
$$
\nu^{\mathscr{C}}_w (\varphi) = 
\begin{cases}
\lim\limits_{n_k \to \infty} f^{n_k}_1 (w), & \text{ if } w \not\in B\\
0  & \text{ if } w \in B. 
\end{cases}
$$
where $\varphi\epsilon \mathscr{C}(Y)$ and $(\varphi_{n_k})_{n_k \in
  \mathbb{N}}$ is a sequence such that $\forall n_k \in \mathbb{N}$,
$\varphi_{n_k} \in D$ and $\varphi_{n_k} \to \varphi$ as $n_k \to
\infty$ in $\mathscr{C}(Y)$. $\forall \; w \in \Omega$, the map
$\nu^\mathscr{C}_w$ defined on $\mathscr{C}(Y)$ as above is clearly
linear and if $w \not\in B$, $\nu^\mathscr{C}_w (\varphi) =
\nu^{\hat{\mathscr{C}}_\lambda}_w (\varphi)$  for all $\varphi \in
\mathscr{C}(Y)$. i.e. $\forall_\lambda w$, $\forall \varphi \in
\mathscr{C}(Y)$, $\gamma^\varphi_w(\varphi) = $
$\nu^{\hat{\mathscr{C}}_\lambda}_w(\varphi)$. Moreover the linear
functional $\nu^\mathscr{C}_w$ is positive because of our assumptions
on  $D$ that the positive elements of $\mathscr{C}(Y)$, can be
approximated by positive elements of $D$. Hence, $\forall w \in
\Omega$, $\nu^{\mathscr{C}}_w$ is a Radon measure on $Y$. 

Since $\forall_\lambda w$, $\forall \varphi \in \mathscr{C}(\varphi) =
\nu^{\hat{\mathscr{C}}_\lambda}_w$ $(\varphi)$ and since $\forall w
\in \Omega$, $\nu^\mathscr{C}_w$ and
$\nu^{\hat{\mathscr{C}}_\lambda}_w$ are Radon measures, it follows
that $\forall_\lambda w$, the measures $\nu^{\mathscr{C}}_w$ and
$\nu^{\hat{\mathscr{C}}_\lambda}_w$  are equal,\pageoriginale i.e.
$$
\forall_\lambda w, \nu^{\mathscr{C}}_w =
\nu^{\hat{\mathscr{C}}_\lambda}_w. 
$$

We have to prove that the measure valued function $\nu^\mathscr{C}$ on
$\Omega$ taking $w$ to $\nu^\mathscr{C}_w$ belongs to $\mathscr{C}$. 

If $\varphi \in \mathscr{C} (Y)$, it is clear that the function
$\nu^\mathscr{C}(\varphi)$ belongs to $\mathscr{C}$ since $\forall $
$n \in \mathbb{N}$, $f^n_1 \in \mathscr{C}$ and $B \in \mathscr{C}$. 

Let $U$ be a non-void open set. Since $Y$ is metrizable, there exists
an increasing sequence $\varphi_n$ of continuous functions on $Y$ such
that $\forall n \in\mathbb{N}$, $0 \leq \varphi_n \leq \chi_U$ and
$\varphi_n(y) \uparrow \chi_U (y)$ {\em for all $y \in Y$}. 

Hence $\forall w \in \Omega$, $\nu^\mathscr{C}_w (\chi_U) =
\lim\limits_{n \to \infty} \nu^\mathscr{C}_w (\varphi_n)$. Therefore,
$\forall U$ open, $\nu^\mathscr{C}\break(\chi_U) \in \mathscr{C}$. 

Now, a standard application of the Monotone class theorem will yield
that $\forall C \in \mathscr{Y}$, the function
$\nu^\mathscr{C}(\chi_C)$ belongs to $\mathscr{C}$. 

Hence the measure valued function $\nu^\mathscr{C}$ belongs to
$\mathscr{C}$ and since $\forall_\lambda w$, $\nu^\mathscr{C}_w =
\nu^{\hat{\mathscr{C}}_\lambda}_w$, our lemma is completely proved.  
\end{proof}

\begin{thm}\label{part1:chap3:thm41}
Let $Y$ have the $J$-compacity metrizability property. Let
$\mathscr{C}$ be a $\sigma$-algebra contained in
$\hat{\mathscr{O}}_\lambda$ and let $\lambda$ restricted to
$\mathscr{C}$ be $\sigma$-finite. Then a conditional expectation of
$\nu$ with respect to $\mathscr{C}$ exists and is unique. 
\end{thm}

\begin{proof}
Let us prove the theorem under the assumption that $Y$ is a compact
metrizable space and $J$ is a finite measure on $\mathscr{Y}$. The
general case will follow along lines similar to case
\ref{part1:chap3:case2} of theorem (\ref{part1:chap3},
\S\ \ref{part1:chap3:sec4}, \ref{part1:chap3:thm38}). 

Since $J$ is a finite measure on $\mathscr{Y}$ and $Y$ is a compact
metrizable space, $J$ is a Radon measure on $Y$. Let
$\hat{\mathscr{C}}_\lambda$ be the completion of $\mathscr{C}$ with
respect to $\lambda$. By case \ref{part1:chap3:case1} of theorem
(\ref{part1:chap3}, \S\ \ref{part1:chap3:sec4}, \ref{part1:chap3:thm38}), a 
conditional expectation $\nu^{\hat{\mathscr{C}}_\lambda}$ of $\nu$
with respect to $\hat{\mathscr{C}}_\lambda$ exists in such a way that
$\forall$ $w \in\Omega$, $\nu^{\hat{\mathscr{C}}_\lambda}_w$ is a
Radon measure on $Y$. 
$$
\int \nu^{\hat{\mathscr{C}}_\lambda}_w (Y) d \lambda (w) = \int \nu_w
(Y) d \lambda (w) = J(Y) < + \infty. 
$$\pageoriginale
Hence, by the above lemma (\ref{part1:chap3},
\S\ \ref{part1:chap3:sec4}, \ref{part1:chap3:lem40}), there exists a
measure 
valued function $\nu^\mathscr{C}$ on $\Omega$ with values in
$\mathfrak{m}^+ (Y, \mathscr{Y})$ such that $\nu^\mathscr{C}\epsilon
\mathscr{C}$ and $\forall_\lambda w$, $\nu^\mathscr{C}_w =
\nu^{\hat{\mathscr{C}}_\lambda}_w$. 

Hence, since $\nu^{\hat{\mathscr{C}}_\lambda}$ is a conditional
expectation of $\nu$ with respect to $\hat{\mathscr{C}}_\lambda
. \nu^\mathscr{C}$ is a conditional expectation of $\nu$ with respect
to $\mathscr{C}$. 


Since the Borel $\sigma$-algebra of a compact metrizable space is
countably generated, we can easily see that if $Y$ has the
$J$-compacity metrizability property, $\mathscr{Y}$ has the
$J$-countability property. Hence by theorem (\ref{part1:chap3},
\S\ \ref{part1:chap3:sec3}, \ref{part1:chap3:thm36}) the 
conditional expectation of $\nu$ with respect to $\mathscr{C}$ is
unique. 
\end{proof}


\section[Existence theorem for disintegration...]{Existence theorem
  for disintegration of a\hfil\break measure. The  
  theorem of M. Jirina}\label{part1:chap3:sec5}

Throughout this section, let $\Omega$ be a topological space,
$\mathscr{O}$ its Borel $\sigma$-algebra and $\lambda$ a positive
measure on $\mathscr{O}$. 

We have already remarked in \S\ \ref{part1:chap3:sec1} of this chapter, that a
disintegration of $\lambda$ with respect to $\mathscr{C}$ is a
conditional expectation of the measure valued function $\delta$ on
$\Omega$ with values in $\mathfrak{m}^+ (\Omega, \mathscr{O})$ taking
$w$ to the Dirac measure $\delta_w$ and vice versa. So, by theorem
(\ref{part1:chap3}, \S\ \ref{part1:chap3:sec3},
\ref{part1:chap3:thm36}) we get the following uniqueness theorem for  
disintegrations and by the theorems (\ref{part1:chap3},
\S\ \ref{part1:chap3:sec4}, \ref{part1:chap3:thm38}) and
(\ref{part1:chap3}, \S\ \ref{part1:chap3:sec4},
\ref{part1:chap3:thm41}), we get the following two theorems for the
existence of disintegrations. More precisely, we have 

\begin{thm}[Uniqueness]\label{part1:chap3:thm42}
Let $\lambda$ be a $\sigma$-finite measure on $\mathscr{O}$ and let
$\mathscr{O}$ have the $\lambda$-countability property. Let $\mathscr{C}$ be a $\sigma$-algebra contained in $\hat{\mathscr{O}}_\lambda$ and let $\lambda$
restricted to $\mathscr{C}$ be $\sigma$-finite. Then, if $\left\{
(\lambda^\mathscr{C}_1)_w \right\}_{w in Omega}$  and $\left\{
(\lambda^\mathscr{C}_2)_w\right\}_{w \in \Omega}$ are two
disintegrations of $\lambda$ with respect to $\mathscr{C}$, we have
$\forall_\lambda w$, $(\lambda^\mathscr{C}_1)_w =
(\lambda^\mathscr{C}_2)_w$.  
\end{thm}

\begin{thm}[Existence(M.Jirina)]\label{part1:chap3:thm43}
Let there exist a sequence $(K_n)_{n \in \mathbb{N}}$ of compact sets
of $\mathscr{O}$ and a set $N \in \mathscr{O}$ such that
\begin{itemize}
\item[{\rm (i)}] $\Omega = \bigcup\limits_{n \in \mathbb{N}} K_n \cup
  N$,\pageoriginale 

\item[{\rm (ii)}]  $\lambda(N) = 0$,  and 

\item[{\rm (iii)}] $\forall \;  n \in \mathbb{N}$, the restriction of
  $\lambda$ to $K_n$ is a Radon measure on $K_n$. 
\end{itemize}

Let $\mathscr{C}$ be a complete $\sigma$-algebra contained in
$\hat{\mathscr{O}}_\lambda$ i.e. let $\mathscr{C} =
\hat{\mathscr{C}}_\lambda$. Then, a disintegration of $\lambda$ with
respect to $\mathscr{C}$ exists. 
\end{thm}

\begin{thm}[Existence]\label{part1:chap3:thm44}
Let $\Omega$ have the $\lambda$-compacity metrizability\break property. Let
$\mathscr{C}$ be a $\sigma$-algebra contained in
$\hat{\mathscr{O}}_\lambda$ and let $\lambda_{|\mathscr{C}}$ be
$\sigma$-finite, where $\lambda_{|\mathscr{C}}$ stands stands for the
restriction on $\lambda$ to $\mathscr{C}$. Then, a disintegration of
$\lambda$ with respect to $\mathscr{C}$ exists and is unique. 
\end{thm}

We remark that the assumptions in the theorem (\ref{part1:chap3},
\S\ \ref{part1:chap3:sec5},  \ref{part1:chap3:thm43}) are
fulfilled if $\lambda$ is a $\sigma$-finite Radon measure on
$\Omega$. $\Omega$ has the $\lambda$-compacity metrizability property
if $\lambda$ is a $\sigma$-finite Radon measure on $\Omega$ and if
$\Omega$ is either metrizable or if every compact subset of $\Omega$
is metrizable. In particular, if $\Omega$ is a Suslin space and if
$\lambda$ is a $\sigma$-finite Radon measure on $\Omega$, then
$\Omega$ has the $\lambda$-compacity metrizability property. 

When $\lambda$ is a Radon probability measure on $\Omega$ i.e.,
$\lambda$ is a Radon measure and $\lambda(\Omega) =1$, the theorem
(\ref{part1:chap3}, \S\ \ref{part1:chap3:sec5},
\ref{part1:chap3:thm43}) is essentially due to M. Jirina \cite{key1}
in the sens that this theorem is an easy consequence of his theorem 3.2, on
page 448 and the `note added in proof' in page 450.

\section[Another kind of existence theorem...]{Another kind of
  existence theorem for conditional expectation 
of measure valued functions}\label{part1:chap3:sec6}

Throughout this section, let $\Omega$ be a topological space,
$\mathscr{O}$ its Borel $\sigma$-algebra, $\lambda$ a positive measure
on $\mathscr{O}$, $Y$ a topological space and $\mathscr{Y}$ its Borel
$\sigma$-algebra. 

If $\mathscr{C}$ is a $\sigma$-algebra of $\hat{\mathscr{O}}_\lambda$,
we saw in Chapter \ref{part1:chap1} how the existence of a disintegration of $\lambda$
with respect to $\mathscr{C}$, implies immediately the existence of
conditional\pageoriginale expectation with respect to $\mathscr{C}$ of
non-negative, extended real valued functions on $\Omega$ belonging
to $\mathscr{O}$ and in Chapter \ref{part1:chap2}, we saw how the existence of a
disintegration of $\lambda$ with respect to  $\mathscr{C}$ implies the
existence of conditional expectation with respect to
$\hat{\mathscr{C}}_\lambda$ of extended real valued
$\lambda$-integrable functions belonging to
$\hat{\mathscr{O}}_\lambda$ and also of Banach space valued
$\lambda$-integrable functions belonging to
$\hat{\mathscr{O}}_\lambda$. In this section, we shall consider the
case of measure valued functions in relation to their conditional
expectations with respect to $\mathscr{C}$, when a disintegration of
$\lambda$ with respect to $\mathscr{C}$ exists. 

\begin{thm}\label{part1:chap3:thm45}
Let $\mathscr{C}$ be a $\sigma$-algebra contained in
$\hat{\mathscr{O}}_\lambda$. Let $(\lambda^\mathscr{C}_w)_{w
  \in\Omega}$ be a disintegration of $\lambda$ with respect to
$\mathscr{C}$. Let $\nu$ be a measure valued function on $\Omega$ with
values in $\mathfrak{m}^+ (Y, \mathscr{Y} )$. Let $\u{\nu \in
  \mathscr{O}}$. Then the measure valued function $\nu^\mathscr{C}$ on
$\Omega$ with values in $\mathfrak{m}^+ (Y, \mathscr{Y})$ defined as
$\nu^\mathscr{C}_w = \int \nu_{w'} \lambda^\mathscr{C}_w(dw')$  is a
conditional expectation of $\nu$ with respect to $\mathscr{C}$. In
particular, a conditional expectation of $\nu$ with respect to
$\mathscr{C}$ exists. 
\end{thm}

\begin{proof}
Let $f$ be a function on $Y$, $f \geq 0$, $f \in \mathscr{Y}$. Consider the  extended real valued function
$\nu(f)$. This function belongs to $\mathscr{O}$ since $\nu \in
\mathscr{O}$. Since $(\lambda^\mathscr{C}_w)_{w \in \Omega}$ is a
disintegration of $\lambda$ with respect to $\mathscr{C}$, the
function $w \to \int \nu_{w'} (f) \lambda^\mathscr{C}_w(dw')$ is a
conditional expectation of the function $\nu(f)$ with respect to
$\mathscr{C}$. Hence the measure valued function $\nu^\mathscr{C}$ on
$\Omega$  with values in $\mathfrak{m}^+ (Y, \mathscr{Y})$ defined as
$\nu^\mathscr{C}_w = \int \nu_w, \lambda^\mathscr{C}_w(dw')$  $\forall
w \in\Omega$, is a conditional expectation of $\nu$ with respect to
$\mathscr{C}$. 
\end{proof}

When $\nu \in \hat{\mathscr{O}}_\lambda$, we cannot apply the above
argument since for $f \geq 0$ on $Y$, belonging to $\mathscr{Y}$,
$\nu(f)$, though belongs to $\hat{\mathscr{O}}_\lambda$, does not in
general belong to $\mathscr{O}$. Hence, the integral $\int \nu_{w'} (f)
\lambda^\mathscr{C}_w(dw')$ does not have a meaning in general for all
functions $f$ on $Y$, $f\geq 0$, $f \in \mathscr{Y}$, since the
measures $\lambda^\mathscr{C}_w$ are measures on $\mathscr{O}$ and not
on $\hat{\mathscr{O}}_\lambda$ for all $w \in \Omega$. 


However,\pageoriginale for functions $\nu$ belonging to
$\hat{\mathscr{O}}_\lambda$, we have the following theorem of
existence of conditional expectations. 

\begin{thm}\label{part1:chap3:thm46}
Let $\mathscr{}$ be a $\sigma$-algebra contained in
$\hat{\mathscr{O}}_\lambda$. Let $(\lambda^\mathscr{C}_w)_{w
  \in\Omega}$ be a disintegration of $\lambda$ with respect to
$\mathscr{C}$. Let $\nu$ be a measure valued function on $\Omega$ with
values in $\mathfrak{m}^+ (Y, \mathscr{Y})$, $\nu$ belonging to
$\hat{\mathscr{O}}_\lambda$. Let $J = \int \nu_w d \lambda(w)$. Let
$J$ be $\sigma$-finite and let $\mathscr{Y}$ have the $J$-countability
property. Then, a conditional expectation of $\nu$ with respect to
$\mathscr{C}$ exists and is unique.
\end{thm}

For the proof of this theorem, we need the following lemma.

\begin{lem}\label{part1:chap3:lem47}
Let $\nu$ be a measure valued function on $\Omega$ with values in
$\mathfrak{m}^+ (Y, \mathscr{Y})$, $\nu$ belonging to
$\hat{\mathscr{O}}_\lambda$. Let $J = \int \nu_w d \lambda(w)$. Let
$J$ be $\sigma$-finite. Let $\mathscr{Y}$ have the $J$-countability
property. Then there exists a measure valued function $\nu'$ on
$\Omega$ with values in $\mathfrak{m}^+ (Y, \mathscr{Y})$ such that
$\u{\nu'\in \mathscr{O}}$ and $\forall_\lambda w$, $\nu'_w = \nu_w$ on
$\mathscr{Y}$. 
\end{lem}

\begin{proof}
Since $J$ is $\sigma$-finite, there exists an increasing sequence
$(E_n)_{n \in \mathbb{N}}$ of sets belonging to $\mathscr{Y}$ such
that $Y = \bigcup\limits_{n \in \mathbb{N}} E_n$ and $J(E_n) < +
\infty \forall \; n \in \mathbb{N}$. Hence, 
$$
\forall \; n \in \mathbb{N}, \quad \forall_\lambda w, \nu_w (E_n) < +
\infty, 
$$

Hence,
$$
\forall_\lambda w, \; \forall n \in \mathbb{N}, \nu_w (E_n) < +
\infty. 
$$
i.e. $\exists$ a set $N_1 \in \mathscr{O}$ with $\lambda(N_1) = 0$
such that if $w \not\in N_1$, $\nu_w(E_n) < + \infty$ for all $n \in
\mathbb{N}$. 

Let $N \in \mathscr{Y}$ with $J(N) = 0$ such that the $\sigma$-algebra
$\mathscr{Y} \cap \complement N$ on $Y' = Y \cap \complement N$ is
countably generated. 

Since $J(N) = 0$, $\exists \;\; N_2 \in \mathscr{O}$ with $\lambda(N_2)=0$
such that if $w \not\in N_2$, $\nu_w(N) =0$. 

Let $\mathscr{C} = (C_n)_{n \in \mathbb{N}}$, $\forall n \in
\mathbb{N}$, $C_n \in \mathscr{Y}'$ generate $\mathscr{Y}'$. We can
assume that $\mathscr{C}$ is a $\pi$-system and that $Y' \in
\mathscr{C}$. 

Consider $\forall\; n \in \mathbb{N}$, $\forall\; m \in \mathbb{N}$,
the function $w \to \nu_w (E_m \cap C_n)$. This function belongs to
$\hat{\mathscr{O}}_\lambda$ and $E_m \cap C_n \in
\mathscr{Y}$. Therefore, there exist functions
$f^{m,n}_1$\pageoriginale and $f^{m,n}_2$ on $\Omega$ belonging to
$\mathscr{O}$ such that $f^{m,n}_1(w) \leq \nu_w(E_m \cap C_n) \leq
f^{m,n}_2(w)$ for all $w \in \Omega$ and the set 
$$N^{m,n} = \left\{ w
\in\Omega \mid f^{m,n}_1 (w) \neq f^{m,n}_2(w)\right\}$$ 
has $\lambda$-measure zero. 

Let $N_3 = \bigcup\limits_{\substack{m\in\mathbb{N} \\ n \in
    \mathbb{N}}} N^{m,n}$. Then $N_3\in \mathscr{O}$ and $\lambda(N_3)
= 0$. 

If $y_0$ is any point of $Y$, define the measure valued function
$\nu'$ on $\Omega$ with values in $\mathfrak{m}^+ (Y, \mathscr{Y})$ as
follows. 
$$
\nu'_w=
\begin{cases}
\nu_w, & \text{ if } w \not\in N_1 \cup N_2 \cup N_3\\
\delta_{y_0}, & \text{ if } w \in N_1 \cup N_2 \cup N_3
\end{cases}
$$

From the definition, it is clear that $\forall_\lambda w$, $\nu'_w =
\nu_w$. To prove the lemma, we have to prove only  that $\nu' \in
\mathscr{O}$. Now, 

$\forall m \in \mathbb{N}, \forall n \in \mathscr{N}, \forall w \in
\Omega$,
{\fontsize{9}{11}\selectfont
\begin{align*}
\nu'_w (E_m \cap C_n) & = \chi_{\complement \{N_1\cup N_2 \cup N_3\}}
(w) \cdot \nu_w (E_m \cap C_n) + \chi_{N_1 \cup N_2 \cup N_3} (w)
\cdot \chi_{E_m\cap C_n} (y_0)\\
& = \chi_{\complement\{N_1 \cup N_2 \cup N_3\}} \cdot f^{m,n}_1 (w) +
\chi_{N_1 \cup N_2 \cup N_3} (w) \cdot \chi_{E_m \cap C_n} (y_0). 
\end{align*}}\relax

Since $f^{m,n}_1 \in \mathscr{O}$ and $N_1 \cup N_2 \cup N_3 \in
\mathscr{O}$, it is clear that the function $\nu' (E_m \cap C_n)$
belongs to $\mathscr{O}$. 

Now fix a $m \in \mathbb{N}$. Let 
$$
\mathscr{B} = \left\{ B \in \mathscr{Y}' \mid \nu' (B \cap E_m) \text{
belongs to } \mathscr{O}\right\}. 
$$

It is clear that $\mathscr{B}$ is a $d$-system. $\mathscr{B}$ contains
the $\pi$-system $\mathscr{C}$ which generates $\mathscr{Y}'$. Hence,
by the Monotone class theorem, $\mathscr{B}= \mathscr{Y}'$. Hence,
since $m$ is arbitrary, $\forall \; m \in \mathbb{N}$, $\forall \;B
\in \mathscr{Y}'$, $\nu' (B\cap E_m)$ belongs to $\mathscr{O}$. Since
$\forall B \in \mathscr{Y}'$, $\forall \; w \in \Omega$, $\nu'_w (B) =
\lim\limits_{m \to \infty} \nu'_w(B\cap E_m)$, it follows that
$\nu'(B)$ belongs to $\mathscr{O}$. 

Let 
$$
A \in \mathscr{Y}, \; A= A \cap Y' \cup A \cap N. 
$$

If\pageoriginale $w \not\in N_1 \cup N_2 \cup N_3$, $\nu'_w(A) =
\nu'_w (A \cap Y')$, since $\nu'_w(A \cap N) = \nu_w (A \cap N)
=0$. Hence,  
$$
\nu'_w(A) = \chi_{\complement \{N_1 \cup N_2 \cup N_3\}} (w). \nu'_w
(A \cap Y') + \chi_{N_1 \cup N_2 \cup N_3} (w). \chi_A(y_0) 
$$
for all $w \in\Omega$. Since $A \cap Y' \in \mathscr{Y}'$, the
function $\nu'(A\cap Y')$ belongs to $\mathscr{O}$. Since $N_1 \cup
N_2 \cup N_3 \in \mathscr{O}$, it follows from the above expression of
$\nu'(A)$, that $\nu'(A)$ belongs to $\mathscr{O}$. 

Since $A$ ia an arbitrary set $\in \mathscr{Y}$, it follows that the
measure valued function $\nu'$ belongs to $\mathscr{O}$. 
\end{proof}

\medskip
\noindent{\textbf{Proof of theorem 46}}
From the above lemma (\ref{part1:chap3}, \S\ \ref{part1:chap3:sec6},
\ref{part1:chap3:lem47}) there exists a measure valued 
function $\nu'$ on $\Omega$ with values in $\mathfrak{m}^+(Y,
\mathscr{Y})$ such that $\u{\nu' \in \mathscr{O}}$ and
$\forall_\lambda w$, $\nu'_w = \nu_w$. From theorem
(\ref{part1:chap3}, \S\ \ref{part1:chap3:sec6},
\ref{part1:chap3:thm45}) 
there exists a conditional expectation ${\nu'}^\mathscr{C}$ with
respect to $\mathscr{C}$ for $\nu'$, since $\nu'\in
\mathscr{O}$. Since $\forall_\lambda w$, $\nu'_w = \nu_w$, it is clear
that ${\nu'}^\mathscr{C}$ is also a conditional expectation of $\nu$
with respect to $\mathscr{C}$. 


The uniqueness follows from theorem (\ref{part1:chap3},
\S\ \ref{part1:chap3:sec3}, \ref{part1:chap3:thm36}).  

Thus, we see, if a disintegration of $\lambda$ with respect to
$\mathscr{C}$ exists, how theorem (\ref{part1:chap3},
\S\ \ref{part1:chap3:sec6}, \ref{part1:chap3:thm45}) guarantees 
immediately the existence of conditional expectations with respect to
$\mathscr{C}$ of measure valued functions belonging to $\mathscr{O}$
and how theorem (\ref{part1:chap3}, \S\ \ref{part1:chap3:sec6},
\ref{part1:chap3:thm46}) guarantees the same for measure 
valued functions belonging to $\hat{\mathscr{O}}_\lambda$ if $J$ is
$\sigma$-finite and if $\mathscr{Y}$ has the $J$-countability property
where $J$ is the integral of $\nu$ with respect to $\lambda$. But
theorems (\ref{part1:chap3}, \S\ \ref{part1:chap3:sec5},
\ref{part1:chap3:thm43}) and (\ref{part1:chap3},
\S\ \ref{part1:chap3:sec5}, \ref{part1:chap3:thm44}) give sufficient  
conditions for the existence of disintegration of $\lambda$. Hence, we
have the following existence theorem of conditional expectations for
measure valued functions. 

\begin{thm}\label{part1:chap3:thm48}
(i) Let $\mathscr{C}$ be a complete $\sigma$-algebra contained in
  $\hat{\mathscr{O}}_\lambda$ and le there exist a sequence $(K_n)_{n
    \in \mathbb{N}}$ of compact sets of $\Omega$ and a set $N
  \in\mathscr{O}$ with $\lambda(N) =0$ such that $\Omega =
  \bigcup\limits_{n \in \mathbb{N}} K_n \cup N$ and the restriction of
  $\lambda$ to $K_n \forall \; n \in \mathbb{N}$ is a Radon measure on
  $K_n$.

Or (ii)\pageoriginale Let $\mathscr{C}$ be an arbitrary
$\sigma$-algebra contained in $\hat{\mathscr{O}}_\lambda$ and let
$\Omega$ have the $\lambda$-compacity metrizability property. 
 \end{thm}

Then, under either of the conditions (i) and (ii), if $\nu$ is any
measure valued function on $\Omega$ with values in $\mathfrak{m}^+
(Y, \mathscr{Y})$, $\u{\nu \in \mathscr{O}}$, a conditional
expectation of $\nu$ with respect to $\mathscr{C}$ exists. 

If $\nu \in \hat{\mathscr{O}}_\lambda$, if $J = \int\nu_w d \lambda
(w)$, if $J$ is $\sigma$-finite, and if $\mathscr{Y}$ has the
$J$-countability property, then under either of the conditions (i) and
(ii), a conditional expectation of $\nu$ with respect to $\mathscr{C}$
exists and is unique. 

\section[Conditions for a given family...]{Conditions for a given family $(\lambda^\mathscr{C}_w)_{w \in
  \Omega}$ of positive measures on $\mathscr{O}$, to be a
  disintegration of $\lambda$ with respect to $\mathscr{C}$ and
  consequences}\label{part1:chap3:sec7}

Throughout this section, let $(\Omega, \mathscr{O}, \lambda)$ be a
measure space and let $\mathscr{C}$ be a $\sigma$-algebra contained in
$\hat{\mathscr{O}}_\lambda$. Let $(\lambda^\mathscr{C}_w)_{w \in
  \Omega}$ be a given family of positive measures on $\mathscr{O}$. We
shall discuss below some necessary and sufficient conditions for this
family to be a disintegration of $\lambda$, with respect to
$\mathscr{C}$. 

\begin{proposition}\label{part1:chap3:prop49}
For $(\lambda^\mathscr{C})_{w \in\Omega}$ to be a disintegration of
$\lambda$, it is necessary and sufficient that the following three
conditions are verified.
\begin{itemize}
\item[{\rm (i)}] The measure valued function $\lambda^\mathscr{C} $
  taking $w \in\Omega$ to $\lambda^\mathscr{C}_w$, belongs to
  $\mathscr{C}$ 

\item[{\rm (ii)}] $\lambda = \int \lambda^\mathscr{C}_w
  d\lambda(w)$ \qquad and 

\item[{\rm (iii)}] $\forall \; A \in \mathscr{C}$, $\forall_\lambda
  w$, $\lambda^\mathscr{C}_w$ is carried by $A$ or by $\complement A$
  according as $w \in A$ or $w \in\complement A$. 
\end{itemize}
\end{proposition}

\begin{proof}{\em Necessity.}
Let $(\lambda^\mathscr{C}_w)_{w \in\Omega}$ be a disintegration of
$\lambda$ with respect to $\mathscr{C}$. Then, by the definition of
disintegration, (i) and (ii) are fulfilled. Let us verify (iii). 

Let $A \in\mathscr{C}$. $\lambda^\mathscr{C} (\complement A)$ is a
conditional expectation of $\chi_{\complement A}$ with respect to
$\mathscr{C}$. Therefore,
\begin{align*}
\int\limits_A \chi_{\complement A}(w) d\lambda(w) & = \int\limits_A
\lambda^\mathscr{C}_w (\complement A) d\lambda(w)\\
\text{i.e. }  0 = \lambda (A \cap \complement A) & =
\int\limits_A \lambda^\mathscr{C}_A (\complement A)
d\lambda(w). 
\end{align*}\pageoriginale

Therefore, $\forall_\lambda w$, $\chi_A(w)$. $\lambda^\mathscr{C}_A
(\complement A) = 0$. Similarly, 
$$
\forall_\lambda w, \chi_{\complement A} (w). \lambda^\mathscr{C}_w (A)
=0. 
$$

This proves that $\forall_\lambda w$, $\lambda^\mathscr{C}_w
(\complement A) = 0$ if $w \in A$ and  $\lambda^\mathscr{C}_w(A) =0$,
if $w \in\complement A$. Hence, $\forall_\lambda w$,
$\lambda^\mathscr{C}_w$ is carried by $A$ if $w \in A$ and is carried
by $\complement A$ if $w \in\complement A$. 


\medskip
\noindent{\textbf{Sufficiency.}} We have to prove that $\forall B \in
\mathscr{O}$. $\lambda^\mathscr{C}(B)$ is a conditional expectation of
$\chi_B$ with respect to $\mathscr{C}$. i.e., we have to prove that
$\forall$ $A \in \mathscr{C}$,
$$
\int\limits_A \chi_B d \lambda = \int\limits_A \lambda^\mathscr{C} (B)
d\lambda; 
$$ 
i.e. we have to prove that 
\begin{align*}
\lambda(A \cap B) = \int\limits_A \lambda^\mathscr{C} (B) d\lambda.\\
\int\limits_A \lambda^\mathscr{C}_w (B) d \lambda(w) = \int\limits_A
\lambda^\mathscr{C}_w (B \cap A)  d \lambda (w) +
\int\limits_{\complement A} \lambda^\mathscr{C}_w (B \cap \complement
A) d \lambda (w). 
\end{align*}

Since $\forall_\lambda w$, $w \in A$, $\lambda^\mathscr{C}_w$ is
carried by $A$,
$$
\forall_\lambda w, \; w\in A, \; \lambda^\mathscr{C}_w (B \cap
\complement A) = 0. 
$$

Hence, 
$$
\int\limits_A \lambda^\mathscr{C}_w (B\cap \complement A) d \lambda(w)
= 0. 
$$

Therefore, 
$$
\int\limits_A \lambda^\mathscr{C}_w (B) d \lambda(w) = \int\limits_A
\lambda^\mathscr{C}_w (A \cap B) d\lambda(w). 
$$

Since
\begin{align*}
\lambda& = \int \lambda^\mathscr{C}_w d \lambda(w),\\
\lambda(A \cap B) & = \int \lambda^\mathscr{C}_w (A \cap B)
d\lambda(w)\\
& = \int\limits_A \lambda^\mathscr{C}_w (A \cap B) d \lambda(w) +
\int\limits_{\complement A} \lambda^\mathscr{C}_w (A \cap B)
d\lambda(w). 
\end{align*}

Since\pageoriginale  $\forall_\lambda w$, $\lambda^\mathscr{C}_w$ is
carried by $\complement A$ if $w \in \complement A$,
$$
\int\limits_{\complement A} \lambda^\mathscr{C}_w (
A \cap B) d \lambda(w) = 0. 
$$

Hence 
\begin{align*}
\lambda(A\cap B) &  = \int\limits_A \lambda^\mathscr{C}_w (A \cap B)
d\lambda(w)\\
& = \int\limits_A \lambda^\mathscr{C}_w (B) d\lambda(w) 
\end{align*}
\end{proof}

\begin{defn}\label{part1:chap3:def50}
Let $(X, \mathscr{X})$ be a measurable space. Let $x \in X$. The {\em
$\mathfrak{X}$-atom of $x$} is defined to be the intersection of all
sets belonging to $\mathfrak{X}$ and containing $x$. 
\end{defn}


\begin{defn}\label{part1:chap3:def51}
Let $(X, \mathfrak{X})$ be a measurable space. We say $\mathfrak{X}$
is {\em countably separating} if there exists a sequence $(A_n)_{n \in
\mathbb{N}}$, $A_n \in \mathfrak{X}$ $\forall n \in \mathbb{N}$, such
that $\forall x \in X$, the $\mathfrak{X}$-atom of $x$ is the
intersection of all the $A_n$'s that contain $x$. 
\end{defn}

It can be easily proved that if $\mathfrak{X}$ is countably generated,
then it is countably separating. 

\begin{proposition}\label{part1:chap3:prop52}
Let (i) the measure valued function $\lambda^\mathscr{C}$ on $\Omega$
taking $w$ to $\lambda^\mathscr{C}_w$ belong to $\mathscr{C}$. 
\begin{itemize}
\item[{\rm (i)}] $\lambda = \int \lambda^\mathscr{C}_w d\lambda(w)$ and 

\item[{\rm (ii)}] $\forall_\lambda w$, $\lambda^\mathscr{C}_w$ is
  carried by the $\mathscr{C}$-atom of $w$.
\end{itemize}

Then $(\lambda^\mathscr{C}_w)_{w \in \Omega}$ is a disintegration of
$\lambda$. 
\end{proposition}

\begin{proof}
Let $(\lambda^\mathscr{C}_w)_{w \in\Omega}$ have (i), (ii) and
(iii). From condition (iii), we see by the definition of the
$\mathscr{C}$-atom of $w$, given $A \in \mathscr{C}$, $\forall_\lambda
w$, $\lambda^\mathscr{C}_w$ is carried by $A$ if $w \in A$ and
$\lambda^\mathscr{C}_w$ is carried by $\complement A$ if $w
\in\complement A$. Thus, the condition (iii) of this proposition
implies the condition (iii) of the proposition (\ref{part1:chap3},
\S\ \ref{part1:chap3:sec7}, \ref{part1:chap3:prop49}). Hence, the
conditions (i), (ii) and (iii) of this proposition 
imply the conditions (i), (ii) and (iii) of the proposition (\ref{part1:chap3},
\S\ \ref{part1:chap3:sec7}, \ref{part1:chap3:prop49}). Hence
$(\lambda^\mathscr{C}_w)_{w \in \Omega}$ is a disintegration of
$\lambda$ with respect to $\mathscr{C}$ 
\end{proof}

\begin{proposition}\label{part1:chap3:prop53}
Let\pageoriginale $\mathscr{C}$ be countably separating. Then, the
conditions (i), (ii) and (iii) of the previous proposition (\ref{part1:chap3},
\S\ \ref{part1:chap3:sec7}, \ref{part1:chap3:prop52}) are necessary
for $(\lambda^\mathscr{C}_w)_{w\in\Omega}$ to be a disintegration of
$\lambda$ with respect to $\mathscr{C}$.  
\end{proposition}

\begin{proof}
Let $\mathscr{C}$ be countably separating and let
$(\lambda^\mathscr{C}_w)_{w \in \Omega}$ be a disintegration of
$\lambda$ with respect to $\mathscr{C}$. Then (i) and (ii) are
obvious. We have to only verify that $\forall_\lambda w$,
$\lambda^\mathscr{C}_w$ is carried by the $\mathscr{C}$-atom of $w$. 

Since $\mathscr{C}$ is countably separating, by definition, there
exists a sequence $(A_n)_{n \in \mathbb{N}}$ of sets belonging to
$\mathscr{C}$ such that $\forall w \in\Omega$, the $\mathscr{C}$-atom
of $w$ is the intersection of all the $A_n$'s that contain $w$. 

By condition (iii) of the proposition (\ref{part1:chap3},
\S\ \ref{part1:chap3:sec7}, \ref{part1:chap3:prop49}), $\forall \; n 
\in \mathbb{N}$, $\exists$ a set $N_n \in \mathscr{O}$ with
$\lambda(N_n) =0$ such that if $w \not\in N_n$,
$\lambda^\mathscr{C}_w$ is carried by $A_n$ if $w \in A_n$ and is
carried by $\complement A_n$ if $w \not\in A_n$.


Let $N = \bigcup\limits_{n \in\mathbb{N}} N_n$. Then $N \in
\mathscr{O}$ and $\lambda(N) = 0$. Let $w \not\in N$ and let $A_w$ be
the $\mathscr{C}$-atom of $w$. Since $\mathscr{C}$ is countably
separating, there exists a sequence $(n_k)$ of natural numbers such
that 
$$
A_w = \bigcup\limits_{A_{n_k} \ni w} A_{n_k}, 
$$
$A_{n_k}$ belonging to the sequence $(A_n)_{n \in \mathbb{N}} \forall
n_k$. 

For every $n_k$, since $w \in A_{n_k}$, $\lambda^\mathscr{C}_w$ is
carried by $A_{n_k}$ and hence is carried by $A_w$ since $A_w =
\bigcap\limits_{A_{n_k \ni w}} A_{n_k}$. 

Since $w \not\in N$ is arbitrary, it follows that $\forall_\lambda
w$, $\lambda^\mathscr{C}_w$ is carried by the $\mathscr{C}$-atom of $w$.
\end{proof}

The following {\em counter example} will show that the condition (iii)
of the proposition (\ref{part1:chap3}, \S\ \ref{part1:chap3:sec7},
\ref{part1:chap3:prop52}), namely $\forall_{\lambda} w$, 
$\lambda^\mathscr{C}_w$ is carried by the $\mathscr{C}$ -atom of $w$
is not necessarily true for a disintegration
$(\lambda^\mathscr{C}_w)_{w \in \Omega}$ of $\lambda$ with respect to
$\mathscr{C}$, without further assumptions on $\mathscr{C}$. 

Let\pageoriginale $\Omega$ be the circle $S^1$ in $\mathbb{R}^2$. Let
$\mathscr{O}$ be the Borel $\sigma$-algebra of $\Omega$. Let $\lambda$
be the Lebesgue measure of $S^1$ on $\mathscr{O}$. Let $\mathscr{C}$
be the $\sigma$-algebra of all symmetric Borel sets. Since $\Omega$ is
a compact metric space and $\lambda$ is a Radon probability measure,
by theorem (\ref{part1:chap3}, \S\ \ref{part1:chap3:sec5},
\ref{part1:chap3:thm44}), a disintegration 
$(\lambda^\mathscr{C}_w)_{w \in \Omega}$ of $\lambda$ with respect to
$\mathscr{C}$ exists and is unique. Consider the family
$(\delta^\mathscr{C}_w)_{w \in \Omega}$ of measures given by
$\delta^\mathscr{C}_w = \frac{1}{2} (\delta_w + \delta_{-w})$. It is
easy to check that $(\delta^\mathscr{C}_w)_{w \in \Omega}$ is a
disintegration of $\lambda$ with respect to $\mathscr{C}$. Since the
disintegration is unique, we have $\forall_\lambda w$,
$\lambda^\mathscr{C}_w = \delta^\mathscr{C}_w$, i.e. $\forall_\lambda
w$, $\forall^\mathscr{C}_w = \frac{1}{2} (\delta_w + \delta_{-w})$. 

Let $\hat{\mathscr{C}}$ be the $\sigma$-algebra generated by
$\mathscr{C}$ and all the $\lambda$-null sets of
$\mathscr{O}$. i.e. let $\hat{\mathscr{C}} = \mathscr{C} V
\eta_\lambda$. By theorem (\ref{part1:chap3},
\S\ \ref{part1:chap3:sec5}, \ref{part1:chap3:thm44}), a disintegration 
$(\lambda^{\hat{\mathscr{C}}}_w)_{w \in \Omega}$ of $\lambda$ with
respect to $\hat{\mathscr{C}}$ exists and is unique. It is clear that
$(\lambda^\mathscr{C}_w)_{w \in \Omega}$ is also a disintegration of
$\lambda$ with respect to $\hat{\mathscr{C}}$. Hence, by uniqueness, 
$$
\forall_\lambda w, \lambda^{\hat{\mathscr{C}}}_w =
\lambda^\mathscr{C}_w = \frac{1}{2} (\delta_w + \delta_{-w}). 
$$

If $w \in \Omega$, the $\hat{\mathscr{C}}$-atom of $w$ is the single
point $w$ itself. Since we see that $\forall_\lambda w$,
$\lambda^{\hat{\mathscr{C}}}_w = \frac{1}{2} (\delta_w +
\delta_{-w})$, $\forall_\lambda w$, $\lambda^{\hat{\mathscr{C}}}_w$ is
carried by the pair of points $w$ and $-w$ and hence not by the
$\hat{\mathscr{C}}$-atom of $w$ for these $w$ for which
$\lambda^{\hat{\mathscr{C}}}_w = \frac{1}{2} (\delta_w +
\delta_{-w})$. Hence, the condition (iii) of proposition
(\ref{part1:chap3}, \S\ \ref{part1:chap3:sec7},
\ref{part1:chap3:prop52}) is violated. 

Let us now state and prove some simple, but useful consequences of the
previous propositions. 

Let $(Z, \mathfrak{z})$ be a measurable space such that $\mathfrak{z}$
is countably separating and let $\forall \; z \in \mathbb{Z}$, the
$\mathfrak{z}$-atom of $z$ be $z$ itself. This means that there exists
a sequence $(Z_n)_{n \in \mathbb{N}}$ of sets belonging to
$\mathfrak{z}$ such that every point $z \in \mathbb{Z}$ is the
intersection of a suitable subsequence of these $Z_n$'s. 

\begin{proposition}\label{part1:chap3:prop54}
Let $h$ be a mapping from $\Omega$ to $Z$ such that $h \in
\mathscr{C}$. Let $(\lambda^\mathscr{C}_w)_{w \in \Omega}$ be a
disintegration of $\lambda$ with respect to $\mathscr{C}$. Then, 
$$
\forall_\lambda w, \; \forall_{\lambda^\mathscr{C}_w} w', \; h(w') = h (w).
$$\pageoriginale
i.e. $\forall_\lambda w$, $h$ is $\lambda^\mathscr{C}_w$ almost
everywhere is a constant equal to $h(w)$. 
\end{proposition}

\begin{proof}
Let $w \in \Omega$. There exists a subsequence $(Z_{n_k})_{n_k \in
  \mathbb{N}}$ of $(Z_n)_{n \in \mathbb{N}}$ such that 
$$
h(w) = \bigcap\limits^{\infty}_{n_k =1} Z_{n_k}. 
$$

Hence, 
$$
h^{-1} (h(w)) = \bigcap\limits^{\infty}_{n_k =1} h^{-1}(Z_{n_k}). 
$$
$\forall \; n \in \mathbb{N}$, $h^{-1}(Z_n) \in \mathscr{C}$ since $h
\in \mathscr{C}$. 

Now, by condition (iii) of the proposition (\ref{part1:chap3},
\S\ \ref{part1:chap3:sec7}, \ref{part1:chap3:prop49}), $\forall 
\; n \in \mathbb{N} $, $\forall_\lambda w$, $\lambda^\mathscr{C}_w$ is
carried by $h^{-1}(Z_n)$ if $w \in h^{-1} (Z_n)$. Hence,
$\forall_\lambda w$, $\forall \; n \in \mathbb{N}$,
$\lambda^\mathscr{C}_w$ is carried by $h^{-1}(Z_n)$ if $w \in
h^{-1}(Z_n)$. 

Therefore, $\forall_\lambda w$, $\lambda^\mathscr{C}_w$ is carried by
the intersection of all those $h^{-1}(Z_n)$ for which $w \in
h^{-1}(Z_n)$. But the intersection of all the $h^{-1}(Z_n)$ for which
$w h^{-1} (Z_n)$ is $h^{-1}(h(w))$. 

Hence, $\forall_\lambda w$, $\lambda^\mathscr{C}_w$ is carried by
$h^{-1}(h(w))$. This precisely means that 
$$
\forall_\lambda w, \; \forall_{\lambda^\mathscr{C}_w}  w', \; h(w') =
h(w). 
$$
\end{proof}

\begin{corollary}\label{part1:chap3:coro55}
Let $(h_n)_{n \in\mathbb{N}}$ be a sequence of functions on $\Omega$
with values in $Z$ such that $\forall$ $n$, $h_n \in
\mathscr{C}$. Then, 
$$
\forall_\lambda w, \; \forall_{\lambda^\mathscr{C}_w} w', \forall n\in
\mathbb{N}, h_n(w')= h_n(w). 
$$
\end{corollary}

\begin{proof}
This is an immediate consequence of the above proposition
(\ref{part1:chap3}, \S\ \ref{part1:chap3:sec7}, \ref{part1:chap3:prop54}). 
\end{proof}

\begin{corollary}\label{part1:chap3:coro56}
Let $\mathscr{O}$ be countably generated. Then  $\forall_\lambda w$,
$\forall_{\lambda^\mathscr{C}_w}$ $w'$, $\lambda^\mathscr{C}_{w'} =
\lambda^\mathscr{C}_w$. 
\end{corollary}

\begin{proof}
Let $(B_n)_{n \in \mathbb{N}}$ be a $\pi$-system generating
$\mathscr{O}$. Define a sequence $(h_n)_{n\in \mathbb{N}}$ of
functions on $\mathscr{O}$ as $h_n (w) =
\lambda^\mathscr{C}_w(B_n)\forall w \in \Omega$. 

Then,\pageoriginale $\forall \; n$, $h_n$ is a function on $\Omega$
with values in the extended real numbers and $h_n \in \mathscr{C}$
$\forall$ $ n \in\mathbb{N}$. Hence, by the above Corollary (\ref{part1:chap3},
\S\ \ref{part1:chap3:sec7}, \ref{part1:chap3:coro55}),
$$
\forall_\lambda w, \forall_{\lambda^{\mathscr{C}}_w} w', \; \forall n
\in \mathbb{N} , \; \lambda^{\mathscr{C}}_{w'} (B_n) =
\lambda^\mathscr{C}_w (B_n). 
$$

A standard application of the Monotone class theorem therefore gives, 
$$
\forall_\lambda w, \; \forall_{\lambda^\mathscr{C}_w} w', \; \forall B
\in \mathscr{O}, \; \lambda^\mathscr{C}_{w'} (B) = \lambda^\mathscr{C}_w(B).
$$
Hence,
$$
\forall_\lambda w, \; \forall_{\lambda^\mathscr{C}_w} w', \;
\lambda^\mathscr{C}_{w'} = \lambda^\mathscr{C}_w. 
$$
\end{proof}

