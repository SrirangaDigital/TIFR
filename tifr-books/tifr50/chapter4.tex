
\part[Measure valued Supermartingales and...]{Measure valued
  Supermartingales and Regular Disintegration of 
  measures with respect to a family of $\sigma$-Algebras}\label{part2}

\chapter{Supermartingales}\label{part2:chap4}

\section{Extended real valued Supermartingales}\label{part2:chap4:sec1}

\setcounter{pageoriginal}{68}
Throughout\pageoriginale this chapter, let $(\Omega, \mathscr{O},
\lambda)$ be a measure space. Let $(\mathscr{C}^t)_{t \in \mathbb{R}}$
be a family of sub $\sigma$-algebras of $\hat{\mathscr{O}}_\lambda$,
which is {\em increasing} in the sense that $\forall \; s$, $t \in
\mathbb{R}$, $s \leq t$, $\mathscr{C}^s \subset \mathscr{C}^t$. Let us
further assume that $\lambda$ restricted to $\mathscr{C}^t$ is
$\sigma$-finite $\forall t \in \mathbb{R}$. It $\mathscr{C}$ is a
$\sigma$-algebra contained in $\hat{\mathscr{O}}_\lambda$ and $f$ is a
function on $\Omega$ with values in $\bar{\mathbb{R}}^+$, belonging to
$\hat{\mathscr{O}}_\lambda$, $f^\mathscr{C}$ will stand for a
conditional expectation of $f$ with respect to $\mathscr{C}$. 

If $f$ is a function from $\Omega \times \mathbb{R}$ to
$\bar{\mathbb{R}}^+$, $\u{f}^t$ will denote $\forall$ $t \in
\mathbb{R}$, the function on $\Omega$ with values in
$\bar{\mathbb{R}}^+$, taking $w \in \Omega$ to $f (w,t)$. $\forall w
\in \Omega$, $\u{f_w}$ will denote the function on $\mathbb{R}$ to
$\bar{\mathbb{R}}^+$, taking $t$ to $f(w,t)$. 

Let $f$ be a function from $\Omega \times \mathbb{R}$ to
$\bar{\mathbb{R}}^+$. We have the following series of definitions. 


\begin{defn}\label{part2:chap4:def57}
$f$ is said to be {\em adapted} to $(\mathscr{C}^t)_{t \in\mathbb{R}}$
  if $\forall \; t$, $f^t \in \mathscr{C}^t$.
\end{defn}

\begin{defn}\label{part2:chap4:def58}
$f$ is said to be {\em right continuous} if $\forall_\lambda w$, $f_w$
  is a right continuous function. 
\end{defn}

\begin{defn}\label{part2:chap4:def59}
$f$ is said to be {\em regulated} if $\forall_\lambda w$, $f_w$ has
  finite right and finite left limits at all points $t \in
  \mathbb{R}$. 
\end{defn}

\begin{defn}\label{part2:chap4:def60}
A function $g$ from $\Omega \times \mathbb{R}$ to $\bar{\mathbb{R}}^+$
is said to be a {\em modification} or a {\em version} of $f$ if
$\forall \; t \in \mathbb{R}$, $\forall_\lambda w$, $g (w,t) =
f(w,t)$. 
\end{defn}

\begin{defn}\label{part2:chap4:def61}
$f$ is said to be a {\em supermartingale} (resp. {\em martingale})
  adapted to $(\mathscr{C}^t)_{t \in \mathbb{R}}$ if 
\begin{itemize}
\item[{\rm (i)}] $f$ is\pageoriginale adapted to $(\mathscr{C}^t)_{t \in
  \mathbb{R}}$  and 

\item[{\rm (ii)}] $\forall \; s$, $t \in \mathbb{R}$, $s \leq t$,
  $\forall_\lambda w$, $(f^t)^{\mathscr{C}^s} (w) \leq f^s(w)$. 
\end{itemize}
(resp.
\begin{itemize}
\item[{\rm (i)}] $f$ is adapted to $(\mathscr{C}^t)_{t \in
  \mathbb{R}}$ and 

\item[{\rm (ii)}] $\forall \; s$, $t \in \mathbb{R}$, $s \leq t$,
  $\forall_\lambda w$, $(f^t)^{\mathscr{C}^s} (w) = f^s(w)$)
\end{itemize}

Note that every martingale is a supermartingale. A simple example of a
supermartingale adapted to $(\mathscr{C}^t)_{t\in \mathbb{R}}$ is
given by any function $f$ from $\Omega \times \mathbb{R}$ to
$\bar{\mathbb{R}}^+$, adapted to $(\mathscr{C}^t)_{t \in \mathbb{R}}$
for which $\forall_\lambda \; w$, $f_w$ is a decreasing function of
$t$. 
\end{defn}


Let $g$ be any function on $\Omega$ with values in
$\bar{\mathbb{R}}^+$ and belonging to $\hat{\mathscr{O}}_\lambda$. Let
$\forall \; t$, $w \to f(w,t)$ be a given conditional expectation of
$g$ with respect to $\mathscr{C}^t$. Then, it is clear that the
function $f$ on $\Omega \times \mathbb{R}$ to $\bar{\mathbb{R}}^+$,
taking $(w,t)$ to $f(w,t)$ is a martingale, adapted to
$(\mathscr{C}^t)_{t \in \mathbb{R}}$. 

If $f$ is a supermartingale adapted to $(\mathscr{C}^t)_{t \in
  \mathbb{R}}$, note that $t \to J^t = \int f^t d \lambda$ is a
decreasing function of $t$. If $f$ is a right continuous
supermartingale, we easily see by applying Fatou's lemma that $t \to
J^t$ is right continuous. 

\begin{defn}\label{part2:chap4:def62}
A supermartingale adapted to $(\mathscr{C}^t)_{t\in \mathbb{R}}$, is
said to be {\em regular} if it is right continuous and regulated. 
\end{defn}

Note that if $f$ is a regular supermartingale adapted to
$(\mathscr{C}^t)_{t \in \mathbb{R}}$, then $\forall_\lambda w$, $f_w$
is a real valued function on $\mathbb{R}$. 

\begin{defn}\label{part2:chap4:def63}
A supermartingale $g$ adapted to $(\mathscr{C}^t)_{t \in \mathbb{R}}$
is said to be a {\em regular modification} of a supermartingale $f$
adapted to $(\mathscr{C}^t)_{t \in \mathbb{R}}$ if $g$ is regular and
is a modification of $f$. 
\end{defn}

\begin{rem}\label{part2:chap4:rem64}
If $f$ is a supermartingale adapted to $(\mathscr{C}^t)_{t \in
  \mathbb{R}}$ and if $g_1$ and $g_2$ are two regular modifications of
$f$, we easily see that $\forall_\lambda w$, $\forall t$, $g_1 (w,t) =
g_2 (w,t)$. In this sense, we say that a regular modification of a
supermartingale adapted to $(\mathscr{C}^t)_{t \in \mathbb{R}}$ is
unique if exists. In particular, if $f$ is a regular supermartingale
adapted to $(\mathscr{C}^t)_{t \in \mathbb{R}}$\pageoriginale and if
$g$ is a regular modification of $f$, then $\forall_\lambda w$,
$\forall t$, $g(w,t) = f(w,t)$. 
\end{rem}

The following theorem is very fundamental in the theory of
supermartingales. It guarantees the existence of a regular
modification for a  supermartingale, under some conditions. Whenever,
we refer to the following theorem, we refer to it as `the fundamental
theorem'.

\medskip
\noindent{\textbf{The Fundamental theorem.}} Let $f$ be a
supermartingale adapted to $(\mathscr{C}^t)_{t \in \mathbb{R}}$ such
that $\forall$ $t \in \mathbb{R}$, $J^t < + \infty$ and $t \to J^t$
right continuous. Let further the family $(\mathscr{C}^t)_{t \in
  \mathbb{R}}$ of $\sigma$-algebras be {\em right continuous} in the
sense that $\forall t \in \mathbb{R}$, $\mathscr{C}^t =
\bigcap\limits_{u > t} \mathscr{C}^u$. Then, there exists  a function
$g$ on $\Omega \times \mathbb{R}$ with values in $\mathbb{R}^+$ such
that $g$ is a regular modification of $f$. 

This theorem is proved in P.A. Meyer \cite{key1} in theorems $T4$ and
$T3$ of Chap. VI, in pages 95 and 94. In that book, the
supermartingales are assumed to take only real values. Hence to deduce
the fundamental theorem from the theorems $T4$ and $T3$ mentioned
above, we observe the following. 

Since $\forall \; t$, $J^t < + \infty$, we have $\forall \; t$,
$\forall_\lambda w$, $f(w,t) < + \infty$. Define a function $h$ on
$\Omega \times \mathbb{R}$ with values in $\mathbb{R}^+$ as follows:
$$
h(w,t) = 
\begin{cases}
f(w,t) & \text{ if } f(w,t) < + \infty.\\
0 & \text{ otherwise. }
\end{cases}
$$

Then, $h$ is a modification of $f$ with values in
$\mathbb{R}^+$. Hence by applying the theorems $T4$ and $T3$ of
Chap. VI of P.A. Meyer \cite{key1}, we get a regular modification for
$h$ and it is also a regular modification of $f$ as well. 

\begin{rem}\label{part2:chap4:rem65}
Let $(\mathscr{C}^t)_{t\in \mathbb{R}}$ be an increasing right
continuous family of $\sigma$-algebras contained in
$\hat{\mathscr{O}}_\lambda$. Let $f$ be a right continuous
supermartingale adapted to $(\mathscr{C}^t)_{t \in \mathbb{R}}$, such
that $\forall \; t \in \mathbb{R}$, $J^t = \int f(w,t) d \lambda (w) <
+ \infty$. Then $f$ is regular. 
\end{rem}

For, by the fundamental theorem, there exists a regular modification
$g$ of $f$ since $J^t < + \infty$ $\forall \; t \in \mathbb{R}$ and $t
\to J^t$ is right continuous by Fatou's lemma. 
\begin{align*}
\forall t, \; \forall_\lambda w, \; g(w,t) = f(w,t)\\
\forall_\lambda w, \; \forall t \in \mathbb{Q}, g(w,t) = f(w,t). 
\end{align*}\pageoriginale

Because $\forall_\lambda w$, $g_w$ and $f_w$ are right continuous, it
follows that $\forall_\lambda w$, $\forall t \in \mathbb{R}$, $g (w,
t) = f(w,t)$. Since $\forall_\lambda w$, $g_w$ is regular, it follows
that $\forall_\lambda w$, $f_w$ is also regular. 

The following procedure is standard and gives a method of finding out
the regular modification of a supermartingale, whenever it exists.

Let $\left( \dfrac{k}{2^n}\right)_{k \in \mathbb{Z}, n \in
  \mathbb{N}}$ be the set of all dyadic rationals. Then, $\forall n
\in \mathbb{N}$, $\mathbb{R} = \bigcup\limits_{k \in \mathbb{Z}}
(\dfrac{k}{2^n}, \dfrac{k+1}{2^n}]$. $\forall n \in \mathbb{N}$,
  define the functions $\tau_n$ from $\mathbb{R}$ to $\mathbb{R}$ as
  follows:

If $t \in\mathbb{R}$, $\tau_n (t) = \dfrac{k+1}{2^n}$ if $t \in
(\dfrac{k}{2^n}, \dfrac{k+1}{2^n}]$. 


Then $\forall t \in \mathbb{R}$, $\tau_n (t) \downarrow t$ as $n \to
\infty$ and if $t$ is a dyadic rational, $\tau_n(t) =t$ for all $n$
large enough. 

Let $f$ be a function on $\Omega \times \mathbb{R}$ with values in
$\bar{\mathbb{R}}^+$. Define another function $\bar{f}$ on $\Omega
\times \mathbb{R}$ with values in $\bar{\mathbb{R}}^+$ as follows:
$$
\bar{f} (w,t) = 
\begin{cases}
\lim\limits_{n\to \infty} f^{\tau_n(t)} (w), & \text{ if the limit
  exists and is finite}.\\
0, & \text{ otherwise.}
\end{cases}
$$

It is obvious that if $t$ is a dyadic rational and if $f(w,t) < +
\infty$, then $\bar{f}(w,t) = f(w,t)$. 

\begin{proposition}\label{part2:chap4:prop66}
Let $f$ be a function on $\Omega \times \mathbb{R}$ with values in
$\bar{\mathbb{R}}^+$. Then 
\begin{itemize}
\item[{\rm (i)}] If the family $(\mathscr{C}^t)_{t\in \mathbb{R}}$ is
  right continuous, and if $f$ is adapted to $(\mathscr{C}^t)_{t\in
    \mathbb{R}}$, so is $\bar{f}$.

\item[{\rm (ii)}] If $f$ is right continuous, and if $\forall_\lambda
  w$, $\forall t$, $f(w,t) < + \infty$, then $\forall_\lambda w$,
  $\forall t$, $\bar{f} (w,t) = f(w,t)$.
\end{itemize}
\end{proposition}

\begin{proof}
\begin{itemize}
\item[{\rm (i)}] Since $f$ is adapted to
  $(\mathscr{C}^t)_{t\in\mathbb{R}}$ and $(\mathscr{C}^t)_{t \in
  \mathbb{R}}$ is right continuous, we can easily see that $\forall \;
  t$, the set \{$w : \lim\limits_{n \to \infty} f^{\tau_n(t)} (w)$
  exists and is finite \} belongs to $\tau^t$. 
From this,\pageoriginale it follows that $\forall
  t$, ${\bar{f}}^t \in \mathscr{C}$ and this means that $\bar{f}$ is
  adapted to $(\mathscr{C}^t)_{t \in \mathbb{R}}$.

\item[{\rm (ii)}] Since $f$ is right continuous, there exists a set
  $N_1 \in \mathscr{O}$ with $\lambda(N_1) = 0$ such that if $w
  \not\in N_1$, $f_w$ is a right continuous function on
  $\mathbb{R}$. Let $N_2 \in \mathscr{O}$, $\lambda(N_2) =0 $ be such
  that if $w \not\in N_2$, $f(w,t) < + \infty$ for all  $t \in
  \mathbb{R}$. Then if $w \not\in N_1 \cup N_2$, for all $t$,
  $\lim\limits_{n \to \infty}$ $f^{\tau_n(t)}(w)$ exists and is equal
  to $f(w,t)$ and hence is finite. Thus, if $w \not\in N_1 \cup N_2$,
  $\bar{f} (w,t) = f(w,t) \forall t\in\mathbb{R}$. 
\end{itemize}

Hence, $\forall_\lambda w$, $\forall t \in\mathbb{R}$, $\bar{f} (w,t)
= f(w,t)$. 
\end{proof}

We have the following obvious corollary.

\begin{corollary}\label{part2:chap4:coro67}
Let $(\mathscr{C}^t)_{t\in \mathbb{R}}$ be a right continuous
increasing family of sub $\sigma$-algebras of
$\hat{\mathscr{O}}_\lambda$. Let $f$ be a supermartingale adapted to
$(\mathscr{C}^t)_{t \in \mathbb{R}}$ and let $g$ be a regular
modification of $f$. Then, $\forall_\lambda w$, $\forall t $, $g (w,t)
= \bar{f} (w, mt)$. In particular, if $f$ is a supermartingale adapted
to $(\mathscr{C}^t)_{t \in \mathbb{R}}$ with $J^t < + \infty$ and $t
\to J^t$ right continuous, then $\forall_\lambda w$, $\forall t$,
$\lim\limits_{n \to \infty} f(w, \tau_n(t))$ exists and is finite and
$\forall \; t$, $\forall_\lambda w$, $f(w,t) =
\bar{f}(w,t)$. Moreover, $\bar{f}$ is a regular modification of $f$. 
\end{corollary}

In the course of proofs of several theorems in this book, the
following important theorem will be needed. We call it the ``{\em
  Upper envelope theorem}'' and whenever we refer to it, we will
refer to it as the ``Upper envelope theorem''. We state this without
proof. 

Let $(X, \mathfrak{X}, \mu)$ be a measure space. Let
$(\mathfrak{X}^t)_{t \in \mathbb{R}}$ be an increasing right
continuous family of $\sigma$-algebras contained in
$\hat{\mathfrak{X}}_\mu$. Let $(f_n)_{n \in \mathbb{N}}$ be an
increasing sequence of regular supermartingales in the sense that
$\forall \; n \in \mathbb{N}$, $f_n$ is a regular super martingale
adapted to $(\mathfrak{X}^t)_{t \in \mathbb{R}}$ such that
$\forall_\mu x$, $\forall t$, $f_n(x,t) \leq f_{n+1} (x,t)$ for all $n
\in \mathbb{N}$. Let $f(x,t) = \sup\limits_{n} f_n(x,t)$. Then $f$ is
a right continuous supermartingale and $\forall_\mu x$, limits from
the left exist at all points.

For a proof see P.A. Meyer \cite{key1}. T16, p.99. 

\section{Measure valued Supermartingales}\label{part2:chap4:sec2}

Let\pageoriginale $(Y,\mathscr{Y})$ be a measurable space. Let $\nu$
be a measure valued function on $\Omega \times \mathbb{R}$ with values
in $\mathfrak{m}^+ (Y, \mathscr{Y})$, taking a point $(w,t)$ of
$\Omega \times \mathbb{R}$ to the measure $\nu^t_w$ on
$\mathscr{Y}$. Let $\forall \; t \in \mathbb{R}$, $\nu^t$ be the
measure valued function on $\Omega$ with values in $\mathfrak{m}^+ (Y,
\mathscr{Y})$ taking $w \in \Omega$ to $\nu^t_w$ and $\forall $ $w
\in\Omega$, let $\nu_w$ be the measure valued function on $\mathbb{R}$
taking $t \in\mathbb{R} $ to the measure $\nu^t_w$. If $f$ is a
function on $Y$, $f\geq 0$, $f \in\mathscr{Y}$, let $\nu(f)$ be the
function on $\Omega \times \mathbb{R}$ with values in
$\bar{\mathbb{R}}^+$ taking $(w,t)$ to $\nu^t_w (f)$, $\forall w
\in\Omega$, let $\nu_w(f)$ be the function on $\mathbb{R}$ taking $t$
to $\nu^t_w (f)$ and $\forall \; t \in \mathbb{R}$, let $\nu^t(f)$ be
the function on $\Omega$ taking $w \in \Omega$ to $\nu^t_w(f)$. 

\begin{defn}\label{part2:chap4:def68}
$\nu$ is said to be a {\em measure valued supermartingale} (resp. {\em
  measure valued martingale}) adapted to $(\mathscr{C}^t)_{t \in
    \mathbb{R}}$ if 
\begin{itemize}
\item[{\rm (i)}] $\forall \; t$, $\nu^t \in \mathscr{C}^t$

\item[{\rm (ii)}] $\forall \; s$, $t \in \mathbb{R}$, $s \leq t$ and
  $\forall$ $A \in \mathscr{C}^s$ and 
$$
\int\limits_A \nu^t_w d \lambda (w) \leq \int\limits_A \nu^s_w d \lambda(w)
$$
\end{itemize}
(resp.\begin{itemize}
\item[{\rm (i)}] $\forall \; t$, $\nu^t \in \mathscr{C}^t$

\item[{\rm (ii)}] $\forall \; s$, $t \in\mathbb{R}$, $s \leq t$ and
  $\forall$ $A \in \mathscr{C}^s$,
$$
\int\limits_A \nu^t_w d\lambda(w) = \int\limits_A \nu^s_w d\lambda(w)
\; )
$$
\end{itemize}
in the sense that $\forall$ function $f$ on $Y$, $f \in \mathscr{Y}$,
$f \geq 0$,
\begin{align*}
\int\limits_A \nu^t_w (f) d\lambda (w) & \int\limits_A \nu^s_w (f) d
\lambda(w)\\
(\text{resp. } \int\limits_A \nu^t_w (f) d\lambda(w) & = \int\limits_A
\nu^s_w (f) d\lambda (w). \; )
\end{align*}
\end{defn}

It is clear that $\nu$ is a measure valued supermartingale (resp.
measure valued martingale) adapted to
$(\mathscr{C}^t)_{t\in\mathbb{R}}$ if and only if $\forall$ function
$f$ on $Y$, $f \in \mathscr{Y}$, $f \geq 0$, $\nu(f)$ is and extended
real valued supermartingale (resp. martingale) adapted to
$(\mathscr{C}^t)_{t \in \mathbb{R}}$. 

\begin{defn}\label{part2:chap4:def69}
$\nu$ is\pageoriginale said to be a {\em regular} measure valued supermartingale
  adapted to $(\mathscr{C}^t)_{t \in \mathbb{R}}$ if $\forall$ $B \in
  \mathscr{Y}$, $\nu(\chi_B)$ is a regular supermartingale adapted to
  $(\mathscr{C}^t)_{t \in \mathbb{R}}$ . 
 \end{defn}

\begin{defn}\label{part2:chap4:def70}
If $\nu$ and $\mu$ are two measure valued supermartingales, adapted to
$(\mathscr{C}^t)_{t \in \mathbb{R}}$, with values in $\mathfrak{m}^+
(Y, \mathscr{Y})$, $\mu$ is said to be a {\em modification} of $\nu$
if $\forall t$, $\forall_\lambda w$, $\mu^t = \nu^t$. 
\end{defn}

\begin{defn}\label{part2:chap4:def71}
If $\nu$ and $\mu$ are two measure valued supermartingales with values
in $\mathfrak{m}^+ (Y, \mathscr{Y})$ and adapted to
$(\mathscr{C}^t)_{t \in \mathbb{R}}$, $\mu$ is said to be a {\em
  regular modification} of $\nu$ if $\mu$ is regular and is a
modification of $\nu$. 
\end{defn}

Let $\nu$ be a measure valued supermartingale adapted to
$(\mathscr{C}^t)_{t \in \mathbb{R}}$, with values in $\mathfrak{m}^+
(Y, \mathscr{Y})$. Let $\forall$ $t \in\mathbb{R}$, $J^t = \int
\nu^t_w d\lambda(w)$. Then, $\forall$ $t$, $J^t$ is a positive measure
on $\mathscr{Y}$. Since $\nu$ is a measure valued supermartingale,
$\forall \; s$, $t \in \mathbb{R}$, $s \leq t$, we have $J^t \leq J^s$
in the sense that $\forall $ $B \in \mathscr{Y}$, $J^t(B) \leq
J^s(B)$. In this sense we say that $t \to J^t$ is decreasing. Let
$\sup\limits_{t \in \mathbb{R}} J^t$ be the set function defined on
$\mathscr{Y}$ as, $\forall$ $B \in \mathscr{Y}$, $(\sup\limits_{t \in
  \mathbb{R}} J^t) (B) = \sup\limits_{t \in \mathbb{R}} J^t (B)$. With
a similar definition for the set functions, $\sup\limits_{t \in
  \mathbb{Z}} J^t$, $\sup\limits_{\substack{t \in \mathbb{Z}\\t \leq
    0}} J^t$, we have $\sup\limits_{t \in \mathbb{R}} J^t =
\sup\limits_{t \in \mathbb{Z}} J^t = \sup\limits_{\substack{t \in
    \mathbb{Z}\\t \leq 0}}$ since $t \to J^t$ is
decreasing. $\sup\limits_{\substack{t \in \mathbb{Z}\\ t \leq 0}} J^t$
is a measure on $\mathscr{Y}$, since it is the supremum of an
increasing sequence of measures on $\mathscr{Y}$. Let us denote this
measure by J. Thus,
$$
J = \sup\limits_{t \in\mathbb{R}} J^t = \sup\limits_{t \in \mathbb{Z}}
J^t = \sup\limits_{\substack{t \in \mathbb{Z}\\t \leq 0}} J^t. 
$$

\section[Properties of regular measure valued
  supermartingales]{Properties of regular measure valued\hfil\break
  supermartingales}\label{part2:chap4:sec3} 

In this section, let $\nu$ be a regular measure valued supermartingale
with values in $\mathfrak{m}^+ (Y, \mathscr{Y})$, adapted to an
increasing right continuous family $(\mathscr{C}^t)_{t \in
  \mathbb{R}}$ of $\sigma$-algebras contained in
$\hat{\mathscr{O}}_\lambda$. Let $J^t = \int \nu^t_w d\lambda(w)$ and
$J = \sup\limits_{t \in \mathbb{R}} J^t$.

\begin{proposition}\label{part2:chap4:prop72}
Let $f$\pageoriginale be a function on $Y$, $f \geq 0$, $f \in \mathscr{Y}$ such
that $f$ is $J$-integrable. Then, $\nu(f)$ is a regular supermartingale
with values in $\bar{\mathbb{R}}^+$, adapted to $(\mathscr{C}^t)_{t\in
\mathbb{R}}$ and $\forall_\lambda w$, $\forall t$, $f$ is
$\nu^t_w$-integrable. 
\end{proposition}

\begin{proof}
We know that $\forall$ $B \in \mathscr{Y}$, $\nu(\chi_B)$ is a regular
supermartingale, with values in $\bar{\mathbb{R}}^+$ adapted to
$(\mathscr{C}^t)_{t \in \mathbb{R}}$. Hence if $s$ is a step function
on $Y$, $s \geq 0$, $s \in \mathscr{Y}$, $\nu(s)$ is again a regular
supermartingale with values in $\bar{\mathbb{R}}^+$ and adapted to
$(\mathscr{C}^t)_{t \in \mathbb{R}}$. Since $f$ is $J$-integrable, $f
\in \mathscr{Y}$ and $f \geq 0$, there exists an increasing sequence
$(s_n)_{n \in \mathbb{N}}$ of step functions such that $\forall$ $n
\in \mathbb{N}$, $0\leq s_n \leq f$, $s_n \in \mathscr{Y}$ and $s_n(y)
\uparrow f(y)$ for all $y \in Y$. 

Hence $\forall$ $t \in \mathbb{R}$, $\forall $ $w \in \Omega$,
$\nu^t_w (s_n) \uparrow \nu^t_w(f)$. Hence $(\nu(s_n))_{n \in
  \mathbb{N}}$ is an increasing sequence of supermartingales with
values in $\bar{\mathbb{R}}^+$, adapted to $(\mathscr{C}^t)_{t \in
  \mathbb{R}}$, with limit as $\nu(f)$. Hence, by the `Upper envelope
theorem'. $\nu(f)$ is a right continuous supermartingale. $\forall$
$t$, $\int \nu^t_w(f) d\lambda(w)  = J^t (f) \leq J(f) <+
\infty$. Hence by remark (\ref{part2:chap4},
\S\ \ref{part2:chap4:sec1}, \ref{part2:chap4:rem65}), $\nu(f)$ is a regular 
supermartingale. Hence $\forall_\lambda w$, $\forall t$, $\nu^t_w(f) <
+ \infty$ and this proves that $\forall_\lambda w$, $\forall t$, $f$
is $\nu^t_w$-integrable. 
\end{proof}

\begin{corollary}\label{part2:chap4:coro73}
Let $f$ be a function on $Y$, with values in $\bar{\mathbb{R}}$, $f
\in \mathscr{Y}$ and $J$-integrable. Then, $\forall_\lambda w$,
$\forall t$, $f$ is $\nu^t_w$-integrable and $\forall_\lambda w$, the
function $\nu_w(f)$ on $\mathbb{R}$ taking $t$ to $\nu^t_w(f)$ is
regulated and right continuous. 
\end{corollary}

\begin{proof}
This follows immediately from the above proposition
(\ref{part2:chap4}, \S\ \ref{part2:chap4:sec3},
\ref{part2:chap4:prop72}) by 
writing $f$ as $f^+ - f^-$ with the usual notation. 
\end{proof}

\begin{proposition}\label{part2:chap4:prop74}
Let $E$ be a Banach space over $\mathbb{R}$. Let $g$ be a step
function on $Y$ with values in $E$, $g \in \mathscr{Y}$ and
$J$-integrable. Then, $\forall_\lambda w$, $\forall t $, $g$ is
$\nu^t_w$-integrable and $\forall_\lambda w$, $\nu_w(g)$, the function
on $\mathbb{R}$ taking $t$ to $\nu^t_w(g)$, is a right continuous and
regulated function on $\mathbb{R}$ with values in $E$. ($A$ Banach
space valued function $f$ on $\mathbb{R}$ is said to be {\em
  regulated} if $\forall $ $t \in\mathbb{R}$,
$\lim\limits_{\substack{s \to t\\s >t}} f(s)$ and
$\lim\limits_{\substack{s\to t\\s <t}}$ f(s) exist in the Banach
space). 
\end{proposition}

\begin{proof}
Let\pageoriginale $g = \sum\limits^n_{i =1} \chi_{A_i} x_i$ where
$\forall$ $i$, $i = 1, \ldots n$, $x_i \in E$, $A_i \in \mathscr{Y}$
and $A_i \cap A_j = \emptyset$ if $i \neq j$. Since $g$ is
$J$-integrable, $\forall$ $i = 1, \ldots n$, $J(A_i) < +
\infty$. Hence, by proposition (\ref{part2:chap4},
\S\ \ref{part2:chap4:sec3}, \ref{part2:chap4:prop72}), $\forall i = 1,
\ldots 
n$, $\nu(A_i)$ is a regular supermartingale and hence $\forall_\lambda
w$, $\forall t$, $i = 1, \ldots, n$, $\nu^t_w(A_i) < + \infty$ and
$\forall_\lambda w$, $\forall i = 1, \ldots n$, $\nu_w(A_i)$ is a
right continuous regulated function on $\mathbb{R}$. 

Hence $\forall_\lambda w$, $\forall t$, $g$ is $\nu^t_w$-integrable
and $\nu^t_w(g) = \sum\limits^n_{i=1} \nu^t_w(A_i)x_i$ for these $w$
and for all $t$ for which $g$ is $\nu^t_w$-integrable.

Thus, $\forall_\lambda w$, $\nu_w(g) = \sum \nu_w(A_i) x_i$ on
$\mathbb{R}$. Hence $\forall_\lambda w$, $\nu_w(g)$ is a right
continuous regulated function on $\mathbb{R}$. 
\end{proof}

\begin{proposition}\label{part2:chap4:prop75}
Let $E$ be a Banach space over $\mathbb{R}$ and let $f$ be a function
on $Y$ with values in $E$, $f \in \mathscr{Y}$ and
$J$-integrable. Then, $\forall_\lambda w$, $\forall t$, $f$ is
$\nu^t_w$-integrable and $\forall_\lambda w$, $\nu_w(f)$ is a right
continuous regulated function on $\mathbb{R}$ with values in $E$. 
\end{proposition}

\begin{proof}
Since $f$ is $J$-integrable, and $\in \mathscr{Y}$, $\forall n$,
$\exists$ a step function $g_n$ on $Y$, $g_n \in \mathscr{Y}$ such
that $\int |g_n -f| d J \leq \dfrac{1}{2^{2n}}$.

Let $h = \sum\limits^\infty_{n=1} 2^n |g_n -f|$. 

Then $h$ is $\geq 0$ and is $J$-integrable on $Y$. 
$$
\forall \;n \in \mathbb{N}, |f| \leq |g_n -f| + |g_n| \leq
\frac{h}{2^n} + |g_n|
$$
By proposition (\ref{part2:chap4}, \S\ \ref{part2:chap4:sec3},
\ref{part2:chap4:prop72}), $\forall_\lambda w$, $\forall t$, $h$ 
is $\nu^t_w$-integrable and by proposition (\ref{part2:chap4},
\S\ \ref{part2:chap4:sec3}, \ref{part2:chap4:prop74}) 
$\forall_\lambda w$, $\forall \; n \in\mathbb{R}$, $g_n$ is
$\nu^t_w$-integrable. Hence, $\forall_\lambda w$, $\forall t$, $f$ is
$\nu^t_w$-integrable. 
$$
|g_n -f| \leq \frac{1}{2^n} h. 
$$
Hence $\forall_\lambda w$, $\forall t$, $|\nu^t_w (g_n) - \nu^t_w(f)|
\leq \dfrac{1}{2^n} \nu^t_w(h)$. Since $h$ is $\geq 0$ and
$J$-integrable, by proposition (\ref{part2:chap4},
\S\ \ref{part2:chap4:sec3}, \ref{part2:chap4:prop72}),
$\forall_\lambda w$, 
$\nu_w(h)$ is a right continuous regulated function on
$\mathbb{R}$. Hence\pageoriginale $\forall_\lambda w$, $\nu_w(h)$ is locally bounded
on $\mathbb{R}$. Hence $\forall_\lambda w$, $\nu^t_w(g_n)$  converges
to $\nu^t_w(f)$ in $E$, as $n \to \infty$, locally uniformly in the
variable $t$. 

By the previous proposition (\ref{part2:chap4},
\S\ \ref{part2:chap4:sec3}, \ref{part2:chap4:prop74}),
$\forall_\lambda w$, 
$\forall n \in \mathbb{N}$, $\nu_w (g_n)$ is right continuous and
regulated. Hence, since $E$ is complete, and  since $\forall_\lambda
w$, the convergence of $\nu^t_w(g_n)$ to $\nu^t_w(f)$ is locally
uniform in $t$, it follows that $\nu_w(f)$ is also right continuous
and regulated. 
\end{proof}


