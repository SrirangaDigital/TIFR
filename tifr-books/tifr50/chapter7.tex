
\chapter{Strong $\sigma$-Algebras}\label{part2:chap7}

\section{A few propositions}\label{part2:chap7:sec1}

We\pageoriginale shall prove in this section two propositions which
will be used later on.

\begin{proposition}\label{part2:chap7:prop95}
Let $(\Omega, \mathscr{O}, \lambda)$ be a measure space and
$\mathscr{C}$, a $\sigma$-algebra contained in
$\hat{\mathscr{O}}_\lambda$. Let $\lambda$ restricted to $\mathscr{C}$
be $\sigma$-finite. Then, the following are equivalent:
\begin{itemize}
\item[{\rm (i)}] $\lambda$ is disintegrated with respect to
  $\mathscr{C}$  by the constant measure valued function $w \to
  \lambda^\mathscr{C}_w = \lambda$. 

\item[{\rm (ii)}] $\lambda$ is ergodic on $\mathscr{C}$ i.e. $\forall
  A \in \mathscr{C}$, $\lambda(A) =0$ or $1$.
\end{itemize}
\end{proposition}

\begin{proof}
Let us assume (i) and prove (ii). 

By the definition of a measure space, $\lambda \not\equiv 0$. (i)
implies that $\lambda$ is a probability measure. By proposition (\ref{part1:chap3},
\S\ \ref{part1:chap3:sec7}, \ref{part1:chap3:prop49}), for any
disintegration $(\lambda^\mathscr{C}_w)_{w \in 
  \Omega}$ of $\lambda$ with respect to $\mathscr{C}$, given $A \in
\mathscr{C}$, $\forall_\lambda w$, $\lambda^\mathscr{C}_w$ is carried
by $A$ or by $\complement A$, according as $w \in A$ or $w \in
\complement A$. Therefore, given $A \in \mathscr{C}$, either
$\lambda(A) = 0$ or if $\lambda(A) >0$, there exists a $w \in A$ such
that $\lambda^\mathscr{C}_w$ is carried by $A$. Applying this to our
special case where $\lambda^\mathscr{C}_w = \lambda \;\; \forall w \in
\Omega$, we see that if $\lambda(A) >0$, $\lambda$ is carried by $A$
and hence $\lambda(A) =1$, since $\lambda$ is a probability
measure. Thus, $\lambda$ is ergodic on $\mathscr{C}$.

Let us now assume (ii) and prove (i). (ii) implies that $\lambda$ is a
probability measure.

To verify that $\lambda$ is disintegrated with respect to
$\mathscr{C}$ by the constant measure valued function $w \to
\lambda^\mathscr{C}_w = \lambda$, according to proposition (\ref{part1:chap3},
\S\ \ref{part1:chap3:sec7}, \ref{part1:chap3:prop49}), it
is\pageoriginale sufficient to verify that 
\begin{itemize}
\item[{\rm (1)}] for any $A \in \mathscr{O}$, $w \to \lambda(A)$
  belongs to $\mathscr{C}$. 

\item[{\rm (2)}] $\lambda = \int \lambda d \lambda(w)$

\item[{\rm (3)}] Given $A \in \mathscr{C}$, $\lambda$ is carried by
  $A$ or by $\complement A$.
\end{itemize}
(1) and (2) are clear and (3) follows from the fact that $\lambda$ is
ergodic. 
\end{proof}

\begin{proposition}\label{part2:chap7:prop96}
Let $(X, \mathfrak{X}, \mu)$ be a measure space. Let $x \to \nu_x$ be
a measure valued function on $X$ with values in a measurable space
$(\Omega, \mathscr{O})$, $\nu \in \mathfrak{X}$. Let $\rho = \int
\nu_x \mu(dx)$. Let $\mathscr{C}$ be a $\sigma$-algebra contained in
$\hat{\mathscr{O}}_\rho \cap \bigcap\limits_{x \in X}
\hat{\mathscr{O}}_{\nu_x}$. Let $\forall_\mu x$, $\nu_x$ have
$(\lambda^\mathscr{C}_w)_{w \in \Omega}$  as disintegration with
respect to $\mathscr{C}$. Then $\rho$ also has
$(\lambda^\mathscr{C}_w)_{w \in \Omega}$ as disintegration with
respect to $\mathscr{C}$.
\end{proposition}

\begin{proof}
We have to only prove that 
$$
\forall A \in \mathscr{C}, \; \chi_A \cdot \rho = \int\limits_A
\lambda^{\mathscr{C}}_w d \rho (w). 
$$

Let $B \in \mathscr{O}$.
\begin{gather*}
\chi_A \cdot \rho(B) = \rho(A \cap B) = \int \nu_x (A \cap B) d\mu
(x).\\
\forall_\mu x, \nu_x(A \cap B) = \int \lambda^\mathscr{C}_w (A\cap B)
d \nu_x(w). 
\end{gather*}
$\int \lambda^\mathscr{C}_w(A \cap B) d \nu_x(w) =\nu_x
(\lambda^\mathscr{C}. (A \cap B)) = \nu_x(\chi_A \cdot
\lambda^\mathscr{C}. (B))$. Therefore, 
\begin{align*}
\int \nu_x (A \cap B) d\mu (x) & = \int \nu_x (\chi_A \cdot
\lambda^\mathscr{C}. (B)) \mu (dx)\\
& = \int \chi_A (w) \cdot \lambda^\mathscr{C}_w (B) \rho (dw)\\
& = (\int\limits_A \lambda^\mathscr{C}_A \rho (dw)) (B). 
\end{align*}
Since $B \in \mathscr{O}$ is arbitrary,
$$
\chi_A \cdot \rho = \int\limits_A \lambda^\mathscr{C}_w \rho(dw). 
$$
\end{proof}

\section{Strong $\sigma$-Algebras}\label{part2:chap7:sec2}

Hereafter\pageoriginale throughout this chapter, we shall make the
following assumptions regarding $\Omega$, $\mathscr{O}$ and $\lambda$.
\begin{itemize}
\item[{\rm (1)}] $\Omega$ is a topological space and $\mathscr{O}$ is
  its Borel $\sigma$-algebra

\item[{\rm (2)}] $\mathscr{O}$ is countably generated

\item[{\rm (3)}] $\lambda$ is a probability measure on $\mathscr{O}$
  and 

\item[{\rm (4)}] $\Omega$ has the $\lambda$-compacity metrizability
  property. 
\end{itemize}

For example if  $\Omega$ is a Suslin space and $\mathscr{O}$ is its
Borel $\sigma$-algebra, (2) is verified and (4) is also true for any
probability measure $\lambda$. 

Let $\bar{\mathscr{O}}$ be the $\sigma$-algebra of all {\em
  universally measurable} sets of $\Omega$. i.e. $\bar{\mathscr{O}} =
\underset{\mu \in \mathfrak{m}^+(\Omega)}{\bigcap\limits
  \hat{\mathscr{O}}} $ where $\mathfrak{m}^+(\Omega)$ where is the set
of all probability measures on $\Omega$.

We remark that our assumptions regarding $\Omega$, $\mathscr{O}$ and
$\lambda$ implies the existence and uniqueness of disintegrations of
$\lambda$ with respect to any $\sigma$-algebra contained in
$\bar{\mathscr{O}}$.

If $\mathscr{H}$ is any $\sigma$-algebra contained in
$\bar{\mathscr{O}}$, let $(\lambda^\mathscr{H}_w)_{w \in \Omega}$
denote a disintegration of $\lambda$ with respect to $\mathscr{H}$. 

\begin{defn}\label{part2:chap7:def97}
Let $\mathscr{S}$ and $\mathscr{C}$ be two $\sigma$-algebras contained
in $\bar{\mathscr{O}}$. We say $\mathscr{C}$ is $\lambda$-{\em
  stronger than} $\mathscr{S}$ if $\forall_\lambda w$,
$\lambda^\mathscr{C}_w$ admits as disintegration with respect to
$\mathscr{C}$, the family of measures $(\lambda^\mathscr{C}_{w'})_{w'
  \in \Omega}$.

The following proposition shows that the property ``$\mathscr{C}$ is
$\lambda$-stronger than $\mathscr{S}$'' does not depend on the chosen
disintegrations. 
\end{defn}

\begin{proposition}\label{part2:chap7:prop98}
Let $\mathscr{S}$ and $\mathscr{C}$ be two $\sigma$-algebras contained
in $\bar{\mathscr{O}}$. Let $(\lambda^\mathscr{C}_w)_{w \in \Omega}$
(resp. $(\lambda^\mathscr{C}_w)_{w \in \Omega}$) and
$(\chi^\mathscr{C}_w)_{w \in \Omega}$ (resp. $(\chi^\mathscr{C}_w)_{w
  \in \Omega}$) be two disintegrations\pageoriginale of $\lambda$ with respect to
$\mathscr{S}$ (resp. $\mathscr{C}$). If $\forall_\lambda w$,
$\lambda^\mathscr{S}_w$ has $(\lambda^\mathscr{C}_{w'})_{w' \in
  \Omega}$ for disintegration with respect to $\mathscr{C}$, then
$\forall_\lambda w$, $\chi^\mathscr{S}_w$ has
$(\chi^\mathscr{C}_{w'})_{w' \in \Omega}$ for disintegration with
respect to $\mathscr{C}$. 
\end{proposition}


\begin{proof}
Since $\forall_\lambda w$, $\lambda^\mathscr{S}_w$ has
$(\lambda^\mathscr{C}_{w'})_{w' \in \Omega}$ for disintegration with
respect to $\mathscr{C}$, we have
\begin{itemize}
\item[{\rm (1)}] $\forall_\lambda w$, $\lambda^\mathscr{C}_w = \int
  \lambda^\mathscr{C}_{w'} d \lambda^\mathscr{S}_w(dw')$ 

\item[{\rm (2)}] $\forall_\lambda w$, given $A \in\mathscr{C}$,
  $\forall_{\lambda^\mathscr{S}_w}$ $w'$, $\lambda^\mathscr{C}_{w'}$
  is carried by $A$ or $\complement A$ according as $w' \in A$ or $w'
  \in\complement A$.
\end{itemize}

To prove the proposition, we have only to prove that 
\begin{itemize}
\item[{\rm (i)}] $\forall_\lambda w$, $\chi^\mathscr{S}_w = \int
  \chi^\mathscr{C}_{w'} d \chi^\mathscr{S}_w(w') $

\item[{\rm (ii)}] $\forall_\lambda w$, given $A \in \mathscr{C}$,
  $\forall_{\chi^\mathscr{S}_w} w'$, $\chi^\mathscr{C}_{w'}$ is
  carried by $A$ or $\complement A$ according as $w' \in A$ or $w' \in
  \complement A$. 
\end{itemize}

Because of the uniqueness of disintegrations, we have
\begin{itemize}
\item[{\rm (a)}] $\forall_\lambda w$, $\lambda^\mathscr{S}_w = \chi^\mathscr{S}_w$

\item[{\rm (b)}] $\forall_\lambda w$, $\lambda^\mathscr{C}_w =
  \chi^\mathscr{C}_w$ and therefore,

\item[{\rm (c)}] $\forall_\lambda w$, $\forall_{\chi^s_w} w'$,
  $\lambda^\mathscr{C}_{w'} = \chi^\mathscr{C}_{w'}$. 
\end{itemize}

It is now easy to see that (i) and (ii) follow from (1) and (2)
because of (a) and (c).

\end{proof}

\begin{proposition}\label{part2:chap7:prop99}
Let $\mathscr{C}$ be a $\sigma$-algebra contained in
$\bar{\mathscr{O}}$. If $\mathscr{C}$ is countably separating (in
particular, if $\mathscr{C}$  is countably generated), it is
$\lambda$-stronger than all its sub $\sigma$-algebras.
\end{proposition}

\begin{proof}
Let $\mathscr{S} \subset \mathscr{C}$ be a $\sigma$-algebra. Since
$\mathscr{O}$ is countably generated, we have
$$ 
\forall_\lambda w, \; \lambda^\mathscr{S}_{w} = \int  \lambda^\mathscr{C}_{w'} \lambda^{\mathscr{S}}_w (dw')
$$
because\pageoriginale of the transitivity of the conditional
expectations. Therefore, since $w' \to \lambda^\mathscr{C}_{w'}$
belongs to $\mathscr{C}$, to prove the proposition, it is sufficient
to prove by proposition (\ref{part1:chap3},
\S\ \ref{part1:chap3:sec7}, \ref{part1:chap3:prop52}) that
$\forall_\lambda w$, 
$\forall_{\lambda^\mathscr{C}_w} w'$, $\lambda^\mathscr{C}_{w'}$ is
carried by the $\mathscr{C}$-atom of $w'$. Since
$(\lambda^\mathscr{C}_w)_{w \in \Omega}$ is a disintegration of
$\lambda$ with respect to $\mathscr{C}$ and since $\mathscr{C}$ is
countably separating, according to proposition (\ref{part1:chap3},
\S\ \ref{part1:chap3:sec7}, \ref{part1:chap3:prop53}),
$\forall_\lambda w'$, $\lambda^\mathscr{C}_{w'}$ is carried by the
$\mathscr{C}$-atom of $w'$. Therefore, $\forall_\lambda w$,
$\forall_{\lambda^\mathscr{S}_{w}} w'$, $\lambda^\mathscr{C}_{w'}$ is
carried by the $\mathscr{C}$-atom of $w'$. 
\end{proof}

\begin{proposition}\label{part2:chap7:prop100}
Let  $\mathscr{C}$ and $\mathscr{S}$ be two $\sigma$-algebras
contained in $\bar{\mathscr{O}}$. If $\mathscr{C}$ is
$\lambda$-stronger than $\mathscr{S}$, it is $\lambda$-stronger than
every sub $\sigma$-algebra of $\mathscr{S}$.
\end{proposition}

\begin{proof}
Let $\mathscr{S}'$ be a $\sigma$-algebra contained in
$\mathscr{S}$. Since $\mathscr{O}$ is countably generated, we have
$\forall_\lambda w$, $\lambda^{\mathscr{S}'}_w = \int
\lambda^\mathscr{S}_{w'}d\lambda^{\mathscr{S}'}_w(w')$   by the
transitivity of the conditional expectations.

Since $\mathscr{C}$ is $\lambda$-stronger than $\mathscr{S}$,
$\forall_\lambda w'$, $\lambda^\mathscr{S}_{w'}$ has
$(\lambda^\mathscr{C}_{w''})_{w'' \in \Omega}$ for disintegration with
respect to $\mathscr{C}$. Therefore, $\forall_\lambda w$,
$\forall_{\lambda_w}^{\mathscr{S}'} w'$, $\lambda^\mathscr{S}_{w'}$
has $(\lambda^\mathscr{C}_{w''})_{w'' \in \Omega}$ for disintegration
with respect to $\mathscr{C}$. Hence by proposition
(\ref{part2:chap7}, \S\ \ref{part2:chap7:sec1}, \ref{part2:chap7:prop96}), 
$\forall_\lambda w$, $\lambda^{\mathscr{S}'}_w$ has
$(\lambda^\mathscr{C}_{w''})_{w'' \in \Omega}$ for disintegration with
respect to $\mathscr{C}$. 

This proves that $\mathscr{C}$ is $\lambda$-stronger than
$\mathscr{S}'$. 
\end{proof}

\begin{defn}\label{part2:chap7:def101}
A $\sigma$-algebra $\mathscr{C}$ contained in $\bar{\mathscr{O}}$ is
said to be {$\lambda$-{\em strong}} if it is $\lambda$-stronger than itself.
\end{defn}


We see by the above proposition, proposition (\ref{part2:chap7},
\S\ \ref{part2:chap7:sec2}, \ref{part2:chap7:prop100}), that a 
$\lambda$-strong $\sigma$-algebra is $\lambda$-stronger than all its
sub $\sigma$-algebras. From proposition (\ref{part2:chap7},
\S\ \ref{part2:chap7:sec2}, \ref{part2:chap7:prop99}) we see that
if a $\sigma$-contained in $\bar{\mathscr{O}}$ is countably
separating, it is $\mu$-strong for any probability measure $\mu$ on
$\mathscr{O}$, if $\Omega$ has the $\mu$-compacity metrizability
property. 

\begin{proposition}\label{part2:chap7:prop102}
Let\pageoriginale $\mathscr{C}$  be a $\sigma$-algebra contained in
$\bar{\mathscr{O}}$. $\mathscr{C}$ is $\lambda$-strong if and only if
one of the two following equivalent conditions is true. 
\begin{itemize}
\item[{\rm (i)}] $\forall_\lambda w$, $\lambda^\mathscr{C}_w$ is
  disintegrated with respect to $\mathscr{C}$ by the constant measure
  valued function $w' \to \lambda^\mathscr{C}_w$.

\item[{\rm (ii)}] $\forall_\lambda w$, $\lambda^\mathscr{C}_w$ is
  ergodic on $\mathscr{C}$. 
\end{itemize}
\end{proposition}

\begin{proof}
The equivalence of (i) and (ii) follows from proposition
(\ref{part2:chap7}, \S\ \ref{part2:chap7:sec1},
\ref{part2:chap7:prop95}). 

Since $\mathscr{O}$ is countably generated, by Corollary
(\ref{part1:chap3}, \S\ \ref{part1:chap3:sec7}, \ref{part1:chap3:coro56}), we have 
$$
\forall_\lambda w, \; \forall_{\lambda^\mathscr{C}_w} w', \;
\lambda^\mathscr{C}_{w'} = \lambda^\mathscr{C}_w. 
$$

Let $\mathscr{C}$ be $\lambda$-strong. Let us prove
(i). $\forall_\lambda w$, $\lambda^\mathscr{C}_w$ is disintegrated by
$(\lambda^\mathscr{C}_{w'})_{w' \in \Omega}$ with respect to
$\mathscr{C}$, and since $\forall_\lambda w$,
$\forall_{\lambda^\mathscr{C}_w} w'$, $\lambda^\mathscr{C}_{w'}$,
$\lambda^\mathscr{C}_{w' } = \lambda^\mathscr{C}_w$, we see that
$\forall_\lambda w$, $\lambda^\mathscr{C}_w$ is disintegrated by the
constant measure valued function $ w' \to \lambda^\mathscr{C}_{w}$,
with respect to $\mathscr{C}$. This is (i)

Let us assume (i) and prove that $\mathscr{C}$ is $\lambda$-strong. 

$\forall_\lambda w$, $\lambda^\mathscr{C}_w$ is disintegrated with
respect to $\mathscr{C}$ by the constant measure valued function $w'
\to \lambda^\mathscr{C}_w$ and $\forall_\lambda w$,
$\forall_{\lambda^\mathscr{C}_w} w'$, $\lambda^\mathscr{C}_{w'} =
\lambda^\mathscr{C}_w$. 

Hence, $\forall_\lambda w$, $\lambda^\mathscr{C}_w$ is disintegrated
with respect to $\mathscr{C}$ by the family
$(\lambda^\mathscr{C}_{w'})_{w' \in\Omega}$. This proves that
$\mathscr{C}$ is $\lambda$-strong.
\end{proof}

In the following proposition, we will be using the fact that if $f$ is
a $\lambda$-integrable function and if $(\mathscr{C}^n)_{n \in
  \mathbb{N}}$ is a decreasing sequence of $\sigma$-algebras contained
in $\bar{\mathscr{O}}$ with $\mathscr{C} = \bigcap\limits_{n=1}
\mathscr{C}^n$, then $\forall_\lambda w$, $f^{\mathscr{C}^n}(w) \to
f^{\mathscr{C}}(w)$. This follows immediately from the convergence
theorem for martingales with respect to a decreasing sequence of
$\sigma$-algebras. See P.A. Meyer \cite{key1}, Chap. V. T21. 

\begin{proposition}\label{part2:chap7:prop103}
Let $(\mathscr{C}^n)_{n \in \mathbb{N}}$ be a decreasing sequence of
$\sigma$-algebras contained in $\bar{\mathscr{O}}$ and let
$\mathscr{C} = \bigcap\limits_{n=l}^\infty \mathscr{C}^n$. If $\forall
\; n \in \mathbb{N}$, $\mathscr{C}^n$ is $\lambda$-strong then
$\mathscr{C}$ is also $\lambda$-strong. 
\end{proposition}

\begin{proof}
We\pageoriginale have to prove that $\forall_\lambda w$,
$\lambda^\mathscr{C}_w$ is disintegrated with respect to $\mathscr{C}$
by $(\lambda^\mathscr{C}_{w'})_{w' \in \Omega}$. We have only to check
that 
$$
\forall_\lambda w, \forall B \in \mathscr{O}, w' \to \int \chi_B
(w'') \lambda^\mathscr{C}_{w'} (dw'')
$$
is a conditional expectation of $\chi_B$ with respect to $\mathscr{C}$
for the measure $\lambda^\mathscr{C}_w$. i.e. we have to only prove
that $\forall_\lambda w$, $\forall B \in \mathscr{O}$, for all $A \in
\mathscr{C}$,
$$
\int\limits_A \chi_B (w') \lambda^\mathscr{C}_w(dw') = \int\limits_A
(\int \chi_B(w'')\lambda^\mathscr{C}_{w'} (dw''))
\lambda^\mathscr{C}_w (dw'). 
$$

To prove this, it is sufficient to prove that 
$$
\forall \; B \in \mathscr{O}, \; \forall_\lambda w, \; \int\limits_A
\chi_B (w') \lambda^\mathscr{C}_w(dw') = \int\limits_A (\int\chi_B
(w'') \lambda^\mathscr{C}_{w'} (dw''))  \lambda^\mathscr{C}_w(dw')
$$
for all $A \in \mathscr{C}$. For if we prove this, an application of
Monotone class theorem will give the result, since $\mathscr{O}$ is
countably generated.

Let $B \in \mathscr{O}$. Since $\lambda$ is a probability measure,
$\chi_B$ is $\lambda$-integrable. Hence,
$$
\forall_\lambda w' \cdot (\chi_B)^{\mathscr{C}^n} (w') \to
(\chi_B)^{\mathscr{C}}  (w'). 
$$
i.e. $\forall_\lambda w'$, $\int \chi_B (w'')
\lambda^{\mathscr{C}^n}_{w'} (dw'') \to \int \chi_B (w'')
\lambda^\mathscr{C}_{w'} (dw'')$. Hence,
$$
\forall_\lambda w, \forall_{\lambda^\mathscr{C}_w} w', \int \chi_B
(w'') \lambda^{\mathscr{C}^n}_{w'} (dw'') \to \int \chi_B (w'')
\lambda^\mathscr{C}_{w'} (dw''). 
$$

Therefore, by Lebesgue's dominated convergence theorem (which is
applicable here since $\forall_\lambda w$,
$\forall_{\lambda^\mathscr{C}_w} w'$, $\lambda^{\mathscr{C}^n}_{w'}$
is a probability measure for all $n$ and hence $\int \chi_B (w'')
\lambda^{ \mathscr{C}^n}_{w'} (dw'') \leq 1$ and 1 is integrable with
respect to $\lambda^\mathscr{C}_w$ for $\lambda$-almost all $w$), we
have $\forall_\lambda w$, for all $A \in \mathscr{C}$, 
$$
\int\limits_A (\int \chi_B (w'') \lambda^{\mathscr{C}^n}_{w'} (dw''))
\lambda^\mathscr{C}_w (dw') \to \int\limits_A (\int \chi_B (w'')
\lambda^\mathscr{C}_{w'} (dw'')) \lambda^\mathscr{C}_w(dw'). 
$$

Since\pageoriginale $\forall$ $n \in\mathbb{N}$, $\mathscr{C}^n$ is
$\lambda$-strong, and hence is $\lambda$-stronger than $\mathscr{C}$,
$\forall_\lambda w$, $\lambda^\mathscr{C}_w$ is disintegrated with
respect to $\mathscr{C}^n$ by $(\lambda^\mathscr{C}_{w'})_{w'
  \in\Omega}$ for all $n\in\mathbb{N}$. Hence, $\forall_\lambda w$,
$\forall B \in \mathscr{O}$, for all $A \in \mathscr{C}$, for all $n
\in \mathbb{N}$, 
$$
\int\limits_A (\int \chi_B (w'') \lambda^{\mathscr{C}^n}_{w'} (dw''))
\lambda^\mathscr{C}_w(dw') = \int\limits_A \chi_B (w')
\lambda^\mathscr{C}_w(dw') . 
$$
Hence $\forall B \in \mathscr{O}$, $\forall_\lambda w$, for all $A
\in\mathscr{C}$, 
$$
\int \chi_B (w') \lambda^\mathscr{C}_w(dw') = \int\limits_A (\int
\chi_B (w'') \lambda^\mathscr{C}_{w'} (dw''))
\lambda^\mathscr{C}_w(dw'). 
$$
This is what we wanted to prove. 
\end{proof}

\section{Choquet-type integral representations}\label{part2:chap7:sec3}

In this section, we shall obtain a Choquet-type integral
representation for probability measures which have a given family of
probability measures as disintegration with respect to a given
$\sigma$-algebra of $\bar{\mathscr{O}}$. 

\begin{defn}\label{part2:chap7:def104}
Let $\mathscr{C}$ be a sub $\sigma$-algebra of
$\bar{\mathscr{O}}$. $\mathscr{C}$ is said to be {\em universally
  strong} if it is $\lambda$-strong for every probability measure
$\lambda$ on $\mathscr{O}$. 
\end{defn}

Let us fix a sub $\sigma$-algebra $\mathscr{C}$ of $\bar{\mathscr{O}}$
which is universally strong, throughout this section. Let
$(\lambda_w)_{w \in \Omega}$ be a family of probability measures on
$\mathscr{O}$. Let us assume that the measure valued function $w \to
\lambda_w$ belongs to $\mathscr{C}$, i.e. $\forall$ $B \in
\mathscr{O}$, $w \to \lambda_w (B) \in \mathscr{C}$. 

$\forall \; w \in \Omega$, consider the set $\{w' \in \Omega \mid
\lambda_{w'} = \lambda_w\}$. This set $\in \mathscr{C}$ and hence is a
union of $\mathscr{C}$-atoms. Hence, we shall call it the molecule of
$w$ and write it as Mol $w$. 

Let $\tilde{\Omega}  = \{w \in \Omega \mid \lambda_w$ is disintegrated
with respect to $\mathscr{C}$ by $(\lambda_{w'})_{w' \in \Omega}$ and
is carried by Mol $w$ $\}$.

Let $\mathscr{K} = \{ \lambda \mid \lambda $ a probability measure on
$\mathscr{O}$ such that $\lambda$ has $(\lambda_w)_{w \in \Omega}$
as\pageoriginale disintegration with respect to $\mathscr{C} ~\}$.

Then $\mathscr{K}$ is a convex set.

\begin{proposition}\label{part2:chap7:prop105}
If $\lambda \in \mathscr{K}$, $\lambda$ is carried by $\tilde{\Omega}$
and $\lambda = \int\limits_\Omega \lambda_w d\lambda(w)$. If $\mu$ is
any probability measure carried by $\tilde{\Omega}$ and if $\rho =
\int\limits_\Omega \lambda_wd\mu(w )$, then $\rho \in \mathscr{K}$.
\end{proposition}

\begin{proof}
Let $\lambda \in \mathscr{K}$. Since $\mathscr{C}$ is universally
strong, it is in particular $\lambda$-strong $\lambda$ has
$(\lambda_w)_{w \in \Omega}$ for disintegration with respect to
$\mathscr{C}$. Hence, $\forall_\lambda w$, $\lambda_w$ is
disintegrated with respect to $\mathscr{C}$ by $(\lambda_{w'})_{w' \in
\Omega}$. Moreover, by Corollary (\ref{part1:chap3},
\S\ \ref{part1:chap3:sec7}, \ref{part1:chap3:coro56}), 
$$
\forall_\lambda w, \; \forall_{\lambda_w} w', \; \lambda_{w'} =
\lambda_w. 
$$
This shows that $\lambda $ is carried by $\tilde{\Omega}$. Since
$\lambda = \int\limits_\Omega \lambda_w \lambda(dw)$ and since
$\lambda$ is carried by $\tilde{\Omega}$, $\lambda =
\int\limits_{\tilde{\Omega}} \lambda_w\lambda(dw)$. The rest of the proposition
follows immediately from proposition (\ref{part2:chap7},
\S\ \ref{part2:chap7:sec1}, \ref{part2:chap7:prop96}). 
\end{proof}

\begin{thm}\label{part2:chap7:thm106}
The extreme points of $\mathscr{K}$ are precisely the measures
$\lambda_w$ where $w \in \tilde{\Omega}$.
\end{thm}

\begin{proof}
Let us first show that $\forall w \in \tilde{\Omega}$, $\lambda_w$ is
extreme.

Let $w \in\tilde{\Omega}$. Then $\lambda_w\in \mathscr{K}$. Let $\mu$
and $\nu$ be two measures $\in \mathscr{K}$ and $t$ be a real number
with $0 < t < 1 $  such that 
$$
\lambda_w = t \mu + (1-t) \nu .
$$

 Mol  $w \in \mathscr{C}$ and $\lambda_w (Mol w) =1$ since $\lambda_w$
 is carried by Mol $w$. Hence $\mu$ and $\nu$ are also carried by Mol
 $w$. By the previous proposition \ref{part2:chap7},
 \S\ \ref{part2:chap7:sec3}, \ref{part2:chap7:prop105}), $\mu = \int 
 \lambda_{w'} \mu (dw')$ and $\nu = \int \lambda_{w'} \nu(dw')$.
 Since $\mu$ and $\nu$ are carried by Mol $w$, the integration is
 actually over only Mol $w$. Thus, 
$$ 
\mu = \int\limits_{Mol\; w} \lambda_{w'} \mu (dw') \text{ and } \nu =
\int\limits_{Mol \; w} \lambda_{w'} \nu (dw').
$$

But for $w' \in Mol \; w$, $\lambda_{w'} = \lambda_w$ and hence $\mu =
\lambda_w = \nu$. 

Thus\pageoriginale $\lambda_w$ is extreme $\forall $ $w \in
\tilde{\Omega}$. 

Now, let $\lambda$ be an extreme point of $\mathscr{K}$. Let us show
that there exists a $w \in \tilde{\Omega}$ such that $\lambda_w  =
\lambda$. 

First of all, we claim that $\lambda$ must be ergodic on
$\mathscr{C}$. For, if not, $\exists A \in \mathscr{C}$ such that $0 <
\lambda (A) < 1$, 
$$
\lambda = \lambda (A) \cdot \frac{\chi_A \cdot \lambda}{\lambda(A)} +
\lambda (\complement A) \cdot \frac{\chi_{\complement A} \cdot
  \lambda}{\lambda(\complement A )}. 
$$

Thus, $\lambda$ is a convex combination of the measures $\dfrac{\chi_A
\cdot \lambda}{\lambda(A)}$ and $\dfrac{\chi_{\complement A} \cdot
  \lambda}{\lambda(\complement A)}$. Since $A \in \mathscr{C}$, and
$\lambda \in \mathscr{K}$, $\dfrac{\chi_A \cdot \lambda}{\lambda(A)}$
also $\in \mathscr{K}$. Similarly, $\dfrac{\chi_{\complement A} \cdot
  \lambda}{ \lambda (\complement A)}$ also belongs to
$\mathscr{K}$. These two measures are not the same since
$\dfrac{\chi_A \cdot \lambda}{\lambda(A)}$ is carried by $A$ and
$\dfrac{\chi_{\complement A} \cdot \lambda}{\lambda(\complement A)}$
is carried by $\complement A$. Thus, we get a contradiction to the
fact that $\lambda$ is extreme. Hence $\lambda$ must be ergodic on
$\mathscr{C}$. Hence by proposition (\ref{part2:chap7},
\S\ \ref{part2:chap7:sec1}, \ref{part2:chap7:prop95}) $\lambda$ is 
disintegrated with respect to $\mathscr{C}$ by the constant measure
valued function $w \to \lambda^\mathscr{C}_w  = \lambda$. It follows
therefore, by uniqueness of disintegrations that $\forall_\lambda w$,
$\lambda_w = \lambda$. By the previous proposition $\lambda$
is carried by $\tilde{\Omega}$. Hence there exists a $w
\in\tilde{\Omega}$ such that $\lambda_w = \lambda$. 
\end{proof}

In view of theorem (\ref{part2:chap7}, \S\ \ref{part2:chap7:sec3},
\ref{part2:chap7:thm106}) and proposition (\ref{part2:chap7},
\S\ \ref{part2:chap7:sec3}, \ref{part2:chap7:prop105})  
we see that the integral representation of $\lambda \in \mathscr{K}$
as $\int\limits_{\tilde{\Omega}} \lambda_w \lambda(dw)$ is indeed of
the same type as the one considered by Choquet. But, we have not
deduced our result from Choquet's theory. 

Let us now prove a kind of uniqueness theorem.

Let $\tilde{\Omega}^\bigdot$ be the quotient set of $\tilde{\Omega}$
by the molecules. Let $p$ be the canonical mapping from
$\tilde{\Omega}$ to $\tilde{\Omega}^\bigdot$. Let $\mathscr{C}^\circ$
be the $\sigma$-algebra on  $\tilde{\Omega}^\bigdot$ consisting of
sets whose inverse image under $p$ belongs to $\mathscr{C}$. If $w \in
\tilde{\Omega}$, let us denote by $\dot{w}$ the element $p(w)$ of
$\tilde{\Omega}^\bigdot$. $\forall \dot{w} \in
\tilde{\Omega}^{\bigdot}$ define the measure $\lambda_{\dot{w}}$ on
$\mathscr{C}^\circ$ as the image measure of $\lambda_w$ under the
mapping $p$ where $w$ is such that $p(w) =\dot{w}$. 

This\pageoriginale is independent of the choice of $w$ in
$p^{-1}(\dot{w})$, for if $p(w_1) = p(w_2)$, then $\lambda_{w_1} =
\lambda_{w_2}$. If $\mu$ is any measure on $\mathscr{C}$, let us
denote by $\dot{\mu}$ its image measure under $p$ on
$\mathscr{C}^\circ$. We note that if $A \in \mathscr{C}^\circ$,
$\lambda_{\dot{w}} (A) = 1$ or $0$ according as $\dot{w} \in A$ or
not. For, if $\dot{w} \in A$, then $w \in p^{-1}(A)$ where $w$ is such
that $p(w) = \dot{w}$. Hence Mol $w \subset p^{-1} (A)$. Therefore,
$\lambda_{\dot{w}}(A) = \lambda_w(p^{-1} (A)) =1 $ since $\lambda_w$
is carried by Mol $w$. If $\dot{w} \not\in A$, $\dot{w} \in
\complement A$ and by the same argument $\lambda_{\dot{w}}
(\complement A) = 1$. Hence $\lambda_{\dot{w}} (A) =0$. 

\begin{thm}\label{part2:chap7:thm107}
Let $\lambda \in \mathscr{K}$. If $\mu$ is any probability measure
carried by $\tilde{\Omega}$ such that $\lambda = \int \lambda_w \mu
(dw)$ , then $\dot{\mu} = \dot{\lambda}$ on $\mathscr{C}^\circ$.
\end{thm}

\begin{proof}
Let $\lambda = \int\limits_{\tilde{\Omega}} \lambda_w \mu (dw)$. We first see that
$\dot{\lambda} = \int\limits_{\tilde{\Omega}^\bigdot}
\lambda_{\dot{w}} \dot{\mu} (d\dot{w})$. For let $A \in
\mathscr{C}^\circ$
\begin{align*}
\dot{\lambda} (A) = \lambda (p^{-1} (A)) & = \int \lambda_w (p^{-1}
(A)) \mu(dw)\\
& = \int \lambda_{\dot{w}} (A) \mu (dw)\\
& = \int \lambda_{p(w)} (A) \mu (dw)\\
& = \int \lambda_{\dot{w}} (A) \dot{\mu} (d\dot{w}).
\end{align*} 
Hence,
$$
\dot{\lambda} = \int_{\tilde{\Omega}^\bigdot} \lambda_{\dot{w}}
\dot{\mu} (d\dot{w}). 
$$

For any $A \in \mathscr{C}^\circ$, $\dot{\lambda}(A) =
\int\limits_{\tilde{\Omega}^\bigdot} \lambda_{\dot{w}} (A) \dot{\mu}
(d\dot{w})$. Since $\lambda_{\dot{w}}(A) =1$ or $0$ according as
$\dot{w} \in A$ or $\complement A$, the integration is actually only
over $A$; i.e.
$$
\dot{\lambda} (A) = \int\limits_A \lambda_{\dot{w}} (A) \dot{\mu}
(d\dot{w}). 
$$
But, $\int\limits_A \lambda_{\dot{w}} (A) \dot{\mu} (d\dot{w}) =
\dot{\mu}(A)$  since $\lambda_{\dot{w}} (A) =1$ for $\dot{w} \in
A$. Hence, 
$$
\dot{\lambda} (A) = \dot{\mu} (A), \; \forall A \in \mathscr{C}^\circ
$$
and therefore, $\dot{\lambda} = \dot{\mu}$. 
\end{proof}

\section{The Usual $\sigma$-Algebras}\label{part2:chap7:sec4}

In\pageoriginale this section, we shall define the $\sigma$-algebras
that occur in the theory of Brownian motion on the real line and prove
some properties of them.

Let $\Omega_{[0,+\infty)}$ be  the set of all real valued continuous
  functions on $[0, + \infty)$. Let $D$ be a countable dense subset of
    $[0,+\infty)$. For $t \in [0,+\infty)$ define the mapping $\pi_t$
        from $\Omega_{[0,+\infty)}$ to $\mathbb{R}$ as $\pi_t(w) =
          w(t)$, where $w$ is an element of
          $\Omega_{[0,+\infty)}$. $\pi_t$ is called the $t'$th
            projection. Let $\mathfrak{I}_D$ be the topology on
            $\Omega_{[0,+\infty)}$ which is the coarsest making all
              the $(\pi_t)_{t \in D}$ continuous. Let $\mathscr{O}_D$ be
              the Borel $\sigma$-algebra of $\Omega_{[0,+\infty)}$ for
                the topology $\mathfrak{I}_D$. It can be easily
                checked that $\mathscr{O}_D$ is the smallest
                $\sigma$-algebra making the projections $(\pi_t)_{t\in
                D}$ measurable. Since $\forall t \in [0,+\infty)$,
                  $\pi_t = \lim\limits_{\substack{t_n\to t\\t_n \in
                      D}} \pi_{t_n}$, it follows immediately that
                  $\mathscr{O}_D$ is also the smallest
                  $\sigma$-algebra making all the projections
                  $(\pi_t)_{t \in[0,+\infty)}$ measurable.

Let $\mathfrak{I}_P$ be the topology of pointwise convergence on
$\Omega_{[0,+\infty)}$ i.e. $\mathfrak{I}_P$ is the coarsest topology
  making all the $(\pi_t)_{t\in [0,+\infty)}$ continuous. Let
    $\mathscr{O}_P$ be the Borel $\sigma$-algebra of
    $\Omega_{[0,+\infty}$ in this topology $\mathfrak{I}_P$. Let
      $\mathfrak{I}$ be the topology of uniform convergence on compact
      sets of $[0,+\infty)$. Let $\mathscr{O}$  be the Borel
        $\sigma$-algebra of $\Omega_{[0,+\infty)}$ in this topology
          $\mathfrak{I}$. We easily see that $\mathfrak{I}_D$ is
          coarser than $\mathfrak{I}_P$ which in turn is coarser than
          $\mathscr{I}$. Hence $\mathscr{O}_D \subset \mathscr{O}_P
          \subset \mathscr{O}$. 

Now, $\Omega_{[0,+\infty)}$ is a separable Fr\'echet space under the
  topology $\mathfrak{I}$ and hence is a Polish space i.e. is
  homeomorphic to a complete separable metric space. It is a theorem
  that the Borel $\sigma$-algebra of a Polish space is the same as the
  Borel $\sigma$-algebra of any coarser Hausdorff topology. Hence
  $\mathscr{O}_D = \mathscr{O}_P = \mathscr{O}$. 

Thus the smallest $\sigma$-algebra making all the
$(\pi_t)_{t\in[0,+\infty)}$ measurable coincides with the Borel
  $\sigma$-algebra of the topology of pointwise convergence on
  $[0,+\infty)$ and it is countably generated since $\mathscr{O}$ is
    countably generated. 

Let\pageoriginale $\mathcal{U}'$ be the $\sigma$-algebra on
$\Omega_{[0,+\infty)}$ generated by $(\pi_s)_{s\leq t}$ where $t \in
  [0,+\infty)$. Let $\Omega_{[0,t]}$ be the space of all real valued
    continuous functions on $[0,t]$. For $0 \leq s \leq t$, let
    $\pi'_s$ be the map defined on $\Omega_{[0,t]}$ as $\pi'_s(w) =
    w(s)$ where $w \in \Omega_{[0,t]}$. As above, we can see that the
    Borel $\sigma$-algebra of $\Omega_{[0,t]}$ for the topology of
    pointwise convergence on $[0,t]$ is the same as the
    $\sigma$-algebra generated by $(\pi's)_{0\leq s \leq t}$. Let us
    denote this $\sigma$-algebra by $\mathscr{O}^t$. As above we can
    see that $\mathscr{O}^t$ is countably generated.

Let $p:\Omega_{[0,+\infty)} \to \Omega_{[0,t]}$ be the restriction
  map. It is easily checked that $\mathcal{U}^t$ is equal to
  $p^{-1}(\mathscr{O}^t)$ where $p^{-1}(\mathscr{O}^t)$ is the
  $\sigma$-algebra consisting of sets $p^{-1}(A)$ as $A$ varies over
  $\mathscr{O}^t$. Since $\mathscr{O}^t$ is countably generated,
  $\mathcal{U}^t$ is also countably generated.

Let $w \in \Omega_{[0,+\infty)}$. The $\mathcal{U}^t$-atom of $w$ is
  $p^{-1} (p(w))$  i.e. the set of all trajectories which coincide
  with $w$ upto time $t$.

\begin{proposition}\label{part2:chap7:prop108}
$A \in \mathcal{U}^t \Longleftrightarrow A \in \mathscr{O}$ and is a union
  of $\mathcal{U}^t$-atoms.
\end{proposition}

\begin{proof}
Let $A \in \mathcal{U}^t$. Then clearly $A \in \mathscr{O}$ and is a
union of $\mathcal{U}^t$-atoms. We have to prove only the other way.

Let  $(f_n)_{n \in \mathbb{N}}$ be a countable number of functions
generating the $\sigma$-algebra $\mathcal{U}^t$. Consider the mapping
$\varphi: \Omega_{[0,+\infty)} \to \mathbb{R}^\mathbb{N}$ given by
  $\varphi(w) = (f_n(w))_{n\in\mathbb{N}}$. Since $\forall$
  $n \in \mathbb{N}$, $f_n \in \mathcal{U}^t$, $\varphi$ also $\in
  \mathcal{U}^t$ and hence is Borel measurable i.e. measurable with
  respect to $\mathscr{O}$. Hence $\varphi (\Omega_{[0,+\infty)})$ is
    a Suslin subset of $\mathbb{R}^\mathbb{N}$ since
    $\Omega_{[0,+\infty)}$ is a Polish space for the topology
      $\mathfrak{I}$ and $\mathscr{O}$ is its Borel
      $\sigma$-algebra. Since $A \in \mathscr{O}$, $A$ is Suslin and
      hence $\varphi(A)$ ia Suslin. Similarly $\varphi(\complement A)$
      is also Suslin. Since $A$ is union of $\mathcal{U}^t$-atoms, it
      is easily checked that
      $\varphi(\Omega_{[0,+\infty)}) \varphi(A)$ is precisely
        $\varphi(\complement A)$. Thus the Suslin subsets $\varphi(A)$
        and $\varphi(\complement A)$ are complementary in
        $\varphi(\Omega_{[0,+\infty)})$ and hence are
          Borel. Since\pageoriginale $A$ is a union of
          $\mathcal{U}^t$-atoms, $A = \varphi^{-1}(\varphi(A))$. Since
          $\varphi \in \mathcal{U}'$ and $\varphi(A)$ is Borel, it
          follows that $\varphi^{-1}(\varphi(A))$ i.e. $A \in
          \mathcal{U}^t$. 
\end{proof}

Let $\forall t \in [0,+\infty)$, $\mathscr{C}^t$ be the
  $\sigma$-algebra $\bigcap\limits_{\epsilon >0} \mathcal{U}^{t+
    \epsilon}$.

\begin{corollary}\label{part2:chap7:coro109}
$A \in \mathscr{C}^t \Longrightarrow A \in \mathscr{O}$ and is a union
  of $\mathscr{C}^t$-atoms.
\end{corollary}


\begin{proof}
If $A \in \mathscr{C}^t$, then $A \in \mathscr{O}$ and is a union of
$\mathscr{C}^t$-atoms. Conversely if $A \in \mathscr{O}$ and is a
union of $\mathscr{C}^t$-atoms, it is $\forall \epsilon > 0$,  a union of
$\mathcal{U}^{t+\epsilon}$-atoms and hence by the above proposition,
$A \in \mathscr{U}^{t+\epsilon} \forall \epsilon >0$. Hence $A
\epsilon \mathscr{C}^t$.
\end{proof}

\begin{proposition}\label{part2:chap7:prop110}
The $\mathscr{C}^t$-atom of $w$ consists of precisely the trajectories
which coincide with $w$ a little beyond $t$. i.e.
$$ 
\mathscr{C}^t-\text{atom of  } w = \{ w' \mid \exists \;\; t_{w'} > t
\text{ such that } w' = w \text{ in } [0, t_{w'}]\}. 
$$
\end{proposition}

\begin{proof}
Let $B = \{w'\mid \exists \;\; t_{w'} >t \text{ such that } w' = w \text{
  in } [0, t_{w'}]\}$
\begin{align*}
B & = \bigcup\limits_{\epsilon > 0} \{w' \mid w' = w\text{ in } [0, t+
\epsilon]\}\\
&  =  \bigcup\limits_{\epsilon > 0}
\{\mathcal{U}^{t+\epsilon}-\text{ atom of } w\}. 
\end{align*}

Let $A$ be any set $\epsilon \mathscr{C}^t$ containing $w$. Then,
$\forall \epsilon >0$, $A \epsilon \mathcal{U}^{t+\epsilon}$ and hence
$A$ contains the $\mathcal{U}^{t+\epsilon}$-atom of $w$, $\forall
\epsilon >0$. Hence $A \supset B$.

Therefore, to prove the proposition, sufficient to prove that $B \in
\mathscr{C}^t$. 
\begin{align*}
B & = \bigcup\limits_{\epsilon >0}
\{\mathcal{U}^{t+\epsilon}-\text{atom  of }w\}\\
& = \bigcup\limits_{n \in\mathbb{N}}
\{\mathcal{U}^{t+\dfrac{1}{n}}- \text{atom of } w\}\\
& = \bigcup\limits_{m \in \mathbb{N}}
\bigcup\limits_{\substack{\frac{1}{n} < \frac{1}{m} \\ n \in
    \mathbb{N}}} \{\mathcal{U}^{t+\frac{1}{n}} -\text{atom of } w\}
\end{align*}
$\forall \; m \in \mathbb{N}$, $\bigcup\limits_{\substack{\frac{1}{n}
    < \frac{1}{m} \\ n \in \mathbb{N}}} \{\mathcal{U}^{t+
  \frac{1}{n}}-\text{atom  of } w \} \in \mathcal{U}^{t+\frac{1}{m}}$
and hence, 
$$
\bigcap\limits_{m \in \mathbb{N}}
\bigcup\limits_{\substack{\frac{1}{n} < \frac{1}{m}\\n \in
    \mathbb{N}}} \{\mathcal{U}^{t+\frac{1}{n}}-\text{atom of } w\}
\epsilon \mathscr{C}^t.
$$
\end{proof}

One\pageoriginale can prove that $\forall t$, $\mathscr{C}^t$ is not
countably separating. For a proof see L. Schwartz \cite{key1}, page
161.

However, $\forall t$, $\mathscr{C}^t$ is universally strong. For,
since $\forall \epsilon >0$, $\mathcal{U}^{t+\epsilon}$ is countably
generated, it is universally strong by proposition (\ref{part2:chap7},
\S\ \ref{part2:chap7:sec2}, \ref{part2:chap7:prop99}). Since
$\mathscr{C}^t = \bigcap\limits_{n \in \mathbb{N}} 
\mathcal{U}^{t+\frac{1}{n}}$, by proposition (\ref{part2:chap7},
\S\ \ref{part2:chap7:sec2}, \ref{part2:chap7:prop103}), it 
follows that $\mathscr{C}^t$ is universally strong $\forall t \in
[0,+\infty)$.

\section{Further results on regular disintegrations }\label{part2:chap7:sec5}

In this section, let us assume that $\Omega$ is a compact metrizable
space and $\mathscr{O}$ is its Borel $\sigma$-algebra. Let
$(\mathscr{C}^t)_{t \in \mathbb{R}}$ be a right continuous increasing
family of $\sigma$-algebras contained in $\overline{\mathscr{O}}$
which is the $\sigma$-algebra of all universally measurable sets of
$\Omega$. Let $\lambda$ be a probability measure on $\mathscr{O}$. 

Note that a unique regular disintegration of $\lambda$ with respect
to $(\mathscr{C}^t)_{t\in \mathbb{R}}$ exits.

Let us assume that $\forall t \in \mathbb{R}$, $\mathscr{C}^t$ is
$\lambda$-strong. 

\begin{thm}\label{part2:chap7:thm111}
Let $(\lambda^t_w)_{\substack{w \in \Omega\\t \in \mathbb{R}}}$ be a
regular disintegration of $\lambda$ with respect to
$(\mathscr{C}^t)_{t \in \mathbb{R}}$. Then $\forall_\lambda w$,
$\forall s$, $\lambda^s_w$ has a regular disintegration with respect
to $(\mathscr{C}^t)_{t \in \mathbb{R}}$ given by 
\begin{align*}
& (t,w') \to \lambda^t_{w'} \text{ for } t \geq s\\
& (t,w') \to \lambda^s_w \text{ for } t < s.
\end{align*}
\end{thm}

\begin{proof}
Let $t$ be fixed. Since $\mathscr{C}^t$ is $\lambda$-strong,
$\forall_\lambda w$, $\lambda^t_w$ has $(\lambda^t_w)_{w' \in \Omega}$
for disintegration with respect to $\mathscr{C}^t$. Since a set of
$\lambda$-measure zero is also $\forall_\lambda w$, $\forall s$, of
$\lambda^s_w$-measure zero, it follows therefore that 

$\forall_\lambda w$, $\forall s$, $\forall_{\lambda^s_w} w'$,
$\lambda^t_{w'}$ has $(\lambda^t_{w''})_{w''\epsilon \Omega}$ for
disintegration with respect to $\mathscr{C}^t$. 

By\pageoriginale theorem (\ref{part2:chap6},
\S\ \ref{part2:chap6:sec2}, \ref{part2:chap6:thm94}), we have 
$$
\forall_\lambda w, \; \forall s \leq t, \; \lambda^s_w = \int
\lambda^t_{w'} \lambda^s_w(dw'). 
$$

Hence, by proposition (\ref{part2:chap7}, \S\ \ref{part2:chap7:sec1},
\ref{part2:chap7:prop96}), $\forall_\lambda w$, $\forall 
s \leq t$, $\lambda^s_w$ has $(\lambda^t_{w''})_{w'' \in \Omega}$ for
disintegration with respect to $\mathscr{C}^t$. Thus, $\forall t$,
$\forall_\lambda w$, $\forall s \leq t$, $\lambda^s_w$ has
$(\lambda^t_{w''})_{w'' \in \Omega}$ for disintegration with respect
to $\mathscr{C}^t$.

Therefore, $\forall_\lambda w$, $\forall t$, a dyadic rational
$\forall \; s \leq t$, $\lambda^s_w$ has $(\lambda^s_{w''})_{w'' \in
  \Omega}$ for disintegration with respect to $\mathscr{C}^t$. 

Now, let $\mu$ be a probability measure on $\mathscr{O}$ and let
$(\delta^t_w)_{w \in \Omega}$ be a disintegration of $\mu$ with
respect to $\mathscr{C}^t$ $\forall$ dyadic $t$. 

For any $t \in \mathbb{R}$, define
$$
\delta^t_w =
 \begin{cases}
\text{vague } \lim \delta^{\tau_n(t)}_w, \text{ if it exists}\\
0, \quad \text{i f not}
 \end{cases}
$$

Then, since $\Omega$ is compact metrizable, we can easily check that\break
$(\delta^t_w)_{\substack{w \in \Omega\\ t \in \mathbb{R}}}$  is a
regular disintegration of $\mu$ with respect to $(\mathscr{C}^t)_{t
  \in \mathbb{R}}$.

Now we have,
$$
\forall_\lambda w, \; \forall t \in \mathbb{R}, \; \lambda^t_w =
\text{ vague } \lim\limits_{n \to \infty} \lambda^{\tau_n (t)}_w. 
$$

Hence, since $\forall_\lambda w$, $\forall s$, $\lambda^s_w$ has a
disintegration $(\lambda^t_{w''})_{w'' \in \Omega}$ with respect to
$\mathscr{C}^t$ for all $t$ dyadic $\geq s$, by the preceding
paragraph, $\forall-\lambda w$, $\forall s$, $\lambda^s_w$ has
$(\lambda^t_{w''})_{w'' \in \Omega}$ for a regular disintegration with
respect to $\mathscr{C}^t$ for all $t \geq s$.

By the theorem of trajectories, i.e. by theorem (\ref{part2:chap6},
\S\ \ref{part2:chap6:sec2}, \ref{part2:chap6:thm93}), 
$$
\forall_\lambda w, \; \forall s, \forall_{\lambda^s_w} w', \; \forall
t \leq s, \; \lambda^t_{w'} = \lambda^t_w. 
$$

In particular,
$$
\forall_\lambda w, \; \forall s, \; \forall_{\lambda^s_w} w' , \;
\lambda^s_{w'} = \lambda^s_w.
$$

Since\pageoriginale $\mathscr{C}^s$ is
$\lambda$-strong. $\forall_\lambda w$ $\lambda^s_w$ is disintegrated
by $(\lambda^s_{w'})_{w' \in\Omega}$ with respect to
$\mathscr{C}^s$. Since $\forall_\lambda w$, $\forall s$,
$\forall_{\lambda^s_w} w'$, $\lambda^s_{w'} = \lambda^s_w$,
$\forall_\lambda w$, $\forall_\lambda w$, $\forall s$, $\lambda^s_w$
is disintegrated with respect to $\mathscr{C}^s$ by the constant
measure valued function $w' \to \lambda^s_w$. Hence, by proposition
(\ref{part2:chap7}, \S\ \ref{part2:chap7:sec1},
\ref{part2:chap7:prop95}), $\forall_\lambda w$, $\forall s$,
$\lambda^s_w$ is 
ergodic on $\mathscr{C}^s$. Hence $\forall_\lambda w$, $\forall s$,
$\forall t \leq s$, $\lambda^s_w$ is ergodic on $\mathscr{C}$. Hence,
again by the proposition (\ref{part2:chap7},
\S\ \ref{part2:chap7:sec1}, \ref{part2:chap7:prop95}), $\forall_\lambda w$, 
$\forall s$, $\forall t \leq s$, $\lambda^s_w$ is disintegrated with
respect to $\mathscr{C}^t$ by the constant measure valued function $w'
\to \lambda^s_w$. 
 
Thus, we have proved that $\forall_\lambda w$, $\forall s$,
$\lambda^s_w$ has with respect to $(\mathscr{C}^t)_{t \in \mathbb{R}}$
a disintegration given by $(t, w') \to \lambda^t_{w'}$ for $t \geq s$
and $(t,w') \to \lambda^s_w$ for $t <s$.

We have to check only that this is a regular disintegration. Since
$(t,w) \to \lambda^t_w$ is a regular disintegration of $\lambda$, we
have to check only the right continuity at the point $s$. i.e., we
have to only prove that 
$$
\forall \; B \in \mathscr{O}, \; \forall_\lambda w, \; \forall s, \;
\forall_{\lambda^s_w} w', \; \lim\limits_{t \downarrow s}
\lambda^t_{w'}(B) = \lambda^s_w(B). 
$$

But this follows immediately from the fact that $\forall_\lambda w'$,
$\lim\limits_{t \downarrow s} \lambda^t_{w'} (B) = \lambda^s_{w'} (B)$
and 
$$
\forall_\lambda w, \; \forall s, \; \forall_{\lambda^s_w} w', \;
\lambda^s_{w'} = \lambda^s_w.
$$
\end{proof}

\begin{corollary}\label{part2:chap7:coro112}
Let $f$ be a regular supermartingale adapted to $(\mathscr{C}^t)_{t
  \in \mathbb{R}}$. Then $\forall_\lambda w$, $\forall s$, $f$ remains
a supermartingale on the set $\Omega \times [s,+\infty)$, for the
  measure $\lambda^s_w$. 
\end{corollary}

\begin{proof}
From the above theorem, we have $\forall_\lambda w$, $\forall s$,
$\forall$ pair $\{t,t'\}$ with $s \leq t \leq t'$,
$\forall_{\lambda^s_w} w'$.
$$
E^{\lambda^s_w} (f^{t'} \mid \mathscr{C}^t) (w') = \lambda^t_{w'}
(f^{t'}) 
$$
where $E^{\lambda^s_w} (f^{t'} \mid \mathscr{C}^t)(w')$ stands for the
value at $w'$ of the conditional expectation  of
$f^{t'}$\pageoriginale with respect to $\mathscr{C}^t$ for the measure
$\lambda^s_w$.

Because $f$ is a supermartingale, $\forall_\lambda w$,
$\lambda^t_w(f^{t'}) \leq f^t(w)$. Hence
$$
\forall_\lambda w, \forall s, \forall_{\lambda^s_w} w' \cdot
\lambda^t_{w'} (f^{t'}) \leq f^t(w'). 
$$

Hence $\forall_\lambda w$, $\forall s$, $\forall$ pair $\{t,t'\}$ with
$s \leq t \leq t'$, $\forall_{\lambda^s_w} w'$, $E^{\lambda^s_w}
(f^{t'} \mid \mathscr{C}^t) (w') \leq f^t (w')$. 

This shows that $\forall_\lambda w$, $\forall s$, $f$ is a
supermartingale on $\Omega \times [s, + \infty)$ for the measure
  $\lambda^s_w$. 

The regularity of $f$ with respect to $\lambda^s_w$ follows from that
of $f$ with respect to $\lambda$.
\end{proof}

\begin{rem}\label{part2:chap7:rem113}
Under the assumptions that $\mathscr{C}^t$ is $\lambda$-strong
$\forall$ $t \in \mathbb{R}$ and $\Omega$, compact metrizable, note
that the above corollary is stronger than theorem (\ref{part2:chap6},
\S\ \ref{part2:chap6:sec2}, \ref{part2:chap6:thm91}).  
\end{rem}


\begin{thebibliography}{99}
\bibitem{key1} Blumenthal, R. M. and Getoor, R.K. :\pageoriginale Markov Processes
  and Potential theory. (Acad. Press, 1968).

\bibitem{key2} Dieudonn\'e, J. : Sur le th\'eor\`eme de Lebesgue
  Nikodym III. (Ann. Univ. Grenoble Sec. Sci. Math. PHys. (N-S) 23,
  (1948), 25-53). 

\bibitem{key3} Doob, J. L. : Stochastic Processes. (John Wiley \& Sons
  1967).

\bibitem{key4} Hille, E, and PHillips, R.S. : Functional Analysis and
  Semi-Groups. (Amer. Math. Soc. Colloq. Publications Vol. XXXI,
  1957). 

\bibitem{key5} Jirina, M. : On Regular Conditional
  Probabilities. (Czech. Math. Journal Vol. 9, (1959), 445-450).

\bibitem{key6} Maharam, D. : On a theorem of Von
  Neumann. (Proc. Amer. Math. Soc. Vol 9(1958), 987-994). 

\bibitem{key7} Meyer, P.A. : Probability and Potentials. (Blaisdell
  publishing Company, 1966).

\bibitem{key8} Schwartz, L. : Surmartingales r\'egli\`eres \`a valeurs
  mesures et d\'esint\'egration r\'eguli\`eres d'une
  mesure. (J. Anal. Math. Vol XXVI (1973) 1-168).

\bibitem{key9} Schwartz. L. : Radon measures on arbitrary topological
  spaces and cylindrical Measures. (Oxford Univ. Press. 1973).

\end{thebibliography}
