 \chapter{}\label{chap3}
 
 There\pageoriginale are obvious analogues of the function spaces
 introduced at the 
 begining of Chapter \ref{chap1} for vector and  matrix valued functions. We
 shall use the same notations for these spaces and norms and scalar
 products on them. For example,  for two vectors $u=(u_j)$ and $v=
 (v_j)$ in $\mathscr{E}^s_{L^2} [0,  h]$,  we define 
 $$
 (u(t), ~ v(t)) = \sum\limits_j (u_j(x, t), ~ v_j(x, t))_s.
 $$
 

 \section{Energy inequalities for symmetric hyperbolic systems}\label{chap3-sec1}%%% 1
  
  Let $A_k(x, t)$ and $B(x, t)$ be matrices (of order $N$) of
  functions. Consider the following system of first order equations. 
\begin{equation}
\frac{\partial}{\partial t} u - \sum A_k (x, t) \frac{\partial}{\partial
  x_k} u -B (x, t) u = f \tag{1.1} \label{chap3-eq1.1}
 \end{equation} 
 where $A_k (x,  t)$ are Hermitian matrices. Suppose that 
$$
A_k (x,  t)
 \in \mathscr{B}^1 [0, h], B (x,t) \in \mathscr{B}^0
             [0,h]\text{~ and~ } f \in  \mathscr{D}^0_{L^2} [0,
               h]. 
$$  

\setcounter{proposition}{0}
\begin{proposition}[Friedrichs]\label{chap3-sec1-prop1} % proposition 1
 Let $u$ be a solution of \eqref{chap3-eq1.1} belonging to
  $\mathscr{D}^1_{L^2} [0,  h]$. Then we have 
\begin{equation}
||u(t)|| \leq \exp (\gamma t)\cdot ||u(0)|| + \int\limits^t_0 \exp
(\gamma (t-s))|| f(s)|| ds \tag{1.2}\label{chap3-eq1.2} 
\end{equation}
where $\gamma$ is a constant depending only on the bounds of $A_k$,
$B$.  
\end{proposition} 

\begin{proof}
Differentiating $|| u(t) ||^2 = u(t)$,  $u(t))$ with respect to $t$ we
have the identity 
$$
\frac{d}{dt}||u (t)||^2 = \left(\frac{du}{dt}(t),  u(t)\right) + \left(u(t),
\frac{du}{dt}(t)\right). 
$$

Since\pageoriginale $A_k$ are Hermitian matrices and since $u
\in \mathscr{D}^1_{L^2} [0, h]$ we obtain from \eqref{chap3-eq1.1} the
relation   
 \begin{align*}
\left(u,  \frac{du}{dt}\right) & =\sum_k \left(u,  A_k \frac{\partial
  u}{\partial x_k}\right) + (u,  Bu + f)\\ 
& = - \sum_k \left(\frac{\partial}{\partial x_k} (A_k u), u\right) + (u,  Bu +
   f)\\ 
& = - \left\{\sum_k \left(A_k \frac{\partial u}{\partial x_k},
   u\right) + \sum_k \left(\frac{\partial A_k}{\partial x_k} u,
   u\right)\right\} +(u,  Bu+ f).   
 \end{align*} 

 Hence $\dfrac{d}{dt}|| u(t) ||^2 = - \sum\limits_k \left(\dfrac{\partial
   A_k} {\partial x_k}\cdot u,  u\right)+ 2 \re (u,  Bu + f)$ 
 $$
 \leq 2 \gamma ||u||^2 + 2 || u || || f ||
 $$
where $\gamma$ is a constant depending only on the bounds of
$\dfrac{\partial A_k}{\partial x_k}$ and $B$.  Hence  
 $$
 \frac{d}{dt} || u(t) || \leq \gamma || u(t) || + ||f||
 $$
which on integration with respect to $t$ yields the required inequality
$$
||u(t)|| \leq \exp (\gamma t) \cdot || u(0)|| + \int\limits^t_0 \exp
(\gamma (t-s)) || f(s) || ds. 
$$

The energy inequality involves the $L^2$-norm of the solution $u$ of
the system in the $x$-space. It is possible to derive the energy
inequality under the weaker assumption that $u \in L^2 (0,
h]$. For this we use the method of regularization in the $x$-space
  of the function $u$ by mollifiers introduced by Friedrichs. We
  recall the notion of mollifiers and a few of their properties which
  we need. 
\end{proof}

\begin{defi*}
{\em Mollifiers of Friedrichs}. Let $\varphi \in \mathscr{D}$
with its support contained in the unit ball $\left \{ |x| < 1 
\right\}$ such that $\varphi (x) \geq 0$ and $ \int \varphi (x) \, dx
=1$.\pageoriginale Then for a $ \delta > 0$ define  
$$
\varphi_\delta (x)= \frac{1}{\delta^n} \varphi
\left(\frac{x}{\delta}\right).   
$$\pageoriginale
are called mollifiers.
\end{defi*}

\begin{proposition}[Friedrichs]\label{chap3-sec1-prop2} % \proposition 2 
 Let $a \in \mathscr{B}^1$ and $u \in L^2$. Denote by
 $C_\delta $ the commutator defined by  
\begin{align}
C_\delta u & = \varphi_\delta  *  \left(a(x)\frac{\partial u}{\partial
  x_j}\right)- a(x) \left(\varphi_\delta  *  \frac{\partial
  u}{\partial x_j}\right)\\  
& = \left[\varphi_\delta *,  a \frac{\partial}{\partial x_j}\right]
u. \tag{1.3} \label{chap3-eq1.3}
\end{align}
\end{proposition}

Than we have 
\begin{enumerate}[(i)]
\item $||C_\delta u || \leq c|| u || $ where $c$ is a constant
  depending only on $\varphi $ and a  

\item $C_\delta u \to 0$ in $L^2$ as $\delta \to 0$. 
\end{enumerate}

Before proving this proposition it will be useful to prove the
following 

\setcounter{lemma}{0}
\begin{lemma}\label{chap3-sec1-lem1}% \lemma 1
If $u \in L^p$ then $\varphi_\delta * u \to u$ in $L^p $ as
$\delta \to 0$. More generally,  if $u \in \mathscr{D}^m_{L^p}
(m=0, 1, \ldots )$ then $\varphi_\delta * u \to u$ in
$\mathscr{D}^m_{L^p}$.  
\end{lemma}

\begin{proof}
Let $\psi_\delta = \varphi_\delta * u-u$. Since $\int \varphi_\delta
(x) dx =1$ we have 
$$
\psi_{\delta}(x)= \int \varphi_\delta (x-y) u (y) dy- u(x)= \int
\varphi_\delta (x-y)(u(y)- u(x))dy.
$$  
\end{proof}

If $p'$ is such that $\dfrac{1}{p}+ \dfrac{1}{p'} = 1$ by H\"older's
inequality we have 
$$
| \psi_\delta (x) | \leq \left(\int \varphi_\delta
(x-y)dy\right)^{1/p'} \left(\int \varphi_\delta (x-y) \big| u(y)-u(x)
\big|^p dy\right)^{1/p}.  
$$

Here we use $\varphi_\delta
=\varphi^{\frac{1}{p'}}_\delta \cdot \varphi^{\frac{1}{p}}_\delta$. Now
since $\int \varphi_\delta (x-y)dy =1$ we have $\int \big|\psi_\delta
(x) \big|^p dx \leq \iint \varphi_\delta (x-y)| u(y)-u(x)|^p dx \, dy=
\iint\limits_{|x-v| \leq \delta} \varphi_\delta (x-y)| u(y)-u(x)|^p dx
\, dy$. 
By a change of variables $x'= x-y$ we obtain 
$$
\int |\psi_\delta (x)|^p
dx \leq \int_{|x'| < \delta} \varphi_\delta (x') dx' \int
|u(y)-u(x'-v)|^p dy.
$$\pageoriginale 

If $\varepsilon (\delta)$ denotes $\sup \int\limits_{|h| \leq
  \delta}|u (y)-u(y + h)|^p dy$ then $\int |\psi_\delta (x)|^p
dx \leq \varepsilon (\delta)$ which tends to 0 as $\delta \to 0$. The second
part is an immediate consequence of this result since
$\left(\dfrac{\partial}{\partial x}\right)^\nu (\varphi_\delta \star u)=
\varphi_\delta \star \left(\dfrac{\partial}{\partial x}\right)^\nu u $
for $\nu |
\leq m$ if $u \in \mathscr{D}^m_{L^p}$. 

\medskip
\noindent
{\bf Proof of Proposition \ref{chap3-sec1-prop2}:}
\begin{equation}
C_\delta u(x)=- \int \varphi_\delta (x-y)(a(x)- a(y)) \frac{\partial
  u}{\partial y_j} (y)dy \tag{1.4}\label{chap3-eq1.4} 
\end{equation}
where the integral on the right is taken in the sense of
distributions. Now we have 
\begin{equation}
C_\delta u= \int \frac{\partial}{\partial y_j} \left\{ \varphi_\delta
(x-y) (a(x)- a(y))\right\} u(y) dy \tag{1.5}\label{chap3-eq1.5} 
\end{equation}
where the integral is taken in the usual sense. In fact the integral
in \eqref{chap3-eq1.5} is equal to  
$$
- \int \frac{\partial a}{\partial y_j}(y) \varphi_\delta (x-y) u(y)dy 
+ \int (a(x)- a(y) \frac{\partial \varphi_\delta}{\partial y_j}(x-y) u
(y) dy; 
$$
we now note that  
$$
|a(x)-a(y)| \leq |a|_\mathscr{B} \delta_1 |x-y|,  \int |x-y||
\frac{\partial \varphi_\delta}{\partial y_j}(x-y)| dx \leq c, 
$$ 
with $c$ independent of $\delta$. Thus it follows from the
Hausdorff-Young theorem that the function represented by the above
integral is majorized in the $L^2$-norm by $c_1 |a|_{\mathscr{B}^1}
||u||$. Now we see that the integration by parts is  justified. In
fact,  the 
two integrals are equal for $u \in \mathscr{D}$. Then for any
$u \in L^2$ the equality is proved by taking a sequence $u_j
\varepsilon \mathscr{D}$ having\pageoriginale for its limit $u$ in
$L^2$. Then 
$C_\delta u_j$ tends to the second integral in the sense of $L^2$. On
the other hand $C_\delta u_j \to C_\delta u$ in the sense of
distributions. This proves (i). 

Since $(a(x)-a(y)) \varphi_\delta (x-y)$ considered, for fixed $x$,
as a function of $y$ has compact support we see that  
$$
\int \frac{\partial}{\partial y_j} \left\{ (a(x)- a (y)) 
\varphi_\delta (x-y) \right\} dy=0. 
$$

Hence 
\begin{align*}
C_\delta u(x) & = \int \frac{\partial}{\partial y_j}\left\{ (a(x)- a
(y)) \varphi_\delta (x-y) \right\} (u(y)- u(x)) dy\\ 
& = - \int \frac{\partial a}{\partial y_j}(y) \varphi_\delta (x-y)
(u(y)-u(x))dy\\
&\quad - \int (a(x)-a(y))\frac{\partial \varphi_j}{\partial x_j}
(x-y)
 (u(y)- u(x)) dy\\ 
& = \phi_1 (x) + \phi_2 (x),  \text{ say}. 
\end{align*}

Now as in the proof of lemma \ref{chap3-sec1-lem1},  we see that
$$
|| \phi_i (x)|| \to 0 \text{ as } \delta \to 0 (i=1,  2). 
$$

In fact, for instance, 
$$
|\phi_2 (x)| \leq | a |_{\mathscr{B}^1} \int |x-y|| \frac{\partial
  \varphi_\delta (x-y)}{\partial x_j}||u(y)- u(x)|dv.  
$$

Since $\int |x| |\dfrac{\partial \varphi_\delta}{\partial x_j}| dx
\leq c$ (independent of $\delta $) we obtain the desired property by
the same reasioning as earlier. As an immediate consequence,  we have 

\setcounter{corollary}{0}
\begin{corollary}\label{chap3-sec1-coro1}% corollary 1
If we assume $a \in \mathscr{B}^m$ and $u \in
\mathscr{D}^m_{L^2}$ in proposition \ref{chap3-sec1-prop2} then 
\begin{enumerate}[\rm(1)]
\item $|| C_\delta u ||_{\mathscr{D}^m_{L^2}}  \leq c ||u
  ||_{\mathscr{D}^m_{L^2}}$,\pageoriginale 

\item $ C_\delta u \to 0$ in $\mathscr{D}^m_{L^2} $ as $\delta \to
  0$, $m = 1, 2,\ldots $ 
\end{enumerate}
\end{corollary}

\begin{proposition}[Friedrichs]\label{chap3-sec1-prop3}%prop 3
 Let $u$ be a solution of \eqref{chap3-eq1.1} belonging\break $L^2 [0,
   h]$ then the 
 inequality \eqref{chap3-eq1.2} 
$$
|| u (t) || \leq \exp (\gamma t)|| u (0) || + \int\limits^{t}_{0}
\exp (\gamma (t-s))|| f (s)|| ds, 
$$
holds, where $\gamma$ is the same constant as in prop. \ref{chap3-sec1-prop1}. 
\end{proposition}

\begin{proof}%proo 0
By regularizing $u$ in the $x$-space by mollifiers $\varphi_\delta$ we
obtain a function belonging to $\mathscr{D}^1_{L^2} [0, h]$ to which
we can apply the Prop. \ref{chap3-sec1-prop1}. Let $u_\delta =
\varphi_\delta *_{(x)} u$. Then  
$$
\frac{\partial u_\delta}{\partial t} = \frac{\partial}{\partial t} 
(\varphi _\delta *_{(x)} u) = \varphi_\delta *_{(x)} \frac{\partial
  u}{\partial t}. 
$$

Form the equation \eqref{chap3-eq1.1} we obtain the following equation
for $u_\delta$ 
$$
\frac{\partial u_\delta}{\partial t} = \sum\limits_{k} \varphi_\delta
*_{(x)} \left(A_k \frac{\partial u}{\partial x_k}\right) 
+ \varphi_\delta *_{(x)} Bu +
\varphi_\delta * f,  
$$
that is 
\begin{align*}
\frac{\partial u_\delta}{\partial t} & = \sum\limits_{k} A_k
\frac{\partial u_\delta}{\partial x_k} + Bu_\delta + f_\delta +
C_\delta u\\ 
\text{where } C_\delta u & = \sum \bigg\{ \varphi_\delta *_{(x)}
(A_k \frac{\partial u}{\partial x_k}) - A_k (\varphi _\delta *_{(x)}
\frac{\partial u}{\partial x_k}) \bigg\}\\
&\quad + \bigg\{  \varphi_\delta
*_{(x)} Bu-R (\varphi_\delta * u)\bigg\}\\ 
& = \sum  \left[ \varphi_\delta *_{(x)}, A_k \frac{\partial}{\partial
    x_k}\right] u + \left[\varphi_\delta *_{(x)}, B\right] u. 
\end{align*}

Applying prop. \ref{chap3-sec1-prop1} to the equation in $u_\delta$ we
obtain since $u_\delta \subset \mathscr{D}'_{L^2} [0,h] $
$$
|| u_\delta (t) || \leq
\exp (\gamma t ) || u_\delta (0) || + \int \exp (\gamma (t-s)) \int ||
f_\delta (s) || + || C_\delta (u) (s)|| ds. 
$$\pageoriginale

Now it follows from the Friedrichs lemma (Prop. \ref{chap3-sec1-prop2}) that
$$
|| (C_\delta u) (s) || \leq c || u (s) || 
$$ 
where $c$ is a constant independent of $\delta$ and $C_\delta u(s) \to
0$ as $\delta\to 0$. By Lebesgue's bounded convergence theorem it
follows that   
$$
\int\limits^{t}_{0} \exp (\gamma (t-s))\cdot (|| f_\delta (s)|| + ||
(C_\delta u) (s) ||) \, ds  
$$
tends to $\int\limits^{t}_{0} \exp (\gamma (t-s)) || f(s) ||ds$. Thus
passing to the limits as $\delta \to 0$ we obtain 
$$
||(u (t) || \leq \exp (\gamma t ) || u(0) || + \int\limits^{t}_{0}
\exp (\gamma (t-s) || f (s) || ds. 
$$
\end{proof}

\section{Some remarks on the energy inequalities}\label{chap3-sec2}%sec 2

In the previous section we obtained estimates for the solutions of
symmetric hyperbolic systems in $L^2$-norm in terms of the $L^2$-norms
of the initial values and of the second member. One can ask whether
such estimates can be proved in the maximum norm and $L^p$-norm for $p
\neq 2$. Littman \cite{key1} has proved that such an energy inequality cannot
hold in the $L^p$-norm for $p\neq 2$. The existence of such an
inequality with the maximum norms of functions and of their
derivatives is related to the propagation of regularity, a form of
Huygens principle for differentiablity. For instance, if $u(0)$ is
$m$ times continuously differentiable is $u(t)$ also $m$ times
continuously differentiable? In general an energy inequality in the
maximum norm does not hold as we shall show by a counter example due
to Sobolev. However, when the dimension of the\pageoriginale $x$-space
is one an 
inequality for solutions of strongly hyperbolic systems is valid in
the maximum norm. This result is due to $T$. Haar. We indicate his
result briefly. 

\smallskip
\noindent
\textbf{\textit{Haar's inequality}}. Consider the system of equations
of the first order  
\begin{equation}
\frac{\partial u}{\partial t} - A (x,t) \frac{\partial u}{\partial x}
- B (x, t) u = f \tag{2.1}\label{chap3-eq2.1}  
\end{equation}
where the matrix $A(x, t)$ is such that det $(\lambda I-A)$ has real
and distinct roots. Then we have the inequality 
\begin{equation}
| u (t) |_0 \leq c (T) \bigg\{ | u (0) |_0 + \sup\limits_{0 \leq t
  \leq T}| f (t) |_0 \bigg\} \tag{2.2}\label{chap3-eq2.2} 
\end{equation}
where $| u (t)|_0 = \sup\limits_{x \in D_0} | u (x, t) |$, $D$  being a
neighbourhood of the origin and $D_0 = D \cap \{t = 0\}$. 

In fact, let $\lambda _1 (x, t), \ldots, \lambda_N (x, t)$ be the
roots of det $(\lambda I-A) = 0$. $A(x, t)$ being diagonalizable there
exists a non-singular matrix $N(x, t)$ such that  
$$
N(x, t) A (x, t) = D (x, t) N (x, t) 
$$ 
where $D(x, t)$ is the diagonal matrix
$$
\begin{pmatrix} 
\lambda_1 (x, t)  & & 0\\
&\ddots&\\ 
0 & & \lambda_{N} (x, t)\\
\end{pmatrix}
$$
and such that $|\det N (x, t)| > \delta > 0$. We have the identity 
$$
\frac{\partial}{\partial t} (N u) = \frac{\partial N}{\partial t} u +
N \frac{\partial  u}{\partial t}. 
$$

Substituting for $\dfrac{\partial u}{\partial t}$ from the given
system the right hand side becomes 
\begin{align*}
\frac{\partial N}{\partial t} u + N.A \frac{\partial u}{\partial x} +
N.B u + Nf & = \frac{\partial N}{\partial t} u + DN
\frac{\partial}{\partial x} u + N.B u + N.f\\ 
& = D \frac{\partial }{\partial x}. (Nu) + B_1 u + N.f,
 \end{align*}\pageoriginale  
where $B_1 = - D \dfrac{\partial N}{\partial x} + NB +
 \dfrac{\partial N}{\partial t}$. If $B_2$ denotes $B_1 N^{-1}$ then $v
 = Nu$ satisfies the system. 
 $$
 \frac{\partial v}{\partial t} = D \frac{\partial v}{\partial x} + B_2
 v + Nf 
 $$
 which can be reduced to an integral equation of the Volterra type and
 then can be solved by successive approximation. Let $(x_0, t_0)$ by
 any point in the $(x, t)$-plane. Let $D$ be the domain enclosed by
 $(t = 0)$, the characteristic curves passing through $(x_0, t_0)$ and
 having the maximum and minimum slopes. Let $D_0 = D \cap (t =
 0)$. One can then show from the integral equation that 
 $$
 | u (x_0, t_0) |\leq c \left\{ \sup\limits_{x \in D_0} | u (x, 0) |+
 \sup\limits_{(x, t) \in D}| f (x, t) |\right\} 
 $$ 
 with a constant $c$ independent of $u$.
 
 That the energy inequality with the supremum norms does not hold in
 general in shown by the following counter example due to
 Sobolev.

\medskip
\noindent
 \textit{Counter example (Sobolev)}. We consider the wave
 operator  
 \begin{equation}
\square \equiv \frac{\partial^2}{\partial t^2} - \sum\limits^{3}_{j=1}
\frac{\partial^2}{\partial x^2_j} \tag{2.3}\label{chap3-eq2.3}
 \end{equation} 
 in $\underbar{R}^3$. We set 
 $$
 E_1 (t,u) = \sup\limits_{x} \left\{\left| \frac{\partial u}{\partial
   t}\right| + \sum_{j} \left|\frac{\partial u}{\partial x_j}\right| \right\}. 
 $$

 We shall show that if $t_0 > 0$ then an inequality
 $$
 E_1 (t_0, u) \leq c E_1 (0, u) 
 $$\pageoriginale
does not hold, which proves that the differentiability of the solution
is not propagated in the $t$-direction. For this purpose, let $\Gamma
(x, t)$ be a fundamental solution of $\square$ such that  
 $$
 \Gamma (x, 0) = 0, \; \frac{\partial \Gamma}{\partial t} (x, 0) =
 \delta   
 $$
 $\delta$ being the Dirac distribution. Let $\varphi \in
 \mathscr{D}$. For an $\in > 0$ define  
 $$
 \varphi^{(\in)} (x) = \varphi \left(\frac{x}{\in}\right). 
 $$

 Extending $\Gamma (x, t)$ to the whole space by setting 
 \begin{align*}
\tilde{\Gamma} (x, t) & = \Gamma (x, t) \text{ for } ~ t \geq 0\\ 
&= - \Gamma (x,-t) ~\text{ for } ~ t \leq 0.
 \end{align*} 

 We obtain a distribution solution of $\dfrac{\partial^2}{\partial
   t^2} - \sum\limits_{j} \dfrac{\partial^2}{\partial x^2_j}$ in the
 whole space $(-\infty < t < \infty)\times \underbar{R}^3$. Setting 
 $$
 u_\in (x, t) = \tilde{\Gamma} (x, t-to) *_{(x)} \varphi^{(\in)} (x) 
 $$
 we obtain a solution of the homogeneous equation which satisfies 
 $$
 \frac{\partial}{\partial t} u_\in (x, to) = \frac{\partial
   \tilde{\Gamma}}{\partial t} (x, t_0 -to) *_{(x)} \varphi^{(\in)} (x)=
 \delta * \varphi^{(\in)} (x) = \varphi^{(\in)} (x) 
 $$
 and $\dfrac{\partial}{\partial x_j} u_\in (x,t_0) = \tilde{\Gamma}
 (x, 0) *_{(x)} \dfrac{\partial}{\partial x_j} \varphi^{(\in)} (x) =
 0$. 
 
Hence $E_1 (t_0, u_\in) = \sup\limits_{x} |\varphi^{(\in)} (x)|$.

On the other hand we first observe that $\Gamma (x, t)$ can be taken
to be $\dfrac{1}{4 \pi t} \delta_{|x|-t}$. Let us choose $a \varphi
\in \mathscr{D}$ with its support contained in the unit ball in
$\underbar{R}^3$ such that $\varphi (0) = 1$ and $|\varphi (x)| \leq
1$. Then $|\dfrac{\partial}{\partial x_j} \varphi^{(\in)} (x) | \leq
\dfrac{\gamma}{\in}$.\pageoriginale Thus for $t = 0$ we see that  
$$
\frac{\partial}{\partial t} u_\in (x, 0) = + \frac{\partial
  \Gamma}{\partial t} (x, t_0) *_{(x)} \varphi^{(\in)} (x) = \left[
  \frac{\partial}{\partial t} (\frac{t}{4 \pi}) \int\limits_{|\xi|=1}
  \varphi^{(\in)} (x-t) ds_\xi ) \right]_{t=t_0} 
$$

Now since $\int\limits_{B_\in (x_0) \cap \{| \xi | =1 \}} ds_\xi =
O(\epsilon^2)$ it follows that $\dfrac{\partial}{\partial t} u_\epsilon (x, 0) =
O(\epsilon)$. We also have $\dfrac{\partial u_\epsilon}{\partial x_j} (x, 0) = O
(\epsilon)$ and so $E_1 (0, u_\epsilon) = O(\epsilon)$ which together
with $E_1 (t_0, u_{\epsilon}) = \sup | \varphi^{(\epsilon)}(x)| =1$ shows
that an energy inequality 
of the type $E_1 (t_0, u_{\epsilon}) \leq cE_1(0, u_{\epsilon})$ does not hold. 


\section{Singular integral operators}\label{chap3-sec3}%sec 3

In this section we introduce the notion of singular integral operators
and recall some of their properties which will be useful in the study
of the existence and uniqueness of solutions of the Cauchy
problem. The following considerations lead us to the notion of
singular integral operators. 

Consider the system of equations 
\begin{equation}
\frac{\partial u}{\partial t} - \sum A_k \frac{\partial u}{\partial
  x_k}- Bu = f \tag{3.1} \label{chap3-eq3.1}
 \end{equation} 
 where $A_k$ and $B$ are matrices whose entires are constants and\,$f
 \in L^2 [0. h]$. We assume \eqref{chap3-eq3.1} to be strongly hyperbolic in the
 sense that the roots of the equation. det $(\lambda I-A.\xi)  = 0$
 are real and distinct. Let the roots be $\lambda_1 (\xi), \ldots,
 \lambda_N (\xi)$ for $\xi \neq 0$. We have the following. 

\setcounter{lemma}{0} 
\begin{lemma}\label{chap3-sec3-lem1}%lemm 1
There exists a non-singular matrix $N(\xi)$ which is homogeneous of
degree zero and bounded such that  
  \begin{enumerate}
\renewcommand{\labelenumi}{\rm(\theenumi)}
\item $| \det N(\xi) | \leq \delta> 0 $ for all $\xi$.

\item $N (\xi) (A.\xi) = D (\xi) N (\xi)$ where $D(\xi)$ is the
  diagonal matrix  
 $$
 D(\xi)=
 \begin{pmatrix}
\lambda_1 (\xi) &  & 0 \\
 &\ddots\\
 0 & &\lambda_{N}(\xi) 
 \end{pmatrix} 
 $$\pageoriginale
  \end{enumerate}
\end{lemma}

 Assume that there exists a solution $u \in L^2 [0, h]$. Then,
 denoting for every fixed $t$, the Fourier transform of $u$ in the
 $x$-space $\hat{u}(\xi, t)$, we obtain the following system of
 ordinary diffenential equations: 
 \begin{equation}
\frac{d}{dt} \hat{u} (\xi, t) = (2 \pi i A.\xi + B) \hat{u} (\xi, t) +
\hat{f} (\xi, t)\tag{3.2} \label{chap3-eq3.2}
 \end{equation} 

 Multiplying both sides of this system by $N(\xi)$ and using
 lemma \ref{chap3-sec3-lem1} we have  
 $$
 \frac{d}{dt} (N\hat{u}) (\xi,t) = (2 \pi i D (\xi) \cdot N (\xi) + N (\xi) 
 B) \hat{u} (\xi,t) + N (\xi) \hat{f} (\xi, t). 
 $$
 $v (\xi, t) = N (\xi) \hat{u} (\xi, t)$ satisfies the system of
 equations 
 \begin{equation}
\frac{dv}{dt} (\xi, t) = (2 \pi i D (\xi) + B' (\xi)) v (\xi,t) + N
(\xi) \hat{f} (\xi, t ), \text{~ where~ } \tag{3.3} \label{chap3-eq3.3}
 \end{equation}
 $$
 B' (\xi) = N (\xi) BN (\xi)^{-1}.\quad \text{Now}  
 $$
 \begin{align*}
\frac{d}{dt} || v (\xi, t)||^2 & = \int \left( \frac{dv}{dt} \cdot  \bar{v}
+ v \cdot \frac{\overline{dv}}{dt} \right) d\xi\\ 
& = \int \bigg\{ 2 \pi ( i D (\xi ) v \cdot  \bar{v}+ v \cdot
\overline{iD (\xi ) v)} + 2 \re (B' v,\bar{v}) \\
& \qquad \qquad + 2 \re N(\xi )
\bar{f} \cdot \bar{v}\bigg\}d \xi \\ 
& = 2 \int \re (B' (\xi ) v \cdot \bar{v} + N (\xi ) \hat{f}
\cdot\bar{v}) (\xi,t) d \xi.  
 \end{align*} 

 Because $N(\xi)$ is bounded and condition (1) of lemma
 \ref{chap3-sec3-lem1} holds. The operators $B'$ is bounded and 
 hence  
 \begin{align*}
\frac{d}{dt} || v (\xi, t)||^2  & \leq 2 \gamma || v || ^2 + 2 \re
(N(\xi ) \hat{f}, v)\\ 
& \leq 2 \gamma || v ||^2 + 2 || N(\xi ) \hat{f} || || v || 
 \end{align*}

 Thus\pageoriginale we obtain
 $$
 || v || \leq \exp (\gamma t) \cdot  || v (\xi, 0)|| + \int\limits^{t}_{0}
 \exp (\gamma (t-s)) || N (\xi) \hat{f} (\xi, s)|| ds. 
 $$ 

 By Plancheral's formula's formula we have    
 $$
 || v (\xi, t) || = || N (\xi) \hat{u} (\xi, t) || \leq c ||u  (t)
 ||. 
 $$
 and again since $N(\xi)$ has a bounded inverse by condition (1) we
 see that  
 \begin{equation}
 || u(t)|| \leq c(h) \bigg\{ || u (0) || + \int\limits^{t}_{0} || f
 (s) || ds \bigg\} \tag{3.4} \label{chap3-eq3.4}
 \end{equation}
 where $c$ is a constant depending only on $h$. 
 
 Now we look at this reasoning without explicitly using the notion of
 Fourier transforms. 
 
 $N(\xi)$ is homogeneous of degree 0 in $\xi$ and so the convolution
 operators $\mathscr{N} (x) *$ defines a bounded operator in the space
 $L^2$ since  
 $$
 || \mathscr{N} (x) * u || = || N (\xi) \hat{u} || \leq c || u ||  
 $$
 by Plancherel's formula. Here $\mathscr{N}(x)$ is the inverse Fourier 
 image of $N(\xi)$. Let $\mathscr{D}(x)$ be the distribution whose
 Fourier image is $D \left(\dfrac{\xi}{|\xi|}\right)$.   
Define the operators $\wedge$ by 
$$
(\widehat{\wedge u})  = |\xi| \hat{u}. 
$$

Then we obtain 
\begin{align*}
\frac{d}{dt} (\mathscr{N} (x) *_{(x)}u) &= 2\pi i \mathscr{D} (x)
*_{(x)} \wedge (\mathscr{N} (x) *_{(x)} u)\\
&\quad + \mathscr{N} (x) *_{(x)}
(Bu) + \mathscr{N} (x) *_{(x)} f. 
\end{align*}

In other words $v = \mathscr{N} *_{(x)} u$ satisfies the system 
$$
\frac{dv}{dt} = 2 \pi i \mathscr{D} *_{(x)} \wedge v + B_1 v +
\mathscr{N} *_{(x)}f, 
$$\pageoriginale
where $B_1 \in \mathscr{L} (L^2, L^2)$ because of condition
(1). Integrating with respect to $t$ in the interval $[0, t]$ we have
the inequality 
$$
|| \mathscr{N} *_{(x)} u || \leq \exp (\gamma t) || \mathscr{N}*_{(x)} u (x,
0) || + \int\limits^{t}_{0} \exp (\gamma (t-s)) || \mathscr{N} *_{(x)}
f (x,s) || ds  
$$
where $\gamma$ is a constant depending only on $A$ and $B$. But there
exists a constant $k$ (depending on $A$) such that  
$$
\frac{1}{k} || u(x, t) || \leq ||\mathscr{N} *_{(x)} u (x, t) || \leq k
||u(x, t) || 
$$
which gives an energy inequality for $u$.

Now in the case of systems with variable coefficients even though we
cannot apply Fourier transforms we may, however, write the system in a
form similar to \eqref{chap3-eq3.2} to which we can apply the above
method to get an energy inequality. For this purpose we introduce the
singular integral operators. 


\section{}\label{chap3-sec4}%sec 4

For a function $f \in L^2 (\underbar{R}^1)$ consider the integral
transform defined by  
\begin{equation}
g (x) = v. p. \int\limits^{\infty}_{-\infty} \frac{f(t)}{x-t}
\, dt. \tag{4.1}\label{chap4-eq4.1} 
 \end{equation} 
 M. Riesz \cite{key1} has proved that the Cauchy principal value defining $g$
 exists and $g \in L^2 (\underbar{R}^1)$. $f \to g$ is a continuous
 linear mapping of $L^2 (\underline{R}^1)$ into itself. In the
 language of the theory of distributions we can write $g =
 v.p. \left(\dfrac{1}{x}\right) * f.\; v.p. \left(\dfrac{1}{x}
 \right)$ is a tempered distribution whose\pageoriginale
 Fourier image is $\chi (\xi) = - \pi i $ for $\xi > 0$ and $\pi i$
 for $\xi < 0$. We observe that $\dfrac{1}{x}$ is homogeneous of
 degree $-1$ and has mean value 0. If $\hat{g}$ and $\hat{f}$ are the
 Fourier images of $g$ and $f$ respectively then $\hat{g} = \partial
 \chi \hat{f}$ and $|| g = \pi || f ||$ by Plancheral's formula. 
 
 Calderon and Zygmund \cite{key1} generalized this theory to functions on
 $\underbar{R}^n$. Let $N(x)$ be a homogeneous function of degree -n
 on $\underbar{R}^n (N (\lambda x) = \lambda^{-n} N(x))$ which is
 smooth in the complement of the origin and has mean value
 $\int\limits_{|x| =1} N(x) d \sigma _x = 0$. Then they proved that $g
 = v.p. N(x) * f \in L^p $ if $f \in L^p$. In particular $f \to g$ is
 a continuous linear map of $L^2$ into itself. This latter fact can be
 seen observing that $v.p. N(x)$ is a tempered distribution, its
 Fourier transform $h (\xi)$ is a homogeneous function of degree 0
 and has mean value $\int\limits_{|\xi| =1} h(\xi) d \sigma_\xi =
 0$. In this paragraph $d\sigma_{x}$ and $d\sigma_{\xi}$ stand for
 normalized volume element of the unit sphere; viz. $d\sigma_{x}=dS_{x}/\text{vol~}S$. 
 
 Conversely, given any homogeneous function $h(\xi)$ of degree $0$
 with mean value 0, if $\gamma (x)$ is its inverse Fourier image we
 can define an integral operators $\gamma *$ by 
 $$
 (\gamma * f) (x) = \int \exp (2 \pi ix \cdot \xi) h (\xi) \hat{f} (\xi)
 d\xi.  
 $$
 
 Now consider the differential operators  
 $$
 L \left(x, \frac{\partial}{\partial x} \right) = \sum a_j (x)
 \frac{\partial}{\partial x_j}. 
 $$

 For a function $u \in \mathscr{S}$ we can write  
 $$
 (Lu)(x) = \int \exp (2 \pi ix \xi) (\sum_j a_j (x) \xi_j) (2\pi i)
 \hat{u} (\xi) d \xi. 
 $$

Denote\pageoriginale $ h(x, \xi ) = 2 \pi i \sum\limits_{j} a_j (x) \xi_j /
|\xi|$. If we define  
$$
(Hf) (x) = \int \exp (2 \pi ix. \xi ) h(x, \xi ) \hat{f} (\xi ) d \xi 
$$
$H$ will be a bounded operator in $ L^2$. In fact, $H$ can be written 
\begin{equation*}
H f = 2 \pi i \sum a_j (x) (R_j *f) \tag{4.2}\label{chap4-eq4.2}
\end{equation*}
where $R_j$ is the inverse Fourier image of $^\xi j / |\xi|$. It
follows that  
$$
 \parallel Hf \parallel \leq 2 \pi \sum | a_j (x)|_0 \parallel R_j * f
 \parallel \leq (2 \pi \sum | a_j| _0 ) \| f \|. 
$$

Now $L$ can be written in the form 
$$
L u = H \wedge u.
$$

We introduce the notation used by Calderon-Zygmund \cite{key1}, \cite{key2}. 

Let $U$ be an open set in $\underbar{R}^n$. A function $u$ defined on
$U$ is said to satisfy a uniform Holder condition of order $ \beta (0
\leq \beta \leq 1)$ if for any $x$, $x' \in U$ we have  
\begin{equation*}
| u(x) - u (x') | \leq c |x-x' |^\beta. \tag{4.3}\label{chap4-eq4.3}
\end{equation*}
$c$ is called the H\"older constant for $u$. We shall denote by
$C_\beta (U)$, $\beta \geq 0$, the class of complex valued continuous   
bounded functions on $U$ with bounded continuous derivatives upto
order $[\beta]$ (the integral part of $\beta$) and with the
derivatives of order $[\beta]$ satisfying a H\"older condition of
order $\beta - [\beta]$. $\mathscr{E}_\xi (\underline{R}^n -\{0\})$
will denote the 
space consisting of complex valued functions $h(\xi)$, $\xi
\in \underbar{R}^n$, homogeneous of degree 0 and infinitely
differentiable in $\underbar{R}^n - \{ 0 \}$ with respect to $
\xi$. This space $ \mathscr{E}_\xi (\underbar{R}^n - \{ 0 \}
)$\pageoriginale is topologized by the family of seminorms defined by   
$$
p_s (h) = \sum\limits_{| \nu | \leq s} \sup_{| \xi | \geq 1} |
(\frac{\partial}{\partial \xi})^\nu h (\xi ) |. 
$$

We say that $h(x, \xi ) \in C_\beta ^\infty$, $\beta \geq 0$, if  
\begin{enumerate}
\renewcommand{\labelenumi}{(\theenumi)}
\item for $\beta = 0$ the function $x \to h(x, \xi ) \in
  \mathscr{E}_\xi (\underbar{R}^n - \{ 0 \})$ is continuous and
  bounded;  

\item for $0 < \beta < 1$, $h (x, \xi ) \in C^\infty _0$ and the
  function $ x \to h(x, \xi ) \in \mathscr{E}_\xi (\underbar{R}^n - \{
  0 \} )$ is uniformly H\"older continuous of order $ \beta$ in the
  sense that for any $\nu$ 
\begin{equation*}
\sup_{| \xi | \geq 1} \left|\left(\frac{\partial}{\partial
  \xi}\right)^\nu h(x, \xi ) - \left(\frac{\partial}{\partial
  \xi}\right)^\nu h(x', \xi ) \right| \leq c_\nu 
|x-x'|^\beta ; \tag{4.4} \label{chap4-eq4.4}
\end{equation*}

\item if $ \beta \geq 1$, $\left(\dfrac{\partial}{\partial
  x}\right)^\nu h(x, \xi ) \in C^\infty _0$ for $ | \nu| \leq \beta $ and
  $\left(\dfrac{\partial}{\partial x}\right)^\nu h(x, \xi ) \in C^\infty _{\beta
  - [\beta]}$ for $ |\nu | = [\beta]$.  
\end{enumerate}
$h (x, \xi )$ being a homogeneous function of $\xi$ can be expanded 
as a series in spherical harmonics. Let $Y_l(\xi)$ be a normalized
real spherical harmonic of degree $l$, that is such that  
\begin{equation*}
\int\limits_{| \xi | = 1} Y_l (\xi )^2 d \sigma_\xi = 1
\tag{4.5}\label{chap4-eq4.5} 
\end{equation*}
and $ Y_{lm}(\xi )$ be a complete orthogonal system of normalized
spherical harmonics of degree $l$. Then we can write  
\begin{equation*}
h(x, \xi ) = a_0 (x) + \sum _{ l \geq 1, m} a_{lm} (x) Y_{lm} ( \xi )
\tag{4.6} \label{chap3-eq4.6}
\end{equation*}
in terms of the spherical harmonics. Then 
\begin{equation*}
a_{lm}(x) = \int\limits_{|\xi | =1} h (x, \xi ) Y_{lm}(\xi )d
\sigma_\xi. \tag{4.7} \label{chap4-eq4.7}
\end{equation*}

Let\pageoriginale $ \widetilde{Y}_{lm}$ denote the inverse Fourier
image of $ Y_{lm}(\xi)$  
$$
\widetilde{Y}_{lm}(x) = \int e^{2 \pi ix. \xi} {Y}_{lm} (\xi ) d \xi.
$$

We define 
\begin{equation*}
(Hf) (x) = a_0 (x) f(x) + \sum_{l,m} a_{l,m}(x) (\widetilde{Y}_{lm}
  *f) (x). \tag{4.8} \label{chap3-eq4.8}
\end{equation*}

Now we have the following estimates due to Calderon and Zygmund: 
\begin{enumerate}[(a)]
\item $| Y_{lm}( \xi ) | \leq c l^{\frac{1}{2}(n-2)}$, $c$ being a
  positive constant; 

\item the number of distinct spherical harmonics $Y_{lm} (\xi)$ of
  degree $l$ is of the order $ l^{n-2}$;  

\item $| a_{lm} (x)| \leq c M l^{- \dfrac{3}{2} n}$ where $ M = \sup
  \limits_{\substack{x \in \underbar{R}^n, | \xi | \geq 1 \\ | \nu | 
      \leq 2n}} | (\dfrac{\partial}{\partial \xi})^\nu h (x, \xi ) | $. 

More generally we have the following sharper estimates. Let $ L$ be
the operator defined by  
$$ 
L(F) = |\xi|^2 (\triangle_\xi F) \text{ where } \triangle _\xi  =
\sum^n_{j =1} \left(\frac{\partial}{\partial \xi_j} \right)^2. 
$$
Then 
\begin{equation*}
a_{lm} (x) = (-1) ^r l^{-r}(l+n-2)^{-r} \int \limits_{|\xi | = 1}
L^r_\xi (h(x, \xi ) Y_{lm} (\xi ) d \sigma_\xi. \tag*{$(4.7)'$} 
\end{equation*}

From this it follows that 
\item $ | a_{lm}(x) | \leq c(n, r) M_{2r} l^{-2r + \dfrac{n}{2}}$ 

\noindent
where $M_{2r} = \sup\limits_{\substack {x \in \underbar{R}^n, | \xi|
    \geq 1 \\ | \nu | \leq 2r}} | (\dfrac{\partial}{\partial \xi})^\nu
h (x, \xi ) |$ 

\item $\sup \limits_{|  \xi |  \geq 1} | (\dfrac{\partial}{\partial
  \xi})^\nu  Y_{lm} ( \xi ) | \leq c (\nu, n) l^{\dfrac{1}{2}(n-2)+ |
  \nu |}$.\pageoriginale 
\end{enumerate}

These estimate show that the series defining $Hf$ is convergent in the
$ L^2$-sense.  

In fact, 
\begin{equation*}
\parallel Hf \parallel \leq (| a_0 (x) | + \sum | a_{lm}(x) |_o |y_{lm}
( \xi ) |_o ) \parallel f \parallel. \tag{4.9} \label{chap3-eq4.9}
\end{equation*}

From (a), (b) and (c) it follows that 
$$
\sum | a_{lm}|_o |Y_{lm}|_o \leq c M \sum_{l} l^{-\frac{3}{2}n+
  \frac{1}{2}(n-2) +n -2} = c M \sum_l l^{-3} < \infty. 
$$ 

Hence 
$$
\parallel H \parallel \leq c M, M \; \text{~ being defined in (c)}. 
$$ 

A singular integral operator was defined by Calderon and Sygmund by
the following equation 
\begin{equation*}
(Hu) (x) = a(x) u(x) + \int k(x, x-y) u (y) dy,
  \tag{4.10}\label{chap3-eq4.10} 
\end{equation*}
where $k(x, z)$ is a complex valued homogeneous function of degree
$-n$ in $z$, of class $\mathscr{E}$ in $ \underbar{R}^n -\{ 0 \}$ in
the $z$-variable for every fixed $x$ and the function $k(x, z)$ has
mean value zero in the $z$-space for every fixed $x$. Let us expand $ k
(x, z)$ in terms of spherical harmonics:  
$$
k(x, z) = \sum a_{lm}(x) Y_{lm} (z') |z|^{-n}, \quad z' = \frac{z}{| z |}\;, 
$$
where $a_{lm} (x) = \int \limits_{| z' | = 1} k (x, z' ) d \sigma _{z'}$. 

Then, taking into account the fact that 
$ \mathscr{F} [Y_{lm}(z' )|z|^{-n} ] = \gamma_1 Y_{lm} (\xi )$,\pageoriginale
$\gamma_1$ being a constant, we define the symbol $\sigma (H)$ as  
\begin{equation*}
\sigma (H) = a_0 (x) + \sum a_{lm} (x) \gamma_1 Y_{lm} (\xi
). \tag{4.11}\label{chap3-eq4.11} 
\end{equation*}

We start from this $\sigma(H)$ in our definition. However the two
definitions are identical since there exists a one to one linear
mapping $\sigma$ of the class of singular integral operators of the
class $C^\infty _\beta$ into the class of functions $ h(x, \xi)$,
$x$, $\xi \in \underbar{R}^n$ homogeneous of degree zero with respect
to $\xi$ and in $ C^\infty_\beta$. $\sigma (H)$ is called the symbol
of the singular integral operator $H$. Thus the series $
\sum\limits_{l, m} a_{lm}(x) Y_{lm} (\xi)$ represents in a sense the 
Fourier transform of $k(x, z)$ with respect to $z$. We recall without
proof the following important theorems on these operators, which we
shall require for later use. For proofs see Calderon-Zygmund $[1,2]$.  

\setcounter{theorem}{0}
\begin{theorem}[Calderon-Zygmund \cite{key1}]\label{chap3-sec4-thm1}%theo 1
 If $H$ is a singular integral operator of
  type $C^\infty_\beta$ then its symbol is a homogeneous function of
  degree zero and of class $C^\infty_\beta$ with respect to $\xi$ in
  $| \xi | \geq 1$. Conversely every function of $x$ and $ \xi$ which
  is homogeneous of degree zero and belongs to the class $
  C^\infty_\beta$ in $| \xi| \geq 1$  is the symbol of a unique singular 
  integral operator of type $ C^\infty_\beta$. If  
$$
M = \sup_{\substack{x \in \underbar{R}^{n}, | \xi | \geq 1 \\ | \nu |
    \leq 2n}}  \big|(\frac{\partial}{\partial \xi})^\nu \sigma (H) (x,
\xi) \big| 
$$
then 
\begin{equation*}
\parallel H f \parallel_p \leq M A_p \parallel f \parallel_p
\tag{4.12}\label{chap4-eq4.12} 
\end{equation*}
where\pageoriginale $A_p$ depends only on $p$ and $n$. 
\end{theorem}

If $h_1 (x, \xi ), h_2 (x, \xi )$ are of class $ C^\infty _\beta in |
\xi| \geq 1$ then it is easy to see that $  h_1 (x, \xi ) +  h_2 (x,
\xi )$ and $  h_1 (x, \xi ) h_2 (x, \xi )$ are also of class $
C^\infty _\beta$ and further if $ | h_2 (x, \xi ) | \geq \delta > 0$
then  $ \dfrac{h_1 (x, \xi)}{h_2 (x, \xi)}$ is also of class $
C^\infty _\beta$.  

\begin{theorem}[Calderon-Zygmund \cite{key2}]\label{chap3-sec4-thm2}%theo 2
 Let $ h(x, \xi) = \sigma (H)$ be of type $ C^\infty_\beta$,
 homogeneous of degree zero in $\mathscr{E}$ then  
\begin{enumerate}
\renewcommand{\labelenumi}{\rm(\theenumi)}
\item for $r \leq \beta$, $Hf \in \mathscr{D}^r_{L^p}$ for $f \in
  \mathscr{D}^r_{L^p} (1< p< \infty)$, and  

\item if $f \in L^p$ and H\"older continuous of order $\alpha
  (\alpha < \beta $ then $ Hf \in L^p $ and H\"older continuous of
  order $ \alpha (1 < p < \infty)$. 
\end{enumerate} 
\end{theorem}

Let $ H^\#$ and $H_1 0 H_2 $ be singular integral operators whose
symbols are respectively $\overline{\sigma (H)}$ and $ \sigma
(H_1)$. $\sigma (H_2)$.  

\begin{theorem}[Calderon-Zygmund]\label{chap3-sec4-thm3}%the 3
If $ \sigma (H_1)$, $\sigma (H_2)$ are independent of $x$ then 
$$
H_1 \circ H_2 = H_1 H_2 = H_2 \circ  H_1 = H_2 H_1
$$
and if $ \sigma (H)$ is independent of $x$ and $| \sigma (H) (\xi) |
\geq \delta > 0$ then $H$ is invertible and its inverse $ H^{-1}$ is
also a singular integral operator. We illustrate by a simple example
the motivation for the definition of the singular integral operators $
H_1 \circ H_2$ and  $ H^{\#}$. Consider the differential operators  
$$
L = \sum_j a_j (x) \frac{\partial}{\partial x_j}, M = \sum_j b_j (x)
\frac{\partial}{\partial x_j}, a_j, b_j \in \mathscr{B}^1. 
$$
\end{theorem}

Then 
$$
LM = \sum_{j, k} a_j(x) b_k (x) \frac{\partial^2}{\partial x_j
  \partial x_k} + \sum_{j, k} a_j (x) \frac{\partial b_k}{\partial
  x_j} \frac{\partial}{\partial x_k}. 
$$

Therefore,\pageoriginale if we define 
$$
L \circ M= \sum_{j, k} a_j (x) b_k (x) \frac{\partial^2}{\partial x_j
  \partial x_k} 
$$
then $LM = L \circ M$ modulo first order operators. Next if we define 
$$
L^{\#} = - \sum \overline{a_j (x)} \frac{\partial}{\partial x_j} 
$$
then $L^* \equiv L^{\#}$ modulo bounded operators.  

These considerations suggest that the product of two singular integral
operators and the conjugate operator $H^*$ will be approximated, in
some sense, by the singular integral operators $H_1 \circ H_2$ and $H^{\#}$
respectively. More precisely we have the following:  

\begin{theorem}[Calderon-Zygmund]\label{chap3-sec4-thm4}%theo 4
 Let $H$ be a singular integral operator of
  type $C^\infty_\beta (\beta >1)$ and $M$ be a bound for $ \sigma
  (H) (x, \xi)$ and its derivatives with respect to the coordinates of
  $\xi$ of order $2n$, the first derivatives of these with respect to
  the coordinates of $x$ and H\"older constants of the latter. Then
  for every $f \in \mathscr{D}^1_{L^p} (1 < p < \infty)$ we have  
{\fontsize{9pt}{11pt}\selectfont
\begin{equation*}
\begin{split}
& \parallel (H \wedge - \wedge H) f \parallel _{L^p} \leq A_p M
 \parallel f \parallel _{L^p}, \; \parallel (H^* \wedge - \wedge H^*) f
\parallel_{L^p} \leq A_p M \parallel f \parallel _{L^p} \\ 
& \parallel (H^* - H^{\#} ) f \parallel_{L^p} \leq A_p M \parallel f
\parallel_{L^p}, \parallel \wedge (H^* H^{\#}) \parallel_{L^p} \leq
A_p M \parallel f \parallel_{L^p}  
\end{split}\tag{4.13}\label{chap3-eq4.13}
\end{equation*}}\relax
where $A_p$ depends only on $p, n, \beta$. Further if $H_1$ and
$H_2$ are two singular integral operators of type $C^\infty_\beta$ and
$f \in \mathscr{E}^1_{L^p} (1 < p< \infty)$ then $ H_1 \circ H_2$ is an
operator of type  $C^\infty_\beta $ and  
\begin{equation*}
\begin{split}
& \parallel (H_1 \circ H_2 - H_1 H_2) \wedge f \parallel_{L^p} \leq A_p M_1
M_2 \parallel f \parallel_{L^p}, \\ 
& \parallel \wedge (H_1 \circ H_2 - H_1 H_2) f \parallel_{L^p} \leq A_p M_1
M_2 \parallel f \parallel_{L^p} 
\end{split}
\tag{4.14} \label{chap4-eq4.14}
\end{equation*}
where\pageoriginale again $ A_p$ depends only on $p, n, \beta$
and $ M_1, M_2$ being defined in the same way as $M$.   
\end{theorem}

We can write differential operators in the form of singular integral
operators as follows: Let $A = \sum\limits_{| \alpha | = m} a_\alpha
(x) \left(\dfrac{\partial}{\partial x}\right)^\alpha$ be a homogeneous
differential operator of order $m$ with coefficients $ a_\alpha (x) $
in $C_\beta$, $\beta \geq 0$. If $ u \in \mathscr{D}^m_{L^2}$ then $
\wedge^m u $ is well defined,  
$$
(\widehat{ \wedge^m u} ) (\xi) = | \xi |^m \hat{u} (\xi)
$$
and $Au = H \wedge^m u$ where $H$ is a singular integral operator
of type $C^\infty_\beta$ and  
\begin{equation*}
\sigma (H) = i^m \sum\limits_{| \propto | = m }a_\propto (x)
\xi^\propto | \xi |^{-m}. \tag{4.15} \label{chap3-eq4.15} 
\end{equation*}

Similarly any general linear differential operator of order $m$
$$
A = \sum\limits_{k \leq m} A_k, A_k = \sum \limits_{| \nu | =k}
a_{k, \nu} (x) \left(\dfrac{\partial}{\partial x}\right) ^\nu
$$ 
with $a_{k, \nu}(x) $ of class $ C_\beta$ can be written as  
\begin{equation*}
Au = \sum H_k \wedge^k u \tag{4.16}\label{chap3-eq4.16}
\end{equation*}
where $H_k$ is a singular integral operator of class $C^\infty_\beta$
and  
\begin{equation*}
\sigma (H_k) = i^k \sum\limits_{| \nu | = k} a_{k, \nu} (x) \xi^\nu |
\xi |^{-k} \tag{4.17} \label{chap3-eq4.17}
\end{equation*}
for every $ u \in \mathscr{D}^m_{L^2}$.

A matrix of operators is called a singular integral matrix if its
elements are singular integral operators and its symbol is the matrix
whose elements are the symbols of the corresponding elements of
the\pageoriginale singular integral matrix. A system of differential
operators can be written as a singular integral matrix.   

\section[Extension of G$\ring{\text{a}}$rding's...]{Extension of G$\ring{\text{a}}$rding's inequality to singular
  integral operators}\label{chap3-sec5} %sec 5  

In this section we prove an inequality for the singular integral
operators whose symbol satisfies a condition of positivity. This is an
analogue of the well know inequality of Garding for elliptic
differential operators. Before stating the inequality we prove some
preliminary results needed in the proof of this inequality. These
results are also of independent interest.  

The following lemma corresponds to the local property of differential
operators, namely, that differential operators decrease supports.  

\setcounter{lemma}{0}
\begin{lemma}[Quasi localisation lemma]\label{chap3-sec5-lem1}%lemm 1
 Let $ \Omega$ be the ball of radius
$2 \eta$ and of centre a point $x_0$ in $ \underbar{R}^n$. Let $H$ be
a singular integral operator whose symbol $\sigma (H) (x, \xi) \in
C^\infty_\beta$, with $\beta > 0$. If $ u \in \mathscr{D}^1_{L^2}$
has its support in the ball of radius $\eta$ and of centre $x_0$ then   
\begin{equation}
\parallel H \wedge u \parallel_{L^2 (C \Omega)} \leq c(n, \eta) M'
\parallel u \parallel \tag{5.1}\label{chap3-eq5.1} 
\end{equation}
where $M' = \sum\limits_{| \nu | \leq 3n +3} \sup\limits_{ x \in
  \underbar{R}^n, | \xi | \geq 1} \big| (\dfrac{\partial}{\partial
  \xi})^\nu \sigma (H) (x, \xi ) \big|$ and $ c (n, \eta)$ is a constant
depending only on $n$ and $\eta$. 
\end{lemma}

\begin{proof}%proo 0
We decompose the operator $\wedge$ as $\wedge = \wedge_1 + \wedge_2$
with $\widehat{\wedge}_1 (\xi ) = \alpha (\xi ) | \xi |$ and $
\widehat{\wedge}_2 (\xi ) =(1- \alpha ( \xi )) | \xi | $ where $
\alpha ( \xi ) \in \mathscr{D}$ such that $\alpha ( \xi ) \equiv 1 $ on
$ | \xi | \leq 1, 0 \leq \alpha (\xi) \leq 1$ and vanishes outside $|\xi|
\leq 2$. $\widehat{\wedge}_2 (\xi)$\pageoriginale is an infinitely
differentiable function, $\widehat{\wedge}_1 (\xi)$ has compact
support and hence $\wedge_1$ is a bounded operator in $ L^2$. So it is
enough to prove that   
$$
\parallel H \wedge_2 u \parallel_{L^2 (C \Omega)} \leq C (n, \eta) M'
\parallel u \parallel. 
$$ 

Let 
\begin{equation}
\sigma (H) (x, \xi ) = a_0 (x) + \sum \limits _{l, m} a_{lm} (x)
Y_{lm} (\xi ) \tag{5.2} \label{chap3-eq5.2}
\end{equation}
be the expansion of $\sigma (H)$ in terms of a complete system of
spherical harmonics $Y_{lm} (\xi)$. Let $Y'_{lm}(x)$ be the
singular integral operator such that  
$$
Y'_{lm}(x) \to Y_{lm} (\xi) \widehat{\wedge}_2 (\xi) 
$$
by Fourier transforms. Then we can write 
\begin{equation}
(H \wedge_2 u) (x) = a_0(x) \wedge_2 u(x) + \sum\limits_{l, m}
  a_{lm}(x) (Y'_{lm}(x) * u). \tag{5.3} \label{chap4-eq5.3}
\end{equation}

First we show that 
\begin{equation}
|Y'_{lm}(x) \leq |x|^{-2p} c (p,n) | Y_{lm} (\xi ) |_{2p} \text{ for }
x \in ^c\{ 0 \} \text{ for } 2p \geq n+2 \tag{5.4}\label{chap3-eq5.4} 
\end{equation}
where $| Y_{lm} (\xi ) |_{2p} = \sum\limits_{| \nu | \leq 2p}
\sup\limits_{| \xi | \geq 1} | (\dfrac{\partial}{\partial \xi})^\nu
Y_{lm}( \xi ) |$.  

In fact, 
$$
Y'_{lm}(x) =  | x |^{-2p} \bigg\{ |x |^{2p} Y'_{lm} (x)\bigg\}
$$
and $| x|^{2p} Y'_{lm} (x)$ is the inverse Fourier image of const 
$$
\triangle^p_\xi (Y_{lm}(\xi) (1- \alpha (\xi)) | \xi |.
$$

 Hence we
have the estimate   
\begin{gather*}
|Y'_{lm}(x) | \leq | x|^{-2p} \left(\frac{1}{2\pi}\right)^p \int |
\triangle^p_\xi 
((1- \alpha (\xi )) Y_{lm} (\xi ) | \xi| ) | d \xi \\ 
\leq | \xi |^{-2p} c (n, p) | Y_{lm}(\xi ) |_{2p} for 2p \geq n + 2. 
\end{gather*}\pageoriginale 

This establishes the assertion \eqref{chap3-eq5.4}.

Now we show that for any $u \epsilon \mathscr{D}$ with support
contained in $\omega = B_\eta (x_0) $ 
\begin{equation}
||~ |x|^{-2p}* u ||_{L^2(C \Omega)} \leq c (n, p, \eta)|| u ||
\tag{5.5}  \label{chap3-eq5.5}
\end{equation}
holds for $p$ satisfying $4 p > n$. 

In fact, 
for $x \epsilon \complement \Omega$, $|| ~|x|^{-2p} * u || = \big|\int
\dfrac{{u(y)}}{|x-y|^{2p}}dy\big|$ 
by Schwarz inequality, 
$$
\leq || u || (\text{ vol } \omega)^{1/2}(\text{ dist. }
(x,\omega))^{-2p}   
$$

Hence 
$ ||~|x|^{-2p} * u ||_{L^2({}^C\Omega)} \leq (\text{ vol}
\omega)^{\frac{1}{2}}||u|| ( \int\limits_{|x| \ge 2 \eta}
\dfrac{dx}{(|x| - \eta )^{4p}})^{\frac{1}{2}}$.  
The integral in the right hand side converges for $4p >n$ which 
proves the assertion \eqref{chap3-eq5.5}. Now \eqref{chap3-eq5.4} and
\eqref{chap3-eq5.5} together assert that  
\begin{align*}
||H \wedge_2 u || _{L^2(C_\Omega)} & \leq (\text{ vol } 
\omega)^\frac{1}{2} c (p,n,\eta) \left(\sum\limits_{l,m}| a_{lm}(x)|_\circ |
Y_{lm} (\xi) |_{2p} \right)|| u ||\\
&  \leq C' (p,n,\eta) M' || u ||.    
\end{align*}

This completes the proof of lemma \ref{chap3-sec5-lem1}. In the proofs
of the following 
results we use a $C^\infty$ partition of unity in $ \underbar{R}^n$. 
$$
\alpha_j \xi \mathscr{D},\;\;  \alpha_j \geq 0, \;\;
\sum\limits_{j}\alpha^2_j = 1. 
$$\pageoriginale

To simplify the arguments we take a partition of unity satisfying the
following conditions: Let $\alpha_0 \varepsilon \mathscr{D}$ whose
support is contained in the ball of redius $\varepsilon$,
$\varepsilon$ being a small number to be determined by the singular
integral operator $H$. Let $\{x^{(j)}\}$ be a sequence of points of
$\underbar{R}^n $ whose coordinates are multiples of $\varepsilon'(=
\varepsilon n^{- \dfrac{1}{2}})$, $\alpha_j(x) = \alpha_0 (x-x^{(j)}
)$, $j = 0, 1,\ldots, x^{(0)} =  (0)$. The support of $\alpha_0$
will be denoted by $\omega_0$ and the ball of centre $x^{(j)} $ and of
radius $2 \varepsilon$ will be denoted by $\Omega_j$. Let  
$$
\alpha(p) = \sum \limits_{|\nu|\leq p } \sup\limits_{x}
\big|\left(\frac{\partial}{\partial x}\right)^\nu \alpha_0 (x) \big|.  
$$
\end{proof}

\begin{lemma}\label{chap3-sec5-lem2}% 2
Let $H$ be a singular integral operator with its symbol $ \sigma (H)\break
(x, \xi) \varepsilon c^\infty_\beta$, with $\beta > 0 $ and
$(\alpha_j) $ be a $C^\infty$ partition of unity as constructed
above. Then for any $u \varepsilon \mathscr{D}_L^1 2$ 
\begin{equation}
\sum_j ||((H \wedge) \alpha_j - \alpha_j (H \wedge )) u ||^2 \leq
\gamma ||u||^2 \tag{5.6} \label{chap3-eq5.6}
\end{equation}

In particular, taking $\sigma (H) = 1$ this wouls imply 
\begin{equation}
\sum_j || [ \wedge , \alpha_j ] u || ^2 \le \gamma || u||^2.
\tag{5.7}\label{chap3-eq5.7} 
\end{equation}
\end{lemma}

Let $\beta \varepsilon \mathscr{D}_\xi$, \; $0 \leq \beta (\xi) \leq 1$
with support contained in $|\xi| < 1 $ which takes the value 1 in a
neighbourhood of the origin. Decompose $\wedge$ into $\wedge =
\wedge_1 + \wedge_2$ where $\hat\wedge_1 (\xi) = \beta (\xi) | \xi |$
and $\hat{\wedge}_2 (\xi) = (1-\beta (\xi) )|\xi|$.
Clearly $|| \wedge_1 u || \leq || u ||$ and hence  
$$
||H \wedge_1 \alpha_j u || \leq || H || \; || \alpha_j u ||_{L^2
  (\Omega_j)} \leq \sup\limits_{x} \big|\alpha_j (x) \big| || H ||\;\;
 || u||_{L^2(\Omega_j)}  
$$\pageoriginale

Hence 
$$
\sum\limits_j || H \wedge_1 \alpha_j u ||^2 \leq \alpha (0)^2 || H
||^2  k || u ||^2  
$$
where $k$ is the maximum number of sets $\{\omega_h\}$
 intersecting at any point and 
 $$
 \sum_j || \alpha_j H \wedge_1 u ||^2 = || H \wedge_1 u ||^2 \leq || H
 ||^2 || u ||^2.  
 $$

 So we have only to consider $\sum\limits_{j}|| [H \wedge_2, \; 
   \alpha_j ]u ||$. Consider the term  
 \begin{gather*}
\varphi_j (x) = [ H\wedge_2, \alpha_j ] u
(x). \tag{5.8}\label{chap3-eq5.8} \\ 
 \varphi_j (x) = \sum\limits_{l, m} a_{lm }(x) \int \tilde{Y}_{lm}
 (x-y) \wedge_2 (x-y) (\alpha_j(y) - \alpha_j (x)) u (y) dy.  
 \end{gather*}

 Let us denote the operator $\tilde{Y}_{lm}* \wedge_2$ by
 $Y'_{lm}$. First of all we consider $||\varphi_j||_{L^2
   (\Omega_j)}$. Expanding $\alpha_j (y) - \alpha_j (x) $ in a Taylor
 series, we obtain  
\begin{equation}
\alpha_j (y) - \alpha_j (x) = \sum\limits_{1\leq | \nu | \leq q-1}\frac{1}{\nu!}
\left(\frac{\partial}{\partial x} \right)^\nu \alpha_j (x) (y-x)^\nu +
\sum\limits_{|\nu | = q } \alpha_{j,\nu} (x, y) (x-y)^\nu
\tag{5.9}\label{chap3-eq5.9}   
\end{equation} 
 where $q$ will be determined later. It follows that 
 $$
 \varphi_j (x) = \sum\limits_{|\nu| \leq q-1} \frac{1}{\nu !}
 (-1)^{|\nu|} (\frac{\partial}{\partial x})^\nu \alpha_j (x) \sum
 \limits_{l,m } a_{lm}(x) (x^\nu Y'_{lm} ) u + \varphi^{(2)}_j (x)  
 $$
where 
 \begin{equation}
\varphi _j^{(2)} (x) = \Sigma a_{lm}(x) \int \alpha_{j ,\nu}
(x,y)(x-y)^\nu Y'_{lm} (x-y) u (y) dy. \tag{5.10} \label{chap3-eq5.10}
 \end{equation} 

Now\pageoriginale the operators $H_\nu = \sum a_{lm} (x) (x^\nu
 Y'_{lm}) $ are 
 singular integral operators which operate on $L^2$  as continuous
 linear operators since $ \sup\limits_{x}| a_{lm }(x) | $ is a rapidly
 decreasing sequence (more precisely, for any positive integer
 $\sigma$ we have  
 $$
 \sum\limits_{1 \geq 0 } l^\sigma \sup\limits_{x} \big|a_{lm}(x) \big| <
 \infty) \text{ (see Calderon-Zygmund  \cite{key1}.}  
 $$

Hence for the first sum, 
 \begin{equation}
\varphi^{(1)}_j (x) = \sum\limits_{|\nu | \leq q-1}
\frac{(-1)^{|\nu|}}{\nu !} (\frac{\partial }{\partial x})^\nu \alpha_j
(x) \cdot H_\nu u \tag{5.11} \label{chap3-eq5.11} 
 \end{equation} 
 and we have 
 \begin{equation}
\big\| \varphi^{(1)}_j \big\|^2_{L^2 (\Omega_j )} \leq c (q) \alpha
(q-1) \sum\limits_{1 \leq | \nu | q-1} \big\| H_\nu u
\big\|^2_{L^2(\Omega_j )}. \tag {5.12}  \label{chap3-eq5.12}
 \end{equation} 

 To majorize the second sum $\varphi^{(2)}_j (x)$  we begin by
 considering a typical term $(x^\nu \cdot Y')* u$. We have 
 \begin{align*}
 |(x^\nu Y') * u | &  = \big| \int (x-y)^\nu Y' (x-y) \cdot u (y) \,
 | dy \big|\\
&\quad \leq \int |(x-y)^\nu Y' (x-y) || u(y) | dy \\ 
 & = \left(\int_{\Omega'_j }+ \int_{C\Omega'_j}\right) | (x-y)^\nu Y' (x-y) || u
 || (y) | dy 
 \end{align*} 
 where $\Omega'_j$ is a sphere of radius $6 \varepsilon$ about
 $x^{(j)}$. The first integral is majorized by  
$$
 \sup\limits_{x} |x^\nu Y' (x) |\;\; || u||_{\Omega'_j}(\text{vol
 }\Omega'_0)^{\frac{1}{2}}  
$$
and the second integral is majorized by 
 $$
 \sup\limits_{x}\big| |x|^{2p} x^\nu Y' (x) \big| \int_{C\Omega'_j}
 \frac{|u(y)|}{|x-y|^{2p}} dy .  
 $$

 Now\pageoriginale $I \equiv \int\limits_{C \Omega'_j}
 \dfrac{|u(y)|}{|x-y|^{2p}} dy \leq \sum\limits_{k} \int\limits_{\omega_k}
 \dfrac{|u(y)|}{|x-y|^{2p}} dy$ where the sum is taken is taken over
 all the $\omega_k$ such that $d(\Omega_j, \omega_k ) \geq 3
 \varepsilon, \Omega_j$ being the support of $\alpha_j$. Hence   
 $$
 I \leq \sum\limits_{k} 2^{2p}d (\omega_k, \Omega_j)^{-2p} \||
 u||_{\omega_k} (\text{vol } \omega_0)^{\frac{1}{2}}. 
 $$

 Hence the second integral is majorized by 
 $$
  \sup\limits_{x}(|x|^{2p}|x^\nu Y' (x)|) (\text{vol
  }\omega_0)^{\frac{1}{2}}2^{2p} \left(\sum_k d (\omega_k,
  \Omega_j)^{-2p}||u||_{\omega_k}\right)  
  $$
 where the $\omega_k$ occuring in the summation are such that
 $d(\omega_k, \Omega_j) \geq 3 \varepsilon $. 
 For $|\nu| = q$ sufficiently large it can be shown that 
 $$ 
 K(\nu ) = \sum\limits_{1 \geq 0}
 \sup\limits_{x}|a_{lm}(x)| \cdot \sup\limits_{x}|x^\nu
 Y'_{lm}(x)  | < \infty  
 $$
and
$$
K (\nu,p) = \sum\limits_{1\geq 0} \sup\limits_x | a_{lm} (x) \big| \cdot
\sup\limits_x \big| \, |x|^{2p} x^\nu Y'_{lm} (x) \big| < \infty
$$
for $p$ sufficiently large. So we have 
{\fontsize{10pt}{12pt}\selectfont
 \begin{align*}
\big|\big|\varphi^{(2)}_j \big|\big|^2_{\Omega_j} & \leq
\int_{\Omega_j} \left(\sum\limits_{|\nu | =q}\sum\limits_{l,m}| a_{lm}(x) | 
\int \alpha_{j,\nu}(x,y)  (x-y)^\nu Y'_{lm}(x-y) \big|\big|u (y) \big|
dy\right) dx\\
&  \leq c \Big( \sum\limits_{|\nu | =q} K (\nu ) ||u||^2_{\Omega'_j}+
K (\nu, p) \left(\sum_k d (\omega_k, \Omega_j
)^{-2p}||u||_{\Omega_j}\right)^2. 
\tag{5.13} \label{chap3-eq5.13}
 \end{align*}}\relax

But by Schwarz inequality we have 
$$
\sum_k d(\omega_k, \Omega_j)^{-2p} || u ||_{\omega_k} \leq \left(\sum_k d
(\omega_k, \Omega_j)^{-2p}\right)^{\frac{1}{2}}\left(\sum\limits_{k}d
(\omega_k,\Omega_j)^{-2p}||u||^2_{\omega_k}\right)^{\frac{1}{2}} 
$$
and\pageoriginale since $(\sum_k d(\omega_k, \Omega_j)^{2p}) < K$, a
constant we obtain after summing over $j$   
\begin{align*}
 \sum\limits_{k,j} ||u||^2_{\omega_k} d (\omega_k,  \Omega_j)^{-2p}
& = \sum\limits_{k}||u||^2_{\omega_k} \sum\limits_{j} d (\omega_k,
 \Omega_j)^{-2p}\\
&  \leq K_p \sum\limits_{k}||u||^2_{\omega_k}\leq K_p
 r ||u||^2 
\end{align*}
where $K_p$ is a constant depending on $p$ and $r$ is the maximum
number of balls $\omega_k$ containing a point of
$\underbar{R}^n$. Substituting in \eqref{chap3-eq5.13} 
$$
\sum\limits_{k}||\varphi^{(2)}_k||^2_{\Omega_k} \leq c ||u||^2
$$
which together with \eqref{chap3-eq5.12} gives the estimate 
 \begin{equation}
 \sum\limits_{k}||\varphi_k ||^2_{\Omega_k}\leq c'
 ||u||^2. \tag{5.14}\label{chap3-eq5.14} 
 \end{equation} 

 It remains to estimate $ ||\varphi_k||_{{}^C\Omega_k} $
 in order to complete the proof of the lemma. For $x \in {}^C
 \Omega_k $ a typical term in the expression for $\varphi_k(x) $ is of
 the form  
 $$
 \psi (x) = \int_{\omega_k}Y'_{lm} (x-y) \alpha_{j} (y) u (y) dy 
 $$
 from which we obtain as before the estimate 
 \begin{align*}
 |\psi (x)| &  \leq \sup\limits_{x}\big|
 |x|^{2p}Y'_{lm}(x)\big| \cdot \int_{\Omega_j} \frac{|u(y) |}{|x-y|^{2p}}dy\\  
  & \leq \sup\limits_{x}\big| |x|^{2p}Y'_{lm}(x)
 \big| \cdot ||u||_{\Omega_j} d (x,\Omega_j )^{-2p}
 |(\text{vol}\,\omega_0)^{\frac{1}{2}}.   
 \end{align*}

Hence 
$$
||\psi||_{{}^C\Omega_k}\leq
\sup\limits_{x}\big| |x|^{2p}Y'_{lm}(x)\big| \cdot ||u||_{\Omega_j}
(\text{vol }\omega_0)^{\frac{1}{2}}(\int\limits_{|x|\leq 
  2_\varepsilon}\frac{1}{d(x,\omega_1)^{4p}} dx)^{\frac{1}{2}}. 
$$\pageoriginale

Taking $4p > n$ and observing that $K(\nu, p) < \infty$ we see that  
 $$
 ||\varphi_k||^2_{{}^C\Omega_k}\leq c''||u||^2 \Omega_j, 
 $$
and again, summing over $k$, 
 \begin{equation}
\sum_k || \varphi_k ||^2_{{}^C\Omega_k} \leq c'' \cdot  r||
u||^2. \tag{5.15}  \label{chap3-eq5.15}
 \end{equation} 

 This completes the proof of the lemma.
 
 The following is an extension to singular integral operators of
 G$\ring{\text{a}}$r\-ding's inequality for elliptic differential operators.  

\setcounter{proposition}{0} 
\begin{proposition}\label{chap3-sec5-prop1}%prop 1
Let $H$ be a singular integral operator such that its symbol
$\sigma(H) = h(x, \xi) \in C^\infty_\beta$ with $\beta > 0 $ 
satisfies  
 \begin{equation}
|h(x,\xi) |\geq \tau > 0 \tag{5.16}\label{chap3-eq5.16}
 \end{equation} 
 for every $x \in \underbar{R}^n $ and every vector $\xi$,
 $\delta$ being a positive constant. Then there exists $a \delta' >0$
 such that  
 \begin{equation}
||H \wedge u ||^2 \geq \delta' || \wedge u ||^2 - \gamma || u||^2
\tag{5.17} \label{chap3-eq5.17}
 \end{equation} 
 for every $u \in \mathscr{D}^1_{L^2} $ where $\gamma$ is a
 positive constant.  
 \end{proposition}

\begin{proof}%proo 0
$H$ being a singular integral operator we know that $||Hu ||\leq\break AM
   ||u|| $ where $A$ is a constant depending only on $n$ and  
$$
M = \sum\limits_{|\nu| \leq 2n} \sup\limits_{x \varepsilon
  \underbar{R}^n, \; |\xi| \ge 1} \big|
(\frac{\partial}{\partial\xi})^\nu \gamma (H) (x, \xi)\big|.  
$$\pageoriginale

Given a $\delta > 0 $ there exists a number $\epsilon  > 0 $ such
that for every $x_0 \in \underbar{R}^n$ and for every $u
\in L^2$  
\begin{equation}
||(H-H(x_0)) u ||^2_{\omega_0}\leq \frac{\delta^2}{4}||u||^2
\tag{5.18} \label{chap3-eq5.18}
 \end{equation} 
 where $H(x_0)$ is the singular integral operator with constant
 coefficients such that $\sigma(H(x_0))(\xi) = \sigma(H) (x_0,
 \xi)$. ($H(x_0)$ is the tangential operator at $x_0$). $\varepsilon$ 
 can be chosen independent of the position of $x_0$. Consider the
 $C^\infty$ partition of unity introduced earlier,  
 $$
 \alpha_j(x) \geq 0, \quad \alpha_j \in \mathscr{D}, \quad \sum
 \alpha^2_j   (x) \equiv 1. 
 $$

 As we have 
 $$
 || H \wedge u||^2 = \sum ||\alpha_j H \wedge  u ||^2  
 $$
 it is sufficient to prove the inequality for $\alpha_j H \wedge u$. 
 \begin{align*}
 ||\alpha_j H \wedge u ||^2 & \geq \frac{1}{2} || H \alpha_j \wedge u
 ||^2 - ||(H \alpha_j - \alpha_j H ) \wedge u ||^2 \\ 
 & \geq \frac{1}{2}|| H \alpha_j \wedge u ||^2 - 2 ||H(\wedge
 \alpha_j - \alpha_j \wedge ) u ||^2\\
&\quad - 2 || ((H \wedge ) \alpha_j -
 \alpha_j (H \wedge )) u ||^2.  
 \end{align*}

 Now  we have 
 \begin{align*}
 \sum\limits_{j} H ( \wedge \alpha_j - \alpha_j \wedge ) u ||^2 &
 \leq \sum_j ||H ||^2 \; || (\wedge \alpha_j -\alpha_j \wedge ) u ||^2 \leq
 c'_1 || H ||^2  || u ||^2 \\ 
&  \leq c_1 ||u||^2
 \end{align*} 
and by lemma \ref{chap3-sec5-lem2}
$$
\sum\limits_{j} || (H  \wedge) \alpha_j - \alpha_j (H \wedge)) u ||^2
\leq c_2 ||u||^2  
$$ 
where\pageoriginale $c_1$ and $c_2$ are constants depending only on
the norm of $H$ and $n$.   

Hence 
\begin{equation}
||H \wedge u||^2 \geq \frac{1}{2} \sum\limits_{j}||H \alpha_j \wedge
u ||^2 - c_3 ||u||^2 \tag{5.21} \label{chap3-eq5.21}
\end{equation} 
and we have only to consider $||H \alpha_j \wedge u ||^2$. 
  
For this purpose let $H(x^{(j)})$ be the singular integral operator
whose symbol is $h(x^{(j)}, \xi)$, so that  
$$
\sigma (H-H(x^{(j)})) = h (x, \xi ) - h(x^{(j)}, \xi). 
$$

So we have  
$$
||H \alpha_j \wedge u ||^2 \geq \frac{1}{2} || H (x^{(j) }) \alpha_j
\wedge u ||^2 - ||(H - H(x^{(j)})) \alpha_j \wedge u ||^2. 
$$ 

From the condition that $|h(x, \xi) | > \delta $ we have 
$$
\frac{1}{2} || H(x^{(j)}) \alpha_j \wedge u ||^2 \geq
\frac{\delta^2}{2}|| \alpha_j \wedge u ||^2.  
$$

As in lemma \ref{chap3-sec5-lem2}, let $\Omega_j$ denote the ball of radius $2
\varepsilon$ and centre $x^{(j)}$.  
We decompose the second term into a sum 
\begin{align*}
||(H-H(x^{(j)})) \alpha_j \wedge u ||^2 & = || (H-H(x^{(j)})) \alpha_j
\wedge u||^2_{\Omega_{j}} \\
& \qquad + || (H-H(x^{(j)})) \alpha_j \wedge u||^2_{{}^C\Omega_j} 
\end{align*}

As mentioned at the begining of the proof, the first term is majorized
by $\dfrac{\delta^2}{4}||\alpha_j \wedge u ||^2$. For the second term
we have  
\begin{align*}
||(H-H(x^{(j)})) \alpha_j \wedge u ||^2_{{}^C\Omega_j} & \leq 2 ||
(H-H(x^{(j)})) (\alpha_j \wedge - \wedge \alpha_j ) u
||^2_{{}^C\Omega_j}\\
& \qquad  + 2 ||(H-H(x^{(j)})) \wedge \alpha_j u ||^2_{{}^C \Omega_j}. 
\end{align*}

By lemma \ref{chap3-sec5-lem1}1,\pageoriginale $||(H-H(x^{(j)} ))
\wedge \alpha_j u 
||^2_{{}^C\Omega_j} \leq c (n, \eta ) M' ||\alpha_j u ||^2 $ and since
$(H-H(x^{(j)}))$ is a singular integral operator we obtain from lemma
\ref{chap3-sec5-lem2} the inequality   
$$
 \sum\limits_{j} (H-H(x^{(j)})) (\alpha_j \wedge - \wedge \alpha_j )
 u ||^2 _{{}^C\Omega_j} \leq c ||u||^2. 
$$

Hence 
\begin{align*}
|| H \wedge u||^2 & \geq \frac{\delta^2}{4}\sum_j ||\alpha_\delta
\wedge u ||^2 - c(n,\eta) M'' \sum_j ||\alpha_j u ||^2 - c ||u||^2 \\ 
& \geq \frac{\delta^2}{4}||\wedge u ||^2 - \gamma || u ||^ 2  
\end{align*} 
which completes the proof of the inequality. 
\end{proof}

\begin{proposition}\label{chap3-sec5-prop-2}% prop 2
Let $H$ be a singular integral operator whose symbol $\sigma (H) =
h(x, \xi) \in C^\infty_\beta$ with $\beta > 0$. Let $h
(x,\xi) $ satisfy the condition  
\begin{equation}
\re h (x, \xi ) \leq - \delta, \; \delta > 0 \text{~ for every~ } x
\in \underbar{R}^n \text{~ and every vector~ }
\xi. \tag{5.19}\label{chap3-eq5.19}  
\end{equation}

Then there exists a $\delta' > 0 $ such that 
\begin{equation}
((H + H^* ) \wedge u, \wedge u) \leq - \delta' || \wedge u ||^2 +
  \gamma || u||^2 \text{~ for~ } u \in \mathscr{D}^1_{L^2}
  \tag{5.20} \label{chap3-eq5.20}
\end{equation}
where $\gamma $ is a constant depending only on $M$, $\delta$ and $n$,
$\delta'(\delta' < \delta)$ can be chosen as near $\delta$ as one
wishes.  
\end{proposition}

\begin{proof}%proo 0
One can write $ H^* \wedge = H^{\#} \wedge + (H^* - H^{\#}) \wedge $. By
Th. \ref{chap3-sec4-thm4} of \S\ \ref{chap3-sec4}, $(H^*-H^{\#})$
is a bounded operator in $L^2$ and hence 
it is enough to prove that for $P = H + H^{\#}$, $(P \wedge u, \wedge
u)$ satisfies an inequality of the required kind. The symbol
$\sigma(P) = h(x,\xi) + \overline{h(x,\xi)}$ is real and $\leq - 2
\delta$. Let $\alpha_j \in \mathscr{D}$, $\alpha_{j}(x) \geq 0$,
$\sum \alpha^2_j (x) \equiv 1$ be a $C^\infty$ partition of  
unity\pageoriginale as in lemma \ref{chap3-sec5-lem1}. Then 
\begin{align*}
(P \wedge u, \wedge u) = \sum\limits_j (\alpha_j P \wedge u, \alpha_j
\wedge u) & = \sum\limits_j (P \alpha_j \wedge u, \alpha_j \wedge u)\\
&\quad - \sum\limits_j ((P \alpha_j - \alpha_j P) \wedge u, \alpha_j \wedge  
u). 
\end{align*}

For any $\epsilon' > 0 $ we have, by Schwarz's inequality
\begin{align*}
(P \alpha_j - \alpha_j P) \wedge u, \alpha_j \wedge u) & \leq || (P
  \alpha_j - \alpha_j P) \wedge u|| \cdot || \alpha_j \wedge u || \\
& \leq \epsilon' || \alpha_j \wedge u ||^2 + \frac{1}{\epsilon'}|| (P 
  \alpha_j-\alpha_j P) \wedge u||^2. 
\end{align*}

From lemma \ref{chap3-sec5-lem2} we have 
{\fontsize{10pt}{12pt}\selectfont
\begin{align*}
\sum\limits_j || (P \alpha_j - \alpha_j P) \wedge u||^2 & \leq 2
\sum\limits_j P(\alpha_j \wedge - \wedge \alpha_j ) u||^2 + 2
\sum\limits_j (P \wedge ) \alpha_j - \alpha_j (P \wedge ) u ||^2 \\ 
&  \leq c' ||u||^2
\end{align*}}\relax
and we have only to estimate $(P \alpha_j \wedge u, \alpha_j \wedge
u)$. Write $P=P(x^{(j)}) + (P-P(x^{j}))$ where, as before,
$P(x^{(j)})$ is the singular integral operator whose symbol is $\sigma
(P) (x^{(j)}, \xi)$. Since $\sigma (P(x, \xi )) \leq - 2 \delta $ we
have  
$$
(P(x^{j})) \alpha_j \wedge u, \alpha_j \wedge u) \leq -2 \delta ||
\alpha_j\wedge u ||^2. 
$$

Again by Schwarz's inequality
\begin{gather*}
\big|(P- P(x^{(j)})) \alpha_j \wedge u, \alpha_j \wedge u )\big| \leq
|| \left\{ P-P(x^{(j)})\right\} \alpha_j \wedge u || \cdot || \alpha_j
\wedge u ||\\  
 \leq \frac{\varepsilon''}{4}|| \alpha_j \wedge u ||^2 +
 \frac{4}{\varepsilon''}|| (P- P(x^{(j)})) \alpha_j \wedge u ||^2. 
\end{gather*}

Now, as in Prop. \ref{chap3-sec5-prop1},
$$
|(P- P(x^{(J)})) \alpha_j \wedge u||^2 \leq \eta (\varepsilon)||
\alpha_j \wedge u||^2 + \mu || (\alpha_j \wedge - \wedge \alpha_j )
u||^2 + \mu ||\alpha_j u||^2. 
$$\pageoriginale

Putting all these inequalities together one sees that 
\begin{gather*}
(P \wedge u, \wedge u) \leq \left(-2 \delta + \varepsilon' +
\frac{\varepsilon''}{4} + \frac{4}{\varepsilon''} \eta (\varepsilon)\right)
\sum\limits_j || \alpha_j \wedge u ||^2 |\\
 + \mu \sum\limits_j ||
(\alpha_j \wedge - \wedge \alpha_j) u||^2 + || u ||^2. 
\end{gather*}

Choosing $\varepsilon' \dfrac{\varepsilon''}{4}$, near $\delta$ and
fixing $\varepsilon$ to have $\dfrac{4 \eta (\varepsilon)}{\varepsilon
  ''}$ small enough to make $-2 \delta + 
\varepsilon' + \dfrac{\varepsilon''}{4}+ \dfrac{4}{\varepsilon''} \eta
(\varepsilon)$ as near $\delta$ as required and using lemma
\ref{chap3-sec5-lem2} the desired inequality follows. 

We shall now prove a lemma which we require later. It is analogous to
lemma \ref{chap3-sec5-lem2}. We define, for any real $s, \wedge^s $ by
$(\widehat{\wedge^s u}) = |\xi |^s \hat{u}$. 
\end{proof}

\begin{lemma}\label{chap3-sec5-lem3}%lemm 3
Let $H$ be a singular integral operator whose symbol  
$\sigma (H)= h (x, \xi ) \varepsilon C^\infty_\beta$, with $\beta =
\infty$. Then for any $u \in L^2$ 
\begin{equation}
||(H \wedge^s - \wedge^s H) \wedge^\sigma u|| \leq c || u || \text{  for }
s, \sigma \geq 0 \text{ with } s + \sigma \leq
1. \tag{5.21} \label{chap3-eq5.21} 
\end{equation}
\end{lemma}

\begin{proof}%proof 0
Let $\alpha \in \mathscr{D}_\xi $ be such that $0 \leq \alpha
(\xi) \leq 1$, $\alpha (\xi )  \equiv 1$ on $|\xi | \leq 1$ and
vanish outside $|\xi| \geq 2$. Writing $|\xi|^s = |\xi|^s \alpha
(\xi)+ |\xi|^s (1- \alpha (\xi ))$  we decompose the operator into a
sum $\wedge^s = \wedge^s_0 + \wedge^s_ 1$ with $\sigma(\wedge^s_0)=
|\xi|^s \alpha (\xi)$ and $\sigma (\wedge^s_1) = |\xi|^s (1- \alpha
(\xi))$. As $|\xi|^s \alpha (\xi)$ has compact support $\wedge^s_0$
defines a continuous linear operator in $L^2$ and hence it is enough
to prove that  
$$
||(H \wedge^s_1- \wedge^s_1 H) \wedge^\sigma u|| \leq c || u ||. 
$$

Expanding $\sigma (H)$ in terms of spherical harmonics $Y_{lm}$ as in
lemma \ref{chap3-sec5-lem2} and taking the inverse Fourier image we have 
$$
H = a_0 (x) + \Sigma a_{lm}(x) \tilde{Y}_{lm} \ast. 
$$\pageoriginale

Let $ P = a(x) \cdot\tilde{Y} \ast $ be a term in the sum. We consider 
$$
(P \wedge^s_1 - \wedge^s_1 P) \wedge^\sigma u = \int (a(x) - a(y))
\wedge^s_1 (x-y) \wedge^\sigma \varphi (y) dy 
$$
where $\varphi (y) = (\tilde{Y} * u)(y)$. Expand $a(x)-a(y)$ in Taylor
series upto order $q, q$ to be determined later: 
$$
a(x)- a(y)=- \sum\limits_{|\leq | \nu | \leq q-1} \frac{1}{\nu !}
\left(\frac{\partial }{\partial x}\right)^\nu a(x) \cdot (y-x)^\nu -
\sum\limits_{|\nu|=q} \frac{a_\nu (x,y)}{\nu !}(y-x)^\nu. 
$$

This gives 
\begin{align*}
(P \wedge^s_1 - \wedge^s_1 P) \wedge^\sigma u =  \sum\limits_{1\leq |
    \nu | \leq q-1 } (-1)^{|\nu| +1} \left(\frac{\partial}{\partial x}\right)^\nu
  a (x) \cdot (x^\nu \wedge^s_1 ) * (\wedge^\sigma \varphi )\\
 + \sum\limits_{|\nu |=q}(-1)^{|\nu|+1} \int \frac{a_\nu (x, y)}{\nu
    !}(x-y)^\nu \wedge^s_1 (x-y )(\wedge^\sigma \varphi ) (y)
  dy. \tag{5.22} \label{chap3-eq5.22}
\end{align*}

We estimate the first sum in \eqref{chap3-eq5.22}. We have 
$$
|\ (\widehat{x^\nu \wedge^s_1})|= \left| \left(\frac{\partial}{\partial
  \xi}\right)^\nu (1 - \alpha (\xi )) |\xi|^s \right|\leq c_\nu (1 +
|\xi |)^{s- |\nu|}.  
$$

Hence
$$
|| (x^\nu \wedge^s_1) * (\wedge^\sigma \varphi)|| \leq C_\gamma || (1 +
|\xi|)^{s- |\nu|} |\xi|^\sigma \hat{\varphi}|| \leq c_\nu || \varphi
|| 
$$
since $s + \sigma \leq 1 $ and $|\nu | \geq 1$. Summing over $\nu$
with $|\nu| \leq q-1$, we obtain  
\begin{equation}
\sum\limits_{|\nu | \leq q-1} || \frac{(-1)^{|\nu | +1}}{\nu !}
\left[\left(\frac{\partial}{\partial x}\right)^\nu a\right] \left[(x^\nu \wedge^s_1 ) *
(\wedge^\sigma \varphi )\right]|| \leq c (q) || \varphi ||  \; |a|_{q-1}
\tag{5.23} \label{chap3-eq5.23}
\end{equation}
where 
$$
|a|_p = \sup_{x, | \nu | \leq p} \left| \left(\frac{\partial}{\partial
  x}\right)^\nu a(x) \right|. 
$$

Since\pageoriginale $|| \varphi || = || \hat{\varphi}|| = || Y(\xi)
\hat{u}|| \leq |Y|_0$. $||u||$ the right hand side of the inequality
\eqref{chap3-eq5.23} is less than or equal to  
$$
c(q) |a|_{q-1} | Y|_0 \cdot||u||. 
$$

Now we estimate the second sum. Write $|\xi|^\sigma$ as 
$$
|\xi|^\sigma = \alpha (\xi) |\xi|^\sigma + (1- \alpha (\xi))
|\xi|^\sigma = \alpha (\xi) |\xi|^\sigma + |\xi| \left\{ (1- \alpha
(\xi)) |\xi|^{\sigma-1} \right\} 
$$
where $\alpha (\xi) \in \mathscr{D}$, $\alpha (\xi) \equiv 1$
in a neighbourhood of the origin. Thus $\wedge^\sigma = B_0 + \wedge
B_1$ where $B_0$ and $B_1$ are bounded operators in $L^2$. Hence we
have only to consider the part containing $\wedge B_1$. Denote by
$\psi_\nu$ the integral 
$$
\psi_\nu (x)= \int a_\nu (x, y) (x-y)^\nu \wedge^s_1 (x-y) \wedge B_1
\varphi (y)dy.  
$$

Now we can write $|\xi|= \Sigma \xi_j \dfrac{\xi_j}{|\xi|}$ and if
$R_j$ denote the Riesz operators defined by $(\widehat{R_j f}) =
\dfrac{\xi_j}{|\xi|} \hat{f}$ we can write $\wedge= \Sigma
\dfrac{\partial}{\partial x_j} R_j$. Substituting for $\wedge $ in
$\psi_\nu (x)$ 
$$
\psi_\nu (x)= - \Sigma_j \int \frac{\partial}{\partial x_j} \left\{
a_\nu (x, y) (x-y)^\nu \wedge^s_1 (x-y ) \right\} \cdot (R_j  B_1 \varphi)
(y)dv. 
$$


We observe that $(x^\nu \wedge^s_1)$ is a bounded function together with
its deri\-vatives of the first order for $|\nu| \geq n+2$. In fact its
Fourier image is $\left(\dfrac{\partial}{\partial \xi}\right)^\nu
\left\{ (1- \alpha (\xi)) |\xi|^s \right\}$ and  
$$
|x^\nu \wedge^s_1 | \leq \int |(\widehat{x^\nu \wedge^s_1})| d \xi 
\leq c_\gamma \int (1 + |\xi|)^{s-|\nu|} d \xi < \infty. 
$$

We can write 
\begin{align*}
\psi_\nu (x) & = \int a_\nu (x,y ) (x-y )^\nu  \wedge^s_1 (x-y)
(\wedge B_1 \varphi ) (y) dy\\ 
& = -\Sigma \Big\{ \int \left[\frac{\partial a_\nu}{\partial y_j} (x,
  y) \right]
(x-y)^\nu \wedge^s_1 (x-y) (R_j B_1 \varphi ) (y) dy\\
& \qquad  + \int a_\nu (x,y) \left[\frac{\partial }{ \partial y_j
  }((x-y)^\nu \wedge_1^s (x-y)) \right] (R_j B_1 \varphi ) (y ) dy.  
\end{align*}\pageoriginale

Set 
\begin{equation}
\psi_\nu (x)= I_1 + I_2. \tag{5.24}\label{chap3-eq5.24}
\end{equation}

We estimate $I_1$ and  $I_2$ separately.
\begin{gather*}
|I_1| \leq \sum\limits_j \big| \int \left[\frac{\partial a_\nu}{ \partial y_j}
  (x, y) \right] (x - y)^\nu \wedge^s_1 (x-y ) (R_j B_1 \varphi )
(y)dy \big|\\  
 \leq |a|_{q+1} \sum\limits_j \int |(x-y)^\nu \wedge^s_1 (x-y)|~| (R_j
 B_1 \varphi ) (y)dy. 
\end{gather*}

The Fourier image of $(1 +|x|^{2 P}) x^\nu \wedge^s_1 (x)$ is 
$$
\left(\frac{1}{2 \pi i}\right)^{|\nu |} \left(\frac{\partial}{\partial
  \xi}\right) \left[ (1- 
  \alpha (\xi ))| \xi|^s \right] + \left(\frac{1}{2 \pi i}\right)^{2p + |\nu|}
\triangle^p_\xi \left(\frac{\partial}{\partial \xi}\right)^\nu \left[(1- \alpha
  (\xi))|\xi|^s \right] 
$$
and hence 
\begin{align*}
|x^\nu \wedge^s_1 (x) | & \leq \frac{1}{1 + |x|^{2p}} \left\{
\left(\frac{1}{2 \pi}\right)^{|\nu|} \int |\left(\frac{\partial}
     {\partial \xi}\right)^\nu 
\left[(1-\alpha (xi )) |\xi^s| \right]| d \xi \right. \\
& \qquad \qquad \left. + \left(\frac{1}{2 \pi}\right)^{2p+|\nu|} \int  
| \triangle^p_\xi \left(\frac{\partial}{\partial \xi}\right)^\nu 
\left[(1- \alpha (\xi))| \xi|^s \right] | d \xi \right\} \\ 
 & \leq \frac{1}{1+|x|^{2p}} (C_1 (\nu ) + C_2 (p, \nu ))
\end{align*}
and similarly we have
{\fontsize{10pt}{12pt}\selectfont
\begin{align*}
\left|\left(\frac{\partial}{\partial x_j}\right) 
(x^\nu \wedge^s_1 (x)) \right| & =
\frac{1}{1+|x|^{2p}} \left|(1 + |x|^{2p} ) \frac{\partial }{\partial x_j}
(x^\nu \wedge^s_1 (x)) \right|\\ 
& \leq  \frac{1}{1+|x|^{2p}} \left\{ \left| \frac{\partial }{\partial x_j}
(x^\nu \wedge^s_1 (x)) \right|+ \left| |x|^{2p}
\left(\frac{\partial}{\partial x_j}\right) 
(x^\nu \wedge^s_1 (x))\right| \right\}\\ 
&\leq \frac{1}{1+|x|^{2p}} (C_2 ( \nu ) + C^2_1 (p, \nu )).
\end{align*}}\relax\pageoriginale

For sufficiently large $p$ the quantities $C_2 (p, \nu )$, $C'_2 (p,
\nu)$ are finite. Thus we have  
\begin{gather*}
|I_1| < |a|_{q+1} \sum\limits_j \int \frac{| (R_j B_1 \varphi )(y)|}{
  1+ |x-y|^{2p}} dy  \tag{5.25}\label{chap3-eq5.25}\\
|I_2| \leq \sum_j \left| \int a_\nu (x, y) \frac{\partial}{\partial y_j} \left[
  (x-y)^\nu \wedge^s_1 (x-y) \right] \cdot (R_j B_1 \varphi
)(y)\right| dy \\  
\leq |a|_q \Sigma \int \left| \frac{\partial }{\partial y_j} \left[(x-y)^\nu
  \wedge^s_1 (x-y) \right]\right| (R_j B_1 \varphi )(y)  dy  \\
|I_2|  \leq M (p)|a|_q \Sigma \int \frac{(R_j B_1 \varphi )(y)}{1+
  |x-y|^{2p}} dy . \tag{5.26}\label{chap3-eq5.26}
\end{gather*}

This leads to the inequality 
$$
||I_1 (x)||_{L^2} \leq |a|_{q+1} \sum^n_{j=1} ||R_j B_1 \varphi
||_{L^2} \left(\int \frac{1}{(1+ |x|^{2p})}dx\right) 
$$
because of the Hausdorff-Young theorem. We have the same kind
estimate for $|| I_2 (x) ||_{L^2}$. 

Hence
\begin{align*}
||\psi_\nu || & \leq C_3 (n)|| (R_j B_1 \varphi)|| \cdot |a|_{q+1} \leq C_4
(n) |a|_{q+1} ||\varphi ||\\ 
& \leq C_4 (n) |a|_{q+1} |Y (\xi)|_0 \cdot  || u||. 
\end{align*}

Now\pageoriginale summing up for all $l, m$ we have for any 
$u \in L^2$ 
\begin{align*}
|| (H \wedge^s_1 - \wedge^s_1 H) \wedge^\sigma u|| & \leq \sum_{l,m}
|| (a_{lm} \wedge^s_1- \wedge^s_1 a_{lm}) \wedge^\sigma (\tilde{Y}_{lm} *
u)||\\ 
& \leq C_5 (n) \left(\sum_{l,m} |a_{lm}|_{n+3}| Y_{lm} (\xi )
|| |_0\right) ||u||\\  
& \leq C_5 (n, s, \sigma ) M ||u ||_{L^2}
\end{align*}
and this completes the proof of the lemma.
\end{proof}

The following is a generalization of Friedrichs' lemma to singular
integral operators (see Mizohota \cite{key1}). 

\begin{proposition}\label{chap3-sec5-prop3}%3
Let $H$ be a singular integral operator such that its symbol $\sigma
(H) = h(x, \xi) \in C^\infty_{1+ \sigma}$, $ \sigma > 0$. Let
$C_\delta u$ denote, for $u \in L^2$, the commutator $[H
  \wedge, \varphi_\delta * ]u $ where $\varphi_\delta$ is the
mollifier of Friedrichs. 

Then
\begin{enumerate}
\renewcommand{\labelenumi}{\rm(\theenumi)}
\item  $|| C_\delta u|| \leq c M' ||u||$

\noindent
where $M'= |a_0 |_{\beta^{1+ \sigma}} + \sum\limits_{l,m}
|a_{lm}|_{\beta^{1+ \sigma}} | Y_{lm}|_{\beta^0}$ and $c$ depends only
on $\varphi$ and $n$ 

\item\quad $C_\delta u \rightarrow 0$ weakly in $L^2$ as $\delta
  \rightarrow 0$. 
\end{enumerate}
\end{proposition}

\begin{proof}%proof
We expand $h(x, \xi)$ in spherical harmonics $Y'_{lm}(\xi)$
$$
h(x, \xi ) = a_0 (x) + \sum_{l,m} a_{lm}(x) Y'_{lm} (\xi) 
$$
and hence we can write, denoting the inverse Fourier image of
$Y'_{lm}$ by $\tilde{Y}_{lm}$  
$$
Hu (x) = a_0 (x) u(x) + \sum_{l,m} a_{lm} (x) (\tilde{Y}_{lm} * u) (x).
$$

To prove (1) it is sufficient to prove it for $u \in
\mathscr{D}$. Now  
\begin{align*}
C_\delta u & = [H \wedge, \varphi_\delta *] u = H \wedge (u *
\varphi_\delta)- (H \wedge u) * \varphi_\delta \\ 
& =  \sum_{l,m } \left\{ a_{lm}(x) (\tilde{Y}_{lm} * \wedge (u *
\varphi_\delta ))- a_{lm}(x) (\tilde{Y}_{lm} * \wedge u)* \varphi_\delta
\right\}.\tag{5.27}\label{chap3-eq5.27} 
\end{align*}\pageoriginale

Consider a typical term of this sum:
$$
a_{lm} (x) (\tilde{Y}_{lm} * \wedge (u * \varphi_\delta )) - (a_{lm}(x) 
(\tilde{Y}_{lm} * \wedge u)) * \varphi_\delta 
$$
and substitute $\Sigma \dfrac{\partial}{\partial x_j} R_j $ for
$\wedge $ where $R_j$ are the Riesz operators. Put $\psi_{lm}(x)=
\tilde{Y}_{lm} * R_j * u$. We have 
\begin{align*}
& a_{lm} (x) (\tilde{Y}_{lm} * \frac{\partial}{\partial x_j} R_j * (u
  * \varphi_\delta )- (a_{lm}(x) (\tilde{Y}_{lm} *
  \frac{\partial}{\partial x_j} R_j * u )) * \varphi_\delta \\ 
&= a_{lm}(x) \left[\frac{\partial}{\partial x_j}(\tilde{Y}_{lm} * R_j * u)
    * \varphi_\delta \right] - (a_{lm} \frac{\partial}{\partial x_j}
  (\tilde{Y}_{lm} * R_j *  u)) * \varphi_\delta \\ 
&= a_{lm}(x) \left[\frac{\partial}{\partial x_j} \psi_{lm}(x) *
    \varphi_\delta \right] - \left[a_{lm} \frac{\partial}{\partial x_j}
    \psi_{lm} \right]* \varphi_\delta\\ 
& = \int \left[a_{lm} (x)- a_{lm} (y) \right] \;
  \left[\frac{\partial}{\partial y_j} 
    \psi_{lm}(y) \right] \varphi_\delta (x-y) dy  
\end{align*}
where the integral is taken in the sense of distributions. By
definition this is  
$$
- \int \frac{\partial}{\partial y_j} \left\{[a_{lm} (x)- a_{lm} (y) ]
\varphi_\delta (x-y) \right\} \psi_{lm} (y)  dy  
$$
where the integral is taken in the usual sense.

Now,
{\fontsize{10pt}{12pt}\selectfont
\begin{align*}
& \int \left|\frac{\partial}{\partial y_j}\left\{ \left[a_{lm}(x)-a_{lm}
    (y) \right] \varphi_\delta (x-y) \right\} \psi_{lm} (y) dy \right|\\  
& \leq \left| \int \psi_{lm} (y)(a_{lm} (x)- a_{lm}(y) )\frac{\partial
    \varphi_\delta}{\partial y_j} (x-y ) dy \right| + \left| \int
  \psi_{lm} (y) \varphi_\delta (x-y) \frac{\partial_{lm}(y)}{\partial
    y_j} dy) \right|\\ 
& \leq || \psi||_{lm} \left\{ 2|a_{lm}|_0 \left|\left| \frac{\partial
    \varphi_\delta}{\partial x_j}\right|\right|_{L^1} +|a_{lm}|_1 \cdot ||
  \varphi_\delta ||_{L^1}\right\}\\ 
& \leq || \psi_{lm}|| \left\{2 |a_{lm}|_0 c_1 (\delta, n) + |a_{lm}|_1
  c_2 (\delta, n)\right\}\\ 
& \leq c(\delta, n)|a_{lm}|_1 \cdot |Y' (\xi ) |_0 \cdot || u || 
\end{align*}}\relax\pageoriginale
which proves (1).

To prove (2) let $v \in L^2$ and consider
{\fontsize{10pt}{12pt}\selectfont
\begin{align*}
& \int v(x) \int \psi_{lm}(y) \frac{\partial}{\partial y_j}\left\{
  (a_{lm}(x)-a_{lm}(y)) \varphi_\delta (x-y) \right\}  dy \; dx\\ 
& = \int v(x) \int \psi_{lm} (y) \left\{ \frac{\partial
    \varphi_\delta}{\partial y_j} (x-y) \cdot (a_{lm} (x)-a_{lm}(y))-
  \varphi_\delta (x-y) \frac{\partial a_{lm}}{\partial y_j}(y) \right\}
   dy \; dx\\ 
& = \int v(x) \int \psi_{lm} (y) \left\{ \sum_k (x_{k}-y_k) \frac{\partial
    a_{lm}}{\partial y_k} (y) \cdot  \frac{\partial \varphi_\delta}{\partial
    y_j}(x-y)+ \sigma (x, y) \frac{\partial \varphi_\delta}{\partial
    y_j} (x-y) \right. \\
& \hspace{1cm}  \left. - \varphi_\delta (x-y) \frac{\partial a_{lm}}{\partial
    y_j}(y) \right\} dy \; dx
\end{align*}}\relax
where $\sigma (x, y) = a_{lm}(x) - a_{lm}(y) - \sum\limits_k (x_k-y_k)
\dfrac{\partial a_{lm}}{\partial y_k}(y)$. Let  
\begin{align*}
k_1 (y, x-y) & =  \sum_{k} (x_k-y_k) \frac{\partial a_{lm}}{\partial
  y_k}(y) \cdot \frac{\partial \varphi_\delta}{\partial y_j}(x-y)-
\varphi_\delta (x-y) \frac{\partial a_{lm}}{\partial y_j}(y)\\ 
& = - \frac{\partial}{\partial x_j} \left\{ \sum (x_k- y_k )
\frac{\partial a_{lm}}{\partial y_k}(y) \cdot \varphi_\delta
(x-y)\right\} \tag{5.28}\label{chap3-eq5.28}  
\end{align*}
and $(5.28)'$ \qquad $k_2 (y, x-y) \geq \sigma (x, v) \dfrac{\partial
  \varphi_\delta}{\partial y_j}(x-v)$ 

Then $|k_2 (y, x-y)| \leq c| a_{lm}(x)|_{1 + \sigma} |x-y|^{1+ \sigma}
\left|\dfrac{\partial \varphi_\delta}{\partial x_j}(x-y)\right|$. 

Applying the Hausdorff-Young inequality we have
{\fontsize{10pt}{12pt}\selectfont
\begin{align*}
||\int v (x) k_2 (y,x-y) dx || & \leq c | a_{lm} |_{1+\sigma} \left(\sum \int
|x-y|^{1+\sigma}\left| \frac{\partial \varphi_\delta}{\partial x_j} (x-y)
\right| dx\right) \cdot ||v|| \\
& = c|a_{lm} |_{1+\sigma}||v|| \varepsilon (\delta )
\tag {5.29} \label{chap3-eq5.29}
\end{align*}}\pageoriginale
where $\varepsilon (\delta) = \sum \int |x|^{1+\sigma}\left|
\dfrac{\partial \varphi_\delta}{\partial x_j} \right| dx \rightarrow 0$ as
$\delta \rightarrow 0$.  
On the other hand we observe that 
$$
\int k_1 (y,z) dz = \int \frac{\partial}{\partial z_j} \left\{ \sum
\limits_{r} z_r \frac{\partial a_{lm}}{\partial
    y_r}(y) \cdot  \varphi_\delta z \right\}dz = 0, 
$$
since $\varphi_\delta$ has compact support. Now consider 
$$
\int \int k_1 (y, x-y) v (x) \psi_{lm} (y) dy \; dx  = \int
\psi_{lm} (y) dy \int k_1 (y, x-y) v (x) dx. 
$$ 

The right hand side can be written after a change of variables $z=
x-y$ in the form  
$$
\int \psi_{lm} (y) dy \int v (y+z) k_1 (y, z) dz.
$$

Schwarz inequality gives 
$$
\big| \int \psi_{lm}(y) dy \int v (y+z) k_1 (y,z) dz \big| \leq ||
\psi_{lm} || ~\left|\left| \int k_1 (y,z) v (y+z) dz \right|\right|. 
$$ 

Since $\int k_1 (y, z) dz = 0$ we can write
$$
\left|\left|\int k_1 (y, z) v (y+z) dz \right|\right| = \left|\left|
\int k_1 (y,z) \big\{ v (y+z) - v (v)\big\} dz \right|\right|.  
$$

We shall now evaluate the right hand side. Let us set  
$$
\varepsilon' (\delta) = \sup\limits_{|h|\leq \delta} \left(\int |v(y  + h)
- v(y) |^2 dx\right)^{\frac{1}{2}}. 
$$

Schwarz inequality shows that 
\begin{align*}
&\left| \int k_1 (y, x-y) (v(x) - v(y)) dx \right|^2\\
&\qquad \leq \left(\int |
k_1 (y, x-y) dx \right) \left(\int | k_1 (y, x-y) || v (x)-v(y)|^2
dx\right). 
\end{align*}\pageoriginale

Clearly $\int |k_1 (y, x- y) | dx \leq c| a_{lm}|_{\beta^1}$ where
$c$ is a constant depending only on $\varphi$ and $\delta$. Hence
integrating both sides of this inequality with respect to $y$ we have  
$$
|| \int k_1 (y, x-y) (v(x) - v(y)) dx ||^2
$$
\begin{align*}
& \leq c|a_{lm}|_{\beta^1} \iint |k_1 (y, x-y) || v(x) - v(y)|^2 dx \;dy\\ 
& =  c|a_{lm}|_{\beta^1} \int\limits_{|z| \leq \delta} dz \int
  |k_1 (x-z,z) || v(x) - v (x-z)|^2  dx\; dz. 
\end{align*}

Since $k_1(y, x-y)$ is a bounded function the right side is less than 
\begin{equation*}
c'|a_{lm}|_{\beta^1} \varepsilon' (\delta )^2 (\text{vol }
\omega_{\delta}). \tag{5.30} \label{chap3-eq5.30}
\end{equation*}
where $\varepsilon' (\delta) \to 0$ as $\delta \to 0$ and
$\omega_{\delta}$ is the ball $|z| \leq \delta$. Combining the
inequalities \eqref{chap3-eq5.29} and \eqref{chap3-eq5.30} we obtain 
\begin{gather*}
 \big| \iint v(x) \psi_{lm} (y) \{ k_1 (y, x -y) + k_2 (y, x-y) \} dy
\; dx \big| \\ 
 \leq || \psi_{lm} || (c|a_{lm}|_{\beta^{1+\sigma}} || v || \varepsilon
(\delta ) + c'' |a_{lm}|_1 \varepsilon' (\delta ))\\ 
 \leq c'' || u ||(|a_{lm}|_{1+\sigma}| Y_{lm}|_o ||v|| \varepsilon
(\delta ) + |a_{lm}|_1 |Y_{lm}|_0 \varepsilon' (\delta )), 
\end{gather*}
which tends to 0 as $\delta \to 0$. This completes the proof of the
proposition. 
\end{proof}

\setcounter{corollary}{0}
\begin{corollary}\label{chap3-sec5-coro1}%coro 1
If we assume $u \in \mathscr{D}^1_{L^2}$ in proposition
\ref{chap3-sec5-prop3} then  
\begin{enumerate}
\renewcommand{\labelenumi}{\rm(\theenumi)}
\item $|| C_{\delta} u ||_{\mathscr{D}^1_{L^2}} \leq c|| u
  ||_{\mathscr{D}^1_{L^2}}$ 

\item $C_{\delta} u \to 0$ weakly in $\mathscr{D}^1_{L^2}$ as $\delta
  \to 0$. 
\end{enumerate}
\end{corollary}

\begin{proof}%proof 0
We\pageoriginale remark that
\begin{equation*}
\frac{\partial}{\partial x_j}(C_\delta u) = C_{\delta}\left(\frac{\partial}
     {\partial x_j}\right) + \left[H^{(j)}_{\wedge},\varphi_{\delta}*
       \right]u  \tag{$\ast$} 
\end{equation*}
where $H^{(j)}$ denotes the singular integral operator defined by 
$$
H^{(j)}_{u} = a_0^{(j)}u + \sum a^{(j)}_{1m}(\tilde{Y}_{1m}*u),
a^{(j)}_{1m} = \frac{\partial}{\partial x_j}a_{lm}, 
$$
or equivalently 
$$
\sigma(H^{(j)}) = a_0^{(j)}(x)+ \sum a^{(j)}_{1m}(x)
Y_{lm} (\xi) \in C^{\infty}_{\sigma} \text{~ with~ } \sigma > 0. 
$$

Now, the latter term of the right hand side in $(*)$ tends to 0 in 
$L^{2}$ as $\delta \to 0$. In fact, 
 \begin{align*}
\left[H^{(j)} \wedge, \varphi_\delta * \right] u  & =
H^{(j)}(\varphi_{\delta^*} 
\wedge u) - H^{(j)}_{\wedge^u}\\
& \quad  + H^{(j)} \wedge u-\varphi_\delta
*(H^{(j)}_{\wedge^u})\text{ and } \wedge u \in L^2. 
\end{align*}

Now applying Proposition (\ref{chap3-sec5-prop3}) to $(*)$ we have the
corollary.  

From Prop. \ref{chap3-sec5-prop1} it can be easily seen that the
following proposition 
holds. This plays the same role as G$\ring{\text{a}}$rding's inequalitv for
differential operators. 

\begin{proposition}%prop 4
Let $\mathscr{H}$  be a square matrix whose elements $H_{jk}$ are
singular integral operators (belonging to $C_\beta^{\infty}$) with
their symbols $\sigma(H_{jk}) = h_{jk}(x, \xi)\in
C_{\beta}^{\infty}$ with $\beta > 0\,(j, k=1,\ldots,N)$. Suppose
$\sigma \mathscr{(H)}$ is the matrix whose element are
$\sigma{(H_{jk})(x, \xi)}$ and satisfies the hypothesis  
\begin{equation*}
|\sigma (\mathscr{H})\alpha | \geq \delta | \alpha | \text{~ for every~ }
x, \xi \in \underbar{R}^n, \delta > 0 \tag{5.31} \label{chap3-eq5.31}
\end{equation*}
where $\alpha = (\alpha_1,\ldots, \alpha_N)$ is a complex vector in
$\underbar{C}^{N}$. Then for every $u = (u_1, \ldots , u_N)
\in \pi \mathscr{D}'_{L^2}$\pageoriginale
\begin{equation*}
||\mathscr{H}\wedge u ||^2 \geq \frac{\delta^2}{8}|| \wedge u||^2
-\gamma_{1} || u||^2,\tag{5.32} \label{chap3-eq5.32}
\end{equation*}
where $\gamma_1$ is a positive constant.
\end{proposition}

\begin{remark*}%rem 0
$|| u ||^2$, for $u=(u_1,\ldots, u_N)\in \pi
  \mathscr{D}^1_{L^{2}}$, denotes $|| u_1||^2+ \cdots || u_N||^2$. 
The proof runs on the same lines as in the proof of the
Prop. \ref{chap3-sec5-prop1}. 
\end{remark*}

\section{Energy inequalities for regularly hyperbolic
  systems}\label{chap3-sec6}%sec 6  

Let $\Omega$ denote the subset $\underbar{R}^n \times [0, h]$ of
$\underbar{R}^{n+1}$. 

\begin{defi*}%defi 0
A first order system of differential operators
\begin{equation*}
M = \frac{\partial}{\partial t} - \sum A_k(x,
t)\frac{\partial}{\partial x_k} \tag{6.1} \label{chap3-eq6.1}
\end{equation*}
is said to be regularly hyperbolic in $\Omega$ if 
\begin{enumerate}
\renewcommand{\labelenumi}{\rm(\theenumi)}
\item $A_k(x, t)$ are bounded,

\item for every $(x, t)\in \Omega$ and $\xi \in
  \underbar{R}^n$ the roots of the systems  
\begin{equation*}
\det  \left(\lambda I - \sum A_k(x,t) \cdot \xi_k\right) = 0
\tag{6.2}\label{chap3-eq6.2} 
\end{equation*}
are real and distinct; further if $\lambda_1(x, t, \xi )\cdots
\lambda_N (x, t, \xi)$ are these roots then  
\begin{equation*}
\inf_{\substack{(x, t)\in \Omega \\ j \neq k }}, | \xi |
=1^{| \lambda_j(x, t, \xi ) -\lambda_k(x, t, \xi ) |>0}
\tag{6.3}\label{chap3-eq6.3} 
\end{equation*}
\end{enumerate}

We write the system \eqref{chap3-eq6.1} in terms of singular integral
operators, by 
putting $\sum A_k (x,t) \dfrac{\partial}{\partial x_k} = i
\mathscr{H}(t)\wedge$ where $\mathscr{H}(t)$ is a matrix of order $N$
of singular integral operators whose symbol is the matrix 
$$
\sigma (\mathscr{H}(t)) = 2 \pi \sum A_k(x,t)\frac{\xi_k}{|\xi|}. 
$$\pageoriginale
\end{defi*}

Thus \eqref{chap3-eq6.1} is written in the form 
\begin{equation*}
M = \frac{\partial}{\partial t}-i \mathscr{H}(t) \wedge \tag*{$6.1)'$} 
\end{equation*}

If the coefficients are such that $A_k = A_k(x, t) \in
\beta^{1+\sigma}[0, h]$ with $\sigma >0$ then for each fixed
$t,\sigma(H)(x, t, \xi)\in C^{\infty}_{1+\sigma},  \sigma >0$. 
\end{proof}

\setcounter{proposition}{0}
\begin{proposition}[Petrowsky]\label{chap3-sec6-prop1} %1
Let $M$ be a regularly hyperbolic system with $A_k
\in \beta^{1+\sigma}[0, h]$. Suppose $A_k(x, t)$ are real. Then there
exists a matrix $\sigma(\mathfrak{N}(t)) =
\sigma(\mathfrak{N})(x, t, \xi)$ except possibly when $n
= 2$ such that 
\begin{enumerate}[\rm(i)]
\item $\sigma(\mathfrak{N}(t))\sigma(\mathscr{H}(t)) =
  \sigma(\mathscr{D}(t)) \sigma(\mathfrak{N}(t))$ where 
$$
\sigma (\mathscr{D}(t)) = 
\begin{pmatrix}
\lambda_1(x,t, \xi) & & 0 \\
 & \ddots & &\\
0  & & \lambda_N(x, t,\xi)
\end{pmatrix}
$$

\item $\sigma(\mathfrak{N}(t)) = \sigma (\mathfrak{N})(x, t, \xi )$ is
  of class $C_{1 + \sigma}$ for every fixed $t$, has real elements
  and further 
\begin{equation*}
 | \det \sigma (\mathfrak{N}(t))| \geq \delta' > 0 \text{~ for every~ }
 (x, t) \varepsilon \Omega , \xi \varepsilon
 \underbar{R}^n.\tag{6.4}\label{chap3-eq6.4}  
\end{equation*}

\item the mapping $t \to \sigma (\mathfrak{N}(t)) \in
  C_{1+\sigma}^{\infty}$ is once continuously differentiable 
\end{enumerate}
\end{proposition}

\begin{proof}%proof 0
Since the roots of \eqref{chap3-eq6.2} det $(\lambda I-\sum A_k \cdot
\xi_k) = 0$ are 
real and distinct it follows that $\lambda_j(x, t, \xi)$ are single
valued functions on $|\xi| = 1$ for every fixed $(x, t)\in
\Omega$. This follows by the principle\pageoriginale of monodromy in
the case $n > 2$ and in the case $n=2$ by virtue of hyperbolicity.  

To see that $\lambda_j(x, t, \xi ) \in C^{\infty}_{1+\sigma}$,
$\sigma > 0$ for fixed $t$ denoting by $$
P (\lambda, x, t, \xi) = 0
$$ 
the characteristic equation 
$$
\det \left(\lambda I-\sum A_k \cdot \xi_k\right) = 0 
$$
we have from the implicit function theorem
$$
\frac{\partial \lambda_j}{\partial x_k} = - \bigg(\frac{\partial P}{\partial
  x_k} \bigg| \frac{\partial P}{\partial \lambda}\bigg)_{\lambda =
  \lambda_j} 
$$
and further $\big|\left(\dfrac{\partial P}{\partial
  \lambda}\right)_{{\lambda} = {\lambda_j}} \big| \geq d^{N-1}$ where
$d=\inf\limits_{\substack{(x, t) 
  \in \Omega, |\xi|=1\\ j\neq k}} |\lambda_j - \lambda_k|$. 
\end{proof}

\smallskip
\noindent
\textbf{Construction of $\sigma(\mathfrak{N}(t))$}. Suppose $n \geq
3$. To find $\sigma (\mathfrak{N}(t))$ such that $\sigma 
(\mathfrak{N}(t)) \sigma (\mathscr{H}(t)) =
\sigma(\mathscr{D}(t))\sigma (\mathfrak{N}(t))$ is the same, if we
write $\sigma (\mathfrak{N} = (n_{jk})$, $\sigma
(\mathfrak{N})=(a_{jk})$, as finding a matrix solution of 
$$
\lambda_j n_{jl} = \sum_k n_{jk}  a_{kl}.
$$

For a fixed $j$ the vector $(n_{j1},\ldots, n_{jN})$ is an eigenvector
of the matrix $A =(a_{jk})$ corresponding to the eigenvalue
$\lambda_j$. Consider the case $\lambda_j = \lambda_1$. We assert that
the space of eigenvectors at the point $(x, t, \xi)$ can be given by
explicit expressions (the space of eigenvectors is one dimensional) in
such a way that this vector is continuous in $(x, t, \xi)$ is class
$C^{\infty}_{1+\sigma}$ and continuously differentiable in $t$. In
fact, if $M_{jk}(t)$ is the $(j, k)$-cofactor of $(\lambda_1 I -A)$ then
$(M_{1j}, M_{2j},\ldots, M_{Nj})$ $(j = 1,\ldots, N)$\pageoriginale
span the space of eigenvectors. As the rank of $(\lambda_1I-A)$ is
$(N-1)$ everywhere one of these is not trival.  

\begin{remark*}%rema 0
In the case where the coefficients $A_k(x, t)$ are not real there will
be topological difficulties in the above reasoning which proves the
existence of smooth $\sigma \mathfrak{N}(x, t, \xi)$. It should however
be observed that the theorem of local existence of smooth $\sigma
\mathfrak{N} (x, t, \xi)$ remains valid. Therefore it would be better
to use a partition of unity to derive energy inequalities for such
systems. Moreover this argument can be applied for more general
hyperbolic systems. (See: Le probl\`eme de Cauchy pour les syst\`emes
hyperboliques et paraboliques, Mem. Coll. Sc., Kyoto
Univ,. Ser. A. Math., 1959). 
\end{remark*}

\begin{proposition}[Energy inequality]\label{chap3-sec6-prop2}%2
Let 
$$
M = \frac{\partial}{\partial t} -  \sum A_k (x, t)
\frac{\partial}{\partial x_k} 
$$
be a regularly hyperbolic system in $\Omega$ with the coefficients
$A_k (x, t)$ satisfying  
$$
A_k \in \mathbb{B}^{1+ \sigma} [0, h],
\; \frac{\partial}{\partial t} A_k \in B^0 [0, h]. 
$$

Suppose $B \in \mathbb{B}^0 [0, h]$, $f \in L^2 [0, h]$
given. Then, if $u \in L^2 [0, h]$ is a solution of  
\begin{equation*}
\frac{\partial u}{\partial t} - \sum A_k (x, t) \frac{\partial
  u}{\partial x_k} - B(x, t) u = f \tag{6.5} \label{chap3-eq6.5}
\end{equation*}
we have the inequality 
\begin{equation*}
|| u (t) || \leq c(h) \left\{ || u (0) || + \int\limits^t_0 || f(s) || ds
\right\}. \tag{6.6} \label{chap3-eq6.6}
\end{equation*}
\end{proposition}

\begin{proof}
First\pageoriginale we assume this $u \in
\mathscr{D}^1_{L^2}[0, h]$. The given system is written in
singular-integral-operator form as   
\begin{equation*}
\frac{\partial u}{\partial t} - i \mathscr{H} (t) \wedge u - B(t) u =  
f. \tag{6.7} \label{chap3-eq6.7}
\end{equation*}

Multiplying this system by the matrix $\mathfrak{N}$ obtained in
Prop. \ref{chap3-sec6-prop1} we obtain  
$$
\frac{\partial}{\partial t} (\mathfrak{N} u) - i \mathfrak{N} (t) 
\mathscr{H}(t)) \wedge u - (\mathfrak{N} B + \frac{\partial
  \mathfrak{N}}{\partial t}) u = \mathfrak{N} f.  
$$

By Prop. \ref{chap3-sec6-prop1} $\mathfrak{N} \circ \mathfrak{N} = D
\circ \mathfrak{N}$ which  implies that   
$$
\mathfrak{N} \mathscr{H} \wedge \equiv \mathscr{D} \mathfrak{N}
\wedge (\text{mod. bounded operators}) 
$$
because $(\mathfrak{N} \mathscr{H}) \wedge \equiv (\mathfrak{N}) \circ
\mathscr{H} \wedge$  (mod bounded operators)  
$$
(\mathscr{D} \mathfrak{N}) \wedge \equiv (\mathscr{D} \circ \mathfrak{N})
\wedge (\text{mod. bounded operators}) 
$$

Also $(\mathscr{D} \mathfrak{N}) \wedge = \mathscr{D} \wedge
\mathfrak{N} +  $ a bounded operator, and hence the new system becomes   
$$
\frac{\partial}{\partial t} (\mathfrak{N} u)= i \mathscr{D} \wedge
(\mathfrak{N} u) + (\mathfrak{N} B + \frac{\partial
  \mathfrak{N}}{\partial t} u + \mathfrak{N} f.  
$$

In otherwords $v = \mathfrak{N} u $ satisfies 
$$
\frac{\partial v}{\partial t} = i \mathscr{D} \wedge v + B_1 u +
\mathfrak{M} f 
$$
where $B_1 = \left(\mathfrak{N} B + \dfrac{\partial \mathfrak{N}}{\partial
  t}\right)$ is a bounded operator in view of
Prop. \ref{chap3-sec6-prop1}. Now   
\begin{align*}
\frac{\partial}{\partial t}(v, v) & = (i \mathscr{D} \wedge v, v) +
(v, i \mathscr{D} \wedge v) + 2 \re (B_1 u + \mathfrak{N} f, v)\\ 
& = i ( \mathscr{D} \wedge  - \wedge\mathscr{D}^\ast) v,v) + 2 \re (B_1
u + \mathfrak{N} f, v).  
\end{align*}
 
But $\wedge \mathscr{D}^* =  \wedge \mathscr{D}^{\#} + $ a bounded
operator, and since $\mathscr{D}$ is real $\mathscr{D}^{\#} =
\mathscr{D}$ and\pageoriginale $\wedge \mathscr{D} = \mathscr{D}\wedge
+$ a bounded operator. Hence $\mathscr{D}\wedge - \wedge
\mathscr{D}$ is a bounded operator and   
$$
\frac{\partial}{\partial t}||v||^2 \leq 2 \gamma_1  ||v||^2 + 2c
||u|| ~||v||+2 ||\mathfrak{N} f||~ ||v||, 
$$
that is 
$$
\frac{\partial}{\partial t}||v|| \leq \gamma ||v|| +c ||u||+||
\mathfrak{N} f||.  
$$

By the regular hyperbolicity we have in view of Prop. \ref{chap3-sec6-prop1}
\begin{equation*}
|det \sigma (\mathfrak{N}(t))|\geq \delta' >
0. \tag{6.4}\label{addchap3-eq6.4}  
\end{equation*}

Hence by the generalized Garding inequality applied to $\mathfrak{N}$
there exist $\delta''> 0$ and $ \beta > 0$ such that 
\begin{equation}
|| \mathfrak{N}\wedge u|| \geq \delta'' || \wedge u||- \beta ||
u||. \tag{6.8}\label{chap3-eq6.8} 
\end{equation}

Define
\begin{equation*}
||| u ||| = || \mathfrak{M} u || + \beta || (\wedge + 1)^{-1} u ||
\tag{6.9} \label{chap3-eq6.9}
\end{equation*}
where $(\wedge+1)^{-1} u \xrightarrow{\mathscr{F}} \dfrac{1}{(1+|\xi|)}
\hat{u}$. It is clear that $||| u||| \leq c_{1}|| u||$ since
$\mathfrak{N}$ and $(\wedge + 1)^{-1}$  are bounded. On the other hand 
$$
\mathfrak{N} u = \mathfrak{N} \wedge (\wedge + 1)^{-1} u + \mathfrak{N}
(\wedge+1)^{-1}u 
$$
implies 
\begin{align*}
|| \mathfrak{N} u || &\geq || \mathfrak{N}\wedge (\wedge  + 1)^{-1} u||-||
\mathfrak{N}(\wedge + 1)^{-1} u||\\
&\quad \geq \delta'' || \wedge(\wedge + 1)^{-1} u|| -\beta ||(\wedge +
  1)^{-1}u|| -|| \mathfrak{N}(\wedge + 1)^{-1}u||\\ 
&\quad \geq \delta'' || \wedge(\wedge + 1)^{-1} u|| -\beta' ||(\wedge +
  1)^{-1}u|| \\ 
&\quad \geq \delta'' ||u|| -(\beta'+ 1)|| (\wedge + 1)^{-1}u|| 
\end{align*}
which\pageoriginale proves that $||| u ||| \geq c_2 || u|| $
consequently the norms $||| u||| $ and $|| u||$ are equivalent. It is
therefore sufficient to prove the energy inequality for the norm $|||
u |||$.  
\begin{align*}
\frac{\partial}{\partial t} ||| u(t)||| & = \frac{\partial}{\partial
  t}(|| \mathfrak{N} u || + \beta || (\wedge + 1)^{-1}u ||) \\
& \leq \gamma || \mathfrak{N} (u) ||+ c || u|| +|| \mathfrak{N} f|| +\beta
\frac{\partial}{\partial t}|| (\wedge+1)^{-1}u
||. \tag{6.10}\label{chap3-eq6.10}  
\end{align*}

Considering $\dfrac{\partial u}{\partial t} = i \mathscr{H}\wedge u +
Bu + f$ 
$$
(\wedge + 1)^{-1} \frac{\partial u}{\partial t} = i(\wedge +
1)^{-1}\mathscr{H} \wedge u + (\wedge + 1)^{-1}(Bu + f) 
$$
but $(\wedge + 1)^{-1}\mathscr{H} \wedge = (\wedge+1)^{-1} \wedge
\mathscr{H}  + (\wedge+1)^{-1}B_{2}$  
where $B_{2}$ is a bounded operator in $L^{2}$ and hence
$$
\frac{\partial}{\partial t}||(\wedge + 1)^{-1}u|| \leq \delta_{o} ||u
|| +|| (\wedge + 1)^{-1}f ||. 
$$

Substituting in the inequality \eqref{chap3-eq6.10} we obtain 
$$
\frac{\partial}{\partial t}||| u (t)||| \leq \gamma' ||| u (t)||| +
||| f |||, 
$$
which, on integration with respect to $t$, gives
$$
||| u (t) ||| \leq ||| u (0) ||| \exp (\gamma' t) + \int^{t}_{0} ||| f
(S) ||| \exp (\gamma'(t-s)) ds. 
$$ 

Since $||| u (t) ||| \sim || u(t)|| $ we obtain the required
inequality 
 $$
|| u (t)|| \leq c(h) \{|| u(0)|| + \int\limits^{t}_{0}|| f (s) || ds.
$$

In the general case in which $u \in L^{2}[0, h]$ we
regularize it by the the mollifiers $\varphi_\delta$ of Friendriche
and apply the above argument to the function\pageoriginale $u_\delta =
\varphi_\delta * {}_{(x)}u$ and pass to the limits as $\delta \to 0$ in
the inequality for $u_\delta$ to obtain the energy inquality for $u$. 
\end{proof}

\begin{remark*}%rem 0
In the above proof the norm $||| u |||$ depends a priori on the
parameter $t$ since it involves the operator $\mathfrak{N}(t)$. When
$t$ runs through a bounded set the constant $\beta$ in the
definition of $||| u |||$ can be chosen to be independent of
$\mathfrak{N}$.  
\end{remark*}

In the following proposition we prove that, if $A_k$ and $B$ are
differentiable of sufficiently high order, then there exists an energy
inequality for higher order derivatives. 

\begin{proposition}\label{chap3-sec6-prop3}%prop 3
Let $M$ be a regularly hyperbolic system with  
$A _k (x, t)\in \mathscr{B}^{\max (1 + \sigma, m)}[0, h]$, $0 <
\sigma < 1$, $\dfrac{\partial}{\partial t} A_k(x, t) \in
\mathscr{B}^{0}[0, h]$. 
Suppose $B(x, t)\in \mathscr{B}^{m}[0, h]$, and $f(x, t)\in
\mathscr{D}^{m}_{L^{2}}[0, h]$ are given. If $u \in
\mathscr{D}^{m}_{L^{2}}[0, h]$ is a solution of 
$$
(M-B) u = f
$$
then 
\begin{equation*}
|| u (t) ||_m \leq c_{m}(h) \left\{ || u(0)||_m  + \int^{t}_{0} ||
f(s)||_m ds\right\}.  \tag{6.11}\label{chap3-eq6.11}
\end{equation*}
\end{proposition}

\begin{proof}
It is sufficient to prove the proposition for the case $m=1$ and the
general case will follow by repeated application of the argument. Let
$\dfrac{\partial u}{\partial x_j} = u^{(j)}$. Then 
$$
M[u^{(j)}] = \sum_k \frac{\partial A_{k}}{\partial x_{j}}(x,t)
\frac{\partial u}{\partial x_{k}} + \frac{\partial B}{\partial x_{j}}
(x,t) u + \frac{\partial f}{\partial x_{j}}, j = 1,2, \ldots,n 
$$
that is $u^{(j)}$ satisfy a regularly hyperbolic system with new $B$ and $f$.
Denoting\pageoriginale $\sum\limits^n_{j=1} ||| u^{(j)} |||$ by
$\varphi_1(t)$ we obtain  
$$
\frac{d \varphi_1}{dt}(t)\leq \gamma_1 \varphi_1 (t) + \sum\limits_j |||
\frac{\partial f}{ \partial x_j} ||| +\sum\limits_j  ||| \frac{\partial
  B}{\partial x_j}u||| 
$$
which on integration yields the required inequality 
$$
|| u (t) ||_1 \leq c_1 (h) \left\{ || u(0)||_1 + \int\limits^t_0 ||
f(s)||_1 ds\right\}. 
$$

In the following we duduce on energy inequality for solutions of a
single regularly hyperbolic differential equation of order $m$.  

Consider the evolution equation
\begin{equation}
\left(\frac{\partial}{\partial t}\right)^m u + \sum_{\substack{j+| \nu| \leq
      m\\j\leq m-1}} a_{j,\nu}(x,t)\left(\frac{\partial}{ \partial
  x}\right)^\nu   \left(\frac{\partial}{\partial t}\right)^j
u=g. \tag{6.12} \label{chap3-eq6.12} 
\end{equation}

The principal part of this is by definition the homogeneous
differential operator of order $m$ 
\begin{equation}
\left(\frac{\partial}{\partial t}\right)^m + \sum_{\substack{|\nu | +j= m\\ j\leq
      m-1}} a_{j,\nu}(x,t)\left(\frac{\partial}{ \partial x}\right)^\nu
  \left(\frac{\partial}{ \partial t}\right)^j  \equiv L
  \tag{6.13} \label{chap3-eq6.13} 
\end{equation}
which we write in the form
$$
L \equiv \left(\frac{\partial}{\partial t}\right)^m + \sum \limits^m_{j=1} h_j
\left(x, t,  \frac{\partial}{\partial x}\right)\left(\frac{\partial}{\partial
  t}\right)^{m-j} 
$$
where $h_j \left(x, t, \dfrac{\partial}{\partial x}\right) = \sum\limits_{| \nu
  | = j} a_{m-j, \nu} (x, t) \left(\dfrac{\partial}{\partial x}\right)^\nu$. The
given operator is said to be regularly hyperbolic if the polynomial
equation 
\begin{equation}
\lambda^m + \sum_{j} h_j (x, t,\xi) \lambda^{m-j} = 0
\tag{6.14}\label{chap3-eq6.14} 
\end{equation}
has real and distinct roots for every $(x, t) \epsilon \Omega$; $| \xi
| =1$. $h_j \left(x, t, \dfrac{ \xi}{| \xi |}\right)$  can be\pageoriginale
considered as the symbol of a singular integral operator $H_j$ and
hence we can represent  
$$
h_j \left(x, t, \frac{ \partial}{ \partial x}\right) = H_j (i \wedge)^j 
$$
and
\begin{equation}
L \equiv \left(\frac{\partial}{\partial t}\right)^m + \sum\limits^m_{j=1} H_j (i
\wedge )^j \left(\frac{\partial}{\partial
  t}\right)^{m-j}. \tag{6.15} \label{chap3-eq6.15} 
\end{equation}

Setting
\begin{align*}
v_1 & = \left(\frac{\partial}{ \partial t}\right)^{m-1} u\\
v_2 & = i(\wedge + 1)\left(\frac{\partial}{\partial t}\right)^{m-2} u\\
v_j & = \{ i (\wedge + 1 ) \}^{j-1}\left(\frac{\partial}{\partial t}\right)^{m-j}
u\\ 
v_m & = \{ i (\wedge + 1) \}^{m-1} u
\end{align*}

We see that  \; $(i \wedge)^{j-1} = (i \wedge )^{j-1}\{ i 
(\wedge + 1) \}^{-(j-1)} \{ i (\wedge + 1) \}^{j-1}$

\hspace{2.6cm} $ = (1 + S_{j-1})\{ i(\wedge + 1) \}^{j-1}$

\smallskip
\noindent
where $\sigma(S_{j-1}) = \left(\dfrac{| \xi|}{1+ | \xi
  |}\right)^{j-1}-1$. $S_{j-1} \wedge$ is a bounded operator in $L^2$. Then  
the principal part is rewritten as 
\begin{align*}
L[u] & =  \left(\frac{\partial}{\partial t}\right)^m u + i \sum H_j \wedge (1 +
S_{j-1}) \{ i(\wedge + 1)\}^{j-1} \left(\frac{\partial}{\partial t }\right)^{m-j}
u\\ 
& = \frac{\partial}{ \partial t} v_1 + i \sum H_j \wedge v_j + i \sum
H_j \wedge S_{j-1} v_j. 
\end{align*}

Then $ v = \begin{pmatrix} v_1 \\  \vdots \\ v_n \end{pmatrix}$
satisfies the system of first order equations 
\begin{equation}
\frac{\partial}{\partial t}  v = i \mathscr{H} \wedge v + B v + f
\tag{6.16} \label{chap3-eq6.16}
\end{equation}
where\pageoriginale
\begin{equation}
\sigma (\mathscr{H}) = 
\begin{pmatrix} 
1 & & \\
&  1_{\ddots} &  \\
-h_1 &  -h_2\cdots -h_{m-1} & -h_m   
\end{pmatrix},\tag{6.17} \label{chap3-eq6.17}
\end{equation}
$B$ a bounded operator and $f = \begin{pmatrix}
 0 \\
 \vdots \\
0\\
 g \end{pmatrix}$. 

Let $P(\lambda) = \det (\lambda I-\sigma (\mathscr{H})) =
\lambda^m+\sum_{j} h_j \left(x, t, \dfrac{\xi}{|\xi|}\right)^{m-j}$. Thus the given
equation is regularly hyperbolic if and only if the associated first
order system is. 

\begin{proposition}\label{chap3-sec6-prop4}%prop 4
Suppose $P(\lambda) = 0$ has real and distinct roots $\lambda_1(x, t,
\xi) < \cdots < \lambda_N (x, t, \xi)$ such that 
\begin{equation}
\inf_{\substack{(x, t) \epsilon  \Omega , |\xi |=1  \\  j \neq k}} \big|
\lambda_j (x, t, \xi ) - \lambda_k (x, t, \xi ) \big| = d > 0
\tag{6.18}\label{chap3-eq6.18}  
\end{equation}
and further the coefficients are such that
\begin{align*}
& a_{j,\nu }\epsilon \mathscr{B}^{1+\sigma}[0, h],
\frac{\partial}{\partial t}a_{j,\nu} \epsilon \mathscr{B}^0 [0,h]
\text{ for } j+|\nu | = m \\ 
& a_{j, \nu} \epsilon \mathscr{B}^0 [0, h] \text{ for } j+| \nu | \leq
m-1. 
\end{align*}
\end{proposition}

Let $g \epsilon L^2 [0, h]$ be given. If $u \epsilon
\mathscr{D}^m_{L^2}[0, h]$ is a solution of \eqref{chap3-eq6.12} then 
\begin{equation}
|| v(t)||' \leq C_0 (h) \left\{ || v(0)||'+ \int^t_0 || f(s)||'
 ds\right\} \tag{6.19} \label{chap3-eq6.19}
\end{equation}
where $|| v(t)||^{12} = \sum\limits^m_{j=1}||(\dfrac{\partial}{\partial 
  t})^{m-j} u ||^2_{j-1}$. 

This proposition is proved easily using the energy inequality for the
associated first order system. 
\end{proof}

\section{Uniqueness theorems}\pageoriginale\label{chap3-sec7}%sec 7

From the energy inequalities obtained in the previous section some
results on the local uniqueness follow immediately. We shall show that
a solution of a homogeneous regularly hyperbolic system of equations
vanishes identically in a cone if the cauchy data is zero. This was
first proved by Holmgren and later made precise by F. John \cite{key1}. 

Consider the first order system of equations 
\begin{equation}
M[u] \equiv \frac{\partial u}{ \partial t}- \sum A_k (x, t)\frac{
  \partial u}{\partial x_k} - B(x, t) u = 0 \tag{7.1}\label{chap3-eq7.1} 
\end{equation}
where $M$ is regularly hyperbolic in $\Omega = \underline{R}^n
\times[0, h]$. 

\setcounter{proposition}{0}
\begin{proposition}\label{chap3-sec7-prop1}%prop 1
Let $M$ be regularly hyperbolic in $\Omega$ with $A_k \epsilon
\mathscr{B}^{1 + \sigma}_{x, t}$,\break $B \epsilon \mathscr{B}^0_{x,
  t}$. If $u \epsilon \mathscr{E}^1_{x, t}$ satisfies $M[u] = 0$ and
$u(x, 0) \equiv 0$ in a neighbourhood $U$ of the origin in
$\underline{R}^n_x$ then $u \equiv 0$ in a neighbourhood of the origin
in $\Omega$. 
\end{proposition}

\begin{proof}
Let $D_\epsilon\subset \Omega$ be the set $\left\{(x, t) \epsilon \Omega: | x|^2
+ t < \epsilon, t \geq 0 \right\}$. We first make a change of variables 
\begin{equation}
t'=t + \sum x^2_j, x'_j=x_j (j=1, \ldots, n). \tag{7.2} \label{chap3-eq7.2}
\end{equation}

Under this transformation let $\tilde{u}(x'_{k}, t') = u(x, t)$ then the
system of equations is transformed into the system 
\begin{equation}
\left(I-2 \sum x'_k \cdot A_k\right)\frac{\partial \tilde{u}}{\partial
  t'} =\sum A_k 
  \frac{ \partial \tilde{u}}{\partial x'_k} + B
  \tilde{u}. \tag{7.3} \label{chap3-eq7.3} 
\end{equation}
$D_\epsilon$ is transformed into a strictly convex domain
$\tilde{D}_\epsilon$ bounded by $t'= \sum x'^2_j$,
$t'=\epsilon$.\pageoriginale $\tilde{u}$ is defined in the domain
$\tilde{D}_\epsilon$ and we extend $\tilde{u}$ outside
$\tilde{u}_\epsilon$ by 0 and we denote this again by
$\tilde{u}$. Clearly $\tilde{u}\epsilon \mathscr{E}^1$ since it
vanishes identically in a neighbourhood of $t'= \sum x'^2_j$. Thus
$\tilde{u}$ has its support in $\tilde{D}_\epsilon$. It follows from
lemma \ref{chap3-sec7-lem1} that if $x'$ is in a small neighbourhood
of the origin (it 
is sufficient to take $2|x'|A$), $(I-2 \in x'_k A_k)$ is
invertible and the eigenvalues of $(I-2 \sum x'_k A_k)^{-1} \sum
A_k \cdot \xi_k$ are real and distinct since those of $\sum A_k \cdot \xi_k$
are. Thus the transformed system remains regularly hyperbolic in
$\tilde{D}_\epsilon$. Extending $A_k(x, t)$, $B(x, t)$ to the whole of
$\underline{R}^n_x[0, h]$ in such a way that the system remains
regularly hyperbolic we obtain $\tilde{M}[u] = 0$ in $\underline{R}^n
\times [0, h]$ (this can be achieved by taking the inverse image by a
suitable differentiable \textit{retraction} of $\underline{R}^n \times
[0, h]$ to $\tilde{D}_\epsilon$. 
\begin{equation}
\frac{\partial \tilde{u}}{\partial t'}= \sum \left(I-2 \sum x'_k \cdot
A_k\right)^{-1} 
\left(A_k \frac{\partial \tilde{u}}{\partial x'_k}\right)+ \left(I-2 \sum x'_k
A_k\right)^{-1}  B\tilde{u}. \tag{7.4} \label{chap3-eq7.4}
\end{equation}
$\tilde{u}$ has Cauchy data zero and hence the energy inequality shows
that $\tilde{u}(x', t')\equiv 0$ and hence $u$ vanishes on
$D_\epsilon$. 
\end{proof}

Similarly it can be proved that $u$ vanishes in $D^{-1}_\epsilon=\{(x,
t): t\leq 0$, $\sum x_j^2 + t < \epsilon$ and this completes the
proof. We now prove the following lemma due to H.F. Weinberger
(Weinberger \cite{key1}). 


\setcounter{lemma}{0}
\begin{lemma}\label{chap3-sec7-lem1}%lemm 1
Suppose $A$ is a constant matrix such that for all real $\xi \neq 0$,
$\det (\lambda I-\sum A_k \cdot  \xi_k) = 0$ has real and distinct roots
$\lambda_1 (\xi) <\ldots< \lambda_N(\xi)$. If $\lambda_{\max}$ denotes
$\sup\limits_{|\xi | = 1}( \lambda_N (\xi))$ and $\alpha =
\left(\begin{smallmatrix}\alpha_1\\ \vdots\\ \alpha_N \end{smallmatrix}\right)$
is a real vector $\neq 0$ with $| \alpha | \leq \dfrac{
  1}{\lambda_{\max}}$ then $\det (\mu B-\sum A_k \cdot  \xi _k) = 0$,
$B=I-A \cdot \alpha$, has real and\pageoriginale distinct roots for
any real $\xi \neq 0$. 
\end{lemma}

\begin{remark*}%rema 0
From the choice of $\alpha$ it follows that $B$ is invertible. 
\end{remark*}

\begin{proof}%proo 0
First we assert that all the eigen values $\nu_k$ of $B$ are
positive. For, they are the roots of  
\begin{align*}
\det (\nu I - B) & = \det (\nu I-(I-A \cdot \alpha ))\\
& = (-1)^N \det ((1- \nu ) I-A \cdot  \alpha ) = 0.
\end{align*}
and hence
$$
1-\nu_k = \lambda_k (\alpha) = |\alpha| \lambda_k
(\frac{\alpha}{|\alpha|}) 
$$
which implies that
\begin{equation}
\nu_k = 1- |\alpha | \lambda_k (\frac{\alpha}{|\alpha|}) > 0
\tag{7.5}\label{chap3-eq7.5}   
\end{equation}
since $\lambda_k (\xi) < \dfrac{1}{|\alpha|}$ on $|\xi| = 1$. Consider  
$$
\det (\mu B-\lambda I- A \cdot \xi) = (-1)^N \det 
((\lambda- \mu) I + A(\xi + \mu \alpha)) = 0  
$$
and let $\varphi_1 (\mu), \ldots, \varphi_N (\mu)$ be the roots of
the equation (with respect to $\lambda$) 
$$
\det ((\lambda -\mu)I + A(\xi+ \mu\cdot \alpha)) = 0 
$$
for a fixed $\xi$. We can write
$$
\det ((\lambda- \mu) I + A(\xi+ \mu \cdot \alpha)) = (\lambda-
\varphi_1 (\mu))\cdots(\lambda-\varphi_N(\mu)). 
$$

Now we assert that
\begin{enumerate}[(i)]
\item  $\varphi_j (\mu)\rightarrow I \infty$ as $\mu \rightarrow \pm
  \infty$

\item $\varphi_j (\mu)$ are strictly increasing functions of $\mu$. Since
  we have
\begin{align*}
\varphi_k (\mu)& -\mu= \lambda_k (-\xi-\mu- \alpha) \text{ or }\\
\varphi_k (\mu ) & = \mu - \lambda_k (\xi + \mu \cdot \alpha )
\tag{7.6}\label{chap3-eq7.6} 
\end{align*}\pageoriginale
it follows that for each fixed $\mu, \varphi_k(\mu)$ are real and
distinct. To show (i) consider det $(\mu B-\lambda I-A \cdot \xi ) = 0$
which implies that det $(B- \dfrac{\lambda}{\mu}I - \dfrac{ A \cdot \xi}{
  \mu})=0$. For a fixed $\xi, \dfrac{\varphi_k (\mu)}{\mu}$ tends
to the eigen values of $B$ as $\mu \rightarrow \infty$ and hence for
large $\mu \varphi_k (\mu)\sim \mu \cdot \nu_k$. Since $\nu_k$ are
positive, $\varphi_k(\mu)$ behaves like $\mu$ for large $\mu$. 
\end{enumerate}

As for (ii), suppose on the contrary there exists a $j_0$ and
$\mu_1, \mu_2$ with $\mu_1 < \mu_2$ such that
$\varphi_{j_{0}}(\mu_1) > \varphi_{j_{0}}(\mu_2)$. Then there exists
$a \lambda_{0}$ such that for three distinct $\mu_1'$, $\mu'_2$,
$\mu'_3$ we have 
$$
\varphi_{j_{0}}(\mu_1') = \varphi_{j_{0}}(\mu'_2)= 
\varphi_{j_{0}}(\mu'_3)=\lambda_0. 
$$

Since each $\varphi_j (\mu)(j \neq j_0)$ contributes at least one root
of det $(\mu B -\lambda_0 I-A \cdot \xi) = 0$ it will have at least $N + 2$
roots. This being an equation of degree $N$ we are lead to a
contradiction. Now putting   
$$
\lambda=0, \det (\mu B-A. \xi)= (-1)^N \varphi_1 (\mu) \varphi_2 
(\mu) \cdots \varphi_N (\mu). 
$$

Since every $\varphi_j (\mu)$ has only one zero and the zeros are
distinct, we have the lemma. 
\end{proof}

\begin{remark*}%rema 0
Since $\lambda_j (-\xi ) = - \lambda_j (\xi )$ for every $j,
\lambda_{\max}$ is positive and equal to  $\sup\limits_{\substack {|
    \xi |=1\\ 1 \leq j \leq N}} | \lambda_j (\xi)|$. 
\end{remark*}

\setcounter{corollary}{0}
\begin{corollary}\label{chap3-sec7-coro1}%coro 1
Let\pageoriginale $M$ be a regularly hyperbolic system in $\Omega =
\underline{R}^n  
\times [0, h]$, $\lambda_j (x, t, \xi)$ be the roots of det $(\lambda
I - A \cdot \xi) = 0$ and let 
\begin{equation}
\lambda_{\max} =\sup_{\substack{| \xi | =1, (x, t) \epsilon \Omega
    \\ 1\leq j \leq N}} | \lambda_j (x, t, \xi
)|. \tag{7.7}\label{chap3-eq7.7}  
\end{equation}

Suppose $S$ is a hypersurface in $\Omega$ passing through a point
$(x_0, t_0)$ and defined by an equation $\varphi (x, t) = 0$, $\varphi
\epsilon \mathscr{E}^2$ with 
\begin{equation}
\left(\frac{\partial \varphi}{ \partial t}\right)^2 \geq  \lambda^2_{\max}\sum
  \left(\frac{\partial \varphi}{\partial
    x_j}\right)^2. \tag{7.8} \label{chap3-eq7.8} 
 \end{equation} 

If $u$ is a $C^1$ solution of $M [u] = 0$ with $u(x, t) = 0$ for $(x, 
 t) \epsilon S$ then $u (x, t)\equiv 0$ in a neighbourhood of $(x_0,
 t_0)$. 
\end{corollary}

\begin{proof}
By a change of coordinates  $x'_j = x_j (1 \leq j \leq n)t' = \varphi
(x, t)$ the system $M$ is transformed into the system 
\begin{equation}
\left(\frac{\partial \varphi}{\partial t}I - \sum A_k \frac{\partial
    \varphi}{ \partial x_k}\right) \frac{\partial \tilde{u}}{\partial t'}=
  \sum A_k (x,t) \frac{\partial \tilde{u}}{\partial x'_k}+ \cdots
  \tag{7.9} \label{chap3-eq7.9}
\end{equation}
where $\tilde{u}$ is, as before, the image of $u$ by this mapping. $S$
is mapped into $t' = 0$. Taking 
$$
\alpha =\left(\frac{\partial \varphi}{\partial x_1}\Bigg| \frac{
  \partial \varphi}{\partial t}, \ldots , \frac{\partial
  \varphi}{\partial x_n} \Bigg| \frac{ \partial \varphi}{\partial
  t}\right) 
$$
the conditions of the lamma \ref{chap3-sec7-lem1} are satisfied
because of the assumptions 
on $\alpha$ and hence $\left(\dfrac{\partial \varphi}{ \partial t} I-  \sum
A_k \dfrac{\partial \varphi} {\partial x_k}\right)$ is invertible. Thus
$\tilde{u}$ satisfies  
\begin{equation}
\frac{\partial \tilde{u}}{\partial t'}= \left(\frac{\partial \varphi}{
  \partial t} I - \sum A_k \cdot \frac{\partial \varphi}{\partial
  x_k}\right)^{-1} \sum A_k \frac{\partial \tilde{u}}{\partial x_k'} +
\ldots \tag{7.10} \label{chap3-eq7.10}
\end{equation}

 This\pageoriginale is again a regularly hyperbolic system since
 \begin{gather*}
 \det \left(\lambda I - \left(\frac{ \partial \varphi}{\partial t} I - \sum A_k
 (\frac{ \partial \varphi}{ \partial x_k})\right)^{-1} \sum A_k \cdot
 \xi_k \right) \\  
 = \det \left(\frac{\partial \varphi}{\partial t}I- \sum A_k \frac{
   \partial \varphi }{\partial x_k}\right)^{-1} \cdot \det
 \left(\lambda\left(\frac{\partial \varphi}{\partial t} I- \sum A_k
 \frac{\partial\varphi}{\partial x_k}\right)-A \cdot \xi \right)  
 \end{gather*} 
 and by the lemma its roots are real and distinct for
  $$
\alpha = (\dfrac{ \partial \varphi}{\partial x_1}\Bigg| \dfrac{
   \partial \varphi}{\partial t}, \ldots, \dfrac{ \partial
   \varphi}{\partial x_n} \Bigg| \dfrac{ \partial \varphi}{ \partial
   t}). 
$$

Thus by the local uniqueness (Prop. \ref{chap3-sec7-prop1})
$\tilde{u}$ vanishes in 
 a neighbourhood of the origin and hence $u$ vanishes identically in a
 neighbourhood of $(x_0, t_0)$. 
\end{proof}

\begin{proposition}\label{chap3-sec7-prop2}%prop 2
Let $M$ be a regularly hyperbolic system in $\Omega = \underline{R}^n
\times[0, h]$, $(x_0, t_0) \epsilon \Omega$ and $C$ be the backward
cone defined by $\Big\{t - t_0 = \alpha_0 | x - x_0|$, $t < t_0 $ where
$\alpha_0 = \dfrac{1}{\lambda_{\max}}\Big\}$. Let $D$ be the interior of
this backward cone belonging to $\Omega$. If $u$ is a $\mathscr{C}^1$
solution of $M[ u ] = 0$ in $D$, continuous upto the cone, and
vanishing on $D_0 = D \cap (t = 0)$, then $u$ vanishes identically in
$D + C$  in particular $u(x_0, t_0) = 0$. 
\end{proposition} 

\noindent
{\it Proof:}~(F. John \cite{key1})%proo 0
 we first remark that $u(x, t)$ vanishes identically in a
 neighbourhood of the hyperplane $t = 0$. Let $S_{\theta}(0 <  \theta
 \leq t^2_0)$ be a one parameter family of hyper-surfaces $\varphi (x,
 t, \theta) = 0$ where 
\begin{equation}
\varphi (x, t, \theta ) = (t - t_0)^2- \alpha ^2_0 | x-x_0|^2- \theta
\tag{7.11} \label{chap3-eq7.11}
\end{equation}

Then $\cup S_\theta \supset D$ and
 \begin{equation}
\left(\frac{\partial \varphi}{ \partial t}\right)^2 \Bigg| \sum
\left(\frac{\partial \varphi}{\partial x_k}^2\right) = 
  \frac{(t-t_0)^2}{\alpha^4_0 |
     x-x_0|^2}= \frac{ \alpha^2_0 | x-x_0 |^2 + \theta}{ \alpha^4_0 |
     x-x_0 |^2}> \frac{1}{ \alpha^2_0}= \lambda^2_{\max}
  \tag{7.12}\label{chap3-eq7.12}  
 \end{equation} 

 Hence,\pageoriginale it follows from the lemma that if $u$ vanishes on
 $S_{\theta_{0}}$ for some $\theta_0$ then it vanishes on $S_\theta$ for
 $\theta$ in a neighbourhood of $\theta_0$. The set of $\theta$ for which
 $u$ vanishes on $S_\theta$ is therefore open. It is also closed and
 non-empty. Hence it is the whole set. Thus $u$ vanishes in the whole
 cone $D + C$. \hfill $\Box$

\setcounter{remark}{0} 
\begin{remark}\label{chap3-sec7-rem1}%rema 1
This result holds also for a single equation of order $m$ and can be
proved by writing it as a system by means of singular integral
operators and applying the above arguments. 
\end{remark} 

\begin{remark}\label{chap3-sec7-rem2}%rema 2
Form Prop. \ref{chap3-sec7-prop2} above it follows that if the
Cauchy data has for support a 
small set containing the origin then the support of the solution lies
in some cone limited by lines whose slope $\dfrac{1}{\alpha}\geq
\lambda_{\max}$. This is interpreted as follows: the maximum speed of
propagation of the disturbance is less than $\lambda_{\max}$. 
\end{remark} 

\begin{remark}%rema 3
The above proposition gives a unique continuation theorem for
solutions of systems of some semi linear equations: 
\begin{equation}
M[u] \equiv \frac{\partial u}{\partial t} - \Sigma A_k (x, t) \frac{
  \partial u}{ \partial x_k}-f (x, t, u)
\tag{7.13} \label{chap3-eq7.13} 
\end{equation}
where $A_k (x, t)$ satisfy the same conditions as in Prop.1 and $f
\epsilon \mathscr{E}^1_{x, t}$. More precisely if $u_1$ and  $u_2$ are
two solutions of $M[u]=0$ such that $u_1(x, 0) = u_2 (x, 0)$ for $x
\epsilon D_0$ then $u_1 \equiv u_2$ in the whole of the cone $D$ with
$D_0$ as base. For, $v = u_1-u_2$ satisfies 
\begin{align*}
& \frac{\partial v}{\partial t}- \sum A_k \frac{\partial v}{\partial x_k}- \{f
(x, t, u_1)-f (x, t, u_2) \} = 0. \tag{7.14} \label{chap3-eq7.14}\\
&  v(x, 0) = 0 \text{~ for~ } x \epsilon D_0
\end{align*}
\end{remark}

By\pageoriginale the mean value theorem $f(x, t, u_1) - f(x, t, u_2) =
B(x, t) (u_1,- u_2)\break = B(x, t) v$, $B(x, t)= \dfrac{\partial f}{\partial
  u} (x, t, u_2 +\theta (u_1 - u_2)$. By
Prop. \ref{chap3-sec7-prop2} we have $v \equiv 0$ 
in $C$ and hence $u_1 \equiv u_2 $ in $D$. 

Finally we apply the method of sweeping a cone by a one parameter
family of surfaces to show that the solutions of second order parabolic
equations have no lacuna. 

Consider a parabolic equation of the second order 
\begin{equation*}
\left(\frac{\partial}{\partial t} -L\right) [ u ] = 0
\tag{7.15}\label{chap3-eq7.15} 
\end{equation*}
where $L = \sum\limits_{j, k = 1}^{n} a_{jk} (x, t)
\dfrac{\partial^2}{\partial x_j \partial x_k} + \sum\limits_j b_j (x,
t) \dfrac{\partial}{\partial x_j} + c(x, t)$ with infinitely
differentiable real coefficients and $a_{jk}$ satisfy further the
condition  
\begin{equation*}
\sum\limits^n_{j, k = 1} a_{jk}(x, t) \xi_j \xi_k \geq \delta (x, t) |
\xi|^2, \tag{7.16}\label{chap3-eq7.16}
\end{equation*}
$\delta (x, t) > 0$, for real $\xi \neq 0$. It is known that the unique
continuation across time like hyperplanes holds in the sense that if
$u$ is a $C^2$ solution of the above parabolic equation with  
$$
u(x, t) \big|_{x_1 = 0} = 0, \;\; \frac{\partial u}{\partial x_1} (x, t)
\big|_{x_1 = 0} = 0. 
$$
in some neighbourhood of the origin in $X_1 = 0$ then $u (x, t) \equiv
0$ in a neighbourhood of the origin in the $(x, t)-$ space (see
Mizohatai \cite{key4}, Memoines of the college of Science, Kyoto University,
1958) 

\begin{proposition}\label{chap3-sec7-prop3}% 3
Suppose\pageoriginale $M$ is a parabolic operator of the second order
defined in 
$\Omega = \underline{R}^n \times [0, h]$ and suppose a $C^1$ solution $u$ of
$M[ u ] = 0$ vanishes on a non-empty open set $\theta$ of $\Omega$
then $u \equiv 0$ in a horizontal component $T$ of $\Omega$ containing
$\theta$. 
 
By horizontal component $T$ of $\theta$ in $\Omega$ we mean the set
$\{ (x, t) \in \Omega \}$ such that there exists an $x'$ with $(x',
t) \in \theta$. 
\end{proposition}

\begin{proof}
Suppose $S$ is a hypersurface defined by an equation
$$
\varphi(x,t) = 0, \varphi \varepsilon \mathscr{E}^2_{x, t} 
$$
such that the tangent space of $S$ at the origin is not paralled to $t
= 0$. Then $\sum |\dfrac{\partial \varphi} {\partial x_j} \big| \neq
0$. Suppose $\dfrac{\partial \varphi}{\partial x_j} \neq 0$; then one
can solve for $x_1$ in a neighbourhood of the origin as $x_1 = \psi
(x_2, \ldots, x_n, t)$. By a change of variables  
$$
t' = t, x'_1 = x_1 - \psi (x_2, \ldots, x_n, t),\ x'_j = x_j( j = 2, 
\ldots, n) 
$$
$S$ will be transformed into $(x'_1 = 0)$ and the form of the equation
remains unaltered. Hence by the remark above the transformed function
$\tilde{u}$ vanishes in a neighbourhood of the origin and hence $u$
vanishes in a neighbourhood of the origin on $S$. We may assume
$\mathscr{O}$ to be a neighbourhood of the origin and consider a
one-parameter family of ellipsoods $S_{\theta}$ defined by  
$$
\varphi (x, t, \theta) = \frac{t^2}{a^2} + \frac{|x|^2}{\theta^2} -1 =
0 (0 < \theta < \infty) 
$$
with the condition that the tangent space to this is not parallet to
$(t = 0)$. Again by the argument of connectedness, as before, we
obtain the proposition. 
\end{proof}

\section{Existence theorems}\pageoriginale\label{chap3-sec8}% sec 8

In this section we prove some theorems on the existence
of solutions of the Cauchy problem for hyperbolic equation. To begin
with we recall the Hille-Yosida theorem on the infinitesimal generator
of a semi group of operators on a Banach space. This is used to assert
the existence of solutions.  

\setcounter{theorem}{0}
\begin{theorem}[Hille-Yosida]\label{chap3-sec8-thm1}%theo 1
 Let $X$ be a Banach space and $A$ be a linear
operator on $X$ with domain of definition $\mathscr{D}_A$ dense in
$X$. Assume that $A$ has the following property: 

{\rm(P)}~ there exists a real number $\varepsilon_0 > 0$ such that
for every real number $\lambda$ with $|\lambda| < \varepsilon_0$ we
have  
\begin{enumerate}
\renewcommand{\labelenumi}{\rm(\theenumi)}
\item $(I - \lambda A)$ is a one to one surjective mapping of
  $\mathscr{D}_A$ onto $X$,  

\item there exists a constant $\gamma > 0$ such that  
$$
|| (I - \lambda A) u|| \geq (1 - \gamma | \lambda |) || u || 
$$
for every $u \in \mathscr{D}_A$. Then for any given $u_0 \in
\mathscr{D}_A$ there exists in $-\infty< t< \infty$ a once
continuously differentiable solution  
\begin{equation*}
\frac{du}{dt} (t) = Au(t) \text{~ with~ } u(0) = u_0
\tag{8.1}\label{chap3-eq8.1} 
\end{equation*}
with values in $\mathscr{D}_A$.
\end{enumerate}
\end{theorem}

\begin{coro*}%coro 0
Let $A$ be a linear operator with domain of definition $\mathscr{D}_A$
dense in $X$ and possessing the property (P) of
Th. \ref{chap3-sec8-thm1}. If $t \to f(t) 
\in \mathscr{D}_A$ is a continuous function of $t$ such that $t\to Af(t)\in X$
is a continuous function of $t$ and a $u_0\in\mathscr{D}_A$ is given
there exists a 
once continuously differentiable solution $u(t)$ (with values in
$\mathscr{D}_A$) of  
\begin{equation*}
\frac{du}{dt} (t) = Au (t) + f(t) \text{ with } u(0) = u_0
\tag{8.2}\label{chap3-eq8.2} 
\end{equation*}\pageoriginale
\end{coro*}

We first consider the case of systems whose coefficients do not depend
on $t$. 

We remark that for a differential operator it is not in general
possible to secure the condition $P(2)$ when we take $L^2$ for the
Banach space $X$ even when \eqref{chap3-eq8.1} is well posed in the space
$L^2$. For, suppose the condition $P(2)$ is satisfied. 
\begin{gather*}
|| (I - \lambda A) u ||^2  = || u ||^2 + \lambda^2 || Au ||^2 -
\lambda (( A + A^*) u, u)\\ 
 \geq (1- \gamma| \lambda|)  || u ||^2.
\end{gather*}

As $|\lambda|$ can be taken arbitrarily small this would imply if $|
\lambda |$ is small that  
\begin{align*}
(( A + A^*) u, u)  & \leq \gamma || u ||^2 \text{ for } \lambda > 0
  \text{ and }\\ 
(( A + A^*) u, u) & \geq - \gamma || u ||^2 \text{ for } \lambda < 0 
\end{align*}
which togeter imply 
$$
|(( A + A^*) u, u)| \leq \gamma || u ||^2 
$$

This would mean, when we take $A = \sum A_k (x)
\dfrac{\partial}{\partial x_k}$, $A_k \in \mathbb{B}'$, that $A_k =
A^*_k$. In fact, $A + A^* = \sum (A_k - A^*_k)
\dfrac{\partial}{\partial x_k}- \dfrac{\partial A^*_k}{\partial x_k}$,
and it is easy to see that the above inequality holds if and only if
$A_k \equiv A^*_k (k = 1, 2, \ldots, n)$. We then proceed to study the
system  
\begin{equation*}
\frac{\partial u}{\partial t}(t) = \sum A_k (x) \frac{\partial u
}{\partial x_k} + B(x) u + f \tag{8.3} \label{chap3-eq8.3}
\end{equation*}

We take for the operator $A$ the differential operator 
\begin{equation*}
A = \sum A_k (x) \frac{\partial}{\partial x} + B(x)
\tag{8.4}\label{chap3-eq8.4} 
\end{equation*}\pageoriginale
in $\mathscr{D}^1_{L^2}$. We take for the domain of definition of $A$
the set  
\begin{equation*}
\mathscr{D}_A = \left\{ u \in \mathscr{D}^1_{L^2} : A u \in
\mathscr{D}^1_{L^2} \right\}. \tag{8.5} \label{chap3-eq8.5}
\end{equation*}

We remark that $\mathscr{D}^2_{L^2} \subset \mathscr{D}_A$ and
consequently $\mathscr{D}_A$ is dense in $\mathscr{D}^1_{L^2}$. A is a
closed operator in the sense that its graph is closed. In fact, let
$u_p \in \mathscr{D}_A$ be a sequence such that $u_p \to u_0$, $Au_p
\to v_0$ in $\mathscr{D}^1_{L^2}$. Since $A$ is a continuous operator
from $\mathscr{D}^1_{L^2}$ into $L^2$ we have $Au_0 = v_0$ in $L^2$
and since the injection of $\mathscr{D}^1_{L^2}$ into $L^2$ is
bi-unique $Au_0 = v_0$ in $\mathscr{D}^1_{L^2}$, that is $u_0 \in
\mathscr{D}_A$. 

\setcounter{proposition}{0}
\begin{proposition}\label{chap3-sec8-prop1}%1
Let
\begin{equation*}
\frac{\partial u}{\partial t} = \sum A_k (x) \frac{\partial
  u}{\partial x_k} + B(x) u + f \tag{8.3}\label{chap3-eq8.3} 
\end{equation*}
be a regularly hyperbolic system in $\Omega = \underbar{R}^n \times [
  0, h]$ with $A_k \in \mathbb{B}^{i + \sigma}$, $B \in
\mathbb{B}^1$ and $f \in \mathscr{D}_A [ 0, h ]$. Then, given $u_0 \in
\mathscr{D}_A$ there exists a unique solution $u \in \mathscr{D}_A [0,
  h]$, which is a differentiable function of $t$ in the sense of $L^2$
with values in $\mathscr{D}_A$  of \eqref{chap3-eq8.3} for which $u (0) = u_0$. 
\end{proposition}

\begin{proof}
We write the system in the singular integral operator form 
\begin{equation*}
\frac{d}{dt} u = ( i \mathscr{H} \wedge + B) u + f
\tag{8.6}\label{chap3-eq8.6} 
\end{equation*}
and $A = i \mathscr{H} \cap + B$. By the condition of regular
hyperbolicity of \eqref{chap3-eq8.3} there exists a bounded singular integral
operator $\mathfrak{N}$ such that  
$$
\mathfrak{N} 0  \mathscr{H} =  \mathscr{D} 0  \mathscr{H}
$$\pageoriginale
where $\mathscr{D}$ is a singular integral matrix whose symbol is  
$$
\sigma (\mathscr{D}) = 
\begin{pmatrix}
\lambda_1 (x, \xi) & & 0\\
&  \ddots & \\
0 & & \lambda_N (x, \xi)
\end{pmatrix}
$$
and $|\det \sigma |(\mathfrak{N}) | > \delta > 0$.

Define a bilinear form by 
\begin{equation*}
(Lu, v) = (\mathfrak{N} \wedge u, \mathfrak{N} \wedge v) + \beta (u,
  v) = (( \lambda \mathfrak{N}* \mathfrak{N}\Lambda + \beta I)u,
  v). \tag{8.7} \label{chap3-eq8.7}
\end{equation*}
for $u, v \varepsilon \mathscr{D}^1_{L^2}$ with a $\beta$ to be
chosen later. $(Lu, u)$ defines a norm equivalent to that of
$\mathscr{D}^1_{L^2}$ for sufficiently large $\beta$. In fact, since
$\mathfrak{N}$ is a bounded operator in $L^2$ we have  
$$
(Lu, u) \leq ||\mathfrak{N}||^2_{\mathscr{L}(L^2, L^2)} || \wedge u 
||^2 + \beta || u ||^2 \leq M || u ||^2_{\mathscr{D}^1_{L^2}}. 
$$

On the other hand by G$\ring{\text{a}}$rding's inequality there exists
$a\ \gamma > 0$ such that  
$$
(Lu, u) \geq \delta' || \wedge u ||^2 - \gamma || u ||^2 + \beta || u
||^2, 
$$
then for sufficiently large $\beta( > \gamma)$ this would imply that  
$$
(Lu, u)\geq c ||u||^2_{\mathscr{D}^1_{L^2}} 
$$
which proves the assertion. We provide $\mathscr{D}^1_{L^2}$ with the
norm $(Lu, u)$. We proceed to verify conditions 1, 2, of the
Hille-Yosida Theorem. 
To\pageoriginale prove condition $P(2)$ we must prove that for real
$\lambda$ near the origin  
\begin{equation*}
(L(I- \lambda A) u, (I- \lambda A)u) \geq (1 - \gamma| \lambda|) (Lu,
  u) \text{~ for every~ } u \in
  \mathscr{D}_A. \tag{8.8} \label{chap3-eq8.8} 
\end{equation*}

To do this we assume at first that $u \in \mathscr{D}^2_{L^2}$
we have then, 
\begin{align*}
(L(I - \lambda A)u, (I - \lambda A)u) & = (Lu, u) + \lambda^2 (LAu,
  Au) - \lambda((LA + A^* L) u, u)\\ 
& \geq (Lu, u) - \lambda ((LA + A^* L) u, u).
\end{align*}

Since $A = i \mathscr{H} \Lambda + B$ we have  
$$
(LA + A^*L) = (\Lambda \mathfrak{N}^* \mathfrak{N} \Lambda + \beta
I)(i\mathfrak{N}\wedge +B)+ (-i
\cap  \mathscr{H}^* + B^*) (\wedge \mathfrak{N}^* \mathfrak{N} \wedge
+ \beta I) 
$$

But $\mathfrak{N} \wedge \mathscr{H} \equiv D \wedge \mathfrak{N} \mod
(\wedge^0)$ where $P_1 \equiv P_2 \mod (\wedge^0)$ means that $P_1 -
P_2$ is a bounded operator in $\mathscr{D}^1_{L^2}$. 

In fact,
\begin{align*}
\mathfrak{N} \wedge \mathscr{H} &\equiv \mathfrak{N} \mathscr{H}
\wedge \equiv (\mathfrak{N} \circ \mathscr{H}) \wedge - (\mathfrak{N} \circ
\mathscr{H} - \mathfrak{N} \mathscr{H}) \wedge\\ 
& \equiv (\mathfrak{N} \circ \mathscr{H}) \wedge \mod (\Lambda^), \;  (\text
  {since } \mathfrak{N} \circ \mathscr{H} = \mathscr{D} \circ \mathfrak{N})
  \\ 
& \equiv (\mathscr{D} \circ \mathfrak{N}) \wedge \equiv \mathscr{D}
  \mathfrak{N} \wedge \equiv \mathscr{D} \wedge \mathfrak{N} \text{mod}
  (\wedge^0). 
\end{align*}

Hence 
\begin{align*}
((LA+A^* L) u, u) & = i \{ (D \wedge \mathfrak{N} \wedge u,
\mathfrak{N} \wedge u) - (\mathfrak{N} \wedge u, \mathscr{D} \wedge
\mathfrak{N} \wedge u)\\
&\quad + 2 \re (\mathbb{B}_1 \wedge u, \mathfrak{N} \wedge u),
\end{align*}
where $\mathbb{B}_1$ is a bounded operator in $L^2$. Now 
$$ 
\mathscr{D} \wedge - \wedge \mathscr{D}*\equiv \mathscr{D} \wedge -
\wedge \mathscr{D}^{\#} \equiv (\mathscr{D} \wedge -
\mathscr{D}^{\#}\wedge)\equiv (\mathscr{D}-\mathscr{D}^{\#}) \wedge. 
$$ 

Since $\mathscr{D}$ is a diagonal matrix and $\sigma (\mathscr{D})$ is
real, we see that $\mathscr{D} = \mathscr{D}^{\#}$. Hence $\mathscr{D}
\wedge - \wedge \mathscr{D}^* \equiv \text{mod}
(\wedge^0)$. Hence\pageoriginale there exists a constant $\gamma_1$
such that   
$$
-\gamma_1 || u ||^2_{\mathscr{D}^1_{L^{2}}} \leq ((L A + A^* L ) u, u)
\leq \gamma_1 || u ||^2_{\mathscr{D}^1_{L^{2}}} 
$$
or equivalently we write following Leray \cite{key1}
$$
-\gamma_1(\wedge + 1)^2 \leq LA + A^* L \leq \gamma_1 (\wedge + 1)^2 
$$
and thus, as $||u ||^2_{\mathscr{D}^1_{L^{2}}}$ and $(Lu, u)$ are
equivalent we obtain 
$$
(L (I - \lambda A) u, \; (I - \lambda A) u) \geq (1- \gamma_1 | \lambda
|) (Lu, u) 
$$
for $|\lambda | < \dfrac{1}{\gamma_1}$.

Next the inequality \eqref{chap3-eq8.8} holds for all $u \in \mathscr{D}_A$
also. Suppose $u \in \mathscr{D}_A$. If $\varphi_{\delta}$ are
mollifiers of Friedrichs then the function $u_{\delta} = u *
\varphi_{\delta}$ belongs to $\mathscr{D}^2_{L^{2}}$ and it follows
from \eqref{chap3-eq8.8} that there exists a constant $\gamma_1$ such
that for some real near the origin  
$$
(L(I - \lambda A) u_{\delta}, \; (I - \lambda A)u_{\delta}) \geq (1 -
\gamma_1 | \lambda |) ( L u_{\delta}, u_{\delta}). 
$$

But
$$
Au_{\delta} \to Au \text{ in } \mathscr{D}^2_{L^{2}} \text{ as }
\delta \to 0. 
$$

In fact, $Au_{\delta} - Au = (Au_{\delta} - \varphi_{\delta} * (A u))
+ (\varphi_{\delta}* (Au) - Au)$ in which the first term tends to 0 in
$\mathscr{D}^1_{L^{2}}$ by Friderich's lemma and the latter term tends
to 0 in $\mathscr{D}^1_{L^{2}}$ since $A u \in
\mathscr{D}^1_{L^{2}}$. Thus condition $P(2)$ of Hille-yosida Theorem is
verified. To prove condition $P(1)$ we must prove that $(I - \lambda A)$
is a one-to-one surjective mapping of $\mathscr{D}_A$ onto
$\mathscr{D}^1_{L^{2}}$ for\pageoriginale sufficiently small
$\lambda$. From \eqref{chap3-eq8.8} it follows that $(I - \lambda A)$
is one-to-one
for $| \lambda | < \dfrac{1}{\gamma_1}$.  

Next $(I - \lambda A) \mathscr{D}_A$ is closed in
$\mathscr{D}^1_{L^{2}}$. For, $(I - \lambda A) u_n \to v_0$ in
$\mathscr{D}^1_{L^{2}}$ for $u_n \in \mathscr{D}_A$ means by
\eqref{chap3-eq8.8} that $u_n$ is a Cauchy sequence for the new norm
hence has a unique limit $u_0$ in $\mathscr{D}^1_{L^{2}}$. Hence $-
\lambda Au_{n} \to v_0 - u_0$ in $\mathscr{D}^1_{L^{2}}$. As $A$ is a
closed mapping $u_0 \in \mathscr{D}_A$ and $(I - \lambda A) u_0 = v_0$. 

Finally we prove that $(I - \lambda A) \mathscr{D}_A$ is dense in
$\mathscr{D}^1_{L^{2}}$. The proof is by contradiction. Suppose $(I -
\lambda A) \mathscr{D}_A$ is not dense in
$\mathscr{D}^1_{L^{2}}$. Then there exists $a \psi \in
\mathscr{D}^1_{L^{2}}$, $\psi \neq 0$ such that $(( I - \lambda A)u,
\psi)_1 = 0$ i.e. $(( \wedge + 1) (I - \lambda A)u$, $(\Lambda + 1)
\psi) = 0$ for all $u \in \mathscr{D}_A$, that is, $(I - \lambda A^*)
(\wedge + 1) \psi_1 =0$ 
where $A^* = - i \wedge \mathscr{H}^* + B^* $ and $\psi_1 = (\wedge +
1) \psi \varepsilon L^2$. 

Now $A^* (\wedge + 1) \psi_1 = (-i \wedge \mathscr{H}^* + B^*)(\wedge
+ 1) \psi_{1} = (\wedge + 1) (-i \wedge \mathscr{H}^* + B^*) \psi_1 + B_0
\psi_1$. 
where $B_0 = -i \wedge (\mathscr{H}^* \wedge - \wedge \mathscr{H}^*) +
(B^* \wedge - \wedge B^*)$. 

Further $A^* (\wedge +1) \psi_1 = (\wedge + 1) (-i \mathscr{H}^{\#}
\wedge + B^* + B_1) \psi_1 + B_0 \psi_1$ 
where $B_1 = (-i \mathscr{H}^{\#} \wedge + i \mathscr{H}^{\#} \wedge)
\varepsilon \mathscr{L} (L^2, L^2)$. 

But $B = B_1 + (\wedge + 1)^{-1} B_0 + B^*$ is a bounded operator in
$L^2$ and hence $(I - \lambda A^*) (\wedge + 1) \psi_1 = 0$ is
equivalent to saying that $[I - \lambda (-i \mathscr{H}^{\#} \wedge +
  \tilde{B}] \psi_1 = 0$, which in turn is equivalent to saying that
$[I - \lambda (-i \mathscr{H}^{\#} \wedge + \tilde{B}] \psi =
0$. Starting from the equation 
\begin{equation*}
\frac{\partial}{\partial t} u = - \sum {}^{t}\overline{A}_k (x)
\frac{\partial}{\partial x_k} u - \tilde{B} u \tag{8.9} \label{chap3-eq8.9}
\end{equation*}
and using \eqref{chap3-eq8.8} after observing that $\psi \in \mathscr{D}_A$
we\pageoriginale obtain an inequality 
\begin{gather*}
(L_1 (I - \lambda  (- i\mathscr{H}^{\#} \wedge + \tilde{B})) \psi,
  (I - \lambda (-i(\mathscr{H}^{\#} \wedge + \tilde{B})) \psi )
  \tag{8.10}\label{chap3-eq8.10}\\ 
\geq (1 - \gamma | \lambda | ) \; (L_1 \psi, \psi )
\end{gather*}
which implies that $||\psi ||=0$ and hence $\psi = 0$ which is a
contradiction to the assumption. 

Now all the conditions of Hille-Yosida theorem for $A=i\mathscr{H}
\wedge + B$ are verified and hence there exists a solution of the
equation 
$$
\frac{d}{dt} u = (i\mathscr{H} \wedge + B) u + f \text{ with } u(0) =
u_0 
$$
with the required properties. 
\end{proof}

In the above proposition we proved the existence of solutions of
regularly hyperbolic systems when $u_0 \in \mathscr{D}_A$ in
particular when $u_0 \in \mathscr{D}^2_{L^{2}}$ and $f
\in \mathscr{D}_A [0, h]$ and so in particular when $f
\in \mathscr{D}^2_{L^{2}}$. This result can be improved as
follows. 

\begin{proposition}\label{chap3-sec8-prop2}%2
Suppose \eqref{chap3-eq8.3} is a regularly hyperbolic system in $\Omega =
\underbar{R}^n \times [0, h]$ with $A_k \in \mathbb{B}^{1 +
  \sigma}$, $B \in \mathbb{B}^1$, $u_0 \in
\mathscr{D}^1_{L^{2}}$ and $f \in \mathscr{D}^1_{L^{2}} [0,
  h]$. Then there exists $u \in \mathscr{D}^1_{L^{2}} [0, h]$
(once differentiable in $t$ in the sense of $L^2$) satisfying the
system in the $L^2$-sense and $u(0) = u_0$.  
Also the following energy inequality holds:
\begin{align*}
(Lu(t), u(t)) \leq \exp (\gamma t) & \cdot  (Lu(0), u(0))\\
&  + \int \limits^t_0
  (L(f(s)), f(s)) \exp (\gamma (t-s))ds \tag{8.12} \label{chap3-eq8.12}
\end{align*}
where $(Lu, u)$ is defined in Prop. \ref{chap3-sec8-prop1}.
\end{proposition}

\begin{proof}
We\pageoriginale regularize $u_0$ and $f$ by mollifiers of Friedrichs
$\varphi_{\delta}$ to obtain $u_0 * \varphi_{\delta} = u^{\delta}_0
\in \mathscr{D}^2_{L^{2}}$, $f * \varphi_\delta = f_\delta \in
\mathscr{D}^2_{L^2} [0, h]$. By prop. \ref{chap3-sec8-prop1} 
applied to $u^{\delta}_0$, $f_{\delta}$ there exists a $u_{\delta}$
continuous and with values in $\mathscr{D}_A$ satisfying  
\begin{equation*}
\frac{\partial}{\partial t} u_{\delta} = \sum A_k (x)
\frac{\partial}{\partial x_k} u_{\delta} + Bu_{\delta} + f_{\delta}
\tag{8.12} \label{chap3-eq8.12}
\end{equation*}
and $u_{\delta} (0) = u^{\delta}_0$. Further $u_{\delta} (t) -
u_{\delta'} (t)$ satisfies the equation 
{\fontsize{10pt}{12pt}\selectfont
$$
\frac{\partial}{\partial t} \left[u_{\delta} (t)-u_{\delta'} (t)
  \right] = \sum A_k (x) \frac{\partial}{\partial x_k} \left[u_{\delta} (t)
  - u_{\delta'} (t) \right] + B[u_{\delta} (t) - u_{\delta'} (t)] +
(f_{\delta} - f_{\delta'})  
$$}\relax
and hence by the energy inequality
\begin{equation*}
|| u_{\delta} (t) - u_{\delta'} (t) ||_1 \leq c(h) \left\{ || u_{\delta}
(0) - u_{\delta'} (0) ||_1 + \int\limits^h_0 ||f_{\delta} (s) -
f_{\delta'} (s) ||_1 ds\right\}, \tag{8.13} \label{chap3-eq8.13}
\end{equation*}
which shows that $\{u_{\delta} (t) \} $ is a Cauchy sequence in the
space of continues functions with values in
$\mathscr{D}^1_{L^2}$. Hence $u_{\delta} (t) \to u(t)$ in the space of
continuous functions with values in $\mathscr{D}^1_{L^2}$. On the
other hand the equation 
$$
u_{\delta} (t) - u^{\delta}_0 = \int\limits^t_0 \{ A u_{\delta} (s) +
f_{\delta} (s) \} ds, \;\; A = \sum A_k \frac{\partial}{\partial x_k} + B 
$$
holds in $L^2$. Passing to the limits in $L^2$ we obtain  
$$
u(t) - u_0 = \int\limits^t_0 \{ Au(s) + f(s) \} ds. 
$$

Differentiating this, we see that the relation  
$$
\frac{d}{dt} u (t) = Au(t) + f(t) 
$$
holds in the sense of $L^2$ where $u \in \mathscr{D}^1_{L^2}
[0, h]$, $\dfrac{\partial u}{\partial t} \in L^2 [0, h]$
respectively.\pageoriginale Consider now 
\begin{align*}
\frac{d}{dt} (Lu_{\delta}, u_{\delta}) & = \left(L \frac{d}{dt} u_{\delta},
u_{\delta}\right) + \left( L u_{\delta},\frac{d}{dt}  u_{\delta}\right) + (L_t'
u_{\delta}, u_{\delta})\\ 
& \leq ((LA+ A^*L) u_{\delta}, u_{\delta}) + 2 \re (Lf_{\delta},
f_{\delta}) + \gamma' (L u_{\delta}, u_{\delta})\\ 
& \leq \gamma (L u_{\delta}, u_{\delta}) + (L f_{\delta}, f_{\delta}).
\end{align*}

Since $u_{\delta} (t)$ and $f_{\delta} (t)$ converge, uniformly in
$t$, to $u(t)$, $f(t)$ respectively in $\mathscr{D}^1_{L^2}$ as $\delta
\to 0$ we have \eqref{chap3-eq8.12}. This completes the proof of the
proposition.  
\end{proof}

\begin{remark*}%rema 0
The above equation is a particular case of one involving singular
integral operators. If in fact we consider an equation  
\begin{align*}
\frac{d}{dt} u(t) & = i \mathscr{H} \wedge u(t) + Bu(t) + f(t)\\
& \equiv A u(t) + f(t), \tag{8.14}\label{chap3-eq8.14}
\end{align*}
with $\sigma (\mathscr{H}) \in C^{\infty}_{1 + \sigma}$, $B
\in \mathscr{L} (L^2, L^2) \cap \mathscr{L}(\mathscr{D}^1_{L^2},
\mathscr{D}^1_{L^2})$, which is regularly hyperbolic, we could prove
an analogous proposition in the same way. We would have to use the
Fridrichs' lemma for singular integral operators, namely, 
\begin{equation*}
[\mathscr{H} \wedge, \varphi_{\delta} *] \to 0 \text{ weakly in }
\mathscr{D}^1_{L^2}. \tag{8.15}\label{chap3-eq8.15} 
\end{equation*}
\end{remark*}

Now we consider the general case of regularly hyperbolic systems when
the coefficients are functions of the variable $t$ also. We use a
method similar to the one of Cauchy-Peano for ordinary differential
equations. 

\begin{theorem}\label{chap3-sec8-thm2}%the 2
Let\pageoriginale $\Omega = \underbar{R}^n \times [0, h]$ and  
\begin{equation*}
\frac{\partial}{\partial t} u = \sum A_k (x, t)
\frac{\partial}{\partial x_k} u + B(x ,t) u + f
\tag{8.16} \label{chap3-eq8.16}
\end{equation*}
be a regularly hyperbolic system in $\Omega$ with $A_k \in
\mathbb{B}^{1+ \sigma} [0, h]$, $B \in \mathbb{B}^1 [0, h]$,
$f \in \mathscr{D}^1_{L^2} [0, h]$. Given a $u_0 \in
\mathscr{D}^1_{L^2}$ there exists a unique solution $u$ of
\eqref{chap3-eq8.16}, in 
the sense of $L^2$, which belongs to $\mathscr{D}^1_{L^2} [0, h]$ and
is differentiable in the sense of $L^2$ for which $u(0) = u_0$. 
\end{theorem}

\begin{proof}
Consider a subdivision
$$
\Delta : 0 = t_0 < t_1 \ldots < t_s = h. 
$$

We define a function $u$ inductively as follows: For $t_{j-1} \leq t
\leq t_j$, $u_{\Delta}(t) = u_j (t)$ where $u_j$ satisfies the system  
\begin{equation*} 
\frac{\partial}{\partial t} u_j = \sum A_k (x, t_{j-1})
\frac{\partial}{\partial x_k} u_j + B(x, t_{j-1}) u_j + f,
u_j(t_{j-1}) = u_{j-1} (t_{j-1}) \tag{8.18} \label{chap3-eq8.18}
\end{equation*}
for $j = 1, \ldots, s$. By Prop. \ref{chap3-sec8-prop3} there exists a
unique solution $u_j 
\in \mathscr{D}^1_{L^2}$ for this system for $j = 1, \ldots,
s$. Thus $u_{\Delta}(t)$ is uniquely determined. We shall show that
$u_{\Delta}$ is uniformly bounded for small subdivisions (subdivisions
of small norms), that is, 
$$
\sup\limits_{t \in [0, h]}|| u_{\Delta} (t) ||_1 \leq M <
\infty. 
$$

It follows from \eqref{chap3-eq8.18} using the given conditions on the
coeficients that  
$$
\sup\limits_{t \in [0, h]}|| \frac{d}{dt} u_{\Delta} (t)
||_{L^2} \leq M' < \infty. 
$$

Hence\pageoriginale $\{ u_{\Delta} (t) \}$ is a bounded set in
$\mathscr{E}^{1}_{L^2} (\Omega)$ as $\Delta$ runs through subdivisions
of small norm. Thus by choosing a suitable subsequence of $\Delta$,
$u_{\Delta} \to u$ weakly in $\mathscr{E}^{1}_{L^2} (\Omega)$ and $u$
satisfies  
\begin{gather*}
\frac{\partial u}{\partial t} = \sum A_k (x, t) \frac{\partial
  u}{\partial x_t} + B(x, t) u+f \tag{8.18}\label{chap3-eq8.18}\\ 
u_{\Delta} \to u, \; \frac{\partial u_{\Delta}}{\partial x_k} \to
\frac{\partial u}{\partial x_k}, \; \frac{\partial u_{\Delta}}{\partial
  t} \to \frac{\partial u}{\partial t} \text{ weakly in } L^2(\Omega ) 
\end{gather*}
and these derivatives are taken in the sense of distribution in
$\Omega$. 

Next we shall show that $u \in \mathscr{D}^1_{L^2} [0, h]$ and
$u(0) = u_0$. For almost all $t$, $u(x,t)$, as a function of $t$ for
each fixed $x$, is absolutely continuous (see Sehwartz
\cite{key1}). Hence we can write   
$$
u(x, t') - u(x, t'') = \int\limits^{t''}_{t'} \frac{\partial
  u}{\partial t} (x, t) \; dt 
$$
the derivative in the right hand side is taken in the distribution
sense. 
By the Schwarz inequality
$$
|u(x, t') - u(x, t'')|^2 \leq | t' - t''| \int\limits^{t''}_{t'}
|\frac{\partial u}{\partial t} (x, t)|^2 \; dt 
$$ 
which on integration with respect to $x$ gives 
$$
|| u(x, t') - u(x, t'')||_{L^2 (\underbar{R}^n)} \leq | t' - 
t''|^{\frac{1}{2}} || \frac{\partial u}{\partial t} (x, t) ||_{L^2
  (\Omega)} 
$$
proving that $u \in L^2 [0, h]$. If $\varphi_{\delta}$ denote
mollifiers of Friedrichs, the function $u = u * \varphi_{\delta}$
satisfies $u_{\delta} \in \mathscr{D}^1_{L^2} [0, h]$ and  
\begin{equation*}
\frac{\partial}{\partial t} u_{\delta}(t) = \sum A_k (x, t)
\frac{\partial u}{\partial x_k} u_{\delta} + B(x, t) u + f +
C_{\delta} u \tag{8.19} \label{chap3-eq8.19} 
\end{equation*}
where\pageoriginale
\begin{equation*}
C_{\delta} = \sum \left[A_k \frac{\partial}{\partial x_k},
  \varphi_{\delta} * \right] + [ B, \varphi_{\delta} *
]. \tag{8.20}\label{chap3-eq8.20}  
\end{equation*}

By Friedrichs' lemma $|| C_{\delta} u ||_1 \leq c || u ||_{1}$ and $||
C_{\delta} u ||_1 \to 0$ as $\delta \to 0$ for fixed $t$. Since
$\int\limits^h_0 || u(x, t) ||_1 dt < \infty$, it follows that
$|C_{\delta} u ||_1$ is integrable, and from Lebesgue's bounded
convergence theorem, we deduce that   
$$ 
\int\limits^h_0 || C_\delta u(x, t) ||_1 \; dt \to 0 \text{ as }
\delta \to 0. 
$$

Now from the energy inequality for the system \eqref{chap3-eq8.19}  
$$
|| u_{\delta} (t) ||_1 \leq c(h) \left\{ || u_{\delta} (0) ||_1 + 
\int\limits^h_0 (||f_{\delta} (s) ||_1 + ||C_\delta u(s)||_1 ) ds \right\} 
$$
it follows that $\sup\limits_{t \in [0, h]}|| u_{\delta} (t)
||_1 \leq M < \infty$. Again $u_{\delta} (t) - u_{\delta'}(t)$
satisfies an equation 
\begin{align*}
\frac{\partial}{\partial t} (u_{\delta} (t) - u_{\delta'} (t)) & = \sum
A_k (x, t) \frac{\partial}{\partial x_k} (u_{\delta} (t) - u_{\delta'}
(t)\\ 
& + B(x, t) (u_{\delta}(t) -u_{\delta'} (t)) + C_{\delta} u(t) -
C_{\delta'} u(t) 
\end{align*}
and we have the energy inequality 
$$
|| u_{\delta} (t) - u_{\delta'}(t) ||_1 \leq c'(h) \left\{ || u_{\delta}(0)
- u_{\delta'} (0) ||_1 + \int\limits^h_0 ||(C_{\delta} - C_{\delta'}) u
(s)||_1 ds \right\} 
$$
which shows that $|| u_{\delta} (t) - u_{\delta'} (t) ||_1 \to 0$ as
${\delta}$, $\delta' \to 0$. So $\{ u_{\delta} (t) \}$ is a Cauchy
sequence in $\mathscr{D}^1_{L^2} [0, h]$ and hence its limit is in
$\mathscr{D}^1_{L^2} [0, h]$. 
By\pageoriginale the uniqueness of limits in $L^2 [0, h]$, $u_{\delta}
\to u$ and $u 
\in \mathscr{D}^1_{L^2} [0, h]$. Since the operation of
restriction is continuous and the restriction of $u_{\Delta}$ to $t =
0$, namely $u_{\Delta} (x, 0))$, is $u_0$ we see that $u(x, 0) =
u_0$. 

Now it only remains to show that $\{ u_{\Delta} (t) \}$ is a bounded
set in $\mathscr{E}^1_{L^2}$. For this we proceed as follows. We use
the norm defined by  
$$
(Lu, u) = (\mathfrak{N} \wedge u, \mathfrak{N} \wedge u) + \beta (u, u) 
$$
for suitable $\beta > 0$ (see (8.7)). $\mathscr{D}^1_{L^2}$ is
provided with this norm. 

By the energy inequalities we have, for $j = 1,\ldots, s$
\begin{align*}
(L (t_{j-1}) u_{\Delta}, u_{\Delta}) & = (L(t_{j-1}) u_j (t), u_j(t))
  \tag{8.21}\label{chap3-eq8.21}\\ 
& \leq \exp (\gamma( t -t_{j-1})) (L (t_{j-1}) u_j (t_{j-1}), u_j
  (t_{j-1}))\\ 
& \qquad + \int\limits^{t_j}_{t_{j-1}} \exp ( \gamma (t-s)) (L(t_{j-1})
  f(s), f(s)) \; ds. 
\end{align*}

The $L(t)$ depends on the $\mathfrak{N} (t)$ which form a bounded set
of singular integral operators and hence by the remark ater
prop. \ref{chap3-sec6-prop2}, \S\ \ref{chap3-sec6}
we can use the same constant $\beta$ to the new norm in
$D^{1}_{L^{2}}$. Further letting $L_{k}=L(t_{k})$
\begin{align*}
(L_j u_{\Delta}, u_{\Delta}) - (L_{j-1} u_{\Delta}, u_{\Delta}) & = ||
  \mathfrak{N} (t_j) \wedge u_{\Delta} ||^2 - ||^{\mathfrak{N}}
  (t_{j-1}) \wedge u_{\Delta} ||^2 \\ 
& \leq C|| (\mathfrak{N} (t_j) - \mathfrak{N}(t_{j-1})||_{\alpha (L^2,
    L^2)}|| u_{\Delta}||^{\alpha}_1. 
\end{align*}

Since $(L_{j -1} u_{\Delta}, u_{\Delta}) \sim || u_{\Delta} ||^2_1$
we have $|| u_{\Delta}||^2_1 \leq k (L_{j-1} u_{\Delta},
u_{\Delta})$ and hence  
\begin{align*}
(L_j u_{\Delta}, u_{\Delta}) & \leq (1 + k|| \mathscr{H}(t_j) -
  \mathfrak{N}(t_{j-1}||_{\alpha (L^2, L^2)}) (L_{j-1}u_{\Delta},
  u_{\Delta})\\ 
&= (1 + \varepsilon (t_{j-1}, t_j)) (L_{j-1} u_{\Delta}, u_{\Delta}).  
\end{align*}\pageoriginale

Using this in the above inequality \eqref{chap3-eq8.21} we have  
\begin{align*}
(L_{s-1} u_{\Delta}, u_{\Delta}) & \leq \exp (\gamma t) (L_0 u (t_0),
u(t_0)) \\
& + \int\limits^h_0 \exp (\gamma(t-s)) (L_0 f(s), f(s)) ds
\prod\limits^{\mathscr{S}}_{j=1} \{1+ \varepsilon (t_{j-1}), t_j \}. 
\end{align*}

But we have, by a will-known inequality,
$$
\prod^{\mathscr{S}}_{j=1} \{1 + \varepsilon (t_{j-1}, t_j) \}
\leq \left(1 + \frac{1}{S} \sum \in (t_{j-1}, t_j)\right)^s \leq e^{\gamma_0} 
$$
where
\begin{align*}
\gamma_0 & = \sup \sum \varepsilon (t_{j-1}, t_j) = \sup
\limits_{\Delta} k  \sum || \mathfrak{N} (t_j) -
\mathfrak{N}(t_{j-1})||_{\alpha (L^2, L^2}))\\
& \leq k \int\limits^h_0 \sup\limits_{x \in R^n} \sum\limits_{| \nu | \leq
  2n} \sup\limits_{| \xi | \geq 1} \left| \left(\frac{\partial}{\partial
  \xi}\right)^{\nu} \frac{\partial}{\partial t} \sigma (\mathfrak{N}) (x, t,
\xi) \right|. 
\end{align*}

Hence $\{u_{\Delta} (t) \}$ is a bounded set in $\mathscr{E}^1_{L^2}$, 
this completes the proof. 

If we assume taht that coefficients and the initial data $u_0$ and $f$
are sufficiently smooth we can improve Theorem
\ref{chap3-sec8-thm2}. We indicate this briefly. 

We assume 
$$
A_k \in \mathbb{B}^2 [0, h], \frac{\partial A_k}{\partial t}
\in \mathbb{B}^0 [0, h], B \in \mathbb{B}^2 [0, h], u_0
\in \mathscr{D}^2_{L^2}, f \in \mathscr{D}^2_{L^2}, [0,
  h] 
$$

From theorem \ref{chap3-sec8-thm2}, we know that there exists a unique
solution $u \in\mathscr{D}^1_{L^2} [0, h]$ of  
\begin{equation*}
M[ u ] = f. \tag{8.16}\label{chap3-eq8.16}
\end{equation*}

Differentiating\pageoriginale with respect to $x_j$ (denoting
$\dfrac{\partial}{\partial x_j}$ by $D_j$) we have  
\begin{equation}
M [ D_j u] - \sum_{k} ~ (D_j A_k ) [ D_k u ]  = D_j f + ~ (D_j B) [ u]
~ (j = 1, 2,\ldots, n) \tag{8.22} \label{chap3-eq8.22} 
\end{equation}
where the second member $D_j f + (D_j B) [ u ] ~ \in
\mathscr{D}^{1}_{L^2} [ 0, h ]$ and $D_j u (0) \in 
\mathscr{D}^{1}_{L^2}$. 
Now \eqref{chap3-eq8.22} is a system of equations with unknown
functions $( D_1 u, 
\ldots, D_n u )$ which has the same principal part $M$. We can show,
without any singnificant modification in the previous argument, that
there exists a unique  solution $( D_1 u, \ldots, D_n u ) \in
\mathscr{D}^{1}_{L^2} [0, h]$. On the other hand, by the energy
inequality, we can see that the system: 
$$
M[v_j] - \Sigma_k ~ ( D_j A_k ) [ v_k ] = g_j \varepsilon ~ L^2 [ 0,
  h ] 
$$
has atmost one solution $v$ in $L^2 [0, h]$. This shows that $u
\in \mathscr{D}^{2}_{L^2} [0, h]$. 
\end{proof}

\setcounter{corollary}{0}
\begin{corollary}\label{chap3-sec8-coro1} %corollary 1.
 Let \eqref{chap3-eq8.16} be a regularly hyperbolic system in the set $\Omega =
 \underbar{R}^n \times [0, T]$ with 
 \begin{gather*}
\left(A_k (x,t), ~ \frac{\partial}{\partial t} A_k (x, t )\right) \in
   ( \mathscr{B}^2 [ 0, T ], ~ \mathscr{B}^1 [ 0, T ] ), \\ 
\left(B (x,t), ~ \frac{\partial}{\partial t} B (x,t )\right) \in (
   \mathscr{B}^2 [ 0, T ], ~ \mathscr{B}^1 [ 0, T ] )  
\end{gather*} 
and  $f (x, t ) \varepsilon \mathscr{D}^{2}_{L^2} [ 0,T]$.
\end{corollary}

Then, given an element $u_0 \in \mathscr{D}^{2}_{L^2}$ there
exists a unique solution  $u \in \mathscr{D}^{2}_{L^2} [0,T]$ of
\eqref{chap3-eq8.3} with $u (0) = u_0$.  

\begin{proof}
Defferentiating both sides of the equation \eqref{chap3-eq8.3} with
respect to $x_j$ in the sense of distributions we have  
$$
\frac{\partial}{\partial x_j} M [ u ] = \frac{\partial}{\partial x_j}
\frac{\partial}{\partial  t} u - \frac{\partial}{\partial x_j} \left(
\sum A_k (x,t) \frac{\partial}{\partial x_k} \right)
-\frac{\partial}{\partial x_j} (B (x,t) u ) = \frac{\partial
  f}{\partial x_j} 
$$\pageoriginale
which can be rewritten as 
\begin{align*}
\frac{\partial}{\partial  t} \left(\frac{\partial u}{\partial x_j}\right) - \sum
& A_k (x,t) \frac{\partial}{\partial x_j} \left( \frac{\partial u}{\partial
  x_j} \right) - \sum \frac{\partial A_k}{\partial x_j} (x,t) 
\frac{\partial u}{\partial x_k} - B (x, t) \frac{\partial u}{\partial
  x_j} \\
& = \frac{\partial B}{\partial x_j} (x,t) u + \frac{\partial
  f}{\partial x_j}. 
\end{align*}

That is,
\begin{equation}
M \left[  \frac{\partial u}{\partial x_j} \right] = -\sum \frac{\partial
  A_k}{\partial x_j} (x, t) \frac{\partial u}{\partial x_k} =
\frac{\partial f}{\partial x_j} + \frac{\partial B}{\partial x_j}
(x,t) u. \tag{8.22}\label{chap3-eq8.22} 
\end{equation}

Setting $\dfrac{\partial u}{\partial x_j} = v_j$ for $j = 1,
\ldots, n$ we obtain a new system  
$$
M \left[ v_j \right] - \sum \frac{\partial A_k}{\partial x_j} (x, t) 
v_k  = \varphi_j 
$$
and we can take for $v_j (0)$ the function $\dfrac{\partial 
  u_0}{\partial x_j} \in \mathscr{D}^{1}_{L^2}$ (the
derivative being taken in the sense of distributions) since $u_o
\in \mathscr{D}^{2}_{L^2}$. 
If we assume $u \in \mathscr{D}^{1}_{L^2} \left[  0, T
  \right]$ it follows then that $\varphi_j \in
\mathscr{D}^{1}_{L^2} \left[  0, T \right]$ since  $f \in
\mathscr{D}^{2}_{L^2} \left[0, T \right]$ and  $B \in
\mathscr{B}^{2} \left[0, T \right]$. Then by Th. 2 there exists a
unique solution (in $L^2$) $v = ( v_1, \ldots , v_n )$ with  $v_j
\in \mathscr{D}^{1}_{L^2} \left[ 0, T \right]$. Hence $u
\in \mathscr{D}^{2}_{L^2} \left[ 0, T \right]$. 
\end{proof} 

\begin{corollary}\label{chap3-sec8-coro2} %corollary 2.
Let \eqref{chap3-eq8.16} be regularly hyperbolic in the set $\Omega$ with  
$$
\left( A_k, \frac{\partial A_k}{\partial t}, \ldots, \left(
\frac{\partial}{\partial t} \right)^m A_k \right) 
\in ( \mathscr{B}^{m}
\left[  0, T \right], \ldots, \mathscr{B}^{0} \left[  0, T \right] ),
~ B \in \mathscr{B}^{m} \left[0, T \right] 
$$
and $f \in \mathscr{D}^{m}_{L^2} \left[ 0, T \right]$; then
given $u_0 \in \mathscr{D}^{m}_{L^2} $ there exists a unique
solution $u$ in\pageoriginale $\mathscr{D}^{m}_{L^2} \left[ 0, T
  \right]$ of \eqref{chap3-eq8.3} with $u (0) = u_0$.  

 This can be proved by successively applying  the argument of
 Corollary \ref{chap3-sec8-coro1}.  

Taking $m = \left[\frac{n}{2}\right] + 2$ we obtain, using Sobolev's lemma, the
following  
\end{corollary}

\begin{corollary}\label{chap3-sec8-coro3} % corollary 3.
 Let \eqref{chap3-eq8.16} be regularly hyperbolic with 
 $$
 \left(A_k,  \frac{\partial A_k}{\partial t},\ldots\right) \in
 \left(\mathscr{D}^{[\frac{n}{2}] +2}_{L^2} \left[  0, T \right], ~
 \mathscr{D}^{[\frac{n}{2}] +1}_{L^2},\ldots\right), \;  B \in
 \mathscr{D}^{[\frac{n}{2}] +1}_{L^2} \left[  0, T \right] 
 $$
 and $f \in \mathscr{D}^{[\frac{n}{2}] +2}_{L^2} \left[  0, T
   \right]$ 
  then, given $u_0 \in  \mathscr{D}^{[\frac{n}{2}] + 2}_{L^2}
  $ there exists a solution $u \in\mathscr{E}^1$ of \eqref{chap3-eq8.16}
  with $u (0) = u_0$, unique in $L^2$. 
\end{corollary} 

\begin{corollary}\label{chap3-sec8-coro4} % corollary 4.
Assume that \eqref{chap3-eq8.16} is regularly hyperbolic in an open
neighbourhood $U$ 
of 0 in $\underbar{R}^{n+1}$, $A_k$, $B \in \mathscr{E} (U)$
then there  exists a neighbourhood $U' \subset U$ such that for any
$u_0 \subset \mathscr{E} (  U \cap \{ t = 0 \} )$, $f \in
\mathscr{E} (U)$ there exists a solution $u \in \mathscr{E}
(U')$ of \eqref{chap3-eq8.16}, unique in $L^2$. 
\end{corollary}

\begin{remark*}%rema 0
If we use a partition of unity the above arguments can be used to
proved results analogous to the above corollaries in the spaces
$\mathscr{E}^m_{L^2 (\loc)}$ in place of $\mathscr{D}^m_{L^2}$. 

Finally we have the following result on the existence of solutions of
a single regularly hyperbolic equation of order $m$. 
\end{remark*}

\begin{corollary}\label{chap3-sec8-coro5} % corollary 5.
 Let 
 \begin{equation*}
L \left[  u \right]  \equiv \left( \frac{\partial}{\partial t}
\right)^{m_u} + \sum\limits_{\substack{ j + | \nu | \leq m  \\ j < m}} a_{j, \nu} (x,t)
\left(\frac{\partial}{\partial  x}\right)^\nu  \left(
\frac{\partial}{\partial t} \right)^j 
u = g  \tag{8.23} \label{chap3-eq8.23}
 \end{equation*} 
be a\pageoriginale regularly hyperbolic equation of order $m$ in a
neighbourhood of the origin with infinitely  differentiable
coefficients $a_{j,\nu}$. Let $g$ be infinitely diffrentiable in a
neighbourhood $U$ of the origin. Then given the initial conditions  
$$
(u_o, u_1, \ldots ,  u_{m-1} )\in \prod \mathscr{E}( U \cap \{ t = 0
\}) 
$$
there exists a solution $u \in \mathscr{E}(U')$ in a neighbourhood
$U'$ such that  
$$
\left(\frac{\partial}{\partial t}\right)^j u (x, 0) = u_j (x), \;\; j
=0, 1,  \ldots, (m-1). 
$$
\end{corollary}

\section[Necessary condition for the well posedness.....]{Necessary
  condition for the well posedness of the Cauchy problem}\label{chap3-sec9}%sec 9  

In chapter \ref{chap2} we considered necessary condition for well posedness of
the Cauchy problem when the coefficients were inependent of $x$. In
Chapter \ref{chap3} we considered some sufficiency condition for well
posedness e.g, hyperbolicity, when the coefficients depended on $x$.  

Now we consider some necessary conditions in this later case. For
simplicity we shall consider single, first order differential
opearator, 
\begin{equation*}
\frac{\partial u}{\partial t} = \sum a_k (x, t) \frac{\partial
  u}{\partial x_k} + b(x, t)u \tag{9.1}\label{chap3-eq9.1}
\end{equation*}
(for a fuller treatement see Mizohata \cite{key3}).

If all $a_k (x, t)$, $b(x, t)$ are real the classical method of
characteristics establishes the well posedness of the Cauchy
problem. However if the $a_k$ are complex the question of existence
was not settled till recently. The characteristic polynomial of the
above equation is $\sum a_k (2 \pi \xi_k) - \lambda$. If this has real
eigenvalues i.e. $a_k$'s are real, the Cauchy problem is well
posed\pageoriginale as 
shown by the results  of Chap III. We shall prove that if there is
$\xi^o$ such that $\im \sum a_k (0, 0) \xi^o \neq 0$ (say $\neq 0$),
the problem is not well posed. The idea of the proof is as follows: we
construct a sequence of solution $u_n (x)$, $n = 1, 2$ for which, on
the hypothesis of well-posedness, we must have $\sup |u_n (x, t)| =
0(n^h)$ while on the other hand by using an energy inequality for a
suitable operator we, must have a minorization by $\exp(n)$ for some
functions closely related to $u'_n$ above which will give a
contradiction. More precisely we shall prove 

\begin{proposition}\label{chap3-sec9-prop1} % \prop 1
Let 
\begin{equation*}
\frac{\partial u}{\partial t} = H \wedge u + b(x, t) u + f
\tag{9.2} \label{chap3-eq9.2} 
\end{equation*}
be an equation in the singular integral form with $\sigma (H) = h (x,
t, \xi)$ satisfying 
\begin{equation*}
 \re h(x,t,\xi ) \le 0 \text{ for all } (x, t, \xi ) \text{ and
} t \to h(x, t,\xi) \in C^\infty_{1 +  \sigma} \tag{9.3}\label{chap3-eq9.3} 
\end{equation*}
is continuous. Then given any $u_o \in \mathscr{D}^1_{L^2}$ and $f \in
\mathscr{D}^1_{L^2}[0, h]$ there exists a unique solution 
$$
u \in \mathscr{D}^1_{L^2}[0, h] \text{~ of  \eqref{chap3-eq9.2} with~ } u(x, 
0) = u_o(x). 
$$
 
 On the contrary, if there exists a $\xi^o$ such that $\re h(x, t,
 \xi^o) > 0$ then the energy inequality cannot be obtained in the
 $L^2$-space. Of course this does not immediately imply that the
 Cauchy problem is not well posed in $\mathscr{D}^\infty_{L^2}$. 
 
 We see\pageoriginale that $a(x, t, - \xi)= - a(x, t, \xi)$ which
 shows that in the case of a differential operator \eqref{chap3-eq9.1} the
 condition $\re a(x, t, \xi)  \equiv 0$ will be necessary for the
 existence theorem. We analyse  this situation more clearly.  
 \end{proposition}
 
\setcounter{theorem}{0}
\begin{theorem}\label{chap3-sec9-thm1} % \the 1
Suppose there exists a real vector $\xi^o \in \underbar{R}^n$, $\xi^o
\neq 0$ and $x^o$ such that $\Iim \sum a_k (x^o, 0) \xi^o_k < 0$. Then
the forward Cauchy problem is not well posed for \eqref{chap3-eq9.1} in
$\mathscr{E}$ or in $\mathscr{D}^\infty_{L^2}$ or in $\mathscr{B}$.  
 \end{theorem} 

\begin{remark*}%rema 0
P.D. Lax \cite{key1} also proved a similar theorem, by using the
characteristic method, that if eigenvalues are simple for the well
posedness of the Cauchy problem it is necessary that the eigenvalues
be real. 
\end{remark*} 
 
 We first prove an energy inequlity for a suitably modified operator
 and then establish two lemmas for commutators which together will
 prove the theorem. Suppose $x^0 = 0$. 
 
 First of all we localize the differential operator given in
 \eqref{chap3-eq9.1}. Suppose $u$ is a solution of \eqref{chap3-eq9.1}
 of class  $\mathscr{E}^1$. Let 
 $\beta (x) \in \mathscr{D}$ with support contained in a small
 neighbourhood of the origin. Now 
 \begin{equation}
\frac{\partial}{\partial t} (\beta  u) = \sum a_k
\frac{\partial}{\partial x_k} (\beta u) +  b (\beta u) - \sum a_k
\frac{\partial \beta}{\partial x_k} u \tag{9.4} \label{chap3-eq9.4}
\end{equation} 

 Since the support of $\beta u$ and of $\dfrac{\partial}{\partial
   x_k}(\beta u)$ are contained in the support of $\beta$ we can
 modify $a_k$ and $b$ outside the support of $\beta$. We can write 
 \begin{equation*}
\frac{\partial}{\partial t} (\beta u) - \sum \tilde{a}_k (x, t)
\frac{\partial}{\partial x_k} (\beta u) - \tilde{b} (x,t) (\beta u)
=- \sum \tilde{a}_k (x,t) \frac{\partial \beta}{\partial x_k} \cdot u
\tag*{$(9.4)'$} 
 \end{equation*} 
 where\pageoriginale $\tilde{a}_k$ and $\tilde{b}$ are equal to $a_k$
 and $b$ respectively on the support of $\beta$ and   
\begin{align*}
{\rm(i)}~  &  \tilde{a}_k,  \; \tilde{b} \in \mathscr{B}^\infty_{x, t} \\
{\rm(ii)}~ &  \im \sum \tilde{a}_k (x,t) \xi^o_k  < - \delta, \; \delta > 0
  \text{ for all }(x,t) \text{ with } x \in  \underbar{R}^n \text{ and
  } 0 \le t \le t_o \tag{9.5} \label{chap3-eq9.5}
\end{align*}

We can assume $|\xi^o| = 1$ if necessary by multiplying by a
suitable constant. There exists a neighbourhood $V$ of $\xi^o$ such
that  
\begin{equation}
\im  \sum \tilde{a}_k (x,t) \xi_k \le - \delta \text{ for } 0 \le t
\le t_o, \xi \in V. \tag{9.6}  \label{chap3-eq9.6}
\end{equation}

Let $\hat{\alpha} \in \mathscr{D}$ with the support contained on $V$
and $\hat{\alpha}(\xi) \equiv 1$ in a neighbourhood of $\xi^o$. Define
$\hat{\alpha}_p$ by 
\begin{equation}
\hat{\alpha}_p (\xi) = \hat{\alpha} ( \frac{\xi}{p}), \alpha_p (x) =
\bar{\mathscr{F}} [\hat{\alpha}_p (\xi ) ]. \tag{9.7} \label{chap3-eq9.7}
 \end{equation} 
 
Convolving both sides of $(9.4)'$ with $\alpha_p$ we obtain
 \begin{align*}
& \left(\frac{\partial}{\partial t} - \sum \tilde{a}_k
   \frac{\partial}{\partial x_k} - \tilde{b} \right) \left(\alpha_p
   *_{(x)} (\beta  u) \right) \equiv L \left[ \alpha_p *_{(x)} (\beta
     u) \right] \\
&= - \left[ \alpha_p *, L \right](\beta u) - \sum \tilde{a}_k
   (\alpha_p 
   *_{(x)}(\beta_k u)) - \sum \left[\alpha_p *_{(x)}, \tilde{a}_k
     \right] (\beta_k  u). \tag{9.8} \label{chap3-eq9.8}
\end{align*}
where $\beta_k = \dfrac{\partial\beta}{\partial x_k}$.
 
We rewrite
 $$
 \sum \tilde{a}_k \frac{\partial}{\partial x_k} (\alpha_p *_{(x)}v) =
 H \wedge (\alpha_p *_{(x)}v) 
 $$
 where $v = \beta u$ and $\sigma (H)= 2 \pi i \sum \tilde{a}_k
 \frac{\xi_k}{|\xi|} =  h(x, t, \xi)$. That is  
 $$
 H \wedge (\alpha_p *_{(x)}v) = \int \exp (2 \pi ix. \xi) \cdot h (x, t,
 \xi)|  \xi | \hat{\alpha}_p  (\xi) \hat{v} (\xi) d \xi. 
 $$\pageoriginale

 This operator depends only on the value of the symbol $h$ on the set
 $\{ \lambda V\}_{\lambda \geq 0}$ since the support of $| \xi |
 \hat{\alpha_p} \hat{v}$ is contained in the set $\{ \lambda V
 \}$. Hence we can modify the symbol $h$ to $h$ outside $\{ 
 \lambda V\}$ as follows: 
\begin{enumerate}[(i)]
\item $\tilde{h} (x, t, \xi) \equiv h(x, t, \xi)$ for $\xi \in \lambda
  V$ 

\item $\re \tilde{h} (x, t, \xi) \ge \delta', \delta' > 0 $. 
 \end{enumerate} 
 
 Thus we have finally an equation
 \begin{equation}
( \frac{\partial}{\partial t} - \tilde{H}\wedge - \tilde{b}) ( \alpha_p
   *_{(x)}(\beta u)) = f \tag{9.9} \label{chap3-eq9.9}
 \end{equation} 
 where $\tilde{H}$ is the singular integral operator whose symbol
 $\sigma (\tilde{H})$ is $\tilde{h}, f$ being the right-hand side of
 \eqref{chap3-eq9.8}. 
 
\setcounter{lemma}{0}
 \begin{lemma}\label{chap3-sec9-lem1}% \lem 1
Suppose $H(t)$ is a singular integral operator of class
$C^\infty_\beta$, $\beta = \infty$ such that  
\begin{equation}
\sigma (H) (x, t, \xi) \ge \delta' > 0. \tag{9.10} \label{chap3-eq9.10}
\end{equation}

Suppose $f \in L^2 [0, h]$ is given. If $u \in \mathscr{D}^1_{L^2}[0, 
  h]$ satisfies 
 \begin{equation}
\frac{\partial}{\partial t} (\alpha_p *_{(x)} u) = H \wedge (\alpha_p
*_{(x)}u) + b(x, t)( \alpha_p *_{(x)}u) +  f
\tag{9.11} \label{chap3-eq9.11} 
 \end{equation} 
 then there exists $a\ \delta'' > 0$ such that 
 \begin{equation}
\frac{d}{dt} \| \alpha_p *_{(x)}u \|  \ge \delta'' p \| \alpha_p
*_{(x)} u \| - \| f \| \tag{9.12}\label{chap3-eq9.12} 
\end{equation} 
for sufficiently large $p$.
\end{lemma}

\begin{proof}
Let\pageoriginale us denote $\alpha_p  *_{(x)}u$ by $v_p$. Then we
have   
$$
\frac{d}{d t} (v_p, v_p) = (( H \wedge  + \wedge H^*)v_p, v_p) + 2 \re 
(bv_p, v_p) + 2 \re (v_p, f). 
$$

But $\wedge H^* = H^{\#} \wedge (\text{mod} \wedge^o)$ implies
 $$
 \frac{d}{dt}(v_p, v_p) =  (( H + H^{\#}) \wedge v_p , v_p) +  2  \re 
 (bv_p, v_p) + 2  \re (v_p, f) + (Bv_p,  v_p), 
 $$
 with $B$ a bounded operator. If $P$ denotes the singular integral
 operator $H + H^{\#}$ then $\sigma (P)\ge 2 s'$. We remark that $(P
 \wedge^s - \wedge^s P) \wedge^\sigma$ is a bounded operator if $s,
 \sigma \ge 0$ and  $s + \sigma \le 1$. Taking $s= \sigma =
 \dfrac{1}{2}$ 
 $$
 P \wedge \equiv \wedge^{\frac{1}{2}} P \wedge^{\frac{1}{2}} (\text{mod~}
 \wedge^0). 
 $$

 Hence 
 $$
 (P \wedge v_p, v_p) = P \wedge^{\frac{1}{2}} v_p,
 \wedge^{\frac{1}{2}} v_p) + ((C v_p, v_p)  
 $$
 where $C = P \wedge - \wedge^{\frac{1}{2}} P \wedge^{\frac{1}{2}}$
 is a bounded operator. Thus 
 \begin{equation}
\frac{d}{d t} (v_p, v_p) \ge \re ((H +H^{\#}) \wedge^{\frac{1}{2}}
v_p, \wedge^{\frac{1}{2}} v_p) - \gamma_1 \| v_p \|^2 -2 v_p \|\;  \| f
\|   \tag{9.13} \label{chap3-eq9.13}
 \end{equation} 
 on the other hand we have by G$\ring{\text{a}}$rding's lemma that  
\begin{equation}
\re ((H +H^{\#}) \wedge^{\frac{1}{2}} v_p, \wedge^{\frac{1}{2}}v_p)
\ge \delta' (\wedge^{\frac{1}{2}}_{p}, \wedge^{\frac{1}{2}} v_p) -
\gamma_2 \|  v_p \| ^2. \tag{9.14} \label{chap3-eq9.14}
 \end{equation} 

  Since the distance of the support of $\hat{v}_p (\xi, t) \equiv
  \hat{\alpha}_p (\xi) \hat{u} (\xi, t)$ from the origin is larger
  than $\sigma p$, $\sigma > 0$, we have by Plancheral's formula 
 \begin{align*}
(\wedge^{\frac{1}{2}} v_p, \wedge^{\frac{1}{2}} v_p) & = \int |\xi
   |\hat{v}_p (\xi, t) |^2 d \xi \\ 
& \ge \sigma p  \int | \hat{v}_p (\xi, t) |^2 d \xi = \sigma p \| v_p
   \|^2  
 \end{align*} 

Thus\pageoriginale we have 
 $$
 \frac{d}{d t} \|v_p \|^2 \ge \sigma^{\delta'} p \|v_p \|^2 - (\gamma_1 +
 \gamma_2) \|v_p \|^2 - 2 \| v_p \|\;\;  \| f \| 
 $$
 which implies
 $$
 \frac{d}{dt} \|v_p \| \ge ( \delta p - \gamma) \|  v_p \| - \| f \|  
 $$ 
 where  $\delta > 0 $, $\gamma > 0$ are constants. Therefore for large
 $p ( \delta p- \gamma) \ge \delta'' _p$, $\delta'' >0$. 
 
For such $p$ we have 
$$
 \frac{d}{dt} \| v_p \| \ge \delta '' p \| v_p \| - \|  f \|. 
$$ 

In other words we have
\begin{equation}
\frac{d}{dt} \| \alpha_p *_{(x)}u \| \ge \delta '' p \| \alpha_p
*_{(x)}u \| - \|  f \| \tag{9.12}\label{chap3-eq9.12} 
\end{equation} 
completing the proof of the lemma.
\end{proof}

\begin{lemma}\label{chap3-sec9-lem2} % \lem 2
If $a \in \mathscr{B}$ and $u \in \mathscr{D}^1_{L^2}$ there exists a
constant $c > 0$ such that  
\begin{gather*}
\| \left[ \alpha_p *, a(x) \frac{\partial}{\partial x_j} \right] u \|  \le c
\left\{ \sum_{1 \le  | \rho | \le m-1} \|  \frac{\partial}{\partial
  x_j} (x^{\rho} \alpha_p) * u \| \right.\\
\left.  +  \left( \sum_{| \rho | = m} \|
\frac{\partial}{\partial x_j}(x^\rho \alpha_p) \|_{L^1} +  \| (x^\rho
\alpha_p) \| _{L^1}\right) \| u \| \right\}. \tag{9.15}\label{chap3-eq9.15} 
\end{gather*}
 \end{lemma}

\begin{proof}
Let $v =  \left[ \alpha_p *, a(x) \dfrac{\partial}{\partial
    x_j}\right] u$; then  
$$
v(x) = \int (a(y) - a(x)) \alpha_p (x-y) \frac{\partial u}{\partial
  y_j} (y) dy. 
$$
 
Expanding\pageoriginale $a(y)- a(x)$ by mean value theorem upto order
$m$, to be determine later,  
 $$
 a(y) - a(x) = \sum\limits_{1 \le | \varrho | \le m-1} \frac{(y -
   x)^\rho}{\rho !} ( \frac{\partial}{\partial x})^\rho a(x) + \sum_{|
   \rho | =m} a_\rho (x,y) (x-y)^\rho 
 $$
 and hence 
 \begin{align*}
 v(x) & = \sum\limits_{1 \le | \rho | \le m-1} \frac{(-1)^\rho}{\rho
   !} (\frac{\partial}{\partial x})^\rho a(x) \frac{\partial}{\partial
   x_j}(x^\rho \alpha_p) * u\\
& \qquad  +  \sum\limits_{|\rho| = m}\int (x-y)^\rho
 \alpha_p (x-y) a_\rho (x,y) \frac{\partial u}{\partial y_j}(y) \; dy.  
 \end{align*}

  Now \quad $\varphi (x) = \int (x-y)^\rho \alpha_p (x-y) a_\rho (x,y)
  \dfrac{\partial u}{\partial y_j} (y) \, dy$ 
 \begin{align*}
& = - \int \frac{\partial}{\partial y_j} \left\{ (x-y)^\rho \alpha_p
   (x-y) a_\rho (x,y) \right\} u(y) \, dy \\ 
& = - \int \left\{ \frac{\partial}{\partial y_j} \left[ (x-y)^\rho
     \alpha_p (x-y) \right] a_\rho (x,y)  \right.\\
& \qquad  \left. +(x-y)^\rho \alpha_p (x-y)
   \frac{\partial}{\partial y_j} a_\rho (x,y) \right\} u(y) \, dy. 
 \end{align*} 

 Hence $\|\varphi (x) \| \le c \| \; |x^\rho \alpha_p | *|u| +  |
 \dfrac{\partial}{\partial x_j} (x^\rho \alpha_p) | *| u |
 \|$. Applying Haus\-dorff-Young inequality to the right hand side we
 obtain the desired inequality. 
 
 Similarly one can prove that if $a \in \mathscr{B}$ and $u \in  L^2$
 then  
\begin{equation*} 
\| [ \alpha_p *, a] u \| \le c \left\{ \sum_{1 \le |  \delta | \le
  m-1} \| (x^\rho \alpha_p )* u \| + ( \sum_{| \rho |=m} \|
x^\rho \alpha_p \|_{L^1}) \| u \| \right\}\tag{9.16} \label{chap3-eq9.16}
\end{equation*} 
where $c$ is a positive constant.
 
 Now\pageoriginale we look  at the terms appearing in the right hand
 side of \eqref{chap3-eq9.15}.  
 
 First of all $\dfrac{\partial}{\partial x_j}(x^\rho \alpha_p) * u$
 has its Fourier image $(2 \pi i \xi_j) (\hat{x^\rho} \alpha_p)
 \hat{u} = (2 \pi i \xi_j) \hat{u}$. const.${\hat{\alpha}}^{\rho}_p
 (\xi)$ which shows, since the support of ${\hat{\alpha}}^{\rho}_p
 (\xi)$ has diameter $\sigma' p$ where $\sigma'$ is a constant
 depending only on $\hat{\alpha}$, that, 
 \begin{equation}
\|  \frac{\partial}{\partial x_j}(x^\rho \alpha_p) * u \| \le c p \|
(x^\rho \alpha_p) * u \|. \tag{9.17} \label{chap3-eq9.17} 
 \end{equation} 

 Next consider $\| x^\rho \alpha_p \|_{L^1}$ for $| \rho | = m$ 
\begin{align*}
\sup \left| x^\rho \alpha_p \right| & \le \text{ const. }  \int 
|\hat{\alpha}^{(\rho )}_p (\xi)  | d \xi 
 = \text{ const. } \int | ( \frac{\partial}{\partial \xi})^\rho
 \hat{\alpha}_p (\xi)  | d \xi \\ 
& = \text{ const. }  (\frac{1}{p})^{| \rho| -n} \int  | (
\frac{\partial}{\partial \xi})^\rho \hat{\alpha}| d \xi. 
 \end{align*}

Similarly $| x |^{2n}| x^\rho \alpha_p | \le$ const
 $\left(\dfrac{1}{p}\right)^{|\rho|+n} \int \left| \Delta^n_\xi (
 \dfrac{\partial}{\partial \xi})^\rho \hat{\alpha}\right| d  \xi$
 which implies that   
 $$
 (1 + |x|^{2n}) |x^\rho \alpha_p |\le \text{ const. }
 \left(\frac{1}{p}\right)^{|\rho|-n}. 
 $$

Hence $\|x^\rho \alpha_p \|_{L^1} \le$ const. $\int \dfrac{dx}{1 + |
  x|^{2n}} \cdot \left( \dfrac{1}{p}\right)^{ | \rho | -n} \le c
\left(\dfrac{1}{p}\right)^{|\rho | -n}$. In the same way one can show that  
 $$
 \| \frac{\partial}{\partial x_j}(x^\rho \alpha_p) \|_{L^1} \le c\left(
 \frac{1}{p}\right)^{| \rho | -n -1}. 
 $$
\end{proof}

Thus we have proved the 

\smallskip
\noindent
{\bf Corollary of Lemma \ref{chap3-sec9-lem2}:}
If\pageoriginale $a \in \mathscr{B}$ and $u \in \mathscr{D}^1_{L^2}$ then 
\begin{equation}
\|  [ \alpha_p *, a (x) \frac{\partial}{\partial x_j} ] u \| \le c p
\sum\limits_{1\le|\rho|\le m-1} \| (x^\rho \alpha_p) * u \| + O \left(
\frac{1}{p^{m-n-1}}\right) \| u \|. \tag{9.18} \label{chap3-eq9.18}
\end{equation} 

This follows from \eqref{chap3-eq9.15}.

Similarly it follows from  \eqref{chap3-eq9.16} that 
\begin{equation*}
\|  [ \alpha_p *, a (x)  ] u \| \le c  \sum\limits_{1\le|\rho|\le m-1}
\| (x^\rho \alpha_p) * u \| + O (\frac{1}{p^{m-n}}) \|
u\|. \tag*{$(9.18)'$}  
\end{equation*}

\begin{lemma}\label{chap3-sec9-lem3}% lemma 3
If $L$ is a differential operator of the first order with its
coefficients in $\mathscr{B}$ 
\begin{equation*}
L = \sum a_k (x) \frac{\partial}{\partial x_k} + b(x)
\tag{9.19}\label{chap3-eq9.19} 
\end{equation*}
then for any $u \in  \mathscr{D}^1_{L^2}$
\begin{equation*}
\|  [ \alpha_p *, L ] u \| \le c  \sum\limits_{1\le|\rho|\le m-1} p\|
(x^\rho \alpha_p) * u \| + O ( \frac{1}{p^{m-n-1}}) \| u
\|. \tag{9.20} \label{chap3-eq9.20}
\end{equation*}

This is an immediate consequence of the inequalities \eqref{chap3-eq9.18} and
$(9.18)'$. More generally one can prove exactlly in the same way 
\begin{equation*}
\| [ ( x^\nu \alpha_p ) *, L] u \| \le c
\sum\limits_{|\nu|+1\le|\rho|\le m-1} \| (x^\rho \alpha_p) * u \| + O
\left( \frac{1}{p^{m-n+ | \nu |}}\right) \| u
\|.\tag{9.21} \label{chap3-eq9.21} 
\end{equation*}
and 
\begin{equation*}
\| [ ( x^\nu \alpha_p ) *, L] u \| \le cp
\sum\limits_{|\nu|+1\le|\rho|\le m-1} \| (x^\rho \alpha_p) * u \| + O
( \frac{1}{p^{m+  1 
    | \nu | - n -}}) \| u \|. \tag{9.22} \label{chap3-eq9.22}
\end{equation*}
for every $u \in  \mathscr{D}^1_{L^2}$.
\end{lemma}

Now we\pageoriginale can complete the 

\smallskip
\noindent
{\bf Proof of Theorem \ref{chap3-sec9-thm1}:~}
We prove this theorem in the spaces $\mathscr{E}$. As we shall see from
the method of proof the same will be valid for the spaces
$\mathscr{D}^\infty_{L^2}$ and $\mathscr{B}$. The proof is by
contradiction. 

Suppose the Cauchy problem is well posed in the spaces. We construct a
sequence of initial conditions $\psi_q (x)$ and consider the
corresponding sequence of solutions $\psi_q(x)$ are defined as
follows: 

Let $V$ be a small a neighbourhood of $\xi^o$ and $\hat{\alpha} \in
\mathscr{D}$ have its support in $V$ with $\hat{\alpha} (\xi) \equiv
1$ in neighbourhood $V'$ of $\xi^o$, $V'  \subset V$. Take $a\
\hat{\psi} \in \mathscr{D}$, $\hat{\psi} (\xi) \neq 0$ with support
contained in $V'$. Denoting  
$$
\hat{\psi}_q ( \xi ) = \hat{\psi}( \xi - q \xi^o) 
$$
we have by taking inverse Fourier transforms
\begin{equation}
\psi_q (x) = \exp (2 \pi iqx. \xi^o ) \psi (x)
\tag{9.23} \label{chap3-eq9.23} 
\end{equation}
$\psi_q \in \mathscr{E}$ (also in $\mathscr{D}^\infty_{L^2}$,
$\mathscr{B}$). Further 
\begin{equation}
\| \psi_q \|_{\mathscr{E}^h} = O (q^h). \tag{9.24}\label{chap3-eq9.24}
\end{equation}
(We remark that \eqref{chap3-eq9.24} holds for the semi-norms in
$\mathscr{D}^\infty_{L^2}$ and $\mathscr{B}$ also). 

By hypothesis of the well posedness, the corresponding solution $u_q
(x,t)$ of \eqref{chap3-eq9.1} having $\psi_q(x)$ as the initial data
is estimated by   
\begin{equation*}
\sup\limits_{K} |u_q (x, t)| = O(q^h) \tag{9.25}\label{chap3-eq9.25}
\end{equation*}
for some fixed $h$ where $K$ is a compact set in the
$(x,t)$-space. Also we see that  
\begin{equation}
\| \alpha_p  * (\beta \psi_p) \| \ge c >
0. \tag{9.26}\label{chap3-eq9.26} 
\end{equation}\pageoriginale

In fact,
\begin{gather*}
\alpha_p *(\beta \psi_p ) =  \beta( \alpha_p * \psi_p) +  [ \alpha_p
  *, \beta ] \psi_p. \\ 
\| \beta(\alpha_p * \psi_p) \|  = \| \hat{\beta} * ( \hat{\alpha}_p
\hat{\psi}_p) \| = \| \hat{\beta} *\hat{\psi}_p \|  = \| \beta \psi_p
\| = \| \beta \psi \| >  0 
\end{gather*}
by using Plancheral's formula and the fact that $\hat{\alpha}_p \equiv
1$ on the support of $\hat{\psi}_p, \psi$ being an analytic function
and $\|\beta \psi ||> 0$. On the other hand it is easy to see that $\|
[\alpha_p *, \beta] \psi_p \| = O\left(\dfrac{1}{p}\right)$. Now we prove this
leads to contradiction as follows. Instead of $\alpha_p * (\beta u)$
in \eqref{chap3-eq9.8} we consider $( x^\nu \alpha_p) * (\beta^\nu
u_p)$ with $| \nu | \le m-1$, $| \nu | \le m + mh$ which form a system
of localisers, $\beta^\mu = \left( \dfrac{\partial}{\partial
  x}\right)^\mu \beta (x)$. Then we have   
$$  
\left( \frac{\partial}{\partial t} - \sum 
\tilde{a}_{k}\frac{\partial}{\partial x_k} - \tilde{b}\right) (( x^\nu
\alpha_p) * ( \beta^\mu u_p)) 
$$  
\begin{align*}
= [L, (x^\nu \alpha_p ) *] (\beta^{\mu} u_p) &- \sum \tilde{a}_k ((
x^\nu \alpha_p ) *  ( \beta^{\mu_k} u_p)) \\ 
&- \sum [(x^\nu \alpha_p ) *, \tilde{a}_k ] ( \beta^{\mu_k} u_p) 
\end{align*}
where $\mu_k = \mu +  e_k$, $\ell_k = (0, \ldots , 1 , \ldots , 0)$
the $kth$ component is 1. 

Applyinginequallity \eqref{chap3-eq9.12} for $(x^\nu \alpha_p) * (
\beta^\mu u_p)$ 
and using inequalities \eqref{chap3-eq9.21}, \eqref{chap3-eq9.22} with
$m = h  + n +2$, we have   
{\fontsize{10pt}{12pt}\selectfont
\begin{gather*}
\frac{d}{dt} \|  ( x^\nu \alpha_p) * ( \beta^\mu u_p) \| \ge \delta''
p \| (x^\nu \alpha_p) * ( \beta^\mu u_p) \| - \| f \| \\ 
\ge \delta'' p \| (x^\nu \alpha_p) * ( \beta^{(\mu)} u_p) \| - cp
\sum\limits_{|\nu|+1\le|\rho\le m-1} || (x^\rho \alpha_p) *  ( \beta^\mu
u_p) \| \\ 
- c \sum\limits_{\substack{|\nu|+1\le|\rho|\le m-1 \\ |\mu'|=|\mu|+1}}
\| (x^\rho \alpha_p)   * ( \beta^{(\mu ')} u_p )\| -c 
\sum\limits_{|\mu'|=|\mu|+1} \| (x^\nu \alpha_p) * ( \beta^{\mu'}u_p)\|
- O \left( \frac{1}{p}\right)   
\end{gather*}}

Now\pageoriginale consider the functions $\theta_p(\nu, \mu) u_p$
defined by  
$$
\theta_p (\nu, \mu )u_p = p^{\theta(|\nu|-|\mu|)} (x^\nu)\alpha_p) *
(\beta^{(\mu)}u_p)  
$$
where $0<\theta<1$. In fact we take $\theta = \dfrac{1}{m}$. We
have from above inequality  
\begin{gather*}
\frac{d}{dt} || \theta_p (\nu, \mu ) u_p || \geq \delta '' p ||
\theta_p (\nu , \mu ) u_p || - cp^{1 - \theta}  \sum_{ |\nu| + 1 \leq
  |\rho| \leq m -1} || \theta_p (\rho , \mu )u_p || \\ 
  -  c p^\theta \sum_{ |\mu | = |\mu | + 1 } || \theta _p (\nu , \mu'
  ) u_p || - c \sum_{\substack{|\nu | + 1 |\leq 1|\rho| \leq m-1\\ |\mu'| =
      |\mu|+1}}  || \theta _p (\rho , \mu') u_p || \\
  - p^{\theta (|\nu | - | \mu |)} O (\frac{1}{p}).
  \tag{9.27}\label{chap3-eq9.27} 
\end{gather*}

Now if $|\mu'| = m + mh $ we have, by \eqref{chap3-eq9.25}
$$
|| \theta_p (\nu , \mu ) u_p || \leq c p^{\theta(|\nu | - | \mu |)}
|| u _p || \leq c' p^{ \theta(|\nu| - |\mu |) + h}.  
$$

But $\theta (|\nu | - |\mu |) \leq \theta (m - 1- m - mh - 1) = \theta 
(-mh - 2) = - h - 2 \theta$ since $\theta = \dfrac{1}{m}$. Thus $||
\theta_p (\nu, \mu )u_p || \leq c p^{-2 \theta }$. Denoting  
$$
S_p (t) = \sum_{\substack{0 \leq | \nu | \leq m - 1  \\ 0 \leq | \mu
    | \leq m + mh} } || \theta_p (\nu, \mu ) u_p (t) ||  
$$
we have from \eqref{chap3-eq9.27} that 
\begin{align*}
\frac{d}{dt} S_p (t) & \geq  \delta''  p S_p (t) - c p ^{1 -\theta }
S_p (t) - O (1) \\ 
& \geq \gamma'' p S_p (t) - O (1) \text{ for large  }  p, r'' > 0.  
\end{align*}

Integrating this with respect to $t$ 
\begin{align*}
S_p (t) & \geq \exp (\gamma'' pt ) S_p(0)  - \int\limits^t_0 \exp (r''
p(t - s)) O (1) ds  \\  
& = \exp (\gamma'' ~ pt )  \left[ S_p (0) -
  O\left(\frac{1}{p}\right)\right].  
\end{align*}\pageoriginale

But $S_p (0) = \sum\limits_{\substack{0 \leq |\nu| \leq m - 1 \\ 0 \leq | \mu
    | \leq m + mh}}  ||  \theta_p (\nu,  \mu ) u_p (0) || \geq ||
\alpha_p * (\beta u_p) (0) || \geq c > 0$ by \eqref{chap3-eq9.26} 
for large
$p$. Hence for every fixed $t$ the function $S_p (t)$ increases
exponentially with respect to $p$ i.e. $S_p (t) \geq c e^{\gamma''{\rm
    pt}}$. On the other hand   
$$
|| \theta_p (\nu , \mu ) u_p (t) || = p^{\theta (|\nu| - |\mu|)} ||
(x^\nu \alpha_p) * (\beta^\mu u_p)||  
$$
and $|| \beta^\mu u_p (t) || = 0 (p)^h$. Hence $S_p (t) \geq c p^k $ 
for a large $k$. In fact $||\theta_p (\nu, \mu ) u_p (t) || \leq
0(p^{h +1 })$ since $\theta | \nu | < 1 )$. This is a
contradiction. This completes the proof of the theorem
\ref{chap3-sec9-thm1}.   
