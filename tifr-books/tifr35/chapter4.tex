\chapter{}\label{chap4} %chapter 4

In\pageoriginale this chapter we briefly discuss the existence of
solutions of the Cauchy problem for parabolic equations.  

In section \ref{chap4-sec1} we introduce parabolic equations of order $m$ in the
$x$-variables and prove an existence theorem when  coefficinets do not
depend on $t$. In section \ref{chap4-sec2} we obtain an energy inequality for
parabolic equations which we use to prove the existence of solutions
of the Cauchy problem for parabolic equation with sufficiently smooth
initial conditions when coefficients depend on $t$ as well.  

\section{Parabolic equations}\label{chap4-sec1}% section 1

Consider the differential equation 
\begin{equation*}
\frac{\partial}{\partial t} u =  \sum_{|\nu|\leq 2m} a_\nu (x)
\left(\frac{\partial}{\partial x }\right)^\nu u + f = A \left(x ,
\frac{\partial}{\partial x}\right) u + f \tag{1.1}\label{chap4-eq1.1}  
\end{equation*}
where $A$ is negative elliptic of order $2m$ in $\bar{R}^n$ in the
sense that  
\begin{equation*}
\re \sum_{|\nu|=2m} a_\nu (x) (i \xi )^\nu \leq - \delta |\xi |^{2m}
\tag{1.2}  \label{chap4-eq1.2}
\end{equation*}
$\delta$ being a positive constant. We assume that the coefficients
$a_\nu$ belong to $\mathbb{B}^{2m}$.  

We prove the existence of a solution of \eqref{chap4-eq1.1} in the
space $L^2$. We 
take for the domain of definition $\mathscr{D}_A$ of $A$ the space
$\mathscr{D}^{ 2m }_{L^2}$.  

\setcounter{proposition}{0}
\begin{proposition}\label{chap4-sec1-prop1}% proposition 1 
For small $\lambda > 0 $ the operator $(I - \lambda A)$ defines a
one-to-one surjective mapping of $\mathscr{D}^{2m}_{ L^2}$ onto $L^2$.  
\end{proposition}

\begin{proof}
For\pageoriginale $u \in\mathscr{D}^{2m}_{L^2} $ and $\lambda>0$ 
\begin{equation*}
|| (I - \lambda A) u ||^2 = || u ||^2 - \lambda ((A+A^*) u, u) +
\lambda^2 || A u ||^2. \tag{1.3}\label{chap4-eq1.3} 
\end{equation*}

Since $A$ is negatively elliptic we have, from
G$\ring{\text{a}}$rding's lemma, that  
\begin{enumerate}[(i)]
\item $- (( A + A^* ) u, u ) \geq \delta || u ||^2_m - \gamma_1 || u
  || ^2 $ 

\item $||Au ||^2 \geq \dfrac{\delta^2 }{2} || \wedge^{2m} u ||^2 -
  \gamma_2 || u ||^2$.  
\end{enumerate}
\noindent
where $\gamma_1$, $\gamma_2$ are positive constants depending on
$\delta$. Hence it follows from \eqref{chap4-eq1.3} that  
\begin{equation}
|| (I - \lambda A) u ||^2 \geq (1 - \gamma_1 \lambda -\gamma_2
\lambda^2) ||u||^2 + \frac{\delta^2}{2} \lambda^2 || \wedge^{2m}
u||^2 , \tag{1.4} \label{chap4-eq1.4}
\end{equation}
which show that for sufficiently small $\lambda, (I - \lambda A)$ is
one-$t$-one from $\mathscr{D}^{2m}_{L^2}$ to $L^2$ and that the image
is closed.  

Next we show that the image $(I - \lambda A) \mathscr{D}^{2m}_{L^2}$
is dense in $L^2$, for $\lambda>0$ small. This is done by 
contradiction. Suppose the image is not dense in $L^2$. Then there
exists a $\psi \in L^2$, $\psi \neq 0$ such that  
$$
(( I - \lambda A) u,  \psi) = 0 \text{ for all } u \in 
\mathscr{D}^{2m}_{L^2}, 
$$
a fortiori for all $u \in \mathscr{D}$. This implies that   
\begin{equation}
(I - \lambda A^*) \psi =  0. \text{~~ Let~ } \psi_1 = (1 - 
  \Delta)^{-m}_\psi.\tag{1.6}\label{chap4-eq1.6} 
\end{equation}

Then $\psi_1 \in \mathscr{D}^{2m}_{L^2}$, $\psi_1 \neq 0$ and  
$$
(I - \lambda A^*) (1- \Delta)^m \psi _1 = 0  
$$

Hence\pageoriginale $((I - \lambda A^*) (1 - \Delta )^m \psi _1,
\psi_1) = || \psi_1 ||^2_m - \lambda (A^* (1 - \Delta)^m \psi_1,
\psi_1 ) = 0$. Now the 
real part of $(A^*(1-\Delta )^m \psi_1, \psi_1)$ is  
$$
\frac{1}{2}(\{A^*(1 - \Delta)^m + (1 - \Delta)^m A \} \psi_1,
\psi_1),  
$$
and since $\left\{A^* (1 - \Delta )^m + (1 - \Delta)^m A \right\}$ is
an elliptic operator of order $4m$, we have by
G$\ring{\text{a}}$rding's lemma,    
\begin{equation}
\frac{1}{2}( \{A^* (1 - \Delta )^m  + (1 - \Delta)^m A\}) \psi_1,
\psi_1) \leq - \frac{\delta}{2} || \wedge^{2m} \psi_1 ||^2 + \gamma_3
|| \psi_1 ||^2. \tag{1.7}  \label{chap4-eq1.7}
\end{equation}

Hence, we have 
\begin{align*}
\re & \{ || \psi_1 ||^2 _m - \lambda (A^* (1 - \Delta)^m  \psi_1,
\psi_1 ) \} \\ 
& \geq || \psi_1 ||^2_m + \lambda (\frac{\delta}{2} || \wedge^{2m}
\psi_1 ||^2 - \gamma_3 || \psi_1 ||^2 ) \\  
& \geq(1 - \lambda \gamma_3 ) || \psi_1 ||^2_m.
\tag{1.8}\label{chap4-eq1.8} 
\end{align*}

This implies that $\psi_1 = 0$ contrary to the assumption, which
proves that $(I - \lambda A)$ is surjective for sufficiently small $\lambda$.  
\end{proof}

\setcounter{corollary}{0}
\begin{corollary}\label{chap4-sec1-coro1}% corollary 1 
If $u \in L^2$ such that $A[u] \in L^2 $ then $u
  \in \mathscr{D}^{2m}_{L^2}$.  
\end{corollary}

\begin{proof}
Since from the Theorem for sufficiently small $\lambda, (I - \lambda
A)$ is surjective it follows that there exists $w
\in\mathscr{D}^{2m}_{L^2}$ such that $(I - \lambda A)  w= (I -
\lambda A )u$. Hence $(I - \lambda A) (w  - u) = 0$. Now in the
course of the proof of the theorem we have shown that $(I - \lambda A)
v =0$, $v \in L^2$ implies $v = 0$. Hence $u = w \in
\mathscr{D}^{2m}_{L^2}$. 
\end{proof}

\begin{proposition}\label{chap4-sec1-prop2}% proposition 2
Given\pageoriginale any initial data $u_0 \in
\mathscr{D}^{2m}_{L^2}$ and any 
second member $f \in \mathscr{D}^{2m}_{L^2} [ 0, h]$ then
there exists a solution $u \in \mathscr{D}^{2m}_{L^2} [ 0, h]$
of \eqref{chap4-eq1.1} such that $u(0) = u_0$  where the deriative
$\dfrac{\partial}{\partial t} u $ is taken in the sense of $L^2$.  
\end{proposition}

\begin{proof}
The prop. \ref{chap4-sec1-prop1} asserts that all the conditions of
Hille-Yosida theorem 
are satisfied taking $X = L^2$, $\mathscr{D}_A =
\mathscr{D}^{2m}_{L^2}$. Hence we have the proposition by the
application of Hille-Yosida theorem. Let us remark that $u, A u
\in L^2 [0,h]$ implies $u \in \mathscr{D}^{2m}_{L^2}
            [0, h]$.   

We have proved the Proposition~\ref{chap4-sec1-prop2} under the
assumption that $f 
\in \mathscr{D}^{2m}_{L^2} [ 0, h]$. We shall improve it by
proving it assuming only 
$$
f \in \mathscr{D}^{m}_{L^2} [ 0,
  h]. 
$$

For this purpose we establish an energy inequality for the
parabolic equation \eqref{chap4-eq1.1}. 
\end{proof}

\section{Energy inequality for parabolic equations}\label{chap4-sec2}% section 2

Consider the parabolic equation 
\begin{equation*}
\frac{\partial}{\partial t} u = \sum\limits_{|\nu|\leq 2m} a_\nu (x)
\left(\frac{\partial }{\partial x}\right)^\nu u+ f = A \left(x,
\frac{\partial }{\partial x}\right) u + f\tag{2.1} \label{chap4-eq2.1}
\end{equation*}

\setcounter{proposition}{0}
\begin{proposition}\label{chap4-sec2-prop1}% proposition 1
Let \eqref{chap4-eq2.1} be a parabolic equation with the coefficients
$a_\nu (x)$ of 
$A$ belonging to $\mathscr{B}^{2m}$  and the second member $f \in
\mathscr{D}^{2m}_{L^2} [0,h]$. If $u \in
\mathscr{D}^{3m}_{L^2} [0,h]$ satisfies \eqref{chap4-eq2.1} then 	   
\begin{equation*}
|| u (t)||^2_{2m} \leq \exp (\gamma_1 t) || u (0) ||^2_{2m} +
\gamma_2 (\delta) \int\limits^t_0 \exp (\gamma (t - s)) || f(s) ||^2_m
ds, \tag{2.2} \label{chap4-eq2.2} 
\end{equation*}
where $\gamma_1$, $\gamma_2$ are positive constants. 
\end{proposition}

\begin{proof}
Consider\pageoriginale
\begin{align*}
\frac{d}{dt} (u (t), u(t))_{2m} &= \left(\frac{d}{dt} u (t),u(t)\right)_{2m} 
+\left(u(t)\frac{d}{dt} u(t)\right)_{2m}\\ 
&=((A +A^*)u, u)_{2m} + 2 \re (f, u)_{2m} \\ 
&= (((1- \Delta)^{2m} A + A^*(1 - \Delta)^{2m}u, u) + 2 \re
(f,u)_{2m}. 	  
\end{align*}

The first term in the right hand side is by
G$\ring{\text{a}}$rdings's ineequality less than  
\begin{equation}
- \frac{\delta}{2} ||\Lambda^{3m} u ||^2 + \gamma_0 || u ||^2_{2m} \leq -
\frac{\delta}{2} || u ||^2_{3m} + \gamma_1 || u ||_{2m}^2
\tag{2.3} \label{chap4-eq2.3} 
\end{equation}
since $(1 - \Delta )^{2m} A $ is an elliptic operator of order
$6m$. Also  
$$
|(f, u)_{2m}| \leq || f ||_m || u ||_{3m} \leq \frac{2}{\delta} ||f
||^2_m + \frac{\delta}{2} || u ||^2_{3m} 
$$
by the inequality between the arithmetic and geometric means. Hence 
$$
\frac{d}{dt} (u, u)_{2m} \leq \left(\frac{\delta}{2} - \delta\right) || u
||^2_{3m} + \gamma_1 || u ||^2_{2m} + \frac{2}{\delta} || f ||^2_m  
$$
that is 
\begin{equation*}
\frac{d}{dt} || u(t) ||^2_{2m} \leq - \frac{\delta}{2} || u ||^2_{3m}
+ \gamma_1 || u ||^2_{2m} + \frac{2}{\delta} || f||^2_m
\tag{2.4} \label{chap4-eq2.4} 
\end{equation*}
a fortiori 
\begin{equation*}
\frac{d}{dt} || u(t) ||^2_{2m} \leq \gamma_1 || u(t) ||^2_{2m} +
\frac{4}{\delta} || f (t) ||^2_m \tag*{$(2.4)'$} 
\end{equation*}
and hence we obtain after integrating with respect to $t$ in $[0, h]$
the required inequality \eqref{chap4-eq2.2}.  
\end{proof}

Next\pageoriginale we obtain the energy inequality of the form
\eqref{chap4-eq2.2} under the 
assumption that $u \in \mathscr{D}^{2m}_{L^2}[0, h]$ instead
of $u \in \mathscr{D}^{3m}_{L^2}[0, h]$. In the case of
hyperbolic systems such an improvement could be achieved easily by
using Friedrichs' lemma. This method will not work in our case since
$A$ is not of the first order. However, as we shall show, by a slight
modification, we can use this method of regularisation by mollifiers.  

As before we estimate the commutators of convolutions with mollifiers
$\varphi_\varepsilon$ of Friedrichs. 

\setcounter{lemma}{0}
\begin{lemma}\label{chap4-sec2-lem1} %lemma 1
For $a \in \mathscr{B}^{2m}$ and $v \in L^2$ denote by
$C_\varepsilon v$ the commutator  
\begin{equation}
C_\varepsilon v = [\varphi^*_\varepsilon, a]
v. \tag{2.5}\label{chap4-eq2.5} 
\end{equation}

Then there exists a constant $\gamma_0$ such that for $|\nu| \leq m$ 
\begin{equation}
|| \left(\frac{\partial }{\partial x}\right)^\nu C_\varepsilon v|| \leq \gamma_0
| a |_{\mathscr{B}^{2m}} \bigg\{\sum\limits_{1\leq|\rho|\leq m} || (x^\rho
| \varphi_\varepsilon)* v ||_\nu + \varepsilon ||\,v\,||. \tag{2.6} \label{chap4-eq2.6}  
\end{equation}
\end{lemma}

\begin{proof}
We have, 
$$
C_\varepsilon v = \int [a(y)- a(x)]\varphi_\varepsilon (x-y) v(y) dy.  
$$

Developing $a(y) - a(x)$ by Taylor's theorem 
$$
a(y) - a(x) = \sum_{1 \leq |\rho | \leq m} \frac{(y - x)^\rho}{\rho !}
\left(\frac{\partial}{\partial x}\right)^\rho a (x) + \sum\limits_{|\rho|=m}
a_\rho (x, y) (y - x)^\rho,  
$$
where since $a \in \mathscr{B}^{2m}$
$$
|\left(\frac{\partial}{\partial x}\right)^{\nu'} a_\rho (x, y ) \bigg| \leq c_1 |
y - x | \; | a |_{\mathscr{B}^{2m}} \text{ for } |\nu'| \leq m-1 
$$
and 
$$
| \left(\frac{(\partial)}{\partial x}\right)^\nu  a_\rho (x, y) | \leq c_2 | 
a|_{\mathscr{B}^{2}}\text{ for  } |\nu | = m .  
$$

In fact,\pageoriginale 
\begin{gather*}
a_\rho (x, y) = \frac{m}{\rho !} \int\limits^1_0 (1 -
\theta)^{m-1}\left\{ a^{(\rho)} (x + \theta (y - x)) - a^{(\rho)} (x)
\right\} d \theta,\\ 
a^{(\rho)} (x) = \left(\frac{\partial}{\partial
  x}\right)^\rho a(x).   
\end{gather*}

Hence 
\begin{align*}
C_\varepsilon v & = \sum\limits_{1 \leq |\rho| \leq m} \frac{(-
  1)^{|\rho|}}{\rho !} \left(\frac{\partial}{\partial x}\right)^\rho a(x)
\left[(x^\rho \varphi_\varepsilon) * v \right]  \\
& \quad + \sum_{|\rho|=m} (-1)^m \int a_\rho (x, y)
(x - y)^\rho \varphi_\varepsilon (x- y) v(y) \;
dy. \tag{2.7} \label{chap4-eq2.7} 
\end{align*}
\eqref{chap4-eq2.7} implies the lemma. Obviously the terms of the
first sum on the 
right hand side contribute to the terms of the sum of the right hand
side of \eqref{chap4-eq2.6}. As far as the second sum is concerned we
remark that 
$\int |x| \big|\left(\dfrac{\partial }{\partial x}\right)^\nu (x^\rho
\varphi_\varepsilon) \big| dx = O (\varepsilon)$ for $ |\nu |\leq m$ and
$|\rho| = m$.  

By Hausdorff-Young inequality the second sum on the right hand side of
\eqref{chap4-eq2.7} is less than $O(\varepsilon) || v || $ and this
completes the proof of the lemma.  
\end{proof}

More generally we have the 

\begin{lemma}\label{chap4-sec2-lem2} %lemma 2 
Let $a \in \mathscr{B}^{2m}$ and $v \in L^2$. If  
\begin{equation*}
C^\nu_{\varepsilon} v = [(x^\nu \varphi_\varepsilon)*, a] v \text{
  for } | \nu | \leq m - 1 \tag{2.8}\label{chap4-eq2.8} 
\end{equation*}
then there exists a constant $\gamma _0 > 0$ such that 
\begin{equation*}
 ||C^\nu_{\varepsilon} v||_m \leq  \gamma_0 |a|_{\mathscr{B}^{2m}}
 \left(\sum_{|\nu|+1\leq|\rho|\leq m} || (x^\rho \varphi_\varepsilon) * v
 ||_m + \varepsilon || v ||\right).  \tag{2.9}\label{chap4-eq2.9}   
\end{equation*}
\end{lemma}

The proof is completely analogous to that of lemma 1 and hence we do
not repeat it here.  

As\pageoriginale a consequence of lemma \ref{chap4-sec2-lem1} and
\ref{chap4-sec2-lem2} we have  

\setcounter{corollary}{0}
\begin{corollary}\label{chap4-sec2-coro1}% corollary 1
If $A = \sum\limits_{|\nu|\leq 2m} a_\nu (x)
(\dfrac{\partial}{\partial x})^\nu $ is a differential operator of
order $2m$ with $a _\nu 
\in \mathscr{B}^{2m}$, then for any $u \in
\mathscr{D}^{2m}_{L^2}$ and for any $|\nu|\leq m$ 
\begin{equation*}
||[A,(x^\nu \varphi_\varepsilon)* ] u ||_m \leq c
\left(\sum_{|\nu|+1\leq\rho\leq m} || (x^\rho \varphi_\varepsilon) * u
||_{3m} + \varepsilon ||u ||_{2m}\right) \tag{2.10} \label{chap4-eq2.10}
\end{equation*}
 where $c = \gamma _0$, $\sup \limits_{\mu} | a_\mu (x)
 |_{\mathscr{B}^{2m}} \gamma _0 > 0$,  is a constant. We remark that
 \eqref{chap4-eq2.10} asserts also that, for any $|\gamma |\geq m$,  
$$
|| [ A, (x^\nu \varphi_\varepsilon) * ] u || \leq c\ \varepsilon || u
||_{2m}.  
$$
\end{corollary}

\begin{proposition}\label{chap4-sec2-prop2}% proposition 2
Let \eqref{chap4-eq2.1} be a parabolic equation of order $2m$ in $\Omega$ with
$a_\nu \in \mathscr{B}^{2m}$ and $ f \in
\mathscr{D}^{2m}_{L^2} [0, h]$. If $u \in
\mathscr{D}^{2m}_{L^2} [0, h]$ satisfies \eqref{chap4-eq2.1} then  
\begin{equation*}
|| u (t) ||^2 _{2m} \leq \exp (\gamma , t)|| u (0) ||^2_{2m} + c
\int\limits^t_0 \exp (\gamma (t - s)) || f (s) ||^2_{m} ds,  
\tag{2.11}\label{chap4-eq2.11}
\end{equation*}
\end{proposition}

\begin{proof}
Consider the function $(x^\nu \varphi_\varepsilon) *_{(x)} u =
u^\nu_{\varepsilon}$ for $0\leq|\nu|\leq m$. Clearly
$u^\nu_\varepsilon \in \mathscr{D}^{3m}_{L^2}  [0, h]$ and
satisfies the system  
\begin{equation*}
\frac{\partial}{\partial t} u^\nu_{\varepsilon} = Au^\nu_\varepsilon +
f^\nu_\varepsilon + [ (x^\nu \varphi_\varepsilon) *(x), A] u, \quad
0 \leq |\nu |\leq m.   \tag{2.12}\label{chap4-eq2.12}
\end{equation*}

Then  inequality \eqref{chap4-eq2.4} of Prop. \ref{chap4-sec2-prop1}
applied to this system gives the system of inequalities  
\begin{align*}
\frac{d}{dt} || u^\gamma_\varepsilon (t) ||^2_{2m} & \leq - \delta' ||
u^\nu_\varepsilon (t)||^2_{3m} + \gamma_1 || u^\nu_\varepsilon
(t)||^2_{2m} + \gamma_2 || 
f^\nu_\varepsilon (t) ||^2_m \\
& \quad  + \gamma_2 || [ (x^\nu
  \varphi_\varepsilon )* _{(x)}, A ] u ||^2_m \text{~ for~ } 0 \leq |
\nu | \leq m. \tag{2.13}\label{chap4-eq2.13} 
\end{align*}

From the corollary \ref{chap4-sec2-coro1} after lemma
\ref{chap4-sec2-lem2} applied to $[(x^\nu 
  \varphi_\varepsilon) *_{(x)}, A]$ we obtain for all\pageoriginale $0
\leq |\nu| \leq m$.   
\begin{align*}
|| [ (x^\nu \varphi_\varepsilon )*_{(x)}, A] u ||_m & \leq C_1
\Big(\sum_{|\nu|+1 \leq|\rho|\leq m} || (x^\rho \varphi_\varepsilon )
*_{(x)} u ||_{ 3m } +  \varepsilon || u ||_m\\ 
 & =C\left(\sum\limits_{|\nu|+1\leq|\rho|\leq m} || u^\rho_\varepsilon
||_{ 3 m} + \varepsilon || u ||_{2m}\right).\tag{2.14}\label{chap4-eq2.14} 
\end{align*}

We define $\nu^\nu_\varepsilon =
\varepsilon^{-\theta|\nu|}u_\varepsilon^\nu$ where $\theta > 0$ is
small constant. Multiplying \eqref{chap4-eq2.13}  by $\varepsilon^{-2
  \theta |\nu|}$ 
and setting $S_\varepsilon(t) = \sum\limits_\nu ||
v^\nu_\varepsilon(t)||^2_{2m}$ we have (after adding for $\nu$ over $0
\leq|\nu|\leq m$ from \eqref{chap4-eq2.14} 
\begin{align*}
\frac{d}{dt} (S_\varepsilon (t)) &\leq -\delta' \sum_\nu ||
v_\varepsilon^\nu (t) ||^2_{3m} + \gamma_1 S_\varepsilon (t) +
\gamma_2 F_\varepsilon (t) \\
&\quad + \gamma_2 \sum_\nu \varepsilon^{-2 \theta |\nu |} C_2
(\sum_{|\nu|+1\leq |\rho |\leq m} ||  
u^\rho_\varepsilon ||^2_{3m} + \varepsilon^2 || u
||^2_{2m}). \tag{2.15}  \label{chap4-eq2.15}
\end{align*}

But
\begin{align*}
\sum_{0\leq|\nu|\leq m} \varepsilon^{-2 \theta |\nu|}
\sum_{|\nu|+1\leq|\rho|\leq m} || u^\rho_\varepsilon||^2_{3m} & =
\sum_\nu \varepsilon^{-2\theta|\nu|} \sum_{|\nu|+1|\leq|\rho|\leq m}
\varepsilon^{2 \theta |\rho|} || v^\rho _\varepsilon||^2_{3m}\\
&  \leq n' \varepsilon^{2 \theta} \sum_\nu \sum_\rho
||v^\rho_\varepsilon||^2_{3m}. 
\end{align*}

Thus 
\begin{align*}
\frac{d}{dt} S_\varepsilon (t) \leq \gamma_1 S_{\epsilon}(t) + \gamma _2
F_\varepsilon(t) & + (\gamma_2 C_2 n' \varepsilon^{2 \theta} - \delta' )
\sum_{0 \leq |\nu | \leq m} || v^\nu_\varepsilon ||^2_{3m}\\ 
&	+ c\ \varepsilon^{2 (1-m \theta )} || u(t) ||^2_{2m} 
\end{align*}

For small $\varepsilon > 0$, $(\gamma_2 C_1  n' \varepsilon^{2
   \theta} - \delta') < 0 $ and hence   
 $$
\frac{d}{dt} S_\varepsilon(t) \leq \gamma_1 S_\varepsilon(t) + 
\gamma_2 F_\varepsilon(t) + O (\varepsilon^{2 (1- m \theta)}), 
$$

Integrating with respect to $t$
\begin{equation*}
S_\varepsilon (t) \leq \exp (\gamma _1 t) S_\varepsilon (0) + \gamma_2
\int\limits^t_0 \exp (\gamma_1 (t-s)) \Big\{F_\varepsilon (s) +
O(\varepsilon^{2(1-m \theta )})\Big\} ds \tag{2.16} \label{chap4-eq2.16}
\end{equation*}\pageoriginale

But
\begin{align*}
 || v^\rho_\varepsilon||^2_{2m} & = ||
u^\rho_\varepsilon||^2_{2m} \varepsilon^{-2 \theta | \rho |} \\ 
& = (\widehat{x^\rho \varphi_\varepsilon}) \hat{u}(\xi)  (1 + |\xi
|)^{2m} ||^2 \varepsilon^{-2 \theta | \rho |} 
\end{align*}
by Plancherel's formular where $\hat{g}$ denotes the Fourier image of
$g$ in the $x$-space and 
 \begin{align*}
 (\widehat{x^\rho \varphi_\varepsilon}) (\xi ) & = \int x^\rho
   \varphi_\varepsilon e^{-2 \pi ix . \xi_{dx}} \\ 
 & = \varepsilon^{|\rho|} \int x^\rho \varphi (x) e^{-2 \pi i
     \in x. \xi} dx 
 \end{align*}

Since $\varphi$ has its support in $| x |< 1$. We have  
$$
|(\widehat{x^\rho \varphi_\varepsilon}) \; (\xi )| \leq
\varepsilon^{|\rho|}\int\varphi (x) \; dx = \varepsilon^{|\rho |} 
$$

Hence 
$$
|| v^\rho_{\varepsilon} ||^2_{2m} \leq \varepsilon^{2|\rho| (1 - \theta )} || u 
||^2_{2m} 
$$
and  
$$
\sum_{0\leq|\rho|\leq m} || v^\rho_\varepsilon||^2_{2m} \leq || u
||^2_{2m} \sum_{0\leq|\rho|\leq m} \varepsilon^{2 |\rho| (1 -
  \theta)}  \leq || u ||^2_{2m} (1 + c \varepsilon^{2 (1 - \theta)}) 
$$
which tends to $|| u ||^2_{2m}$ as $ \varepsilon \to 0$. Hence
$S_\varepsilon(t) \to || u (t)||^2_{2m}$ as $\varepsilon \to 0$. Also
$F_{\varepsilon}(t) \to || f(t) ||^2_m$. Hence on taking limits as
$\varepsilon \to 0$ 
we have  
$$
|| u(t) ||^2_{2m} \leq \exp (\gamma_1 t) || u (0) ||^2_{2m} + \gamma_2
\int\limits^t_0 \exp (\gamma_1 (t- s)) || f (s) ||^2_m ds.  
$$

This\pageoriginale completes the proof of proposition.
\end{proof}

Finally we consider the case parabolic systems in which the
coefficients are functions of $(x, t)$ in $\Omega$. Let  
\begin{equation*}
\frac{\partial}{\partial t} u - \sum_{|\nu |\leq 2 m} a_\nu (x, t)
(\frac{\partial}{\partial x})^\nu u = f \tag{2.17}\label{chap4-eq2.17} 
\end{equation*}
be a parabolic equation of order $2m$. That is we assume that 
$$
A = \sum_{|\nu|\leq 2m} a_\nu (x, t) (\frac{\partial }{\partial
  x})^\nu  
$$
is uniformaly negatively elliptic in $\Omega\ (\Omega = \{ (x, t)| x
\in \underbar{R}^n , 0 \leq t \leq h)$. This means that  
$$
\re \sum_{|\nu|=2m} a_\nu (x, t) (i \xi)^\nu \leq - \delta |\xi|^{2m}  
$$
for all $(x, t) \in \Omega$, $\xi \in \underbar{R}^n$,
$\delta > 0$.  

\begin{proposition}\label{chap4-sec2-prop3} %proposition 3
Let \eqref{chap4-eq2.17} be a parabolic system in $\Omega$ with $a_\nu
\in\mathscr{B}^{2m}\break [0, h]$ and $f \in
\mathscr{D}^{m}_{L^2} [0, h]$. Then, given a $u_0 \in  
\mathscr{D}^{2m}_{L^2}$  there exists $u \in
\mathscr{D}^{2m}_{L^2} [0, h]$ satisfying \eqref{chap4-eq2.17}, with
$u \Big|_{t=0} 
=u_0$, and which satisfies the energey inequality \eqref{chap4-eq2.11}. 
\end{proposition}

\begin{proof}
Let $0 = t_0 < t_1 \cdots < t_k = h$ be a subdivision of $[0,h]$ of
equal length. We define $u_1(t), \ldots, u_k(t)$ in $[t_0, t_1],
\ldots, [t_{k-1}, t_k]$ by the following conditions  
\begin{align*}
\frac{du_1}{dt} & = A(t_0) u_1 + f, \; u_1 (t_0) = u_0 \quad \text{for}
\quad t_0 \leq t \leq t_1 \\ 
\frac{du_2}{dt} & = A(t_1) u_2 + f, \; u_k (t_{1}) = u_{1} (t_{1})
\quad \text{for} \quad  t_1 \leq t \leq t_2 \\ 
\frac{du_k}{dt} & = A(t_{k-1}) u_k + f, \; u_2 (t_{1}) = u_{1}
(t_{1})  \quad \text{for} \quad t_{k-1} \leq t \leq t_k.  
\end{align*}

We\pageoriginale denote by $u^{(k)} (t)$ the function which in
$t_{j-1} \leq t \leq t_j$ is equal to $u_j (t)$. It is easy to see that
$\{ u^{(k)} (t)\}$ is a uniformly bounded set. More precisely it is a
bounded set in the Hilbert space $\mathscr{E}^{2m, 1}_{L^2}(\Omega)$,
consisting of all the functions $u \in L^2$ such that
$$
\frac{\partial u}{\partial t} \in L^2,
\left(\frac{\partial}{\partial x}\right)^\nu u \in L^2  \text{~ for~ }
|\nu| \leq 2m,    
$$
where the derivatives are taken in the sense of
distributions. $\mathscr{E}^{2m, 1}_{L^2} (\Omega)$ provided with the
scalar product  
$$
(u, v)_{\mathscr{E}^{ 2m, 1}_{L^2}(\Omega)} = (u, v)_{L^2 (\Omega)} +
\left(\frac{\partial u}{\partial t}, \frac{\partial v}{\partial
  t}\right)_{L^2 (\Omega)} + \sum_{|\nu| \leq 2m}
\left(\left(\frac{\partial}{\partial x}\right)^\nu u,
\left(\frac{\partial}{\partial x}\right) v\right)_{L^2 (\Omega)}  
$$
is a Hilbert space. Hence $\left\{u^{(k)} (t) \right\}$ has a weak
limit in  $\mathscr{E}^{2m, 1}_{L^2}(\Omega)$, say $u(x, t) \cdot u(x,
t)$ satisfies the equation  
\begin{equation*}
\frac{\partial u}{\partial t} = Au + f \tag{2.18}\label{chap4-eq2.18}
\end{equation*}
in the sense of distributions. We shall now show that $u \in
\mathscr{D}^{2m}_{L^2} [0, h]$. We know that $u \in L^2 [0, h
]$. If $\varphi_\varepsilon$ be mollifiers of Friedrichs consider the
equation  
$$
\frac{\partial}{\partial t}((x^\nu \varphi_\varepsilon) *_{(x)}
u) =  A ((x^\nu \varphi_\varepsilon)) *_{(x)}u ) + (x^\nu
\varphi_\varepsilon)  *_{(x)} f + \left[ (x^\nu \varphi_\varepsilon)
  *_{(x)}, A\right]u   
$$
for $|\nu |\leq m$. The functions $u^\nu_\varepsilon = (x^\nu
\varphi_\in)*_{(x)} u$ form a Cauchy sequence as $\varepsilon \to
0$. This can be proved by an argument similar to the one in
Prop. \ref{chap4-sec2-prop2}. It  can\pageoriginale also be shown that  
\begin{align*}
u^\nu_\varepsilon & \to u(t) \text{~ in~ } \mathscr{D}^{2m}_{L^2}
\text{~ for~ } \nu = 0, \\ 
&  \to 0 \text{~ in~ } \mathscr{D}^{2m}_{L^2} \text{~~ otherwise } 
\end{align*}
uniformly in $t$. This proves that the energy inequality \eqref{chap4-eq2.11}
holds in this case also.  

Recent work by P. Sobolevskii develops the semi-group theory for the
equations of the parabolic type by using fractional powers. Equations
of parabolic type in Banach space, Trudy Moscov Mat, Obsc. 10(1961),
297 - 350. 
 \end{proof}
