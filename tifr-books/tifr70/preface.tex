\thispagestyle{empty}
\begin{center}
{\Large\bf Lectures on}\\[5pt]
{\Large\bf Partial Differential Equations}
\vskip 1cm

{\bf By}\\[5pt]
{\large\bf G.B. Folland}
\vfill

{\bf Tata Institute of Fundamental Research}

{\bf Bombay}

{\bf 1983}
\end{center}

\eject

\thispagestyle{empty}
\begin{center}
{\Large\bf Lectures on}\\[5pt]
{\Large\bf Partial Differential Equations}
\vskip 1cm

{\bf By}\\[5pt]
{\large\bf G.B. Folland}\\[5pt]
{Lectures delivered at the}\\[5pt]
{\bf Indian Institute of Science, Bangalore}\\[15pt]
{under the}\\[5pt]
{\bf T.I.F.R.  -- I.I.Sc. Programme in Applications of}\\[5pt]
{\bf Mathematics}
\vfill

{\bf Notes by}\\[5pt]
{\large\bf K.T. Joseph and S. Thangavelu}
\vfill

{Published for the}\\
{\large\bf Tata Institute of Fundamental Research Bombay}\\
{\large\bf Springer-Verlag}\\
{Berlin Heidelberg New York}\\
{\large\bf 1983}
\end{center}

\eject


\thispagestyle{empty}
\begin{center}
{\bf Author}\\[10pt]
{\large\bf G.B. Folland}\\[5pt]
{University of Washington}\\[5pt]
{Seattle, Washington 98175}\\[5pt]
{\bf U.S.A.}

\vfill
{\bf\copyright Tata Institute of Fundamental Research, 1983}
\vfill

\rule{\textwidth}{.5pt}

ISBN 3-540-12280-X Springer-Verlag, Berlin, Heidelberg. New York

ISBN 0-387-12280-X Springer-Verlag, New York. Heidelberg. Berlin

\rule{\textwidth}{.5pt}
\vfill

\parbox{0.7\textwidth}{
No part of this book may be reproduced in any form by print, microfilm
or any other means without written permission from the Tata Institute
of Fundamental Research, Colaba, Bombay 400 005}
\vfill

Printed by N.S. Ray at The Book Center Limited,

Sion East, Bombay 400 022 and published by H. Goetze,

Springer-Verlag, Heidelberg, West Germany

\vskip 1cm

{\large\bf Printed in India}
\end{center}

\eject



\chapter*{Preface}


\addcontentsline{toc}{chapter}{Preface}

	This book consists of the notes for a course I gave at the
        T.I.F.R. Center in Bangalore from September 20 to November
        20, 1981. The purpose of the course was to introduce the
        students in the Programme in Application of Mathematics to the
        applications of Fourier analysis-by which I mean the study
        of convolution operators as well as the Fourier transform
        itself-to partial differential equations. Faced with the
        problem of covering a reasonably broad spectrum of material in
        such a short time, I had to be selective in the choice of
        topics. I could not develop any one subject in a really
        thorough manner; rather, my aim was to present the essential
        features of some techniques that are well worth knowing and to
        derive a few interesting results which are illustrative of
        these techniques. This does not mean that I have dealt only
        with general 
        machinery; indeed, the emphasis in Chapter $2$ is on very
        concrete calculation with distributions and Fourier
        transforms-because the methods of performing such calculations
        are also well worth knowing. 

	If these notes suffer from the defect of incompleteness, they
        posses the corresponding virtue of brevity. They may therefore
        be of value to the reader who wishes to be introduced to some
        useful ideas without having to plough through a systematic
        treatise. More detailed accounts of the subjects discussed here
        can be found in the books of Folland [1], Stein [2],
        Taylor [3], and Treves [4]. 

No specific knowledge of partial differential equations or Fourier
Analysis is presupposed in these notes, although some prior
acquittance with the former is desirable. The main prerequisite is a
familiarity with the subjects usually gathered under the
rubic ``real analysis'': measure and integration, and the elements of point set
topology and functional analysis. In addition, the reader is expected
to be acquainted with the basic facts about distributions as presented,
for example, in Rudin [7]. 

I wish to express my gratitude to professor K.G. Ramanathan for
inviting me to Bangalore, and to professor S.Raghavan and the staff of
the T.I.F.R. Center for making my visit there a most enjoyable
one. I also wish to thank Mr S. Thangavelu and Mr K.T. Joseph for
their painstaking job of writing up the notes. 


