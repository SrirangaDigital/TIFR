
\chapter{An Application to Complete Intersections}\label{chap5}

AS\pageoriginale AN APPLICATION of the results we have proved till
now, we prove a result on complete intersecions:
\begin{THM*}
Let $X$ be a smooth closed subscheme of $\mathbb{P}_k^n$ of
codimension 2. Let us further assume that chark $=0$ and $n\geq
2d^\circ(X)$. Then $X$ is a complete intersection.
\end{THM*}

\begin{proof}
If $n\leq 5$ then $d^\circ X=1$ or 2. So we see immediately that $X$
is a complete intersection. So we can assume $n\geq 6$. Then by
Barth's theorem, $\Pic X=\mathbb{Z}.O_X(1)$. Since $X$ is a local
complete intersection, ($X$ is smooth) $\omega_X$ is a line bundle and
hence $\omega_X=O_X(r)$ for some integer $r$. By Serre's lemma
[Chapter \ref{chap4}] $X$ is the scheme of zeros of a section of a
rank two vector bundle on $\mathbb{P}^n$. So we see that the theorem
is essentially a criterion for decomposability of a rank two vector
bundle on $\mathbb{P}^n$. 
\begin{itemize}
\item [i)] We will first prove that the integer $r$ above is
  negative. By Bertini's theorem we can cut $X$ by a linear subspace
  of $\mathbb{P}^n$ of codimension $(n-3)$ to get a smooth curve $C$
  in $\mathbb{P}^3$. Then 
$$
\omega_C=O_C(r+n-3).
$$

Since $H^\circ(C,O_C)^v=H^1(C,\omega_C)=H^1(C,O_C(r+n-3))$, we see
that 
$$
H^1(C,O_C(r+n-3))\neq 0.
$$

By Castelnuovo's theorem,
$$
r+n-3\leq d-3.
$$

Since $n\geq 2d$, we see that $r$ is negative.
\item [ii)] Applying Kodaira's vanishing theorem, we see:
$$
H^i(X,O_X(m))=0,m<0,0\leq i<\dim X=n-2.
$$\pageoriginale

By duality we get,
$$
H^i(X,O_X(m))=H^{n-2-i}(X,O_X(r-m)).
$$

Since $r$ is negative we see that,
$$
H^i(XO_X(m))=0
$$
for every $m$ and $0< i<\dim X$.
\item [iii)] Let $\Spec A$ be the cone of $X$ (in the given embedding
  in $\mathbb{P}^n$.) Let 
$$
\bar{A}=\underset{m}{\oplus} H^\circ(X,O_X(m)).
$$

We will show that $\bar{A}$ is a Gorenstein ring. By (ii) we see that
$\bar{A}$ is Cohen-Macaulay. Since 
$$
\omega_{\bar{A}}=\underset{m}{\oplus}H^\circ(X\omega_X(m))=\oplus
H^\circ (XO_X(m+r))=\bar{A}(r),
$$
$\bar{A}$ is Gorenstein.
\item [iv)] Let $\Spec R=\Spec k[X_\circ,X_1,\ldots X_n]$ be the cone
  over $\mathbb{P}^n$. So we have a surjection $R\longrightarrow A$
  and $\bar{A}$ is a finite $A$-module and hence a finite
  $R$-module. Let $s=$ minimum number of generators of $\bar{A}$ over
  $R$. Since $\bar{A}$ is a Cohen-Macaulay module of dimension $n-1$,
  we have a resolution, 
$$
0\longrightarrow R^m\longrightarrow R^t\longrightarrow
R^s\longrightarrow\bar{A}\longrightarrow 0.
$$

Since $\bar{A}$ is Gorenstein, $s=m$. So the minimal resolution for
$\bar{A}$ is,
$$
0\longrightarrow R^s\longrightarrow R^{2s}\longrightarrow R^s
\longrightarrow \bar{A}\longrightarrow 0.
$$

\item [v)] We claim that $s\leq d-2. d=d^\circ X$.

We can choose $X_i$'s above in such a manner that $\bar{A}$ is
integral over,
$$
R'=k[X_\circ,\ldots,X_{n-2}]
$$\pageoriginale
the inclusion $R'\longrightarrow\bar{A}$ is a graded
homomorphism. Since $\bar{A}$ is Cohen-Macaulay we see that $\bar{A}$
is locally free over $R'$. But since $\bar{A}$ is a graded $R'$-module
it is actually free and $rk_{R'}\bar{A}=d$.

Since $X$ can be assumed to be not contained in any hyperplane, we see
that the images of $X_{n-1}$ and $X_n$ in $\bar{A}$ form a part of a
minimal set of generators of $\bar{A}$ over $R'$. So we see that the
$R'$ module got as the cokernel of, $R\longrightarrow\bar{A}$, is
generated by atmost $d-3$ elements. \ie $\bar{A}$ is generated over
$R$ by at most $d-2$ elements. So 
$$
s\leq d-2.
$$

\item [vi)] Now we claim that $s=1$. 

Since outside the closed point (vertex) of $R, A$ and $\bar{A}$ are
isomorphic, we have that the $(s-1) \times (s-1)$ minors of the matrix
defining the map $R^{2s}\longrightarrow R^s$, defines the vertex of
$R$ as its set of zeroes. So the codimension of the scheme defined by
the $(s-1) \times (s-1)$ minors of this $s  \times 2s$ matrix in $R$ is $n+1$. But
from general principles, codimension of the variety defined by the
$z \times z$ minors of an $r \times t$ matrix is less than or equal to
$(r-z+1)(t-z+1)$. So here, 
$$
n+1\leq (2s-(s-1)+1)(s-(s-1)+1)=(s+2).2.
$$

Since $s\leq d-2$, we see that, $n+1\leq 2d$ which is a contradication
to our hypothesis. Therefore $s=1$. 

So we see that the composite map $R\longrightarrow A\longrightarrow
\bar{A}$ is surjective. \ie $A=\bar{A}$. So $A$ is Gorenstein and
codimension two in $R$ and hence a complete intersection. 
\end{itemize}
\end{proof}



%\addcontentsline{toc}{section}{Bibliography}
\begin{thebibliography}{99}\pageoriginale
\bibitem{key1} ABHYANKAR, S.S., Algebraic Space Curves, University of
  Montreal, Lecture Notes.
\bibitem{key2} BARTH, W., Transplanting cohomology classes in complex
  projective space, Amer. J. Math., 92 (1970), 951--967.
\bibitem{key3} FOSSUM, R., et al: The Homological Algebra of Trivial
  Extensions of Abelian Categories with application to Ring
  Theory. Preprint series 72/73 -Aarhus Universitet.
\bibitem{key4} GAETA, F.: Quelques progres recents dans Ia
  classification des varieties algebriques dun espace projectif
  Deuxieme Collogue de Geometrie Algebrique Liege. C.B.R.M., 145--181
  (1952).
\bibitem{key5} HALPHEN, G.-H.: Memoire sur la classification des
  courbes gauches algebriques, J. Ec. Polyt. 52 (1882), 1--200.
\bibitem{key6} HORROcKS, G: Vector bundles on a punctured spectrum,
  Proc. Lond. Math. Soc. 14 (1964), 689--713.
\bibitem{key7} KODAIRA, K.,: On a differential-geometric method in the
  theory of analytic stacks, Proc. Nat. Acad. Sci. USA 39 (1953),
  1268--1273.
\bibitem{key8} MUMFORD, D.: Pathologies III, Amer. J. Math. 89 (1967),
  94--104.
\bibitem{key9} MURTHY, M.P., and TOWBER, J.: Algebraic vector bundles
  over $\mathbb{A}^3$ are trivial, Inventiones Math. 24, 173--189
  (1974).
\bibitem{key10} NOETHER, M.: Zur Grundlegung der Theorie der
  Algebracschen Raumeurven, Verlag der komgliehen Akademie der
  Wissensehaften, Berlin (1883).
\bibitem{key11} PESKINE,\pageoriginale C. and L. SZPIRO: Liaison des
  varietes algebriques I, Inventiones Math., 26, 271--302, (1974).
\bibitem{key12} QUILLEN, D.: Projective Modules over Polynomial Rings
  Inventiones Math., 36, 166--172 (1976).
\bibitem{key13} RAO, A. PRABHAKAR: Liaison among curves in Projective
  3-space, Ph.d. Thesis, University of California, Berkeley.
\bibitem{key14} SAINT-DONAT, B.,: On Petri's analysis of the linear
  system of quadrics through a canonical curve. Math. Ann. 206 (1973),
  157--175.
\bibitem{key15} SERRE J-P.: Sur la topologic des varietes algebriques
  en caracteristique p, Symposium Int. de Topologie Algebraica, Mexico
  (1958), 24--53.
\end{thebibliography}
