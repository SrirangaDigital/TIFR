\chapter{Introduction}

\markboth{Introduction}{Introduction}

THESE NOTES ARE the outcome of a series of lectures I gave 
in the winter of 1975--76 at the Tata Institute of Fundamental Research,
Bombay. The object of the research, we --D. FERRAND, L. GRUSON,
C. PESKINE and I--started in Paris was, roughly speaking {\it to find
out the equations defining a curve in projective 3-space} (or in
affine 3-space or of varieties of codimension two in projective
n-space.) I took the opportunity given to me by the Mathematics
Department of T.I.F.R, to try to put coherently the progress made by
the four of us since our paper \cite{key11}. Even though we are scattered over
the earth now, (RENNES, LILLE, OSLO and BOMBAY:) these notes should be
considered as the result of common of all of us. I have tried in the
quick description of the chapters to obey the ``Redde Caesari quae
sunt Caesaris.'' 

Chapter \ref{chap1} contains certain prerequisites like duality,
depth, divisors etc. and the following two interesting facts:
\begin{itemize}
\item [i)] An example of a reduced curve in $\mathbb{P}^3$ with no
{\it imbedded} smooth deformation (an improvement on the counter
example ``6.4'' in \cite{key11} which was shown to me by G. Ellingrud
from Oslo who also informed me that it can be found in
M. Noether \cite{key10}). 
\item [ii)] A proof that every locally complete intersection curve in
$\mathbb{P}^3$ can be defined by four equations.

Chapter \ref{chap2} is my personal version of the theory of conductor
for a curve. A long time ago, O. Zariski asked me what my
understanding of Gorenstein's theorem was and this chapter is my
answer; even though it contains no valuations and I wonder if it will
be to the taste of Zariski. In it I first recall classical
facts known since Kodaira, through duality. The three
main points are as follows:

If $X$ is a smooth surface, projective over a field k, C, a reduced
irreducible curve on $X, \overline{X}\xrightarrow{g} X$, a finite
composition of dilations, such that the proper transform
$\overline{C}$ of $C$ on $\overline{X}$ is smooth, one has:
\begin{itemize}
\item [a)] the conductor $\underline{f}$ is related to dualizing
sheaves by 
$$
\underline{f}=g_*\omega_{\overline{C}}\otimes\omega_C^{-1}
$$
\item [b)] Gorenstein's theorem is a simple consequence of
$H^1(\mathbb{P}^2,\break O_{\mathbb{P}^2})=0$. 

\item [c)] Regularity of the adjoint system is equivalent to
$H^1(\overline{X},\break O_{\overline{X}}(\overline{C}))=0$.



We conclude the chapter with a counter-example which is new in the
litterature:
\item [d)] A curve $C$ on a surface $X$ over a field of characteristic
$p\geq 5$, such that 
\begin{itemize}
\item [i)] $O_X(C)$ is ample.
\item [ii)] Kodaira vanishing theorem holds \ie $H^1(X, O_X(-C))=0$. 
\item [iii)] Regularity of the adjoint does not hold.
$$
\text{\ie}\hspace{3cm} H^1(\overline{X}, O_{\overline{X}}(-\overline{C}))\neq
0.
$$

We also give the proof - shown to us by Mumford - that such a
situation cannot occur in zero characteristic; \ie $H^1(X,
O_X(-C))\simeq H^1(\overline{X}, O_{\overline{X}}(-\overline{C}))$
over characteristic zero fields. 

Chapter \ref{chap3} contains two classical theorems by
Castelnuovo. These theorems have been dug out of the litterature by
L. Gruson. My only effort was to write them down (with Mohan
Kumar). The point, in modern language, is to give bounds for Serre's
vanishing theorems in cohomology, in terms of the degree of the
given curve in $\mathbb{P}^3$. The two results are the following:

If $C$ is a smooth curve in $\mathbb{P}^3$, $J$ its sheaf of ideals
and $d$ its degree, then 
\begin{itemize}
\item [a)] $H^2(\mathbb{P}^3, J(n))=0\qquad n\geq\frac{d-1}{2}$
\item [b)] $H^1(\mathbb{P}^3, J(n))=0\qquad n\geq d-2$
\end{itemize}
\end{itemize}
\end{itemize}
\end{itemize}

The reader who is interested in equations defining a curve canonically
embedded may read the version of Saint-Donat \cite{key14} of Petri's
theorem, in which coupling the above results with some geometric
arguments, he gets the complete list of equations of such a curve. (In
general they are of degree 2, but here we only get that the degree is
less than or equal to three.)

In Chapter \ref{chap4} we give an answer to an old question of
Kronecker (and Severi): a local complete intersection curve in affine
three space is set theoretically the intersection of two (algebraic)
surfaces. We also give the projective version of D. Ferrand: a local
complete intersection curve in $\mathbb{P}^3$ is set-theoretically the
set of zeroes of a section of a rank two vector bundle. Unfortunately
such vector bundles may not be decomposable. The main idea - which is
already in \cite{key11}, example 2.2 - is that if a curve $C$ is
``liee'' to itself by a complete intersection, then the ideal sheaf of
the curve $C$ in $O_X$ is - upto a twist - the dualising sheaf
$\omega_C$ of $C$ (\cite{key11}, Remarque 1.5). Starting from that, we
construct an extension of $O_C$ by $\omega_C$, with square of
$\omega_C$ zero, and then a globalisation of a theorem of
R. Fossum \cite{key3} finishes the proof. The globalisation is harder
in the case of D. Ferrand. It must be said that the final conclusion
in $\mathbb{A}^3$ has been made possible by Murthy-Towker \cite{key9}
(and now Quillen-Suslin \cite{key12}) theorem on triviality of vector
bundles on $\mathbb{A}^3$. Going back to rank two-vector bundles on
$\mathbb{P}^3$ we have now three ways of constructing them:

\begin{itemize}
\item Horrock's
\item Ferrand's
\item and by projection of a canonical curve in $\mathbb{P}^3$
\end{itemize}

It will be interesting to know the relations between these
families. We take this opportunity to ask the following question: Can
one generalise Gaeta's theorem (for \eg \cite{key11} Theorem 3.2) in
the following way:

Is every smooth curve in $\mathbb{P}^3$ liee by a finite number of
``liaisons'' to a scheme of zeroes of a section of a rank two-vector
bundle?\footnote{these questions have now been answered by A.P. Rao
(the first negatively) in his paper: ``Liaisons among curves in
$\mathbb{P}^3$'' Inventiones Math. 1978.}

Or as R. Hartshorne has suggested: ``What are the equivalence
{\small classes} of curves in $\mathbb{P}^3$, modulo the equivalence relation
given by ``liaison''. A start in this direction has been taken by his
student. A. Prabhakar Rao (Liaison among curves in Projective 3-space,
Ph.D. Thesis \cite{key13}).\footnotemark[\thefootnote]

I have news from Oslo, saying that L. Gruson and C. Peskine are
starting to understand the mysterious chapter \ref{chap3} of Halphen's
paper \cite{key5}. I hope they will publish their results soon. These
works and the yet unpublished notes of D. Ferrand on self-liaison
would be a good piece of knowledge on curves in 3-space.

N. Mohan Kumar has written these notes and it is a pleasure for me to
thank him for his efficiency, his remarks and his talent to convert
the ``franglais'' I used during the course to ``good English''. The
reader should consider all the ``gallissisms'' as mine and the
``indianisms'' him. It has been a real pleasure for me to work with
him and to drink beer with him in Bombay - a city which
goes far beyond all that I had expected, in good and in
bad. I thank the many people there who gave me the opportunity of
living in India and also made my stay enjoyable - R. Sridharan,
M.S. Narasimhan, R.C. Cowsik, S. Ramanan and surprisingly Okamoto from
Hiroshima University. The typists of the School of Mathematics have
typed these manuscripts with care and I thank them very much. I also
thank Mathieu for correcting the orthographic mistakes and Rosalie -
Lecan for the documentation she helped me with.

