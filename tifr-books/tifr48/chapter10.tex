
\chapter{Quasi-conformal Mappings}\label{chap10}

\begin{defi*}
  M\"obius\pageoriginale $n$-space is the one point compactification of eucli\-dean
  $n$-space $\mathbb{R}^n$, it will be denoted by $\mathbb{R}^n \cup
  \{ \infty\}$.
\end{defi*}

$GM(n)$ the M\"obius group of M\"obius $n$-space is the group of
transformations generated by ``inversion'' in the sphere $S^n$
$$
\eta_1^2 + \eta_2^2 + \cdots + \eta_{n+1}^2=1.
$$

If we set $\eta_i = \frac{y_i}{y_i}(i=1, \ldots , n+1)$ then $S^n$ is
realized as the projective variety $y^2_\circ- y^2_1-----y^2_{n+1}=0$,
and one can prove that $GM(n)$ becomes identified with $0(1, n+1)/\pm
1 (\cite{17}p. 57)$.

\begin{thm} \label{chap10:thm10.1}
  The subgroup $G'$ of $GM(n)$ which stabilizes the hemisphere
  $S_-(\eta_{i+1}< 0)$ is isomorphic to $GM(n-1)$ under the
  restriction homomorphism into its action on the equatorial $n-1$
  sphere $\eta_{n+1}=0$. Moreover $G'$ operates transitively on $S_-$
  and keeps invariant a positive definite quadratic differential form
  $dS^2$. Under stereographic projection from $(0, 0, \ldots, 0, 1)$,
  $S_-$ maps onto the unit ball $x_1^2 + \cdots + x^2_n < 1$ and its
  invariant metric $dS^2$ upto a constant factor becomes
  $\frac{dx^2}{1-|x|^2}$, where $dx$ is usual euclidean metric. \hfill
  (loc. cit. pp. 58-59) 
\end{thm}

The unit ball $|x|<1$ with metric $\frac{dx^2}{1- |x|^2}$ where $dx$
is euclidean metric, is a Riemannian space called the hyperbolic
$n$-space, the isotropy subgroup at a point is $0(n)$ (In this
realization of hyperbolic space, the isometries of hyperbolic metric
preserve euclidean angles).

Hence\pageoriginale the spaces have constant curvature.

We introduce following notations:

Let $V$, $W$ be two Riemannian spaces and let $\varphi: V \to W$ be a
homeomorphism. 

Let
\begin{align*}
  L_\varphi (p, r) & = \sup\limits_{d (p, q)=r} d (\varphi (p),
  \varphi(q))\\
  l_\varphi (p, r) & = \inf\limits_{d (p, q)=r} d (\varphi (p),
  \varphi(q))\\
  H_\varphi(p) & = \varlimsup\limits_{r \to 0} \frac{L_\varphi(p,
    r)}{l_\varphi(p, r)}\\
  I_\varphi (p) & = \varlimsup\limits_{r \to 0} \frac{L_\varphi(p,
    r)}{r}.\\ 
J_\varphi (p) &= \varlimsup\limits_{r \to 0} \frac{m (\varphi
  (T_r(p)))}{(m T_r (p))}
\end{align*}
$m$ is the Hausdorff measure. *** and *** where for any subset $E$ of
$V$, $T_r(E)$ denotes the tubulor neighbourhood of $E$ of radius $r$.
$$
T_r E= \{ v; v \in V\quad d (v, E) \leq r\}
$$

\begin{defis*}
  *** is said to be \textit{quasi-conformal} iff there exists a
  constant $B$ with $H \varphi (p) \leq B ~\forall p \in V$.
\end{defis*}

A quasi-conformal is said to be \textit{$k$-quasi-conformal} iff
$H_\varphi (p) \leq k$ for almost all $p \in V$.

The foregoing definition is not well-suited for proving some of the
basic theorems concerning quasi-conformal mappings. The development
below leads to an alternative definition of quasi-conformal mapping in
terms of the modulus of a shell.

\begin{defis*}
  $A$\pageoriginale  \textit{shell} ~$D$ in M\"obius $n$-space $\mathbb{R}^n \cup \{
  \infty\}$ is an open connected set whose complement consists of two
  connected components $C_\circ$ and $C_1$. $A$ shell not containing
  the point $\infty$ is called a \textit{shell in} $\mathbb{R}^n$. The
  component $C_1$ of its complement which contains $\infty$ is the un
  \textit{bounded} component and the other component $C_\circ$ will be
  referred to as the \textit{bounded} component.
\end{defis*}

For a shell $D$ in M\"obius-$n$ space, we define its \textit{conformal
  capacity}
$$
C(D) = \inf\limits_{u} \int_D |\triangledown u|^n dD
$$
where $u$ varies over $C^1$-functions with $u(c_\circ)= 0 u(c_1)=1$ 
$C_\circ , C_1$ being connected components of the complement of $D$. We
will call such a function $u$ a \textit{smooth admissible
  function}. It is easy to see that $C(D)$ is invariant under
conformal mapping, since the integral $\int_D |\triangledown u|^n d D$
is invariant.

Let $C_{n-1}$ denote the area of the surface of the unit
$n$-ball. Then we define
$$
\mod D= \left(\frac{C_{n-1}}{C(D)} \right)^{\frac{1}{n-1}}.
$$ 

\begin{example} \label{chap10:exp10.2}
  If $D_{a, b}=m \{ x, x \in \mathbb{R}^n a <|x|>b\}$ then 
  $$
  C(D_{a,b}) = C_{n-1} \left(\log \frac{b}{a} \right)^{- (n-1)}
  ~\text{and}~ \mod D_{a,b}= \log \frac{b}{a}.
  $$
\end{example}

\begin{proof}
  Let\pageoriginale $u$ be a smooth admissible function for $D_{a, b}$ then
  $$
  1 \leq \int \int\limits_a^b |\triangledown u| dr = \int\limits_a^b
  |\triangledown u| r^{\frac{n-1}{n}} r^{-\frac{n-1}{n}} dr.
  $$
\end{proof}

By H\"older's inequality
$$
1\leq \int\limits_a^b |\triangledown  u| dr <
\left(\int\limits_a^b|\triangledown u|^n r^{n-1} dr \right)^{1/n}
\left( \int_a^b r^{-1} dr\right)^{\frac{n-1}{n}}
$$
raising to the power $n$, and integrating over all rays
\begin{align*}
  c_{n-1} & \leq \left( \int_D |\triangledown u|^n  dD\right)
  \left(\log \frac{b}{a}\right)^{(n-1)}\\
  \therefore \quad C (D_{a, b}) & \geq c_{n-1} \left(\log
  \frac{b}{a}\right)^{- (n-1)}.
\end{align*}

On the otherhand by taking smooth admissible approximations of the
function
$$
u=
\begin{cases}
  \qquad0 & |x|\leq a\\
  \frac{\log x - \log a}{\log b - \log a} & a \leq |x|\leq b\\
  \qquad 1 & b \leq |x|
\end{cases}
$$
we get 
\begin{align*}
  C(D_{a,b}) \leq \int |\triangledown u|^n dD & = C_{n-1} \left(\log
  \frac{b}{a}\right)^{-n} \int^b_a \left(\frac{1}{r} \right)^n r^{n-1}
  dr\\
  & = C_{n-1} \left( \log \frac{b}{a}\right)^{- (n-1)}\\
  \therefore \quad C(D_{a, b}) & = C_{n-1} \left( \log \frac{b}{a}\right)^{-(n-1)}.
\end{align*}

Therefore\pageoriginale
$$
\mod (D_{a, b})= \log \frac{b}{a}.
$$

\begin{defi*}
  Let $D$, $D'$ be two shells with $C'_\circ \supset C_\circ$ and
  $C_1' \supset C_1$ then we say ``\textit{$D'$ separates the boundary
    of $D$}''. Clearly in this case $C(D')\geq C(D)$ and $\mod D'\leq
  \mod D$. 
\end{defi*}

\begin{lemma} \label{chap10:lem10.3}
  Let $S_r = \{ x | x \in \mathbb{R}^n, |x|=r \}$ and let $u$ be a
  $C^1$ function on $S_r$ then there exists a constant $A$ depending
  only on $n$ such that
  $$
  (CSC_{S_r}u)^n \leq A. r \int_{S_r} |\triangledown u|^n d S_r
  $$
  (For a proof see p.69 \cite{17}).
\end{lemma}

\begin{lemma}[Loewner] \label{chap10:lem10.4}
  Let $D$ be a shell in M\"obius $n$-space and let $C_\circ$, $C_1$
  denote the connected components of the complement of $D$, then
  $C(D)>0$ if neither $C_\circ$ nor $C_1$ consists of a single point.
\end{lemma}

\begin{proof}
  Choose a point $p$ in $\mathbb{R}^n$ such that $S_r$, the sphere
  with center at $p$ and radius $r$ meets $C_\circ$ and $C_1$ for all
  $r$ with $0< r_1 < r < r_2$ then
  \begin{multline*}
    \int_D |\triangledown u|^n d D = \int_n |\triangledown u|^n dx
    \geq \int_{D_{r_1, r_2}} |\triangledown u|^n dx\\
    = \int\limits^{r_2}_{r_1} \int\limits_{S_r} |\triangledown u|^n d
    \sigma dr ~\text{where}~ d \sigma ~\text{is}~ n-1 ~\text{measure
      on}~ S_r.
  \end{multline*}
\end{proof}

By\pageoriginale the previous lemma
$$
\int\limits_{S_r} |\triangledown u|^n d \sigma \geq A^{-1} r^{-1}
(CSC_{S_r} u)^n = A^{-1} r^{-1}.
$$

Thus $\int\limits_D |\triangledown u|^n dD~ A^{-1}
\int\limits_{r_1}^{r_2} \frac{dr}{r} = A^-1 \log \frac{r_2}{r_1}$ for
  all smooth admissible functions $u$. Hence $C(D)\geq A^{-1} \log
  \frac{r_2}{r_1} > 0$.

\begin{defi*}
  $A$ continuous function $f$ on the interval $0 \leq x \leq b$ is
  called \textit{absolutely continuous} if its derivative
  $\frac{df}{dx}$ exists almost everywhere and is integrable and
  $\int\limits_{x_\circ}^{x_1} \frac{df}{dx} dx= f(x_1)- f(x_\circ)$
  for all $a \leq x_\circ, x_1 \leq b$.
\end{defi*}

$A$ function $u$ on an open subset $D$ of $\mathbb{R}^n$ is called
\textit{$ACL$ in $D$}, if in any closed ball lying in $D$ it is
absolutely continuous on almost all lines in the ball parallel to the
coordinate axes.

\begin{notns}
  \begin{align*}
    E_+ & = \left\{ x; x \in \mathbb{R}^n \quad x_n > 0\right\}\\
    S^+_r & = S_r \cap E_+.
  \end{align*}
\end{notns}

\begin{lemma} \label{chap10:lem10.5}
  If $u$ is an $ACL$ function on $E^+$ then 
  $$
  \int\limits_a^b \left({}^{OSC}_{~S^+_r} u \right)^n \frac{dr}{r} \leq
  2A \int_{E_+} |\triangledown u|^n dx.
  $$
\end{lemma}
This is a slight generalization of Lemma \ref{chap10:lem10.3}, for a
proof see pp. 72-73 \cite{17}.

\begin{lemma} \label{chap10:lem10.6}\pageoriginale
  $$
  \displaylines{I^n_\varphi (p) \leq (H_\varphi (p))^n J_\varphi
    (p)\cr
  \text{and} \hfill I^n_{\varphi^{-1}} (\varphi (p))\quad (H_\varphi
  (p))^n J_{\varphi^{-1}} (\varphi (p)).\hfill }
  $$
\end{lemma}

\begin{proof}
  The first inequality comes from
  $$
  \left( \frac{L_\varphi (p, r)}{r}\right)^n = \left(\frac{L_\varphi
    (p, r)}{l_{\varphi}(p, r)} \right)^n \left( \frac{l_\varphi (p, r)}{r}\right)^n. 
  $$ 
  The proof of the second inequality is similar.
\end{proof}

\begin{remark*}
  It can be proved that if $\varphi$ is differentiable at $p$ then 
  $$
  I^n_\varphi (p) \leq (H_\varphi (p))^{n-1} J_\varphi (p).
  $$
\end{remark*}

\begin{lemma} \label{chap10:lem10.7}
  Let $\varphi$ be a quasi-conformal mapping then $\varphi$ exists
  almost everywhere.
\end{lemma}

\begin{proof}
  By the previous lemma
  $$
  I^n_\varphi (p) \leq (H_\varphi (p))^n J_\varphi(p).
  $$

  By hypothesis $H_\varphi (p) < B ~\forall p$. By Lebesgue's theorem
  (Saks \cite{19} p. 115)
  \begin{align*}
    & J_\varphi (p) < \infty ~\text{a.e.}~\\
    \therefore \quad & I_\varphi (p) < \infty ~\text{a.e.}\\
    \text{i.e.} \quad & \varlimsup_{q \to p} \frac{\varphi (q) -
      \varphi(p)}{|q-p|} < \infty ~\text{a.e.,}
  \end{align*}

  By the Radamacher-Stepnoff theorem (\cite{19} pp. 310-312)
  $\dot{\varphi}$ exists a.e.,
\end{proof}

\begin{lemma} \label{chap10:lem10.8}
  Let\pageoriginale $D$, $D'$ be open in $\mathbb{R}^n$ and let $\varphi: D \to D'$
  be homeomorphism of $D$ into $D'$. Let $p$ be a hyperplane in
  $\mathbb{R}^n$, if $H_\varphi(p)< k$ for $p \in D-p$, then $\varphi$
  in $ACL$ on $D$ and $\varphi^{-1}$ is $ACL$ on
  $\varphi^{-1}(D)$. \hfill (See \cite{17} for a proof.)
\end{lemma}

\begin{defi*}
  Given a shell $D$; a continuous function $u$ on $\bar{D}$, $ACL$ in
  $D$ is said to be \textit{admissible} if $u (L_\circ \cap
  \bar{D})=0$ and $u (C_1 \cap \bar{D})=1$, $C_\circ, C_1$ being
  connected components of the complement of $D$. 
\end{defi*}

\begin{lemma} \label{chap10:lem10.9}
  $$C(D) = \inf_{u~\text{admissible}} \int_D |\triangledown  u|^n dD$$
  \hfill (See \cite{17} pp. 64 for a proof).
\end{lemma}

\begin{lemma} \label{chap10:lem10.10}
  Let $\varphi: D \to D'$ be a homeomorphism of shells, if is $ACL$
  and $I^n \varphi \leq k^{n-1} J_\varphi$ almost everywhere, then
  $\mod \varphi(D) \leq k \mod D$.
\end{lemma}

\begin{proof}
  Given $u$ an admissible function on $D$, set $u'= u \circ
  \varphi^{-1}$ then $u \leftrightarrow u'$ is bijective
  correspondence between admissible functions on $D$ and $D'$.
\end{proof}

\begin{align*}
  \triangledown (u) (p) & = \varlimsup_{q \to p} \frac{|u (q) - u
    (p)|}{|q-p|} \\
  & = \frac{u(q) - u(p)}{|\varphi (q) - \varphi(p)|}\cdot
  \frac{|\varphi (q)- \varphi (p)|}{|q-p|}\\
  & = |\triangledown (u') \varphi (p)| I_\varphi (p).\\
  \therefore \quad C(D) & \int|(\triangledown (u'))(\varphi(p))|^n
  k^{n-1} J_{\varphi}(p).\\
  & = k^{n-1} \int_{D'} |\triangledown u'|^n dD\\
  \therefore \quad & \quad C(D) \leq  k^{n-1} C(D')\\
  \therefore \quad & \mod D' \leq k \mod D.
\end{align*}

We\pageoriginale now define the \textit{spherical symmetrization} of a
shell for the purpose of obtaining a rough quantitative estimate for
the modulus of a shell.

Let $L$ denote the ray $\{ (t, 0, \ldots 0) - \infty < t \leq 0\}$ in
$\mathbb{R}^n$, and let $E$ be a set, open or closed, in
$\mathbb{R}^n$. For each sphere $S_r = \{ x, x\in \mathbb{R}^n$,
$|x|=r\}$ place along $S_r$a spherical cap (of dimension $n-1$) with
center at $S_r \cap E$. Take the cap open if $E$ is open closed if $E$
is closed, and equal to $S_r$ if $S_r \subset E$. The resulting set is
denoted by $E^*$. Clearly $E^*$ is open \resp closed,
\resp connected) if $E$ is open (\resp closed, \resp connected).

\begin{defi*}
  Let $D$ be a shell in $\mathbb{R}^n$. The spherical symmetrization
  of $D$ is the set $D^\circ= (D \cup C_\circ)^* - C_\circ^*$.
\end{defi*}
where $C_\circ$ is the bounded component of $D$. It is clear that
$D_\circ$ is a shell.

\begin{thm} \label{chap10:thm10.11}
  $C(D) \geq C(D^\circ)$
\end{thm}

  The proof of this theorem makes  use of the isoperimetric
  inequalities for both euclidean and spherical space (cf. Mostow,
  loc. cit, p.87). Intuitively the result is plausible because the
  spherical symmetrization of $D$ is a ``smoothing'' of $D$ and hence
  admissible function for\pageoriginale $D^\circ$ need to be ``twist less'',
  accordingly $C(D^\circ)\leq C(D)$. 


In the proof of next lemma, we will estimate the modules of a shell by
comparing it with a special shell which generalizes a special slit
plane domain considered by Teichmuller.

\begin{defi*}
  The \textit{Teichmuller shell} $D_+ (b)$ is the shell in
  $\mathbb{R}^n$ whose complementary components consist of the segment
  $-1 \leq x_1 \leq 0$, $x_2 = \cdots = x_n =0$ and the ray $b\leq x_1
  < \infty$, $x_2= x_3 = \cdots - x_n=0$ where $b > 0$.
\end{defi*}

\begin{lemma} \label{chap10:lem10.12}
  $\varphi: R \to R'$ be a homeomorphisms of domains in
  $\mathbb{R}^n$, assume $\mod \varphi(D)\leq k \mod D$, then
  $H_\varphi < C^k$, where $C$ depends only on $n$.
\end{lemma}

\begin{proof}
  For $p \in R$, we consider the spherical shell $D_{l_\varphi(p, r),
    L_\varphi (p, r)}$ centered at $\varphi (p)$. Let $D=
  \varphi^{-1}(D_{l_\varphi (b, r), L_\varphi (p, r)})$ then $\log
  \frac{L_\varphi (p, r)}{l_\varphi (p, r)}=\break \mod D_{l(p, r), L(p, r)}
  \leq k \mod D \leq k \mod D^\circ$ (by \ref{chap10:thm10.11})
       [$D^\circ$ is spherical symmetrization of $D$]
       $$
       \leq k \mod D_\tau (1).
       $$
       Since $D^\circ$ \textit{separates the boundaries of} $D_\tau (1)$.
\end{proof}

Set $C= \mod D_\tau (1)$.

Then \quad $\frac{L_\varphi (p, r)}{l_\varphi (p, r)} = C^k$
$$
\therefore \quad H_\varphi (p) \leq C^k \forall p \in R.
$$

\begin{note}
  The idea of comparing $\mod D$ with $\mod D_\tau (1)$ is due to
  $A$. Mori cf. his posthumous paper in the Transaction of the AKS
  V. 84 (1957) pp. 56-77.
\end{note}

Putting\pageoriginale together Lemmas \ref{chap10:lem10.8},
\ref{chap10:lem10.10} and \ref{chap10:lem10.12}, we can now assert 

\begin{thm} \label{chap10:thm10.13}
  Let $\varphi : E \to E'$ be a homeomorphism of domains in
  $\mathbb{R}^n$~ ${}^n$. Then $\varphi$ is quasi-conformal iff
  \begin{enumerate}[(1)]
    \item $\varphi$ is $ACL$ in $E$.
      \item For all shells $D \subset E$, $k^{-1}\mod D \leq \mod
        \varphi (D) \leq k \mod D$, for some constant $k$. 
  \end{enumerate}
\end{thm}

  We now prove two theorems that are of central importance for our
  main theorem.

\begin{thm} \label{chap10:thm10.14}
  Let $\varphi$ be a quasi-conformal mapping of an open ball in
  $\mathbb{R}^n$ onto itself. Then $\varphi$ extends to a
  homeomorphism of the closed ball.
\end{thm}

\begin{proof}
  Mapping the domain of $\varphi$ onto the upper half space $X=$\break $\{
  (x_1\ldots x_n), x_n > 0 \}$ via a M\"obius transformation, the
  theorem is seen to be equivalent to the assertion a quasi conformal
  mapping $\varphi : X \to Y= \{ y: |y|< 1\}$ extends to a continuous
  mapping at any point $x$ of the boundary of $X$. for convenience, we
  take $x=0$.
\end{proof}

The proof is by contradiction. If $\lim\limits_{p \to 0} \varphi (p)~
(p \in X)$ does not exist, we can find two sequences $\{ p_k\}$ and
$\{ q_k\}$ in $X$ approaching 0 with $\lim\limits_{k \to \infty}
\varphi (p_k)= p', \lim\limits_{k \to \infty} \varphi (q_k)= q'$ and
$|q' - p'|=a > 0$. Denoting by $\overline{pq}$ the line segment
joining two points $p$ and $q$, we select points $p_\circ'$ and
$q_\circ'$ in $Y$ such that $d(\overline{p'_\circ p'_k},
\overline{q'_\circ q_k})> a$ for all large $k$, where $p'_k= (p_k),
q'_k= (q_k)$. Set $p_\circ= \varphi^{-1} (p'_\circ)$, $q_\circ =
\varphi^{-1} (q'_\circ)$. Then for $\sup (|p_k|, |q_k|)< r < \inf
(p_\circ, q_\circ)$, the hemisphere $S^+_r= \{ x; |x|= r, x_n> 0$
\pageoriginale meets the curves $\varphi^{-1} \overline{(p_\circ' p'_k)}$ and
$\varphi^{-1}\overline{(q'_\circ q'_k)}$. For each such $r$ at least
one of the coordinate functions of $\varphi(x) = (\varphi_1 (x), \ldots
\varphi_n (x))$ satisfies 
$$
{}^{OSC}_{~S^+_r} \varphi_i > a/\sqrt{n}.
$$

Hence 
$$
\sum_i \int_\circ^\infty \left({}^{OSC}_{~S^+_r}+ \varphi_i \right)^n
\frac{dr}{r} = \infty.
$$

By Lemma (\ref{chap10:lem10.8}), $\varphi_i$ is $ACL$ in $X$. Applying
Lemma \ref{chap10:lem10.5}, we get for each $i=1, \ldots , n$
\begin{align*}
  \int_\circ^\infty \left( {}^{OSC}_{~S^+_r} \varphi_i\right)^n
  \frac{dr}{r} & \leq 2A \int_X |\triangledown \varphi_i|^n dx \leq 2A
  \int_X I^{n}_\varphi dx\\
  & \leq 2A \int_X K^{n-1} J_\varphi dx \leq 2A K^{n-1} \int_Y dy.
\end{align*}
This yields a contradiction.

\begin{thm} \label{chap10:thm10.15}
  Let $\varphi$ be a $k$-quasi conformal mapping of an open ball $B^n$
  in $\mathbb{R}^n$, onto itself, $n\geq 2$, and let $\varphi_\circ$
  denote the boundary homeomorphism induced by $\varphi$. Then
  $\varphi_\circ$ is $C^k$-quasi conformal where $c = \mod D_\tau (1)$
  depends only on $n$.
\end{thm}

\begin{proof}
  By mapping $B^n$ onto upper half space $E_+$ via M\"obius
  transformation we can replace $B^n$ by $E_+$ in the theorem. By
  previous theorem $\varphi$ extends to \pageoriginale the
  boundary. Let $\varphi$ also denote its extension by symmetry to
  $\mathbb{R}^n$. $\varphi$ is $k$-quasi conformal in
  $\mathbb{R}^n$-hyperplane $x_n=0$. Hence $\varphi$ is $ACL$ in
  $\mathbb{R}^n$ by Lemma \ref{chap10:lem10.8}.
\end{proof}

We have $H_\varphi (p) \leq k$ a.e. in $\mathbb{R}^n$.

By Lemma \ref{chap10:lem10.6} and the remark following it
$$
(I_\varphi (p))^n \leq k^{n-1} J_\varphi (p)\quad a.e. 
$$

Therefore for any shell $D$ in $\mathbb{R}^n \mod \varphi (D)\leq k
\mod D$ by Lemma \ref{chap10:lem10.10}. Applying Lemma
\ref{chap10:lem10.12}, we get that $\varphi_\circ$ is $C^k$-quasi
conformal. 

The following two Lemmas round out prerequisites for our main theorem

\begin{lemma} \label{chap10:lem10.16}
  Let $\varphi: S^n \to S^n$ be a 1-quasi conformal map then $\varphi$
  is a M\"bius transformation if $n > 2$ \hfill (See \cite{17} pp. 101-102.)
\end{lemma}

\begin{lemma} \label{chap10:lem10.17}
  Let $\varphi$ be a quasi conformal mapping of a domain of
  $\mathbb{R}^n$ into $\mathbb{R}^n, n> 1$. Then $m (\varphi(E))=
  \int_E J_\varphi dx$ for any measurable set $E$ in the domain of
  $\varphi$. \hfill (cf. loc. cit. p. 94).
\end{lemma}

Now we prove theorem \ref{chap9:thm9.3}.

\begin{theorem*}
  Let $G= 0 (1, n)/ \pm 1$, $n>2$ and let $X$ be the associated
  symmetric Riemannian space. Let $\Gamma$, $\Gamma'$ be discrete
  subgroups such that $G/\Gamma$ and $G/\Gamma'$ have finite Haar
  measure. Let $\varphi:X \to X$ be a homeomorphism and $\theta:
  \Gamma \to \Gamma'$ an isomorphism such that $\varphi(\gamma x)=
  \theta (\gamma) \varphi(x)$ for all $\gamma \in \Gamma$, $x \in
  X$. Assume that $\varphi$ is quasi-conformal. Then\pageoriginale
  $\varphi$ induces a diffeomorphism $\varphi_\circ$ of the boundary
  component $X_\circ$ of the Satake compactification of $X$ and
  moreover $\varphi_\circ G \varphi_\circ^{-1}=G$ as transformations
  of $X_\circ$.
\end{theorem*}

\begin{proof}
  The symmetric space $X$ is the hyperbolic $n$-space which we
  identify with the open unit ball $B^n: |x|< 1$ in $\mathbb{R}^n$
  with metric $ds^2_H= \frac{|dx|^2}{1- |x|^2}$. the Satake
  compactification of $X$ then can be identified with the closed unit
  ball and $X_\circ$ is its bounding sphere $S^{n-1}$.
\end{proof}

Quasi-conformality of $\varphi$ with respect to $dS_H$ implies that
$\varphi$ is quasi-conformal with respect to $|dx|$. so in view of
Theorem \ref{chap10:thm10.14}, $\varphi$ extends to a homeomorphism of
the closed ball. Let $\varphi_\circ$ be the restriction of this
extension to the boundary $X_\circ= S^{n-1}$. By Theorem
\ref{chap10:thm10.15} and Lemma \ref{chap10:lem10.7}, $\varphi_\circ$
is almost everywhere differentiable. Furthermore since $X$ is dense
in $X \cup X_\circ$, $\varphi_\circ(\gamma x)= \theta(\gamma)
\varphi_\circ (x), \forall \gamma \in \Gamma$ and $x \in
X_\circ$. Also note that $G$ which is the full group of isometriese of
$X$ acts canonically on $X_\circ$ and conversely from the
identification of $GM(n-1)$ with $G$ (cf. \cite{17} p. 57 and p. 98),
it follows that each M\"obius transformation of $S^{n-1}$ extends to a
unique isometry of $X$. We replace $X_\circ$ by $\mathbb{R}^{n-1}\cup
\{ \infty\}$ via stereographic projection. Let $\psi$ denote the
homeomorphism of $\mathbb{R}^{n-1} \cup \{ \infty\}$ onto itself
induced by $\varphi_\circ$. Let $A$ be the 1-parameter subgroup of $G$
corresponding to the 1-parameter subgroup of M\"obius transformations
of $S^{n-1}$ obtained from the homotheties $x \mapsto \lambda x$ (with
$\lambda \in \mathbb{R}^+, x \in \mathbb{R}^{n-1}$) and $\infty
\mapsto \infty$ of $\mathbb{R}^{n-1} \cup \{ \infty\}$.

Let $p$ be a point at which the differential $\psi$ exists; we can
assume that $p=0$ for convenience. We identify the tangent space
to\pageoriginale $\mathbb{R}^{n-1}$ at 0 with $\mathbb{R}^{n-1}$ in
the usual way.

Define 
\begin{align*}
  f:G & \to \Hom_\mathbb{R} (\mathbb{R}^{n-1}, \mathbb{R}^{n-1})
  ~\text{by}\\
  f(g) & = (\psi g) (0)
\end{align*}

Set $F(g)= (t_f(g) f(g)) (\det {}^t f(g) f(g))^{-1/m}$ whose
$m=n-1$. Since a linear map $L$ is conformal iff 
$$
\frac{<L(x), L(y)>}{||L(x)|| \cdot || L(y)||} = \frac{<x, y>}{||x||
  \cdot ||y||}
$$

For any two orthogonal unit vectors $x$, $y$ we deduce from $<x+y,
x-y>=0$ that $0 = <L (x-y), L (x+y)>= || L(x) ||^2 - ||
L(y)||^2$. Thus $L$ maps the unit ball into a ball and 
\begin{center}
  ${}^t LL = (\det t_{LL})^{1/m} \cdot Id $ ~where~ $m=$ dimension of
  the vector space.
\end{center}

Thus if $L$ is conformal we have $({}^t LL) (\det {}^tLL)^{-1/m}=
Id$. Moreover $L$ is $K$-quasi-conformal iff the ratio of largest to
the smallest eigenvalue of ${}^tLL$ is $K^2$.

From the above it follows that $f(g)$ is a conformal mapping of the
tangent space at 0 iff $F(g)=$ identity. One can check that $F(ga)=
F(g)$, $\forall a \in A$ and $F(\gamma g)= F(g)$ for $\gamma \in
\Gamma$. Moreover $F$ is a measurable mapping of $G$ into $\Hom
(\mathbb{R}^{n-1}, \mathbb{R}^{n-1})$. It has bounded (by $K^2$ for
some $K$) entries almost everywhere since $\psi$ is
$k$-quasi-conformal for some $K$. Therefore $F$ gives rise to an
element of $\mathscr{L}^2 (G/\Gamma, \Hom (\mathbb{R}^{n-1},
\mathbb{R}^{n-1}))$; which we again denote by $F$. For an element
$\wedge \in \mathscr{L}^2 (G/\Gamma, \Hom (\mathbb{R}^{n-1},\break
\mathbb{R}^{n-1}))$ let norm $|| \wedge||^2 = \int_{G/\Gamma} \Tr
({}^t \wedge (g)\cdot \wedge (g)) d \mu$.

$G$\pageoriginale operates on $\mathscr{L}^2 (G/\Gamma, \Hom (\mathbb{R}^{n-1},
\mathbb{R}^{n-1}))$ via $(Z. f)(g)=f(gz)$ unitarily and we have $A.F
=F$.

Hence by Lemma \ref{chap5:lem5.2},  

$G.F=F$ i.e. $F$ is constant almost everywhere. In particular 
$$
F(gk)=F(g), \forall k \in 0 (n-1)
$$
i.e., the group of rotations about 0. ${}^t k= k^{-1}$ implies by the
special choice of $F$ that
$$
F(g)= F(gk)= k^{-1} G(g)k.
$$

Since $F(g)$ commutes with $0 (n-1)$, we conclude
$$
F(g) = \text{const.}~ Id.
$$

Since the matrix $F(g)$ is positive definite and of determinant 1, the
constant must equal 1.

Therefore $\psi$ is 1-quasi-conformal and therefore $\psi$ is M\"obius
transformation by Lemma \ref{chap10:lem10.16}.

In particular \quad $\varphi_\circ G \varphi_\circ^{-1}= G$.

\begin{thebibliography}{99}
\bibitem{1}{Araki} On root systems and an infinitisimal classification
  of irreducible symmetric spaces. J. of Math. Osaka City U. Vol. 13
  (1962) pp. 1-34.

\addcontentsline{toc}{chapter}{Bibliography}

\bibitem{2}{A. Borel} Gropes Lin\'eaires algebraiques. Ann of Math. 64
  (1956) pp. 1-19.
\bibitem{3}{$\rule{1cm}{.5pt}$ } Density properties of certain
  subgroups of s.s. groups without compact component. Ann. of Math. 72
  (1960) pp.179-188. 
\bibitem{4}{A. Borel \& J. Tits} Groupes R\'eductifs. Publ. IHES
  No. 27, pp 659-755.
\bibitem{5}{C. Chevalley} Theorie des Groupes de Lie, Tom II, Hermann
  \& cie, Paris, 1951.
\bibitem{6}{$\rule{1cm}{.5pt}$} Theorie des Groupes de Lie, Tom III
  Hermann \& cie, Paris, 1955.
\bibitem{7}{H. Frustenberg} A Poission Formula for Lie groups. 
\bibitem{8}{Helgason} Differential Geometry and symmetric spaces,
  Academic Press, New York, 1962.
\bibitem{9}{G. Hochschild} Structure of Lie groups, Holdenday, San
  Francisco, 1965.
\bibitem{10}{N. Jacobson} Lie Algebras, Interscience, New York, 1962.
\bibitem{11}{C.C. Moore} Ergodicity of flows on homogrneous
  spaces. Amer. J. of Math. 88(1965), pp. 154 - 178.
\bibitem{12}{G.D. Mostow} A new proof of E. Cartan's
  Theorem. Bull. Amer. Math. Soc. (1949) pp. 969-980.
\bibitem{13}{G.D. Mostow} Self adjoint groups. Ann. of Math. 62 (1955)
  pp. 44-55.
\bibitem{14}{$\rule{1cm}{.5pt}$} Some new decomposition theorems for
  s.s. groups, Memoirs No. 14 AMS (1955).
\bibitem{15}{$\rule{1cm}{.5pt}$} Fully irreducible subgroups of
  algebraic groups, A.J. of Math 78 (1956) pp. 200-221.
\bibitem{16}{$\rule{1cm}{.5pt}$} Homogeneous spaces with finite
  invariant measure Ann. of Math. 75 (1962) pp. 17-37.
\bibitem{17}{$\rule{1cm}{.5pt}$} Publications IHES No. 34, pp. 53-104.
\bibitem{18}{M. Rosenlicht} Some rationality questions on algebraic
  groups. Ann. Mat. Pura Appl. 43 (1957) pp. 24-50.
\bibitem{19}{S. Saks} Theory of Integrals. Hafner, New York.
\bibitem{20}{I. Satake} Representation and compactification of Symm
  Riemann Spaces, Ann. of Math. 71(1960) pp. 77-110.
\bibitem{21}{$\rule{1cm}{.5pt}$} Reductive algebraic groups over a
  perfect field. J. Math. Soc. of Japan 15 (1963) pp. 210-235.
\bibitem{22}{J.P. Serre} Lie algebras and Lie groups, Benjamin, New
  York, 1965.
\bibitem{23}{$\rule{1cm}{.5pt}$} Algebra de Lie S.S. complexes,
  Benjamin, New York, 1966.
\bibitem{24}{J. Tits} Classification of algebraic
  S.S. groups. Proc. of Symposium on pure Math. Vol. 19 pp. 33-61, AMS 1966.
\end{thebibliography}
