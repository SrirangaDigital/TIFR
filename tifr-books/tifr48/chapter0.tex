\setcounter{chapter}{-1}
\chapter{Preliminaries}\label{chap0} %%% chap0

We\pageoriginale start with two definitions.

\heading{Symmetric spaces.}

A Riemannian manifold $X$ is said to be symmetric if $\forall x
\in X$, there is an isometry $\sigma_x$ such that $\sigma_x(x)=x$
and $\forall t \in T_x \quad \sigma^\circ_x(t)=- t$, where $T_x$
is the tangent space at $x$ and $\sigma^\circ_x$ denotes the
differential of $\sigma_x$.

\heading{Locally symmetric spaces.}

A Riemannian manifold $X$ is locally symmetric if $\forall x \in
X$, is a neighborhood $N_x$ which is a symmetric space under induced
structure.

\begin{remark*}
  Simply connected covering of a locally symmetric complete Riemannian
  space is a symmetric space (see Theorem 5.6 and
  Cor 5.7 pp.187-188
  \cite{8}). 
\end{remark*}

Now we give an example which suggests that the last condition (that is
there are no direct factors of dimensions 1 or 2 in the statement of
the conjecture given in the introduction is in a sense necessary).

\begin{example}\label{chap0:exp0.1} %% 0.1
  $Y, Y'$ Compact Riemann surfaces of the same genus $>1$ and which are
  not conformally equivalent. By uniformization theory, the simply
  connected covering space of such Riemann surfaces is analytically
  equivalent to the interior $X$ of the unit disc in the complex
  plane. Then $Y= \Gamma\backslash X, Y' = \Gamma'\backslash X$ where
  $\Gamma$, $\Gamma'$ are fundamental groups of $Y$, $Y'$
  respectively. The elements of $\Gamma$, $\Gamma'$ operate
  analytically on $X$. Letting $G$ denote the group of conformal
  mappings of $X$ into itself, we have $\Gamma$, $\Gamma' \subset
  G$. It is well known that $G$ is also the group of isometries of $X$
  with respect to the hyperbolic metric $ds^2 = \frac{dz^2}{1-z^2}$
  and\pageoriginale with respect to this metric $X$ is symmetric
  Riemannian space of negative curvature. Hence $Y$, $Y'$ are locally
  symmetric spaces of negative curvature.
\end{example}

If $Y$, $Y'$ were isometric then they would be conformally equivalent,
which would be \textit{a contradiction}.

We list some facts about linear algebraic groups, these are standard
and the proofs are readily available in literature. Perhaps the use of
algebraic groups is not indispensable, however we hope that this will
simplify the treatment. 

Let $K$ be an algebraically closed field. For our purpose, we need
only consider the case $K= \mathbb{C}$, the field of complex numbers, 

\heading{Definitions.}

\textit{Algebraic set}: $A$ subset $A$ of $K^n$ is said to be
algebraic if it is the set of zeros of a set of polynomials in $K[X_1,
\ldots , X_n]$.

If $A$ is a subset of $K^n$, then $I(A)$ will denote the ideal of
$K[X_1, X_2, \ldots,\break X_n]$ consisting of the polynomials which vanish
at every point of $A$.

\textit{Zariski topology on $K^n$}: The closed sets are algebraic
sets.

\textit{Field of definition of a set}: Let $k$ be a subfield of $K$
and $A$ a subset of $K^n$. If $I(A)$ is generated over $K$ by
polynomials in $k[X_1, \ldots X_n]$ then $A$ is said to be defined
over $k$ or $k$ is a field of definition of $A$.

$A$ subgroup of the group $GL(n, K)$ of non-singular $n \times n$
matrices over $K$ is \textit{algebraic} if it is the intersection with
$GL(n, K)$ of a Zariski closed\pageoriginale subset of the set of all $n \times n$
matrices $M(n, K)$.

An algebraic group $G$ is a $k$-group if $G$ is defined over $k$,
where $k$ is a subfield of $K$.

\heading{Terminology.}

If $k = \mathbb{R}$ or $\mathbb{C}$, we shall refer to the usual
euclidean space topology as the $\mathbb{R}$-topology for
$G_{\mathbb{R}}$ or for $G_{\mathbb{C}}$.

For a $k$-group $G$ we write,
$$
G_k = G \cap GL (n, k).
$$

\begin{thm}\label{chap0:thm0.2} %%% 0.2
  $G$ be an algebraic group then the Zariski connected component of
  identity is a Zariski-closed, normal subgroup of $G$ of finite
  index. (\cite{5} Th 2, Chap. II, pp.86-88). 
\end{thm}

\begin{thm}[Rosenlicht]\label{chap0:thm0.3} %%% 0.3
  If $k$ is a infinite perfect field, $G$ a connected $k$-group then
  $G_k$ is Zariski dense in $G$ (\cite{18} pp.25-50). 
\end{thm}

\begin{prop}\label{chap0:prop0.4} %%% 0.4
  If $k$ is a perfect field, any $x \in GL (n, k)$ can be written
  uniquely in the form [Jordan normal form] $x = s \cdot u$ where $s$
  is semi-simple and $u$ is unipotent; $s$, $u$ commute. (use Th. 7,
  pp.71-72 [72])
\end{prop}

\begin{thm}\label{chap0:thm0.5} %%% 0.5
  If $k$ is a field of characteristic zero and $G$ an algebraic
  $k$-group then there is a decomposition $G= M.U$ (semi-direct
  product) where $U$ is a normal unipotent $k$-subgroup, $M$ is a
  reductive $k$-subgroup. Moreover any reductive $k$-subgroup of $G$
  is conjugate to a subgroup of $M$ by an element in $U_k$. (Th.7.1,
  pp.217-218, \cite{15}).
\end{thm}

\begin{prop}\label{chap0:prop0.6} %%% 0.6
  If $U$ is a unipotent algebraic subgroup of an algebraic group
  defined over a field $k$ of characteristic zero, then 
  \begin{enumerate}[\rm 1.]
    \item $U$\pageoriginale is connected (\cite{2} \S 8, p.46).
      \item $U$ is hypercentral [Engel-Kolchin] (see LA 5.7
        \cite{22}).
        \item $U_{\mathbb{R}}$ is connected in the
          $\mathbb{R}$-topology if $k \subseteq R$.
  \end{enumerate}
\end{prop}

\begin{prop}\label{chap0:prop0.7} %%% 0.7
  An abelian reductive group over algebraically closed field is
  diagonalizable. 
\end{prop}

\begin{defi*}
  A connected abelian reductive group is called a \textit{torus}.
\end{defi*}

\begin{thm}\label{chap0:thm0.8} %%% 0.8
  Let $G$ be an algebraic $k$-group. Then
  \begin{enumerate}[\rm 1.]
    \item The maximal tori are conjugate by an element of $G$.
      \item Every reductive element of $G_k$ lies in a $k$-torus.
        \item A maximal $k$-torus is a maximal torus.
          \item Any maximal torus is a maximal abelian subgroup if $G$
            is connected and reductive.
  \end{enumerate}
\end{thm}

\begin{defis*}
  A reductive element $x \in G L (n, k)$ is called $k-split$ (or
  $k$-reductive) if $y \in GL (n, k)$ such that $y xy^{-1}$ is
  diagonal, this is equivalent to saying that all the eigen-values of
  $x$ are in $k$. 
\end{defis*}

$A$ torus $T$ is called $k-split$ if $y \in GL (n, k)$ with $yT
y^{-1}$ diagonal, equivalently if each element of $T_k$ is $k-split$.

Let $G$ be a reductive group, $\ring{G}$ its Lie algebra and $T$ be a
maximal torus.

Consider the adjoint representation
\begin{align*}
  G & \to \Aut \ring{G}\\
  x & \mapsto \Ad_x
\end{align*}
then\pageoriginale  $\ring{G} = \sum^{\ring{G}}_\alpha$ \quad $\alpha
\in \Hom (T, \mathbb{C}^*)$

\noindent where $\ring{G}_\alpha = \left\{ y | y
\in \ring{G} \quad \Ad_x (y) = \alpha (x) \,y\, \forall x \in T
\right\} \Hom (T, \mathbb{C}^*)$ being\break abelian we will use additive
notation.

\begin{thm}\label{chap0:thm0.9} %%% 0.9
  Let $\phi = \{ \alpha | \alpha \in \Hom (T, \mathbb{C}),
  \ring{G}_\alpha \neq 0, \alpha \neq 0 \}$ then is called the
  \textit{set of roots of $G$ on} $T$ and we have 
  \begin{enumerate}[\rm 1.]
    \item $\alpha \in \phi \Rightarrow - \alpha \in
      \phi$
      \item $\alpha \in \Phi \Rightarrow \dim
        \varepsilon_\alpha =1$.
        \item $\left[\ring{G}_\alpha, \ring{G}_\beta \right]=
          \ring{G}_{\alpha +}$ if $\alpha, \beta, \alpha |\in \phi$
          
          $\left[\ring{G}_\alpha, \ring{G}_\beta \right]= 0$
          if $\alpha + \beta \notin \phi$
          \item There exists a linearly independent set $\triangle
            \in \phi$ such that the roots are either
            non-negative integral linear combination or a non positive
            integral linear combination of elements in
            $\triangle$. Such a subset is called a \textit{fundamental
              system} of roots on $T$.
  \end{enumerate}
\end{thm}

\begin{remark*}
  A fundamental system of roots can be obtained as follows. Take any
  linear ordering of $\Hom (T, \mathbb{C}^*)$ compatible with
  addition. Let $\triangle$ be the set $\{ \alpha \big| \alpha
  \in \phi, \alpha$ not a sum of two positive elements in
  $\phi \}$.
\end{remark*}

\begin{notns}
        Let $G$ be a group and $A$ a subset of
        $G$, then $Z(A)$ will denote the centralizer and Norm $(A)$ the
        normalizer of $A$.
\end{notns}

If $A$ and $B$ are two subsets of $G$
$$
A[B] = \{ a b a^{-1} \in A, b \in B\}.
$$

\begin{defi*}
  Let\pageoriginale $T$ be a maximal torus a of a connected reductive group
  $G$. $Z(T)$ operates trivially on $\phi$. The group $W=
  \frac{\text{Norm} (T)}{Z(T)}$ is called the \textit{Weyl group} of $G$.
\end{defi*}

\begin{thm}\label{chap0:thm0.10} %%% 0.10
  The \textit{Weyl group} operates simply transitively on the set of
  fundamental systems of roots.
\end{thm}

\begin{defis*}
  A reductive element $x \in G$ is $k-regular$ if $\forall y
  \in k$ - reductive, $\dim Z(x) \leq \dim Z(y)$.  
\end{defis*}

A reductive element is called \textit{singular} if it is not
$\mathbb{R}$-regular. 

Let $V$ be a $K$-subspace of $K^m$ and let $k$ be a subfield of $K$,
then $V$ is a $k-subspace$ if $V=K(V \cap k^m)$i.e., $V \cap k^m$
generates the space over $K$.

Let $G$ be a connected reductive $k$-group and ${}_kT$ a maximal
$k$-split torus. Consider the adjoint representation of ${}_k T$ on $G$.

Then $\ring{G}= \displaystyle{\sum_\alpha} \ring{G}_\alpha$ \quad
$\Hom ({}_k T, \mathbb{C}^*)$.

Each $\ring{G}_\alpha$ is a $k$-subspace.

The following analogue of the Theorem 0.9 is true.

\begin{thm}\label{chap0:thm0.11} %%% 0.11
  Let ${}_k \phi= \{ \alpha \big| \ring{G}_\alpha \neq 0, \alpha \neq 0
  \}$. Then
  \begin{enumerate}[\rm 1.]
    \item $\alpha \epsilon_k \phi \Rightarrow - \alpha
      \epsilon_k \phi$
      \item $\left[\ring{G}_\alpha, \ring{G}_\beta \right] =
        \ring{G}_{\alpha + \beta}$ if $\alpha, \beta, \alpha + \beta
        \epsilon_k \phi$

        $\left[\ring{G}_\alpha, \ring{G}_\alpha \right]=0$ if $\alpha
        + \beta \epsilon_k \phi$
        \item There exists a linearly independent subset ${}_k
          \triangle \subset {}_k \phi$ such that the roots are either
          non-negative\pageoriginale integral linear combination or non-positive
          integral linear combination of elements from ${}_k \triangle
          \cdot {}_k \triangle$ is called a fundamental system of
          restricted roots.
  \end{enumerate}
\end{thm}

Let $G$ be a connected reductive $k$-group, let ${}_k T$ be a maximal
$k$-split torus in $G$ and let $T$ be a maximal $k$-torus containing
${}_k T$. Let $\triangle$ be a fundamental system of roots on $T$ and
${}_k \triangle$ a fundamental system of restricted roots on ${}_k T$. We
call $\triangle$ and ${}_k \triangle$ \textit{Coherent} if the elements
in ${}_k \triangle$ are restriction of roots in $\triangle$. If one
introduces ordering of the sets $\phi$ and ${}_k \phi$ via lexicographic
ordering with respect to $\triangle$ and ${}_k \triangle$ respectively,
the resulting orders are Coherent the sense: If $\alpha \in
\phi$ and $\alpha \big| {}_k T > 0$ then $\alpha > 0$.

The existence of Coherent $\triangle$ and ${}_k \triangle$ can be seen as
follows. Let $X= \Hom (T, \mathbb{C}^*)$, the group of rational
characters of $T$. Then $X$ is a free abelian group. Ann ${}_k T$, the
subgroup of characters which are trivial on ${}_k T$ is a direct summand
of $X$ since ${}_k T$ is connected. Therefore, one can choose a basis
$\chi_1, \ldots \chi_r$ for $X$ such that $\chi_1 , \ldots , \chi_s$
is a base for Ann ${}_k T$. Now introduce lexicographic ordering on $X$
with respect to this base. The resulting order on $\phi$ clearly has
the property: If $\alpha$ and $\beta$ have the same restrictions to
${}_k T$ and if $\alpha > 0$, then $\beta> 0$. Consequently, there is
induced an order on ${}_k \phi$ compatible with addition. The
corresponding fundamental systems $\triangle$ and ${}_k \triangle$ are
Coherent. 

\begin{notns}
  Let $\triangle' \subset {}_k \triangle$
\begin{align*}
  \{ \triangle'\} & = \mathbb{Z}-\text{linear span of}~ \triangle'\\
  \alpha \triangle' & = \bigcap_{\alpha \in \triangle'} \ker
  \alpha \subset {}_k T
\end{align*}
\end{notns}\pageoriginale

Choose an ordering such that ${}_k \triangle$ consists of positive roots.

Put
$$
\ring{N} (\triangle')= \sum_{\substack{\alpha > 0\\\alpha \notin \{
    \triangle'\}}} \ring{G}_\alpha 
$$

$N (\triangle')$ be the complex analytic subgroup of $G$ with Lie
algebra\break $\ring{N} (\triangle')$. Since $\forall x \in
\displaystyle{\sum_{\alpha > 0}} \ring{G}_\alpha$ is nilpotent, $N
(\triangle')$ is a unipotent group.

Let 
\begin{align*}
  G (\triangle') & = Z({}^\perp \triangle')\\
  P (\triangle') & = \text{Norm}~ (N (\triangle'))\\
  N & = N (\phi)~ P  = P (\phi) ~\phi = \text{empty set}.\\
  M_k' & = \text{norm}~({}_k T)~ M_k  = Z({}_k T) = G (\phi)\\
  k^W & = M_k'/ M_k.
\end{align*}

The group $k^W$ is called \textit{Little Weyl Group} or
\textit{relative Weyl group} and this operates transitively on the set
of fundamental systems of restricted roots.

\begin{lemma}\label{chap0:lem0.12} %% 0.12
  $$
  P (\triangle') = G(\triangle') \cdot N (\triangle').
  $$
  $G(\triangle')$ is a maximal reductive subgroup of $P (\triangle')$
  and $N (\triangle')$ is a maximal normal unipotent subgroup (called
  unipotent radical of $P(\triangle')$).
\end{lemma}

\begin{thm}[Bruhat's decomposition]\label{chap0:thm0.13} %%% 0.13
  Let\pageoriginale $G$ be a connected reductive $k$-group. Then
  \begin{enumerate}[\rm 1.]
    \item $G_k = N_k \cdot M'_k \cdot N_k$
      \item The natural map $M'/M \to P_k \backslash G_k /P_k$ is
        bijection.
        \item Any unipotent $k$-subgroup of $G$ is conjugate to a
          subgroup of $N$ by an element in $G_k$.
          \item Any $k$-subgroup containing $P$ equals $P
            (\triangle')$ for some $\triangle' \subset \triangle$ ($P$
            is minimal parabolic $k$-subgroup). (see \cite{4} or
            \cite{21}). 
  \end{enumerate}
\end{thm}

\begin{remark*}
  $Z(T)$ is a connected subgroup. More generally if $S$ is any torus
  in $G$ then $Z(S)$ is connected.
\end{remark*}

Now we consider for a moment the special case that $k$ is
alge\-brai\-cally closed.

In this case $G(\phi)= Z(T)=T$. Since $T$ is a maximal abelian
subgroup and $P= TN$ is solvable. Clearly $P$ is connected. It follows
at once from assertion 4 of the previous theorem that the connected
component of the identity in $P (\triangle')$ contains $P$ and
therefore it is seen to coincide with $P (\triangle')$. In particular,
every subgroup of $G$ containing $TN$ is connected, and $TN$ is a
maximal connected solvable subgroup. 

\begin{defi*}
  A maximal connected solvable subgroup of an algebraic\break group is
  called a \textit{Borel subgroup}. A subgroup containing a
  \textit{Borel subgroup} is called \textit{Parabolic}.
\end{defi*}

\begin{thm}\label{chap0:thm0,14} %%% 0.14
  The\pageoriginale Borel subgroups of an algebraic group are conjugate under an
  inner automorphism.
\end{thm}

\begin{thm}\label{chap0:thm0.15} %%% 0.15
  If $G$ is a connected reductive $k$-group and ${}_k \triangle$ is a
  fundamental system of restricted roots on a maximal $k$-split torus
  ${}_k T$, then $P(\triangle')$ is parabolic for any $\triangle' \subset
  {}_k \triangle$. 
\end{thm}

\begin{proof}
  Let $T$ be a maximal $k$-torus containing ${}_k T$. Since all
  fundamental systems are conjugate under the Weyl group, it is
  possible to find a fundamental system $\triangle$ on $T$ which is
  coherent with ${}_k \triangle$. Let $\phi^+$ denote the set of positive
  roots on $T$ defined by a lexicographic ordering with respect to
  $\triangle$. Then for any $\triangle' \subset {}_k \triangle$, the Lie
  algebra of $P(\triangle')$ contains $\ring{G}_\alpha$ for all
  $\alpha \in \phi^+$. It follows directly that
  $P(\triangle')$ is parabolic. 
\end{proof}

\begin{remark*}
  A subgroup $Q$ is parabolic iff $G/Q$ is a complete variety;\break
  equivalently, in case $k= \mathbb{C}$ if $G/Q$ is compact in the
  $\mathbb{R}$-topology. 
\end{remark*}

From the conjugacy of Borel subgroups and the theorem above, it is
seen that any parabolic $k$-subgroup is conjugate to $P(\triangle')$
for some $\triangle' \subset {}_k \triangle$. Also any parabolic subgroup
containing the maximal $k$-split torus ${}_k T$ is $\omega
[P(\triangle')] \omega^{-1}$ with $\omega \in N({}_k T)$. 

Since a reductive element of a connected reductive group is
$k$-regular iff it lies in a single maximal torus, we see

\begin{prop}\label{chap0:prop0.16} %%% 0.16
  A reductive element of a connected reductive $k$-group $G$ is $k$
  regular iff it lies in at least one and at most finitely many
  parabolic $k$-subgroups of $G$, not equal to $G$.
\end{prop}
