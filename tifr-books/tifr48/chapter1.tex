
\chapter{Complexification of a real Linear Lie Group}\label{chap1} %%% chap1

Let\pageoriginale $G$ be a Lie subgroup of the Lie group of all automorphisms of a
real vector space $V$. Let $V_{\mathbb{C}}$ denote the
complexification of $V$ (i.e., $V_{\mathbb{C}}= V \otimes \mathbb{C}$)
we identify the elements of $\dot{G}$, the lie algebra of $G$, with
endomorphisms of $V$. We let $\dot{G}_{\mathbb{C}}$ denote the
complification of the Lie algebra $\dot{G}$ and let
$G^\circ_{\mathbb{C}}$ denote the analytic group of automorphisms of
$V_{\mathbb{C}}$ that is determined by $\dot{G}_{\mathbb{C}}$. We identify
the endomorphisms of $V$ with their unique endomorphism extension to
$V_{\mathbb{C}}$, so that we have $\dot{G} \subset
\dot{G}_{\mathbb{C}}$ and $G^\circ \subset
G^\circ_{\mathbb{C}}$. Where $G^\circ$ is connected component of
identify in $G$. 

\begin{defis*}
  By the \textit{complexification of a real linear Lie group} $G$ is
  meant $G^\circ_{\mathbb{C}}$. $G$, it will be denoted by $G_{\mathbb{C}}$.
\end{defis*}

By a \textit{f.c.c. group} we mean a topological group with finitely
many connected components.

Suppose that $G_*$ is a semisimple f.c.c. Lie subgroup of $GL (n,
\mathbb{R})$. Then $\dot{G}_*
\otimes_{\mathbb{R}}\mathbb{C}=\dot{G}_{\mathbb{C}}$ is
semisimple. Hence $\dot{G}_\mathbb{C}=[\dot{G}_{\mathbb{C}},
  \dot{G}_{\mathbb{C}}]$ is an algebraic Lie algebra [Th. 15,
  pp. 177-179 \cite{5}]. Since a Zariski-connected subgroup of $GL(n,
\mathbb{C})$ is topologically connected, it follows that the complex
analytic analytic semisimple group $G^\circ_{\mathbb{C}}$is algebraic,
and therefore $G_* \cdot G^\circ_{\mathbb{C}}$ is algebraic. Thus we
have

\begin{thm} \label{chap1:thm1.1} %%% 1.1
  The Zariski closure in $GL(n, \mathbb{C})$ of the semisimple
  f.c.c. Lie subgroup of $GL (n, \mathbb{R})$ is its
  complexification. 
\end{thm}

\begin{defi*}
  A subset $S$ of $GL(n, \mathbb{R})$ is said to be
  \textit{selfadjoint} if $t_S=S$ where $t_S= \{ g\big| t_g
  \in S$, ($t_g$ transpose of $g$) $\}$.
\end{defi*}

\begin{thm}\label{chap1:thm1.2} %%% thm1.2
  Let\pageoriginale $G_*$ be a semisimple f.c.c. Lie subgroup of $GL(n, \mathbb{R})$
  then $\exists x \in GL n, \mathbb{R})$ such that $x G_*
  x^{-1}$ is self adjoint. (for a proof see \cite{12}).
\end{thm}

\begin{notns}
  $S(n)$ will denote the set of all real $n \times n$ symmetric matrices
  and $P(n)$ the set of real positive definite symmetric matrices.
\end{notns}

For any $g\in GL (n, \mathbb{R})~ g= (g^t g)^{\frac{1}{2}}(g^t
g)^{-\frac{1}{2}}\cdot g$ with $(g^t g)^{\frac{1}{2}}\in P(n)$
and $(g^t g)^{-\frac{1}{2}}g \in 0 (n, \mathbb{R})$. 

\begin{thm}\label{chap1:thm1.3} %%% thm1.3
  Let $G_*$ be a self adjoint Lie subgroup of $GL (N, \mathbb{R})$. If
  $G_*$ is of finite index in $F_\mathbb{R}$, $F$ an algebraic
  $\mathbb{R}$ group (equivalently $G^\circ_*=
  (F_\mathbb{R})^\circ$). Then
  \begin{enumerate}[\rm 1.]
    \item $G_* = \{ G_* \cap P(n) \}\cdot \{ G_* \cap 0 (n,
      \mathbb{R})\}$
      \item $G_* \cap 0(n, \mathbb{R})$ is a maximal compact subgroup
        of $G_*$
        \item $G_* \cap P(n)= \exp (\dot{G}_* \cap S(n))$ (see \cite{12}).
  \end{enumerate}
\end{thm}

\begin{lemma}\label{chap1:lem1.4} %%% 1.4 lem
  Let $G_*$ be a real analytic self adjoint subgroup of $GL(n,
  \mathbb{R})$, $G$ its Zariski closure in $GL (n, \mathbb{C})$. Let
  $A$ be a maximal connected abelian subgroup in $G_* \cap P(n)$. Let
  $T$ be the Zariski closure of $A$ in $GL (n, \mathbb{C})$, then $T$
  is a maximal $\mathbb{R}$-split torus in $G$ and $A= (T\mathbb{R})^\circ$.
\end{lemma}

\begin{proof}
  $A$ being a commutative group of $\mathbb{R}$-diagonalizable
  matrices, is $\mathbb{R}$-diagonalizable. Therefore $T$, its Zariski
  closure is $\mathbb{R}$-diagonalizable and hence an abelian subgroup
  of $G$.
\end{proof}

Since $A$ is self adjoint, the centralizer $Z(A)$ of $A$ in $G$ and
therefore also $G_* \cap Z(A)$ are self adjoint.

By\pageoriginale the previous theorem
$$
G_* \cap Z(A) = \left\{ G_* \cap Z(A) \cap P(n) \right\} \circ \{ G_*
\cap Z(A) \cap 0(n, \mathbb{R})\}
$$

By maximality of $A$
$$
G_* \cap Z(A) \cap P(A)=A.
$$

Hence 
$$
Z(A) \cap G_* = A \cdot \{ G_* \cap Z(A) \cap 0 (n, \mathbb{R})
$$

Since $T \subset Z(A)$, we have
$$
(T_{\mathbb{R}})^\circ = A \circ \{ (T_\mathbb{R})^\circ \cap 0(n, \mathbb{R})\}
$$

Also since $(T_{\mathbb{R}})^\circ$ is diagonalizable over
$\mathbb{R}$, $(T_\mathbb{R})^\circ \cap 0(n, \mathbb{R})$ is finite
and as $(T_\mathbb{R})^\circ$ is connected, this consists of identity
matrix alone.

Thus $(T_\mathbb{R})^\circ=A$.

\begin{lemma}\label{chap1:lem1.5} %% 1.5lemma
  Let $G_*$ be a semi-simple self adjoint analytic subgroup of $GL (n,
  \mathbb{R})$ and let $G$ be its Zariski-closure. Let $K_\mathbb{R}=
  G \cap 0(n, \mathbb{R})$, $E = G_* \cap P(n)$ and $A$ as above, then
  $$
  K_\mathbb{R} [A]=E.
  $$
\end{lemma}

\begin{proof}
  Evidently $K_\mathbb{R} [A] \subset E$. We will prove the other
  inclusion. 
\end{proof}

First we show that if $e$, $p \in P(n)$ and
$\epe^{-1}\in P(n)$ then $\epe^{-1}=p$.

By\pageoriginale the theorem \ref{chap1:thm1.3} we have
\begin{align*}
  Z(p) &= \{ Z(p) \cap P(n)\} \cdot \{ Z(p) \cap 0(n,
  \mathbb{R})\}\hspace{1cm}  \cr
  \text{and}\hspace{1cm} Z(p) \cap P(n) & = \exp \{ Z(p) \cap S(n)\}
\end{align*}
where $Z(p)$ is centralizer of $p$.

Since 
$$
\displaylines{
  \epe^{-1} \in P(n) \quad \epe^{-1} = {}^t(\epe^{-1})=
  e^{-1}p\, e\cr
  \text{so} \hfill e^2 p = pe^2 ~\text{i.e.,}~  e^2 \in Z(p)\hfill
}
$$

Since
\begin{align*}
  e^2 & = \Exp (X) ~\text{for some}~ X \in \ring{Z}(p)\cap  S(n)\\
  e & = \Exp \frac{1}{2} X ~\text{therefore}~ e \in Z(p) \cap
  P (n).\\
  \therefore \qquad ep & = pe ~\text{i.e.}~ epe^{-1} = p, ~\text{as asserted}. 
\end{align*}

Now if $p \in E$ \quad $p = \Exp X$ for some $X \in
\dot{G} \cap S(n)$. 

The Zariski closure of one parameter group $\Exp t X$ is a torus which
is contained in a maximal $\mathbb{R}$-split torus (say $S$).

By conjugacy of maximal $\mathbb{R}$-split tori, $\exists x
\in G$ with $x (T_\mathbb{R})^\circ x^{-1}=
(S_\mathbb{R})^\circ$ where $T$ is the Zariski closure of $A$ in
$G$. By the previous lemma $(T_\mathbb{R})^\circ= A$ and hence $p
\in x A x^{-1}$.

As 
\begin{equation*}
  G= E \circ K_\mathbb{R}\qquad  (\text{see Th \ref{chap1:thm1.3}})
\end{equation*}
we have $x = ek$ with $e \in E$, $k \in K_\mathbb{R}$.
$$
x a x^{-1} =p \qquad \text{for some}~ a \in A.
$$

Thus $ekak^{-1} e^{-1}=p$ \quad but $kak^{-1} \in P(n)$ hence
$kak^{-1} =p$.

\begin{remark*}
  If\pageoriginale $B$ is a maximal connected abelian subgroup in $K_* =
  (K_\mathbb{R})^\circ$ then an argument similar to the one used in
  the above proof yields: $K_* [B] = K_*$.
\end{remark*}

\medskip
\noindent \textbf{Weyl chambers.} The connected components of
$\displaystyle{A - \bigcup_{\alpha \in \phi} \ker \alpha}$,
where $\phi$ is a restricted root system on $T$, are called the Weyl
chambers associated with $G_*$ and $A$.

If $\triangle$ is a fundamental system of restricted roots, then
$A_\triangle= \{ a \big| a \in A, \alpha (a)> 1 \forall \alpha
\in \triangle\}$ is a Weyl chamber. Observe that $(\text{Norm}~
T)_\mathbb{R}$ operates on $A$, for $(\text{Norm}~ T)_\mathbb{R}$
operates on $T_\mathbb{R}$ and hence on $(T_\mathbb{R})^\circ=A$.

If $0 \neq X_\alpha \in \dot{G}_\alpha$ then $ \forall h \in T$
\begin{align*}
  Ad ~h (X_\alpha) & = h~ X_\alpha h^{-1}= \alpha (h) X_\alpha\\
  {}^t(h X_\alpha h^{-1})& = (h^{-1})^t X_\alpha h= \alpha (h)
  {}^tX_\alpha \hspace{3cm}\\
  \text{i.e.} \hspace{2cm} h {}^t X_\alpha h^{-1} & = (\alpha(h))^{-1} {}^tX_\alpha
\end{align*}
this proves that ${}^tX_\alpha \in \dot{G}_{-\alpha}$.

Let $h_\alpha= [X_\alpha, {}^tX_\alpha]$ then $h_\alpha$, $X_\alpha$,
${}^tX_\alpha$ is a base for 3 dimensional split Lie algebra over
$\mathbb{R}$. By taking a suitable multiple of $X_\alpha$, we can
assume that 
$$
[h_\alpha, X_\alpha] = 2 X_\alpha, [h_\alpha, {}^tX_\alpha]= - 2^tX_\alpha
$$ 
then \qquad $\Exp \pi/2 (X_\alpha - {}^tX_\alpha) \in$ (Norm $T$).

Since\pageoriginale $X_\alpha - {}^tX_\alpha$ is skew symmetric it actually belongs
to $(\text{Norm}~ T) \cap K_*$.

$Ad \Exp \pi/2 (X_\alpha - {}^tX_\alpha)$ is reflection in the Wall
corresponding to $\alpha$ , of the Weyl chamber.

This shows that $Ad[(\text{Norm}~ T)_\mathbb{R} \cap K_*]$ contains
the reflections in all the Walls of the Weyl chambers.

\begin{thm}\label{chap1:thm1.6} %%% 
  $E= K_* [\bar{A}_\triangle]$.
\end{thm}

\begin{proof}
  $K_* [\bar{A}_\triangle] = K_* [(\text{Norm}~ A \cap K_*)
    [\bar{A}_\triangle]]$ 
\end{proof}

Since $Ad (\text{Norm}~ A \cap K_*)$ contains the reflections in all
the walls of Weyl chambers $(\text{Norm}~ A \cap K_*)
[\bar{A}_\triangle]= A$.
$$
\therefore \quad K_* [\bar{A}_\triangle] = K_* [A].
$$

Let $X \in \dot{E}$ and let $Y$ be an $\mathbb{R}$-regular
element in $A$, then since $K_*$ is compact, $\exists k \in K_*$ such that
$$
\displaylines{d (X, k[Y]) = d(X, K_* [Y])\cr
  \text{where} \hfill d (\widetilde{X}, \widetilde{Y}) = \Tr 
  ~(\widetilde{X}- \widetilde{Y})^2 \hfill \cr
  \text{then} \hfill d (k [X], Y)= d(X, k[Y]) \leq d(k[X], l [Y]),
  \forall l \in K_* \hfill }
$$
therefore $\forall Z \in \dot{K}_*$ the real valued function
\begin{align*}
  f_Z : t & \mapsto d(k[x], \Exp tZ [Y])\\
  & = \Tr [h [X]- \Exp (tZ)Y \Exp (- tZ)]^2
\end{align*}
is minimum at $t=0$.

$$
\therefore ~ \frac{\partial f_2}{\partial t}_{t=0} =0.
$$
which\pageoriginale gives
$$
\Tr (h [X]-Y) [Y, Z]=0 ~\text{but since}~ \Tr Y[Y, Z]=0
$$
we have 
$$
= \Tr Z[k[X], Y] = \Tr k [X][Y, Z]=0, \forall Z \in K_*
$$
hence 
$$
[k [X], Y]=0.
$$

$\mathbb{R}$-regularity of $Y$ implies
\begin{align*}
  & Z(Y) \cap \dot{G} \cap S(n) = \dot{A}\\
  \therefore \quad  &k [X] \in  \dot{A}
\end{align*}
this proves that
\begin{center}
  $K_* [A]\supset E$. The other inclusion is obvious.
\end{center}

\begin{defi*}
  An algebraic $k$-group is said to be $k$-compact if it contains no
  $k$-split connected solvable subgroup, that is a connected group
  that can be put in triangular form over $k$.
\end{defi*}

\begin{remark*}
  If $G$ is a reductive algebraic $k$-group then the following three
  conditions are equivalent.
  \begin{enumerate}
    \item $G$\pageoriginale is $k$-compact
      \item $G_k$ has no unipotent elements
        \item the elements of $G_k$ are reductive.
  \end{enumerate}
\end{remark*}

\begin{exr}
  Prove the above equivalences.

  [Hint $(1) \Rightarrow (2) \Rightarrow (3)$ is obvious prove $(3)
    \Rightarrow (1)$ by showing: not $(1) \Rightarrow$ not (3).]
\end{exr}

The following digression is included just for fun, we need it only in
the case $k=\mathbb{R}$.

\begin{thm}\label{chap1:thm1.7} %%% 1.7
  Let $k$ be a loc. compact field of characteristic 0. Then $G$ is
  $k$-compact iff it is compact in the $k$-topology.
\end{thm}

\begin{proof}
  $(\Rightarrow)$ Let $V$ be the underlying vector space [i.e. $G$ is
    a subgroup of $\aut v$]
\end{proof}

Let $E_d$ be the set of $d$ dimensional subspaces of $V$. Then there
is a canonical imbedding $E_d \hookrightarrow \mathbb{P} (\wedge^d
(V))$ which makes $E_d$ a closed subvariety of the projective variety
$\mathbb{P} (\wedge^d (V))$. The product $\displaystyle{\prod^n_{d=1}}
E_d$ is a closed subvariety of $\displaystyle{\prod^n_{d=1} (\wedge^d
  (V))}$ (which, by Segre imbedding, itself is a closed subvariety of
a projective space $\mathbb{P}$ of sufficiently large
dimension). Hence $\displaystyle{\prod^n_{d=1} E_d}$ is a compact set.

The set $W = \left\{ (\omega_1, \ldots \omega_n)\Big| (\omega_1, \ldots ,
\omega_n) \displaystyle{\prod^n_{d=1}} E_d \right\} \omega_1 \subset
\omega_2 , \ldots \alpha \omega_n$ is a closed subvariety (it is
called the Flag manifold) of $\displaystyle{\prod^n_{1}}E_d$.

$G$ operates on $W$. For a $k$-rational point $\omega \in
W_k$. Let $T_\omega$ be the stabalizer of $\omega$ in $G$ then
$G/T_\omega=G. \omega$. Since $T_\omega$ is $k$-triangularizable\pageoriginale
(hence solvable) and as $G$ is $k$-compact $(T_\omega)^\circ= \{
e\}$. Hence $T_\omega$ is finite [In an algebraic group the connected
  component of identity is of finite index see Th. \ref{chap0:thm0.2}].
Therefore $G.= G/T$ has dimension equal to that of $G$.

Let $D \displaystyle{\mathop{\cap}_{g \in G_k} T_{g \omega}}$ since
$T_{g \omega}= g T_\omega g^{-1}$. $D$ is a finite (therefore
discrete) normal subgroup and so it is central.

Since $T_\omega$ is finite we can choose $g_1\cdots g_r$ such that
$$
D= \bigcap^r_{i=1} T_{g_i \omega}
$$
let $u_i = g_i \cdot \omega$ and $W^i$ be the Zariski closure of $G
\cdot u_i$ in the projective variety.
$$
G ~\text{acts on}~ \prod^r_{i=1} G. u_i \left(\subset \prod^r_{i=1} w^i \right).
$$

Since $D$ acts trivially we get a faithful action of $G' = G/D$
on $\displaystyle{\prod^r_{i=1}(G. u_i)}$.

Let $v= (u_1 \cdots u_r)$ and let $V$ be the Zariski closure of the
$G'$ orbit $G'. v$ of $v$. Then since $V$ is a irreducible closed set
and $G'v$ is open (by Chevalley's theorem in algebraic geometry) $\dim
(\widetilde{V} - G'v) \leq \dim \widetilde{V}= \dim$ ($G'$ orbit of a
$k$-rational element). Therefore $\widetilde{V}- G'v$ has no
$k$-rational points.

So $\widetilde{V}_k = (G' v)_k= G'_k v$

$\therefore ~ \widetilde{V}_k$ is compact in $k$-topology.

Since\pageoriginale the differential of the map $G'_k \to G'_k. v$ is surjective by
the implicit function theorem for loc. compact fields, this map is
compact.

But $G_k \to G'_k$ is open (again by implicit function th.) and so
$\frac{G_k}{D} =$ Image of $G_k$ in $G'_k$ is open (and therefore a
closed subgroup). This proves that $\frac{G_k}{D}$ is compact, but
since $D$ is finite $G_k$ is compact in $k$-topology.

The converse is also true. For if $G_k$ is compact in $k$-topology $G$
cannot have a unipotent subgroup. (Any unipotent group is isomorphic
as an algebraic variety to $K^r$ and its set of $k$-rational point is
$k^r$ which is not compact). This proves that any element of $G_k$ is
reductive and this by the preceding remark implies that $G_k$ is
$k$-compact. 
