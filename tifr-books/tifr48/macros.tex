\usepackage{graphicx,xspace,fancybox,yhmath}
\newcounter{pageoriginal}
\marginparwidth=10pt
\marginparsep=10pt
\marginparpush=5pt
%\renewcommand{\thepageoriginal}{\arabic{pageoriginal}}
\newcommand{\pageoriginale}{\refstepcounter{pageoriginal}\marginpar{\footnotesize\xspace\textbf{\thepageoriginal}}
} 
\let\pageoriginaled\pageoriginale

\swapnumbers

\newtheorem{thm}{Theorem}[chapter]
\newtheorem{prop}[thm]{Proposition}
\newtheorem{lemma}[thm]{Lemma}
\newtheorem{coro}[thm]{Corollary}

\newtheoremstyle{example}{10pt}{10pt}{ }%
{}{\bfseries}{.}{ }{}
\theoremstyle{example}

\newtheorem{example}[thm]{Example}

\swapnumbers

%%%%%%%%%%%%%%%%%%
\newtheoremstyle{remark}{10pt}{10pt}{ }%
{}{\bfseries}{.}{ }{}
\theoremstyle{remark}

\newtheorem{definition}{Definition}
\newtheorem{Lemma}{Lemma}
\newtheorem{conj}{Conjecture}
\newtheorem{remark}{Remark}
\newtheorem{remarks}{Remarks}
\newtheorem{examples}{Examples}

\newtheorem{proof of theorem}{Proof of theorem}
%\newtheorem{prob}{Problem}	
%\newtheorem{problem}{Problem}

%%%%%%%%%%%%%%%%%%%
\newtheoremstyle{nonum-it}{}{}{\itshape}{}{\bfseries}{\bf.}{ }%
{\thmname{#1}\thmnote{ \bfseries#3}}
\theoremstyle{nonum-it}	

\newtheorem{lemma*}{Lemma}	
\newtheorem{theorem*}{Theorem}	
\newtheorem{prop*}{Proposition}	
\newtheorem{coro*}{Corollary}


%%%%%%%%%%%%%%%%%%%%%

\newtheoremstyle{nonum-roman}{}{}{}{}{\bfseries}{\bf.}{ }%
{\thmname{#1}\thmnote{ \bfseries#3}}
\theoremstyle{nonum-roman}	

\newtheorem{defi*}{Definition}
\newtheorem{defis*}{Definitions}
\newtheorem{remark*}{Remark}	
\newtheorem{remarks*}{Remarks}
\newtheorem{example*}{Example}
\newtheorem{examples*}{Examples}
\newtheorem{notation}{Notation}
\newtheorem{exr}{Exercise}
\newtheorem{notns}{Notations}
\newtheorem{note}{Note}

%%%%%%%%%%%%%%%%%%%%%


\newcommand*\mycirc[1]{%
  \tikz[baseline=(C.base)]\node[draw,circle,inner sep=.7pt](C) {#1};\:
}

\DeclareMathOperator{\Sec}{Sec}
\DeclareMathOperator{\aut}{Aut}
\DeclareMathOperator{\epe}{epe}
\DeclareMathOperator{\Exp}{Exp}
\DeclareMathOperator{\Tr}{Tr\,}
\DeclareMathOperator{\supp}{supp}
\DeclareMathOperator{\Supp}{Supp}
\DeclareMathOperator{\Hom}{Hom}
\DeclareMathOperator{\re}{Re}
\DeclareMathOperator{\Aut}{Aut}
\DeclareMathOperator{\Ad}{Ad}
%\DeclareMathOperator*{\lt}{lt}

\def\heading#1{\medskip\noindent\textbf{\large{#1}\medskip}}

\def\uub#1{\underline{\underline{#1}}}
\def\ub#1{\underline{#1}}
\def\oob#1{\overline{\overline{#1}}}
\def\ob#1{\overline{#1}}
\def\ophi{\overset{o}{\phi}}
\def\oval#1{\text{\cornersize{2}\ovalbox{$#1$}}}


\font\bigsymb=cmsy10 at 4pt
\def\bigdot{{\kern1.2pt\raise 1.5pt\hbox{\bigsymb\char15}}}
\def\overdot#1{\overset{\bigdot}{#1}}

\makeatletter
\renewcommand\subsection{\@startsection{subsection}{2}{\z@}%
                                     {-3.25ex\@plus -1ex \@minus -.2ex}%
                                     {-1.5ex \@plus .2ex}%
                                     {\normalfont}}%

\renewcommand\thesection{\@arabic\c@section}
%\renewcommand\thesection{\thechapter.\@arabic\c@section}
%\renewcommand\thesubsection{({\thechapter.\thesection.\@arabic\c@subsection})}
\renewcommand{\@seccntformat}[1]{{\csname the#1\endcsname}\hspace{0.3em}}
\makeatother

\def\fibreproduct#1#2#3{#1{\displaystyle\mathop{\times}_{#3}}#2}
\let\fprod\fibreproduct

\def\fibreoproduct#1#2#3{#1{\displaystyle\mathop{\otimes}_{#3}}#2}
\let\foprod\fibreoproduct


\def\cf{{cf.}\kern.3em}
\def\Cf{{Cf.}\kern.3em}
\def\eg{{e.g.}\kern.3em}
\def\ie{{i.e.}\kern.3em}
\def\iec{{i.e.,}\kern.3em}
\def\idc{{id.,}\kern.3em}
\def\resp{{resp.}\kern.3em}


\def\bA{\mathbf{A}}
\def\bB{\mathbf{B}}
\def\bC{\mathbf{C}}
\def\bD{\mathbf{D}}
\def\bE{\mathbf{E}}
\def\bF{\mathbf{F}}
\def\bG{\mathbf{G}}
\def\bH{\mathbf{H}}
\def\bI{\mathbf{I}}
\def\bJ{\mathbf{J}}
\def\bK{\mathbf{K}}
\def\bL{\mathbf{L}}
\def\bM{\mathbf{M}}
\def\bN{\mathbf{N}}
\def\bO{\mathbf{O}}
\def\bP{\mathbf{P}}
\def\bQ{\mathbf{Q}}
\def\bR{\mathbf{R}}
\def\bS{\mathbf{S}}
\def\bT{\mathbf{T}}
\def\bU{\mathbf{U}}
\def\bV{\mathbf{V}}
\def\bW{\mathbf{W}}
\def\bX{\mathbf{X}}
\def\bY{\mathbf{Y}}
\def\bZ{\mathbf{Z}}

\newcommand\Bpara[4]{%
\begin{picture}(0,0)%
        \setlength{\unitlength}{1pt}%
        \put(#1, #2){\rotatebox{#3}{\raisebox{0mm}[0mm][0mm]{%
        \makebox[0mm]{$\left\{\rule{0mm}{#4pt}\right.$}}}}%
\end{picture}}

%%%%%%%%%%%%%%%%%%

\theoremstyle{romancase}
\newtheorem{romancase}{Case}
\renewcommand{\theromancase}{\Roman{romancase}}

\theoremstyle{step}
\newtheorem{step}{Step}
\renewcommand{\thestep}{\Roman{step}}


\makeatletter
\renewcommand\chaptermark[1]{\markboth{\thechapter. #1}{\thechapter. #1}}

\def\cleardoublepage{\clearpage\if@twoside \ifodd\c@page\else
    \thispagestyle{empty}\hbox{}\newpage\if@twocolumn\hbox{}\newpage\fi\fi\fi}

\renewcommand\tableofcontents{%
    \if@twocolumn
      \@restonecoltrue\onecolumn
    \else
      \@restonecolfalse
    \fi
    \chapter*{\contentsname
        \@mkboth{%
           \contentsname}{\contentsname}}%
    \@starttoc{toc}%
    \if@restonecol\twocolumn\fi
    }
\makeatother

\marginparsep=10pt
\marginparwidth=18pt

