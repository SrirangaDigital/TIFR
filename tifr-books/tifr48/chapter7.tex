
\chapter{Automorphisms of $\Phi^*$}\label{chap7}

\begin{lemma} \label{chap7:lem7.1}
  Let\pageoriginale $G$ be an $\mathbb{R}$-group with no compact factors. Let
  $\tau:\dot{T} \to \dot{T}$ be a automorphism stabilizing
  ${}_\mathbb{R} T$ and $\Phi^*$ then $\tau$ preserves the Killing form.
\end{lemma}

\begin{proof}
  $$
  \dot{T} = (\dot{J}_1 \cap \dot{T}) + (\dot{J}_2 \cap \dot{T}) +
  (Z(\dot{J}) \cap \dot{T})
  $$
  we know that (i) $\tau$ preserves $B^*= \displaystyle{\sum_{\alpha
      \in \Phi^*}} \alpha^2$  (ii) the three subspaces $\dot{J}_1 \cap
  \dot{T}, \dot{J}_2\cap\dot{T}$ and $Z(\dot{J}) \cap \dot{T}$ are
  stable under $\tau$ and (iii) if $W^*$ is the subgroup of Weyl group
  stabilizing ${}_\mathbb{R} T$ then $\dot{J}_1 \cap \dot{T}$ and
  $\dot{J}_2 \cap \dot{T}$ are irreducible under $W^*$.
\end{proof}

Since $B^*$ and $B$ are preserved by $W^*$, $B_i=C_i B^*_i~ i= 1, 2$. $c_i
\neq 0$. Here $B_i, B_i^*(i=1,2)$ are restrictions to $\dot{J}_i \cap
\dot{T}$ of $B, B^*$ respectively. Let $B_3, B_3^*$ are restrictions
to $Z(\dot{J})_i \cap \dot{T}$ of $B, B^*$ respectively, then $B_3=
B_3^*$. 

Hence on $T$, $B= B^*_3+ C_1 B_1^*+ C_2B_2^*$. As $\tau$ preserves
$B_1^*, B_2^* \& B_3^*$ it also preserves $B$.

\begin{lemma} \label{chap7:lem7.2}
  Let $G$ be an $R$-group without compact factors and let $\tau:
  \dot{T} \to \dot{T}$ be an automorphism stabilizing ${}_{\mathbb{R}}
  \dot{T}$ and $\Phi^*$ then $\tau$ is restriction to $\dot{T}$ of an
  automorphism of $\dot{G}$.
\end{lemma}

\begin{proof}
  Let $W'$ be the subgroup of Weyl group of $T$ generated by $\{
  \sigma_\alpha,\break \alpha \in \Phi^*\}$. We shall first prove that $W=
  W'$. Given $B \in \{ \triangle_\circ \}$, $<\beta, \Phi^*>\neq 0$
  since $G$ has no $\mathbb{R}$-compact factors and hence $\{G_{\pm
    \alpha}, \alpha \in \Phi^*\}$ generates $G$.
\end{proof}

We can find $\alpha \in \Phi^*$ with $<\beta,\, \alpha > < 0$.

As\pageoriginale 
\begin{align*}
  & \sigma_\alpha (\beta)  = \beta + q (\beta, \alpha) 
  \alpha ~\text{where}~ q (\beta, \alpha)= \frac{-2 <\beta,
    \alpha>}{<\alpha , \alpha>} > 0\\
  & \sigma_{\sigma_\alpha} (\beta)  \in \Phi^*  \\
  \therefore \quad & \sigma_{\sigma_\alpha(\beta)}  \in W'
  ~\text{but}~ \sigma_{\sigma_\alpha (\beta)} = \sigma_\alpha
  \sigma_\beta \sigma_\alpha^{-1}\\
  \therefore \quad & \sigma_\beta  = \sigma_\alpha^{-1}
  \sigma_{\sigma_\alpha (\beta)} \sigma_\alpha \in W' ~\text{for all}~
  \beta \in \{ \triangle_z\}\\
  \therefore \quad & W' = W.
\end{align*}

$\tau \sigma_\alpha \tau^{-1} = \sigma_{\tau(\alpha)}$ is a reflection
since for $\alpha \in \Phi^* \tau (\alpha) \in \Phi^*$
\begin{align*}
  \tau W' \tau^{-1} & = W'\\
  \therefore \quad \tau W \tau^{-1} & = W.
\end{align*}

Thus $\tau$ permutes reflections in $W$, i.e. $\tau$ permutes the set
$\{ \sigma_\alpha, \alpha \in \Phi \}$.
\begin{align*}
  \therefore & \quad \tau \Phi = \Phi\\
  \therefore & \quad \tau ~\text{extends to an automorphisms of }~~
  \dot{G}. 
\end{align*}
