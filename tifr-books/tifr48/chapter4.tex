
\chapter{Discrete Subgroups}\label{chap4}

In\pageoriginale this and the following sections we will use the following
notations. 

$G$ will denote a semi-simple (complex analytic) algebraic
$\mathbb{R}-$group. $G_\mathbb{R}= G \cap GL (n, \mathbb{R})$ and $G_*
= G^\circ_{\mathbb{R}}$. For any subset $S$ of $G$, $S^*$ and
$\bar{S}$ are respectively the Zariski closure and the closure in
$\mathbb{R}$-topology of $S$ in $G$.

We state the following useful Theorem, for a proof the reader is
referred to \cite{3} or \cite{16}.

\begin{thm} \label{chap4:thm4.1}
  Let $G$ be a connected algebraic $\mathbb{R}-$group with no
  $\mathbb{R}$-compact factors and let $\Gamma$ be a
  $\mathbb{R}$-closed subgroup  of $G_\mathbb{R}$. If
  $G_{\mathbb{R}/\Gamma}$ has an $G_{\mathbb{R}}$-invariant finite
  measure, then $\Gamma$ is Zariski dense in $G$.
\end{thm}

Here after we assume that $\Gamma$ is a closed subgroup of
$G_\mathbb{R}$ such that $G_{\mathbb{R}/\Gamma}$ has an
$G_{\mathbb{R}}$ invariant finite measure and $G$ has no
$\mathbb{R}$-compact factors.

Now we prove a few ``density'' results.

\begin{lemma} \label{chap4:lem4.2}
  If $\Gamma_\circ$ is the set of reductive $\mathbb{R}$-regular
  elements in $\Gamma$ then $\Gamma^*_\circ= G$.
\end{lemma}

\begin{proof}
  We first show that $\Gamma_\circ$ is non-empty.
\end{proof}

Fix an element $x$ of $LA^t, t > 1$, let $U$ be a symmetric
nbd. of 1 in $G$. Then since the sett $x^n U \Gamma$. **** have same
non zero measure and since the total measure is finite, at least two
of\pageoriginale them intersect
$$
\displaylines{\text{let} \hfill  x^m U \Gamma \cap  x^k U \Gamma \neq
  \phi ~\text{for}~ k > m\hfill\cr
  \text{then} \hfill \Gamma \cap U x^{k-m} U \neq \phi \hfill \cr
  \text{i.e. for some} \hfill n \geq 1 \hfill \cr
  \Gamma \cap U x^n U \neq \phi. \cr
  \text{but} \hfill U x^n U \subset U [x^n] \cdot U^2. \hfill }
$$

If $U$ is sufficiently small by Lemma \ref{chap3:lem3.4} $U[x^n]\cdot
U^2 \subset G[LA^t]$. This implies that
$$
\Gamma \cap G [LA^t] \neq \phi
$$
$\therefore$ \qquad $\Gamma_\circ \supset \Gamma \cap G [LA^t]$ is
non-empty.

Let $\gamma_\circ \in \Gamma_\circ$, if $\gamma \in \Gamma-
S_{y_\circ}$ then by Theorem \ref{chap3:thm3.2},
$$
\gamma \gamma_\circ^n \in G[LA^t] ~\text{for all}~ n > n_\circ
(\gamma). 
$$

Set 
$$
\displaylines{B_\circ = \left\{\gamma_\circ^n; n> n_\circ
  (\gamma)\right\} \cr
  \text{then} \hfill B_\circ \cdot B_\circ \subset B_\circ
  ~\text{hence}~ B^*_\circ \cdot B^*_\circ = B^*_\circ. \hfill}
$$

Since the ideal of polynomials vanishing on $B_\circ$ is stable under
translation by $x \in B^*_\circ$ and therefore under translation by
$x^{-1}$ for $x \in B^*_\circ$ (see Lemma 1 on p. 80 [50]), we
have
$$
(B^*_\circ)^{-1} \subset B^*_\circ
$$

Therefore\pageoriginale
$$
\displaylines{(B^*_\circ)^{-1} (B^*_\circ) \subset B^*_\circ.\cr
  \therefore \hfill 1 \in B^*_\circ. \hfill }
$$

Also since 
\begin{align*}
  \gamma B_\circ & \in \Gamma_\circ\\
  \gamma B^*_\circ &\subset \Gamma^*_\circ\\
  \therefore \qquad \gamma^1 & = \gamma \in \Gamma^*_\circ.
\end{align*}

This proves that $\Gamma- S_{y_\circ} \subset \Gamma^*_\circ$. But
since $S_{y_\circ}$ is a Zariski closed proper subset of $G$,
$\Gamma-S_{y_\circ}$ is Zariski dense and therefore
$\Gamma_\circ^*=G$.

\begin{lemma} \label{chap4:lem4.3}
  Let $\gamma_1 \in \Gamma$, set 
  $$
  \displaylines{
  \Gamma_1 = \left\{ \gamma \big|\gamma \in
  \Gamma, \gamma, \gamma^n \in \Gamma_\circ  ~\text{for}~  n > n_\circ
  (\gamma)\right\}\cr
  \text{and} \hspace{1.9cm}\Gamma_2 = \left\{ \gamma^n \big|\gamma \in \Gamma_1
  \quad n \geq n_\circ (\gamma) \right\} \hfill \cr
  \text{then} \hspace{1.8cm} \Gamma^*_{\circ}= \Gamma^*_\circ=G.\hfill}
  $$
\end{lemma}

\begin{proof}
  Since the Zariski closure of $\{ x^n, n \geq n_\circ\}$ for $x \in G$
  is a group (see the proof of previous lemma) $x: \{ x^n, n \geq
  n_\circ^*\}$. 
\end{proof}

This shows that 
$$
\Gamma_1 \subset \Gamma^*_2.
$$

Hence it is sufficient to prove that $\Gamma_1$ is Zariski dense in
$G$. Given $y \in \Gamma_\circ$, since by Lemma \ref{chap3:lem3.6},
$S_y$ does not contain\pageoriginale any conjugacy classes, $\exists \gamma$ such
that $\gamma[\gamma_1] \notin S_y$.

$\therefore ~T_y = \left\{ \gamma, \gamma |\gamma_1| \in S_y\right\}$
is a proper algebraic subset of $G$. 

For any $\gamma \in \Gamma - T_y$, $\gamma[y_1] \notin S_y$ so 
$$
\displaylines{\hspace{1.5cm}\gamma [\gamma_1] y^n \in \Gamma_\circ
  ~\text{for all}~ n > m_\circ (\gamma)\hfill \cr
  \hspace{1.5cm}\gamma \gamma_1 \gamma^{-1} y^n \in \gamma_\circ ~\therefore~
  \gamma_1 \gamma^{-1} y^n \gamma \in \gamma^{-1} \Gamma_\circ \gamma
  = \Gamma_\circ\hfill \cr
  \text{i.e.} \hspace{1cm} \gamma_1 (\gamma^{-1} [y])^n \in
  \Gamma_\circ ~\text{for}~ n > m_\circ (\gamma))\hfill \cr
  \text{i.e.} \hspace{1cm}\gamma^{-1} [y] \in \Gamma_1 ~\text{if}~ \gamma \notin
  T_y\hfill \cr
  \therefore \hspace{1.2cm} \gamma \notin T^{-1}_y ~\text{implies}~
  \gamma [y]\in \gamma_1 \hfill \cr
  \therefore \hspace{1.2cm} \Gamma_1^* \supset (\Gamma- T^{-1}_y)^*
             [y].\hfill }
$$ 

Bur \qquad $\Gamma - T^{-1}_y$ is dense in $G$, so
\begin{align*}
  & y \in G^* [y] \subset \Gamma_1^*\\
  \therefore \qquad & \Gamma_\circ \subset \Gamma_1^*.
\end{align*}

By the previous lemma we have $\Gamma_1^*= \Gamma_\circ^* = G$.

Following is a refinement of the above result
\begin{lemma} \label{chap4:lem4.4}
  Let $S$ be a proper algebraic subset of $G$, let $n$ be a positive
  integer and $\gamma_1 \in \Gamma$. Then $\exists \, \gamma_\circ \subset
  \Gamma_\circ -S$ such that $\gamma_\circ, \gamma_\circ^2, \ldots ,
  \gamma_\circ^n$ and $\gamma_1 \gamma_\circ, \gamma_1 \gamma_\circ^2,
  \ldots , \gamma_1 \gamma_4^n \in \Gamma_\circ -S$.
\end{lemma}

\begin{proof}
  $\forall m, S_m = \{ x \big| x \in G, \gamma_1 x^m \in S\} \cup \{ x
  \big|x\in G, x^m \in S\}$\pageoriginale is a proper algebraic subset of $G$.
\end{proof}

Hence $S_1\cup  S_2\cup \ldots \cup S_n$ is a proper algebraic subset. Since
$\Gamma_2$ is Zariski dense, we can find a $\gamma_\circ$ in $\Gamma_2-
S_1 \cup S_2 \cdots \cup S_n$. Obviously such a $\gamma_\circ$
satisfies the requirements of the lemma.

\begin{lemma} \label{chap4:lem4.5}
  Let $G$ be a semi-simple $\mathbb{R}$-group and let ${}_\mathbb{R}T$
  be a maximal $\mathbb{R}-$split torus. Let $T$ be a maximal
  $\mathbb{R}-$torus containing ${}_\mathbb{R}T$. Set $A =
  ({}_\mathbb{R}T)^\circ \mathbb{R}$, $H= (T_{\mathbb{R}})^\circ$. Then
  $$
  (G_* [\Gamma] \cap H)^* =T.
  $$
\end{lemma}

\begin{proof}
  $Z(A)_{\mathbb{R}}= Z(_{\mathbb{R}}T)_{\mathbb{R}}= L.A$.
\end{proof}

Since
\begin{align*}
  H & = (H \cap L) \cdot A\\
  L^\circ [H] & = L^\circ [H \cap L] = L^\circ. A\\
  \text{and} \hspace{1cm} G_* [G_* [\Gamma] \cap H] & = G_* [\Gamma
    \cap G_* [H]]\hspace{4cm}\\
  & = G_* [\Gamma \cap G_* [L^\circ. A]]\\
  & \supset G_* [\Gamma_\circ] \supset \Gamma_\circ.
\end{align*}

By taking Zariski closure we get, since $G^*_* = G$
$$
G= \Gamma^*_\circ = (G_* [G_* [\Gamma]\cap H])^* = G [(g_*[\Gamma]
  \cap H^*)].
$$

Therefore
\begin{multline*}
  \dim G =\dim G [(G_* [\Gamma]\cap H)^*] 
  = \dim G/{Z(x)} + \dim (G_* [\Gamma] \cap H)^*\\
  \text{for some}~ x \in H.
\end{multline*}

Since\pageoriginale $\dim Z(x) \geq \dim T$, we find $\dim (G_* (\Gamma) \cap H)^* =
\dim T$ and thus $(G_* [\Gamma]\cap H)^*=T$.
