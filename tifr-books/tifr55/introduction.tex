\chapter{Introduction}


SIEVE METHODS, beyond that of Eratosthenes and of Legendre, can be
considered to have started with the works Brun (small sieve) and of
Linnik (large sieve). In first part of these lectures we confine
ourselves to an introduction to the large sieve and a survey of its
applications. Under Chapter $0$ we give a historical introduction to
the theory of the large sieve pertaining to the works, upto the first
paper of Bombieri (1965), covering a period of twenty-five years. 

Regarding the relative powers of elementary sieve methods and the
analytical methods one usually considers that the latter should be
more powerful. Further it has generally held that large sieve is than
the small sieve (also that Selberg's sieve always supersedes Brun's
sieve). But history has shown that such views are not totally
correct. 

As for the first point we elucidate the connection between the
elementary large sieve method and the analytical methods in number
theory by recalling briefly some of those basic methods used in number
theory. 
\begin{enumerate}[1)] 
\item In multiplicative number theory one has the important problem of
  finding asymptotic formulae for the sums $\sum\limits_{n \leq x} a_n
  $ as $x \to \infty$, where $a_n s$ are values of some
  number-theoretic functions. Introducing the function $F(s)=
  \sum\limits_{n=1}^{\infty} a_n n^{-s}. s \in \mathbb{C}$, the
  connection with analytical methods is brought about through the
  following formula, due to Dirichlet: 
$$
\sum_{n \in x}a_n = \frac{1}{2 \pi i} \int\limits_{Re~ s = c} F(s)
\frac{x^s}{s} ds  (x> 1 , \text{ non-integral }). 
$$
where $c$ is a real number exceeding the abscissa of convergence
(finite in practice) of $F(s)$. (Hence  and in what follows
integration along straight lines are always in the direction of
non-decreasing real and imaginary parts.) To clarify the use of this
formula let us take $F(s)= \zeta^2 (s)$, which corresponds to the
problems of finding the asymptotic formula (with error term) for
$\sum\limits_{n \leq x}d(n)$. In this case one can take for $c$ above
any value greater than $1$. One can show now that the major
contribution to the above integral comes from the part $|~ Im ~s|\leq
T$, provided $T$ is suitably large in relation to $x$ (we also choose
$c$ sufficiently close to $1$). Next shifting the line of integration
to the left one has that the remaining (major) part is, by Cauchy's
theorem. 
$$
=2 \pi i R + (\int\limits_{\substack{Re~ s =s~ \\ |~ Im ~s | \leq ~T}}
+\int\limits_{\substack{s~ \leq~ Re~ s\geq ~c \\ ~ Im ~s ~ = T}}-
\int\limits_{\substack{s \leq ~Re~ s\geq c \\  ~Im ~s  = T}}) \zeta^2
(s) \frac{x^s}{s}ds, 
$$
where $(0 \neq) \zeta < 1$ and $R$ denotes the sum of residues of the
integer and at its poles within the rectangle bounded by the lines $|~
Im~ s|=T, Re~ s = \zeta$ and $Re~ s=c$. It turns out that the estimate
for the first integral above dominates those of the other two and is
itself `negligible' provided $T$ is not too large in relation to $x$;
in otherwords, if $T$ is appropriately chosen, then $R$ is the main
term of the asymptotic formula for $\sum\limits_{n \leq x} d(n), x \to
\infty$. Thus, in this case, we are led to the following 
$$
\sum_{n \in x} d(n) = x~ \log ~ x + c_0 x+ 0(x^{\theta}), x \to
\infty, 
$$
with some constants $c_o$ and $\theta, 0 < \theta < 1$. (Clearly, the
restriction that $x$ is not an integer can easily be dropped here.) 

\item As to additive number theory we again consider a simple case
  only. Let $\gamma$ be an (infinite) set of non-negative integers and
  let $a_n$ be the characteristic function of $\gamma$; i.e., 
$$
a_n=
\begin{cases}
1 \text{ if } n \in \gamma,\\ 0 \text{ if } n \not\in \gamma.
\end{cases}
$$

\noindent
Introducing the function
$$
f(z)= \sum_{n=0}^{\infty} a_n z^n
$$
(so that the power-series has radius of convergence $=1$) the formula
$$
\sum_{\substack{n_1 \in \gamma , n_2 \in \gamma \\ n_1+ n_2 =N}} 1=
\frac{1}{2 \pi i} \int\limits_{|z| = r< 1} f^2 (z) z^{-N-1}dz , N\in N 
$$
provides the analytical connection with the problem of finding (the
asymptotic formula for ) the number of representations of $N$as a sum
of two numbers of $\gamma$. For instance, when $\gamma$ is the set of
primes this corresponds to the well-known Goldbach problem. Then it
turns out that the major contribution to the above (integral) comes
out of the points $z$ with arguments `close' to fractions with `small'
denominators, while $r$ approaches $1$ as $N \to \infty$, and the set
of such points $z$ constitute the `major arcs' and the remaining parts
are termed `minor arcs'. Thus again we see that to get information
about our problem one has to move close to the singularities (on the
unit circle) of our function. 

The functions introduced in $1)$ and $2)$ above are particular
instances of general Dirichlet series 
$$
G(s)= \sum_{n=0}^{\infty} a_n e^{-\lambda_n s} \quad a_n \in \mathbb{C}, s \in \mathbb{C},
$$
where $\lambda_1 < \lambda_2 < \lambda_3 < \cdots \to \infty$. Still
the above problems have this essential difference: Under $1)$ we
encountered isolated singularities and on the other hand, regarding
$2)$ one knows from gap theorems that, for examples, for  the function
$\sum\limits_{p} z^p$ the unit circle is the natural
boundary. However, both the cases illustrate the principle of the
analytical methods in that the singularities of the associated function
are the sources of arithmetical information regarding the concerned
problem, in the sense that heavier the singularity more is its
contribution to the main term. 

The idea of the second of the methods sketched above, the 'Hardy -
Littlewood method', goes back to Hardy and Ramanujan. It has been
later developed in a series of papers by Hardy and Littlewood. (A
variant of this method, introduce by I.M Vinogradov, which uses finite
sums instead of series, allows integration over the unit circle.) I
this method, the aforementioned principle is reflected in that the
contribution, of the major arcs (i.e., neighbourhoods of heavier
singularities), the so-called 'singular series' of the problem,
determines the main term.  

We are now in a position to indicate as to how the 'elementary' large
sieve method can be regarded as being analogous to the corresponding
analytical approach In its basic form the large sieve relates the
mean-square contribution from the mid-points of the major arcs (the
sources of arithmetical information) of the size  of the associated
function with the mean-square integral, similar to the singular series
(above) being related to the integral over the unit circle. Thus the
method links up an arithmetical information with the gross mean-square
(= number of  elements in $\gamma$, if $a_n$ is the characteristic
function of $\gamma$). This is to suggest that this elementary' sieve
method can be considered analogous to analytical methods.  
\end{enumerate}

