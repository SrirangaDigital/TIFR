
\chapter{The Large Sieve for Character Sums}\label{chap3}%chap 3

THERE\pageoriginale IS another version of the large sieve which
concerns with the averaging of character sums (cf. \eqref{eq3.2} below). In
this chapter we give three such results which are readily obtained
from Theorem \ref{chap2-thm2.5}. We prove first 

\setcounter{section}{3}
\setcounter{theorem}{0}
\begin{theorem}\label{chap3-thm3.1}%the 3.1
Let $Q \in \mathbb{N}$. For any character $\chi \mod q$ and for any
complex numbers $a_n$, satisfying 
\begin{equation*}
a_n=0, \text{ unless }(n,q)=1 \forall q \leq Q, \tag{3.1}\label{eq3.1}
\end{equation*}
write
\begin{equation*}
X(\chi):=\sum_{M < n \leq M + N} a_n \chi(n). \tag{3.2}\label{eq3.2}
\end{equation*}
\end{theorem}

Then, we have
\begin{equation*}
\sum_{q \leq Q} \frac{1}{\varphi (q)} \sum _{\chi \mod q}|\tau (\chi)|^2
|X(\chi)|^2 \leq (N+Q^2)\sum_{M<n \leq M+N} |a_n|^2, \tag{3.3}\label{eq3.3} 
\end{equation*}
\begin{equation*}
\sum_{q \leq Q}(N+\frac{3}{2}qQ)^{-1} \frac{1}{\varphi (q)} \sum_{\chi
  \mod q}|\tau (\chi)|^2|X (\chi)|^2 \leq \sum_{M < n \leq M+N} |a_n|^2,
\tag{3.4}\label{eq3.4} 
\end{equation*}
and
\begin{equation*}
\sum_{q \leq Q} \log \frac{Q}{q} \sum^{*}_{\chi \mod q}|X(\chi)|^2
\leq (N+Q^2) \sum_{M< n\leq M+N} |a_n|^2, \tag{3.5}\label{eq3.5} 
\end{equation*}
where $\tau(\chi)$ is defined by \eqref{eq1.60} and
\begin{equation*}
\sum_{\chi \mod q}^{*}:=\sum_{\substack{\chi \mod q \\ \chi \text{ primitive}}}
\tag{3.6}\label{eq3.6} 
\end{equation*}

\begin{remark*}%rem
The condition \eqref{eq3.1} is not usually a severe restriction, since
in applications either this fulfilled or the extra terms arising in
the other case are separately estimated to be small. 
\end{remark*}

\begin{proof}%pro
First of all, for $(n,q)=1$, it follows from \eqref{eq1.61}
\begin{equation*}
\varphi(\bar{\chi})a_n \chi(n) = \sum_{\ell=1}^{q} \bar{\chi}(\ell)a_n
e(n\frac{\ell}{q}),\tag{3.7}\label{eq3.7} 
\end{equation*}
which\pageoriginale holds, by \eqref{eq3.1}, also for $(n,q)>1$. Now
\eqref{eq3.7} gives 
\begin{equation*}
\tau (\bar{\chi}) X (\chi) =
\sum^{q}_{\ell=1}\bar{\chi}(\ell)U(\frac{\ell}{q}), \tag{3.8}\label{eq3.8}
\end{equation*}
where $U(x)$ is defined through \eqref{eq2.9}. Multiplying each side of
\eqref{eq3.8} by its complex conjugate, summing over all character $\chi \mod
q$ and using \eqref{eq1.57} with $u_\ell=(\dfrac{\ell}{q})$. we get
the identity   
\begin{equation*}
\frac{1}{\varphi (q)} \sum_{\chi \mod q}|\tau (\chi)|^2|X(\chi)|^2=
\mathop{\sum{}'}_{\ell 
  =1}^{q} |U(\frac{\ell}{q})|^2, \tag{3.9}\label{eq3.9}  
\end{equation*}
(cf. Bombieri and Davenport \cite{key2}) and further use of this in
Theorem \ref{chap2-thm2.5} yields \eqref{eq3.3} and \eqref{eq3.4}.

Next, for a character $\chi \mod q$ let $f$ be its conductor, and let $\chi$
be induced by $\chi^* \mod f$. Then by the assumption \eqref{eq3.1}
(cf. \eqref{eq1.59}),   
\begin{equation*}
X(\chi) = X(\chi^*), \tag{3.10}\label{eq3.10}
\end{equation*}
and using Lemma \ref{chap1-lem1.1} as well as
\begin{equation*}
\frac{f}{\varphi (f)}(\sum_{\substack{ r \leq x \\ (r,f)=1}})
\frac{\mu^2 (r)}{\varphi (r)}) > \log x \text{~ for~ } x > 0,
\tag{3.11}\label{eq3.11} 
\end{equation*}
we obtain, in our notation \eqref{eq3.6},
\begin{equation*}
\begin{cases}
\sum_{q \leq Q} \frac{1}{\varphi (q)} \sum_{\chi \mod q} |\tau
(\chi)|^2 |X(\chi)|^2= \sum_{f \leq Q} \frac{f}{\varphi (f)}
\sum_{\substack{r \leq Q/f \\ (r,f)=1}} \frac{\mu^2 (r)}{\varphi (r)}
\sum^*_{\chi \mod f}\\
 |X(\chi)|^2 \geq \sum_{f \leq Q} \log \frac{Q}{f}
\sum^*_{\chi \mod f}|X(\chi)|^2.  
\end{cases}\tag{3.12}\label{eq3.12}
\end{equation*}

Thus \eqref{eq3.5} is a consequence of \eqref{eq3.3}.

Regarding the quality of the results in Theorem \ref{chap3-thm3.1} we
note that, in 
view of the identity \eqref{eq3.9}, the estimates \eqref{eq3.3} and
\eqref{eq3.4} are 
capable of improvements only along with sharpening of Theorem
\ref{chap2-thm2.5}. On the other hand, the statement \eqref{eq3.5}
leaves a gap (even
through the inequality \eqref{eq3.11} is capable of an asymptotic
formulation.  
\end{proof}

In\pageoriginale the case when only primitive characters occur in both
\eqref{eq3.3} and \eqref{eq3.4} our condition \eqref{eq3.1} can be
removed to prove the next 

\begin{theorem}\label{chap3-thm3.2}%the 3.2
For any character $\chi \mod f$, $r \in \mathbb{N}$ and for arbitrary
complex numbers $a_n$, set 
\begin{equation*}
X_r (\chi):=\sum_{M<N \leq M+N} a_n \chi(n) c_r(n)
\tag{3.13}\label{eq3.13} 
\end{equation*} 
 where $c_r(n)$ is given by \eqref{eq1.29}. Then, we have
 \begin{equation*}
\sum_{\substack{rf \leq Q \\ (r,f)=1}} \frac{f}{\varphi{rf}} \sum^*_{\chi
  \mod f}|X_r(\chi)|^2 \leq (N+Q^2) \sum_{M < n \leq M+N}|a_n|^2,
\tag{3.14}\label{eq3.14} 
\end{equation*} 
and
\begin{equation*}
\sum_{\substack {rf \leq Q \\ (r,f) =1}}(\frac{N}{f} +
\frac{3}{2}rQ)^{-1} \frac{1}{\varphi (rf)} \sum^*_{\chi \mod
  f}|X_r(\chi)|^2 \leq \sum_{M < n \leq M+N}|a_n|^2. \tag{3.15}\label{eq3.15} 
 \end{equation*} 
\end{theorem}

\begin{remark*}%rem
Observe that, under the condition \eqref{eq3.1}, we have $c_r(n)$ appearing
in the non-zero terms of \eqref{eq3.13} as $\mu(r)$ and so,
\eqref{eq3.14} also leads to \eqref{eq3.5} in view of \eqref{eq3.12}. 
\end{remark*} 

\begin{proof}%pro
We have for any $q \in \mathbb{N}$, by \eqref{eq1.57},
\begin{equation*}
\frac{1}{\varphi(q)} \sum_{\chi \mod q}\bigg|
\sum_{\ell=1}^{q}\overline{\chi}(\ell )U(\frac{\ell}{q})
\bigg|^{2}= \mathop{\sum{}'}\limits_{\ell =1}^{q}| U
(\frac{\ell}{q})|^2. \tag{3.16}\label{eq3.16} 
\end{equation*} 

Now if $\chi \mod q$ is induced by $\chi^* \mod f$ ($f$ conductor of
$\chi$). we have, on using \eqref{eq1.68} and \eqref{eq1.59}, 
\begin{equation*}
q = rf, \chi(\ell) = \chi^*(\ell) \text{~ for~ } (\ell, q) =
1. \tag{3.17}\label{eq3.17} 
\end{equation*}

Therefore, summing \eqref{eq3.16} over $q \leq Q$ gives
\begin{equation*}
\sum_{q \leq Q} \mathop{\sum{}'}\limits^{q}_{\ell =1}|U(\frac{\ell}{q})|^2 \geq
\sum_{\substack{rf \leq Q \\ (r,f) = 1}} \frac{1}{\varphi (rf)}
\sum_{\chi \mod f}^*|
    \mathop{\sum{}'}\limits_{\ell=1}^{q} \overline{\chi}(\ell) 
U(\frac{\ell}{q})|^2. \tag{3.18}\label{eq3.18}   
\end{equation*}


For any primitive character $\chi \mod f$, $q=rf$, $(r,f)=1$ and any
$\ell$, $(\ell,q)=1$, on writing 
\begin{equation*}
\ell = \lambda r + \mu f, (\lambda,f) = 1, (\mu ,r) = 1,
\tag{3.19}\label{eq3.19}
\end{equation*}\pageoriginale
we will have
\begin{equation*} 
\begin{cases}
\mathop{\sum{}'}\limits_{\ell =1}^q \bar{\chi} (\ell) U (\frac{\ell}{q})  =
\sum\limits_{M < n \leq M+N}  a_n \mathop{\sum{}'}\limits_{\lambda = 1}^f
\overline{\chi}(\lambda r) \mathop{\sum{}'}\limits_{\mu = 1}^r  e(n
(\frac{\lambda}{f} + \frac{\mu}{r})) = \\ 
=\bar{\chi}(r) \sum\limits_{M < n \leq M+N} a_n
\mathop{\sum{}'}\limits^{f}_{\lambda = 
  1} \bar{\chi}(\lambda)e(n \frac{\lambda}{f})
\mathop{\sum{}'}\limits^{f}_{\mu = 1} 
e(n \frac{\mu}{r}) = \bar{\chi} (r) \tau (\bar{\chi}) X_r (\chi),  
\end{cases}\tag{3.20} \label{eq3.20}
\end{equation*}
because of \eqref{eq1.62} and \eqref{eq1.29}. Since $(r,f) = 1$, we
have (by \eqref{eq1.42}) that $|\bar{\chi}(r)| = 1$ and further by
Lemma \ref{chap1-lem1.1} that $|\tau(\bar{\chi})|^2 =  f$. Thus
\eqref{eq3.20} and \eqref{eq3.18}, on using \eqref{eq2.86}, prove the
part \eqref{eq3.14}. The proof of \eqref{eq3.15} is the same in the
before summing over $q$ in \eqref{eq3.16} we need multiply by the
factor $(N+\dfrac{3}{2} q Q)^{-1}$ and at the and employ
\eqref{eq2.87} instead of \eqref{eq2.86}. 

Since obviously
\begin{equation*}
c_1(n) = 1, \tag{3.21}\label{eq3.21}
\end{equation*}
we obtain, by retaining only the parts with $r=1$ in the expressions
occurring on the left-hand sides of Theorem \ref{chap3-thm3.2}, as a
particular case 
\end{proof}

\begin{theorem}\label{chap3-thm3.3} %The 3.3
For any character $\chi\mod q$ and for any complex numbers $a_n$, define
\begin{equation*}
X(\chi): = ~ \sum_{M < n\le M+N}a_n \chi(n). \tag{3.22}\label{eq3.22}
\end{equation*}
\end{theorem}

Then, we have
\begin{equation*}
\sum_{q \leq Q} \frac{q}{\varphi (q)}  \sum_{x \mod q}^*  |X(\chi)|^2
\leq (N+Q^2) \sum_{M < n \leq M+N} |a_n|^2 \tag{3.23}\label{eq3.23} 
\end{equation*}
and 
\begin{equation*}
\sum_{q \leq Q} (\frac{N}{q} + \frac{3}{2}Q)^{-1} \frac{1}{\varphi}(q)
\sum^*_{\chi \mod q}  |X(\chi)|^2 \leq \sum_{M < n\leq M+N}
|a_n|^2. \tag{3.24}\label{eq3.24} 
\end{equation*} 

\medskip
\begin{center}  
{\bf NOTES}\pageoriginale 
 \end{center} 
 
The version of the large sieve discussed in this chapter occurs for the
first time in Bombieri \cite{key1} (see, however, R\'enyi \cite{key2};
cf \eqref{eq0.32}). Simplifications of the proof and improvement of
the quality of the result were made by Davenport and Halberstam
\cite{key1}, who also obtained there the first weighted form. However,
as has been mentioned earlier and can also be seen from \eqref{eq3.9}
and \eqref{eq3.18}, the results of this chapter are more or less direct
consequences of those in Chapter \ref{chap2}.

A first result with conditions \eqref{eq3.1} was given by Bombieri and
Davenport \cite{key2} (and for \eqref{eq3.5}, see Bombieri \cite{key6}
(Th\'eor\`eme 8)) where they also prove a generalization of
\eqref{eq3.3} which, via Theorem \ref{chap2-thm2.5} under the
assumptions of Theorem \ref{chap3-thm3.1}, becomes 
\begin{equation*}
\sum_{\substack {q\leq Q \\(q,k)=1}}  \frac{1}{\varphi(q)} \sum_{\chi
  \mod q} |\tau (\chi)|^2 | \sum_{\substack {M < n \leq M+N \\  n
    \equiv \ell (\mod k)}} a_n \chi(n)|^2 \leq (\frac{N}{k} + 1 + Q^2)
\sum_{M < n \leq M+N} |a_n|^2  \tag{3.25}\label{eq3.25} 
\end{equation*} 
where $k \in \mathbb{N}$ and $(\ell,k) = 1$.
  
\eqref{eq3.11}:~ By Chapter \ref{chap1}, {\bf 1.}~ We have that, for $x > 0$, 
{\fontsize{10pt}{12pt}\selectfont
\begin{equation*}
\begin{cases}
\frac{f} {\varphi(f)} \sum_{\substack{r \le x \\ (r,f)=1}} \frac{\mu^2(r)}{\varphi(r)} =\prod_{p|f} (1-\frac{1}{p})^{-1} \sum_{\substack{r \le x \\ (r,f)=1}} \frac{\mu^2(r)}{r} \prod_{p|r}(1-\frac{1}{p})^{-1}=\\
= \prod_{p|f} (1+ \sum^{\infty}_{\nu =1} p^{-\nu}) \sum_{\substack {r
    \le x \\(r,f)=1}} \frac{\mu^2 (r)} {r} \prod_{p|r} (1+
\sum^{\infty}_{\nu =1} p^{-\nu}) \ge \sum_{n \le x} \frac{1}{n} > \log
x,  
\end{cases}\tag{3.26}\label{eq3.26} 
\end{equation*}}\relax
on using, for $x \geq 1$, if $N \leq x < N + 1$,
\begin{equation*}
\sum_{n \leq x} \frac{1}{n} \geq \sum_{n \leq N} \int\limits_n^{n+1} \frac{dt}{t} = \log (N+1) > \log x  \tag{3.27}\label{eq3.27}
\end{equation*}
(cf. van Lint and Richert \cite{key1}).

Clearly,\pageoriginale analogous to the derivation of \eqref{eq3.5} from
\eqref{eq3.3} one can 
obtain a corresponding result from \eqref{eq3.4}. In the special case $a_n =
\Lambda (n)$, this has been done by Montgomery and Vaughan
       \cite{key2} (cf. \eqref{eq6.29}).  

The extension to sums involving Ramanujan's sum $c_r(n)$, namely,
Theorem \ref{chap3-thm3.2} which contains Theorem \ref{chap3-thm3.3},
is due to A. Selberg (\cite{key6}, cf. Bombieri \cite{key6}
(Th\'eor\`eme 7A)).  

We obtained \eqref{eq3.23} from \eqref{eq3.14} by keeping only the
part with $r = 1$. On the otherhand, by taking the part corresponding
to $f = 1$, we get 
\begin{equation*}
\sum_{q \leq Q} \frac{1}{\varphi (q)}| \sum_{M < n \leq M+N} a_n c_q
(n)|^2 \leq  \sum_{q\leq Q} \mathop{\sum{}'}\limits_{\ell -1}^q |U
(\frac{\ell}{q})|^2 \tag{3.28}\label{eq3.28} 
 \end{equation*} 
which should be compared with Wolke's result (\eqref{eq2.105}).
 
The main importance of Selberg's generalization \eqref{eq3.14} is due to its
application in proving density theorems for Dirichlet's $L$-functions
(cf. Chapter \ref{chap6}, 2.). There the strongest known results
(cf. Montgomery \cite{key8}, Motohashi \cite{key11} and Jutila
\cite{key12} are based on 
the following strinking identity again due to A. Selberg: 
\begin{equation*}
\begin{cases}
L(s,\chi) M(s,\chi,\psi_r) = \sum_{n=1}^\infty \frac{\chi(n)}{n^s}
\psi_r (n) \sum_{d|n} \xi_{d}, (\chi : = \chi\mod f)\\ 
\text{for any set of } \xi_d \in \mathbb{C}, \xi_d = 0(1).  
\end{cases}\tag{3.29} \label{eq3.29}
\end{equation*} 
where
\begin{equation*}
M(s, \chi,\psi_r) : = \sum_{n=1}^\infty \frac{\chi(n) \xi_n \psi_r
  (n)}{n^s} \prod_{p|\frac{r}{(r,n)}} (1- \frac{\chi(p)}{p^{s-1}}),
\tag{3.30}\label{eq3.30}    
\end{equation*}
\begin{equation*}
\psi_r (n) : = \mu ((r,n)) \varphi ((r,n)), \tag{3.31}\label{eq3.31}
\end{equation*}   
and
\begin{equation*}
\mu (r) \neq 0. \tag{3.32} \label{eq3.32}
\end{equation*}

Actually, \eqref{eq3.14} is employed in the weaker from
\begin{equation*}
\sum_{\substack{rf \leq Q \\ (r,f) =1}} \frac{\mu^2(r)f}{\varphi (rf)}
\sum_{\chi \mod f}^* \Big| \sum_{M < n \leq M+N} a_n \chi(n) \psi_r (n) \Big
|^2 \leq (N+Q^2) \sum_{M < n\leq M + N} \Big| a_n \Big|^2. 
\tag{3.33}\label{eq3.33}  
\end{equation*}\pageoriginale


In fact, the expression on the left-hand side of \eqref{eq3.33} is the part
of the sum in \eqref{eq3.14} corresponding to squarefree $r$'s, since (under
\eqref{eq3.32}) one has one has readily, by \eqref{eq1.32}, 
\begin{equation*}
c_r(n) = \sum_{d|(r,n)} \mu (\frac{r}{(r,n)} \frac{(r,n)}{d}) d = \mu
(\frac{r}{(r,n)}) \varphi ((r,n)) = \mu (r) \psi_r (n) \; (\mu (r) \neq
0). \tag{3.34}\label{eq3.34} 
\end{equation*}

Motohashi \cite{key14} has in turn shown that \eqref{eq3.33} can be
generalized to 
give the following estimate: Let $\chi_j \mod f_j$, $f_j \leq F (j = 1,
\ldots,J)$, be distinct primitive characters. Then 
{\fontsize{9pt}{11pt}\selectfont
\begin{equation*}
\sum_{r \leq R} \sum_{\substack{j \leq J \\ (f_j, r)=1}}  \frac{\mu^2
  (r) f_j} {\varphi (rf_j)}| \sum_{M < n \leq M + N } a_n \chi_j (n)
\psi_r (n)|^2 \leq (N+O(JFR^2 \log (FR))) \sum_{M < n \leq M
  +n}|a_n|^2 \tag{3.35}\label{eq3.35} 
\end{equation*}}\relax
which, for sufficiently small $J$, improves upon \eqref{eq3.33}.

Slightly more general forms than those of Theorem \ref{chap3-thm3.1}
and \ref{chap3-thm3.3} 
(namely, without the restriction \eqref{eq3.1} or removing the limitation to
summing over only primitive characters) are possible but at the
expense of the quality of the estimates (cf. Bombieri \cite{key1} and
Davenport and Halberstam \cite{key1}). 

For an estimate of averages involving real characters only, see jutila
\cite{key6}. 

Finally, there are result concerning averages of characters sums,
which are useful when combined with large sieve estimates in some
applications. They can be obtained without employing results of
Chapter \ref{chap2} (cf. Montgomery \cite{key5}  (Theorems
6.2 and 6.3)). We 
mention, as an example, (Montgomery \cite{key5} (Theorem
6.2)):  
\begin{equation*}
\sum_{\chi \mod q} |X(\chi)|^2 \leq \varphi (q)
(1+\left[\frac{N-1}{q}\right]) \sum_{\substack{M < n \leq M + N
    \\ (n,q) =1}} |a_n|^2. \tag{3.36}\label{eq3.36} 
\end{equation*}

For\pageoriginale a proof, split $X(\chi)$ into $1+ \left[
  \dfrac{N-1}{q}\right]$ parts of length $q$ (introducing additions
$a_n$'s $= 0$, if necessary). For each part $X_1(\chi)$, say, it results
from \eqref{eq1.57}, with obvious appropriate choices for $u_\ell$, that 
\begin{equation*}
\sum_{\chi \mod q} |X_i(\chi)|^2  = \varphi (q) \sum_{\substack{n \in
    \mathscr{I}_i \\ (n,q)=1}} |a_n|^2, \tag{3.37}\label{eq3.37} 
\end{equation*}
where $\mathscr{I}_i$ denotes the range of $n$ in $X_i(\chi)$. Now \eqref{eq3.36}
following on using Minkowski's and Cauchy's inequalities. Likewise,
but with a more complicated yet still elementary, reasoning
(Montgomery \cite{key5} (Theorem 6.3)) one obtains    
\begin{equation*}
\sum_{\chi \mod q} |X (\chi^*)|^2 \leq q(1+ \left[\frac{N-1}{q}\right]
) \sum_{\substack{M < n \leq M+N \\ (n,q) =1}} |a_n|^2
\tag{3.38}\label{eq3.38} 
\end{equation*}
where for each $\chi \mod q$, $\chi^*$ denotes the primitive character
which induces $\chi$. 
