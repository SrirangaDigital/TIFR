
\chapter[The Large Sieve for Dirichlet Polynomials...]{The Large Sieve for Dirichlet Polynomials and Dirichlet
  Series}\label{chap4}%4 

STILL\pageoriginale ANOTHER from of the large sieve,as has been noted
Davenport, can be obtained with respect to Dirichlet polynomials 
\begin{equation*}
\sum_{n=1}^{N} a_n n^{-s}, s = \sigma + \text{ it } (\sigma , t \text{
  reals }),a_n \in \mathbb{C}, \tag{4.1}\label{eq4.1} 
\end{equation*}
as an application of Lemma \ref{chap2-lem2.1}.

\setcounter{section}{4}
\setcounter{theorem}{0}
\begin{theorem}\label{chap4-thm4.1} %The 4.1
Let $T_0$, $T(>0)$ and $t_r$ be real numbers satisfying
\begin{equation*}
T_0 = t_0 < t_1 < \cdots <t_R < t_{R+1} = T_0 + T,
\tag{4.2}\label{eq4.2} 
\end{equation*}
and put
\begin{equation*}
\delta : = \min_{0 \leq r \leq R} t_{r+1} - t_r \tag{4.3}\label{eq4.3}
\end{equation*}

Then, we have
\begin{equation*}
\sum_{r=1}^R \bigg|\sum_{n=1}^N a_n n^{-it_r}\bigg|^2 \leq (T+4N \log
N) (\log N + \delta^{-1}) \sum_{n=1}^N |a_n|^2, \forall a_n  \in
\mathbb{C}. \tag{4.4}\label{eq4.4} 
\end{equation*}
\end{theorem}

\begin{proof} %Pro
By \eqref{eq4.2} and \eqref{eq4.3}, we have $(R+1) \delta \leq T$ and
so we can suppose that $N \geq 2$. Also, by \eqref{eq4.3}, the
intervals  $\left[ t_r - \dfrac {\delta}{2}, t_r + \dfrac{\delta}{2},
  \right]$, $1 \leq r \leq R$, do not overlap. Therefore, taking in
Lemma \ref{chap2-lem2.1} 
\begin{equation*}
f(u) = \sum_{n=1}^N  a_n n^{-iu}, u = t_r (1 \leq r \leq R)
\tag{4.5}\label{eq4.5} 
\end{equation*}
and proceeding as in \eqref{eq2.28}, we obtain
\begin{equation*}
\sum_{r=1}^R |f(t_r)|^2 \leq (\int\limits_{T_0}^{T_0 +T} |f(t)|^2
dt)^{\frac{1}{2}}  \int\limits_{T_0}^{T_0 +T} |f'(t)|^2
dt)^{\frac{1}{2}} + \delta ^{-1} \int\limits_{T_0}^{T_0 +T} |f(t)|^2
dt. \tag{4.6}\label{eq4.6} 
\end{equation*} 

Now
\begin{equation*}
\int\limits_{T_0}^{T_0+T}|f(t)|^2 dt = T \sum_{N=1}^N |a_n|^2 +
\sum_{\substack{m,n = 1\\ m \neq n}} a_m \bar{a}_n
\int\limits_{T_0}^{T_0+T} (\frac{n}{m})^{it} dt. \tag{4.7}\label{eq4.7} 
\end{equation*}\pageoriginale 

The second sum, on noting that
\begin{equation*}
\bigg| \log \frac{n}{m} \bigg| \geq \frac{|n-m|} {\max (n,m)}
\tag{4.8}\label{eq4.8} 
\end{equation*} 
and using the arithmetic-geometric inequality, may be estimated by 
\begin{equation*}
2 \sum_{\substack{m,n = 1\\m \neq n}}^N \frac{|a_m{^a}{_n}|}{|\log
  \frac{n}{m}|} \leq N \sum_{\substack{m,n=1\\ m \neq n}}^N
\frac{1}{|n-m|} (|a_m|^2 + |a_n|^2). \tag{4.9}\label{eq4.9} 
\end{equation*} 

Due to symmetry in $m$ and $n$ (of these expressions), it is enough
to consider the factor of $|a_n|^2$ on the right-hand side. This
equals 
\begin{equation*}
\sum_{1 \leq m < n} \frac{1}{n-m} + \sum_{n < m \leq N} \frac{1}{n-m}
=  \sum_{1 \leq k < n} \frac{1}{k} + \sum_{1 \leq k \leq N-n}
\frac{1}{k}\leq 2 \log N. \tag{4.10}\label{eq4.10} 
\end{equation*} 

Therefore, the expression \eqref{eq4.9} can further be estimated by
\begin{equation*}
4 N \log N \sum_{n=1}^{N} |a_n|^2. \tag{4.11}\label{eq4.11}
\end{equation*} 
 
Using this in \eqref{eq4.7}, we get
\begin{equation*}
\int\limits_{T_0}^{T_0+T}|f(t)|^2 dt \leq (T + 4N \log N) \sum_{n=1}^N
|a_n|^2. \tag{4.12} \label{eq4.12}
\end{equation*}
  
Clearly, for $f'$ instead of $f$ \eqref{eq4.12} holds with an extra factor
of $\log^2 N$ on the right-hand side; in otherwords, we have 
\begin{equation*}
\int\limits_{T_0}^{T_0+T}|f(t)|^2 dt \leq (T + 4N \log N) \sum_{n=1}^N
|a_n|^2. \tag{4.13}\label{eq4.13} 
\end{equation*}   
\end{proof}

Now \eqref{eq4.4} follows from \eqref{eq4.6}, \eqref{eq4.12} and
\eqref{eq4.13}. 
    
For applications in the direction of the classical mean-value theorems
for the Dirichlet series, the powerful tools of Montgomery and Vaughan
\cite{key1}, namely \eqref{eq2.49} and \eqref{eq2.77}, can be
remodelled to the following Lemma.  

\setcounter{section}{4} 
\setcounter{lemma}{0} 
\begin{lemma}\label{chap4-lem4.1} %lem 4.1
Let\pageoriginale $\lambda_{1}, \ldots, \lambda_R$  be distinct real
number and set 
\begin{equation*}
\Delta :\min_{r\neq s}|\lambda_r - \lambda_s | \tag{4.14}\label{eq4.14}
\end{equation*} 
and
\begin{equation*}
\Delta_r : \min _{\substack{s \\ s\neq r}}  |\lambda_r - \lambda_s
|. \tag{4.15}\label{eq4.15} 
\end{equation*} 


Then
\begin{equation*}
\bigg| \mathop{\sum_{r = 1}^{R} \sum_{s=1}^{R}}_{r\neq s}  \frac{u_r
  \bar{u}_s}{\lambda_r - \lambda_s}\bigg| \le \pi \Delta^{-1}
\sum_{r=1}^R |u_r|^2, \forall u_r \in
\mathbb{C}. \tag{4.16}\label{eq4.16} 
\end{equation*}
and
\begin{equation*}
\bigg| \mathop{\sum_{r=1}^{R} \sum_{s=1}^R}_{r\neq s}  \frac{u_r
  \bar{u}_s}{\lambda_r - \lambda_s}\bigg| \leq \frac{3}{2} \pi
\sum_{r=1}^R |u_r|^2, \Delta^{-1}_r, \; \forall u_r \in 
\mathbb{C}. \tag{4.17} \label{eq4.17} 
\end{equation*}
\end{lemma}

\begin{proof} % proof
Let $\epsilon > 0$ denote a small number to be suitably restricted below. Put
\begin{equation*}
x_r = \epsilon \lambda_r  (1 \le r \le R). \tag{4.18}\label{eq4.18} 
\end{equation*}

Then, we have in the notation of \eqref{eq2.10} and \eqref{eq4.14}, for all
sufficiently small $\epsilon$, 
\begin{equation*}
\delta = \min_{r \neq s}  || \epsilon (\lambda_r - \lambda_s ) || =
\epsilon \min_{r \neq s} |\epsilon (\lambda_r - \lambda_s) | = \epsilon \Delta,
\tag{4.19}\label{eq4.19}  
\end{equation*}
and, similarly in the notation of \eqref{eq2.75} and \eqref{eq4.15}.
\begin{equation*}
\delta_r = \epsilon \Delta_r\quad (1 \leq r \leq
R). \tag{4.20}\label{eq4.20} 
\end{equation*}

Multiplying both sides of \eqref{eq2.49} and \eqref{eq2.77} by $\pi
\epsilon$ and further using \eqref{eq4.18}, \eqref{eq4.19} and
\eqref{eq4.20}, we obtain Lemma \ref{chap4-lem4.1} as 
a consequence of  
\begin{equation*}
\lim\limits_{\epsilon \to + 0}  \frac{\pi \epsilon} {\sin (\pi (x_r -
  x_s))} = \frac{1} 
    {(\lambda_r - \lambda_s )}. \tag{4.21} \label{eq4.21}
\end{equation*}
\end{proof}

We obtain almost immediately from Lemma \ref{chap4-lem4.1} the following 

\begin{theorem}\label{chap4-thm4.2} %the 4.2
Under the assumptions and notation of Lemma \ref{chap4-lem4.1}, we have
\begin{equation*}
\int\limits_{-T}^T \Big|\sum_{r=1}^R a_r e^{i \lambda_rt}
\Big|^2 dt = 2 (T + \theta_1 \pi \Delta^{-1}) \sum_{r=1}^R |a_r|^2,\quad
\forall a_r \in \mathbb{C}, \tag{4.22} \label{eq4.22}
\end{equation*}\pageoriginale
and
\begin{equation*}
\int\limits_{-T}^T \Big| \sum_{r=1}^R a_r e^{i \lambda_rt}
\Big|^2 dt  = \sum_{r=1}^R |a_r|^2 \;  (2T + 3\theta_2 \pi \Delta^{-1}_r),\;\;
\forall a_r \in \mathbb{C},  \tag{4.23}\label{eq4.23} 
\end{equation*}
where
\begin{equation*}
|\theta_j| \leq 1, j=1, 2. \tag{4.24}\label{eq4.24}
\end{equation*}
\end{theorem}

\begin{proof} % pro
The integral of the theorem is
\begin{equation*}
\sum_{r,s=1}^R a_r \bar{a}_s \int\limits_{-T}^T e^{i(\lambda_r -
  \lambda_s )t}{_{dt = 2T}} \sum_{r=1}^R|a_r|^2 + \sum_{r=1}^r
\sum_{\substack{s=1 \\ r \neq s}}^R a_r \bar{a}_s \frac{e^{i(\lambda_r
    \lambda_s) T}
-e^{-i(\lambda_r -\lambda_s) T}}{i(\lambda_r - \lambda_s
  )}. \tag{4.25}\label{eq4.25}  
\end{equation*}

Now application of \eqref{eq4.16} and \eqref{eq4.17} with the choices
\begin{equation*}
u_r = a_r e^{\pm i \lambda_r T} \tag{4.26}\label{eq4.26}
\end{equation*}
to the double sum in \eqref{eq4.25} yield \eqref{eq4.22} and
\eqref{eq4.23} respectively.  

The most interesting use of Theorem \ref{chap4-thm4.2} is when
applied, in its from \eqref{eq4.23}, to Dirichlet series via Dirichlet
polynomials. Taking 
\begin{equation*}
\lambda_r = - \log r, \tag{4.27}\label{eq4.27}
\end{equation*} 
we find, by \eqref{eq4.8},
\begin{equation*}
\Delta_r^{-1} \leq r+1, \tag{4.28}\label{eq4.28}
\end{equation*} 
and so get, by \eqref{eq4.23},
\begin{equation*}
\int\limits_{-T}^T \bigg|\sum_{r=1}^R a_r r^{-it} \bigg|^2 dt
= \sum_{r=1}^R |a_r|^2 (2T + 30_3 \pi (r+1)), \; |\theta_3 | \leq 1,\quad
\forall R \in \mathbb{N}. \tag{4.29}\label{eq4.29} 
\end{equation*}

Further, if we impose the condition
\begin{equation*}
\sum_{r=1}^\infty r |a_r|^2 < \infty , \tag{4.30}\label{eq4.30}
\end{equation*} 
we can conclude from \eqref{eq4.29} that the Dirichlet polynomials
\begin{equation*}
\sum_{r=1}^R a_r r^{-it} \tag{4.31}\label{eq4.31}
\end{equation*}\pageoriginale 
converge in the mean to the Dirichlet series
\begin{equation*}
\sum_{r=1}^\infty a_r r^{-it} \in L_2 (-T,
T). \tag{4.32}\label{eq4.32} 
\end{equation*} 
\end{proof}

Thus we derive from \eqref{eq4.23} the following important result. 
 
\begin{theorem}\label{chap4-thm4.3} %the 4.3
For $a_n \in \mathbb{C}$, suppose that
\begin{equation*}
\sum_{r=1}^\infty n |a_n|^2 < \infty. \tag{4.33}\label{eq4.33}
 \end{equation*}
\end{theorem}
 

 Then
\begin{equation*}
 \int\limits_{-T}^T \bigg|\sum_{n=1}^\infty a_n n^{-it}\bigg|^2
 dt =  \sum_{n=1}^\infty |a_n|^2 ~(2T + 3 \theta \pi (n+1)),
 \tag{4.34}\label{eq4.34}  
 \end{equation*} 
  where
 \begin{equation*}
 |\theta | \leq 1. \tag{4.35}\label{eq4.35}
 \end{equation*} 

\medskip
\begin{center} 
{\bf NOTES}
\end{center}

Theorem \ref{chap4-thm4.1}, due to Davenport, was published by
Montgomery \cite{key2}. 

Lemma \ref{chap4-lem4.1}, Theorems \ref{chap4-thm4.2} and
\ref{chap4-thm4.3} are due to Montgomery and Vaughan \cite{key1}
(Theorem 2 and Corollaries 2 and 3). 

\eqref{eq4.1}:~ Clearly, it is no restriction to have the result of
this chapter with $\sigma = 0$. Theorems \ref{chap4-thm4.2} and
\ref{chap4-thm4.3}: Obviously, it is possible to obtain with the help
of Gallagher's Lemma \ref{chap2-lem2.1} (cf. \eqref{eq4.6}) results,
corresponding to Theorems \ref{chap4-thm4.2} and \ref{chap4-thm4.3},
in with the integral is replaced by a sum over a set of well-spaced
points. 

\eqref{eq4.32}:~ For the reasoning leading to \eqref{eq4.32}, cf,
Titchmarsh, E.C., \textit{The Theory of Functions} (Oxford),
pp. 386--387. 


For a discussion of $\rho (N,T)$ in the general inequality 
\begin{equation*}
\int\limits^T_{-T} \mid \sum^N_{n=1} a_n n^{-it} \mid^2 dt \leq \rho
(N,T)
\sum^N_{n=1} \mid a_n \mid^2, \tag{4.36}\label{eq4.36} 
\end{equation*}
see Elliott \cite{key7}.
