\chapter{Notation}


In general all the notation employed in these lecture are either
standard or are given explicitly at the place of their first
occurrence. So we limit our seives here to a description of the former
type followed by the ones of the other kind (along with the place of
their first occurrence in parenthesis, for the convenience of
reference). A  reference $(A.B)$ to a part in these lectures stands
for ``'formula $B'$of 'chapter $A'$''. 

The letter $p$ (with or without affixes) denotes invariably a prime
number. An 'almost prime' $P_r$ (cf. $(12.8))$, for a given integer $r
\leq 1$, is a natural number with not more than $r$ prime factors
(counted with multiplicity). The greatest common divisor of two
integers $m$ and $n$ is denoted by $(m,n)$. For an integer $n$, the
divisor function $d(n)$ $(cf.p.ii)$ denotes the number of (positive)
divisors of $n$. We also use von Mangoldt's function $\wedge (n)$
($cf. (1.75)$) defined as log $p$ or $0$ according as $n$ is a power
of (some prime) $p$ or not. As usual, for a real $x,[x]$ denotes the
greatest integer not exceeding $x$. Euler's constant $\gamma$
(cf. $(8.33)$) is $(\lim_{x \to \infty}(\sum\limits_{1 \leq n \leq x}
n^{-1}- \log x))$.  

The order notation $O , o \ll , \gg$ have their customary meaning and
the dependence of the implied constant on some auxiliary parameter
$(s)$ is (when essential) given explicitly. (The notation $\asymp$ ,
meaning $\gg \ll$, occurs only once in connection with ($(0.51)$.) 

The symbols $\mathbb{N}, \mathbb{Z}, \mathbb{R}, \mathbb{C}$ denote
respectively the sets of natural numbers, integers, real numbers,
complex numbers endowed with their natural (basic)
structures. Regarding intervals (of reals or integers ) the convention
of using  the brackets ), (to indicate the excluded end-point $(s)$
and ], [ to indicated the included end point $(s)$ is adopted. Also we
    $|A|$ to denote the cardinality of a finite set (or
    sequence)$\mathcal{A}$). 

Finally, the notation explicitly introduced in the lectures: We have
$||x||$ $(cf.(2.10)), e(u) (cf. (0.52))$. The number-theoretic
functions $\nu (n)\break (cf. (1.9))$. $\mu(n) (cf.(1.8)). q(n) (cf.(1.19)),
c_q (n) (cf.(1.29)), \varphi (n)\break (cf. (1.33))$ and $r(n)(cf. (6.71))$
occur more than once. Also for the summation conventions
$(\sum\limits_{\ell =1}^{q}, \sum\limits_{\ell \mod q})$ and $
{\sum\limits^{{*}}_{\chi \mod q}}$ see respectively $(1.29)$ and
$(3.6)$. The following conditions are of repeated use in the second
part of these lectures: 
\begin{gather*}
(\Omega_1 )(cf. (9.16_a) \text{ or } (9.16_b)); (\Omega _2(k.L))
  (cf. (11.3)); (\Omega _2(k)) (cf.(11.4));\\ 
(R) (cf.(11.9)); (\Omega_\circ )(cf.(11.11)); (Q) (cf. (11.23)); (R(1,
  \alpha))(cf. (11.62)). 
\end{gather*}



