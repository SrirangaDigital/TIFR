
\chapter[Some Applications of the Small Sieve.....]{Some Applications
  of the Small Sieve in the Case $\omega(p) =
  \dfrac{p}{p-1}$}\label{chap10}%10  

AS\pageoriginale ALREADY mentioned at the beginning of Chapter \ref{chap9} we
present here 
some applications of the results obtained there in the chapter for
our purposes. Theorem \ref{chap9-thm9.3}, which is not contained in the
arithmetical form of the large sieve (cf. the remark made subsequent to
\eqref{eq9.17}), is of particular interest. 

We have the following two interesting (cf. Notes) applications,\break which
we shall quote without any details of proof. 

\setcounter{section}{10}
\setcounter{theorem}{0}
\begin{theorem}\label{chap10-thm10.1}
We have, as $N \to \infty$,
 \begin{gather*}
|\{p:p \leq N, p+h = p' \} \leq 4 \mathfrak{S} (h) \frac{N}{\log^2 N} \{ 1+O (
\frac{ \log \log N}{\log N}) \},\\
 \forall h \in \mathbb{Z}, h \neq 0, \; h
\equiv 0 \mod  2.\tag{10.1}\label{eq10.1} 
 \end{gather*} 
uniformly in $h$, and also 
\begin{gather*}
| \{p:p \leq N,N - p= p' \} \leq 4 \mathfrak{S} (h) \frac{N}{\log^2 N}
\{ 1+O ( \frac{ \log \log N}{\log N}) \},\\
\text{ for }  N \equiv 0 \mod
2.\tag{10.2}\label{eq10.2} 
\end{gather*}
 where $\mathfrak{S}$ is defined through \eqref{eq9.42}.
\end{theorem}

More generally one also has 

\begin{theorem}\label{chap10-thm10.2}
Let $A>0$ and let $a,b,k,\ell$ be integers satisfying 
\begin{equation*}
ab \neq 0, (a,b)= 1 , ab \equiv 0 \mod  2.\tag{10.3}\label{eq10.3}
\end{equation*}
and 
\begin{equation*}
(k,\ell)=1, 1 \leq k \leq \log^A x (\mathbb{R} \ni x \geq
  x_0).\tag{10.4}\label{eq10.4} 
\end{equation*}
\end{theorem}

Then we have, uniformly in $a,b,k$ and $\ell$, as $x \to \infty$
{\fontsize{10pt}{12pt}\selectfont
\begin{equation*}
|\{ p:p \leq x, p \equiv \ell  \mod  k, ap+b = p' \}| \leq 4
\mathfrak{S} (abk) \frac{x}{\varphi(k)\log^2 x} \{ 1+O_A (\frac{\log
  \log x}{\log x}) \}.\tag{10.5}\label{eq10.5}  
\end{equation*}}\relax

A much more delicate application is the next theorem which shall be used 
in\pageoriginale Chapter \ref{chap13}. Though some of the
majorizations in the proof of Theorem \ref{chap10-thm10.3} are crude,
many others 
involve rather delicate considerations. Before coming to the
formulation of this theorem we shall obtain some useful auxiliary
results. 

\setcounter{lemma}{0}
\begin{lemma}\label{chap10-lem10.1} 
We have
\begin{equation*}
\sum_{q \le x} \frac{\mu^2(q)}{q} h^{\nu (q)} \leq_{h} ( \text {
  log } x + 1)^h \text { for } x \ge 1, h \in
\mathbb{N}. \tag{10.6}\label{eq10.6}  
\end{equation*}
\end{lemma}

\begin{proof}
Consider
\begin{equation*}
\sum_{a \le x} \frac{\mu^2 (q)}{q} h^{\nu (q)} \le \prod_{p \le x} (1
+ \frac{h}{p}) \le \prod_{p \le x} (1 +
\frac{1}{p})^h. \tag{10.7}\label{eq10.7} 
\end{equation*}
 
Now \eqref{eq10.6} is apparent in view of the well-known formula due
to Mertens, 
\begin{equation*}
\prod_{p \le x} (1-\frac{1}{p}) = \frac{e^{-\gamma }}{\log x} (1 + 0
(\frac {1}{\log x})). \tag{10.8}\label{eq10.8} 
\end{equation*}
on noting that $(1 + \dfrac{1}{p}) \le (1 - \dfrac{1}{q})^{-1}$.
\end{proof}

\begin{lemma}\label{chap10-lem10.2}%lemma 10.2
Let $A > 0$ and let $h$, $k \in \mathbb{N}$. Let $K \le \log^A x$ for
sufficiently large $x$. Set 
\begin{equation*}
E(x, d) : = \underset{(\ell, d)=1}{\max} |\pi (x; d, \ell) - \frac{li\
  x}{\varphi (d)} |. \tag{10.9}\label{eq10.9} 
\end{equation*}
\end{lemma}

Then for any $U_1 (>0)$ there exists a value $C_1 = C_1 (U_1, h, A)$
such that 
\begin{equation*}
\sum_{d < \frac{\sqrt{x}}{k \log^{C_1}x}} \mu^2 (d) h^{\nu (d)} E(x,
dk) \ll_{U_1, h, A} \frac{x}{\varphi (k)
  \log^{U_1}x}. \tag{10.10}\label{eq10.10}  
\end{equation*}

\begin{proof}
By the rough estimate (cf. \eqref{eq10.9})
\begin{equation*}
E(x, d_1) \ll \frac{x}{d_1}\quad \text{for}\quad d_1 \le x
\tag{10.11}\label{eq10.11}  
\end{equation*}
and an application of the Cauchy's inequality followed by extensions
of ranges for variables in the resulting summations, we see that the
expression on the left-hand side in \eqref{eq10.10} is 
\begin{equation*}
\ll (\frac{x}{k})^{\frac{1}{2}} ( \sum_{d \le x} \frac{\mu^2 (d) h^{2
    \nu (d)}}{d})^{\frac{1}{2}} (\sum_{d <
  \frac{\sqrt{x}}{\log^{C_1}x}} E(x,
d))^{\frac{1}{2}}. \tag{10.12}\label{eq10.12}  
\end{equation*}\pageoriginale

In view of the bound given by \eqref{eq10.6} for the first sum here, we make
the choice of $C_1$, for given $U_1$, by means of \eqref{eq6.23} such that
the second sum is bounded by $x (\log x)^{- (2U_1+h^2+A)}$. Then,
continuing the estimation in \eqref{eq10.12}. we obtain further 
\begin{equation*}
\underset{U_1, h, A} \ll (\frac{x}{k})^{\frac{1}{2}} (\log
x)^{\frac{1}{2}h^2} (x (\log x)^{(2U_1 + h^2 + A})^{\frac{1}{2}}
\underset{U_1, h, A} \ll x(\log x)^{U_1 - \frac{1}{2}A}
(\varphi(k))^{-\frac{1}{2}}. \tag{10.13}\label{eq10.13} 
\end{equation*}
which proves \eqref{eq10.10} because $\varphi(k) \le k \le (\log x)^A$.
\end{proof}

Now we are in a position to prove the main theorem of this chapter.

\begin{theorem}\label{chap10-thm10.3}%the
Let $v$ be a real number satisfying
\begin{equation*}
v > 3. \tag {10.14}\label{eq10.14}
\end{equation*}
\end{theorem}

Let $h (\neq 0)$ satisfy, being determined with respect to
sufficiently large $x$. 
\begin{equation*}
h \in \mathbb{Z}, h \equiv 0 \mod 2,\quad \text{and either}\quad h = [x]
\quad\text{or}\quad 0<|h| \le x^{1/3}. \tag{10.15}\label{eq10.15} 
\end{equation*}

Then, as $x \to \infty$, holds
\begin{equation*}
\begin{cases}
C_v(x,h): = |\{ |h-p_1 p_2 p_3|: |h-p_1 p_2 p_3| =p, x^{1/v}\\ 
\le
p_1<x^{1/3} \le p_2 < p_3, p_1 p_2 p_3 \le x \} | \le \\ 
\le 4 c(v) \mathfrak{G} (h) \frac{x}{\log^2 x} \left \{ 1+O_v(\frac{\log \log
  x}{\log x}) \right \}. 
\end{cases}\tag{10.16} \label{eq10.16}
\end{equation*}
where $\mathfrak{G}$ is defined by \eqref{eq9.42} and
\begin{equation*}
c(v):= \int\limits^{1/3}_{1.v} \frac{\log (2 - 3 \alpha )}{\alpha
  (1-\alpha )} d \alpha \tag{10.17}\label{eq10.17} 
\end{equation*}
(Note that the $O$-constant in \eqref{eq10.16} depends at most on $v$.)

\begin{proof}%proof
Let us consider (the finite sequence)
\begin{equation*}
\mathscr{B} : = \{ b:b = pd, d \in \mathscr{D}, p \le \frac{x}{d}
\}.\tag{10.18}\label{eq10.18} 
\end{equation*}
where \pageoriginale
\begin{equation*}
\mathscr{D}:=\bigg\{ d:d= p_1 p_2, x^{1/v} \leq p_1 < x^{1/3} \leq p_2
\leq \sqrt{\frac{x}{p_1}} \bigg\}. \tag{10.19}\label{eq10.19}
\end{equation*}

We note that each $d \in \mathscr{D}$ (has a unique representation as
$p_1 p_2$ and) satisfies  
\begin{equation*}
x^{\frac{1}{3}} < x^{\frac{1}{3}+ \frac{1}{v}} \leq d = p_1 p_2 <
\sqrt{p_1 x} < x^{\frac{2}{3}}.\tag{10.20}\label{eq10.20} 
\end{equation*}
and so 
\begin{equation*}
|\mathscr{D}|< x^{\frac{3}{2}}. \tag{10.21}\label{eq10.21}
\end{equation*}

With a view to determine a suitable approximation to $|\mathscr{B}|$, we
use the formula (taking $\rho  = \dfrac{1}{v}$ and $\rho =
\dfrac{1}{3}$) 
\begin{equation*}
\sum_{x^\rho \leq p < y}\frac{1}{p}= \log (\frac{\log y}{\rho \log
  x})+ O (\frac{1}{\rho \log x}) \text{ for } y \geq x^{\rho},
\tag{10.22}\label{eq10.22} 
\end{equation*}
for Stieltjes integration to obtain (with $\rho = \dfrac{1}{3}$)
\begin{equation*}
\sum_{x^{\frac{1}{3}} \leq p_2 < \sqrt{\frac{x}{p_1}}} \frac{1}{p_2
  \log \frac{x/ p_1}{p_2}}= \int\limits^{x/p_1}_{\sqrt{x^{1/3}}}
\frac{d\eta}{\eta \log \eta \log \frac{x/p_1}{\eta}}+ \circ (\frac{1}{\log
  ^2 x}).\tag{10.23}\label{eq10.23} 
\end{equation*}

Further multiplying \eqref{eq10.23} by $\dfrac{1}{p_1}$ and summing over
$x^{1/v} \leq p_1 < x^{1/3}$ we get, by use of \eqref{eq10.22} (with $\rho
= 1/v$), 
\begin{align*}
\begin{cases}
\sum\limits_{d \in \mathscr{D}} \frac{1}{d \log \frac{x}{d}}  & =
\int\limits_{x^{1/v}}^{x^{1/3}} \frac{d \xi}{\xi \log \xi}
\int\limits_{x^{1/3}}^{(\frac{x}{\xi})^{1/2}} \frac{d
  \eta}{\eta \log \eta \log (\frac{x}{\xi\eta})} + O_v
(\frac{1}{\log^2 x})= \\ 
& = \frac{1}{\log x} \int\limits_{1/v}^{1/3} \frac{d \alpha}{\alpha}
\int\limits^{(1 - \alpha)/2}_{1/3} \frac{d
  \beta}{\beta(1-\alpha-\beta)}+ O_\nu (\frac{1}{\log^2 x})=\\ 
& = \frac{1}{\log x} \int\limits_{\frac{1}{v}}^{\frac{1}{3}}
\frac{\log(2-3\alpha)}{\alpha (1-\alpha)} d\alpha + O_v
(\frac{1}{\log^2 x})= \frac{c(v)}{\log x}+ O_v (\frac{1}{\log^2 x}),
\end{cases} \tag{10.24} \label{eq10.24}
\end{align*}
where we have put
\begin{equation*}
\xi = x^{\alpha}, \eta = x^{\beta}. \tag{10.25}\label{eq10.25}
\end{equation*}
and have also used the notation \eqref{eq10.17}. Thus since every $d
\in \mathscr{D}$ is $< x^{2/3}$ (cf. \eqref{eq10.20}),\pageoriginale
by \eqref{eq10.18} and the prime-number theorem (in a weak from) one has 
\begin{equation*}
|\mathscr{B}|= \sum_{d \in \mathscr{D}} \sum_{p \leq \frac{x}{d}}1 =
\sum_{d \in \mathscr{D}} \frac{x}{d \log (\frac{x}{d})}
(1+O(\frac{1}{\log x}))= \frac{c(v)x}{\log x}(1+O_v(\frac{1}{\log
  x})), \tag{10.26}\label{eq10.26} 
\end{equation*}
on using \eqref{eq10.24}. (Note that the formula \eqref{eq10.26} counts the
numbers in $\mathscr{B}$ according to the multiplicity of their
occurrence.) 
\end{proof}

Towards the estimate \eqref{eq10.17}, we naturally consider
\begin{equation*}
S(\mathscr{A}, \mathfrak{p}_h, z), \tag{10.27}\label{eq10.27}
 \end{equation*} 
 where
\begin{equation*}
\mathscr{A}:= \bigg\{ |h-p_1 p_2 p_3|: x^{1/v}\leq p_1 < x^{1/3} \leq
p_2< p_3, p_1, p_2,p_3 \leq x\bigg\}, \tag{10.28}\label{eq10.28} 
 \end{equation*} 
 \begin{equation*}
\mathfrak{p}_h : = \{ p:p \nmid h \}, \tag{10.29}\label{eq10.29}
 \end{equation*} 
 and (we make the choice)
 \begin{equation*}
z^2 = x^{\frac{1}{2}} \ell^{-18} \tag{10.30}\label{eq10.30}
  \end{equation*}  
with the abbreviation
 \begin{equation*}
\ell:L = \log x. \tag{10.31}\label{eq10.31}
 \end{equation*} 
 
Instead of $\mathscr{A}$ it is more convenient (cf. \eqref{eq10.60})
to work with 
\begin{equation*}
\mathscr{A}^* := \{ |h-p_1 p_2 p_3 |: x^{1/v} \leq p_1 < x^{1/3} \leq
p_2 < \sqrt{\frac{x}{p_1}}, p_1 p_2 p_3 \leq x
\}. \tag{10.32}\label{eq10.32} 
\end{equation*}

Now the primes $p_i (i = 1,2,3)$ occurring in $\mathscr{A}$ satisfy
$p_1 p^2_2 \leq x$ and so $p_2 < \sqrt{\dfrac{x}{p_1}}$, which shows
that $\mathscr{A}$ is contained in $\mathscr{A}^*$ (even with regard to
multiplicity of numbers in it). Therefore  
\begin{equation*}
S(\mathscr{A}, \mathfrak{p}_h , z) \leq S(\mathscr{A}^*, \mathfrak{p}_h
, z).\tag{10.33}\label{eq10.33}  
\end{equation*}

Note also that all the elements  of $\mathscr{A}^*$ have the same type
of representation from among\pageoriginale 
\begin{equation*}
h- p_1 p_2 p_3, p_1 p_2 p_3 -h \text{~ or~ } |h|+ p_1 p_2 p_3
\tag{10.34}\label{eq10.34} 
\end{equation*} 
according as $h =[x]$, $|h|\leq x^{1/3}$ with $h>0$ or $h<0$
respectively (cf. \eqref{eq10.15}) and hence, in particular, that the
multiplicity of a number in $\mathscr{A}^*$ is exactly the multiplicity
of the corresponding $p_1 p_2 p_3 $ in $\mathscr{B}$. 
 
Next, we prepare for an application of Theorem \ref{chap9-thm9.3} with
respect $\mathscr{A}^*$, $\mathfrak{p}_h$ and $z$. Comparing the
definitions of $\mathscr{A}^*$ and $\mathscr{B}$ (through $\mathscr{D}$) we
see that, for $(q,h)=1$ with $\mu (q) \neq 0$, by \eqref{eq1.53} (in the
notation of Chapter \ref{chap9}) one has  
\begin{align*}
\begin{cases}
|\mathscr{A}^*_q| & = \sum\limits_{\substack{b \in \mathscr{B} \\ b \equiv h
    \mod q}}1 = \frac{1}{\varphi(q)} \sum\limits_{\chi \mod q}
\bar{\chi}(h) \sum\limits_{b \in \mathscr{B}}\chi (b)=\\ 
& =\frac{1}{\varphi (q)} \sum\limits_{b \in \mathscr{B}}\chi_0 (b) +
\frac{1}{\varphi (q)} \sum\limits_{\substack{\chi \mod q \\ \chi \neq
    \chi_0}} \bar{\chi}(h) \sum\limits_{b \in \varphi} \chi(b),  
\end{cases}\tag{10.35}\label{eq10.35}
\end{align*}
in view of the remark involving \eqref{eq10.34}. From here we get
 \begin{equation*}
|R_q|= || A^*_q \Big|- \frac{1}{\varphi (q)}|\mathscr{B}| \Big| \leq
\frac{1}{\varphi  (q)} \sum_{\substack{\chi \mod q \\ \chi \neq
    \chi_0}} |\sum_{b \in   \mathscr{B}} \chi (b)|+
\frac{1}{\varphi(q)} \sum_{\substack{b \in 
    \mathscr{B} \\ (b.q)> 1}} 1 = :  R^*_q, \tag{10.36}\label{eq10.36} 
\end{equation*} 
say. In particular
\begin{equation*}
|\mathscr{A}^*| = |\mathscr{A}^*_1|= |\mathscr{B}|,
\tag{10.37}\label{eq10.37} 
 \end{equation*} 
 which is clear otherwise also. Therefore we make the choices
 \begin{equation*}
\omega (p) = \frac{p}{p-1} \text{for} p \in \mathfrak{p}_h
\tag{10.38}\label{eq10.38} 
\end{equation*} 
and, by \eqref{eq10.26},
\begin{equation*}
X= |\mathscr{B} | = c(v) \frac{x}{\log x} (1 + O_v (\frac{1}{\log
  x})). \tag{10.39}\label{eq10.39} 
\end{equation*} 
 
Now, $S(\mathscr{A}, \mathfrak{p}_h, z)$ counts all numbers of (the
set in) \eqref{eq10.16} which exceed $z$ and so (because of
\eqref{eq10.33}) applying Theorem \ref{chap9-thm9.3} with $K = h$ for
$\mathscr{A}^*$ it follows, by \eqref{eq9.41},\pageoriginale
\eqref{eq10.39} and \eqref{eq10.30}. 
 \begin{equation*}
C_v (x,h)\leq 4c (v) \mathscr{B} (h) \frac{x}{\log^2 x} \{ 1+O_v (\frac{\log
  \log x}{\log x})\}+ \sum_\circ,\tag{10.40}\label{eq10.40} 
 \end{equation*} 
 where (cf. \eqref{eq10.36})
 \begin{equation*}
\sum_\circ := z+ \sum_{\substack{q < z^2 \\ (q, h)=1}} \mu^2 (q) 3^{\nu
  (q)}|R_q|. \tag{10.41}\label{eq10.41}
 \end{equation*}
 
The rest of the proof concerns with an estimation of $\sum_0$ which
shows this contribution to \eqref{eq10.40} as being of the nature of
an error-term. The sum in \eqref{eq10.41} is, by Cauchy's inequality
and \eqref{eq10.36}, 
\begin{equation*}
\leq (\sum_{\substack{q < z^2 \\ (q, h)=1}} \mu^2 (q) 9^{\nu
  (q)}|R_q|)^{\frac{1}{2}} (\sum_{\substack{q < z^2 \\ (q, h)=1}}
\mu^2 (q) R^*_q)^{\frac{1}{2}}. \tag{10.42}\label{eq10.42} 
\end{equation*}  
  
Now trivially, from \eqref{eq10.35}, \eqref{eq10.26}, we have
(cf. \eqref{eq10.15}, cf.  also \eqref{eq10.11}) 
\begin{equation*}
|R_q| \ll_v (\frac{x+x^{1/3}}{q}+ \frac{x}{\varphi (q) \log x}) \ll_v
\frac{x}{q} \tag{10.43}\label{eq10.43} 
\end{equation*}
so that the first sum in \eqref{eq10.42} is, by Lemma
\ref{chap10-lem10.1}, 
\begin{equation*}
= O_v (x \log ^9 x). \tag{10.44}\label{eq10.44}
\end{equation*}
(Observe that the first term of the middle expression in \eqref{eq10.43} is
obtained by using the fact that the multiplicity of any member of
$\mathcal{A}^*$ is absolutely bounded (cf. \eqref{eq10.34}).) Next we deal
with the simpler part, of the second sum in \eqref{eq10.42}, arising from
the second term defining $R^*_q$. We have  for $q < x$ with $q$
squarefree,  
{\fontsize{10}{12}\selectfont
\begin{equation*}
\begin{cases}
\sum\limits_{\substack{b \in \mathscr{B} \\(b,q)>1}} 1 \leq
\sum\limits_{\substack{d \in \mathscr{D}\\ p|q}} 1 +
\sum\limits_{\substack{x^{1/v} \leq p_1 , p_2 < x^{\frac{1}{2}} \\ p_1 |q}}
\sum\limits_{p<\frac{x}{p_1 p_2}} 1 \leq \nu (q) (|\mathscr{D}|+ x ^{1
  - \frac{1}{v}} \sum\limits_{x^{1/v}\leq p_2 < x ^{\frac{1}{2}}}
\frac{1}{p_2})\\ \ll_v x^{1-1/v} \log x, 
\end{cases}\tag{10.45}\label{eq10.45}
\end{equation*}}
on using \eqref{eq10.21}, \eqref{eq10.22}, \eqref{eq10.15} and the
trivial estimate 
\begin{equation*}
\nu (q) \leq \frac{\log q}{\log 2} \; \forall q \in
\mathbb{N}. \tag{10.46}\label{eq10.46} 
\end{equation*}\pageoriginale

Hence \eqref{eq10.45} leads to the estimate, for the part of the sum
in \eqref{eq10.42} which in under consideration via \eqref{eq8.32} 
\begin{equation*}
\ll_v x^{1- \frac{1}{v}} \log x \sum_{d < z^2} \frac{\mu^2
  (q)}{\varphi (q)} \ll_v x^{1-\frac{1}{v}} \log ^2
x. \tag{10.47}\label{eq10.47}  
\end{equation*}

Now collecting together the bounds \eqref{eq10.42}, \eqref{eq10.44}
and \eqref{eq10.47} we see that because of the choice
\eqref{eq10.30}. 
\begin{equation*}
\sum_0^2 \ll_v z^2 + x \log^9 x(x ^{1-\frac{1}{v}} \log ^2 x + \sum_1)
\ll_v x^{2- \frac{1}{v}} \log^{11} x+x \log ^9 x. \sum_1,
\tag{10.48}\label{eq10.48}  
\end{equation*}
where (cf. \eqref{eq10.36})
\begin{equation*}
\sum_i : = \sum_{q < z^2} \frac{\mu^2 (q)}{\varphi (q)}
\sum_{\substack{\chi \mod q \\ \chi \neq \chi_0}} | \sum_{b \in
  \mathscr{B}} \chi(b)|. \tag{10.49}\label{eq10.49}  
\end{equation*}

Observe that $h$ has no longer a part to play in the sequel.

Transition to primitive characters in \eqref{eq10.49} yields
(cf. \eqref{eq1.59}), 
on writing $q=rf$ ($f$: conductor $\chi \mod q$) 
\begin{equation*}
\begin{cases}
\sum_1 = \sum\limits_{r <z^2} \frac{\mu^2 (r) }{\varphi (r)}
\sum\limits_{\substack{f< \frac{z^2}{r} \\ (r,f)=1}} \frac{\mu^2
  (f)}{\varphi (f)} \sum^*\limits_{\substack{\chi \mod f \\ \chi \neq
    \chi_0}} | \sum\limits_{\substack{b \in \mathscr{B} \\(b,r)=1}}
\chi(b) | \leq \\ \sum\limits_{r < z^2} \frac{\mu^2 (r)}{\varphi (r)}
\sum\limits_{f < z^2}\frac{\mu^2 (f)}{\varphi
  (f)}\sum^*\limits_{\substack{\chi \mod f \\ \chi \neq \chi_0}} |
\sum\limits_{\substack{b \in \mathscr{B} \\(b,r)=1}} \chi (b)| =
\sum\limits_{r < z^2} \frac{\mu^2 (r)}{\varphi (r)} \sum_2 (r),  
\end{cases}\tag{10.50}\label{eq10.50}
\end{equation*}
where (on replacing $f$ by $q$)
\begin{equation*}
\sum_2 (r) : = \sum_{q < z^2} \frac{\mu^2 (q)}{\varphi (q)}
\sum_{\substack{\chi \mod q \\ \chi \neq \chi_0}} | \sum_{\substack{b
    \in \mathscr{B}\\(b,r)=1}} \chi (b)|.\tag{10.51}\label{eq10.51} 
\end{equation*}
(In the remaining part we also use the abbreviation \eqref{eq10.31}
wherever convenient.)  

By the Siegel-Walfisz theorem (cf. \eqref{eq8.2}) we have for any
character $\chi \neq \chi_0 \mod q$,\pageoriginale in view of
\eqref{eq1.44} and \eqref{eq10.46}, 
\begin{gather*}
 \big|\sum_{\substack{p<y \\ p \not \mid r}} \chi (p) \big| \leq
 \frac{\log r}{\log 2}+ \big| \sum_{\ell =1}^{q} \chi (\ell) \pi (y;q,
 \ell ) \big| \ll \frac{\log r}{\log 2} + y. \varphi (q) \ell^{-3g}\\
 \text{ uniformly for } q \ll \log^g y \text{ as } y \to \infty,
 \tag{10.52}\label{eq10.52} 
\end{gather*}
so that (with $g = 17$)
{\fontsize{10}{12}\selectfont
\begin{equation*}
\begin{cases}
\sum\limits_{q \leq \ell ^{17}} \frac{\mu^2 (q)}{\varphi (q)}
\sum\limits_{\substack{\chi \mod q \\ \chi \neq \chi_0}} \bigg|
\sum\limits_{\substack{b \in \mathscr{B} \\(b,r)=1}} \chi (b) \bigg| =
\sum\limits_{q \leq \ell^{17}} \frac{\mu^2 (q)}{\varphi
  (q)}\sum\limits^*_{\substack{\chi \mod q \\ \chi \neq \chi_0}} \bigg|
\sum\limits_{\substack{d \in \mathscr{D} \\(d,r)=1}} \chi (d)
\sum\limits_{\substack{p \leq x/d \\ p \nmid r}d} \chi (p)| \leq \\ 
\ll \sum\limits_{q \leq \ell ^{17}} \sum\limits_{d \in \mathscr{D}}
\frac{x}{d} \varphi (q) \ell^{-51} \ll_v x \ell^{-16}.  
\end{cases}\tag{10.53}\label{eq10.53}
\end{equation*}}

Hence 
\begin{equation*}
\sum_2 (r) \ll_v \sum_3 (r) + x \ell^{-16}, \tag{10.54}\label{eq10.54}
\end{equation*}
where
\begin{equation*}
\sum_3 (r) : = \sum_{\ell^{17} < q<z^2} \frac{\mu^2 (q)}{\varphi (q)}
\sum^*_{\chi \mod q} \bigg|\sum_{\substack{b \in \mathscr{B} \\(b,r)=1}}
\chi (b)\bigg|.\tag{10.55}\label{eq10.55} 
\end{equation*}

For an estimation of \eqref{eq10.55} we use contour integration and the
hybrid sieve. To this and we put 
\begin{equation*}
T=x^3, \tag{10.56}\label{eq10.56}
\end{equation*}
as well as
\begin{equation*}
a = 1+ \ell^{-1}. \tag{10.57}\label{eq10.57}
\end{equation*}

Further let
\begin{equation*}
\ell^{17} < w \leq z^2 \tag{10.58}\label{eq10.58}
\end{equation*}
and also note that the \textit{supposition}
\begin{equation*}
x=[x]+ \frac{1}{2} \tag{10.59}\label{eq10.59}
\end{equation*}
involves no loss of generality. We also introduce the Dirichlet series 
\begin{gather*}
P=P_r (s, \chi ) := \sum_{\substack{p \leq w^2 \\ p \nmid r}}
\frac{\chi (p)}{p^s}, Q=Q_r (s ,\chi ):=\sum_{\substack{p > w^2 \\ p
    \not\mid r}} \frac{\chi (p)}{p^s},\\
 D=D_r (s ,\chi ) :=
\sum_{\substack{d \in \mathscr{D} \\ (d, r)=1}} \frac{\chi (d)}{d^s}
(Re s > 1). \tag{10.60}\label{eq10.60} 
\end{gather*}\pageoriginale

Now, by \eqref{eq10.19} and a well-known formula (using
\eqref{eq10.59}) 
\begin{equation*}
\sum_{\substack{b \in \mathscr{B} \\ (b , r)=1}} \chi (d) = \frac{1}{2 \pi
  i} \int\limits_{a-iT}^{a+iT} (P + Q)D(s) \frac{x^s}{s} ds+ O(\frac{x
  \log x}{T}). \tag{10.61}\label{eq10.61} 
\end{equation*}

Splitting the integral here into two parts, corresponding to $(PD)(s)$
and $(QD)(s)$ respectively, and shifting the line of integration of
the former only to $\sigma = \dfrac{1}{2}$ we obtain, on using
\eqref{eq10.57}. 
{\fontsize{10}{12}\selectfont
\begin{equation*}
\begin{cases}
\sum\limits_{\substack{b \in \mathscr{B} \\ (b,r)=1}} \chi (b) \ll
x^{\frac{1}{2}} \int\limits_{-T}^{T} | (PD)(\frac{1}{2} + it, \chi)|
\frac{dt}{1+|t|} + \frac{x}{T} \int\limits_{a}^{1/2}
|(PD)(\sigma \pm iT, \chi)| d \sigma + \\ +  x
\int\limits_{-T}^{T}|(QD)(a + it , \chi )| \frac{dt}{1+ |t|} + \frac{x
  \log x}{T}.  
\end{cases}\tag{10.62}\label{eq10.62}
\end{equation*}}

For the second term on the right-hand side the crude estimate
\begin{equation*}
|(PD) (\sigma \pm it , \chi )| \leq \sum_{p \leq w^2 } \sum_{d \in
  \mathscr{D}} 1 \leq w^2 |\mathscr{D}| \leq x^2 \text{for}
\frac{1}{2} \leq \sigma \leq a, \tag{10.63}\label{eq10.63} 
\end{equation*}
obtained from \eqref{eq10.61}, \eqref{eq10.59}, \eqref{eq10.31} and
\eqref{eq10.22}, suffices. 

Towards an estimation of the remaining terms we introduce a notation
(for convenience of description). For an arbitrary function $f (s,
\chi )$, $s = \sigma + it$ we  set (with respect to $r$, $w$, $T$ and $z$ as
before) 
\begin{equation*}
M (\sigma , f) : = \sum_{\substack{q < w \\ (q, r)=1}}
\frac{q}{\varphi (q)} \sum^*_{\chi \mod q} \int\limits_{-T}^{T} |f
(\sigma + it , \chi )| \frac{dt}{1+|t|}, \tag{10.64}\label{eq10.64} 
\end{equation*}
and observe that, by Cauchy-Schwarz inequality, for any two o(such)
functions $I_1$ and $f_2$ one has 
\begin{equation*}
M(\sigma,  f_1 f_2)\leq (M (\sigma, f_1^2 ))^{\frac{1}{2}} (M (\sigma,
f^2_2))^{-\frac{1}{2}}. \tag{10.65}\label{eq10.65} 
\end{equation*}

We would also need the estimate (valid with an absolute $\ll$-constant)
{\fontsize{9pt}{11pt}\selectfont
\begin{equation*}
M(\sigma, f^2)\ll \sum_{n=1}^{\infty}(w^2 \log x + n)|a_n|^2 n
^{-2\sigma}, \text{for} f=f(s,\chi )= \sum_{n=1}^{\infty}
\frac{a_n}{n^s}\chi(n) \text{ if } \sum_{n=1}^{\infty} \frac{|a_n|}{n ^\sigma}
< \infty, \tag{10.66}\label{eq10.66} 
\end{equation*}}\relax\pageoriginale
which is obtained by partial integration in Theorem \ref{chap5-thm5.1}
and using\break \eqref{eq10.56}. 

By \eqref{eq10.62}, \eqref{eq10.64}, \eqref{eq10.55} and
\eqref{eq10.56} there holds 
\begin{gather*}
\sum_{\ell^{17}<q<w} \frac{q}{\varphi (q)} \sum^*_{\chi \mod
  q}|\sum_{\substack{b \in \mathscr{B} \\ (b,r)=1}} \chi(b) | \ll (x
M(\frac{1}{2}, P^2 ) M (\frac{1}{2}, D^2))^{\frac{1}{2}}\\
+ x (M (a ,
Q^2) M (a, D^2 ))^{\frac{1}{2}}+w^2. \tag{10.67}\label{eq10.67} 
\end{gather*}

Now we apply \eqref{eq10.66} to the four $M$'s occurring in
\eqref{eq10.67} to obtain in succession (with absolute $\ll$-constants) 
\begin{equation*}
M(\frac{1}{2} , P^2)\ll \sum_{p \leq w^2} (w^2 \log x + p)p^{-1} \ll
w^2 \log^2 x, \tag{10.68}\label{eq10.68} 
\end{equation*}
because of \eqref{eq10.60}, \eqref{eq10.58} and \eqref{eq10.30}),
\begin{equation*}
M(\frac{1}{2}, D^2) \ll \sum_{d \in \mathscr{D}} (w^2 \log x +d)d^{-1} \ll
w^2 \log^2 x+ x^{\frac{2}{3}}, \tag{10.69}\label{eq10.69} 
\end{equation*}
using in addition \eqref{eq10.21} further (cf. \eqref{eq10.57})
\begin{equation*}
M(a, Q^2) \ll \sum_{p > w^2} (w^2 \log x +p)p^{-2a} \log x,
\tag{10.70}\label{eq10.70} 
\end{equation*}
and lastly 
\begin{equation*}
M(a, D^2) \ll \sum_{d \in \mathscr{D}} (w^2 \log x +d)d^{-2a} \ll w^2
(\log^2 x) x^{-\frac{1}{3}}+ \log x. \tag{10.71}\label{eq10.71} 
\end{equation*}

Using the four estimates \eqref{eq10.68}--\eqref{eq10.71} in
\eqref{eq10.67} we arrive (cf. \eqref{eq10.58}, \eqref{eq10.30}) 
\begin{equation*}
\begin{cases}
\sum\limits_{\ell^{17}< q<w} \frac{q}{\varphi (q)} \sum^*\limits_{\chi
  \mod q} | \sum\limits_{\substack{b \in \mathscr{B} \\ (b,r)=1}} \chi(b)
| \ll (x w^2 \log^2 x (w^2 \log^2 x+x^{\frac{2}{3}}))^{\frac{1}{2}}\\
+ x (\log^2 x (w^2 x ^{-\frac{1}{3}}))^{\frac{1}{2}} + w^2 \ll
x^{\frac{1}{2}} w \log x (w \log   + x + x^{\frac{1}{3}})\\
+ x \log x(w
x ^{-\frac{1}{6}}+1 ) + w^2 \ll x \log x + wx^{5}{6} \log x + w^2
x^{\frac{1}{2}} \log ^2 x,  
\end{cases}\tag{10.72}\label{eq10.72}
\end{equation*}
valid uniformly in $r$. This gives, by partial summation and
\eqref{eq10.30}, 
\begin{equation*}
\sum_{\ell^{17}<q<z^2} \frac{1}{\varphi (q)} \sum^*_{\chi \mod q} |
\sum_{\substack{b \in \mathscr{B} \\ (b,r)=1}} \chi (b) | \ll x
\ell^{-16}+ x^{\frac{5}{6}} \ell^2 \ll x \ell^{-16} \tag{10.73}\label{eq10.73} 
\end{equation*}\pageoriginale
uniformly in $r$.

Note that the left-hand side expression in \eqref{eq10.73} majorizes $\sum_3
(r)$ of \eqref{eq10.55}. Therefore, it  follows from \eqref{eq10.54} that 
\begin{equation*}
\sum_2 (r) \ll_v x \ell^{-16}, \tag{10.74}\label{eq10.74}
\end{equation*}
which when used in \eqref{eq10.50} yields (by means of \eqref{eq8.32})
the estimate 
\begin{equation*}
\sum_1 \ll_v x \ell^{-15}. \tag{10.75}\label{eq10.75}
\end{equation*}

Thus we obtain, on using \eqref{eq10.75} in \eqref{eq10.48},
\begin{equation*}
\sum_0 \ll_v x (\log x)^{-3},\tag{10.76}\label{eq10.76} 
\end{equation*}
from which in view of \eqref{eq10.40}, follows the estimate in
\eqref{eq10.16}. This completes the proof if Theorem
\ref{chap10-thm10.3}.  

\begin{center}
\textbf{NOTES}
\end{center}

\ref{chap10-thm10.1}:~
The results of this theorem were proved first under the assumption of
the generalized Riemann hypothesis by Wang \cite{key2}, and the
unconditional proof of these results (with an application to have
\eqref{eq6.77} with a $ \theta < \dfrac{1}{2}$) is due to Bombieri and
Davenport \cite{key1}. The main terms of \eqref{eq10.1} and
\eqref{eq10.2} are 4 times 
the conjectured asymptotic formulae (without error terms), for the
respective problems, of Hardy and Littlehood. In this context it is
significant to mention that Montgomery has pointed out (in
correspondence) that a decrease of the factor 4 here any constant
$c<4$ would have the same consequences regarding the Siegel-zeros as
has been remarked in connection with the Brun-Titchmarsh theorem
(cf. Notes of Chapter \ref{chap8}, under \eqref{eq8.21}). 


Theorem \ref{chap10-thm10.2}:~
See\pageoriginale Klimov \cite{key1}, \cite{key2}, and Halberstam and
Richert \cite{key1}  

(Theorem 3.12).  

\eqref{eq10.8}:~ The formula \eqref{eq10.8} though not needed in this
full force for Lemma \ref{chap10-lem10.1}, is of much use in later chapters. 

Lemma \ref{chap10-lem10.2}:~
cf. Halberstam and Richert \cite{key1} (Lemma 3.5)

Theorem \ref{chap10-thm10.3}:~
This result is the most essential part of Chen's proof of his famous
theorem with respect to Goldach's conjecture. For $v=10$ it
occurs in Chen \cite{key1}, Halberstam and Richert \cite{key1}
(Chapter \ref{chap11}, with 
some simplifications due to P.M. Ross) (with the weights $\wedge (n)$),
and in Ross \cite{key1}. We shall use Theorem \ref{chap10-thm10.3} 
with $v=8$ in Chapter
\ref{chap13} and thereby obtaining the advantage of dealing with elementary
functions in connection with the lower bound estimation via Selberg's
sieve. (cf. Notes of Chapter \ref{chap13}, preceding \eqref{eq13.27}.) 

\eqref{eq10.17}:~ For later use we obtain an estimate for
$c(v)$. First, note that the function $f(x)=x^2 -2x \log x$ is
increasing for $x \geq 1$ (since the derivative $f'(x) = 2(x-1-\log x)
\geq 0$ for $x \geq 1$). So $f(x) \geq f(1)$, for $x \geq 1$, which
means that $x^2 -2x \log x \geq 1$; i.e., 
\begin{equation*}
\log x \leq \frac{(x-1)(x+1)}{2x} \text{ for } x \geq
1. \tag{10.77}\label{eq10.77} 
\end{equation*}
(However, observe also that \eqref{eq10.77} follows from 
\begin{equation*}
\int\limits_{a}^{b} F(y) dy \leq \frac{(b-a)}{2} (F(b) + F(a))
\text{ for any convex function } F, b \geq a,\tag*{(10.77)$'$}
\end{equation*}
on taking $F(y)=
y^{-1}$ and $a=1$, $b=x(\geq 1)$.) Hence  \eqref{eq10.77},
  with $x = 2-3 \alpha$, gives  
\begin{equation*}
\frac{\log (2-3 \alpha)}{\alpha (1-\alpha)} \leq \frac{1-3\alpha}{2}
\frac{3}{\alpha (2-3 \alpha)}= \frac{3}{4} (\frac{1}{\alpha}-
\frac{3}{2-3 \alpha}) \text{ for } 0 < \alpha \leq
\frac{1}{3},\tag{10.78}\label{eq10.78} 
\end{equation*}
from which, we obtain
\begin{equation*}
c(v)= \int\limits_{\frac{1}{V}}^{\frac{1}{3}}\frac{\log (2-3
  \alpha)}{\alpha (1-\alpha)} d \alpha \leq \frac{3}{4} \log (\alpha
(2-3 \alpha)) |^{\alpha = \frac{1}{3}}_{\alpha = \frac{1}{V}} =
\frac{3}{4} \log (\frac{v^2}{3(2v -3)}). \tag{10.79}\label{eq10.79} 
\end{equation*}

\eqref{eq10.61}:~ Titchmarsh,\pageoriginale $E.C$. \textit{The theory
  of the Riemann zeta function} (Oxford), 

Lemma 3.12.

The effective way of treating the remainder terms by an explicit use
of analytical methods. as done here subsequent to \eqref{eq10.61}. is of
recent origin. It occurs in the papers Barban and Vehow \cite{key1} (see
Motohashi \cite{key10}) Hooley \cite{key2}, \cite{key5}. Huxley
\cite{key5}, Chen \cite{key1}. Motohashi \cite{key8}. Halberstam and
Richert \cite{key1}. Goldfeld \cite{key4}, 
and Ross \cite{key1}. Excepting the first of these paper all the others
employ this method for the purpose of some applications only. In that
paper Barban and Vehov \cite{key1} sketch a proof (- a rigorous proof was
given by Motohashi \cite{key10}-) of the following surprisingly uniform
result: 

If $x > z$, $\log z \gg \log z_1$, $z_1 \ge z$ and
\begin{equation*}
\lambda_d : =
\begin{cases}
\mu(d) & if d \le z \\
\mu(d) \frac{\log (\frac{Z_i}{d})}{\log (\frac{Z_i}{d})}& if z < d \le z_1.
\end{cases}\tag{10.80}\label{eq10.80}
\end{equation*}
then we have
\begin{equation*}
\sum_{1 \le n \le x} (\sum_{\substack{d|n \\ d \ge z_1 }} \lambda_ d)^2
\frac{x}{\log (\frac{z_1}{z})}.\tag{10.81}\label{eq10.81} 
\end{equation*}
