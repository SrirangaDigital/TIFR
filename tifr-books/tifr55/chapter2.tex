
\chapter{The Large Sieve}\label{chap2}

WE\pageoriginale START by recalling the notation introduction in
Chapter \ref{chap0} (cf. \eqref{eq0.1} and \eqref{eq0.52}). Let
$\gamma$ be a set 
consisting of $S$ integers from an interval of length $N$:  
\begin{equation*}
\gamma \subset (M,M+N], M \in \mathbb{Z}, N \in \mathbb{N}, S: =
  |\gamma |. \tag{2.1}\label{eq2.1} 
\end{equation*}
and let
\begin{equation*}
T(x): = \sum_{n \in \gamma} e(nx). \tag{2.2}\label{eq2.2}
\end{equation*}

Then, as has been indicated in Chapter \ref{chap0}, the large sieve in
its basic form is concerned with the estimation, of the expression  
\begin{equation*}
\sum_{q \leq Q} \mathop{\sum{}'}\limits_{\ell =1}^q \bigg|T( \frac{\ell}{q})
\bigg |^2 
\tag{2.3}\label{eq2.3} 
\end{equation*}
in terms of $N$, $Q$ and $S$, of the form
\begin{equation*}
\sum_{q \leq Q} \mathop{\sum{}'}\limits_{\ell =1}^q \bigg |T (
\frac{\ell}{q} \bigg|^2 \leq C(N,Q)S. \tag{2.4}\label{eq2.4}   
\end{equation*}

The simplest approach to a bound of the type in \eqref{eq2.4} is now due to
Gallagher \cite{key1}. Gallagher's starting point is the following lemma
which occurs in the earlier work of Hardy and Littlewood, and of
Sobolev (for several variables): 

\setcounter{section}{2}
\setcounter{lemma}{0}
\begin{lemma}\label{chap2-lem2.1} %lem 2.1
Let $u$ and $\delta( > 0)$ be real numbers, and let $f(x)$
(complex-valued) be continuous on $\bigg[u - \dfrac{\delta}{2}, u+ 
  \dfrac{\delta}{2} \bigg]$ with a continuous derivative in $( u -
\dfrac{\delta}{2}, u + \dfrac{\delta}{2})$. Then  
\begin{equation*}
|f(u)|^2 \leq \int\limits_{u- \frac{\delta}{2}}^{u+
  \frac{\delta}{2}}|f(x) f'(x)| dx + \delta^{-1} \int\limits_{u-
  \frac{\delta}{2}}^{u+ \frac{\delta}{2}} |f(x)|^2 dx. \tag{2.5}\label{eq2.5} 
\end{equation*}
\end{lemma}

\begin{proof} %pro
Put
\begin{equation*}
F(x)=f^2(x). \tag{2.6}\label{eq2.6}
\end{equation*}

Then, by partial integration, we find the identity 
\begin{equation*}
\begin{cases}
F(u) &= \delta^{-1} \int\limits_{u- \frac{\delta}{2}}^{u} (x(u- \frac{\delta}{2})) F' (x) dx+ \delta^{-1} \int\limits_{u}^{u + \frac{\delta}{2}}(x-(u+ \frac{\delta}{2})) F' (x) dx +\\
& + \delta^{-1} \int\limits^{u+ \frac{\delta}{2}}_{u- \frac{\delta}{2}} F (x) dx.
\end{cases}\tag{2.7}\label{eq2.7}
\end{equation*}\pageoriginale

Hence 
\begin{equation*}
|F (u)| \leq \frac{1}{2} \int\limits^{u + \frac{\delta}{2}}_{u-
  \frac{\delta}{2}}|F'(x)| dx + \delta^{-1} \int\limits^{u+
  \frac{\delta}{2}}_{u- \frac{\delta}{2}} |F(x)| dx, \tag{2.8}\label{eq2.8} 
\end{equation*}
which gives \eqref{eq2.5} on using \eqref{eq2.6}.

Gallagher's use of the above lemma enables one to obtain a bound of
the type in \eqref{eq2.4} when the expression on the left there is extended
in two directions; namely, when the terms of the sum in \eqref{eq2.2} have
weights (arbitrary complex numbers) $a_n$ and the set of (fractional)
points $\dfrac{\ell}{q}$ in \eqref{eq2.4} are replaced by a finite set of
real numbers which are distinct modulo $1$. 
\end{proof}

\setcounter{section}{2}
\begin{theorem}\label{chap2-thm2.1} %the 2.1
For any complex numbers $a_n$, $M < n\leq M + N$, set 
\begin{equation*}
U(x):= \sum_{M < n \leq M + N} a_n e(nx). \tag{2.9}\label{eq2.9}
\end{equation*}
\end{theorem}

Let $x_1, \ldots , x_R$ be real numbers which are distinct $\mod 1$ and put 
\begin{equation*}
\delta : = \min_{\substack{r,s\\ r \neq s}} || x_r -x_s || , \text{ if
  } R \geq 2, || x || : = \min_{k \in \mathbb{Z}} |x-k|; \delta := \infty
, \text{ if }  R= 1. \tag{2.10}\label{eq2.10} 
\end{equation*}

Then
\begin{equation*}
\sum_{r=1}^R |U (x_r) |^2 \leq ( \pi N+ \delta^{-1}) \sum_{M < n \leq
  M+N}|a_n|^2. \tag{2.11}\label{eq2.11}
\end{equation*}

We shall present the proof of this theorem in all its detail. For some
proofs of large sieve inequalities, in particular, for the proof of
Theorem \ref{chap2-thm2.1} which is based on Lemma \ref{chap2-lem2.1} 
(cf. also the proof of \eqref{eq2.95}), it is a advantageous
(cf. \eqref{eq2.30})\pageoriginale  to consider a shifted
interval (usually, an interval symmetric about the point zero)
instead of the interval $(M + M + N]$. Then the general case is easily
  derived by reversing the shifting procedure. The idea of the last
  step is contained in the following
  
\setcounter{section}{2}
\begin{lemma}\label{chap2-lem2.2} %lem 2.2
Let $N \in \mathbb{N}$, let $x_1, \ldots, x_R (R \geq 2)$ be real
numbers which are district $\mod 1$ and put 
\begin{equation*}
\delta := \min_{\substack{r,s\\ r \neq s}} || x_r -x_s
||. \tag{2.12}\label{eq2.12} 
\end{equation*}

Set
\begin{equation*}
V(x): = \sum_{-\frac{N}{2}< m \leq \frac{N}{2}} b_{m}e (mx), b_m \in
\mathbb{C}. \tag{2.13}\label{eq2.13} 
\end{equation*}
\end{lemma}

Then the inequality
\begin{equation*}
\sum_{r=1}^{R}| V (x_r)|^2 \leq \Delta (N, \delta )
\sum_{-\frac{N}{2}<m \leq \frac{N}{2}} |b_m|^2,\quad \forall b_m \in
\mathbb{C} \tag{2.14} \label{eq2.14} 
\end{equation*}
with some (positive) function $\Delta (N, \delta)$ depending only on
$N$ and $\delta$ implies that for any $M \in \mathbb{Z}$, 
\begin{equation*}
\sum_{r=1}^{R}| U (x_r)|^2 \leq \Delta (N, \delta ) \sum_{M < n \leq M
  + N} |a_n|^2 \; \forall a_n \in   \mathbb{C} \tag{2.15}\label{eq2.15} 
\end{equation*}
where 
\begin{equation*}
U(x): = \sum_{M< n \leq M+N} a_n e(nx).\tag{2.16}\label{eq2.16}
\end{equation*}

\noindent
{\bf Proof of lemma \ref{chap2-lem2.2}.}~
If $U(x)$ is defined by \eqref{eq2.16}, let us consider
{\fontsize{10pt}{12pt}\selectfont
\begin{equation*}
\begin{cases}
V(x):&= e(- (M+\left[ \frac{N+1}{2} \right] )x)U(x) = \sum\limits_{M< n \leq
  M+N} a_ne((n-M- \left[ \frac{N+1}{2} \right] )x) =\\ 
&= \sum\limits_{ -\left[ \frac{N+1}{2} \right] < m \leq - \left[
  \frac{N+1}{2} \right] +N} b_m e(mx)= \sum\limits_{- \frac{N}{2} < m \leq
  \frac{N}{2}} b_m e(mx), 
\end{cases}\tag{2.17} \label{eq2.17}
\end{equation*}}\relax
where\pageoriginale
\begin{equation*}
b_m =a_{m+M+ \left[ \frac{N+1}{2} \right]}, -\frac{N}{2} < m \leq
\frac{N}{2}. \tag{2.18}\label{eq2.18} 
\end{equation*}

Now $| U(x)| = |V(x)|$ for all real $x$ from which we easily see that
\eqref{eq2.15} is an immediate consequence of \eqref{eq2.14}. 

Now we are in a position to prove Theorem \ref{chap2-thm2.1}.

\medskip
\noindent
{\bf Proof of Theorem \ref{chap2-thm2.1}.}~
To start with we dispose off the case $R=1$. This is easily done
in view of \eqref{eq2.10} and Cauchy's inequality: 
\begin{equation*}
|U(x_1)|^2 \leq N \sum_{M <n \leq M+N} |a_n|^2. \tag{2.19}\label{eq2.19}
\end{equation*}

So we assume that 
\begin{equation*}
R \geq 2 \tag{2.20}\label{eq2.20}
\end{equation*}
and further, because of Lemma \ref{chap2-lem2.2}, we shall consider
$V$ instead of $U$. Since $V(x)$ is of period 1 we can also suppose
that  
\begin{equation*}
0 \leq x_1 < x_2 < \ldots < x_R < 1. \tag{2.21}\label{eq2.21}
\end{equation*}

By the pigeon-hole principle follows easily that 
\begin{equation*}
\delta \leq \frac{1}{R}. \tag{2.22}\label{eq2.22}
\end{equation*}

Now, from Lemma \ref{chap2-lem2.1} with $f = V$ and $u = x_r$, one has 
\begin{equation*}
|V(x_r)|^2 \leq \int\limits_{I_r}|V(x)V'(x)|dx + \delta^{-1}
\int\limits_{I_r}|V(x)|^2 dx, \tag{2.23}\label{eq2.23} 
\end{equation*}
where
\begin{equation*}
I_r : =[x_r -\frac{\delta}{2}, x_r +
  \frac{\delta}{2}]. \tag{2.24}\label{eq2.24}  
\end{equation*}

By \eqref{eq2.12}, we have for $r \neq s$
\begin{equation*}
\delta \leq || x_r -x_s || \leq |x_r - x_s|, \tag{2.25}\label{eq2.25}
\end{equation*}
so\pageoriginale that our intervals $I_r$ do not overlap and their
total length, i.e., length of  
\begin{equation*}
\bigcup_{r=1}^R I_r, \tag{2.26}\label{eq2.26}
\end{equation*}
equals
\begin{equation*}
\delta R \leq 1, \tag{2.27}\label{eq2.27}
\end{equation*}
on recalling \eqref{eq2.22}. Summing now over $r$ in \eqref{eq2.23}
(and using that both $V(x)$ and $V'(x)$ have period 1) we can replace the
integration on the right over \eqref{eq2.26} by $\int\limits_{0}^1$. 
Thus we get, by employing Schwarz's inequality.
\begin{equation*}
\sum_{r=1}^R |V(x_r)|^2 \leq ( \int\limits_{0}^1 |V(x)|^2 dx)^{1/2}
(\int\limits_{0}^1 |V'(x)|^2 dx)^{\frac{1}{2}} + \delta^{-1}
\int\limits_{0}^1 |V(x)|^2 dx. \tag{2.28}\label{eq2.28} 
\end{equation*} 

Now, it follows from \eqref{eq1.28} that 
\begin{equation*}
\int\limits_{0}^1 |V(x)|^2 dx = \sum_{- \frac{N}{2}< m \leq
  \frac{N}{2}} |b_m|^2 \tag{2.29}\label{eq2.29} 
\end{equation*}
and also
\begin{equation*}
\int\limits_{0}^1 |V'(x)|^2 dx =\sum_{- \frac{N}{2}< m \le
  \frac{N}{2}} |2b_m \pi_m|^2 \le N^2 \pi^2 \sum_{- \frac{N}{2}< m \le
  \frac{N}{2}}|b_m|^2. \tag{2.30}\label{eq2.30} 
\end{equation*}

Hence \eqref{eq2.28} yields
\begin{equation*}
\sum_{r=1}^R|V(x_r)|^2 \le (\pi N+ \delta^{-1} )\sum_{- \frac{N}{2}< m
  \le \frac{N}{2}}|b_m|^2. \tag{2.31}\label{eq2.31} 
\end{equation*}

This is \eqref{eq2.14} with 
\begin{equation*}
\Delta (N, \delta) = \pi N + \delta^{-1}. \tag{2.32}\label{eq2.32}
\end{equation*}

Therefore, by lemma \ref{chap2-lem2.2}, \eqref{eq2.15} with
\eqref{eq2.32} gives \eqref{eq2.11} thereby 
completing the proof of Theorem \ref{chap2-thm2.1}. 

Now we discuss the result of Theorem \ref{chap2-thm2.1}. Due to the
presence of the 
factor\pageoriginale $\delta^{-1}$ the efficiency of \eqref{eq2.11},
as of other large 
sieve inequalities that we shall consider, depends on the information
as to how `well-spaced' (in the sense of \eqref{eq2.10} the points $x_r$
are. 

The simplest case is, for any $(2 \le) R \in \mathbb{N}$. with 
\begin{equation*}
x_r = \frac{r}{R},\quad 1\leq r \leq R \tag{2.33}\label{eq2.33}
\end{equation*} 
so that 
\begin{equation*}
\delta= \frac{1}{R}. \tag{2.34}\label{eq2.34}
\end{equation*}

Now Theorem \ref{chap2-thm2.1} gives 
\begin{equation*}
\sum_{r=1}^R \bigg| U(\frac{r}{R}) \bigg|^2 \le (\pi N+R ) \sum_{M< n
  \le M+N}|a_n|^2 \tag{2.35}\label{eq2.35}  
\end{equation*}

In the more interesting case 
\begin{equation*}
x_r = \frac{\ell}{q}, \; 1 \le \ell \le q \le Q, ~ (\ell ,q)=1,
\tag{2.36}\label{eq2.36} 
\end{equation*}
the set of Farey fractions of ord\'er $Q$, the points  are not quite so
well-spaced. We find, on assuming that $Q \geq 2$, for any two
distinct Farey fractions in \eqref{eq2.36} 
\begin{equation*}
|| \frac{\ell}{q}- \frac{\ell'}{q'}||= \bigg| \bigg| \frac{\ell q' - q
  \ell'}{qq'}\bigg| \bigg| \geq \frac{1}{qq'} \ge \frac{1}{Q^2},
\tag{2.37}\label{eq2.37} 
\end{equation*}
i.e.
\begin{equation*}
\delta \ge \frac{1}{Q^2}. \tag{2.38}\label{eq2.38}
\end{equation*}

Hence we get from Theorem \ref{chap2-thm2.1} and \eqref{eq2.19} the following 

\begin{theorem}\label{chap2-thm2.2} %the 2.2
Let $M \in \mathbb{Z}$, $N \in \mathbb{N}$ and let $a_n(M < n \le M+N)$
be arbitrary complex numbers. Then  
\begin{equation*}
\sum_{q \leq Q} \sum_{\ell = 1}^{q}{'}| U(\frac{\ell}{q}) |^2 \leq
(\pi N+Q^2) \sum_{M< n \leq M+N}|a_n|^2 \; \forall Q \in \mathbb{N},
\tag{2.39}\label{eq2.39} 
\end{equation*}
where $U(x)$ is defined through \eqref{eq2.9}.
\end{theorem}

Regarding\pageoriginale the quality of the preceding results we make
the following 
remarks. Under the assumptions of Theorem \ref{chap2-thm2.1}, recall
the estimation \eqref{eq2.15}: 
\begin{equation*}
\sum_{r=1}^{R}|U(x_r)|^2 \le \Delta (N, \delta ) \sum_{M < n \le
  M+N}|a_n|^2. \tag{2.40}\label{eq2.40} 
\end{equation*}

In some application the bound in \eqref{eq2.11} is quite
satisfactory. However, if we are inserted in better estimates we need
apply more effective tools in the proof.  

In order to see some necessary conditions for having a general result
of the form \eqref{eq2.40} first note that if $x_1=0$ and $a_n=1$ for $M < n
\le M+N$ then the left hand side is at least $|U(x_1)|^2 =N^2$, so that 
\begin{equation*}
\Delta (N, \delta) \ge N.\tag{2.41}\label{eq2.41}
\end{equation*}

Next, since $\delta$ is invariant under a translation of the set $x_1,
\ldots, x_R$ by any given $x \in \mathbb{R}$, \eqref{eq2.40} would also
simply that 
\begin{equation*}
\sum_{r=1}^{R}|U(x_r +x)|^2 \leq \Delta (N,\delta ) \sum_{M < n
  \leq M+N} |a_n|^2 \text{~ for every~ } x \in
\mathbb{R}. \tag{2.42}\label{eq2.42}  
\end{equation*} 

Integrating this with respect to $x$ over an interval of length 1 and
using \eqref{eq1.28}, we see that  
\begin{equation*}
R \sum_{M < n \leq M+N}|a_n|^2 = \sum_{r=1}^{R} \int\limits_{0}^{1}|U
(x_r +x)|^2 dx \leq \Delta (N, \delta ) \sum_{M< n \leq
  M+N}|a_n|^2. \tag{2.43} \label{eq2.43}
\end{equation*}

Therefore from our example \eqref{eq2.33} of equally-spaced points follows
that  
\begin{equation*}
\Delta (N, \delta )\geq \delta^{-1}. \tag{2.44}\label{eq2.44}
\end{equation*}

Furthermore, Bombieri and Davenport \cite{key3} have given examples from
which one gets 
\begin{equation*}
\Delta (N, \delta )\ge N+\delta^{-1} -1. \tag{2.45}\label{eq2.45}
\end{equation*}

These remarks can be considered as negative ones.

In the positive direction the following result, due to Montgomery and
Vaughan \cite{key2}, leaves only a small gap when compared with
\eqref{eq2.45}. Since $N$ is usually\pageoriginale 
large in applications this difference in minor.
 
\begin{theorem}\label{chap2-thm2.3} %the 2.3
Under the hypotheses of Theorem \ref{chap2-thm2.1}, we have 
\begin{equation*}
\sum_{r=1}^{R} |U(x_r)|^2 \leq ( N+\delta^{-1}) \sum_{M < n \leq M+N}
|a_n|^2. \tag{2.46}\label{eq2.46} 
\end{equation*}
\end{theorem}

(In what follows we assume that $R \geq 2$ (cf. \eqref{eq2.20}).)

Their proof uses the principle of duality due to Hellinger-Toeplitz,
namely that for an $R \times N$ matrix $(c_{rn})$ with complex entries
and a constant $A$ 
\begin{equation*}
\sum_{n=1}^{N}| \sum_{r=1}^{R} c_{rn}v_r |^2 \le A \sum_{r=1}^{R}
|v_r|^2, \; \forall v_r \in  \mathbb{C} \tag{2.47}\label{eq2.47} 
\end{equation*} 
implies that (and is implied by)
\begin{equation*}
\sum_{r=1}^{R}| \sum_{n=1}^{N} c_{rn}w_n |^2 \leq A \sum_{n=1}^{N}
|w_r|^2,\forall w_n \in  \mathbb{C} \tag{2.48}\label{eq2.48} 
\end{equation*}
and the following extension of Schur's result regarding Hilbert's
inequality (Montgomery and Vaughan \cite{key1}): 

\begin{lemma}\label{chap2-lem2.3} %lem 2.3
Under the hypotheses of Theorem \ref{chap2-thm2.1}, we have 
\begin{equation*}
|\sum_{r=1}^{R} \sum_{\substack{s=1 \\ s \neq r}}^{R} u_r \bar{u}_s
cosec ( \pi(x_r -x_s )) | \le \delta^{-1} \sum_{r=1}^{R}|u_r |^2,
\forall u_r \in  \mathbb{C}. \tag{2.49}\label{eq2.49} 
\end{equation*}
\end{lemma}

\noindent
{\bf Proof of Lemma \ref{chap2-lem2.3}.}~
First we can impose the normalization condition
\begin{equation*}
\sum_{r=1}^{R} |u_r|^2 =1. \tag{2.50}\label{eq2.50}
\end{equation*}

Also we can assume, since the double-sum in \eqref{eq2.49} is a
skew-hermitian form, that the $(u)$ which makes the left-hand side
there maximum satisfies 
\begin{equation*}
\sum_{\substack{r=1 \\ r \neq s}} u_r \text{ cosec } (\pi (x_r-x_s)) = i \lambda u_s, \quad 1 \leq s \leq R \tag{2.51}\label{eq2.51}
\end{equation*}
with some (real) $\lambda$. Thus it suffices to show (under  \eqref{eq2.50}
and \eqref{eq2.51}) that 
\begin{equation*}
| \lambda |\leq \delta^{-1} \tag{2.52}\label{eq2.52}
\end{equation*}\pageoriginale

Further, we can assume that all the $x_r$'s lie in the interval
(0,1] (without any loss of generality). 

We have, by \eqref{eq2.15} and \eqref{eq2.50},
\begin{equation*}
\begin{cases}
|\lambda |^2 &= \sum\limits_{s=1}^{R} |\sum\limits^R_{\substack{r=1
    \\ r \neq s}} ur  \text{ cosec } (\pi (x_r-x_s))|^2=\\ 
&= \sum\limits_{s=1}^{R} \sum\limits^R_{\substack{r=1 \\ r \neq s}}
\sum\limits^R_{\substack{t=1 \\ t \neq s}} u_r \bar{u}_t \text{ cosec } (\pi_r
-x_s )) ~ \text{ cosec } (x(x_t- x_s)) =\\ 
&= \sum_1 +\sum_2 ,
\end{cases} \tag{2.53}\label{eq2.53}
\end{equation*} 
where
\begin{equation*}
\begin{cases}
\sum_1  &= \sum\limits_{s=1}^{R} |\sum\limits^R_{\substack{r=1 \\ r
    \neq s}} | u_r|^2 \; cosec^2 (\pi( x_r-x_s ))\\ 
\text{ and } &\\
\sum_2 &= \sum\limits_{\substack{r=1 \\ r \neq t}}^{R} \sum\limits_{t=1}^{R} u_r
\bar{u}_t \sum\limits_{\substack{s = 1 \\ s \neq r \\ s \neq t}}^{R} \text{
  cosec } (\pi( x_r-x_s )) \text{ cosec } (\pi ( x_t-x_s )). 
\end{cases} \tag{2.54}\label{eq2.54}
\end{equation*}

Using the identity (if $R \geq 3$-for $R = 2$ the inequality \eqref{eq2.63}
below is trivial, because, then $\sum_2 =0$) 
{\fontsize{10}{12}\selectfont
\begin{equation*}
\begin{cases}
\cosec \; \theta_1 \cosec \theta _2 &= \cosec (\theta _1-\theta _2)(\cot \theta _2- \cot \theta _1), if \theta _1 \theta _2 (\theta _1-\theta _2)\neq 0.\\
& -\pi < \theta _1,\theta _2,\theta _1-\theta _2 < \pi,
\end{cases}
\end{equation*}}
we see that 
\begin{equation*}
\sum_2 = \sum_3 - \sum_4, \tag{2.55}\label{eq2.55}
\end{equation*}
where
\begin{equation*}
\begin{cases}
\sum_3 =\sum\limits_{\substack{r=1 \\ r \neq t}}^{R} \sum\limits_{t=1}^{R} u_r
\bar{u}_t ~ cosec (\pi(x_r -x_t )) \sum\limits_{\substack{s=1 \\ s \neq r, s
    \neq t}}^{R} cot (\pi (x_t- x_s))\\ 
\text{and} &\\
\sum_4 =\sum\limits_{\substack{r=1 \\ r \neq t}}^{R}
\sum\limits_{t=1}^{R} u_r \bar{u}_t\; cosec (\pi (x_r -x_t ))
\sum\limits_{\substack{s=1 \\ s \neq r, s \neq t}}^{R} cot (\pi (x_t-
x_s)).\\ 
\end{cases} \tag{2.56}\label{eq2.56}
\end{equation*}

Denoting\pageoriginale
\begin{equation*}
b_r : = \sum_{\substack{s=1 \\ s \neq r}}^{R} cot (\pi (x_r -x_s )),
\quad 1 \le r \le R, \tag{2.57}\label{eq2.57} 
\end{equation*}
we will have
\begin{equation*}
\sum_{3}= \sum_{\substack{r=1 \\ r \neq t}}^{R} \sum_{t=1}^{R} u_r \bar{u}_t ~ cosec (\pi(x_r -x_t )) b_t - \sum_ {31} \tag{2.58}\label{eq2.58}
\end{equation*}
with
\begin{equation*}
\sum_ {31}\sum_{\substack{r=1 \\ r \neq t}}^{R} \sum_{t=1}^{R} u_r
\bar{u}_t \cosec (\pi(x_r -x_t )) \cot (\pi (x_t -x_r)) = \text{ Re } 
\sum_{31}. \tag{2.59}\label{eq2.59} 
\end{equation*}

Now, from \eqref{eq2.58} and \eqref{eq2.15}, we obtain
\begin{equation*}
\sum_3 + \sum_{31}= \sum_{r=1}^{R} \bar{u}_t b_t i \lambda u_t = i
\lambda \sum_{t=1}^{R} b_t|u_t |^2. \tag{2.60}\label{eq2.60} 
\end{equation*}

Treating  $\sum_4$ of \eqref{eq2.56} similarly, we also get
\begin{equation*}
\sum_3+ \sum_{31} = \sum_4 +\sum_{41}, \sum_{41} =-\sum_{31},
\tag{2.61}\label{eq2.61} 
\end{equation*}
so that, from \eqref{eq2.55} and \eqref{eq2.59}. follows
\begin{equation*}
\sum_2 =-2 \text{ Re } \sum_{31}. \tag{2.62}\label{eq2.62}
\end{equation*}

Hence 
\begin{equation*}
|\sum_2 | \leq 2 | \sum_{31}| \leq  \sum_{\substack{r=1 \\ r \neq
    t}}^{R} \sum_{t=1}^{R} (|u_r|^2 | + |u_t|^2) | \cosec^2 (\pi ( x_r
-x_t )) \cos )\pi (x_r-x_t))|, \tag{2.63}\label{eq2.63} 
\end{equation*}
which yields in view of symmetry, by \eqref{eq2.53} and \eqref{eq2.54},
\begin{equation*}
|\lambda|^2 \leq \sum_{\substack{r=1 \\ r \neq t}}^{R} \sum_{t=1}^{R}
(|U_r|^2 \cosec^2 (\pi(x_r -x_t))(1+2| \cos (\pi (x_r- x_t
))|). \tag{2.64}\label{eq2.64} 
\end{equation*}

Observing here that $\sin^2(\pi \theta) = \sin^2(\pi || \theta ||)$ and
$| \cos (\pi \theta)| =\break | \cos (\pi||\theta ||)|$ (for any real
$\theta$) and employing the inequality 
\begin{equation*}
\cosec^2 \theta (1+2 \cos \theta) \leq 3 \theta^{-2} \text{ for }
0<\theta \leq \frac{\pi}{2}, \tag{2.65}\label{eq2.65} 
\end{equation*}\pageoriginale
we obtain further
\begin{equation*}
|\lambda |^2 \leq 3 \pi^{-2} \sum_{r=1}^{R}|u_r|^2 \sum_{\substack{t=1 \\ t \neq r}}^{R} ||x_r-x_t||^{-2}. \tag{2.66}\label{eq2.66}
\end{equation*}

Since the value of the inner sum here is unaltered when $x$'s are
translated by integers, we can arrange, for any given
$x_r$. translates of all $x_t$'s $(t \neq r)$ to lie in the interval
$(x_r-\dfrac{1}{2},x_r+\dfrac{1}{2})$. Then the inner sum in \eqref{eq2.66}
is easily majorized, because of \eqref{eq2.10}, by $2
\sum\limits_{j=1}^{\infty}(\delta j)^{-2}$. So (on recalling \eqref{eq2.50})
holds  
\begin{equation*}
|\lambda|^2 \leq 6 \pi^{-2} \delta^{-2} \sum
^{\infty}_{j=1}j^{-2}=\delta^{-2}. \tag{2.67} \label{eq2.67}
\end{equation*}

This is \eqref{eq2.52} and thereby Lemma \ref{chap2-lem2.3} is
completely proved. 

Now we are in a position to deduce Theorem \ref{chap2-thm2.3} from
\eqref{eq2.47} to \eqref{eq2.49}. 

\medskip
\noindent
\textbf{Proof of Theorem \ref{chap2-thm2.3}.}~ Taking
\begin{equation*}
c_{rn}=e((M+n)x_r),\qquad w_n=a_n, \tag{2.68}\label{eq2.68}
\end{equation*}
it suffices for a proof of \eqref{eq2.46}, in view of \eqref{eq2.47}
and \eqref{eq2.48}, to show that 
\begin{equation*}
\sum^{N}_{n=1} | \sum^{R}_{r=1}c_{rn} v_r |^2 (N+ \delta^{-1})
\sum^{R}_{r=1}|v_r|^2, \; \forall v_r \in  \mathbb{C}. \tag{2.69}\label{eq2.69} 
\end{equation*}

Expanding the left-hand side here we obtain from the diagonal terms
the contribution 
\begin{equation*}
N \sum^{R}_{r=1}|v_r|^2. \tag{2.70}\label{eq2.70}
\end{equation*}

The remaining part amounts to
\begin{equation*}
\sum^{R}_{\substack{r=1 \\ \quad r \neq s}} \sum^{R}_{s = 1} v_r
\bar{v}_S \sum_{M < n \leq M+N} e(n(x_r-x_S)), \tag{2.71}\label{eq2.71} 
\end{equation*}
and the inner sum here is
\begin{equation*}
\frac{1}{2}i \{ e(
(M+\frac{1}{2})(x_r-x_S))-e((M+N+\frac{1}{2})(x_r-x_S))\} \cosec
(\pi(x_r-x_S)). \tag{2.72}\label{eq2.72} 
\end{equation*}\pageoriginale

We apply \eqref{eq2.49} twice with the choices
\begin{equation*}
u_r=v_re((M+\frac{1}{2})x_r) \text{ and } u_r=V_r
e((M+N+\frac{1}{2})x_r). \tag{2.73}\label{eq2.73} 
\end{equation*}

Then, because of the factor $\dfrac{1}{2}$ in \eqref{eq2.72}, we obtain that
the contribution \eqref{eq2.71} is  
\begin{equation*}
\leq \delta^{-1}\sum^{R}_{r=1} |v_r|^{2}. \tag{2.74}\label{eq2.74}
\end{equation*}
and this in combination with \eqref{eq2.70} yields \eqref{eq2.69}.

Montgomery and Vaughan \cite{key2} have also obtained a more
sophisticated form of the large sieve, which has turned out to be
extremely powerful in arithmetical applications. The weights attached
here enable one to take care of the irregular spacing of Farey
fraction (cf. Theorems \ref{chap2-thm2.5} and \ref{chap2-thm2.6} below). 

\begin{theorem}\label{chap2-thm2.4}%the 2.4
Under the assumption of Theorem \ref{chap2-thm2.1}, put
\begin{equation*}
\delta_r : = \min_{\substack{s\\ s \neq r}}||x_r-x_s||. \tag{2.75}\label{eq2.75}
\end{equation*}
\end{theorem}

Then
\begin{equation*}
\sum_{r=1}^{R}(N+\frac{3}{2}\delta_r^{-1})^{-1}|U(x_r)|^{2} \leq
\sum_{M < n \leq M+N}|a_n|^2. \tag{2.76}\label{eq2.76} 
\end{equation*}

The proof is very similar to that of Theorem \ref{chap2-thm2.3}. The
essential change is the following version, involving $\delta'_r s$, of Lemma
\ref{chap2-lem2.3} (Montgomery and Vaughan \cite{key1}): 

\begin{lemma}\label{chap2-lem2.4}%lem 2.4
Under the hypothesis and notation of Theorem \ref{chap2-thm2.4}, there
holds 
\begin{equation*}
| \sum^{R}_{\substack{r=1\\ \quad r \neq s}} \sum_{s=1}^{R}u_r
\bar{u}_S \text { cosec }(\pi (x_r-x_S))| \leq \frac{3}{2}
\sum^{R}_{r=1}|u_r |^2 \delta^{-1}_r,  \; \forall u_r \in
\mathbb{C}. \tag{2.77}\label{eq2.77}  
\end{equation*}
\end{lemma}

Here we shall only conclude Theorem \ref{chap2-thm2.4} using
\eqref{eq2.47}, \eqref{eq2.48} and \eqref{eq2.77}. 

\medskip
\noindent
\textbf{Proof of Theorem \ref{chap2-thm2.4}.}~ Now\pageoriginale we
put, instead of \eqref{eq2.68}, 
\begin{equation*}
c_{rn}=(N+\frac{3}{2}\delta_r^{-1})^{-\frac{1}{2}}e((M+n)x_r), w_n  =
a_n. \tag{2.78}\label{eq2.78} 
\end{equation*}

Then the diagonal terms, from the expression on the left of
\eqref{eq2.47}, contribute 
\begin{equation*}
N \sum_{r=1}^{R}(N+\frac{3}{2}\delta_r^{-1})^{-1}|v_r|^2, \tag{2.79}\label{eq2.79}
\end{equation*}
and the remaining part is
\begin{equation*}
\mathop{\sum_{r=1}^{R} \sum_{s=1}^{R}}_{r\neq s} v_r(N +
\frac{3}{2}\delta_r^{-1})^{-\frac{1}{2}} \bar{v}_s(N +
\frac{3}{2}\delta_s^{-1})^{-\frac{1}{2}} \sum_{M < n \leq M+N}
e(n(x_r-x_s)). \tag{2.80}\label{eq2.80} 
\end{equation*}

Again the inner sum is given by \eqref{eq2.72}. Now we apply
\eqref{eq2.77} twice with the choices 
{\fontsize{10pt}{12pt}\selectfont
\begin{equation*}
u_r=v_r(N+
\frac{3}{2}\delta_r^{-1})^{-\frac{1}{2}}e((M+\frac{1}{2})x_r) \text{
  and }u_r=v_r(N+\frac{3}{2}\delta^{-1}_r)^{-\frac{1}{2}} e((M+N+
\frac{1}{2})x_r). \tag{2.81}\label{eq2.81} 
\end{equation*}}\relax

Then, because of the factor $\dfrac{1}{2}$ in \eqref{eq2.72}, we obtain that
the contribution \eqref{eq2.80} is 
\begin{equation*}
\leq \frac{3}{2}\sum_{r=1}^{R} |v_r|^2(N +
\frac{3}{2}\delta^{-1}_r)^{-1}\delta^{-1}_r. \tag{2.82}\label{eq2.82} 
\end{equation*}

Now, this together with \eqref{eq2.79} proves \eqref{eq2.47} with $A=1$ for our
choice \eqref{eq2.78} of $c_{rn}$. Therefore \eqref{eq2.48} with the
above choice \eqref{eq2.78} of $w_n$ yields \eqref{eq2.76}, thereby
proving Theorem \ref{chap2-thm2.4}.  

Let us now specialize again, We assume that $Q \geq 2$, because the
theorems that follow are trivially true for $Q=1$ as before
(cf. \eqref{eq2.19}). Take 
\begin{equation*}
x_r=\frac{\ell}{q},1 \leq \ell \leq q \leq Q, (\ell,q)=1,
\tag{2.83}\label{eq2.83} 
\end{equation*}
so that we have (cf. \eqref{eq2.37}), for any two distinct Farey
fractions of \eqref{eq2.83}. 
\begin{equation*}
\bigg|\bigg|\frac{\ell}{q}-\frac{\ell'}{q'}\bigg|\bigg| \geq
\frac{1}{qq'} \geq \frac{1}{qQ} \geq \frac{1}{Q^2}
\tag{2.84}\label{eq2.84} 
\end{equation*}
which shows that in Theorem \ref{chap2-thm2.3} and Theorem
\ref{chap2-thm2.4} the quantities $Q^{-2}$ and $q^{-1}Q^{-2}$
are\pageoriginale permissible lower bounds for $\delta$ and
$\delta_r$, respectively. Therefore we obtain from these theorems 

\begin{theorem}\label{chap2-thm2.5}%the 2.5
For any complex number $a_n$, put
\begin{equation*} 
U(x):=\sum_{M<n \leq M+N} a_n e(nx). \tag{2.85}\label{eq2.85}
\end{equation*} 
\end{theorem}

Then
\begin{equation*}
\sum_{q \leq Q} \mathop{\sum{}'}_{\ell=1}^{q} \left|U(\frac{\ell}{q})\right|^2
\leq (N+Q^2) \sum_{M < n\leq M+N} |a_n|^2 \tag{2.86}\label{eq2.86} 
\end{equation*}
and
\begin{equation*}
\sum_{q \leq Q}(N+\frac{3}{2}qQ)^{-1}
\mathop{\sum{}'}^{q}_{\ell=1}|U(\frac{\ell}{q})|^2 
\leq \sum_{M <n \leq M+N} |a_n|^2. \tag{2.87}\label{eq2.87} 
\end{equation*}

Finally, returning to the beginning of the this chapter, i.e.,
\begin{equation*}
a_n=
\begin{cases}
1 &  \text{if }~ n\in \gamma,\\
0 & \text{if }~ n\notin \gamma,
\end{cases} \tag{2.88}
\end{equation*}
where $U(x)\equiv T(x)$, we note that Theorem \ref{chap2-thm2.5}
contains, with respect to \eqref{eq2.4}, the following 

\begin{theorem}\label{chap2-thm2.6}
Let $\gamma$ be a set of $S$ integers from an interval $(M, M+N]$ and put
\begin{equation*}
T(x):=\sum_{n \in \gamma}e(nx). \tag{2.89}\label{eq2.89}
\end{equation*}
\end{theorem}

Then
\begin{equation*}
\sum_{q \leq Q} \mathop{\sum{}'}_{l=1}^{q}\left|T(\frac{\ell}{q})\right|^2 \leq
(N+Q^2)S \tag{2.90}\label{eq2.90}
\end{equation*}
and
\begin{equation*}
\sum_{q \leq Q}(N+\frac{3}{2}qQ)^{-1}
    \mathop{\sum{}'}_{\ell=1}^{q}\left|T(\frac{\ell}{q})\right|^2\leq
    S. \tag{2.91}\label{eq2.91}  
\end{equation*}

\medskip
\begin{center}
{\bf NOTES}
\end{center}

The (explicit) qualitative version of \eqref{eq2.39} occurs for the first
time in Bombieri \cite{key1} with the factor $7\max (N,Q^2)$. His method
may be considered to\pageoriginale be a refinement of Linnik's, 

An improvement of Bombieri's factor as well as the extension from
Farey fractions to well-spaced points, i.e., \eqref{eq2.40}, is due to
Davenport and Halberstam \cite{key1}. Their substantial improvement of the
method is based on a convolution of $U(x)$ with a suitable auxiliary
function, an idea introduced by Roth \cite{key2}. Subsequent developments
and discussions along these lines were given by Bombieri and Davenport
\cite{key2}, \cite{key3} (and by Liu \cite{key1}). The paper of
Bombieri and Davenport \cite{key3} contains also various
investigations under different assumptions 
about the relative orders of $N$ and $\delta$. In particular, they
have proved the only result that still supersedes, under certain
conditions, the Theorem $2.3$; namely 
\begin{equation*}
\Delta (N, \delta)\leq \delta^{-1}+270 N^3 \delta^2 \text{~ if~ } N
\delta \leq \frac{1}{4}, \tag{2.92}\label{eq2.92} 
\end{equation*}
which is superior to \eqref{eq2.46} if $(N \delta)^2 < \frac{1}{270}$.

The best general result known to date, Theorem \ref{chap2-thm2.3}, is due to
Montgomery and Vaughan \cite{key2}. They have pointed out (cf. Montgomery
and Vaughan \cite{key1}) that it is possible to replace the factor
$\delta^{-1}$ in \eqref{eq2.49}, and so also in \eqref{eq2.46}, by
$\delta^{-1}-c$ 
for some $c>0$. Indeed, from the remarks subsequent to \eqref{eq2.66} we see
that the inner sum in \eqref{eq2.66} can be more precisely estimated as
follows: Let $J_1$ and $J_2$ denote the number of translates of
$x'_ts$ which lie in the intervals $(x_r-\dfrac{1}{2},x_r)$ and $(x_r,
x_r+\dfrac{1}{2}]$ respectively, so that $J_1 + J_2=R-1$. Then the
above-mentioned inner sum is at most (cf. \eqref{eq2.10}) 
\begin{equation*}
\begin{cases}
\delta(\sum\limits^{J_1}_{j=1} \frac{1}{j^2} + \sum\limits^{J_2}_{j=1}
\frac{1}{J^2})
\leq \delta^{-2} (\frac{\pi^{-2}}{3}- \int^{\infty}_{J_1+1}
\frac{du}{U^2}- \int \limits_{J_2+1}^{\infty} \frac{du}{u^2})=\\ 
=\delta^{-2}(\frac{\pi^2}{3}- \frac{(R+1)}{(J_1+1)(J_2+1)})
\leq\delta^{-2}(\frac{\pi^2}{3}-\frac{4}{{R+1}}) \leq \frac{\pi^2}{3}
\delta^{-2}-\frac{8}{3}\delta^{-1} 
\end{cases} \tag{2.93}\label{eq2.93}
\end{equation*}	
on\pageoriginale using $R \geq 2$, which implies in view of
\eqref{eq2.22} that $(R+1)\leq \dfrac{3}{2}\delta^{-1}$. Now, on
combining \eqref{eq2.66}, \eqref{eq2.93} and \eqref{eq2.50} we obtain  
\begin{equation*}
|\lambda|^2 \leq \delta^{-2}-\frac{8}{\pi^2}\delta^{-1} \leq (\delta^{-1}-\frac{4}{\pi^2}). \tag{2.94}\label{eq2.94}
\end{equation*}

Thus one can take, for instance, $c=\dfrac{4}{\pi^2}$ in the above
remark. However, according to \eqref{eq2.45}, it is not possible to obtain
these general estimates with any $c>1$. The essential tools for the
proofs of the Theorem \ref{chap2-thm2.3} and \ref{chap2-thm2.4} were
developed in Montgomery and 
Vaughan \cite{key1} (cf. Montgomery \cite{key6}). Earlier similar approaches had
been discussed in the work of Elliott \cite{key7}, Mathews
\cite{key1}, \cite{key2}, \cite{key3},
and Kobayashi \cite{key1}. 

An intermediate result of Bombieri (\cite{key4} and p.~17  of
\cite{key6}, namely 
\begin{equation*}
\Delta (N, \delta^{-1})\leq N+ 2 \delta^{-1},  \tag{2.95}\label{eq2.95}
\end{equation*}
is based on Theorem \ref{chap0-thm0.1}. Under the assumption of
Theorem \ref{chap2-thm2.1}, considering the sum
$\sum_{-N}^{N}a_ne(nx)$ (cf. Lemma \ref{chap2-lem2.2}) instead of
$U(x)$, he takes the Hilbert space $\ell^2$ of sequences $\alpha =
\{\alpha _n \}$ with $(\alpha, \beta):=
\sum_{-\infty}^{\infty}\alpha_n \overline{\beta}_n$, $||\alpha ||^2
=\sum_{-\infty}^{\infty}|\alpha_n|^2$. Choosing
{\fontsize{10}{12}\selectfont
\begin{equation*}
L\in \mathbb{N},f=
\begin{cases}
a_n \text{ if }|n|\leq N,\\
0 \text{ if } |n|>N,
\varphi_r =
\begin{cases}
e(-nx_r) &\text{ if } |n| \leq N\\
(\frac{N+L-|n|}{L})^{1/2}e(-nx_r) &\text{ if } N< |n| \leq N+L.\\
0 &\text{ if } |n|>N+L,
\end{cases}
\end{cases}  \tag{2.96}\label{eq2.96}
\end{equation*}}
one gets
\begin{equation*}
||f||^2 =\sum_{-N}^{N}|a_n|^2, (f,\varphi_r )= \sum_{-N}^{N} a_n e(n
x_r), \; r = 1,\ldots ,R.  \tag{2.97}\label{eq2.97} 
\end{equation*}

Bombieri proves that for each $r$, $1 \leq r \leq R$
\begin{equation*}
\sum_{S=1}^{R}|(\varphi_r, \varphi_s )|\leq 2N + L +
\frac{\pi^2}{12}.\frac{1}{L \delta^2},  \tag{2.98}\label{eq2.98} 
\end{equation*}

Now\pageoriginale \eqref{eq2.95} follows from Theorem
\ref{chap0-thm0.1} or from \eqref{eq0.59}, by taking 
\begin{equation*}
L=\left[\frac{1}{\delta} \right]. \tag{2.99}\label{eq2.99}
\end{equation*}
\begin{description}
\item[\eqref{eq2.10}]: or $||X||:=|x-[x+\dfrac{1}{2}]|$.

\item[\eqref{eq2.41}, \eqref{eq2.44}]: cf. Montgomery and Vaughan
  \cite{key2}. 

\item[\eqref{eq2.47}, \eqref{eq2.48}]: cf. Hardy, Littlewood and
  Polya, \textit{Cambridge}  Theorem $288$.

\item[\eqref{eq2.51}]: cf. Mirsky, L., \textit{An Introduction to
  Linear Algebra}(Oxford), p.~388 Theorem \underline{$12.6.5$}. 

\item[\eqref{eq2.65}]: We rewrite \eqref{eq2.65} in the more convenient form:
\end{description}

\begin{equation*}
3 \sin^2 \theta \geq \theta^2 (1+2 \cos \theta) \quad \text{for} \quad  0 \leq
\theta \leq \frac{\pi}{2}  \tag{2.100}\label{eq2.100} 
\end{equation*}

Now, from the series expansion of $\sin \theta$ and $\cos \theta$, we have
\begin{equation*}
\sin \theta \geq \theta -\frac{1}{6}\theta^3 , \cos \theta \leq 1-
\frac{1}{2} \theta^2+ \frac{1}{24} \theta^4 \text{ for } 0 \leq \theta
\leq \frac{\pi}{2}, \tag{2.101} \label{eq2.101}
\end{equation*}
since $(m!)^{-1} \theta^m \geq ((m+2)!)^{-1} \theta^{m+2}$, if $m \geq 1$ and $\theta^2 \leq 6$. This gives \eqref{eq2.100} on verifying
\begin{equation*}
3(1-\frac{\theta^2}{6})^2 = 1+2(1-\frac{\theta^2}{2}
+\frac{\theta^4}{24}). \tag{2.102} \label{eq2.102}
\end{equation*}

\eqref{eq2.76}:~ In connection with this very effective form of the
large sieve we have in Montgomery and Vaughan \cite{key2} the remark
that it may be that the constant $\dfrac{3}{2}$ in \eqref{eq2.77}, and
consequently also in \eqref{eq2.76}, can be replaced by $1$. Actually,
their work contains the constant $\dfrac{\sqrt{12
    +\sqrt{78}}}{\pi}(=1.45282 \cdots)$ (and some slight improvements)
instead of the aforementioned constant $\dfrac{3}{2}$ (cf. notes of
Chapter \ref{chap8} under \eqref{eq8.36}). For a previous result of
this type, see Montgomery \cite{key5} (Therefore
\ref{chap4-thm4.1}). A first weighted form of this sieve occurs in
Davenport and Halberstam \cite{key1} (cf. also Davenport \cite{key1}
Liu \cite{key1} and Montgomery \cite{key5}. (Theorem
\ref{chap2-thm2.4}). 

By\pageoriginale comparing the results of this chapter with those of
the previous 
forms \eqref{eq0.54} gives in Chapter \ref{chap0}, it is natural to ask the
following question (Erd\"os \cite{key3}) Consider (in the notation of
Ch. \ref{chap0})  
\begin{equation*}
\sum_{p \leq Q} \sum_{l=1}^{p-1}|T(\frac{l}{p})|^2 \leq C_1(N,Q)S,
\tag{2.103}\label{eq2.103} 
\end{equation*}
where now only a `negligible' proposition of terms remain on the
left-hand side. Then are there better results for $C_1(N,Q)$ than
those for $C(N,Q)$, (cf. \eqref{eq2.4}) or, more specifically, can one expect a
gain of a factor log $Q$ here? Erd\"os \cite{key3} (cf. Erd\"os and
R\'enyi \cite{key1})
proved that if $Q \leq \sqrt{N}$ this is true for \textit{almost all}
sets $\gamma$. Wolke \cite{key2} (cf. Wolke \cite{key1}) has proved a slightly
weaker estimate which holds also for (more general) $U(x)$,
(cf. \eqref{eq2.9}), and for \textit{all} sets $\gamma$, but under the
severe condition $N \leq Q(\log Q)^{\delta}$ for some $\delta>0$. On
the other hand Erd\"os \cite{key3} (cf. Erd\"os and R\'enyi
\cite{key1}) has shown that 
$C_1(N,Q)$ is of the same order of magnitude as $C(N,Q)$, if $Q$ is of
a higher order than $\sqrt{N \log N}$. For further literature in this
connection, see Elliott \cite{key5}, \cite{key7}. The result is
generally speaking, 
that (except under special circumstances) we cannot have a better
estimate in \eqref{eq2.103} than that for \eqref{eq2.4}. 

Another attempt, at sharpening the large sieve, is due to Burgess
(\cite{key1}). He proved that for any set $\mathscr{Q} \cup \mathbb{N}$ 
\begin{equation*}
\sum\limits_{\substack{q \leq Q\\ q \in
    \mathscr{Q}}}\mathop{\sum{}'}\limits_{\ell=1}^{q}|U(\frac{\ell}{q})| \;  \ll
(Q|\mathscr{Q}|)(N+(Q|\mathscr{Q}|))(\sum_{M < n \leq M + N}|a_n|^2),
\tag{2.104} \label{eq2.104}
\end{equation*}
from which saving is made when $\mathscr{Q}$ is a sparse set. (Note also that
here $U$ occurs to the first power on the left-hand side.) 

In the other direction one may ask for general lower bounds, for
instance in \eqref{eq2.103}. In view of \eqref{eq0.53} such results would mean
that `general sequences cannot be too well-distributed in almost all
arithmetic progressions'. The first result\pageoriginale 
in this context is given by
Roth \cite{key1}, For further literature concerning this question, we refer
to Roth \cite{key3}, \cite{key5}, Choi \cite{key1}, Montgomery
\cite{key5} (Chapter \ref{chap5}) and 
Huxley \cite{key5}. Wolke \cite{key10} (cf. Wolke \cite{key9}) has
stated the explicit lower bound (in the notation \eqref{eq1.29}) 
\begin{equation*}
\sum_{\substack{q \leq Q\\ q \in
    \mathscr{Q}}}\mathop{\sum{}'}\limits_{\ell=1}^{q}|U(\frac{\ell}{q})|^2 
\geq \sum\limits_{\substack{q \leq Q \\ q \in \mathscr{Q}}}
\frac{1}{\varphi(q)} \bigg 
| \sum_{M < n \leq M+N} a_n c_q(n)\bigg |^2 \text{~ for any set~ } \mathscr{Q}
\subset \mathbb{N}. \tag{2.105}\label{eq2.105} 
\end{equation*}

However, \eqref{eq2.105} should rather be considered as a reduction of
the expression on the right-hand side to the large sieve
(cf. \eqref{eq3.28}). 
