


\setcounter{chapter}{-1}
\chapter{History of the Large Sieve}\label{chap0}%\chap 

LET\pageoriginale $\gamma$ BE a set of $| \gamma |$ integers contained in an
interval of length $N$: 
\begin{equation*}
\gamma \subset (M, M+N), M \in \mathbb{Z}, N \in \mathbb{N},S : = |\gamma |. 
\tag{0.1}\label{eq0.1}
\end{equation*}

Setting
\begin{equation*}
S(q,\ell): = \sum_{\substack{n \in \gamma \\ n \equiv \ell \mod q}} 1,
\tag{0.2}\label{eq0.2}
\end{equation*}
we have
\begin{equation*}
\sum_{\ell =1}^{q} S(q,\ell)= S,
\tag{0.3}\label{eq0.3}
\end{equation*}
or that
\begin{equation*}
\sum_{\ell =1}^{q} (S(q, \ell )- \frac{S}{q})=0.
\tag{0.4}\label{eq0.4}
\end{equation*}

Confining, for the moment, our attention to primes $q=p$ only,\break \eqref{eq0.4}
tells us that the quantity 
\begin{equation*}
D(p): = \sum_{\ell =1}^{p} (S(p, \ell )- \frac{S}{p})^2
\tag{0.5}\label{eq0.5}
\end{equation*}
measures how uniformly the set $\gamma$ is distributed among the
residue\break classes $\mod p$. Such an information. is of great importance
in various problems. 

A uniform and non-trivial bound of the form
\begin{equation*}
(D) \hspace{2cm}
\sum_{p \leq Q} p D(p) \leq K (N,Q,S) \hspace{2cm}
\tag{0.6}\label{eq0.6}
\end{equation*}
by uniform we mean here that while $K$ may depend on $N,Q$ and $S$, it
should be independent of the particular structure of the set
$\gamma$-makes it possible to draw the following general
conclusions. (Note that actually the supposition $M=0$ here
involves\pageoriginale no loss of generality.) 

\begin{enumerate}
\renewcommand{\theenumi}{\Alph{enumi}}
\renewcommand{\labelenumi}{\bf(\theenumi)}
\item Let $N$, $Q$ and $S$ be given. If every set $\gamma$
  (cf. \eqref{eq0.1}) is so uniformly distributed over the residue
  classes $\mod p$ as 
expressed by $(D)$, then for most of the $p$'s $D(p)$ must be
small. This remark tells us since $D(p)$ is bound to be large if many
residue classes $\mod p$ do not contain any element of $\gamma$, the
statement that $S(p,\ell)=0$ for `many' $\ell$'s $\mod p$ can be true
for only `few' (`exceptional') primes $p \leq Q$. 

We can express this remark in a quantitative form. Let $\omega(p)$ be
an (integer-valued) function satisfying 
\begin{equation*}
0< \omega (p) < p,\tag{0.7}\label{eq0.7}
\end{equation*}
and now we ask for the number of $p's \leq Q$, for which \textit{at
  least} $\omega (p)$ residue classes $\mod p$ do not contain any
element of our set $\gamma$. Let us denote the set of these
`exceptional' primes by $\mathfrak{p}$, and set 
\begin{equation*}
\min_{p \in \mathfrak{p}} \frac{\omega (p)}{p}=\delta = \delta
(\mathfrak{p}).\tag{0.8}\label{eq0.8} 
\end{equation*}

One has then for each $p \in \mathfrak{p}$
\begin{equation*}
p D(p) \geq p \omega (p)\frac{S^2}{p^2} \geq  \delta
S^2,\tag{0.9}\label{eq0.9} 
\end{equation*} 
and so
\begin{equation*}
\sum_{p \in \mathfrak{p}} p D(p) \geq \delta S^2
|\mathfrak{p}|.\tag{0.10}\label{eq0.10} 
\end{equation*}

Trivially
\begin{equation*}
\sum_{p \in \mathfrak{p}} p D(p) \leq \sum _{p \leq Q}
pD(p),\tag{0.11}\label{eq0.11} 
\end{equation*}
and therefore \eqref{eq0.6} and \eqref{eq0.10} give
\begin{equation*}
(A) \hspace{2cm}
|\mathfrak{p}| \leq \frac{K(N,Q,S)}{\delta
    S^2} \hspace{2cm} \tag{0.12}\label{eq0.12} 
\end{equation*}

For\pageoriginale the remaining primes $p$ i.e., $p \not\in
\mathfrak{p}$ and $p \leq Q$, less than $\omega(p)$ residue classes
$\mod p$ are devoid of numbers of $\gamma$. Consequently, for these
primes $p$ each of at least $p- \omega (p)$ residue classes $\mod p$
contains atleast one element of $\gamma$ 

\item The preceding
result may also be considered as a sieve problem. In order to see
this. let us start (cf. \eqref{eq0.1}) with the set of numbers 
\begin{equation*}
M+1 , \ldots , M+N,\quad M \in \mathbb{Z}, N \in
\mathbb{N}.\tag{0.13}\label{eq0.13} 
\end{equation*}

Now for certain primes $p \leq Q$, $p \in \mathfrak{p}$ say, strike out
of numbers (\ref{eq0.13}) all those numbers which are situated in any of
certain $\omega(p)$, where $\omega(p)$ satisfies \eqref{eq0.7}, of the $p$
residue classes $\mod p$. Let the remaining set of numbers be our set
$\gamma$. We obtain \eqref{eq0.12} again under the present situation, since
we have used for its proof only that at least $\omega(p)$ residue
classes $\mod$ $p$ (for each $p \in \mathfrak{p}$) contain no element
of $\gamma$. Next resolving \eqref{eq0.12} with respect to $S$- as shall be
seen later to be possible we get an upper bound for $|\gamma|$ with
respect to our set $\gamma$ above. It is the type  
\begin{equation*}
(B) \hspace{2cm}
 (|\gamma|=) S \leq K_1 (N,Q,|\mathfrak{p}|, \delta
  ). \hspace{2cm} \tag{0.14}\label{eq0.14} 
\end{equation*}

\item Our remark at the beginning of $A$ had been that a non-trivial
  estimate of the type $(D)$ implies $D(p)$ is small for most of the
  $p's$ under consideration. If we view this as the statement that
  `most often' $\dfrac{S}{p}$ is a good approximation to $S(p, \ell)$
  then its quantitative version leads to a more precise formulation
  than that under $A$ and consequently than that under $B$.. In fact,
  we obtain a result of the type of
  \v{C}ebys\v{e}v's inequality. 
\end{enumerate}

To this end we introduce a function $c(p)$ satisfying
\begin{equation*}
c(p) \geq 1 \tag{0.15}\label{eq0.15}
\end{equation*}
and put 
\begin{equation*}
\max\limits_{p\leq Q} c(p) = c =c_Q.  \tag{0.16}\label{eq0.16}
 \end{equation*} \pageoriginale
 
Now we ask for the existence of the inequality 
\begin{equation*}
|S(p,\ell)-\frac{S}{P}|<\frac{S}{pc(p)}\tag{0.17}\label{eq0.17} 
\end{equation*}

More precisely,  we ask for the number of primes  $p\leq Q$ for  which
the  inequality \eqref{eq0.17} does  not hold for at least $\omega
(p)$ (cf. \eqref{eq0.7}) residue classes $\ell\mod p$. Let us denote
again by $\mathfrak{p}$ the set of such exceptional primes. Then, in the
notation of \eqref{eq0.8}, 
\begin{equation*}
 p D(p)\ge \omega(p) \frac{S^2}{p^2 c^2
 (p)}\ge \frac{\delta}{c^2}S^2, \quad 
 \forall p \in \mathfrak{p}.\tag{0.18}\label{eq0.18} 
 \end{equation*}
 
Consequently, in view of \eqref{eq0.11}. \eqref{eq0.6} gives for the number of
 exceptional primes $p$ the upper  bound 
\begin{equation*}
(C) \hspace{2cm}
 |\mathfrak{p}|\leq \frac{c^2}{\delta S^2}
  K(N,Q,S). \hspace{2cm}\tag{0.19}\label{eq0.19} 
\end{equation*}

This result may be phrased as follows: \eqref{eq0.6} implies that for every
set $\gamma$ (of \eqref{eq0.1}) one has for all $p\leq Q$. save for atmost 
$$
\frac{c^2}{\delta S^2}K(N,Q,S)
$$
primes, and for all residue classes $\ell \mod p$ with the exception
of less then $\omega (p)$ of them 
\begin{equation*}
S(p,\ell) = \frac{S}{P}+ \theta \frac{S}{p c (p)},\quad \text{where}\quad
|\theta| < 1.\tag{0.20}\label{eq0.20}  
\end{equation*}

Choosing $c(p)=1$, we see that each exceptional prime of $A$, is also
an exceptional prime of $C$. and conversely. Therefore $C$. is a
generalization of $A$. and hence also of $B$. 


Although the uniformity of \eqref{eq0.6}  with  respect to $\gamma$ may be
considered a defect\pageoriginale (because it includes all `bad' sets)
it has the 
advantage of drawing all the conclusions (of which we have given three
general examples in {\bf A}, {\bf B}.  and {\bf C}.) valid for all sets  ,
including    those which are otherwise not readily accessible. The
quality of the $K's$  known for \eqref{eq0.6} allows  us to  obtain
results which are not available by the use of other methods.  

\eqref{eq0.6} is of interest for $Q\leq N$ only. $A$ trivial estimate
is obtained in the following way. By \eqref{eq0.5} and  \eqref{eq0.3}
we have 
\begin{equation*}
D(p)= \sum ^p_{\ell =1} S^2(p ,\ell) - \frac{S^2}{p},\tag{0.21}\label{eq0.21}
\end{equation*}
and trivially
\begin{equation*}
S(p,\ell) \leq \sum_{\substack{M<n\leq M+N\\n \equiv \ell \mod p}}1\leq
\frac{N}{p} +1 \leq 2 \frac{N}{p} \;\; (p \leq N).\tag{0.22}\label{eq0.22} 
\end{equation*}

Using this and \eqref{eq0.3} again in \eqref{eq0.21} we find that 
\begin{equation*}
pD(p)\leq 2N \sum _{\ell =1}^p S(p,\ell) = 2
NS,\tag{0.23}\label{eq0.23} 
\end{equation*}
and so
\begin{equation*}
(D_0) \hspace{2cm} \sum _{p \leq Q} p D(p) \leq 2NQS \quad (Q\leq
  N). \hspace{2cm} \tag{0.24}\label{eq0.24} 
\end{equation*}

As we have seen above (A),(B) and (C) are more or less different
versions of general types of results based on (D). Linnik \cite{key1}
was the first to consider such problems. With respect to $B$. he made
the remark that when  striking out an (absolutely) \textit{bounded}
number of residue classes mod $p$ the sieve method of Brun (or  of
Selberg) is applicable. However, this is no longer true if, as is
permissible under {\bf B.}, $\omega (p)$ is,  for example, an increasing
function of $p$. For this very reason, Linnik named his method of
treating A. and B.	 

\medskip
\noindent
\textbf{``The Large Sieve''.}

Linnik\pageoriginale \cite{key1} proved that
\begin{equation*}
(A_1) \hspace{2cm} |\mathfrak{p}|\leq20 \pi \frac{N}{\delta^2 S} \quad
  \text{for}\quad Q=\sqrt{N}, \hspace{2cm} \tag{0.25}\label{eq0.25}
\end{equation*}
and consequently
\begin{equation*}
(B_1) \hspace{2cm}  S\leq 20 \pi \frac{N}{\delta^2|\mathfrak{p}|},
  \quad\text{for}\quad Q= \sqrt{N}\hspace{2cm} \tag{0.26}\label{eq0.26}
\end{equation*}

As an example he takes $\gamma$ to be the set of primes $\leq N$,
i.e., $M =0$ and  $S=\pi (N)$, and $\omega(p)= p^{3/4}$. Then $\delta
\geq N^{-1/8}$ and $(A_1)$ yields the non-trivial estimate: 
\begin{equation*}
|\mathfrak{p}| \leq 20 \pi\frac{N}{N^{-1/4}\pi (N)} \leq 80 N^{1/4}
\log N,\quad \text{for}\quad N \ge N_0\tag{0.27}\label{eq0.27} 
\end{equation*}

Although the large sieve first occurred in the forms $(A_1)$ and
$(B_1)$, I would prefer, in particular, in view of later developments,
to call (D) \textit{the} large sieve and rather consider results of
type (A), (B),(C) and others as applications of the theory of the
large sieve. 

Following Linnik it was R\'enyi's (\cite{key1}, \cite{key2})  merit to
generalize the 
large sieve method in several respects. Simultaneously he noticed the
fundamental importance of (D), and also treated the more precise
version (C) for the first time. 

Generalizing \eqref{eq0.2}, for arbitrary complex numbers $a_n$ we set 
\begin{equation*}
\tilde{S}(q,\ell) : = \sum _{\substack{M<n\leq M+N\\ n\equiv \ell
    \mod q}}a_{n}\tag{0.28}\label{eq0.28} 
\end{equation*}
and
\begin{equation*}
\tilde{S}:= \tilde{S}(1,1)= \sum_{M<n \leq
  M+N}a_{n}.\tag{0.29}\label{eq0.29}  
\end{equation*}
(By taking for $a_n$ the characteristic function of $\gamma$ in
\eqref{eq0.28} we get \eqref{eq0.2}.) 

Let\pageoriginale $\mathfrak{p}$ be an arbitrary set of primes $p\leq
Q$. R\'enyi's 
paper \cite{key2} has implicitly the following (explicit) generalization of
(D) (with $Q< \sqrt{N}$): 
\begin{equation*}
(\tilde{D_1}) \qquad 
\begin{cases}
\sum\limits_{p \in f}pq \sum\limits ^{pq}_{\ell=1}|\tilde{S}(pq,\ell)
-\frac{\tilde{S}(q,\ell)}{P}|^2 \leq
\frac{1}{2\varepsilon}\sum\limits_{M<n\leq M+N} a^2_n\\ 
+ 
 \frac{4\pi^2\in^2N^4m^2}{3}|\mathfrak{p}| \text{ for }
 Q<\sqrt{\frac{N}{q}},
\end{cases}\tag{0.30} \label{eq0.30}
\end{equation*}
where $a'_n s$ are $\geq 0$, $m = \underset{M < n \leq M+N}{\max} a_n$, $q$ is a
squarefree number not divisible by any $p \in \mathfrak{p}$, and $0 <
\epsilon < \dfrac{1}{2N}$.  

The use of a set $\mathfrak{p}$ of primes here is particularly
suitable for applications of the type (A), (B) and (C), because
$\mathfrak{p}$ can serve as the of exceptional primes, and the factor
$|\mathfrak{p}|$ instead of $Q$ improves the estimate. From this
generalization to composite moduli and arbitrary coefficients he
derived (\cite{key1}, (Lemma 1)) a (C)-type result about 
\begin{equation*}
| \tilde {S}(pq,\ell ) - \frac{\tilde{S}(q, \ell)}{p}| <
\frac{\tilde{S}}{pq c(pq)}.\tag{0.31}\label{eq0.31}
\end{equation*}

From here he succeeded (\cite{key1}, (Lemma 2)) in making the large sieve
applicable also in the estimation of certain averages of character
sums, i.e., sums of the form 
\begin{equation*}
\sum_{M<n \leq M+N}\chi(n) a_n .\tag{0.32}\label{eq0.32} 
\end{equation*} 

Turning back to the case $q =1$,  $a_n = 
\begin{cases} 
1 \text{ for } n \in \gamma, \\
 0  \text{ for } n \notin \gamma , 
\end{cases}$
i,e. to \eqref{eq0.17}, then
R\'enyi's $C$-result corresponding to \eqref{eq0.19} is 
\begin{equation*}
(C_{1}) \hspace{2cm} |\mathfrak{p}| \leq \frac{3 \pi N^2
c^3}{2 \delta^{3/2}S^2}  \quad \text{for}\quad Q
< \sqrt{N}. \hspace{2cm} \tag{0.33}\label{eq0.33}   
\end{equation*}

By using a different method of proof, R\'enyi (\cite{key7}, (18)) next proved 
\begin{equation*}
(D_1)  \hspace{2.4cm}\sum_{p< (N/12)^{1/3}} p D (p) \leq 2
NS \hspace{2.4cm} \tag{0.34}\label{eq0.34}  
\end{equation*}
and\pageoriginale then applied this to C. and A. obtaining
(\cite{key7}, (Theorem 3)) 
\begin{equation*}
(C_2) \hspace{2cm} |\mathfrak{p}|\leq \frac {2 N c^2}{\delta S} \quad 
  \text{for} \quad Q < (\frac{N}{12})^{1/3} \hspace{2cm} \tag{0.35}\label{eq0.35} 
\end{equation*}
and (\cite{key7}, (Corollary 1))
\begin{equation*}
(A_2) \hspace{2cm} |\mathfrak{p}| \leq \frac{2N}{\delta S} \quad
  \text{for}\quad Q <
  (\frac{N}{12})^{1/3}. \hspace{2cm} \tag{0.36}\label{eq0.36}   
\end{equation*}

Finally R\'enyi improved his method and considered the large sieve in
its (D)-version as a special statistical statement
(\cite{key4}, \cite{key8}, \cite{key10}, \cite{key12}) and he showed
(\cite{key4})  
\begin{equation*}
(D_2) \hspace{2.4cm} \sum _{p \leq \frac{1}{2} N^{1/3}} p D (p) \leq 9
  NS \hspace{2.4cm} \tag{0.37}\label{eq0.37}  
\end{equation*}
and from this (\cite{key4}, (Theorem 3))
\begin{equation*}
(C_{3}) \hspace{2cm} |\mathfrak{p}| \leq \frac{9 N c^2}{\delta S}
  \quad\text{for}\quad Q
  < \frac{1}{2}N^{1/3} \hspace{2cm} \tag{0.38}\label{eq0.38}   
\end{equation*}

The last results above, though stronger than Linnik's, are still valid
only for a smaller range for the primes, i,e., for smaller values of
$Q$. In \cite{key9} (cf. Halberstam and Roth \cite{key1} (Ch. IV,
Theorem $6'$)) he prepared the ground for the extension 
\begin{equation*}
(D_{3}) \hspace{1.1cm}  \sum_{p \leq Q} p D(p) < (N +Q^3)S \quad\text{for}\quad
  Q < \sqrt{N}. \hspace{1.2cm} \tag{0.39}\label{eq0.39}  
\end{equation*}

Barban has been the first to prove a (D)-result by using Linnik's
original method. He showed (\cite{key8}, (Theorem 1)) 
\begin{equation*}
(D_{4})\qquad \sum_{p \leq Q}pD(p) \leq 2 \pi \frac{NQ}{K_0}S+
  \frac{K^2_o}{Q} S^2\; \text{ for } \;  1<K_0 < \min (Q,
  \frac{N}{Q}),\tag{0.40}\label{eq0.40}  
\end{equation*}
and for $(\tilde{D}_1)$ in the case $q =1$ (\cite{key10}, (1.3);  for
$q>1$ cf. \cite{key10}, (Theorem 3.1)) 
\begin{equation*}
(\widetilde{D}_{2}) \qquad 
\begin{cases}
\sum_{p \in \mathfrak{p}} p \sum _{\ell =1}^p |\tilde{S}(p, \ell ) -
\frac{\tilde{S}}{p}|^2 \leq \frac{1}{\varepsilon} \sum\limits_{M<n \leq
M+N} a^2_n +\\ 
+\frac{4 \pi^2 \in^2 N^2 \tilde{S}^2} {3}|\mathfrak{p}|  \quad
\text{for}\quad  Q < N, \qquad \tag{0.41}\label{eq0.41}  
\end{cases}
\end{equation*}\pageoriginale
subject to the conditions $0< \epsilon < \min (\dfrac{1}{2 \pi
  N},\dfrac{1} {2Q^2}), a _N \ge 0$. The particular case $a_n =
\begin{cases}
1 \text{ for } n \in \gamma ,\\
0 \text{ for } n \notin \gamma,
\end{cases}$
(i.e., $\tilde {S}= S$) of this result contains $(D_4)$ and from it
Barban (\cite{key10}. (Theorem 1.1)) derived for $A$.
\begin{equation*}
(A_3) \hspace{2cm} |\mathfrak{p}| \leq  20 \frac{N}{\delta^{3/2}S}
  \quad\text{for}\quad Q = \sqrt {N}. \hspace{2cm} \tag{0.42}\label{eq0.42}  
\end{equation*}

The importance of $(D_4)$ lies in the fact that there is no longer any
restriction on $Q$, apart from the natural one, namely $Q<N$, implied
by the $K_0$-condition. 

There is another generalization to a weighted form  of $(D)$ due to
Halberstam and Roth \cite{key1} (Ch. IV, Theorem 5). They proved that for
arbitrary weights $\delta _p$, satisfying  
\begin{equation*}
0 < \delta_p < \frac{1}{2  p^Q}\tag{0.43}\label{eq0.43} 
\end{equation*}
one has
\begin{equation*}
(D*) \hspace{2cm} \sum_{p \leq Q} p \delta_p D(p) \ll S+N^4 \sum_{p \leq
    Q}\delta_p^3 . \hspace{2cm}\tag{0.44} \label{eq0.44} 
\end{equation*}

A very important progress was made by Roth \cite{key2},  who succeeded in
proving  
\begin{equation*}
(D_5)\qquad \sum_{p \in \mathfrak{p}} p D(p) \ll (N+Q^2 \log K_0 ) S
  +S^2 |\mathfrak{p}| K_0^{-2} \quad  \text{for}\quad K_0 \ge
  2.\tag{0.45}\label{eq0.45}  
\end{equation*}
where again $\mathfrak{p}$ denotes an arbitrary set of primes $p \leq
Q$, and there is no restriction on $Q$. This result includes in
particular (\cite{key2}, (9)) 
\begin{equation*}
(D_6) \hspace{2.4cm} \sum _{p \leq \sqrt{\frac{N}{\log N}}} p D(p) <
  NS, \hspace{2.4cm} \tag{0.46}\label{eq0.46}  
\end{equation*}\pageoriginale
as well as  (\cite{key2}, (Corollary 1))
\begin{equation*}
(D_7) \qquad   \sum _{p \leq Q}p D(p) \ll  SQ^2 \log Q \quad
  \text{for}\quad Q \ge \sqrt{\frac{N}{\log
      N}}. \qquad  \tag{0.47}\label{eq0.47}  
\end{equation*}

With respect to (C) Roth (\cite{key2}, (Corollary 2)) derived, also
noticing that it is more appropriate - as can be seen from
\eqref{eq0.18} to introduce  
\begin{equation*}
\beta : = \frac {1}{4} \underset{p \leq Q} \min \frac{\omega (p)}{p
  c^2 (p)}\tag{0.48}\label{eq0.48}  
\end{equation*}
instead of $\delta$ and $c$ separately, from $(D_5)$ 
\begin{equation*}
(C_4)\qquad\qquad  |\mathfrak{p}| \ll \frac{N+Q^2 \log \frac{1}{\beta}}{\beta
    S} \quad \text{for}\quad Q < N. \qquad\qquad  \tag{0.49}\label{eq0.49}  
\end{equation*}

It is not too simple a matter to make a comparison of the  various
basic (D)-results of the large sieve. However, the main features
are the following ones.The very effective estimate (for $K(N,Q,S)$ of
\eqref{eq0.6}) 
\begin{equation*}
\ll NS \tag{0.50}\label{eq0.50} 
\end{equation*}
which was known by Renyi for $Q$ upto $N^{1/3}$ (cf.  \eqref{eq0.34},
\eqref{eq0.37}, \eqref{eq0.39}) has been extended by Roth
(cf. \eqref{eq0.46}) upto  $Q=\sqrt{\frac{N}{log N}}$. For values of
$Q$ beyond $\sqrt{N}$, \eqref{eq0.40} and \eqref{eq0.47} still yield
non-trivial estimates (compare \eqref{eq0.24}) upto the vicinity of
$N$. Apart from the factor  $\log Q$ and
the $\ll$-constant, \eqref{eq0.47} is in most cases the  better
estimate. Only for $Q \asymp N^2S)^{1/5}$ does \eqref{eq0.40} yield  
\begin{equation*}
\ll Q^2 S \tag{0.51}\label{eq0.51} 
\end{equation*}
 which is \eqref{eq0.47} without the factor $\log Q$.  On the other
 hand, the same result is implied by \eqref{eq0.45} if moreover $ S
 \ll Q \log Q$. 
 
As\pageoriginale far as the methods of proof for the large sieve are
concerned there are different ways of approach. 
 
Recalling our notation introduced in the beginning of this chapter
(i.e., considering, for simplicity, the case 
 $ a_n=
 \begin{cases}
1  \text{ for } n \in \gamma ,\\
0  \text{ for } n \notin \gamma, 
 \end{cases}$ only) the first method, used in the aforementioned basic
 papers of Linnik \cite{key1} and of R\'enyi \cite{key2}. is based on
 a treatment of the exponential sum 
\begin{equation*}
T(x) : \sum _{n \in \gamma}e(nx)\cdot  x \in \mathbb{R}, \quad e (u): =
e^{2\pi iu} \tag{0.52}\label{eq0.52}  
\end{equation*}

Here the essential use is made of the Farey dissection from the method
of Hardy and Littlewood and of Parseval's formula. In fact, the close
connection with $T(x)$ stems from the identity (cf. \eqref{eq7.1}) 
\begin{equation*}
\sum^{p-1}_{\ell =1} |T (\frac{\ell} {p})|^2 = p \sum^p _{\ell =1}(S(p,
\ell)-\frac{S}{P})^2= p D (p),\tag{0.53} \label{eq0.53} 
\end{equation*}
the second equality being \eqref{eq0.5}. For the proof of the first
equality we need only note that the left-hand side expression 
\begin{align*}
=-S^2+ \sum _{\ell =1}^p | T(\frac{\ell}{p})|^2 & = -S^2+ \sum_{\ell =
1}^p \sum
_{n_{1}\in\gamma} \sum_{n_{2}\in \gamma}e(\frac{\ell}{p}(n_1-n_2))\\ 
&= -S^2 + \sum _{n_{1}\in \gamma }\sum_{n_{2}\in \gamma} ( \sum^{p}_{\ell}e (\frac {\ell}{p}(n_1-n_2)))
\end{align*}
and further, by \eqref{eq1.22}. the inner sum in the last expression
$= p$ or $= 0$ according as $(n_1-n_2)$ is $\equiv 0$ or $\nequiv 0
\mod p$, so that our expression 
$$
=-S^2 + \sum _{\ell =1}^p S^2 (p,\ell ) = p(\sum _{\ell =1}^p
(S(p,\ell )- \frac{S}{P})^2) 
$$
by \eqref{eq0.3}. Hence the form (D), which we have considered so far
to be the basis  of the large sieve, amounts to asking for an upper
bound for  
\begin{equation*}
 \sum_{p \leq Q}\sum_{\ell =1}^{p-1} |
 T(\frac{\ell}{p})|^2.\tag{0.54}\label{eq0.54}  
\end{equation*}

Another\pageoriginale method of proof that should be mentioned here
takes a more 
general point of view and may serve to simplify the understanding of
the large sieve method. It relates the problems with certain results
in an inner product space. This has been already developed by R\'enyi in
his early papers (\cite{key4}, \cite{key7}), where he refers to Boas
\cite{key1} and also 
to Bellman \cite{key1}  for their extensions of classical results to
`quasi-orthogonal' functions. Roth's work \cite{key2} has used king of
combination of both methods. 

Further methods and details of proofs with respect to the theory of
the large  sieve will not be given here, but rather would  be
mentioned in appropriate chapters, in part icular under chapter
\ref{chap2}. However, since we will not have the opportunity to use the second
method mentioned above, we shall present here a basic result due to
A. Selberg (cf. Bombieri \cite{key4}). The proof is very elegant and the
result seems to me to be most suitable for giving an idea of this
method, which consists then in choosing appropriate functions in an
application of \eqref{eq0.55}.  

\begin{theorem}\label{chap0-thm0.1}%the 0.1
Let $f, \varphi_1 ,\ldots , \varphi_R$ be elements of an inner product
space over $\mathbb{C}$. Then
\begin{equation*}
\sum_{r=1}^R \frac{|(f , \varphi_r) |^2}{\sum_{s=1}^{R}|(\varphi_r,
  \varphi_S)|} \leq ||f||^2.\tag{0.55}\label{eq0.55} 
\end{equation*}
\end{theorem}

\begin{proof}%pro
For any complex numbers $c_r$, $1 \leq r \leq R$, we have by the
Bessel's inequality argument, 
\begin{equation*}
\begin{cases}
||f||^2 -2 Re \sum\limits_{r=1}^R c_r \overline{(f , \varphi_r)}
+ \sum\limits_{r,s=1}^R c_r \overline{c}_S(\varphi_r, \varphi_S)=\\ 
= || f- \sum\limits_{r=1}^{R} c_r \varphi_r || ^2 \ge 0.
\end{cases}\tag{0.56}\label{eq0.56}
\end{equation*}

Using here
\begin{equation*}
\begin{cases}
\sum\limits_{r,s=1}^R c_r \overline{c}_S
(\varphi_r, \varphi_s)\leq \sum\limits_{r,s=1}^R
(\frac{1}{2}|c_r|^2+ \frac{1}{2}|c_S|^2)|(\varphi_r, \varphi _S)|=\\ 
= \sum\limits_{r=1}^R |c_r|^2 \sum\limits^R_{s=1} |(\varphi _r, \varphi_s)|
\end{cases}\tag{0.57}\label{eq0.57}
\end{equation*}\pageoriginale
and then choosing 
\begin{equation*}
c_r = \frac{(f,\varphi_r)}{\sum\limits_{s=1}^R | ( \varphi_r , \varphi_s)|}
\tag{0.58}\label{eq0.58} 
\end{equation*}
the proof is completed.
\end{proof}

From Theorem \ref{chap0-thm0.1} or variants of it the main tools for
this (second) method can be derived (cf. Montgomery \cite{key5} (pp. 4-8)
and Huxley \cite{key7} (pp. 29-30)). A simple example is that as an
immediate consequence of \eqref{eq0.55} we obtain, under the same
assumptions. `Bellman's inequality' (Bombieri \cite{key3}) 
\begin{equation*}
 \sum_{r=1}^{R} |(f, \varphi _r )|^2 \leq ||f||^2 \underset{1 \leq
 r \leq R }\max \sum_{s=1}^R | (\varphi _r, \varphi _s
 )|.\tag{0.59}\label{eq0.59} 
\end{equation*}
\eqref{eq0.55} as well as \eqref{eq0.59} generalize Bessel's
inequality, or (for $R=1$) Schwarz's inequality, in an inner product
space to which they reduce when $\{ \varphi_r\}$ happen to be
orthonormal. 

The importance of this method for the modern development of the large
sieve has been noticed by Bombieri, Gallagher and $A$. Selberg
(cf. Bombieri \cite{key3}, \cite{key4}). 

The further development in the theory of the large sieve has shown
that with the work of Roth one had already come close to best
possible results. The next decisive step in this direction was made in
an important paper by Bombieri \cite{key1} (also independently in a paper by
A.I.  Vinogradov \cite{key1}. Apart from the important deductions he
made from his result and other details (not to mention here), the main
features  of this progress in theoretical respect were (i) the
extension of his ((D)-type) result from an  estimate of \eqref{eq0.54}
to an estimate (of the larger gum) where\pageoriginale 
the summation is extended over
all  natural numbers $q \leq Q$ instead of over only primes $p\leq Q$
(cf. \eqref{eq2.3}). (ii) keeping thereby not only the quality of Roth's
result but even removing a log-factor, and (iii) obtaining an
explicit $\ll$-constant. which is of considerable importance in
certain applications. 

After having recalled some arithmetical results in Chapter \ref{chap1}  we
shall take up this modern version of the large sieve in Chapter
\ref{chap2}. Applications to character sums are possible, as has  been
already mentioned in \eqref{eq0.32}. and will be  treated in Chapter
\ref{chap3}. Important further applications of the large sieve, not mentioned
so far, to Dirichlet series which were first noticed by Davenport
(cf. Montgomery \cite{key2})  are dealt with in Chapter
\ref{chap4}. This theme is 
continued in Chapter \ref{chap5} where certain `hybrid' forms of the large
sieve for applications to Dirichlet series also occur. Chapter \ref{chap6} is
devoted to a survey on special applications of the large sieve   to
Dirichlet series and also to some problems of number theory: In
Chapter \ref{chap7} we shall turn to the large sieve in its
arithmetical form (B). Lastly, in Chapter \ref{chap8} we give an
application of the large 
sieve to a special problem, Viz. Brun-Titchmarsh theorem. of number
theory. 

So far very little has been said about the applications of the large
sieve. As can be seen from the description of the contents of the
following chapters given above  there is a great variety of
applications. Many of these are of a `statistical' nature, in the
sense that they are concerning certain averages. Many results of
number theory are known to be consequences of certain unproved
hypotheses. However, in many cases we are able to apply the
aforementioned statistical statements  to obtain, strikingly, the same
results as those which one gets by assuming certain still unproved
hypotheses. 
  
 Apart from a first application in the construction of a non-basic
 essential component (Linnik \cite{key2}). Linnik \cite{key3} showed
 the power of his new method, via (A), in\pageoriginale a result about
 the least quadratic non-residue $\mod p$. R\'enyi's  first
 application of the large sieve yielded a surprising result in the
 direction of Goldbach's conjecture. In fact,   R\'enyi has been the
 first to prove that every sufficiently large even number can be
 written as a sum of a prime  and a number consisting of a
 \textit{bounded} number of prime factors. 
  
We shall not mention other applications in this historical
introduction but rather defer them to later (appropriate)
chapters. However, in keeping with the title of this chapter,
following the Notes for  this chapter we add a list of references to
works, upto the first paper of Bombieri, in chronological
order. This list also includes papers, upto this point, which deal
only with the applications of the large sieve. 
  
There are also generalizations of the large sieve in  various
directions. Some of these papers are given  in a second list of
references following the above one. 

\medskip  
\begin{center}
\textbf{NOTES}
\end{center}  
  
In order not to make this purely historical introduction too long,  we
have selected only some significant results that seem to be
illuminating for our way of presenting the subject. For further
information the reader is referred to the surveys in Barban \cite{key10},
Halberstam and Roth \cite{key1}.  Davenport \cite{key1}, Roth
\cite{key4} Montgomery \cite{key5}. Huxley \cite{key7}  and Bombieri
\cite{key5}, \cite{key6}.  

\eqref{eq0.25}, \eqref{eq0.26}: By using $|e^{ix}-1|  \leq |x|$
instead of Linnik's estimate $|e^{ix}-1| \leq e |x|$ the constant $20$
can be replaced by $4$.
 
\eqref{eq0.33}: R\'enyi's paper (\cite{key2}, (Lemma 1)) contains
only the special case $ \omega (p)= p ^{8/9}$, $c=p^{1/9}$.  However,
as he has  pointed out   (\cite{key7},  (Theorem 2)) his method also
applies to the general case. 

\eqref{eq0.39}: Note\pageoriginale that by $(D_0)$, \eqref{eq0.39} holds
also for $Q\geq\sqrt{N}$.
 
\eqref{eq0.40}: His method gives actually a factor $\dfrac{1}{3}$ for
the second term, and also the condition on $K_0$ may be slightly
relaxed. 
 
\eqref{eq0.39}, \eqref{eq0.40}: In general $(D_3)$ yields the better
result for $Q$ in the vicinity of $N^{1/3}$ whereas $(D_4)$ becomes
superior when $Q$ tends to $\sqrt{N}$. Also $(D_4)$ has been the first
result to yield non-trivial estimates upto the vicinity of $N$. 
  
\eqref{eq0.42}: Actually, the more appropriate choice of
$\epsilon=\dfrac{\sqrt{s}}{2\pi N}$ yields the factor $3\pi$ instead of
20, and a simple refinement of the proof gives even
$\dfrac{3}{2}\pi$. A consequence of this remark is, when applied to
$C$.,
\begin{equation*}
(C_{5})\qquad |\mathfrak{p}|\leq \frac{3}{2}\frac{\pi
    Nc^3}{\delta^{3/2}S} \text{ for } Q=\sqrt{N},\tag{0.60}\label{eq0.60}
\end{equation*}
a result which is always better than R\'enyi'es $((C_i))$.
 
\eqref{eq0.44}: Following Barban's method of proof for $(D_4)$ one
notices that in $(D^*)$ the factor $N^4$ can be replaced by $N^2S^2$. 

\eqref{eq0.56}: Here we have used that 
\begin{equation*}
\begin{cases}
||f||^2=(f,f),(f,g+h)=(f,g)+(f,h),\\
(cf,g)=c(f,g),(g,f)=\overline{(f,g)}
\end{cases}\tag{0.61}\label{eq0.61}
\end{equation*}
and (hence also)
\begin{equation*}
(f,cg)=\bar{c}(f,g). \tag{0.62}\label{eq0.62}
\end{equation*}

For an application of Theorem \ref{chap0-thm0.1} see the notes
following Chapter~\ref{chap2}.  

\begin{center}
{\bf{History of the large sieve. References in\\ chronological order}}.
\end{center}

Linnik \cite{key1}, Boas \cite{key1}, Linnik \cite{key2}, \cite{key3},
Bellman \cite{key1},  R\'{e}nyi \cite{key1}, \cite{key2},
\cite{key3}, \cite{key4}, \cite{key5}, \cite{key6}, \cite{key7},
\cite{key8}, Bateman, Chowla and  Erd\"os \cite{key1}, Kubiliyus
\cite{key1}, R\'{e}nyi \cite{key9}, Wang \cite{key1}, R\'{e}nyi
\cite{key10}, \cite{key11},  Stepanov \cite{key1},\pageoriginale Hua
\cite{key1}, R\'{e}nyi \cite{key12}, Barban \cite{key1}, Linnik
\cite{key4}, Barban \cite{key2}, Erd\"{o}s \cite{key1}, Barban
\cite{key3}, Gelfond and Linnik \cite{key1}, Pan \cite{key1},
Erd\"{o}s \cite{key2}, Pan \cite{key2}, Wang \cite{key2}, Levin
\cite{key1}, Barban \cite{key4}, \cite{key5}, \cite{key6}, Pan
\cite{key3}, Rieger \cite{key3}, Pan \cite{key4}, Barban \cite{key7},
Roth \cite{key1}, Levin \cite{key2}, Barban \cite{key8}, \cite{key9},
M. and S. Uchiyama \cite{key1}, Barban \cite{key10}, Wang, Hsieh and
Yu \cite{key1},  Halberstam and Roth \cite{key1}, Roth \cite{key2},
Buchstab \cite{key1}, A.I. Vinogradov \cite{key1}, Bombieri
\cite{key1}, 

\medskip 
\begin{center} 
\underline{Extensions of the large sieve} 
\end{center}


Andruhaev \cite{key1}, Fogels \cite{key1}, Goldfeld \cite{key3},
Hlawka \cite{key1}, \cite{key2}, Huxley \cite{key1}, \cite{key3},
\cite{key4}, Johnsen \cite{key1}, Rieger \cite{key1}, \cite{key2},
Samandarov \cite{key1}, Schaal \cite{key1}, Wilson \cite{key1}.
