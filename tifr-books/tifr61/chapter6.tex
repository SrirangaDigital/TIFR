
\chapter{Full Theory of Water Waves}\label{chap6}

IN\pageoriginale CHAPTERS \ref{chap4} AND \ref{chap5} we considered the approximate shallow water theory which has some advantages of simplicity, but also some inadequacies such as the behavior off-shore from a sloping beach. Here we deal with the full theory and some of its solutions.

\section{Conservation equations and the boundary value problem}\label{chap6:sec6.1}

Consider a 3-dimensional flow of water on a general sloping bottom. We assume that the water is inviscid and neglect surface tension. Let $V$ be a volume element enclosed in a smooth surface $S$ in the fluid. Let $\underline{u}(\underline{x},t)$ be the velocity of the fluid particle at the position $\underline{x}$ at time $t$. Let $\rho(\underline{x},t)$ and $p(\underline{x},t)$ denote the density and pressure of the fluid particle at $\underline{x}$ at time $t$. The outward drawn unit normal to the surface $S$ is denoted by $\underline{n}$. The equation of conservation of mass is 
\begin{equation}
\frac{d}{dt}\int\limits_V\rho\,dV=-\int\limits_S\rho n_ju_j\,dS. \tag{6.1}\label{chap6:eq6.1}
\end{equation}

We assume that all the quantities are sufficiently smooth. Then using the divergence theorem we obtain
\begin{equation}
\int\limits_V\frac{\partial\rho}{\partial t}\, dV+\int\limits_V\frac{\partial} {\partial x_j}\left(\rho u_j\right)\,dV=0.\tag{6.2}\label{chap6:eq6.2}
\end{equation}

Here the summation convention is used for repeated indices and this will be adopted throughout.

Since\pageoriginale \eqref{chap6:eq6.2} is true for all volume elements $V$ and the functions are smooth it follows that 
\begin{equation}
\frac{\partial\rho}{\partial t}+\frac{\partial}{\partial x_j}\left(\rho u_j\right)=0.\tag{6.3}\label{chap6:eq6.3}
\end{equation}

Equation \eqref{chap6:eq6.3} is the mass conservation equation or `continuity equation'.

We now derive the momentum equation. If $F$ denotes the body force per unit mass acting on the volume $V$, then the conservation of momentum in the $i^{\rm th}$ direction can be expressed by
\begin{equation}
\frac{d}{dt}\int\limits_V\rho u_i\,dV= -\int\limits_S\left(\rho u_i\right)n_ju_j\,dS-\int\limits_Sn_ip\,dS+\int\limits_V F_i\rho \,dV\tag{6.4}\label{chap6:eq6.4}
\end{equation}

The first term on the right hand side of \eqref{chap6:eq6.4} denotes the transport of momentum, the second term the surface force acting on $S$ and the third term the body force acting on $V$. Equation \eqref{chap6:eq6.4} can be written as 
$$
\frac{d}{dt}\int\limits_V\rho u_i\,dV+\int\limits_S\left(\rho u_iu_j+p\delta_{ij} \right)n_j\,dS=\int\limits_V F_i\rho \,dV.
$$

Using the divergence theorem, we have 
$$
\int\limits_V\left\{\frac{\partial}{\partial t}\left(\rho u_i\right)+\frac{\partial}{\partial x_j}\left(u_iu_j+p\delta_{ij}\right)-\rho F_i\right\}\,dV=0.
$$

Since this is true for all $V$ we obtain 
\begin{equation}
\frac{\partial}{\partial t}\left(\rho u_i\right)+\frac{\partial}{\partial x_j} \left(\rho u_iu_j\right)+\frac{\partial p}{\partial x_i}=\rho F_i. \tag{6.5}\label{chap6:eq6.5}
\end{equation}

Under normal conditions for water waves it is reasonable to assume that $\rho =$ constant. This together with \eqref{chap6:eq6.3} implies 
\begin{equation}
\frac{\partial u_j}{\partial x_j}=0.\tag{6.6}\label{chap6:eq6.6}
\end{equation}\pageoriginale

Using \eqref{chap6:eq6.6} and $\rho=$ constant, the momentum equation \eqref{chap6:eq6.5} can be written as
\begin{equation}
\frac{\partial u_i}{\partial t}+u_j\frac{\partial u_i}{\partial x_j}= -\frac{1}{\rho}\;\frac{\partial p}{\partial x_i}+F_i. \tag{6.7}\label{chap6:eq6.7}
\end{equation}

In water waves the body force $\underline{F}$ is the gravitational acceleration $g$ acting vertically downwards. We shall eventually use a mixed notation with the vertical coordinate replaced by $y$. With that in mind (but not making the change yet), we denote the unit vector in the vertical direction by $\underline{j}$. Then 
\begin{equation}
\underline{F}=-g\underline{j}.\tag{6.8}\label{chap6:eq6.8}
\end{equation}

\subsection*{\bf Potential flow}.

In vector notation, the equations \eqref{chap6:eq6.6} and \eqref{chap6:eq6.7} are 
\begin{align}
& \nabla.\underline{u}=o,\tag{6.9}\label{chap6:eq6.9}\\
& \frac{\partial\underline{u}}{\partial t}+(\underline{u}.\nabla)\underline{u} = -\frac{1}{\rho}\nabla p-g\underline{j}.\tag{6.10}\label{chap6:eq6.10}
\end{align}

The vorticity is defined by 
\begin{equation}
\underline{\omega}=\nabla\times\underline{u};\tag{6.11}\label{chap6:eq6.11}
\end{equation}
if \eqref{chap6:eq6.9} is first written as 
\begin{equation}
\frac{\partial u}{\partial t}+\nabla\left(\frac{1}{2}\underline{u}^2\right)+ \underline{\omega}\times\underline{u}=-\frac{1}{\rho}\nabla p-g\underline{j}, \tag{6.12}\label{chap6:eq6.12}
\end{equation}
and then the curl is taken, we obtain
\begin{equation}
\frac{\partial\omega}{\partial t}+(\underline{u}.\nabla)\underline{\omega}= (\underline{\omega}.\nabla)\underline{u}.\tag{6.13}\label{chap6:eq6.13}
\end{equation}

It\pageoriginale is clear that $\underline{\omega}\equiv 0$ is a solution of this vorticity equation. Furthermore, if $\underline{\omega}=0$ for $t=0$ then (under mild conditions on $\nabla\underline{u}$) $\underline{\omega}\equiv 0$ for all $t$. 

Hence we assume that $\underline{\omega}\equiv 0$; \ie the flow is `irrotational'. This implies 
\begin{equation}
\underline{u}=\nabla\phi\tag{6.14}\label{chap6:eq6.14}
\end{equation}
for some scalar field $\phi$; $\phi$ is called the velocity potential.

The first equation \eqref{chap6:eq6.9} then gives 
\begin{equation}
\nabla^2\phi=0,\tag{6.15}\label{chap6:eq6.15}
\end{equation}
and the second equation \eqref{chap6:eq6.10} becomes
\begin{align*}
&\nabla\left(\frac{\partial\phi}{\partial t}\right)+\nabla\left( \frac{1}{2}u^2\right)+\nabla\left(\frac{p}{\rho}\right)+ \nabla(gy)=0,\\
\ie\qquad &\frac{\partial\phi}{\partial t}+\frac{1}{2}(\nabla\phi)^2+ \frac{p-p_0}{\rho}+gy=\quad\text{function of}\quad t.
\end{align*}

Without loss of generality we can set this function of $t$ to be zero since otherwise it can be absorbed in $\phi$, but it will be convenient to keep an arbitrary constant $p_0$. We have 
\begin{equation}
\frac{p_0-p}{\rho}=\frac{\partial\phi}{\partial t}+\frac{1}{2}(\nabla\phi)^2+ gy.\tag{6.16}\label{chap6:eq6.16}
\end{equation}

Relation \eqref{chap6:eq6.16} gives the pressure $p$ in terms of the potential $\phi$. 

\subsection*{\bf Boundary conditions}.

At the bottom surface (or any other fixed solid surface), the normal velocity must be zero, so we have 
\begin{equation}
\underline{u}.\underline{n}=\frac{\partial\phi}{\partial n}=0. \tag{6.17}\label{chap6:eq6.17}
\end{equation}

At\pageoriginale the free surface of the water we give two conditions. They are coupled, but we may think of one as essentially determining the free surface, and the other as a boundary condition for \eqref{chap6:eq6.15}. 

The first one is obtained from the defining property of the free surface, namely that 
\begin{equation}
\text{normal velocity of the surface}=\text{normal velocity of the fluid}. \tag{6.18}\label{chap6:eq6.18}
\end{equation}

To implement this, let the water surface at time $t$ be given by 
\begin{equation}
f\left(x_1,x_2,x_3,t\right)=0.\tag{6.19}\label{chap6:eq6.19}
\end{equation}

In terms of $f$ we have:
\begin{equation}
\begin{aligned}
& \text{Unit normal vector}\quad \underline{n}=\frac{\nabla f}{|\nabla f|},\\
& \text{Normal velocity of surface}\quad = -\frac{f_t}{|\nabla f|}.
\end{aligned}\tag{6.20}\label{chap6:eq6.20}
\end{equation}

(To show \eqref{chap6:eq6.20} we consider two successive positions of the surface at times $t$ and $t+dt$. Points $\underline{x}$ on the first and $\underline{x}+\underline{n}\,ds$ on the second are separated by distance $ds$ along the normal. We have 
\begin{align*}
 f(\underline{x},t)=0 &\\
 f(\underline{x}+\underline{n}\,ds,t+dt) &=f(\underline{x},t)+(\underline{n}. \nabla f)\,ds+f_t\,dt+\cdots\\
  &=0.
\end{align*}

Therefore
$$
\frac{ds}{dt}=-\frac{f_t}{|\nabla f|},
$$
and this is the normal velocity). Thus \eqref{chap6:eq6.18} implies
\begin{equation}
\begin{aligned}
& \underline{u}.\frac{\nabla f}{|\nabla f|}=-\frac{f_t}{|\nabla f|},\\
\ie\qquad & f_t+u_1\frac{\partial f}{\partial x_1}+u_2\frac{\partial f}{\partial x_2}+u_3\frac{\partial f}{\partial x_3}=0. 
\end{aligned}\tag{6.21}\label{chap6:eq6.21}
\end{equation}\pageoriginale

We now introduce the mixed notation, in which $y$ is the vertical coordinate and $v$ is the vertical velocity, and take $x_3\equiv y,u_3\equiv v$. Then if $f$ is specialized to 
$$
f\left(x_1,x_2,x_3,t\right)=\eta\left(x_1,x_2,t\right)-y,
$$
\eqref{chap6:eq6.21} becomes 
$$
\eta_t+\phi_{x_1}\eta_{x_1}+\phi_{x_2}\eta_{x_2}=v
$$
At the surface
\begin{gather}
y=\eta\left(x_1,x_2\right),\tag{6.22}\label{chap6:eq6.22}\\
\intertext{we have}
\eta_t+\phi_{x_1}\eta_{x_1}+\phi_{x_2}\eta_{x_2}=v.\tag{6.23}\label{chap6:eq6.23}
\end{gather}

The second boundary condition (if surface tension is ignored) is that the pressure in the water must equal the pressure in the air at the interface. Since the changes in air pressure are small (because its density is small), it is a good approximation to take the air pressure to be a constant $p_0$. If this is taken as the $p_0$ in \eqref{chap6:eq6.16}, the boundary condition becomes
$$
\phi_t+\frac{1}{2}(\nabla\phi)^2+g\eta =0.
$$
Thus the full boundary value problem is formulated as follows 
\begin{equation}
\left.
\begin{aligned}
& \nabla^2\phi=0,\\
&\frac{\partial\phi}{\partial n}=0\quad\text{at the bottom surface},\\
&\phi_t+\frac{1}{2}(\nabla\phi)^2+g\eta=0\quad\text{at}\quad y=\eta\left(x_1, x_2,t\right),
\end{aligned}
\right\}\tag{6.24}\label{chap6:eq6.24}
\end{equation}
where\pageoriginale the surface $y=\eta(x_1,x_2,t)$ is determined by
\begin{equation}
\eta_t+\phi_{x_1}\eta_{x_1}+\phi_{x_2}\eta_{x_2}=\phi_y. \tag{6.25}\label{chap6:eq6.25}
\end{equation}

\section{Linearized theory}\label{chap6:sec6.2}

Assume $\phi,\nabla\phi,\phi_t,\eta$, etc., are all small, \ie consider small disturbances. To the first order approximation the boundary conditions at the free surface become
\begin{equation}
\left.
\begin{aligned}
\eta_t&=\phi_y,\\
\phi_t&+g\eta=0,
\end{aligned}
\right\}\text{at}\quad y=0.\tag{6.26}\label{chap6:eq6.26}
\end{equation}

These can be combined into
\begin{equation}
\phi_{tt}+g\phi_y=0\quad\text{at}\quad y=0.\tag{6.27}\label{chap6:eq6.27}
\end{equation}

Assuming the bottom surface to be horizontal, the boundary condition at the bottom surface becomes
\begin{equation}
\phi_y=0\quad\text{at}\quad y= -h_0.\tag{6.28}\label{chap6:eq6.28}
\end{equation}

In the case of one dimensional waves, let
\begin{equation}
\phi=\Phi(y)e^{ikx-i\omega t}\tag{6.29}\label{chap6:eq6.29}
\end{equation}
be a solution of \eqref{chap6:eq6.15}, \eqref{chap6:eq6.27}, \eqref{chap6:eq6.28}. Then the ordinary differential equation satisfied by $\Phi$ is 
\begin{equation}
\Phi_{yy}-k^2\Phi=0,\tag{6.30}\label{chap6:eq6.30}
\end{equation}
and we have
\begin{align}
\text{Free surface:}\quad \Phi_y &-\frac{\omega^2}{g}\Phi=0\quad\text{at}\quad y=0,\tag{6.31}\label{chap6:eq6.31}\\
\text{Bottom:}\quad \Phi_y &=0\quad\text{at}\quad y=-h_0. \tag{6.32}\label{chap6:eq6.32}
\end{align}

We\pageoriginale note that
\begin{equation}
\Phi=\cos h k\left(h_0+y\right),\tag{6.33}\label{chap6:eq6.33}
\end{equation}
is a solution of \eqref{chap6:eq6.30} and satisfies the boundary condition \eqref{chap6:eq6.32}. The boundary condition at the free surface \eqref{chap6:eq6.31} is satisfied provided
\begin{equation}
\omega^2=gk\tan h \,kh_0.\tag{6.34}\label{chap6:eq6.34}
\end{equation}

Equation \eqref{chap6:eq6.34} is an important relation called the ``dispersion relation''. In \eqref{chap6:eq6.34} $k$ denotes the wave number, the frequency of the wave and $c=\frac{\omega}{k}$ the phase velocity. From \eqref{chap6:eq6.34} we obtain
\begin{equation}
c^2=\frac{g}{k}\tan h\,kh_0.\tag{6.35}\label{chap6:eq6.35}
\end{equation}

From formula \eqref{chap6:eq6.35}, we note that the phase velocity depends on $k$, which means that for a general disturbance the waves will disperse. The equations \eqref{chap6:eq6.26}, \eqref{chap6:eq6.29} and \eqref{chap6:eq6.33} imply 
\begin{align*}
\eta &= -\frac{1}{g}\phi_t|_y=0\\
&=\frac{i\omega}{g}\cos h\,kh_0e^{ikx-i\omega t}.
\end{align*}

Hence a solution to the problem \eqref{chap6:eq6.24} in the linearized theory is 
\begin{equation}
\left.
\begin{aligned}
\eta&=Ae^{ikx-i\omega t},\\
\omega^2 &= gk\tan h\,kh_0,\\
&= -\frac{ig}{\omega}\;\frac{A\cos h \,k(h_0+y)}{\cos h\,kh_0} e^{ikx-i\omega t},
\end{aligned}
\right\}\tag{6.36}\label{chap6:eq6.36}
\end{equation}
where\pageoriginale $A=\frac{i\omega}{g}\cos h\,kh_0$.

In the shallow water (long wave) theory the wave length $\lambda=\frac{2\pi}{k}$ is large compared with $h_0$; therefore $kh_0\ll 1$. As $kh_0\to 0,\tan h\,kh_0\simeq kh_0$, and we have 
$$
\omega\simeq\pm\sqrt{gh_0}k.
$$

Hence
\begin{equation}
c\simeq\frac{\omega}{k}\simeq\pm\sqrt{gh_0}.\tag{6.37}\label{chap6:eq6.37}
\end{equation}

This is what we found in linear shallow water theory. The solution in this case is nondispersive and hyperbolic. This shows that the additional terms change the character of the wave. For a full discussion of the relation of shallow water theory to the full theory, see \cite{key1} Section 13.10.

In the other extreme of deep water, \ie $kh_0\gg 1$,
\begin{equation}
\omega\simeq\pm\sqrt{gk}\quad\text{and}\quad c\simeq\pm \sqrt{\frac{g}{k}}. \tag{6.38}\label{chap6:eq6.38}
\end{equation}

Equation \eqref{chap6:eq6.38} tells us that the waves are still dispersive but simplifies the formula. Here
$$
\phi\simeq -\frac{ig}{\omega}Ae^{ky}e^{ikx-i\omega t}.\,(y<0)
$$

The above formulae give good results even when the depth is only about twice the wave length.

\subsection*{\bf Initial value problem}.

We want to find the solution for $\eta$ when the initial conditions
\begin{equation}
t=0:
\begin{cases}
\eta =\eta_0(x)\\
\eta_t=\eta_1(x)
\end{cases}\tag{6.39}\label{chap6:eq6.39}
\end{equation}\pageoriginale
are given. From \eqref{chap6:eq6.36} we see that $\eta$ will be of the form 
\begin{gather}
\eta=\int\limits_{-\infty}^\infty A_1(k)e^{ikx-iW(k)t}\,dk+\int\limits_{-\infty}^\infty A_2(k) e^{ikx-iW(k)t}\,dk, \tag{6.40}\label{chap6:eq6.40}\\
\intertext{where}
W(k)=\left\{gk\tan h\,kh_0\right\}^{1/2}.\tag{6.41}\label{chap6:eq6.41}
\end{gather}

Equations \eqref{chap6:eq6.39} and \eqref{chap6:eq6.41} imply
\begin{align*}
\eta_0 &= \int\limits_{-\infty}^\infty \left\{A_1(k)+A_2(k)\right\}e^{ikx}\,dk,\\
\eta_1&= \int\limits_{-\infty}^\infty iW(k)\left\{-A_1(k)+A_2(k)\right\} e^{ikx}\,dk.
\end{align*}

Using Fourier inversion theorem, we obtain
\begin{equation}
\begin{aligned}
A_1 &=\frac{1}{2}\int\limits_{-\infty}^\infty \left\{\eta_0(x)-\frac{i}{W}\eta_1 (x)\right\}e^{-ikx}\,dx,\\
A_2 &= \frac{1}{2}\int\limits_{-\infty}^\infty \left\{\eta_0(x)+\frac{i}{W} \eta_1 (x)\right\}e^{-ikx}\,dx.
\end{aligned}\tag{6.42}\label{chap6:eq6.42}
\end{equation}

Equations \eqref{chap6:eq6.41}, \eqref{chap6:eq6.42} give the solution for the initial value problem.


