\chapter{Lecture}\label{lec23} %%%23

\setcounter{section}{11}
\section{Regularity of Green's Kernels}\label{lec23:sec12}%sec 12.

\subsection{}\label{lec23:sec12:subsec1}

In\pageoriginale \S\ \ref{lec7:sec3:subsec5} we have defined Green's kernel of the operator $A$
associated with an elliptic sesquilinear from $a(u, v)$ on $V$ such
that $H^m_o (\Omega) \subset V \subset H^m (\Omega), Q$ being $L^2
(\Omega)$ say. We recall that $a(u, v)$ begin $V$ elliptic, $A$ is an
isomorphism of $N$ onto $Q'$. Hence $A^{-1} = G$ is an isomorphism of
$Q'$ onto $N$. Since $\mathscr{D} (\Omega)$ is dense in $Q'$, by
Schwartz's kernel theorem $A^{-1} = G$ is given by $G_{x, y}
\in \mathscr{D}' (\Omega_x \times \Omega_y). G_{x, y}$ is
called the kernel of the operator $A$.  

Let $a^* (u, v) = \overline{a(v, u)} ; a^* (u, v)$ is also
$V$ elliptic and defines a space $N^*$ and an operator $A^*$. Let
its kernel be $G^*_{x, y}$. If $T_{x, y} \in \mathscr{D}'
(\Omega_x, \Omega_y)$, then $T_{y, x}$ will be defined by setting on
the everywhere dense set $\mathscr{D} (\Omega_x) \times \mathscr{D}
(\Omega_y) $ in $\mathscr{D}(\Omega_x \times \Omega_y)$.  
$$
T_{y, x} (\varphi (x) \cdot \psi (y)) = T_{x, y } (\psi (x)
\varphi (y)).  
$$

We denote $\mathscr{D} (\Omega_x) $ by $D_x, \mathscr{D} (\Omega_x
\times \Omega_y)$ by $\mathscr{D}_{x, y}$ and so on. We have the 
\begin{proposition}\label{lec23:sec12:subsec1:prop12.1} %proposition 12. 1
  $G_{xy} = G^*_{y, x}$. 
\end{proposition}

Let $\varphi, \psi \in \mathscr{D} (\Omega_x)$. We have to
verify that $\langle G \varphi, \bar{\psi} \rangle = \langle \varphi,
G^* \psi \rangle$. Let $G \varphi = u \in N$ and $G^* \psi = w
\in N^*$. Then $\varphi = Au$ and $\psi = A^* w$. Hence we have
to verify that $\overline{\langle u, A^* w \rangle}= \langle A \psi, \bar{\omega} \rangle$. This follows since $\langle
u, \overline{A^* w} \rangle = \overline{a^* (w, u)} = a(u, w) = \langle A u,
\bar{w}\rangle$.
  
\begin{definition}\label{lec23:sec12:subsec1:def12.1} %definition 12. 1
  An element $G_{x, y} $ in $\mathscr{Q}' (\Omega_x \times \Omega_y)$
  will be called a {\em kernel}.  
\end{definition}

\begin{definition}\label{lec23:sec12:subsec1:def12.2} % definition 12. 2
  $A$\pageoriginale kernel is semi-regular, with respect to $x$, if $G_{x, y}$ is given
  by a $C^\infty$ function of $\Omega_x$ into $\mathscr{D}'_y$. We write
  it then as $G(x)_y$ or $G_y(x)$. 
\end{definition}

\begin{definition}\label{lec23:sec12:subsec1:def12.3}%defin 12.3
  A kernel is regular if it is semi-regular with respect to $x$ and $y$.
\end{definition}

\begin{definition}\label{lec23:sec12:subsec1:def12.4}%defi 12.4
  A kernel is very regular if it is regular and a $C^\infty$ function
  outside the diagonal. 
\end{definition}

If $G_{x, y}$ is semi-regular it is an element of $\mathscr{E}_x
\hat{\otimes} \mathscr{D}^{'}_{y} = \mathscr{E} (x,
\mathscr{D}^{'}_{y})$. Hence it defines a mapping $G : \mathscr{D}_y
\to \mathscr{E}_z$ given by $\int G(x)_y \varphi (y) dy \in
\mathscr{E}(x)$ for $\varphi (y) \in \mathscr{D}_y$. Conversely
by Schwartz's kernel theorem, any linear mapping $G : \mathscr{D}_y
\to \mathscr{E}_x$ is given by a semi-regular kernel. 

We now with to consider conditions on $a(u, v)$ so that the kernel
$G_{x, y}$ be very regular.
 
\begin{definition}\label{lec23:sec12:subsec1:def12.5}%defi 12.5
  A partial differential operator $A$ defined in $\Omega$ with
  $C^\infty$ coefficients is said to by {\em hypo-elliptic} if for $s
  \in \mathscr{D}^{'} (\Omega)(AS)_{\theta} \in
  \mathscr{E}(\theta)$ implies that $S \in \mathscr{O}$,
  for every $0 \subset \Omega$. 
\end{definition}

For example, if $a(u, v)$ is $V$ elliptic, $\Omega$ is bounded with
smooth boundary, then by the results of \S\ \ref{lec13:sec8}, it follows that $A$ is hypo-elliptic. 

\begin{theorem}\label{lec23:sec12:subsec1:thm12.1}%the 12.1
  Let $|a (u, u)| \ge \alpha |u |^2_V$ and the operators $A$ and $A^*$
  be hypo-elliptic. Then $G_{x, y}$ is very regular. 
\end{theorem}

\begin{proof}
First we prove $G_{x, y}$ is regular. We know $G$ is an isomorphism of
$Q'$ onto $N$. Let $\varphi \in \mathscr{D}(\Omega)$. Then $G
\varphi = u \in N$ and $Au =  \varphi$. By hypo-ellipticity of
$A$ it follows that $u \in \mathscr{E}$. Hence $G$ defines a
mapping of $\mathscr{D}(\Omega)$ onto $N \cap \mathscr{E}$. By closed
graph theorem this mapping is continuous and hence Schwartz's kernel
theorem is given by a semi-regular kernel $G(x)_y$. Hence $G_{x, y} =
G^* (x)_y$. Similarly, since $A^*$\pageoriginale is hypo-elliptic $G^{*}_{x, y} =
G^{*}(x)_y$. But by proposition 12.1, $G_{x, y} =  G^{*}_{y, x}$ and
hence $G_{x, y} = G^{*}_{x}(y)$. This shows that $G_{x, y}$ is
regular. 
\end{proof}

To complete the proof we have to show that outside the diagonal it is
a $C^\infty$ function. 

Let $O_1$ and $O_2 \subset \Omega$ be two open sets such
that $O_1 \cap O_2 = \phi$. Let $T \in
\mathscr{E'}(\mathscr{O}_2)$. Then $\int G_x (y) T_y d \mu$ is defined
and is an element of $\mathscr{D}_x$ say $S_x$ such that
$AS=T$. Restricting $A, S, T$ to $O_1$ since $T = 0$ on
$O_1$ by hypo-ellipticity of $A$ on $O_1$, we have $S
O_1 \in \mathscr{E}(O_2)$. Hence $T \to \int G_x (y) T_y d
\mu$ is a mapping of $\mathscr{E}'(O_2)$ onto
$\mathscr{E}(O_1)$, which on account of the closed graph
theorem, is continuous. Hence 
$$
G \in L (\mathscr{E'}(O_2) ; \mathscr{E} (O_1)
\simeq \mathscr{E}(O_1) \hat{\otimes} \mathscr{E}(O_1 =
\mathscr{E}(\theta_1 \times \theta_2). 
$$

That is to say $G$ is $C^\infty$ in $O_1 \times
O_2$. Since $O_1$ and $O_2$ are any two open set
such that $O_1 \cap O_2 = \phi, G$ is a $ C^\infty $
function outside the diagonal. 

\begin{remark*}
  For a more detailed study, see Malgrange \cite{k12}.

  2. The extension of the mapping $G:Q' \to N$.
\end{remark*}

Under the hypothesis of theorem \ref{lec23:sec12:subsec1:thm12.1}, $G$ defines an algebraic
isomorphism of $Q' \cap \mathscr{E}$ onto $N \cap \mathscr{E}$. For if
$f \in Q' \cap \mathscr{E}$ and $Gf = u \mathscr{E} N$ . Then
$Au = f$. Hence by hypo-ellipticity of $A$ it follows that $u
\in \mathscr{E}$. Conversely if $u \in N \cap
\mathscr{E}$, then $Au = f \in Q' \cap \mathscr{E}$ and $Gf =
u$. If we could apply closed graph theorem, then it would follow that
$G$ is a topological isomorphism of $Q' \cap \mathscr{E}$ onto $N \cap
\mathscr{E}$, the intersections being given as usual the upper bound
topology. This is so, for example, if $Q$ is a Banach space. We have
then the 
\begin{theorem}\label{lec23:sec12:subsec1:thm12.2}%theo 12.2
  If\pageoriginale the closed graph theorem is applicable $G \in \mathscr{L}
  (Q' \cap \mathscr{E}, N \cap \mathscr{E})$ and is an
  isomorphism. Similarly $G^* \in \mathscr{L} (Q' \cap
  \mathscr{E}, N \cap \mathscr{E})$ and is an isomorphism. 
\end{theorem}

\subsection{}\label{lec23:sec12:subsec2}

Now we wish to consider the transpose $G$. Let us first consider the
Dirichlet problem so that $V = H^m_o = Q \cdot \mathscr{D}
(\Omega)$ is dense in $V$ and $N =V$. Hence by transposing $G$ we have
an isomorphism $^t G : V' + \mathscr{E}' \to V' +
\mathscr{E}'$. However since $\mathscr{D}(\Omega)$ is not always dense
in $N$, the dual of $N$ is not a space of distributions and hence we
do not consider directly the transport of $G$. Here the sums of
locally convex topological vector space $A$ and $B$ subspaces of an
algebraic vector space $F$ is topologized as follows : we consider the
mapping $(a, b) \to a+ b$ of $A \times B$  onto $A+B$ and put on $A +
B$ the finest locally convex topology such that this mapping is
continuous. If $Z$ is the kernel, then $A + B \approx A \times B / Z$. 

\begin{theorem}\label{lec23:sec12:subsec2:thm12.3}%theo 12.3
  Under the hypothesis of theorem \ref{lec23:sec12:subsec1:thm12.1}, if further, for every $S
  \in Q' \cap \mathscr{E}'$ there exists a sequence $\varphi_n
  \in \mathscr{D} (\Omega)$ such that $\varphi_n \to S$ in $Q'
  \cap \mathscr{E}'$, then $G : Q' \to N$ can be extended by continuity
  to $G : Q'  + \mathscr{E}' \to N + \mathscr{E}'$. 
\end{theorem}

\noindent Proof due to L. Schwartz (unpublished). 
We define first $G$ on $Q + \mathscr{E}'$. $G$ is already defined on
$Q'$. By theorem \ref{lec23:sec12:subsec1:thm12.1} is very regular and is given by $\int G
(x)_y \varphi (\varphi) dy \varphi (\varphi) \in Q'$ We cannot
use this at once to define it on $\mathscr{E}'$, for then the
integral itself is not in $\mathscr{E}'$. We proceed then as follows :
Let $\alpha (x, y) \in \mathscr{E} (\Omega_x \times \Omega_y)$
be a function with support in a neighbourhood of diagonal and equal to
$1$ in another neighbourhood of diagonal. Let 
$$
H_{x, y} = \alpha(x, y)G_{x, y} = H(x)_y = H_x (y).
$$

Hence\pageoriginale $H_{x, y}$ is regular. It is easily seen, since $G_{x, y}$ is a
$C^\infty$ function outside the diagonal that $A_x H_{x, y} - \delta
(x)_y = L (x, y) \in \mathscr{E} (\Omega_x \times
\Omega_y)$. Now let $T \in \mathscr{E}' (\Omega)$ with compact
support $K$ say. Since $\int\rho(x)_y T= T$, we have $T = LT + AHT$ where 
\begin{align*}
LT_x & = \int L (x, y) T_y dy \in \mathscr{E}.\\
(H, T)_x & = \int H_x (y) T_y dy \in \mathscr{D}'.
\end{align*}

But the supports of the mapping $y \to H_x (y)$ and $y \to L (x, y)$
are contained in the support of  $\alpha(x, y)$. By choosing the
support of near enough the diagonal, we may have the support of $HT$ and $LT$ in any arbitrary neighbourhood of the support
of $T$. Hence $HT \in \mathscr{E}'$ and $LT \in
\mathscr{D}$. We define $\tilde{G}T = HT + GLT \in
\mathscr{E}' + N$. We have to verify that if $\varphi \in
\mathscr{D} (\Omega)$, then $\tilde{G}(\varphi) = G (\varphi)$. This
follows as in general  
$$
A \tilde{G}T = AHT + AGLT = T - LT + LT = T.
$$

If $\varphi \in \mathscr{D} (\Omega), \tilde{G} \varphi = H
\varphi + GL \varphi \in \mathscr{D} + _N \subset_N$ and $A
\tilde{G} \varphi = \varphi$. Since $A$ is an isomorphism, $\tilde{G}
\varphi = G \varphi$. 

Now $G$ is continuous from $\xi^{'}_{k}(\Omega)$ into $\mathscr{E'} +
N$. This proves that $\tilde{G}$ does not depend on $\alpha$ and
$\tilde{G}$ can be extended to $\mathscr{E}' (\Omega)$ so that
$\tilde{G} : \mathscr{E}' (\Omega) \to \mathscr{E'} (\Omega) + N$ is
continuous. 

We denote now $\tilde{G}$ be $G$ itself. $G$ defines then a continuous
mapping $\theta$ from $Q' \times \epsilon' \to N + \epsilon'$ by
$\theta (f, s) = Gf + GS$. If we prove $\theta$ is zero on the kernel
of $Q' \times \mathscr{E'} \to Q' + \mathscr{E'}$, we shall have
proved that $\theta$ defines a mapping $G$ of $Q' + \epsilon' \to N
+ \mathscr{E}'$  as required. Let $f \in Q'$ and $S
\in \mathscr{E'}$ such that $f + S = 0$. Hence $f \in
Q' \cap \mathscr{E}'$. By assumption $(2)$, there exists $\varphi_n
\in \mathscr{D} (\Omega)$ which converges in $Q'$ and
$\mathscr{E'}$ to $f$. Hence - $\varphi_n$ converges to $S$ in
$\mathscr{E}'$ and we have $Gf + GS = \lim (G \varphi_n + G(\varphi_n)) = 0$.\pageoriginale 

\begin{coro*}
  We have $A. G = I_{Q' + \mathscr{E'} }$ and $G. A = I_{N +
    \mathscr{E'}}$ where $I_{Q' + \mathscr{E}'}$ and $I_{N +
    \mathscr{E'} }$ are the identity maps $Q' + \mathscr{E'}$ and $N +
  \mathscr{E'}$ respectively. 
\end{coro*}
For
$$
A(Gf + GS) = f + AGS = f + S
$$
and $G. A (u + S) = G Au + G~A~S = u + G~A~S$.

Since $GAS = S$ on $\mathscr{D} (\Omega)$ it is so on
$\mathscr{E'}$. This proves the  
\begin{theorem}\label{lec23:sec12:subsec2:thm12.4}%theo 12.4
Under the hypothesis of theorem \ref{lec23:sec12:subsec2:thm12.3}, $A$ is a topological isomorphism
from $N + \mathscr{E'}$ to $Q' + \mathscr{E'}$. 
\end{theorem}

The uniqueness of $G_x(y)$ is given by the following
\begin{theorem}\label{lec23:sec12:subsec2:thm12.5}% 12.5
  Under the hypothesis of theorem \ref{lec23:sec12:subsec2:thm12.3}, for $y \in \Omega$,
  $G_x (y)$ is defined as the solution of  
  \begin{align*}
    & A_x (G_x(y) = \delta_x (y)\\
    & G_x (y) \in N + \mathscr{E}'.
  \end{align*}
\end{theorem}

Consider $y \to \delta_x (y) \in \mathscr{E} (\Omega_y,
\mathscr{E'} (\Omega_x))$. Let $G(\delta_x(y)) = G_x (y)$. Then $y \in 
G_x(y)$ is a $C^\infty$ function from $\Omega \to N + \epsilon'$ we have $A_x (G_x(y)) = \delta_x(y)$, and $G_x(y)$ is the only distribution to verify the
equation in $N + \mathscr{E'}$. 
