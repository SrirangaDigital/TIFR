\chapter{Lecture}\label{lec4}
 
\setcounter{section}{2}
\subsection{The mapping \texorpdfstring{$\gamma$}{gamma}}\label{lec4:sec2:subsec4}
 
Let\pageoriginale $\mathscr{H}^m(\Omega)=H^m\Omega \cap 
\mathscr{D}(\bar{\Omega})$.  For 
 the function $f \in \mathscr{H}^m(\Omega)$, the restriction
 of $f$ to the boundary $\Gamma$ of $\Omega$ defines a function
 $\gamma f$ on $\Gamma$. We wish to know for what for what spaces
 $X(\Gamma)$ of function on $\Gamma$, this mapping $\gamma$ of
 $\mathscr{H}^L (\Omega)$ to $X(\Gamma)$ is continuous. If $\gamma$ is
 continuous, we can extend $\nu$ to $H^1(\Omega)$ if, for example,
 $\Omega$ has $1$-extension property, and $\gamma u$ for $u
 \in H'(\Omega)$ may be considered as generalized boundary value
 of $u$. 

 \begin{theorem}\label{lec4:sec2:subsec4:thm2.5}%the 2.5
   Let $\Omega = \{ x_n >0 \}$ so that $\Gamma = \{ x_n =0 \}$. Let
   $X(\Gamma)=L^2 (\Gamma)=L^2 (R^{n-1})$. Then $u \to \gamma u$ is a
   continuous mapping of $\mathscr{H}^1(\Omega)\to L^2 (\Gamma)$, i.e.,
   there exists a unique mapping $\gamma : H^1(\Omega)\to L^2(\Gamma)$
   which on $\mathscr{H}^1 (\Omega)$ is restriction. 
 \end{theorem} 

 \begin{proof}
   Let $x'=(x_1, \ldots ,x_{n-1})$. Let $\mathscr{O}(x_n)$ be a function
   defined for $x_n>0$, zero for $x_n >1$, and $0< \mathscr{O}(x_n)<1$ in
   $(0,1)$. We have $|u(x',0)|^2 =- \int ^\infty _0
   \dfrac{\partial}{\partial x_n}(u(x)\bar{u}(x)\theta (x_n))dx$. Hence 
 \end{proof} 

 \begin{align*}
   \int\limits _{R^{n-1}}|u (x',0)|^2 dx' & =-\int \limits_ \Omega
   \frac{\partial}{\partial x_n}(\theta (x_n)u(x)\bar{u}(x))dx\\ 
   &=\int \limits_ \Omega \theta '|u|^2 dx -\int \limits_ \Omega \theta
   (\frac{\partial u}{\partial x_n} \bar{u} + \frac{\partial
     \bar{u}}{\partial x_n})dx. 
 \end{align*} 
 So by Schwartz's inequality,
 \begin{align*}
\int_ \Gamma |u(x',0)|^2 dx' & \le C(\int _\Omega |u|^2 dx + \int
\limits_ \Omega |\frac{\partial u}{\partial x_n}|^2 dx),\\ 
 & \le C \parallel u \parallel ^2_1.
 \end{align*}

 This means $\gamma$ is continuous.
  \begin{remark*}
    Let $A= \dfrac{\partial}{\partial x_n}$. Then $H(A; \Omega) \cap
    \mathscr{D}(\bar{\Omega})$ is dense in $H(A; \Omega)$, and by the same
    method as above, $u \to \gamma u$ is continuous from\pageoriginale $H(A; \Omega)$ to
    $L^2(\Gamma)$. 
  \end{remark*} 
 
 In the text few propositions, we are going to determine the image and
 the kernel of $\gamma$. 
 
 We have seen that $u \in H^m (R^n)$ if and only if $u
 \in L^2$ and $(1+|\xi|^m)\hat{u} \in
 L^2$. Generalizing we define $H^ \alpha (R^n)$ for non-integer
 $\alpha >0$. 

 \begin{definition}\label{lec4:sec2:subsec4:def2.4}%defi 2.4
   {\em $u \in H^ \alpha (R^n)$} if and only if $u \in
   L^2$ and $(1+|\in|^ \alpha) \tilde{u} \in L^2$. On $H^
   \alpha (R^n)$, we put the topology defined by the norm 
   $$
   (u,u)_{H^\alpha (R^n)}= ((1+|\xi|^ \alpha)
   \hat{u})_{L^2(\mathbb{R}^n)}. 
   $$ 
 \end{definition}

 \begin{theorem}\label{lec4:sec2:subsec4:thm2.6}%theo 2.6
   Let $\Omega =\{ x_n >0 \}$. For $u \in H^1 (\Omega)$, we have 
 \end{theorem} 
 \begin{enumerate}[1)]
 \item $\gamma u \in H^{\frac{1}{2}}(\Gamma)$, and 
 \item $u \to \gamma u$ is continuous mapping of $H^1(\Omega)$
   onto $H^{\frac{1}{2}}(\Gamma)$. 
 \end{enumerate}
 \begin{proof}
   Let $\xi =(\xi_1, \ldots, \xi_{n-1})_{n-1}$
   and $\hat{u}(\xi', x_n)= \int \limits_{R^{n-1}} e^{-2 \pi i
     x'. \xi'}u(x',x_n)dx'$ be truncated Fourier transform of
   $u(x)$. Since 
  \end{proof}  
  $$
  \mathscr{F}\left(\frac{\partial u}{\partial x_i}\right)= \xi _i \hat{u}
  \xi L^2 (\xi' ,x_n), i=1, \ldots ,n-1, 
  $$
   we have 
  \begin{enumerate}[1)]
  \item $(1+ |\xi'|)u \in L^2 (\xi ',x_n)$.

    Further, since $\dfrac{\partial \hat{u}}{\partial
      x_n}=\dfrac{\partial \hat{u}}{\partial x_n}$ we have  
  \item $\dfrac{\partial \hat{u}}{\partial x_n} \in
    L^2(\xi', x_n)$. 
  \end{enumerate} 
  
  Now, as in theorem \ref{lec4:sec2:subsec4:thm2.5}, $|\hat{u}(\xi',0)|^2=- \int _o
  ^\infty (\hat{u}\bar{\hat{u}}\theta)dx_n$. Hence  $\int\limits
  _\Gamma (1+ |\xi|)|\hat{u}$ $(\xi', 0)|^2 d \xi '= -
  \int\limits _{R^+_n}(1+ |\xi'|) \dfrac{\partial}{\partial
    x_n}(\hat{u} \bar{\hat{u}}\theta) dx < \infty $ by Schwartz's
  inequality and (\ref{lec1:sec1:subsec2:eq1}) and (2).  
  
Hence $\gamma~ u~ \in ~H^{\frac{1}{2}}(\Gamma)$.
 
  We\pageoriginale now prove the second point. $f \in
  H^{\frac{1}{2}}(\Gamma)$ of and only if $(1 +
  |\xi|^{\frac{1}{2}}) \hat{f} \in L^2(\mathbb{Z}^n)$. We
  have to look for a function $u \in H^1(\Omega)$ such that
  $\gamma u =f$. Let  
$$
U(\xi',x_n)=\exp (-(+|\xi'|)x_n) \hat{f}(\xi')
\text{ for } x_n > 0, 
$$
and $u = \mathscr{F}_{\xi}(U(\xi',x_n))$. We prove
that $u \xi H^1$ and $\gamma u =f$. The only not completely
trivial point is to verify that $\dfrac{ \partial u}{\partial x_n}
\xi L^2$.  
$$
\frac{ \partial u}{\partial x_n} = -(1+|\xi|) \exp
(-(1+|\xi|x_n) \hat{f}(\xi'). 
$$

Hence 
\begin{align*}
\int | \frac{ \partial u}{\partial x_n}|^2 dx'_n & =
(1+|\xi|^2|\hat{f}(\xi')|^2 \int \limits^{\infty}_{o}
\exp (-(1+|\xi|)x_n)dx_n\\ 
& = (1+|\xi||\hat{f}(\xi')|^2.
\end{align*}

Since $f \in H^{\frac{1}{2}} (\Gamma)$, we have $ \int
\limits_{\Omega}|\dfrac{ \partial u}{\partial x_n}|^2 dx $ is finite. 

\begin{theorem}\label{lec4:sec2:subsec4:thm2.7}%theo 2.7
  $\gamma~ u  = 0$ if and only if $u~ \in~ H^1_o (\Omega)$.
\end{theorem}
\begin{proof}
    \begin{enumerate}[(a)]
    \item $ u \in H^1_o (\Omega) \Longrightarrow u =0$ for we
      have $u = \lim \varphi_k$ in $H^1 (\Omega) \varphi_k
      \in \mathscr{D}(\Omega). \gamma u = \lim \gamma
      (\varphi_k) $ in $L^2 (\Gamma)=0$.   
    \item Conversely to prove that $\gamma~ u = 0$ implies $u \in~
      H^1_O (\Omega)$ we require the  
\end{enumerate}
\end{proof}

\begin{lemma*}%lemm 0
  Let $\Omega = \{ x_n > 0 \}, u \in H^1_O (\Omega), \phi
  \in \mathscr{D}(\bar{\Omega})$. Then $\phi u \in H^1
  (\Omega)$, and $\gamma(\phi u) = \gamma(\phi) \gamma(u)$. 
\end{lemma*}

\begin{proof}
We know $\mathscr{H}^1 (\Omega)$ is dense in $H^1 (\Omega)$. Hence
there exists $u_k \in \mathscr{H}^1(\Omega)$, such that $u_k
\to u$ in $H^1 (\Omega)$. Now, $\phi u_k \to \phi u$ in $H^1 (\Omega)$
and since $\gamma(\phi u_k) = \gamma(\phi) \gamma(u_k)$, we have
$\gamma(\phi u) = \gamma(\phi) \gamma(u)$. 
\end{proof}

Coming\pageoriginale back to the proof of the theorem, let $a(x)$ be a $C^\infty$
function $1$ on the unit ball, $0$ outside another ball, and $0 \leq a
(x) \leq 1 $ else-where. Then if we define $a_j (x) = a (x / j )$, we
have $a_j u \to u $ in $ H^1 (\Omega)$. Hence if we prove that $ a_j u
\in H^1_0 (\Omega)$, we shall have proved that $u \in
H^1_o (\Omega)$. Since $a_j u $ has compact support, and since $\gamma
\langle a_j u \rangle = a_j \gamma u = 0$, this means, we may assume,
that in addition to $\gamma u = 0, u$ has compact support in $\bar
{\Omega}$. Let $\mathscr{O}_k (x_n)$ be a function defined for $x_n >
0$,  
$$
\mathscr{O}_k (x) = 
\begin{cases}
  0 &\text{for}~ 0 < x_n < 1/k \\
  \text {linear} & \text{for}~ 1/k \leq x_n \leq 2/k \\
  1 &\text{for}~ x_n > 2/k. 
\end{cases}
$$

Then $\widetilde{\mathcal{O}_k u} \in H^1 (R^n)$. By
regularization, we may assume that $\mathscr{O}_k u \in H^1_0
(\Omega)$. We now prove $\mathscr{O}_k u \to u $ in $H^1 (\Omega)$. We
have  
$$
\theta_k \frac{ \partial u }{\partial x_i} = \frac { \partial}
      {\partial x_i } (\theta_k (u)) \to \frac{\partial u }{\partial
        x_i} \text { for } 1 \leq i \leq n-1. 
$$

We have to prove then that $\theta'_k (x_n) u + \theta_k
\dfrac{\partial u } {\partial x_n} \to \dfrac{\partial u } {\partial
  x_n}$; that is to say, $\theta'_k (x_n) u \to 0$. Now  
$$
\theta'_k (x_n) =
\begin{cases}
0 &\text{for } x_n < 1/k \text { and }  x_n > 2/k \\
k &\text{for } 1/k < x_n < 2/k.
\end{cases}
$$

Further $ u (x', x_n ) = - \int^{x_n}_{0} \dfrac{\partial u }
{\partial t } (x', t)$ dt. Hence  
\begin{align*}
| u (x', x_n) |^2 & \leq x_n \int^{x_n}_{0} | \frac{\partial u}
{\partial t } (x', t) |^2 dt \\ 
& \leq 2/k \int^{x_n}_{0} | \frac{\partial u} {\partial t } (x', t)
|^2 dt  ~\text{if}~ x_n > 2/k \\ 
& \qquad 2/k.C \hspace{2cm} \text { if } x_n > 2/k \tag{1}\label{lec4:sec2:subsec4:eq1} 
\end{align*}
Also\pageoriginale
\begin{align*}
\int |\theta'_k (x_n) |^2 | u(x', x_n) |^2 dx_n & = \int\limits^{2/k}_{0}|
\theta'_k u|^2 dx \\ 
& \le 2k \int\limits^{2/k}_{0} dx_n \int\limits^{x_n}_{0}
\left|\frac{\partial u}{\partial t}(x', t)\right|^2 dx' \\ 
& =2k \int\limits^{2/k}_{0} \left|\frac{\partial u }{ \partial t}(x', t)\right|
^2 dt \int\limits^{2/k}_{t} dx_n \\ 
& \qquad \text{(by changing the order of integration)} \\
& \le 4  \int\limits^{2/k}_{0} |\frac{\partial u}{\partial x_n}|^2 dx_n.
\end{align*}

Hence $\int |\theta'_k u|^2 dx \rightarrow 0$ by (\ref{lec4:sec2:subsec4:eq1}), which
completes the proof. 

Thus we see that $H'_o (\Omega)$ is the space of functions which are
weakly zero on the boundary. 

The above results can be generalized to spaces $H^m(\Omega)$. Let
$\Omega = \{ x_n > 0\}; u \in H^m(\Omega)$, if and only if $D^p(u)
\in H^1 (\Omega)$, for $|p| \le m -1$. Hence $\gamma D^p u$
can be defined as above for $|p| \le m -1$. we have the  
\begin{theorem}\label{lec4:sec2:subsec4:thm2.8}%the 2.8
$u \in H^m_o(\Omega)$ if and only if $\gamma (D^p u) = 0$ for
  $|p| \le m-1$.  
\end{theorem}

In fact, we can say something more,
\begin{exer*}
Let $u \in H^m_o(\Omega)$ and $u \in D^p_x$, with
$|q|$ arbitrary.  
\end{exer*}

Then
$$
\gamma(D^q_{x'} , u) = D^q_{x'}, (\gamma u).
$$

\subsection{} \label{lec4:sec2:subsec5} %%% 2.5

Let $\Omega$ be an open set of $R^n$ such
that $(a)\Omega$ has 
$1$-extension property, and $(b)$ the boundary $\Gamma$ of $\Omega$ is
an ($n -1$) dimensional $C^1$ manifold. In the case $\Omega$ is
bounded $(b)$ implies $(a)$. On $\Gamma$ we have an intrinsic
measure. We denote by $L^2_{\loc}(\Gamma)$ the space of square summable
functions on every compact of $\Gamma$ with respect to this measure. 

\begin{theorem}\label{lec4:sec2:subsec5:thm2.9}%theo 2.9
Under\pageoriginale the above hypothesis on $\Omega$ (i.e.)
   \begin{enumerate}[\rm a)]
      \item $\Omega$ possesses $1$ extension property
      \item $\Gamma$ is an $(n -1)$ dimensional $C^1$ manifold,
   \end{enumerate}
there exists a unique continuous map $\gamma : H^1 (\Omega)
\rightarrow L^2_{loc}(\Gamma)$ which on functions of $\mathscr{H}^{1}
(\Omega)$ coincides with the restriction to $\Gamma$. 
\end{theorem}

\begin{proof}
Form (a) it follows that $\mathscr{H}'(\Omega)$ is dense in
$\mathscr{H}' (\Omega)$ and hence the\break uniqueness will follow from the
existences. Let $\Gamma_2$ be any compact of $\Gamma$. We observe that
by using a $C^1$ partition of unity the problem is reduced to a local
one, that is to say, we may assume that the support of $\varphi
\in \mathscr{H}^1(\Omega)$ is contained in an open set $0$ and
that there exists a homeomorphism $\psi$ as in theorem \ref{lec3:sec2:subsec3:thm2.4}. Further
we may assume, without loss of generality, that $\Gamma_2 \subset
0$. Let $a$ be a $C^\infty$ function in $R^n$ with compact support in
$0$ and which is $1$ on $\gamma_2$. Then $au \psi^{-1} \in
H(\mathbb{Z}^n)$ (in the notation of theorem \ref{lec3:sec2:subsec3:thm2.4}), and $\gamma au
\psi^{-1} \in L^2(W_o)$. Since $\psi$ defines an isomorphism
of $L^2 (\Gamma_2)$ into $L^2(W_o), \psi (\gamma a u \psi^{-1})
\in L^2(\Gamma_2)$. Now on $\Gamma_2$ we have $\gamma u =
\psi(\gamma au \psi^{-1})$ which proves that $\gamma$ is continuous
mapping of $\mathscr{H}^1(\Omega)$ into $L^2(\Gamma_2)$ which
completes the proof. 
\end{proof}

\begin{remark*}
  A complete generalization of theorem \ref{lec4:sec2:subsec4:thm2.7} is due to N. Aronszajn [\ref{k2:e1}].
\end{remark*}
