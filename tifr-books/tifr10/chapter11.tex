
\chapter{Lecture}\label{lec11}%cha 11

\setcounter{section}{6}

\subsection{}\label{lec11:sec6:subsec5} 

We\pageoriginale now consider another kind of
sesquilinear forms giving rise to the same operator $A =
\sum\limits_{i, j=1}^n \dfrac{\partial}{\partial x_j}(g_{ij}(x)
\dfrac{\partial}{\partial x_i}) + g_i (x)\dfrac{\partial}{\partial
  x_j}+ g_i \dfrac{\partial}{\partial x_i} + g_o (x)$. 

Let $\Omega$ be an open set with the boundary $\Gamma$ having a $C^1
(n-1)$ dimensional  piece $\Sigma$. Let $\gamma \, u$ be the extension of
functions in $H^1 (\Omega)$ to $\sum$ as defined in \S\ \ref{lec4:sec2:subsec4}. 

On $H^1(\Omega)$ consider the sesquilinear form 
$$
a(u, v) = (u, v)_g+ \sum \left(g_i \dfrac{\partial u}{\partial x_i},
v\right)_o + \sum (g_o u, v)_o + \int\limits_\Sigma ~\gamma ~u
\overline{\gamma ~u}d \sigma, 
$$
 where $d \sigma$ is the intrinsic measure on $\Sigma$. The operator
 associated with it is the same $A$ as before. To consider the
 ellipticity of this from we require some definitions. 

\begin{definition}\label{lec11:sec6:subsec5:def6.1}  %definition 6.1
  Let $\Omega$ be a bounded {\em connected} open set; we shall say that
  $\Omega$ is of Nykodym type if there exists a constants $P(\Omega)> 0$
  such that the following inequality holds for all $u \in H^1
  (\Omega)$. 
\end{definition}

\begin{equation*}
  \int\limits_{\Omega} |u|^2 dx - \frac{1}{mes \Omega}\Big |\int u dx
  P(\Omega)|u|^2_1 . \tag{1}\label{lec11:sec6:subsec5:eq1}  
\end{equation*}

The inequality (\ref{lec11:sec6:subsec5:eq1}) is called Poincare inequality. We admit without
proof the  
\begin{theorem}\label{lec11:sec6:subsec5:thm6.7} %them 6.7
  Every $\Omega$ with ``smooth boundary'' is of Nykodym type. (For
  proof, see Deny \cite{k7}). 
\end{theorem}

Another interpretation of the inequality (\ref{lec11:sec6:subsec5:eq1} ) is obtained by observing that 
$$
\int |u|^2 dx - \frac{1}{\mes \Omega}\bigg|\int u dx \bigg |^2
$$
is the minimum of $\big | + c_o \big |_0$ for all constants $c$.

For\pageoriginale
\begin{align*}
  \big| u +c \big|^2_0  &= |u|^2_o + \bar{c}\int {}_u dx + c
  \int \bar{u} dx + |c|^2 \mes \Omega \\ 
  & = \frac{1}{\mes \Omega} \left(c + \int {}_u dx \right) \left(\bar{c} +
  \int \bar{u} dx \right)+
  |u|^2_o - \frac{1}{\mes \Omega}\bigg| \int udx \bigg|^2 
\end{align*} 

Thus (\ref{lec11:sec6:subsec5:eq1}) means Inf $| u + c|^2_o \le P (\Omega) |u|^2_1$.

\begin{theorem}\label{lec11:sec6:subsec5:thm6.8}  %Theorem 6.8
  Let $\Omega$ be a domain of Nykodym type with the boundary $\Gamma
  a(n-1)$ dimensional $C^1$ manifold. Then the form  
  $$
  a(u, v) = (u, v)_g + ~ \int\limits_{\Gamma}~ \gamma~ u \overline{\gamma ~
  v}~ d~ \sigma 
  $$
  is $V$-elliptic on $H^1(\Omega)$.
\end{theorem}

\begin{proof}
Since $\re  (a(u, u)) = \re  (a(u, u))_g + \int |\gamma u|^2 d \sigma
\geq \alpha |u|^2_1 + \int\limits_\Gamma |\gamma u|^2 d \sigma$ to prove the
$V$-ellipticity of $a(u, v)$ it is enough to prove that there exists a
$\beta > 0$ such that  
$$
\alpha |u|^2_1 + \int |\gamma u|^2 d \sigma \leq \beta || u || ^2_1, 
$$
or that 
$$
\int |\gamma u |^2 d \sigma + |u |^2_1 \geq \beta_1 || u ||^2_1.
$$

Let $[u, v]= (u, v)_1 +  \int \gamma u \overline{\gamma v} d
\sigma$. $[u, v]$ is a 
continuous sesquilinear form on $H^1(\Omega)$ since $[u, u] = 0$
implies $|u|^2_1 =0$ and $\int |\gamma~ u|^2 d \sigma = 0$, we have $u
= c$, a constant for $|u|^2_1 = 0$ and $c = 0$ for $\int |\gamma u|^2
= 0$. That is to say $[u, u] = 0$ implies $u = 0$. In fact, we have
the  
\end{proof}

\begin{lemma*}
  $[u, v]$ defines a Hilbertian structure on $H^1(\Omega)$. 
\end{lemma*}

Assuming the lemma for a moment, we see that on account of the closed
graph theorem, the two norms $\sqrt{[u, v]}$ and $\sqrt{(u, v)_V}$ are
equivalent. Hence $[u, u]\geq \beta || u || ^2_1$ which was to be
proved.  

To complete the proof we have to prove the lemma, i.e., that under the
scalar product $[~ ], H^1 (\Omega)$ is complete.  

Let\pageoriginale $u_k$ be a Cauchy sequence for the scalar product [~]. Then
$\dfrac{\partial u_k}{\partial x_i}, i = 1,\ldots, n$, and $\gamma
u_k$ are Cauchy sequences in $L^2(\Omega)$, and $L^2(\Gamma)$
respectively. Hence $\dfrac{\partial u}{\partial x_i} \rightarrow f_i,
i = 1,\ldots, n$ in $L^2(\Omega)$ and $\gamma u_k \rightarrow g$ in
$L^2(\Gamma)$. Since $\Omega$ is of Nykodym type from (\ref{lec11:sec6:subsec5:eq1}), we
have  
$$
\displaylines{\hfill
\int\limits_\Omega | u_k - \frac{1}{\mes\Omega} | \int ~ u ~
dx |^2 |^2 ~ dx \le P | u |^2_1 \hfill\cr
\text{i.e.,} \hfill
\int\limits_{\Omega} | ~ u_k - c_k |^2 \le P | u_k |^2_1 ~\text{where
}~ c_k = \frac{1}{\mes ~  \Omega} \int ~ u_k ~ dx.\hfill}  
$$

Since $u_k$ is a Cauchy sequence in $L^2 (\Omega),u_k - c_k$ is a
Cauchy sequence in $L^2(\Omega)$. Hence $u_k - c_k \rightarrow v$ in
$L^2(\Omega)$ and $\dfrac{\partial v}{\partial x_i} = \lim
\dfrac{\partial u_k}{\partial x_i} = f_i$. Hence $u_k - c_k
\rightarrow v$ in $H^1(\Omega)$ and so $\gamma(u_k - c_k) \rightarrow
v$ in $L^2 (\Gamma)$. Since $\gamma u_k \rightarrow g$ in
$L^2(\Gamma), c_k \rightarrow c$. However $u_k = (u_k - c_k) +
c_k$. Hence $u_k \rightarrow v + c$ in $H^1(\Omega)$ under the norm [
], which proves the lemma. 

\subsection{Formal interpretation:}\label{lec11:sec6:subsec6} 

If $\Omega$ is of Nykodym type with a smooth $(n-1)$ dimensional
boundary $\Gamma$, we have just proved that the form $a(u,v) =
(u,v)_g + (\gamma u, \gamma v)_0$ is elliptic on $H^1(\Omega)$. The
operator $A$ that it defines is $A = - \sum \dfrac{\partial}{\partial
  ~ x_i} \left(g_{ij}(x)\dfrac{\partial}{\partial ~ x_j}\right)$ and $u
\in N$ implies $a(u,v) = (Au,v)_o$ for all $v ~ \in ~
V$. Now formally, 
$$
\int\limits_\Omega ~ A ~ u ~\bar{v} ~ d ~ x = a(u,v) + \int
\frac{\partial u}{\partial \eta_A}\bar{v}~ d ~\sigma 
$$
where $\dfrac{\partial u}{\partial \eta_A} = \sum ~ g_{ij} ~
\dfrac{\partial u}{\partial x_j} \cos (n,x_i),(n,x_i)$ being the angle
between the outer normal and $x_i$. Thus $u ~\in~ N$ implies
formally $\dfrac{\partial u}{\partial \eta_A} = 0$. 

\subsection{Complementary results}\label{lec11:sec6:subsec7}

\textit{Boundary\pageoriginale value problems of oblique type for} $\Omega = \{ x_n
> 0\}$. For general theory, see Lions [\ref{k10:e4}]. Let $\Gamma$ be the
boundary of $\Omega : \{ x_n = 0 \}$. We recall the definition of the
spaces $H^\alpha(\Gamma)$ for $\alpha$ real defined in \S\ \ref{lec4:sec2:subsec5}.
 $H^\alpha (\Omega) = \{ f ~ L^2(\Gamma) \}$ such that $(1 + |\xi
|^\alpha) \hat{f} ~ \in ~ L^2 (\Gamma)$, where $\hat{f}$
is the Fourier transform of $f$. We have proved in \ref{lec4:sec2:subsec4}, that there
exists a unique mapping $\gamma : H^1 ~ (\Omega) ~ \rightarrow ~
H^{\frac{1}{2}}(\Gamma)$ which on $\mathscr{D}\bar{(\Omega)}$ is
the restriction to $\Gamma$ and this mapping is \textit{onto}. 

\begin{theorem}\label{lec11:sec6:subsec7:thm6.9}% theorem6.9
  The dual of $H^\alpha (\Gamma)$ is $H^{-\alpha}(\Gamma)$.
\end{theorem}

\begin{proof}% proof
Let $\mathscr{F}(H^\alpha(\Gamma))$ be the space of Fourier
transforms of
$H^\alpha(\Gamma)$. $\mathscr{F}(H^\alpha(\Gamma))$ consists
of functions $\hat{f} ~ \in ~ L^2 (\Gamma)$ such that $(1 +
| \xi |^\alpha)\hat{f} \in L^2(\Gamma)$. Hence its dual
consists of functions $\hat{g} ~ \in ~ L^2 (\Gamma)$ such
that $\dfrac{1}{1 + | \xi |^{\alpha}} ~\hat{g}~ \in~ L^2
(\Gamma)$, i.e., $(1 + | \xi |^{-\alpha}) ~ \hat{g} ~ \in
~ L^2 (\Omega)$. Hence the dual of
$\mathscr{F}(H^{-\alpha}(\Gamma))$ is
$\mathscr{F}(H^{-\alpha}(\Gamma))$ which proves the theorem. 
\end{proof}

Let $\Lambda = \sum\limits_{i=1}^{n-1} \alpha_i ~
\dfrac{\partial}{\partial ~ x_i}$ with $\alpha_i$ \textit{real
  constants}. We call $\Lambda$ a \textit{tangential operator}. 

\begin{lemma}\label{lec11:sec6:subsec7:lem6.2}% lemma 6.2
  $\Lambda$ is a continuous linear mapping of
  $H^{\frac{1}{2}}(\Gamma)$ into $H^{\frac{-1}{2}}(\Gamma)$. 
\end{lemma}
\begin{proof}% proof
It is enough to prove that $\dfrac{\partial}{\partial x_i}$ is a
continuous linear mapping from $H^{\frac{1}{2}}(\Gamma)$ into
$H^{-\frac{1}{2}}(\Gamma)$ or that $\mathscr{F}
\left(\dfrac{\partial}{\partial x_i}\right)$ is continuous from
$\mathscr{F}(H^{\frac{1}{2}}(\Gamma))$ into\break
$\mathscr{F}\left(H^{-\frac{1}{2}}(\Gamma)\right)$. Let $f \in
H^{\frac{1}{2}}(\Gamma)$. Then $(1 +  | \xi |^{\frac{1}{2}})
\hat{f} \in L^2 (\Gamma)$, and so
$\mathscr{F}\left(\dfrac{\partial f}{\partial x_i}\right) = 2 \pi ~i ~\xi_i ~
\hat{f} ~ \in ~ H^{-\frac{1}{2}}(\Gamma)$. Since the
mapping $g \rightarrow \xi_i g$ is continuous, from
$\mathscr{F}(H^{\frac{1}{2}}(\Gamma ))$ into
$\mathscr{F}(H^{-\frac{1}{2}}(\Gamma))$ the proof is complete. 
\end{proof}

From lemma \ref{lec11:sec6:subsec7:lem6.2} we see that $\langle \Lambda \gamma u,
\overline{\gamma v} \rangle$ is defined for all $u,v \in 
H^1(\Omega)$. Further we have the 
\begin{lemma}\label{lec11:sec6:subsec7:lem6.3}%lemma 6.3
  $\re  (\Lambda \gamma ~ u, \gamma ~u) = 0$\pageoriginale for all $u ~ \in ~
  H^1(\Omega)$. For by Fourier transform $\re  ~ \langle
  \dfrac{\partial}{\partial x_i} ~ \gamma u, \overline{\gamma u}
  \rangle = \re  ~ \int ~ 2 ~ \pi ~ i ~ \xi_i ~ | ~ \gamma ~ \hat{u} ~
  |^2 ~ d ~ \xi$. 
\end{lemma}

Let $a(u,v) = (u,v)_1 + \lambda (u,v)_0 + \langle \Lambda ~ \gamma ~
u, \overline{\gamma v} \rangle$ for $u,v ~ \in ~
H^1(\Omega)$. From lemma \ref{lec11:sec6:subsec7:lem6.2}, we see that $a(u,v)$ is a continuous
sesquilinear form on $H^1(\Omega)$. 

\begin{lemma}\label{lec11:sec6:subsec7:lem6.4} % lemma 6.4
  If $\lambda > 0, a(u,v)$ is $H^1(\Omega)$ elliptic. For $\re  (a(u,u))
  = | u |^2_1 + \lambda | u |^2_0 \ge \min (\lambda,1) || u ||_1$. 
\end{lemma}

From theorem \ref{lec5:sec3:subsec2:thm3.1}, we have the 
\begin{theorem}\label{lec11:sec6:subsec7:thm6.10}% theorem 6.10
  The operator associated with $a(u,v)$ is $-\triangle + \lambda$ and $
  - \triangle + \lambda $ is an isomorphism from $N$ onto
  $L^2(\Omega)$. 
\end{theorem}

To get a formal interpretation of the problem, we have to see that $ u
~ \in ~ N$ means. $u ~\in ~ N$ if and only if 
$$
(( - \triangle + \lambda) u,v)_0 = (u,v)_1 + (\Lambda ~\gamma ~
u,\overline{\gamma u})_o + \lambda(u,v)_o. 
$$

By Green's formula, $(-\triangle ~ u, v)_0 = (u,v)_1 +
\int_{\Gamma}\dfrac{\partial u}{\partial x_n} \bar{v}$. Hence $u ~
\in ~ N$ implies \textit{formally} $\dfrac{\partial ~
  u}{\partial x_n}(x_1, \ldots ,x_n, 0) = \Lambda \gamma ~ u$, a
condition of oblique derivative. 
