
\chapter{Lecture}\label{lec14}%% 14

\setcounter{section}{8}

\subsection{}\label{lec14:sec8:subsec3}
 
We\pageoriginale now proceed to prove the theorem \ref{lec13:sec8:subsec2:thm8.1}. First we prove a lemma of
fundamental character which will help to establish an inductive
procedure to prove the theorem. 

\begin{lemma}\label{lec14:sec8:subsec3:lem8.1}% lemma 8.1
  Let $A$ be a uniformly elliptic differential operator of order
  $2m$. Let $u ~ \in ~ \mathscr{L}^m(\Omega)$ and let $Au
  \in \mathscr{L}^{-m+1}$ (usually $Au \in \mathscr{L}^{-m}$
  only). Then $u ~ \in ~\mathscr{L}^{m+1}$. 
\end{lemma} 
 
 \begin{proof}
   We prove the lemma in two steps. In the first one it will be shown
   that it is enough to prove the lemma assuming $A$ and $Au$ to have
   compact support, for which the assertion will be proved in the second
   step. 
 \end{proof} 

\begin{step}\label{lec14:sec8:subsec3:step1}% step 1
  The lemma is equivalent to ``if $u ~ \in ~ K^m$ and $Au
  \in K^{-m+1}$, then $u \in K^{m+1}$''. 
 \end{step} 
 
The direct part is evident. To prove the converse, let $u ~
\in ~ \mathscr{L}^m$ be such that $Au~  \in ~
\mathscr{L}^{-m+1}$. For any $\varphi ~ \in ~
\mathscr{D}(\Omega), v = \varphi ~ u ~ \in ~
\mathscr{L}^m$. Now $Au = A(\varphi~ u) = A ~ u + \sum\limits_{|p|~\le
  ~ 2m, ~ |q| ~ \le ~ 2m-1} ~ D^p ~ D^q ~ u$. 

\medskip
\noindent
Since for $|q| \le 2m-1, D^q u \in \mathscr{L}^{-m+1}$ and by
assumption, $A u \in \mathscr{L}^{-m+1}$, it follows that $Av
~\in~ \mathscr{L}^{-m+1}$. Since $\varphi$ has compact
support, $v$ and $Av$ are in $K^{-m+1}$. Hence $v \in
K^{m+1}$. Since this is true for every $\varphi ~ \in ~
\mathscr{D}(\Omega), v ~ \in ~ \mathscr{L}^{-m+1}$. 

Now we prove the
\begin{step}\label{lec14:sec8:subsec3:step2}% step 2
If $u \in K^m$ and $Au \in K^{-m+1}$, then $u
\in K^{m+1}$. 
\end{step}

We have to prove that $\dfrac{\partial u}{\partial x_i} \in
H^m(\Omega)$. A general method to prove this, here and in later
occasions, will be to estimate the difference\pageoriginale quotients of $u$. Let $h
= (h,0,\ldots,0)$ and $u^h(x) = \dfrac{1}{h}~ u(x+h) - u(x)$ which
exists if $h$ is small enough. Now we establish the following:  
\begin{enumerate}
\renewcommand{\theenumi}{\alph{enumi}}
\renewcommand{\labelenumi}{\theenumi)}
\item $A(u^h) - (Au)^h = \sum (-1)~  ^{p}D^p ~ (a^h_{pq}~ D^q u(x+h))$.
\item $A(u^h)$ is bounded in $K^{-m}$.
\item $u^h$ is bounded in $H^m$.
\end{enumerate}

Assuming for a moment that $a)$, $b)$, $c)$ are proved, we complete the
proof of the lemma. Since $u^h$ is bounded in $H^m$, it is a weakly
compact and hence there exists $h_i~ \rightarrow ~ 0$ such that
$u^{h_i} ~ \rightarrow g$ weakly in $H^m(\Omega)$. On the other hand,
$u^{h_i} ~ \rightarrow ~ \dfrac{\partial u}{\partial ~x_i}$ in
$\mathscr{D}'$. Hence $\dfrac{\partial~ u}{\partial~ x_i} = g ~
\in~  H^m$, i.e., $u ~ \in ~ H^{m+1}$. Since $u$ has
compact support, $u ~ \in ~ K^{m+1}$. 

Now the prove $a)$, $b)$, $c)$.
\begin{enumerate}[a)]
\item We verify easily that $(\alpha  f)^h - \alpha f^h  =
  \alpha^h ~ f(x+ h)$. Applying this term by term in $(Au)^h -
  A(u^h)$ we obtain $(a)$.  
\item On account of $(a)$, to prove that  $A(u^h)$ is bounded in
  $K^{-m}$ it is enough to prove that $(Au)^h$ and each of
  $D^p(a^h_{pq} D^q u(x+h))$ are bounded in $H^{-m}$. Since $Au = g
  \in K^{-m+1}$, $\dfrac{\partial g}{\partial x_1} ~\in  ~K^{-m}$
  and since $(Au)^h \rightarrow \dfrac{\partial g}{\partial x_1},
  (Au)^h$ is a convergent sequence in $K^{-m}$ and so is bounded,
  Further, since $a^h_{pq} \in C^\infty$ as $h \rightarrow 0,
  a^h_{pq} \rightarrow \dfrac{\partial}{\partial x_1} a_{pq}(x)
  \in C^\infty$ uniformly on every compact set. Also $D^q
  u(x+h) \rightarrow D^q u(x)$ in $L^2$. Hence $a^h_{pq}D^q u(x+h)$
  converge in $L^2$. Since $D^p$ are derivatives of order than or
  equal to $m$, $D^p(a^h_{pq} D^q u(x+h))$ converge in $H^{-m}$, and
  hence in $K^{-m}$. This proves $(b)$.  
\item Since by $(b), A(u^h)$ is bounded, we have
\begin{align*}
\langle A ~ u^h, u^{-h} \rangle & \le || A(u^h) ||_{H^{-m}} || u^h ||_m\\
& \le c_1 || u^h ||_m
\end{align*}
\end{enumerate}

On\pageoriginale account of Garding's inequality, we have 
$$
\re  ~ a(u^h,u^h) + \lambda | u^h |^2_0 \ge \alpha || u^h ||^2_m
$$
on every compact set. As $h \rightarrow 0$, we may assume that all
$u^h$ have their support in a fixed compact set. Hence 
$$
\alpha || u^h ||^2_m \le \lambda | u^h |^2_0 + c_1 || u^h ||_m.
$$

\noindent
Since $u \in K^m, u^h \rightarrow \dfrac{\partial u}{\partial
  x_i}$ in $L^2$ and so $| u^h |_o$ is bounded. Further we have 
$$
c_1 || u^h ||_m \le \frac{c_1^2}{2} + \frac{\alpha}{2} || u^h ||^2_m.
$$

Hence $\dfrac{\alpha}{2}|| u^h ||^2_m\le c_3$ which proves $u^h _m$
is bounded, and this completes the proof of lemma \ref{lec14:sec8:subsec3:lem8.1}.  

\begin{lemma}\label{lec14:sec8:subsec3:lem8.2}% lemma 8.2
  Let $u \in \mathscr{L}^m$, and $Au \in
  \mathscr{L}^{-m+1+j}$.  Then $u \in \mathscr{L}^{m+j+1}$ for
  every non-negative integer $j$. 
\end{lemma}

\begin{proof}% proof
  Lemma \ref{lec14:sec8:subsec3:lem8.1} proves the lemma for the case $j =0$; assuming it proved
  for integers upto $j = 1$, we prove it for $j$. Since
  $\mathscr{L}^{-m+j+1}  \subset$ $\mathscr{L}^{-m+j}, Au ~ \in
  ~ \mathscr{L}^{-m+j+1}$ implies that $Au ~ \in
  \mathscr{L}^{-m+j}$ and hence by induction hypothesis that $u ~
  \in ~ \mathscr{L}^{m+j}$. Now $D ~ Au - A ~ Du = A'u$ where
  $A'$ is a differential operator of order $2m$. Since $Au ~ \in
  ~ \mathscr{L}^{-m+j+1}, D ~ Au \in ~ \mathscr{L}^{-m+j}$ and
  since $u \in \mathscr{L}^{m+j}, Au ~\in ~
  \mathscr{L}^{m+j}$. Hence $A(Du) ~ \in ~
  \mathscr{L}^{-m+j}$. But $Du ~ \in \mathscr{L}^m$ as $u
  \in ~ \mathscr{L}^{m+j}, j ~ \ge ~ 1$. Hence by lemma \ref{lec14:sec8:subsec3:lem8.1},
  $Du ~ \in ~ \mathscr{L}^{m+j}$, i.e., $u \in ~
  \mathscr{L}^{m+j+1}$. 
\end{proof}

Lemma \ref{lec14:sec8:subsec3:lem8.2} can be put in a slightly better form of 

\begin{lemma*}[{\boldmath{$8.2'$}}]
  Let $u \in \mathscr{L}^{m}$ and $Au \in \mathscr{L}^r$, then $u \in \mathscr{L}^{r+2m}$. 
\end{lemma*}

For, if $r \le -m$, the lemma is trivial and if $r > -m$ we have $r =
- m+j$ and lemma $8.2'$ follows at once from lemma \ref{lec14:sec8:subsec3:lem8.2}. 

Now\pageoriginale we complete the proof of theorem \ref{lec14:sec8:subsec3:lem8.1}. We have to prove that if
$T \in \mathscr{D}'$  and $A ~ T ~ \in
~\mathscr{L}^r$, then $T \in \mathscr{L}^{r+2m}$. Let $O, O_1$ be
two relatively compact open sets such that $O \subset O_1 \subset \Omega$. On
account of a theorem of Schwartz, $T_0 = \sum D^p f_p$ where $f_p$ are
continuous in with support contained in $O_1$. By theorem \ref{lec3:sec2:subsec1:thm2.1}, $T_0
~ \in ~ H^{-\beta}(O)$. Now $\triangle^{m+\beta}$ where
$\triangle$ is the Laplacian is on account of theorem \ref{lec2:sec1:subsec6:thm1.3} is an
isomorphism of $H^{m+\beta}(O)$ onto $H^{-m+\beta}_o$. Hence there
exists $u ~ \in ~ H^{m+\beta}(0)$ such that
$\triangle^{m+\beta} ~ u = T_0$. Applying lemma $8.2'$ to
$\triangle^{m+\beta}$, we have $u ~ \in
\mathscr{L}^{2m+\beta}(0)$ as the order of $\triangle$ is
$2(m+\beta),T_0 ~ \in ~ \mathscr{L}^{-\beta}$, and $u ~
\in ~ \mathscr{L}^{m+\beta}(0)$. Now $(A ~ T_0) = (A ~
\triangle^{m+\beta} ~ u) \in ~ \mathscr{L}^r(0)$. The order of
$B = A ~ \triangle^{m+\beta}$ is $4m+2\beta$ and $B$ is uniformly
elliptic. As $u ~ \in ~ Z^{2m+\beta}$ and $Bu ~ \in
\mathscr{L}^r$, applying lemma \ref{lec14:sec8:subsec3:lem8.2}, we have $u ~ \in ~
\mathscr{L}^{r+4m+2\beta}$. Hence $T = \triangle^{m+\beta} ~ u ~
\in ~ \mathscr{L}^{r+2m}$. 

\subsection{Some remarks}\label{lec14:sec8:subsec4} %%% 8.4

We remark that theorem \ref{lec13:sec8:subsec2:thm8.2} implies theorem \ref{lec13:sec8:subsec2:thm8.1} trivially though in
the course of the proof, we proved theorem \ref{lec13:sec8:subsec2:thm8.1}, before proving
theorem \ref{lec13:sec8:subsec2:thm8.2}. This raises a vague question what properties which are
true for uniformly elliptic differential equations can be upheld for
the elliptic ones. For instance, we know for Dirichlet's problem for
bounded domains with smooth boundary Fredholm's alternative holds if
the operator is uniformly elliptic. In the case $n = 2$, we have the
following counter example of Bicadze \cite{k4}. 

Consider the Dirichlet problem in the unit circle for the operator $A
= \dfrac{1}{4} \left(\dfrac{\partial}{\partial ~ x} +
i\dfrac{\partial}{\partial ~ y}\right)^2$. A is elliptic but is not
uniformly elliptic, for the associated form has $\xi^2 - \eta^2$, as
its real part. We prove that the space of $u$ such that $Au = 0, u =
0$ on the boundary is\pageoriginale not finite dimensional and hence that Fredholm
alternative does not hold. $Au = 0$ means $\dfrac{\partial^2 ~ u}
{\partial \bar{z}^2} =0$, where $\dfrac{\partial}{\partial \bar{z}} =
\dfrac{\partial}{\partial ~ x} + i \dfrac{\partial}{\partial ~ y}$ and
hence $\dfrac{\partial ~ u}{ \partial \bar{z}}$ is holomorphic in the
unit circle. Hence $ u = f + \bar{z} g$ where $f$ and $g$ are
holomorphic in the unit circle. But $u = 0$ on the boundary $z \bar{z}
= 1$ . Hence $0= z ~ u = z ~ f + g$ on the boundary, and hence $g = -
zf$ everywhere as $f$ and $g$ are holomorphic in the unit circle. Thus
$u = (1 - z \bar{z})f(z)$ is a solution of the above problem for any
holomorphic $f(z)$ which shows that the space of $u$ such that $Au =
0, u = 0$, on the boundary, is not finite dimensional. 

For complementary results, see Schechter \cite{k15} and a forthcoming
paper by Agmon, Douglis, Nirenberg. 
