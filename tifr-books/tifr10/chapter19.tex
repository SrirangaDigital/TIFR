
\chapter{Lecture}\label{lec19}%%%19
\setcounter{section}{10}

The\pageoriginale problem of formulated in the last lecture would loose its interest
if $\mathscr{A}\cup $ were not independent of the extension
$\mathscr{A}$ of $A$ that we have chosen. We prove that in fact this
is the case for some kinds of domains.  

Let $\mathscr{A}$ and $\mathscr{A'}$ be two extensions of $A$. Let
$\in H^{-k}_{\bar{\Omega}}$. We have then $\langle \mathscr{A} \cup
- \mathscr{A} \cup , \bar{v}\rangle = \langle \cup. , (\overline{\mathscr{A}^* -
  \mathscr{A}^*})v >$ for $v \in H^{k+2m} (R^n)$. Since
$\mathscr{A}^* = \mathscr{A}^*$ on $\Omega , w = (\mathscr{A}* -
\mathscr{A}^*) v$ is such that $w_\Omega = 0$. Now in order that
$\mathscr{A} \cup = \mathscr{A}' \cup$ it is sufficient to assume some
sort of density of $(\mathscr{A}^* - \mathscr{A}^{*'})v$, for $v
\in H^{k+2m} (R^n)$ in $H^k (R^n)$, i. e. , of $w \in
H^k (R^n)$ such that $w_{\Omega} = 0$. This can be done by having the
following definition.  

\begin{definition}\label{lec19:sec10:subsec2:def10.3}%defini 10. 3
  $\Omega$ is $k$-sufficiently regular if $w \in H^k (R^n)$ is
  such that $w_\Omega = 0$. Then there exists $g \in H^k_o
  (\sigma \bar{\Omega})$ such that $= \tilde{g}$.  
\end{definition}

Assuming then $\Omega$ to be sufficiently regular, we have
$\mathscr{A}^* - \mathscr{A}^{*'} v = w = \lim \varphi_i$ in $H^k_o
([ \bar{\Omega})$ with $\varphi_j \in \mathscr{D} ([
\bar{\Omega})$. But since $\cup = 0$ on $[ \bar{\Omega}, \langle \cup ,
  \bar{\varphi}_j \rangle = 0$. Hence $\langle  ( \mathscr{A} -
  \mathscr{A}^*) \cup,
  \bar{v}\rangle = 0$ for all $v \in H^{k+2m} (R^n)$, and so
  $\mathscr{A} u = \mathscr{A}' \cup$.  

\begin{definition}\label{lec19:sec10:subsec2:def10.4}%definition 10. 4
  If $\Omega$ is $k$-sufficiently regular, the problem will be called
  Visik-Soboleff problems.  
\end{definition}

\subsection{}\label{lec19:sec10:subsec3}

We prove now the uniqueness and existence theorem for the
Visik-Soboleff problems.  

\begin{theorem}\label{lec19:sec10:subsec3:thm10.2}%theorem 10. 2
~
  \begin{enumerate}[(1)]
    \item Let $\Omega$ be a domain in $R^n$ such that 
      \begin{enumerate}[(a)]
      \item $\Omega$ has $k$ and $k+2m$ extension property;
      \item $\Omega$ is $k$ and $k+ 2m$ sufficiently regular. 
      \end{enumerate}
      
    \item Let\pageoriginale $V$ be such that $H^m_o (\Omega) \subset V \subset
      H^m(\Omega)$ and $a (u, v)$ be a sesquilinear form on $V$ satisfying
      $a(u, v) \geq \alpha || u ||^2_m$ and such that there exists
      $\mathscr{A}_{pq} \in \mathscr{D}_{L \infty} (\Omega)$ such
      that $\mathscr{A}_{pq} = O_{pq}$ on $\Omega$.  
    \item  Let the operator $A^*$ defined by $a^* (u, v) = \overline{a(v,
      u)}$ be such that $A^* u \in H^k (\Omega)$ imply $u
      \in H^{k + 2m}$. Then Visik-Soboleff problem admits a unique
      solution i. e. , given $T \in H^{-k + 2m} (\bar{\Omega}$,
      there exists a unique $u \in H^{-k } (\bar{\Omega)}$ such that
      $A u - T \in M^k$.
  \end{enumerate}
\end{theorem}

\begin{proof} % pro
  To prove this theorem we shall require some lemmas. Let $(H^k
  (\Omega))'$ be the dual of $H^k (\Omega)$. We do not identify
  $(H^k(\Omega))'$ with any space of distributions for $\mathscr{D}
  (\Omega)$ is not dense in general in $H^k (\Omega)$. We know the
  restriction map $v \to v_{\Omega}$ of $H^k (R^n)$ into $H^k (\Omega)$
  is continuous. The transpose of this mapping is a mapping of $H^k
  (\Omega))'$ into $(H^k (R^n)) = (H_o^k (R^n)) = (H^k_o(R^n)) = H^{-k}
    (R^n)$, given explicitly by $\langle \pi_k, T, \bar{v} \rangle = (T
  \bar{v}_{\Omega})$ for $v \in H^k (R^n)$. Further if
  $v_{\Omega} = 0, \langle \pi_k T, \bar{v} \rangle = 0$, i. e. , $\pi_k
  T = 0$ on $[ \bar{\Omega}$ so that the support of $\pi_k$ is
  contained in $\bar{\Omega}$. Hence $\pi_k$ is a continuous mapping of
  $(H^k (\Omega))'$ into $H^{-k}_{\bar{\Omega}}$.  
\end{proof}

Using (1) - (a), and (b) of the theorem we prove the fundamental
\begin{lemma*} %lem 
  The mapping $T \to \pi_k T$ of $(H^k (\Omega))'$ into
  $H^{-k}_{\bar{\Omega}}$ is an isomorphism.  
\end{lemma*}

\begin{proof}% pro
  We build explicitly the inverse. On account of the $m$-extension
  property of $\Omega$, there exists a continuous mapping $u \to P(u)$
  of $H^k (\Omega)$ into $H^k (R^n), $ with $P \mu = u$ a. e.  on
  $\Omega$. Let $S \in H^{-k}_{\bar{\Omega}}$. Then the
  semi-linear form $u \to \langle S, \bar{Pu} \rangle$ is continuous on
  $H^k (\Omega)$ and hence defines an element $\bar{\omega} S
  \in (H^k (\Omega))'$ so that $\langle S, \overline{P (u)}
  \rangle = (\bar{\omega}_k S) ~ (\bar{u})$  
  
  The\pageoriginale mapping $S \to \bar{\omega}_k S$ is obviously continuous. The
  lemma will be proved if we prove 
  
  $ a) \quad \bar{\omega}_k - _{-k} T = T$, and $b) \quad \pi_{k}
  \bar{\omega}_k S = S$.  
\end{proof}

\begin{enumerate}[a)]
\item We have $\bar{\omega}_k \pi_k T(\bar{u}) = \langle \pi_k T,
  \overline{Pu} \rangle = T((\overline{Pu})_\Omega) = T (\bar{u})$.  
\item For $v \in H^k (R^n)$ we have
\end{enumerate}
$$
\langle \pi_k \bar{\omega}_k S, \bar{v} \rangle = (\bar{\omega}_k
S)(\bar{v}_{\Omega}) = \langle S, \bar{P} (v_\Omega) \rangle.  
$$

Let $w = P(v ~ )$ we have to prove
$$
\langle S, \bar{w} \rangle = \langle S, \bar{v} \rangle. 
$$

Let $g = w - v$; we have $g \in H^k (R^n)$ and $g_\Omega =
W_\Omega = 0$. By $1 b)$ we have $g = \tilde{h} $ with $h \in
H^q_o ( \sum (\bar{\Omega)}$. Hence 
$$
\langle S, \bar{g} \rangle = \langle S, \tilde{h } \rangle =
\lim\limits_{0} \langle S, \varphi_j \rangle, \varphi_j \in
\mathscr{D} (\in \bar{\Omega}) 
$$
since $\zeta \in H^{-k}_{\bar{\Omega}}$. 

Hence $\langle S, w \rangle = \langle S, \bar{v} \rangle$ which
completes the proof of lemma. To complete the proof of the theorem,
given $ T \in H^{-k + 2m}(\bar{\Omega})$ we have to determine
$\cup \in H^{-k})_{\bar{\Omega}} $ such that $\mathscr{A} V - T
\in M^k$; $\langle \mathscr{A}u - T, \bar{v} \rangle = 0$ for
$v \in H^{k+2m}(R^n)$ such that $v_\Omega \in N^*$; or
such $aU$ that  
\begin{equation*}
  \langle U, \overline{\mathscr{A}^* v} \rangle = \langle{T, \bar{v}}
  \rangle. \tag{1}\label{lec19:sec10:subsec3:eq1} 
\end{equation*}

By hypothesis $1) ~ (a), ~ (b)$, on account of the above lemma, there
exist isomorphisms 	$\bar{\omega}_k, \bar{\omega}_{k + 2m}$ of
$H^{-k}_{\Omega}$ and $H^{-(k+2m)}_{\bar{\Omega}}$ into $(H^k
(\Omega))'$ and $(H^{k+2m}(\Omega))'$ respectively. Let
$\bar{\omega}_k U = u, {\bar{\omega}}_{k+2m} T = t$. Then 
$$
\langle U, \overline{\mathscr{A}^* v} \rangle = \langle \pi_k u,
\overline{\mathscr{A}^* v} \rangle = u((\overline{\mathscr{A}^*
  v)}_\Omega) = u (\overline{A^* v_\Omega}) 
$$
and $\langle T, \bar{v} \rangle = \langle{\pi}_{k + 2m} t, \bar{v}
\rangle = t(\bar{v}_\Omega )$, so that from (\ref{lec19:sec10:subsec3:eq1}) our problem will be
solved if given $t = \bar{\omega}_{k + 2m} T$ we can determine $u
\in (H^k (\Omega))'$ such that  
\begin{equation*}
u(\overline{A^* v_\Omega}) = t (\bar{v}_\Omega), \text { for all } v
\in H^{k + 2m}(R^n) \text { such that } v_\Omega \in
N^*. \tag{2}\label{lec19:sec10:subsec3:eq2} 
\end{equation*}

We\pageoriginale prove that the problem can be still simplified in as much as we
need prove (\ref{lec19:sec10:subsec3:eq2}) only for $w \in H^{k + m} (\Omega)
\in N*$. Indeed on account of $(k + 2m)$ extension property of
$\Omega$, $w \in H^{k+2m}(\Omega) \cap N^*$ is a $v _\Omega$
when $v = P(w)$. Hence our problem is reduced to: Given $ t =
\bar{\omega}_{k+2m} T$ determine $u \in(H^k (\Omega))'$ such
that 
$$
u (\overline{A* v}) = t(\bar{v}) \text { for all } v \in H^{k
  + 2m}(\Omega) \cap N^*.  
$$

Now we use (3). Let $f \in H^k$. Then $A^* v = f$ has a
unique solution $G^*f \in N^*$ which on account of the
hypothesis (3) of the theorem is in $H^{k + 2m}(\Omega)$. If $ f \to
0$ in $H^k (\Omega), G^*f \to 0$ in $H^{k + 2m}(\Omega)$. Hence by the
closed graph theorem, $G^*$ is a continuous mapping of $H^k$ into $H^{k
  + 2m}(\Omega) \cap N^*$ in the topology of $H^{k + 2m}
(\Omega)$. Since $t ~ (H^{k + 2m}(\Omega))' f \to t (\overline{G^*f})$
is a continuous semi-linear mapping on $H^k (\Omega)$ and hence there
exists a unique $u \in (H^k (\Omega))'$ such that  
$$
u (\overline{A^* v}) = u (\bar{f}) = t (G^*f) = t (\bar{v}). 
$$
\begin{remark*} % remark
  $U$ depends continuously on $T$. 
\end{remark*}
