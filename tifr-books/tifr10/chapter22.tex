\chapter{Lecture}\label{lec22}%% 22
\setcounter{section}{11}

\subsection{}\label{lec22:sec11:subsec3}
Now\pageoriginale we come to the

\noindent \textbf{Second step.} 
  We wish to prove $(\phi D_\tau u)^{-h} $ is
  bounded in $H^2 (\Omega)$. To do this we consider $a((\phi D_\tau
  u)^{-h}, v)$ and prove the 
  \begin{lemma}\label{lec22:sec11:subsec3:lem11.6} %lem 11. 6
    $\big | a(( \phi D_\tau u )^{-h}, v ) \big | \leq c || v ||_2$ for $v
    \in \vartheta$ such that $D_\tau \bar{u} \in H^2$.  
  \end{lemma}
  
  \begin{proof} %pro
    We write
    \begin{align*}
      a((\phi D_\tau u)^{-h}, v ) & = a((\phi D_\tau u)^{-h}, v + a(\phi
      D_tau u, v^h) - b(D_\tau u, v^h)\\ 
      & - a(D_\tau u, \phi v^h). 
    \end{align*}
    
    As in the previous cases, we have straight forward estimates except
    for $a(D_\tau u, \phi v^h)$. Now 
    $$
    a(D_\tau u, \phi v^h) = a (D_\tau u, \phi v^h) + a(u, D_\tau (\phi
    v^h)) - a (u, D_\tau (\phi v^h)) 
    $$
    which exists since $D_\tau v \in H^2$. Again as in the
    previous cases, only non-trivial part is to prove that $\big | a(u,
    D_\tau (\phi v^h)) \big | \leq c || v ||_2$.  
  \end{proof}

To do this we have to correct $D_\tau (\phi v^h)$ by the
\begin{lemma}\label{lec22:sec11:subsec3:lem11.7} %lem 11. 7
There exists $w_h \in H^2 (\Omega), w_h = 0$ near $\partial
\Omega - \sum$ such that  
\begin{enumerate}[\rm i)]
\item $D_\tau (\phi v^h) - w_h \in \vartheta$. 
\item $|| w_h ||_1 \leq c || v ||_2$. 
\item $\big | a(u, w_h ) \big | \leq c || v ||_2$. 
\end{enumerate}
\end{lemma}

Admitting this for a moment, we have
\begin{align*}
  a(u, D_\tau (\phi v^h)) & = a(u, w_h) + a(u, D_\tau (\phi v^h) - w_h)\\
  & = a(u, w_h) + (f, D_\tau (\phi v^h) - w_h)
\end{align*}
and we have the lemma \ref{lec22:sec11:subsec3:lem11.6} as usual. 

So we have to prove now lemma \ref{lec22:sec11:subsec3:lem11.7}. If $w_h$ have to verify 
\begin{enumerate}[i)]
\item Then we can write :
\end{enumerate}
$$
B(w_h) = B D_\tau (\phi v^h) = f_h + \sum_{i = 1}^{n - 1} D_i g^i_h,
D_i = \frac{\partial}{\partial x_i}, ~\text { say }.  
$$

We\pageoriginale shall prove that one can choose $f_h$ and $g^i_h$ such that
\begin{enumerate}[a)]
\item $f_h$ have their support in a fixed compact and are bounded in
  $H^{\frac{1}{2}} (\Gamma)$ by $c || v ||_2$.  
\item $g^i_h \in H^{3/2} (\Omega)$, and are bounded in
  $H^{\frac{1}{2}} (\Omega)$ by $c || v ||_2$ with support in a fixed
  compact.  
\end{enumerate}

Assuming (a) and (b) we prove that $w_h$ satisfying (i), (ii),
and (iii) can be found. For by (a) and lemma. \ref{lec21:sec11:subsec2:lem11.1}, there
exists $w^o_h \in H^2 (\Omega)$ such that $w^o_h (x'. 0) = 0,
\dfrac{\partial}{\partial x_n} w^o_h (x', 0) = f_h, w^o_h$ are bounded
in $H^2 (\Omega)$ by $c || v ||_2$ 
 and $w^o_h$ vanish near $\partial \Omega - \sum$. Similarly on
 account of (b) and lemma \ref{lec21:sec11:subsec2:lem11.1}, there exists $w'_h \in
 H^2 (\Omega)$ with $w^i_h (x' , 0) = 0, \dfrac{\partial}{\partial
   x_n} w^i_h (x', 0) = g^i_h, w_h^i$ bounded in $H^2 (\Omega)$ by $c
 || v ||_2$ and $w^i_h \equiv 0$ near $\partial \Omega -
 \sum$. Setting $w_h = w^o_h + \sum_{i = 1}^{n - 1} D_i w^i_h$ we see
 that (i) and (ii) are at once satisfied. Further to verify
 (iii) we have $\big | a (u, w^o_h \big | \leq c || v ||_2$ and it
 remains to estimate $a(u, D_i w^i_h)$. But since $D_\tau u
 \in H^2$ and since $w^i_h = 0$ near $\partial \Omega - \sum$
 by integration by parts, we get $\big |a (u, D_i w^i_h) \big | \leq c
 || v ||_2$.  

We have still to verify (a) and (b), we indicate which parts of
$B(D_\tau (\phi v^h))$ are to be taken as $f_h$ and which as $g^i_h$
and prove each time that they are bounded by $c || v ||_2$ in the
appropriate spaces. We have 
\begin{gather*}
  B(D_\tau (\phi v^h) = B((D_\tau \phi ) v^h ) + B(\phi D_\tau v^h)\\
  B((D_\tau \phi ) v^h) = (D_\tau \phi ) B v^h +
  \left(\frac{\partial}{\partial x_n} D_\tau \phi \right) v^h + \sum \alpha_i
  \left(\frac{\partial}{\partial x_i} (D_\tau \phi )\right) v^h,  
\end{gather*}
and 
$$
Bv^h = -\sum\alpha^h_i \frac{\partial v}{\partial x_i} (x' + h, 0),
\text{ since } (Bv)^h = 0. 
$$ 

Then we take $B((D_\tau \phi )v^h)$ as part of $f_h$; it is
$H^{\frac{1}{2}}$ and is bounded by $c || v ||_2$.  

Now\pageoriginale \quad $B(\phi (D_\tau v^h)) = \phi B(D_\tau v^h) + \dfrac{\partial
  \phi}{\partial x_n} (x', 0) D_\tau v^h + \sum \alpha \dfrac{\partial
  \phi}{i \partial x_i} D_\tau v^h$.  

We consider each of the summands separately :
$$
\frac {\partial \phi}{\partial x_n} (D_\tau v^h) = D_\tau
\left(\frac{\partial \phi}{\partial x_n} v^h\right) - \left(D_\tau \frac{\partial
  \phi}{\partial x_n}\right) v^h.  
$$

We take $D_\tau \dfrac{\partial \phi}{\partial x_n}) v^h$ as part of
$f_n$ and it is seen that it satisfies $(a)$. For $D_\tau
\left(\dfrac{\partial \phi}{\partial x_n} v^h\right)$ we take it as part of
$D_\tau g^i_h$. It is also seen that $g^i_h$ satisfies
(b). Similarly we consider $\displaystyle{\sum_i} \dfrac{\partial \phi}{\partial x_i \partial_{x_i}} D_\tau v^h$. It remains only to consider $\phi B (D_\tau 
v^h)$. Now since $(Bv)^h = 0$, we have  
$$
\phi B(D_\tau v^h ) = - \phi \left(\sum D_\tau \alpha_i \frac{\partial
  v^h}{\partial x_i}+ D_\tau \alpha_o \right) v^h),  
$$
and they are to be taken as parts of $f_h$. This completes the proof
of lemma \ref{lec22:sec11:subsec3:lem11.6}.  

To complete the proof of the theorem, we require the
\begin{lemma}\label{lec22:sec11:subsec3:lem11.8} %lemma 11. 8
  $|| D_\tau (\phi_u)^{-h} ||_2 \leq c$. 
\end{lemma}

Again we require corrections $w'_h$ as follows:
\begin{lemma}\label{lec22:sec11:subsec3:lem11.9} %lem 11. 9
  There exists $w'_h \in H^2 (\Omega)$ vanishing near $\partial
  \Omega - \sum$ and such that 
  \begin{enumerate}[(1)]
  \item $D_\tau (\phi u)^{-h} - w'_h \in \vartheta$
  \item $|| w'_h ||_2 \leq c$. 
  \end{enumerate}
\end{lemma}

Assuming the existence of such $w'_h$ and considering
$$
a(D_\tau (\phi u)^{-h} - w'_h, D_\tau (\phi u^{-h} - w'_h)
$$
and using the ellipticity of $a(u, u)$ we obtain lemma \ref{lec22:sec11:subsec3:lem11.8}, as in
the lemma \ref{lec21:sec11:subsec2:lem11.4}, after observing that lemma \ref{lec22:sec11:subsec3:lem11.6} can be applied
though $v = D_\tau (\phi u)^{-h} - w'_h$ is such that $D_\tau v
\notin | H^2 (\Omega)$. To see this last point we prove that in
$\vartheta$, $v's$ such that $D_\tau v \in H^2 (\Omega)$ are
dense.  

Let $v \in \vartheta$ and $v (x' , 0) = f$. Let $\psi
\in \mathscr{D}(\Omega)$ be such that $\varphi_\alpha f$ in
$H^{3/2} (\Gamma)$, and let $\psi_\alpha = - \sum \alpha
\dfrac{\partial \varphi}{\partial x_i} - \alpha_o \varphi_\alpha$.  

Now\pageoriginale it is possible to find $v_\alpha \in H^2 (\Omega) \cap
\vartheta (\bar{\Omega})$ such that $v_\alpha (x', 0)=
\varphi_\alpha, \dfrac{\partial v}{\partial x_n} = \varphi_\alpha, v
\equiv 0$ near $\partial \Omega - \sum$ and $v_\alpha \to v$ in
$H^2$. By the choice of $\psi_\alpha, v$ belongs to $\vartheta$ and
the result follows.  

To prove lemma \ref{lec22:sec11:subsec3:lem11.9}, we observe
\begin{gather*}
  (\phi u)^{-h} = \phi u^{-h} + \phi^{-h} u(x' - h, x_n), \text { and } \\
  D_\tau (\phi u)^{-h} = (D_\tau \phi)u ^{-h} + \phi D_\tau u^{-h} +
  (D_\tau \phi^{-h} u(x' - h, x_n) + \phi^{-h} D_\tau u(x-h).  
\end{gather*}

We first define $w_h^1$ such that
\begin{enumerate}[a)]
\item $(D_\tau \phi^{-h}) u(x-h) + \phi^{-h} D_\tau u (x-h) - w^1_h
  \in \vartheta$,  and
\item $|| w^1_h ||_2 \leq c$. 
\end{enumerate}

To verify $(a)$ we choose $w^1_h$ so that
\begin{align*}
  w^1_h (x', 0) & = ((D_\tau \phi)^{-h} u(x-h) + \phi^{-h}D_\tau
  u(x-h))_{x_n} = 0. \\ 
  \text{and}\qquad  \frac{\partial \omega'_h (x', 0)}{0 x_n} & = \left(\left(
  \frac{\partial}{\partial x_n}\right)^{-h} u (x-\lambda) + \phi^{-h}
  \frac{\partial}{\partial x_n} u (x -\lambda)\right) x_n=\partial 
\end{align*}

Since $D_\tau u \in H^2 (\Omega)$ the right hand side in the
first expression ie in $H^{3/2} (\Gamma)$, and is bounded. Similarly
the one in the second expression is in $H^{\frac{1}{2}}(\Gamma)$, and
is bounded, and by lemma \ref{lec21:sec11:subsec2:lem11.1} the existence of $w^1_h$ is
proved. Next we find $w^2_h$ so that 
\begin{enumerate}[a)]
\item $(D_\tau \phi)u ^{-h} - w^2_h \in \vartheta$ and 
\item $|| w^2_h ||_2 \leq c$. 
\end{enumerate}

The existence of such $w^2_h$ is assured by lemma \ref{lec21:sec11:subsec2:lem11.4}. Finally we
define $w^3_h$ so that 
\begin{enumerate}[a)]
\item $\sigma D_\tau u^{-h} - w^3_h \in \vartheta$
and 
\item $|| w^3_h ||_2 \leq c$. 
\end{enumerate}

To verify (a), we should have $B(\phi (D_T u^{-h})) = Bw^3_h$. But
\begin{gather*}
B(\phi (D_\tau u^{-h})) = \phi B(D_T u^{-h}) + \frac{\partial
  \phi}{\partial x_n} D_\tau u^{-h} + \sum \alpha_i \frac{\partial
  \phi}{\partial x_i} D_\tau u^{-h}. 
\end{gather*}

Since\pageoriginale $Bu = 0$ and $D_\tau B u = 0$, we have
$$
B(D_\tau u) + \sum_{i=1}^{n-1} (D_\tau \alpha_i ) D_i u + (D_\tau \alpha_o ) u =0. 
$$

Hence on the support of $\phi$, 
\begin{gather*}
(B(D_\tau u))^{-h} + \sum ((D_\tau \alpha_i) D_i u)^{-h} + ((D_\tau
  \alpha_o) u)^{-h} = 0, ~\text{i.e.,} \\ 
(B(D_\tau u^{-h} + \sum \alpha_i^{-h} D_\tau u(x-h) + \sum (D_\tau
  \alpha_i D_i u)^{-h} + ((D_\tau \alpha_o) u)^{-h} = 0; 
\end{gather*}
so that we have to find $w^3_h$ satisfying
\begin{align*}
B w^3_h = - \phi \sum \alpha_i^{-h} D_\tau u(x-h) & - \phi \sum
((D_\tau \alpha_i) D_i u)^{-h} - \phi ((D_\tau \alpha_o ) u)^{-h}\\ 
& + \frac{\partial \phi}{\partial x_n} D_\tau u^{-h} + \sum \alpha_i +
\frac{\partial \phi}{\partial x_i} D_\tau u^{-h} = g_h.  
\end{align*}

Hence it is enough to find $w^3_h$ such that
$$
\begin{cases}
w^3_h (x', 0) = 0\\
\frac{\partial w^3_h}{\partial x_n}(x', 0 ) = g_h ~ \text{defined above. }
\end{cases}
$$

Since $D_\tau u \in H^2 (\Omega), g_h \in H^{3/2}
(\Gamma)$ and is bounded in $H^{\frac{1}{2}} (\Gamma)$. Hence by lemma
\ref{lec21:sec11:subsec2:lem11.1} we have the existence of such $w^3_h$. This completes the
proof of the theorem \ref{lec21:sec11:subsec2:thm11.2}.  

\medskip
\noindent\textbf {Final Remarks. } 
~
\begin{enumerate} [(1)]
\item By using Stampanhia \cite{k17} and Lions [\ref{k10:e6}], Campanato \cite{k6} has
  proved the regularity at the boundary for Picone problems.  
\item For another method for Dirichlet conditions with constant
  coefficients in two dimensions, and very general conditions on the
  boundary, see Agmon \cite{k1}.  
\end{enumerate}
