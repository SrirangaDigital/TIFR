
\chapter{Lecture}\label{lec17}%%% 17
\setcounter{section}{9}

\subsection{Completion of the proof of theorem \ref{lec15:sec9:subsec1:thm9.1}}
\label{lec17:sec9:subsec5}

We\pageoriginale are now in a position to complete the proof of theorem 
\ref{lec15:sec9:subsec1:thm9.1}. Our
problem is to prove that if $\underline{u}$ is such that $Au = f
\in H^m (\Omega^\epsilon )$, then $u \in H^{2m}
(\Omega^\epsilon)$ for every $\in > 0$, where $A =\Sigma
(-1)^{|p|}D^p (a_{pq} D^q)$ with $a_{pq} \in \mathscr{D}
(\bar{\Omega})$. We have already proved that $D^p_\tau u \in
L^2 (\Omega)$ for $|p| \leq m$. We now have to consider $D^p u$. We
denote the derivatives with respect to $x_n$ by $D_y$. In this part of
proof only ellipticity of $A$ is required, and \textit{boundary
  conditions will not be necessary}. We write also $\Omega$ for
$\Omega^\epsilon$.  

Now
\begin{align*}
  Au & = \sum (-1)^m D^m_y g+ \sum_{r 
    \leq m-1 , |p| \leq   2m-r} (x) D^m_y D^p u \hspace{1cm}\tag{1}
  \label{lec17:sec9:subsec5:eq1}\\
  \text{where}~ \hspace{2cm} g &= \sum_{|p| \leq m} (x) D^p u = \beta
  (x) D^m_y u + \cdots \tag{2}\label{lec17:sec9:subsec5:eq2}
\end{align*}

We prove now
\begin{lemma}\label{lec17:sec9:subsec5:lem9.8}% Lemma9.8
  $\re  \beta (x) \geq \alpha > 0 $. 
\end{lemma}

\begin{lemma}\label{lec17:sec9:subsec5:lem9.9}%lemma 9.9
  $g \in H^m (\Omega)$. 
\end{lemma}

\begin{lemma}\label{lec17:sec9:subsec5:lem9.10}%lemma  9.10
  $D^{m+1}_y u \in  L^2 (\Omega )$. 
\end{lemma}

Using these lemmas and lemmas \ref{lec16:sec9:subsec4:lem9.7}, since already it is proved that
$D^p_\tau u \in H^m (\Omega)$, for $|p| \leq m$, we have the  
\begin{corollary}\label{lec17:sec9:subsec5:coro9.1}%corollary 9. 1
  $u \in H^{m+1} (\Omega)$. 
\end{corollary}

\begin{description}
  \item [Proof of the lemma 9.8]%Proof of the lemma 9.8
  It is easily checked that $\beta (x) = a_{\rho \rho} (x)$, $\rho = (0,
  \ldots,\break 0, m)$. Since $\re \Sigma a_{pq}(x) \xi^p \xi^q \geq \alpha
  \xi^{2m}$, taking $\xi = (0, \ldots , 0, 1)$, we have $\re  \beta (x)
  = a_{pq} (x) \geq \alpha > 0$.  
\item [Proof of the lemma 9.9] %Proof of the lemma 9.9
    On account of lemma \ref{lec16:sec9:subsec4:lem9.7}, it is enough to prove (a) $D^m_y g 
    \in L^2 (\Omega )$, (b) $D^\lambda_\tau g \in L^2
    (\Omega )$, $| \lambda | = m$.  
  
  \begin{enumerate}[\rm a)]
  \item follows\pageoriginale from (\ref{lec17:sec9:subsec5:eq1}) for $Au \in L^2$ and $D^r_y D^p_\tau
    u = D^r_y D^q_\tau (D^{q'}_\tau u ) \in L^2 (\Omega )$, $|r|
    \leq m-1, |p| = 2m-r |q|+ r \leq m, |q'| \leq m$ 
    for by proposition \ref{lec16:sec9:subsec3:prop9.2}, $D^{q'}_\tau u \in H^m
    (\Omega)$ and $| q | + r \leq m$.  
  \item follows from (\ref{lec17:sec9:subsec5:eq2}) since we have $D^\lambda _\tau g =
    \sum\limits_{p \quad m, \quad m} \alpha D^p D^\lambda_\tau u$ and
    $D^\lambda_\tau u \in H^m ( \Omega )$ by proposition \ref{lec16:sec9:subsec3:prop9.2}.  
  \end{enumerate}
\item [Proof of lemma9.10] %Proof of lemma 9. 10
    From $(2)$, we have 
    $$
    D_y g = \beta D^{m+1}_y u + (D_y \beta ) D^m_y u + \sum\limits_{| p |
      \leq m+1, | p_n | \leq m} \alpha (x) D^p u.  
    $$

    From lemma \ref{lec17:sec9:subsec5:lem9.9}, $D_y g \in L^2; D^m_y u \in L^2$ as
    $u \in H^m (\Omega )$ and the last sum is in $L^2$ as seen in
    lemma \ref{lec17:sec9:subsec5:lem9.9}. Now, by lemma 
    \ref{lec17:sec9:subsec5:lem9.8}, we have $D^{m+1}_y u \in L^2$.  
\end{description}

Thus, having proved that $u \in H^{m+1} (\Omega )$. There are
two ways in which we could possibly carry the induction. However, the
easier one of proving that $u \in H^{m+k} (\Omega )
\Rightarrow u \in H^{m+k+1} (\Omega )$ does not work for if we
take $D^{k+1}_y g$ we get terms like $\sum\limits_{\rho \leq k+1,| p |
  \leq m, :p_n \leq m-1} \alpha D_y D^p_u$ about which we
cannot say anything at once unless $k = 0$.  

We proceed in a slightly different way. We prove first 
\begin{enumerate}[a)]
\item $D^\lambda_\tau D^{m+1}_y u \in L^2 (\Omega )$ with
  $|\lambda| = k$, and  
\item assuming $D^\lambda_\tau D^{m+1}_y \rho_u \in L^2
  (\Omega )$ for $ | \lambda | \leq k - \rho + 1$,  
  we prove that $D^\mu_\tau K^{m + \rho + 1}_y u \in L^2 $ for
  $| \mu | \leq k - \rho$.  
\end{enumerate}

(a) From (\ref{lec17:sec9:subsec5:eq2}) we have 
$$
D^\lambda_\tau D_y g = \beta D^\lambda_\tau D^{m + 1}_y u + \sum_{|p|
  \leq m+ k + 1, p_n \leq m} \alpha D^p u.  
$$

By\pageoriginale lemma \ref{lec17:sec9:subsec5:lem9.9}, $D^\lambda_\tau D_y g \in L^2$. Since $k+1
\leq m$, and $p_n \leq m$, $D^p u = D^\rho_\tau D^q u $ with $| q |
\leq m$. Hence the last sum is in $L^2$, and $D^\lambda_\tau D^{m+1}_y
u \in L^2 (\Omega )$ as $\re  \beta (x) \geq \alpha > 0$.  

b) Again from (\ref{lec17:sec9:subsec5:eq2}), 
$$
D^\mu_\tau D^{\rho+1}_y g = \beta D^{\mu}_\tau D^{m+\rho + 1}_y u +
\sum_{| q | \leq m+ | \mu | + \int + 1, |q_n| \leq m+ p} \alpha D^q u.  
$$

We have $| q | \leq m+1 k +1, |q_n| \leq m+\rho$; hence by induction
hypothesis, the sum is in $L^2 (\Omega )$. Since $|\mu| + \rho + 1
\leq k + 1 \leq m, D^m_\tau D^{\rho + 1}_y g \in L^2$. Hence
$D^\mu_y D^{m+ \rho + 1}_y u \in L^2$. This proves $D^{2m}_y u
\in L^2$.  

Since we have already proved $D^p_\tau u \in L^2 , |p| \leq
2m$, by lemma \ref{lec16:sec9:subsec4:lem9.7}, we have $\underline{u \in H^{2m} (\Omega
  )}$.  

\subsection{Other results}\label{lec17:sec9:subsec6} 

Theorem \ref{lec15:sec9:subsec1:thm9.1} is but a first step in considering the regularity at
the boundary. We prove now the  
\begin{theorem}\label{lec17:sec9:subsec6:thm9.2}%theorem 9. 2
  Hypothesis being same as in theorem \ref{lec15:sec9:subsec1:thm9.1}, if $f \in H^k
  (\Omega )$ and $Au = f$, then $u \in H^{2m + k} (\Omega )$.  
\end{theorem}

Theorem \ref{lec15:sec9:subsec1:thm9.1} corresponds to the case $k = 0$. The proof of this theorem
is similar in its development to the proof of theorem \ref{lec15:sec9:subsec1:thm9.1}. First by
making use of local mappings we prove that it is enough to make the
proof in the case of a cube $]0, ~ 1[^n$.  

Next to prove $u \in H^{2m + k}$ we have to prove $D^p_\tau u$
and $D^{p_n}_y u $ are in $H^m (\Omega )$ for $| p | \leq m+k$ and $
p_n \leq m+k$ respectively. The third step not involving the boundary
conditions is essentially the same as in the previous
considerations. We consider briefly the second step by proving the  
\begin{lemma}\label{lec17:sec9:subsec6:lem9.11}%lemma 9.11
  $D^p_\tau u \in H^m (\Omega^\epsilon)$\pageoriginale for $| p | \leq m+k$. 
\end{lemma}

We have proved this lemma for $k = 0$; we assume it to be true for $1,
\ldots , k-1$, and prove it for $k$. As before we consider the
difference quotients $(D^r_\tau u)^{-n} $ with $| r | = m+k+1$, and
prove that they are bounded in $H^m (\Omega^\epsilon )$. It is
actually enough to show that $(\phi D^r u)^{-h}$ is bounded where
$\phi \in \mathscr{D} (\bar{\Omega})$ vanishing near $\partial
\Omega - \sum$. As before, we consider the identity  
$$
a((\phi D^r_\tau u)^{-h} v) = a((\phi D^r_\tau u)^{-h}, v) + a(\phi
D^r_\tau u, v^h) - b(D^r_\tau u, v) -a (D^r_\tau u, \phi v^h) 
$$
where $b(u, v) = a(\phi u, v) -a (u, \phi v)$. 

By induction hypothesis we may assume that $D^p_\tau u \in H^m
(\Omega^\epsilon )$ for $|p| \leq m+k-1$. Using this and almost the
same manipulation as in proposition \ref{lec15:sec9:subsec2:prop9.1}, we prove that $a((\phi
D^r_\tau u)^{-h}, v) + a (\phi D^r_\tau u, v^h) $ and $ b (D^r_\tau u,
v )$ are bounded in $H^m (\Omega^\epsilon )$ by $c|| v ||_m$. To
prove $a(D^r_\tau u, \phi v^h)$ is bounded we write  
\begin{multline*}
  a(D^r_\tau u, \phi v^h) = \left[ a(D^r_\tau u, \phi v^h) + (-1)^{|r| - 1}
    a (u, D^r_\tau (\phi v)^h) \right]\\
  + (-1)^{|r|-1} a(u, D^r (\phi v)^h). 
\end{multline*}

The first sum is proved to be bounded again by the same methods as
those used in proposition \ref{lec15:sec9:subsec2:prop9.1}. However, to prove $e(u, D^r_\tau \phi v^h)$ is
bounded, we cannot use at once $a(u, D^r_\tau \phi v^h) = (f, D^r_\tau
\phi v^h)$, for $D^q_\tau v$ is not necessarily in $H^m (\Omega )$ for
$|q| \leq r$. However, by regularization, it is seen that such
$\underline{v}$ that $D^q_\tau v \in H^m (\Omega)$ for $|q|
\leq r$ are dense in $H^m (\Omega )$. It is enough then to prove
$a(D^r_\tau u, \phi v^h)$ is bounded for such $v'$s and then we have
$a(u, D^r_\tau (\phi v^h))_o, |r| = m+k-1$. Now, since $f \in
H^k$ we can integrate the last expression $k$-times by parts\pageoriginale and
obtain 
\begin{align*}
  a(u, D^r_\tau (\phi v)^h) & = | (-1)^k (D^k_\tau f, D^{m-1}_\tau (\phi v)^h)\\
  &\leq c| v |_o \leq c || v ||_m. 
\end{align*}

Having proved then that $a((\phi D^r_\tau u)^{-h}, v)$ is bounded by
$c || v ||_m$ in $H^m (\Omega )$ by putting $v = \phi (D^r_\tau
u)^{-h}$, we obtain, as usual, by ellipticity, that $|| \phi (D^r_\tau
u)^{-h} ||_m \leq c$ and by now standard arguments that $D^{m+1}_\tau
u \in H^m (\Omega^\epsilon )$.  

From theorem \ref{lec8:sec4:subsec2:thm4.4} we have $H^\rho (\Omega) \subset \mathscr{E}^\circ
(\Omega)$ if $a\rho > n$. Further if $\Omega$ has $\rho$-extension
property from theorem 
\ref{lec8:sec4:subsec2:thm4.4} $H^\rho (\Omega ) \subset \mathscr{E}^\circ (\bar{\Omega})$. Hence, by
using theorem \ref{lec17:sec9:subsec6:thm9.2}, we have the  
\begin{theorem}\label{lec17:sec9:subsec6:thm9.3}%theorem 9. 3
  Under the hypothesis of theorem \ref{lec15:sec9:subsec1:thm9.1}, if $2k > n$, then $u$ is in
  $\mathscr{E}^{2m} (\bar{\Omega})$.  
\end{theorem}

In this case $u$ is a \textit{usual} solution of Neumann problem. 

\begin{remarks*}
  Analogous proof applies for Dirichlet's problem. Now the question
  arises for what spaces $V$ such that $H^n_o \subset V \subset H^m$
  can we apply the above methods for proving regularity at the
  boundary. One of the crucial steps in above proof was the
  manipulation of difference quotients $v^h$ and hence the subspace of
  $V$ consisting of functions which vanish near the boundary $\partial
  \Omega - \Sigma$ must be invariant for translations. For spaces $V$
  given by conditions like $\bigg\{ u, \dfrac{\partial u}{\partial
    \eta}, \ldots , \dfrac{\partial^k u}{\partial_\eta 
    k} = 0, k \leq n-1 \bigg\}$, this condition is satisfied. However, for
  spaces $V \subset H^m, m \geq 2$, determined by conditions like
  $\alpha (x) u + \beta (x) \dfrac{\partial u}{\partial x_n} = 0$ on
  $\Sigma$, this condition is obviously not satisfied. Nevertheless by
  changing a little the method of proof the\pageoriginale regularity theorems have
  been proved by Aronsza jn-Smith for the spaces $V$ given by
  conditions like $\alpha (x) u + \beta (x) \dfrac{\partial u}{\partial
    x_n} = 0$. We shall consider these methods in later lectures.  
\end{remarks*}

\subsection{An application of theorem \ref{lec15:sec9:subsec1:thm9.1}}
\label{lec17:sec9:subsec7}  

Let $A$ and $N$ be as in theorem \ref{lec15:sec9:subsec1:thm9.1}. On $N$ we consider the metric
$| u |_N = || u ||_m + | Au |_o$ and on $H^{2m} \cap N$, the metric
$|| u ||_{2m} + || u ||_m + | Au |_o$ which defines on $H^{2m} \cap N$
the upper bound topology. The inclusion mapping $H^{2m} \cap N \to N$
is continuous and since from theorem \ref{lec15:sec9:subsec1:thm9.1} it is onto, it is an
isomorphism. Hence 
$$
\displaylines{\hfill
|| u ||_{2m} + || u ||_m + | Au |_o \geq c (|| u ||_o + | Au |_o). \hfill\cr
\text{i.e. ,}\hfill 
| Au |_o + || u ||_m \geq \gamma' || u ||_{2m}. \hfill} 
$$

This is equivalent to 
$$
| Au |_o + | u |_o \geq \gamma ' || u ||_{2m}, 
$$
in the case of strongly $m$-regular open sets (which is the case in
theorem \ref{lec15:sec9:subsec1:thm9.1}).  

This is proved directly by Ladyzenskya for the case $m = 2$, and by
Guseva for the general case. [ ~ ]. To obtain the regularity at the
boundary from these inequalities, one has to prove moreover a
non-trivial density theorem.  

