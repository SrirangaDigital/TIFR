
\chapter{Lecture}\label{lec20}
\setcounter{section}{10}

\subsection{Application}\label{lec20:sec10:subsec4}

We\pageoriginale now consider some applications of the above theory bringing out how
the usual non-homogeneous boundary value problems are particular case
of Visik-Soboleff problems.  

Let $V$ be such that $H^1_o (\Omega) \subset V \subset H^1
(\Omega)$. and $a(u, v) = (u, v)_1 + \lambda (u, v)_o$ for $\lambda >
0$. The operator $A$ associated with $a(u, v)$ is then by \S\ \ref{lec7:sec3:subsec5}
 $\mathcal{F} - \triangle + \lambda$. Let $\mathscr{A} = - \triangle +
\lambda$. Since $a(u, v)$ is hermitian $A = A^*$ and $n =
N^*$. Visik-Soboleff problem reads now for $M^o$ as: Given $T
\in H^{-2}_{\bar{\Omega}}$ determine $U \in L^2
(\bar{\Omega})$ such that $ - \triangle u + \lambda v - T \in
^\circ$. From theorem \ref{lec19:sec10:subsec3:thm10.2}, it follows that this problem admits a
solution, say for example, if $\Omega$ has smooth boundary.  

Now we take a particular $T = \tilde{f} + S$ where $f \in L^2
(\Omega)$ and $S \in H^{-2}_\Gamma$. Since the support of
$\mathscr{A}u - T$ is in $\Gamma$ restricting to $\Omega$ we see $Au =
f$ where $u = U_\Omega$. Further $\langle \mathscr{A} U - T, \bar{v}
\rangle = 0$ for all $v \in H^2 (R^n)$ such that $v_\Omega
\in N^*$. Hence for such $v$,  
\begin{align*}
  \langle - \triangle U + \lambda U, \bar{v} \rangle & = \langle T,
  \bar{v} \rangle = \langle \tilde{f}, \bar{v} \rangle + \langle S,
  \bar{v} \rangle\\ 
  & = (- \widetilde{(\triangle U) + \lambda u}), \bar{v}	 \rangle +
  \langle S, \bar{v} \rangle.  
\end{align*}

Formally, by Green's formula, 
\begin{align*}
  \langle - \triangle U + \lambda U, \bar{v} \rangle & = \int_\Omega -
  (\triangle + \lambda) U \bar{v} dx = \int_\Omega U(- \triangle +
  \lambda ) \bar{v} dx\\ 
  = \int_\Omega u. (- \triangle + \lambda ) \bar{v} dx & = - \int u
  \frac{\partial \bar{v}_\Omega}{\partial \eta} d \sigma +
  \int \frac{\partial u}{\partial \eta} \bar{v}_\Omega d \sigma +
  \int_\Omega (- \triangle + \lambda ) u \bar{v}_\Omega dx 
\end{align*}
and $\langle \widetilde{(- \triangle u + \lambda u)}, \bar{v} \rangle =
\int_\Omega (- \triangle + \lambda ) u \bar{v} dx$.  

Hence the original problem is formally equivalent to: given $S
\in H^{-2}$, find $u \in L^2 (\Omega)$ such that
$\int_\Gamma \left(\dfrac{\partial u}{\partial \eta} \bar{v} - u
\dfrac{\partial \bar{v}}{\partial \bar{\eta}} \right) d \sigma = \langle S,
\bar{v} \rangle$ where $v \in H^2(R^n)$ such that $v_\Omega
\in N^*$.  

We\pageoriginale now take some particular cases of $S$. 
\begin{enumerate}[1)]
\item Let $g$ and $h \in L^2 (\Gamma )$. If $\varphi
  \in \mathscr{D} (R^n)$ the mapping $\varphi \to \int_\Gamma
  g \bar{\varphi}d \sigma - \int h \dfrac{\partial
    \bar{\varphi}}{\partial \eta}d$ is a continuous linear mapping on
  $\mathscr{D} (R^n)$ with the topology of $H^2 (R^n)$. For if
  $\varphi \to 0$ in $H^2 (R^n), \dfrac{\partial \varphi}{\partial
    \eta} \to 0$ in $H^1 (R^n)$ and since the mapping $\gamma$ from
  $H^1 (R^n)$ to $L^2 (\Gamma )$ is continuous, $\int_\Gamma g
  \bar{\varphi} d \sigma$ and $\int_\Gamma h(\dfrac{\partial
    \bar{\varphi}}{\partial \eta}) d \sigma$ tend to zero in $L^2
  (\Omega)$. Hence this mapping defines $S \in H^{-2}
  (\Gamma)$. Then (1) reads 
  \begin{equation*}
    \int_\Gamma \left(\frac{\partial u}{\partial n} \bar{v } - u \frac{\partial
      \bar{v}}{\partial n}\right) d \sigma = \int_\Gamma g \bar{v}d - \int_\Gamma
    h \frac{\partial \bar{v}}{\partial n} d \sigma \tag{2}\label{lec20:sec10:subsec4:eq2} 
  \end{equation*}
  Let now
  \begin{enumerate}[a)]
  \item $V = H^1 (\Omega)$. Then $\dfrac{\partial v}{\partial n}= 0$ and
    $(1)$ means $\int_\Gamma \dfrac{\partial u }{\partial n} \bar{v} d
    \sigma = \int_\Gamma g \bar{v} d \sigma$ for all $v$. Hence
    $\dfrac{\partial u}{\partial n} = g$, on $\Gamma$. Hence the problem
    solved is  
    \begin{equation*}
      \triangle \quad \lambda \quad - u + u = f, \frac{\partial u}{\partial
        n } = g \text { on } \Gamma. \tag{3}\label{lec20:sec10:subsec4:eq3} 
    \end{equation*}
  \item $V = H^1_o (\Omega)$. Then $\gamma v = 0$, and (\ref{lec20:sec10:subsec4:eq2}) becomes
    \begin{equation*}
      - \triangle u + \lambda u = f, ~ u = h \text { on } \Gamma. \tag{4}\label{lec20:sec10:subsec4:eq4}
    \end{equation*}
  \end{enumerate}
\item Another example would be to take $g \in H^{-3/2}
  (\Omega)$ and $h \in H^{-\frac{1}{2}} (\Gamma)$. Then the
  mapping $\varphi \to \langle g, \bar{\varphi} \rangle - \langle h,
  \gamma \left(\dfrac{\partial \bar{\varphi}}{\partial n}\right \rangle$ is
  continuous on $\mathscr{D} (R^n)$ with the topology of $H^2 (R^n)$
  for as we shall prove later on $\gamma \varphi \to 0$ in $H^{3/2}
  (\Omega)$ and $\gamma \left(\dfrac{\partial \varphi}{\partial_n}\right) \to 0$
  in $H^{\frac{1}{2}} (\Omega)$, as $\varphi \to 0$ in $H^2
  (R^n)$. This defines a $S \in H^{-2} (\Gamma)$. With this
  $S$ and  
\begin{enumerate}[a)]
\item $V = H^1$, the formal problem solved is $- \triangle u + \lambda
  u = f, \dfrac{\partial u}{\partial n} = g \in
  H^{-\frac{3}{2}} (\varphi)$.  
\item $V = H^{\frac{1}{2}}, \ldots ~ - \Delta u + \lambda u = f, u =
  h \in H^{-\frac{1}{2}}(\Omega)$.  
\end{enumerate}
\end{enumerate} 

These are the problems studied by the Italian School. The principal
problem is to give a precise meaning to (\ref{lec20:sec10:subsec4:eq3}), (\ref{lec20:sec10:subsec4:eq4}) and so on. (See
Magenes \cite{k11}.  

\section{Aronszajn and Smith Problems\texorpdfstring{\protect\footnotemark[1]}{1}}\label{lec20:sec11}

\footnotetext[1]{The author's thanks are due to Professors Aronszajn and
  Smith for lending him an unpublished manuscript concerning these
  problems.} 

\subsection{Complements on \texorpdfstring{$H^m (\Omega)$}{Hm(Omega)}}\label{lec20:sec11:subsec1}

In\pageoriginale \S\ \ref{lec4:sec2:subsec4}, we have defined a mapping $\gamma$ of $H^1 (\Omega)$
onto $H^{\frac{1}{2}} (\Gamma)$ where $\bar{\Omega}= \{ x_n > o \}$ and
$\Gamma = \{x_n ~ = 0\}$. Now we prove the 
\begin{proposition}\label{lec20:sec11:subsec1:prop11.1} % proposition 11. 1
  Let $\Omega = \{ x_n > 0 \}$ and $\Gamma = \{ x_n = 0 \}$. Then the
  mapping $\gamma$ maps $H^m (\Omega)$ onto $H^{m - \frac{1}{2}}
  (\Gamma)$ for all $m$.  
\end{proposition}

\begin{proof} %proof
  We denote $x' = (x_1, \ldots , x_{n-1}), x_n = y, \xi' = (\xi_1,
  \ldots , \xi_{n - 1})$.  
\end{proof}

Here we shall prove that the mapping is into. That it is onto will
follow from a more general theorem to be proved later on. Since
$\Omega$ is $m$-extendible and since $\gamma$ on $ \mathscr{D}
(\bar{\Omega})$, the restrictions of functions $u (x', y)$ of
$\mathscr{D} (R^n) $ is $u (x', 0)$, it is enough to prove that the
mapping $u \to u (x', 0)$ is a continuous mapping of
$\mathscr{D}(R^n)$ with the topology of $H^m_o(R^n)$, into $H^{m -
  \frac{1}{2}} (\Gamma)$. This we do by using Fourier transform.  

Let $\mathscr{F} (u(x)) = \bar{u} (\xi) = \int e^{-2 \pi i x. } u (x)
dx$ be the Fourier transform of $u$. Then $u(x) = \int e^{2 \pi ix. }
v (\xi ) d \xi$ and so $u (x', 0) = \int e^{2 \pi \xi'. } ~ v(\xi,
\xi_n) d \xi d \xi_n$. Hence  
\begin{equation*}
\mathscr{F}'_{x'} u (x', 0) = \int v(\xi' \xi_n) d \xi_n. \tag{1}\label{lec20:sec11:subsec1:eq1}
\end{equation*}

We have now to prove that the mapping $\mathscr{F} (u) = v
\mathscr{F}_{x'} (u(x', 0))$ is continuous from $\mathscr{F}
(\mathscr{D})$ with the topology of $\hat{H}(m)$ into $\mathscr{F}
(H^{m - \frac{1}{2}} (\Gamma))$ or that $v \to (1 + | \xi |^{m -
  \frac{1}{2}}) \mathscr{F}_{x'} (u (x', 0))$ is continuous from
$\mathscr{F} (\mathscr{D})$ with the topology of $\hat{H}^m $ into
$L^2$.  

Hence we have to prove, using (\ref{lec20:sec11:subsec1:eq1}), that
$$
\int (1 + | \xi | ^{2m-1} d \xi' \Big| \int v (\xi', \xi_n ) d \xi_n
\Big |^2 \leq c \int (1 + | \xi |^{2m}) | v (\xi) \big|^2 d \xi.  
$$
Now\pageoriginale 
\begin{align*}
  \Big | \int v (\xi' , \xi_n ) d \xi_n \Big |^2 & = \Big | \int v (\xi'
  , \xi_n ) (1+ |\xi|^m) (1 + |\xi|^m) -1 d \xi _n |^2\\ 
  & \leq \int | v (\xi', \xi_n) |^2 ( 1 + |\xi |^m)^2 d \xi_n \frac{d
    \xi_n }{(1 + |\xi|^m)^2}\\ 
  & \leq c \int \frac{d \xi_n}{(1 + | \xi |^2 )^m} \int |v|^2 (1 + |\xi
  |^m )^2 d \xi_n. \tag{2}\label{lec20:sec11:subsec1:eq2} 
\end{align*}

Putting
\begin{equation}
   \xi_n = \sqrt {1 + \xi'^{2}} t, \int \frac{d
    \xi_n}{(1+ \overline{|\xi|}^2)^{m - \frac{1}{2}}} =\int \frac{dt}{(1 +
    t^2)^m} \tag{3}\label{lec20:sec11:subsec1:eq3} 
\end{equation}

Hence from (\ref{lec20:sec11:subsec1:eq2}) and (\ref{lec20:sec11:subsec1:eq3}), 
\begin{align*}
  \int (1 + |\xi' |^{2m-1}) & \Big| \int v (\xi' , \xi_n ) d \xi_n \Big
  |^2\\ 
  & \leq c \int (1 + |\xi' |^{2m - 1}) \frac{d \xi'}{(1+|\xi'|^2)^{m
      - \frac{1}{2}}} \int | v|^2 (1 + | \xi |^m)^2 d \xi_n\\ 
  & \leq c \int |v |^2 (1 + |\xi|^m)^2 d \xi 
\end{align*}
as was to be proved. 

Now, if $u \in H^m (\Omega)$, we have $\dfrac{\partial
  u}{\partial x_n} \in H^{m-1} (\Omega)$ and by the above
proposition, $\gamma \left(\dfrac{\partial u}{\partial x_n}\right) \in H^{m
  - \frac{1}{2}} (\Gamma)$. Hence we have the 
\begin{coro*} % corollary
$\gamma_\partial (u) = \gamma (D^j_y u) \in H^{m - j -
    \frac{1}{2}}(\Gamma)$ for $j = 1, \ldots , m - 1$.  
\end{coro*}

Now, let $\overset{\to}{\gamma} (u) = (\gamma_o u, \ldots , \gamma_{m
  -1} u)$ and $F = H^{m - \frac{1}{2}} (\Gamma) \times \ldots \times H^{m - j -
  \frac{1}{2}} (\Gamma)$ with the product Hilbertia an structure.  

\begin{theorem}\label{lec20:sec11:subsec1:thm11.1} % theorem 11. 1
  The mapping $u \to \overset{\to}{\gamma} (u)$ of $H^m (\Omega)$ onto
  $F$ with kernel $H^m_o (\Omega)$.  
\end{theorem}

From the above proposition, it follows that $\overset{\to}{\gamma}$
maps $H^m (\Omega)$ into $F$ and that its kernel is $H^m_o
(\Omega)$. To prove that $\overset{\to}{\gamma}$ is onto it is enough
to show that if $\overset{\to}{f} = (0, \ldots , f_j, \ldots , 0)
\in F$ with $f_j \in H^{m - j - \frac{1}{2}}
(\Gamma)$, then there exists $u \in H^m (\Omega)$ such that
$\gamma_j u = f_j $ and $\gamma_k u = 0$ for $k \neq j$ and $k \leq m
- 1$.  

Taking\pageoriginale Fourier transforms in $x'$, with $\xi' = (\xi_1, \ldots ,
\xi_{n - 1})$ we have to find $v (\xi', y)$ such that 
\begin{enumerate}[1)]
\item $(1 + | \xi |^m) v (\xi', y) \in L^2 (, y)$. 
\item $D^m_y v(\xi' , y) \in L^2 (\xi' , y), $ and
\item (a)  $D^j_y v (\xi' , 0) = \hat{f}_j (\xi')$

  (b) $D^k_y v (\xi' , 0) = 0$ for $k \neq j, k \leq m - 1$. 
\end{enumerate}

Put \qquad $\phi (\xi' , y) = \dfrac{1}{j !} y^j e^{-1 + | \xi' | y}
\hat{f}_j (\xi')$.  

Let $t = ( 1 + | \xi' |) y$. Then put
$$
v (\xi' , y) = \phi (\xi' , y ) (1 + \alpha_1 t + \cdots + \alpha_{m
  - j - 1} t ^{m - j - 1}).  
$$

By direct computation, we have
\begin{gather*}
  D^k_y \phi (\xi', 0) = 0 \text { for } k \leq j - 1, ~D^j_y \phi (\xi'
  , 0) = \hat{f}_j (\xi' ), \text { and }\\ 
  D^{j+1}_y (\xi' , 0) = \frac{1}{j !} C^j _{j + 1} D^j (y^j)(-1)^l (1 +
  |\xi'|)^l \hat{f}_j (\xi').  
\end{gather*}

Hence $D^k_y v(\xi' , 0) = 0$, for $k \leq j - 1$, $D^j_y (\xi' , 0) =
\hat{f}_j (\xi)$, and 
$$
D^{j+1}_y v(\xi' , 0) = C^j_{j + 1} (-1)^l (1 + | \xi'|)^l F_j ( ~ ) +
(j + 1) C^j_{j +l-1} (1 + |\xi'|)^{l-1} (_1 1+ |\xi' |) \hat{f}_j
(\xi').  
$$
$\alpha_1, \ldots , \alpha_{m - j - 1}$ are determined by $m-j-1$
conditions that $D^{j+1}_y v (\xi', 0) = 0$ for $l = 1, \ldots ,
m-j-1, . \alpha_1, \ldots , \alpha_{m-j-1}$ are then well-determined
independent of $\xi', $ e. g. , $(j + 1) \alpha_1 = C^j_{j + 1}$ and
so on.  

It remains to verify
\begin{enumerate} [(a)]
\item $( 1 + |\xi' |^m) t^k \phi (\xi' , y) \in L^2$ for $k
  \leq m - j -1$, and 
\item $D^m_y (t^k \phi (\xi' , y) \in L^2$. 
\end{enumerate}
\begin{enumerate}[(a)]
\item We have to consider $\Big | (1 + | \xi' |^m t^k \phi (\xi' , y)
  \Big |_o$.
\end{enumerate}
\begin{align*}
 \int \Big(1 + |\xi'|^{2m}\Big) & \Big| \hat{f}_j (\xi' ) \Big|^2 (1 +
  |\xi'|^{2k}) d \xi' \int^\infty_o y^{2j + 2k} e^{-(1+ |\xi'|)y} dy\\ 
  &= \int (1 + | \xi' |^{2m} | \hat{f}_j (\xi') |^2 (1+|\xi'|^{2k}) d
  \xi' \int^\infty_o \frac{t^{j+2k_e {-t}dt}}{(1 + |\xi' |^{2j + 2k +
      1}}\\ 
  & \tag*{by putting  $(1 + | \xi' |) y = t$.}\\ 
  & \leq c \int ( 1 + |\xi'|^{2m-2j-1} | \hat{f}_j ( \xi') |^2 d \xi' <
  \infty, \text { since } f_j \in H^{m - j - \frac{1}{2}}
(\Gamma).  
\end{align*}
\begin{enumerate}
\renewcommand{\theenumi}{\alph{enumi}}
\renewcommand{\labelenumi}{\theenumi)}
\setcounter{enumi}{1}
\item We\pageoriginale have to consider $D^m_y (t^k y^j e^{-(1 + |\xi'|)y} \hat{f}_j
  (\xi'))$. This is a sum of terms 
  $$
  t^{k-r}(1 + | \xi |^r) y^{j-1 }(1 + |\xi|^{m-r-1} e^{(1 + |\xi'|)y}
  \hat{f}_j (\xi') \text { for } \underset{1 = 1, \ldots , j. }{r = l,
    \ldots, k}.  
  $$
  
  Hence we have to consider
  \begin{align*}
    \int & (1 + |\xi'|)^{2k-2r} (1 + | \xi'|)^{2m-2l} | \hat{f}_j |^2
    d\xi' \int y^{2j + 2l + 2k - 2r} e^{-2 (1 + | \xi' | y)} dy\\ 
    & = \int (1 + | \xi |)^{2k - 2r}(1 + |\xi'| )^{2m - 2l} |\hat{f}_j |^2 d
    \xi' \frac{1}{(1 + | \xi' |)^{2j-2l+2k-2r+1}} \\ 
    &\tag*{$t^{2j-2l+2k-2r} e^{-2t}d$ by putting $t = (1 + |\xi'| ) y$}\\
    & \leq c \int (1 + | \xi'|)^{2m-2j-1} | \hat{f}_j |^2 d \xi' < \infty, 
  \end{align*}
  which proves the theorem. 
\end{enumerate}
