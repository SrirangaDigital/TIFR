
\chapter{Lecture}\label{lec6} %%% 6

\setcounter{section}{3}
\subsection{Examples: \texorpdfstring{$\Delta + \lambda, \lambda >0, \Delta = \sum\limits_{i=1}^n
  \dfrac{\partial^2 u}{\partial x_i^2}$}{Delta}}\label{lec6:sec3:subsec3}

Let\pageoriginale $\Omega$ be an open set in $R^n$ and $H^1(\Omega), H^1_0 (\Omega)$
be as defined before. Let $V$ be a closed subspace of $H^1$ such that
$H^1_0 \subset V \subset H^1$. The metric on $V$ is one induced by
$H^1 : (u, v)_V = (u, v)_1$. Let $Q$ be $L^2 (\Omega)$ with the
involution $f \rightarrow \bar{f}$. Then $V \subset Q$ and is dense in
$Q$. On $V$ consider the sesquilinear form 
$$
a(u, v) = (u, v)_1 + (u, v), \lambda > 0.
$$

Then $a(u, v)$ is continuous on $V \times V$ and 
\begin{align*}
\re  (a(u, u)) & = |u|^2_0 + \lambda |u|^2_1 \ge \min (1, \lambda)
(|u|^2_0+ |u|^2_1)\\ 
& = \alpha \|u\|^2_1 , \quad \alpha > 0.
\end{align*}

Hence $a(u, v)$ is V-elliptic. Hence, for a given $f \in
L^2(\Omega) = Q'$ we have $u \in V$ such that $a(u, v) = <f,
\bar{v}>$ for all $v \in V$. We determine $N$ and $A$
explicitly in this case. 

\begin{proposition}\label{lec6:sec3:subsec3:prop3.1}%prop 3.1
~
\begin{enumerate}[\rm 1)]
\item $A = - \Delta u + \lambda u$ {\rm for} $u \in N \lambda > 0$
\item $u \in N \Leftrightarrow 
  {\begin{cases}  
           u \in
           V, \Delta u \in L^2 & {\rm and}\\ 
    (- \Delta u, v)_0 = (u, v)_1 & {\rm for ~all } \quad v \in
    V. \end{cases} } $  
\end{enumerate}
\end{proposition}

\begin{proof}%proof
We know $u \in N$ if and only if $u \in V$ and the
mapping $v \rightarrow a(u, v)$ is continuous on $V$ with the topology
of $Q$. Further since $a(u, v)$ is V-elliptic, for $f \in L^2$
there exists $u \in N$ such that $a(u, v) = <Au, \bar{v}> =
<f, \bar{v}>$. Let $v =\varphi \in \mathscr{D}(\Omega)$. Then 
\begin{align*}
 (u, \varphi)_1 + \lambda (u, \varphi)_0 &= (Au, \varphi )_0.\\
\text{Now},\qquad (u, \varphi )_1 = \Sigma \left(\frac{\partial u} {\partial
  x_i}, \frac{\partial \varphi}{\partial x_i}\right)_0  &= <- \Sigma
\frac{\partial^2 u}{\partial x_i^2}, \varphi>\\ 
 &= <- \Delta u, \varphi>.
\end{align*}

Hence\pageoriginale $<- \Delta u, \bar{\varphi} > + \lambda < u, \bar{\varphi}> = <
Au, \bar{\varphi}> = <f, \bar{\varphi}>$ for all $\varphi \in
\mathscr{D} (\Omega)$. This means $A = - \Delta + \lambda$ and $-
\Delta u + \lambda u = f$. Since $f \in L^2$ and $u
\in L^2$ we have $\Delta u \in L^2$. Further, if $u
\in N, a(u, v) = <Au, \bar{v} > $ and hence 
\begin{align*}
(u, v)_1 + \lambda (u, v)_0 & = (- \Delta u, v)_0 + \lambda (u, v)_0\\
\text{which gives}\qquad (u, v)_1 &= (-\Delta u, v)_0.
\end{align*}

Conversely if $u$ satisfies the above conditions, since $-\Delta u +
\lambda u \in L^2$ the mapping $v \rightarrow a(u,v) = (u,v)_1
+ \lambda(u, v)_0 = (-\Delta u + \lambda u, v)_0$, is continuous on
$V$ in the topology induced by $Q$. Hence $u \in N$. 
\end{proof}

Now we give a \textit{formal} interpretation of $u \in N$. The
correct meaning of this interpretation will be brought out later
on. Assuming the boundary $\Gamma$ of $\Omega$ to be smooth, we have,
by a formal Green's formula, 
$$
(- \Delta u, v)_0 = - \int_{\Omega} \Delta u. \bar{v} dx =
\int_{\Gamma}\frac{\partial u} {\partial n} \bar{v} d \sigma + (u,
v)_1 
$$
where $\dfrac {\partial u}{\partial n}$ is the normal
derivative. However if $u \in N$, by proposition \ref{lec6:sec3:subsec3:prop3.1}, we
have  
$$
(- \Delta u, v)_0 = (u,v)_1.
$$

Hence $u \in N$ implies $\int \dfrac {\partial u}{\partial n}
\bar{v} d \sigma = 0$. 

We now take particular cases of $V$ and interpret this formal result.
\begin{enumerate}[1)]
\item Let $V = H^1$. $u\in H^1$ is not a boundary condition,
  neither is $\Delta u \in L^2$. However, $\int\limits_{\Gamma}
  \dfrac {\partial u}{\partial n} \bar{v} d = 0$ for every $v
  \in H^1$, is a boundary condition, and formally this means
  $\dfrac{\partial u} {\partial n} = 0$, i.e., $u \in N$
  implies the normal derivative vanishes. 
\item Let\pageoriginale $V = H^1_0. u \in H^1_0$ implies $u =0$ on the
  boundary, and hence is a boundary condition. $\Delta u \in
  L^2$ is not a boundary condition and $\int\limits_{\Gamma} \dfrac{\partial
    u} {\partial n} \bar{v} d \sigma  $ is always zero for $v
  \in H^1_0$. 
\item Let $\Gamma_1$ be the subset of $\Gamma$ and define $V$ to
  consist of function $u \in H^1$ such that $\gamma u = 0$ on
  $\Gamma_1. V$ is a closed subspace of $H^1$. $u \in N$ if and
  only if $u \in V$, that is to say, $\gamma u = 0$ on $
  \Gamma_1$; this is a boundary condition, $\Delta u \in L^2$
  which is not a boundary condition, and $\int\limits_{\Gamma}
  \dfrac{\partial u}{\partial n} \bar{v} d \sigma = 0$ for $v
  \in V$, but since $\gamma v = 0$ on $\Gamma_1$, this means
  $\int\limits_{\Gamma-Q} \dfrac{\partial u} {\partial n} \bar{v} d \sigma =
  0$ for all $v \in V$. This means again formally $\dfrac
  {\partial u} {\partial n} = 0$ on $\Gamma - \Gamma_1$. So formally
  the condition is $u = 0$ on $\Gamma_1$ and $\dfrac {\partial
    u}{\partial n} = 0$ on $\Gamma - \Gamma_1$. 
\end{enumerate}

We call 1), 2) and 3) weak homogeneous, Neumann, Dirichlet and
mixed Dirichlet-Neumann problems respectively. 

We may state the above results in the
\begin{theorem}\label{lec6:sec3:subsec3:thm3.2}% thm 3.2
  If  $\lambda > 0, \Omega$ an arbitrary open set in $R^n$, then the
  equation $-\Delta u + \lambda u = f$ with $f \in L^2 (\Omega)$
  has a unique solution with homogeneous boundary data. 
\end{theorem}

\begin{remarks*}%rem 0
  {\em Non-homogeneous problems:} Corresponding to the homogeneous
  problems considered above, we may consider non-homogeneous ones in
  which not necessarily vanishing boundary values are prescribed. We
  shall show formally that this can be reduced to a homogeneous case
  together with a problem of first order partial differential equation. 
\end{remarks*}

\begin{problem}\label{lec6:sec3:subsec3:prob3.2}%problem 3.2  %%% (for book 3.2)
  Given\pageoriginale $F \in L^2$ and $G \in V$ such that $\Delta G
  \in L^2$ determine $u$ such that $- \Delta U + \lambda U = F $
  and $U - G \in N$. 
\end{problem}

\begin{theorem}\label{lec6:sec3:subsec3:thm3.3}%thm 3.3
  Problem \ref{lec6:sec3:subsec3:prob3.2} admits a unique solution for $\lambda > 0$.
\end{theorem}

\begin{proof}%proof
  Put $u = U-G$. Then we have to seek $u$ such that
  $$
  - \Delta u + \lambda u = F - (- \Delta + \lambda) G = f, \text{ say }.
  $$
  
  Since $f \in L^2$, there exists unique $u \in N$ by
theorem \ref{lec6:sec3:subsec3:thm3.2}. 
\end{proof}

In the case $V$ is as in examples 1), 2) and 3) respectively, this
means $\dfrac{\partial u}{\partial n}$ on $\Gamma, G = U$ on $\Gamma$
and $U = G$ on $\Gamma_1$ and $\dfrac{\partial U}{\partial n} =
\dfrac{\partial G}{\partial n}$ on $\Gamma - \Gamma_1$ respectively. The
above solution of problem \ref{lec6:sec3:subsec3:prob3.2} implies then that if we wish to
determine $U$ with $\dfrac{\partial U}{\partial n}, U$, given on the
boundary, we have only to determine $G \in L^2$ satisfying
$\Delta G \in L^2$ and $\dfrac{\partial U}{\partial n} =
\dfrac{\partial G}{\partial n}$ and $U = G$ on the respective parts of
the boundary.
 
\begin{remark*}
  If we take $V= H^1_0, Q = H^{-1}$ so that  $Q' = H^{-1}$ we have
\end{remark*}

\begin{theorem}\label{lec6:sec3:subsec3:thm3.4}%thm 3.4
  Given a distribution $T \in H^{-1}$, there exists a unique
  solution $U \in H^1_0$ such that $-\Delta u + \lambda u = T$. 
\end{theorem}

\begin{remark*}
  Roughly speaking we may say that the boundary conditions are
  introduced by means of the following two conditions :$(a) u
  \in V, (b) (- \Delta u, v)_0 = (u, v)_1$. The two extreme
  cases are $H^1_0$ (Dirichlet) and $H^1$ (Neumann) wherein, in the
  first case, only $u \in V$ is the boundary condition, and in
  the second one, $(- \Delta u, v)_0 = (u, v)_1$ is the boundary
  condition. The condition $u \in V$ may be considered to be
  \textit{stable} and the  other one \textit{unstable}. Heuristically
  this may be justified as follows: if we consider smooth functions in
  $H^1(\Omega) $ such that $\dfrac{\partial u}{\partial n}= 0$, on
  completion this property no longer holds, so we may say this condition
  is unstable,\pageoriginale while in the second case, the completion of smooth
  functions vanishing on boundary still possesses this property in a
  weaker sense.   
\end{remark*}

\begin{exercise}\label{lec6:sec3:subsec3:exr1}%exercise 1
With the hypothesis as in theorem \ref{lec5:sec3:subsec2:thm3.1}, if $a(u, v)$ is V-elliptic,
the existence of $u \in V$ such that $a(u, v) = <f, \bar{v}>$
for all $v \in V$ and any $f \in Q'$ can be carried on
the following lines: Since the mappings $v \rightarrow <f, \bar{v}>$
and $v \rightarrow a(u, v)$ are continuous on $V$ with the topology
induced by $Q$, there exists $Kf$ and $\tilde{A}f$ in $V$ such that 
$$
a(u, v) = (\tilde{A}u, v)_V <f, \bar{v}> = (Kf, v)_V.
$$

Hence to solve the problem we require $\tilde{A}u = Kf$. This is
proved if we prove $\tilde{A}$ is an isomorphism of $V$ onto $V$. 
\end{exercise}

\begin{exercise}\label{lec6:sec3:subsec3:exr2}%exercise 2
The same results as in theorem \ref{lec5:sec3:subsec2:thm3.1} is true on a weaker assumption that 
$$
|a(u, v)| \geq \alpha |u|^2_V,~ \alpha > 0.
$$
\end{exercise}

