
\chapter{Lecture}\label{lec13}%% 13
\setcounter{section}{7}

\subsection{Applications}\label{lec13:sec7:subsec4}

Let\pageoriginale $ a(u,v) = \sum_{|p|,|q| \le m} \int~ a_{pq} D^q (u)\bar{D}^p ~ v
~ dx \text{ with }  a_{pq} ~ \in ~ L^\infty$ 

and 
$$
A(u,v) = \sum_{|p|,|q| = m } ~ a_{pq} D^{q} ~ (u) ~ \bar{D}^p ~ v ~ dx
$$
be the leading part of $a(u,v)$.

\setcounter{theorem}{7}
\begin{theorem}\label{lec13:sec7:subsec4:thm7.8}% theorem 7.8
  Let $(a) \Omega$ be strongly $m$-regular and $(b) \re  A(u,u) \ge
  \alpha | u |^2_m$ for some $\alpha > 0$ and for all $u ~ \in
  ~ H^m(\Omega)$. Then there exists $\lambda$ such that $\re  ~ a(u,u) +
  \lambda | u |^2_0 \ge \beta || u ||^2_m$ for some $\beta > 0$, and for
  all $u ~ \in ~ H^m$. 
\end{theorem}

\begin{proof}% proof
  We have 
  $$
  \re  ~ a(u,v) = \re  ~ A(u,u) + \re  ~ \rho (u,u)
  $$ 
  where 
  $$
  \rho (u,v) = \sum_{|p|\le m,} \int a_{pq} D^q u D^p v ~ dx ~ |q| \le m
  \text{ and } |p| + |q| \le 2m - 1. 
  $$
\end{proof}

Every term of $\rho (u,v)$ is majorized by $c || u ||_m || u ||_{m-1}$
and so $\re  ~ \rho (u,u) \le c_1 || u |~_m || u ||_{m-1}$. Hence 
$$
\re  ~ a(u,u) \ge \alpha | u |^2_m - c_1 ~ || u ||_m ~ ||u ||_{m-1}.
$$

We have then to prove that we can find $\lambda$ such that there
exists $\beta$ satisfying  
\begin{equation}
X = ~ \alpha | u |^2_m - c_1 || u ||_m ~ || u ||_{m-1} + \lambda | u
|^2_0 \ge \beta || u ||^2_m \tag{1}\label{lec13:sec7:subsec4:eq1} 
\end{equation}

Since $\Omega$ is strongly m-regular, using definition for any
$\in > 0$, there exists $c(\in)$ such that $|| u
||_{m-1}  \le \in || u ||_m + c(\in) | u |_o$. Hence\pageoriginale
$c_1 || u ||_m ||u ||_{m-1} \le c_1 \in || u ||^2_m +
c(\in) || u ||_m | u |_o$. Since $\Omega$ is m-regular also
$|| u ||_m$ is equivalent to $| u |_m + | u |_o$. Hence $c_1 || u ||_m
|| u ||_{m-1} \le c_2 ~ \in (| u |_m + | u |^2_0) +
c'(\in)(|u|_m|u|_o + |u |^1_o)$. So $X \ge \alpha | u |^2_m -
c_2 ~ \in ~ (|u|^2_m + | u |^2_o) - c'(\in) (|u|_m
|u|_o + |u|^2_o + |u|^2_o$. Now $2| u |_m| u |_o \le ~ \in_1 ~
| u |^2_m + \dfrac{1}{\in_1} | u |^2_o$ for any
$\in_1$. Hence 
$$
X \ge \left(\alpha - ~c_2 ~ \in - \dfrac{\in_1
  c'(\in)}{2}\right) ~  | u |^2_m + (\lambda - c'' (\in) +
\frac{1}{\in_1}) | u |^2_o. 
$$

First we choose $\in$ so that $\alpha - c_2 \in =
\frac{\alpha}{2}$. This determines $c(\in)$ and
$c'(\in)$. Then we choose $\in_1$ so small that
$\in_1 c'(\in) < \alpha / 4 $, and then $\lambda$ so
large that $\lambda - c''(\in) + \dfrac{1}{\in_1} > 0
$. 

Then $X \ge \beta_1 (|u|^2_m + | u|^2_o)$ and by m-regularity of
$\Omega$, $X \ge \beta || u ||^2_m$ as it was required to be proved. 

\section{Regularity in the Interior}\label{lec13:sec8} % section 8

\subsection{}\label{lec13:sec8:subsec1}

Having established the existence and uniqueness of weak solutions of
certain elliptic differential equations, we turn now to consider
their regularity problem, that is to say, to see whether in the
equation $Au = f$ sufficient regularity of $f$ will imply some
regularity of $u$. First, we shall investigate when $u$ is regular in
the interior of the given domain $\Omega$ and next we shall consider
when $u$ is regular in $\bar{\Omega}$ in some sense. 

To formulate the problem of interior regularity, we shall require some
definitions of new spaces. 

We recall having defined in \S\ \ref{lec3:sec2:subsec1}, that $H^{-r} ~ (\Omega) =
(H^r(\Omega))'$, for positive $r$. If $0$ is an open set in $\Omega$
and if $u$ is a function in $\Omega$, $u_0$ will denote the
restriction of $u$ to $0$. 

\begin{definition}\label{lec13:sec8:subsec1:def8.1}% definition 8.1
  Let\pageoriginale $\Omega$ be an open set in $R^n. \mathscr{L}^r =
  H^r_{loc}(\Omega)$ for any integer $r$, consists of functions $u$
  which for any relatively compact $0 \subset \Omega$ are such that $u_0
  ~ \in ~ H^r (0)$, $r$ integer $\ge 0 \text{ or } < 0$. 
\end{definition}

Let $K_n$ be an increasing sequence of closures of relatively compact
open sets $0_n$ covering $\Omega$.  Let $p_n = || u_{0_n} ||_r$ be the
norms in $H_{0_n}^r$ of $u_{0_n}$. $p_n'$s are semi-norms in
$\mathscr{L}^r$. On $\mathscr{L}^r$ we put the locally compact
topology determined by the semi-norms $p_n$.  

\begin{definition}\label{lec13:sec8:subsec1:def8.2}% definition 8.2
  $K^r ~ r$, any integer will denote the space of $u ~ \in ~
  H^r$ with compact support. 
\end{definition}

On $K^r$ we put the natural inductive limit topology. A sequence $u_n$
converges in $K^r$ if all $u_n$ have their supports in a fixed compact
$A$ and all $u_n \rightarrow 0$ in $H^r (A)$ . We see easily $(Z^r)' =
K^{-r}$. 

\begin{proposition}\label{lec13:sec8:subsec1:prop8.1}% proposition 8.1
  $\mathscr{E}'(\Omega) = \bigcup\limits_{r \in Z}K^r (\Omega)$
  algebraically. 
\end{proposition}

\begin{proof}% proof
By definition $\cup ~ K^r ~ (\Omega) ~ \subset ~ \mathscr{E}' ~
(\Omega)$. We have to prove only that if $T ~ \in ~
\mathscr{E}'$ then $T~ \in ~ K^r$ for some $r$. Now by a
theorem of Schwartz, $T ~ \in ~ \mathscr{E}' (\Omega)$ implies
$T = \sum\limits_{|p|\le \mu}D^p f_p$ where $f_p$ are continuous and
have a compact support. Hence $f_p ~ \in ~ L^2 (\Omega)$ and
by theorem \ref{lec3:sec2:subsec1:thm2.1}, $T ~ \in ~ H^{-\mu}(\Omega)$. This means that
$T ~ \in ~ K^r (\Omega)$, where $ r = - \mu$. 
\end{proof}

\begin{proposition}\label{lec13:sec8:subsec1:prop8.2}% proposition 8.2
  Let $B = \sum\limits_{|p|\le \mu} b_p ~ (x) ~ D^p$ with $b_p ~
  \in ~ \mathscr{E}$. Then $B$ is a continuous linear mapping of
  $\mathscr{D}, \mathscr{E},\mathscr{D}',\mathscr{E}'$ into itself and
  also a continuous linear mapping of $\mathscr{L}^r (\Omega)$ into
  $\mathscr{L}^{r-\mu}(\Omega)$ and $K^{r}(\Omega)$ into
  $K^{r-\mu}(\Omega)$. 
\end{proposition}

\begin{remark*}
  It is not true, however, that $B$ is continuous from $H^r$ to $H^{r-\mu}$.
\end{remark*}

\begin{proof}% proof
  The\pageoriginale first assertion is trivial and the last one follows if we prove the
  middle one. Let $f ~ \in ~ \mathscr{L}^r(\Omega)$. Since $b_p
  ~ \in ~ \mathscr{E}$ on $\Omega$, $b_p'$s and their
  derivatives are bounded on $0$ so that it is enough to prove that
  $D^\mu ~ f ~ \in ~ H ~ ^{r-\mu}(0)$.  We may assume $0$ to be
  an open set with smooth boundary. If $\mu < r$ and $r > 0$ we have the
  result from the definition. If $\mu > r$, then by integration by parts
  for $g \in H^{\mu -r}(0)$. 
  $$
  \langle D^{\mu} ~ f, g \rangle = (-1)^{\mu - r} \langle D^\mu ~ f,
  D^{\mu - r} ~ g \rangle 
  $$
  exists and hence $D^m ~f $  is a continuous linear function on $H^{\mu
    - r} ~ (0)$, i.e., $D^\mu$  $f ~ \in ~ H^{r-\mu} ~ (0)$. 
\end{proof}

\subsection{Statements of theorems}\label{lec13:sec8:subsec2} %%% 8.2

Let
\begin{equation}
  A = \sum_{|p|,|q|\le m} (-1)^p D^p(a_{pq}(x)D^q),
  a_{pq} ~ \in ~ \mathscr{E} ~ (\Omega) \tag{1}\label{lec13:sec8:subsec2:eq1} 
\end{equation}

\begin{definition}\label{lec13:sec8:subsec2:def8.3}% definition 8.3
  A is {\em{uniformly elliptic}} in $\Omega$ if given any compact $K
  \subset \Omega$ we have an $\alpha_{K}> 0$ such that 
  \begin{align*}
    \re  ~ \left(\sum ~ a_{pq}(x) \xi ~ ^{p}\xi^q\right) & \ge ~ \alpha ~
    _{K}|\xi|^{2m} \text{ for all } x ~ \in ~ K \text{ and all }\\
    \xi & = (\xi_1, \ldots, \xi_n ) \in ~ R^n.\tag{2}\label{lec13:sec8:subsec2:eq2} 
  \end{align*}
\end{definition}

\begin{remark*}% remark
  If $A$ is uniformly elliptic, Garding's inequality (Theorem \ref{lec12:sec7:subsec1:thm7.1}) is
  true on every compact $K$.  
\end{remark*}

Let 
\begin{equation}
  B = \sum_{|p|\le \mu} b_p(x) ~ D^p, b_p ~ \in ~
  \mathscr{E}(\Omega) \tag{3}\label{lec13:sec8:subsec2:eq3} 
\end{equation}

\begin{definition}\label{lec13:sec8:subsec2:def8.4}% definition 8.4
  $B$ is elliptic in $\Omega$ if $\sum b_p(x) \xi^p = 0$ with $\xi ~
  \in ~ R^n$, implies $\xi = 0$. 
\end{definition}

We see at once that a uniformly elliptic operator is elliptic. The
converse, however, is inexact. For example, in the case $n =2$, $B =
\dfrac{\partial}{\partial x_1} + i \dfrac{\partial}{\partial x_2}$ is
elliptic, but evidently is not uniformly elliptic being not of even
order. 

In\pageoriginale this and the next lecture, we shall prove the following two
theorems on the regularity in the interior. 

\begin{theorem} \label{lec13:sec8:subsec2:thm8.1}% theorem 8.1
  Let $A$ be a uniformly elliptic operator of order $2m$ in $\Omega$. If
  for some $T ~ \in ~ \mathscr{D}' (\Omega), A ~ T ~
  \in~ \mathscr{L}^r(\Omega)$ for some fixed $r$, then $T
  \in \mathscr{L} ~ ^{r+2m}(\Omega)$. 
\end{theorem}

\begin{theorem}\label{lec13:sec8:subsec2:thm8.2}% theorem 8.2
  Let $B$ be an elliptic differential operator of order $\mu$ in
  $\Omega$. If for some $T ~ \in ~ \mathscr{D}' (\Omega), B ~ T
  ~ \in ~\mathscr{L}^r (\Omega)$ for some fixed $r$, then $T ~
  \in   ~\mathscr{L}^{r+\mu} (\Omega)$. 
\end{theorem}

From these theorems, the regularity in the classical sense will follows by the 

\begin{coro*}
  Let $B$ be an elliptic operator of order $\mu$. If for some $T ~
  \in ~ \mathscr{D}'$, $B ~ T ` \in ~ \mathscr{E}$, then
  $T ~ \in ~ \mathscr{E}$. 
\end{coro*}

For $B ~ T ~ \in ~ \mathscr{E}$ means $B ~ T ~ \in ~
\mathscr{L}^r $ for all $\mu$. Hence by the theorems $T ~ \in
~ \mathscr{L}^{r+\mu}$ for every $r$. Hence all the derivatives of $T$
will be functions which means $T ~ \in ~ \mathscr{E}$. 

Before proving these theorems, we shall establish some connections
between elliptic and uniformly elliptic operators. Using these, we
shall prove that theorem \ref{lec13:sec8:subsec2:thm8.1} implies theorem \ref{lec13:sec8:subsec2:thm8.2}, and then we
shall occupy ourselves in the proof of theorem \ref{lec13:sec8:subsec2:thm8.1}. 

\begin{proposition}\label{lec13:sec8:subsec2:prop8.3}% proposition 8.3
  Let $n \ge 3$. If $B$ is elliptic, then $B$ is of even order (See
  Schechter \cite{k15}). 
\end{proposition}

Let $x ~ \in ~ \Omega$ and $\sum\limits_{|p| = \mu} ~
b_p(x)\xi^p = Q (\xi)$, That $B$ is elliptic at $x$ means that only
real zero of $Q(\xi)$ is $\xi = 0$. We prove $Q(\xi)$ must be of even
degree. By a non-singular linear transformation, if necessary, we may
assume $\xi_n$ has degree $\mu$. 

Let\pageoriginale $\xi' = (\xi_1,\ldots,\xi_{n-1}) \neq 0$ be a point in $R^{n-1}
\subset \mathscr{C}^n$. Let $Q(\xi_n)$ be the polynomial in $\xi_n$
obtained by substituting $(\xi_1,\ldots, \xi_{n-1})$ by $\xi'$. Then
$Q(\xi_n)$ has $n$ complex roots all of which have imaginary part
$\neq 0$, for otherwise $(\xi', \xi_n)$ would be a real non-trivial
zero of $Q(\xi)$. Let $\pi_+$ and $\pi_-$ be number of roots of
$Q(\xi_n)$, with positive and negative imaginary parts
respectively. On account of homogeneity of $Q(\xi)$ if we put $\xi'' =
- \xi'$, the number of positive roots of $Q(\xi'', \xi_n)$ will be
$\pi_-$ and negative roots will be $\pi_+$.  Let $\xi', \xi''$ be
joined by an are not passing through the origin which is possible
because $n-1 \ge 2 $. From a classical theorem the roots of $Q(\xi_n)$
can be continued from $\xi'$ to $\xi''$ continuously.  Now at no point
on the are $\xi', \xi''$ can $\xi_n$ be real on account of
ellipticity. So the positive roots at $\xi'$ are continued into
positive roots at $\xi''$. Hence $\pi_+ = \pi_-$ and $\mu = \pi_+ +
\pi_- = 2 \pi_+$ is even. 

\begin{proposition}\label{lec13:sec8:subsec2:prop8.4}% proposition 8.4
  Let $B$ be an elliptic differential operator with real
  coefficients. Then $B$ is uniformly elliptic in $\Omega$. 
\end{proposition}

\begin{proof}
  Let $\sum\limits_{|p| = \mu} ~ b_p (x) \xi^p = P (x,\xi)$. Since $B$
  is elliptic, $P(x,\xi) = 0$ implies $\xi = 0$. Hence on $|\xi| = 1,
  P(x,\xi)$ for fixed $x$ keeps same sign which we may assume $>
  0$. Hence $\dfrac{P(x,\xi)}{|\xi|^\mu} \ge \alpha$ for fixed $x$. If
  now $K$ is any compact, $P(x,\xi)$ is continuous on the compact $K  \times
  ~ |\xi | = 1$ and hence $\dfrac{P(x,\xi)}{|\xi|^\mu} > \alpha$ for
  all $x ~ \in ~ K$ and all $\xi ~ \neq ~ 0$. If we put $-\xi$
  for $\xi$ we get the inequality multiplied by $(-1)^\mu$ hence $\mu$
  is even. 
\end{proof}

\begin{proposition}\label{lec13:sec8:subsec2:prop8.5}% proposition 8.5
  Theorem \ref{lec13:sec8:subsec2:thm8.1}\pageoriginale $\Rightarrow$ Theorem \ref{lec13:sec8:subsec2:thm8.2}
\end{proposition}

\begin{proof}% proof
  Let $B = \sum\limits_{|p|\le \mu} ~ b_p (x) D^p$. Put $\bar{B} =
  \sum\limits_{|p|\le \mu} \overline{b_p(x)}D^p$. Let  
  $$
  A = \bar{B}B = \sum_{|p| = |q|=\mu} ~ b_p(x)~ \overline{b_p(x)} ~ D^p
  ~ D^q + \cdots 
  $$
\end{proof}

$A$ is of even order and if $P(x,\xi)$ is its associated form, then
$P(x,\xi) = | \Sigma b_p(x) \xi~ ^{p}|^2$. Further since $B$ is
elliptic, $A$ also is. By proposition \ref{lec13:sec8:subsec2:prop8.4}, $A$ is  then uniformly
elliptic. Let now $T ~ \in ~ \mathscr{D}'$ such that $BT
\in \mathscr{L}^r$. Hence $A~ T = \bar{B} B ~ T ~ \in
~ \mathscr{L}^{r-\mu}$. If theorem \ref{lec13:sec8:subsec2:thm8.1} is true, then $T ~
\in ~ \mathscr{L}^{r+\mu}$ proving theorem \ref{lec13:sec8:subsec2:thm8.2}. 
