\chapter{A Theory of Mellin Transformations}\label{chap1}

IN\pageoriginale THIS CHAPTER, we develop a theory of Mellin transformation for the
multiplicative group of an arbitrary local field, Although that is the
only case in which we shall be interested, we shall start by recalling
the definition of the Mellin transformation in the general case.

We shall assume results in the theory of locally compact abelian
groups and Fourier transforms.

\section{Generalities}\label{chap1:sec1}

\subsection{}\label{chap1:sec1:subsec1} 
Let $G$ be a locally compact abelian group and $G^\ast$, the Pontrjagin
dual of G consisting of the group $\Hom(G,\mathbb{C}_1^\times)$ of all
continuous homomorphisms of G into the group $\mathbb{C}_1^\times)$ of
complex numbers of absolute value1. For $g\epsilon G$ and $g^\ast
\epsilon G^\ast$,we write $<g,g^\ast>=g^\ast(g)$.

Let $dg$ be the Haar measure on $G$ (unique upto a positive scalar factor)
and $L^1(G)$ the space of complex-valued functions which are
integrable on $G$. For $F$ in $L^1(G)$, the Fourier transform $F^\ast$ is
defined by 
\begin{equation*}
  F^\ast(g^\ast)=\int\limits_GF(g)<g, g^\ast>dg\quad for\quad
  g^\ast\epsilon G^\ast; 
\end{equation*}
then $F^\ast$ belongs to the space $L^\infty(G^\ast)$, which is the
completion relative to the uniform norm of the space of complex-valued
continuous functions on $G^\ast$ with compact support. 

Let $\Lambda(G)$ be the space of continuous functions $F$ in $L^1(G)$
whose Fourier transform $F^\ast$ is in $L^1(G^\ast)$; then $F
\mapsto F^\ast$ is a bijection of $\Lambda(G)$ on
$\Lambda(G^\ast)$. The Haar measure $dg^\ast$ on $G^\ast$ can be
normalized so that $(F^\ast)^{\ast}(g)=F(-g)$ for every
$F\epsilon\Lambda(G)$ and $g\epsilon G$. This is just the Fourier
inversion theorem and the measure\pageoriginale $dg^{\ast}$ is said to be the
{\em dual} of dg. 

\subsection{}\label{chap1:sec1:subsec2} %1.2

Let $\Omega(G)=\Hom (G,\mathbb{C}^\times)$, the group of continuous
homomorphisms of $G$ into the group $\mathbb{C}^\times$ of non-zero complex
numbers. An element $\omega$ of $\Omega(G)$ is called a {\em
  quasicharacter} of G. The fact that
$\mathbb{C}^\times=\mathbb{R}_+^\times \times \mathbb{C}_1^\times$ (direct product) induces
a corresponding decomposition for any $\omega$ in $\Omega(G)$. Indeed,
let $r(g)=|\omega(g)|$ for any $g\in G$ and then
$g\mapsto\omega(g)/r(g)$ is in $G^{\ast}$ so that for every g in G we
have
\begin{equation*}
  \omega(g)=r(g)<g,g^{\ast}>\tag{1}\label{chap1:sec1:subsec2:eq1} 
\end{equation*}
for a unique $g^{\ast}$ in $G^{\ast}$.

For any complex-valued function $F$ on $G$ and any $\omega$ in
$\Omega(G)$, let us define
\begin{equation*}
  Z(\omega)=\int\limits_GF(g)\omega(g)dg\tag{2}\label{chap1:sec1:subsec2:eq2}
\end{equation*}
where, for the moment, we say nothing about the existence of the
integral and merely remark that we have a function Z defined on a
certain subset of $\Omega(G)$. We call the map $M:F\rightarrow Z$
the {\em Mellin transformation}.

Substituting (\ref{chap1:sec1:subsec2:eq1}) in (\ref{chap1:sec1:subsec2:eq2}), we have
\begin{equation*}
  Z(\omega)=\int\limits_GF(g)r(g)<g,g^{\ast}>dg
\end{equation*}
and the Fourier inversion theorem yields the following

\setcounter{theorem}{1}
\begin{theorem} \label{chap1:sec1:subsec2:thm2} %%% thm1.2
  If $dg^{\ast}$  is the dual of the measure dg  on a locally compact
  abelian g group $G,\in \Omega(G)$  and $F:G\rightarrow
  \mathbb{C}$ are given, then under the assumption that the function
  $g \mapsto F(g) r(g)$ is in $\wedge (G)$, the Mellin transform is
  defined and further 
  \begin{equation*}
    F(g)r(g)=\int\limits_{G^{\ast}} Z(\omega)<g,-g^{\ast}>dg,
  \end{equation*}
  i.e.\pageoriginale
  \begin{equation*}
    F(g)=\int\limits_{G^\ast}Z(\omega)\omega(g)^{-1}dg.
  \end{equation*}
\end{theorem}

\subsection{} \label{chap1:sec1:subsec3} %%% 1.3

If we take $G=\mathbb{R}{_+^\times}$, then $\Omega(G)$ consists of
$\omega(x)=x^s$ with $s=\sigma+it(i=\sqrt{-1})$ s and t being
real. Now $r(x)=x^{\sigma}$ and $<x,t>=x^{it}=e^{ity}$ with $y=\log x$
(or equivalently $x=e{^y}$). The map $x\mapsto y=\log x$ gives a
bijection ${\mathbb{R}_+^\times}{\xrightarrow{\sim}}\mathbb{R}$. Now 
\begin{equation*}
  Z(s)=\int\limits_0^\infty F(x)x^sd\log x =
  \int\limits_{-\infty}^\infty F(e^y)e^{\sigma
    y}\mathfrak{e}(\frac{1}{2\pi}yt)dy. 
\end{equation*}

The Lebesgue measure dy on $\mathbb{R}$ is its own dual, relative to
$(y,y')\mapsto \mathfrak{e}(yy')$. Thus Fourier's inversion theorem
gives, under the assumption that $F(x)x^{\sigma}\in
\wedge(\mathbb{R}_+^\times)$ (or equivalently that
$F(e^y)e^{\sigma y}\in\wedge(\mathbb{R}))$
\begin{equation*}
  F(y)e^{\sigma y}= \int\limits_{-\infty}^\infty
  Z(s)\mathfrak{e}(-\frac{1}{2\pi}yt)\frac{1}{2\pi}dt
\end{equation*}
i.e.
\begin{equation*}
  F(x)=\frac{1}{2\pi}\int\limits_{-\infty}^\infty
  Z(s)x^{-\sigma-it}dt=\frac{1}{2\pi i}\int\limits_{\sigma-\infty
    i}^{\sigma+\infty i}Z(s)x^{-s}ds
\end{equation*}
which is just the assertion of the Theorem above for
$G=\mathbb{R}{_+^\times}$.

\subsection{}\label{chap1:sec1:subsec4}%1.4

In the special case $G=\mathbb{R}{_+^\times}$ discussed above, the fact
that $\Omega(\mathbb{R}{_+^\times})\backsimeq \mathbb{C}$ plays an
important role; we are thereby in a position to use the theory of
functions of one complex variable. We would like to know if we have,
in a general situation, $G^{\ast}\subset\Omega(G)$ with $G^{\ast}$
being a real Lie group (possibly without separability)and $\Omega(G)$
its complexification. This is, in fact, true if and only if $G$ is
compactly generated (i.e. generated by a compact neighbourhood
of $0$). From the structure of locally compact abelian groups, we know
that $G$ is compactly generated if and only if there exists a unique
maximal compact subgroup $C$ with $G\backsimeq
C\times\mathbb{R}^{m}\times\mathbb{Z}^{n}$. But then $G^{\ast}\backsimeq
C^{\ast}\times\mathbb{R}{^m}\times(\mathbb{C}{_1^\times})^{n}$ where $C^{\ast}$ is
discrete and $\Omega(G)\backsimeq
C^{\ast}\times \mathbb{C}^{m}\times\mathbb({C}^\times)^{n}$. In this situation, one
can expect to find a good theory of Mellin transforms. As a matter of
fact, we shall take $G$ to be the group $K^\times$ of non-zero elements of
a local\pageoriginale field $K$, where, by a {\em local field}, we mean $\mathbb{R}$
or $\mathbb{C}$ or a finite extension of Hensel's field $\Theta_p$ of
$p$-adic numbers or the field of quotients of the ring of formal
power-series with a finite coefficient field. In the case of
$\mathbb{R}$-fields $K$ (i.e.$K=\mathbb{R}$ or $\mathbb{C}$), we have
$K^{*}\backsimeq K{_1^\times}x\mathbb{R}$ where $K{_1^\times}$ is the compact
group of elements of $K$ of absolute value $1$. In the last two cases
which we refer to as $p$-fields, $K^\times\backsimeq K{_1^\times}\times \mathbb{Z}$
where $K{_1^\times}$ is a similarly defined compact group in $K$; here $K$ is
not always compactly generated while $K^\times$ is compactly generated.

\section{Asymptotic Expansions}\label{chap1:sec2}

Keeping in view some applications, later on, like the determination of
the asymptotic behaviour of certain integrals naturally associated
over fields with forms of higher degree and of the nature of the
so-called ``local singular series'' for such forms, we now introduce a
general notion of asymptotic expansions.

Let $Y$ be a topological space and $x_{\infty}$, a point of the closure
$\overline{X}$ of a sub-space $X$ of $Y$ such that $X$ is separable at
$x{_\infty}$(i.e. the system of neighbourhoods at $X{_\infty}$ has a
countable base). Let $\varphi{_0}, \varphi{_1},\varphi{_2},\ldots$ be a
given sequence of complex-valued functions on X such that for every
$k\ge0$, we have 
\begin{enumerate}
\renewcommand{\theenumi}{\roman{enumi}}
\renewcommand{\labelenumi}{(\theenumi)}
\item $\varphi_{k}(x) \neq 0$ for all $x$ in X which are different
  from $x{_{\infty}}$ and sufficiently close to $x{_{\infty}}$ and 
\item $\varphi_{k+1}(x)= 0(\varphi_{k}(x))$ as $x$ tends to
  $x_{\infty}$ i.e. for given $\epsilon > 0,
  |\varphi_{k+1}(x)|\le\epsilon|\varphi_{k}(x)|$ for $x$ in $X$ close
  enough to $x_{\infty}$. We then say that a complex valued function
  $f$ on $x$ has an {\em asymptotic expansion as} $x$(in $X$) {\em tends
    to} $x_{\infty}$, if there exists a sequence $\{a_{n}\}_{n\ge0}$ of
  complex numbers such that for every $k\ge0$ and all $x$ in X close
  to $x_{\infty}$,
\end{enumerate}
\begin{equation*}
  f(x)=\sum_{i=0}^{k}a_{i}\varphi_{i}(x)+\mathcal{O}(\varphi_{k+1}(x))
  \tag{3}\label{chap1:sec2:eq3}   
\end{equation*}
i.e:\pageoriginale $|f(x)-\sum_{i=0}^{k}a_{i}\varphi_{i}(x)|\le C|\varphi_{k+1}(x)|$
for a constant $C>0$ independent of $x$. In symbols we denote this by 
\begin{equation*}
  f(x)\approx\sum_{k=0}^{\infty}a_k\varphi_k(x)~\text{as}~ x\rightarrow
  x_{\infty}\tag{4}\label{chap1:sec2:eq4}
\end{equation*}

The symbol $\mathcal{O}$ and $\mathsf{o}$ are in the sense of Hardy-Littlewood.
For the given sequence $\{\varphi_{n}\}$,condition $(i)$ clearly
implies the uniqueness of $a_{0},a_{1},a_{2},\ldots$ and therefore an
asymptotic expansion for $f$ is unique, if it exists. Further, it is
sufficient to assume that (\ref{chap1:sec2:eq3}) holds for a cofinal
set of natural numbers $k_{1}<k_{2}<\ldots$ 

Let $Y=\mathbb{R}^{n}$ and $X$, an open set in $\mathbb{R}^{n}$.
Taking $x_{1},\ldots,x_{n}$ as coordinates in $\mathbb{R}^{n}$, let
for $\alpha=(\alpha_{1},\ldots,\alpha_{n})$ with non-negative integers
$\alpha_{1},\ldots,\alpha_{n}$,$D^{\alpha}$ denote the differential
operator $\frac{\partial^{\alpha{_1}+\ldots+\alpha_{n}}}{\partial
  x_{1}^{\alpha_{1}}\ldots\partial x_{n}^{\alpha_{n}}}$. If, in the
foregoing, $f,\varphi_{0},\varphi_{1}, \varphi_{2},\ldots$ are
infinitely differentiable and if, further, for every $\alpha$,
\begin{equation*}
  (D^{\alpha}f)(x)\approx\sum_{k=k_{\alpha}}^{\infty}a_{k}(D^{\alpha}\varphi_{k})(x)\quad
  ~\text{as}\quad x\rightarrow x_{\infty}
\end{equation*}
where we same assume that $D^{\alpha}\varphi_{k}$ vanish for $0\le
k\le k_{\alpha}$ and for $x$ close enough to $x_{\infty}$, then we say
that the {\em asymptotic expansion} of $f$ as $x$ tends to
$x_{\infty}$ termwise differentiable.

\begin{remarks*}
  It may happen that the $a'_{k} s$ in (\ref{chap1:sec2:eq4}) depend on a
  parameter. Then we can talk of the {\em uniformity of the asymptotic
  expansion } relative to such a parameter and similarly of the
  {\em termwise differentiability} including the differentiation
  with regard to the parameter. 
\end{remarks*}

Let  us illustrate the notion of
  asymptotic expansion for $f$, with some examples.

\begin{example} \label{chap1:sec2:exp1} %%% exp1
  Take\pageoriginale $X=Y=\mathbb{R}, x_{\infty}=0,f\epsilon
  C^{\infty}(\mathbb{R})$and $\varphi_{k}(x)=(x)^{k}$ for
  $k=0,1,2,,\ldots$ Then the Maclaurin expansion for $f$ at $x=0$ given
  just an asymptotic expansion for $f$ as $x$ tends to $0$, namely
  \begin{equation*}
    f(x)\approx\sum_{k=0}^{\infty}\frac{f^{k}(0)}{k!}x^{k}\quad as \quad
    x\rightarrow 0
  \end{equation*}
  (Note that $\approx$ does not, in general, imply equality). Further,
  the asymptotic expansion is termwise differentiable, as is
  obvious. This example is quite simple but all the important features
  of an asymptotic expansion are incorporated here. We now give a
  slightly more complicated example.
\end{example}

\begin{example}\label{chap1:sec2:exp2} %%%exp 2
  Let $0\le \lambda_{0} < \lambda_{1} < \lambda_{2} <
  \ldots$ be a monotonic increasing sequence of non-negative real
  numbers with no finite accumulation point and let
  $m_{0},m_{1},m_{2},\ldots$ be a sequence of natural numbers. Take
  $y=\mathbb{R}, X=\mathbb{R}_{+}^{*}, x_{\infty}=0$ and
  \begin{equation*}
    x^{\lambda_{0}}(\log
    x)^{m_0-1},\ldots,x^{\lambda_{0}},x^{\lambda_{1}} (\log
    x)^{m_1-1},\ldots,{x^{\lambda_{1}}},\ldots
  \end{equation*}
  as $\varphi_0(x),\varphi_1(x),\ldots$ It is easy to check that
  condition (i) and (ii) are fulfilled. We can thus talk of the
  following asymptotic expansion for a function $f$ as $x$ tends to $0$
  , namely
  \begin{equation*}
    f(x)\approx\sum_{k=0}^{m_k}\quad\sum_{m=1}^{m_k}a_{k,m}x{}^
    {\lambda_{k}}(\log
    x)^{m-1}\quad as \quad x\rightarrow 0
  \end{equation*}
  If we set
  \begin{equation*}
    R_{k}(x)=f(x)-\sum_{i=0}^{k}\quad\sum_{m=1}^{m_i}a_{i,m}x{}^{\lambda_{i}}(\log
    x)^{m-1}
  \end{equation*}
  then (\ref{chap1:sec2:eq3}) can be replaced for {\em this} sequence $\{\varphi_{n}\}$
  by
  \begin{equation*}
    R_{k}(x)=\mathsf{o}(x^{\rho})\quad as \quad x\rightarrow 0
    ~\text{for every} \quad
    \rho<\lambda_{k+1}\tag*{$(3')$}\label{chap1:sec2:eq3'} 
  \end{equation*}
  and\pageoriginale for every $k\ge0$. This is quite easy to verify. In fact,
  (3) $\Rightarrow(3)'$ since
  $x^{-\rho}R_{k}(x) = x{}^{\lambda_{k+1}{-\rho}}\mathcal{O}((\log
  x)^{m_{k+1}{-1}})=\mathsf{o}(1)$ in view of $\lambda_{k+1}-\rho$ being
  strictly positive. On the other hand, $(3)'\Rightarrow (3)$ since
  $R_{k}(x)-R_{k+1}(x)= \sum\limits_{m=1}^{m_{k+1}}
  a_{k+1,m}x^{\lambda_{k+1}}(\log x )^{m-1}\mathcal{O}(x^{\lambda_{k+1}}(\log
  x) {}^{m_{k+1}-1})$ as $x\rightarrow 0$ while
  $R_{k+1}(x)=\mathsf{o}{(x^{\rho})}$ for $\lambda_{k+1}<\rho<\lambda_{k+2}$
  implying that $R_{k}(x)=\mathcal{O}(x^{\lambda_{k+1}}(\log
  x){}^{m_{k+1}-1})+\mathsf{o}(x^\rho)=\mathcal{O}(x^{\lambda _{k+1}}(\log
  x){}^{m_{k+1}-1})$.
\end{example}

\section{The Classical Case} \label{chap1:sec3}

\subsection{The Statement of a Theorem}\label{chap1:sec3:subsec1}

We are interested in explicitly defined function spaces on which the
Mellin transformation is an isomorphism. We begin with the classical
case where $G$ is the multiplication group $\mathbb{R}_{+}^{*}$ of
positive real numbers.

 Let $0\le\lambda_{0}<\lambda_{1}<\ldots$ be a strictly increasing
 sequence of non-negative real numbers with no finite limit point and
 let $m_0,m_1,m_2,\dots$ be a sequence of natural numbers. We
 introduce two function spaces $\mathcal{F}$ and $\mathcal{Z}$ as
 follows.The space $\mathcal{F}$ is defined as the set of all
 complex-valued functions $F$ on $\mathbb{R}_{+}^\times$ such that (i)
 $F\epsilon C^\infty(\mathbb{R}_{+}^\times)$, (ii) $F(x)$ behaves like a
 Schwartz function as $x$ tends to $\infty$ i.e.,
 $F^{(n)}=\mathsf{o}(x^{-\rho})$ as $x\rightarrow \infty$ for every $\rho$
 and every $n\ge 0$ and
 (iii)$F(x)\approx\sum\limits_{k=0}^{\infty}\sum\limits_{m=1}^{m_{k}}
 a_{k,m}x^{\lambda_{k}}{(\log x)}^{m-1}$ as $x\rightarrow 0$ (with
   constants $a_{k,m}$) and this asymptotic expansion is termwise
   differentiable. $\mathcal{F}$ is clearly nonempty, since the
   function $F=0$ is in $\mathcal{F}$. If $\lambda_n=n$ for every $n\ge0$,
   then $F(x)=e^{-x}$ (for $x\ge0$) is in the corresponding
   $\mathcal{F}$. The space $\mathcal{Z}$ consists of all meromorphic
   functions $Z(s)$ of the complex variable $s$ for which
\begin{enumerate}
\item $Z(s)$ has poles at most at the points
  $-\lambda_0,-\lambda_1,-\lambda_2,\dots,$
\item
  $Z(s)-\sum\limits_{m=1}^{m_{k}}\frac{b_{k,m}}{(s+\lambda_{k})^{m}}$\pageoriginale
  for suitable constants $b_{k,m}$ is holomorphic in a neighbourhood
  of the point $s=-\lambda_{k}$ and 
\item for every polynomial $P(s)$in s, the function $P(s)Z(s)$ is
  bounded in every vertical strip $B_{\sigma_{1},\sigma_{2}}=\{s=\sigma
    + ti \epsilon\mathbb{C}\vert\sigma_{1}\le\sigma\le\sigma_{2}\}$ for
  $-\infty<\sigma_{1}<\sigma_{2}<\infty$ after deleting small
  neighbourhoods of $-\lambda_{k}$ for every $k$ from
  $B_{\sigma_{1},\sigma_{2}}$. Under the correspondence M between
  $\mathcal{F}$ and $\mathcal{Z}$ in Theorem \ref{chap1:sec3:subsec1:thm1} below, Euler's
  gamma function $\Gamma(s)$ corresponds in $\mathcal{Z}$ to the
  function $F(x)=e^{-x}(x\ge 0)$ in $\mathcal{F}$ when $\lambda_{n}=n$
  for every $n\ge 0$.
\end{enumerate}
We shall now state and prove a theorem which has many good
applications 

\begin{theorem}\label{chap1:sec3:subsec1:thm1} %%% thm3.1
  We have $M:\mathcal{F}\xrightarrow{\sim}\mathcal{Z}$. More precisely,
  for any $F$ in $\mathcal{F}$,
  \begin{equation*}
    (MF)(S)=\int\limits_{0}^{\infty}F(x)x^{s}d\log x
  \end{equation*}
  defines a holomorphic function on $\mathbb{C}_{+}=\{s = \sigma+ti
  \epsilon\mathbb{C}\vert\sigma> 0\}$ and its meromorphic continuation
  belongs to $\mathbb{Z}$. Conversely, if $Z$ is in $\mathcal{Z}$, then 
  \begin{equation*}
    (M^{-1}Z)(x)=\frac{1}{2\pi i}\int\limits_{\sigma-\infty
      i}^{\sigma+\infty i} Z(s)x^{-s} ds 
  \end{equation*}
  gives rise to a function in $\mathbb{F}$ independently of $\sigma$ for
  $\sigma> 0$. Moreover 
  \begin{equation*}
    b_{k,m}=(-1)^{m-1}(m-1)!a_{k,m}\tag{5}\label{chap1:sec3:subsec1:eq5}
  \end{equation*}
  for every $k$ and $m$.
\end{theorem}

\subsection{}\label{chap1:sec3:subsec2} %%% subsec3.2

Before we commence the proof of Theorem \ref{chap1:sec3:subsec1:thm1}, let us make a comment on
its statement but, what is more important, namely, also a few remarks
to simplify later arguments for the proof of the theorem. 

For a moment, denote $M^{-1}Z$ in the statement of Theorem
\ref{chap1:sec3:subsec1:thm1} by NZ 
instead. As soon as we know that $M(\mathscr{F})\subset\mathscr{Z}$
and $N(\mathscr{Z})\subset\mathscr{F}$, then NM is identity on
$\mathscr{F}$ and $MN$ is identity on $\mathscr{Z}$.This follows from
Fourier's inversion theorem.\pageoriginale Therefore we can write $N=M^{-1}$.

Let $D=\sum\limits_{k=0}^{t}p_{k}(x)\frac{d^{k}}{dx^{k}}$ be a
differential operator with coefficients $p_{k}(x)$ which are
polynomials in $x$ having complex coefficients, we say that $D$ is
homothety-invariant if, for every $f\epsilon
C^{\infty}(\mathbb{R}_{+}^\times$ and $\lambda > 0$, we have $D(f(\lambda
x))=(Df)(\lambda x)$ for all $x$. Clearly D is such an operator, if
and only if $p_{k}(x)=c_{k}x^{k}$ for every k. The space of such
homothety-invariant differential operators is spanned over
$\mathbb{C}$ by $x^k\frac{d^{k}}{dx^{k}}(k = 0,1,2,\ldots)$.On the
other hand,we have
\begin{equation*}
  \left(x\frac{d}{dx}\right)^2{\displaystyle{\mathop{=}^{\text{defn}}}}
  \left(x\frac{d}{dx}\left(x\frac{d}{dx}\right)\right) =
  x\frac{d}{dx}+x^{2} \frac{d^{2}}{dx^{2}}   
\end{equation*}
and generally, if 
$$
\displaylines{\hfill 
  \left(x\frac{d}{dx}\right)^n=\sum\limits_{i=1}^{n} c_{n,i}x^{i}
  \frac{d^{i}}{dx^{i}},\hfill \cr 
  \text{then}\hfill 
  \left(x\frac{d}{dx}\right)^{n+1}=\sum\limits_{i=1}^{n+1}c_{n+1,i}x^{i}
  \frac{d^{i}}{dx^{i}}\hfill}
$$   
where $c_{n+1,i}=i c_{n, i}+ c_{n,i-1}$. Thus the space of the
homothety-invariant differential operators above is just the
$\mathbb{C}$-span of the differential operators $1,
x\frac{d}{dx},(x\frac{d}{dx})^2,\ldots$ 

We now remark that
$\mathscr{F}$ is stable under homothety-invariant differential
operators(with polynomial coefficients). For this, it suffices to
prove, in view of the foregoing, that $\mathscr{F}$ is invariant under
$x\frac{x}{dx}$ and in turn to verify condition (iii) for $xF'$
whenever $F\epsilon\mathscr{F}$. Now
$F(x)\approx\sum\limits_{k=0}^{\infty}\sum\limits_{i=1}^{m_{k}}$ $a_{k,j}
x^{\lambda_{k}}(\log  
  x)^{j-1}$ for $F\epsilon\mathscr{F}$ as $x\rightarrow 0$ implies that
  $F'(x)\approx\sum\limits_{k=0}^{\infty}\sum\limits_{j=1}^{m_{k}}
  a'_{k,j}x^{\lambda}$ $k^{-1}(\log x)^{j-1}$ where
    $a'_{k,j}=\lambda_{k}a_{k,j}+ja_{k,j+1}$ for $1\le j\le m_{k}-1$and
    $a'_{k,m_{k}}=\lambda_{k}a_{k,m_{k}}$. Thus, for every $F\epsilon \mathscr{F}
  ,x  F'(x)\approx\sum\limits_{k=0}^{\infty}
  \sum\limits_{j=1}^{m_{K}}a'_{k,j}x^{\lambda_{k}}(\log  
    x)^{j-1}$ as $x\rightarrow 0$ and so $xF'\epsilon\mathscr{F}$.

For\pageoriginale $\mathscr{Z}$, we have a corresponding property, namely that
$\mathscr{Z}$ is stable  under multiplication by polynomials in
$s$. This is quite obvious, since for
$z\epsilon\mathscr{Z}$, multiplication by polynomials only serves to
improve the situation with regard to conditions (1) and (2), while
there is nothing new to be verified regarding condition (3). 
 
We assert further that for $F\epsilon\mathscr{F}$, 
\begin{equation*}
  M(x^{k}F^{(k)}(x))(s)=(-1)^{k}(s+k-1)\ldots
  s(MF)(s)\tag{6}\label{chap1:sec3:subsec2:eq6} 
\end{equation*}
for every $s\epsilon\mathbb{C}_{+}$and $k\ge1$. For $k=1$, the formula
is clear since  
\begin{align*}
  M(xF'(x))(s) & =\int\limits_{0}^{\infty}xF'(x)x^sd\log x\\
  &
  =F(x)x^{s}{\displaystyle{\mathop{\beth}_{0}^{\infty}}} - s
  \int\limits_{0}^{\infty}F(x)x^{s}d \log x\\ 
  & =-s(MF)(S), ~\text{since the first term is}~ 0. 
\end{align*}

Assuming (\ref{chap1:sec3:subsec2:eq6}) with $k-1$ in place of $k(\ge 2)$, we have again, from
integration by parts,  
\begin{align*}
  M(x^{K}F^{(k)}(x))(s)&
  =F^{(k-1)}(x)x^{s+k-1}{\displaystyle{\mathop{\beth}_{0}^{\infty}}} -
  {(s+k-1)}M(x^{k-1}F^{(k-1)}(x))(s)\\ 
  & =0+(-1)^{k}(s+k-1)\ldots s(MF)(s).
\end{align*}
For the a sake of completeness, we include the following well-known
lemma on the interchange of differentiation and integration.  

\setcounter{lemma}{1}
\begin{lemma} \label{chap1:sec3:subsec2:lem2}%%% lem 3.2
  Let $X$ be a locally compact measure space with $dx$ a Borel measure
  on $X$ and let, for an interval $I=(a,b)\subset\mathbb{R},f(x,t)$ be
  a continuous function on $X\times 1$ satisfying the conditions: 
  \begin{equation*}
    f(x,t)\epsilon L^{1}(X,dx) ~\text{for every}~ t\epsilon I,
  \end{equation*}
  \begin{align*}
    & \frac{\partial}{\partial t}f(x,t) ~\text{is continuous and}\\
    & \Big|\frac{\partial}{\partial t}f(x,t) \Big|\le \varphi(x)
    ~\text{for a}~ \varphi \in L^{1} (X,dx)
    ~\text {for every}~ t\epsilon I. 
  \end{align*}
\end{lemma}

Then\pageoriginale
\begin{equation*}
  \frac{d}{dt}\int\limits_{X}f(x,t) dx =
  \int\limits_{X}\frac{\partial}{\partial t} f(x,t) dx. 
\end{equation*}

\begin{proof}
  The hypotheses imply the existence of the integrals $(\delta(t)
  {\displaystyle{\mathop{=}^{\text{def}}}})\int\limits_{X}$ $f(x,t)dx$ and
  $(\psi(t){\displaystyle{\mathop{=}^{\text{def}}}})\int\limits_{X}
  \frac{\partial}{\partial  
    t}f(x,t)dx$. On the other hand, for every $t,t'$ in I, we have
  $f(x,t')=f(x,t)+(t'-t)\frac{\partial}{\partial t}f(x,\tau)$ for some
  $\tau$ between $t$ and $\tau$. Therefore 
  \begin{equation*}
    |(\Phi(t')-\Phi(t))/(t'-t)-\Phi(t)|\le\int\limits_{X}
    |\frac{\partial}{\partial t}f(x,\tau),-\frac{\partial}{\partial t}f(x,t)|dx
  \end{equation*}
  and the lemma follows from Lebesgue's theorem.
\end{proof}

\subsection{Proof of Theorem 3.1}\label{chap1:sec3:subsec3}% subsec3.3

(i) First we show that for every $F\epsilon \mathscr{F},Z=MF$ is in
$\mathscr{Z}$.

Let $s=\sigma + ti \epsilon \mathbb{C}_{+}$ is and $0 <
\sigma_{1}\le\sigma \le \sigma_{2}< \infty$. Taking $\epsilon,n$ with
$0<\epsilon<\sigma_{1}$and $n>\sigma_{2}$,define
\begin{equation*}
  \varphi(x)=
  \begin{cases}
    x^{\sigma_{1}-\epsilon}
    {\displaystyle{\mathop{max}_{0<x\le1}}}(x^{\epsilon}|F(x)|), 0<x\le1\\
    x^{\sigma_{2}-n}{\displaystyle{\mathop{max}_{x\le1}}}(x^{n}|F(x)|),
    x>1 
  \end{cases}
\end{equation*}

Then it is easy to verify that $\varphi \epsilon
L^{1}(\mathbb{R}_{+}^\times,d \log x)$ and further $\varphi$ dominates
$F(x)x^{s}$.Therefore the integral defining $Z(s)$ converges
absolutely and the integrand being holomorphic, it follows, in view of
the arbitrary nature of $\sigma_{1}$ and $\sigma_{2}$, that $Z(s)$ is
holomorphic in $\mathbb{C}_{+}$. Since for every $\sigma,\sigma_{2}$
with $\sigma\le\sigma_{2}<\infty$, $F(x)x^{s}$ is dominated by a
function in $L^{1}((1,\infty),d \log x)$,say the restriction of the
function $\varphi$ above to $(1,\infty)$  it follows similarly that
$\int\limits_{1}^{\infty}F(x)x^{s}d(\log x)$ is indeed an entire
function of $s$, Now for every $k$ and $\rho<\lambda_{k+1}$,
\begin{align*}
R_{k}& =F(x)-\sum\limits_{i-0}^{k} \sum\limits_{j=1}^{m_{i}}
a_{i,j}x^{\lambda_{i}}(\log x)^{j-1}\\
    & = \mathsf{o}(x^{\rho}) \text {by
  condition}(3)'\tag{7}\label{chap1:sec3:subsec3:eq7} 
\end{align*}

Therefore\pageoriginale in $(0,1],R_{k}(x)x^{s}$ is dominated by the function
\begin{equation*}
\psi(x){\displaystyle{\mathop{=}^{def}}} x^{\sigma_{1}+\rho}
    {\displaystyle{\mathop{max}_{0<x\le 1}}} x^{-\rho}|R_{k}(x)|
\end{equation*}
 which clearly belongs to $L^{1}((0,1],d(\log x))$ for every
   $\sigma_{1}$ with $-\lambda_{k+1}<\sigma_{1}\le\sigma$ and
   $-\sigma_{1}<\rho<\lambda_{k+1}$. Thus, for similar reasons as
   above, $\int\limits_{o}^{1}R_{k}(x)x^{s}d(\log x)$ is holomorphic
   in $s(=\sigma+ti)$ for $\sigma>-\lambda_{k+1}$. Finally, for any
   $\lambda\ge0$ and $\sigma>0$,we have
\begin{equation*}
\int\limits_{0}^{1}x^{s+\lambda}d(\log x)=\frac{1}{s+\lambda}
\end{equation*}
Differentiating the integral $j-1$ times with respect to $s$ for $j\ge
1$, we get (in view of Lemma \ref{chap1:sec3:subsec2:lem2})that
\begin{equation*}
  \int\limits_{0}^{1}x^{s+1}(\log x)^{j-1} d(\log x)
  =\frac{(-1)^{j-1}(j-1)!}{(s+\lambda)^{j}}\tag{8}\label{chap1:sec3:subsec3:eq8}
\end{equation*}

Putting together all the facts above and using
(\ref{chap1:sec3:subsec3:eq7}) and (\ref{chap1:sec3:subsec3:eq8}), we
have
\begin{align*}
  Z(s) & = \left(\int\limits_{0}^{1}+\int\limits_{1}^{\infty}F(x)x^{s} d(\log
  x)\right)\\
  &
  =\sum\limits_{i=0}^{k}\sum\limits_{j=1}^{m_{i}}
  \frac{b_{i,j}}{(s+\lambda_{i})^{j}} +\int\limits_{0}^{1}R_{k}(x)x^{s}d(\log 
  x)+\int\limits_{0}^{\infty}F(x)x^{s}d(\log
  x)\tag{9}\label{chap1:sec3:subsec3:eq9} 
\end{align*}
with $b_{ij}=(-1)^{j-1}(j-1);a_{ij}$ where the second term in
(\ref{chap1:sec3:subsec3:eq9}) 
represents a holomorphic function of $s(=\sigma+ti)$ for $\sigma>
-\lambda_{k+1}$ and the third term is an entire function of $s$. Since
$k$ is arbitrary, we have thus verified conditions
(\ref{chap1:sec1:subsec2:eq1}) and (\ref{chap1:sec1:subsec2:eq2}) for
$Z$ to belong to $\mathscr{Z}$. The existence of a function in
$L^{1}([1,\infty))$ dominating the integrand shows that the third term
  in (\ref{chap1:sec3:subsec3:eq9}) is bounded for $s$ in any vertical strip\pageoriginale
  $B_{\sigma_{1},\sigma_{2}}$; for a similar reason, the second term
is also bounded in $B_{\sigma_{1}, \sigma_{2}}$ (with $k$ chosen
  correspondingly) while the first term is bounded, if we delete
  there from neighbourhoods of the points
  $-\lambda_{0},-\lambda_{1},-\lambda_{2},\ldots$. Thus condition
  (\ref{chap1:sec2:eq3}) for $Z$ to be in $\mathscr{Z}$ is fulfilled with $P(s)\equiv
  1$. To verify condition (\ref{chap1:sec2:eq3}) for arbitrary $P(s)$ we may assume,
  without loss of generality, that
  $P(s)=\sum\limits_{i=0}^{n}(-1)^{i}a_{i}s(s+1)\ldots(s+i-1)$ with
  $a_{i}$ in $\mathbb{C}$.Then working with
  $\sum\limits_{i=0}^{n}a_{i}x^{i}F^{(i)}(x)$ in place of $F$ above,
  as is legitimate, we may, in view of the remarks in
  \S \ref{chap1:sec3:subsec2},
  conclude that $P(s)Z(s)$ is bounded in every vertical strip
  $B_{\sigma_{1}},{\sigma_{2}}$ .

 (ii) Let us now prove the converse that for every
  $Z\epsilon \mathscr{Z}, F=M^{-1}$\break  $Z\epsilon \mathscr{F}$. 

  Let, for
  $0<\sigma_{1}<\sigma_{2}$, L denote the boundary (traversed
  anticlockwise) of the rectangle in the s-plane with vertices at
  $\sigma_{2}+t_{0}i,\sigma_{1}+t_{0}i,\sigma_{1}-t_{0}i,\sigma_{2}-t_{0}i(for\quad
  t_{0}>0)$. Let $x>0$; then by Cauchy's theorem,
  $\int\limits_{L}Z(s)x^{-s}ds=0$, since the integrand is holomorphic
  in $\mathbb{C}_{+}$.But, in view of the growth condition  satisfied
  by Z in vertical strips, the integral taken over the horizontal
  sides of $L$ tends to $0$ as $t_{0}$ tends to infinity. Thus
\begin{equation*}
\int\limits_{\sigma_{1}-\infty i}^{\sigma_{1}+\infty
  i}Z(s)x^{-s}ds=\int\limits_{\sigma_{2}-\infty i}^{\sigma_{2}+\infty i}Z(s)x^{-s}ds
\end{equation*}
enabling us to conclude that, for $x>0$,
\begin{equation*}
  F(x)=\frac{1}{2\pi i}\int\limits_{\sigma-\infty i}^{\sigma+\infty
    i}Z(s)x^{-s}ds\tag{10}\label{chap1:sec3:subsec3:eq10} 
\end{equation*}
is defined independently of $\sigma(>0)$.The growth condition
(\ref{chap1:sec2:eq3}) 
satisfied by $Z$ ensures the absolute convergence of the integrals
involved above. Using Lemma \ref{chap1:sec3:subsec2:lem2}, we obtain
for $k\ge 1$, that 
\begin{equation*}
x^{k}F^{(k)}(x)=\frac{1}{2\pi i}\int\limits_{\sigma-\infty
  i}^{\sigma+\infty i}(-1)^k
s(s+1)\ldots(s+k-1)Z(s)x^{-s}ds\tag{11}\label{chap1:sec3:subsec3:eq11} 
\end{equation*}

Condition\pageoriginale (\ref{chap1:sec2:eq3}) for $Z$ again guarantees the absolute
convergence of the 
integral and we conclude that $F\epsilon
C^{\infty}(\mathbb{R}_{+}^\times)$ and further F behaves at infinity like
a Schwartz function, since $(11)$ implies that 
\begin{equation*}
  |x^{\sigma+ k} F^{(k)}(x)|\le
  \frac{1}{2\pi}\int\limits|s(s+1)\ldots(s+k-1)Z(s)|dt<\infty
\end{equation*}

We are left with proving condition (iii) for F to be
in$\mathscr{F}$. Choose $\rho$ such that
$\lambda_{k}<\rho<\lambda_{k+1}$ and let, for $t_{0}>0$, the contours
$L_{1}$ and  $L_{2}$  be defined respectively as the boundary of the
rectangle with vertices at $\sigma+t_0, i-\rho+t_{0}i,-\rho
-t_{0}i,\sigma-t_{0}i$ and the union of the line segments
$\{\sigma-ti|t_{0}\le t<\infty\}$, $\{u-t_{0}i|-\rho\le u\le \rho\}$,
$\{-\rho+ti|-t_{0}\le t\le t_{0}\}, \{u+t_{0}i|-\rho\le u\le \sigma\}$ and
$\{\sigma+ti|t_{0}\le t<\infty\}$. While the contour $L_{1}$ is traversed
in the counter-clock-wise direction, the contour $L_{2}$ is covered
clockwise from $\sigma-i\infty$. From (\ref{chap1:sec3:subsec3:eq10}), we have
\begin{equation*}
F(x)=\frac{1}{2\pi i}(\int\limits_{L_{1}}+\int\limits_{L_{2}})Z(s)x^{-s}ds.
\end{equation*}

The integration over $L_{1}$ gives, by Cauchy's theorem, the sum of
the residues of the integrand at the  poles inside $L_{1}$. Condition
(\ref{chap1:sec2:eq3}) for Z implies that the contribution to the integral over $L_{2}$
from the horizontal segments tends to $0$ as $t_{0}$ tends to
$\infty$. Making $t_{0}$ tends to $\infty$, we obtain for $F(x)$ the
expression 
\begin{equation*}
\frac{1}{2\pi i} \int\limits_{\sigma-\infty i}^{\sigma+\infty
  i}Z(s)x^{-s}ds=\frac{1}{2\pi i}\int\limits_{-\rho-\infty
  i}^{-\rho+\infty i}Z(s)x^{-s}ds+S\tag{12}\label{chap1:sec3:subsec3:eq12}
\end{equation*}
where $S$ is the sum of the residues of $Z(s)x^{-s}$ at the points
$s=-\lambda_{0},-\lambda_{1}$, $-\lambda_{2},\ldots$ The residue at
$s=-\lambda_{i}$ is just the coefficient of $(s+\lambda_{i})^{-1}$in
the power-series expansion at ${-\lambda_{i}}$ of the function
$x^{\lambda_{i}}e^{-(s+\lambda_{i})\log
  x}\sum\limits_{m=1}^{m_{i}} b_{i, m} (s+\lambda_{i})^{-m}$ and is\pageoriginale
seen to be equal to $\sum\limits_{m=1}^{m_{i}}x^{\lambda_{i}}(\log
x)^{m-1}a_{i,m}$ with $a_{i,m}$ precisely as given by
(\ref{chap1:sec3:subsec1:eq5}). In view of
this, the integral on the right hand side of (\ref{chap1:sec3:subsec3:eq12}) is just $R_{k}(x)$
introduced earlier. However, the integral can be directly majorised by
$\frac{1}{2\pi}x^{\rho}\int\limits_{-\infty}^{\infty} |Z(-\rho+
ti)|dt$; we thus see for every $k\ge 1$, that
$R_{k}(x)=\mathcal{O}(x^{\rho})$ where $\rho$ is arbitrary but subject to
the condition that $\lambda_{k}<\rho<\lambda_{k+1}$. Consequently,
$R_{k}(x)=\mathsf{o}(x^{\rho})$ for every $\rho<\lambda_{k+1}$ and for every
$k\ge 1$.

 We have finally to prove the termwise differentiability of the
 asymptotic expansion of $F$ as $x\rightarrow 0$. By
 (\ref{chap1:sec3:subsec3:eq11}) with $k=1$ and
 our remarks in \S \ref{chap1:sec3:subsec2}, $-sZ(s)$ is given in
 $\mathbb{Z}$ and 
 condition (\ref{chap1:sec1:subsec2:eq2}) is fulfilled for it with
\begin{equation*}
  b'_{k,m}=
  \begin{cases}
    \lambda_{k}b_{k,m} -b_{k, m+1} & \text{for} \quad 1\le m\le m_{k}-1\\
    \lambda_{k}b_{k,m} &\text{for}\quad m=m_{k}
  \end{cases}
\end{equation*}
in place of $b_{k,m}$. This follows from the fact that
$-sZ(s)=(\lambda_{k}-(s+\lambda_{k}))Z(s)=(\lambda_{k}-(s+\lambda_{k}))\sum\limits_{m=1}^{m_{k}}\frac{b_{k},m}{(s+\lambda_{k})}^{m}+$a
function holomorphic at $-\lambda_{K}$. Applying the arguments above to
$-sZ(s)$ in place of $Z(s)$, we obtain, as indicated the function
$xF'(x)$ in place of $F(x)$ and further
\begin{equation*}
  xF'(x)\approx
  \sum\limits_{k=0}^{\infty}\sum\limits_{m=1}^{m_{k}}
  \widetilde{a}_{k,m} x^{\lambda_{k}}(\log x)^{m-1}
\end{equation*}
where $\widetilde{a}_{k,m}=(-1)^{m-1}b'_{k,m}/(m-1)!$. But using
(\ref{chap1:sec3:subsec1:eq5}),
we see that $\widetilde{a}_{k,m}$ is the same as $a'_{k,m}$ defined in
\S \ref{chap1:sec3:subsec2}. From our remarks in \S \ref{chap1:sec3:subsec2},this  means
precisely that the asymptotic expansion of $F(x)$ as $x\rightarrow 0$
can be differentiated termwise once. Iteration of this procedure
(using (\ref{chap1:sec3:subsec2:eq6})), gives us that termwise
differentiation is valid any 
number of times and Theorem \ref{chap1:sec3:subsec1:thm1} is
completely proved. 

\section{The Case of $\mathbb{R}$-fields}\label{chap1:sec4} %% sec4

\subsection{}\label{chap1:sec4:subsec1} %%% subsec4.1

In\pageoriginale this section, we shall prove the analogues of
Theorem \ref{chap1:sec3:subsec1:thm1} for the
cases when $G$ is multiplicative group $K^{\ast}(=K~\backprime~ \{0\})$ of an
$\mathbb{R}$-field $K$, i.e, for $K=\mathbb{R}$ or $\mathbb{C}$.
 
We first fix some notation applicable to any local field $K$. For
 $a\epsilon K$, we define the modulus $|a|_{K}$ of a by 
\begin{equation*}
  |a|_{K}=
  \begin{cases}
    \text{the  rate of change of the measure in $K$ under}~ x \rightarrow
    ax\\ 
    \hspace{6cm}\text{for}~ x\epsilon K ~\text{and}~ a\neq 0\\ 
    0,\quad  \text{for} a=0.
  \end{cases}
\end{equation*}

It is well-known that $|a|_{\mathbb{R}}=|a|$ and
$|a|_{\mathbb{C}}={|a^{2}|}$ where $|\;|$ denotes the usual absolute
value in $\mathbb{R}$ or $\mathbb{C}$. 

 We had introduced in \S \ref{chap1:sec1:subsec2} the group $\Omega(K^\times)$ of
 quasicharacters of $K$ as the group (with the compact open topology) of
 continuous homomorphisms of $K^\times$ into $\mathbb{C}^\times$. Then
 $\Omega(K^\times)^{0}$, the connected component of the identity,
 consists of $\omega_{s}$ for $s(=\sigma+ti)\epsilon \mathbb{C}$
 defined by $\omega_{s}(x)=|x|_{k}^{s}$ for every $x\epsilon
 K^\times$. For $\omega \epsilon \Omega(K^*)$,the associated
 quasicharacter $|\omega(x)|$ is given by
 $|x|_{k}^{\sigma(\omega)}=\omega_{\sigma(\omega)}(x)$, $\sigma(\omega)$
 being in $\mathbb{R}$, we also set $\Omega_{+}(K^\times)=\{\omega
   \epsilon \Omega(K^*); \sigma(\omega) > 0\}$ and for an
 $\mathbb{R}$-field $K$, introduce the angular component $ac(x)$ for
   $x\epsilon K^\times$ by $ac(x)=x/|x|$. For $x \epsilon \mathbb{R}^\times$,
   we have $ac(x)= sgn(x)$, the usual sign-function on $\mathbb{R}$.

\subsection{The Case $K=\mathbb{R}$}\label{chap1:sec4:subsec2} %%%%4.2

 Let us assume, as before, that $0\le \lambda_{0}<\lambda_{1}< \ldots$
 is a given strictly increasing sequence of non-negative real numbers
 with no finite accumulation point and $\{m_{k}\}_{k\ge 0}$ a given
 sequence of natural numbers. Correspondingly we define for
 $K=\mathbb{R}(or G=\mathbb{R}^\times)$ the function-spaces $\mathscr{F}$
 and $\mathscr{Z}$ as follows. The space
 $\mathscr{F}(=\mathscr{F}(\mathbb{R}^\times))$ is defined as the set of
 all complex-valued functions $F$ such that 
\begin{enumerate}
\renewcommand{\theenumi}{\roman{enumi}}
\renewcommand{\labelenumi}{(\theenumi)}
\item $F\epsilon C^{\infty}({\mathbb{R}^\times})$,\pageoriginale

\item $F$ behaves like a  Schwartz function of $x$ as $|x|$ tends to
  infinity i.e. $F^{(n)}=\mathsf{o}(|x|^{-\rho})$ as $|x|\rightarrow
  \infty$ for every $\rho $ and every $n\ge 0$, and 

\item $F(x)\thickapprox \sum\limits_{k=0}^{\infty}
  \sum\limits_{m=1}^{m_{k}} a_{k, (sgn x)}|x|^{\lambda_{k}}(\log
  |x|)^{m-1}$ as $|x|\rightarrow 0$ and this asymptotic expansion is
  termwise differentiable, we write 
  \begin{equation*}
    a_{k,m}(u)=a_{k,m,o}+ua_{k,m,1}\tag{13}\label{chap1:sec4:subsec2:eq13}
  \end{equation*}
  for $u=+1$ or $-1$; this may be regarded as the (trivial) Fourier
  expansion of function on the group $\{1,-1\}$.
\end{enumerate}

  The group $\Omega(\mathbb{R}^\times)=\{\omega_{s}(sgn)^p;s\epsilon
  \mathbb{C}, p=0,1\}$ consists of two copies of $\mathbb{C}$ indexed
  by $p=0$ or $1$. 

The space
$\mathscr{Z}(=\mathscr{Z}(\Omega(\mathbb{R}^\times))$ is defined as the
set of all complex-valued functions $Z$ on $\Omega(\mathbb{R}^\times)$
such that
\begin{enumerate}
\renewcommand{\labelenumi}{(\theenumi)}
\item $Z(\omega_s(sgn)^p)$ is meromorphic on
  $\Omega(\mathbb{R}^\times)$ with poles at most for
  $s=-\lambda_0,-\lambda_1,-\lambda_2,\ldots$,
\item $Z(\omega_s(sgn)^p)
  -\sum\limits_{m=1}^{m_{k}}\frac{b_{k,m,p}}{(s+\lambda_{k})^m}$  is
    holomorphic for $s$ close to $-\lambda_{k}$, for every $k\ge 0$
    and 
\item for every polynomial $P\epsilon\mathbb{C}[s]$and every
  $\sigma_{1},\sigma_{2},$ the function $p(s)$ $Z(\omega_{s}(sgn)^{p})$
  is bounded for $s$ in a vertical strip $B_{\sigma_{1},\sigma_{2}}$,
  with neighbourhoods of the points $-\lambda_{0},-\lambda_{1},\ldots$
  removed therefrom.
\end{enumerate}
The following theorem is an immediate consequence of Theorem 3.1 and
the definitions given above.

\setcounter{theorem}{1}
\begin{theorem}\label{chap1:sec4:subsec2:thm2} %%% thm4.2
 We have a bijective correspondence $M_{\mathbb{R}}$(abbreviated
as $M$) between $\mathscr{F}=\mathscr{F}(\mathbb{R}^\times)$ and
$\mathscr{Z}=\mathscr{Z}(\Omega(\mathbb{R}^\times))$. More precisely, if
$F\epsilon \mathscr{Z}$, then 
\begin{equation*}
(MF)(\omega)=\int\limits_{\mathbb{R}^\times}F(x)\omega(x)d^\times x
~\text{with}~ d^\times x=\frac{dx}{2|x|_{\mathbb{R}}}
\end{equation*}
defines\pageoriginale a holomorphic function on $\Omega_{+}(\mathbb{R}^\times)$ which
has a meromorphic continuation belonging to $\mathscr{Z}$, conversely,
for $z\epsilon \mathscr{Z}$, the integral
\begin{equation*}
(M^{-1}Z)(x)=\sum\limits_{p=0,1}(\frac{1}{2\pi
    i}\int\limits_{\sigma-\infty i}^{\sigma+\infty i}
  Z(\omega_s(sgn)^P)|x|_{\mathbb{R}}^{-s} ds)(sgn x)^{-p}
\end{equation*}
defines a function $F$ in $\mathscr{F}$ independently of $\sigma$, for
$\sigma > 0$. Furthermore
\begin{equation*}
b_{k,m,p}=(-1)^{m-1}(m-1)!a_{k,m,p}
\end{equation*}
for every $k,m$ and $p$.
\end{theorem}

\begin{proof}
  By introducing suitable definitions, we shall reduce ourselves
  to the situation in \S \ref{chap1:sec3}.
  For the given $F$ and for $x\epsilon {\mathbb{R}_{+}^\times}$, let us
  define
  \begin{equation*}
    F_p(x)=\frac{1}{2}(F(x)+(-1)^{p}F(-x)),p=0,1.
    \tag{14}\label{chap1:sec4:subsec2:eq14} 
  \end{equation*}
  Then $F(ux)=F_{0}(x)+uF_1(x)$ for $u=\pm 1$ and therefore, for every
  $x\epsilon \mathbb{R}^\times$,we have
  \begin{equation*}
    F(x)=F_0(|x|)+(sgn\quad x)F_1(|x|)\tag{15}
    \label{chap1:sec4:subsec2:eq15}
  \end{equation*}
  similarly, for $Z$ defined on $\Omega(\mathbb{R}^\times)$, let us introduce
  a function $Z_p$on $\mathbb{C}$ for $p= 0,1$ by the prescription 
  \begin{equation*}
    Z_p(s)=Z(\omega_s(sgn)^p)\quad \text{for}\quad s\epsilon
    \mathbb{C} \tag{16} \label{chap1:sec4:subsec2:eq16}
  \end{equation*}
  Then, in a purely formal fashion, we have 
  \begin{equation*}
    M_{\mathbb{R}}F=Z\Leftrightarrow MF_p=Z_p ~\text{for}~ p=0 ~\text{and}~ 1.
  \end{equation*}
  In fact, let $M_{\mathbb{R}}F=Z$. Then, by the definition of $Z_p$,
  \begin{align*}
    Z_p(s) & =\int\limits_{\mathbb{R}^\times} F(x) |x|_\mathbb{R}^s(sgn\quad
    x)^p\frac{dx}{2|x|_{\mathbb{R}}}\\
    & =\int\limits_{0}^{\infty}\frac{1}{2}(F(x)+(-1)^{p}F(-x))x^{s}d(\log
    x)\\
    & =(M F_p)(s)
  \end{align*}
  for\pageoriginale $p=0,1,$ Conversely, let $MF_p=Z_p$ for $p=1,2$, Then,any $\omega$
  in $\Omega(\mathbb{R}^\times)$ being of the form $\omega_s(sgn)^p$ for
  $p=0$ or $1$ and $s$ in $\mathbb{C}$, $Z(\omega)=Z_p(s)$ for a unique
  $p=0$ or $1$. But $Z_p(s)=(MF_p)(s)$ and we have only to retrace our
  steps in the foregoing to conclude that
  $Z(\omega)=(M_{\mathbb{R}}F)(\Omega)$.
\end{proof}
Finally, looking at the relations
(\ref{chap1:sec4:subsec2:eq14}) - (\ref{chap1:sec4:subsec2:eq16}) above between $F$ and
$F_0,F_1$ and between $Z$ and $Z_0,Z_1$,we conclude from the
definitions of the various function spaces that 
\begin{align*}
  F\epsilon\mathscr{F}(\mathbb{R}^\times) &  \Rightarrow F_0,F_1\epsilon \mathscr{F}\\
  &  \Rightarrow Z_0,Z_1\epsilon \mathscr{Z}
  \qquad \text{by Theorem \ref{chap1:sec3:subsec1:thm1}}\\
  &  \Rightarrow Z \epsilon
  \mathscr{Z}(\Omega(\mathbb{R}^\times))
\end{align*}
Similarly, we have
\begin{align*}
  Z\epsilon\mathscr{Z}(\Omega(\mathbb{R}^\times)) & \Rightarrow
  Z_0,Z_1\epsilon \mathscr{Z}\\
  & \Rightarrow
  F_0,F_1\epsilon \mathscr{F} \qquad \text{by
    Theorem \ref{chap1:sec3:subsec1:thm1} }\\
  & \Rightarrow F\epsilon
  \mathscr{F}(\mathbb{R}^\times).  
\end{align*}

From the uniqueness of the coefficients in asymptotic expansions and
the definitions (\ref{chap1:sec4:subsec2:eq13}) and
(\ref{chap1:sec4:subsec2:eq14}), it follows that for $p=0,1,$ the
coefficients $a_{k,m,p}$ correspond to $F_{p}(x)$ in
$\mathscr{F}$$(for \mathbb{R}_+^\times)$ in the same way as $a_{k,m}$ to
$F\epsilon\mathscr{F}(\mathbb{R}^\times)$ for every $k,m.$ Similarly from
(\ref{chap1:sec4:subsec2:eq16}), the coefficients $b_{k,m,p}$ feature in the expansion of
$Z_p(s)$ at $s=-\lambda_k$ for $p=0,1$. The\pageoriginale last assertion of the
theorem now follows from Theorem \ref{chap1:sec3:subsec1:thm1} together with the fact that,
for $p=0,1,$ the function $F_p$ and $Z_p$ correspond to each other
(under the correspondence $M$ in \S \ref{chap1:sec3:subsec1}.) 

\subsection{The Case $K=\mathbb{C}$}\label{chap1:sec4:subsec3} %%% subsec4.3

 We shall now prove an analogue of Theorem
 \ref{chap1:sec3:subsec1:thm1} for the complex case,  
Let, as before, $\le \lambda_0\le\lambda_1<\lambda_2<\ldots$ be
a strictly increasing sequence of non-negative real numbers with no
finite accumulation point and $\{m_k\}_{k\ge0}$ a sequence of natural
numbers. Let $\mathscr{F}$ be the associated space of complex-valued
functions $F$ on $\mathbb{C}^\times$ such that (i) $F\epsilon
\mathbb{C}^\infty(\mathbb{C}^*)$ (ii) $F(x)$ behaves like a Schwartz function as
$|x|_{\mathbb{C}}$ tends to infinity, namely, $\frac{\partial^{2+b}
  F}{\partial x^a \partial
  \overline{x}^b}(x)=\mathsf{o}(|x|_{\mathbb{C}}^{-\rho})$ as 
$|x|_{\mathbb{C}}\rightarrow \infty$, for every $\rho$ and for every
$a,b, \ge 0$ and (iii) $F(x)\thickapprox \sum\limits_{k=0}^\infty
\sum\limits_{m=1}^{m_k}a_{k,m}(ac(x)).|x|_{\mathbb{C}}^{\lambda_{k}}(\log
|x|_{\mathbb{C}})^{(m-1)}$ as $|x|\rightarrow 0$ with $a_{km}\epsilon
C^\infty(\mathbb{C}_{1}^\times)$ for every $k,m$, is an asymptotic
expansion which is term\-wise differentiable and uniformly in $ac(x)$
even after termwise differentiation (in the sense that for every $k\ge
0$ and for every $\rho<\lambda_{k+1}$ and for given $\epsilon>
0$, there exists $\delta$ independent of $ac(x)$ such that for
$|x|_{\mathbb{C}}<\delta$, we have
\begin{equation*}
|R_k(x)|=
\left|\sum\limits_{i=0}^{k}\sum\limits_{m=1}^{m_{i}}a_{i,m}(ac(x))|x|_{\mathbb{C}}^{\lambda_i}(\log
|x|_{\mathbb{C}})^{m-1}\right|\le\epsilon|x|_{\mathbb{C}}^{\rho} 
\end{equation*}
and further, similar inequalities hold even after the termwise
differentiation of the asymptotic expansion). Since  for every $k,m$ we
have $a_{k,m}\epsilon C^\infty(\mathbb{C}_{1}^\times)$, we have the
Fourier expansion $a_{k,m}(u)= \sum\limits_{p \epsilon
  \mathbb{Z}}a_{k,m,p}u^{p}$. 

Putting $|x|_{\mathbb{C}}=r$ and
$u=ac(x)= \mathfrak{e} (\theta)$ for $x\epsilon \mathbb{C}^*$, we have
$x=r^{\frac{1}{2}}$.u 
and
\begin{equation*}
  d^\times x{\displaystyle{\mathop{=}^{\text{def}}}} \frac{dx}{2\pi
    |x|_{\mathbb{C}}}= d(\log r)d\theta.
\end{equation*}
It\pageoriginale is easy to check that
\begin{equation*}
r\frac{\partial}{\partial r}=\frac{1}{2}(x\frac{\partial}{\partial
  x}+\overline{x}\frac{\partial}{\partial \overline{x}}) \text{and}
\frac{\partial}{\partial\theta}=2\pi i(x\frac{\partial}{\partial
  x}-\overline{x}\frac{\partial}{\partial
  \overline{x}})\tag{17}\label{chap1:sec4:subsec3:eq17} 
\end{equation*}

Suppose $D$ is a differential operator of the form $\sum\limits_{0\le
  a,b\le n}p_{a,b}(x,\overline{x})$ $\frac{\partial^{a+b}}{\partial x^a\partial
  \overline{x}^b}$ with polynomials $P_{a,b}$ in $x,\overline{x}$ having
coefficients in $\mathbb{C}$. We call $D$ homothety-invariant, if for
every $F\epsilon C^\infty(\mathbb{C}^\times)$ and $t\epsilon \mathbb{C}^\times$,
we have $(DF)$ $(tx)=D(F(tx))$ identically in $x$. The space of such
homothety-invariant differential operators is easily seen to be
spanned over $\mathbb{C}$ by differential operators of the form
$(x\frac{\partial}{\partial x})^a(\overline{x}\frac{\partial}{\partial
  \overline{x}})^b$ for $a,b\ge 0$ in $\mathbb{Z}$ and hence, in view
of (\ref{chap1:sec4:subsec3:eq17}), the differential operators
$D_{a,b}=\frac{1}{(2\pi 
  i)^b}r^a\frac{\partial^{a+b}}{\partial r^{a}\partial\theta^{b}}$ for
$a,b,\ge 0$ in $\mathbb{Z}$ also generate the above space over
$\mathbb{C}$.

 For $F\epsilon C^\infty(\mathbb{C}^\times)$ and for fixed
 $r=|x|_{\mathbb{c}}$, we have the (absolutely convergent) Fourier
 expansion 
\begin{equation*}
  F(x)=\sum\limits_{p\epsilon\mathbb{Z}}F_{p}(r)u^p.
  \tag{18}\label{chap1:sec4:subsec3:eq18} 
\end{equation*}
 We shall characterise $F$ being in $\mathscr{F}(\mathbb{C}^\times)$ in
 terms of an alternative set of conditions involving Fourier
 coefficients $F_{p}$ in (\ref{chap1:sec4:subsec3:eq18}) as follows:  
\begin{enumerate}
\renewcommand{\theenumi}{\Roman{enumi}}
\renewcommand{\labelenumi}{(\theenumi)}
\item $F\epsilon C^\infty(\mathbb{C}^\times)\Leftrightarrow F_{p}\epsilon
C^\infty(\mathbb{R}_+^\times)$ for every $p\epsilon \mathbb{Z}$ and
further, for every $a,b,\ge0$ in $\mathbb{Z}$ $r_1,r_2$ with
$0<r_{1}<r_{2}<\infty$, we have
\begin{equation*}
{\displaystyle{\mathop{\sup}_{p\in \mathbb{Z},r_{1}\leq r\leq
      r_{2}}}}|p^{b}F^{(a)}_{p}(r)|<\infty\tag{19}\label{chap1:sec4:subsec3:eq19}
\end{equation*}
Moreover, when these equivalent conditions hold, the Fourier expansion
(\ref{chap1:sec4:subsec3:eq18}) is termwise differentiable,
i.e. 
\begin{multline*}
r^{-a}D_{a,b}F(x)=\frac{1}{(2\pi i)^b}\frac{\partial^{a+b}}{\partial
  r^{a}\partial \theta^{b}}F(x)=\sum\limits_{p\epsilon
  \mathbb{Z}}p^{b} F_{p}^{(a)}(r)u^p\\ 
\text{for every} a,b\ge 0
~\text{in}~ \mathbb{Z}\tag{20}\label{chap1:sec4:subsec3:eq20} 
\end{multline*}

\item $F\epsilon C^{\infty}(\mathbb{C}^\times)$\pageoriginale and $F(x)$ behaves like a
  Schwartz function as $r\rightarrow \infty \Leftrightarrow
  \{F_p\}_{p\epsilon \mathbb{Z}}$ is as in I above and further, for
  every $a,b, \ge 0$ and every $\sigma$, we have
\begin{equation*}
  {\displaystyle{\mathop{\sup}_{p\epsilon \mathbb{Z},r\ge
        1}}}|r^{\sigma}p^bF^{(a)}_{p}(r)|<\infty \tag{21}
  \label{chap1:sec4:subsec3:eq21}
\end{equation*}
\item $F\epsilon C^{\infty}(\mathbb{C}^\times)$ and the asymptotic expansion
  $F(x)\thickapprox
  \sum\limits_{k=0}^{\infty}\sum\limits_{m=1}^{\infty}a_{k,m}(u)$ $r^{\lambda_{k}}(\log 
    r)^{m-1}$ as $r \rightarrow 0$ is termwise differentiable uniformly
    for $u$ in
    $\mathbb{C}_{1}^\times\Leftrightarrow \{F_{p}\}_{p}\epsilon\mathbb{Z}$
    is as in I above; for every $p$, there exists an asymptotic
    expansion $F_{p}(x)\thickapprox \sum\limits_{k=0}^{\infty}
    \sum\limits_{m=1}^{m_{k}}a_{k,m,p}r^{\lambda_{k}} (\log r)^{m-1}$
      as $r \rightarrow 0 $ which is termwise differentiable and
      further, for every $k \ge 0,a,b\ge 0$ in $\mathbb{Z}$ and
      $\sigma > -\lambda_{k+1}$,we have 
\begin{equation*}
  {\displaystyle{\mathop{\sup}_{p\epsilon \mathbb{Z},0<r\le 
        1}}}|r^{{\sigma}+a}p^bR_{k,p}^{(a)}(r)|<\infty\tag{22} 
  \label{chap1:sec4:subsec3:eq22}  
\end{equation*}
where, for $x\epsilon \mathbb{C}^\times$,
\begin{equation*}
R_{k}(x){\displaystyle{\mathop{=}^{def}}}F(x)-\sum\limits_{i=0}^{k}\sum\limits_{m=1}^{m_{i}}
a_{i,m}(u)r^{\lambda_{i}}(\log r)^{m-1}=\sum\limits_{p\epsilon \mathbb{Z}}R_{k,p}(r)u^P
\end{equation*}
with
\begin{equation*}
R_{k,p}(r)=F_{p}(r) -\sum\limits_{i=0}^{k}\sum\limits_{m=1}^{m_{i}}a_{i,m,p}r^{\lambda_{i}}{(\log r)}^{m-1}
\end{equation*}
In (\ref{chap1:sec4:subsec3:eq21}), we may replace $r^{\sigma}p^bF^{(a)}_{p}(r)$\text{by}
$r^{\sigma}(D_{a,b}F)_{p}(r)$ and similarly, in (\ref{chap1:sec4:subsec3:eq22}),
$r^{\sigma+a}p^bR_{k,p}^{(a)}(r)$ by
$r^{\sigma}(D_{a,b}R_{k})_{p}(r)$. 

We now given the proof of assertions, I, II, III above.
\end{enumerate}

\noindent \textbf{Proof of I.} We shall use
Lemma \ref{chap1:sec3:subsec2:lem2} in the following special form: 
namely if, for every $p\epsilon \mathbb{Z}, f_p(t), f_{p}^{'}(t)$ are continuous
on $I=(a,b)\subset\mathbb{R}$ such that 
\begin{gather*}
\sum\limits_{p \epsilon \mathbb{Z}}|f_{p}(t)|<\infty,|f_{p}^{'}(t)|\le
c_{p},\sum\limits_{p \epsilon \mathbb{Z}}c_{P}<\infty \text{for every}
t\epsilon I, \text{then}\\
\frac{d}{dt}\left(\sum\limits_{p\epsilon\mathbb{Z}}f_{p}(t)\right) = \sum\limits_{p
  \epsilon \mathbb{Z}}f'_{p}(t)\tag{23}\label{chap1:sec4:subsec3:eq23} 
\end{gather*}

If\pageoriginale $F\epsilon C^{\infty}({\mathbb{C}^\times})$, then, for $r^{-a}D_{a,b}F$,
   we have corresponding to (\ref{chap1:sec4:subsec3:eq18}), the Fourier expansion 
\begin{equation*}
(r^{-a}D_{a,b}F)(x)(=\frac{1}{(2\pi i)^b}\frac{\partial^{a+b}F}{\partial
    r^a \partial\theta^b}(x))=\sum\limits_{p \epsilon \mathbb{Z}}\widetilde{F}_{p}(r)u^P\tag{24}\label{chap1:sec4:subsec3:eq24} 
\end{equation*}
converging absolutely (and hence uniform for $u$ in
$(\mathbb{C}_{1}^\times)$, for every $a,b,\ge 0$ in
$\mathbb{Z}$. Therefore, for every $p\epsilon \mathbb{Z}$ we have
\begin{align*}
\widetilde{F}_{p}(r) & =\int\limits_{0}^{1}r^{-a}D_{a,b}F(x)\mathfrak{e}(-p\theta)d\theta
\text{by definition of} \widetilde{F}_{p}\\
                    &  =p^b\int\limits_{0}^{1}\frac{\partial^{a}}{\partial
    r^a}F(x)\mathfrak{e}(-p\theta)d\theta, ~\text{on integrating by
    parts}~ b ~\text{times},\\
                    &  =p^b\frac{d^a}{dr^a}\left(\int\limits_{0}^{1}F(x)
\mathfrak{e}(-p \theta)d\theta\right), ~\text{using
  (\ref{chap1:sec4:subsec3:eq23})}\\
                    &  =p^bF_{p}^{(a)}(r)
\end{align*}

Thus, for the expression on the left hand side of
(\ref{chap1:sec4:subsec3:eq19}), we have the
bound ${\displaystyle{\mathop{\sup}_{r_{1}\le{|x|}_{\mathbb{C}}\le
      r_{2}}}}|r^{-a}D_{a,b}F(x)|$ which is certainly finite. We have
therefore proved (\ref{chap1:sec4:subsec3:eq19}) and
(\ref{chap1:sec4:subsec3:eq20}). 

 To prove the reverse implication in I, we use
 (\ref{chap1:sec4:subsec3:eq19}) with $b+2$ in
 place of $b$ and the fact that
 ${\displaystyle{\mathop{\sup}_{r_{1}\le r\le
       r_{2}}}}|F_{0}^{(a)}(r)|\le c < \infty$ for every $a,b\ge 0$ in
 $\mathbb{Z}$. Then, for a constant $c'$ which may be taken to satisfy
 $c'\ge c$, we have 
\begin{equation*}
\sum\limits_{p\epsilon \mathbb{Z}}|p^bF^{(a)}(r)|\le
c'(1+2\sum\limits_{p=1}^{\infty}p^{-2})< \infty \quad \text{for}\quad r_{1}\le
r_1\le r \le r_2.
\end{equation*}

Applying (\ref{chap1:sec4:subsec3:eq23}) a sufficient number of times, we get
$(r^{-a}D_{a,b}F)(x)=\sum\limits_{p\epsilon \mathbb{Z}}p^bF_{p}^{(a)}(r)u^p$.

\noindent \textbf{Proof of II.} It\pageoriginale is easy to check that any $F$ in
$C^\infty(\mathbb{C}^\times)$ behaves like a Schwartz function as
$|x|_{\mathbb{C}}\rightarrow \infty$ if and only if, for every
$\sigma$ and every $a,b,\ge 0$ in $\mathbb{Z}$, we have
\begin{equation*}
{\displaystyle{\mathop{\sup}_{r\ge
      1}}}\left|r^{\sigma}\frac{\partial^{a+b}F}{\partial
  r^{a}\partial\theta^b}(x)\right|<\infty 
\end{equation*}
 However, the last condition implies (\ref{chap1:sec4:subsec3:eq21}) for every $\sigma$ and
 every $a, b\ge 0$ in $\mathbb{Z}$ in view of
 (\ref{chap1:sec4:subsec3:eq20}). Conversely, in 
 can be deduced by assuming (\ref{chap1:sec4:subsec3:eq21}) with
 $(b+2)$ in place of $b$, just as above.

\noindent \textbf{Proof of III.} A given $F$ in
$C^\infty(\mathbb{C}^\times)$ has an 
asymptotic expansion as $r\rightarrow 0$ which is termwise
differentiable uniformly in $u$, if and only if, for every $a,b,k\ge
0$ in $\mathbb{Z}$ and every $\sigma>-\lambda_{k+1}$,
\begin{equation*}
{\displaystyle{\mathop{\sup}_{0<|x|_{\mathbb{C}}\le
      1}}}\left|r^{\sigma}\frac{1}{(2\pi
  i)^b}r^a\frac{\partial^{a+b}R_{k}}{\partial r^a \partial
  \theta^b}(x)\right|<\infty 
\end{equation*}

But proceeding exactly  as above (in the proof of I) the last
condition is seen to be equivalent to (\ref{chap1:sec4:subsec3:eq22})
being valid for every 
$a,b,k\ge 0$ in $\mathbb{Z}$ and every $\sigma > -\lambda_{k+1}$.

 Finally, let us remark that $\mathscr{F}(\mathbb{C}^\times)$ is stable
 under homothety-invari\-ant differential operators, as is clear from
 the foregoing. 

 We may now proceed to define the function-space
 $\mathscr{Z}(\Omega(\mathbb{C}^\times))$. First, we observe that
 $\Omega(\mathbb{C}^\times)=\{\omega_{s}(ac)^p;s\epsilon \mathbb{C},
 p\epsilon \mathbb{Z}\}$consists of a countable number of copies of
 $\mathbb{C}$ indexed by $p$ in $\mathbb{Z}$. The space
 $\mathscr{Z}(\Omega(\mathbb{C}^\times))$ is defined as the set of all
 complex-valued functions $Z$ such that
\begin{enumerate}
\item $Z$ is meromorphic on $\Omega(\mathbb{C}^\times)$ with poles at most
  when $s=-\lambda_{0}$, $-\lambda_{1},-\lambda_{2},\ldots$,
\item $Z(\omega_{s}(ac)^p)-\sum\limits_{m=1}^{m_k}
  \frac{b_{k,m,p}}{(s+\lambda_{k})^m}$ is holomorphic for $s$ close
  enough to $-\lambda_{k}$, where $b_{k,m,p}$ are constants and $k \ge
  0$ is arbitrary in $\mathbb{Z}$; and
\item for\pageoriginale any $p\epsilon \mathbb{Z}$ and $s$ belonging to any vertical
  strip $B_{\sigma_{1},\sigma_{2}}$ with neighbourhoods of the points
  $-\lambda_{0},-\lambda_{1},-\lambda_{2},\ldots$. removed therefrom
  and for every polynomial $P(s,p)$ in $s$ and $p$ with coefficients
  in $\mathbb{C}$ we have 
\begin{equation*}
|P(s,p)Z(\omega_{s}(ac)^p)|\le c''\tag{25}\label{chap1:sec4:subsec3:eq25} 
\end{equation*}
for a constant $c''$ depending on $P,Z,\sigma_{1},\sigma_{2}$ and the
neighbourhoods of $-\lambda_{0},-\lambda_{1},\lambda_{2},\ldots$
removed but neither on $s$ nor on $p$.

For $Z\epsilon \mathscr{Z}(\Omega(\mathbb{C}^\times))$, let us define the
functions $Z_{p}$ by $Z_{p}(s)=Z(\omega_{s}$ $(ac)^p)$ for $s\epsilon
\mathbb{C}$ and $p \epsilon \mathbb{Z}$. We may thus look upon $Z$ as
a collection of functions $Z_{p}$ for $p\in \mathbb{Z}$. It is
clear that $\mathbb{Z}(\Omega(\mathbb{C}^\times))$ is stable under
$Z_{p}\mapsto QZ_{p}$ where $Q$ is an arbitrary polynomial in $p$ and
$s$, for every $p\epsilon \mathbb{Z}$. 
 We are now in a position to state and prove 
\end{enumerate}

\begin{theorem}\label{chap1:sec4:subsec3:thm3}  %%%% thm4.3
  We have a bijective correspondence $M_{\mathbb{C}}$ between
  $\mathscr{F}(\mathbb{C}^\times)$ and
  $\mathbb{Z}(\Omega(\mathbb{C}^\times))$. more precisely, for any $F$ in
  $\mathscr{F}(\mathbb{C}^\times)$,
  \begin{equation*}
    (M_{\mathbb{C}}F)(\omega)=\int\limits_{\mathbb{C}^\times}F(x)\omega(x)d^\times
    x
    \text\quad{with}\quad d^\times
    x=\frac{dx}{2\pi|x|_{\mathbb{C}}}\tag{26}\label{chap1:sec4:subsec3:eq26} 
  \end{equation*}
  defines a holomorphic function on $\Omega_{+}(\mathbb{C}^\times)$ and its
  meromorphic continuation is in
  $\mathscr{Z}(\Omega(\mathbb{C}^\times))$. Conversely, for any
  $Z\epsilon\mathscr{Z}(\Omega(\mathbb{C}^\times))$ and $x\epsilon
  {\mathbb{C}^\times}$ 
  \begin{equation*}
    (M_{\mathbb{C}}^{-1}Z)(X)=\sum\limits_{p\epsilon
      \mathbb{Z}}\left(\frac{1}{2\pi i} \int\limits_{\sigma-\infty
      i}^{\sigma-\infty i}Z(\omega_{s}(ac)^p)|x|_{\mathbb{C}^{-s}
      ds}\right)ac(x)^{-p} 
  \end{equation*}
  defines a function in $\mathscr{F}(\mathbb{C}^\times)$ independently of
  $\sigma$ for $\sigma > 0$ moreover, for every $k\le 0,m\le 1$ and $p$
  in $\mathbb{Z}$, we have 
  \begin{equation*}
    b_{k,m,p}=(-1)^{m-1}(m-1)!a_{k,m-p}.
  \end{equation*}
\end{theorem}

\begin{proof}
As in the case of $\mathbb{R}^\times$,we do fall back on
Theorem \ref{chap1:sec3:subsec1:thm1} for 
the proof of this theorem; however, an additional complication is
caused here by the maximal compact subgroup ${\mathbb{C}_{1}^\times}$ of
$\mathbb{C}^\times$ being infinite, unlike the case of $\mathbb{R}^\times$ and\pageoriginale
forces us to seek, every time, estimates which are uniform with regard
to $p$.
\end{proof}

From I, II, III above, we know. for $F\epsilon
\mathscr{F}(\mathbb{C}^\times)$, that $F_{p}\epsilon \mathscr{F}$ (for
$\mathbb{R}_{+}^\times$) for every $ p \epsilon \mathbb{Z}$ and further,
$D_{a,b}F\epsilon \mathscr{F}(\mathbb{C}^\times)$ for every $a,b \ge 0$
in $\mathbb{Z}$. Substituting (\ref{chap1:sec4:subsec3:eq18}) into
(\ref{chap1:sec4:subsec3:eq26}). with $d^\times x=d(\log 
r)d \theta$), we have (on integrating first with respect to
$\theta$, $(M_{\mathbb{C}}F)_p=M({F_{-p}})$. On the other hand, let us
  define, for $Z_P$, the function $Z_{P}^{\sharp}$ by
\begin{equation*}
  Z_{p}^{\sharp}(s)=(-1)^{a+b}(s+a-1)\ldots s.p^{b}Z_{p}(s)\tag{27}
  \label{chap1:sec4:subsec3:eq27} 
\end{equation*}
for $a,b\ge 0$ in $\mathbb{Z}$ and let $D=D_{a,b}$. Again,
substituting now (\ref{chap1:sec4:subsec3:eq24}) into
(\ref{chap1:sec4:subsec3:eq26}) with $DF\epsilon
\mathscr{F}(\mathbb{C}^\times)$ in place of $F$ we have similarly,
\begin{align*}
  ((M_{\mathbb{C}}(DF))_{P})(s)& =(M(r^{a}\widetilde{F}_{-p}))(s)\\
  & =(M((-p)^{b}r^{a}F_{p}^{(a)}(r)))(s)\qquad
  \text{by (\ref{chap1:sec4:subsec3:eq20})}\\ 
  &  =(-p)^{b}(-1)^{a}(s+a-1)\ldots s(M(F_{p}))(s)\qquad \text{by
    (\ref{chap1:sec3:subsec2:eq6})}\\
  & =(-1)^{a+b}p^{b}(s+a-1)\ldots s(M(F_{-p}))(s).
\end{align*}

Thus, if $Z=M_{\mathbb{C}} F$, then
\begin{equation*}
  Z_{p}^{\sharp}=(M_{\mathbb{C}}(DF))_{p}.\tag{28}\label{chap1:sec4:subsec3:eq28} 
\end{equation*}
Let us assume now that $F\epsilon \mathscr{F}(\mathbb{C}^\times)$ and show
that $M_{\mathbb{C}}F$ gives a function in
$\mathscr{Z}(\Omega(\mathbb{C}^\times))$. From above, $F_{p}\epsilon
\mathscr{F}$(for $\mathbb{R}_{+}^\times)$ for every $p\epsilon
\mathbb{Z}$ and therefore $(M_{\mathbb{C}}F)_{p}=M(F_{p-})\epsilon
\mathscr{Z}$ for $\mathbb{R}_{+}^\times$ by
Theorem \ref{chap1:sec3:subsec1:thm1}. On the other
hand, for $\sigma >-\lambda_{k+1}$,we have from
(\ref{chap1:sec3:subsec3:eq9}),
\begin{multline*}
  (MF_{p})(s)=\sum\limits_{i=0}^{k}\sum\limits_{m=1}^{m_{i}}
  \frac{(-1)^{m-1}(m-1)!a_{i,m,p}}{(s+\lambda_{i})^m}\\ 
  + \int\limits_{0}^{1}R_{k,p}(r)r^{s}d(\log r) + \int\limits_{1}^{\infty}
  F_{P}(r)r^{s}d(\log r).
\end{multline*}

Now
 ${a_{i,m}}(u)=\sum\limits_{p\epsilon\mathbb{Z}}a_{i,m,p}u^{p}$ for
$u\epsilon {\mathbb{C}}_{1}^\times$ and therefore,
$|a_{i,m,p}|\le||a_{i,m}||_{\infty}$ (the supremum norm), for every
$p$. Thus

\begin{multline*}
  \left|\sum\limits_{i=0}^{k}\sum\limits_{m=1}^{m_{i}}
  \frac{(-1)^{m-1}(m-1)!{a_{a,m,p}}}{{(s+\lambda_{i})}^{m}|} \right|\\
  \le \sum\limits_{i=0}^{k} \sum\limits_{m=1}^{m_{i}}(m-1)!| |a_{i,m}|
  |_{\infty}\epsilon^{-m}  ~\text{for}~ s+\lambda_{i}\ge\epsilon>0. 
\end{multline*}
Further,\pageoriginale
by (\ref{chap1:sec4:subsec3:eq22}), $|R_{k,p}(r)r^{s}|\le c_{1}$,
$r^{{\sigma}_{1}{-\sigma_{0}}}$ with 
$c_{1}={\displaystyle{\mathop{\sup}_{0 < r \le 1,p \epsilon
      \mathbb{Z}}}}|r^{\sigma_0}R_{k,p}(r)|$ 
for $-\lambda_{k+1}<\sigma_{o}<\sigma_{1}\le \sigma $ and we have
consequently
\begin{equation*}
\left|\int\limits_{0}^{1}R_{k,p}(r)r^sd(\log r)\right|\le \frac{c_1}{\sigma_{1}-\sigma_{0}}
\end{equation*}

Again, using (\ref{chap1:sec4:subsec3:eq21}), we have 
\begin{align*}
\left|F_{p}(r)r^s\right| & \le {\displaystyle{\mathop{\sup}_{r\ge1,p\epsilon\mathbb{Z}}}}\left|r^{n}.F_{p}(r)\right|.r^{\sigma_{2}-n}\\
               & = c_{2}.r^{\sigma_{2}^{-n}}, \text{say}
\end{align*}
leading to 
\begin{equation*}
  \left|\int\limits_{1}^{\infty}f_{p}(r)r^{s}d(\log r)\right|\le
\frac{c_{2}}{n-\sigma_{2}}, ~\text{for}~ \sigma\le\sigma_{2}<n.
\end{equation*}
putting these together,we conclude that for every $p$ in
$\mathbb{Z},((M_{\mathbb{C}}F)_{p}))(s)$ is bounded for $s$ in vertical
strips $B_{\sigma_{1},\sigma_{2}}$ with neighbourhoods of
$-\lambda_{0},-\lambda_{1}$, $-\lambda_{2},\ldots$ removed therefrom and
for $-\lambda_{k+1}<\sigma_1<\sigma<\infty$ and all $k\ge 0$. To
prove the corresponding assertion for
$P(s,p)((M_{\mathbb{C}}F)_{p})(s)$ with arbitrary polynomials
$P(s,p)$, we may assume, without loss of generality, that
$P(s,p)=(-1)^{a+b}p^{b}(s+a-b)\ldots s$for some $a,b\ge 0$ in
$\mathbb{Z}$. Now we can work with $D_{a,b}F$ instead of $F$ and the
required assertion follows in view of
(\ref{chap1:sec4:subsec3:eq28}). The verification for 
$M_{\mathbb{C}}F$ to belong to $\mathbb{Z}$ is nothing but checking
the conditions 1), 2), 3)  for $(M_{\mathbb{C}}F)_{p}$ which are all
satisfied as seen above.

 Conversely, let us assume that $Z$ is in $\mathscr{Z}$. Then, for
 every $p$ in $\mathbb{Z}$, we have $Z_{p}\epsilon \mathscr{Z}(\text{for}~
 \mathbb{R}_{+}^\times)$ and by Theorem \ref{chap1:sec3:subsec1:thm1},
 $F_{p}{\displaystyle\mathop{=}^{\text{def}}}M^{-1}(\mathscr{Z}_{p})$
 is in $F(\text{for}~ \mathbb{R}_+^*$. Therefore,
 for $\sigma > 0$, 
\begin{equation*}
  2\pi r^{\sigma}F_{p}(r)=\int\limits_{-\infty}^{\infty}Z_{-p}(\sigma
  + ti)r^{-ti}dt. 
\end{equation*}

On\pageoriginale the other hand, there exists a constant $c_{3}>0$ depending on
$\sigma$ such that, for every $t\epsilon \mathbb{R}$ and $p\epsilon
\mathbb{Z}$,
\begin{equation*}
\text{max}(1,|t|^{2})|Z_{p}(\sigma+ti)|\le c_{3},
\end{equation*}
and hence we have 
\begin{equation*}
2\pi
r^{\sigma}|F_{p}(r)|\le\int\limits_{-\infty}^{\infty}|Z_{p}(\sigma+ti)|dt
\le 4c_{3}.
\end{equation*}

Replacing $Z$ by $Z^{\sharp}$ for which
$(Z^{\sharp})_{p}=Z_{p}^{\sharp}$ given by
(\ref{chap1:sec4:subsec3:eq27}) from every $p 
\epsilon \mathbb{Z}$, we obtain for $\sigma>0$ and every
$D_{a,b}$that 
\begin{equation*}
  {\displaystyle{\mathop{\sup}_{p\epsilon \mathbb{Z},r>0}}}
  |r^\sigma(D_{a,b}F)_{p}(r)|<\infty.
\end{equation*}

But if we choose $\sigma$ with $-\lambda_{k+1}<\sigma <-\lambda_{k}$
then we have, by our arguments concerning (\ref{chap1:sec3:subsec3:eq12}) in
\S \ref{chap1:sec3:subsec3},that
\begin{equation*}
2\pi r^{\sigma}R_{k,p}(r)=\int\limits_{-\infty}^{\infty}Z_{-p}(\sigma+ti)dt.
\end{equation*}

On the other hand, there exists a constant $c_{4}$ depending on
$\sigma$ such that 
\begin{equation*}
max(1,|t|^{2})|Z_{-p}(\sigma+ti)|\le c_{4}
\end{equation*}
for every $t\epsilon\mathbb{R}$ and every $p\epsilon\mathbb{Z}$.This
gives first
\begin{equation*}
2\pi r^{\sigma}|R_{k,p}(r)|\le4c_{4},
\end{equation*}
and, as before, on replacing  $Z$ by $Z^{\sharp}$, we get
\begin{equation*}
  {\displaystyle{\mathop{\sup}_{p\epsilon\mathbb{Z},r\ge
        0}}}|r^{\sigma}(D_{a,b} R_{k})_{p}(r)|<\infty 
\end{equation*}
for every $D_{a,b}$ with $a,b\ge 0$ in $\mathbb{Z}$ and every $\sigma$
with $-\lambda_{k+1}<\sigma<-\lambda_{k}$. This implies
(\ref{chap1:sec4:subsec3:eq22}) and
putting  all these together, we conclude from I, II, III that
$M_{\mathbb{C}}^{-1}Z\epsilon \mathscr{F}(\mathbb{C}^\times)$.

 The\pageoriginale last assertion of Theorem \ref{chap1:sec4:subsec3:thm3}
 follows from the fact that 
 $(M_{\mathbb{C}}F)_{p}$ $=M{(F_{-p})}$ for every $p\epsilon \mathbb{Z}$ and
   from the corresponding assertion of Theorem
   \ref{chap1:sec3:subsec1:thm1}. 

 Our theorem is now completely proved.

\begin{Remark*}
  The above theory is essentially one-dimensional, since\break 
  $\Omega(K^\times)$ for $K=\mathbb{R}, \mathbb{C}$ (or even a $p$-field) is
  a one-dimensional complex Lie group and we have used function theory
  only in the case of one complex variable. To generalise our results to
  general compactly generated abelian groups, there seems to be another
  problem, namely, that of guessing the right kind of asymptotic
  expansions to be employed,for which perhaps one needs the powerful
  intuition of a good applied mathematician!
\end{Remark*}

\section{The Case of $p$-fields}\label{chap1:sec5} 

\subsection{}\label{chap1:sec5:subsec1}  

We deal now with a bijective correspondence between function
spaces $\mathscr{F}$ and $\mathscr{Z}$ associated, in a similar
manner, with p-fields which, as we recall, are just finite algebraic
extensions of Hensel's field $\mathscr{O}_{p}$ of p-adic numbers or
fields $\mathbb{F}((t))$ of Laurent series in one variable $t$ with
coefficients in a finite field $\mathbb{F}$.

 We first recall certain well-known facts about $p$-fields. 

 Let $K$ be a $p$-field, $R=\{x \epsilon K;|x|_{k}\le 1 \}$, the maximal
 compact subring of $K$ and $P=\{x\epsilon K;|x|_{k}< 1\}$ the
 (unique)maximal ideal of $R$. Then the residue field $R/P$ is a
 finite field $\mathbb{F}_{q}$ consisting of $q$ elements (with $q$
 equal to some power of $p$ in the former case and equal to the
 cardinality of $\mathbb{F}$ in latter case). In fact, if
 $K=\mathbb{F}((t))$, then $R=\mathbb{F}[[t]]$, the ring of
 power-series in $t$ over $\mathbb{F},P=t \mathbb{F}[[t]]$, and
 further $\mathbb{F}$ and $R/P$ are isomorphic\pageoriginale and have the same
 cardinality, say $q$, so that we can write $\mathbb{F}_{q}$ instead
 of $\mathbb{F}$ in this case. 

 We take $dx$ to be a Haar measure on $K$ which is so normalised as to
 make the measure $m(R)$ of $R$ equal to $1$. Now
 $R={\displaystyle{\mathop{\amalg}_{a}}}(P+a)$, the disjoint union of
 finitely many cosets $P+a$ of $R$ modulo $P$ and $m(P+a)=m(P)$ for
 every $a$. Therefore we have
\begin{gather*}
1=m(R)=\sum\limits_{a \mod P}m(P+a)=[R:P]m(P)\\
i.e,\qquad m(P)=q^{-1}.
\end{gather*}

More generally, for any $e\epsilon \mathbb{Z}$, we have
\begin{align*}
m(P^{e}) & =
\begin{cases}
1/[R:p^{e}]=1/([R:P][P:P^{2}]\ldots[P^{e-1}:P^{e}])& \text{for } e\ge0\\
[P^{e}:R]=[P^{e}:P^{e+1}]\ldots[P^{-1}:R] & \text{for }~ e < 0
\end{cases}\\
& = q^{-e}
\end{align*}

For $a\epsilon K^\times$, we define the order $ord(a)$ of a by
$aR=P^{ord(a)}$. Then $|a|_{K}$ introduced in \S \ref{chap1:sec4} as the rate
of change of measure in $K$ under $x\mapsto ax$, is just
$\frac{m(aR)}{m(R)}=m(P^{ord(a)})$, i.e, $|a|_{k}=q^{-ord(a)}$.

 The group $R^\times$ of units in $R$ is precisely $R\backslash
 P=\{x\epsilon K;|x|_{K}=1\}$ and $m(R^\times)=m(R)-m(P)=1-q^{-1}$.

 We define $d^\times x$ as the Haar measure on $K$, with the obvious
 normalisation, namely, the one that makes the measure of $R^\times$ to be
 1. It is now easy to see that
\begin{equation*}
  d^\times x=\frac{1}{1-q^{-1}}\frac{dx}{|x|_{K}}
\end{equation*}

From the definition of $ord(a)$ for $a\epsilon K^*$ above, it is clear
that we have an exact sequence
$$
\{1\}\to
R^{\ast}\xrightarrow{j}K^{\ast}\xrightarrow{``\text{ord}''}\mathbb{Z}\to \{0\}
$$
where\pageoriginale $j$ is the inclusion map and ``ord'' is the map $a\mapsto
ord(a)$. For any $e$ in $\mathbb{Z}$ the complete inverse image in
$K^*$ under the ``ord'' map i.e., $\{a\epsilon
K;|a|_{k}=q^{-e}\}=p^{e} \backprime p^{e+1}$ is a compact open subset of $K$ and
called the $e^{th}$ wall (around 0).

 We choose a  cross-section for the map ``ord'':$K^{\ast}\rightarrow
 \mathbb{Z}$, above by selecting an elements $\pi$ in $K^{\ast}$ with
 ord $\pi=1$, It is to be noted that $\pi$ is not unique but once
 chosen, it is fixed once for all.

 For every $x\epsilon K^{\ast}$, we define the angular component
 $ac(x)$ as $x\pi^{-ord(x)}$ in $R^{\ast}$ and again the definition is
 not intrinsic in view of the ambiguity about the choice of $\pi$.

 The map $x\mapsto (ord (x), ac(x))$ gives, from above, an isomorphism
 $K^{\ast}\backsimeq \mathbb{Z} \times \mathbb{R}^{\ast}$. Thus
 $\Omega(K^{\ast})\backsimeq \mathbb{C}^{\ast}\times
 (R^{\ast})^{\ast}$ under the isomorphism
 $\omega\Leftrightarrow(z,\mathcal{X})$ where, for $e\epsilon
 \mathbb{Z}$ and $u\epsilon R^{\ast}$, we have
 $\omega(\pi^{e}u)=z^{e}\mathcal{X}(u)$. We could have written
 instead, for any $x=\pi^{e} u\epsilon
 K^{\ast},\omega(x)=|x|_{k}^{s}\mathcal{X}(ac(x))(=q^{-es}$ $\mathcal{X}(u))$. But
 $|x|_{K}$ for $x\in K^{\ast}$ always belongs to the set
 $\{q^{e};e\in \mathbb{Z}\}$ and we have therefore to take $s$ only
 modulo $2\pi i/\log q$. As a result, the right parameter to be taken
 is actually $z=q^{-s}$ and not $s$. As stated above, $\Omega(K^{\ast})$
 consists of countably many copies of $\mathbb{C}^{\ast}$ indexed by
 $(R^{\ast})^{\ast}$, the discrete dual of $R^{\times}$ (which now
 corresponds to the indexing set $\mathbb{Z}$ in the case of
 $K=\mathbb{C}$). For any function $Z$ on $\Omega(K^{\times})$ and for
 $\omega\epsilon \Omega(K^{\times})$ with
 $\omega(\pi^{e}u)=z^{e}\mathcal{X}(u)$ as mentioned above, we write
 $Z_{\mathcal{X}}(\omega)$ for
 $Z(\omega)$. Let $\Omega_{+}(K^{\times})=\{\omega \epsilon
 \Omega(K^{\ast});\sigma(\omega)>0\}$.

 We now proceed to define the spaces $\mathscr{F}(K^{\times})$ and
 $\mathscr{Z}(\Omega(K^{\ast}))$ for $p$-fields $K$. While doing so, we
 guided by the existence of accepted analogues, for $p$-fields,\pageoriginale of
 $C^{\infty}$ functions behaving like Schwartz functions on
 $\mathbb{R}$-fields and also by the subsequent applications we have
 in mind.

\subsection{}\label{chap1:sec5:subsec2} %%% subsec5.2

Let $\Lambda=\{\lambda \epsilon \mathbb{C}; \lambda ~\text{modulo}~ 2\pi
  i/\log q; Re\lambda \ge 0\}$ be a given finite set and
  $\{m_{k};k \epsilon \wedge\}$ be a given set of natural numbers. Let
$\mathscr{F}(K^{\times})$ be the space of complex-valued functions $F$ on
$K^{\times}$ such that 
\begin{enumerate}
\renewcommand{\theenumi}{\roman{enumi}}
\renewcommand{\labelenumi}{(\theenumi)}
\item $F$ is locally constant i.e, for any $x\epsilon k^{\times}$, there
  exists a natural number $n$ depending on $x$ such that $F(x)=F(y)$
  for all $y$ with $y-x\epsilon xP^{n}$,
\item $F(x)=0$ for $|x|_{k}\gg 1$(i.e, for all sufficiently large
  $|x|_{K}$),and 
\item for $|x|_{K} \ll 1$(i.e, for all sufficiently small $|x|_{K})$,
  we have
\begin{equation*}
  F(x)=\sum\limits_{\lambda \epsilon
    \Lambda}\sum\limits_{m=1}^{m_{\lambda}}a_{\lambda,m}(ac(x))|x|_{k}^{\lambda}(\log
  |x|_{k})^{m-1}\tag{29}\label{chap1:sec5:subsec2:eq29} 
\end{equation*}
with locally constant functions $a_{\lambda,m}$ on $\mathbb{R^{\times}}$.
\end{enumerate}

Since $a_{\lambda,m}(u)$ are locally constant on $R^{\times}$, we may
write 
\begin{equation*}
  a_{\lambda,m}(u)=\sum\limits_{\mathcal{X} \epsilon
    (R^{\times})^{\ast}}a_{\lambda,m,\mathcal{X}}\mathcal{X}(u)
  \tag{30}\label{chap1:sec5:subsec2:eq30}  
\end{equation*}
as a finite Fourier series, with constant coefficients
$a_{\lambda,m,\mathcal{X}}$.

 For any $F\epsilon \mathscr{F}(K^{\times})$, we know that its
 restriction to the $e^{th}$ wall $p^{e} ~\backprime~$ $p^{e+1}$ is locally constant
 and therefore we have, for $u\epsilon R^{\times}$,the finite Fourier
 series
\begin{equation*}
F(\pi^{e}u)=\sum\limits_{\mathcal{X}\epsilon(R^{\times})^{\ast}}c_{e,\mathcal{X}}
\mathcal{X}(u)\tag{31} \label{chap1:sec5:subsec2:eq31}
\end{equation*}
with constant coefficients $c_{e, \mathcal{X}}$.The set
$\{\mathcal{X}\epsilon(R^{\times})^{\ast};c_{e,\mathcal{X}}\neq 0$ for
some $e$ in $\mathbb{Z}\}$ is finite. Indeed, for all
$e<0$in $\mathbb{Z}$ with $|e|$ large, we have $F(\pi^{e}u)=0$ for all
$u$ and therefore $c_{e,\mathcal{X}}=0$ for all such $e$. From
(\ref{chap1:sec5:subsec2:eq29}) 
and (\ref{chap1:sec5:subsec2:eq30}) we have, for all sufficiently large $e$,
\begin{equation*}
  c_{e,x}=\sum\limits_{\lambda \epsilon \wedge}
  \sum\limits_{m=1}^{M_{\lambda}} a_{\lambda,m,\mathcal{X}} ~ q^{-\lambda
  e}(-e \log q)^{m-1} \tag{32}\label{chap1:sec5:subsec2:eq32}
\end{equation*}
and\pageoriginale the number of $a_{\lambda,m}$ involved and hence from
(\ref{chap1:sec5:subsec2:eq30}), the
number of $\mathcal{X}$ with $a_{\lambda,m, \mathcal{X}}\neq 0$ is
finite. For the remaining finitely many $e$ again, the number of
$\mathcal{X}$with $c_{e,\mathcal{X}}\neq 0$ is finite, since
(\ref{chap1:sec5:subsec2:eq31}) is a finite sum. 
 The space $\mathscr{Z}(\Omega(K^{\times}))$ is defined as the set of
 complex-valued functions $Z$ on $\Omega(K^{\times})$ for which 
\begin{enumerate}
\renewcommand{\labelenumi}{(\theenumi)}
\item for every $\mathcal{X} \epsilon(R^{\times})^{\ast}$, there exist
  constants $b_{\lambda, m, \mathcal{X}}$ such that
  $Z_{\mathfrak{x}}(z)-\sum\limits_{\lambda \epsilon \wedge}
  \sum\limits_{m=1}^{m_{\lambda}}\frac{b_{\lambda,m,
      \mathcal{X}}}{(1-q^{-\lambda_{z}})^m}$ is a polynomial in $z$
  and $z^{-1}$with complex co-efficients, and  
\item for almost all $\mathcal{X},Z_{\mathcal{X}}(z)$ is identically zero.\\
      Let, for every $m\ge 1$, the numbers $e_{m,1},\ldots, e_{m,n}$
      be defined by the following identity in $t$: 
\end{enumerate}

\begin{equation*}
  t^{m-1}=\sum\limits_{j=1}^{m}e_{m,j} \frac{(t+j-1)(t+j-2) \ldots
    (t+1)}{(j-1)!}\tag{33}\label{chap1:sec5:subsec2:eq33} 
\end{equation*}

Actually $e_{m,j}$ are integers and further, for $m=1,2\ldots$
\begin{equation*}
  e_{m,m}=(m-1)!;e_{i,j}=0\quad\text{for}\quad
  i<j,\tag{34}\label{chap1:sec5:subsec2:eq34} 
\end{equation*}

\subsection{}\label{chap1:sec5:subsec3} %%% subsec5.3

We may now state and prove
\setcounter{theorem}{2}
\begin{theorem}\label{chap1:sec5:subsec3:thm3} %%% thm5.3
  We have an isomorphism $M_{k}$(or briefly, $M$) between the spaces
  $\mathscr{F}(K^\times)$ and $\mathscr{Z}(\Omega(K^\times))$ for any
  $p$-field $K$. More precisely, for any $F\epsilon
  \mathscr{F}(K^\times )$.
  \begin{equation*}
    (M_{K}F)(\omega)=\int\limits_{K^\times }F(x)\omega(x)d^\times x
  \end{equation*}
  defines a function on $\Omega_{+}(K^\times )$ and the meromorphic
  continuation is in $\mathscr{Z}(\Omega(K^\times ))$. Conversely, if $Z$
  is in $\mathscr{Z}(\Omega(K^\times ))$, then
  \begin{equation*}
    (M_{K}^{-1}Z)(x)~\text{is given by}~ \sum\limits_{\mathcal{X}
      \epsilon(R^\times )^{\ast}}
    {\displaystyle{\mathop{(\text{Residue}}}}_{z=0}
    (Z_{\mathcal{X}}(z)z^{-ord(x)-1}))
    \mathcal{X}(ac(x))^{-1}\tag{35}\label{chap1:sec5:subsec3:eq35} 
  \end{equation*}
  and\pageoriginale defines a function in $\mathscr{F}(K^\times )$. Moreover, we have
  \begin{equation*}
    b_{\lambda,m \mathcal{X}} = \sum\limits_{j=m}^{m_{\lambda}}
    e_{j,m}(-\log q)^{j-1}
    a_{\lambda,j,\mathcal{X}^{-1}}\tag{36} \label{chap1:sec5:subsec3:eq36} 
  \end{equation*}
  for every $\lambda,m $ and $\mathcal{X}$, where $e_{j,m}$ are given
  by (\ref{chap1:sec5:subsec2:eq33}).
\end{theorem}

\begin{proof}
  First, let $F$ be given in $\mathscr{F}(K^\times )$ and
  $\omega\epsilon\Omega_{+}(K^\times )$, so that
  $\omega(\pi^{e}u)=z^{e}\mathcal{X}(u)$ and
  $|z|=q^{-\sigma(\omega)}<1$. Now
  \begin{align*}
    (M_{K}F)_{\mathcal{X}}(z) & =\int\limits_{K^\times}F(x)\omega(x)d^\times x\\
    & =\sum\limits_{e \epsilon\mathbb{Z}}\int\limits_{R^\times
    }F(\pi^{e}u)z^{e}\mathcal{X}(u)d^\times
    u\tag{37} \label{chap1:sec5:subsec3:eq37} \\ 
    & =\sum\limits_{e \epsilon \mathbb{Z}}c_{e,\mathcal{X}^{-1}}
    z^{e}\tag{38}\label{chap1:sec5:subsec3:eq38} 
  \end{align*}
  on substituting (\ref{chap1:sec5:subsec2:eq31}) and using the
  orthogonality relations 
  $\int\limits_{R^\times }\mathcal{X}'(u)\mathcal{X}(u)$ $d^\times u=1$ or
  $0$ according as $\mathcal{X}'=\mathcal{X}^{-1}$ or not. All the steps
  used above are justified, recalling that $|z|<1$. 
\end{proof}

As remarked earlier, the number of $\mathcal{X}$ in
$(R^\times )^* $ with the property that $c_{e,\mathcal{X}}\neq 0$
for some $e$ (depending on $\mathcal{X}$) in $\mathbb{Z}$
finite. Hence, expect for finitely many $\mathcal{X}$, we have
$(M_{K}F)_{\mathcal{X}}=0$. In view of (\ref{chap1:sec5:subsec3:eq37})
(\ref{chap1:sec5:subsec3:eq38}), (\ref{chap1:sec5:subsec2:eq31}) and
(\ref{chap1:sec5:subsec2:eq32}), we
have
\begin{equation*}
  (M_{k}F)_{\mathcal{X}}(z)=\sum\limits_{e}' c_{e,\mathcal{X}^{-1}}
  z^{e} + \sum\limits_{e=e_{0}}^{\infty} 
    \left\{\sum\limits_{\lambda \epsilon
      \wedge}\sum\limits_{m=1}^{m_{\lambda}}a_{\lambda,m,\mathcal{X}^{-1}}q^{-\lambda
      e}(-e \log q)^{m-1}\right\} z^{e}
  \end{equation*}
  where$\sum\limits_{e}'$ is a finite summation and $e_{0}\ge 0$ in
  $\mathbb{Z}$ depends on $F$. Therefore, writing $\equiv $ to imply
  equality modulo $\mathbb{C}[z,z^{-1}]$, we have
  \begin{align*}
    (M_{k}F)_{\mathcal{X}}(z) \equiv&  \sum\limits_{e=0}^{\infty}
    \left(\sum\limits_{\lambda  \epsilon
      \wedge}^{\infty}\sum\limits_{m=1}^{m_{\lambda}}
    a_{\lambda,m,\mathcal{X}^{-1}} q^{-\lambda
      e}(-e \log q)^{m-1}\right)z^{e}\\
     = \sum\limits_{e=0}^{\infty} & \left(\sum\limits_{\lambda \epsilon
      \wedge}\sum\limits_{m=1}^{m_{\lambda}}a_{\lambda,m,\mathcal{X}^{-1}}q^{-\lambda
      e}(-\log q)^{m-1}\sum\limits_{j=1}^{m} e_{m,j}
  \begin{pmatrix}
    e+j-1\\
    j-1
  \end{pmatrix}
   \right)z^{e},\\
   & \tag*{using \ref{chap1:sec5:subsec2:eq33},}\\ 
    = \sum\limits_{\lambda \epsilon \wedge} & \sum\limits_{j=1}^{m_{\lambda}}
    \left(\sum\limits_{m=j}^{m_{\lambda}} e_{m,j}(-\log
    q)^{m-1}a_{\lambda,m,\mathcal{X}^{-1}} \right) \sum\limits_{e=0}^{\infty}
  \begin{pmatrix}
    e+j-1\\
    j-1
  \end{pmatrix}
   (q^{-\lambda}z)^{e}     
  \end{align*}
  on\pageoriginale reversing the order of the summations over $j$ and $m$ in the
  preceding line and then interchanging them with the summation over
  $e$. We remark that this is justified $|z|<1$ and the summations over
  $m,j $and $\lambda$ are al finite, using
  (\ref{chap1:sec5:subsec3:eq36}) to define 
  $b_{\lambda,m \mathcal{X}}$ and the fact that
  $|q^{\lambda}z|\le|z|<1$,we have, finally,
  \begin{equation*}
    (M_{K}F)_{\mathcal{X}}(z)\equiv \sum\limits_{\lambda\epsilon
      \wedge}\sum\limits_{j=1}^{m_{\lambda}}\frac{b_{\lambda,j,~
        \chi}}{(1-q^{-\lambda}z)^{j}}  
  \end{equation*}
  and therefore $M_{k}F$ is indeed in $\mathscr{Z}(\Omega(K^\times ))$.

 Conversely, let $Z$ be given in $\mathscr{Z}(\Omega(K^\times ))$, so
 that, for every $\mathcal{X}\epsilon(R^\times)^{\ast}$, we have
 $Z_{\mathcal{X}}(z)=\sum\limits_{e \epsilon
   \mathbb{Z}}d_{\chi,e} z^{e}$. where $0<|z|<1$ and further,
 in view of  conditions (\ref{chap1:sec1:subsec2:eq1}) and
 (\ref{chap1:sec1:subsec2:eq2})  satisfied by $Z$, we have the
 following:
\begin{align*}
& \left\{\mathcal{X}\epsilon(R^\times)^*;d_{\mathcal{X},e}\neq 0~\text{for some}~
 e ~\text{in}~ \mathbb{Z}\right\} \text{is a finite set} ,
 d_{\mathcal{X},e}= 0 ~\text{for}~ e<0 
 ~\text{in}\\ 
 & \mathbb{Z} ~\text{with}~ |e| ~\text{large and}
  d_{\mathcal{X},e}=\sum\limits_{\lambda \epsilon
   \wedge}\sum\limits_{m=1}^{m_{\lambda}}b_{\lambda,m,\mathcal{X}}
  \begin{pmatrix}
    e+m-1\\
    m-1
  \end{pmatrix}
 q^{-\lambda e}\\
& \text{for all large}\quad e>0~\text{in}~
 \mathbb{Z}\tag{39}\label{chap1:sec5:subsec3:eq39} 
\end{align*}

From (\ref{chap1:sec5:subsec2:eq31}) and
(\ref{chap1:sec5:subsec3:eq38}), by reversing the process, we get, for 
$e\epsilon \mathbb{Z}$ and $u\epsilon R^\times $ that
\begin{align*}
  (M_{K}^{-1}Z)(\pi^{e}u)&
  =\sum\limits_{\mathcal{X}}d_{\mathcal{X},e}\mathcal{X}(u)^{-1}\\  
  &  =\sum\limits_{\mathcal{X}\epsilon(R^\times )^{\ast}}
  (\text{Residue}_{z=0}(Z_{\mathcal{X}}(z)z^{-e-1}))\mathcal{X}(u)^{-1}
  \tag{40}\label{chap1:sec5:subsec3:eq40} 
\end{align*}

 It\pageoriginale is now clear that ${M_{K}^{-1}}Z$ satisfies conditions (i) and (ii)
 defining $\mathscr{F}(K^\times)$. Also, for any given $\mathcal{X}$
 and $\lambda$, equations (\ref{chap1:sec5:subsec3:eq36}) for $1\le
 m\le m_{\lambda}$ considered 
 as a set of equations in the $m_{\lambda}$ unknowns $a_{\lambda, j,
   \mathcal{X}^{-1}}(1\le j\le m_{\lambda})$ are solvable, in view of
 (\ref{chap1:sec5:subsec2:eq34}). From
 (\ref{chap1:sec5:subsec3:eq40}) and (\ref{chap1:sec5:subsec3:eq39}),
 we have, for all large $e>0$in $\mathbb{Z}$.  
\begin{align*}
  (M_{K}^{-1}Z)(\pi^{e}u)& =\sum\limits_{\mathcal{X}}
  \left\{\sum\limits_{\lambda \epsilon
    \wedge}\sum\limits_{m=1}^{m_{\lambda}} b_{\lambda,m,\mathcal{X}}
  \binom{e+m-1}{m-1} q^{-\lambda,e}\right\}\mathcal{X}(u)^{-1}\\
   & =\sum\limits_{\mathcal{X}} \left[ \sum\limits_{\lambda\epsilon
    \wedge}\sum\limits_{m=1}^{m_{\lambda}} \left\{
    \sum\limits_{j=m}^{m_{\lambda}} e_{j,m}(-\log q)^{j-1}a_{\lambda,
    j,\mathcal{X}^{-1}}\right\} \right.\\
   & \hspace{4cm} \left. {\binom{e+m-1}{m-1}} q^{-\lambda
      e}\right]\mathcal{X}(u)^{-1}, 
\end{align*}
using the solvability of equations (\ref{chap1:sec5:subsec3:eq36})
with $1\le m\le m_{\lambda}$ 
for $a_{\lambda,j,\mathcal{X}^{-1}}$. Again, reversing the order of
the summations over $j$ and $m$ and then using
(\ref{chap1:sec5:subsec2:eq30}) and (\ref{chap1:sec5:subsec2:eq33}), we 
have, for all large $e>0$ in $\mathbb{Z}$,  
\begin{align*}
  (M_{K}^{-1}Z)(\pi^{e}u)& =\sum_{\lambda \epsilon \wedge}
  \sum_{j=1}^{m_{\lambda}} a_{\lambda, j}(u)^{-\lambda e}
  \left\{\sum_{m=1}^{j}e_{j,m} \binom{e+m-1}{m-1}\right\} (-\log q)^{j-1}\\
  & =\sum_{\lambda \epsilon \wedge}\sum_{m=1}^{m_{\lambda}}
  a_{\lambda,m} (ac(x))|x|_{k}^{\lambda}(\log |x|_{k})^{m-1}, 
  \text{ with }  x=\pi^{e}u. 
\end{align*}

Thus, condition (\ref{chap1:sec5:subsec2:eq29}) is proved for $M_{k}^{-1} Z$ with $|x|_{K}\ll 1$,
implying that $M_{K}^{-1}Z$ is in $\mathscr{F}(K^\times )$.  

\begin{remark}\label{chap1:sec5:subsec3:rem1} %%% rem 1
  In (\ref{chap1:sec5:subsec2:eq29}), we have assumed (while defining
  $\mathscr{F}(K^\times )$) 
  that the functions $a_{\lambda,m}$ on $R^\times $ are locally
  constant. However, we shall now show that this is actually a
  consequence of $F$ being locally constant on $K^\times $. For this,
  is suffices to prove that the functions 
  \begin{equation*}
    |x|_{k}^{\lambda}(\log |x|_{K})^{m-1} \text{ for } \lambda
    \epsilon \wedge ~\text{and}~ 1\le m\le m_{\lambda} 
  \end{equation*}
  on $\{|x|_{k}=q^{-e}$; $e$ sufficiently large in $\mathbb{Z}\}$ are
  linearly independent over $\mathbb{C}$, If possible, let
  $a_{\lambda, m, 1}$ be complex constants such that 
  \begin{equation*}
    \sum_{\lambda \epsilon \wedge}\sum_{m=1}^{m_{\lambda}}
    a_{\lambda,m,1} |x|_{k}^{\lambda}(\log|x|_{k})^{m-1}=0\tag{41} 
    \label{chap1:sec5:subsec3:eq41}
  \end{equation*}
  for\pageoriginale all $|x|_{k}\ll 1$. Then, proceeding exactly as in the proof of
  the first half of Theorem \ref{chap1:sec5:subsec3:thm3}, we obtain
  $b_{\lambda,m, 1}$ related 
  to $a_{\lambda, m, 1}$ as in (\ref{chap1:sec5:subsec3:eq36}) with $\mathcal{X}=1$ such that 
  \begin{equation*}
    \sum_{\lambda \epsilon
      \wedge}\sum_{m=1}^{m_{\lambda}}\frac{b_{\lambda,m,1}}{(1-q^{-\lambda}z)^m}\equiv
    0 \pmod{\mathbb{C}[z,z^{-1}]}\tag{42} \label{chap1:sec5:subsec3:eq42}
\end{equation*}
Since, on the left side, $(1-q^{-\lambda}z)^{-m}$ has a pole
$q^{\lambda}$ and the points $q^{\lambda}$for $\lambda \epsilon
\wedge$ are all distinct, (\ref{chap1:sec5:subsec3:eq42}) necessarily
implies that $b_{\lambda, 
  m, 1}$ are all 0. Therefore, in view of (\ref{chap1:sec5:subsec3:eq36}) with
$\mathcal{X}=1$, the constants $a_{\lambda, m,1}$ in
(\ref{chap1:sec5:subsec3:eq41})  are all 0.
\end{remark}

\begin{remark}\label{chap1:sec5:subsec3:rem2} %%% rem2
  Formula (\ref{chap1:sec5:subsec3:eq35}) for the inverse transform
  $M_{K}^{-1}$ looks, on the 
  face of it, quite different from its counterpart in the case of
  $\mathbb{R}$-fields. But the analogy will be obvious, if we rewrite
  it, for $\sigma > 0$, as  
  \begin{equation*}
    (M_{K}^{-1}Z)(x)=\sum_{\mathcal{X}\epsilon(R^\times )^{\ast}}
    (\frac{\log q}{2 \pi i} \int\limits_{\sigma-\frac{\pi i}{\log
        q}}^{\sigma+\frac{\pi i}{\log q}}
    Z_{\mathcal{X}}(q^{-s})|x|_{k}^{-s}ds) \mathcal{X}(ac(x))^{-1}
  \end{equation*}
  where now $\pi=3.14\ldots$ (the length of the circumference of a
  circle of unit diameter!). In fact, this is immediate from
  (\ref{chap1:sec5:subsec3:eq35}),
  since $Z_{\mathcal{X}}(z)^{-e-1}$ is holomorphic in $0<|z|<1$ and
  for $\sigma >0$, we have 
  \begin{align*}
    \text{ Residue }_{z=0}(Z_{\mathcal{X}}(z)z^{-e-1})& =\frac{1}{2\pi
      i} \oint\limits_{|z|=q^{-\sigma}}Z_{\mathcal{X}}(z)z^{-e}d(\log z)\\
    &  =\frac{\log q}{2 \pi i}\int\limits_{\sigma-\frac{\pi i}{\log
        q}}^{\sigma+\frac{\pi i}{\log q}}Z_{\mathcal{X}}(q^{-s})|x|_{k}^{-s}ds
  \end{align*}
  (on setting $z=q^{-s}$).
\end{remark}
\newpage

\begin{center}
  \textbf{APPENDIX}\\[10pt]
  \textbf{POISSON FORMULA OF HECKE TYPE}
\end{center}

This\pageoriginale appendix is based on a preprint entitled ``On a generalization of
the Fourier transformation'' by T. Yamazaki \cite{Yam}, where the special
case corresponding to $\mathfrak{x}=1- \frac{1}{m}$ for
$m=2,4,6,\ldots$ below  has been discussed. The Poisson formula that
we shall derive here goes back, in its simplest form, to
Hecke \cite{Hec 1}. 
Our generalisation, besides being of interest on its own, may be
expected to provide, together with a suitable enlargement of this
theory so as to cover all the local fields, the `metaplectic group' to
be associated with diagonal forms of $m\ge2$. In this connection, we
should also refer to a series of papers by $T$. Kubota where he has
generalised the Fourier transformation and proved a poisson formula
for his transformation(See \cite{Kub}).

\setcounter{section}{0}
\section{The Unitary Operator $W$} %%% appsec1

 We recall Theorem \ref{chap1:sec3:subsec1} on the bijective correspondence between the
 spaces $\mathscr{F}$ and $\mathscr{Z}$ associated with
 $\mathbb{R}_{+}^\times$.

 Choosing $\lambda_{k}=k,m=1$ for $k=0,1,2\ldots$, let us remark the
 space for $\mathbb{R}_{+}^\times $ can be characterised also as the
 space of $F\epsilon C^{\infty}(\mathbb{R}_{+}^\times )$ such that
 $F(x)$ behaves like a Schwartz function as $x\rightarrow \infty$ and
 further $F \epsilon C^{\infty}([0,\infty))$ by defining
   $F(0)=\lim\limits_{x \rightarrow 0}F(x)$. The last condition is
   seen to be equivalent to $F$ in $C^{\infty}(\mathbb{R}_{+}^\times )$
   having a termwise differentiable asymptotic expansion
   $F(x)\approx \sum_{n=0}^{\infty}a_{n}x^{n}$ as $x \rightarrow
   0$; one has merely to define $F(0)=a_{o}$, in order to show that
   conditions (iii) for $F$ in $\mathscr{F}$ implies
   ``$FC^{\infty}([0,\infty))$''. The spaces $\mathscr{F}$ can
     also be described as the space of complex-valued functions
     obtained by restricting to $R_{+}^\times $, Schwartz functions on
     $\mathbb{R}$ (i.e, $C^{\infty}$ functions $G(x)$ on $\mathbb{R}$
     which behave like Schwartz\pageoriginale functions as $|x|\rightarrow
     \infty$); this can be seen by using, for example, a theorem of
     $H$. Whitney \cite{Whi} that $C^{\infty}$ functions on closed subsets
     of $\mathbb{R}^{n}$ admit $C^{\infty}$ extensions to the whole of
     $\mathbb{R}^{n}$.
 
 The space $\mathscr{Z}$ for $\mathbb{R}_{+}^\times $ can be described
 alternatively as the family of complex-valued functions $Z(s)$ on
 $\mathbb{C}$ such that $Z(s)(\Gamma(s))^{-1}$ is an entire function of $s$
 and further, for every polynomial $P\epsilon \mathbb{C}[s]$, the
 function $P(s)Z(s)$ is bounded in any vertical strip
 $B_{\sigma_{1}},\sigma_{2}$ with neighbourhoods of the points
 $0,-1,-2,\ldots$ removed therefrom. This is clear, since
 $1/\Gamma(s)$ is an entire function  of $s$ with  simple zeros at
 $0,-1,-2,\ldots$

  For any $a>0$, the function $F(x)=e^{-ax}$ is clearly in
  $\mathscr{F}$ and further, 
\begin{equation*}
  M(e^{-ax})(s)=\int\limits_{0}^{\infty}e^{-ax}x^{s}d(\log
  x)=a^{-s}\Gamma(s)\tag{43}\label{chap1:sec5:subsec3:eq43} 
\end{equation*}
almost by the definition of $\Gamma(s)$. More generally, for any
$\tau$ in $\mathbb{C}$ with $Im(\tau)>0$, we get, by the principle of
analytic continuation, from  (\ref{chap1:sec5:subsec3:eq43}), that 
\begin{equation*}
  M(e^{i \tau x}(s)=(-i \tau)^{-s} \Gamma (s)
\end{equation*}
where $(-i\tau)^{-s}=e^{-s Log(-i \tau)}$ with $Log$ denoting the
principal branch of the logarithm. 
 We also recall the following asymptotic formula for the
 gamma-function: namely, 
\begin{equation*}
  |\mathsf{r}(s)|=(2\pi)^{\frac{1}{2}} |t|^{\sigma-\frac{1}{2}} \exp
  \left(-\frac{\pi}{2}|t|\right)
  (1+\mathsf{o}(1))\tag{44}\label{chap1:sec5:subsec3:eq44}  
\end{equation*}
for $s=\sigma+ti$ in any vertical strip $B_{\sigma_{1}}, \sigma_{2}$,
as $|s|\rightarrow \infty$. 

 Let us fix, once for all, a real number $\mathfrak{x}>0$. Then, from
 (\ref{chap1:sec5:subsec3:eq44}), it is immediate that
\begin{equation*}
  \left|\frac{\Gamma(s)}{\gamma(\mathfrak{x}-s)}\right| =
  |t|^{2\sigma-\mathfrak{x}}
  (1+\mathsf{o}(1))\tag{45}\label{chap1:sec5:subsec3:eq45} 
\end{equation*}
for\pageoriginale $s=\sigma+ti$  in a vertical strip $B_{\sigma_{1},\sigma_{2}}$.

 For any $Z$ in $\mathscr{F}$, we define $Z^\times $ by
\begin{equation*}
  Z^\times (s)=\Gamma(s)\frac{Z(\mathfrak{x}-s)}{\Gamma(\mathfrak{x}-s)}
\end{equation*}

Then $Z(s)/\Gamma(s)=z(\mathfrak{x}-s)/\Gamma(\mathfrak{x}-s)$ therefore,
an entire function. For any polynomial $P$, we have $P(s)Z(s)$ bounded
in vertical strips with neighbourhoods of $0,-1,-2,\ldots$ removed
therefrom and (\ref{chap1:sec5:subsec3:eq45}) then implies the same
property for $Z^\times $ 
instead of $Z$. Thus $Z^\times $ is again in $\mathscr{Z}$ and
moreover it is  trivial to check that $(Z^\times)^\times)=z$.

For any $Z$ in $\mathscr{Z}$, if we define 
\begin{equation*}
  ||Z||^{2}=\frac{1}{2\pi}\int\limits_{-\infty}^{\infty}
  \left|z(\frac{1}{2}\mathfrak{x}+ti)\right|^{2}dt
\end{equation*}
then the growth condition on $Z$ implies $||Z||<\infty$ and it is
clear that $||\quad||$ is a norm in $\mathscr{Z}$. Since the left hand
side of (\ref{chap1:sec5:subsec3:eq45}) is $1$ for
$s=\frac{1}{2}\mathfrak{x}+ti$, it is easily 
seen that $||Z^\times ||=||Z||$ for every $Z$ in $\mathscr{Z}$. We can
define, similarly, for $F$ in $\mathscr{F}$. 
\begin{equation*}
  ||F||^{2}=\int\limits_{0}^{\infty}|F(x)|^{2}x^{\mathfrak{x}}d \log x.
\end{equation*}

The integral is finite since $\mathfrak{x}>0$ and $F$ behaves like a
Schwartz function at infinity. We have again a norm in $\mathscr{F}$
given by $||\quad ||$ but, as we shall see, it is the same as one
obtained by transporting to $\mathscr{F}$ the  norm from $\mathscr{Z}$
under $M^{-1}$:$\mathscr{Z}\xrightarrow\backsim\mathscr{F}$.

We assert that $||F||^{2}=||MF||^{2}$. In fact, if $Z=MF$, then 
\begin{equation*}
  Z \left(\frac{1}{2}\mathfrak{x}+ti\right) =
  \int\limits_{0}^{\infty}F(x)x^{\frac{\mathfrak{x}}{2}},x^{ti}d\log x
\end{equation*}
is\pageoriginale just the Fourier transform of $F(x)x^{\mathfrak{X}/2}$. Since
$\frac{1}{2\pi}dt$ and $d \log x$ are dual measures, the Plancherel
theorem gives 
\begin{equation*}
  (||MF||^{2}=)||Z||^{2}=||F||^{2}. 
\end{equation*}

 We now define an operator $W:\mathscr{F}\rightarrow\mathscr{F}$ as
 the composite of the operators $M,Z\mapsto Z^\times $ and $M^{-1}$;
 more explicitly, we have, for any $F$ in $\mathscr{F}$, 
\begin{equation*}
  WF=M^{-1}((MF)^\times )\tag{46}\label{chap1:sec5:subsec3:eq46}
\end{equation*}

From $(Z^\times)^{\times}$, it is immediate that $W^{2}F=F$. Further
\begin{equation*}
  ||WF||=||M^{-1}((MF)^\times )||=||(MF)^\times ||=||MF||=||F||.
\end{equation*}

Thus $W$ is a unitary operator of order $2$ on the pre-Hilbert space
 $\mathscr{F}$. 
   We shall make use of the following formula, later on: namely, for
   any $F\epsilon \mathscr{F}$, 
   \begin{equation*}
     (WF)(0)=\frac{(MF)(\mathfrak{X})}{\Gamma(\mathfrak{X})}
     \tag{47}\label{chap1:sec5:subsec3:eq47}  
\end{equation*}

  This is easy to prove; in fact, it $(WF)(x)\approx
  a'_{0}+a'_{1}x+\ldots$ as $x\rightarrow 0$, then, by
  Theorem \ref{chap1:sec3:subsec1},
  $(M(WF))(S)-\frac{a'_{0}}{s}$ is holomorphic at $s=0$ which implies
  that $\lim\limits_{s\rightarrow
    0}\frac{(M((WF))(S)}{\Gamma(s)}=a'_{o}=WF(0)$. But
  $\frac{(M(WF))(s)}{\Gamma(s)}=\frac{(MF)(\mathfrak{X}-s)}{\Gamma
    (\mathfrak{X}-s)}$  
  and taking limits on both sides as $s\rightarrow 0$, we see that
  (\ref{chap1:sec5:subsec3:eq47}) is immediate. 

\begin{remarks*}
  One can give another proof of (\ref{chap1:sec5:subsec3:eq47}) by
  using the relation  
  \begin{equation*}
    (WF)(x)=\int\limits_{0}^{\infty}k(xy)F(y)y^{\mathfrak{X}} d\log y
  \end{equation*}
  where
  $k(x)=\sum\limits_{n=0}^{\infty}\frac{(-x^{n})^{n}}{n!\Gamma
    (n+\mathfrak{X})}
  =x^{-\frac{1}{2}(\mathfrak{X}-1)}J_{\mathfrak{X}-1}(2x^{\frac{1}{2}})$  
  and $J_{\mathfrak{X}-1}$ is the usual Bessel function of order
  $\mathfrak{X}-1$. 

 It should be stressed that $\mathfrak{X}$ is involved in the
 definition of $W$. 
\end{remarks*}

\section{A Poisson Formula} %%% appsec2

 Consider\pageoriginale a triple $\{\lambda, \mathfrak{X},\gamma\}$, where $\lambda$
 and $\mathfrak{X}$ are positive real numbers and $\gamma$ is a complex
 number. Following Hecke, we say that a complex-valued function
 $\varphi$ on $\mathbb{C}$ is a function of signature $\{\lambda,
 \mathfrak{X}, \gamma\}$ if 
\begin{enumerate}
\renewcommand{\labelenumi}{\theenumi)}
\item $(s-\mathfrak{X})\varphi(s)$ is an entire function of $s$ and
  further $\varphi$ is (at most) of polynomial growth in any vertical
  strip $B_{\sigma_{1},\sigma_{2}}$ for every
  $\sigma_{1}<\sigma_{2}$.
\item $R(s)=(\frac{\lambda}{2\pi})^{s}\Gamma(s) \varphi(s)$
  satisfies the function equation $R(s)=\gamma R(\mathfrak{X}-s)$, and
\item for all sufficiently large $\sigma,\varphi(s)$, is represented
  by an absolutely convergent Dirichlet series
  $\sum\limits_{n=1}^{\infty}\frac{a_{n}}{n^{s}}$. 
\end{enumerate}
 Our condition $(1)$ above is a variant of Hecke's assumption 
 \cite{Hec 1}):$1)'(s-\mathfrak{X})\varphi(s)$ is an entire function genus,
 Conditions 1), 2), 3) above are together equivalent to conditions
 $1)'$, 2), 3). With regard to 3),  we should mention that there
 exist Dirichlet series e.g, $\sum(-1)^{n}/(\sqrt{n}(\log n)^{s})$
 convergent for all values of $s$, but never absolutely convergent. If
 there exists a function $\varphi \neq 0$, of signature $\{\lambda,
 \mathfrak{X}, \gamma\}$, then condition (2) above implies
 $\gamma^{2}=1$, as is obvious on applying $s\mapsto\mathfrak{X}-s$ to
 the functional equation.

\begin{theorem*}
  Suppose that $\varphi \neq 0$, is a functional of signature $\{\lambda,
  \mathfrak{X}, \gamma\}$ and further, let
  \begin{equation*}
    a_{0}=\gamma\left(\frac{\lambda}{2\pi}\right)^{\mathfrak{X}}
    \Gamma(\mathfrak{X})\text{Residue}_{s=\mathfrak{X}}\varphi(s).
  \end{equation*}

  Then we have, for every $F$, in $\mathscr{F}$, the Poisson formula
  \begin{equation*}
    \sum\limits_{n=0}^{\infty}a_{n}(WF)\left(\frac{2\pi
      n}{\lambda}\right)=\gamma\sum\limits_{n=0}^{\infty}a_{n}F\left(\frac{2\pi
      n}{\lambda}\right) 
  \end{equation*}
\end{theorem*}

\begin{proof}
  From (\ref{chap1:sec5:subsec3:eq46}) and the functional equation for $R(s)$, we get 
  \begin{equation*}
    \frac{(M(WF))(s)}{\Gamma(s)}R(s)=\frac{(MF) (\mathfrak{X}-s)}
         {\Gamma(\mathfrak{X}-s)}\gamma R{(\mathfrak{X}-s)}
  \end{equation*}
  i.e.\pageoriginale
  \begin{equation*}
    (M(WF))(s)\left(\frac{\lambda}{2\pi}\right)^{s}\varphi(s)
    =\gamma(MF)(\mathfrak{X}-s)\left(\frac{\lambda}{2\pi}
    \right)^{\mathfrak{X}-s}\varphi(\mathfrak{X}-s)
    \tag{48}\label{chap1:sec5:subsec3:eq48} 
  \end{equation*}
\end{proof}

  Since $\varphi$ has got at most a simple pole at
  $s(=\sigma+ti)=\mathfrak{X}$ and holomorphic everywhere else, we see,
  in view of Theorem \ref{chap1:sec3:subsec1} applied to $WF$ (with $\lambda_{k}=k,
  m_{k}=1$ for all $k\ge 0$) that the left side of
  (\ref{chap1:sec5:subsec3:eq48}) is holomorphic
if $s\neq \mathfrak{X},0, -1, -2, \ldots$ For similar reasons, the
right side of (\ref{chap1:sec5:subsec3:eq48}) is holomorphic for $\mathfrak{X}-s\neq, 0, -1, -2,
\ldots$ i.e, for $s\neq 0, \mathfrak{X},
\mathfrak{X}+1,\mathfrak{X}+2,\ldots$. Thus the function of $s$
represented by (\ref{chap1:sec5:subsec3:eq48}) is holomorphic except for $s=0,=\mathfrak{X}$ at
most. Moreover, from condition (1) satisfied by $\varphi $ and an
application of Theorem \ref{chap1:sec3:subsec1} to $F$ and $WF$ again, we see that, even
after multiplication by a polynomial $P(s)$, both sides of
(\ref{chap1:sec5:subsec3:eq48}) 
remain bounded in any given vertical strip $B_{\sigma_{1},\sigma_{2}}$
with neighbourhoods of the poles above removed therefrom. Thus, we
have, in particular, for any $\sigma $ with $0<\sigma<\mathfrak{X}$,
that 
\begin{multline*}
  \frac{1}{2\pi i}\int\limits_{\sigma-\infty i}^{\sigma-\infty
    i}(M(WF))(s)\left(\frac{\lambda}{2\pi
  }\right)^{s}\varphi(s)ds=\frac{\gamma}{2 \pi i}\\
  \int\limits_{\sigma-\infty i}^{\sigma+\infty i}(MF)
  (\mathfrak{X}-s)\left(\frac{\lambda}{2\pi}\right)^{\mathfrak{X}-s}
  \varphi(\mathfrak{X}-s)ds,\tag{49}\label{chap1:sec5:subsec3:eq49}
\end{multline*}
 both integrals converging absolutely. On the left side of
 (\ref{chap1:sec5:subsec3:eq49}), we 
 can shift the line integration from $Re(s)=\sigma$ to
 $Re(s)=\sigma_{0}$ for any sufficiently large $\sigma_{0}>0$,
 provided that we take into account the residue of the integrand at
 $s=\mathfrak{X}$. The nice growth conditions satisfied by the
 integrands (in vertical strips) make, as before, the integrals on
 horizontal segments of fixed length tend to zero, as the segments
 recede to infinity on either side of the real axis. In this manner,
 we obtain that the left hand side of (\ref{chap1:sec5:subsec3:eq49})
 is equal to 
\begin{multline*}
  \frac{1}{2\pi i}\int\limits_{\sigma_{0}-\infty i}^{\sigma_{0}+\infty
  i}(M(WF))(s)\sum\limits_{n=1}^{\infty}a_{n}\left(\frac{\lambda}{2\pi
    n} \right)^{s}ds-(M(WF))(\mathfrak{X})\\
  \left(\frac{\lambda}{2\pi}
  \right)^{\mathfrak{X}}\text{Residue}_{s=\mathfrak{X}}\varphi(s)
\end{multline*}
 which\pageoriginale is the same as
 $\sum\limits_{n=1}^{\infty}a_{n}(WF)\left(\frac{2\pi
   n}{\lambda}\right)-\gamma a_{o}F(0)$. We have used here, the
 relation (\ref{chap1:sec5:subsec3:eq47}) with $WF$ in place of  $F$, the definition of $a_{o}$
 and the fact that $\gamma=\pm 1$; further, we have interchanged the
 integration on the line $Res=\sigma_{o}$ and the summation over $n$
 as justifiable without difficulty and also appealed to Theorem
 \ref{chap1:sec3:subsec1}. By applying entirely similar arguments to
 the right hand side 
 of (\ref{chap1:sec5:subsec3:eq49}), we see that it is equal to
 $\gamma\left(\sum\limits_{n=1}^{\infty}a_{n}F\left(\frac{2\pi
   n}{\lambda}\right)-\gamma a_{o}(WF)(0)\right)$ and our theorem is
 completely proved.

\section{Relation with Hecke's Theory} %%% appsec3

 As we have remarked earlier, the theorem proved above, in a special
 but quite typical case, goes back to Hecke whose theory of the
 correspondence between Dirichlet series with functional equations and
 modular forms is well-known. We now make the connection of our
 theorem with Hecke's explicit. 

  For a complex number $\tau$ with $Im(\tau)>0$, let us set
  $F(x)=e^{i\tau x}$ for $x\epsilon \mathbb{R}_{+}^\times $. 
  Then we know, from, above, that
  $(MF)(s)=\Gamma(s)(-i\tau)^{-s}$ and this gives us
\begin{align*}
  (W(e^{i\tau
    x}))(x)& {\displaystyle{\mathop{=}^{defn}}}(M^{-1}(\Gamma(s)\frac{(M(e^{i\tau
      x}))(\mathfrak{X}-s)}{\Gamma(\mathfrak{X}-s)}))(x)\\
           & =(M^{-1}(\Gamma(s)(-i\tau)^{-(\mathfrak{X}-s)}))(x)\\ 
           & =(M^{-1}(\Gamma(s)(-i\tau)^{-\mathfrak{X}}
              \left(\frac{i}{\tau}\right)^{-s}))(x)\\
           & =(i\tau)^{-\mathfrak{X}}e^{-ix/\tau}.  
\end{align*}

 Thus,\pageoriginale taking $F(x)=e^{i\tau x}$ in the theorem above, we get the
 relation 
\begin{equation*}
  \sum\limits_{n=0}^{\infty}a_{n}\mathfrak{e}
  \left(-\frac{n}{\lambda\tau} \right)=\gamma(-i\tau)^{\mathfrak{X}}
  \sum\limits_{n=0}^{\infty}a_{n} \mathfrak{e}\left(\frac{n\tau}{\lambda}\right)
\end{equation*}
which becomes more transparent, on being restated as follows, Namely,
under Hecke's correspondence
\begin{equation*}
  \varphi(s)=\sum\limits_{n=1}^{\infty}\frac{a_{n}}{n^{s}}\mapsto 
  f(\tau)=\sum\limits_{n=0}^{\infty}a_{n}\mathfrak{e}
  \left(\frac{n\tau}{\lambda}\right)
\end{equation*}
with $a_{o}=\gamma\left(\frac{\lambda}{2\pi}\right)^{\mathfrak{X}}
\Gamma(\mathfrak{X})\text{Residue}_{s=\mathfrak{X}}\varphi(s)$,
the functional equation $R(s)=\gamma R(\mathfrak{X}-s)$ of
$R(s)=\left(\frac{\lambda}{2 \pi}\right)^{s}\Gamma(s)\varphi(s)$
corresponds to the transformation-law
$f\left(-\frac{1}{\tau}\right)=\gamma(-i\tau)^{\mathfrak{X}}f(\tau)$
which merely says that the function $f(\tau)$ obviously holomorphic in
$\tau$ for $\text{Im}\,(\tau)>0$ behaves like a modular form of weight
$\mathfrak{X}$ under the transformations from the subgroup of
$(PSL_{2})(\mathbb{R})$ generated by $\tau\mapsto \tau+\lambda$ and
$\tau \mapsto-1/\tau$. This was indeed  the starting point of Hecke
theory (\cite{Hec 2}).

 We\pageoriginale go back now, to the general case starting from an arbitrary $F$ in
 $\mathscr{F}$ and proceed to give an interpretation of the set-up
 above as follows. For any $\mathscr{U}$ in $\mathbb{R}$, let us
 denote by $\xi(\mathscr{U})$, the multiplication by $e^{i \mathscr{U}
   x}$ of any function in $\mathscr{F}$ and further define
 $\eta(\mathscr{U})$ on $\mathscr{U}$ on $\mathscr{F}$ by 
\begin{equation*}
  \eta(\mathscr{U})F=W\xi(-\mathscr{U})W\quad F
\end{equation*}
for any $F$. Let $M_{p}(\mathbb{R})$(respectively $M_{p}(Z)$) denote
the subgroup of the group of unitary automorphisms of $\mathscr{F}$
generated by $\xi(\mathscr{U})$ and $\eta(\mathscr{U})$ as
$\mathscr{U}$ varies over $\mathbb{R}$ (respectively over
$\mathbb{Z}.\lambda$). Finally, let us fix a function $\varphi \neq 0$
of signature $\{\lambda, \mathfrak{X}, \gamma\}$ and define for any $F$
in $\mathscr{F}$, a function $\theta_{F}$ on $M_{p}(\mathbb{R})$ by 
\begin{equation*}
\theta_{F}(g)=\sum\limits_{n=0}^{\infty}a_{n}(gF)\left(\frac{2 \pi
  n}{\lambda}\right)\quad \text{for}\quad g \epsilon M_{p}(\mathbb{R})
\end{equation*}

Then , for any $g_{0}$ in $M_{p}(\mathbb{Z})$,
\begin{equation*}
  \theta_{F}(g_{o}g)=\theta_{F}(g) ~\text{for every}~
  g\epsilon M_{p}(\mathbb{R}).
\end{equation*}
 Also, it can be proved that
 $w=\mathfrak{e}\left(\frac{\mathfrak{X}}{4}\right) W$ belongs to
 $M_{p}(\mathbb{R})$. Moreover, if $(M_{p})'(\mathbb{Z})$ denotes the
 subgroup of $M_{p}(\mathbb{R})$ generated by $\xi(u)$ for all
 $\mathscr{U}$ is $\mathbb{R}$ and $w$, then
\begin{equation*}
  \theta(g_{o}'g)=\mathcal{X}(g_{o}')\theta_{F}(g)\quad 
  (g\epsilon M_{p}(\mathbb{R}))
\end{equation*}
 for every $g_{o}'$ in $(M_{P}')(\mathbb{Z})$, where $\mathcal{X}$ is
 a character of $M_{p}(\mathbb{Z})$. All the assertions above except
 the one about $w$ being in $M_{p}(\mathbb{R})$, are consequences of
 the Poisson formula.

 The\pageoriginale group $M_{p}(\mathbb{R})$ is known as the Metaplectic group and
 our interpretation above becomes significant in view of the
 following  fact that can be proved: namely lat $\mu$ denote the
 cyclic subgroup of $\mathbb{C}_{1}^\times $ generated by
 $\mathfrak{e}(\mathfrak{X})$; then $\mu$ is contained in
 $M_{p}(\mathbb{R})$ and under the correspondence
\begin{equation*}
\xi(\mathscr{U})\mapsto \binom{1\quad \mathscr{U}}{0\quad 1},
\eta(\mathscr{U})\mapsto \binom{1\quad 0}{\mathscr{U}\quad 1}
\end{equation*}

 We have
\begin{equation*}
  M_{p}(\mathbb{R})/\mu\xrightarrow\sim PSL_{2}(\mathbb{R})
  \text{or}SL_{2}(\mathbb{R})
\end{equation*}
 according as $\mathfrak{X}$ is or is not the quotient of an integer by
 an integer. 

\begin{remark*}
  By using the same function $k(x)$ as in the remarks at the end of
  \S 1 above, we have
  \begin{equation*}
    (\eta(t)F)(x)=(it)^{-\mathfrak{X}}e^{ix/t}\int\limits_{0}^{\infty}
    k\left(\frac{xy}{t^{2}}\right)e^{iy/t}F(y)y^{\mathfrak{X}}d\log y
  \end{equation*}
  for every $t\neq 0$ in $\mathbb{R}$ and every $F$ in $\mathscr{F}$;
  this integral representation of $(\eta(t)F)(x)$ can be used to
  determine the structure of $M_{p}(\mathbb{R})$ stated above.
\end{remark*}
