\chapter{Poisson Formula of Siegel-Weil Type}\label{chap4} %%%% {chap4}

\section{Formulation of a Poisson
  Formula}\label{chap4:sec1} %%% sec{chap4-sec1}

In\pageoriginale this Chapter, we formulate a Poisson formula of general type which
corresponds to forms of higher degree in the same manner as the
Siegel-Weil formula (\cite{Wei 5}) is related to quadratic forms; the
classical Poisson summation formula is again a special case of such a
Poisson formula and it is well-known (\cite{Tat}) that in this classical
formula lies embedded a substantial part of the validity of the
functional equation of Hecke L-series. Our Poisson formula, as in the
case of Weil, is to be regarded as an identity between two tempered
distributions on adelized vector spaces arising from global fields. We
shall formulate sufficient conditions for the validity of the general
Poisson formula. Moreover, we shall use the so-called ``adelic
language'' (\cite{Wei 3}), essentially to simplify the presentation.

\subsection{Standard Notation}\label{chap4:sec1:subsec1} %%%% subsec{1.1}

Let $k$ be a {\em global field} \ie an algebraic number field of
finite degree over the rational number field or an algebraic function
field of one variable with finite constant field. For any non-trivial
absolute value $|\;|_{v}$ on $k$, denote by $k_{v}$ the completion of
$k$ with respect to the metric $d(a,b)=|a-b|_{v}$ on $k$. Let us
assume that $|\;|_{v}$ is normalized, as already described. We call
$v$ {\em archimedean} or {\em non-archimedean} according as $k_{v}$ is
an $\mathbb{R}$-field or a $p$-field; archimedean\pageoriginale
valuations arise only in the case of algebraic number fields. We
choose exactly one representative (and indeed normalised as well) in
each ``equivalence class'' of valuations $v$. By $S$, we shall mean,
in this article, a finite set of valuations $v$ which always includes
the archimedean valuations. For non-archimedean $v$, we denote the
valuation ring by $R_{v}$ or $R$ and its maximal ideal by $P_{v}$ or
simply $P$.

\subsection{Adelization}\label{chap4:sec1:subsec2}%%% subsec{1.2}

A subset $U$ of an affine space $X$ of dimension $n$ is called {\em
  locally $k$-closed} (or {\em quasi-$k$-affine}) if $U=V\backslash W$
for $k$-closed subsets $V$, $W$ in the affine space. If $V$ is defined
by the equations $f_{1}(x)=\ldots=f_{r}(x)=0$ and $W$ by
$g_{1}(x)=\ldots=g_{s}(x)=0$ for $f_{1},\ldots,f_{r}$,
$g_{1},\ldots,g_{s}$ in $k[x_{1},\ldots,x_{n}]$, then $a$ belongs to
$U$ if and only if $f_{1}(a)=\ldots=f_{r}(a)=0$ and further
$g_{j}(a)\neq 0$ for at least one $j$ with $1\leq j\leq s$. If we
define $U_{v}=U_{k_{v}}=\{x\in k^{n}_{v};f_{1}(a)=\ldots=f_{r}(a)=0$
and $g_{j}(a)\neq 0$ for at least one $j$ with $1\leq j\leq s\}$, then
$U_{v}$ is clearly locally compact. Further, for non-archimedean $v$,
we see that $U^{0}_{v}=\{a\in R^{(n)};f_{1}(a)=\ldots=f_{r}(a)=0$ and
at least one $g_{j}(a)$ is a unit of $R_{v}$ for $1\leq j\leq s\}$ is
compact. For any $S$ described in \S \ref{chap4:sec1:subsec1} above,
$\prod\limits_{v\in S}U_{v}$ is locally compact and
$\prod\limits_{v\not\in S}U^{0}_{v}$ is compact, so that
$$
U_{S}{\displaystyle{\mathop{=}^{\text{defn.}}}}\prod_{v\not\in S}U^{0}_{v}\times
\prod_{v\in S}U_{v}
$$
is locally compact. For two such $S$, $S'$ with $S\subset S'$, clearly
$U_{S}$ is open in $U_{S'}$.

For any locally $k$-closed $U$, we define the {\em adelization}
$U_{A}$ of $U$ by $U_{A}=\xrightarrow[S]{\lim}U_{S}$ and $U_{S}$ is open in
$U_{A}$ for every $S$. Then $U_{A}$ is a locally compact
space\pageoriginale and it is not hard to see that $U_{A}$ depends
only on $U$ (although the requirement that for $a\in U^{0}_{v}$,
$g_{j}(a)$ is a unit for at least one $j$, may look disturbing!). We
can show that adelization is a ``functor'' and, in particular, for any
$k$-morphism $j:U\to U'$ of locally $k$-closed subsets $U$, $U'$ in
affine spaces, we have a unique continuous map $j_{A}:U_{A}\to
U'_{A}$. The set $U_{k}$ of $k$-rational points on $U$ can be imbedded
in $U_{A}$ through the diagonal imbedding and it is, in fact, discrete
in $U_{A}$.

\subsection{Examples}\label{chap4:sec1:subsec3} %%%% subsec{1.3}

We shall illustrate the foregoing with some examples.
\begin{enumerate}
\renewcommand{\theenumi}{\roman{enumi}}
\renewcommand{\labelenumi}{(\theenumi)}
\item Let $X$ be the affine $n$-space, so that $X_{k}\simeq k^{n}$. We
  denote $X_{A}$, for $n=1$, by $k_{A}$, the ring of $k$-adeles; for
  any $n$, $X_{A}\simeq (k_{A})^{n}$ and $X_{A}/X_{k}$ is compact. If
  $n=1$, $k=\mathbb{Q}\hookrightarrow \mathbb{Q}_{A}$ and any $t$ in
  $\mathbb{Q}_{A}$ is of the form $(t_{\infty},t_{p},\ldots)$ with
  $t_{\infty}\in \mathbb{R}$ and $t_{p}$ is in the ring
  $\mathbb{Z}_{p}$ of $p$-adic integers for almost all
  (non-archimedean) $p$. If $<t_{p}>$ denotes the ``fractional part''
  of $t_{p}$, then $<t_{p}>=0$ for almost all $p$ so that
  $t'=t-\sum\limits_{p}<t_{p}>$ in $\mathbb{Q}_{A}$ has its
  non-archimedean components $t'_{p}$ in $\mathbb{Z}_{p}$ for every
  $p$. It is now immediate that $\mathbb{Q}_{A}/\mathbb{Q}\simeq
  \mathbb{R}/\mathbb{Z}\times \prod\limits_{p\text{
      prime}}\mathbb{Z}_{p}$ and hence, it is compact. The imbedding
  of $\mathbb{Q}$ as a discrete subgroup of $\mathbb{Q}_{A}$
  corresponds precisely to the imbedding $\mathbb{Z}\hookrightarrow
  \mathbb{R}$. 

\item For $f(x)\in k[x_{1},\ldots,x_{n}]$ and $i$ in $k$, we have
  introduced $U(i)$ as $f^{-1}(i)\backslash C_{f}$; thus, a belongs to
  $U(i)$ if and only if $f(a)=i$ and further, $\dfrac{\partial
    f}{\partial x_{j}}(a)\neq 0$ for at least one $j$. Clearly $U(i)$
  is a locally $k$-closed subset and hence $U(i)_{A}$ is defined.
\end{enumerate}

\subsection{Tamagawa Measure}\label{chap4:sec1:subsec4} %%%% subsec{1.4}

Suppose\pageoriginale $U$ is a non-singular locally $k$-closed subset of affine
$n$-space and $\omega$, an everywhere regular differential form of the
highest degree on $U$ vanishing nowhere and moreover, defined over
$k$. We shall be dealing only with situations where such a form
$\omega$ exists; for example if $U=X$,
$\omega(x)=dx_{1}\Lambda\ldots\Lambda dx_{n}$ in Example (i) above and
$\omega(x)=\theta_{i}(x)$ for $U=U(i)$ in Example (ii).

By a procedure (which may not, however, work always), we can associate
to $\omega$, the so-called {\em Tamagawa measure} $|\omega|_{A}$ on
$U_{A}$ as follows. Let us start from the Borel measure $|\omega|_{v}$
on $U_{v}$ associated with $\omega$; in particular, on the open set
$U^{0}_{v}$, we have a measure, say $m_{v}$. Let us assume that the
(infinite product) measure $\bigotimes\limits_{v\not\in
  S}|\omega|_{v}$ exists on $\prod\limits_{v\not\in S}U^{0}_{v}$;
this, incidentally, happens to exist if and only if the infinite
product $\prod\limits_{v\not\in S}m_{v}(U^{0}_{v})$ is absolutely
convergent. On the other hand, we always have the product measure
$\bigotimes\limits_{v\in S}|\omega|_{v}$ on the (finite) product
$\prod\limits_{v\in S}U_{v}$. Thus, we can define, for every $S$, the
measure $|\omega|_{A}$ on $U_{S}=\prod\limits_{v\not\in
  S}U^{0}_{v}\times \prod_{v\in S}U_{v}$ by
$|\omega|_{A}=(\bigotimes\limits_{v\not\in S}|\omega|_{v})\otimes
(\bigotimes\limits_{v\in S}|\omega|_{v})$. However, we have, still, to
remove the ambiguity that arises in our definition of $|\omega|_{v}$
above, due to our not having fixed a Haar measure on $k^{n}_{v}$, for
$n=1,2,\ldots$. For this purpose, let us proceed as follows. Let us
choose a non-trivial character $\psi:k_{A}/k\to
\mathbb{C}^{\times}_{1}$. Then $k_{A}$ may be identified with its dual
by setting $<a,b>=\psi(ab)$ for $a$, $b\in k_{A}$ and further, 
$k\simeq (k_{A}/k)^{\ast}$\pageoriginale via the map $c\mapsto
\psi(ct)$. Let $j_{v}:k_{v}\to k_{A}$ be the imbedding which sends any
$t$ in $k_{v}$ to the adele with $t$ as the $v$-th component and $0$
elsewhere; we set $\psi_{v}=\psi\circ j_{v}$. On $X_{v}=k^{n}_{v}$, we
take as our measure, the $n$-fold product $|dx|_{v}$ of the measure on
$k_{v}$ which is self-dual relative to $(t,t')\mapsto \psi_{v}(tt')$; 
then, this is the same as the measure on $X_{v}=k^{n}_{v}$ self-dual
relative to $((x_{1},\ldots,x_{n}),(y_{1},\ldots,y_{n}))\mapsto
\psi_{v}(x_{1}y_{1}+\cdots+x_{n}y_{n})$. Moreover, $\psi_{v}=1$ on
$R_{v}$ and nontrivial on $P^{-1}_{v}$, for all but finitely many $v$;
further, $m(X^{0}_{v})=1$ for such $v$. The measure $|dx|_{A}$ always
exist on $X_{A}$ and has the characteristic property that
$X_{A}/X_{k}$ has measure $1$; on each open subgroup $X_{S}$ defined
above, $|dx|_{A}$ is just the product of the measures $|dx|_{v}$, in
the usual sense.

\subsection{The Schwartz-Bruhat Space
  $\mathscr{S}(X_{A})$}\label{chap4:sec1:subsec5} %%%% subsec{1.5}

If $k$ is an algebraic function field of one variable with finite
constant field, then $X_{A}$ is a locally compact abelian group with
arbitrarily large and small compact open subgroups in the sense of
Chapter II, \S\ \ref{chap2:sec1:subsec2}. The Schwartz-Bruhat space $\mathscr{S}(X_{A})$
is thus already familiar to us and so is the dual, $\mathscr{S}(X_{A})'$ of
tempered distributions.

Let, on the other hand, $k$ be an algebraic number field and the
degree $[k:\mathbb{Q}]$ of $k$ over $\mathbb{Q}$ be finite. If
$S_{\infty}$ denotes the set of all archimedean valuations on $k$, we
write, for the affine space $X$ of dimension $n$,
\begin{align*}
& X_{\infty}=\prod_{v\in S_{\infty}}X_{v}\simeq
\mathbb{R}^{n[k:\mathbb{Q}]}\\
& X_{0}(=X_{\text{finite}})=\varinjlim_{S} \prod_{v\in S\backslash
  S_{\infty}}X^{0}_{v}. 
\end{align*}
Let\pageoriginale us remark that $X_{0}$ is a locally compact abelian
group with arbitrarily large and small compact open subgroups and
therefore the Schwartz-Bruhat space $\mathscr{S}(X_{0})$ is already known
to us. Moreover, we are also quite familiar with the Schwartz-Bruhat
space $\mathscr{S}(X_{\infty})$. We now define the {\em Schwartz-Bruhat
  space associated with} $X_{A}$ by
$\mathscr{S}(X_{A})=\mathscr{S}(X_{0})\otimes_{\mathbb{C}}\mathscr{S}(X_{\infty})$;
every element of this space is a finite linear combination of
$\Phi_{0}\otimes \Phi_{\infty}$ with $\Phi_{0}\in\mathscr{S}(X_{0})$ and
$\Phi_{\infty}\in \mathscr{S}(X_{\infty})$. We define the ``topological
dual'' $\mathscr{S}(X_{A})'$ of $\mathscr{S}(X_{A})$ as the space of all
$\mathbb{C}$-linear functionals on $\mathscr{S}(X_{A})$ such that
$T(\Phi_{0}\otimes \Phi_{\infty})$ depends continuously on
$\Phi_{\infty}$ in $\mathscr{S}(X_{\infty})$ for every fixed $\Phi_{0}$ in
$\mathscr{S}(X_{0})$. Let $\{T_{n}\}_{n}$ be a sequence contained in
$\mathscr{S}(X_{A})'$ such that, for every $\Phi\in \mathscr{S}(X_{A})$,
$\lim\limits_{n}T_{n}(\Phi)$ exists; denoting this limit by $T(\Phi)$,
we see trivially that $T$ is $\mathbb{C}$-linear. Moreover, for every
fixed $\Phi_{0}$ in $\mathscr{S}(X_{0})$ and any $\Phi_{\infty}\in
\mathscr{S}(X_{\infty})$, we have
$T(\Phi_{0}\otimes\Phi_{\infty})=\lim\limits_{n\to
  \infty}T_{n}(\Phi_{0}\otimes\Phi_{\infty})$, by the definition of
$T$. Now since $\Phi_{\infty}\mapsto T_{n}(\Phi_{0}\otimes
\Phi_{\infty})$ belongs to $\mathscr{S}(X_{\infty})'$ for every fixed
$\Phi_{0}\in\mathscr{S}(X_{0})$, the well-known completeness of
$\mathscr{S}(X_{\infty})'$ entails that $\Phi_{\infty}\mapsto
T(\Phi_{0}\otimes \Phi_{\infty})$ belongs to
$\mathscr{S}(X_{\infty})'$. Thus $\mathscr{S}(X_{A})'$ is complete. It is
called the {\em space of tempered distributions} on $X_{A}$, as usual.

\subsection{Poisson Formula}\label{chap4:sec1:subsec6} %%%% subsec{1.6}

We are now in a position to formulate the general Poisson formula
associated with $f(x)\in k[x_{1},\ldots,x_{n}]$ and with
$\mathscr{S}(X_{A})$. We first make the simple remark that for every
$i^{\ast}\in k$, $\psi(i^{\ast}f(x))\in\mathscr{S}(X_{A})'$, if we simply
define
$\psi(i^{\ast}f(x))(\Phi)=\int\limits_{X_{A}}\Phi(x)\psi(i^{\ast}f(x))|dx|_{A}$,
for every $\Phi$ in $\mathscr{S}(X_{A})$ and note that the integral
converges absolutely.

Let\pageoriginale us formulate a series of assumptions.
\begin{itemize}
\item[(PF-1)] The infinite sum $\sum\limits_{i^{\ast}\in k}\psi(i^{\ast}f(x))$
  belongs to $\mathscr{S}(X_{A})'$, \ie (equi\break valently) {\em the
    Eisenstein-Siegel} series $\sum\limits_{i^{\ast}\in
    k}\int\limits_{X_{A}}\Phi(x)\psi(i^{\ast}f(x))|dx|_{A}$ associated
  with $f(x)$ converges absolutely for every $\Phi$ in
  $\mathscr{S}(X_{A})$.

\item[(PF-2)$'$] For every $i$ in $k$, the measure $|\theta_{i}|_{A}$
  exists on $U(i)_{A}$.

\item[(PF-2)$''$] If $j:U(i)_{A}\hookrightarrow X_{A}$ is induced by
  $U(i)\hookrightarrow X$, then for every $i$ in $k$, the {\em global
  singular series} $j_{\ast}(|\theta_{i}|_{A})$ or simply
  $|\theta_{i}|_{A}$ {\em associated with $f(x)$ and} $i$ exists in
  $\mathscr{S}(X_{A})'$ \ie (equivalently)
  $\int\limits_{U(i)_{A}}\Phi|\theta_{i}|_{A}$ is absolutely
  convergent for every $\Phi$ in $\mathscr{S}(X_{A})$.

\item[(PF-3)] The infinite sum $\sum\limits_{i\in k}|\theta_{i}|_{A}$
  belongs to $\mathscr{S}(X_{A})'$. 
  
  We say that the {\em Poisson formula holds for $f(x)$} (relative to
  $k$) if {\em all the assumptions} (PF-1), (PF-2)$'$, (PF-2)$''$,
  (PF-3) {\em are valid} and {\em further}
\item[(PF-4)]
  \begin{equation*}
    \sum_{i\in k}|\theta_{i}|_{A}=\sum_{i^{\ast}\in
      k}\psi(i^{\ast}f(x))\tag{114}\label{chap4:sec1:subsec6:eq114}
  \end{equation*}
\end{itemize}

The simplest example of a Poisson formula is furnished by considering
the case when $n=1$, $f(x)=x$ so that
$\Phi^{\ast}(x)=\int\limits_{k_{A}}\Phi(y)\psi(xy)|dy|_{A}$. Then
\ref{chap4:sec1:subsec6:eq114} reads
$$
\sum_{i\in k}\delta(x-i)=\sum_{i^{\ast}\in k}\psi(i^{\ast}x)
$$
since $|\theta_{i}|_{A}$ is just the Dirac measure supported at
$i$. But the relation above is the same as saying
\begin{equation*}
  \sum_{i\in k}\Phi(i)=\sum_{i^{\ast}\in
    k}\Phi^{\ast}(i^{\ast})\tag{115}\label{chap4:sec1:subsec6:eq115}
\end{equation*}
for every $\Phi\in \mathscr{S}(k_{A})$. As remarked already, there
remains, built into the simple-looking classical Poisson formula
(\ref{chap4:sec1:subsec6:eq115}), quite a substantial part of the proof of\pageoriginale
the functional equation of Hecke's $L$-series rephrased in adelic
language.

\section[Criteria for the Validity of the...]{Criteria for the Validity of the Poisson Formula and
  Applications}\label{chap4:sec2}%%%% {sec2}

\subsection{}\label{chap4:sec2:subsec1} %%%% subsec{2.1}

Sufficient conditions for the validity of the Poisson formula stated
in the last section are provided by the following theorem which is the
second substantial theorem in our theory for forms of higher degree.

\begin{theorem}\label{chap4:sec2:subsec1:thm1} %%\label{thm2.1}
  Let $f(x)$ denote a homogeneous polynomial of degree $m\geq 2$ in $n$
  variables $x_{1},\ldots,x_{n}$ with coefficients in a global field $k$
  of characteristic not dividing $m$ and let us assume that a tame
  $k$-resolution of the projective hypersurface defined by $f(x)=0$
  exists. Then the Poisson formula holds for $f(x)$ relative to $k$ if
  the following two conditions are satisfied:
  \begin{itemize}
  \item[\rm(C1)] the codimension of $C_{f}$ in $f^{-1}(0)\geq 2$ (\ie
    equivalently, codim $(C_{f})\geq 3$);
  \item[\rm(C2)] there exist $\sigma>2$ and a finite set $S$ of valuations
    $v$ of $k$, such that, for every $i^{\ast}$ in $k_{v}\backslash
    R_{v}$ and $v\not\in S$,
    $$
    |F^{\ast}_{v}(i^{\ast})|\leq |i^{\ast}|^{-\sigma}_{v}.
    $$
    where
    $F^{\ast}_{v}(i^{\ast})=\int\limits_{X^{0}_{v}}\psi_{v}
    (i^{\ast}f(x))|dx|_{v}$.  
  \end{itemize}
\end{theorem}

The proof of this theorem will be given later. We merely remark that
there exists a conjecture to replace condition (C2) by a geometric
condition; this will be explained later. Condition (C1) is easy to
verify and simply means that the hypersurface defined by $f(x)=0$ is
irreducible and normal.

\subsection{Applications}\label{chap4:sec2:subsec2} %%%% subsec{2.2}

We enumerate a series of applications of
Theorem (\ref{chap4:sec2:subsec1:thm1}),
assuming that $k$ has characteristic $0$, just for the sake of
simplicity.
\begin{itemize}
\item[A.1)] Let\pageoriginale $f(x)$ be strongly non-degenerate \ie
  let $f(x)$ be homogeneous with the critical set $C_{f}=\{0\}$. Then
  $(C_{1})$ is equivalent to the condition $n\geq 3$, since
  $\codim_{f^{-1}(0)}(C_{f})=n-1$. Further,
  from \ref{chap3:sec5:subsec3:eq108}),  we know
  that, for all but finitely many $v$ and for every $i^{\ast}\in
  k_{v}\backslash R_{v}$.
  \begin{equation*}
    |F^{\ast}_{v}(i^{\ast})|\leq
    |i^{\ast}|^{-n/m}_{v}\tag*{$(108)'$}\label{chap4:sec2:subsec2:eq108'} 
  \end{equation*}
  as a consequence of Deligne's theorem (\cite{Del}); for $\ord(i^{\ast})$
  divisible by $m$, actually equality holds
  in \ref{chap4:sec2:subsec2:eq108'}. Thus (C2) is
  equivalent to $n>2m$ and the Poisson formula holds for strongly
  non-degenerate $f(x)$ if $n>2m$. In any case, the existence of a tame
  resolution is obvious here and even, in the case of function fields
  $k$, the Poisson formula holds, provided that $m$ is not divisible by
  the characteristic of $k$.
  
\item[A.1)$'$] If $f(x)$ is a non-degenerate quadratic form, we are
  just in the situation A.1)~above with $m=2$. Thus, the Poisson
  formula holds for $n>4$. Incidentally, condition C2)~is easy to
  verify, since 
  \begin{align*}
    |F^{\ast}_{v}(i^{\ast})|^{2} &= \int\limits_{X^{0}_{v}\times
      X^{0}_{v}}\psi_{v}(i^{\ast}(f(x)-f(y))|dx\Lambda dy|_{v}\\
    &=\left(\int\limits_{R_{v}^{(2)}}\psi_{v}(i^{\ast}x_{1}y_{1})|dx_{1}\Lambda
    dy_{1}|\right)^{n}\\
    &= |i^{\ast}|^{-n}_{v}
  \end{align*}
  for every $i^{\ast}$ in $k_{v}\backslash R_{v}$ and for all but
  finitely many $v$; we may replace $f(x)$ by
  $\sum\limits_{i}u_{i}x^{2}_{i}$ with $u_{i}\in R^{\times}_{v}$, in the
  computation above.
  
\item[A.2)] If\pageoriginale $f(x)$ is just a homogeneous polynomial
  of degree $m$ in $n$ variables with coefficient in $k$, then by
  applying the techniques of Birch and Davenport for forms in many
  variables (\cite{Bir}), it can be shown that, for any $\epsilon>0$,
  $$
  |F^{\ast}_{v}(i^{\ast})|\leq
  |i^{\ast}|_{v}^{-(\codim(C_{f})/2^{m-1}(m-1)[k:\mathbb{Q}])+\epsilon} 
  $$
  for every $i^{\ast}$ in $k_{v}\backslash R_{v}$ and all but finitely
  many $v$. The condition (C1) is already fulfilled if we require that
  $$
  \codim(C_{f})>2^{m}(m-1)[k:\mathbb{Q}](\geq 2)
  $$
  and consequently the Poisson formula holds, subject to this
  requirement being satisfied. For further details, we refer to our
  paper \cite{Igu 8}.
\end{itemize}


\subsection{Further Applications of the Poisson
  Formula}\label{chap4:sec2:subsec3}  %%% subsec{2.3}

\begin{itemize}
\item[A.3)] Let $G$ be a ``$k$-form'' of $SL_{2m}$ (\ie an algebraic
  group defined over $k$ and isomorphic to $SL_{2m}$ over an algebraic
  closure $\ob{k}$ of $k$) and let $\rho:G\to GL(X)$ be a ``$k$-form''
  of the $m(2m-1)$ -  dimensional fundamental representation of $G$,
  where $X$ is a certain affine space. If $G$ is $k$-isomorphic to
  $SL_{2m}$, then we may identify $X$ with the vector space of
  $2m$-rowed skew-symmetric matrices $x$ on which the action of $G$
  via $\rho$ is described by $x\to \rho(g)\cdot x=g x^{t}g$ for
  $g$ in $G$. Upto a factor from $K^{\times}$, there exists a unique
  invariant of degree $m$, with coefficients in $k$; this invariant is
  precisely the {\em Pfaffian} $\Pf(x)$ with its well-known property
  that $\Pf(g x^{t}g)=\det(g)\Pf(x)$ for any $g$ in
  $GL_{2m}$. In this case, it is known that
  $F^{\ast}_{v}(i^{\ast})=\sum\limits^{m-1}_{i=1}c_{i}|i^{\ast}|^{-(2i+1)}_{v}$,
  where $c_{i}=\sum\limits_{\substack{1\leq j\leq m\\ j\neq
      i}}(1-q^{-(2j+1)})/(1-q^{-2(j-i)})$. In\pageoriginale the
  general case, there exists a unique invariant $f(x)$ which, for all
  but finitely many $v$, coincides with $\Pf(x)$ after a non-singular
  linear transformation with coefficients in $R_{v}$. As a result, the
  above formula for $F^{\ast}_{v}(i^{\ast})$ is valid even in the
  general case and consequently, we have
  $$
  |F^{\ast}_{v}(i^{\ast})|\leq |i^{\ast}|_{v}^{-3+\epsilon}
  $$
  for every $i^{\ast}$ in $k_{v}\backslash R_{v}$ and for all but
  finitely many $v$. Further 
  $$
  \codim_{f^{-1}(0)}(C_{f})=5
  $$
  and thus the Poisson formula is valid in this case. Complete details
  may be found in our paper \cite{Igu 1}.
  
\item[A.4)] Let $G$ be a $k$-form of the connected, simply connected,
  simple group of type $E_{6}$ and let $\rho:G\to GL(X)$ be a $k$-form
  of the $27$-dimensional fundamental representation of $G$. Then
  there exists a cubic invariant $f(x)$ known as ``det'' (in the
  theory of Jordan algebras) and unique upto a factor from $k^{\ast}$,
  as before (\cite{Jac}). In this case, it can be shown that
  $$
  \codim_{f^{-1}(0)}(C_{f})=9\quad\text{and}\quad
  |F^{\ast}_{v}(i^{\ast})|\leq |i^{\ast}|^{-5+\epsilon}_{v}
  $$
  and hence the Poisson formula holds. The inequality above is a
  consequence of the more precise assertion:
  \begin{equation*}
    F^{\ast}_{v}(i^{\ast})=\frac{1-q^{-9}}{1-q^{-4}}|i^{\ast}|^{-5}_{v}-
    q^{-4}\frac{1-q^{-5}}{1-q^{-4}}|i^{\ast}|^{-9}_{v}
    \tag{116}\label{chap4:sec2:subsec3:eq116} 
  \end{equation*}
  Formula\pageoriginale \ref{chap4:sec2:subsec3:eq116}) may be found
  in \cite{Mar}, \S\ 14. 

\item[A.5)] Consider the complete polarization $\det(a,b,c)$ of the
  cubic form ``det'' in example A.4) above; the form $\det(a,b,c)$ is
  symmetric trilinear in $a$, $b$, $c$ and
  $\det(a,a,a)=3!\det(a)$. Taking a non-degenerate symmetric bilinear
  form $Q(a,b)$ with coefficients in $k$, on the same space, define
  $a\times b$, $a^{\sharp}$ by
  $$
  Q(a\times b,c)=\det (a,b,c),\quad a^{\sharp}=(1/2)(a\times a).
  $$
  Then, for a suitable choice of $Q(a,b)$, we have
  $$
  (a^{\sharp})^{\sharp}=\det(a)\cdot a.
  $$
  Let $X$ be the $(2(27+1)=)56$ - dimensional space consisting of
  $x=(a,b,\alpha,\beta)$ where $a$, $b$ are elements of the
  $27$-dimensional space mentioned in A.4) and $\alpha$, $\beta$ are in
  $\mathbb{C}$. After Freudenthal, we take the quartic
  $$
  f(x)=Q(a^{\sharp},b^{\sharp})+\alpha
  \det(b)+\beta\det(a)-(1/4)(Q(a,b)-\alpha\beta)^{2}. 
  $$
  The group of automorphisms of $f(x)$ has two connected components and
  the connected component $G$ of the identity is a $k$-form of the
  connected, simply connected simple group of type $E_{7}$ and
  corresponds to its ``first fundamental representation''. Conversely,
  for a $k$-form of type $E_{7}$ with its $56$-dimensional fundamental
  representation defined over $k$, there exists only one invariant,
  namely $f(x)$, upto a factor from $K^{\times}$.
\end{itemize}

For the quartic $f(x)$ above, we have
$$
\codim_{f^{-1}(0)}(C_{f})=10\quad\text{and}\quad
|F^{\ast}_{v}(i^{\ast})|\leq |i^{\ast}|_{v}^{-5-1/2+\epsilon}
$$
and\pageoriginale therefore, the Poisson formula holds in this
case. The estimate for $F^{\ast}_{v}(i^{\ast})$ follows from the more
precise relation:
\begin{align*}
  F^{\ast}_{v}(i^{\ast}) &=
  \frac{(1-q^{-14})(1-q^{-18})}{(1-q^{-4})(1-q^{-17})}
  \gamma(i^{\ast})|i^{\ast}|^{-5-1/2}_{v}\\[5pt]
  &\quad -q^{-4}\frac{(1-q^{-10})(1-q^{-14})}{(1-q^{-4}) (1-q^{-9})}
  \gamma(i^{\ast})|i^{\ast}|_{v}^{-9-1/2}\\[5pt] 
  &\quad +(q^{-14}+c(i^{\ast}))|i^{\ast}|^{-14}_{v}, 
\end{align*}
where 
$$
\gamma(i^{\ast})=
\begin{cases}
  1, &\text{for } \ord(i^{\ast}) \text{ even}\\
  \chi(-\pi^{-\ord(i^{\ast})}i^{\ast})q^{1/2}g_{\chi}, & \text{for }
  \ord(i^{\ast})\text{ odd} 
\end{cases}
$$
with $\chi\neq 1$, $\chi^{2}=1$ and
$$
c(i^{\ast})=
\begin{cases}
  q^{-13}\dfrac{(1-q^{-1})(1+q^{-13})(1-q^{-14})}{(1-q^{-9})(1-q^{-17})}, 
  \text{ for } \ord(i^{\ast})\text{ even}\\[10pt] 
  q^{-17-1/2}\dfrac{(1-q^{-1})(1+q^{-4})(1-q^{-14})}{(1-q^{-9})(1-q^{-17})}
  \gamma(i^{\ast}), \text{ for } \ord(i^{\ast})\text{ odd.}
\end{cases}
$$
For further details, one may refer to \cite{Igu 9}.

\section{The Siegel Formula}\label{chap4:sec3} %%% sec3

In this section, we shall explain the significance of the Poisson
formula in the simplest case and then derive the Siegel formula due to
Weil, as a consequence of the Poisson formula for quadratic forms.

\begin{Remark*}
  Except\pageoriginale for the case A.1) and hence also A.1)$'$, we have
  assumed that $k$ has characteristic $0$. But actually, G.R.\@ Kempf
  has explicitly constructed, in the fall of 1974, $k$-resolutions for
  the hypersurface defined by $f(x)=0$, in the cases A.3), A.4) and
  A.5), for an {\em arbitrary} field $k$. Since conditions C1), C2) are
  valid in the case of function fields as well, we see, by incorporating
  the results of Kempf, that the Poisson formula holds, if the
  characteristic of $k$ does not divide $m!$ in the case A.3) and if $k$
  has characteristic different from 2, 3 in the cases A.4) and
  A.5). However, we might mention that, without any restriction on the
  characteristic of $k$, the Poisson formula is valid in the case A.3).
\end{Remark*}

\subsection{Statement of the Siegel Formula}\label{chap4:sec3:subsec1} %%% subsec{3.1}

Let $k$ be a global field of characteristic (different from 2), $X$
the affine $n$-space and let $f(x)$ be a non-degenerate quadratic form
in $n\geq 3$ variables $x_{1},\ldots,x_{n}$ with coefficients in
$k$. Let $G$ be the special orthogonal group $SO(f)$ of $f$,
consisting of all linear transformations on $x_{1},\ldots,x_{n}$ which
leave $f(x)$ invariant and have determinant $1$. Let $dg$ be an
invariant differential form on $G$ of (maximal) degree
$\frac{1}{2}n(n-1)$, defined over $k$. Then, for all but finitely many
valuations $v$ of $k$, we have
$$
m(G^{0}_{v})=\int\limits_{G^{0}_{v}}|dg|_{v}=
\begin{cases}
\prod\limits_{1\leq i\leq(n-1)/2}(1-q^{-2i}), & \text{for } n \text{
  odd}\\[5pt]
\prod\limits_{1\leq i<n/2}(1-q^{-2i})\cdot (1-\chi(d)q^{-n/2}), & \text{for }
n \text{ even} 
\end{cases}
$$
where $\chi^{2}=1$, $\chi\neq 1$ and the {\em discriminant} $d$ of
$f(x)$ is defined to be $(-1)^{n(n-1)/2}\det (t_{ij})$, if
$f(x,y)=f(x+y)-f(x)-f(y)$ is written as
$\sum\limits^{n}_{i,j=1}t_{ij}x_{i}x_{j}$ with\pageoriginale
$t_{ij}=t_{ji}$. In particular, (for $n\geq 3$) we see, from above,
that $\prod\limits_{v\not\in S}m(G^{0}_{v})$ is absolutely convergent,
where $S$ is the finite set of exceptional $v$ mentioned above; thus
the (adelic) measure $|dg|_{A}=\bigotimes'_{v}|dg|_{v}$ exists. The
{\em Siegel formula (for the special orthogonal group} G) {\em due to
  Weil} (\cite{Wei 5}) states that
\begin{equation*}
  (1/2)\int\limits_{G_{A}/G_{k}}\left(\sum_{\xi\in
    X_{k}}\Phi(g\xi)\right)|dg|_{A}=\Phi(0)+ \sum_{i^{\ast}\in k}
  \int\limits_{X_{A}}\Phi(x)\psi(i^{\ast}f(x))|dx|_{A}\tag{117}
  \label{chap4:sec3:subsec1:eq117}
\end{equation*}
for every $\Phi$ in $\mathscr{S}(X_{A})$ and $n>4$ (with $\psi\in
(k_{A}/k)^{\ast}$ chosen already).

\subsection{}\label{chap4:sec3:subsec2} %%%% subsec{3.2}

In the course of deriving the Siegel formula, we shall freely use the
following theorems.
\begin{enumerate}
\renewcommand{\theenumi}{\roman{enumi}}
\renewcommand{\labelenumi}{(\theenumi)}
\item {\em Minkowski-Hasse theorem:} $U(i)_{A}\neq \phi
  \Rightarrow U(i)_{k}\neq \phi$ {\em for every} $i\in k$ (for every
  $n\geq 1$).
\item {\em Witt's theorem:} $\xi_{i}\in U(i)_{k}(\neq \phi)\Rightarrow
  G_{k}\cdot \xi_{i}=U(i)_{k}$ {\em for every $i\in k$ and further
    $G_{A}\cdot\xi_{i}=U(i)_{A}$} (if $n\geq 3$).
\item {\em For $n=3,4$,}
  $\tau_{k}(G){\displaystyle{\mathop{=}^{\text{def}}}} 
  \int\limits_{G_{A}/G_{k}}|dg|_{A}=2$. 
\item {\em For the twisted Fourier transformation $\hat{\;}$ on
  $X_{A}$ defined by
  $\hat{\Phi}(x)=\int\limits_{X_{A}}\Phi(y)\psi(f(x,y))|dy|_{A}$,
  we have}
$$
\psi(i^{\ast}f(x))~\hat{}=\psi(-(i^{\ast})^{-1}f(x))
$$
{\em for $n\geq 1$ and every $i^{\ast}\in k^{\times}$, where
  $f(x,y)=f(x+y)-f(x)-f(y)$.} 
\end{enumerate}

Among the theorems mentioned above, (i) and (ii) are well-known. For
the proof of (iii), we refer to Theorem 3.7.1 in the lectures
\cite{Wei 3} of Weil. For the proof of (iv), we refer to Th\'eor\`eme
2 and Th\'eor\`eme 5 in the paper \cite{Wei 4}\pageoriginale of Weil and
Theorem 3 in our paper on ``Harmonic analysis and theta functions'',
Acta Math. 120(1968), 187-222.

\subsection{Tamagawa Number}\label{chap4:sec3:subsec3} %%% \subsec {3.3}

We recall that, on any algebraic group $G$ defined over an arbitrary
field $k$, there exists a left $G$-invariant differential form $dg\neq
0$, which is of degree equal to the dimension of $G$ and further,
defined over $k$; we shall simply call such a form a {\em gauge form
  on $G$.} A gauge form is automatically everywhere regular and
non-vanishing on $G$. More generally, suppose that $U$ is a
non-singular algebraic variety defined over $k$ with a $k$-rational
point $\xi$, such that $G$ acts transitively on $U$ through
$(g,x)\mapsto g\cdot x$ and further, the action of $G$ is defined over
$k$; then the stabiliser $H$ at $\xi$ is an algebraic subgroup of $G$
defined over $k$. In such a case, we simply write $G/H=U$ and call $U$
an algebraic homogeneous space defined over $k$. A $G$-invariant
differential form $\theta\neq 0$ on $U$ which is of degree equal to
the dimension of $U$ and defined over $k$, is called a {\em gauge form
  on} $U$; such a form does not exist, in general. But if it exists,
then any other gauge form on $U$ differs from $\theta$ only by a
scalar factor from $k^{\times}$. We observe that if $G$ is the special
orthogonal group $SO(f)$ of a quadratic form $f(x)$ over $k$, then
$\theta_{i}$ is a gauge form on $U(i)$ for every $i$ in $k$.

Let now $k$ be a global field and let $k_{A}$, $k^{\times}_{A}$ be its
adele and idele groups respectively, as before. For any idele
$t=(\ldots,t_{v},\ldots)\in k^{\times}_{A}$, define its modulus
$|t|_{A}$ as the rate of measure change in $k_{A}$ under the
multiplication by $t$; then $|t|_{A}=\prod\limits_{v}|t_{v}|_{v}$. For
any $c$ in $k^{\times}\hookrightarrow k^{\times}_{A}$, multiplication by
$c$ gives an automorphism of the compact group $k_{A}/k$; hence, for
$c\in k^{\times}$, we have $|c|_{A}=1$. This fact (well-known as the
product formula for global fields) has\pageoriginale a far-reaching
implication: namely, if a gauge form $\theta$ exists on $U=G/H$ and
if, further, the Tamagawa measure $|\theta|_{A}$ also exists on
$U_{A}$, then, for any other gauge form $\omega$ on $U$,
$|\omega|_{A}$ (exists and) is the same as $|\theta|_{A}$, since
$\omega=c\theta$ for $c\in k^{\times}$ and
$|\omega|_{A}=|c\theta|_{A}=|c|_{A}|\theta|_{A}=|\theta|_{A}$. Thus
$|\theta|_{A}$ depends only on $U$ and $k$. Furthermore, if $dg$, $dh$
are the gauge forms on $G$, $H$ and if $G_{A}\cdot \xi=U_{A}$, then we
will get $G_{A}/H_{A}=U_{A}$ and $|dg|_{A}$, $|dh|_{A}$,
$|\theta|_{A}$ automatically ``match together'' in the sense that
\begin{equation*}
  \int\limits_{G_{A}}\varphi(g)|dg|_{A}=\int\limits_{U_{A}}
  \left(\int\limits_{H_{A}}\varphi(gh)|dh|_{A}\right)
  |\theta(g\cdot\xi)|_{A}\tag{118}\label{chap4:sec3:subsec3:eq118} 
\end{equation*}
for any continuous function $\varphi$ on $G_{A}$ with compact
support. These are the excellent properties of the Tamagawa measures.

On the other hand, for any $\gamma$ in $G_{k}$, we have
$d(g\gamma)=c\cdot dg$ with $c\in k^{\times}$ and hence $|dg|_{A}$ is
right $G_{k}$-invariant. Therefore, an intrinsic measure $\tau_{k}(G)$
can be introduced as
$$
\tau_{k}(G)=m(G_{A}/G_{k})=\int\limits_{G_{A}/G_{k}}|dg|_{A}
$$
and $\tau_{k}(G)$ is called the {\em Tamagawa number} of $G$ relative
to $k$ (usually under the assumption that
$\tau_{k}(G)<\infty$). Theorem (iii) in \S\ \ref{chap4:sec3:subsec2}
(that we have 
assumed) merely states that, for $n=3$ or $4$, we have
$\tau_{k}(SO(f))=2$. Furthermore, if $\mathbb{G}_{a}$ denotes the
affine line considered as an additive algebraic group (defined over
the prime field), then we have
$\tau_{k}(\mathbb{G}_{a})=m(k_{A}/k)=1$. In order to proceed further,
we shall prove the following elementary lemma.

\setcounter{lemma}{1}
\begin{lemma}\label{chap4:sec3:subsec3:lem2}  %%%\label{lem3.2}
  Suppose that a locally compact group $G$ splits as a semi-direct
  product of $H$ by $G':G=G'\cdot H$. Let $dg$, $dg'$, $dh$ denote
  left-invariant measures  on\pageoriginale $G$, $G'$, $H$ respectively
  such that $dg=dg'\otimes dh$. Let $\Gamma$, $\Gamma'$, $\Delta$ denote
  discrete subgroups of $G$, $G'$, $H$ respectively such that
  $\Gamma=\Gamma'\cdot \Delta$. Assume that $dg$, $dg'$, $dh$ are also
  right-invariant under $\Gamma$, $\Gamma'$, $\Delta$ respectively and
  further that, for every $\gamma$ in $\Gamma'$, $d(\gamma
  h\gamma^{-1})=dh$. Then we have
  \begin{equation*}
    \int\limits_{G/\Gamma}\varphi(g) dg = \int\limits_{G'/\Gamma'}
    \left(\int\limits_{H/\Delta} \varphi(g'h)dh\right)dg'\tag{119}
    \label{chap4:sec3:subsec3:eq119}
  \end{equation*}
  for every continuous function $\varphi$ on $G/\Gamma$ with compact
  support. 
\end{lemma}

\begin{proof}
  We can find a continuous function $\varphi_{0}$ on $G$ with compact
  support such that
  $$
  \varphi(g)=\sum_{\gamma\in\Gamma}\varphi_{0}(g\gamma)
  $$
  for every $g$ in $G$. Similarly, if we define $\varphi_{1}(g)$ as
  $\sum\limits_{\delta\in\Delta}\varphi_{0}(g\delta)$ then we get 
  \begin{multline*}
  \int\limits_{G/\Gamma}\varphi(g)dg=\int\limits_{G} \varphi_{0} (g)
  dg = \int\limits_{G'}\left(\int\limits_{H}\varphi_{0} (gh) dh\right)
  dg'\\
  =\int\limits_{G'/\Gamma'} \left(\sum_{\gamma'\in\Gamma'}
  \int\limits_{H/\Delta} \varphi_{1}(g'\gamma'h)dh\right)dg'. 
  \end{multline*}
If we apply to the inner integral the measure-preserving automorphism
$h\to {\gamma'}^{-1}h\gamma'$ of $H/\Delta$,
then (\ref{chap4:sec3:subsec3:eq119}) is immediate.
\end{proof}

We now go back to our previous notation and denote by $G$, $G'$, $H$
algebraic groups defined over a global field such that $G$ is the
semi-direct product of $H$ by $G'$. Let $dg$, $dg'$, $dh$ denote gauge
forms on $G$, $G'$, $H$. Then we can apply
Lemma \ref{chap4:sec3:subsec3:lem2} to
$G_{A}$, $G'_{A}$, $H_{A}$, $G_{k}$, $G'_{k}$, $H_{k}$, $|dg|_{A}$,
$|dg'|_{A}$, $|dh|_{A}$ instead of $G$, $G'$, $H$, $\Gamma$,
$\Gamma'$, $\Delta$, $\Delta'$, $\Delta$, $dg$, $dg'$, $dh$
respectively. In fact, all the conditions in the lemma are
satisfied. Since we can replace $\varphi$
in (\ref{chap4:sec3:subsec3:eq119}) by any
non-negative measurable function on $G/\Gamma$, we get, on putting
$\varphi=1$, the formula
\begin{equation*}
  \tau_{k}(G)=\tau_{k}(G')\tau_{k}(H)\tag{120}
  \label{chap4:sec3:subsec3:eq120} 
\end{equation*}
Since\pageoriginale $\tau_{k}(\mathbb{G}_{a})=1$, formula
(\ref{chap4:sec3:subsec3:eq120}) 
implies that $\tau_{k}(G)=\tau_{k}(G')$, whenever $H$ is
$k$-isomorphic to a direct product of $\mathbb{G}_{a}$.

\subsection{Proof of the Siegel Formula}\label{chap4:sec3:subsec4} %%%% subsec{3.4}

We proceed to prove the Siegel formula given by
\ref{chap4:sec3:subsec1:eq117}. 

Applying induction on $n$, let us assume that the Tamagawa number of
the special orthogonal group of any non-degenerate quadratic form over
$k$ in $n'$ variables with $3\leq n'<n$ is $2$; in spite of the fact
that $n>4$ (as stated after formula (\ref{chap4:sec3:subsec1:eq117})), our induction
hypothesis is quite legitimate by virtue of Theorem (iii) assumed in
\S\ \ref{chap4:sec3:subsec2}.

With the same notation as in Theorem (ii) assumed in
\S\ \ref{chap4:sec3:subsec2}, we
have clearly
\begin{equation*}
  X_{k}\backslash \{0\}=\coprod\limits_{i\in
    k}U(i)_{k}=\coprod\limits_{i\in k}{}'G_{k}\cdot
  \xi_{i}\tag{121} \label{chap4:sec3:subsec3:eq121} 
\end{equation*}
where the accent (over the symbol for disjoint union) indicates that
$i$ runs only over the subset of $k$ which ensures $U(i)_{k}\neq
\phi$. For any such $i$, we set
$$
H_{i}= \text{ the stabiliser of $G$ at } \xi_{i}.
$$
We then assert that $\tau_{k}(H_{i})=2$ for every such $i$. The proof
runs as follows. If $i\neq 0$, let $X'=\{x\in X;f(\xi_{i},x)=0\}$ and
$f'$ the restriction of $f$ to $X'$. Then we have $SO(f')=H_{i}$ and
since $X'$ has dimension $n-1\geq 4$, we see that $\tau_{k}(H_{i})=2$,
by the induction hypothesis. Let now $i=0$. Then choosing $\eta$ from
$X_{k}$ with $f(\xi_{0},\eta)=1$ and putting
$\eta_{0}=-f(\eta)\xi_{0}+\eta$, we obtain $\eta_{0}\in X_{k}$,
$f(\eta_{0})=0$ and $f(\xi_{0},\eta_{0})=1$. Let us now take $X'$ to
be the subspace of $X$ defined by $f(\xi_{0},x)=f(\eta_{0},x)=0$ and
$f'$ to be the restriction of $f$ to $X'$. Then, as in the previous
case, $f'$ is non-degenerate and $H_{0}$ can be verified to be the
semidirect product\pageoriginale by $SO(f')$ of the group $H$ of
matrices of the form
$$ 
\begin{pmatrix} 
  1 & -f'(c) & -{}^{t}cT'\\
  0 & 1 & 0\\
  0 & c & 1_{n-2}
\end{pmatrix}
$$
where $c$ is an arbitrary $(n-2)$-rowed column, $f'(x')=1/2^{t}x'T'x'$
for $x'\in X'$ and $1_{n-2}$ is the $(n-2)$-rowed identity
matrix. Since $H$ is isomorphic to the $(n-2)$-fold product of
$\mathbb{G}_{a}$ and since $n-2\geq 3$, it follows by applying
(\ref{chap4:sec3:subsec3:eq120}), that
$\tau_{k}(H_{0})=\tau_{k}(SO(f'))=2$, in view of the 
induction hypothesis once again.

For $\xi_{i}\in U(i)_{k}$, the map $g\mapsto \xi_{i}$ of $G$ to $U(i)$
induces the continuous injection $G_{A}/(H_{i})_{A}\to U(i)_{A}$ and
Witt's theorem assumed in \S\ \ref{chap4:sec3:subsec2} guarantees the
surjectivity of 
the latter map. Thus we have a homeomorphism
$$
G_{A}/(H_{i})_{A}\simeq U(i)_{A}
$$
(in view of a well-known theorem in the theory of topological groups).

We are now ready to start the proof of the Siegel formula. The left
hand side of (\ref{chap4:sec3:subsec1:eq117}) is the same as
$$
\tau_{k}(G)\Phi(0)+\int\limits_{G_{A}/G_{k}}\left(\sum_{\xi\in
  X_{k}\backslash \{0\}}\Phi(g\xi)\right)|dg|_{A},
$$
on pulling out the term corresponding to $\xi=0$. By
(\ref{chap4:sec3:subsec3:eq121}), we
have
\begin{multline*}
  \sum_{\xi\in X_{k}\backslash \{0\}}\Phi(g\xi)=\sum_{i\in
    k}\sum_{\xi\in U(i)_{k}}\Phi(g\xi)=\sum_{i\in k}{}'\sum_{\xi \in
    G_{k}\cdot \xi_{i}}\Phi(g\xi)\\
  =\sum_{i\in k}{}'\sum_{\gamma\in
    G_{k}\mod (H_{i})_{k}}\Phi(g\gamma\xi_{i}).
\end{multline*}
Therefore, integrating over $G_{A}/G_{k}$, we get 
\begin{align*}
  \int\limits_{G_{A}/G_{k}}\left(\sum_{\xi\in
    X_{k}\backslash\{0\}}\Phi(g\xi)\right)|dg|_{A} &= \sum_{i\in
    k}{}'\int\limits_{G_{A}/(H_{i})_{k}}\Phi(g\xi_{i})|dg|_{A}\\
  & = \sum_{i\in  k}{}'\tau_{k}(H_{i}) \int\limits_{U(i)_{A}} \Phi|
  \theta_{i} |_{A};\tag{122}\label{chap4:sec3:subsec3:eq122}
\end{align*}\pageoriginale
we have used here the fact that integrating over $G_{A}/(H_{i})_{k}$
with respect to $|dg|_{A}$ is the same as integrating over
$(H_{i})_{A}/(H_{i})_{k}$ relative to $|dh|_{A}$ coming from a gauge
form $dh$ on $H_{i}$ and then over $G_{A}/(H_{i})_{A}=U(i)_{A}$
relative to $|\theta_{i}|_{A}$. We recall now that $\tau_{k}(H_{i})=2$
for every $i$ in $k$ such that $U(i)_{k}\neq \phi$.

In view of the Minkowski-Hasse theorem, $U(i)_{k}=\phi$ implies
$U(i)_{A}=\phi$ and hence the integral over $U(i)_{A}$ is $0$; hence
we may legitimately remove the accent over the symbol of summation in
(\ref{chap4:sec3:subsec3:eq122}). In this way, we get, for $n>4$ and $\Phi$ in
$\mathscr{S}(X_{A})$, that
\begin{equation*}
  \begin{split}
    \int\limits_{G_{A}/G_{k}}\left(\sum_{\xi\in
      X_{k}}\Phi(g\xi)\right)|dg|_{A} &= \tau_{k}(G)\Phi(0)+2\sum_{i\in
      k}\int\limits_{U(i)_{A}}\Phi|\theta_{i}|_{A}\\
    &= \tau_{k}(G)\Phi(0)+2\sum_{i^{\ast}\in
      k}\int\limits_{X_{A}}\Phi(x)\psi(i^{\ast}f(x))|dx|_{A}. 
  \end{split}\tag{123}\label{chap4:sec3:subsec3:eq123}
\end{equation*}
The second step in (\ref{chap4:sec3:subsec3:eq123}) is a consequence of the Poisson formula
\ref{chap4:sec1:subsec6:eq114}. On the other hand, for $\Psi(x)=\Phi(gx)$, we have
\begin{align*}
  \hat{\Psi}(x) &= \int\limits_{X_{A}}\Phi(gy)\psi(f(x,y))|dy|_{A}\\
  &= \int\limits_{X_{A}}\Phi(gy)\psi(f(gx,gy))|dy|_{A}\quad (y\to
  g^{-1}y)\\
  &= \int\limits_{X_{A}}\Phi(y)\psi(f(gx,y))|dy|_{A}\\
  &= \hat{\Phi}(gx).
\end{align*}
The\pageoriginale usual Poisson formula implies that
\begin{equation*}
  \sum_{\xi \in X_{k}}\Phi(g\xi)=\sum_{\xi\in
    X_{k}}\Psi(\xi)=\sum_{\xi\in X_{k}}\hat{\Psi}(\xi)
  =\sum_{\xi\in
    X_{k}}\hat{\Phi}(g\xi).\tag{124} \label{chap4:sec3:subsec3:eq124}
\end{equation*}
Therefore, the left hand side of (\ref{chap4:sec3:subsec3:eq123}) is invariant if $\Phi$ is
replaces by $\hat{\Phi}$. By Theorem (iv) assumed in
\S\ \ref{chap4:sec3:subsec2}, we have
\begin{equation*}
  \int\limits_{X_{A}}\hat{\Phi}(x)\psi(i^{\ast}f(x))|dx|_{A}
  =\int\limits_{X_{A}}\Phi(x)\psi(-i^{\ast^{-1}}f(x))|dx|_{A}
  \tag{125} \label{chap4:sec3:subsec3:eq125}  
\end{equation*}
for every $i^{\ast}$ in $k^{\times}$. The invariance under $\Phi\to
\hat{\Phi}$ of the right hand side of (\ref{chap4:sec3:subsec3:eq123})
implies, in view of (\ref{chap4:sec3:subsec3:eq125}), the invariance under $\Phi\to
\hat{\Phi}$ of
$$
\tau_{k}(G)\Phi(0)+2\int\limits_{X_{A}}\Phi(x)|dx|_{A}
=\tau_{k}(G)\Phi(0)+2\hat{\Phi}(0). 
$$

Hence $\tau_{k}(G)$ is necessarily equal to $2$ and the Siegel formula
is proved. 

\begin{Remark*}
  In the course of proving the Siegel formula, we have tacitly used the
  fact that $\tau_{k}(G)<\infty$. However, following H.\@ Ariturk
  (\cite{Ari}), the finiteness of $m(G_{A}/G_{k})$ can be established
  inductively as follows. In view of Theorem (iii) assumed in
  \S\ \ref{chap4:sec3:subsec2}, the induction hypothesis entails that the Tamagawa
  number of the special orthogonal group of a non-degenerate quadratic
  form over $k$ in $n'$ variables with $3\leq n'<n$ is not only finite
  but, in fact, equal to $2$. Now we can certainly find $\Phi\neq 0$ in
  $\mathscr{S}(X_{A})$ such that $\Phi\geq 0$ and $\Phi(0)=0$; it is clear
  then that $\hat{\Phi}(0)>0$. Moreover, as is well-known, we can
  also choose $\varphi$ in $\mathscr{S}(X_{A})$ such that $\varphi\geq 0$
  and, further, $|\hat{\Phi}(x)|\leq \varphi(x)$ for every $x$ in
  $X_{A}$. By going through the previous argument for such $\Phi$, we
  have
  \begin{align*}
    \int\limits_{G_{A}/G_{k}}\left(\sum_{\xi\in
      X_{k}}\Phi(g\xi)\right)|dg|_{A} &=
    \int\limits_{G_{A}/G_{k}}\left(\sum_{\xi\in X_{k}\backslash \{
      0\}}\Phi(g\xi)\right)|dg|_{A}\\ 
    &= \sum_{i\in
      k}\tau_{k}(H_{i})\int\limits_{U(i)_{A}}\Phi|\theta_{i}|_{A}\\
    &= 2\sum_{i\in
      k}\int\limits_{U(i)_{A}}\Phi|\theta_{i}|_{A}\quad\text{(by
      induction)}\\
    &<\infty, \text{ since the Poisson formula holds for}\\
    &\qquad\quad \text{$f(x)$ and hence (PF-3) is valid.}
  \end{align*}\pageoriginale
  On the other hand,
  \begin{align*}
    \int\limits_{G_{A}/G_{k}}\left|\sum_{\xi\in
      X_{k}\backslash\{0\}}\hat{\Phi}(g\xi)\right||dg|_{A} & \leq
    \int\limits_{G_{A}/G_{k}}\left(\sum_{\xi\in
      X_{k}\backslash\{0\}}\varphi(g\xi)\right)|dg|_{A}\\ 
    &= 2\sum_{i\in k}\int\limits_{U(i)_{A}}\varphi|\theta_{i}|_{A}\\
    &<\infty,
  \end{align*}
  in a similar manner. In other words, $\sum\limits_{\xi\in
    X_{k}}\Phi(g\xi)$ and $\sum\limits_{\xi\in
    X_{k}\backslash\{0\}}\hat{\Phi}(g\xi)$ for the chosen $\Phi$ are
  both in $L^{1}(G_{A}/G_{k})$ and hence the same is true of their
  difference as well. Thus the constant function $\hat{\Phi}(0)>0$
  is in $L^{1}(G_{A}/G_{k})$ which implies that $m(G_{A}/G_{k})<\infty$.
\end{Remark*}

\section{Other Siegel Formulas}\label{chap4:sec4}%%% sec4}

\subsection{General Comments}\label{chap4:sec4:subsec1} %%% subsec4.1

In order that we may be able to carry over the method of proof for the
Siegel formula for orthogonal groups to other cases, we need to have,
besides the\pageoriginale Poisson formula
\ref{chap4:sec1:subsec6:eq114}), suitable 
generalizations of Theorems (i), (ii), (iii) and (iv) assumed in
\S\ \ref{chap4:sec3:subsec2}. In the situations covered by A.3), A.4),
and A.5) in 
\S\ \ref{chap4:sec2:subsec3}, we have, indeed, $U(i)_{k}\neq \phi$ for
every $i$ in 
$k$; therefore, there is no need to look for an analogue of Theorem
(i) assumed in \S\ \ref{chap4:sec3:subsec2}. Theorem (ii) assumed in
\S\ \ref{chap4:sec3:subsec2} 
can be replaces by the following general theorem (stated without
proof).

\begin{theorem}\label{chap4:sec4:subsec1:thm1} %%% {thm4.1}
  Let $U=G/H$ denote an algebraic homogeneous space defined over a
  global field $k$ and assume that both $G$ and $H$ are connected and
  simply connected. Then $U_{k}$ decomposes into finitely many disjoint
  $G_{k}$-orbits $G_{k}\cdot \xi_{\alpha}$ and $U_{A}$ becomes the
  disjoint union of $G_{A}$-orbits $G_{A}\cdot \xi_{\alpha}$ for the
  same $\xi_{\alpha}$, with each $G_{A}\cdot \xi_{\alpha}$ open in $U_{A}$.
\end{theorem}

We wish to emphasise that in the cases covered by A.3), A.4) and A.5)
in \S\ \ref{chap4:sec2:subsec3}, every $G$-orbit in $X$ defined over
$k$ satisfies 
the conditions in the above theorem. Even in the case of quadratic
forms corresponding to A.1)$'$ of \S\ \ref{chap4:sec2:subsec2}, the
stabilisers turn 
out to be connected and simply connected, if we use $\Spin(f)$ and $G$
instead of $SO(f)$. A systematic exposition (with references) of this
subject can be found in \cite{Igu 2}. We recall, in this context, a {\em
  conjecture of Weil} which states that $\tau_{k}(G)=1$ for any
connected simply connected algebraic group $G$ defined over a global
field $k$. Since the stabilisers (other than $G$) have smaller
dimension, we can hope to verify this conjecture inductively at least
in some cases.

We have already explained the nature of the analogue in the general
case of Theorem (iii) of \S\ \ref{chap4:sec3:subsec2}. As for Theorem (iv), we remark
that a generalization of the formula
$\psi(i^{\ast}f(x))\hat{}=\psi(-(i^{\ast})^{-1}f(x))$ is not known and
may even not exist. (In\pageoriginale this connection, we refer to our
paper \cite{Igu 6}). Therefore, following Mars (\cite{Mar}), we proceed as
follows. We choose a symmetric (or skew-symmetric) non-degenerate
bilinear form $Q(x,y)$ on $X\times X$ with coefficients in $k$, such
that $G$ becomes invariant under $g\mapsto {}^{t}g$, where, for any
$g$ in $GL(X)$, we define the adjoint ${}^{t}g$ by the relation
$Q(gx,y)=Q(x,{}^{t}gy)$. Such a choice of $Q(x,y)$ is always possible;
for example, we may take $Q(x,y)=f(x,y)$ in A.1)$'$ of
\S\ \ref{chap4:sec2:subsec2}. As in that case, we define the (twisted) Fourier
transformation $\Phi\to \hat{\Phi}$ by
$$
\hat{\Phi}(x)=\int\limits_{X_{A}}\Phi(x)\psi(Q(x,y))|dy|_{A}\text{
  \ for \ } x\in X_{A}.
$$
Then, for every $g$ in $GL(X)_{A}$, we have
$$
\sum_{\xi\in X_{k}}\Phi(g\xi)=|\det(g)|^{-1}_{A}\sum_{\xi\in
  X_{k}}\hat{\Phi}({}^{t}g{}^{-1}\xi); 
$$
this is just the usual Poisson formula. Since the automorphism
$g\mapsto {}^{t}g^{-1}$ of $G$ gives rise to measure-preserving
homeomorphism of $G_{A}/G_{k}$ with itself, we see that
$$
I(\Phi)=\int\limits_{G_{A}/G_{k}}\left(\sum_{\xi \in
  X_{k}}\Phi(g\xi)\right)|dg|_{A}
$$
satisfies the relation
\begin{equation*}
I(\Phi_{t})=|t|^{-n}_{A}I(\hat{\Phi}_{t-1})\tag{126}
\label{chap4:sec4:subsec1:eq126} 
\end{equation*}
where $t\in k^{\times}_{A}$ and $\Phi_{t}(x)=\Phi(tx)$. For $t=1$, this
relation reduces to the invariance of $I(\Phi)$ under the Fourier
transformation $\Phi\to \hat{\Phi}$. The idea of Mars is that,
without looking for an analogue of Theorem (iv) or rather of the
invariance under $\Phi\to \hat{\Phi}$ of
$$
\sum_{i^{\ast}\in k^{\times}}\int\limits_{X_{A}}\Phi(x)\psi(i^{\ast}f(x))|dx|_{A},
$$
one\pageoriginale may consider, instead, the relation
\begin{equation*}
\lim\limits_{|t|_{A}\to \infty}I(\Phi_{t})=\lim\limits_{|t|_{A}\to
  \infty}|t|^{-n}_{A}I(\hat{\Phi}_{t^{-1}})\tag{127}
 \label{chap4:sec4:subsec1:eq127} 
\end{equation*}
after expressing both sides of (\ref{chap4:sec4:subsec1:eq126}) by ``orbital integrals''
and incorporating the Poisson formula. For instance, if we now go back
to the case of a quadratic form $f(x)$ in $n>4$ variables, then
(\ref{chap4:sec4:subsec1:eq127}) will yield either the relation
$\tau_{k}(G)\Phi(0)=2\Phi(0)$ or the relation
$\tau_{k}(G)\Phi(0)=\Phi(0)$ for every $\Phi$ in $\mathscr{S}(X_{A})$,
according as $G=SO(f)$ or $\Spin (f)$. (In the latter case, we need,
of course, to assume correspondingly, that $\tau_{k}(\Spin (f))=1$ for
$n=3$, $4$).

\subsection{Other Siegel Formulas}\label{chap4:sec4:subsec2} %%%% subsec{4.2}

We now indicate, in some detail, how one may obtain a Siegel formula
in the general case; actually in the cases covered by A.3), A.4) of
\S\ \ref{chap4:sec2:subsec3} such a generalisation exists. Using Theorem
\ref{chap4:sec4:subsec1:thm1}, we have, for every $\Phi$ in $\mathscr{S}(X_{A})$,
\begin{equation*}
  \int\limits_{G_{A}/G_{k}}\left(\sum_{\xi\in
    U_{k}}\Phi(g\xi)\right)|dg|_{A}=\int\limits_{U_{A}}\in
  \Phi|\theta_{U}|_{A}\tag{128} \label{chap4:sec4:subsec2:eq128}
\end{equation*}
where $\theta_{U}$ is a gauge form on the homogeneous space $U$ and
further, $\epsilon(x)=\tau_{k}(H_{\alpha})$ if $x$ is in $G_{A}\cdot
\xi_{\alpha}$. In all cases, we know that $U(i)$ is a $G$-orbit for
every $i$ in $k$ and further, $f^{-1}(0)$ is a union of finitely many
$G$-orbits. From (\ref{chap4:sec4:subsec2:eq128}), we see that
\begin{equation*}
  I(\Phi)=\tau_{k}(G)\Phi(0)+\sum_{V}\int\limits_{V_{A}}\in
  \Phi|\theta_{V}|_{A}+\sum_{i\in
    k}\int\limits_{U(i)_{A}}\in\Phi|\theta_{i}|_{A}\tag{129}
  \label{chap4:sec4:subsec2:eq129}
\end{equation*}
letting $V$ run over all $G$-orbits in $f^{-1}(0)$ other than $U(0)$
and $\{0\}$. Further, we also have the Poisson formula 
$$
\sum_{i\in k}\int\limits_{U(i)}\Phi|\theta_{i}|_{A}=\sum_{i^{\ast}\in
  k}\int\limits_{X_{A}}\Phi(x)\psi(i^{\ast}f(x))|dx|_{A}.
$$\pageoriginale
Let us define the ``completed Eisenstein-Siegel series'' $E(\Phi)$ for
$\Phi$ in $\mathscr{S}(X_{A})$ by
$$
E(\Phi)=\Phi(0)+\sum_{V}\int\limits_{V_{A}}\Phi|\theta_{V}|_{A}+\sum_{i^{\ast}\in
  k}\int\limits_{X_{A}}\Phi(x)\psi(i^{\ast}f(x))|dx|_{A}
$$
where $V$ runs over the same $G$-orbits as in
(\ref{chap4:sec4:subsec2:eq129}). Then a 
(conjectural) {\em Siegel formula} in the general case, is the simple
(- looking) relation that, for every $\Phi$ in $\mathscr{S}(X_{A})$,
\begin{equation*}
  I(\Phi)=E(\Phi).\tag{130}\label{chap4:sec4:subsec2:eq130} 
\end{equation*}

In the case covered by A.3) of \S\ \ref{chap4:sec2:subsec3}, it is
known that 
$\tau_{k}(G)=\epsilon(x)=1$ and therefore, the Siegel formula
(\ref{chap4:sec4:subsec2:eq130}) is indeed valid. As for A.4) and
A.5), we apply the method 
of Mars explained in \S\ \ref{chap4:sec4:subsec1}. Then we do get
$\tau_{k}(G)=\epsilon(x)=1$ as far as A.4) is concerned and therefore
the Siegel formula (\ref{chap4:sec4:subsec2:eq130}) is valid here. Similarly, we can uphold
the Siegel formula also for A.5), provided, however, that
$\epsilon(x)=1$ \ie if the above-mentioned conjecture of Weil is
verified for the stabilisers.

\begin{Remark*}
  For any $u\in k_{A}$, $\Phi(x)\to \psi(uf(x))\Phi(x)$ represents a
  unitary operator of $\mathscr{S}(X_{A})$. The unitary operators of
  $\mathscr{S}(X_{A})$ defined by
  $$
  \Phi(x)\to \Psi(x)=\psi(uf(x))\hat{\Phi}(x)\to \hat{\Phi}(-x)
  $$
  which are obtained by conjugating unitary operators above with respect
  to $\Phi\to \hat{\Phi}$, generate, together, with them, what might
  be called a {\em metaplectic group} associated with\pageoriginale
  $f(x)$. Let us denote this group by $Mp_{A}$ and its subgroup obtained
  by restricting $u$ to $k$, by $Mp_{k}$. Then $I(\Phi)$ is invariant
  under the action of $Mp_{k}$. Therefore, if the Siegel formula
  (\ref{chap4:sec4:subsec2:eq130}) is valid, then the correspondence
  $$
  Mp_{A}\ni \mathbb{S}\to E(\mathbb{S}\Phi)\in\mathbb{C}
  $$
  defines a continuous function on $Mp_{A}$ which is left invariant or
  ``automorphic'' under $Mp_{k}$. In the case when $f(x)$ is a quadratic
  form and $Q(x,y)=f(x,y)$, $Mp_{A}$ is just the {\em metaplectic group
    of Weil}; the invariance of $E(\mathbb{S}\Phi)$ under
  $Mp_{k}=SL_{2}(k)$ gives the adelic version of the automorphic
  behaviour of the classical Eisenstein series.
\end{Remark*}

\subsection{A Final Comment}\label{chap4:sec4:subsec3} % %%%%%%% subsec4.3

In the case covered by A.1) of \S\ \ref{chap4:sec2:subsec3} \ie when $f(x)$ is a
strongly non-degenerate form of degree $m\geq 2$ in $n>2m$ variables
with $m$ not divisible by the characteristic of $k$, there is no
Siegel formula in the strict sense. In this case, we still take
$$
E(\Phi)=\Phi(0)+\sum_{i^{\ast}\in
  k}\int\limits_{X_{A}}\Phi(x)\psi(i^{\ast}f(x))|dx|_{A} 
$$
as the completed Eisenstein-Siegel series but there is no group over
which we can take the average of
$$
I_{0}(\Phi)=\sum_{\xi\in X_{k}}\Phi(\xi).
$$
However, at least a subgroup of what might be called a metaplectic
group is readily available. In fact, we convert the product
$P=\mathbb{G}_{a}\times \mathbb{G}_{m}$ into an algebraic group by
defining
$$
(u,t)(u',t')=(u+t^{m}u',tt').
$$
(Here\pageoriginale $\mathbb{G}_{m}$ is the multiplicative group
$\mathbb{G}_{a}\backslash \{0\}$). For $(u,t)\in P_{A}=k_{A}\times
k^{\times}_{A}$, and $\Phi$ in $\mathscr{S}(X_{A})$, we define
$$
((u,t)\cdot \Phi)(x)=|t|^{n/2}_{A}\psi(uf(x))\Phi(tx).
$$
Then $(I_{0}-E)$ $((u,t)\cdot\Phi)$ becomes a $P_{k}$-invariant
continuous function on $P_{A}$ and we can show that it vanishes at
every ``$k$-rational boundary point'' of $P_{A}$ in the following
sense:
$$
(I_{0}-E)((u,t)\cdot\Phi)=O(|t|^{m-n/2}_{A})\text{ \  as \ }
|t|_{A}\to \infty
$$
and moreover,
$$
(I_{0}-E)((u,t)\cdot\Phi)=O(|t|^{n/2-m}_{A})\text{ \ as \ } |t|_{A}\to
0
$$
but subject to the restriction that $(u+i^{\ast})t^{-m}$ remains in a
compact subset of $k_{A}$ for some $i^{\ast}$ in $k$. If we specialise
$f$ to be a quadratic form and $k$ to be $\mathbb{Q}$, the theorem
above gives the well-known behaviour of theta series as the variable
in the complex upper half-plane approaches the rational points on the
real axis. In the general case, the theorem above suggests that
$(I_{0}-E)((u,t)\cdot\Phi)$ remains bounded as $|t|_{A}$ tends to
$0$. We have observed that this conjecture will have as its
consequence a generalisation of the Minkowski-Hasse theorem (namely
$U(i)_{A}\neq \phi$ implying that $U(i)_{k}\neq\phi$). For the
details, we refer to our paper \cite{Igu 5}.

\section{Siegel's Main Theorem for Quadratic
  Forms}\label{chap4:sec5}  %%%% \label{chap4-sec5}

As it is well-known, the main theorem of Siegel in the analytic theory
of quadratic forms is equivalent to the existence of an identity
between a certain weighted average of theta series and an Eisenstein
series, when one considers non-degenerate\pageoriginale integral
quadratic forms in more than four variables. We shall see in this
article how this identity can be derived from the Siegel formula for
quadratic forms due to Weil, quite explicitly in the positive-definite
case.

\subsection{Siegel's Main Theorem}\label{chap4:sec5:subsec1} %%%% subsec {5.1} 

For any matrix $M$, we denote its transpose by ${}^{t}M$ and its
determinant (whenever it makes sense) by det $M$. If $M$ is a real
$n$-rowed symmetric non-singular matrix, we say that $M$ is of {\em
  signature} $(p,q)$ if exactly $p$ eigenvalues of $M$ are positive
(and consequently $q=n-p$ eigenvalues are negative); $M$ is called
{\em positive-definite} or {\em indefinite} according as $q=0$ or
$pq>0$.

Two $n$-rowed integral symmetric matrices $T$, $T'$ are said to be
{\em in the same} (equivalence) {\em class} (respectively {\em narrow
  class}) if $T={}^{t}AT'A$ for an integral matrix $A$ of determinant
$\pm 1$ (respectively $1$). We say that two non-singular $n$-rowed
integral symmetric matrices $T$, $T'$ of the same signature {\em
  belong to the same genus}, if, for every prime number $p$, there
exists a matrix $A_{p}$ with entries in the ring $\mathbb{Z}_{p}$ of
$p$-adic integers and of determinant $1$, such that $T={}^{t}ApT'Ap$;
we remark that this notion of a genus is the same as that of Siegel,
in view of the fact every symmetric matrix $M$ over $\mathbb{Z}_{p}$
admits an $A$ over $\mathbb{Z}_{p}$ such that ${}^{t}AMA=M$ and $\det
A=-1$ (See \cite{Sie 1}, I, Hilfssatz 19]. Clearly any two $T$, $T'$ as
  above which are in the same class, belong to the same genus as well;
  thus the genus of $T$ splits into classes. From the reduction theory
  of quadratic forms due to Minkowski, Hermite and Siegel, it is known
  that the genus of any non-singular integral symmetric matrix $T$
  consists of only a finite number of classes and therefore, only
  finitely many narrow classes. For the narrow classes, we choose a
  complete set of representatives say $T_{1}(=T)$,
  $T_{2},\ldots,T_{r}$. We may further\pageoriginale assume that if,
  for one of these $T_{i}$, there exists no integral matrix $A$ of
  determinant $-1$ with ${}^{t}AT_{i}A=T_{i}$, then, along with
  $T_{i}$, the matrix $T^{\ast}_{i}={}^{t}DT_{i}D$ also occurs among
  the $r$ representatives for a fixed integral $D$ of determinant
  $-1$; we may take, for example, $D$ to be a diagonal matrix with
  $-1$ as its first diagonal element and $1$ as the remaining diagonal
  elements. To each $T_{i}$, $1\leq i\leq r$, we associate the
  quadratic form $f_{i}(x)=1/2{}^{t}xT_{i}x$, where $x$ now denotes
  the $n$-rowed column with elements $x_{1},\ldots,x_{n}$; further, we
  write $f(x)$ instead of $f_{1}(x)$.

The special orthogonal group $SO(f)$ of $f(x)=1/2{}^{t}xTx$, consists
of all matrices $A$ with ${}^{t}ATA=T$ and $\det A=1$. It acts, in a
natural fashion, on the affine $n$-space $X$. Denoting $SO(f)$, for
the present, by $G$, it is known that $G$ is a semi-simple linear
algebraic group (for $n>2$) defined over $\mathbb{Q}$; the adele group
$G_{A}$ has the property that $G_{A}/G_{\mathbb{Q}}$ has finite
measure. We assume that the measure $\mu$ on $G_{A}$ is normalized so
that
\begin{equation*}
  \mu(G_{A}/G_{\mathbb{Q}})=1.\tag{131}\label{chap4:sec5:subsec1:eq131}
\end{equation*}
The Siegel formula for orthogonal groups due to Weil may now be stated
once again: namely, for $n>4$ and for any $\Phi$ in $\mathscr{S}(X_{A})$,
we have
\begin{equation*}
  \int\limits_{G_{A}/G_{\mathbb{Q}}}\left(\sum_{\xi\in
    X_{\mathbb{Q}}}\Phi(g\xi)\right)d\mu(g)=\Phi(0)+\sum_{i^{\ast}
    \in\mathbb{Q}}\int\limits_{X_{A}}\Phi(x)\psi(i^{\ast}f(x))|dx|_{A}. 
  \tag{132} \label{chap4:sec5:subsec1:eq132} 
\end{equation*}

Let $T$ be positive-definite, unless otherwise stated and let
$T_{1},\ldots,T_{r}$ be the above-mentioned representatives of the
narrow classes in the genus of $T$. Corresponding to each
$f_{i}(x)=1/2{}^{t}xT_{i}x$ and a complex variable $\tau$ with
$\Iim(\tau)>0$, let us define the theta series $v_{i}$ by 
$$
\nu_{i}(\tau)=\sum_{\xi\in\mathbb{Z}^{n}}\mathfrak{e}(\tau
f_{i}(\xi)).
$$\pageoriginale 
It is not hard to show that this series converges absolutely,
uniformly when $\Iim \tau\geq \epsilon>0$. Thus $v_{i}(\tau)$ is a
holomorphic function of $\tau$ for $\Iim(\tau)>0$ and furthermore,
$v_{i}(\tau)$ depends only on the class of $T_{i}$. For $T_{i}$ and
$T^{\ast}_{i}$, in particular, we have the same $\nu_{i}(\tau)$. Let
$e_{i}$ (respectively $e^{+}_{i}$) denote the number of integral
matrices $A$ of determinant $\pm 1$ (respectively $1$) such that
${}^{t}AT_{i}A=T_{i}$; then $e_{i}$ is obviously finite and at least
equal to $2$. Moreover, $e_{i}=\delta_{i}e^{+}_{i}$ where
$\delta_{i}=2$ or $1$ according as $T_{i}$ admits an integral $A$ with
${}^{t}AT_{i}A=T_{i}$, $\det A=-1$ or otherwise. With our
understanding above, only one of $T_{i}$ or $T^{\ast}_{i}=T_{i}[D]$
occurs among the $r$ representatives if $\delta_{i}=2$, while both of
them occur for $\delta_{i}=1$.

The analytic formulation of the main theorem of Siegel for
representation of integers by $T$ for $n>4$ may now be given: namely,
\begin{multline*}
  \left(\sum^{r}_{i=1}\nu_{i}(\tau)/e^{+}_{i}\right)/
  \left(\sum^{r}_{i=1}1/e^{+}_{i}\right)\\
  =1+\dfrac{\mathfrak{e}(n/8)}{\sqrt{\det
      T}}\sum_{c,d}c^{-n/2}\left(\sum_{\xi \mod
    c}\mathfrak{e}\left(-\frac{d}{c}f(\xi)\right)\right)
  (c\tau+d)^{-n/2}\tag{133}\label{chap4:sec5:subsec1:eq133} 
\end{multline*}
where, on the right hand side, the summation is over all pairs of
coprime integers $c$, $d$ with $c\geq 1$ and further, with
$cdf(\xi)\in\mathbb{Z}$ for all $\xi\in \mathbb{Z}^{n}$. In view of
our remarks above, it is not difficult to check that the left hand
side of (\ref{chap4:sec5:subsec1:eq133}) is the same as the ``analytic genus-invariant'' of
$T$ in the sense of Siegel; formula (\ref{chap4:sec5:subsec1:eq133}) is thus the same as
formula (\ref{chap3:sec2:subsec5:eq83}) in the above-mentioned
fundamental paper  (\cite{Sie 1}, I).

\subsection{}\label{chap4:sec5:subsec2}%%% subsec{5.2}

We wish to prove that the Siegel formula (\ref{chap4:sec5:subsec1:eq132}) for
positive-definite integral quadratic forms yields the analytic
identity (\ref{chap4:sec5:subsec1:eq133}), if we merely specialise $\Phi$ in\pageoriginale
(\ref{chap4:sec5:subsec1:eq132}) suitably. But first, let us refer to another familiar
notion of ``classes in a genus'' associated with an arbitrary linear
algebraic group over a global field.

Let $G$ be a linear algebraic group defined over a global field $k$ of
characteristic $0$. The adele group $G_{A}$ contains only finitely
many distinct double cosets modulo $G_{\Omega}$, $G_{k}$ where
$G_{\Omega}$ is just the open subgroup denoted earlier by
$G_{S_{\infty}}$:
\begin{equation*}
  G_{A}=\coprod^{r}_{i=1}G_{\Omega}g_{i}G_{k}
  \tag{134}\label{chap4:sec5:subsec2:eq134}
\end{equation*}
The number $r=r_{k}(G)$ is known sometimes as the class number of
$G$. We shall prove, in \S\ \ref{chap4:sec5:subsec6}, that
\begin{equation*}
  r_{\mathbb{Q}}(SL_{n})=1\tag{135}\label{chap4:sec5:subsec2:eq135}
\end{equation*}
Actually, more generally, even $r_{k}(SL_{n})=1$. If $G\subset GL(V)$
for a vector space $V$ of dimension $n$, then $G_{A}$ acts in a
natural fashion, on the ``lattices'' in $V_{k}$; the orbit of a
``lattice'' under $G_{A}$ (respectively $G_{k}$) constitutes the {\em
  genus} (respectively {\em class}) of the ``lattice'' and the number
of classes in a genus is finite.

Going back to the $n$-rowed integral positive-definite matrix $T$ once
again, we take $G=SO(f)$, for the associated $f(x)=1/2{}^{t}xTx$. From
(\ref{chap4:sec5:subsec2:eq134}) and
(\ref{chap4:sec5:subsec2:eq135}), we have 
\begin{equation*}
  (SL_{n})_{A}=(SL_{n})_{S_{\infty}}\cdot (SL_{n})_{\mathbb{Q}}.
  \tag*{$(135)'$}\label{chap4:sec5:subsec2:eq135_1}
\end{equation*}
Thus any $g$ in $SO(f)_{A}\subset (SL_{n})_{A}$ can be written as
\begin{equation*}
g=A\cdot C^{-1},\quad A\in(SL_{n})_{S_{\infty}},\quad C\in
(SL_{n})_{\mathbb{Q}}.\tag{136}\label{chap4:sec5:subsec2:eq136}
\end{equation*}
Since\pageoriginale ${}^{t}CTC={}^{t}(g^{-1}A)Tg^{-1}A={}^{t}ATA$, it
follows that ${}^{t}ATA$ is in the genus of $T$; its class depends
only on the double coset $G_{\Omega}gG_{\mathbb{Q}}$. Conversely, if
$T'$ is in the genus of $T$, then, in $(GL_{n})_{A}$, $T'={}^{t}ATA$
for some $A$ in $(SL_{n})_{S_{\infty}}$; by the Minkowski-Hasse
theorem, we know then that $T'={}^{t}CTC$ for some $C$ in
$(SL_{n})_{\mathbb{Q}}$ and thus $T'$ corresponds to the double coset
$G_{\Omega}(AC^{-1})G_{\mathbb{Q}}$. Therefore, for our $r$ (narrow)
class-representatives $T_{1},\ldots,T_{r}$ above, we may take
$T_{i}={}^{t}A_{i}TA_{i}={}^{t}C_{i}TC_{i}$, $1\leq i\leq r$
corresponding to $g_{i}=A_{i}C^{-1}_{i}$ as in
(\ref{chap4:sec5:subsec2:eq136}) and the
double coset decomposition {134} for $SO(f)_{A}$. From
(\ref{chap4:sec5:subsec1:eq131}), (\ref{chap4:sec5:subsec2:eq134}) and
the invariance of $\mu$, we obtain 
\begin{align*}
1 &= \sum^{r}_{i=1}\mu(G_{\Omega}~g_{i}G_{\mathbb{Q}}/G_{\mathbb{Q}})\\
&=
\sum^{r}_{i=1}\mu(g^{-1}_{i}G_{\Omega}~g_{i}G_{\mathbb{Q}}/G_{\mathbb{Q}})\\
&= \sum^{r}_{i=1}\mu(g^{-1}_{i}G_{\Omega}~g_{i}/(G_{\mathbb{Q}}\cap
g^{-1}_{i}G_{\Omega}~g_{i}))\\
&= \sum^{r}_{i=1}\mu(G_{\Omega}/(g_{i}G_{\mathbb{Q}}g^{-1}_{i}\cap
G_{\Omega}))\\
&= \mu(G_{\Omega})\sum^{r}_{i=1}1/e^{+}_{i}
\end{align*}
in view of the fact that $g_{i}G_{\mathbb{Q}}g^{-1}_{i}\cap
G_{\Omega}\simeq A^{-1}_{i}G_{\Omega}A_{i}\cap
C^{-1}_{i}G_{\mathbb{Q}}C_{i}=SO(f_{i})_{\mathbb{Z}}$ is of order
$e^{+}_{i}$. Thus 
\begin{equation*}
  \mu(G_{\Omega})=1/\left(\sum^{r}_{i=1}1/e^{+}_{i}\right).
  \tag{137}\label{chap4:sec5:subsec2:eq137}
\end{equation*}

\subsection{}\label{chap4:sec5:subsec3} %%% subsec{5.3}

For $\Phi\in\mathscr{S}(X_{A})$ in (\ref{chap4:sec5:subsec1:eq132}), we take
$\Phi=\Phi_{0}\otimes \Phi_{\infty}$ in
$\mathscr{S}(X_{0})\bigotimes\limits_{\mathbb{C}}\mathscr{S}(X_{\infty})$,
following the notation of Chapter IV, \S\ \ref{chap4:sec1:subsec5}1.5, with 
\begin{align*}
\Phi_{0} &= \text{ the characteristic function of }
\prod_{p}X^{0}_{p}\text{ and }\\
\Phi_{\infty}(x_{\infty}) &= \mathfrak{e}(f(x_{\infty})\tau).
\end{align*}\pageoriginale
The left hand side of (\ref{chap4:sec5:subsec1:eq132}) becomes just
\begin{equation*}
  \sum^{r}_{i=1}\int\limits_{G_{\Omega}/(g_{i}G_{\mathbb{Q}}g^{-1}_{i}\cap
    G_{\Omega})}\left(\sum_{\xi\in
    X_{\mathbb{Q}}}\Phi(gg_{i}\xi)\right)d\mu(g)
  \tag{138}\label{chap4:sec5:subsec3:eq138} 
\end{equation*}
using the same arguments that led us to the proof of
(\ref{chap4:sec5:subsec2:eq137}). The
innermost series in (\ref{chap4:sec5:subsec3:eq138}) is the same as $\sum\limits_{\xi\in
  X_{\mathbb{Q}}}\Phi(g'A_{i}\xi)$ with $g'=(g'_{0},g'_{\infty})$ in
$G_{\Omega}$. If now we use the definition of $\Phi$, then the
last-mentioned series is seen to be precisely
$$
\sum_{\xi\in\mathbb{Z}^n}\mathfrak{e}(f(g'_{\infty}A_{i,\infty}\xi)\tau)=\sum_{\xi\in
  \mathbb{Z}^{n}}\mathfrak{e}(f_{i}(\xi)\tau)=\nu_{i}(\tau). 
$$
It is now immediate that (\ref{chap4:sec5:subsec3:eq138}) is exactly
$\mu(G_{\Omega})\sum\limits_{i=1}v_{i}(\tau)/e_{i}^{+}$ and
therefore, by (\ref{chap4:sec5:subsec2:eq137}), the left hand side of
(\ref{chap4:sec5:subsec1:eq132}) is the same as that of
(\ref{chap4:sec5:subsec1:eq133}).

\subsection{}\label{chap4:sec5:subsec4}%%% subsec{5.4}

What remains to be proved then is only that
(\ref{chap4:sec5:subsec1:eq132}) has the same
right hand side as (\ref{chap4:sec5:subsec1:eq133}), for $\Phi$ as chosen as above. Since
$\Phi(0)=1$, we are reduced to showing that the series on the right
hand side of (\ref{chap4:sec5:subsec1:eq132}) is identical with the
series over $(c,d)$ in (\ref{chap4:sec5:subsec1:eq133}). 

Since $(k_{A}/k)^{\ast}\simeq k$, the right hand side is independent
of the choice of $\psi$ and therefore we may assume that
\begin{align*}
  & \psi_{p}:Q_{p}\to Q_{p}/\mathbb{Z}_{p}\hookrightarrow
  \mathbb{R}/\mathbb{Z}\simeq \mathbb{C}^{\times}_{1}\quad\text{and}\\
  & \psi_{\infty}(x_{\infty})=\mathfrak{e}(-x_{\infty}). 
\end{align*}

Any\pageoriginale $i^{\ast}$ in $\mathbb{Q}$ may be written as $-d/c$
with coprime $c$, $d$ in $\mathbb{Z}$ and further $c\geq 1$. The
integral over $X_{A}$ in \ref{chap4:sec5:subsec1:eq132} becomes
\begin{equation*}
  \left(\prod_{p}\int\limits_{X^{0}_{p}} \psi_{p}
  \left(-\frac{d}{c}f(x) \right)|dx|_{p}\right)\cdot
  \int\limits_{X_{\infty}}\mathfrak{e}(f(x)(\tau+d/c))
  |dx|_{\infty}.\tag{139}\label{chap4:sec5:subsec4:eq139} 
\end{equation*}
The integral over $X^{0}_{p}$ above may be rewritten as 
\begin{equation*}
  \sum_{\xi\mod  c\mathbb{Z}p} ~ \int\limits_{c\mathbb{Z}^{n}_{p}}
  \psi_{p} \left(-\frac{d}{c}f(\xi+x)\right)|dx|_{p}.
  \tag{140}\label{chap4:sec5:subsec4:eq140} 
\end{equation*}
Since
$\psi_{p}\left(-\dfrac{d}{c}f(\xi+x)\right)=\psi_{p}\left(-\dfrac{d}{c}f(\xi)\right)\psi_{p}\left(-\dfrac{d}{c}f(x)\right)$,
the integral over $c\mathbb{Z}^{n}_{p}$ in
(\ref{chap4:sec5:subsec4:eq140}) is (upto the
factor $\psi_{p}\left(-\dfrac{d}{c}f(\xi)\right)$ equal to
$$
|c|^{n}_{p}\int\limits_{\mathbb{Z}^{n}_{p}}\psi_{p}(-cdf(x))|dx|_{p}=
\begin{cases}
|c|^{n}_{p}\text{ if } cdf(x)\text{ is } \mathbb{Z}_{p}-\text{valued
  on } \mathbb{Z}^{n}_{p}\\
0, \text{ otherwise}
\end{cases}
$$
on noting that $\psi_{p}(-cdf(x))$ is a character of
$\mathbb{Z}^{n}_{p}$. Hence the infinite product over primes $p$ in
(\ref{chap4:sec5:subsec4:eq139}) is precisely
\begin{equation*}
  c^{-n}\sum_{\xi\mod
    c}\mathfrak{e}\left(-\frac{d}{c}f(\xi)\right)
  \tag{141}\label{chap4:sec5:subsec4:eq141} 
\end{equation*}
assuming, of course, the $cdf(x)$ is $\mathbb{Z}$-valued on
$\mathbb{Z}^{n}$. On the other hand, the integral over $X_{\infty}$ in
(\ref{chap4:sec5:subsec4:eq139}) is clearly equal to 
\begin{align*}
  & \int\limits_{\mathbb{R}^{n}} \mathfrak{e}(1/2^{t}(Mx) (Mx)
  (\tau+d/c)) |dx|_{\infty}\text{ with } {}^{t}MM=T\\
  &= |\det  M|^{-1}\left(\int\limits_{\mathbb{R}} \mathfrak{e}(1/2
  (\tau+d/c) t^{2})dt\right)^{n}\\ 
  &=(\det T)^{-1/2}\left(\sqrt{i/(\tau+d/c)}\right)^{n}\\
  &= (\det
  T)^{-1/2}\mathfrak{e}(n/8)c^{n/2}(c\tau+d)^{-n/2}
  \tag{142}\label{chap4:sec5:subsec4:eq142} 
\end{align*}

Taking\pageoriginale into account (\ref{chap4:sec5:subsec4:eq139}),
(\ref{chap4:sec5:subsec4:eq140}), (\ref{chap4:sec5:subsec4:eq141}) and
(\ref{chap4:sec5:subsec4:eq142}), it is clear that our assertion at the
beginning of \S\ \ref{chap4:sec5:subsec4}{5.4} is established. With this, the proof of the
fact that the Siegel formula (\ref{chap4:sec5:subsec1:eq132}) implies
(\ref{chap4:sec5:subsec1:eq133}) is also complete.

\subsection{}\label{chap4:sec5:subsec5} %%%% subsec{5.5}

If we consider an $n$-rowed integral symmetric matrix $T$ of signature
$(p,q)$ with $0<q=n-p<n$, then Siegel's main theorem takes the
following form. For the special orthogonal group $G=SO(f)$ with
$f(x)=1/2{}^{t}xTx$, $G_{\infty}$ is no longer compact;
$G_{\mathbb{Z}}$ is no more finite. One now considers the space
$\mathscr{H}$ of all the {\em majorants} $M$ ot $T$, which are just
$n$-rowed symmetric positive-definite matrices such that
$M^{-1}SM^{-1}S$ is equal to the $n$-rowed identity matrix. The group
$G_{\infty}$ acts on $\mathscr{H}$ via the maps $M\mapsto
{}^{t}g_{\infty}Mg_{\infty}$ for $g_{\infty}\in G_{\infty}$ and indeed
transitively; the stabiliser of any point in $\mathscr{H}$ is a
maximal compact subgroup of $G_{\infty}$. Let $\mu$ be a measure on
$\mathscr{H}$ such that $\mu(\mathscr{H}/G_{\mathbb{Z}})=1$. The main
theorem of Siegel (\cite{Sie 3}) now reads: for $n>4$,
\begin{multline*}
  \int\limits_{\mathscr{H}/G_{\mathbb{Z}}}\left(\sum_{\xi\in
    X_{\mathbb{Z}}}\mathfrak{e}(f(\xi)\text{Re } (\tau)+i{}^{t}\xi M\xi\Iim
  (\tau))\right)d\mu(M)\\ 
  =1+|\det
  T|^{-1/2}\oplus \left(\frac{p-q}{8}\right)\sum_{c,d}c^{-n/2}\\
  \left(\sum_{\xi\mod c}\oplus \left(-\dfrac{d}{c}f(\xi)\right) 
  (c\tau+d)^{-p/2}(c\ob{\tau}+d)^{-q/2}\right) 
\end{multline*}
where $c$, $d$ run over all pairs of coprime integers with $c\geq 1$
and $cdf(\xi)\in\mathbb{Z}$ for all $\xi\in\mathbb{Z}^{n}$. This
formula can again be derived from (\ref{chap4:sec5:subsec1:eq132}), with a proper choice of
$\Phi$; actually we may take $\Phi=\Phi_{0}\otimes\Phi_{\infty}$ where
$\Phi_{0}$ is, as before, the characteristic function of
$\prod\limits_{p}X^{0}_{p}$ but
$\Phi_{\infty}(x_{\infty})=\exp(-2\pi{}^{t}x_{\infty}Mx_{\infty})$.

\subsection{}\label{chap4:sec5:subsec6}%%% subsec{5.6}

We now give an indirect but hopefully instructive proof of the
assertion \ref{chap4:sec5:subsec2:eq135_1} used in the course of our
arguments in \S\ \ref{chap4:sec5:subsec2}. 

First,\pageoriginale we give an outline of the proof of the fact that
the Tamagawa number 
\begin{equation*}
  \tau_{k}(SL_{n})=\int\limits_{(SL_{n})_{A}/(SL_{n})_{k}}|dg|_{A}=1
  \tag{143}\label{chap4:sec5:subsec5:eq143} 
\end{equation*}
for any $n\geq 1$ and any global field $k$. Since the assertion
(\ref{chap4:sec5:subsec5:eq143})
 is clear for $n=1$, assume that it has been proved with $r\leq n-1$
 in place of $n$, for $n>1$. Let us now take $G$ to be $SL_{n}$ and
 $X$ to be the same as the space $M_{n,1}$ of $n$-rowed vectors. For
 any field $K$, we have
$$
X_{K}\backslash \{0\}=G_{K}\cdot e_{1}
$$
where now $e_{1}$ is the unit vector with $1$ as its first element and
$0$ elsewhere. The stabiliser $H$ in $G$ of the vector $e_{1}$ is a
semi-direct product of $SL_{n-1}$ and the $(n-1)$-fold product of the
additive group $\mathbb{G}_{a}$. Hence the induction hypothesis yields
$\tau_{k}(H)=1$. For any $\Phi$ in $\mathscr{S}(X_{A})$, it can be shown
that
\begin{equation*}
  \int\limits_{G_{A}/G_{k}}\left(\sum_{\xi\in
    X_{k}}\Phi(g\xi)\right)|dg|_{A}=\tau_{k}(G)\Phi(0)
  +\tau_{k}(H)\Phi^*(0)\tag{144}\label{chap4:sec5:subsec5:eq144} 
\end{equation*}
where $\Phi^{\ast}$ is the Fourier transform of $\Phi$. The left hand
side of (\ref{chap4:sec5:subsec5:eq144}) can be seen to be invariant under $\Phi\mapsto
\Phi^{\ast}$, in view of the (classical) Poisson formula
(\ref{chap4:sec1:subsec6:eq115}). But then the invariance of the right
hand side under 
$\Phi\mapsto \Phi^{\ast}$ implies that $\tau_{k}(G)=\tau_{k}(H)=1$,
which establishes (\ref{chap4:sec5:subsec5:eq143}).

Coming back to the proof of \ref{chap4:sec5:subsec2:eq135_1}, we
need, in view of (\ref{chap4:sec5:subsec1:eq133}), 
only to show that the measure of
$$
(SL_{n})_{S_{\infty}}(SL_{n})_{\mathbb{Q}}/(SL_{n})_{\mathbb{Q}}=\prod_{p}SL_{n}(\mathbb{Z}_{p})\times
SL_{n}(\mathbb{R})/SL_{n}(\mathbb{Z}) 
$$
is\pageoriginale precisely $1$. It is known that
$$
\mu(SL_{n}(\mathbb{Z}_{p}))=(1-p^{-n})\ldots (1-p^{-2})
$$
and therefore
$$
\mu\left(\prod_{p}SL_{n}(\mathbb{Z}_{p})\right)=1/\prod^{n}_{i=2}\zeta (i)
$$
where $\zeta$ is the Riemann zeta function. In order to complete the
proof of \ref{chap4:sec5:subsec2:eq135_1}, we have only to show that
\begin{equation*}
  \mu(G_{\mathbb{R}}/G_{\mathbb{Z}})=\mu(H_{\mathbb{R}}/H_{\mathbb{Z}})
  \cdot\zeta(n).\tag{145}\label{chap4:sec5:subsec5:eq145} 
\end{equation*}
Now, analogous to (\ref{chap4:sec5:subsec5:eq144}), we also have the
relation 
\begin{equation*}
  \int\limits_{G_{\mathbb{R}}/G_{\mathbb{Z}}}\left(\sum_{\xi}
  \Phi_{\infty}(g\xi)\right)|dg|_{\infty}=\mu(H_{\mathbb{R}}/
  H_{\mathbb{Z}})\Phi^{\ast}_{\infty}(0)
  \tag{146}\label{chap4:sec5:subsec5:eq146}
\end{equation*}
for $\Phi_{\infty}$ in $\mathscr{S}(X_{\infty})$, where $\xi$ runs over
all ``primitive vectors'' \ie over all the elements of the orbit
$SL_{n}(\mathbb{Z})e_{1}$ of the vector $e_{1}$ above. We may assume
that $\Phi_{\infty}\geq 0$, $\Phi_{\infty}\neq 0$, so that
$\Phi^{\ast}_{\infty}(0)>0$. If we replace $\Phi_{\infty}(x)$ in
(\ref{chap4:sec5:subsec5:eq146}) by $\Phi_{\infty}(tx)$ for $t>0$,
then we obtain, instead, 
\begin{equation*}
  \int\limits_{G_{\mathbb{R}}/G_{\mathbb{Z}}}\left(\sum_{\xi}\Phi_{\infty}(gt\xi)
  \right)|dg|_{\infty}=\mu(H_{\mathbb{R}}/H_{\mathbb{Z}})t^{-n}
  \Phi^{\ast}_{\infty}(0).\tag{147}\label{chap4:sec5:subsec5:eq147} 
\end{equation*}
Taking $t=1,2,3,\ldots$ and summing up both sides of
(\ref{chap4:sec5:subsec5:eq147}) over $t$, we get
\begin{equation*}
  \int\limits_{G_{\mathbb{R}}/G_{\mathbb{Z}}}\left(\sum_{\xi\in
    X_{\mathbb{Z}}
    \backslash\{0\}}\Phi_{\infty}(g\xi)\right)|dg|_{\infty}=\mu
  (H_{\mathbb{R}}/H_{\mathbb{Z}})\zeta(n)\Phi^{\ast}_{\infty}(0).
  \tag{148}\label{chap4:sec5:subsec5:eq148}
\end{equation*}
Replacing $\Phi_{\infty}(x)$ in (\ref{chap4:sec5:subsec5:eq148}) by
$\Phi_{\infty}(t'x)$ for 
$t'>0$ and letting $t'$ tend to $0$, we have 
$$
\int\limits_{G_{\mathbb{R}}/G_{\mathbb{Z}}}\left(\int\limits_{X_{\mathbb{R}}}\Phi_{\infty}(gx)|dx|_{\infty}\right)|dg|_{\infty}=\mu(H_{\mathbb{R}}/H_{\mathbb{Z}})\zeta(n)\Phi^{\ast}_{\infty}(0).
$$\pageoriginale 
But now the left hand side is the same as
$\mu(G_{\mathbb{R}}/G_{\mathbb{Z}})\Phi^{\ast}_{\infty}(0)$ and
{145} is proved.

\section[Proof of the Criterion for the Validity...]{Proof of the Criterion for the Validity of the Poisson
  Formula}\label{chap4:sec6}%%% sect6 

We devote this section to giving an outline of the proof of Theorem
\ref{chap4:sec2:subsec1:thm1}. 

Let us recall that $k$ is a global field, $\psi$, a non-trivial
character of $k_{A}/k$, $f(x)\in k[x_{1},\ldots,x_{n}]$ a homogeneous
polynomial of degree $m\geq 2$. We had further assumed that $m$ is not
divisible by the characteristic of $k$ and that a tame $k$-resolution
of the projective hypersurface defined by $f(x)=0$ exists. Further,
the two criteria stated are:
\begin{itemize}
\item[(C1)] $r=\codim_{f^{-1}(0)}C_{f}\geq 2$; and

\item[(C2)] there exist $\sigma>2$ and a finite set $S$ of valuations
  $v$ of $k$, such that, for every $i^{\ast}$ in $k_{v}\backslash
  R_{v}$ and $v\not\in S$, we have
$$
|F^{\ast}_{v}(i^{\ast})|=\left|\int\limits_{X^{0}_{v}}\psi_{v}(i^{\ast}f(x))|dx|_{v}\right|\leq
|i^{\ast}|^{-\sigma}_{v} 
$$
\end{itemize}
We have to show that (C1) and (C2) imply the validity of
(PF-1),\ldots,(PF-3) and (PF-4).

\subsection{Geometric Part of the Proof}\label{chap4:sec6:subsec1} %%% subsec6.1

Let $D$ be a positive divisor on an irreducible non-singular algebraic
variety $X$ and $h:Y\to X$ be a $K$-resolution of $D$ for any
arbitrary field $K$. Let $\{(N_{i},\nu_{i})\}_{i}$ denote the
numerical date of $h$ at a point $b$ in $Y$, in the sense of Chapter
III, \S\ \ref{chap3:sec2:subsec2}. We say that $h$ has property $(P_{0})$, at $b$, if
$\nu_{i}\geq N_{i}$ for every $i$ and\pageoriginale if, further, the
equality holds for at most one $i$. If, moreover, in case such
equality holds, say $\nu_{i_{0}}=N_{i_{0}}$, we have
$\nu_{i_{0}}=N_{i_{0}}=1$, then $h$ is said to {\em have property
  $(P)$ at $b$.} It may happen that $h$ has property $(P_{0})$ at
every point of $Y_{K}$ but does not have property $(P)$ at some point
of $Y_{K}$. We have, however, the following

\begin{lemma}\label{chap4:sec6:subsec1:lem1} %%% \label{lem6.1}
  If $h$ (is tame and) has property $(P_{0})$ everywhere, then $h$ has
  necessarily property $(P)$ everywhere.
\end{lemma}

\begin{proof}
  By restricting $h$ to a Zariski open neighbourhood of $h^{-1}(h(b))$
  for a point $b$ in $Y$, we may assume that $X$ is affine and hence
  that $D=(g)$ for some function $g$ regular everywhere on $X$;
  moreover, we may suppose that there exists a ``gauge form'' $dx$ on
  $X$, namely an everywhere regular differential form of degree
  $n=\dim(X)$ on $X$, vanishing nowhere.
\end{proof}

If the lemma is false, then, for the numerical data
$\{(N_{i},\nu_{i})\}_{i}$ at some point in $Y$, we have
$\nu_{i}>N_{i}$ for $i\neq i_{0}$ and $\nu_{i_{0}}=N_{i_{0}}\geq 2$;
without loss of generality, let $i_{0}=1$. Thus there exists an
irreducible component $E_{1}$ of $h^{\ast}(D)$ with $\nu_{1}=N_{1}\geq
2$ and $N_{1}(=N_{E_{1}})$ not divisible by the characteristic of $K$,
since $h$ is tame. We shall construct a regular $(n-1)$-form
$\theta\neq 0$ on $E_{1}$, as follows. At an arbitrary point $b$ on
$E_{1}$, there exist local coordinates $y_{1},\ldots,y_{n}$ of $Y$
centred at $b$ such that 
$$
g\circ h=\epsilon\prod_{i}y^{N_{i}}_{i},\quad
h^{\ast}(dx)=\eta\prod_{i}y^{\nu_{i}-1}_{i}dy_{1}\Lambda\ldots\Lambda
dy_{n}
$$
with $\epsilon$, $\eta$ regular and non-vanishing around $b$ and
$y_{1}=0$ being the local equation for $E_{1}$. Then, for the
$(n-1)$-form
$$
\Theta=\left(N_{1}\epsilon+y_{1}\frac{\partial\epsilon}{\partial
  y_{1}}\right)^{-1}\eta\prod_{i>1}y^{\nu_{i}-N_{i}-1}_{i}dy_{2}\Lambda\ldots\Lambda
dy_{n} 
$$
on $Y$ regular along $E_{1}$, we have 
$$
h^{\ast}(dx)=d(g\circ h)\Lambda {\Theta}.
$$\pageoriginale
Also, locally around $b$, the restriction $\theta$ of $\Theta$ to
$E_{1}$, \ie
$$
\theta=\text{ the restriction to } E_{1} \text{ of }
(N_{1}\epsilon)^{-1}\eta\prod_{i>1}y^{\nu_{i}-N_{i}-1}_{i}dy_{2}\Lambda\ldots\Lambda
dy_{n}
$$
is well-defined (\ie independent of the choice of local coordinates),
different from $0$ and regular on $E_{1}$.

We recall that $h$ can be factored into monoidal transformations each
having irreducible non-singular centre and $E_{1}$ is ``created'' at
one such stage, say $h':Y'\to X'$; we have tacitly used the fact that
$\nu_{1}\geq 2$ and hence $h$ is not biregular along $E_{1}$. Let $Z$
denote the centre of $h'$ and let $E'=(h')^{-1}(Z)$; then $E'$ is
irreducible and non-singular and the restriction of $Y\to Y'$ to
$E_{1}$ gives a birational morphism $g':E_{1}\to E'$. Let $\theta'$
denote the image of $\theta$ under $g'$. Then $\theta'$ is different
from $0$ and everywhere regular on $E'$; if $\theta'$ is not regular
on $E'$, then $g^{-1}(C')$, for any component $C'$ of its polar
divisor, becomes a component of the polar divisor of $\theta$,
contradicting the regularity of $\theta$. If $\pi=h'|_{E'}$, then
$\pi:E'\to Z$ makes $E'$ into a fibre bundle with the projective space
$\mathbb{P}_{t-1}$ of dimension $t-1$ as fibre; here $t=$ (codimension
of $Z$) $\geq 2$. Choosing $b'$ in $E'$ at which $\theta'$ does not
vanish and putting $a'=\pi(b')$, we choose a local gauge form $dz$ on
$Z$ around $a'$ and write $\theta'=\pi^{\ast}(dz)\Lambda\rho$ with a
$(t-1)$ form $\rho$ on $E'$ regular along $F=\pi^{-1}(a')$. This is
possible and further, the restriction $\rho_{F}$ of $\rho$ to $F$ is
well-defined, different from $0$ and regular on $F$. But $F$ being
isomorphic to $\mathbb{P}_{t-1}$, there is no regular form other than
$0$. This contradiction proves the lemma.

The following lemma is quite elementary.

\begin{lemma}\label{chap4:sec6:subsec1:lem2}  %%\label{lem6.2}
  Let\pageoriginale $f(x)$ be a homogeneous polynomial of degree $m$ in
  $n$ variables $x_{1},\ldots,x_{n}$ with coefficients in (an arbitrary
  field) $k$. Let $X_{0}$ (respectively $X$) denote the projective
  (respectively affine) spaces with $(x_{1},\ldots,x_{n})$ as its
  homogeneous (respectively affine) coordinates and $X^{\sharp}$, the
  projective space with $(1,x_{1},\ldots,x_{n})$ as its homogeneous
  coordinates. Let $H_{\infty}=X^{\sharp}\backslash X$ and $f^{\sharp}$
  denote the rational function on $X^{\sharp}$ defined by $f(x)$. Let us
  assume that a $k$-resolution $h_{0}:Y_{0}\to X_{0}$ of the projective
  hypersurface defined by $f(x)=0$ exists. Then $h_{0}$ gives rise to a
  $k$-resolution $h^{\sharp}:Y^{\sharp}\to X^{\sharp}$ of
  $((f^{\sharp})_{0},H_{\infty})$ such that the numerical data of
  $h^{\sharp}$ are the numerical data of $h_{0}$ at some point of
  $Y_{0}$ possibly augmented by $(m,n)$; here
  $(f^{\sharp})=(f^{\sharp})_{0}-mH_{\infty}$ is the divisor of
  $f^{\sharp}$ on $X^{\sharp}$. 
\end{lemma}

\begin{proof}
Clearly $(1,x_{1},\ldots,x_{n})\mapsto (x_{1},\ldots,x_{n})$ gives a
rational map $\mathscr{S}:X^{\sharp}\to X_{0}$ over $k$, regular except at
$(1,0,\ldots,0)$. If $g:Z\to X^{\sharp}$ is the quadratic
transformation centred at $(1,0,\ldots,0)$, then, $h_{1}=\varphi\circ
g:Z\to X_{0}$ is a $k$-morphism. If $Y^{\sharp}$ is the fibre product
$Y_{0}\times_{X_{0}}Z=\{(y,z)\in Y_{0}\times Z;h_{0}(y)=h_{1}(z)\}$,
define $h^{\sharp}:Y^{\sharp}\to X^{\sharp}$ as the composite of
$h_{0}\times 1:Y^{\sharp}\to Z$ and $g:Z\to X^{\sharp}$. For $X_{0}$
(respectively $Y_{0}$), we have $k$-open coverings by
$X_{1},\ldots,X_{n}$ (respectively $Y_{1},\ldots,Y_{n}$) where $X_{1}$
is the affine space obtained from $X_{0}$ by letting $x_{i}=0$ define
the hyperplane at infinity and $Y_{i}=h^{-1}_{0}(X_{i})$ for $1\leq
i\leq n$. Let $A_{i}$ (respectively $\mathbb{P}_{1,i}$) be the affine
line (respectively projective line) with coordinate ring $k[1/x_{i}]$
(respectively homogeneous coordinate ring $k[t,tx_{i}]$ with a new
variable $t$) for $1\leq i\leq n$. Then $X$, $A_{i}\times X_{i}(1\leq
i\leq n)$ form a $k$-open covering of $X^{\sharp}$ and
$\mathbb{P}_{1,i}\times X_{i}(1\leq i\leq n)$ form a $k$-open covering
of $Z$. For $Y^{\sharp}$, we have a $k$-open covering by
$\mathbb{P}_{1,i}\times Y_{i}$.
\end{proof}

We show that $h^{\sharp}$ has the required property. Let
$a^{\sharp}=h^{\sharp}(b^{\sharp})$ for a point $b^{\sharp}$ of
$Y^{\sharp}$. We have only to show that the irreducible components of
$(h^{\sharp})^{\ast}$ $((f^{\sharp})_{0}+H_{\infty})$
are\pageoriginale defined over $k(b^{\sharp})$ and mutually
transversal at $b^{\sharp}$ and that the numerical data of
$h^{\sharp}$ at $b^{\sharp}$ are the same as those of $h_{0}$ at a
point of $Y_{0}$ except possibly for the addition of $(m,n)$. By
changing the indices, we may assume that
$b^{\sharp}\in\mathbb{P}_{1,1}\times Y_{1}$; thus we may write
$b^{\sharp}=(a_{1},b)$ with $a_{1}$ in the universal field or
$a_{1}=\infty$ and $b$ in $Y_{1}$. Let
$$
u_{2}=x_{2}/x_{1},\ldots,u_{n}=x_{n}/x_{1}.
$$
Then there exist local coordinates $(v_{2},\ldots,v_{n})$ of $Y_{0}$
centred at $b$ and defined over $k(b)$ such that
\begin{align*}
  f(1,u_{2},\ldots,u_{n}) &=\epsilon\prod_{i>1}v^{N_{i}}_{i},\\
  du_{2}\Lambda\ldots\Lambda du_{n} &= \eta
  \prod_{i>1}v^{\nu_{i}-1}_{i}dv_{2}\Lambda\ldots\Lambda dv_{n}
\end{align*}
where $\epsilon$, $\eta$ are regular and non-vanishing around
$b$. First, let $a^{\sharp}\in H_{\infty}$ \ie $a_{1}\neq
\infty$. Then we put $y_{1}=x_{1}-a_{1}$,
$y_{2}=v_{2},\ldots,y_{n}=v_{n}$ and $(y_{1},\ldots,y_{n})$ form local
coordinates of $Y^{\sharp}$ centred at $b^{\sharp}$ and clearly
defined over $k(b^{\sharp})$ such that
\begin{align*}
  f(x_{1},\ldots,x_{n}) &=
  \epsilon(y_{1}+a_{1})^{m}\prod_{i>1}y^{N_{i}}_{i}\\
  dx &= \eta(y_{1}+a_{1})^{n-1}\prod_{i>1}y^{\nu_{i}-1}_{i}\cdot dy.
\end{align*}
Since $(x_{1},\ldots,x_{n})$ form local coordinates on $X^{\sharp}$
around $a^{\sharp}$ and $f(x_{1},\ldots,\break x_{n})=0$ gives a local
equation for $(f^{\sharp})_{0}$, $h^{\sharp}$ has the required
property at $b^{\sharp}$. On the other hand, let $a_{1}=\infty$, then
setting $y_{1}=1/x_{1}$, $y_{2}=v_{2},\ldots,y_{n}=v_{n}$, we see that
$(y_{1},u_{2},\ldots,u_{n})$ (respectively $(y_{1},\ldots,y_{n})$ form
local coordinates on $X^{\sharp}$ (respectively $Y^{\sharp}$) around
$a^{\sharp}$ (respectively centred at $b^{\sharp}$) and defined over
$k(b^{\sharp})$. Moreover, $f(1,u_{2},\ldots,u_{n})=0$ (respectively
$y_{1}=0$) gives local equations for $(f^{\sharp})_{0}$ (respectively
$H_{\infty}$);\pageoriginale also we have 
\begin{align*}
  & f(1,u_{2},\ldots,u_{n})=\epsilon\prod_{i>1}y^{N_{i}}_{i},\\
  & dy_{1}\Lambda du_{2}\Lambda\ldots\Lambda
  du_{n}=\eta\prod_{i>1}y^{\nu_{i}-1}_{i}\cdot dy
\end{align*}
This finally establishes the required property of $h^{\sharp}$ at
$b^{\sharp}$.

We remark that the addition of $(m,n)$ to the numerical data results
if and only if $a_{1}=0$ \ie if and only if
$a^{\sharp}=h^{\sharp}(b^{\sharp})$ has $(1,0,\ldots,0)$ as its
homogeneous coordinates.

\subsection{Key Lemmas}\label{chap4:sec6:subsec2} %%% subsec{6.2}

Let $f(x)$ be a polynomial in $x_{1},\ldots,x_{n}$ with coefficients
in a global field $k$, such that $C_{f}\subset f^{-1}(0)$. Defining
$X$, $X^{\sharp}=X\cup H_{\infty}$ and $f^{\sharp}$ (the rational
function extending $f$ to $X^{\sharp}$) as in Lemma {lem6.2}, let
us assume that a tame $k$-resolution $h^{\sharp}:Y^{\sharp}\to
X^{\sharp}$ of $((f^{\sharp})_{0},H_{\infty})$ exists. Then the
restriction of $h^{\sharp}$ to $Y=f^{-1}(X)$ gives a tame
$k$-resolution $h:Y\to X$ of the hypersurface $f(x)=0$. In the
following lemma, $\mathscr{D}(X_{v})$ stands for the space of
Schwartz-Bruhat functions on $X_{v}$ with {\em compact support}; thus
$\mathscr{D}(X_{v})=\mathscr{S}(X_{v})$ for non-archimedean $v$.

\begin{lemma}\label{chap4:sec6:subsec2:lem3} %%%% {lem6.3}
  For given $b$ in $Y_{v}$, choose $\Phi_{v}\geq 0$ in
  $\mathscr{D}(X_{v})$ with $\Phi_{v}(h(b))>0$ and {\em(i)} assume that
  existence of $F_{\Phi_{v}}(0)$ or {\em(ii)} make the stronger
  assumption that
  \begin{equation*}
    F_{\Phi_{v}}(0)=\int\limits_{U(0)_{v}}\Phi_{v}|\theta_{0}|_{v}.
    \tag{149}\label{chap4:sec6:subsec2:eq149}
  \end{equation*}
  Then $h$ has property $(P_{0})$ or property $(P)$ at $b$ depending on
  {\em(i)} or {\em(ii)} above. Conversely, if $h$ has property $(P)$ at
  every point of $Y_{v}$, then (\ref{chap4:sec6:subsec2:eq149}) holds for every $\Phi_{v}$ in
  $\mathscr{D}(X_{v})$; further, if $h^{\sharp}$ has property $(P)$ at
  every point of $Y^{\sharp}_{v}$, then
  (\ref{chap4:sec6:subsec2:eq149}) holds for every 
  $\Phi_{v}$ in $\mathscr{S}(X_{v})$.
\end{lemma}

\begin{proof}
Using\pageoriginale the notation of Chapter III,
\S\ \ref{chap3:sec3:subsec2}, there
exist, for the 
$k_{v}$-resolution $h:Y_{v}\to X_{v}$, local coordinates
$y_{1},\ldots,y_{n}$ of $Y$ centred at $b$ and defined over $k_{v}$
such that
$$
f\circ h=\epsilon\prod_{i}y^{N_{i}}_{i},\quad
h^{\ast}(dx)=\eta\prod_{i}y^{\nu_{i}-1}_{i}\cdot dy
$$
with $\epsilon$, $\eta$ regular and non-vanishing around $b$. Now,
$\epsilon$, $\eta$, $y_{1},\ldots,y_{n}$ give rise to $k_{v}$-analytic
functions on an open neighbourhood, say $V$, of $b$ in $Y_{v}$. By
choosing $V$ smaller still, if necessary, we may assume that already
$\epsilon\eta$ is non-vanishing on $V$ and also that, for the given
$\Phi_{v}$.
$$
|\epsilon(y)|^{s}_{v}|\eta(y)|_{v}\Phi_{v}(h(y))\geq c>0
$$
for a suitable constant $c$ independent of $y$ in $V$ and $s$ with
$-1\leq s\leq 0$. Then, for all such $s$,
\begin{equation*}
  Z_{\Phi_{v}}(\omega_{s})\geq
  c\int\limits_{V}\prod_{i}|y_{i}|_{v}^{N_{i}s+\nu_{i}-1}|dy|_{v}
  \tag{150}\label{chap4:sec6:subsec2:eq150} 
\end{equation*}
On the other hand, from the existence of $F_{\Phi_{v}}(0)$, we can
conclude, in view of Chapter II, \S\ \ref{chap2:sec3:subsec1}, that the function
$(s+1)Z_{\Phi_{v}}(\omega_{s})$, which is positive for $-1<s\leq 0$,
is also continuous and bounded in this range. On being multiplied by
$s+1$, the right hand side of (\ref{chap4:sec6:subsec2:eq150}) is asymptotic to a constant
multiple of $(s+1)/\prod\limits_{i}(N_{i}s+\nu_{i})$ as $s\to -1$ from
the right. Therefore, $\nu_{i}\geq N_{i}$ for every $i$ and the
equality $\nu_{i}=N_{i}$ can hold for at most one $i$. In other words,
the existence of $F_{\Phi_{v}}(0)$ entails that $h$ has property
$(P_{0})$ at $b$.
\end{proof}

We shall show next that (\ref{chap4:sec6:subsec2:eq149}) implies
$\nu_{i_{0}}=N_{i_{0}}=1$. For that purpose, we shall first obtain an
expression for $F_{\Phi_{v}}(0)$ without
assuming \ref{chap4:sec6:subsec2:eq149}. We take
an arbitrary $b'$ from $Y_{v}$ at which $\Phi_{v}(b')>0$; then what we
have shown for $b$ is applicable\pageoriginale to $b'$. For the sake
of simplicity, we shall write $b$ again for $b'$ and use the same
notation as above. By changing indices, we may suppose that
$i_{0}=1$. Let
$\epsilon_{1}\left(N_{1}\epsilon+y_{1}\dfrac{\partial\epsilon}{\partial
  y_{1}}\right)^{-1}\eta$, locally around $b$ in $Y_{v}$. Then, for
every $i\neq 0$ in $k$, we have, $h^{\ast}(\theta_{i})=$ the
restriction of
$\epsilon_{1}\prod_{j>1}y^{\nu_{j}-N_{j}-1}_{j}dy_{2}\Lambda\ldots\Lambda
dy_{n}$ to $(f\circ h)^{-1}(i)$. Furthermore, if we denote $E_{1}$ by
$E$, then the everywhere regular differential form $\theta$ on $E$
(obtained by restriction as in Lemma (\ref{chap4:sec6:subsec1:lem1}) exists. If
$E_{v}\neq\phi$, then $E$ and $\theta$ are both defined over
$k_{v}$. Moreover, $\theta$ gives rise to a positive Borel measure
$|\theta|_{v}$ on $E_{v}$, with $E_{v}$ as its exact support.

We may take the neighbourhood $V$ of $b$ above small enough so that
$\epsilon_{1}\neq 0$ on $V$. Then for every locally constant or
$C^{\infty}$ function $\varphi_{0}$ on $V$ with compact support, we
have
$$
\lim\limits_{i\to 0}\int\limits_{(f\circ
  h)^{-1}(i)}\varphi_{0}|h^{\ast}(\theta_{i})|_{v}=\int\limits_{E\cap
  V}\varphi_{0}|\theta|_{v}. 
$$
We know that $h$ is biregular at every point of $h^{-1}(U(0))$; if
$\nu_{1}=N_{1}=1$ as long as only points of $V$ are considered, we
have
$$
h^{-1}(U(0))=E\backslash \cup \text{ Hyperplanes defined by } y_{j}=0
\text{ for } N_{j}\geq 1.
$$
Hence we get
$$
\int\limits_{E\cap
  V}\varphi_{0}|\theta|_{v}=\int\limits_{h^{-1}(U(0))_{v}}\varphi_{0}|h^{\ast}(\theta_{0})|_{v}. 
$$
We take a finite covering of the preimage, under $h$, of the support
of $\Phi_{v}$, by means of open sets $V$ as above. Choosing a
partition of unity $(p_{V})_{V}$ subordinate to this covering, we
apply the observation above to each $\varphi_{0}=(\Phi_{v}\circ
h)p_{V}$ and obtain 
$$
F_{\Phi_{v}}(0)=\int\limits_{U(0)_{v}}\Phi_{v}|\theta_{0}|_{v}+\sum_{\nu_{E}\geq
  2}\int\limits_{E_{v}}(\Phi_{v}\circ h)|\theta|_{v}.
$$
But\pageoriginale $|\theta|_{v}$ has $E_{v}$ as its exact support and
therefore, the validity of (\ref{chap4:sec6:subsec2:eq149}) now implies that no $E$ with
$\nu_{E}=N_{E}\geq 2$ can pass through a point $b'$ where
$\Phi_{v}(b')>0$. Therefore (\ref{chap4:sec6:subsec2:eq149}) implies
that $h$ has property (P) at $b$.

Conversely, suppose that $h$ has property $(P)$ at every point of
$Y_{v}$. Then certainly (\ref{chap4:sec6:subsec2:eq149}) holds for every $\Phi_{v}$ in
$\mathscr{D}(X_{v})$. If, further, $h$ has property $(P)$ at every
$b^{\sharp}$ in $Y^{\sharp}_{v}$, then we choose local coordinates
$y_{1},\ldots,y_{n}$ on $Y^{\sharp}$ centred at $b^{\sharp}$ and
defined over $k_{v}$ such that, upto a regular and non-vanishing
function around $b^{\sharp}$, $f^{\sharp}\circ h^{\sharp}$ becomes a
product of (possibly negative) powers of $y_{1},\ldots,y_{n}$ and
further, $(h^{\sharp})^{\ast}(dx)$ is another such power-product in
$y_{1},\ldots,y_{n}$ multiplied by a local gauge form around
$b^{\sharp}$. Suppose, for example, the exponent of $y_{i}$ in
$f^{\sharp}\circ h^{\sharp}$ is negative. Then the ``infinite
divisibility'' of $\Phi_{v}$ by a local equation for $H_{\infty}$
implies the infinite divisibility of $\Phi_{v}\circ h^{\sharp}$ by
$y_{i}$. Thus the component with $y_{i}=0$ as a local equation becomes
negligible and (\ref{chap4:sec6:subsec2:eq149}) holds for every
$\Phi_{v}$ in $\mathscr{S}(X_{v})$.

In the sequel, we shall assume as in the beginning of
\S\ (\ref{chap4:sec6}), that $k$ is a global field, $f(x)$ in
$k[x_{1},\ldots,x_{n}]$ is homogeneous of degree $m(\geq 2 \text{
  and})$ not divisible by the characteristic of $k$ and further, a
tame $k$-resolution $h_{0}$ of the projective hypersurface defined by
$f(x)=0$ exists. We shall also use the notation of Lemma
(\ref{chap4:sec6:subsec1:lem2}).

%\setcounter{lemma}{5}
\begin{lemma}\label{chap4:sec6:subsec2:lem4} %%% {lem6.4}
  If $F^{\ast}_{v}$ is in $L^{1}(k_{v})$ for all but finitely many
  valuations $v$ of $k$, then $h^{\sharp}$ has property $(P)$ everywhere.
\end{lemma}

\begin{proof}
  Since $h_{0}$ is tame by assumption, all irreducible components of the
  total transform on $Y_{0}$ of the projective hypersurface defined by\break
  $f(x)=0$ are defined over the separable closure $k_{s}$ of $k$. In
  view of the quasi-compactness of the Zariski topology and the Zariski
  density of the set of $k_{s}$-rational points on any irreducible
  variety\pageoriginale defined over $k_{s}$, we can choose a finite
  number of $k_{s}$-rational points of $Y_{0}$ such that the set of the
  numerical data at these points contains the numerical data at every
  point of $Y_{0}$. Let $k_{1}$ denote the extension field of $k$
  obtained by adjoining (the coordinates of) all the finitely many
  points mentioned above; then $k_{1}$ is a finite separable extension
  of $k$ and as a result, there exist infinitely many non-archimedean
  valuations on $k_{1}$ with degree $1$ relative to $k$ (This is a
  consequence of the face that the zeta function $\zeta_{k}(s)$ of a
  global field is holomorphic for $\text{Re\;}(s)>1$ and has a pole at
  $s=1$). We can choose one such valuation so that its restriction, say
  $w$, to $k$ is not one of the finitely many valuations excluded in the
  hypothesis; then the field $k_{1}$ is contained in $k_{w}$ and
  $F^{\ast}_{w}$ is in $L^{1}(k_{w})$. We have thus achieved a situation
  where the numerical data of $h^{\sharp}$ at any point of $Y^{\sharp}$
  turn out to be also the numerical data of $h$ at some point of
  $Y_{v}\cap h^{-1}(X^{0}_{w})$. Since $F^{\ast}_{w}$ is in
  $L^{1}(k_{w})$, $F_{v}(0)$ exists and by Lemma
   \ref{chap4:sec6:subsec2:lem3}, $h$ has
  property $(P_{0})$ at every point of $Y_{w}\cap
  h^{-1}(X^{0}_{w})$. But, then, by the construction, $h^{\sharp}$ has
  property $(P_{0})$ everywhere and hence, by Lemma
  \ref{chap4:sec6:subsec1:lem1}, has 
  property $(P)$ everywhere and the proof is complete.
\end{proof}

As a consequence, by the last assertion in Lemma
\ref{chap4:sec6:subsec2:lem3}, 
(\ref{chap4:sec6:subsec2:eq149}) holds good, implying that
$F^{\ast}_{\Phi_{v}}$ is in 
$L^{1}(k_{v})$ for every $\Phi_{v}$ in $\mathscr{S}(X_{v})$ and for every
valuation $v$ of $k$.

\subsection{Standard Lemmas}\label{chap4:sec6:subsec3} %%% subsec{6.3}

For a non-archimedean valuation $v$ of $k$ and the characteristic
function $\Phi_{v}$ of $X^{0}_{v}$, we shall denote $F_{\Phi_{v}}$
simply by $F_{v}$; then $F^{\ast}_{\Phi_{v}}=F^{\ast}_{v}$ is the
Fourier transform of $F_{v}$. Further, if $F^{\ast}_{v}$ is in
$L^{1}(k_{v})$, then
$$
F_{v}(i)=\int\limits_{k_{v}}F^{\ast}_{v}(i^{\ast})\psi_{v}(-ii^{\ast})|di^{\ast}|_{v}, 
$$
for every $i$ in $k_{v}$. In the following two lemmas, we shall assume
that $v$ is non-archimedean,\pageoriginale $\psi_{v}=1$ on $R_{v}$,
$\psi_{v}$ is non-constant on $P^{-1}_{v}$ and further $f(x)\in
R_{v}[x_{1},\ldots,x_{n}]$. Anyway, these conditions hold good for all
but finitely many $v$.

\setcounter{lemma}{4}
\begin{lemma}\label{chap4:sec6:subsec3:lem5} %%% {lem6.5}
  If there exists $\sigma>1$ such that $|F^{\ast}_{v}(i^{\ast})|\leq
  |i^{\ast}|^{-\sigma}_{v}$ for every $i^{\ast}$ in $k_{v}\backslash
  R_{v}$, then, for every $i$ in $R_{v}$, we have
  $$
  |F_{v}(i)-1|\leq c, q^{-(\sigma-1)}_{v}
  $$
  for a constant $c$ depending only on $\sigma$.
\end{lemma}

\begin{proof}
By our assumptions on $v$ above, $F^{\ast}_{v}=1$ on $R_{v}$. Since
$X^{0}_{v}$ has measure $1$ and since $F^{\ast}_{v}$ is in
$L^{1}(k_{v})$ in view of the hypothesis, we have
$$
F_{v}(i)-1=\int\limits_{k_{v}\backslash
  R_{v}}F^{\ast}_{v}(i^{\ast})\psi_{v}(-ii^{\ast})|di^{\ast}|_{v}
$$
for every $i$ in $R_{v}$. Thus
\begin{align*}
|F_{v}(i)-1| &\leq \int\limits_{k_{v}\backslash
  R_{v}}|i^{\ast}|^{-\sigma}_{v}|di^{\ast}|_v\\
&=(1-q^{-1}_{v})(1-q^{-(\sigma-1)})^{-1-(\sigma-1)}q_{v}\\
&\leq (1-2^{-(\sigma-1)})^{-1}q^{-(\sigma-1)}_{v}
\end{align*}
and the lemma is proved.
\end{proof}

For $i$ in $R_{v}$, we recall that $U(i)^{0}_{v}$ is a compact subset
of $U(i)_{v}$ defined as follows. Namely, if $i\neq 0$, then
$U(i)^{0}_{v}$ is just $U(i)_{v}\cap X^{0}_{v}$ and for $i=0$, it is
the subset of $U(i)_{v}\cap X^{0}_{v}$, where, in addition,
$\grad_{x}f\not\equiv 0(\mod P_{v})$. We also recall that $r$ is the
codimension of $C_{f}$ in $f^{-1}(0)$.

\begin{lemma}\label{chap4:sec6:subsec3:lem6} %%% \label{lem6.6}
  Under\pageoriginale the same assumptions as in
  LEMMA \ref{chap4:sec6:subsec3:lem5} , we
  have, for every $i$ in $R_{v}$,
  \begin{equation*}
    \left|\int\limits_{U(i)^{0}_{v}}|\theta_{i}|_{v}-1\right|\leq
    c'(q^{-(\sigma-1)}_{v}+q^{-r}_{v})
    \tag{151}\label{chap4:sec6:subsec3:eq151}  
  \end{equation*}
  with a constant $c'$ independent of $i$ and $v$.
\end{lemma}

\begin{proof}
Since the left hand side of (\ref{chap4:sec6:subsec3:eq151}) is at most equal to
$|F_{v}(i)|+1\leq c+2$ (in view of Lemma \ref{chap4:sec6:subsec3:lem5}), we may exclude
some more finitely many valuations while proving this lemma. We may,
in particular, assume that $m\not\equiv 0(\mod P_{v})$. If $N_{e}(i)$
(respectively $N^{0}_{e}(i)$) is the number of $\xi$ in $X^{0}_{v}$
modulo $P^{e}_{v}$ such that $f(\xi)\equiv i(\mod P^{e}_{v})$
(respectively $f(\xi)\equiv i(\mod P^{e}_{v})$ and
$\grad_{\xi}f\not\equiv 0\pmod{P_{v}}$), for $e=0,1,2,\ldots$ then
$$
q^{-(n-1)e}_{v}N_{e}(i)=\int\limits_{P^{-e}_{v}}F^{\ast}_{v}
(i^{\ast})\psi_{v}(-ii^{\ast})|di^{\ast}|_{v}.
$$
As in the proof of Lemma \ref{chap4:sec6:subsec3:lem5}, we have
$$
|q^{-(n-1)}_{v}N_{1}(i)-1|\leq cq^{-(\sigma-1)}_{v}
$$
for a constant $c$ independent of $i$ and $v$. But it is elementary to
prove that $N_{1}(i)=N^{0}_{1}(i)$ for $i\not\in P_{v}$ and further
that $q_{v}^{-(n-1-r)}(N_{1}(i)-N^{0}_{1}(i))$ is bounded uniformly
for $i$ in $R_{v}$ and all $v$ (\cite{Lang-Weil}). Also, we have
$q^{-(n-1)}_{v}\break N^{0}_{1}(i)=\int\limits_{U(i)^{0}_{v}}|\theta_{i}|_{v}$
for all but finitely many $v$. Putting these together, the estimate
(\ref{chap4:sec6:subsec3:eq151}) follows.
\end{proof}

\begin{lemma}\label{chap4:sec6:subsec3:lem7} %%%% \label{6.7}
Suppose that, for every $v$ in $S_{\infty}$, $\sigma_{v}>0$ and
further that, for every $i^{\ast}$ in $k_{v}$ and every $\Phi_{v}$ in
$\mathscr{S}(X_{v})$, we have
$$
|F^{\ast}_{\Phi_{v}}(i^{\ast})|\leq c(\Phi_{v})\max
(1,|i^{\ast}|_{v})^{-\sigma_{v}}
$$
with\pageoriginale a constant $c(\Phi_{v})>0$. Let us write
$$
X_{\infty}=\prod_{v\in S_{\infty}}X_{v},\quad
|dx|_{\infty}=\bigotimes_{v\in S_{\infty}}|dx|_{v},\quad
k_{\infty}=\prod_{v\in S_{\infty}}k_{v}
$$
and let $C_{\infty}$ denote a compact subset of
$\mathscr{S}(X_{\infty})$. Then there exists a constant $c\geq 0$
depending on $C_{\infty}$ such that
\begin{equation*}
  \left|\int\limits_{X_{\infty}}\left(\prod_{v\in
    S_{\infty}}\psi_{v}(i^{\ast}_{v}f(x))\right)\Phi_{\infty}(x)
  |dx|_{\infty}\right|\leq c\prod_{v\in
    S_{\infty}}\max(1,|i^{\ast}|_{v})^{-\sigma_{v}}
  \tag{152}\label{chap4:sec6:subsec3:eq152} 
\end{equation*}
for every $i^{\ast}=(i^{\ast}_{v})_{v}$ in $k_{\infty}$ and every
$\Phi_{\infty}$ in $C_{\infty}$.
\end{lemma}

\begin{proof}
We first remark that, on the left hand side
of \ref{chap4:sec6:subsec3:eq152}, 
$\Phi_{\infty}$ is not necessarily of the form
$\bigotimes\limits_{v\in S_{\infty}}\Phi_{v}$ (with $\Phi_{v}$ in
$\mathscr{S}(X_{v})$); this is just the ``raison d'etre'' for this lemma.

If we put
$$
T_{i^{\ast}}(\Phi_{v})=\max(1,|i^{\ast}|_{v})^{\sigma_{v}}\cdot
F^{\ast}_{\Phi_{v}}(i^{\ast}), 
$$
then we get a subset $\{T_{i^{\ast}}\}_{i^{\ast}}$ of
$\mathscr{S}(X_{v})'$ parametrized by $k_{v}$. The hypothesis means
precisely that this subset is ``bounded'' \ie
$|T_{i^{\ast}}(\Phi_{v})|\leq c(\Phi_{v})$ for every $i^{\ast}$ in
$k_{v}$. The proof of the lemma now follows from the two well-known
facts given below:
\begin{enumerate}
\renewcommand{\theenumi}{\roman{enumi}}
\renewcommand{\labelenumi}{(\theenumi)}
\item For any bounded subset $B$ of $\mathscr{S}(\mathbb{R}^{p})'$ and any
  compact subset $C$ of $\mathscr{S}(\mathbb{R}^{p})$, we have
  $$
  \sup\limits_{T\in B,\Phi\in C}|T(\Phi)|<\infty;
  $$
\item For\pageoriginale bounded subsets $B$, $B'$ respectively of
  $\mathscr{S}(\mathbb{R}^{p})'$, $\mathscr{S}(\mathbb{R}^{q})'$, the set
  $\{T\otimes T';T\in B,T'\in B'\}$ is bounded in
  $\mathscr{S}(\mathbb{R}^{p+q})'$ 
\end{enumerate}
\end{proof}

\begin{lemma}\label{chap4:sec6:subsec3:lem8} %%%%{lem6.8}
  Let $\sigma_{v}>1$ for all $v$ and further, for all but finitely many
  $v$, let $\sigma_{v}\geq \sigma>2$. Then
  $$
  \sum_{i^{\ast}\in
    k}\prod_{v}\max(1,|i^{\ast}+j^{\ast}|_{v})^{-\sigma_{v}}
  $$
  is bounded as $j^{\ast}$ varies over $k_{A}$.
\end{lemma}

\begin{proof}
Since $k_{A}/k$ is compact, we may restrict $j^{\ast}$ to a compact
subset of $k_{A}$. If $v$ is non-archimedean and if $j^{\ast}_{v}$ is
in $R_{v}$, then
$$
\max(1,|i^{\ast}+j^{\ast}_{v}|_{v})=\max(1,|i^{\ast}|_{v})
$$
for every $i^{\ast}$ in $k_{v}$. We may therefore assume $j^{\ast}=0$
in the foregoing. In the special case, when $k=\mathbb{Q}$ and
$\sigma_{v}=\sigma(>2)$ for all $v$, we have
$$
\prod_{v}\max (1|i^{\ast}|_{v})=\max(c,|d|)
$$
for $i^{\ast}=d/c$ with $c\geq 1$ and $(c,d)=1$. By a simple
calculation, it can be shown that
$$
\sum_{i^{\ast}\in\mathbb{Q}}\prod_{v}\max(1,|i^{\ast}|_{v})^{-\sigma}=\zeta(\sigma)(4\zeta(\sigma-1)-\zeta(\sigma))<\infty
$$
where $\zeta(s)$ is the Riemann zeta function. For the case of any
global field $k$, the proof can be found in \cite{Igu 1}, page {187}
and \cite{Igu 3}, page {226}. 
\end{proof}

\subsection{Proof of Theorem
  \ref{chap4:sec2:subsec1:thm1}}\label{chap4:sec6:subsec4} %%%% subsec6.4 

We are now ready to prove that the criteria (C1) and (C2) imply the
validity of the Poisson formula.

From\pageoriginale (C1), (C2) and Lemma \ref{chap4:sec6:subsec3:lem6},
we see that 
(PF-2)$'$ is valid \ie the restricted product measure
$|\theta_{i}|_{A}$ exists on $U(i)_{A}$, for every $i$ in
$k$. Further, by Lemma \ref{chap4:sec6:subsec3:lem5}, (PF-2)$''$ is
valid \ie the image 
measure under $U(i)_{A}\to X_{A}$ is tempered. On the other hand, for
every $\Phi$ in $\mathscr{S}(X_{A})$ and every $i^{\ast}$ in $k_{A}$, we
set
$$
F^{\ast}_{\Phi}(i^{\ast})=\int\limits_{X_{A}}\Phi(x)\psi(i^{\ast}f(x))|dx|_{A}.
$$
We recall that $\Phi$ is a finite sum of functions of the form
$\Phi_{0}\otimes \Phi_{\infty}$, in our earlier notation. We restrict
$\Phi_{\infty}$ to compact subset $C_{\infty}$ of
$\mathscr{S}(X_{\infty})$; then there exists a constant $c\geq 0$
depending on $\Phi_{0}$ and $C_{\infty}$ such that
$$
\sum_{i^{\ast}\in k}|F^{\ast}_{\Phi}(i^{\ast}+j^{\ast})|\leq
c\sum_{i^{\ast}\in
  k}\prod_{v}\max(1,|i^{\ast}+j^{\ast}|_{v})^{-\sigma}<\infty
$$
for every $j^{\ast}$ in $k_{A}$, as a consequence of (C2) and Lemmas
\ref{chap4:sec6:subsec2:lem3},
\ref{chap4:sec6:subsec2:lem4}, \ref{chap4:sec6:subsec3:lem7},
\ref{chap4:sec6:subsec3:lem8}. This 
establishes the validity of (PF-1) \ie the absolute convergence of the
Eisenstein-Siegel series. Since $\sum\limits_{i^{\ast}\in
  k}F^{\ast}_{\Phi}(i^{\ast}+j^{\ast})$ represents a continuous
periodic function of $j^{\ast}$, we get the Fourier expansion
\begin{equation*}
\sum_{i^{\ast}\in k}F^{\ast}_{\Phi}(i^{\ast}+j^{\ast})=\sum_{i\in
  k}c_{\Phi}(i)\psi(ij^{\ast})\tag{153}\label{chap4:sec6:subsec4:eq153}
\end{equation*}
where 
$$
c_{\Phi}(i)=\int\limits_{k_{A}}F^{\ast}_{\Phi}(j^{\ast})
\psi(-ij^{\ast})|dj^{\ast}|_{A},  
$$
provided that the series on the right hand side of
(\ref{chap4:sec6:subsec4:eq153})
converges absolutely. This is certainly the case if the series is just
a finite sum.

If $\Phi$ is of the form $\bigotimes\limits_{v}\Phi_{v}$ with
$\Phi_{v}$ in $\mathscr{S}(X_{v})$ for every $v$ and further $\Phi_{v}$
equal to the characteristic function of $X^{0}_{v}$ for all but
finitely many $v$, then (C2) together\pageoriginale with Lemmas
\ref{chap4:sec6:subsec2:lem3},
\ref{chap4:sec6:subsec2:lem4}, \ref{chap4:sec6:subsec3:lem5}
implies that 
\begin{align*}
  c_{\Phi}(i) &=
  \int\limits_{k_{A}}F^{\ast}_{\Phi}(i^{\ast})\psi(-ii^{\ast})|di^{\ast}|_{A}\\ 
  &=\prod_{v}\int\limits_{k_{v}}F^{\ast}_{\Phi_{v}}(i^{\ast})
  \psi_{v}(-ii^{\ast})|di^{\ast}|_{v}\\ 
  &=\prod_{v}F_{\Phi_{v}}(i)\\
  &= \int\limits_{U(i)_{A}}\Psi|\theta_{i}|_{A}. 
\end{align*}
Therefore, this expression $c_{\Phi}(i)$ is valid for every $\Phi$ in
$\mathscr{S}(X_{A})$, by continuity and linearity. Now putting
$j^{\ast}=0$ in \ref{chap4:sec6:subsec4:eq153}, we deduce that
\begin{equation*}
  \sum_{i^{\ast}\in k}F^{\ast}_{\Phi}(i^{\ast})=\sum_{i\in
    k}\int\limits_{U(i)_{A}}\Phi|\theta_{i}|_{A}
  \tag{154}\label{chap4:sec6:subsec4:eq154} 
\end{equation*}
provided that (PF-3) holds \ie the right hand side of
(\ref{chap4:sec6:subsec4:eq154})  is
absolutely convergent. As remarked earlier, this is certainly true, if
the series is a finite sum, as, for example, in the case of $\Phi$
having compact support. Let us now consider the general case of $\Phi$
in $\mathscr{S}(X_{A})$, not necessarily having compact support. For any
such $\Phi$, we know that there exists $\mathscr{S}\geq 0$ in
$\mathscr{S}(X_{A})$ such that $|\Phi(x)|\leq \varphi(x)$ for every $x$ in
$X_{A}$. It is sufficient therefore to show that the right hand side
of (\ref{chap4:sec6:subsec3:eq152}) converges for every $\Phi\geq 0$ in
$\mathscr{S}(X_{A})$. Let us therefore assume that $\Phi\geq 0$ and
moreover, take $\Phi$ to be of the form $\Phi_{0}\otimes
\Phi_{\infty}$ where $\Phi_{0}$ is the characteristic function of a
large compact open subset of $X_{0}$ and $\Phi_{\infty}\geq 0$ in
$\mathscr{S}(X_{\infty})$. We can easily find a monotone increasing
sequence $\{\Phi_{\infty,n}\}_{n}$ of non-negative $C^{\infty}$
functions on $X_{\infty}$ with compact support which converges to
$\Phi_{\infty}$ in $\mathscr{S}(X_{\infty})$; then
$\{\Phi_{\infty,n}\}_{n}\cup \Phi_{\infty}$ is a compact subset of
$\mathscr{S}(X_{\infty})$. We\pageoriginale put $\Phi_{n}=\Phi_{0}\otimes
\phi_{\infty,n}$ for $n=1,2,3,\ldots$,; then every $\Phi_{n}$ is in
$\mathscr{S}(X_{A})$ and has compact support. Further the monotone
increasing sequence $0\le \Phi_{1}\leq \Phi_{2}\leq\ldots$ converges
to $\Phi$. Therefore, for a constant $c\geq 0$, we have, in view of
Lemma \ref{chap4:sec6:subsec3:lem7},
$$
\sum_{i\in k}\int\limits_{U(i)_{A}}\Phi_{n}|\theta_{i}|_{A}\leq
\sum_{i^{\ast}\in k}|F^{\ast}_{\Phi_{n}}(i^{\ast})|\leq c<\infty
$$
for $n=1,2,3,\ldots$. Since the left hand side is the integral of
$\Phi_{n}$ over the (disjoint) union of $U(i)_{A}$ for all $i$ in $k$,
we obtain, by Lebesgue's theorem on a monotone increasing sequence of
non-negative functions, that
$$
\sum_{i\in
  k}\int\limits_{U(i)_{A}}\Phi|\theta_{i}|_{A}=\lim\limits_{n\to
  \infty}\sum_{i\in
  k}\int\limits_{U(i)_{A}}\Phi_{n}|\theta_{i}|_{A}\leq c.
$$
The proof of the validity of the Poisson formula is now complete.

\begin{Remark*}
  We observe that the proof of our theorem becomes much simpler in the
  function-field case, since the consideration of $X_{\infty}$
  disappears. We might, moreover, mention that we could have applied to
  the left hand side of (\ref{chap4:sec6:subsec4:eq153}), the
  result of Weil on page 7 of 
  \cite{Wei 5} and thereby avoided the use of the monotone sequence
$\{\Phi_{n}\}_{n}$. 
\end{Remark*}

\subsection{A Conjecture}\label{chap4:sec6:subsec5} %%% subsec{6.5}

We conclude this series of lectures by stating a conjecture: namely,
if, with the notation of Lemma \ref{chap4:sec6:subsec1:lem2}, we
assume that the 
resolution $h^{\sharp}:Y^{\sharp}\to X^{\sharp}$ of
$((f^{\sharp})_{0},H_{\infty})$ is tame and further that
$\lambda_{1}=\min\limits_{\nu_{E}\geq 2}(\nu_{E}/N_{E})>1$, then for
any $\sigma<\lambda_{1}$, we have
$$
|F^{\ast}_{\Phi_{v}}(i^{\ast})|\leq c(\Phi_{v})\max
(1,|i^{\ast}|_{v})^{-\sigma}
$$
for every $\Phi_{v}$ in $\mathscr{S}(X_{v})$, all $i^{\ast}$ in $k_{v}$
and all $v$. It is very likely that {\em we can take
  $c(\Phi_{v})=1$\pageoriginale 
  for the characteristic function $\Phi_{v}$ of $X^{0}_{v}$, for all
  but finitely many $v$.} If this conjecture is true, then condition
(C2) can be replaced by the simple geometric condition
$\lambda_{1}>2$.


\begin{thebibliography}{99}\pageoriginale
\bibitem{Ari} ARITURK H.: The Siegel-Weil Formula for Orthogonal
  Groups, Thesis, Johns Hopkins University, 1975.

\bibitem{Ati} ATIYAH M.F.: Resolution of singularities and division of
  distributions, Comm.\@ pure and appl.\@ Math.\@ 23(1970), 145-150.

\bibitem{Ber-Gel} BERNSTEIN I.N.\@ and GEL'FAND S.I.: Meromorphic
  property of the functions $P^{\lambda}$, Functional Analysis and its
  applications, 3(1969), 68-69.

\bibitem{Bir} BIRCH B.J.: Forms in many variables, Proc.\@ Royal
  Soc.\@ A, 265(1962), 245-263.

\bibitem{Bor} BOREL A.: Linear Algebraic Groups, Benjamin, 1969.

\bibitem{Bor-Sha} BOREVICH Z.I.\@ and SHAFAREVICH I.R.: Number Theory,
  Academic Press, 1966.

\bibitem{Bru} BRUHAT F.: Distributions sur un groupe localament
  compact et applications \`a l'\'etude des repr\'esentations des
  groupes $p$-adiques, Bull.\@ Soc.\@ Math.\@ France 89(1961), 43-75. 

\bibitem{Dav} DAVENPORT H.: Analytic methods for diophantine equations
  and diophantine inequalities, Ann Arbor, 1962.

\bibitem{Del} DELIGNE P.: La conjecture de Weil I, Inst.\@ Haut.\@
  Etud.\@ Sci.\@ 43(1973), 273-307.

\bibitem{Erd} ERD\'ELYI A.: Asymptotic Expansions, New York, 1956.

\bibitem{Gau} GAUSS C.F.: Recherches Arithm\'etiques, Paris,  1807.

\bibitem{Gel-Shi} GEL'FAND I.M.\@ and SHILOV G.E.: Les Distributions,
  Paris, 1962.

\bibitem{God} GODEMENT R.: Domaines fondamentaux des groupes
  arithm\'etiques, S\'em.\@ Bourbaki, n$^{\circ}$257, 1962/63, 1-25.

\bibitem{Hec 1} HECKE E.: Uber die Bestimmung Dirichletscher Reihen
  durch ihre Funktionalgleichung, Math.\@ Ann.\@ 112(1936), 664-699:
  Mathem.\@ Werke, Vandenhoeck u.\@ Ruprecht, 1959, 591-626.

\bibitem{Hec 2} HECKE E.: Uber Modulfunktionen und die Dirichletschen
  Reihen mit Eulerscher Produktentwicklung I, Math.\@ Ann.\@ 114
  (1937), 1-28; II, ibid.\@ 316-351: Mathem.\@ Werke, Vandenhoeck u.\@
  Ruprecht, 1959, 644-707.

\bibitem{Hir} HIRONAKA H.: Resolution of singularities of an algebraic
  variety over a field of characteristic zero, Ann.\@ Math.\@
  79(1964), 109-326.

\bibitem{Hor 1} H\"ORMANDER\pageoriginale L.: On the division of distributions by
  polynomials, Ark.\@ Math.\@ 3(1958), 555-568.

\bibitem{Hor 2} H\"ORMANDER L.: Fourier integral operators I, Acta
  Math.\@ 127(1971), 79-183.

\bibitem{Igu 1} IGUSA J.-I.: On the arithmetic of Pfaffians, Negoya
  Math.\@ J.\@ 47(1972), 169-198.

\bibitem{Igu 2} IGUSA J.-I.: Geometry of absolutely admissible
  representations, Number Theory, Algebraic Geometry and Commutative
  Algebra in honor of Y.\@ Akizuki, Kinokuniya, Tokyo (1973), 373-452.

\bibitem{Igu 3} IGUSA J.-I.: On a certain Poisson formula, Nagoya
  Math.\@ J.\@ 53(1974), 211-233.

\bibitem{Igu 4} IGUSA J.-I.: Complex powers and asymptotic expansions
  I, Functions of certain types, J.\@ reine angew.\@ Math.\@
  268/269(1974), 110-130; II, Asymptotic expansions, J.\@ reine
  angew.\@ Math.\@ 278/279(1975), 307-321.

\bibitem{Igu 5} IGUSA J.-I.: A Poisson formula and exponential sums,
  J.\@ Fac.\@ Sci., Univ.\@ Tokyo, 23(1976), 223-244.

\bibitem{Igu 6} IGUSA J.-I.: Some observations on higher degree
  characters, Amer.\@ J.\@ Math.\@ 99(1977), 393-417.

\bibitem{Igu 7} IGUSA J.-I.: On the first terms of certain asymptotic
  expansions, Complex Analysis and Algebraic Geometry, Iwanami Shoten
  and Cambridge Univ.\@ Press (1977), 357-368.

\bibitem{Igu 8} IGUSA J.-I.: Criteria for the validity of a certain
  Poisson formula, Algebraic Number Theory, Japan Soc.\@ Prom.\@
  Sci.\@ (1977), 43-65.

\bibitem{Igu 9} IGUSA J.-I.: Exponential sums associated with a
  Freudenthal quartic, J.\@ Fac.\@ Sci., Univ.\@ Tokyo, 24(1977),
  231-246. 

\bibitem{Jac} JACOBSON N.: Structure and representations of Jordan
  algebras, Amer.\@ Math.\@ Soc.\@ Colloq.\@ Publ.\@ 39, Providence,
  1968. 

\bibitem{Jea} JEANQUARTIER P.: D\'eveloppement asymptotique de la
  distribution de Dirac attach\'ee \`a une fonction analytique, C.R.\@
  Acad.\@ Sci., Paris 271(1970), 1159-1161.

\bibitem{Kub} KUBOTA T.: On an analogy to the Poisson summation
  formula for generalized Fourier transformation, J.\@ reine angew.\@
  Math.\@ 268/269 (1974), 180-189. 

\bibitem{Lang-Weil} LANG S.\@\pageoriginale and WEIL A.: Number of points of
  varieties in finite fields, Amer.\@ J.\@ Math, 76(1954), 819-827.

\bibitem{Lig} LIGHTHILL M.J.: Introduction to Fourier Analysis and
  Generalised Functions, Cambridge, 1958.

\bibitem{Mac} MACKEY G.W.: The Laplace transform for locally compact
  abelian groups, Proc.\@ Net.\@ Acad.\@ Sci., U.S.A.\@ 34(1948), 156-162.

\bibitem{Mal 1} MALGRANGE B.: Int\'egrales asymptotiques et
  monodromie, Ann.\@ Sci.\@ \'Ecole Norm.\@ Super., $4^{\text{e}}$
  Ser.\@ 7(1974), 405-430.

\bibitem{Mal 2} MALGRANGE B.: Le polyn\^ome de Bernstein d'une
  singularit\'e isol\'ee, Fourier Integral Operators and Partial
  Differential Equations, Colloq.\@ Int., Nice, 1974; Springer-Verlag
  Lecture Notes Math.\@ 459(1975), 98-119.

\bibitem{Mar} MARS J.G.M.: Let nombres de Tamagawa de certains groupes
  exceptionnels, Bull.\@ Soc.\@ Math.\@ France 94(1966), 97-140.

\bibitem{Mel} MELLIN H.: Abriss einer einheitlichen Theorie der
  Gamma-und der hypergeometrischen Funktionen, Math.\@ Ann.\@
  68(1910), 305-337.

\bibitem{Pon} PONTRJAGIN L.: Topologische Gruppen, I, II, B.G.\@
  Teubner Verlag.\@ Leipzig.\@ 1957.

\bibitem{Sch} SCHWARTZ L.: Th\'eorie des distributions, II, Hermann,
  Paris, 1951.

\bibitem{Sie 1} SIEGEL C.L.: \"Uber die analytische Theorie der
  quadratischen Formen I, Ann.\@ Math.\@ 36(1935), 527-606; II, Ann.\@
  Math.\@ 37(1936), 230-263; III, Ann.\@ Math.\@ 38(1937), 212-291:
  Gesamm Abhand., Springer-Verlag, 1966, Bd.I, 326-405, 410-443, 469-558.

\bibitem{Sie 2} SIEGEL C.L.: Symplectic geometry, Amer.\@ J.\@ Math.\@
  65(1943), 1-86; Gesamm.\@ Abhand., Springer-Verlag, 1966, Bd.II, 274-359.

\bibitem{Sie 3} SIEGEL C.L.: On the theory of indefinite quadratic
  forms, Ann.\@ Math.\@ 45(1944), 577-622; Gesamm.\@ Abhand.,
  Springer-Verlag, 1966, Bd.II, 421,466.

\bibitem{Sie 4} SIEGEL C.L.: A mean value theorem in geometry of
  numbers, Ann.\@ Math.\@ 46(1945), 313-339; Gesamm.\@ Abhand.,
  Springer-Verlag, 1966, Bd.III, 39-46.

\bibitem{Sie 5} SIEGEL C.L.: Lectures on Quadratic Forms, Tata
  Institute of Fundamental Research, 1956.

\bibitem{Sie 6} SIEGEL C.L.: Analytische Zahlentheorie, Lecture Notes,
  Gottingen, 1963. 

\bibitem{Tai} TAIBLESON M.H.:\pageoriginale Fourier Analysis on Local
  Fields, Princeton Univ.\@ Press and Univ.\@ Tokyo Press, 1975.

\bibitem{Tat} TATE J.: Fourier Analysis in Number Fields and Hecke's
  Zeta-functions, Thesis, Princeton, 1950; Algebraic Number Theory,
  Academic Press, 1967, 305-347.

\bibitem{Wei 1} WEIL A.: L'Int\'egration dans les Groupes Topologiques
  et ses Applications, Hermann, Paris, 1940.

\bibitem{Wei 2} WEIL A.: Sur quelques r\'esultats de Siegel, Summa
  Brasil, Mat.\@ 1(1945), Fasc.\@ 4, 21-39.

\bibitem{Wei 3} WEIL A.: Lectures on Adeles and Algebraic Groupes,
  Institute for Advanced Study, 1961.

\bibitem{Wei 4} WEIL A.: Sur certains groupes d'op\'erateurs
  unitaires, Acta. Math.\@ 111(1964), 143-211.

\bibitem{Wei 5} WEIL A.: Sur la formula de Siegel dans la th\'eorie
  des groupes classiques, Acta.\@ Math.\@ 113(1965) 1-87.

\bibitem{Wei 6} WEIL A.: Fonction zeta et distributions, S\'em.\@
  Bourbaki, n$^{\circ}$ 312(1966), 1-9.

\bibitem{Whi} WHITNEY H.: Analytic extensions of differentiable
  functions defined in closed sets, Trans.\@ Amer.\@ Math.\@ Soc.\@
  36(1934), 63-89.

\bibitem{Yam} YAMAZAKI T.: On a generalization of the Fourier
  transformation, Preprint.
\end{thebibliography}
