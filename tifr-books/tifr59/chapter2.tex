
\chapter{Dual Asymptotic Expansions}\label{chap2} %%%% chap2

THIS\pageoriginale CHAPTER IS essentially concerned with asymptotic expansions
which are dual to the asymptotic expansions mentioned in chapter I,
in a certain sense. We shall deal with the following two questions
about the spaces $\mathscr{F}$ and $\mathscr{Z}$ with the additional
assumption that $\lambda_{o}>0$ in the case of $\mathbb{R}$-fields $K$
and $Re(\lambda)>0$ for every $\lambda \epsilon \wedge $ for $p$-fields
$K$; namely, 
\begin{enumerate}
\renewcommand{\theenumi}{\roman{enumi}}
\renewcommand{\labelenumi}{(\theenumi)}
\item characterising the space $(\omega^{-1}\mathscr{F})^{\ast}$
  = $\{$ the Fourier transform \break $(\omega^{-1}F)^{\ast}F$ for $F\epsilon
    \mathscr{F}\}$ and
\item giving necessary and sufficient conditions for the function
  $\omega^{-1}F$ to have its Fourier transform as well in $L^{1}$, in
  terms of $\omega^{-1}F$ and $Z=MF$.
\end{enumerate}
 The reasons for our careful discussion of asymptotic expansions in
 chapters I and II will become evident when we deal, later on, with
 the local arithmetic theory of forms of higher degree and a poisson
 formula of the Siegel-Weil type for such forms. At this stage, by way
 of pointing  out just one reason, we shall merely highlight the
 following important problem. For locally compact abelian groups
 $X$and $G$ with a continuous map $f:X\rightarrow G$, Weil has defined,
 for any $\Phi$ in the space $\mathscr{Y}(X)$ of Schwartz-Bruhat
 functions on $X$ and any $g^{\ast}$ in the dual $G^{\ast}$ of $G$, 
\begin{equation*}
  F_{\Phi}^{\ast}(g^{\ast})=\int\limits_{X}\Phi(x)<f(x), g^{\ast}>dx
\end{equation*}
 where\pageoriginale $dx$ is the Haar measure on $X$ and
 $<f(x),g^{\ast}>=g^{\ast}(f(x))$. It is still a fundamental problem
 to determine precisely when, for a given $f$, the function
 $F_{\Phi}^{\ast}$ (clearly uniformly continuous on $G^{\ast}$)
 actually belongs to $L^{1}(G^{\ast})$ for every $\Phi$ in
 $\mathscr{Y}(X)$. In his well-known paper \cite{Wei 5}, Weil calls this
 the ``condition A`` and clarifies its meaning for quadratic forms
 $f$, in the case of local fields and adele-spaces. For forms of
 higher degree, one can similarly give necessary and sufficient
 conditions involving an invariant associated with the surface
 determined by $f$, at least in the case when $G=K$.

\section{Fourier Transforms of Quasi-characters}\label{chap2:sec1} %%% sec1

 We shall quickly recall well-known results concerning Fourier
 transforms of quasi-characters. 

\subsection{Notation}\label{chap2:sec1:subsec1} %%% subsec 1.1

 Let $X$ be a finite-dimensional vector space over a local field $K$
 with $\Psi$, a nontrivial character of $K$. Let $[x, y]$ be a
 symmetric non-degenerate $K$-bilinear form $X \times X$, $|dx|$ denote a
 Haar measure on $X$ and let $X$ be identified with its dual
 $X^{\ast}$. 

 For every $\Phi$ in $L^{1}(X)$, we define
\begin{equation*}
  \Phi^{\ast}(x)=\int\limits_{x}\Phi(y)\Psi([x, y])\quad|dy|
\end{equation*}
If $\wedge(X)$ is the space of continuous functions $\Phi$ in
$L^{1}(X)$ such that $\Phi^{\ast}$ is also in $L^{1}(X)$,then the
measure $|dx|$ can be normalised uniquely in such a manner that
$(\Phi^{\ast})^{\ast}(x)=\Phi(-x)$ for every $\Phi$ belonging to
$\wedge(X)$. We shall then call\pageoriginale $|dx|$ the self-dual measure relative
to $\Psi([x,y])$.

 If $\mathfrak{e}(\bigdot)=exp(2 \pi \sqrt{-1}\bigdot)$.then for
 $\Psi$, we have the standard choice
\begin{equation*}
  \Phi(x)=
  \begin{cases}
    \mathfrak{e}(x) ~\text{for}~ x\epsilon K=\mathbb{R}\\
    \mathfrak{e}(2 Re(x)) ~\text{for}~ x\epsilon K=\mathbb{C}
  \end{cases}
\end{equation*}
In the case of $p$-fields $K$, we take a $\Psi$ which is $1$ on $R$ and
non-trivial on $P^{-1}$. Further,if $[x,y]=xy$ for $x,y\epsilon K$,
then $|dx|$ is the same as the measure $dx$ in chapter I, namely
\begin{equation*}
  dx=
  \begin{cases}
    \text{usual measure for}~ K=\mathbb{R}\\
    \text{twice the usual measure for } K=\mathbb{C}\\
    \text{the measure with } M(R)=1 ~\text{for $p$-fields} K
  \end{cases}
\end{equation*}

\subsection{The Space $\mathscr{Y}(X)$ of Schwartz-bruhat Functions
  on $X$ and the Space $\mathscr{Y}(X)'$ of Tempered Distributions on
  $X$}\label{chap2:sec1:subsec2} %%% subsec1.2

The space $\mathscr{Y}(\mathbb{R}^{n})$ of Schwartz functions on
$\mathbb{R}^{n}$ consists of all $C^{\infty}$ functions from
$\mathbb{R}^{n}$ to $\mathbb{C}$ such that for every linear
differential operator $D=\sum a_{\alpha\beta}x^{\alpha}D^{\beta}$ with
$x^{\alpha}=x_{1}^{\alpha_{1}},\ldots x_{n}^{\alpha_{n}},
D^{\beta}=\frac{\partial_{1}^{\beta_{1}+\ldots\beta_{n}}}{\partial
  x_{1}^{\beta_{1}}\ldots\partial
  x_{n}^{\beta_{n}}},\alpha_1,\ldots,\alpha_{n},$
$\beta_{1},\ldots\break\beta_{n}\ge 0$ in $\mathbb{Z}, a_{\alpha
  \beta}\epsilon \mathbb{C}$ and $x=(x_{1},\ldots,x_{n})\epsilon
\mathbb{R}^{n}$, we have
$||D\quad\Phi||_{\infty}={\displaystyle{\mathop{\sup}_{x\epsilon
      \mathbb{R}^{n}}}}$\break 
 $|D\Phi(x)|<\infty$. We topologize $E=\mathscr{Y}(\mathbb{R}^{n})$ with
respect to the family of seminorms $\Phi\mapsto||D\Phi||_{\infty}$, as
$D$ varies over all such linear differential operators with polynomial
coefficients. Thus, a sequence $\{\Phi_{n}\}_{n\ge0}$ in $E$ tends to
$0$ if (and only if), for every $D$ as above, the sequence
$\{||D\Phi_{n}||_{\infty}\}_{n}$\pageoriginale of real numbers tends to $0$ as
$n\rightarrow \infty$. The same topology can also be obtained as
follows. Namely, we rearrange
$\{||(x^{\alpha}D^{\beta})(\Phi)||_{\infty};
\alpha=(\alpha_{1},\ldots,\alpha_{n}),
\beta=(\beta_{1},\ldots,\beta_{n}),\alpha_{i},\beta_{j}\ge 0$ in
$\mathbb{Z} ~\text{for all}~ i, j\}$ as $c_{1}(\Phi)$, $c_{2}(\Phi)$,
$\dots$, and define
\begin{equation*}
  ||\Phi||=\sum\limits_{m=1}^{\infty}\frac{1}{2^{m}}\quad
  \frac{c_{m}(\Phi)}{1+c_{m}(\Phi)}. 
\end{equation*}
Then we have
\begin{enumerate}
\renewcommand{\theenumi}{\roman{enumi}}
\renewcommand{\labelenumi}{(\theenumi)}
\item $||\Phi||\quad =||-\Phi||\ge 0$ with equality only when $\Phi=0$;
\item $||\Phi+ \psi||\le ||\Phi||+||\psi||$ for every $\Phi, \psi$ in $E$; and 
\item the scalar multiplication $(\lambda,\Phi)\mapsto\lambda\Phi$
  from $\mathbb{C}\times E$ to $E$ is continuous. 
\end{enumerate}
  While (i) and (ii) make $E$ a metric space with $||\Phi -\psi||$ as
  the distance between  $\Phi$ and $\psi$ and $||\quad||$ being fairly
  close to a norm, (iii) in addition, makes $E$ a topological vector
  space. It is direct verification to check that $E$ is complete. The
  dual $E'$ consisting of all complex-valued $\mathbb{C}$-linear
  continuous maps of $E$ is called the space of tempered distributions
  on $X$; since $E$ satisfies   (i), (ii), (iii) and is complete, by
  the Banach-Steinhaus theorem. 

 We now take up the definition of $\mathscr{Y}(X)$, $\mathscr{Y}(X)'$
 for the case of $p$-fields $K$.  Let $G$ be a locally compact abelian
 group. We say that $G$ has arbitrarily large (respectively
 arbitrarily small) compact open subgroups $H$, if for every compact
 subset $C$ of $G$ (respectively neighbourhood $V$ of $0$) there
 exists such an $H$  containing $C$ (respectively contained in
 $V$). Whenever $G$ has arbitrarily large and arbitrarily\pageoriginale small
 compact open subgroups, we define the Schwartz-Bruhat space
 $\mathscr{Y}(G)$ as the space of all complex-valued functions $\Phi$
 on $G$ which are locally constant (i.e, constant in some
 neighbourhood of every point) and which have, in addition, compact
 support. If, for any subset $S$ of $G$, we denote by $\varphi_{S}$ the
 characteristic function of $S$, then $\mathscr{Y}(G)$ is just the
 $\mathbb{C}$-span of the functions $\varphi_{a+H}$ with a varying
 over $G$ and $H$ over subgroups described above. We may take
 $\mathscr{Y}(G)'$ as just the algebraic dual of $\mathscr{Y}(G)$,
 which consists of all $\mathbb{C}$-linear function on
 $\mathscr{Y}(G)$; it is not necessary to topologies $\mathscr{Y}(G)$,
 in order to define the dual space $\mathscr{Y}(G)'$.

 Both in the case of $\mathbb{R}$-fields and $p$-fields $K$, the Fourier
 transformation $\Phi \mapsto \Phi^{\ast}$ gives an automorphism of
 $\mathscr{Y}(X)$ and hence an automorphism $T\mapsto T^{\ast}$ of the
 dual $\mathscr{Y}(X)'$, if we set $T^{\ast}(\Phi)=T(\Phi^{\ast})$ for
 every $\Phi\epsilon\mathscr{Y}(X)$.

 The following two definitions are needed for our purposes, later
 on. Let $\varphi \epsilon L_{\text{loc}}^{1}(X)$ i.e, $\varphi$ is a
 complex-valued locally integrable function on $X$ which has, in
 addition, at most polynomial growth (at infinity) whenever $K$ is an
 $\mathbb{R}$-field. We can then define a tempered distribution
 $T_{\varphi}\epsilon\mathscr{Y}(X)'$ by simply setting, for every
 $\Phi$ in $\mathscr{Y}(X)$,
 \begin{equation*}
   T_{\varphi}(\Phi)=\int\limits_{X}\Phi(x)\varphi(x)\quad|dx|
   \tag{50}\label{chap2:sec1:subsec2:eq50} 
\end{equation*}
For the case of $p$-fields, only the convergence of  the  integral needs
to be verified and this is taken care of by $\Phi$ having compact
support and $\varphi$ being locally integrable. For
$\mathbb{R}$-fields, the convergence of the integral at infinity is
made possible by $\Phi$ being Schwartz function and $\varphi$ being at
most of polynomial growth there;\pageoriginale as for the continuity, it is easy to
check that $T_{\varphi}(\Phi)\rightarrow 0$ as $\Phi\rightarrow 0$ in
$\mathscr{Y}(X)$. 

 We remark that under the imbedding $\Phi\rightarrow
 T_{\Phi},\mathscr{Y}(X)$ is dense in $\mathscr{Y}(X)'$. Given $T$ in
 $\mathscr{Y}(X)'$, $U$ open in $X$ and $f\epsilon
 L_{\text{loc}}^{1}(U)$, we say that $T=f$ in $U$, if 
\begin{equation*}
  T(\Phi)=\int\limits_{U}\Phi(x)f(x)\quad|dx|
  \tag{51}\label{chap2:sec1:subsec2:eq51}
\end{equation*}
 for every $\Phi$ in $\mathscr{Y}(X)$ with support of $\Phi$ contained
 in $U$. 

\subsection{}\label{chap2:sec1:subsec3} %%% subsec1.3

In the case when $K$ is a $p$-filed, we shall have occasion to use the
following facts frequently. 

 We shall assume $\psi$ to be standard i.e, $\psi$ is $1$ on $R$ and
 $\psi$ is non-constant on $P^{-1}$. For any
 $\mathcal{X}(R^{\times})^{\ast}$, we denote by $e_{\mathcal{X}}$ the
 smallest natural number $e$ such that $\mathcal{X}$ is 1 on the set
 $1+P^{e}$; thus, by definition, we have always $e_{\mathcal{X}}\ge
 1$. We assert that the following formulae are valid: 
\begin{equation*}
  \int\limits_{R^{\times}}\mathcal{X}(u)(\pi^{-e}u)du =
  \begin{cases}
    1-1/q &\text{for } e\le 0 ~\text{and } \mathcal{X}=1 \\
    -1/q &\text{for } e\le 1 ~\text{and }\mathcal{X}=1 \\
    0 &\text{for } e>1 ~\text{and }\mathcal{X}=1\\
    0 &\text{for } e\neq e_{\mathcal{X}} ~\text{and } \mathcal{X}\neq 1
  \end{cases}\tag{52}\label{chap2:sec1:subsec3:eq52}
\end{equation*}

 Formulate (\ref{chap2:sec1:subsec3:eq52}) for $\mathcal{X}=1$ are straightforward. We shall
 therefore prove only that integral is $0$ for $\mathcal{X}\neq 1$ and
 $e\neq e_{\mathcal{X}}$. First, if $e>e_{\mathcal{X}}$, then the
 integral is the same as 
\begin{align*}
  & =\sum_{a\epsilon R^{\times}, \mod p^{e-1}} \int\limits_{p^{e-1}}
  \mathcal{X}(a(1+x))\psi(\pi^{-e}a(1+x))dx\\
  & =\sum_{a\epsilon R^{\times}, \mod P^{e-1}} \mathcal{X}(a) 
  \psi(\pi^{-e}a)\int\limits_{P^{e-1}}\psi(\pi^{-e}x)dx
\end{align*}

 The\pageoriginale last integral over $P^{e-1}$ is $0$, since $P^{e-1}\ni x
 \rightarrow\psi(\pi^{-e}x)$ is a non-trivial character. Similarly, if
 $e<e_{\mathcal{X}}$ and $e_{\mathcal{X}}\ge 2$, then the integral on the
   left side of (\ref{chap2:sec1:subsec3:eq52}) is equal to
\begin{align*}
  & =\sum_{a\epsilon R^{\times},\mod P^{e_{\mathcal{X}^{-1}}}}
  \int\limits_{P^{e_{\mathcal{X}-1}}}\mathcal{X}(a(1+x))\psi(\pi^{-e}a(1+x))dx\\
  & =\sum\limits_{a\epsilon R^{\times},\mod
  P^{e_{\mathcal{X}}-1}} \mathcal{X}(a)\psi(\pi^{-e}a)
  \int\limits_{p^{e_{\mathcal{X}} -1}}\mathcal{X}(1+x)dx 
\end{align*}
 and the integral over $P^{e_{\mathcal{X}^{-1}}}$ is $0$ since
 $P^{e_{\mathcal{X}^{-1}}}\ni x\rightarrow \mathcal{X}(1+x)$ is a
 non-trivial character of $P^{e_{\mathcal{X}^{-1}}}$. Finally, if
 $e<e_{\mathcal{X}}=1$, the left hand side of
 (\ref{chap2:sec1:subsec3:eq52}) is just 
 $\int\limits_{R^{\times}}\mathcal{X}(u)du=0$, since $\mathcal{X}\neq
 1$.

We have the following supplement to (\ref{chap2:sec1:subsec3:eq52}). For any
$\mathcal{X}$(possibly equal to 1) in $(R^{\times})^{\ast}$, let us
define
\begin{equation*}
  g_{\mathcal{X}}=\int\limits_{R^{\times}}\mathcal{X}(u)\psi
  (\pi^{-e\mathcal{X}}u)du\tag{53}\label{chap2:sec1:subsec3:eq53}
\end{equation*}
Then, trivially, we have
\begin{equation*}
  \overline{g}_{\mathcal{X}}=\mathcal{X}(-1)g_{\mathcal{X}^{-1}}
  \tag{54}\label{chap2:sec1:subsec3:eq54}
\end{equation*}
Moreover, it is clear from (\ref{chap2:sec1:subsec3:eq52}) that 
\begin{equation*}
  g_{1}=-q^{-1}\tag{55}\label{chap2:sec1:subsec3:eq55}
\end{equation*}
 If we write $u=a+\pi^{e_{\mathcal{X}}}v$ and let a run over
 representatives of $R^{\times}$ modulo $P^{e_{\mathcal{X}}}$ and $v$
 over $R$, then we can rewrite (\ref{chap2:sec1:subsec3:eq53}) as 
\begin{equation*}
  g_{\mathcal{X}}=q^{-e_{\mathcal{X}}}\sum\limits_{a\epsilon
    R^{\times}\mod
    p^{e_{\mathcal{X}}}}  \mathcal{X}(a) \psi
  (\pi^{-e_{\mathcal{X}}}a) \tag{56}\label{chap2:sec1:subsec3:eq56}  
\end{equation*}
Thus $q^{e_{\mathcal{X}}}g_{\mathcal{X}}$ is a standard Gaussian sum
and for $\mathcal{X}\neq 1$, we know indeed that 
\begin{equation*}
  |g_{\mathcal{X}}|^{2}=q^{-e_{\mathcal{X}}}\tag{57}\label{chap2:sec1:subsec3:eq57}  
\end{equation*}

\subsection{}\label{chap2:sec1:subsec4}   %%% subsec1.4

 We are now in a position to recall the following well-known table of
 Fourier transforms of quasicharacters; for $K=\mathbb{R}$, one may
 refer to M.J.Lighthill \cite{Lig} or I.M.Gel'fand and G.E. Shilov
 (\cite{Gel-Shi}).
 
 For\pageoriginale any quasicharacter $\omega$ in $\Omega_{+}(K^{\times})$, i.e, with
 $\sigma(\omega)>0$, the function $\omega\omega_{1}^{-1}$ is in
 $L_{loc}^{1}(k)$ as may be verified at once from the convergence of
 the integral$\int|x|_{k}^{\sigma-1}dx$ taken over a compact
 neighbourhood of $0$. Thus $T_{\omega\omega_{1}^{-1}}$ in the sense
 of (\ref{chap2:sec1:subsec2:eq50}) is defined and therefore, also
 $(T_{\omega\omega_{1}^{-1}})^{\times}$, although the latter is not, in
 general, of the form $T_{\varphi}$ for some $\varphi$. But if we
 restrict it to $K^{\times}$, then 
\begin{equation*}
  (T_{\omega\omega_{1}^{-1}})^{\times}\quad = b(\omega)\omega^{-1}
  \text{on } K^{\times}\tag{58}\label{chap2:sec1:subsec4:eq58}
\end{equation*}
 for a constant $b(\omega)$ explicitly described below: namely, let
 $\psi$ be standard and let $d=1/2$ or $1$ according as $K=\mathbb{R}$
 or $\mathbb{C}$-field  $K$ and for $\omega=\omega_{s}(ac)^{p}$,we
 have
\begin{equation*}
  b(\omega){\displaystyle{\mathop{=}^{defn}}}b_{p}(s)=i^{|p|}
  (2d\pi)^{d(1-2s)}\frac{\Gamma(ds+|p|/2)}{\Gamma(d(1-s)+|p|/2)}
\end{equation*}
 and for a $p$-field $K$, we have, for $\omega=\omega_{s}\mathcal{X}$,
\begin{equation*}
  b(\omega){\displaystyle{\mathop{=}^{defn}}}.b_{\mathcal{X}}(s)=
  \begin{cases}
    \frac{1-q^{-(1-s)}}{1-q^{-s}}&\text{for }\mathcal{X}=1\\
    g_{\mathcal{X}}q^{e}{\mathcal{X}}^{s}&\text{for }\mathcal{X}\neq 1
  \end{cases}\tag{59}\label{chap2:sec1:subsec4:eq59}
\end{equation*}

 A transparent proof for the complete table (\ref{chap2:sec1:subsec4:eq59}) can be found in
 Weil's expos\'{e} (\cite{Wei 6}) and we shall give a brief outline  of his
 method. 
 
The natural action of $K^{\times}$ on $K$ induces the action of
 $K^{\times}$ on $\mathscr{Y}(K)$ given by $\Phi\mapsto g\Phi$ with
 $(g^{\phi})(x)=\Phi(g^{-1}x)$ and hence, by duality, the action on
 $\mathscr{Y}(K)'$ given by $T\mapsto gT$ with
 \begin{equation*}
   (gT)(\Phi)=T(g^{-1}\Phi),i,e. (gT)(\Phi(x))=T(\Phi(gx))
 \end{equation*}
 for\pageoriginale $\Phi$ in $\mathscr{Y}(K)$. It can be immediately verified that for
 $T=T_{\omega\omega_{1}^{-1}}$ (in the sense of (\ref{chap2:sec1:subsec2:eq50})) with
 $\omega\epsilon\Omega_{+}(K^{\times})$, we have
 \begin{align*}
   (gT)(\Phi)& =\int\limits_{k}\Phi(gx)(\omega\omega_{1}^{-1})(x)|dx|\\
   &
   =\omega(g)^{-1}\int\limits_{k}\Phi(x)(\omega\omega_{1}^{-1})(x)|dx|\quad{using
     x\mapsto g^{-1}x}\tag{60}\label{chap2:sec1:subsec4:eq60}\\
   & =\omega(g)^{-1}T(\Phi) 
\end{align*}
for every $g\epsilon K^{\times}$ and $\Phi \epsilon \mathscr{Y}(K)$.
 
For a given $\rho\epsilon\Omega(K^{\times})$, suppose
 $\triangle\epsilon \mathscr{Y}(K)'$ satisfies the condition $g
 \triangle =\rho(g)^{-1}\triangle$ for every $g$ in $K^{\times}$. Then
 we shall write $\triangle \epsilon P(K,K^{\times};\rho)$.

 Taking the restriction of $\triangle$ to $K^{\times}$ i.e, by requiring
 $\Phi$ in $\triangle(\Phi)$ to have support contained in $K^{\times}$,
 we get a distribution on $K^{\times}$, which is relatively invariant
 under the action of $K^{\times}$ (i.e, picking up a factor $\rho(g)$
 under the multiplication by $g$ in $K^{\times}$). Then, by a method
 similar to the proof of  the uniqueness of a relatively invariant
 measure (on $K^{\times}$), we can show that 
\begin{equation*}
  \triangle~\text{on}~ k^{\times}\quad = c(\rho\omega_{1}^{-1}) (x)dx
  \tag{61}\label{chap2:sec1:subsec4:eq61}
\end{equation*}
for a complex constant $c$ which may be $0$. For instance, this proof
goes as follows in the case of a $p$-field $K$: if we put 
\begin{equation*}
\triangle'(\Phi)=\triangle(\rho^{-1}\Phi)
\end{equation*}
 for every $\Phi$ in $\mathscr{Y}(K)$ with support of $\Phi$ contained
 in $K^{\times}$, then $\triangle'$ becomes even invariant under the
 action of $K^{\times}$. Since, for every $e\ge 1$, we have 
\begin{equation*}
  \varphi_{1+p^{e}}(x)=\sum\limits_{g\epsilon 1+p^{e},\mod
  p^{e+1}}\varphi_{1+p^{e+1}}(g^{-1}x) 
\end{equation*}
 we\pageoriginale obtain, by the invariance of $\triangle'$, that
\begin{align*}
  q^{e}.\triangle'(\varphi_{1+p^{e}})
  & = q^{e+1}\triangle'(\varphi_{1+p}e+1)\\
  & = \text{a constant} c', ~\text{say};
\end{align*}
then, for every a in $K^{\times}$ and $e\ge 1$, we have
\begin{equation*}
  \triangle'(\varphi_{a(1+p^{e})})=c'q^{-e}=c'
  \int\limits_{K^{\times}}\varphi_{a(1+p^{e})}(x)\frac{dx}{|x|}_{k}
\end{equation*}

Then implies
\begin{equation*}
  \triangle'=c'\frac{dx}{|x|_{k}} i.e,
  \triangle=c'(\rho\omega_{1}^{-1})(x)dx ~\text{in}~ K^{\times}.
\end{equation*}

Coming back to $\triangle\epsilon P(K,K^{\times}; \rho)$, we see, from
the simple relation
$(g^{-1}\Phi)^{\times}=\omega_{1}(g)^{-1}g^{\Phi^{\ast}}$ for every $\Phi$
in $\mathscr{Y}(K)$, that
\begin{equation*}
  \triangle \epsilon P(K, K^{\times};\rho) \Rightarrow
  \triangle^{\times}
  \epsilon(K,K^{\times},\rho^{-1}\omega_{1})
\end{equation*}
Taking $\triangle=T_{\omega\omega_{1}^{-1}}$ for  $\omega\epsilon
\lambda_{+}(K^{\times}) $ and $\omega$ instead of $\rho$, we can see, in
view of (\ref{chap2:sec1:subsec4:eq60}) and
(\ref{chap2:sec1:subsec4:eq61}), that the relation
(\ref{chap2:sec1:subsec4:eq58}) is valid. The 
function $b(\omega)$ can be determined by evaluating both sides of
(\ref{chap2:sec1:subsec4:eq58}) at a suitable $\Phi$. For example, in
the case of a  $p$-field 
$K$, we can take for $\Phi$, the characteristic function of
$1+p^{e}\mathcal{X}$ and then we get the expression for
$b_{\mathcal{X}}(s)$ in (\ref{chap2:sec1:subsec4:eq59}). We shall give some details in the case
of a $p$-field $K$: if $\varphi_{E}$ denotes the characteristic function
of a compact open subset $E$ of $K$, then, for any $a$, in $K$, and
$e$ in $\mathbb{Z}$, we have
\begin{align*}
  \varphi_{a+P^{e}}^{\times}(x)& =\int\limits_{a+p^{e}}\psi(xy)dy 
  =\psi(ax)\int\limits_{P^{e}}\psi(xy)dy\\
  & =\psi(ax)\left\{\underset{\displaystyle{0 \text{
        otherwise},}}{m(P^{e})\text{ if }x\epsilon P^{e}}\right\}\\
  \text{i.e.}\quad  \varphi_{a+P^{e}}^{\times}(x)
  & = q^{-e}\psi(ax)\varphi_{P^{-e}}(x)\tag{62}\label{chap2:sec1:subsec4:eq62}
\end{align*}
If\pageoriginale $\sigma (\omega)>0$ and $\mathcal{X}=1$, then for
$\Phi=\varphi_{1+P}$, we obtain, using (\ref{chap2:sec1:subsec3:eq52}), that 
\begin{align*}
  b(\omega)=\int\limits_{P^{-1}}\psi(x)|x|_{k}^{s-1}dx
  & =\sum\limits_{e=-1}^{\infty}q^{-es}\int\limits_{R^{\times}}\psi(\pi^{e}u)du\\
  & =-q^{s-1}+(1-q^{-1})\sum\limits_{e=0}^{\infty}q^{-es}\\
  & =-q^{s-1}+\frac{1-q^{-1}}{1-q^{-s}}\\
  & =\frac{1-q^{s-1}}{1-q^{-s}}
\end{align*}
 If, on the other hand, $\mathcal{X}\neq 1$, then for
 $\Phi=\varphi_{1+P} e_{\mathcal{X}}$, we get again, in view of
   (\ref{chap2:sec1:subsec3:eq52}), that 
\begin{align*}
  b(\omega)&
  =\int\limits_{P^{-e}_{\mathcal{X}}}\psi(x)|x|_{K}^{s-1}\mathcal{X}(ac(x))dx\\ 
  & =\sum\limits_{e=-e_{\mathcal{X}}}^{\infty}q^{-es}
  \int\limits_{R^{\times}}\psi(\pi^{e}u)\mathcal{X}(u)du\\
  & =q^{e_{\mathcal{X}^{s}}} g_{\mathcal{X}}
\end{align*}

\subsection{}\label{chap2:sec1:subsec5} %%% subsec1.5

 For a moment, we shall write $\mathcal{X}$ for $(ac)^{p}$ as well and
 correspondingly denote $b_{p}(s)$ as $b_{\mathcal{X}}(s)$. If $Re(s)>0$,
 then $\varphi(x)=\mathcal{X}(ac(x))|x|_{K}^{s-1}(\log$ $|x|_{K})^{m-1}$
 is locally integrable on $K$ (and at most of polynomial growth at
 infinity for $\mathbb{R}-\text{fields }K)$. Further, for the Fourier
 transform of $\varphi$ (which we shall identify with the distribution
 $T_{\varphi}$), we have the relation
\begin{gather*}
(\mathcal{X}(ac(x))|x|_{K}^{s-1}(\log|x|_{K})^{m-1})^{\times}\quad
  \text{on}~K^\times =\tag{63}\label{chap2:sec1:subsec5:eq63}\\
  =\sum\limits_{j=1}^{m}(-1)^{j-1}\binom{m-1}{j-1}\frac{d^{m-j}
    b_{\mathcal{X}}(s)}{ds^{m-j}} \mathcal{X}(ac(x))^{-1}|x|_{K}^{-s}
  (\log|x|_{K})^{j-1}
\end{gather*}
(in\pageoriginale the sense of the definition at the end of
\ref{chap1:sec1:subsec2}). The proof of 
(\ref{chap2:sec1:subsec5:eq63}) is as follows. If $\sigma(\omega)>0$, then 
\begin{equation*}
  \left(\frac{d^{m-1}}{ds^{m-1}}(\omega\omega_{1}^{-1})\right)^{\times}
  =\frac{d^{m-1}}{ds^{m-1}}(\omega\omega_{1}^{-1})^{\times}\tag{64}\label{chap2:sec1:subsec5:eq64} 
\end{equation*}
and moreover, the left hand side of (\ref{chap2:sec1:subsec5:eq63}) is
the same as that of 
(\ref{chap2:sec1:subsec5:eq64}); this is an identity in $\mathscr{Y}(K)'$. If now we restrict
the right hand side of (\ref{chap2:sec1:subsec5:eq64}) to
$K^{\times}$, then in view of (\ref{chap2:sec1:subsec4:eq58}), it
becomes equal to  
\begin{equation*}
\frac{d^{m-1}}{ds^{m-1}}(b(\omega)\omega^{-1})(x)=\frac{d^{m-1}}{ds^{m-1}}(b_{\mathcal{X}}(s)_{\mathcal{X}}(ac(x))^{-1}|x|_{K}^{-s});
\end{equation*}
 on applying the Leibnitz formula, it coincides with the right hand
 side of (\ref{chap2:sec1:subsec3:eq52}). 
 
 If we take $\mathcal{X}=1$, $s=1$, $m=1$ in
 (\ref{chap2:sec1:subsec5:eq63}), the left hand 
 side becomes just $1^{\times}$ or the Dirac distribution $\delta_{0}$
 with $\{0\}$ as support while the right hand side is $0$ since
 $b_{1}(1)=0$, by (\ref{chap2:sec1:subsec4:eq59}); thus $\delta_{0}=1$ on $K^{\times}$, which is
 indeed true!

\section{The Space
  ($\omega_{1}^{-1}\mathscr{F})^{\times}$}\label{chap2:sec2}  %%% sec2

\subsection{Statement of a Theorem}\label{chap2:sec2:subsec1} %%% subsec 2.1

\begin{enumerate}
\renewcommand{\theenumi}{\Alph{enumi}}
\renewcommand{\labelenumi}{(\theenumi)}
\item  Given a strictly increasing sequence $\{\lambda_{k}\}_{k\ge 0}$ of
positive real numbers with no finite accumulation point and a sequence
$\{m_{k}\}_{k\ge 0}$ of natural numbers, the space
$(\omega_{1}^{-1}\mathscr{F})^{\times}=\{(\omega_{1}^{-1}G)^{\times};G\epsilon
\mathscr{F}\}$ in  the case of an $\mathbb{R}$-field $K$, consists of
all complex-valued\pageoriginale $C^{\infty}$ function $F^{\#}$ on $K$, with a
termwise differentiable uniform asymptotic expansion of the form
\begin{equation*}
  F^{\#}(x)\thickapprox \sum\limits_{k=0}^{\infty}
  \sum\limits_{m=1}^{m_{k}}a_{k,m}^{\#}(ac(x))|x|_{K}^{-\lambda_{k}}
  (\log|x|_{K})^{m-1}\tag{65}\label{chap2:sec2:subsec1:eq65}
\end{equation*}
as $|x|_{K}\rightarrow \infty$ with $a_{k,m}^{\#}$ denoting a
$C^{\infty}$ function on $K_{1}^{\times}$. Furthermore, if $F^{\#}$ is
the same as the Fourier transform $F^{\times}$ of
$F\epsilon\omega_{1}^{-1}\mathscr{F}$ which has the termwise
differentiable asymptotic expansion 
\begin{equation*}
F(x)\thickapprox
\sum\limits_{k=0}^{\infty}\sum\limits_{m=1}^{m_{k}}a_{k,m}(ac(x))|x|_{K}^{\lambda_{k}-1}(\log |x|_{K})^{m-1}\tag{66}\label{chap2:sec2:subsec1:eq66}
\end{equation*}
as $|x|_{K}\rightarrow 0$, then (\ref{chap2:sec2:subsec1:eq65}) is the
termwise Fourier transform of the expansion (\ref{chap2:sec2:subsec1:eq66}). 

\item Given a finite set $\wedge=\{\lambda \mod 2\pi i/\log q;
  Re(\lambda)>0\}$ and natural numbers $m_{\lambda}$ for every
  $\lambda$ in $\wedge$, the space
  $(\omega_{1}^{-1}\mathscr{F})^{\times}=\{(\omega_{1}^{-1}G)^{\times}$;
  $G\epsilon
  \mathscr{F}\}$ in the case of a $p$-field $K$, consists of all
  complex-valued locally constant functions $F^{\#}$ on $K$ such that
\begin{equation*}
  F^{\#}(x)=\sum\limits_{\lambda \epsilon \wedge}
  \sum\limits_{m=1}^{m_{\lambda}}a_{\lambda,m}^{\#}ac(x))
  |x|_{k}^{\lambda}(\log|x|_{K})^{m-1}\tag{67}\label{chap2:sec2:subsec1:eq67}
\end{equation*}
for all sufficiently large $|x|_{K}$. Further, if $F^{\#}$ is the
Fourier transform $F^{\times}$ of an $F$ in $\omega_{1}^{-1}\mathscr{F}$
for which
\begin{equation*}
F(x)=\sum\limits_{\lambda \epsilon
  \wedge}\sum\limits_{m=1}^{m_{\lambda}}a_{\lambda,
  m}(ac(x))|x|_{K}^{\lambda-1}(\log
|x|_{K})^{m-1}\tag{68}\label{chap2:sec2:subsec1:eq68} 
\end{equation*}
 for all sufficient small $|x|_{K}$, then
 (\ref{chap2:sec2:subsec1:eq67}) is obtained from 
 (\ref{chap2:sec2:subsec1:eq68}) by taking the  Fourier transform termwise.
\item In both the cases $(A)$ and $(B)$, the inverse of the map
  $F\mapsto F^{\times}$ from  $\omega_{1}^{-1}\mathscr{F}$ to
  $(\omega_{1}^{-1}\mathscr{F})^{\times}$ is given by
\begin{equation*}
  F(x)=\lim\limits_{r\rightarrow \infty}\int\limits_{|y|_{K}\le r}
  F^{\times}(y)\psi(-xy)dy, \text{ for } x\epsilon K^{\times}
  \tag{69}\label{chap2:sec2:subsec1:eq69} 
\end{equation*}
\end{enumerate}

\begin{Remarks*}
 For\pageoriginale the proof of part (A) of Theorem \ref{chap2:sec2:subsec1} in the case
 $k=\mathbb{R}$, the reader may refer to \cite{Lig}, Chapter 4,
 \S \ref{chap4:sec4:subsec2}, \S \ref{chap4:sec4:subsec3}, A partial
 form of (A) for 
 $K=\mathbb{R}$ is the principal result proved by Lighthill who has
 shown that for $F$ in $\omega_{1}^{-1}\mathscr{F}$ the Fourier
 transform $F^{\times}$ has an asymptotic expansion of the form
 (\ref{chap2:sec2:subsec1:eq65})
 as $|x|\rightarrow \infty$, without discussing its termwise
 differentiability, however. In this context, we merely remark that
 the differentiability of $F^{\times}$ is immediate and the termwise
 differentiability of the asymptotic expansion for $F^{\times}(x)$ as
 $|x|_{K}\rightarrow \infty$ follows from the stability of
 $\omega_{1}^{-1}\mathscr{F}$ and
 $(\omega_{1}^{-1}\mathscr{F})^{\times}$ under any homothety-invariant
 differential operator. In fact, for $K=\mathbb{R}$, we get
 $(DF)^{\times}={-F}^{\times}-DF^{\times}$ with $D=x\frac{d}{dx}$ and
 therefore, the asymptotic expansions of $-(DF^{\times})(x)$ as
 $|x|\rightarrow \infty$ is obtained simply as the sum, taken
 termwise, of the asymptotic expansion of $F^{\times}(x)$ and
 $(DF)^{\times}(x)$ as $|x|\rightarrow \infty$; we can easily verify
 that this is the same as the asymptotic expansion of $F^{\times}(x)$ as
 $|x|\rightarrow \infty$ being termwise differentiable once and by
 repeated applications of $D$, the asymptotic expansion of
 $F^{\times}(x)$ is seen to be termwise differentiable (i.e, any number
 of times). The proof of part (A) for the case$K=\mathbb{C}$ is left
 as an exercise. We therefore give only the proof for (B) and our
 proof given below for (B) is an adaptation to the case of $p$-fields,
 of Lighthill's proof mentioned above; actually, in the case of
 $p$-fields, we can also give a proof entirely avoiding the concept of
 distributions and using the properties of the integral considered in
 (\ref{chap2:sec1:subsec3:eq52}).
 
 Before we proceed to consider assertions (B) and (C) of
 Theorem \ref{chap2:sec2:subsec1},
 let us prove the following lemma.
\end{Remarks*}

\begin{lemma}\label{chap2:sec2:subsec1:lem1} %%% lem2.1
  For\pageoriginale arbitrary $s$ in $\mathbb{C}$ and $e_{o}$ in
  $\mathbb{Z}$, let 
  \begin{equation*}
    \varphi(x)=
    \begin{cases}
      \mathcal{X}(ac(x))|x|_{K}^{s-1}(\log |x|_{K})^{m-1}
      & \text{ for ord }(x)<e_{0}\\
      \qquad\qquad 0 & \text{ for ord}(x)\ge e_{o}
    \end{cases}
  \end{equation*}
  where $\mathcal{X}\epsilon (R^{\times})^{\ast}$. Then
  $\varphi^{\times}=0$, in $K\backslash p^{-e}\mathcal{X}^{e_{o}+1}$
\end{lemma}

\begin{proof}
If $(a+P^{e})\cap P^{-e_{\mathcal{X}}-e_{0}+1}\neq \phi$, then
ord$(a)<-e_{\mathcal{X}}-e_{0}+1$; for otherwise,
ord(a)$\ge-e_{\mathcal{X}}-e_{0}+1=t$, say and
  $(a+P^{e})\cap P^{t}=(a+P^{e})\cap(a+P^{t})=a+P^{max(e,t)}\neq
\phi$.

 Every $\Phi$ in $\mathscr{Y}(K)$ with support contained in
 $K\backslash P^{-e_{\mathcal{X}}-e_{0}+1}$ is a linear combination of
 functions of the form $\varphi_{a+P^{e}}$ with $(a+P^{e})\cap
 P^{-e_{\mathcal{X}}-e_{0}+1}=\phi$.
 Therefore, in order to prove the lemma, it suffices to show that
 $T_{\varphi}^{\times}(\varphi_{a+P^{e}})=0$ for $a+P^{e}$ disjoint with
 $P^{-e_{\mathcal{X}}-e_{0}+1}$. And,  indeed we have then
\begin{align*}
  T_{\varphi}^{\times}(\varphi_{a+P^{e}})
  & {\displaystyle{\mathop{=}^{defn}}} 
  T_{\varphi}(\varphi_{a+Pe}^{\times})\\
  & =q^{-e}\int\limits_{P^{-e}}\varphi(x)\psi (ax)dx,\qquad
  \text{ by \ref{chap2:sec1:subsec4:eq62}}\\
  & =q^{-e}\sum\limits_{-e \le j <e_{0}}q^{-js}(-j \log
  q)^{m-1}\int\limits_{R^{\times}}\mathcal{X}(u)\psi(a \pi^{j} u)du.
\end{align*}
\end{proof}

Since ord(a)$<-e_{\mathcal{X}}-e_{0}+1$ from above, we have ord $(a
\pi^{j})=ord(a)+j<-e_{\mathcal{X}}-e_{0}+1+e_{0}-1=-e_{\mathcal{X}}$. Therefore,
by (\ref{chap2:sec1:subsec3:eq52}), each one of the last mentioned
integrals vanishes and the lemma is proved.

\heading{Proof of assertions (B) and (C) of Theorem \ref{chap2:sec2:subsec1}.}

 We do not prove assertion (B) in its entirety. We shall only start
 from any $F$ in $\omega_{1}^{-1}\mathscr{F}$, with the expansion
 (\ref{chap2:sec2:subsec1:eq68}) for sufficiently small $|x|_{K}$,
 say, for $ord(x)\ge e_{0}$,  and\pageoriginale show that
 (\ref{chap2:sec2:subsec1:eq67}) holds for the Fourier transform
 $F^{\times}$ of 
 $F$ is in $L^{1}(K)$ and further $F$ has compact support. For any
 function $f$ in $L^1(K)$ with compact support, we see trivially that
   $f^{\times}$ is locally constant; in fact, for some $t\ge 1$, we know
   that $f$ vanishes outside $P^{-t}$ and 
\begin{equation*}
  f^{\times}(x+z)=\int\limits_{P^{-t}}f(y)\psi((x+z)y)dy =
  \int\limits_{P^{-t}}f(y)\psi(xy)\psi(zy)dy=f^{\times}(x)
\end{equation*}
  for $z\epsilon P^{t}$ and every $x\epsilon K$. Thus $F^{\times}$ is
  locally constant on $K$. 

 Let us define $\varphi_{0}: K\rightarrow\mathbb{C}$ by
\begin{equation*}
-\varphi_{0}(x)=
\begin{cases}
\text{ Right hand side of (\ref{chap2:sec2:subsec1:eq68})} & \text{for
    ord }(x)<e_{0}\\ 
  \qquad\qquad 0  & \text{for ord }(x)\ge e_{0}        
\end{cases}
\end{equation*}
 Then $\varphi_{0}\epsilon L_{\loc}^{1}(K)$ and by
 Lemma \ref{chap2:sec2:subsec1},
 $\varphi_{0}^{\times}$ vanishes for all sufficiently large
 $|x|_{K}$. If, now, we define the function $\Phi$ by setting
 $\Phi(0)=0$ and for $x\epsilon K^{\times}, \Phi(x)$ by 
\begin{equation*}
  F(x)=\text{(right hand side of (\ref{chap2:sec2:subsec1:eq68}))}~
  +\varphi_{0}(x)+\Phi(x),\tag{70}\label{chap2:sec2:subsec1:eq70} 
\end{equation*}
  then clearly $\Phi(x)$ is again locally constant and furthermore,
  since $\Phi(x)=F(x)$ for ord$(x)<e_{0}$, $\Phi$ has compact support,
  os that $\Phi\epsilon\mathscr{Y}(K)$. Applying the Fourier transform
  to (\ref{chap2:sec2:subsec1:eq70}), we get, on using
  (\ref{chap2:sec1:subsec5:eq63}), that 
\begin{align*}
  F^{\times}& =\left(\sum\limits_{\lambda \epsilon
    \wedge}\sum\limits_{m=1}^{m_{\lambda}}a_{\lambda,m}(ac(x))|x|_{k}^{\lambda-1}(\log
  |x|_{K})^{m-1}\right)^*+\varphi_{0}^{\times}+\Phi^{\times}\\
  & =\sum\limits_{\lambda \epsilon
    \wedge}\sum\limits_{m=1}^{m_{\lambda}}a_{\lambda,m}^{\#}(ac(x))
  |x|_{k}^{-\lambda}(\log |x|_{K})^{m-1}
\end{align*}
 for all sufficiently large $|x|_{K}$, where
\begin{equation*}
  a_{\lambda, m}^{\#}(u)=\sum\limits_{\mathcal{X} \epsilon
    ({R^{\times}})^*}\left(\sum\limits_{j=m}^{m_{\lambda}}(-1)^{m-1}
  \binom{j-1}{m-1}a_{\lambda,
    j,\mathcal{X}}\frac{d^{j-m}b_{\mathcal{X}}(\lambda)}{ds^{j-m}}\right)
  \mathcal{X}^{-1}(u)\tag{71}\label{chap2:sec2:subsec1:eq71}
\end{equation*}
 Thus\pageoriginale $F^{\times}$ has the expansion (\ref{chap2:sec2:subsec1:eq67}),
 for all large $|x|_{K}$, 
 that is obtained from (\ref{chap2:sec2:subsec1:eq68}) by termwise application of Fourier
 transform. This also proves that any $F^{\#}$ in
 $(\omega_{1}^{-1} \mathscr{F})^{\times}$ has an expression of the type
 (\ref{chap2:sec2:subsec1:eq67}) for all sufficiently large $|x|_{K}$. 

 It now remains for us to prove assertion $(C)$ of
 Theorem \ref{chap2:sec2:subsec1} again
 only partly. We first remark that the integral in (\ref{chap2:sec2:subsec1:eq69}) exists since
 $F^{\times}$ is $C^{\infty}$ or locally constant, we shall prove
 (\ref{chap2:sec2:subsec1:eq69}) 
 only for a $p$-field $K$. The right hand side of
 (\ref{chap2:sec2:subsec1:eq69}) is just 
 \begin{align*}
 \lim\limits_{e\rightarrow \infty}
 \int\limits_{P^{-e}}\psi(-xy)\left(\int\limits_{K}\psi(yz)F(z)dz\right)dy
 & =\lim\limits_{e\rightarrow
   \infty}\int\limits_{K}\left(\int\limits_{P^{-e}}\psi((z-x)y)dy\right)F(z)dz\\
 & =\lim\limits_{e\rightarrow \infty}q^{e}\int\limits_{x+P^{e}}F(z)dz,
 \end{align*}
 since the inner integral over $P^{-e}$ is $q^{e}$ or $0$ according
 as $z-x\epsilon P^{e}$ or otherwise. But, for $x\neq 0$, $F$ is
 constant on $x+P^{e}$ for sufficiently large $e$ and therefore the
 limit above is just $F(x)$.

\begin{Remarks*}
(1) For any $F$ in $\omega_{1}^{-1}\mathscr{F}$ as above with the
  asymptotic expansion (\ref{chap2:sec2:subsec1:eq66}) or
  (\ref{chap2:sec2:subsec1:eq68}) as $|x|_{K}\rightarrow 0$, we
  have, for the Fourier transform $F^{\times}$ a corresponding
  asymptotic expansion (\ref{chap2:sec2:subsec1:eq65}) or
  (\ref{chap2:sec2:subsec1:eq67})  as $|x|_{K}\rightarrow
  \infty$. As remarked in (\ref{chap2:sec2:subsec1:eq71}) for the case
  of $p$-fields, the Fourier coefficients
  $a_{\lambda,m,\mathcal{X}},a_{\lambda,m,\mathcal{X}}^{\#}$
  respectively of the coefficients
  $a_{\lambda,m}(u),a_{\lambda,m}^{\#}(u)$ in the asymptotic expansion
    of $F$ and $F^{\times}$, are related as follows: namely, for every
    $\lambda$ and $i\le m\le m_{\lambda}$,
\begin{equation*}
  a_{\lambda,m,\mathcal{X}^{-1}}^{\#}=(-1)^{m-1}\sum\limits_{j=m}^{m_{\lambda}}
  \binom{j-1}{m-1}\frac{d^{j-m}b_{\mathcal{X}}}{ds^{j-m}}
  (\lambda)a_{\lambda,j, \mathcal{X}}.\tag*{$(71)'$}\label{chap2:sec2:subsec1:eq71'}
\end{equation*}
If we fix $\lambda$ and $\mathcal{X}$ in $(K_{1}^{\times})^{\times}$ and
arrange $a_{\lambda,j,\mathcal{X}}(1\le j\le m_{\lambda})$ and
$(-1)^{m-1}$ $a_{\lambda,m,\mathcal{X}^{-1}}^{\#}(1\le m \le m_{\lambda}$
as $M_\lambda$-rowed column vectors, then the coefficient-matrix
corresponding to the relations \ref{chap2:sec2:subsec1:eq71'} becomes an upper triangular
matrix of $m_{\lambda}$ rows and\pageoriginale columns, with all the diagonal
entries to $b_{\mathcal{X}}(\lambda)$ and the entries just above the
diagonal equal to
${b'}_{\mathcal{X}}(\lambda)$, $2{b'}_{\mathcal{X}}(\lambda)$, 
$\ldots(m_{\lambda}-1){b'}_{\mathcal{X}}(\lambda)$. Therefore, 
if $b_\mathcal{X}(\lambda)=0$, then certainly $a_{\lambda,
  m_{\lambda},\mathcal{X}^{-1}}^{\#}$ is $0$ and moreover,
$a_{\lambda, 1, \mathcal{X}}$ disappears from the expression for
$a_{\lambda, m, \lambda^{-1}}^{\#}$ for $1\le m <m_{\lambda}$. Thus
the asymptotic expansion of $F^{\times}(x)$ as $|x|_{K}\rightarrow 0$
although $F$ is determined by $F^{\times}$, on $K^{\times}$ in view of
(\ref{chap2:sec2:subsec1:eq69}). For $F(x)=e^{-\pi x^{2}}$, the
asymptotic expansion of 
$F^{\times}(=F)^{\times}$ as $|x|\rightarrow \infty$ is just
$F^{\times}(x)\thickapprox 0$ while $F(x)\thickapprox
\sum\limits_{n=0}^{\infty}(-\pi)^{n}x^{2n}/n$! as $|x|\rightarrow$
tends to zero!

 For an $\mathbb{R}$-field $K$ and $\mathcal{X}=(ac)^{p}$, we know
 from  (\ref{chap2:sec1:subsec4:eq59}), that $b_{\mathcal{X}}(s)=0$ if and only if
   $s=1+\frac{|p|}{2d}+\frac{n}{d}$ for $n=0,1,2\ldots$; for such
   $s,b_{\mathcal{X}}'(s)\neq 0$.  Thus the rank of the
 $m_{\lambda}$-rowed matrix above is $m_{\lambda}-1$. If $K$ is a
 $p$-field, then again from $(59)$, $b_{\mathcal{X}}(s)=0$, if and only
 if $\mathcal{X}=1$, $q^{s}=q$; in this case, $b_{\mathcal{X}}'(s)\neq
 0$. Hence, at least for $p$-fields of characteristic $0$, the rank of
 the above-mentioned matrix is $m_{\lambda}-1$.

(2) Let us define $m_{K}$ to be 2, $2 \pi$ or $1-q^{-1}$ according
 as $K=\mathbb{R}, \mathbb{C}$ or a $p$-field so that, in all cases, we
 get
\begin{equation*}
d^{\times}x\text{ (the normalised Haar measure on $K$) } =\frac{dx}{m_{K}|x|_{K}}
\end{equation*}

 For $\lambda_{0}>0$ in the case of $\mathbb{R}$-fields and
 $Re(\lambda)>0$ for all $\lambda$ in the case of $p$-fields, let for
 any $F\epsilon \omega_{1}^{-1}\mathscr{F}$, $F^{\times}$ be its Fourier
 transform and $Z=M(m_{K}\omega_{1}F)$. If we accept
 Theorem \ref{chap2:sec2:subsec1},
 then $F^{\times}$ determines $F$ on $K^{\times}$ by
 (\ref{chap2:sec2:subsec1:eq69}) and $F$ in
 turn, determines $Z$. Similarly, if we start from $Z$, then
 $F,F^{\times}$ are determined, Thus any one of $F,F^{\times}, Z$ can be
 chosen arbitrarily in $\omega_{1}^{-1}\mathscr{F}$,
 $(\omega_{1}^{-1}\mathscr{F})^{\times}$ and $\mathscr{Z}$ respectively\pageoriginale
 and the other two are determined. But, we shall seldom start from
 $F^{\times}$, so that the lack of a complete proof of
 Theorem \ref{chap2:sec2:subsec1}  in
 these lectures will not present any discontinuity.
\end{Remarks*}

\section{The Space $(\omega_{1}^{-1}\mathscr{F})^{\times}\cap
  L^{1}(K)$}\label{chap2:sec3} %%% sec3

\subsection{}\label{chap2:sec3:subsec1} %%% subsec3.1

  Let $0<\lambda_{0}< \lambda_{1}<,\ldots$, $Re\lambda>0$ for $\lambda
  \epsilon \wedge$ and $M_{k}$ be as in \S \ref{chap2:sec2:subsec1}, and let
  $F\epsilon \omega_{1}^{-1}\mathcal{F}$ and $F^{\times}$, the Fourier
  transform of $F$. Then although $F$ is in $L^{1}(K)$,$F^{\times}$ may
  not be in $L^{1}(K)$. We proceed to characterise the subspace
  $(\omega_{1}^{-1}\mathscr{F})^{\times}\cap L^{1}(K)$ by the following
\begin{theorem}\label{chap2:sec3:subsec1:thm1} %%% thm3.1
  For $F\epsilon \omega_{1}^{-1}\mathscr{F}$, the following assertions
  are equivalent:
  \begin{enumerate}
    \renewcommand{\theenumi}{\rm \alph{enumi}}
    \renewcommand{\labelenumi}{(\theenumi)}
    \item $F^{\times}\epsilon L^{1}(K):$
      \item
        $F(0){\displaystyle{\mathop{=}^{defn}}}\lim\limits_{|x|_{K}\rightarrow
  0}F(x)$ exists;
\item \noindent 
  \begin{tabular}[t]{l}
  $Z_{p}(s)(p\neq 0), (s+1)Z_{0}(s)$ \\
  $Z_{\chi}(q^{-s})(\chi\neq 1), (1-q^{-(s+1)})Z_{1}(q^{-s})$   
  \Bpara{0}{10}{180}{15}  
  \end{tabular}
  \begin{tabular}[b]{p{3cm}}
    {\em are holomorphic for $Re(s)\geq -1$;}
  \end{tabular}

\item 
$F(0)=\frac{1}{m_{K}}
\begin{cases}
\lim\limits_{s\rightarrow -1}(s+1)Z_{0}(s)\\
\lim\limits_{s\rightarrow -1}\left(1-q^{-(s+1)}\right)Z_{1}(q^{-s})
\end{cases}
$
\end{enumerate}
and $F^{\times}\epsilon L^{-1}(K)$.
\end{theorem}

 To prove the theorem, we will show  that $a)\Rightarrow b)\Rightarrow
 c)\Rightarrow d)$. We first need to prove a lemma, to take
 care of the proof in the case of $p$-fields, 

\begin{lemma}\label{chap2:sec3:subsec1:lem1} %%%% lem3.1
  Let $f(z)$ be meromorphic in $|z|\le r$ for $r>0$, and let
  $f(z)=\sum\limits_{n\epsilon \mathbb{Z}}s_{n}z^{n}$ be its Laurent
  expansion around $z=0$, Then $a=\lim\limits_{n\rightarrow
    \infty}r^{n}s_{n}$ exists\pageoriginale and is finite if and only if
  $f(z)-\frac{a'}{1-z/r}$ is holomorphic in $0 <|z|\le r$ for a constant
  $a'$, in which case $a'=a$.
\end{lemma}

\begin{proof}
  We merely sketch a proof of this lemma which is rather elementary. If
  $g(z)=\sum\limits_{n\epsilon \mathbb{Z}}t_{n}z^{n}$ is a Laurent
  expansion of $g$ meromorphic in $0<|z|\le r$, then $g$ is holomorphic
  in $0<|z|\le r$ if and only if $\overline{\lim\limits_{n\to
      \infty}}|t_{n}|^{1/n}<r^{-1}$; if $\lim\limits_{n\rightarrow
    \infty}|t_{n}|^{i/n}<r^{-1}$ then clearly $\lim\limits_{n\rightarrow
    \infty}r^{n}t_{n}=0$. Thus $f(z)-\frac{a}{1-z/r}$ is holomorphic in
  $0<|z|\le r$, then $a<\infty $ and $a'=a$.

  Now, if $a<\infty$, then $f$ is holomorphic in $0<|z|<r$. Let us
  refer to ``$a=\lim\limits_{n\rightarrow\infty}r^{n}s_{n}$ exists and
  is finite'' as the property $P$ of $f$. If $f$ has property $P$,then
  so does $Rf$, for any polynomial $R$ in $\mathbb{C}[z]$. We now
  assert that $f$ cannot have a pole at any $\alpha \neq r$ on $|z|=r$;
  for otherwise,
  $(1-\alpha^{-1}z)^{-1}=\sum\limits_{n=0}^{\infty}\alpha^{-n}z^{n}$
  will have the property $P$, which gives a contradiction. Moreover,
  $f$ can have at most a simple pole at $z=r$; for, otherwise,
  $(1-z/r)^{-2}=\sum\limits_{n=0}^{\infty}(n+1)r^{-n}z^{n}$ will also
  have the property $P$, leading to a contradiction,This proves the
  lemma.
\end{proof}

\heading{Proof of Theorem 3.1} 

a) $\Rightarrow$ b). This is straightforward. In fact, if $G,G^{\times}$
are dual groups with dual measures $dg, dg^{\times}$, then, for $F_{0}$
in $L^{-1}(G; dg)$ with its Fourier transform  $F_{0}^{\times}$ in
$L^{-1}(G^{\times};dg^{\times})$, we have
$F_{0}(g)=(F_{0}^{\times})^{\times}(-g)$ at every $g$ where $F_{0}$ is
continuous. Thus , in our situation, $F(x)=(F^{\times})^{\times}(-g)$ at
every $g$ where $F_{0}$ is continuous. Thus, in our situation,
$F(x)=(F^{\times})^{\times}(-x)$ for $x\epsilon K^{\times}$ and
$F(0)=\lim\limits_{|x|_{K}\rightarrow 0}F(x)$ exists, since
$(F^{\times})^{\times}$ is continuous.

b)$\Rightarrow$ c) $\Rightarrow$ d)(for an $\mathbb{R}$-field $K$). We
know that $F$ has an asymptotic expansion as in (66) as
$|x|_{K}\rightarrow 0$ with $\lambda_{0}>0$; it follows that
$F(0)=\lim\limits_{|x|_{K}\rightarrow 0}F(x)$ exists if and only if
$\lambda_{k}\ge 1$\pageoriginale for every $k$ and, in addition, when
$\lambda_{k}=1$ for some $k$, then $m_{k}=1$,
$a_{k,1}=a_{k,1}(1)$. This is equivalent to saying that
$F(x)\approx F(0)+\sum\limits_{\lambda_{k}>1}
\sum\limits_{m=1}^{m_{k}} a_{k,m}(ac(x))|x|_{K}^{\lambda_{k}-1}(\log
|x|_{K})^{m-1}$. In view of Ch. I \S 4, this is equivalent to
$Z_{P}(s)$ for every $p\neq 0$ and $Z_{0}(s)-\frac{m_{K}{f(0)}}{s+1}$
being holomorphic for $Re(s)\ge -1$, where now
$Z=M_{K}(m_{K}\omega_{1}F)$, since $(m_{K}\omega_{1}F)(x)\approx
m_{K}F(0)|x|_{K}+m_{K}\sum\limits_{\lambda_{k}>1}\sum\limits_{m=1}^{m_{k}}
a_{k,m}(ac(x))|x|_{K}^{\lambda_{k}}(\log|x|_{K})^{m-1}$
as $|x|_{K}\rightarrow 0$ and further $b_{k,m,p}=(-1)^{m-1}(m-1)!a_{k,
  m, -p}$. This, in turn, implies that $F^{\times}(x)\approx
\sum\limits_{\lambda_{k}>1}\sum\limits_{m=1}^{m_{k}}a_{k,
  m}^{\sharp}(ac(x))|x|_{K}^{-\lambda_{k}}(\log |x|_{K})^{m-1}$ as
$|x|_{K}\rightarrow \infty$, in view of Theorem
\ref{chap2:sec2:subsec1} above, \ref{chap2:sec2:subsec1:eq71'} 
and the fact that $b_{0}(1)=0$. Such an asymptotic expansion clearly
implies that $F^{\times}$ is in $L^{1}(K)$. 

b)$\Rightarrow$ c) $\Rightarrow$ d) (for a $p$-field $K$) If we write
$F(\pi^{e}u)=\sum\limits_{\mathcal{X}}c_{e,
  \mathcal{X}}\mathcal{X}(u)$ for $e\epsilon \mathbb{Z}$ and $u
\epsilon R^{\times}$, then
$(m_{K}\omega_{1}F)(\pi^{e}u)=m_{K}\sum\limits_{\mathcal{X}}(q^{-e}c_{e,\mathcal{X}})
\mathcal{X}(u)$ and in view of (\ref{chap1:sec5:subsec3:eq38}), we have
$Z_{\mathcal{X}}(z)=m_{K}\sum\limits_{e\epsilon
  \mathbb{Z}}c_{e,\mathcal{X}}{-1}(q^{-1}z)^{e}$. Now b) is
  equivalent to ``$\lim\limits_{e \rightarrow \infty}c_{e,
    \mathcal{X}}$ is equal to $F(0)$ for $\mathcal{X}=1$and to $0$ for
  $\mathcal {X}=1$ and to $0$ for $\mathcal{X}\neq 1$''. This, in
  turn, is equivalent to $Z_{\mathcal{X}}(z)$ for every $\mathcal{X}
  \neq 1$ and $Z_{1}(z)-\frac{m_{K}F(0)}{1-q^{-1}z}$ being holomorphic
  in $0<|z|\le q$, because of Lemma \ref{chap2:sec3:subsec1:lem1} This, again, is equivalent
  to the fact that 
\begin{equation*}
  F(x)=F(0)+\sum\limits_{\substack {\lambda \epsilon
      \wedge\\ Re(\lambda)>1}}\sum\limits_{M=1}^{m_{\lambda}}a_{\lambda,m}
  (ac(x))|x|_{K}^{\lambda-1} (\log|x|_{K})^{m-1}
\end{equation*}
for all sufficiently small $|x|_{K}$. Just as above, this new implies
$F^{\times}(x)= \sum\limits_{\substack {\lambda \epsilon
    \wedge\\ Re(\lambda)>1}}\sum\limits_{m=1}^{m_{\lambda}}a^{\sharp}_{\lambda,
  m}(ac(x))|x|_{K}^{-\lambda}(\log|x|_{K})^{m-1}$ for\pageoriginale all large enough
$|x|_{K}$, so that finally $F^{\times}$ is in $L^{1}(K)$. This completes
the proof Theorem \ref{chap2:sec3:subsec1:thm1}.
