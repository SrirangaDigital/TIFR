\chapter*{Introduction}

\addcontentsline{toc}{chapter}{Introduction}

ONE\pageoriginale OF THE principal objectives of modern number theory must be to
develop the theory of forms of degree more than two,to the same
satisfactory level in which the theory of quadratic forms is found
today as the cumulative work of several eminent mathematicians and
especially of C.L. Siegel. The importance of forms of higher degree as
a serious number-theoretic object was pointed out already by
Gauss(\cite{Gau}, \S\ 266) as follows:''\ldots il suffira d'avoir
recommend\'ece champ vaste \'a l'attention des g\`eom\'etres,
o\'u ils pourront trouver un tr\'es-beau sujet d' exercer leurs
forces, et les moyens de donner \'a l'Arithmetique transcendent de
tr\'es-beaux d\'eveloppements''. It is unbelievable and highly
remarkable how remarkable how young Gauss was able to predict the
existence of a full-fledged theory of quadratic forms as it obtains
today. 

With a view to give some idea as to what may constitute such a theory
of forms of higher degree, we shall describe two problems. Before we
state the first one, let as fix the notation. Let
$f(x)=f(x_{1},x_{2},\ldots,x_{n})$ be a form i.e, a homogeneous
polynomial degree $m\ge 2$ in n variables $x_{1},\ldots,x_{n}$ with
coefficients in the ring $\mathbb{Z}$ of integers. Let p be a fixed
prime, ``$e$'' a non-negative integer and ``$u$'' an integer such that the g.c.d
$(u,p^e)$ is l. Let us define the generalized Gaussian sum
\begin{equation*}
  F^\ast(u/p^e) = \frac{1}{p^{ne}}\sum_{\xi \mod p^e}((u/p^e)f(\xi))
\end{equation*}
where $\xi\in\mathbb{Z}^n$ and $e(\quad)=\exp(2\pi\sqrt{-1})$. The
first problem is to develop a theory of such exponential sums
associated with $f$ and $p$. In this direction, we have the following
\underline{general theorem}:

\noindent there\pageoriginale exists an integer $e_o\ge0$ such that for every $e\ge e_o$,
$F^\star{(u/p^e)}=a$ fixed linear combination of expression of the
form
\begin{equation*}
  \mathcal{X}(u)(p^e)^{-\lambda}(log p^e)^j
\end{equation*}
with $\mathcal{X}$ equal to a Dirichlet character having a power of p
as its condu\-ctor, $0\leq j < n$ and $Re(\lambda)$ being a positive
rational number. 
If now we arrange $Re(\lambda)$ for $lambda$ occurring above in
ascending order as 
\begin{equation*}
  0<\lambda_1<\lambda_2<\cdots
\end{equation*}
(where $\lambda_1,\lambda_2,\ldots$ are invariants of the affine
hypersurface determined by the zeros of f), then for any $\epsilon>0$,
we have, for every $e\geq 0$,
\begin{equation*}
  F^\star(u/p^e)\leq c(p^e)^{-\lambda_1+\epsilon}
\end{equation*}
where $c$ is a constant independent of $e$ and $u$. A \underline{conjecture}
in this connection states that except for a finite number of primes p
depending on f and $\epsilon$, we can replace the constant c above by
$1$, if $\lambda_1 > 1$.
In all the cases where $F^\star(u/p^e)$ can be closely examined or
calculated, this conjecture has been verified. Indeed, in the
particular case of interest when the projective hypersurface S defined
$f(x)=0$ is non-singular, the conjecture is valid with $\lambda_1=n/m$
(and further 0 instead of $\epsilon$). The proof of the conjecture in
this case depends on Deligne's solution of the Riemann Weil hypothesis
for zeta functions associated with non-singular projective varieties
defined over finite fields.

In order to deal with the conjecture when S may not be non-singular,
one approach may be to isolate the relevant part of Deligne's work
used to prove\pageoriginale the conjecture and examine whether the assumption of S
being non-singular may be dropped.

We now state the second problem we referred to earlier in the context
of a theory of forms of higher degree. It is the following well-known
conjecture:

\noindent if $m=3$, then $f(x)=0$ has a non-trivial integral solution provided 
that $n>9$.

Davenport (\cite{Dav}) has proved a similar statement with $n>15$ in place of
the condition $n>9$. To bring down $15$ to $9$ by improving
Davenport's method seems to be almost impossible. We have recently
found a new approach to this conjecture in the non-singular case; this
is a direct generalization of the method used by Weil to establish the
Minkowski-Hasse theorem for quadratic forms connecting the global
representability with local representability. Weil's method depends on
a certain poisson formula and the use of the ``metaplectic
group''(which is a covering of $SL_2$). We have indeed a
generalization of such a poisson formula to forms of higher degree. If
our approach does really settle the conjecture for forms of higher
degree, then we shall be rewarded with a generalization of the
beautiful relation between quadratic forms and modular functions. We
believe in the existence of a new satisfactory theory of forms of
higher degree and hope that the solution of the conjectures above will
occupy important positions in such a theory.
