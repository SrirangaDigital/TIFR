\chapter{}\label{chap3}

\section{Vector bundles}\label{chap3:sec1} %  section 1

\begin{defi*}
  Let\pageoriginale $X$ and $E$ be hausdorff spaces and $p$: $E \to X$ a 
  continuous map. 1 Theotriple $(E, pX)$ is called a 
  continuous complex (real) vector bundle of rank $q$ if the
  following conditions are satisfied.  
  \begin{enumerate}[(i)]
  \item For $x \in X$, $E_x=p^{-1} (x)$ is a vector space of dimension
    $q$ over $\mathbb{C} (\mathbb{R})$. 
  \item If $x \in X$, there is a neighbourhood $U$ of $x$ and a
    homeomorphism $h$ of $E_U=P^{-1}(U)$ onto $U \times \mathbb{C}^q (U
    \times \mathbb{R}^q)$ such that if $\pi$ is the projection of $U
    \times \mathbb{C}^q (U \times \mathbb{R}^q)$ onto $U$, we have
    $\pi (h(y))=x$ if $y \in E_x$ and $h|E_x$ is a
    $\mathbb{C}(\mathbb{R})$-isomorphism of $E_x$ onto $\{ x \} \times
    \mathbb{C}^q (\{ x\} \times \mathbb{R}^q)$. 
  \end{enumerate}
\end{defi*}

If $E$ and $X$ are $C^k$ manifolds $(1 \leq k \leq \infty)$, if $p$ is
a $C^k$ map and if the isomorphisms $h_U$ can be chosen to be $C^k$
diffeomorphisms, $p: E \to X$ is called a $C^k$ bundle (or
differentiable bundle of class $C^k$). 

If $X$ is a real (complex) analytic manifold real (complex) analytic
vector bundle can be defined in the same way. Complex analytic
bundles are also called holomorphic vector bundles. A vector
bundle of rank $1$ is called a line bundle. 

It follows from the definition that if $p:E \to X$ is a complex vector
bundle of rank $q$ there exists an open covering $\{ U_i \}$ of $X$
and homeomorphisms $\varphi_i$: $E_{U_i} \to U_i \mathbb{C}^q$ such
that if $U_{ij} = U_i \cap U_j$, then  $\varphi_j \circ \varphi^{-1}_i
: U_{ij}\times \mathbb{C}^q \to U_{ij} \times \mathbb{C}^q$
is a homeomorphism and $\varphi_j \circ \varphi^{-1}_i (x, y) = (x,
g_{ij}(x)v)$ where, for each $x$, $g_{ij}(x)$ is in $GL(q
,\mathbb{C})$ and $g_{ij}(x)g_{jk}(x)=g_{ik}(x)$ for $x \in
U_{ijk}=U_i \cap U_j \cap U_k$. Clearly $g_{ij}:U_{ij} \to GL(q,
\mathbb{C})$\pageoriginale is continuous. The $g_{ij}$ are called transition map (or
transition functions) of the bundle. 

If the bundle is $C^k$ (or real or complex analytic), the transition
maps $g_{ij}:U_{ij} \to GL(q, \mathbb{C})$ are $C^k$ (or real or
complex analytic). 

Conversely let $X$ be a hausdorff topological space, $\{ U_{i}\}_{i
  \in I}$ an open covering of $X$ and $g_{ij}: U_{ij} \to GL(q,
\mathbb{C})$ continuous map satisfying $g_{ij}(x)g_{jk}(x)=g_{ik}(x)$
for $x \in U_{ijk}(=U_i \cap U_j \cap U_k)$. Then let $S$ be the
topological sum $\displaystyle\mathop{\cup}_{i} \{U_i \times
\mathbb{C}^q \times \times i\}$. Define an 
equivalence relation $\sim$ on $S$ by $(x, v, i)\sim (x', v' ,j)$ if
$x=x'$ and $v'=g_{ij}(x).v$. It is easily verified that the
equivalence relation is open and that the graph is closed. Hence $E
=S/ \sim$ is hausdorff. Let $p':S \to X$ be defined by
$p'(x,v,i)=x$. Clearly equivalent points have the same image in $X$ so
that $p'$ defines a map $p:E \to X$. Then $p^{-1}(U_i)= \{ \overline
{(x,v,i)}|x \in U_i,v \in \mathbb{C}^q \}$ and hence $p^{-1}(U_i)$ is
``isomorphic'' to $U_i \times \mathbb{C}^q$ and $p: E \to X$ is a
complex vector bundle of rank $q$. Thus a vector bundle $p: E \to X$
is characterised by an open covering $\{ U_i \}$ of $X$ such that
$p^{-1}(U_i)$ is isomorphic to $U_i \times \mathbb{C}^q$, and the
transition maps $g_{ij}:U_{ij} \to GL(q, \mathbb{C})$. If $X$ is a
$C^k$ manifold, and the $g_{ij}$ are $C^k$, maps, then the vector
bundle constructed above is also $C^k$. A similar remark applies to
real and complex analytic vector bundles. 

(Compare the above construction with the introduction of the topology
on the tangent bundle as given in Chap. II \S\ \ref{chap2:sec1}). 

\begin{defi*}
  Let\pageoriginale $p:E \to X$, $p':E' \to X$ be two complex vector bundles on
  $X$. A bundle map or a homomorphism $h:E \to E'$ is a continuous
  map $h:E \to E'$ such that for any $x \in X$, $h | E_x [=p^{-1}(x)]$
  is a $\mathbb{C}$-linear map into $E'_x [=p^{-1}(x)]$. If in
  addition $h$ is a homeomorphism (so that $h|E_x$ is an isomorphism
  onto $E'_x),h$ is called an isomorphism. $E$ and $E'$ are isomorphic
  if there is an isomorphic of $E$ onto $E'$. 
\end{defi*}

Similar definitions apply to $C^k$, real analytic and holomorphic
bundle maps and isomorphisms. 

\begin{remark*}
  Let a vector bundle $p: E \to X$ be given by the open covering $\{
  U_i\}_{i \in I}$ and the transition map $g_{ij}:U_{ij}\to GL (q,
  \mathbb{C})$. Let $\{ V_\alpha \}_{\alpha \in A}$ be a refinement of
  $\{ U_i \}$ and $\tau :A \to I$ a map such that $V_\alpha \subset
  U_{\tau(\alpha)}$. Consider the vector bundle $p' :E' \to X$ where
  $E' =S'/\sim$, $S'= \{ (x,v, \alpha)| x \in V_\alpha, v \in
  \mathbb{C}^q\}$ constructed with the transition maps $g'_{\alpha
    \beta}=\beta_{\tau(\alpha)\tau(\beta)} | V_\alpha \cap V_
  \beta$. The map $\tau$ defines a continuous map $h':S' \to S$,
  viz. $h' (x,v, \alpha)=(x,v,\tau (\alpha))$. It is easily verified
  that $h'$ maps equivalent points into equivalent points, and so
  define a continuous map $h:E' \to E$. This map is easily seen to be
  an isomorphism of the vector bundles $E$ and $E'$.  
\end{remark*}

\setcounter{proposition}{0}
\begin{proposition}\label{chap3:sec1:prop1}% Prop 1
  Let $p:E \to X$ and $p':E' \to X$ be two vector bundles given by
  the open coverings $\{ U_i\}_{i \in I}, \{ V_\alpha \}_{\alpha \in
    A}$ and transition map $(g_{ij})$, $(g'_{\alpha \beta})$. Then a
  necessary and sufficient condition that the two vector bundles are
  isomorphic is the following: there exists a common refinement $\{
  W_k \}_{k \in K}$ of $\{ U_i \}$ and $\{V_\alpha \}$, refinement
  maps $\tau_1:K \to I$, $\tau_A:K \to A$, [i.e. $W_k \subset
    U_{\tau_I}(k)\cap V_{\tau_A}(k)$] and continuous maps\pageoriginale $h :W_k
  \to GL(q, \mathbb{C})$ such that if $g_{kl}$, $g'_{kl}$ denote the
  restrictions to $W_{kl}$ of $g_{\tau_I (k),\tau_I (l)}$, $g'_{\tau
    A(k), \tau_A (l)}$ respectively, we have 
  $$
  h_l g_{kl} h^{-1}_k=g'_{kl} \text{ on } W_{kl}.
  $$
\end{proposition}

\begin{proof}
  Let $p:E \to X$ and $p':E' \to X$ be isomorphic and $h:E \to E'$ and
  isomorphic between  then. Let $\{ W_{k}\}$ be a common refinement of
  $\{ U_i \}$ and $\{ V_ \alpha \}$. In view of the remark made above,
  we may suppose that $E$ and $E'$ are constructed using the covering
  $\{ W_k \}$ and the transition maps $g_{kl}$, $g'_{kl}$
  respectively. Let $\varphi_k:E_{W_k} \to W_k \times \mathbb{C}^q$
  and $\varphi'_k:E'_{W_k} \to W_k \times \mathbb{C}^q$ be the
  isomorphisms corresponding to $E$ and $E'$ respectively. Let $h'_k
  =\varphi'_k oho \varphi^{-1}_k:W_k \times \mathbb{C}^q \to W_k
  \times \mathbb{C}^q$. Define $h_k:W_k \to GL(q, \mathbb{C})$ by
  the formula 
  $$
  h'_k(x,v)=(x,h_k (x),v).
  $$
  Then since $h'_k \circ \varphi_k \circ \varphi^{-1}_1 \circ h'^{-1}_1=\varphi'_k
  \circ \varphi'^{-1}_l$ and $(x, g_{kl}(x)v)=\varphi_k \circ \varphi^{-1}_1
  (x, v)$, we obtain at once the relation $h_k
  g_{lk}h^{-1}_1=g'_{lk}$. For the converse, suppose that $p: E \to X$
  and $p': E' \to X$ are two vector bundles and let $\{ W_k \}$ be a common
  refinement of the covering $\{ U_i \}$, $\{ V_ \alpha \}$
  corresponding to $E$ and $E'$ respectively. If there exists map
  $h_k:W_k \to GL (q, \mathbb{C})$, satisfying $h_l g_{kl} h_k^{-1}$,
  let $\varphi_k : E_{W_k} \to W_k \times \mathbb{C}^q$ and $\varphi'_k :
  E'_{W_k} \to W_k \times \mathbb{C}^q$ be the isomorphisms corresponding
  to $E$ and $E'$ respectively. Then $h : E \to E'$ is defined as
  follows: let $h'_k : W_k \times \mathbb{C}^q \to W_k \times 
  \mathbb{C}^q$ be the isomorphism defined by 
  $$
  h'_k (x,v)=(x,h_k(x)v);
  $$
  set\pageoriginale $h^{(k)}=\varphi'^{-1}_k \circ h'_k \circ \varphi_k$ on $E_{W_k}$. We
  have $h^{(k)}=h^{(l)}$ on $E_{W_{kl}}$ because of the formula $h_l
  g_{kl} h^{-1}_k=g'_{kl}$. 
\end{proof}

\begin{examples*}
  \begin{enumerate}[(1)]
  \item Let $I_q=X \times \mathbb{C}^q$ and $p: I_q \to X$ be the
    projection $p$, $(x,v) = x$. Then $p:I_q \to X$ is a complex
    vector bundle of rank $q$ and is called the trivial vector
    bundle of rank $q$. A bundle of rank $q$ is trivial if it is
    isomorphic to $I_q$. Since given a vector bundle $p:E \to X$,
    every point $x \in X$ has a neighbourhood $U$ such that $E_U$ is
    isomorphic to $U \times \mathbb{C}^q$, every vector bundle is
    locally trivial. 

  \item Let $p_1 :E _1 \to X$ and $p_2:E_2 \to X$ be two vector
    bundles of rank $q_1$ and $q_2 $ respectively. Set $F= \bigcup
    \limits_{x \in X}(E_{1x} \otimes E_{2x})$ and define the map $p:F
    \to X$ by $(E_{1x} \otimes E_{2x})=x$. For any $x \in X$, there
    exists a neighbourhood $U$ such that $E_{1U}$ and $E_{2U}$ are
    isomorphic to $U \times \mathbb{C}^{q_1}$ and $U \times
    \mathbb{C}^{q_2}$ respectively. Let $\varphi_1 : E_{1U} \to U
    \times \mathbb{C}^{q_1}$ and $\varphi_2 : E_{2U} \to U \times
    \mathbb{C}^{q_2}$ be such isomorphisms. Define $\varphi :F_U \to
    \times \varphi_1 : E_{1U} \to U \times \mathbb{C}^{q1+q2}$ by
    $\varphi (e_{1x} \otimes e_{2x})=(x, \bar{\varphi_1}
    (e_{1x}\bar{\varphi_2} (e_{2x}))$, where $e_{ix} \in E_{ix}$ and
    $\bar{\varphi _i}(e_i)$ is the projection on $\mathbb{R}^{q_i}$ of
    $\varphi_i(e_i)$; here $e_i\in E_{iU}$. Clearly there exists a
    unique topology on $F$ such that
    above maps are homeomorphisms and $p:F \to X$ is a vector
    bundle. The transition maps $g_{ij}$ of $F$ are given by
    $g_{ij}=g^1_{ij} \oplus g^2_{ij}$ and $g^2_{ij}$ are transition
    maps of $E_1$ and $E_2$ respectively. $F$ is called the direct (or
    Whitney) sum of $E_1$ and $E_2$ and we write $F=E_1 \oplus E_2$. 
  \end{enumerate}
\end{examples*}

If $p:E \to X$ and $p':E' \to X$ are complex vector bundles, then
$\bigcup \limits_{x \in X} E_x \oplus E_{x'}\bigcup \limits_{x \in X}
Hom (E_x,E'_x)$ and $\bigcup \limits_{x \in X}\wedge^p E_x$ can be
given, in\pageoriginale the same way, suitable topologies so as to make them vector
bundles. They are denoted by $E \otimes E'$, Hom $(E, E')$,
$\overset{p}\wedge E$ respectively. When $E'$ is a trivial bundle of
rank 1, Hom$(E_x, E'_x)=E^*_x$ is the dual of $E_x$ and we wrote $E^*$
for the corresponding bundle. $E \otimes E'$ is called the tensor
product of $E$ and $E', \wedge^p E$, the p-fold exterior products of
$E$. 

We remark explicitly that if $g_{ij}$, $g'_{ij}$ are transition maps
of $E$, $E'$ relative to a covering $\{ U_i \}$, those of $E \otimes
E'$ are $g_{ij} \otimes g'_{ij}$ (Kronecker or tensor product of
matrices), those of $E^*$ are $(t_{g_{ij}})^{-1}$, $t_A$ denoting the
transpose of the matrix $A$.  In particular if $E'$ is a line bundle,
$E \otimes E'$ has transition maps $g'_{ij}.g_{ij}$. If we apply this
to a line bundle $E$ its dual $E^*=E'$, we see that \textit{if}
$\underline{E}$ \textit{is a line bundle, $E \otimes E^*$ is
  trivial}. 

This isomorphism is intrinsically defined as follows: for $x \in X$,
we have a bilinear map $E_x \oplus E^*_x \to \mathbb{C}$, viz. $(e
\oplus e^*) \to e^* (e)$. This defines a linear map $E_x \otimes E^*_x
\to \mathbb{C}$, and so a map $h: E \otimes E^* \to 1_1$; $h$ is an
isomorphism. 

Other examples of vector bundles are the following. If $V$ is a
$C^k$ manifold of dimension $n$, $T(V)= \bigcup\limits_{a \in V} T_a (V)$ is a
vector bundle of class 
$C^{k-1}$ and rank $n$; it is called the tangent bundle of $V$. The
bundle of $p$-forms on $V$ is the space $\wedge^p T^* (V)= \bigcup
\limits_{a \in V}\wedge^p T^*_a (V)$. [Note that $\wedge^p T^*(V)$ is in
  fact the p-fold exterior product of $T^* (V)$.] 

Let $E,F,E',F'$ be vector bundles on $X$, $h:E \to F$, $h' :E' \to F'$
bundle maps. For any $x$, we have a linear map $h_x \otimes h'_x:E_x
\otimes E'_x \to F_x \otimes F'_x$; this defines a bundle map $h
\otimes h' : E \otimes E' \to F \otimes F'$. In the same way, we
have a transpose bundle map $h^* : F^* \to E^*$ and a map $\wedge^p 
h : \wedge^p E \to \wedge^p F$. [If $h$, $h'$ are $C^k$, analytic,
  holomorphic, so are $h \otimes h',h^*,\wedge^p h$.] 

\begin{defis*}%defi 0
  (1) Let\pageoriginale $V$ be a $C^k$ manifold $p:E \to V $ a $C^k$ vector bundle
  and $U$ an open set in $V$. Then a $C^k$ section $s $ of $E$ on $U$
  is a $C^k$ map $s : U \to E$ such that $p~ o~ s$ = identity on $U$. 

  $C^k (U, E)$ denotes the set of all $C^k$ sections of $E$ over
  $U$. Analytic (holomorphic) sections of analytic (holomorphic) bundles
  are similarly defined. 
  
  (2) The support of a section $s$ of $E$ over $U$ is defined
    to be the closure $U$ of $\left\{ x | x \epsilon U, s (x) \neq 0
    \right\}$ [$0$ stands for the zero of the vector space $E_x$]. The
    set of $C^k$ sections on $U$ saving compact support in $U$ is
    denoted by $C^k_0 (U, E)$. 
\end{defis*}
  Note that if $E = 1_q$, $C^k (U,1_q)$ can be canonically identified
  with the space of $C^k$ maps $U \to  \mathbb{C}^q (\mathbb{R}^q)$. Let
$E$ be a vector bundle, $\left\{ U_i\right\}$ a covering of $X,
\varphi_i : E_{U{_i}} \to U_i \times \mathbb{C}^q$ isomorphisms and
$g_{ij}: U_{ij} \to GL (q,\mathbb{C})$ the corresponding transition
maps. If $s : X \to E$ is a section, we have elements $s_i \epsilon
C^k (U_i, 1_q)$, $viz. s_i = \varphi_i \circ s$ and hence mappings
$\sigma_i : U_i \to \mathbb{C}^q$; since $\varphi_j \circ \varphi^{-1}_i \circ
s_i = s_j$ on $U_{ij}$, we have $\sigma_j = g_{ij} \sigma_i$ on
$U_{ij}$. Conversely, mapping $\sigma_i : U_i \to \mathbb{C}^q $ with
$\sigma_j = g_{ij} \sigma_i$ on $U_{ij}$ define a section $s : X \to
E$. This section is $C^k$, analytic, holomorphic, according as the
$\sigma_i$ are $C^k$, analytic, holomorphic. 

We denote the set of $C^k$ maps $U \to \mathbb{C}^q (\text{ or
}\mathbb{R}^q)$ by $C^{k,q} (U)$; those with compact support by
$C_0^{k,q} (U)$. 

\section{Linear differential operators: the theorem of
  Peetre}\label{chap3:sec2}  %sec 2

In what follows, $V$ is a $C^\infty$ manifold and all vector bundles
over $V$ are $C^\infty$, real vector bundles. $C^\infty_0 (V, E)$
denotes the set of $C^\infty$ sections\pageoriginale of $E$ over $V$ having compact
support. 

\begin{defi*}%defi 0
  Given a $C^\infty$ manifold $V$ and vector bundles $p_1 : E \to V$
  and $p_2 :F \to  V$, a differential operator $L$ from $E$ to $F$
  (written $L : E \to F$) is an $\mathbb{R}$ linear map $L :
  C^\infty_0 (V, E) \to C^0 (V, F)$ such that supp. $(Ls) \subset$
  $\supp (s)$ for every $s \epsilon C^\infty_0 (V, E)$. These are
    also called operators (or sheaf maps). Note that $L$ does not
    define a bundle map $E\to F$. 
\end{defi*}

\begin{remarks*}%rem  0
  A differential operator gives rise to an $\mathbb{R}$ linear map $L
  : C^\infty (V, E) \to C^0 (V, F)$ as follows. For $x \epsilon V$,
  let $U$ be a relatively compact neighbourhood of $x$. Let $\varphi$
  be a $C^\infty$ function $\varphi : V \to \mathbb{R}$ such that for
  $y$ in a nighbourhood of $x$, $\varphi (y) = 1$ and $\varphi (y) = 0
  $ for $y \notin U$. Then for any $s \epsilon C^\infty (V, E)$, we
  set 
  
  $(Ls)(x) = L (\varphi s) (x)$; since $\varphi$ has compact support,
  $L (\varphi s)$ is well defined. $(Ls)(x)$ is independent of the
  $\varphi$ chosen since $L$ does not increase supports.  
\end{remarks*}

If $E$, $E'$ are $C^\infty$ vector bundles of rank $q$, $q'$
respectively, and if, $U$ is a coordinate neighbourhood of $V$ such
that $E_U$ and $E_U'$ are trivial then $C^\infty_0 (U, E)$ can be
identified with $C^{\infty, q}_0 (U)$, the set of $q$ tuples of
$C^\infty$ functions with compact support in $U$. A linear
differential operator defines then an $\mathbb{R}$ linear map
$L:C^{\infty,q}_0 (U) \to C^{0,q'} (U)$. 

\setcounter{lemma}{0}
\begin{lemma}\label{chap3:sec2:lem1}%lemm 0
  Let $V$ be a $C^\infty$ manifold and $U$,a coordinate neighbourhood
  on $V$, (coordinate system $x_1 ,\ldots,x_n$). Let $L$ be a
  defferential operator $C^{\infty,q}_0 (U) \to  C^{0,p} (U)$ [i.e. an
    operator from  $1_q \to 1_p$ on $U$]. Then for any point $a
  \epsilon U$, there exists a neighbourhood $U'$ of $a$, a positive
  integer $m$ and a constant $C > 0$, such that   
  $$
  || Lf ||_0 \le  C || f||_m \text{ for any } f \epsilon
  C_0^{\infty,q} (U'-\{a\}). 
  $$
  We\pageoriginale recall that the norms on $C^{k,q} (U)$ are defined by
  $$
  || f ||_m = \sum_{|\alpha| \le m} \sum^q_{i=1} \sup |D^\alpha f_i
  (x)| \text{ if } f = (f_1 ,\ldots,f_q). 
  $$
\end{lemma}

\begin{proof}
  Let $a \epsilon U$ and suppose that the lemma does not hold. Let
  $U_0$ be a neighbourhood of a, relatively compact in $U$. Then there
  exists an open set $U_1 \subset \subset (U_0 - \{a\})$ and $f_1
  \epsilon C^{\infty,q}_0 (U_1)$ such that 
  $$
  ||Lf_1||_0 > 2^2 || f_1||_1.
  $$
\end{proof}

Now consider the open neighbourhood $(U_0 - \bar{U}_1)$ of $a$; by our
assumption there exists an open set $U_2,U_2 \subset \subset (U_0 -
\bar{U}_1 - \{a\})$, and $f_2 \epsilon C^{\infty,q}_0 (U_2)$ such
that 
$$
||Lf_2||_0 > 2^{2.2} ||f_2||_2. 
$$

By induction we have a sequence of open sets $\{U_k\}$ with $\bar{U}_k
\subset \{U_0 - a\}$ and $\bar{U}_k \bigcap \bar{U}_1 = \phi$ if $k
\neq 1$ and $f_k \epsilon C^{\infty,q}_0 (U_k)$ with $||Lf_k||_0 >
2^{2k} ||f_k||_k$ Let $f=\sum\limits^\infty_{k=1} \dfrac{2^{-k} f_k}
{||f_k||_k}$. Since $\sum \dfrac { 2^{-k} f_k} {||f_k||}$ is
convergent in the $C^\infty$ topology, $f \epsilon C^{\infty,q}_0
(U_0)$ and $ f |U_k = \dfrac {2^{-k} f_k} {||f_k||_k}$ so that $L(f) |
U_k = 2^{-k} L (f_k)| U_k \big/ ||f_k||_k$. Since $||Lf_k||_0 > 2^{2k}
||f_k||_k$, we have a sequence $(x_k)$, $x_k \epsilon U_k$ such
that 
$$
\big| Lf_k (x_k)\big| \rangle 2^{2k} ||f_k||_k.
$$

Hence\pageoriginale $|Lf (x_k)| \ge 2^k$. But $Lf$ is continuous in $U$, while $(Lf)
(x_k)$ is unbounded and $\{x_k\}$ lie in the relatively compact subset
$U_0 $ of $U$. This is contradiction, so that the lemma is
established. 

\begin{theorem*}[(Peetre)]%theo 0
  Let $V$ be a $C^\infty$ manifold and $E,F$, $C^\infty$ vector
  bundles of rank $q$ and $p$ respectively. Let $L $be a differential
  operator $C^\infty_0 (V,\break E)$ $\to C^0 (V, F)$  and let $U$ be a
  coordinate neighbourhood such that $E_U$ and $F_U$ are trivial. We
  identify $C^k(U, E)$ with $C^{k,q}(U)$. Then for any relatively
  compact open subset $\Omega$ of $U$ there exists a positive integer
  $m$ and continuous functions $a_\alpha$ on $\Omega, |\alpha| \le m$
  [with values in the space of linear maps from $\mathbb{R}^q$ to
    $\mathbb{R}^p,i.e p \times q$ matrices] such that for $f
  \epsilon C^{\infty,q}(U)$ and $x \epsilon \Omega$ we have 
  \begin{equation}
    (Lf)(x) = \sum_{|\alpha \le m|} a_\alpha (x) (D^\alpha f)
    (x).\tag{2.1}\label{chap3:sec2:eq2.1} 
  \end{equation}
\end{theorem*}

\begin{proof}
  Let $\Omega'$ be an open subset of $\Omega$. We shall first prove
  equation (\ref{chap3:sec2:eq2.1}) on $\Omega'$ with the following additional
  assumption: there exists a constant $C > 0$ and an integer $m$ such
  that if $f \epsilon C_0^{\infty, q} (\Omega')$, we have 
  \begin{equation}
    || Lf ||_0 \le C || f ||_m. \tag{2.2}\label{chap3:sec2:eq2.2}
  \end{equation}
\end{proof}

First we remark that if $\varphi \epsilon C^{\infty,q}_0 (\Omega')$
and if $\varphi$ is m-flat $a \epsilon \Omega'$ then $(L \varphi)
(a) = 0$. In fact by \S\ 5 Chapter $I$, there exists a sequence
$\{f_\nu\}$ of functions in $C_0^{\infty,q} (\Omega')$ such that
$f_\nu (x) = 0$ for $x$ in a neighbourhood of $a$ and $|| \varphi-
f_\nu||_n^{\Omega'} \to 0$. Since $\supp (Lf_\nu) \subset \supp f_\nu$
we have $Lf_\nu (a) = 0$ and because of the inequality (2.2), $(L
\varphi) (a) = \lim\limits_{\nu \to \infty} (Lf_\nu) (a) =0$. For $a
\epsilon \Omega'$ and $f \epsilon C^{\infty,q} (\Omega')$,
Taylor's formula\pageoriginale gives us the following: 
$$
f (x) = \sum_{|\alpha| \le m} \frac{(x-a)^\alpha} {\alpha !}. D^\alpha f (a) + g (x),
$$
where $g$ is $m$-flat at $a$. Hence by the remark above,
\begin{align*}
  (Lg) (a) & = 0 \\
  \text{i.e} \qquad (Lf) (a) & =\sum_{|\alpha| \le m} \frac{L[(x-a)^\alpha
      D^\alpha f(a)] (a)} {\alpha !}. 
\end{align*}

In what follows we write elements of $C^{\infty,q}$ as columns. Let $f
=\begin{bmatrix} f_1 \\ \vdots \\ f_q\end{bmatrix}$, $f_i$ being
$C^\infty$ functions and let $e_k =  \begin{bmatrix} 0
  \\ \vdots\\ 1\\ 0\end{bmatrix}$, where 1 occurs in the $k^{th}$
  place. 
\begin{align*}
  \text{ Then } \qquad \frac{(x-a)^\alpha D^\alpha f(a)} {\alpha !} &
  = \sum_{1 \le k \le q} \frac{D^\alpha f_k (a). (x-a)^\alpha
    e_k}{\alpha !}. \qquad \text{ Hence }\\ 
  (Lf) (a) & = \sum_{|\alpha| \le m} \sum_{1 \le k \le q} \frac
  {D^\alpha f_k (a)} {\alpha !} L [(x-a)^\alpha e_k] (a). 
\end{align*}
[Recall our remark that $L$ can be applied to $C^\infty$ functions
  which are not compactly supported.] Now 
$$
(x-a)^\alpha e_k = \sum_{\beta \le \alpha} (^\alpha_\beta) x^\beta
(-a)^{\alpha-\beta} e_k, 
$$
and, by definition, $L (x^\beta e_k)$ is continuous on $U$ (and not
just on $\Omega'$). Hence $(L (x-a)^\alpha e_k) (a)$ is a continuous
function of a in $U$ and can be identified with a $p$-tuple  
$$
[L (x-a)^\alpha e_k] (a) = \begin{bmatrix} a^{1k}_{\alpha}(a)
  \\ \vdots\\ a^{pk}_\alpha (a)\end{bmatrix}. 
$$

Thus,\pageoriginale if $\Omega'$ is an open subset of $U$ and if there exists $m$,
$C$ such that 
$$
|| Lf ||_0 \le C || f ||_m \text{ for } f \in C^{\infty,q}_0
(\Omega'), 
$$
there exist continuous functions $a_\alpha$ \underline{on $\Omega$} such that
$$
(Lf) (x) = \sum_{|\alpha|\le m} a_\alpha (x) D^\alpha \text{ for } f
\epsilon C^{\infty,q} (\Omega'), x \epsilon \Omega'. 
$$

Moreover if $Lf =\sum a_\alpha D^\alpha f$ for all $f \epsilon
C^{\infty,q} (W)$, where $W$ is an open subset of $\Omega$, the
$a_\alpha$ are uniquely determined on $W$ by $L$. Consequently, if
suffices to prove that every a $\epsilon \Omega$ has a
neighbourhood $W$ such that (\ref{chap3:sec2:eq2.1}) holds for all $f \epsilon
C^{\infty, q} (W)$. Now, by the remark above and Lemma
\ref{chap3:sec2:lem1} there is a $W$ such that 
$$
(Lf) (x) = \sum_{|\alpha| \le m} a_\alpha (x) D^\alpha f (x), x
\epsilon W - \{ a \}, f \epsilon C^{\infty,q} (W-\{a\}) 
$$
where the $a_\alpha$ are continuous in $W$. Since, for $f \epsilon
C^{\infty,q} (W)$, both sides of this equation are continuous in $W$,
the result is proved. 

\noindent
\textbf{Note.} A result even somewhat more general than the one proved
here is due to J. Peetre \cite{35}. 

\section{The Cauchy Kovalevski Theorem}\label{chap3:sec3} %sec 3

\setcounter{lemma}{0}
\begin{lemma}\label{chap3:sec3:lem1} %lemm 1
  Let $D = \left\{ w \epsilon \mathbb{C} \big| |w| < R\right\}$ and
  $h$, be a holomorphic function on $D$. If, for $w \epsilon D$,
  $|h' (w)| \le A |w|^r$ and $h (0) = 0$, then 
  $$
  |h (w)| \le A \frac{|w|^{r+1}}{r+1}.
  $$
\end{lemma}

\begin{lemma}\label{chap3:sec3:lem2}%lemm 2
  Let\pageoriginale $D = \left\{ w \epsilon \mathbb{C} \big| |w| < R\right\}$ and
  $h$ be a holomorphic function on $D$. If 
  \begin{gather*}
    h (0) = 0, \left| h' (w) \right| <\frac{A}{(R - |w|)^{r+1}} \text{
      for } w \epsilon D,\\  
    then \qquad \left| h (w) \right| < \frac{A}{r (R-|w|)^r}.
  \end{gather*}
  The proof of the above lemmas follows at once from the
  equation $\int_0^w h'(z) dz = h(w)$.  
\end{lemma}

\begin{lemma}\label{chap3:sec3:lem3}%lemm 3
  If $D =\left\{ w \epsilon \mathbb{C} \big| |w| < R\right\}$ and
  $h$ s is a holomorphic function on $D$, and if 
  \begin{align*}
    & \left| h (w) \right| < \frac{A}{(R-|w|)^r}, \text{ for } w
    \epsilon D,\\ 
    \text{ then }  \qquad & \left| h' (w) \right| < \frac{3A (r+1)}
         {(R-|w|)^{r+1}}. 
  \end{align*}
\end{lemma}

\begin{proof}
  Let $w_0 \epsilon D$ and $0 <\epsilon < R- |w_0|$. Then by
  Cauchy's inequality we have 
  $$
  \left| h'(w_0) \right| \le \frac{1}{\varepsilon} \sup_{|w-w_0| =
    \varepsilon} \left|  h (w)\right|. 
  $$
  Hence $ \left| h'(w_0) \right| \le \dfrac{A}{\varepsilon \{R - |w_0|
    -\varepsilon\}^r}$ for any $\varepsilon$ with $o < \varepsilon <
  R-|w_0|$. 

  Take $\varepsilon = \dfrac {R-|w_0|} {r+1}$.

  Then $\left| h'(w_0) \right| \le \dfrac{A (r+1)} {(R-|w_0|)^{r+1}}
  . \left(\dfrac{r+1}{r}\right)^r$. 
  
  Hence\pageoriginale $\left| h'(w_0) \right| <  \dfrac {3 A (r+1)} {(R -|w_0|)^{r+1}}$.
\end{proof}

\begin{theorem*}[(Cauchy-Kovalevski)]%theo 0
  Let $\Omega = \left\{ (z_1 ,\ldots, z_n) \epsilon \mathbb{C}^n
  \Big| |z_i| < r_i\right\}$. Let $g$, $\varphi$: $\Omega \to
  \mathbb{C}^q$ be holomorphic functions on $\Omega$ and let $\beta =
  (\beta_1 ,\ldots,\break \beta_n)$, $\beta_i \epsilon \mathbb{Z}^+$ with
  $\beta_n > 0$. Let $\alpha$ run over the multiindices with $|\alpha|
  \le |\beta|, \alpha_n < \beta_n$, and suppose that for each $\alpha$
  is given a holomorphic map $a_\alpha$ of $\Omega$ into the space of
  $q \times q$ matrices. Then exists a neighbourhood $U$ of $0$ and a
  unique holomorphic functions $f$ on $U$, $f : U \to \mathbb{C}^q$
  such that 
  \begin{equation}
    D^\beta f (z) = \sum_{\substack{|\alpha|\le|\beta|\\ \alpha_n <
        \beta_n}} a_\alpha (z). D^\alpha f (z) + g
    (z),\tag{3.1}\label{chap3:sec3:eq3.1}  
  \end{equation}
  and
  \begin{equation}
    \left(\frac{\partial} {\partial z_i}\right)^l (f-\varphi) = 0 \text{ for }
    z_i = 0 \text{ and } 0 \le l < \beta_i.\tag{3.2}\label{chap3:sec3:eq3.2} 
  \end{equation}
\end{theorem*}

\begin{proof}
  We may suppose without loss of generality that $r_i \le 1$ and that
  $\varphi = 0$, since if $h = f -\varphi$, the problem then would be
  to solve the equation 
  $$
  D^\beta h = \sum_{\substack{|\alpha|\le|\beta|\\ \alpha_n <
      \beta_n}} a_\alpha D^\alpha h + g', 
  $$
  with $\qquad (\dfrac{\partial}{\partial z_i})^l (h) = 0$, for $z_i =
  0$ and $0 \le l < \beta_i$,  
  where $g'$ is holomorphic on $\Omega$. We may further suppose that
  the $a_\alpha$ are bounded\pageoriginale in $\Omega$. 
\end{proof}

We first remark that for a holomorphic function $h$ on $\Omega$, there
exists a unique holomorphic function $u$ on $\Omega$ such that 
\begin{gather*}
  D^\beta u =h \\
  \text{ and } \qquad \left(\frac{\partial} {\partial z_i}\right)^l u = 0 \text{
    for } z_i = 0 \text{ and } 0 \le l < \beta_i. 
\end{gather*}

To prove this it is enough to show that there exists a unique
holomorphic function $u$ on $\Omega$ such that $\dfrac {\partial u}
{\partial z_i} = h$ and $u =0$ on $z_1 =0$. But this is immediate; we
must set $u (z) = \int^z_0 h(\zeta) d \zeta$. We define holomorphic
functions $f_k :\Omega \to \mathbb{C}^q$ as follows: $f_0 (z) = 0$ for
$z \epsilon \Omega$, and $f_k (z)$, for $k \ge 1$, is defined, by
induction, as the unique holomorphic solution of 
$$
\displaylines{\hfill 
 D^\beta f_k = \sum_{\substack{ |\alpha|\le|\beta| \\ \alpha_n <
     \beta_n}} a_\alpha D^\alpha f_{k-1} +g  \hfill \cr
 \text{with} \hfill \left(\dfrac{\partial}{\partial z_i}\right)^l f_k
 = 0 \text{ for } z_i = 0 \text{ and } 0 \le l < \beta_i \hfill }
$$

It is clear from the remark made above that $\left\{f_k\right\}_{k \ge
  0}$ are defined on $\Omega$. Let $u_0(z) = 0$, $u_k (z) = f_k(z) -
f_{k-1} (z)$. Then $u_k$ satisfies 
\begin{equation*}
   D^\beta u_{k+1} = \sum_{\substack{ |\alpha|\le|\beta| \\ \alpha_n <
       \beta_n}} a_\alpha D^\alpha u_k \tag{3.3}\label{chap3:sec3:eq3.3} 
\end{equation*}
 with  $ \left(\frac{\partial}{\partial z_i}\right)^l
   u_k = 0$  for $z_i = 0,0 \le l <  \beta_i$. 

 Let $\rho (z) = (r_1 - |z_1|) \cdots (r_{n-1} - |z_{n-1}|)$ and let
 $|\beta| = m$. We shall now prove\pageoriginale that there exists a constant A such
 that the relations (\ref{chap3:sec3:eq3.3}) imply the estimates 
 \begin{equation}
   \left| D^\beta u_r (z) \right| \le \frac{A^r |z_n|^r}{ \{\rho
     (z)\}^{mr+1}}.\text{ for } z \epsilon \Omega
   . \tag{3.4}\label{chap3:sec3:eq3.4}   
 \end{equation} 
 
Assume that (\ref{chap3:sec3:eq3.4}) holds for $r=k$. Then
$$
\left| D^\beta u_k (z) \right| \le \frac{A^k |z_n|^k}{ \{\rho
  (z)\}^{mk+1}}. 
$$

Applying Lemma \ref{chap3:sec3:lem1} with respect to $z_n$, $(\beta_n - \alpha_n)$
 times and Lemma \ref{chap3:sec3:lem2} with respect to $z_i$, $\beta_i$ times for $1 \le
 i \le n-1$, we have, since $r_i \le 1$, 
 \begin{multline*}
 \left| \left(\frac{\partial} {\partial z_n}\right)^{\alpha_n} u_k (z)
 \right| \le\\ 
 \frac{A^k |z_n|^{k+\beta_n -\alpha_n}}{(k+1) ,\ldots, (k+\beta_n
   -\alpha_n) \prod\limits_{q}^{n-1} (r_i - |z_i|)^{mk+1} (mk) ,\ldots,
   (mk +1 \beta_i)}  
\end{multline*}
hence, since $\alpha_n < \beta_n, \left| \left(\dfrac{\partial} {\partial
   z_n}\right)^{\alpha_n} u_k (z)\right| \le \dfrac{A^k |z_n| ^{k+1}} {[\rho
     (z)]^{mk+1}}$. $k^{-(m-\alpha_n)}$ 

  Now using Lemma \ref{chap3:sec3:lem3}  with respect to $z_i$, $\alpha_i$ times, for $1
  \le i \le n-1$, we obtain 
 $$
 \left| D^\alpha u_k (z)\right| \le \frac{3^m A^k |z_n|^{k+1}} {[\rho
     (z)] ^{m(k+1)+1}}. \left[\frac{m(k+1) +1} {k}\right]^{m-\alpha_n}. 
 $$
  Hence by equation (\ref{chap3:sec3:eq3.3}), since the $a_\alpha$ are bounded,
 $$
 \left| D^\beta u_{k+1} (z)\right| \le \frac{A^k |z_n|^{k+1}} {[\rho
     (z)] ^{m(k+1)+1}}. 3^m. M \left[\frac{m(k+1) +1}
   {k}\right]^{m-\alpha_n} 
 $$
 for some constant $M$ (independent of $\alpha$ and $k$).
 
 Hence\pageoriginale if $A=\sup\limits_k$. $3^m$. $M  \left[\dfrac{m(k+1) +1}
   {k}\right]^m$, the inequality (\ref{chap3:sec3:eq3.4}) is
 proved. Consequently, if $z$ 
 satisfies $|z_n| < \left[b \rho (z)\right]^m, \sum\limits_k \left|
 D^\beta u_k (z)\right|$ is convergent. Hence there exists a
 neighbourhood $U$ of $0$ such that $\sum\limits_k |D^\beta u_k|$ is
 uniformly on $U$. This clearly implies that $f_k$ is uniformly
 convergent on $U$ and if $f (z) = \lim\limits_{k \to \infty} f_k
 (z)$, $f(z)$ is a holomorphic function which satisfies equation
 (\ref{chap3:sec3:eq3.1})
 with the initial conditions (\ref{chap3:sec3:eq3.2}). Again if $f$ and $f'$ are two
 holomorphic solutions of (\ref{chap3:sec3:eq3.1}) satisfying
 (\ref{chap3:sec3:eq3.2}) let $f(z) - f'(z) = u 
 (z)$. Then if $u_k (z) = u(z)$, $k \ge 1$, we have  
$$
 \displaylines{\hfill
   D^\beta u_{k+1} (z) = \sum a_\alpha D^\alpha u_k\hfill \cr
   \text{ and } \hfill \left(\frac{\partial} {\partial z_i}\right)^l u_k = 0
   \text{ for } z_i = 0 \text{ and } 0 \le l \le \beta_i , \hfill }
$$
 i.e $(u_k)$ satisfies equations (\ref{chap3:sec3:eq3.3}) and by the
 discussion above, 
 there exists a neighbourhood $U'$ of $0$ such that $\sum |u_k|$ is
 uniformly convergent on $U'$. But this implies that $u(z) = u_k (z) =
 0$, which proves the uniqueness of the solution. 
 
 \section{Fourier transforms, Plancherel's theorem}\label{chap3:sec4} %sec 4
 
 \begin{defis*}
   (1)~{\em{If  $f \epsilon  L' (\mathbb{R}^n)$, the fourier
       transform of $f$, denoted by $\hat{f}$, is defined by}} 
   $$
   \hat{f} (\xi) = \frac{1}{(2 \pi)^{n/2}}
   \int\limits_{\mathbb{R}^n} f (x) e^{-ix \xi}{dx}. 
   $$
   (2)~Let\pageoriginale $\mathscr{S}$ be the set of $C^\infty$ functions $f$ on
   $\mathbb{R}^n$ such that for any polynomial $P$  on $\mathbb{R}^n$ and
   any $\alpha$, we have, $\sup\limits_{x \epsilon \mathbb{R}^n}
   \left| P (x) D^\alpha f (x) \right| < \infty$. The space $\mathscr{S}$
   is called Schwartz space. 
 \end{defis*}

 \begin{remarks*}%rem 
   \begin{enumerate}[(1)]
   \item For every $p \ge 1$, $\mathscr{S} \subset L^p$ and
     $\mathscr{S}$ is dense in $L^p $ (in $L^p$ norm) if $p <
     \infty$. 
   \item If $f \epsilon \mathscr{S}$, $D^\alpha f \epsilon
     \mathscr{S}$ for every $\alpha$. 
   \item Any function in $\mathscr{S}$ is bounded. 
   \item If $f \epsilon \mathscr{S}$, it is verified by integration
     by parts that (i) $(D^\alpha f)\hat{(\xi)} = i^{|\alpha|}
     \xi^\alpha \hat{f} (\xi)$ and (ii) $D^\alpha \hat{f} (\xi) =
     \left\{ (-i \xi)^\alpha f (\xi) \right\}$. 
   \item For any $f \epsilon L'$. 
     $$
     \sup\limits_{ \xi  \epsilon \mathbb{R}^n} \left| \hat{f}
     (\xi)\right| \le \sup\limits_{\xi \epsilon \mathbb{R}^n} \int
     \left| e^{-ix \xi} f(x)\right| dx. 
     $$
   \item It follows from remarks (4) and (5) that if $f
     \epsilon \mathscr{S}$, so is $\hat{f}$.   
   \end{enumerate}
 \end{remarks*} 

\setcounter{proposition}{0}
\begin{proposition}[Inversion formula]\label{chap3:sec4:prop1}%prop 1
   If $f \epsilon \mathscr{S}$, we have, 
   $$
   f (y)  = \frac{1}{(2 \pi)^{\frac{n}{2}}} \int\limits_{\mathbb{R}^n}
   \hat{f} (\xi) e^{iy \xi} d \xi. 
   $$
\end{proposition} 

\begin{proof}
  Let $\varphi \epsilon \mathscr{S}$. Consider
  $\int\limits_{\mathbb{R}^n} \varphi (\xi) \hat{f} (\xi) e^{iy \xi} d
  \xi$ 
  $$
  = \frac{1}{(2 \pi) ^{\frac{n}{2}}} \int_{\mathbb{R}^n} \varphi (\xi)
  e^{iy \xi} (\int_{\mathbb{R}^n} f (x) e^{-ix \xi} dx) d \xi. 
  $$
 \end{proof}

By Fubini's theorem, we have
\begin{align*}
  \int \varphi (\xi) \hat{f} (\xi) e^{iy \xi} d \xi & = \frac{1}{(2
    \pi)^{\frac{n}{2}}} \int f (x) dx \int \varphi (\xi) e^{-i (x-y)
    \xi} d \xi\\ 
  & = \frac{1}{(2 \pi)^{\frac{n}{2}}} \int f (y +t) e^{it \xi} \varphi
  (\xi) d \xi dt ~[x-y =t] \\ 
  & = \int f (y +t)) \hat{\varphi} (t) dt. \tag{4.1}\label{chap3:sec4:eq4.1}
\end{align*} 
 
Now,\pageoriginale set $\varphi (\xi) = \psi (\epsilon \xi)$, where $\psi
\epsilon \mathscr{S}$. Then, as is easily verified, we have 
$$
\hat{\varphi} (t) = \varepsilon ^{-n} \hat{\psi} (\frac{t}{\varepsilon}).
$$
Hence  
\begin{align*}
  \int \varphi (\xi)   \hat{f} (\xi) e^{i y \xi} d \xi
  &= \int \psi (\varepsilon \xi) \hat{f} (\xi) e^{iy \xi} d \xi\\
  &= \int f (y + t) \varepsilon ^{-n} \hat{\psi}
  \left(\frac{t}{\varepsilon}\right) dt\hspace{1cm}\\
  \text{i.e}\hspace{2cm}
  \int\limits_{\mathbb{R}^n} \psi (\varepsilon \xi) \hat{f} (\xi) e^{iy
    \xi} d\xi &= \int\limits_{\mathbb{R}^n} f (y+\varepsilon t)
  \hat{\psi} (t) dt. 
\end{align*}

Since $f$ and $\psi \epsilon \mathscr{S}$, we can take the limits
as $\varepsilon \to 0$ under the integrals, so that  
$$
\psi (0) \int_{\mathbb{R}^n} \hat{f}(\xi) e^{iy \xi} d\xi = f (y)
\int_{\mathbb{R}^n} \hat{\psi} (t) dt. 
$$

If we set $\psi (t) = e^{-\frac{t^2}{2}}$, it is easily verified that $\psi (0)
= (2 \pi)^{\dfrac{n}{2}} \int_{\mathbb{R}^n}\break \hat{\psi} (t) dt$, so
that 
$$
(2 \pi)^{-\frac{n}{2}} \int\limits_{\mathbb{R}^n} \hat{f}(\xi) e^{iy
  \xi} d \xi = f(y). 
$$

\begin{coro*}%coro 0
  For\pageoriginale $f$ and $\varphi \epsilon \mathscr{S}$, we have,
  $$
  \int\limits_{\mathbb{R}^n} \varphi (\xi) \hat{f} (\xi) d \xi =
  \int\limits_{\mathbb{R}^n} f (\xi) \hat{\varphi} (\xi).dt 
  $$
  This follows from equation (\ref{chap3:sec4:eq4.1}) on putting $y = 0$.
\end{coro*} 

\begin{remark*}%rema 0
  As is evident from the proof, (\ref{chap3:sec4:eq4.1}) holds
  whenever $f \epsilon 
  L^1 (\mathbb{R}^n)$ and $\varphi \epsilon \mathscr{S}$. 
\end{remark*} 

\begin{lemma*}%lem 0
  If $f \epsilon \mathscr{S}$, we have $||f||_{L^2} = ||\hat{f}||_{L^2}$
  \begin{align*}
    \big[ \text{ For } g \epsilon L^p, ||g||_{L^p} & = \text{ norm
        of } g \text{ in } L^p\\ 
      &  = \left(\int |g (\xi)|^p d  \xi\right)^{\frac{1}{p}} \text{ when } 1 \le
      p < \infty,\\ 
      || g ||_{L^\infty} & = ess. \sup |g (x)|.  \big]
  \end{align*}
\end{lemma*} 

\begin{proof}
  It follows from the inversion formula that for $f \epsilon
  \mathscr{S} \big|, \hat{\hat{f}} (-y ) = f (y)$. Define $\psi$:
  $\mathbb{R}^n \to \mathbb{C}$ by 
  \begin{align*}
    \psi (t) & = \overline{\left\{ \hat{f} (t) \right\}} ; \text{ we have } \\
    \hat{f} (t) & =\int f (\xi) e^{-it \xi} d \xi\\
    & = \overline{\left(\int \bar f (\xi) e^{it \xi} d \xi\right)}.
  \end{align*}
  
  Hence 
  \begin{equation*}
    \bar{\hat{f}} (t)  = \hat{\bar{f}} (-t)
    \tag{4.2}\label{chap3:sec4:eq4.2}
  \end{equation*}
    i.e
    $$
    \hat{\psi} (t) = \bar{f} (t). 
      $$
  
  By\pageoriginale the corollary to the inversion formula,
  $$
  \int f (t) \hat{\psi} (t) dt = \int \psi (t) \hat{f} (t) dt.
  $$
  Now $\psi (t) = \bar{\hat{f}}$ and $\hat{\psi} (t) = \bar{f}$, from
  which it follows that 
  $$
  ||f||_{L^2} = ||\hat{f}||_{L^2}.
  $$
\end{proof} 

\begin{defi*}%defi 0
  If $f \epsilon L^p$ for some $p \ge 1, \hat{f}$ is defined as the
  linear functional on $\mathscr{S}$, for which  $\hat{f} (\psi) =
  \int f (t) \hat{\psi} (t) dt, \psi \epsilon \mathscr{S}$. If
  there exists a function $g \epsilon L^p$ (for some $p$ with $1
  \le p \le \infty$), such $\int g (t) \psi (t) dt = \int f (t)
  \hat{\psi} (t) dt$ for all  $\psi \epsilon \mathscr{S}$, we shall
  identify $f$ with the function $g$. Note that if $f \epsilon
    L^1$, this compatible with the definition given at the beginning
    (as follows the remark after the inversion formula). 
\end{defi*} 

\begin{theorem*}[(Plancherel)]%theo 0
  If $f \epsilon L^2$, then there exists $g \epsilon L^2$ such
  that the linear map $\hat{f}$ is given by 
  \begin{alignat*}{4}
    &&\hat{f} (\psi) & = \int \psi (t) g (t) dt \hspace{4cm} \\
    \text{and} &\hspace{3cm}& ||f ||_{L^2} & = ||g||_{L^2}.\\
  \end{alignat*}
  {\em In other words}, $\hat{f} \epsilon L^2$ and $||f||_{L^2} =
  ||\hat{f}||_{L^2}$. 
\end{theorem*}
 
\begin{proof}
  Since $\mathscr{S}$ is dense in $L^2$, there is a sequence $\{f_\nu
  \}$ of function in $\mathscr{S}$ such that $|| f_\nu - f ||_{L^2}
  \to 0$. It follows from the lemma above that 
  $$
  ||f_\nu - f_\mu||{_{L^2}} = ||\hat{f}_\nu - \hat{f}_\mu||_{L^2}
  $$
  so\pageoriginale that $|| \hat{f}_\nu - \hat{f}_\mu||_{L^2} \to \text{ as } \nu,
  \mu \to \infty$. Hence there exists $g \epsilon L^2$ such that 
  $$
  || \hat{f}_\nu - g||_{L^2} \to 0.
  $$
\end{proof} 
 
Clearly $|| f ||_{L^2} = ||g||_{L^2}$.
 
Now for any $\psi \epsilon \mathscr{S}$, we have
$$
\int f_k (t)\hat{\psi} (t) dt = \int \hat{f}_k (t) \psi (t) dt.
$$
Since $||f_k - f||_{L^2} \to 0$ and $|| \hat{f}_k - g||_{L^2} \to 0$,
we have, taking limits as $k \to \infty$, 
$$
\int f (t) \hat{\psi} (t) dt = \int g(t) \psi (t) dt.
$$

\begin{remark*}%rema 0
  The inversion formula can be written
  $$
  \hat{\hat{f}}(-y) = f (y), \text{ for } f \epsilon \mathscr{S}.
  $$
  It is an immediate consequence of Plancherel's theorem that this
  relationship holds if $f \epsilon L^2$. Further, if $f
  \epsilon L^1$, then as in the proof of the inversion formula, we
  have, for $\psi \epsilon \mathscr{S}$, 
  $$
  \int \psi (\varepsilon \xi)\hat{f} (\xi) e^{iy \xi} d \xi = \int f
  (y + \varepsilon t) \hat{\psi} (t) dt, 
  $$
  so that if we suppose that we have also $\hat{f} \epsilon L^1$ we
  may take limits as $\varepsilon \to 0$, [the term on the right
    converges to $f \int \hat{\psi} (t) dt$ in $L^1$ norm]. 
\end{remark*} 

From\pageoriginale this we conclude that $ \hat f (- y) = (- y)$. [This implies in
  particular that $f$ is then bounded and continuous.]

\begin{proposition}\label{chap3:sec4:prop2}%prop 2
  If $f$, $g \epsilon L^1$, then $\int\limits_{\mathbb{R}^n } | f
  (x - y ) g (y )|dy < \infty $ for almost all $x$  and if  
  \begin{gather*}
    (f * g ) (x ) = \int  f (x - y )g (y ) dy,\\
    || f * g ||_{L^1} \leq || f ||_{L^1} . || g ||_{L^1}.
  \end{gather*}
\end{proposition}

\begin{proof}
  It is enough to prove the proposition for $f \ge 0$ and $g \ge
  0$. We have by Fubini's theorem  
  \begin{align*}
    \int dx \int f (x - y ) g (y ) dy & = \int g (y ) dy \int f (x - y ) dx. \\
    & = \left(\int f (x) dx \right) ~\left(\int g (y) dy \right) < \infty
  \end{align*}
  and the proposition follows. 
\end{proof}

\begin{proposition}\label{chap3:sec4:prop3}%prop 3
  If $f$, $g \epsilon \mathscr{S}$, $f * g \epsilon \mathscr{S}$ and 
  $$
  (f * g \hat {)} = (2 \pi )^\frac{n}{2} \hat{f} \hat{g}.
  $$
\end{proposition}

\begin{proof}
  It is clear that $f * g \epsilon \mathscr{S}$. Now,
  \begin{align*}
    (f * g \hat{)} (x )  & = \left(2 \pi \right)^{-\frac{n}{2}} \int e^{- ixt }
    dt \int f (t - y ) g (y ) dy \\ 
    & = \left(2 \pi \right)^{-\frac{n}{2}} \int g (y )  dy \int f (t - y ) e^{-
      ixt }  dt \\ 
    & = \left(2 \pi \right)^{-\frac{n}{2}} \int  g (y) e^{- ixy } dy \int f (t
    )  e^{- ixt } dt \\ 
    & = \left(2 \pi \right)^{\frac{n}{2}} \hat{f } (x). \hat{g} (x).
  \end{align*}
\end{proof}

\begin{coro*}%coro 0
  For\pageoriginale $f$, $g \varepsilon  \mathscr{S}$, we have, 
  $$
  (fg \hat{)} = (2 \pi)^{- \frac{n}{2}} \hat{f} * \hat{g}.
  $$
  This follows from the above proposition and the inversion formula.
\end{coro*}

\begin{remark*}%rema 0
  In fact, the above result is true  for $f  \epsilon L^i$, $i =
  1$, $2$ and $g \epsilon \mathscr{S}$. 
\end{remark*}

\begin{proof}
  Let $\{ f_{\nu} \}$  be a sequence in $\mathscr{S}$ such that 
  $$
  ||  f_{\nu} -  f||_{L^i} \to 0
  $$
  Then $ \qquad \qquad( f_{\nu} . g)^{\wedge} = (2 \pi)^{ -
    \dfrac{n}{2}} \hat{f}_{\nu} * \hat{g}$. 
  \begin{align*}
    \text{ If } \qquad f \in L^2, = \pi^2 \hat{f}_{\nu} * \hat{g} (t)
    & - \hat{f} * \hat{g} (t) \\ 
    &=  \int (\hat{f}_{\nu} - \hat{f}) (t-y) \hat{g} (y) dy,
  \end{align*}
  and using Schwarz's inequality,
  $$
  \lim _{\nu \to \infty} \hat{f}_{\nu} * \hat{g}(t) = \hat{f} * \hat{g} (t).
  $$
\end{proof}

If $f \epsilon L^1$ and $|| f_{\nu} - f||_{L^1} \to 0$. then $
\hat{f}_{\nu} \to \hat{f}$ uniformly, so that $\hat{f}_{\nu} * \hat{g}
(t) \to \hat{f} * \hat{g}(t)$ uniformly and hence $\lim\limits_{\nu
  \to \infty} \hat{f}_{\nu} * \hat{g} (t) = \hat{f} * \hat{g} (t)$.
Further, since $g$ is bounded, $f_{\nu} g \to f g$ in $L^i$, so that
$(f_{\nu} g \hat{)} (t) \to (f g) \hat{)}(t)$ [pointwise for $i = 1$,
  in $L^2$ for $i =2$]. 

Hence, if $f  \epsilon L^1 $  or $ f  \epsilon L^2$, we have,
\begin{align*}
  (fg\hat{)} (t) = \lim_{\nu \to \infty} (f_{\nu} g\hat{)}(t) & =
  \lim_{\nu \to \infty} (2 \pi)^{-n/2} \hat{f}_{\nu} * \hat{g}(t)\\ 
  & = (2 \pi)^{-n/2} \hat{f}  *  \hat{g}(t). 
\end{align*} 

\section{The Sobolev spaces $H_{m, p}$}\label{chap3:sec5}%sec 5

In\pageoriginale this section we have given proofs of the most important results in
$L^{p}$; however since we shall need only the $L^{2}$ statements, we
have included simple proofs in this special case (based on
Plancherel's theorem). 

Let $\Omega$ be an open set in $\mathbb{R}^{n}$, $p$, $a$ real number,
$p \ge 1, q,m$ integers, $q > 0, m \geq 0 $. Let  $f = (f_1, \ldots
,f_q) : \Omega \to \mathbb{C}^{q}$ be a $C^{\infty}$ map. Consider
the space$  \{f: \Omega \to \mathbb{C}^{q}$, $f \in C^{\infty}(\Omega)
| \sum\limits_{\substack{|\alpha | \le m \\ 1 \le i \le q}}\int |
D^\alpha f _i(x)|^p dx < \infty \}$. 

Define a norm $|f|_{m,p}$ on this space by
$$
|f|^p_{m, p} = \sum_{|\alpha| \leq m} \sum_{1 \leq i \leq q} \int
|D^{\alpha} f_i|^p dx. 
$$

We shall write $|f|^{\Omega}_{m,p}$ for this norm when its dependence
on $\Omega$ is relevant. The completion of the above space is called
the Sobolev space $H_{m,p}(\Omega)$. If a sequence $\{ f_{\nu}\}$ of
$C^{\infty}$ functions converges in $H_{m,p}(\Omega)$, the sequence
$D^{\alpha} f_{\nu}$ is convergent in $L^p$, to $f^{\alpha}$, say. The
limit of $f_{\nu}$ in  $H_{m,p}(\Omega)$ is denoted by $f$ and
$f^{\alpha}$ is called the derivative of order $\alpha$ of $f$, and we
write $D^{\alpha}f = f^{\alpha}$. [We shall see below that $ f
  ^{\alpha} $ is independent of the sequence $\{f_\nu\}]$. We shall
denote $H_{m,2}(\Omega)$ by $H_m(\Omega)$. For a mapping $f =
(f_1,\ldots, f_q)$: $\Omega \to \mathbb{C}^q$, we write $f \in L^{p}$
if $f_i \epsilon L^{p}$ for each $i$: for $f \epsilon L^{p}$, we
define $||f|| _{L^{p}}$ by. 
$$
|| f || ^{p}_{L^{p}} = \sum_{i=1}^{q}  || f_i ||^{p}_{L^{p}}.
$$

Let\pageoriginale $C_{o}^{\infty,q}$ be the subspace of $H_{m,p}(\Omega)$, of
$c^{\infty}$ functions $g:Omega \to \mathbb{C} ^{q}$, with compact
support. Then the closure of $C_{o}^{\infty,q}$ in $H_{m,p}(\Omega)$
is denoted by $H^0_{m,p}(\Omega)$. 

For vectors $v_1$, $v_2 \in \mathbb{C}^{q}$ (or $ \mathbb{R}^q)$ we
shall denote by $(v_1,v_2)$, the usual scalar product, i.e. if
$v_i=(v^1_i, \ldots, v_i^{q})$, then $  (v_1,v_2)= \sum_{k=1}^{q}
v_1^{k}  v^{\overline{k}}_2$ ; similarly, for mappings $f= (f_1,
\ldots ,f_q) , g = (g_1,\ldots,g_q) : \Omega \to \mathbb{C}^{q}$,
we write 
$$
(f,g)= \sum_{i=1}^{q} \int_{\Omega}f_i(x) \overline{g_i(x)}dx.
$$

\begin{defis*}%defi 0
  (1)~ If $f \in L^{p}$ and if $ f \in H_{m,p}(\Omega')$ for
    every relatively compact subset $\Omega'$ of $ \Omega$, then $f$ is
    said to be strongly differentiable, upto order $m$, in $L^{p}$. If
    $p=2$, we speak simply of strong differentiability. 

  (2)~If $f \in L^{p}$ and if there exist functions
    $h^{\alpha}$ in $L^{p}, | \alpha| \leq m$, such that for any $g
    \epsilon C_{\circ}^{\infty,q}$, 
  $$
  \int _{\Omega}(f(x), D^{\alpha}g(x)) dx =(-1)^{|
    \alpha|}\int_{\Omega}(h^{\alpha}(x),g(x)) dx, 
  $$
  then $f$ is said to have weak derivatives upto order $m$ in $L^p$
  and the $h^{\alpha}$ are called the weak derivatives of $f$.
\end{defis*}

\begin{remark*}%rema 0
  \begin{enumerate}[(1)]
  \item If $\int _\Omega(h^\alpha(x), g(x))  dx = \int
    _{\Omega}(h^{'\alpha}(x)$, $g(x))dx$ for all functions $g
    \epsilon C^{\infty}_{0}$, clearly $h^\alpha (x) = h^{'\alpha}(x)^{\Omega}$
    almost everywhere and hence the weak derivatives of $f$, if they
    exist, are uniquely determined. 
  \item If\pageoriginale a function in $L^p$ has strong derivatives upto order $m$
    they are weak derivative of $f$. This follows at once from
    Holder's inequality. In particular, if $f_{\nu} \epsilon
    C^{\infty,q}$ and $f_{\nu}\to f $ in $H_{m, p}$, the limits $
    \lim\limits_{\nu \to \infty} D^{\alpha}f_{\nu}$ in $L^{p}$ are
    independent of the sequence $\{f _{\nu}\} $, being weak
    derivatives of $f$. 
  \item Let $0 \le m' \le m$ and $f \epsilon H_{m,p}$. Then there
    exists a sequence $\{f _{\nu}\}$ of $C^{\infty}$ functions such
    that $f_{\nu}\to f $ in $H_{m,p}$. But this implies that
    $f_{\nu}\to f $ in $H_{m',p}$ and if $D^{\alpha}f_{\nu}\to
    f^{\alpha}$ in $L^{p}$, $f^{o}=f$ almost everywhere. Hence there
    exists a map $i$: $H_{m',p}(\Omega)\to H_{m,p}(\Omega)$ with $i(f)$
    = Limit in  
    $H_{m',p}(\Omega) $ of ${\{f_\nu}\}$. Further if $i (f) = 0$ in
    $H_{m',p} (\Omega)$, then $f^{o}=0$ in $L^{p}$. Now, for $g
    \epsilon C^{\infty, q,}_{o}$ we have  
    $$
    \int _{\Omega}(D^{\alpha}f_{\nu}(x),g(x)) dx = (-1)^{|\alpha|}
    \int _{\Omega}(f_{\nu}(x),D^{\alpha}g(x)) dx\, (| \alpha| \le m) 
    $$
    and by Holder's inequality,
    \begin{multline*}
      \int _{\Omega}(D^{\alpha}f (x),g(x)) dx =(-1)^{| \alpha |}\\ 
      \int (f^{o}(x), D^{\alpha}g(x)) dx =o  \text{ for any } g \in
      C^{\infty, q }_{o}. 
    \end{multline*}
    Hence  $D^{\alpha}f =0$, for $| \alpha |\le m$ i.e. the map $
    i:H_{m,p}(\Omega) \to H_{m',p}(\Omega)$ is an injection. Of course
    $i$ maps $ \overset{\circ}{H}_{m',p}$ into $\overset{\circ}{H}_{m',p}$. 
  \item If $f \epsilon H_{m,p}(\Omega)$ and $\varphi \epsilon C
    ^{\infty,1}_{o}$, then $\varphi f \epsilon H_{m,p}{\Omega}$ and
    $D^{\alpha}(\varphi f) = \sum _{\beta \le \alpha} \binom{\alpha}{\beta}$
    $D^{\beta}\varphi D^{\alpha-\beta}f $. 
  \end{enumerate}
\end{remark*}

\begin{proof}
  If ${\{f_\nu}\}$ is a sequence of $C^{\infty}$ functions converging
  to $f$ in $H_{m, p},\varphi f_{\nu}$ $\to \varphi f$ in $H_{m, p}$,
  i.e.\pageoriginale 
  $$
  D^\alpha (\varphi f _\nu) \to D^\alpha (\varphi f ) \text { in } L^p.
  $$
  
  Hence \qquad $D^\alpha (\varphi f) = \lim\limits_{\nu \to \infty} ~
  D^\alpha (\varphi f_\nu ) = \sum\limits_{\beta \le \alpha} ~
  (^{\alpha}_{\beta}) D^\beta \varphi D^{\alpha - \beta} f$. 
\end{proof}

\begin{enumerate} 
\item[(5)] If $f \in \overset{\circ}{H}_m (\Omega)$, there exists a
  sequence $\{f_\nu\}$ of $C^\infty$ functions with compact support
  $\subset \Omega$, such that $f_\nu \to f$ in $H_m (\Omega)$. If we
  extend $f_\nu$ to functions on $\mathbb{R}^n$ by setting $f_\nu (x)
  = 0$ for $x \notin \Omega$, then $\{f_\nu\}$ is convergent in $H_m
  (\mathbb{R}^n)$, to $f'$ say. We define $i' : \overset{\circ}{H}_m
  (\Omega) \to \overset{\circ}{H}_m (\mathbb{R}^n)$, by $i' (f) =
  f'$. Then $i'$ is injective and preserves norms. 
\item[(6)] If $\Omega$ is bounded, we have, for any $f \in C^{\infty,
  q}_o (\Omega)$, $f(x) = \int\limits^{x_1}_{-M} \dfrac{\partial
  f}{\partial x_1}$ $(t, x_2, \ldots , x_n)$ dt, for large $M$, so that 
  $$
  || f ||_{L^p} \le C(\Omega) || \frac{\partial f}{\partial x_1} ||_{L^p}.
  $$  
  It follows that for any $f \in \overset{\circ}{H}_{m, p} (\Omega)$, we have,
  $$
  | f |_{m, p} \le C_m (\Omega) \sum_{| \alpha | = m} | D^\alpha f |_{o, p}.
  $$
  (This is sometimes called Poincare's inequality.)
\end{enumerate}

\setcounter{lemma}{0}
\begin{lemma}\label{chap3:sec5:lem1} % lem 1
  Let $\varphi \ge 0$ be a $C^\infty$ function with $\supp \varphi
  \subset \{ x \big| | x | < 1 \}$ and $\int\limits \varphi dx =
  1$. Let $\varphi_ \varepsilon (x) = \varepsilon^{-n} \varphi
  (\dfrac{x}{\varepsilon})$. Then if $\Omega$ is an open set in
  $\mathbb{R}^n$ and, for $x \in \mathbb{R}^n$, we set $\varphi_
  \varepsilon * f(x)= \int_{\Omega} \varphi _ \varepsilon (x- y) f(y)
  dy$, then  
  \begin{enumerate}[(i)]
  \item  for any $f \in L^p ( \Omega)$, $\varphi_ \varepsilon * f \to
    f$ in $L^p (\Omega)$ and (ii) for any $f \in \overset{\circ}{H}_{m, p}
    (\Omega)$ 
  \end{enumerate}
  $$
  D^\alpha (\varphi_ \varepsilon *  f) = \varphi_ \varepsilon * D^\alpha f.
  $$
\end{lemma}

\begin{proof}
  If\pageoriginale we extend $f$ to $\mathbb{R}^n$ by setting $f(x) = 0$ for $x
  \notin \Omega$, we have  
  \begin{align*}
    (\varphi_\varepsilon * f - f) (x) & = \int \varphi_\varepsilon
    (x-y ) [f (y) - f(x) ] dy\\ 
    & = \int\limits_{| y | \le \varepsilon} \varphi_ \varepsilon (y) [
      f(y+x) - f(x) ] dy, 
  \end{align*}
  so that, by Holder's inequality, if $p'^{-1} = 1 -  p^{-1}$,
{\fontsize{10}{12}\selectfont
  \begin{align*}
    || \varphi_ \varepsilon * f - f ||_{L^p} & \le \big \{
    \int\limits_{| y | \le \varepsilon} [ \varphi_ \varepsilon (y)
    ]^{p'} dy \big \}^{\frac{1}{p'}} \big \{\int\limits_{| y | \le
      \varepsilon}  dy \int | f (x+y) - f(x) |^p dx \big
    \}^{\frac{1}{p}}\\ 
    & \le C. || \varphi ||_{L^{p'}} . \sup\limits_{| y | \le
      \varepsilon} \big\{\int | f(x+y) - f(x) |^p dx \big \}^{\frac{1}{p}}
    \to 0 \text { as}~  \varepsilon \to 0.  
  \end{align*}}
\end{proof}

[Note that $\big\{ \int\limits_{| y | \le \varepsilon} [\varphi_
    \varepsilon (y) ]^{p'} dy \big \}^{\frac{1}{p'}} = \varepsilon ^{-
    \frac{n}{p}}. || \varphi ||_{L^{p'}}$; that the last term tends to
  zero is trivial if $f$ is continuous with compact support and
  follows for general $f \in L^p$ since continuous functions with
  compact  supports are dense in $L^p$]. 

If $f \in \overset{\circ}{H}_{m, p}  (\Omega) $, let $\{f_\nu\}$ be a
sequence of $C^\infty$ functions with compact support converging to
$f$ in $_{m, p} (\Omega)$. Then  
\begin{align*}
  D^\alpha (\varphi_ \varepsilon * f) (x) & = \int_{\Omega}
  D^\alpha \varphi_ \varepsilon (x- y) f(y) dy\\ 
  & = \lim_{\nu \infty} ~ \int_{\mathbb{R}^n} D^\alpha \varphi_
  \varepsilon (x- y) f_\nu (y) dy\\ 
  & = \lim_{\nu \to \infty} \int\limits_{\mathbb{R}^n} \varphi_
  \varepsilon (x- y) D^\alpha f_\nu (y) dy\\ 
  & = \int\limits_{\Omega} \varphi_ \varepsilon (x- y) D^\alpha f(y) dy\\
  & = \varphi_ \varepsilon * D^\alpha f (x).
\end{align*}

\begin{remark*} % rem 
  This\pageoriginale proposition, when $p = 2$, follows immediately from
  Plancherel's theorem. In fact, if we extend $f$ to $\mathbb{R}^n$ by
  setting it $= 0$ outside $\Omega$, we have, 
  \begin{align*}
    (\varphi_ \varepsilon * f)^\wedge (\xi) &  = (2 \pi
    )^{\frac{n}{2}} \hat{\varphi}_ \varepsilon (\xi). \hat{f} (\xi) =
    (2 \pi)^{\frac{n}{2}} \hat{\varphi} (\varepsilon \xi ) \hat{f}
    (\xi)\\ 
    & \to (2 \pi)^{\frac{n}{2}} \hat{\varphi}(0) \hat{f}(\xi) = \hat{f}
    (\xi) \text { in } L^2, \text {as } \varepsilon \to 0.  
  \end{align*}
\end{remark*}

\setcounter{proposition}{0}
\begin{proposition}\label{chap3:sec5:prop1} % prop 1
  If $f \in H_{m, p}(\Omega)$ and if the $D^\alpha f$, for $| \alpha |
  \le m$, are strongly differentiable upto order $m' $ in $L^p$, then
  $f$ is strongly differentiable upto order $m + m '$ in $L^p$. 
\end{proposition}

\begin{proof}
  It is enough to prove the proposition for a function $f$ with
  compact support $\subset \Omega',  \Omega'$ being a relatively
  compact open subset of $\Omega$. If $\varphi_ \varepsilon$ is
  defined as in the lemma above, then $\varphi_ \varepsilon * f(x) =
  \int_\Omega \varphi_ \varepsilon (x- y) f(y) dy$ is a $C^\infty$
  function of $x$ and for $| \alpha | \le m$, we have by (ii) in the
  lemma above, 
  $$
  D^\alpha (\varphi_ \varepsilon * f) = \varphi_ \varepsilon * D^\alpha f.
  $$
\end{proof}

Again since $D^\alpha f \in \overset{\circ}{H}_{m', p}  (\Omega)$, we have,
for $| \alpha | \le m, | \beta | \le m'$ 
$$
D^{\alpha + \beta} (\varphi_ \varepsilon * f) = D^\beta [D^\alpha
  (\varphi_ \varepsilon * f) ] = D^\beta (\varphi_ \varepsilon *
D^\alpha f) = \varphi_ \varepsilon * D^\beta (D^\alpha f) 
$$
(the last two equations hold because of Lemma \ref{chap3:sec5:lem1}).

If $f_\nu (x) = \varphi_{1/ \nu}* f(x)$, then by Lemma \ref{chap3:sec5:lem1} (i),  $f_\nu
\to f$ in $H_{m + m', p} (\Omega')$ and hence the proposition is
proved. 

\begin{proposition}\label{chap3:sec5:prop2}% prop 2
  Let\pageoriginale $\Omega$ be a bounded open set in $\mathbb{R}^n$. If $\varphi_
  \varepsilon$ is defined as in the lemma above, then  for any $f \in
  \overset{\circ}{H}_{m, p}  (\Omega)$, we have  
  $$
  | \varphi_ \varepsilon * f - f |_{m - 1, p} \le A \varepsilon ||
  \varphi ||_{L^{p'}} | f | m, p,  \text {where} \frac{1}{p}+
  \frac{1}{p'} = 1 
  $$
  and $A$ is a constant depending on $\Omega$.
\end{proposition}

\begin{proof}
  We shall first prove that if $\Omega'$ is a bounded open set with $\Omega
  \Subset \Omega'$, and $\varepsilon$ is small enough, then for $f \in
  \overset{\circ}{H}_{m, p}  (\Omega)$, 
  $$
  | \varphi_ \varepsilon * f - f |^{\Omega'}_{0, p'} \le A \varepsilon
  || \varphi ||_{L^{p'}} | f |_{1, p}. 
  $$
\end{proof}

Since $C^{\infty, q}_0 (\Omega)$ is dense in $\overset{\circ}{H}_{1, p}
(\Omega) $, it is enough to prove this inequality for $f \in C^{\infty
  , q}_0 (\Omega)$. We have  
$$
\displaylines{\hfill 
  f(x+y) - f(x) = \sum_{i = 1}^n y_i \int^1_0 \frac{\partial f}{\partial
    x_i} (x+ ty) ~  dt \hfill \cr
  \text{so that}\hfill  
  | f(x+y) - f(x) |^p \le n^p \sum_{i = 1}^n | y_i |^p \int\limits_0^1
  \frac{\partial f}{\partial x_i} (x+ ty ) \big |^p dt. \hfill }
$$
Hence, if $g_y (x) = f(x+y ) - f(x)$, we have
\begin{align*}
  || g_y ||^p_{L^p} & \le n^p \sum_{i = 1}^n | y_i |^p \int\limits_0^1
  dt \int\limits_{\mathbb{R}^n} \big | \frac{\partial f}{\partial x_i}
  (x+ty) \big |^p dx\\ 
  & \le \left(n^p \sum_1^n | y_i |^p \right) | f |^p_{1, p},
\end{align*} 
so that 
\begin{equation}
  || g_y ||^p_{L^p} \le n^{p +1} \varepsilon^p | f|^p_{1, p} \text {
    if } | y | \le \varepsilon. \tag{5.1}\label{chap3:sec5:eq5.1} 
\end{equation} 
Now\pageoriginale 
\begin{align*}
 \varphi_ \varepsilon * f(x) - f(x) & =
  \int\limits_{\mathbb{R}^n} \varphi_ \varepsilon (x- y) [ f(y) - f(x)
  ] dy\\ 
  &= \int\limits_{\mathbb{R}^n} \varphi_ \varepsilon (y) [ f(x+y) - f(x) ] dy. 
\end{align*}
 
Since $\supp. \varphi_ \varepsilon$ \quad $\subset \big \{x \big | |x
| < \varepsilon \big \}$, this gives 
$$
\varphi_ \varepsilon * f(x) - f(x) = ~ \int\limits_{| y | <
  \varepsilon} \varphi_ \varepsilon (y) [ f(x+y) - f(x) ] dy. 
 $$
 
If $p > 1$, we use Holder's inequality and obtain
$$
\big | \varphi_ \varepsilon * f(x) - f(x) \big | \le \left( \int | \varphi_
\varepsilon (y) |^{p'} dy\right)^{\frac{1}{p'}} \left(\int\limits_{| y | <
  \varepsilon} | f (x+y) - f(x) |^p dy\right)^{\frac{1}{p}}. 
$$
Since, as is easily verified,
$$
\displaylines{\hfill 
  \left(\int | \varphi_ \varepsilon (y) |^{p'} dy\right)^{\frac{1}{p'}} =
  \varepsilon ^{- \frac{n}{p}} || \varphi ||_{L^{p'}}  \hfill \cr
  \text{this gives} \hfill 
  | \varphi_ \varepsilon * f(x) - f(x) |^p \le \varepsilon^{-n} ||
  \varphi ||^p_{L^{p'}} \int | f (x+y) - f(x) |^p dy.} 
 $$
 This inequality clearly holds also if $p = 1$, if we replace $||
 \varphi ||_{L^{p'}}$ by $v$. $|| \varphi ||_{L^{\infty}} =
 v$. $\sup\limits_{x} | \varphi (x) |$, where $v = \int\limits_{| y |
   < 1} dy$. Hence  
\begin{align*}
   \int\limits_{\Omega'} \big | \varphi_ \varepsilon * f(x) & - f(x)
   \big |^p dx \le v. \varepsilon^{-n} || \varphi ||^p_{L^{p'}}
   \int\limits_{\Omega'} dx \int\limits_{| y | < \varepsilon} | f
   (x+y) - f(x) |^p dy\\ 
   &= v. \varepsilon^{-n} || \varphi ||^p _{L^{p'}} \int\limits_{| y |
     < \varepsilon} || g_y ||^p_{L^p} dy\\ 
   & \le A^p \varepsilon^p || \varphi ||^p_{L^{p'}} | f |^p_{1, p}
   \text { (because of (\ref{chap3:sec5:eq5.1}))}\\ 
   & \left [ A^p = v^2 . n^{p+1} \right ].
\end{align*}  
  
Hence\pageoriginale $| \varphi_ \varepsilon * f - f|_{0, p} \le A \varepsilon ||
\varphi ||_{L^{p'}} ~ . |f | _{1, p}$ for some constant $A$. 

Now, for $| \alpha | \le m - 1, D^\alpha f \in  \overset{\circ}{H}_{1, p}$
and  
$$
| \varphi_ \varepsilon * D^\alpha f - D^\alpha f |^{\Omega'}_{0, p}
\le A \varepsilon ~ . || \varphi ||_{L^{p'}} | D^\alpha f |_{1, p}, 
$$
which prove the proposition.

\begin{lemma}\label{chap3:sec5:lem2} % lem 2
  Let $\Omega$ be a bounded open set in $\mathbb{R}^n $ and $k$, a
  continuous function with compact support. Then for $f \epsilon
  L^p (\Omega)$, the function  
  $$
  (Kf) (x) = \int_\Omega k(x-y) f(y) dy \in L^p (\mathbb{R}^n )
  $$
  and the operator $K : L^p (\Omega) \to L^p (\mathbb{R}^n)$ is
  completely continuous. 
\end{lemma}

\begin{proof}
  The first part is obvious since $K f$ is clearly continuous and with
  compact support, with support $\subset \{a +b | a \in \Omega$, $b
  \in \supp. k  \}$, which  is relatively compact. Further, by
  Holder's inequality, $Kf$ is uniformly bounded on the set $|| f
  ||_{L^p} \le 1$. By Ascoli's theorem, it suffices to prove that the
  family $Kf$, $|| f ||_{L^p} \le 1$, is equicontinuous. If $\eta
  (\varepsilon) = \sup\limits_{| a - b| \le \varepsilon} | k(a) - k(b)
  |$, we have  
  $$
  | (Kf ) (x) - (Kf ) (x') | \le \eta (|x - x'|) || f ||_{L^1} \le A_p
  \eta (| x - x'|) || f ||_{L^p} 
  $$ 
  (since\pageoriginale $\Omega$ is bounded), which proves the lemma.
\end{proof}

\setcounter{theorem}{0}
\begin{theorem}[Rellich]\label{chap3:sec5:thm1} %the 1
  Let $\Omega$ be a bounded open set in
  $\mathbb{R}^n$ and $0 \le m' < m$. Then the natural map $i :
  \overset{\circ}{H}_{m, p} (\Omega) \to \overset{\circ}{H}_{m' , p}
  (\Omega)$ is completely continuous.  
\end{theorem}

\begin{proof}
  Let $\Omega'$ be a bounded open set, $\bar{\Omega} \subset
  \Omega'$. We have only to prove that the natural map $j :
  \overset{\circ}{H}_{m' , p} (\Omega) \to \overset{\circ}{H}_{m' , p}
  (\Omega') $ composite of $i$ and the isometry $\overset{\circ}{H}_{m' ,
    p} (\Omega) \to \overset{\circ}{H}_{m' , p} (\Omega')$ is completely
  continuous. 
\end{proof}

For any operator $T$ between these two spaces, we set 
$$
|| T|| =\sup\limits_{f \neq 0}\cdot  \dfrac{| Tf |_{m' , p}}{| f|_{m,
    p}}.
$$ 

Let $T_ \varepsilon$ be the operator $T_ \varepsilon (f) (x) =
\varphi_ \varepsilon * f(x)$. If $\varepsilon$ is sufficiently small,
$T_ \varepsilon (f) \in \overset{\circ}{H}_{m' , p} (\Omega')$, and
because of Prop.~2, we have  
$$
|| T_ \varepsilon - j || \to 0 \quad \text { as } \varepsilon \to 0.
$$
Since the uniform limit of completely continuous operators is
completely continuous, the theorem follows at once from
Lemma \ref{chap3:sec5:lem2}. 

\begin{proposition}\label{chap3:sec5:prop3} % prop 3
  There exist positive constants $C_1$ and $C_2$ such that for any $f
  \in \overset{\circ}{H}_{m} (\mathbb{R}^n)$, 
  $$
  C_1 \int (1+ | \xi |^2)^m | \hat{f} (\xi ) |^2 d \xi \le | f |^2_m
  \le C_2 \int ( 1+ | \xi |^2)^m | \hat {f} (\xi) |^2 d \xi 
  $$
  $(| \hat{f} |$ denotes the norm in $\mathbb{C}^q )$.
\end{proposition}

\begin{proof}
  Since\pageoriginale functions with compact support are dense in
  $H_{m}^0(\mathbb{R}^n)$, it is enough to prove the proposition for
  $f$ with compact support. Now, 
  \begin{align*}
    | f |^2_m & = \sum_{| \alpha | \le m} \sum_{i \le q} \int
    |D^\alpha f_i (x) |^2 dx\\ 
    & = \sum_{| \alpha | \le m} \sum_{i \le q} | D^{\hat{\alpha}} f_i
    |^2_0, \text { by Plancherel's theorem, since } D^\alpha f \in
    |L^2. 
  \end{align*}
\end{proof}

Hence $|f|^2_m =\sum\limits_{| \alpha | \le m} \sum\limits_{i \le q} |
\xi^\alpha |^2 | \hat{f}_i (\xi) |^2 d \xi $. 

Now there exist constants $C_1$ and $C_2$ such that 
$$
C_1 (1+ | \xi |^2 )^m \le \sum_{| \alpha | \le m} |\xi^\alpha |^2 \le
C_2 (1+ | \xi |^2)^m \text {  for  } \xi \in \mathbb{R}^n  
$$
and hence the proposition.

\begin{remark*} % rem
  When $ p = 2$, Theorem \ref{chap3:sec5:thm1} can be proved very simply, using
  Plancherel's theorem. In fact, if $f \in \overset{\circ}{H}_m (\Omega) $
  and $| f|_m \le 1$, clearly $\hat{f}$ is bounded and so is
  $\dfrac{\partial \hat{f}}{\partial \xi_k} = (2 \pi)^{-n/2} \int i
  x_k e^{i \xi x } f(x) dx$ since $\Omega$ is bounded, so that the
  $\{\hat{f}\}$ form an equicontinuous family. Hence given a sequence
  $\{f_\nu \}, | f_\nu | _m \le 1$, we may select a subsequence
  $\{f_{\nu _k}\}$ such that $\{\hat{f} _{v_k}\}$ converges uniformly
  on compact sets of $\mathbb{R}^n$. Now, 
  $$
  | f_{\nu_p} - f_{\nu_q} |^2_{m - 1} \le C_2 \int\limits_{\mathbb{R}} (1+ |
  \xi |^2 )^{m-1} | \hat {f}_{v_p} - \hat {f}_{v_q} |^2 d \xi. \text {
    Given } \varepsilon > 0, 
  $$ 
  we may choose $A$ that $1+|\xi |^2 > \dfrac{1}{\varepsilon}$ for
  $| \xi | > A$, so that  
  $$
  \int\limits_{ | \xi | > A} (1+ | \xi |^2 )^{m -1} | \hat{f}_{\nu_p}
  - \hat{f}_{v_q} |^2 d \xi \le C_3 \varepsilon | f_{\nu_p} - f_{\nu_q}
  |^2_m < 2 C_3 \varepsilon. 
  $$
  while,\pageoriginale if $p, q$ are large, $\int\limits_{| \xi | \le A} (1+ | \xi
  |^2 )^{m-1} | \hat{f}_{v_p} - \hat{f}_{v_q} |^2 d \xi < \varepsilon$
  since $\{\hat{f}_{v_p}\}$ converges uniformly on compact sets. This
  shows that $\{f_{\nu_p}\}$ converges in $H_{m - 1, p}$. 
\end{remark*}

\begin{proposition}\label{chap3:sec5:prop4}% prop 4
  We have 
  $$
  H_{m}^0 (\mathbb{R}^n) = H_m (\mathbb{R}^n) = \left\{f | f \in
  L^2, \int\limits_{\mathbb{R}^n} (1+ | \xi |^2 )^m | \hat{f} (\xi )
  |^2 d \xi < \infty \right\}.
  $$ 
\end{proposition}

\begin{proof}
  Let $f \in H_m (\mathbb{R}^n)$ and $\varphi, a $ be $C^\infty$
  function with compact support, $\varphi : \mathbb{R}^n \to
  \mathbb{R}$ such that $\varphi (x) = 1$ for $| x | \le 1$ and $0 \le
  \varphi (x) \le 1$. 
\end{proof}

Let $\varphi_\nu (x) = \varphi \left(\dfrac{x}{\nu}\right)$. Then $\varphi_\nu
(x) \to 1$ and each $\varphi_\nu$ has compact support. By remark (3)
above, 
$$
D^\alpha \varphi_\nu f = \sum_{\beta \le \alpha} \binom{\alpha}{\beta}
D^\beta \varphi_\nu D^{\alpha - \beta} f.  
$$

Since $D^\beta \varphi_\nu$ are bounded and tend to zero (for $|
\beta | \ge 1)$ everywhere, it follows from Lebesgue's theorem on
bounded convergence that $D^\beta \varphi_\nu . D^{\alpha - \beta } f
\to 0$ in $L^2$ for $| \beta | \ge 1$, so that  
$$
D^\alpha (\varphi_\nu f) \to D^\alpha f \text { in } L^2, \text { for
} | \alpha | \le m. 
$$

Hence $\varphi_\nu f \to f$ in $H_m (\mathbb{R}^n)$ and since
$\{\varphi_\nu\}$ is a sequence of functions with compact support, it
follows that $f \in \overset{\circ}{H}_m (\mathbb{R}^n)$. This proves that
$H_m (\mathbb{R}^n) = \overset{\circ}{H}_m (\mathbb{R}^n)$. It is clear
from Proposition \ref{chap3:sec5:prop3}  that  
$$
\overset{\circ}{H}_m (\mathbb{R}^n) \subset \left\{f \big | f \in L^2,
\int\limits_{\mathbb{R}^n} (1+ | \xi |^2 )^m | \hat{f} (\xi ) |^2 d
\xi < \infty \right\}.  
$$

Conversely,\pageoriginale if $f \in L^2 $ and $\int\limits_{\mathbb{R}^n}$ $(1+ |
\xi |^2)^m | \hat{f} (\xi) |^2 d \xi < \infty$, $(1+|\xi|^2)^{\frac{m}{2}}
\hat{f} (\xi ) \in L^2$. Hence there exists a sequence $\hat{g}_\nu$
in $\mathscr{S}$ such that $\hat{g}_\nu (\xi) \to (1+ | \xi
|^2)^{\frac{m}{2}} \hat{f} (\xi)$ in $L^2$. Let $h_\nu \in
\mathscr{S}$ be such that its Fourier transform $\hat{h}_\nu =
\hat{g}_\nu \big / (1+ | \xi |^2)^{m/2}$ [which exists by the
  inversion theorem]. Then $h_\nu \in H_m$ and
$\int\limits_{\mathbb{R}^n} (1+ | \xi |^2 )^m | \hat {h}_\nu (\xi )-
\hat{h}_\mu (\xi) |^2 d \xi \to 0$ as $\mu, \nu \to \infty$, i.e., by
Proposition \ref{chap3:sec5:prop3}, $h_\nu$ is convergent in $H_m$. It
is clear that 
$h_\nu \to f$ in $L^2$. Hence $f \in H_m (\mathbb{R}^n)$, which proves
the proposition.  

\begin{lemma}\label{chap3:sec5:lem3} % lem 3
  Let $\eta$ be the map $\eta : \mathbb{R}^+ \times S^{n-1} \to
  \mathbb{R}^n - \{0\}$, given by $\eta (t, x) = tx = y$. Then there
  exists an $(n - 1)$ form $\omega $ on $S^{n-1}$ such that $\eta ^*
  (dy_1 \wedge \cdots \wedge dy_n) = t^{n-1} dt \wedge \omega$. [A
    point in $\mathbb{R}- \{0\}$ is denoted by $y = (y_1, \ldots ,
    y_n)$.] 
\end{lemma}

\begin{proof}
  In fact if $x_1, \ldots , x_n$ are the restrictions to $S^{n-1}$ of
  the coordinate functions in $\mathbb{R}^n$, we may take $\omega =
  \sum\limits_{k = 1}^n x_k dx_1 \wedge \cdots \wedge d\hat{x}_k
  \wedge \cdots \wedge dx_n$. (The hat over a term means that the term
  is omitted.) 
\end{proof}

\begin{remark*} % rem
  Since $\int\limits_U dy _1 \wedge \cdots \wedge dy_n$, over any
  non-empty open set  $U \subset \mathbb{R}^n - 0$ is positive, we
  have $\int\limits_{S^{n-1}} \omega \neq 0$. 
\end{remark*}

\begin{theorem}[Sobolev's lemma]\label{chap3:sec5:thm2}% theo 2
  Let $\Omega$ be an open set in $\mathbb{R}^n$ and $m >
  \dfrac{n}{p}$. Then for any compact set $K \subset \Omega$,there
  exists a constant $C_K$ such that for any $C^\infty$ function $f :
  \Omega \to \mathbb{C}^q$ with supp: $f \subset K$, we have 
  $$
  \sup\limits_{x \in K} | f(x) | \le C_{K, m} | f |_{m, p}.
  $$
\end{theorem}

\begin{proof}
  We\pageoriginale may suppose that $\Omega = \mathbb{R}^n$. Further, we can choose
  a compact set $K'$ such that for any $x \in K, g(y) = f(x+y)$ is a
  $C^\infty$ function with $\supp. g \subset K'$. Hence it is enough to
  prove that there exists a constant $C$ such that for $C^\infty$ $f$
  with supp. $f \subset K$, we have, 
  $$
  | f(0) | \le C | f |_{m, p}.
  $$ 
\end{proof}

Let $\eta : \mathbb{R}^+ \times S^{n-1} \to \mathbb{R}^n - \{0\}$ be the map
as defined in the lemma above. 

Let $f(y) = g(tx)$, where $y = tx $ for $ y \neq 0$, $t \in
\mathbb{R}^+, x \in S^{n-1} $ and $g(0) = f(0)$. 

Then $f_i (0) = C_1 \int\limits_{0}^M \dfrac{\partial g_i
  (tx)}{\partial t^m} t^{m-1} dt $ for some constants $M = M_K$ and
$C_1 = C_1 (m)$. 

Multiplying by $\omega$ and integrating over $S^{n-1}$, we have 

\begin{align*}
  f_i (0) \int\limits_{S^{n-1}} \omega &= C_1 \int\limits_{S^{n-1}}
  \int\limits_{0}^M \frac{\partial^m g_i (tx)}{\partial t^m} t^{m-1}
  dt \wedge \omega\\ 
  & = C_1 \int\limits_{S^{n-1}} \int\limits_0^M t^{m-n}
  \frac{\partial^m g(tx)}{\partial t^m} t^{n-1} dt \wedge \omega. 
\end{align*} 
 
Since $\int\limits_{S^{n-1}} \omega \neq 0$, this gives 
\begin{equation}
  f_i (0) = C_2 \int\limits_{| y | < M} t^{m-n} \frac{\partial^m g_i
    (tx)}{\partial t^m} dy \tag{5.2}\label{chap3:sec5:eq5.2} 
\end{equation}
for some constant $C_2$ and $t = | y |$.

Now\pageoriginale for $p > 1$, using Holder's inequality,
$$
| f_i (0) | \le C_2 \left(\int\limits_{| y | < M} t^{(m-n)p'}
dy\right)^{\frac{1}{p'}} \left( \int\limits_{| y | < M} \big | \frac{\partial^m
  g_i (tx)}{\partial t^m} \big |^p dy\right)^{\frac{1}{p}} 
$$
where $\dfrac{1}{p'} + \dfrac{1}{p} = 1$. Hence 
$$
| f_i (0) | \le C_2. \left( \int\limits_{S^{n-1}} \int\limits_0^M t^{(m-n)
  p'} t^{n-1} dt \wedge \omega\right)^{\frac{1}{p'}} \left( \int\limits_{|  y | <
  M} \big | \frac{\partial^m g_i (tx)}{\partial t^m} \big |^p
dy\right)^{\frac{1}{p}}. 
$$

Since $m > \dfrac{n}{p}$, we have $(m- n) p' + n- 1 > -1$ and hence
$\int\limits_{S^{n-1}} \int\limits_0^M$ $t^{(m - n)p'} t^{n-1} dt \wedge
\omega < \infty$. Now 
$$
 \frac{\partial^m g_i (tx)}{\partial t^m} =\sum_{| \alpha | \le m}
 q_\alpha (y) D^\alpha f_i (y), 
$$
where $q_\alpha (y)$ are bounded functions of $y$ and hence there
exists a constant $C_3$ such that 
$$
\int\limits_{| y | < M} \big | \frac{\partial^m g_i (tx)}{\partial
  t^m} \big |^p dy \leq C_3 (|f |_{m, p})^p. 
$$

Hence $| f (0)| \le C_4 | f|_{m, p}$, for some constant $C_4$ depending
on $K$. This proves the theorem for $ p > 1$. 

If $p = 1, m \ge n$, it follows immediately from that
(\ref{chap3:sec5:eq5.2}) $| f(0) | \le C_K | f|_{m, 1}$ for a constant $C_K$ 

\setcounter{corollary}{0}
\begin{corollary}\label{chap3:sec5:coro1} % coro 1
  If $\Omega$ is an open set in $\mathbb{R}^n$, and $K$ is a compact
  subset of $\Omega$, then, for any $f \in C^{\infty, q}(\Omega)$, we
  have 
  $$
  \sup_{x \in K} | f (x) | \le C_{K, \Omega, m} |f |_{m, p} \text { for } m > n/p.
  $$
\end{corollary}

\begin{proof}
  Apply\pageoriginale Theorem \ref{chap3:sec5:thm2} to $\eta f$, where
  $\eta$ is a fixed function with 
  compact support in $\Omega$, which is $= 1$ on $K$. 
\end{proof}

\begin{corollary}\label{chap3:sec5:coro2} % coro 2
  If $\Omega$ is an open subset of $\mathbb{R}^n$ and $m >
  \dfrac{n}{p}$, then if $f \in H_{m, p}(\Omega)$, there exists a
  function $g \in H_{m, p} (\Omega)$ such that $f = g$ almost
  everywhere and $g$ has continuous derivatives of all orders  
  $$
  \le ~ m - \left[\frac{n}{p}\right ] - 1.
  $$
\end{corollary}

\begin{proof}
  By multiplying $f$ by a suitable function, we may suppose that $f
  \in \overset{\circ}{H}_{m, p} (\Omega)$; moreover, we may suppose, that
  $\Omega$ is bounded. Let $\{f_\nu\}$ be a sequence of $C^\infty$
  functions, with compact support in $\Omega$, converging to $f$ in
  $H_{m, p}(\Omega)$. Clearly $D^\alpha (f_\nu - f_\mu ) \in H_{m - |
    \alpha | , p}$, for $0 \le | \alpha | \le m$, and if $m - | \alpha
  | > \dfrac{n}{p}$, and $K \subset \Omega$ is compact, we have, by
  Sobolev's  lemma,
  $$
  \supp_{x \in K} | D^\alpha f_\nu (x) - D^\alpha f_\mu (x) | \le C_K
  | f_\nu - f_\mu |{m, p}. 
  $$
\end{proof}

Hence $D^\alpha f_\nu$ is uniformly convergent on $K$, for $|\alpha | <
m - \dfrac{n}{p}$; if $g = \lim f_\nu$, this implies that $g$ has
continuous derivatives upto order $\le m -\left[\dfrac{n}{p} \right ]
- 1$. 

\begin{remark*} % rem
  The proof of Sobolev's lemma, for $p = 1$ or $2$,  simplifies as
  follows If $p = 1, f_i (x) = \int\limits_{-M}^x \cdots
  \int\limits_{-M}^{x_n} \dfrac{\partial^n f_i (t_1, \ldots ,
    t_n)}{\partial x_1 \cdots \partial x_n} dt_1 \cdots dt_n$ for a
  constant $M$ depending on $K$. 
\end{remark*}

Hence\pageoriginale $| f(x) | \le A | f|_{n, 1} \le A | f |_{m, 1} $, for $m \ge n$
and a constant $A$. 

Further, by Holder's inequality applied to this formula, we get
$$
| f(x) | \le C_{K, p} | f |_{m, p} \text { if } m \ge n, \text { and } p \ge 1.
$$

Thus, the statement that any $f \in H_{m, p}$ has continuous
derivatives of order $\le m - n$ is trivial. If $ p = 2$, by the
remark following the inversion formula in \S\ 4, 
\begin{align*}
  f_i (x) & = \frac{1}{(2 \pi )^{n / 2}} \int e^{ix \xi} \hat{f}_i (\xi) d \xi\\
  & = \frac{1}{(2 \pi )^{n / 2}} \int e^{ix \xi} (1+ | \xi |^2
  )^{\dfrac{m}{2}} \frac{\hat{f}_i (\xi)}{(1+ | \xi |^2 )^{m/2} } d
  \xi , 
\end{align*}
and by Schwarz's inequality
$$
| f_i (x) | \le A \left( \int (1+ | \xi |^2 )^{-m} d \xi
\right)^{\dfrac{1}{2}} \left (
\int | \hat{f}_i (\xi) |^2 (1+ | \xi |^2)^m d \xi\right)^{\dfrac{1}{2}}  
$$
for some constant $A$.

Now, \quad $\int\limits_{\mathbb{R}^n} (1+ | \xi |^2 )^{-m} d \xi <
\infty$ if $m > \dfrac{n}{2}$. 

Hence it follows from Prop. \ref{chap3:sec5:prop2}, that for $m > \dfrac{n}{2}, | f(x) |
\le B | f |_{m, 2}$, for some constant $B$. This latter proof applies
to a such larger class of functions than functions with support in a
fixed compact set. 

Rellich's lemma remains true if we replace $\overset{\circ}{H}_{m, p}
(\Omega)$ by $H_{m, p} (\Omega)$ if the boundary of $\Omega$ is
sufficiently smooth (see Rellich \cite{37}). 

Several proofs of Sobolev's lemma have been given; Sobolev \cite{42}
obtained several very precise inequalities. However most of these
proofs are more complicated than the one given here. 

\section[Elliptic differential operators:...]{Elliptic differential operators: the inequalities of
  G$\ring{\text{a}}$rding and Friedrichs}\label{chap3:sec6} 

In\pageoriginale what follows, $\Omega$ is an open set in $\mathbb{R}^n$ and $L$ is
a linear differential operator, $L : C^{\infty, q}_0 (\Omega) \to
C^{\infty, p}_0 (\Omega)$. 

\begin{defi*} % def 
  \begin{enumerate} [ (1)]
  \item If $L$ can be written as $Lf = \sum\limits_{| \alpha | \le m}
    a_\alpha . D^\alpha f$, with continuous mappings $a_\alpha$ of
    $\Omega$ into the space of $p \times q$ complex matrices, and if
    there exists $\alpha$ such that $| \alpha | = m$ and $a_\alpha
    \nequiv 0$ on $\Omega$, then $L$ is said to have order $m$ on
    $\Omega$. 
  \item If $L$ is a differential operator of order $m$ on $\Omega$,
    for $\xi \in \mathbb{R}^n$, the characteristic polynomial of $L$ is
    defined by $p (x, \xi) = \sum\limits_{| \alpha | =m} \xi^\alpha
    a_\alpha (x)$; it is a mapping of $\Omega \times \mathbb{R}^n$
    into the space of $p \times q $ matrices. 
  \item If $p (x, \xi)$ is the characteristic polynomial of $L$ and if
    for any $\xi \in \mathbb{R}^n$, $\xi \neq 0$ and $x \in \Omega$,
    the  map $p(x, \xi) : \mathbb{C}^q \to \mathbb{C}^p$ is injective,
    then $L$ is said to be elliptic. 
  \item If $p = q$ and $p(x, \xi)$ is the characteristic polynomial of
    $L$ and if for any $\xi \neq 0, \xi \in \mathbb{R}^n, x \in \Omega$
    and $v \in \mathbb{C}^q, v \neq 0$, we have $Re (p (x, \xi ) v, v)
    \neq 0$, then $L$ is said to be strongly elliptic. 
  \item If $p = q$, $L$ is of order $m$, and $p (x, \xi)$ is its
    characteristic polynomial, and if there exists a constant $c> 0$
    such that for any $\xi \in \mathbb{R}^n, x \in \Omega$ and $v \in
    \mathbb{C}^q, Re (p (x, \xi) v, v) \ge c | \xi |^m | v |^2$, then
    $L$ is said to be uniformly strongly elliptic. 
  \end{enumerate}
\end{defi*}

If $n > 1$, then a strongly elliptic operator (or its negative) is
uniformly strongly elliptic an any connected subset $\Omega' \subset
\subset \Omega$. 

In\pageoriginale fact, since $S^{n-1}$ is connected, Re\,$(p(x, \xi) v, v)$ has
constant sing on $\Omega' \times S^{n-1}$. 

Further, if $n > 1$, then any strongly elliptic operator is of even
order. In fact, for fixed $x$ and $v \neq 0, Q (\xi) = Re (p (x, \xi)
v, v)$ is a homogeneous polynomial of degree $m = $ order $L$. It is
clear that for almost all values of $a, b \in \mathbb{R}^n$, the
polynomial $Q(a+ \lambda b)$ of the real variable $\lambda$ has degree
$m$, hence has a real zero if $m$ is odd. If $n > 1$, we may choose
$a, b$ such that $a+ \lambda b \neq 0$ for all real $\lambda$ and $Q$
would then have a real, non-trivial root. 

Let $L_1$ and $L_2$ be differential operators, $L_1 : C^{\infty, q}_0
(\Omega) \to C^{0, p} (\Omega) $ and $L_2 : C^{\infty, p}_0 (\Omega)
\to C^{0, r} (\Omega)$, then if $L_1$ can be written as $L_1 f =
\sum\limits_{| \alpha | \le m} a_\alpha D^\alpha f, a_\alpha$ being
$C^\infty$ functions with values in $p \times q$ matrices, then we
define $L_2 \circ L_1 : C^{\infty, q}_0 (\Omega) \to C^{0, r} (\Omega)$ by 
$$
(L_2  \circ L_1) (f) (x) = (L_2 (L_1 f) ) (x).
$$
We also write $L_2. L_1 $ for $L_2 \circ L_1$.

Let $L_2$ be given by
$$
(L_2 f) (x) = \sum_{| \beta | \le m'} b_\beta (x) D^\beta f(x), \text
{for } f \in C^{\infty, p} (\Omega). 
$$ 

Then $L_2 \circ L_1$ is given by
$$
\displaylines{\hfill 
  (L_2 \circ L_1) (f)(x) = \sum_{| \gamma | \le m + m'} c_\gamma (x)
  D^\gamma f(x)\hfill \cr
  \text{where}\hfill  
  c_\gamma (x) = \sum_{\alpha + \beta = \gamma} b _\beta (x) a_\alpha
  (x) \text{ for } | \gamma | \le m + m'.\hfill } 
$$

Hence\pageoriginale $L_2 \circ L_1$ has order $\le m + m'$ and if $p_1 (x, \xi), p_2 (x,
\xi)$ are the characteristic polynomials of $L_1$ and $L_2$
respectively, the characteristic polynomial $p(x, \xi)$ of $L_2 \circ L_1$
is given by 
$$
p(x, \xi)  = p_2 (x, \xi) . p_1 (x, \xi), \text{ unless } p_2 (x, \xi)
. p_1 (x, \xi) = 0 \text { for all } x \text { and } \xi. 
$$

If $L_1$ and $L_2$ are elliptic differential operators and $L_2 \circ
L_1$ is defined as above, then $L_2 \circ  L_1$ is elliptic. This obvious
since if $p_1 (x, \xi), p_2 (x, \xi)$ are injective, $p(x, \xi)$ is
injective. 

Let $L$ be a differential operator of order $m, L: C^{\infty, q}_0 \to
C^{o, p}$ and  $Lf= \sum\limits_{| \alpha | \le m} a_\alpha D^\alpha f$,
where $a_\alpha$ are $C^\infty$ functions on $\Omega$. Then we define
the (formal) adjoint operator $L^* : C^{\infty, p}_0 \to C^{0, q}$ by 
$$
(Lf, \varphi) = (f, L^* \varphi) \text { for any } f \in C^{\infty,
  q}_0 (\Omega) \text {and } \varphi \in C^{\infty, p}_0 (\Omega).  
$$ 

We shall show that the operator $L^*$ exists and is unique.

If for $\varphi_1, \varphi_2 \in C^{\infty, q}_0 (\Omega), (\varphi_1,
f) = (\varphi_2, f) $ for every $f \in C^{\infty, q}_0 (\Omega)$, then
clearly $\varphi_1 = \varphi_2$. Hence $L^* \varphi$, if it exists, is
unique. 

Since $\varphi $ and $f$ are $C^\infty$ functions with compact supports,
\begin{align*}
  (Lf, \varphi) & = \sum_{\alpha} \sum_{i = 1}^p \sum_{j = 1}^q
  a^{ij}_\alpha (x) D^\alpha f_j (x) \overline{\varphi_i (x)} dx\\ 
  & = \sum_{\alpha} (-1)^{ | \alpha |} \sum_{i = 1}^p \sum_{j = 1}^q
  \int f_j (x) \overline{D^\alpha (a^{\overline{ij}}_\alpha
    (x). \varphi_i (x))} dx\\ 
  & = \sum_\alpha (-1)^{| \alpha |} (f(x), D^\alpha (
  \overline{t_{a_\alpha} (x)}. \varphi (x))), 
\end{align*}
where\pageoriginale $t_a$ is the transpose of the matrix $a$. Hence if we define
$L^* \varphi $ by 
$$
\displaylines{\hfill 
  L^* \varphi = \sum_{\alpha} (-1)^{| \alpha |} D^\alpha
  [t_{\overline{a_\alpha}}\cdot \varphi],\hfill \cr
  \text{we have,} \hfill 
  (f, L^* \varphi) = (Lf, \varphi), \text {for } f \in C^{\infty, q}_0
  (\Omega) \text{ and } \varphi \in C^{\infty, p}_0 (\Omega).\hfill } 
$$

This prove the existence and uniqueness of the adjoint operator $L^*$:
$C^{\infty, p}_0 (\Omega) \to C^{\infty, q}_0 (\Omega)$. Moreover order
of $L^* = $ order of $L$. Further, for $| \alpha | = m$, 
$$
D^\alpha (^t \bar{a}_\alpha \varphi) = t_{\bar{a}_\alpha}. D^\alpha
\varphi + \sum_{| \beta | < m} b_\beta D^\beta \varphi ~ , b_\beta
$$
being functions on  $\Omega$  with values in $q \times
p$ matrices. 

Hence if $p^* (x, \xi)$ is the characteristic polynomial of $L^*$,
$$
p^* (x, \xi) = (-1)^m \sum_{| \alpha | = m} \xi^{\alpha 
  t_{\bar{a_\alpha}}} (x) = (-1)^m t_{\overline {p(x, \xi)}}. 
$$

\begin{remark*} % rem
  If $L$ is an elliptic operator of order $m$, $L$: $C^{\infty, q}_0
  \to C^{\infty, p}_o$ and if $L^*$: $C^{\infty, p}_0 \to C^{\infty ,
    q}_0$ is the adjoint of $L$, then the operator $(-1)^m L^*$. $L$
  is strongly elliptic. 
\end{remark*}

\begin{proof}
  If $A = (-1)^m L^*$. $L$, $p(x, \xi)$, $p^* (x, \xi)$ and $p' (x,
  \xi)$ are the characteristic polynomials of $L$, $L^*$ and $A$
  respectively, and if $\xi \in \mathbb{R}^n$, $x \in \Omega$, $v \in
  \mathbb{C}^q$, $\xi \neq 0$, $v \neq 0$, have, 
  \begin{align*}
    Re (p' (x, \xi ) v, v ) & = Re (( - 1)^m p^* (x, \xi). p(x, \xi )v, v)\\
    & = Re (p(x, \xi) v, p(x, \xi) v) > 0.
  \end{align*}
\end{proof}

\begin{coro*} % coro
  If\pageoriginale $\Omega'$ is relatively compact in $\Omega$ and $L$: $C^{\infty,
    q}_0 (\Omega) \to C^{\infty, p}_0 (\Omega)$ is an elliptic
  operator, of order, $m$, $(-1)^m L^* \circ L$ is uniformly strongly
  elliptic on $\Omega'$, of even order, namely $2m$. 
\end{coro*}

We remark further that if $L$ is an elliptic operator $L$: $C^{\infty,
  q}_0 \to C^{\infty, q}_0$ (i.e. if $q = p)$, then $L^*$ is also
elliptic. In fact, for $\xi \neq 0$, $p(x, \xi)$ is an automorphism of
$\mathbb{C}^q$ and hence so is $t_{\overline{p(x, \xi)}}= (-1)^m p^*
(x, \xi)$.   

\setcounter{proposition}{0}
\begin{proposition}\label{chap3:sec6:prop1}
  Let $\Omega$ be an open set in $\mathbb{R}^n$. Then for $\varepsilon
  > 0$, there exists a constant $C(\varepsilon)$ such that for any $f
  \in \overset{\circ}{H}_m (\Omega)$, $(m > 0)$, we have  
  $$
  | f|^2_{m-1} \le \varepsilon | f |^2_m + C(\varepsilon) | f |^2_0.
  $$
\end{proposition}

\begin{proof}
  It is enough to prove the inequality for $C^\infty$ functions $f$
  %%%%% doubt ref 3 & 5
  with compact support $\subset \Omega$. By proposition 3, \S\ 5, there
  exists a constant $C_2$ such that  
  $$
  | f |^2_{m-1} \le C_2 \int\limits_{\mathbb{R}^n} (1+ | \xi
  |^2)^{m-1} | \hat{f} (\xi ) |^2 d \xi. 
  $$
\end{proof}

Now given $\varepsilon$, there exists $C' (\varepsilon)$ such that
$$
\left(1+ | \xi |^2\right)^{m-1} \le \frac{\varepsilon}{C_2} \left(1+ |
\xi |^2\right)^m + 
C'(\varepsilon ) \text { for } \xi \in \mathbb{R}^n. 
$$
Hence $| f |^2_{m-1} \le \varepsilon \int\limits_{\mathbb{R}^n} (1+ |
\xi |^2 )^m | \hat{f} (\xi) |^2 d \xi + C(\varepsilon)$. $| f |^2_0$,
which by Proposition 3, \S\ 5, proves the required inequality. 
%%%%%%%% doubt ref 3 & 5

\setcounter{theorem}{0}
\begin{theorem}[Garding's inequality]\label{chap3:sec6:thm1} % the 
  Let $L$ be a uniformly strongly elliptic differential operators of
  even order $2m$ on $\Omega$, $\Omega$ being an open\pageoriginale set in
  $\mathbb{R}^n$. Then for any relatively compact open subset $\Omega'$
  of $\Omega$, there exist constants $C> 0$ and $B>0$ such that for
  any $C^ \infty$ function $f$: $\Omega \to \mathbb{C}^q$ with
  $\supp. f \subset \Omega'$, we have 
  $$
  Re (-1)^m (Lf, f) \le C | f |^2_m - B | f |^2_0.
  $$
\end{theorem}

\begin{proof}
We shall prove the theorem in three steps.
\end{proof}

\begin{step}\label{chap3:sec6:stepI} % step I
  Let $L$ be given by
  $$
  Lf = \sum_{| \alpha | \le 2 m} a_\alpha D^\alpha f,
  $$
  where the $a_\alpha$ are constant matrices. Then we have, by
  Plancherel's theorem, 
  $$
  (Lf, f ) = (\hat{Lf}, \hat{f}).
  $$
\end{step}

Now
\begin{align*}
  \hat{L f} (\xi) & = \sum_{| \alpha | \le 2m} a_\alpha
  D^{\hat{\alpha}} f (\xi )\\ 
  & = (-1)^m \sum_{| \alpha | = 2m} a_\alpha . \xi^\alpha \hat{f}
  (\xi) + \sum_{| \alpha | \le 2m -1} a_\alpha i^{| \alpha
    |}. \xi^{\alpha} \hat{f} (\xi). 
\end{align*}

Clearly the characteristic polynomial is independent of $x$; we denote
it by $p(\xi)$. Then  
$$
(Lf, f) = (-1)^m \int\limits_{\mathbb{R}^n} (p (\xi ) \hat{f} (\xi),
\hat{f} (\xi )) d \xi + \sum_{ | \alpha | < 2m}
\int\limits_{\mathbb{R}^n} (i^{| \alpha |} a_\alpha \xi^{\alpha }
\hat{f} (\xi), \hat{f} (\xi)) d. \xi 
$$

Since $L$ is uniformly strongly elliptic on $\Omega$, there exists, by
definition, a constant $C_1$ such that  
$$
Re (p (\xi ) v, v) \ge C_1 | \xi |^{2m} | v |^2 \text { for } \xi \in
\mathbb{R}^n \text { and } v \in \mathbb{C}^q. 
$$

Hence\pageoriginale 
$$
Re(-1)^m (\hat{Lf}, \hat{f} \ge C_1 \int\limits_{\mathbb{R}^n} |
\xi |^{2m} | \hat{f} (\xi) |^2 d \xi - M_1 \int\limits_{\mathbb{R}^n}
(1+ | \xi | )^{2m - 1} | \hat{f} (\xi ) |^2 d \xi 
$$
where $M_1$ is a constant, depending only on the matrices $a_\alpha$,
$| \alpha | \le 2m - 1$. Let $A$ be a constant such that  
$$
C_1 | \xi |^{2m} - M_1 (1+ | \xi|)^{2m -1} \ge C_2 ( 1 + | \xi |^2 )^m
$$
for $| \xi | \ge A$ and a suitable constant $C_2 > 0$. Then 
\begin{align*}
  Re (-1)^m (Lf, f ) & \ge C_2 \int\limits_{| \xi | > A} (1+ | \xi |^2
  )^m | \hat{f} (\xi) |^2 d \xi\\ 
  & \qquad  - M \int\limits_{| \xi | \le A} (1 +
  | \xi | )^{2m - 1} | \hat{f} (\xi) |^2 d \xi\\ 
  & \ge C_2 \int\limits_{\mathbb{R}^n} (1+ | \xi |^2 )^m | \hat{f}
  (\xi ) |^2 d \xi\\ 
  & \qquad - C_2 \int\limits_{| \xi | \le A} (1+ | \xi |^2)^m
  | \hat{f} (\xi) |^2 d \xi\\ 
  & \qquad - M \int\limits_{| \xi | \le A} (1+ | \xi
  | )^{2m - 1} | \hat{f} (\xi) |^2 d \xi.
\end{align*}

Let $B$ be constant such that 
$$
C_2 (1+ | \xi |^2)^m + M (1+ | \xi |)^{2m -1} < B \text { for } | \xi | \le A.
$$

Then 
$$
Re (-1)^m (Lf, f) \ge C_2 \int\limits_{\mathbb{R}^n} (1+ | \xi |^2)^m
| \hat{f} (\xi) |^2 d \xi - B \int\limits_{\mathbb{R}^n} | \hat{f}
(\xi) |^2 d \xi. 
$$

%%%% doiubt ref 3 & 5

By Proposition 3, \S\ 5 there exists a constant $C$ such that 
$$
C_2 \int\limits_{\mathbb{R}^n} (1+ | \xi | ^2)^m | \hat{f} (\xi ) |^2
d \xi \ge C | f |^2_m 
$$
i.e.\pageoriginale the inequality is proved when $L$ has constant
coefficients.  

\begin{step}\label{chap3:sec6:stepII}% step II
  We shall now prove that for any $x_0 \in \Omega$, there exist a
  relatively compact neighbourhood $U$ of $x_0$, $U \subset \Omega$,
  and constants $C$, $B$ such that for any $C^\infty ~ f $: $\Omega
  \rightarrow \mathbb{C}^q$ with $\supp. f \subset U$, we have  
  $$
  \re ~ (-1)^m ~ (Lf, f) \geqq  C ~ | f |^2_m - B ~ | f |^2_0.
  $$
\end{step}

We may write $L$ as $L =\sum\limits_{i=1}^k ~ (B_i)^* A_i$ for some
$k$, where $A_i$ and $B_i$ are differential operators of orders  $\leq
m$. For any $x_0 \in \Omega$, $A_i$  and $B_i$ can be written as $A_i
= A^0_i + A'_i$, $B_i = B^0_i + B_1'$, where $B^0_i$, $A^0_i$ are
differential operators with constant coefficients and $A'_i$, $B'_i$
are differential operators whose coefficients vanish at $x_0$. Then 
\begin{multline*}
(Lf, f) =\\ 
  \sum_i (A^0_i ~ f, B^0_i ~ f) + \sum_i (A'_i ~ f,  B^0_i ~
  f) + \sum_i (A^0_i  ~ f, B'_i ~ f) + \sum_i (A'_i f, B'_i ~F) 
\end{multline*}

Let $L^0$ be the differential operator with constant coefficients, defined by 
$$
L^0 = \sum (B^0_i)^* ~ A^0_i.
$$

Since the coefficients of $A'_i$ and $B'_i$ vanish at $x_0$ and are
continuous functions on $\Omega$, given $\varepsilon > 0$ there exists
a relatively compact neighbourhood $U$ of $x_0$, $U \subset \Omega$
such that for $f$ with supp. $f \subset U$, 
$$
| B'_i  ~ f |^U_0 + | A'_i f |^U_0 \leq \varepsilon | f |_m.
$$

Then\pageoriginale
$$
\re ~ (-1)^m (Lf, f) \geqq \re (-1)^m ~ (L^0 f, f) - \varepsilon ~ M ~ | f |^2_m
$$
where $M$ is a constant depending on $A_i$, $B_i$. Now by the result
in Step \ref{chap3:sec6:stepI}, there exist constants $C'$, $B'$ such that for $C^\infty f $
with $\supp f \subset U$, 
$$
\re (-1)^{m} (L^{o} f, f) \geq C' | f |^2_m - B' | f |^2_0.
$$

Hence $\re ~ (-1)^m (Lf, f) \geqq (C' - \varepsilon ~ M) | f |^2_m -
B' | f |^2_0$; since $\varepsilon \rightarrow 0$ as $U$ shrinks to
$x_0$, our assertion is proved. 

\begin{step}\label{chap3:sec6:stepIII}% step III
  This is the general case. By step \ref{chap3:sec6:stepII} above, for any relatively
  compact open subset $\Omega'$ of $\Omega$, there exist points $x_i$,
  $1 \leq i \leq N$, and neighbourhoods $U_i$ of $x_i$, $\cup U_i
  \supset \bar{\Omega}'$ and constants $C$, $B$ such that for a
  $C^\infty ~ f$ with $\supp. f \subset ~ U_i$, $\re (-1)^m (Lf, f)
  \geqq C | f |^2_m - B | f |^2_0$. We write, as in II, $L
  =\sum\limits_{i=1}^k (B_i)^* ~ A_i$, where $A_i$, $B_i$ are
  differential operators of orders $\leq m$. Let $\eta_k$ be
  $C^\infty$ functions, $\eta_k$: $\Omega \rightarrow \mathbb{R}$,
  with $\supp. \eta_k \subset U_k$, $0 \leq \eta_k (x) \leq 1$, and
  $\sum  \eta^2_k (x) = 1$ for $x \in \Omega'$. (The $\eta_k$ exist :
  see Chap. I, \S \ref{chap1:sec2}.) We first remark that if $\varphi $ is a
  $C^\infty$ function with compact support and $\triangle$ is a
  differential operator of order $m$, then 
  $$
  \triangle (\varphi f) - \varphi \triangle f = \sum_{ | \alpha | < m}
  ~ a_\alpha ~ D^\alpha ~ f, 
  $$
  where the $a_\alpha$ are continuous function with compact supports
  (depending on $\varphi$). Now for $C^\infty ~f$ with $\supp. f
  \subset \Omega'$, 
  $$
  | \eta_k ~ f |^2_m \leq \frac{1}{C} ~(-1)^m ~ \re. \sum (A_i  ~
  \eta_k ~ f, B_i ~ \eta_k ~f) + \frac{B}{C} | \eta_k ~ f |^2_0. 
  $$
\end{step}

By\pageoriginale the remark made above, there exists a constant $C_1$ depending on
$A_i$, $B_i$, such that  
\begin{multline*}
(-1)^m \re. \sum_i (A_i ~ \eta_k ~ f, B_i ~ \eta_k ~ f) \leq (-1)^m\\
\re \sum_i (\eta_k ~ A_i ~ f, \eta_k ~ B_i ~ f) + C_1 | f |_m . | f
|_{m-1}. 
\end{multline*}

Hence 
$$
| \eta_k f|^2_m \leq \frac{1}{C} (-1)^m \re. \sum_i (A_i f, \eta^2_k ~
B_i f) + \frac{C_1}{C} |f|_m | f |_{m-1} + \frac{B}{C} | \eta_k ~ f
|^2_0. 
$$

Since $D^\alpha ~ \eta_k ~ f = \eta_k ~ D^\alpha ~ f +
\sum\limits_{\beta < \alpha} ~(^\alpha _\beta) ~ D^\beta ~ f ~
D^{\alpha - \beta} ~ \eta_k$, we have 
$$|\sum\limits_k | ~ \eta_k ~
f|^2_m - | f |^2_m | \leq C_2 | f |_m. | f |_{m-1}$$ 
for some constant $C_2$. 

Hence summing over $k$,
$$
| f |^2_m \leq C_3 (-1)^m \re. (Lf, f) + C_4 ~ | f |_m ~ | f |_{m-1} +
C_5 | f |^2_0 
$$
for some constants $C_3$, $C_4$, $C_5$. Now
$$
| f |_m ~ | f |_{m-1} \leq \frac{1}{2}. (\varepsilon | f |^2_m +
\frac{1}{\varepsilon} | f |^2_{m-1}), 0 < \varepsilon < \frac{1}{2}. 
$$

Hence
$$
| f |^2_m - \frac{\varepsilon}{2} | f |^2_m \leq C_3 (-1)^m \re (Lf,
f) + \frac{C_4}{2 \varepsilon} | f |^2_{m-1} + C_5 | f |^2_0. 
$$

By Proposition \ref{chap3:sec6:prop1}, there exists a constant $C_6$ such that 
$$
| f |^2_{m-1} \leq \varepsilon^2 | f |^2_m + C_6 | f |^2_0.
$$

Hence $(1- \varepsilon )$. $| f |^2_m \leq C_3 (-1)^m \re$. $(Lf, f) +
C_7 | f |^2_0$ which\pageoriginale proves the theorem. 

\begin{remark*}% remark 
  If $L$ is a uniformly strongly elliptic differential operator of
  order $2m$ which is homogeneous and has constant coefficients,
  i.e. $L = \sum\limits_{| \alpha | = 2m} a_\alpha ~ D^\alpha$, then
  the above inequality holds in a stronger form, i.e. for $\Omega'
  \subset\subset \Omega$, there exists a constant $C$ such that for $C^\infty
  ~ f$  with $\supp. f \subset \Omega'$, 
  $$
  (-1)^m \re. (Lf, f) \geq C. | f |^2_m.
  $$
\end{remark*}

\begin{proof}% proof
  We have, as in Step \ref{chap3:sec6:stepI} above,
  $$
  (Lf, f) = (\hat{Lf}, \hat{f}) = (-1)^m \int (p(\xi) \hat{f}
  (\xi), \hat{f}(\xi)) ~ d\xi, 
  $$
  and there exists constant $C$ such that
  $$
  \re. (-1)^m (Lf, f) \geq C. \int\limits_{\mathbb{R}^n} | \xi |^{2m}
  ~ | \hat{f} (\xi) |^2 ~ d\xi. 
  $$
\end{proof}

By Plancherel's theorem
$$
| D^\alpha ~ f |^2_0 = | D^{\hat{\alpha}}f |^2_0 =
\int\limits_{\mathbb{R}^n} | \xi^{2\alpha} | | \hat{f} (\xi) |^2 ~ d
\xi. 
$$

Hence $\re. (-1)^m (Lf, f) \geq C' \sum\limits_{| \alpha | = m} ~ |
D^\alpha ~ f |^2_0$ and since $f$ is a $C^\infty$ function with
compact support $\subset \Omega'$ we have $\sum\limits_{| \alpha | =
  m} | D^\alpha ~ f |^2_0 \geq C'' | f |^2_m$ for some constant $C'' >
0$ (Poincare's inequality; see Remark (6) after the Definitions in \S
5), which proves the required inequality. 

\begin{proposition}\label{chap3:sec6:prop2}% propsiton 2
  If\pageoriginale $\Omega$ is a bounded open subset of $\mathbb{R}^n$ and $m$ is an
  integer $>0$, then for any $A > 0$, there exists a constant $C$ such
  that for $f \in H^0_m (\Omega)$, 
  $$
  \int\limits_{\mathbb{R}^n} ~ (1 + | \xi |^2 )^m | \hat{f} (\xi) |^2
  ~ d\xi \leq C ~\int\limits_{| \xi | > A} ~ (1+|\xi|^2)^m|
  \hat{f}(\xi)|^2 ~ d\xi. 
  $$
\end{proposition}

[This Proposition may be looked upon as a stronger version, in the
  case $p = 2$, of Poincare's inequality (Remark (6) at the beginning
  of \S\ 5)]. 

\begin{proof}% proof
  If the proposition is false, there exists a sequence $\{
  f_\nu\}_{\nu \geq 1}$ of $C^\infty$ functions with compact support
  $\subset \Omega$, such that $| f_\nu |_m = 1$ and  
  \begin{equation}
    \int\limits_{| \xi | > A} (1 + | \xi |^2)^m | \hat{f}_\nu ~ ( \xi
    ) |^2 d ~ \xi \rightarrow 0 \text{ as } \nu \rightarrow \infty
    . \tag{6.1}\label{chap3:sec6:eq6.1} 
  \end{equation}
\end{proof}

By Rellich's lemma the map $i$: $\overset{o}{H}_m (\Omega )
\rightarrow \overset{o}{H}_0 (\Omega )$ is completely
continuous. Hence we may assume that $\{f_\nu \}$ converges in $L^2$
to $f$ any. Now $\hat{f}_\nu (\xi)$ is an analytic function of $\xi$,
$\nu \geq 1$, and since $f \in \overset{o}{H}_0 (\Omega)$ and $\Omega$
is relatively compact, $f \in L^1$; clearly 
$$
\hat{f} (\xi) = (2\pi)^{-n/2} \int\limits_{\Omega} e^{- i ~ \xi ~ x} f(x) ~ dx
$$
is analytic in $\xi$ since $\Omega$ is bounded; moreover $\hat{f}_\nu$
converges uniformly to $\hat{f}$ on compact subsets of
$\mathbb{R}^n$. 

Now, because of assumption (6. 11), for every compact set $K$ with $K
\subset \{ \xi | | \xi | > A\}$, we have $\int\limits_{K} |
\hat{f}_\nu ~ (\xi) |^2 ~ d ~ \xi ~ \rightarrow 0$. Hence
$\int\limits_K | \hat{f}(\xi) |^2 ~ d ~\xi = 0$, so that $\hat{f}(\xi)
= 0$ for $\xi \in K$ and hence $f = 0$ since\pageoriginale we may choose $K$ such
that $\overset{o}{K} \neq \phi$. Hence $\hat{f}_\nu$ converges to zero
uniformly on compact sets so that $\int\limits_{| \xi | \leq A} (1+ |
\xi |^2)^m | \hat{f}_\nu (\xi)|^2 ~ d\xi \rightarrow 0$ and by
assumption $\int\limits_{| \xi | > A} ~ ( 1 + | \xi |^2)^m |
\hat{f}_\nu (\xi)|^2 ~ d \xi \rightarrow 0$. 

But $| f |_m = 1$ and thus we have a contradiction. This proves the
proposition. 

\setcounter{lemma}{0}
\begin{lemma}\label{chap3:sec6:lem1}% lemma 1
  Let $\Omega$, $\Omega'$, $\Omega''$ be open sets in $\mathbb{R}^n,
  \Omega'' \subset\subset \Omega' \subset\subset \Omega$. Let $\varphi$ be in
  $C^\infty (\Omega )$ such that $\varphi (x) = 1$ for $x$ in a
  neighbourhood of $\bar{\Omega''}$ and $\varphi (x) = 0$ for $x
  \notin \Omega'$. Then for any $\varepsilon > 0$, there exists a
  constant $C(\varepsilon )$, such that  for $k \geq 1$ and for $f \in
  C^\infty (\Omega )$, 
  $$
  \sum_{| \beta | = k} | \varphi^k ~ D^\beta ~ f|^2_0 \leq \varepsilon
  \sum_{| \beta | = k +1} | \varphi^{k+1} ~ D^\beta ~ f |^2_0 +
  C(\varepsilon) \sum_{| \beta | = k -1} | \varphi^{k-1} ~ D^\beta ~
  f|^2_0 
  $$
  [$\varphi^0$ stands for $1$ on $\Omega'$, $0$ outside].
\end{lemma}

\begin{proof}% proof 
  It is enough to prove that for $k \geq 1$, and $| \beta | = k$ we have 
  $$
  | \varphi^k ~ D^\beta ~ f |^2_0 \leq \varepsilon \sum_{| \alpha | =
    k + 1} | \varphi^{k+1} ~ D^\alpha ~ f|^2_0 + C(\varepsilon) \sum_{
    | \alpha | = k -1} | \varphi^{k-1} ~ D^\alpha f|^2_0. 
  $$
\end{proof}

Now
$$
(\varphi^k ~ D^\beta ~ f, \varphi^k ~ D^\beta ~ f) = (D^\beta ~ f,
\varphi^{2k} ~ D^\beta ~ f) 
$$
Let $\beta = \gamma + e$, where $| e | = 1$. Then
\begin{align*}
  (\varphi^k ~ D^\beta ~ f, \varphi^k ~ D^\beta f) & = - (D^\gamma ~
  f, D^e. \varphi^{2k} D^\beta ~ f)\\ 
  & = - (D^\gamma ~ f, 2k ~ \varphi^{2k-1} D^e \varphi . D^\beta f) -
  (D^\gamma ~ f, \varphi^{2k} ~ D^{\beta + e}  ~ f).\\ 
  & = - (\varphi^{k-1} ~ D^\gamma ~ f, 2k ~ \varphi^k ~ D^e ~ \varphi
  . D^\beta ~ f)\\ 
  & \hspace{2cm}- (\varphi^{k-1} ~ D^\gamma ~ f, \varphi^{k+1} ~
  D^{\beta + e} ~ f). 
\end{align*}

By\pageoriginale Schwarz's inequality, this gives,
$$
| \varphi^k ~ D^\beta ~ f |^2_0 \leq | \varphi^{k-1} ~ D^\gamma ~
f|_0. C_1 | \varphi^k ~D^\beta ~ f |_0 + | \varphi^{k -1} ~ D^\gamma ~
f|_0. | \varphi^{k+1} ~ D^{\beta + e} ~ f |_0 
$$
for a constant $C_1$ depending on $\varphi$. Now
$$
\displaylines{\hfill 
  | \varphi^{k-1} ~ D^\gamma ~ f|_0. | \varphi^k ~ D^\beta ~ f|_0 \leq
  \frac{1}{2} \left\{ \frac{\varepsilon}{C_1} | \varphi^k ~ D^\beta ~ f|^2_0
  + \frac{C_1}{\varepsilon} | \varphi^{k-1} ~ D^\gamma ~ f|^2_0 \right\}
  \hfill \cr
  \text{and}\hfill  
  | \varphi^{k-1} ~ D^\gamma ~ f|_0. | \varphi^{k+1} ~ D^{\beta + e} ~
  f|_0 \leq \frac{1}{2} \left\{ \varepsilon | \varphi^{k+1} D^{\beta + e} f
  |^2_0 + \frac{1}{\varepsilon} | \varphi^{k-1} ~ D^\gamma ~f
  |^2_0 \right\}.\hfill } 
$$

Hence 
$$
( 1-\varepsilon) | \varphi^k ~ D^\beta ~ f|^2_0\leq \varepsilon |
\varphi^{k+1} ~ D^{\beta + e} ~ f|^2_0 + C(\varepsilon ). |
\varphi^{k-1} ~ D^\gamma ~ f|^2_0. 
$$
This proves our assertion.

\begin{theorem}[Friedrichs' inequality]\label{chap3:sec6:thm2}% theorem 2
  Let $\Omega$ be a bounded open set in $\mathbb{R}^n$ and $L$, an
  elliptic differential operator on $\Omega$ of order $m$, given by $L
  = \sum\limits_{| \alpha | \leq m} ~ a_\alpha ~ D^\alpha$. Let $r$ be
  an integer, $r \geq 0$. 
\end{theorem}

\begin{enumerate}[(I)]
\item If\pageoriginale $a'_\alpha ~ s$ are constant, there exists a constant $C$
  such that for $f \in C^\infty (\Omega)$, 
  $$
  | f |_{m+r} ~ \leq ~ C ~ | Lf |_r.
  $$
\item For any $x_0 \in \Omega$, there exists a neighbourhood $U$ of
  $x_0$ and a constant $C_1$ such that for any $C^\infty ~ f$ with
  $\supp. f \subset U$, we have 
  $$
  | f |_{m+r} \leq C_1 ~ | Lf |_r.
  $$
\item There exists a constant $C_2$ such that for any $f \in C^\infty
  (\Omega ), \supp. f \subset \Omega $, 
  $$
  | f |_{m+r} \leq C_2 \{ | Lf |_r + | f |_0 \}.
  $$
\item If $\Omega''$, $\Omega'$ are open subsets of $\Omega$, $\Omega''
  \subset\subset \Omega' \subset\subset \Omega$, then there exists a
  constant $C_3$ such that for $f \in C^\infty (\Omega)$, 
  $$
  | f |^{\Omega''}_{m+r} ~ \leq C_3 \{ | Lf |^{\Omega'}_r + | f |^{\Omega'}_0 \}.
  $$
\end{enumerate}

The proofs of this theorem are completely parallel to those of
G$\ring{\text{a}}$rd\-ing's inequality, but in this form do
not follow at once from Theorem \ref{chap3:sec6:thm1}.
  In the case $r = 0$ (III) and (I) and (II) with the inequality $| f
  |_m \leq C | Lf |_0 $ replaced by $| f |_m \leq C \{  | Lf |_0 + | f
  |_0\}$ follows at once from G$\ring{\text{a}}$rding's inequality applied to
  $\triangle = (-1)^m ~ L^* ~ L$. 

\begin{proof}% proof
  \begin{enumerate}
  \item[(I)] Since $L$ is elliptic, there exists a constant $B_1 > 0$
    such that  
    \begin{equation}
      | p(\xi ) . v | \geq B_1 | \xi |^m | v | \quad \text{ for }  \xi
      \in \mathbb{R}^n \text{ and } v \in
      \mathbb{C}^q. \tag{6.2}\label{chap3:sec6:eq6.2}  
    \end{equation}
  \end{enumerate}
\end{proof}

Let\pageoriginale $L_1 = \sum\limits_{| \alpha | = m} ~ a_\alpha ~ D^\alpha$ and
$L_2 = \sum\limits_{| \alpha | < m} ~ a_\alpha ~ D^\alpha$. Then there
exists a constant $M$ depending on $L_2$ such that  
\begin{equation}
  |\widehat{L_2 f} (\xi ) |^2 \leq M. (1 + | \xi |^2)^{m-1} ~
  |\hat{f}(\xi) |^2. \tag{6.3}\label{chap3:sec6:eq6.3} 
\end{equation}

Also there exists a constant A such that

$\dfrac{B^2_1}{2} | \xi |^{2m}  - M(1 + | \xi |^2 )^{m-1} ~ \geq ~ B_2
~ (1 + | \xi |^2 )^m \quad \text{ for } | \xi | > A \text{ where }
B_2$ is a suitable constant $ > 0$. By \S\ 5,
Proposition \ref{chap3:sec5:prop3}, we have 
\begin{align*}
  | Lf |^2_r  & \geq c' \int\limits_{\mathbb{R}^n} ( 1+ | \xi |^2)^r
  | \widehat{Lf} (\xi ) |^2 d~ \xi\\ 
  & \geq c' \int\limits_{| \xi | > A}
  ( 1+ | \xi |^2)^r | \widehat{L_1f}(\xi ) + \widehat{L_2 f} (\xi)
  |^2 d ~ \xi \\ 
  & \geq c' \int\limits_{ | \xi | > A} ~(1 + | \xi |^2)^r \{
  | \frac{1}{2} p (\xi ) \hat{f} (\xi ) |^2 - M (1+ | \xi |^2)^{m-1} |
  \hat{f} (\xi )|^2 \} ~ d ~ \xi\\ 
  & \hspace{2cm} ( \text{ since } | a + b  |^2 \geq
  | \frac{1}{2} | a |^2 - | b |^2 \text{ for } a, b \in \mathbb{C}) \\ 
  & \geq  c' \int\limits_{| \xi | > A} ~ (1 + | \xi |^2 )^r \{
  \frac{1}{2} B_1 | \xi |^{2m} - M(1+ | \xi |^2 )^{m-1} \} | \hat{f}
  | (\xi ) |^2 ~ d ~ \xi\\ 
  & \tag*{\text{(by  ~(\ref{chap3:sec6:eq6.2})~  and
  | ~(\ref{chap3:sec6:eq6.3})).}}  
\end{align*}

Now by the choice of $A$, 
$$
| Lf |^2_r \geq B_2 c' \int\limits_{| \xi > A} ( 1+ | \xi |^2 )^{m+r}
|\hat{f} (\xi)|^2 d \xi 
$$
and hence by Proposition \ref{chap3:sec6:prop2}, there exists a
constant $c$ such that 
$$
| Lf | _r \geq  c | f |_{m+r}.
$$

\begin{enumerate}
\item[(II)] Let\pageoriginale $L = L_0 + L_1$ where $L_0$ and $L_1$ are differential
  operators of orders $\leq m$ such that $L_0$ has constant
  coefficients of $L_1$ vanish at $x_0$. Since the coefficients of
  $L_1$ are continuous functions on $\Omega$, there exists a
  neighbourhood $U$ of $x_0$ such that for any $C^\infty ~ f$ with
  $\supp. f \subset U$, $| L_1 f |^2_r \leq \dfrac{\varepsilon}{2} | f
  |^2_{m+r}$, $\varepsilon$ depending on $U$ and tending to zero as
  $U \rightarrow \{ x_0 \}$\footnote{If $r > 0$, this involves 
    integration by parts.}. Then  
  $$
  | L_0 f + L_1 f |^2_r \geq \frac{1}{2} | L_0 f |^2_r - | L_1 ~ f |^2_r.
  $$
  By (I) there exists a constant $B$ such that 
  $$
  | L_0 ~f|^2_r \geq B | f |^2_{m+r}.
  $$
  Hence
  $$
  | Lf |^2_r \geq ( \frac{B}{2} - \varepsilon ) | f |^2_{m+r}.
  $$
\item[(III)] Because of (II), there is a finite covering $\{ U_1, \ldots
  , U_h \}$ of $\bar{\Omega}'$ such that if $\supp ~ f \subset U_i$
  for some $i$, then  
  $$
  | Lf |_r \geq C | f |_{m+r}.
  $$
  Let $\supp f \subset \Omega'$ and $\varphi_1, \ldots, \varphi_h$ be
  $C^\infty$ functions, $0 \leq \varphi_i \leq 1$, $\supp \varphi_i
  \subset U_i$, $\sum \varphi^2_i = 1$ on $\Omega'$. Now, since $\supp
  ~ f \subset \Omega'$, we have $\big| ~ | f |^2_{m+r} - \sum_i |
  \varphi_i ~f|^2_{m+r} \big| \leq C' | f |_{m+ r -1}$, and $\big| ~ |
  Lf |^2_r - \sum\limits_i | L (\varphi_i ~ f) |^2_r \big| \leq C' | f
  |_{m+r-1}$, so that, since 
  $$
  | L (\varphi_i ~ f) |_r \geq C | \varphi_i ~ f |_{m+r}, \text{ and }
  | f |_{m+r-1} \leq \varepsilon | f |_{m+r} + C(\varepsilon) | f |_0, 
  $$
  we\pageoriginale obtain the required inequality.
\item[(IV)] Define a $C^\infty$ function $\varphi$ on $\Omega$ such
  that $\varphi(x) = 1$ for $x$ in a neighbourhood of $\bar{\Omega}''$
  and $\varphi (x) = 0$ for $x \notin  \Omega'$. Then if $f \in
  C^\infty (\Omega )$, $m' = m + r$, we have 
  $$
  D^\alpha (\varphi^{m'} ~ f) = \sum_{\beta \leq \alpha} ( ^\alpha
  _{\beta}) ~ (D^\beta ~ \varphi^{m'}) ~ D^{\alpha - \beta} ~ f. 
  $$
\end{enumerate}

Now, $D^\beta ~ \varphi^{m'} (x) = C_\beta (x) \varphi^{m' - | \beta
  |} (x)$, where $C_\beta (x)$ is a $C^\infty$ function and $| C_\beta
(x) | \leq A_1$ for some constant $A_1$. Then  
\begin{gather*}
  D^\alpha (\varphi^{m+r} ~ f) = \varphi^{m+r} ~ D^\alpha ~ f +
  \sum_{\substack{\beta \leq \alpha\\ \beta \neq 0}} ~ C'_\beta ~
  \varphi^{m+r - | \beta | } ~ D^{\alpha - \beta } ~ f\\ 
  \text{ where } C'_\beta = ( ^\alpha _{\beta}) C_\beta .
\end{gather*}

Now squaring both sides,  using Schwarz's inequality and summing over
$\alpha$, $| \alpha | \leq m+r$, we obtain, 
\begin{equation}
  \big| ~ |  \varphi^{m+r} ~ f|^2_{m+r} ~ -\sum_{| \alpha | \leq m+r}
  | \varphi^{m+r} ~ D^\alpha ~ f |^2_0 \big| \leq A_2 \sum_{| \beta |
    \leq m+r} | \varphi^{| \beta |}.  D^\beta ~ f |^2_0
  \tag{6.4}\label{chap3:sec6:eq6.4}  
\end{equation}
for a suitable constant $A_2$.

By Part III above, we have, since $\supp. \varphi^{m'} ~f \subset \Omega'$,
\begin{equation}
  | \varphi^{m+r}  ~ |^2_{m + r} \leq C \{  | L (\varphi^{m+r} ~ f)
  |^2_r + ( | f |_0^{\Omega'})^2 \} . \tag{6.5}\label{chap3:sec6:eq6.5} 
\end{equation}

By a repeated application of Lemma \ref{chap3:sec6:lem1} to $|
\varphi^{| \beta |} ~ 
D^\beta ~ f |^2_0$ for $| \beta | < m+r$, we have 
\begin{equation}
  \sum_{| \beta | < m+r } | \varphi^{| \beta |} ~ D^\beta ~ f|^2_0
  \leq \varepsilon \sum_{| \beta | = m+r}  | \varphi^{m+r} ~ D^\beta ~
  f |^2_0 + C (\varepsilon ) ( | f
  |_0^{\Omega'})^2. \tag{6.6}\label{chap3:sec6:eq6.6}  
\end{equation}

It\pageoriginale follows from (\ref{chap3:sec6:eq6.4}),
(\ref{chap3:sec6:eq6.5}) and (\ref{chap3:sec6:eq6.6}) that 
$$
\displaylines{ 
  \sum_{| \alpha | \leq m+r} | \varphi^{m+r} ~ D^\alpha ~f|^2_0 \leq C
  \left\{ | \varphi^{m+r} ~ Lf|^2_r + \left(| f |^{\Omega'}_0
  \right)^2 \right\}\hfill\cr
  \hfill +
  \varepsilon \sum_{| \alpha | \leq m+r} | \varphi^{m+r} ~ D^\alpha ~
  f|^2_0 + C (\varepsilon ) \left(| f |^{\Omega'}_0\right)^2 \cr
  \text{so that}\hfill  
  \sum_{| \alpha | \leq m + r } | \varphi^{m+r} ~ D^\alpha  ~ f|^2_0
  \leq C_2 \left\{ \varphi^{m+r} ~ Lf|^2_r + \left(| f |^{\Omega'}_0 \right)^2
  \right\} \hfill } 
$$
for a suitable constant $C_2$.

Since $\varphi (x) = 1$ for $x \in \Omega''$ and $\supp. \varphi
\subset \bar{\Omega}'$, the theorem follows. 
\begin{remark*} % remark
  As in the remark following G$\ring{\text{a}}$rding's inequality, parts (I) and (II)
  of Theorem 2 can be proved for homogeneous elliptic operators $L$,
  without appealing to Proposition \ref{chap3:sec6:prop2}; the
  reasoning is the same.  
\end{remark*}

The proofs given in this section are essentially those of Garding
\cite{11} and Friedrichs \cite{10}. 

\section[Elliptic operators with $C^\infty$...]{Elliptic operators
  with $C^\infty$ coefficients: the regularity theorem}\label{chap3:sec7} 

Let $\Omega$ be an open set in $\mathbb{R}^n$.
\begin{defi*}% definition
  If $L$ is an elliptic differential operator $L : C^{\infty,q}_0$,
  $(\Omega ) \rightarrow C^{\infty, p}_0 (\Omega )$ and $f \in H_0
  (\Omega )$, we define $(Lf)$ as a linear\pageoriginale functional on $C^{\infty,
    p}_0\break (\Omega )$, by 
  $$
  (Lf) (\varphi) = (f, L^* \varphi) \text{ for } \varphi \in
  C^{\infty,p}_0 (\Omega). 
  $$
\end{defi*}

Further, $(Lf)$ is said to be in $H_0 (\Omega)$ (or $H_m (\Omega)$ or
to be strongly differentiable) if there exists $g \in H_0 (\Omega ) $
(or $H_m(\Omega )$ or which is strongly differentiable) such that
$(Lf) (\varphi ) = (g, \varphi ) = (f, L^* \varphi )$ for any $\varphi
\in C^{\infty, p}_0 (\Omega )$. 

In what follows upto the regularity theorem, $L$ denotes a uniformly
strongly elliptic operator of order $2m$ with $C^\infty$ coefficients,
$L$: $C^{\infty, q}_0\break (\Omega ) \rightarrow C^{\infty, q}_0 (\Omega )
$ and $L = \sum\limits_{i = 1}^r ~ B^*_i A_i$, $A_i$, $B_i$ being
differential operators of orders $\leq m$. For $\varphi$, $\psi \in
H_m (\Omega )$, we define $Q(\varphi, \psi )$ by $ Q(\varphi , \psi ) =
\sum\limits_{i=1}^r (A_i \varphi, B_i \psi )$. Let $\Omega' \subset\subset
\Omega$ and $h \in \mathbb{R}, h \neq 0$ be so small that $(x_1,
\ldots,\break x_n) \in \Omega'$ implies $(x_1 + h, x_2, \ldots, x_n) \in
\Omega$. We write $(x + h)$ for $(x_1+h,x_2, \ldots, x_n)$. For $g \in H_m
(\Omega )$, we define $g^h$: $\Omega' \rightarrow \mathbb{C}^q$ by
$g^h(x) = \dfrac{g(x+h) - g(x)}{h}$. 

\setcounter{lemma}{0}
\begin{lemma}\label{chap3:sec7:lem1}% lemma 1
  \begin{enumerate}[(a)]
  \item If $\eta \in  C^{\infty, 1}_0 (\Omega)$, there exists a
    constant $C$ such that for any $f \in H_0 (\Omega)$, and $h$ small
    enough, we have 
    $$
    | ( \eta ~ f )^h - \eta ~ f^h |_0 \leq C | f |_0.
    $$
  \item For $f \in H^0_m (\Omega )$, there is a constant $C > 0$ such
    that $| f^h |_{m-1} \leq C | f |_m$. 
  \end{enumerate}
\end{lemma}

\begin{proof}% proof
\begin{align*}
    (\eta f)^h (x) - (\eta ~ f^h) (x) &= \frac{\eta (x+h) - \eta
      (x)}{h} f(x+h)\\ 
    & = \eta^h (x) ~ f(x + h).
\end{align*}
\end{proof}

This\pageoriginale proves (a). The proof of (b), for $f \in C^{\infty, q}_0$,
follows at once from 
$$
f^h (x) = \int\limits_0^1 \frac{\partial f}{\partial x_1} (x_1 + th,
x_2, \ldots , x_n) ~ dt, 
$$
and, for $f \in \overset{o}{H}_m(\Omega )$, by passage to the closure.

\setcounter{proposition}{0}
\begin{proposition}\label{chap3:sec7:prop1}% proposition 1
  If $f \in \overset{o}{H}_m (\Omega )$ and has compact support, and
  there exists a constant $C$ such that for any $\varphi \in
  C^{\infty, q}_0 (\Omega ), | Q (f, \varphi) | \leq C | \varphi
  |_{m-1}$, then $f \in H_{m+1}(\Omega )$.  
\end{proposition}

\begin{proof}% proof
  We have
  $$
  Q (f^h, \varphi ) = \sum_i (A_i ~ f^h, B_i ~ \varphi ).
  $$
  Since $D^\alpha ~ f^h = (D^\alpha ~ f)^h$, it follows from Lemma
  \ref{chap3:sec7:lem1} (a) that $| A_i ~ f^h - (A_i ~ f)^h |_0 \leq C_1 | f |_m$ for a
  constant $C_1$ depending on $L$. Hence 
  $$
  Q(f^h, \varphi) = \sum ((A_i f)^h, B_i  \varphi) + 0 (| \varphi |_m . |f |_m).
  $$
\end{proof}

Now 
\begin{align*}
  ((A_i ~ f)^h, B_i \varphi ) & = - (A_i f, (B_i \varphi)^{-h})\\
  & = - (A_i f, B_i \varphi^{-h}) + 0(| \varphi|_m)\, (\text{Lemma
    \ref{chap3:sec7:lem1} (a)}). 
\end{align*}

Hence 
$$
Q(f^h, \varphi) = - Q(f, \varphi^{-h}) + 0 ( | \varphi |_m);
$$
by hypothesis, 
$$
| Q (f, \varphi^{-h}) \leq C | \varphi^{-h} |_{m-1} \leq C' | \varphi
|_m (\text{ Lemma \ref{chap3:sec7:lem1} (b)}). 
$$

Hence\pageoriginale there exists a constant $C_2$ such that 
$$
| Q (f^h, \varphi) | \leq C_2 | \varphi |_m.
$$

This holds for any $\varphi \in C^{\infty,q}_0 (\Omega)$. Since $f^h$
has compact support $\subset \Omega$, choose a sequence $\{
\varphi_\nu\}$ of functions in $C^{\infty, q}_0 (\Omega )$,
$\varphi_\nu \rightarrow f^h$ in $H_m$. Then, we have, 
$$
| Q(f^h, \varphi_\nu) | \leq C_2 | \varphi_\nu |_m
$$
and passing to the limit,
\begin{equation}
  | Q (f^h, f^h) | \leq C_2 | f^h |_m. \tag{7.1}\label{chap3:sec7:eq7.1}
\end{equation}

Now by Garding's inequality, there exists a constant $B$ such that
$$
| \varphi_\nu |^2_m \leq B(-1)^m \re (L \varphi_\nu, \varphi_\nu) | +
| \varphi_\nu|^2_0, 
$$
hence
$$
| \varphi_\nu |^2_m \leq B | Q (\varphi_\nu, \varphi_\nu) | + |
\varphi_\nu |^2_0; 
$$
taking limits as $\nu \rightarrow \infty$ and using
(\ref{chap3:sec7:eq7.1}), this gives  
$$
| f^h |^2_m \leq B ~ C_2 | f^h |_m + | f |^2_0.
$$

Hence there exists a constant $M$ such that 
$$
| f^h |_m \leq M.
$$

Consider $f^h$ for sufficiently small $h$. This is a bounded set in
the Hilbert space $H_m(\Omega)$; hence there is a sequence $\{
h_\nu\}$, $h_\nu \rightarrow 0$ such that $f ^{h}_\nu$ is weakly
convergent to a function $g$ in $H_m (\Omega)$. Also $f^h \rightarrow
\dfrac{\partial f}{\partial x_1}$ in $H_0(\Omega)$. This implies that
$\dfrac{\partial f}{\partial x_1} \in H_m (\Omega)$. Similarly\pageoriginale we can
show that $\dfrac{\partial f}{\partial x_i}, i \geq 2$ are in
$H_m(\Omega)$. Hence it follows from Proposition \ref{chap3:sec5:prop1}, \S\ 5, that $f$ is
$(m+1)$ times strongly differentiable, since $f$ has compact support,
$f \in H_{m+1} (\Omega )$. 

\begin{proposition}\label{chap3:sec7:prop2}% proposition 2
  Suppose $f \in H_m (\Omega )$ and for a  given $r$, $0 < r \leq m$,
  there exists a constant $C$ such that $| Q (f, \varphi) | \leq C |
  \varphi |_{m-r}$ for any $\varphi \in C^{\infty, q}_0(\Omega)$; then
  $f$ is $(m+r)$ times strongly differentiable.  
\end{proposition}

\begin{proof}% proof
  We shall prove the proposition by induction.
\end{proof}

\noindent
\textbf{Case $r = 1$.} Suppose that 
$$
| Q (f, \varphi ) | \leq C | \varphi |_{m-1}.
$$
We assert that for any $\eta \in C^{\infty, 1}_0$, there is a constant
$C'$ such that  
$$
| Q (\eta f, \varphi ) | \leq C' | \varphi |_{m-1};
$$
the case $r = 1$ of Proposition \ref{chap3:sec7:prop2} then follows from Proposition
$1$. To prove the existence of $C'$, we note that  
\begin{align*}
  (A_i ~ \eta ~ f, B_i \varphi) & = (\eta ~ A_i ~ f, B_i \varphi ) +
  (A' f, B_i \varphi) 
\end{align*}
[where $A'$ has order $\leq m-1$ and has coefficients with compact
  support]
\begin{align*}   
  & = (A_i f, B_i \eta \varphi ) + (A' f, B_i \varphi ) + (A_i f, B' \varphi )
\end{align*}
where $B'$ has order $\leq m-1$. Clearly
$$
| (A_i f, B' \varphi ) | \leq C'' | \varphi |_{m-1}.
$$

Now we can write $B_i =\sum\limits_{k} ~ D_k ~ B''_k$ where $B''_k$
have order $\leq m-1$, $D_k$ has ordered. 

Since\pageoriginale $f \in H_m (\Omega )$, we then have,
$$
| A' f, B_i \varphi) | = | \sum (D^{*}_{k} A' f, B''_{k} \varphi ) |
\leq C''  | \varphi |_{m-1}. 
$$

Hence 
$$
Q(\eta f, \varphi ) = Q(f, \eta \varphi ) + 0 (| \varphi |_{m-1})
$$
and the result follows.

Let us now suppose that the result is proved for $r = k -1 > 0$; then
$f$ is $(m+k-1)$ times strongly differentiable; by restricting
ourselves to $\Omega' \subset \subset\Omega$, we may then suppose that $f \in
H_{m+k-1} (\Omega )$. Let $| \beta | = 1$. Now since $f \in H_{m+k
  -1}(\Omega)$, we have 
\begin{align*}
  Q(D^\beta f, \varphi ) & = \sum (A_i D^\beta f, B_i \varphi )\\
  & = \sum (D^\beta A_i ~ f, B_i \varphi ) + \sum (A'_i ~ f, B_i \varphi )
\end{align*}
where the $A'_i$s are differential operators of order $\leq m$. Hence
\begin{align*}
  Q(D^\beta f, \varphi ) & = - \sum (A_i ~ f, D^\beta  B_i \varphi ) +
  \sum (A'_i f, B_i \varphi )\\ 
  & =  - \sum (A_i f, B_i D^\beta \varphi ) + \sum (A''_i f, B''_i \varphi )
\end{align*}
where $A''_i$ and $B''_i$ are differential operators such that
ord. $A''_i \leq m+k -1$ ord. $B''_i \leq m -k +1$ [for this last
  equality, write $B_i ~ D^\beta - D^\beta ~ B_i$ as a linear
  combination $\sum L_j ~ L'_j$ where ord. $L_j \leq k -1$, ord. $L'_j
  \leq m-k+1$ and use the fact that $f \in H_{m+k-1}(\Omega)$, to
  shift $L_j $ to $A_i$]. Since 
$$
  | Q (f, \varphi) | \leq C | \varphi |_{m-k},
$$
this\pageoriginale gives 
$$
  \left| \sum (A_i D^\beta f, B_i \varphi )\right| \leq C | D^\beta \varphi
  |_{m-k} + C_1 | \varphi |_{m-k+1} 
$$
for some constant $C_1$.

Hence 
$$
| Q (D^\beta ~ f, \varphi ) | \leq C_2 | \varphi |_{m-k+1}
$$
and by the induction hypothesis, $D^\beta ~ f$ is $(m+k-1)$ times
strongly differentiable. By Proposition \ref{chap3:sec5:prop1} of \S\
5 this implies that $f$ is $(m+k)$ times strongly differentiable. 

\begin{proposition}\label{chap3:sec7:prop3}% proposition 3
  If $f \in H_m (\Omega )$ and $Lf$ is $r$ times strongly
  differentiable, them $f$ is $(2m + r)$ times strongly
  differentiable. 
\end{proposition}

\begin{proof}% proof
  We shall prove the proposition by induction; by restricting
  ourselves to $\Omega' \subset\subset \Omega$, we may suppose that $Lf \in
  H_r (\Omega )$. Let $Lf = g \in H_0 (\Omega )$; then for $\varphi
  \in C^{\infty, q}_0 (\Omega )$, by definition, 
  $$
  \displaylines{\hfill 
  Q(f, \varphi) = (f, L^* \varphi) = (g, \varphi ); \hfill \cr
  \text{so that} \hfill  
  | Q (f, \varphi ) | \leq C | \varphi |_0 \text{ for some constant }
  C > 0.\hfill }
  $$
\end{proof}

Now using Proposition \ref{chap3:sec7:prop2} with $r = m$, we conclude that $f$ is $2m$
times strongly differentiable. Let us suppose that proposition is true
for $r = k$. Then if $Lf \in H_{k+1}(\Omega)$, by the induction
hypothesis, $f$ is $(2m+k)$ times strongly differentiable. For $|
\beta | \leq k + 1$,  

$L(D^\beta) - D^\beta L = \triangle_\beta$ is a differential operator
of order $\leq 2m + k$, and since  $f$ is $(2m+k)$ times strongly
differentiable, we have 
$$
LD^\beta f = D^\beta (Lf) + \triangle_\beta f \text { and } D^\beta
(Lf) + \triangle_\beta f \in H_0 (\Omega'). 
$$

Therefore\pageoriginale by what we have proved above $(D^\beta f)$ is $2m$ times
strongly differentiable. This, together with
Proposition \ref{chap3:sec5:prop1}, \S\ 5
implies that $f$ is $(2m + k +1)$ times strongly differentiable.

\begin{proposition}\label{chap3:sec7:prop4} % prop 4
  Let $\triangle$ denote the operator $\sum\limits_{i = 1}^n
  \dfrac{\partial^2}{\partial x^2_i}$ acting on $q$-tuples of $C^\infty$
  functions. If $\varphi \in H_0 (\mathbb{R}^n)$ and $r \geq 1$ is an
  integer, there exists $\varphi' \in H_{2r} (\mathbb{R}^n)$ such that
  $(I - \triangle )^r \varphi' = \varphi$, $I$ being the identity.  
\end{proposition}

\begin{proof}
  By Plancherel's theorem $\hat{\varphi} \in H_0
  (\mathbb{R}^n)$. Define $\varphi'$ by  
  $$
  \hat{\varphi'} (\xi) = \frac{\hat{\varphi} (\xi)}{(1+ \xi^2_1 +
    \cdots + \xi^2_n)^r} \in H_0 (\mathbb{R}^n). 
  $$
\end{proof}

Then 
$$
\int\limits_{\mathbb{R}^n} (1+ | \xi |^2)^{2r} | \hat{\varphi'} (\xi )
|^2 d \xi = | \hat{\varphi }|_0 < \infty. 
$$

Hence by Proposition \ref{chap3:sec5:prop4}, \S\ 5, $\varphi' \in H_{2r}$. Moreover using
the fact that $D^{\hat{\alpha}} f (\xi) = i^{| \alpha |} \xi^\alpha
\hat{f} (\xi)$ and the inversion formula we see immediately that  
$$
(I - \triangle )^r \varphi' (\xi ) = \varphi (\xi ).
$$
In the next theorem, $L$ need no longer have the properties stated at
the beginning. 

\begin{theorem*}[(The regularity theorem)]
  If $L$ is an elliptic differential operator of order $m$ with
  $C^\infty$ coefficients, $L$: $C^{\infty, q}_0 (\Omega ) \to
  C^{\infty, p}_0 (\Omega )$ and for an $f \in H_0 (\Omega )$, $Lf = g
  $ is in $H_0 (\Omega)$, and if $g \in C^\infty$, so is $f$. 
\end{theorem*}

\begin{proof}
  Let\pageoriginale $A = (-1)^m L^* \circ L$; by restricting ourselves to $\Omega'
  \subset \subset \Omega$, we may suppose that $A$ is uniformly
  strongly elliptic and $A$: $C^{\infty, q}_0 \to C^{\infty,
    q}_0$. Then we shall prove that for any $f \in H_0 (\Omega)$, if
  $A f \in C^\infty$, then $f \in C^\infty$. Since $(-1)^m L^*o L f =
  (-1)^m L^* (Lf) \in C^\infty$, this will imply the theorem. We may
  extend $f$ to $\mathbb{R}^n$ by $f(x) = 0$ for $x \notin
  \Omega$. Let $r$ be a positive integer, $r \geq m$. Then by
  Proposition \ref{chap3:sec7:prop4}, there exists a $q$-tuple $f^{(r)} \in H_{2r}
  (\mathbb{R}^n)$ such that if $\triangle = \sum\limits_{i = 1}^n
  \dfrac{\partial^2}{\partial x^2_i}$, $(I - \triangle) ~ f^{(r)} =
  f$. Consider $B = (-1)^r A(I - \triangle)^r$. Then $B$ is uniformly
  strongly elliptic and is of order $2 (m + r) $; further  
  $$
  B ~ f^{(r)} = (-1)^r A ~ f \in C^\infty.
  $$
\end{proof}

Since $r \geq m$, $f^{(r)} \in H_{m + r}$ and hence, by Proposition
\ref{chap3:sec7:prop3}, $f^{(r)}$ is $2m + 2r + s$ strongly
differentiable for any $s > 
0$. Hence $f$ is $2m + s$ times strongly differentiable for any $ s >
0$. 

It follows from the corollary to Sobolev's lemma that $f$ has
continuous derivatives of order $\leq 2 m + s - n$ for any $s$, and
hence $f \in C^\infty$.  

\begin{remark*}
  We have in fact proved the following proposition in the case when
  $L$ is strongly elliptic of even order. 
\end{remark*}

\begin{proposition}\label{chap3:sec7:prop5} % prop 5
  Let $L$ be an elliptic operator of order $m$, and $ f \in H_0
  (\Omega )$. If $Lf$ is $r$ times strongly differentiable, $r$ being
  an integer $\geq 0$, then $f$ is $r + m$ times strongly
  differentiable. 
\end{proposition}

The above proposition, for arbitrary $L$ can be reduced, to the case
of strongly elliptic operators of even order by considering
$\triangle_1 = L^* L$, if $r \geq m$ (= order of $L$). The general
case requires the use of the space $H_{-k} (\Omega)$\pageoriginale which is the
dual of $\overset{o}{H}_k (\Omega ), k > 0$. We do not enter into the
details. 

The proof of the regularity theorem given here is a somewhat
simplified version of that of Nirenberg \cite{32}. There are now several
other proofs available. The oldest, which operates with ``fundamental
solutions'' was proposed by $L$. Schwartz \cite{39}; very strong theorems
that can be obtained by this method will be found in H\"ormander
\cite{17}. The first proof using only `a priori' estimates is due to
Friedrichs \cite{10} (who proves, however, only a slightly weaker
assertion). Other proofs are due to $F$. John \cite{19} and $P$. Lax
\cite{24} ; that of Lax is both brief and elegant. Schwartz has recently
given another very elegant and very general proof, which operates,
however, with singular integral operators; see \cite{40a}. There is a
vast literature that has sprung up around this theorem and its
generalizations (particularly the so called ``regularity at the
boundary''). References may be found in \cite{1}. 

\section{Elliptic operators with analytic coefficients}\label{chap3:sec8}

\setcounter{lemma}{0}
\begin{lemma}\label{chap3:sec8:lem1} % lem 1
  If $K$ is a compact set in $\mathbb{R}^n$ and if $K_ \varepsilon =
  \{x | d (x, K) < \varepsilon\}$, there exists $\varphi_ \varepsilon
  \in C^{\infty, 1}_0 (\mathbb{R}^n)$ such that $ 0 \leq \varphi_
  \varepsilon \leq 1$, $\varphi_ \varepsilon (x) = 1$ for $x
  \epsilon K$, $\supp. \varphi_ \varepsilon \subset K_{2
    \varepsilon} $ and $| D^\alpha \varphi_ \varepsilon | \leq
  \dfrac{C_\alpha}{\varepsilon^{| \alpha |}}$ for some constants
  $C_\alpha $ independent of $\varepsilon$ and $K$. 
\end{lemma}

\begin{proof}
  Let $\varphi$ be a $C^\infty$ function such that
  $\int\limits_{\mathbb{R}^n} \varphi (x) dx = 1$ and $\varphi \geq
  0$, $\supp. \varphi \subset \{x | || x || < 1 \}$. Let $\chi_
  \varepsilon (x) = 1$ for $x \in K_\varepsilon$ and $\chi_
  \varepsilon (x) = 0$ for $x \notin K_\varepsilon$. 
\end{proof}

Let $\varphi_ \varepsilon (x) = \varepsilon^{-n} \int \varphi
\left(\dfrac{x-y}{\varepsilon}\right)$. $\chi_ \varepsilon (y) ~ dy$. 

Then\pageoriginale clearly $\varphi (x)  = 1$ for $x \in K$ and $\supp. \varphi_
\varepsilon \subset K_{2 \varepsilon}$. Also 
$$
D^ \alpha \varphi_ \varepsilon (x) = \varepsilon^{- | \alpha|}
\varepsilon^{-n} \int D^ \alpha \varphi \left(\frac{x-y} {\varepsilon}\right)
\chi_ \varepsilon (y) dy. 
$$

Hence $| D^ \alpha \varphi_\varepsilon (x) | \leq \varepsilon^{-|
  \alpha |} \int | D^ \alpha \varphi (y) | dy$, which proves the
lemma. 

\noindent
\textbf{Notation.} In what follows, $R$, $\rho$ are real numbers, $0 <
\rho < \min \{1, R\}$, and $M_ \rho (f)$ is given by  
$$
[M_ \rho (f)]^2 = \int\limits_{|x| < R - \rho} | f(x) |^2  ~ dx.
$$

\setcounter{proposition}{0}
\begin{proposition}\label{chap3:sec8:prop1}% proposition 1
  Let $L$ be an elliptic operator of order $m$, with $C^ \infty$
  coefficients, on $\{x \Big| |x| < R + \delta \}$. Then there exists a
  constant $C$ (independent of $f$, $\rho$, $\rho_1$) such that for
  $\rho$, $\rho_1 > 0$ and $f \in C^{\infty, q}$, we have 
  $$
  \rho^m M_{\rho + \rho_1} (D^\alpha f) \leq C \left\{ \rho^m M \rho_1
  (Lf) + \sum_{|\beta| < m} \rho^{|\beta|} M \rho_1 (D^{\beta}
  f)\right \}. 
  $$
  for $|\alpha| = m$.
\end{proposition}

\begin{proof}
  By Lemma \ref{chap3:sec8:lem1} above there exists a $C^ \infty$ function $\varphi$ on
  $\mathbb{R}^n$ such that $\varphi(x)  = 1$ for $| x | < R - \rho -
  \rho_1$, $0 \leq \varphi (x) \leq 1, \supp. \varphi \subset \{ x
  \Big| |x| < R - \rho_1 \}$ and $| D^ \alpha \varphi| \leq
  \dfrac{C_\alpha}{\rho^{|\alpha|}}$ where $C_\alpha$ are constants
  independent of $\rho$ and $\rho_1$. 
  
  By Friedrichs' inequality, (Part III), there exists a constant
  $C_1$, independent of $f$ and $\rho_1$ such that 
  $$
  |D^ \alpha \varphi f |_0 \leq C_1 \{ | L (\varphi f)|_0 + | \varphi f|_0 \}.
  $$
\end{proof}

Let\pageoriginale 
$$
L = \sum_{|\lambda| \leq m} a_\lambda D^\lambda.
$$ 

Then
$$
L(\varphi f ) = \phi Lf + \sum_{\substack{\beta < \lambda \\ | \lambda
    | \leq m}} a_\lambda \binom{\lambda}{\beta} D^ {\lambda - \beta}
(\varphi ) D^ \beta f . 
$$

Since $a_ \lambda$ are $C^ \infty$ in $| x | < R + \delta$ and $| D^
\alpha \varphi | \leq \dfrac{C_ \alpha}{\rho |\alpha|}$, there exist
constants $C_{\lambda , \beta}$, independent of $\rho$ such that 
$$
| a_ \lambda \binom{\lambda}{\beta} D^ {\lambda - \beta} \varphi | \leq
\frac{C_ {\lambda, \beta}}{\rho^{|\lambda- \beta|}} ~ \text{ for } ~
|x| \leq R. 
$$

Hence
$$
|D^ \alpha \varphi f |_0 \leq C_2 \left\{ M \rho_1 (Lf) + \sum_{|\beta| <
  m} \beta^{-m+ |\beta|} M \rho_1 (D^ \beta f) \right\} 
$$
for a constant $C_2$ and this proves the proposition.

\begin{proposition}\label{chap3:sec8:prop2}%proposition 2
  Let $\Omega$ be an open set in $\mathbb{R}^n$, $0 \in \Omega$, and
  let $L$: $C^ {\infty , q} (\Omega ) \to C^{\infty, q} (\Omega)$ be
  an elliptic operator of order $m$ with coefficients which are
  analytic in $\Omega$ (note that $L = \sum\limits_{|\lambda| \leq m}
  a_\lambda D^ \lambda$ where $a_ \lambda$ are $q \times q$ matrices
  of analytic functions). Then if $R > 0$ and $R_1 > R$ are
  sufficiently small there exists a constant $A > 0$ such that for all
  $\rho$, $0 < \rho < \min (1, R)$, and $f \in C^{\infty , q} (\Omega
  )$, we have, for $| \alpha | \leq mr$, $r = 1, 2, \ldots$, 
  \begin{equation}
    \rho^{| \alpha |} M_{|\alpha | \rho} (D^\alpha f) \leq A^{|\alpha
      | + 1} \left\{ \sum^r_{s= 1} |L^s f |_0^{R_1} \rho^{(s-1)m} + |
    f|^{R_1}_0 \right\}; \tag{8.1}\label{chap3:sec8:eq8.1} 
  \end{equation} 
  here $|f|^{R_1}_0 = \int \limits_{|x|< R_1} | f(x) |^2 dx$; $L^s$
  denotes the iterate of $L$, $s$ times. 
\end{proposition}

\begin{proof}
  We\pageoriginale choose $R_1$ so small that the $a_\lambda$ have holomorphic
  extensions to the polycylinder $|z| \leq R_1$. Let $C_1 =
  \sum_{\substack{|\lambda|\leq m\\ | z | \leq R_1}} \sup . |a_
  \lambda (z) | $; then we have (Cauchy's inequality) 
  \begin{equation}
    \sum_{| \lambda | \leq m} | D^ \alpha a_ \lambda (x) | \leq C_1
    \alpha ! \rho^{-| \alpha |} ~\text{for} ~ |x| \leq R - \rho
    . \tag{8.2}\label{chap3:sec8:eq8.2} 
  \end{equation}
\end{proof}

Let
$$
\sum^r_{s=1} | L^s f |^{R_1}_0 \rho^{(s-1)m} + | f|^{R_1}_0 = S_r (f).
$$
We first remark that 
\begin{equation}
\rho^m S_r (Lf) \leq S_{r+1} (f). \tag{8.3}\label{chap3:sec8:eq8.3}
\end{equation}

We shall prove the proposition by induction. For $r = 1$, i.e. for
$|\alpha| \leq m$, we apply Friedrichs' inequality, Part
$IV$. There-exists a constants $C_2$ such that 
$$
M_0 (D^ \alpha f) \leq C_2 \{ |L f|^{R_1}_0 + |f|^{R_1}_0 \} ~
\text{for} ~ |\alpha| \leq m. 
$$

Hence
$$
\rho^{|\alpha|} M_{| \alpha | \rho} (D^\alpha f) \leq C_2 \{
|Lf|^{R_1}_0 + | f |^{R_1}_0 \}, 
$$
so that (\ref{chap3:sec8:eq8.1}) is true for $r = 1$ if $A \geq C_2$. Now let $mr < |
\alpha| \leq m (r+1)$, $ r \geq 1$, and assume that (\ref{chap3:sec8:eq8.1}) is already
proved for all $\beta$ with $| \beta | < | \alpha |$. 

Let $\alpha = \alpha_0 + \alpha '$ where $| \alpha_0 | = m$. Then we
have by Proposition $1$ with $\rho_1 = (| \alpha| - 1) \rho$,
(and $\alpha_0$ in place of $\alpha$)  
\begin{align*}
&\rho^{|\alpha |} M_{|\alpha | \rho} (D^ \alpha f) \leq\\ 
&C \left\{
  \rho^{|\alpha |} M_{(|\alpha | - 1) \rho} (LD^{\alpha '} f) +
  \sum_{|\beta | < m} \rho^{|\beta | + |\alpha ' |} M_{(|\alpha | -
    1)_\rho} \left(D^{\beta + \alpha'} f\right) \right\}
  \tag{8.4}\label{chap3:sec8:eq8.4}  
\end{align*}

Further,\pageoriginale
$$
D^{\alpha'} Lf = L D^{\alpha'} f + \sum_{|\lambda | \leq m}
\sum_{\gamma < \alpha '} (^{\alpha '}_{\gamma}) D^{\alpha - \gamma} a_
\lambda D^{\gamma + \lambda} f. 
$$

Now, for $|x| \leq R - mr \rho$, we have
$$
|D^{\alpha ' - \gamma} a_{\lambda}(x) \leq C_1 (\alpha ' - \gamma) !
(\rho mr)^{-|\alpha' - \gamma |}. 
$$
and $\binom{\alpha '}{\gamma} \dfrac{(\alpha ' - \gamma )!} {|\alpha ' -
  \gamma |} \leq \left(\dfrac{|\alpha' |}{mr}\right)^{|\alpha' - \gamma |}
\leq1$ since $|\alpha ' | = | \alpha | - m \leq mr$. Hence for $|x|
\leq R - mr \rho$, a fortiori for $|x| \leq R - (| \alpha | - 1)
\rho$, we have, 
\begin{equation}
  | D^{\alpha '} Lf - LD^{\alpha '} f | \leq  C_1 \sum_{|\lambda |
    \leq m} \sum_{\gamma < \alpha'} \rho^{-|\alpha ' - \gamma |} |
  D^{\gamma + \lambda} f |. \tag{8.5}\label{chap3:sec8:eq8.5} 
\end{equation}

Hence, by (\ref{chap3:sec8:eq8.4}) and (\ref{chap3:sec8:eq8.5}), we
have for $mr < | \alpha | \leq m(r+1)$, 
\begin{multline*}
  \rho^{|\alpha |} M_{|\alpha | \rho} (D^\alpha f) \leq C \left\{
  \rho^{|x|} M_{|\alpha '| \rho} (D^{\alpha '} Lf) + \sum_{|\beta | <
    m} \rho^{|\beta| + \alpha '} M_{|\beta + \alpha ' | \rho}
  (D^{\beta + \alpha '} f)\right.\\   
  \left.+ C_1 \sum_{| \lambda | \leq m}
  \sum_{\gamma < \alpha '} \rho^{m + | \gamma |} M_{(m + |\gamma |
    )\rho} (D^{\gamma + \lambda} f) \right\}. \tag{8.6}\label{chap3:sec8:eq8.6} 
\end{multline*}

We can apply our induction hypothesis to each of the three terms in
brackets on the right. 

The first term is $\leq \rho^m A^{|\alpha ' | + 1} S_r (L f) \leq
A^{\alpha ' | + 1} (f)$.  

The second $\leq \sum\limits_{|\beta | < m} A^{|\beta +  \alpha '|/+1}
 S_{r+1} (f)$ and similarly for the third. This gives 
$$
\rho^{|\alpha |} M_{|\alpha| \rho} (D^ \alpha f) \leq A^{|\alpha | +
  1} S_{r+1} (f) \left\{ \frac{C}{A^m} + C \sum_{|\beta | \leq m}
\frac{1}{A} + C'_1  \sum_{\gamma < \alpha '} A^{- |\alpha' - \gamma |}
\right\}. 
$$

Now\pageoriginale
$$
\sum_{\gamma < \alpha'} A^{-|\alpha ' - \gamma |} = \sum_{0 < \beta
  \leq \alpha '} A^{-|\beta |} \leq A^{-1} \sum_{|\beta |  \geq 0}
A^{-|\beta |}  
$$
which clearly $\to 0$ as $A \to \infty$. Hence we can choose $A \geq
C_2$ so large that 
$$
\frac{C}{A^m} + C \sum_{|\beta | \leq m} \frac{1}{A} + C'_1
\sum_{\gamma < \alpha '} A^{-|\alpha ' - \gamma |} < 1 , 
$$
which gives us (\ref{chap3:sec8:eq8.1}).

\setcounter{theorem}{0}
\begin{theorem}[T. Kotake -M.S. Narasimhan]\label{chap3:sec8:thm1} %theorem 1
  Let $L$: $C^{\infty, q} (\Omega ) \to C^{\infty, q} (\Omega )$ be an
  elliptic operator of order $m$ with analytic coefficients. If $f \in
  C^{\infty , q} (\Omega )$ and for any $\Omega ' \subset \subset
  \Omega$, there exists a constant $M > 0$, such that 
  $$
  |L^r f |^{\Omega'}_0 \leq M^{r+1} (rm) ! ,
  $$
  then $f$ is analytic in $\Omega$.
\end{theorem}

\begin{proof}
  We may suppose that $0 \in \Omega$; it suffices moreover to show
  that $f$ is analytic in a neighbourhood of $0$. We choose $R_1$ such
  that Proposition \ref{chap3:sec8:prop1} is true; we the have 
  $$
  \displaylines{\hfill 
  |L^r f|^{R_1}_\circ \leq M^{r+1} (rm) !\hfill \cr
  \text{so that}\hfill 
  S_r (f) \leq \sum^r_{s = 1} \rho^{(s-1)m} M^{s+1} (sm)! + M.\hfill }
  $$

  If\pageoriginale $(r-1)m < | \alpha | \leq$ rm, we choose $\rho = \dfrac{c}{|\alpha
    |}$, where $c$ is small. Since, then $(sm) ! \rho^{(s-1)m} \leq
  (rm)^{2m}$ for $s \leq r$, we conclude that $S_r (f) \leq B^{r+1}_1 $
  for a suitable constant $B_1$. By Proposition \ref{chap3:sec8:prop2}, this implies that $
  |f|_k^{R-c} \leq B^{k+1}_2 k^k$; if $K$ is a compact subset of the set
  $|x| < R-c$, it follows from (the weak form of) Sobolev's lemma that 
  $$
  \sup_{x \in K} |D^\alpha f(x) | \leq B_3 B^{k+n+1}_2 (k+n)^{k+n} ~
  \text{ if } ~ | \alpha | = k. 
  $$
\end{proof}

Stirling's formula shows then that
$$
\sup_{x \in K} | D^\alpha f (x) | \leq  B^{k+1}_4 k!   \text{ if } |
\alpha | = k. 
$$
so that $f$ is analytic in $| x | < R - c$ by Chap I, \S\ 1.

\begin{lemma}\label{chap3:sec8:lem2}%lemma 2
  Let $L$ be any differential operator of order $m$, with coefficients
  which are holomorphic $q \times q$ matrices on $D = \{ z \in
  \mathbb{C}^n \big| |z_i | < r_i \leq 1 \}$ and let $f$ be a bounded
  holomorphic map $D \to \mathbb{C}^q$. Then there exists a constant
  $A$ such that 
  $$
  | L^r f(z) | \leq \frac{(3A)^{r+1} (mr)!}{\prod (r_i - | z_i
    |)^{mr}} ~ \text{ for } ~ z \in D. 
  $$
\end{lemma}

\begin{proof}
  We shall prove the lemma by induction. For $r =0$, the lemma is
  trivial. Assume that it is true for $r = k-1$. Then 
  $$
  | L^{k-1} f(z) | \leq \frac{(3A)^{k} \{m(k-1)\}!}{\prod (r_i - | z_i
    |)^{m(k-1)}} ~ \text{ for } ~ z \in D. 
  $$
\end{proof}

Let $\sum\limits_{|\alpha | \leq m} |a_\alpha (z) | \leq A$, where $L
= \sum a_\alpha D^\alpha$. We have, by Lemma \ref{chap3:sec3:lem3}, \S\ 3, 
$$
| D^\alpha L^{k-1} f(z)| \leq \frac{3(3A)^{k} (mk)!}{\Pi (r_i - |
  z_i|)^{m(k-1) + | \alpha |}} 
$$
and\pageoriginale since $\sum\limits_{| \alpha | \leq m} | a_\alpha (z) | \leq A$ on
$D$, this implies that 
$$
| L^k f(z) | \leq \frac{(3A)^{k+1} (mk)!}{\Pi (r_i - | z_i |)^{mk}}.
$$

\begin{theorem}[Petrovsky]\label{chap3:sec8:thm2}%theorem 2
  If $L$ is an elliptic operator of order $m$, with analytic
  coefficients on $\Omega$, and if $Lf$ is analytic, then $f$ is
  analytic. 
\end{theorem}

\begin{proof}
  By replacing $L$ by $L^* L$ if necessary, we may suppose that $L$ is
  an operator $C^{\infty , q} (\Omega ) \to C^{\infty, q} (\Omega )$
  with analytic coefficients. Let $L = \sum\limits_{|\alpha | \leq m}
  a_\alpha D^\alpha$. We may assume that $\Omega = \{ x \big| |x_i | <
  r_i \}$ and that $a_\lambda$ and $Lf$ extend to holomorphic functions
  on $D = \{ z \in \mathbb{C} \big| |z_i | < r_i \}$. Let $g =
  Lf$. Then it follows from Lemma \ref{chap3:sec8:lem2} that for any compact subset $K$
  of $\Omega$, there exists a constant $M$ such that 
  $$
  | L^r f |^K_0 \leq M^{r+1} (mr) !
  $$
  Theorem \ref{chap3:sec8:thm2} then follows from
  Theorem \ref{chap3:sec8:thm1}.
\end{proof}

\noindent
\textbf{Note:} We indicate briefly how the proof of
Theorem \ref{chap3:sec8:thm1}
simplifies in the special case needed for Theorem \ref{chap3:sec8:thm2}. We use
inequalities (\ref{chap3:sec8:eq8.4}) and
(\ref{chap3:sec8:eq8.5}). But now since $g = Lf$ is analytic, we 
apply the Cauchy inequalities to a holomorphic extension of $g$ and
conclude that 
$$
\displaylines{\hfill 
  | D^\alpha Lf | \leq \frac{C_3 \alpha !}{(mr \rho )^{|\alpha |}}
  \text{ in } |x| \leq R - mr \rho \hfill \cr
  \text{so that} \hfill 
  \rho^{|\alpha '|} M_{(|\alpha| - 1) \rho} (D^{\alpha '} Lf) \leq
  C_4;\hfill }
$$
this\pageoriginale leads easily to the estimate

$\rho^{|\alpha |} M_{|\alpha | \rho} (D^\alpha f) \leq A^{|\alpha |
  +1}$ for all $\alpha$, (A now depends on $f$) and the proof is
completed as before. The point is that one  does not need the somewhat
complicated Part $IV$ of Friedrichs' inequality. The main theorem of
this section (Theorem \ref{chap3:sec8:thm2}) is a special case of results of Petrovsky
\cite{36} who considered also non-linear systems of differential
equations. His proof is however very difficult. The main idea in the
proof given here is contained in the paper of Morrey-Nirenberg
\cite{29}. The proof by Koteke-Narasimhan \cite{23} of Theorem $1$ involves
more careful analysis although it is also based on the idea of
Morrey-Nirenberg. 

\section{The finiteness theorem}\label{chap3:sec9} 

Let $V$ be an oriented $C^\infty$ manifold, $E$, $F$, $C^\infty$
vector bundles of ranks $q$, $p$ respectively over $V$. Let $L$ be a
differential operator $L: C^\infty_0 (V, E) \to C^\infty (V, F)$. All
coordinates systems considered will be assumed to be
\textit{positive}. 

\begin{defi*}
  \begin{enumerate}[(1)]
  \item The order of $L$ at a point $a \in V$ is the largest integer
    $m$ such that $L(F^m s) (a) \neq 0$ for some $f \in m^\infty_a$
    and some section $s \in C^\infty_0(V, E)$. 
  \item The order of $L$ on $V$ is defined to be $\max \limits_{a \in
    V}$ order of $L$ at $a$. 
  \item A differential operator $L$ of order $m$ is said to be
    elliptic if, for $a \in V$, and every (real valued) $f \in
    m^\infty_a$ such that\pageoriginale $(df) (a) \neq 0$, we have $L(f^m s) (a)
    \neq 0$ for every $s \in C^\infty_0 (V, E)$ for which $s(a) \neq
    0$. 
  \end{enumerate}
\end{defi*}

Note that if $f \in m^\infty_0$ and $s(a) = 0$, then $L(f^m s)(a) =
0$. Further if $(df)(a) = 0$, $L(f^m s) (a) = 0$ for any $s$. Hence
$L(f^m s)(a)$ defines a map (not linear) from $E_a \otimes T^*_a (V)
\to F_a$; this gives rise to a $C^\infty$ map $\sigma (L)$: $E \otimes
T^* (V) \to F$ (which preserves fibres). This map is called the
\textit{symbol of} $L$ (and replaces the characteristic polynomial
which we considered earlier). 

\begin{remarks}
  \begin{enumerate}[(1)]
  \item We shall prove that the definition (2) above is consistent
    with the definition (1) of \S\ 6. Let $E$, $F$ be trivial and for
    $a \in V$, let $U_a$ be a coordinate neighbourhood of $a$ and let
    $L$ be given by $L = \sum\limits_{|\alpha | \leq m_1} a_\alpha
    D^\alpha$, $a_{\alpha'} \nequiv 0$ for some $| \alpha'| =
    m_1$. Then it is enough to show that the order of $L$ on $U_a =
    m_1$. But this follows at once from  
    \begin{enumerate}[(i)]
    \item $(D^\alpha f^m) (a) = 0$ for $|\alpha| < m$, if $f \in m^\infty_a$ and
    \item $(D^\alpha f^m) (a) = (m!)(\dfrac{\partial f}{\partial
      x_1})^{\alpha_1} (a) \cdots (\dfrac{\partial f}{\partial
      x_n})^{\alpha_n} (a)$ for $|\alpha| = m$, if $f \in
      m^\infty_a$. 
    \end{enumerate}
  \item If $L$ has order $m$ on $V$, with the same notation as in the
    remark (1), 
  $$
  L(f^m s) (a) = m! \sum_{|\alpha | = m} \xi^\alpha a_\alpha (a) S(a),
  \xi= \left(\frac{\partial f}{\partial x_1}, \ldots , \frac{\partial
    f}{\partial x_n} \right)
  $$
  Hence it follows that the definition (3) above and the definition (3)
  of \S 6 are consistent. 
  \end{enumerate}
\end{remarks}

\begin{examples*}
  \begin{enumerate}[(1)]
  \item Let\pageoriginale $V$ be a $C^\infty$ manifold of dimension $n$, and let
    $\overset{p}{A}$ denote the set of $p$ differential forms on
    $V$. Then $\overset{p}{A} =
    C^\infty (V, E^p)$, where $E^p$ is a vector bundle of rank
    $(^n_p)$ over $V$ for which the fibre at $a \in V$ is $\overset{p}{\wedge}
    T^*_a (V)$. The exterior differentiation $d : \overset{p}{A} \to
    \overset{p}{A} p+1$ is
    a differential operator of degree $1$. If $p = 0$, for $f \in
    m^\infty_a$ and $g \in \overset{\circ}{A}$, we have 
    $$
    d(fg) (a) = (df) (a) g(a) + f(a) (dg) (a).
    $$

    Hence if $(df) (a) \neq 0, d(fg) (a) \neq 0$ whenever $g(a) \neq
    0$ i. e. $d : \overset{\circ}{A} \to \overset{1}{A}$ is elliptic. 
  \item Let $V$ be a complex manifold of complex dimension $n$ and let
    $\varepsilon^{p, q}$ denote the set of all differential forms of
    type $(p, q)$. Then $\varepsilon^{p, q} = C^{\infty} (V, E^{p,
      q})$, where $E^{p, q}$ is a vector bundle of rank $(^n_p)
    (^n_q)$ over $V$; [$ E^{p, q}$ is a bundle whose fibre over $a \in
      V$ is the space $\varepsilon^{p, q}_a$ of complex convectors of
      type $(p, q)$ at $a$]. 
  \end{enumerate}
\end{examples*}

Clearly $\bar{\partial}$: $\varepsilon^{p, q} \to \varepsilon^{p,
  q+1}$ is a differential operator of order 1. Let $q = 0$, $ f \in
m^\infty_a$ and $(df) (a) \neq 0$. Since $f$ is real valued, we have 
$$
(df) (a) = \overline{(\bar{\partial} f)} (a) + (\bar{\partial} f) (a)
$$
and hence $(df) (a)  = \phi$ implies $(\bar{\partial} f) (a) \neq
0$. If $g \in \varepsilon^{p, 0}$ and $g(a) \neq 0\, \bar{\partial} (fg)
(a) = (\bar{\partial} f) (a) g(a) \neq 0$, since $g(a)$ is of type
$(p, 0)$. i.e. $\bar{\partial}$: $\varepsilon^{p, 0} \to
\varepsilon^{p, 1}$ is elliptic. 

In what follows, $\overset{n}{A}(V)$ is the (complex) line bundle $\wedge^n
\mathcal{J}^* (V)$, where $\mathcal{J}^* (V)$ is the bundle of complex
covectors\pageoriginale on $V$ i. e. for $a \in V$, $\mathcal{J}^*_a (V) = T^*_a (V)
\otimes_{\mathbb{R}} \mathbb{C}$ and $E'$ is the vector bundle on $V$,
given by 
$$
E' = E^* \otimes A^n (V).
$$

Since $E' = E^*_a \otimes A^n (V)$, we have a map $\eta$: $E_a \times
E'_a \to A^n_a (V)$, given by  
$$
\eta (x, y^* \otimes \omega_a) = (x, y^*)\omega_a \in A^n_a (V).
$$

Now for any open subset $U$ of $V$, $\eta$ defines a map (which we
again denote by $\eta$) 
$$
\displaylines{\hfill 
  \eta: \Gamma (U, E) \times \Gamma (U, E') \to \Gamma (U, A^n(V))\hfill
  \cr 
  \text{given by}\hfill 
  \eta (s, s') (a) = \eta (s(a), s'(a)).\hfill }
$$

If one of $s$ and $s'$ has compact support, we define $\langle s$, $s'
\rangle$ by 
$$
\langle s, s' \rangle = \int \limits_V \eta (s, s').
$$

\begin{remarks*}
  \begin{enumerate}[(1)]
  \item Since for all line bundle $D$, $D \otimes D^*$ is
    (canonically) trivial, it follows that 
    \begin{align*}
      (E')' & = E'^* \otimes \overset{n}{A} (V)\\
      & = E \otimes (\overset{n}{A}(V))^* \otimes \overset{n}{A} (V) \simeq E.
    \end{align*}
  \item  If $\tau$: $E \to V \times \mathbb{C}^q$ is an isomorphism of
    $E$ with the trivial bundle and $^{t_{\tau^{-1}}}$: $E^* \to V
    \times \mathbb{C}^q$, the associated isomorphism of the duals and
    if 
    $$
    \tau(x) = a \times (x_1 , \ldots , x_q), ^{t_{\tau^{-1}}} (y^*) =
    a \times (y_1 , \ldots , y_q),  
    $$
    then\pageoriginale $y^* (x) = \sum x_i y_i$. We shall also write $\tau (x)$ for
    the projection $(x_1 , \ldots , x_q)$ of $\tau(x)$ on
    $\mathbb{C}^q$. 
  \end{enumerate}
\end{remarks*}

\setcounter{lemma}{0}
\begin{lemma}\label{chap3:sec9:lem1}  %lemma 1
  If $L$ is differential operator $C^\infty_0 (V, E) \to C^\infty (V,
  F)$ then there exists a unique differential operator 
  $$
  L' : C^\infty_0 (V, F') \to C^\infty (V, E'), \text{ such that }
  $$
  \begin{equation}
    \langle s, L' \sigma \rangle = \langle Ls, \sigma \rangle ~ \text{
      if }~ s \in C^\infty (V, E) \sigma \in C^\infty_0 (V,
    F'). \tag{9.1}\label{chap3:sec9:eq9.1}    
  \end{equation}
\end{lemma}

\begin{proof}
  It is clear that an operator $L'$, if it satisfies
  (\ref{chap3:sec9:eq9.1}), is local 
  (i.e. $\supp. L' \sigma \subset \supp. \sigma$) and is uniquely
  determined. We have therefore only to prove the existence
  locally. Let $U$ be a positive coordinate neighbourhood with
  coordinates $(x_1, \ldots , x_n)$. We remark that any $\sigma \in F'
  _a = F^*_a \otimes \overset{n}{A}_a (V)$ can be uniquely written as  
  $$
  \sigma = g \otimes (dx_1 \wedge \cdots \wedge dx_n)_a, g \in F^*_a.
  $$
\end{proof}

Suppose now that $\tau_E$: $E_U \to U \times \mathbb{C}^q$, $\tau _F$:
$F_U \to U \times \mathbb{C}^q$ are isomorphisms and $\tau^*_E$: $E^*_U
\to U \times \mathbb{C}^q$ is the transpose inverse. We suppose that,
in terms of the isomorphism $\tau_E$, $\tau_F$, $L$ is written  
$$
L = \sum_{|\alpha | \leq m} a_{\alpha} D^{\alpha} ~\text{on}~ U.
$$

We define $L' \sigma$  by $L' \sigma = \lambda_{\sigma} dx_1 \wedge
\cdots \wedge dx_n$ where $\tau^*_E (\lambda_{\sigma}) = \bar{L}^*
(\tau^*_F (g))$, if $\sigma = g \otimes dx_1 \wedge \cdots \wedge
dx_n$. (For any operator $A = \sum C_{\alpha} D^{\alpha}$, we denote
by $\bar{A}$ the operator $\sum \bar{C}_{\alpha} D^\alpha$.) We have, if
$\sigma \in C^\infty_0 (U, F'), s \in C^\infty (V, E)$, 
\begin{align*}
  \langle Ls, \sigma \rangle  & = \int\limits_U (L \tau_E (s),
  \tau^*_F (g)) dx_1 \wedge \cdots \wedge dx_n\\ 
  & = \int\limits_U ( \tau_E (s), L^* \tau^*_F (g)) dx_1 \wedge \cdots
  \wedge dx_n\\ 
  & = \langle s, L' \sigma \rangle .
\end{align*}

\begin{defi*} % 
The\pageoriginale $L'$ defined by Lemma \ref{chap3:sec9:lem1} is
called the transpose of the operator $L$. 
\end{defi*}

\begin{remarks*}
If rank $E=$ rank $F$ and $L$ is elliptic, then $L'$ is elliptic.
\end{remarks*}

If $p$: $E \to V$ is a vector bundle, in what follows a section $s$:
$V \to E$, is a map $V \to E$ (not necessarily continuous) such that  
$$
p ~ \circ ~ s = \text { identity on } V.
$$

\begin{defi*}
  A section $s$: $V \to E$ is said to be locally in $H_m$ is every
  point $a \in V$ has a coordinate neighbourhood $U$ such that there
  is an isomorphism $\tau$: $E_U \to U \times \mathbb{C}^q$ for which
  $\tau \circ s$ is in $H_m (U)$. 
\end{defi*}

[We may speak of locally measurable, integrable sections in the same
  way.] The theorems proved in \S\ \ref{chap3:sec7}, \S\
\ref{chap3:sec8}, extend to differential 
operators between vector bundles. We state those results that we
need. The proofs are immediate, and the details will be omitted. 

If $L : C^0_0 (V, E) \to C^\infty (V, F)$ is an elliptic differential
operator, then for any locally (square) integrable section $s$ of $E$
on the open set $U \subset V$, $Ls$ denotes the linear functional on
$C^\infty_0 (U, F') $ defined by 
$$
(Ls) (s') = \langle s, L' s' \rangle \text{ for } s' \in C^\infty_0 (U, F').
$$

If\pageoriginale there exists $\sigma$ which is a locally square integrable (or
$C^\infty, \ldots$) section of $F$ on $U$ for which 
$$
(Ls) (s') = \langle \sigma , s' \rangle  \text{ for } s' \in C^\infty_0 (U, F'),
$$
we say that $Ls$ is locally square integrable (or $C^\infty , \ldots$).

\medskip
\noindent
\textbf{Regularity theorem.} \textit{If $L$ is an elliptic operator,
  $L$: $C^\infty_0 (V,\break E) \to C^\infty (V, F)$ and $s$ is a locally
  square integrable section of $E$ such that $Ls$ is $C^\infty$, then
  $s$ is itself $C^\infty$ (i. e. equal almost everywhere to a
  $C^\infty$ section)}. 

\medskip
\noindent
\textbf{Analyticity theorem} \textit{Let $V$ be an analytic manifold,
  $E, F$ analytic vector bundles on $V$ and $L$ an elliptic operator
  from $E$ to $F$ with analytic coefficients (i. e. for any analytic
  section $s$: $U \to E$, $Ls$ is an analytic section $U \to F$). Then
  if $s$ is a locally square integrable section such that $Ls$ is
  analytic, then $s$ is itself analytic}. 

Let $K$ be a compact set in $V$. Then $H_m (K, E)$ denotes the set of
sections $s : V \to E$, which are locally in $H_m$ for which $\supp. s
\subset K$. Let $\mathscr{U} = \{ U_1 ; \ldots , U_h \}$ be a finite
covering of $K$, $U_i$ being coordinate neighbourhoods such that $E$
restricted to a neighbourhood $U'_i$ of $\bar{U}_i$ is trivial. Let
$\tau_i$: $E_{U'_i} \to U'_i \times \mathbb{C}^q$ be isomorphisms. Let
$\varphi_i$ be $C^\infty$ functions with $\supp. \varphi_i \subset
U_i$ and $\sum \varphi_i = 1$ in a neighbourhood of $K$. Then for $s
\in H_m (K, E)$, $\tau_i (\varphi_i s) \in H_m (U_i)$ and $|\tau_i
(\varphi_i s) |^2_m < \infty$. We\pageoriginale define the norm $|s|_{m,
  \mathscr{U}}$ by $|s|^2_{m, \mathscr{U}} = \sum \limits_{i=1}^h
\tau_i (\varphi_i s)|^2_m$. Then $H_m(K, E)$ is a complete normed
linear space and in fact a Hilbert space. 

Let $\mathscr{H}$ denote the Hilbert space $\oplus^h H_m (U_i)$ and
$\eta$: $H_m (K, E) \to \mathscr{H}$ the map given by $\eta(s) =
\oplus \tau_i (\varphi_i s)$. Clearly $\eta$ is an isometry of $H_m(K,
E)$ onto a closed subspace of $\mathscr{H}$. 

We also have $\tau_i (s | U_i) \in H_m (U_i)$ and if $|| s ||_{m,
  \mathscr{U}}$ denotes 
$$(\sum | \tau_i(s|U_i)|^2_m)^{\frac{1}{2}}, H_m (K, E)$$ 
is a complete normed linear space with the norm $|| ~ ||_{m, \mathscr{U}}$. 

Clearly $|s|_{m, \mathscr{U}} \leq c || s ||_{m , \mathscr{U}}$ and
\begin{align*}
  |\tau_i (s | U_i)|_m & \leq \sum^h_{j=1} |\tau_i (\varphi_j s | U_i) |_m\\
  & \leq C \sum^h_{j = 1} |\tau_j (\varphi_j s | U_j)|_m
\end{align*}
where $C$ is a constant depending on the isomorphisms $\tau_i$.

Hence the two norms are equivalent. It is easy to see that if
$\mathscr{U}_1$ and $\mathscr{U}_2$ are two finite coverings having
the same properties as $\mathscr{U}, | ~ |_{m , \mathscr{U}_1}$ and $|
~ |_{m, \mathscr{U}_2}$ are equivalent. 

\medskip
\noindent
\textbf{Rellich's lemma}. The natural injection $i$: $H_m (K, E) \to
H_{m-1} (K, E)$ is completely continuous. 

This follows at once from the result of \S\ \ref{chap3:sec5} and the
definition of the norms on $H_r (K, E)$. 

\setcounter{proposition}{0}
\begin{proposition}\label{chap3:sec9:prop1} %pro 1
  For\pageoriginale any continuous linear functional $l$ on $H_0 (K, E)$, there
  exists a unique $s' \in H_0 (K, E')$ such that $l(s) = \langle s$,
  $s' \rangle $ for any $s \in H_o (K, E)$. 
\end{proposition}

\begin{proof}
  It is clear that if there exists $s' \in H_o (K, E')$ such
  that $l(s) = \langle s$, $s' \rangle $ for any $s \in H_o (K, E')$
  then $s'$ is unique. Let $U$ be a coordinate neighbourhood such that
  $F$ restricted to a neighbourhood $U'$ of $\bar{U}$ is trivial, then
  it is enough to show that there exists $s' \in H_o (U, E)$ such that 
  $$
  l(s) =  \langle s,s' \rangle \text{ for any } s \in H_o (U, E).
  $$
\end{proof}

If $\tau : E_{U'} \to U' \times \mathbb{C}^q$ is an isomorphism, let
$\tau^* : E^*_{U'} \to U' \times \mathbb{C}^q$ be the corresponding
isomorphism of $E^*_{U}$. Let $s \in H_o (U, E)$ correspond to $\tau
(s) = (s_1 , \ldots , s_o)$; then $(s_1, \ldots , s_q) \in L^2
(U)$. Then by the theorem of Riesz since $L^2 (U)$ is a Hilbert space,
there exists $t = (t_1, \ldots , t_q) \in L^2 (U)$ such that 
$$
l(s) = (s, t) = \int\limits_{U} \sum_{i=1}^{q} s_i \bar{t_i} dx_1
\wedge \cdots \wedge dx_n. 
$$

Let $s' \in H_o (U, E')$ be given by
$$
s' = (\tau')^{-1} (\bar{t_1}, \ldots , \bar{t_q}) \otimes dx_1 \wedge
\cdots \wedge dx_n. 
$$

Then clearly
$$
l(s) = \int\limits_{U} \sum s_i \bar{t_i} dx_1 \wedge \cdots \wedge dx_n =
\langle s, s' \rangle. 
$$

\begin{remark*}%rem 0
  If\pageoriginale $L$ is a differential operator of order $m$, $L: C^{\infty}_o
  (V, E) \to C^{\infty} (V, F)$ and if $K$ is a compact subset of $V$,
  $L$ gives rise to a map $L_K : H_m (K, E) \to H_o (K,F)$. 
\end{remark*}

We shall need the following result

\setcounter{theorem}{0}
\begin{theorem}\label{chap3:sec9:thm1} %the 1
  If $H_1$, $H_2$ are Hilbert spaces and if $A$, $B$ are continuous
  linear maps, $H_1 \to H_2$ such that $A$ is injective and $A(H_1)$
  is closed while  $B$ is completely continuous, then $(A+B) (H_1)$ is
  closed and the kernel of $(A+B)$  is of finite dimension. 
\end{theorem}

\begin{proof}
  It follows from the closed graph theorem that $A^{-1}: A (H_1) \to
  H_1$ is continuous. Let $A + B = T$. If the kernel of $T$ is of
  infinite  dimension, there exists an orthonormal sequence $(x_n)$ in
  $H_1$ such that $Tx_n = 0$. By the complete continuity of $B$, there
  exists a subsequence $(x_{n_{k}})$ of $(x_n)$ such that $Bx_{n_{k}}$
  is convergent. Hence $A$. $x_{n_{k}} \to Ax_0$. It follows from the
  continuity of $A^{-1}$ that $x_{n_{k}} \to x_o$ which contradicts
  the hypothesis that $(x_n)$ are orthonormal. This proves that the
  kernel of $T$ is of finite dimension. 
\end{proof}

Let $N$ be the kernel of $T$ and let $M$ be the orthogonal complement
of $N$ in $H_1$ and let $\widetilde{T}$ be the restriction of $T$ to
$M$. Clearly $\widetilde{T}$ is continuous and injective. It is enough to
prove that $\widetilde{T}^{-1}$ defined on $T(H_1)$ is continuous. Let
$y_n \in T (H_1), y_n \to 0$ and $y_n = \widetilde{T} x_n$, $x_n \in M$. If $x_n
\not\to  0$, we may assume $|| x_n || \ge \rho > 0$ for some positive
number $\rho$. Put $z_n = \dfrac{x_n}{||x_n||}$; then $\widetilde{T} z_n
\to 0$. Let $(z_{n_k})$ be a subsequence of $z_n$ such that
$Bz_{n_{k}}$ is convergent. Then $Az_{n_{k}}$ is convergent and let
$Az_{n_{k}} \to Az_o$. It follows that $z_{n_{k}} \to z_o$ and
obviously $|| z_o || =1$. But $\widetilde{T} z_n \to 0$, i.e. $\widetilde{T}
z_o = 0$ and this is a contradiction. 

\begin{theorem}\label{chap3:sec9:thm2} %the 2
  Let\pageoriginale $V$ be an oriented $C^{\infty}$ manifold  and $E$, $F$,
  $C^{\infty}$ vector bundles on $V$ of rank $q$ and $p$
  respectively. Let $L$ be an elliptic differential operator of order
  $m$, $L$: $C^{\infty}_{o} (V, E) \to C^{\infty} (V,F)$ and $K$, a
  compact subset of $V$. Then $L_K$: $H_m (K, E) \to H_o (K, F)$ has a
  closed image  and the kernel of $L_K$ has finite dimension. 
\end{theorem}

\begin{proof}
  Let $A: H_m (K, E) \to H_o (K, F) \oplus H_{m-1} (K,E)$ be the map
  $Au = (Lu) \oplus i (u)$, where $i$: $H_m (K, E) \to H_{m-1} (K, E)$
  is the natural injection. By Friedrichs' inequality (Part $III$),
  for any $a \in V$, there exists a neighbourhood $U_i$ and a constant
  $C$ such that $|\varphi_i f|_m \leq C (|L \varphi_i f| _0 + |
  \varphi_i f|_0)$ for $C^{\infty} \varphi_i$ with $\supp. \varphi _i
  \subset U_i$. 
\end{proof}

It follows that we have an inequality of the form
$$
|f|_{m , \mathscr{U}} \leq C \{ |Lf|_{o, \mathscr{U}} + || f ||_{m-1,
  \mathscr{U}}\}. 
$$
(with respect to a suitable covering $\mathcal{U}$ of $K$). Since
$||_{r, \mathcal{U}}, || ||_{r, \mathcal{U}}$ are equivalent, this
leads to an inequality 
$$
|f|_m \leq C_1 \{ |Lf |_o + |f|_{m-1} \}.
$$

Now, since $i$ is an injection, so is $A$. Further, because of the
above inequality, $A (H_m (K, E))$ is closed. Let $B: H_m (K, E) \to
H_o (K, F) \oplus H_{m-1}$ $(K, E)$ be the map $Bu = 0 \oplus i(u)$. By
Rellich's lemma, $B$ is completely continuous. Hence, by Theorem $1$,
$A- B = L_K \oplus 0$ has a closed image and a finite dimensional
kernel. The theorem clearly follows from this. 

\begin{proposition}\label{chap3:sec9:prop2} %pro 2
  Let\pageoriginale $V$ be a compact oriented $C^{\infty}$ manifold, $E$, $F$,
  $C^{\infty}$ vector bundles of rank $q$, $p$ respectively on
  $V$. Let $L$ be an elliptic differential operator, $L$: $C^{\infty}
  (V, E) \to C^{\infty} (V, F)$. Let $L (H_m (V, E)) = M$. Then $M =
  \{ s \in H_o (V, F) | <s, s'> = 0$ for every  $s' \in H_o (V,
  F')$ such that  $L' s = 0 \}$. 
\end{proposition}

\begin{proof}
  Let $N = \{ s \in H_o (V, F) | <s$, $s'> = 0$ for $s' \in H_o (V,
  F')$, $L' s' =0 \}$ [the equation $L' s' = 0$ means, of course, that
    $<Lu$, $s'> =0$ for all $u \in C^{\infty} (V, E)$]. By definition
  of the equation $L' s' = 0$ we have $M \subset N$. Suppose that $M
  \neq N$, then since $M$ is closed by Theorem \ref{chap3:sec9:thm2}, there is a
  continuous linear functional $l$ on $H_0 (V, F)$ such that $l (M) =
  0$, but $l (N)\neq 0$. Now, there is $s' \in H_o (V,F')$ such that
  $l(s) = <s$, $s'>$ for $s \in H_o (V, F)$. Since $l(m) =0$, we have
  $<Lu$, $s'> =0$ if $u \in C^{\infty}(V, E)$. But this means
  precisely that $L' s' =0$ and by definition of $N$, we have $l(N)
  =0$, a contradiction. 
\end{proof}

The same reasoning gives the following

\begin{prop*}[{\boldmath $2'$}]\label{chap3:sec9:prop3'} %pro 2'
  Let $V$ be an oriented $C^{\infty}$ manifold, $E$, $F\, C^{\infty}$
  vector bundles on $V$ and $L$: $C^{\infty}_0 (V, F) \to C^{\infty}
  (V, F)$ an elliptic operator. Let $K$ be a compact subset of $V$ and
  $s \in H_o (K, F)$ be such that $<s$, $s'> = 0$ for any $s' \in H_o
  (K, F')$ with $L' s' = 0$ on $\overset{\circ}{K}$. Then there is
  $\sigma \in H_m (K, E)$ with $L \sigma = s$. 
\end{prop*}

\begin{proposition}\label{chap3:sec9:prop3} %pro 3
  If $V$ is a compact $C^{\infty}$ manifold, $E$, $F$, $C^{\infty}$
  vector bundles of the same rank on $V$, $L$ is an elliptic
  differential operator $L: C^{\infty} (V, E) \to C^{\infty} (V,F)$,
  then the image of $L$is of finite codimension 
\end{proposition}

\begin{proof}
  Consider\pageoriginale the operator $L_V$: $H_m (V, E) \to H_0 (V, F)$ and let
  $L_V$ $[H_m (V, E)] = M$. By Proposition \ref{chap3:sec9:prop2}, $M
  = \{s \in H_o (V, F) 
  | <s$, $s'> = 0$ for every $s' \in H_0 (V, F')$ such that $L' s' =
  0\}$. Hence it follows that cokernel $L_V \simeq $ kernel $L'_V
  (L'_V : H_m (V, F') \to H_0 (V,E'))$. Since rank $E=$ rank $F, L'$
  is also elliptic, so that by Theorem \ref{chap3:sec9:thm2}, kernal $L'_V$ has finite
  dimension. Now, if $s' \in H_0 (V, F')$ and $L' s' = 0$, we have $s'
  \in C^{\infty}$. Hence $M \cap C^{\infty} (V, F) = L
  (C^{\infty}(V,E))$. Since $M$ has finite codimension in $H_0 (V,F),
  L (C^{\infty} (V,E))$ has finite codimension in $C^{\infty} (V,F)$ 
\end{proof}

\begin{remark*}
  It can actually be shown that we have
  $$
  C^{\infty} (V,F) / L [C^{\infty} (V,E)] \simeq \text{ kernel } L'_V.
  $$
\end{remark*}

\begin{defi*}%defi 0
  If $V$ is a compact oriented $C^{\infty}$ manifold $E$, $F$ are
  $C^{\infty}$ bundles of the same rank, $L: C^{\infty} (V,E) \to
  C^\infty (V, F)$ an
  elliptic operator, the integer dim. (kernal $L$) - dim. (cokernel
  $L$) is called the index of $L$. 
\end{defi*}

The study of the index of elliptic operators has recently become very
important and has led to beautiful relationships between topology and
analysis. 

The results provided in the section are due mainly to $L$. Schwartz.

\section[The approximation theorem and its...]{The approximation theorem and its application to open Riemann
  surfaces}\label{chap3:sec10} %sec 10 

\begin{defi*}%defi 0
  Let $V$ be a manifold, and $S$ a subset of $V$. $\hat{S}$ denotes
  the union of $S$ with the relatively compact connected components of
  $V-S$. 
\end{defi*}

\begin{remarks*}%rem 0
  \begin{enumerate}[(1)]
  \item If\pageoriginale $K$ is compact, $\hat{K}$ is compact. For if $U$ is a
    relatively open set containing, $K$, let $\Omega_1, \ldots ,
    \Omega_h$ be open connected sets that $\cup \Omega _i \supset
    \partial U, \Omega_i \cap K = \phi$. Then there exists at most $h$
    relatively compact connected components of $V-K$ which are not
    contained in $U$; hence $\hat{K}$ is relatively compact and since $V -
    \hat{K} = \cup \{$ unbounded components  of $V-K \}$, $V - \hat{K}$
    is open i.e. $\hat{K}$ is compact. 
  \item If $K$ is a compact subset of an open set $\Omega$ and if $V-
    \Omega$ has no compact components then $\hat{K} \subset
    \Omega$. For if $U_{\alpha}$ is a bounded component of $V-K$, not
    contained in $\Omega$, let $a \in U_{\alpha}$ and $a \not\in
    \Omega$. If $V_a$ is the connected component of $V - \Omega$
    containing $a$, we have $V_a \subset U_{\alpha}$, hence $V_a$ is
    relatively compact and thus we have a contradiction. 
  \item If $S_1 \subset S_2$, it is easy to see that $\hat{S}_1
    \subset \hat{S}_2$. 
  \item If $U$ is open set then $\hat{U}$ is also open; this fact is
    not so trivial and since we shall not  need it, we omit the proof;
    the same applies to 
  \item If $K$ is a compact set and $K = - \hat{K}$, then $K$ has a
    fundamental system of open (compact) neighbourhoods $U(L)$ such
    that $U = \hat{U} (L = \hat{L})$. 
  \end{enumerate}
\end{remarks*}

\setcounter{lemma}{0}
\begin{lemma}\label{chap3:sec10:lem1}%lemm 1
  Let $V$ be an oriented $C^{\infty}$ manifold, $E$, $F$ $C^{\infty}$
  vector bundles and $L$: $E \to F$, an elliptic differential operator
  with $C^{\infty}$ coefficients. If $\Omega$ is an open set on $V$
  and if $Lf= 0$ on $V$ and $Lf_{\nu} =0$ on $V$, the following
  are equivalent. 
  \begin{enumerate}[(i)]
  \item $f_{\nu} \to f$ in $L^2$ locally on $\Omega$.
  \item $f_{\nu} \to f$\pageoriginale uniformly on compact subsets of $\Omega$.
  \item $f_{\nu}$ and $D^{\alpha} f_{\nu}$, for every $\alpha$,
    converge to $f$ and $D^{\alpha} f$ respectively, uniformly on
    compact subsets of $\Omega$. (Note that because of the regularity
    theorem $f, f_{\nu}$ are $C^{\infty}$.) 
  \end{enumerate}
\end{lemma}

\begin{proof}
  We may suppose that $E$, $F$ are trivial and that $V$ is an open aet
  in $\mathbb{R}^n$. Let $K \subset U \subset \subset U' \subset
  \subset \Omega, K$ being compact and $U$, $U'$ open. Let $r>
  0$. Then by Friedrichs' inequality (Part $IV$) there is a constant
  $C$ such that for any $g \in C^{\infty} (\Omega)$, 
  $$
  |g|^{U}_{m+r} \leq C\{ |Lg|^{U'}_r + |g|_o ^{U'}\}.
  $$
\end{proof}

If $f_{\nu} \to f$ in $L^2 (U')$ and $Lf_{\nu} = 0$ on $V$ and $Lf =
0$ on $\Omega$ this gives 
$$
|f_{\nu} - f'|^{U}_{m+r} \leq C |f_{\nu} - f|_o ^{U'} \to \text{ for
  every } r \geq 0. 
$$

By Sobolev's lemma, there exists a constant $C_K$ such that
$$
|| f_{\nu} - f || _r^K \geq C_K |f_{\nu} - f| ^{U}_{m+r+n},
$$
and hence (i) implies (iii). Since trivially (iii) implies (ii) and
(ii) implies (ii), the lemma is proved. 

\setcounter{theorem}{0}
\begin{theorem}[Malgrange-Lax]\label{chap3:sec10:thm1}%the 1
  Let $V$ be an oriented real analytic manifold, $E$, $F$ analytic
  vector bundles of the same rank and $L$: $E \to F$, an elliptic
  operator of order $m$, with analytic coefficients. Then if $\Omega$
  is an open set in $V$ and if $V - \Omega$ has no compact connected
  components, then any $f \in C^{\infty} (\Omega , E)$ with $Lf =0$ on
  $\Omega$ can be approximated\pageoriginale uniformly on compact subsets of
  $\omega$, by solutions $s \in C^{\infty} (V,E)$ of the equation $Ls
  = 0$. 
\end{theorem}

\begin{proof}
  Let $K$ be a compact set in $\Omega$. Then by the remarks (1) and
  (2) above, $\hat{K}$ is compact and $\hat{K} \subset \Omega$. 
\end{proof}

Let $K'$ be a compact set in $V$ such that $\hat{K} \subset
\overset{\circ}{K'}$. Let
$$
A(K') = \{ f \in H_o (K', E) | Lf = 0 \text{ on } \overset{\circ}{K'} \}
$$
and $S(K) = \{ f | Lf = 0$ in a neighbourhood of  $\hat{K}
\}$. Consider the map $\eta$: $H_o (V, E) \to H_o (\hat{K}, E)$, given
by 
$$
\eta (s) = 
\begin{cases} 
  s & \text{on}~ \hat{K} \\ 
  0 & \text{outside}~ \hat{K}. 
\end{cases}
$$ 

If $\eta (A(K')) = M$, we shall prove that $M$
is dense in $\eta (S(K)) \{$clearly  $M \subset \eta
(S(K))\}$. Let $l$ be a continuous linear functional on $H_o (\hat{K}, E)$
such that $l(s) = 0$ for $s \in M$. By \S\ 9,
proposition \ref{chap3:sec9:prop1}, there
exists $u \in H_o (\hat{K}, E')$ such that $l(s) = < s$, $u >$ for $s \in
H_o (K, E)$. Then $< s$, $u > = 0$ for every $s$ with $Ls =0$ on
$\overset{\circ}{K'}$. Hence by \S\ 9. Proposition $2'$, there exists $v \in H_m (K',
F')$ such that $L' v = u$. 

Now $\supp u \subset \hat{K}$ i.e. $L'v = 0$ on $V- \hat{K}$. Hence by
the analyticity theorem, $v$ is analytic on $V - \hat{K}$. Hence by
the analyticity theorem, $v$ is analytic on $V- \hat{K}$. But supp
. $v \subset K'$ and $V- \hat{K}$ has no relatively compact connected
components; hence $v=0$ on $V- \hat{K}$, i,e, $v \in H_m (\hat{K},
F')$. For any $s \in S(K)$, let $U$ be a neighbourhood of $\hat{K}$ so
that $s$ is defined and $Ls = 0$ on $U$. Then $< \eta (s), u > = < s$,
$u    >  = < s$, $L'v >_U = < Ls$, $v >_U = 0$ (since supp. $v
\subset \hat{K}$), i.e. $l(s) = 0$ for any $s \in \eta (S(K))$. By the
Hahn- Banach Theorem, this implies that $M$ is dense in $\eta
(S(K))$. Thus if $Lf = 0$ in a neighbourhood of $\hat{K}$, there\pageoriginale
exists a sequence of functions $\{ f_{\nu} \}$ in $A(K')$ such that
$f_{\nu} \to f$ in $H_o (\hat{K}, E)$; by Lemma \ref{chap3:sec10:lem1}, $f _{\nu} \to f$
uniformly on compact sets in $(\hat{K})^0$. 

Let $\{K_r \}$ be a sequence of compact sets such that $\cup K_r = V$
and $\hat{K} \subset K^o_1$, $\hat{K}_1 \subset \Omega$, $\hat{K}_r
\subset K^o _{r+1}$ for $r \geq 1$. Then if $Lf =0$ in a neighbourhood
of $\hat{K}_1$ there exists $f_1 \in A(K_2)$ such that 
$$
|| f-f_1 ||^{\hat{K}} < \varepsilon /2
$$

By induction, we have a sequence $f _{\nu} \in A(K_{\nu +1})$ such
that $|| f_{\nu} - f_{\nu +1} || ^{\hat{K}_\nu} <
\dfrac{\varepsilon}{2^{\nu}}$; of course, $f_{\nu} \in C^{\infty}
(K^{o}_{\nu +1})$. 

Define $g$ on $V$ by $g = g_r  \equiv f_r + \sum\limits_{s=
  r+1}^{\infty} (f_s - f_{s-1}) (= \lim\limits_{s \to \infty} f_s)$ on
$K_r$; clearly the series converges uniformly on compact sets of $V$,
and we have $g_r = g_{r+1}$ on $K_r$. Moreover $Lg =0$; in fact, for
any section $u \in C^{\infty}_o (V, F')$ we have $(Lg ) (u) = < g$,
$L'u > = \lim\limits_{s \to \infty} < f_s$, $L'u > = \lim\limits_{s
  \to \infty}  < L f_s$, $u > = 0$ [We have $< f_s$, $L' u > = < Lf_s
  u > $ if $\supp$. $u \subset K_s$.] It is clear that $|| f-g
||^{\hat{K}} < \varepsilon$. 

(The fact that $Lg =0$ also follows from Lemma \ref{chap3:sec10:lem1}.)

\begin{remarks*}%rem 0
  (1)~ It follows from Theorem \ref{chap3:sec10:thm2} and remark (5) after the definition
  of $\hat{S}$ that the following proposition holds. 
\end{remarks*}

  \begin{proposition}\label{chap3:sec10:prop4}%pro 4
    If $K$ is a compact set such that $K = \hat{K}$, then any solution
    of the equation $Ls =0$ in a neighbourhood of $K$ can be
    approximated, uniformly on $K$, by solutions of the equation on
    $V$. 
  \end{proposition}
  
  (2)~ It can be proved that the condition that $V - \Omega$ have no
  compact component is also necessary for every solution on $\Omega$
  to be approximable\pageoriginale by solutions on $V$. The proof depends on the
  existence theory for equations $Ls = f$, $f$ being given, which we
  have not treated. See Malgrange \cite{27}. 
  
  (3)~ Let $V$ be a complex manifold of complex dimension $n$ and let
  $\varepsilon^{p, q}$ denote the set of differential forms of type
  $(p, q)$. Consider $\bar{\partial}$: $\varepsilon^{p, 0} \to
  \varepsilon^{p,1}$, discussed in example (2) of \S\
  \ref{chap3:sec9}. In particular, 
  if $p = 0$ and if the rank of $(0, 0)$ forms = rank of $(0, 1)$
  forms, i.e. if $n = 1$, we may apply Theorem $1$ to
  $\bar{\partial}$: $\varepsilon^{0, 0} \to \varepsilon^{0, 1}$ and
  obtain the following result. 


\begin{theorem}[Runge theorem for open Riemann surfaces: H. Behnke
    -K.Stein]\label{chap3:sec10:thm2}%the2
  If $V$ is an open Riemann surface, (i.e. a connected, non compact
  complex manifold of complex dimension $1$) and if $\Omega$ is an
  open subset of $V$ such that $V - \Omega$ has no complete connected
  components, then if $f$ is a holomorphic function on $\Omega$, for
  any compact subset $K$ of $\Omega, f$ is the uniform limit on $K$ of
  a sequence of holomorphic functions on $V$. 
\end{theorem}

Note that when $V$ is an open set in $\mathbb{C}$, the condition on
$\Omega$ is also necessary. in fact it is seen easily that if $\{
f_{\nu}\}$ is a sequence of holomorphic functions on $V$, converging
uniformly on compact subsets of $\Omega$, then $\{f_{\nu} \}$
converging uniformly on compact subsets of $\hat{\Omega}$. It follows
that any holomorphic function on $\Omega$ which can approximated by
holomorphic functions on $V$, admits a holomorphic extension to
$\hat{\Omega}$. If $\Omega \neq \hat{\Omega}$, this is not the case
for at least one holomorphic function on $\Omega$e.g. $\dfrac{1}{z-a}$
where $a \in \hat{\Omega} - \Omega$. One can further use the Runge
theorem to prove this latter statement also when $V$ is an arbitrary
open\pageoriginale Riemann surface, so that the condition is necessary for any open
Riemann surface. 

\begin{defi*}%defi 0
  Let $V$ be a complex manifold and let $\mathscr{H} = \mathscr{H}
  (V)$ denote the set of all holomorphic function on $V$. $V$ is said
  to be a Stein manifold if the following three condition are
  satisfied. 
  \begin{enumerate}[(i)]
  \item $\mathscr{H}$ separates points.
  \item For any point $a \in V$, there exists functions in
    $\mathscr{H}$, which form a system of local coordinates in a
    neighbourhood of $a$. 
  \item For any compact subset $K$ of $V$, the set
    $\hat{K}_{\mathscr{H}} = \{ x \in V  | f(x) \leq \sup\limits_{y
    \in K}$. $|f(y)|$, for every $f \in \mathscr{H} \}$, is compact. 
  \end{enumerate}
\end{defi*}

\begin{theorem}[Behnke-Stein]\label{chap3:sec10:thm3}%the 3
  Every open Riemann surface $V$ is a Stein manifold
\end{theorem}

\begin{proof}
  For $a$, $b \in V$, $a \neq b$, let $(U_1, \varphi_1 )$, $(U_2,
  \varphi_2)$ be coordinate neighbourhood of a and $b$ such that $U_1
  \cap U_2 = \phi$ and 
  $$
  \varphi _1 (U_1) = \{ z \in \mathbb{C} |z| < 1 \} = \varphi_2 (U_2).
  $$
\end{proof}

If $U'_1 = \{ x \in U_1 \Big| |\varphi_2 (x)| < r < 1 \}$ and $U'_2 = \{ x
\in U_2 \Big| |\varphi_2 (x)| < r < 1 \}$, then $V- U'_1$ and $V - U'_2$
are connected and so is $V-U'_1 - U'_2$. Hence if $\Omega = U'_1 \cup
U'_2$, $V - \Omega$ has no compact connected and by
Theorem \ref{chap3:sec10:thm2}, any
holomorphic function on $\Omega$, can be approximated uniformly on
compact subsets of $\Omega$, by functions in $\mathscr{H}$. 

Let\pageoriginale $f$ be given by
\begin{gather*}
  f(x) = 0 ~\text{for}~ x \in U'_1\\
  \text{and} = 1 \text{for} x \in U'_2.
\end{gather*}

Then $f$ is holomorphic on $\Omega$ and hence there exists $g
\epsilon \mathscr{H}$ such that 
$$
|| f- g ||^{U'_1}, || f -g ||^{U'_2} < \frac{1}{2},
$$
i.e. $\qquad |g (a)| < \dfrac{1}{2}$ and $|g (b)|> \dfrac{1}{2}$ and
hence $\mathscr{H}$ separates points. For $a \in V$, let $f$ be a
holomorphic function in a neighbourhood $W$ of a with $(df)$ $(a) \neq
0$. Then $f$ gives local coordinates at $a$. Let $(U, \varphi)$ be a
coordinate neighbourhood, $U \subset W$ such that $\varphi (U) = \{ z
\in \mathbb{C} \Big| |z| < 1 \}$. Then, with the same notation as above,
$V-U'$ is connected and since $f$ is holomorphic on $U'$, there exists
$g \in \mathscr{H}$ such that $|| f-g ||^{U'} < \varepsilon$. Since
uniform convergence of holomorphic functions implies the uniform
convergence of their derivatives, if $\varepsilon$ is small enough, we
have $(dg) (a) \neq 0$, so that $g$ gives local coordinates at a. We
shall now prove that for a compact set $K$ in $V$, $\hat{K} =
\hat{K}_{\mathscr{H}}$. Let $a \not\in \hat{K}$; then $a \in
U_{\alpha}$, $U_{\alpha}$ being component of $V-K$ which is not
relatively compact. Let $(U, \varphi)$ be a   coordinate
neighbourhood of a such that $\varphi (U) = \{ z \in \mathbb{C} \Big| |z|
< 1 \}$, $\bar{U} \subset U_{\alpha}$. Let $S$ be a discrete unbounded
set, contained in $U_{\alpha}$ and let $S' = S \cup \bar{U}$. Then
$S'$ is a closed set, $S' \subset U_{\alpha}$. Hence there exists a
closed connected set $A$ such that $S' \subset A \subset
U_{\alpha}$. Let $L$ be a compact neighbourhood of $\hat{K}$ such that
$L \cap A = \phi$. Then clearly $A \cap \hat{L} = \phi$ and $\hat{L}$
is a neighbourhood of\pageoriginale $\hat{K}$, $V - \hat{L}$ has no relatively
compact connected  component. Clearly $V - \{ \hat{L} \cap \bar{U} \}$
has no relatively  compact connected component. Let $f$ be defined on
a neighbourhood of $\hat{L} \cup \bar{U}$ by $f(x) = 0$ for $x$ near
$\hat{L} f(x) =1$ for $x$ near $\bar{U}$. Then $f$ is holomorphic in a
neighbourhood of $\hat{L} \cup \bar{U}$. According to the proof of
Theorem \ref{chap3:sec10:thm1} (for the operator  $\bar{\partial}$: $\varepsilon^{0, 0}
\to \varepsilon^{0, 1}$) $f$ is the limit of holomorphic functions on
$V$ in $H_o (\hat{L} \cup \bar{U}$, $\varepsilon^{0, 0})$. Since $K' =
\hat{K} \cup \{a\}$is contained in the interior of $\hat{L} \cup
\bar{U}$, and $L^2$convergence implies uniform convergence on compact
subsets of the interior, $f$ is the uniform limit, on $K'$, of
holomorphic functions on $V$. Hence there exists $g \in \mathscr{H}$
such that 
$$
\displaylines{\hfill 
  |g(x)|< \frac{1}{2} ~\text{for}~ x \in \hat{K}\hfill \cr
  \text{and}\hfill 
  |g(a)| > \frac{1}{2},\hfill }
$$
so that $a \not\in \hat{K}_{\mathscr{H}}$. Hence
$\hat{K}_{\mathscr{H}} \subset \hat{K}$. It follows from the theorem
of maximum modulus that $\hat{K} \subset \hat{K}_{\mathscr{H}}$. 

The main Theorem \ref{chap3:sec10:thm1} is due to Malgrange \cite{27}
and Lax \cite{25}. The
application to open Riemann surfaces is essentially as in Malgrange
\cite{27}. The original treatment of Behnke - Stein \cite{2} is quite different,
and rather more difficult, but enables one to solve also the so called
First and Second Problems of Cousin'' on arbitrary open Riemann
surfaces with little extra effort. 


\begin{thebibliography}{99}
\bibitem{1}{Agmon,} \textit{A. Douglis and L.Nirenberg}.\pageoriginale Estimates
  near the boundary for solutions of elliptic partial differential
  equations satisfying general boundary conditions, I,
  \textit{Comm. pure appl. Math}. 12(1959), 623-727. 
\bibitem{2}{Behnke-K. Stein}. Entwicklung anatylischer Funktionen auf
  Riemannsschen Flmchen, \textit{ Math. Annalen}, 120 (1948),
  430-461. 
\bibitem{3}{E. Bishop}, Mappings of partially analytic spaces,
  \textit{Am. J. Math}. 83 (1961), 209-242. 
\bibitem{4}{J.C. Burkill}, \textit{Lectures on approximation by
  polynomials}, T.I.F.R. 1959 (Appendix). 
\bibitem{5}{E. Cartan}, \textit{Lecons sur les invariants integraux},
  Hermann, paris, 1958 (Chap. VII, \S\ 3, p.71). 
\bibitem{6}{H. Cartan}, Varieties analytiques reelles et varieties
  analytiques complexes, \textit{ Bull. Soc. Math. France}. 85 (1957),
  77-99. 
\bibitem{7}{C. Chevalley}, \textit{Theory of Lie groups}, Princeton
  University Press, 1944. 
\bibitem{8}{E. A. Coddington and N. Levinson}, \textit{Theory of
  ordinary differential  equations}, Mc. Graw Hill, 1955. 
\bibitem{9}{J.Dieudonne}, Une generalisation des espaces
  compacts. \textit{J. Math. pure et appl}. 23 (1944), 65-76. 
\bibitem{10}{K. O. Friedrichs}, On the differentiability of the
  solutions of linear elliptic differential
  equations,\textit{Comm. pure app. Math}. 6 (1953), 299-325. 
\bibitem{11}{L. Garding},\pageoriginale Dirichlet's problem for linear elliptic
  partial differential equations, \textit{Math. Scandinavica},
  1(1953), 55-72. 
\bibitem{12}{G. Glaeser}, Etudes de quelques algebres Tayloriennes,
  \textit{ J.d'Analyse (Jerusalem)}, 6(1958), 1-124. 
\bibitem{13}{ H. Grauert}, On Levi's problem and the imbedding of
  real analytic manifolds. \textit{Annals of Math}. 68 (1958),
  460-472. 
\bibitem{14}{ M. Herve}, \textit{ Several complex variables},
  T.I.F.R., Bombay, 1963. 
\bibitem{15}{M. W. Hirsch}, On imbedding  differentiable manifolds in
  Euclidean spaces. \textit{Annals of Math}. 73 (1961), 566-571. 
\bibitem{16}{H. Hopf}. Zur Topologie der komplexen
  Mannigfaltigkeiten, \textit{Studies and Essays presented to
    R. Courant (Interscience NY} 1948), 167-185. 
\bibitem{17}{L. Hormander}, \textit{Linear partial differential
  operators}, Springer, 1963. 
\bibitem{18}{W. Hurewicz and H. Wallman}, \textit{Dimension theory},
  Princetong University Press (1948), Chap. IV. 
\bibitem{19}{F. John}, \textit{Place waves and spherical means
  applied to partial differential equations}, Interscience,
  $N.Y$. 1995. 
\bibitem{20}{M. Kervaire}, A manifold which does not admit any
  differentiable structure, \textit{Comm, Math. Helvoetici}, 34
  (1960), 257-270. 
\bibitem{21}{K. Kodaira and D.C. Spencer}, On deformations of complex
  analytic structures Parts I, II, \textit{Annals of Math}. 67(1958),
  328-466. 
\bibitem{22}{J. K. Koszul}, \textit{Lectures on fibre bundles and
  differential geometry}, T.I.F.R., Bombay, 1960. 
\bibitem{23}{T. Kotake- M.S. Narasimhan}, Regularity theorems for
  fractional powers of a linear elliptic operator,
  \textit{Bull. Soc. Math. France}, 90(1962), 449-471. 
\bibitem{24}{P. Lax},\pageoriginale On Cauchy's problem for hyperbolic equations
  and the differentiability of solutions of elliptic equations,
  \textit{Comm. pure} appl. Math. 8(1955), 615-633. 
\bibitem{25}{P. Lax}, A stability theorem for abstract differential
  equations and its application the study of the local behaviour of
  solutions of elliptic equations, \textit{Comm. pure
    app. Math}. 9(1956), 747-766. 
\bibitem{26}{B. Malgrange}, \textit{Ideals of differentiable
  functions}, to appear. 
\bibitem{27}{B. Malgrange}, Existence et approximation des solutions
  des equations aux derivees partielles et des equations de
  convolution, \textit{Annales de l'Institut Fourier}, 6(1955-56),
  271-355. 
\bibitem{28}{J. Milnor}, On manifolds homeomorphic to the 7-sphere,
  \textit{Annals of Math}. 64 (1956), 399-405. 
\bibitem{29}{C.B. Morrey - L. Nirenberg}, On the analyticity of the
  solutions of linear elliptic systems of partial differential
  equations, \textit{ Comm. pure app. Math}. 10(1957), 271-290. 
\bibitem{30}{ A.P. Morse}, The behaviour of a function on its
  critical set, \textit{ Annals of Math}. 40(1939), 62-70. 
\bibitem{31}{R. Narasimhan}, Imbedding of holomorphically complete
  complex spaces, \textit{Am. J. Math}. 82(1960), 917-934. 
\bibitem{32}{L. Nirenberg}, Remarks on strongly elliptic partial
  differential equations, \textit{Comm. pure. app. Math}. 8(1955),
  648-674. 
\bibitem{33}{K. Nomizu}, \textit{Lie groups and differential
  geometry}, Publications of the math. Society of Japan, 1956. 
\bibitem{34}{K. Oka},\pageoriginale Sur les fonctions analytiques de plusieurs
  vauiables, $I$. Domaines convexes par rapport aux fonctions
  rationelles, \textit{Journal of Science, Hiroshima Univ}. 6(1936),
  245-255. 
\bibitem{35}{J. Peetre}, Rectifications a l'article ``Une
  caracterisation abstraite des operateurs  differentiels''
  \textit{Math. Scandinavica}, 8(1960), 116-120. 
\bibitem{36}{I. G. Petrovsky}, Suro l'analyticite des systemes $d'$
  equations differ- entielles, \textit{Mat. Sbornik}, 5(1939), 3-68. 
\bibitem{37}{F. Rellich}, Ein Satz uber mittlere Konvergenz,
  \textit{Gottinger Narchrichten}. (1930), 30-35. 
\bibitem{38}{A. Sard}, The measure of critical values of
  differentiable maps, \textit{Bull. Amer. Math. Soc}. 45(1942),
  883-890. 
\bibitem{39}{L. Schwartz},  \textit{Theorie des distributions},
  Vol. 1 and 2, Hermann, Paris, 1950/51. 
\bibitem{40}{L. Schwartz}, \textit{Lectures on complex analytic
  manifolds}, T.I.F.R, Bombay 1965. 
\bibitem{40a} {L. Schwartz}, Les travaux de Seeley sur les
  operateurs integraux singuliers sur une variete,
  \textit{Sem. Bourbaki}, 1963/64, Expose 269. 
\bibitem{41}{J.P. Serre}, Expose 18 in Seminaire H. Cartan, 1953-54. 
\bibitem{42}{S. L. Sobolev}. Sur un theoreme d'analyse fonctionelle,
  \textit{Math. Sbornik}, 4(1938), 471-496. 
\bibitem{43}{A. Weil}, Sur les theoremes de de Rham,
  \textit{Comm. Math. Helvetici}, 26 (1952), 119-145. 
\bibitem{44}{H. Weyl}, Uber die Gleichverteilung von Zahlen mod Eins,
  \textit{Math. Ann}. 77 (1916), 313-352. 
\bibitem{45}{H. Whitney},\pageoriginale Analytic extensicns of differentiable
  function defined in closed sets, \textit{Trans. Amer. Math. Soc}. 36
  (1934), 63-89. 
\bibitem{46}{H. Whitney}, A function not constant on a connected set
  of critical points, \textit{Duke Math. J}. 1 (1935), 514-517. 
\bibitem{47} {H. Whitney}, Differentiable manifolds, \textit{Annals
  of Math}. 37(1936), 645-680. 
\bibitem{48}{H. Whitney}, The self- intersections of a smooth
  n-manifold in 2n-space, \textit{Annals of Math}. 45 (1944),
  220-246. 
\bibitem{49} {H. Whitney}, The singularities of a smooth n-manifold
  in $(2n-1)$-space, \textit{Annals of Math}. 45 (1944), 247-293. 
\bibitem{50}{H. Whitney}, \textit{Geometric integration theory},
  Princeton Univ. Press, 1957. 
\end{thebibliography}
