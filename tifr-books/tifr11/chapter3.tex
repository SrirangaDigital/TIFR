\chapter{Spaces of distributions \texorpdfstring{$\mathscr{H}$}{H}}\label{chap3}

\setcounter{section}{3}
\setcounter{definition}{0}
\begin{definition}\label{chap3:def3.1}
A\pageoriginale space of distributions $\mathscr{H}$ is, by
definition, an $ELC$ which is contained in $\mathscr{D}'$ as a linear
subspace with a finer topology: that is to say, the injection
$i:\mathscr{H} \to \mathscr{D}'$ is continuous.  
\end{definition}

\begin{definition}\label{chap3:def3.2}
A space of distributions $\mathscr{H}$ is said to be normal if
$\mathscr{D}$ is contained in $\mathscr{H}$ with a finer topology and
$\mathscr{D}$ is dense in $\mathscr{H}$.
\end{definition}

\setcounter{section}{3}
\setcounter{prop}{0}
\begin{prop}\label{chap3:prop3.1}
If $\mathscr{H}$ is a normal space of distributions,
\begin{itemize}
\item[1)] the space $\mathscr{H}_c'$ is a normal space of
  distributions, and
\item[2)] the space $\mathscr{H}_\delta'$ (the dual of $\mathscr{H}$
  with the topology of uniform convergence of bounded sets of
  $\mathscr{H}$) is a space of distributions.
\end{itemize}
\end{prop}

\begin{proof}
The space $\mathscr{H}'$ is first of all a subspace of
$\mathscr{D}'$. In fact, if $T\in \mathscr{H}'$ and
$\tilde{T}=T|\mathscr{D}$ ($T$ restricted to $\mathscr{D}$), since the
topology of $\mathscr{D}$ is finer than the topology induced by
$\mathscr{H}$, $\tilde{T}$ is a continuous linear functional on
$\mathscr{D}$. Hence $\tilde{T}\in \mathscr{D}'$. The mapping $T \to
\tilde{T}$ is an injection, for if $\tilde{T}=0$, we have $T=0$
because $\mathscr{D}$ is dense in $\mathscr{H}$. Thus we see that
$\mathscr{H}'$ is a subspace of $\mathscr{D}'$.

Since the injections $\mathscr{D}\to \mathscr{H} \to \mathscr{D}'$ are
continuous, we have by transposition $\mathscr{D}_c' \leftarrow
\mathscr{H}_c' \leftarrow (\mathscr{D'})_c'$ are continuous. But
$\mathscr{D}_c'= \mathscr{D}'$ and
$(\mathscr{D'})_c'=\mathscr{D}$. Hence $\mathscr{D}_c'\leftarrow
\mathscr{H}_c'\leftarrow \mathscr{D}$ are continuous. Also since
$\mathscr{H} \to \mathscr{D}'$ is an injection, $\mathscr{D}$ is dense
in $\mathscr{H}'$ if we take the weak topology
$\sigma(\mathscr{H}',\mathscr{H})$. Now $\mathscr{H}_c'$ and
$\mathscr{H}'$ with the topology $\sigma(\mathscr{H}',\mathscr{H})$
have the same dual (Mackey's Theorem). It follows therefore that the
linear subspace $\mathscr{D}$ which is dense in $\mathscr{H}'$ in the
topology $\sigma(\mathscr{H}',\mathscr{H})$ is also dense in the
topology of $\mathscr{H}_c'$. (See \pageoriginale Bourbaki, EVT~IV, \S~2,
no.~3, Cor.~1). This completes the proof of (1).

The proof of (2) is, in fact, trivial. The continuity of the
injections $\mathscr{D}\to \mathscr{H}\to \mathscr{D}'$ gives the
continuity of the injections $\mathscr{D}_\delta'\leftarrow
\mathscr{H}_\delta' \leftarrow (\mathscr{D'})_\delta'$. But
$\mathscr{D}_\delta'= \mathscr{D}'$ and $(\mathscr{D'})_\delta'=\mathscr{D}$.
\end{proof}
\noindent{\textbf{The space $\mathscr{H}(E)$:}}

Let $\mathscr{H}$ be a space of distributions. Let $E$ be an $ELC$.

\begin{definition}\label{chap3:def3.3}
The space $\mathscr{H}(E)$ consists of all $E$-valued distributions 
$\overrightarrow{T}$ which have the following property. ${}^t
\overrightarrow{T}:E_c'\to \mathscr{D}'$ maps actually $E_c'$ into
$\mathscr{H}$ and is a continuous map of $E_c'$ into $\mathscr{H}$. We
have $\mathscr{H} (E)$ $\approx \mathscr{L}(E_c', \mathscr{H})$. 
\end{definition}

\begin{definition}\label{chap3:def3.4}
Let $\mathscr{H}$ be any linear subspaces of $\mathscr{D}'$. We say
that an $E$-valued distribution $\overrightarrow{T}$ belongs scalarly
to $\mathscr{H}$ if ${}^t\overrightarrow{T}:E_c'\to \mathscr{D}'$
actually maps $E'$ into $\mathscr{H}$. In other words, for every
$\overleftarrow{e}' \in E'$, we have $\langle \overrightarrow{T},
\overleftarrow{e}'\rangle \in \mathscr{H}$. 
\end{definition}

\begin{definition}\label{chap3:def3.5}
We say that a space of distributions $\mathscr{H}$ has the $\mathcal{E}$-
property, if for every locally convex, Hausdorff, complete vector
space any $E$-valued distribution $\overrightarrow{T}$ which scalarly
belongs to $\mathscr{H}$ belongs to $\mathscr{H}(E)$.
\end{definition}

\begin{prop}\label{chap3:prop3.2}
If $\mathscr{H}$ has the $\mathcal{E}$-property, every subspace of
$\mathscr{H}$ with the induced topology has also the $\mathcal{E}$-property.
\end{prop}

\begin{proof}
Let $\mathscr{K}$ be a linear subspace of $\mathscr{H}$ with the
induced topology.Let $\overrightarrow{T}$ be an $E$-valued
distribution, with $E$ a complete $ELC$, satisfying
$\langle\overrightarrow{T},\break \overleftarrow{e}'\rangle \in \mathscr{K}$
for every $\overleftarrow{e}'\in E'$. We have to show that
$\overrightarrow{T} \in \mathscr{K}(E)$. In\pageoriginale  other
words, we have to show that ${}^t\overrightarrow{T}:E_c'\rightarrow
\mathscr{D}'$ takes $E_c'$ into $\mathscr{K}$ and is continuous. Now
$\langle \overrightarrow{T}, \overleftarrow{e}'\rangle \in
\mathscr{K}$ for every $\overleftarrow{e}'$ merely means that
${}^t\overrightarrow{T}(\overleftarrow{e}') \in \mathscr{K}$ for every
$\overleftarrow{e}' \in E'$. Hence ${}^t\overrightarrow{T}: E_c'
\rightarrow \mathscr{D}'$ maps $E_c'$ into $\mathscr{K}$. Since
$\mathscr{K}\subset \mathscr{H}$ and $\mathscr{H}$ has the
$\mathcal{E}$-property ${}^t\overrightarrow{T}: E_c' \rightarrow
\mathscr{H}$ is continuous. Now, the topology of $\mathscr{K}$ is the
induced topology and ${}^t\overrightarrow{T}(E_c')\in
\mathscr{K}$. Hence ${}^t\overrightarrow{T}:E_c' \rightarrow
\mathscr{K}$ is continuous.     
\end{proof}

\begin{prop}\label{chap3:prpo3.3}
Suppose $\mathscr{H}$ satisfies the following conditions
\begin{itemize}
\item [(1)] $\mathscr{H}$ is a normal space of distributions.
\item [(2)] $\mathscr{H}$ has a fundamental system of
  $\mathscr{D}'$-closed neighbourhood of $0$ in $\mathscr{H}$, that is
  to say, $\mathscr{H}$ has a fundamental system of neighbourhoods of
  $0$ which are closed in the topology induced from $\mathscr{D}'$.
\item [(3)] The bounded sets of $\mathscr{H}$ are relatively
  compact. Then $\mathscr{H}$ has the $\mathcal{E}$-property.
\end{itemize} 
\end{prop}

\begin{proof}
Let $E$ be any complete $ELC$ and let $\overrightarrow{T}$ be any $E$-valued
distribution scalarly belonging to $\mathscr{H}$, that is to say,
${}^t\overrightarrow{T}: E_c' \rightarrow \mathscr{D}'$ maps $E_c'$
into $\mathscr{H}$. Since ${}^t\overrightarrow{T}: E_c' \rightarrow
\mathscr{D}'$ is continuous, ${}^t\overrightarrow{T}: E_c' \rightarrow
\mathscr{H}_\mathscr{D'}$ is continuous, where
$\mathscr{H}_\mathscr{D'}$ is the space $\mathscr{H}$ with the
topology induced by $\mathscr{D}'$. The topology on $\mathscr{H}$ is
finer than the topology induced by $\mathscr{D}'$. To prove the
$\mathcal{E}$-property we have to show that
${}^t\overrightarrow{T}:E_c' \rightarrow \mathscr{H}$ is
continuous. According to (2), if we prove that
${}^t{\overrightarrow{T}}^{-1} (W)$ is a neighbourhood of $0$ in $E_c'$
for any convex, stable neighbourhood $W$ of $0$ in $\mathscr{H}$ which
is $\mathscr{D}'$ closed, it will follow that ${}^t\overrightarrow{T}:
E_c' \rightarrow \mathscr{H}$ is continuous. Since $W$ is
$\mathscr{D}'$-closed and since ${}^t\overrightarrow{T}: E_c'
\rightarrow \mathscr{H}_\mathscr{D'}$ is continuous,
${}^t{\overrightarrow{T}}^{-1} (W)$ is closed in
$E_c'$. ${}^t{\overrightarrow{T}}^{-1} (W)$\pageoriginale is a convex,
stable set of $E_c'$. Since $W$ is absorbing,
${}^t{\overrightarrow{T}}^{-1} (W)$ is also absorbing. Since
${}^t{\overrightarrow{T}}^{-1}(W)$ is a convex closed set in $E_c'$ it
is also closed in $E'$ with the weak topology and since it is convex,
stable, absorbing and weakly closed, it is a neighbourhood of $0$ in
the strong topology on $E'$, that is in $E_\delta'$. Hence
${}^t\overrightarrow{T}: E_\delta' \rightarrow \mathscr{H}$ is
continuous. The injection $\mathscr{H} \rightarrow \mathscr{D}'$ is
continuous. Hence the transpose $\mathscr{D} \rightarrow
\mathscr{H}_c'$ is continuous and the image is dense in
$\mathscr{H}_c'$. Because of (3) we have
$\mathscr{H}_\delta'=\mathscr{H}_c'$. Also
$\overrightarrow{T}:\mathscr{H}_\delta' \rightarrow
(E_\delta')_\delta'$ is continuous. Let $E''$ be the bidual of
$E$. The topology $\mathcal{E}$ of uniform convergence on
equicontinuous subsets of $E'$ is coarser than the topology of
$(E_\delta')_\delta'$. Hence $\overrightarrow{T}:\mathscr{H}_\delta'
\rightarrow E_\mathscr{E}''$ is continuous. Hence the composite
$\mathscr{D} \rightarrow \mathscr{H}_\delta' \rightarrow
E_\mathcal{E}''$ is continuous. The image of $\mathscr{D}$ by the
composite is contained in $E$ and on $E$, $E_\mathcal{E}''$ induces on
$E$ the same topology as the initial topology of $E$. Since the image
of $\mathscr{D}$ is dense in $\mathscr{H}_\delta'$, the image of
$\mathscr{H}_\delta'$ in $E_\mathcal{E}''$ is contained in the closure
of $E$ in $E_\mathcal{E}''$. But $E$ being complete we have
$\overrightarrow{T}: \mathscr{H}_\delta' \to E_\mathcal{E}''$ maps
$\mathscr{H}_\delta'$ in $E$. The topology of $E$ being the one induced
by $E_\mathcal{E}''$, we have
$\overrightarrow{T}:\mathscr{H}_\delta'=\mathscr{H}_c' \to E$
continuous. Hence ${}^t\overrightarrow{T}:E_c' \to
(\mathscr{H}_c')_c'$ is continuous. But $(\mathscr{H}_c')_c'$ is the
same as $\mathscr{H}$ with a finer topology. Therefore
${}^t\overrightarrow{T}: E_c' \to \mathscr{H}$ is continuous.   
\end{proof}

This proves our proposition.
