
\chapter{Representations of semi-groups (contd.)}\label{chap19}

Let\pageoriginale, from now on, $G_+$ denote a closed convex cone in
$R^n$, containing the origin. We shall assume that $G_+$ is the
closure of its interior. $G_+$ is a topological semi-groups.
\setcounter{section}{19}
\setcounter{definition}{0}
\begin{definition}\label{chap19:def19.1}
A distribution $T$ on $R^n$ is said to be $G_+$ summable if $T$ has
its support in $G_+$ and $T=\sum D^p\mu_p$ where $\mu_p$ are summable
measures on $R^n$. The space of $G_+$ summable distributions will be
denoted by $\mathscr{D}_{L^1}'(G_+)$. 

If $T\in\mathscr{D}_{L^1}'(G_+)$ and $T'\in\mathscr{D}_{L^1}'(G_+)$
then $T*T'\in\mathscr{D}_{L^1}'(G_+)$. If $T=\sum D^p\mu_p, T'=\sum
D^q\nu_q$ then $T*T'=\sum D^{p+q}\mu_p *\nu_q$. If
$\alpha\in\mathscr{D}(R^n)$ with support in $G_+$ and $T\in
\mathscr{D}_{L^1}'(G_+)$ then $\alpha * T$ has {\bf its support in}
$G_+$ and $\alpha *T d x$ is a summable measure {\bf in $G_+$}. 
\end{definition}
\begin{definition}\label{chap19:def19.2}
A sequence $\{T_j\}$ of $G_+$ summable distributions is said to
converge to zero strictly if $T_j=\sum\limits_{|p|\leq m}D^p\mu_p, j$
with $m$ independent of $j$, and $\{\mu_{p, j}\}$ is, for every $p$, a
sequence of summable measures tending to zero strictly. 

If $\{T_j\}$ and $\{S_j\}$ are two sequences of summable distributions
strict\-ly tending to zero, then $\{T_j *S_j\}$ tends to zero strictly. 

We have seen how one can define $U(\mu)$ for $\mu\in
\mathscr{M}_{G_+}$. Now we shall see how one can define $U(T)$ for $T
\in \mathscr{D}_{L^1}'(G_+)$. However $U(T)$ will not, in general, be
defined on the whole of $E$. But the domain $E_T$ of $U(T)$ will be a
dense subspace of $E$.
\end{definition}

\begin{definition}\label{chap19:def19.3}
Let $\mathscr{H}$ be the filter having for a base the sets
$A_{\varepsilon}$,\break for every $\varepsilon > 0$, defined as follows: 
$$
A_{\varepsilon}= \left\{\begin{array}{l}
\mbox{ $\alpha|\alpha\in(R^n)$, Support of
  $\alpha\subset G_+, \alpha\geq 0$,  }\\
\mbox{ $1-\varepsilon\leq\int\alpha\leq 1+\varepsilon$ and support of
  $\alpha$ contained  }\\
\mbox{ in a $\varepsilon$-neighbourhood of $0$. }
\end{array}\right\}
$$\pageoriginale
\end{definition}

\begin{definition}\label{chap19:def19.4}
Let $T \in \mathscr{D}_{L^1}'(G_+)$. Let $E_T$ be the set of elements
$x$ of $E$ such that $\lim\limits_{\mathscr{F}}U(\alpha *T)x$ exists
in $E$ and for $x \in E_T$ define $U(T)x$ to be
$\lim\limits_{\mathscr{F}}U(\alpha
*T)x$. ($\lim\limits_{\mathscr{F}}$ denotes the limit as $\alpha \to
\delta$ following the filter $\mathscr{F}$. $U(\alpha *T)x$ has a
meaning for every $x\in E$, since $\alpha *T$ has its support in $G_+$
and defines a summable measure in $G_+$). 
\end{definition}

\begin{definition}\label{chap19:def19.5}
We define $\mathscr{D}(G_+)$ to be the subspace of functions in
$\mathscr{D}(R^n)$ whose supports are contained in $G_+$ and
$\mathscr{D}_{L^1}(G_+)$ to be the space of $C^\infty$ functions on
$R^n$ with supports in $G_+$ and with summable derivatives of all orders.
\end{definition}

\setcounter{section}{19}
\setcounter{prop}{0}
\begin{prop}\label{chap19:prop19.1}
For $T\in\mathscr{D}_{L^1}'(G_+),\rho\in\mathscr{D}_{L^1}(G_+)$ and
$x\in E_T$ we have $U(\rho *T)=U(\rho)\circ U(T)x$.
\end{prop}

\begin{proof}
$\rho$ being a summable function with support in $G_+\rho(x)dx$ is a
  summable measure, with support in $G_+$. We denote this measure also
  by $\rho$. $U(\rho)$ and $U(\alpha *T)$ are continuous linear
  operators in $E$ and we have 
$$
U(\rho)\circ U(\alpha *T)x=U(\rho *\alpha * T)x
$$
Now, let $\alpha \to \delta$ following the filter $\mathscr{F}$. Since
$U(\rho)$ is a continuous map of $E$ in $E$. We have 

$\lim\limits_{\mathscr{F}}U(\rho)\circ U(\alpha * T)x=U(\rho)\circ
U(T)x$ for every $x\in E_T$. Also $\rho *\alpha *T=\alpha *\rho * \;
T$ (commutativity), and as $\alpha \to \delta$ following $\mathscr{F},
\alpha *\rho *T \to \rho *T$ strictly (Lemma
\ref{chap17:lem17.2}). Hence $\lim\limits_{\mathscr{F}}U(\rho *(\alpha
  *T))x =U(\rho *T)x$. Thus $U(\rho *T)x=U(\rho)\circ U(T)x$ for every
  $x\in E_T$. 
\end{proof}

\begin{propspl*}\label{chap19:prop19.1'}
Let\pageoriginale $S, T\in\mathscr{D}_{L^1}'(G_+)$, $x\in
  E_T$. Then $x\in E_{S*T}$ if and only if $U(T).x\in E_S$ and if it
is so 
$$
U(S*T)x=U(S)\circ U(T)x.
$$
\end{propspl*}

\begin{proof}
If $S$ is an integrable distribution and $\alpha\in\mathscr{D}
(G_+)$. Then $\alpha *S\in\mathscr{D}_{L^1}(G_+)$. Take for $\rho$ the
element $\alpha *S$ of $\mathscr{D}_{L^1}(G_+)$ in the previous
proposition. Since $U(\alpha *S*T)x=U(\alpha *S)\circ U(T)x$, whatever
be $x\in E_T$, we see that if any one of $\lim\limits_{\mathscr{F}}
U(\alpha *S*T)x$ and $\lim\limits_{\mathscr{F}}U(\alpha *S)U(T)x$
exists, the other also exists and we have the equality of the two
limits. Thus if $x\in E_T$ we have $x\in E_{S*T}$ if and only if
$U(T)x\in E_S$ and then $U(S*T)x=U(S)U(T)x$. 
\end{proof}

\setcounter{cor}{0}
\begin{cor}\label{chap19:cor1}
If $\varphi\in\mathscr{D}_{L^1}(G_+)$, the element $U(\varphi)x\in
E_T$ for every $T\in\mathscr{D}_{L^1}'(G_+)$. Moreover $U(T)
U(\varphi)x=U(T*\varphi)x$.
\end{cor}
\begin{cor}\label{chap19:cor2}
${\bigcap\limits_{T\in\mathscr{D}_{L^1}'}}^{E_T}(G_+)$ is dense in
  $E$. In fact, if $\varphi\in\mathscr{D}(G_+)$ tends to
  $\delta_e=\delta$ following the filter $\mathscr{F},U(\varphi)x\to
  Ix=x$. 

\noindent In particular, we have also $E_T$ dense in $E$ for every
$T\in \mathscr{D}_{L^1}'(G_+)$. 
\end{cor}

\begin{prop}\label{chap19:prop19.2}
The mapping $(T,x)\to U(T)x$ is a closed mapping. That is to say, if
$\{T_j\}$ is a sequence of summable distributions with supports in
$G_+$ and tending strictly to $T$ and if $\{x_j\}$ is a sequence of
elements of $E$ tending to $x$ and if $U(T_j)x_j$ has a meaning for
each $j$ and if $\lim U(T_j)x_j=y$ in $E$, then $U(T)x$ has a meaning
and $U(T)x=y$. (The word mapping is not used here in the usual
sense. $U(T)x$ need not be defined for every $x$).
\end{prop}

\begin{proof}
We shall, in fact, prove a result somewhat stronger than the one that
we have stated. Even when $U(T_j)x_j\to y$ weakly in $E$, we shall
show\pageoriginale that $x\in E_T$ and that $U(T)x=y$. For $\alpha\in
\mathscr{D}(G_+)$, $U(\alpha *T_j)x_j=U(\alpha)\circ U(T_j)x_j$ from
Proposition \ref{chap19:prop19.1}. For $\alpha$ fixed in
$\mathscr{D}(G_+), \alpha * T_j\to\alpha *T$ in the sense of strict
convergence of measures. Hence $U(\alpha *T_j)$ remains in an
equicontinuous set of linear maps of $E$ in $E$ and tends to $U(\alpha
*T)$ in $\mathscr{L}_s(E, E)$.

If $\{V_j\}$ is a sequence of elements lying in an equicontinuous set
of linear maps of $E$ in $E$ and if $V_j\to V$ in $\mathscr{L}_s(E,
E)$ and if $x_j\to x\in E$, the sequence $V_jx_j\to Vx$ in $E$. In
fact,
$$
Vx-V_jx_j=(V-V_j)x+V_j(x-x_j)
$$
Since $[V_j]$ is equicontinuous and $x-x_j\to 0$ in $E$, $V_j(x-x_j)
\to 0$ in $E$. ($[V_j]$ denotes the set of the linear maps
$V_j$). Since $V-V_j\to 0$ in $\mathscr{L}_s(E,E)$, for every fixed
$x\in E$, $(V-V_j)x\to 0$ in $E$.

Taking for $V_j$ the sequence $U(\alpha *T_j)$ we see that $U(\alpha *
T_j)x_j\to U(\alpha *T)x$ in $E$ as $j\to\infty$. Since $U(\alpha)$ is
a continuous linear map of $E$ in $E$ it is also weakly
continuous. Hence if $U(T_j)x_j\to y$ weakly in $E$, $U(\alpha)\circ U
(T_j)x_j\to U(\alpha)y$ in $E$ weakly. But $U(\alpha)\circ U(T_j)x_j
\to U(\alpha *T)x$ strongly in $E$. Hence we must have $U(\alpha)\circ
U(T_j)x_j\to U(\alpha)y$ strongly in $E$ and $U(\alpha)y=U(\alpha
*T)x$. If $\alpha\to\delta$ following the filter
$\mathscr{F},U(\alpha)y \to y$ in $E$. Hence
$\lim\limits_{\mathscr{F}} U(\alpha * T)x$ exists and is equal to
$y$. That is to say, $U(T)x$ has a meaning and $y=U(T)x$. 
\end{proof}

\noindent {\bf Corollaries:}
\begin{itemize}
\item [1)] For each $T\in\mathscr{D}_{L^1}'(G_+), U(T)$ is a closed
  operator. For, if $x_j\to x$ in $E$ and $U(T)x_j\to y$ in $E$,
  choosing $T_j=T$ in the above proposition, we see that $U(T)x$ has a
  meaning and $U(T)x=y$. 
\item [2)] If\pageoriginale $x$ is an element of $E$ such that 
$$
\underset{\mathscr{F}}{\text{weak} \lim} U(\alpha *T)x=y
$$
exists, then $x$ belongs to the domain of $U(T)$ and $U(T)x=y$ and $y=
\lim\limits_{\mathscr{F}} U(\alpha *T)x$ in $E$. 
\item [3)] If $S_j$ is a filter of summable distributions strictly
  converging to $\delta$ and if $U(S_j*T)x$ is defined for every $j$
  and if $U(S_j*T)x\to y$ weakly in $E$, then $x\in E_T$ and $\lim
  U(S_j*T)=U(T)x$.
\item [4)] For defining the operator corresponding to $T\in
  \mathscr{D}_{L^1}'$, even if we choose for the filter $\mathscr{F}'$
  a filter finer than the filter $\mathscr{F}$ used above and require
  the limit to exist only weakly we will get the same operator $U(T)$,
  i.e., the domain of $U(T)$ will not be enlarged. For if
  $\mathscr{F}'=\varphi_i$ is a finer filter, then $\varphi_i*T$ will
  tend to $T$ strictly. If $\underset{\mathscr{F}}{\text{weak} \lim}
  U(\varphi_i*T)x=y$ exists, then by the proposition $\lim U(\alpha_j
  *T)x$ exists and is equal to $y$. 
\end{itemize}

We thus see that the definition for $U(T)$ we have chosen is the most
general one; it gives the largest possible domain of definition for
$U(T)$. 


