
\chapter{Topological tensor products (contd.)}\label{chap13}

We\pageoriginale shall give an intrinsic characterisation of the topology on the
tensor product defined already. Let $L$ and $M$ be two $E L C$s over
$C$. Let $U$ and $V$ be neighbourhoods of the zero elements in $L$ and
$M$. Let $\Gamma(U\otimes V)$ be the convex, stable envelope of the
set $U\otimes V$ in $L\otimes M$, $U\otimes V$ being the set of points
$(u\otimes v,\; u \in U,\; v \in V)$. The set $\Gamma (U \otimes V)$
is an absorbing set in $L\otimes M$. In fact any element of $L\otimes
M$ is of the form $\sum l_\nu \otimes m_\nu, \; l_\nu \in L$ and
$m_\nu \in M$, the sum written being a finite sum. $U$ absorbs $l_\nu$
and $V$ absorbs $m_\nu$. Hence $U\otimes V$ absorbs each of the
elements $l_\nu \otimes m_\nu$. Hence the convex, stable, envelope
$\Gamma(U\otimes V)$ absorbs any finite sum of the elements of the
form $l_\nu \otimes m_\nu$. We can take the sets $\Gamma(U\otimes V)$
for a fundamental system of neighbourhoods of $0$ in a certain locally
convex topology on $L\otimes M$. This topology is precisely the
topology $\pi$. Let us denote the topology just defined by $\tau$.

\setcounter{section}{13}
\setcounter{prop}{0}
\begin{prop}\label{chap13:prop13.1}
The topologies $\pi$ and $\tau$ are identical.
\end{prop}

\begin{proof}
If we show that the topology $\tau$ is the finest locally convex
topology on $L\otimes M$ such that the canonical bilinear map $\eta :
L \times M \to L \otimes M$ is continuous, we are through. Obviously
$\eta$ is continuous, for if we take any neighbourhood of $0$ in
$L\otimes M$ it contains a set of the form $\Gamma(U\otimes V)$, $U$ a
neighbourhood of $0$ in $L$ and $V$ a neighbourhood of $0$ in $M$. Now
$\eta^{-1}(\Gamma(U\otimes V) \supset U \times V$ and this is a
neighbourhood of $(0, 0)$ in $L \times M$. 

Suppose\pageoriginale $\theta$ is any topology on $L \otimes M$ such
that $L \times M \xrightarrow{\eta} \foprod{L}{M}{\theta}$ is
continuous and $\theta$ is locally convex. Any neighbourhood of $0$ in
$\foprod{L}{M}{\theta}$ contains a convex, stable neighbourhood
$W$ of $0$. If we show that $W$ is a neighbourhood of $0$ even for
$\tau$ we are through. Since $\eta : L \times M \to
\foprod{L}{M}{\theta}$ is continuous, there exist neighbourhoods 
$U$ and $V$ of the zero elements of $L$ and $M$ such that $U\otimes V
\subset W$. Since $W$ is convex and stable $\Gamma(U\otimes V)\subset
W$. Hence $W$ is a neighbourhood of $0$ in $\tau$. 
\end{proof}

\noindent {\bf Seminorms.} Let $U$ and $V$ be convex, stable
neighbourhoods of the zero elements in $L$ and $M$ respectively. Let
$p$ be the seminorm associated with $U$ and $q$ the seminorm
associated with $V$. Let $r$ be the seminorm associated with $\Gamma
(U \otimes V)$. It is possible to prove that $r$ is the same as the
tensor product of the two seminorms $p$ and $q$ defined as follows: $p
\otimes q (\mathscr{E})= \Inf \sum\limits_\nu p(x_\nu)q(y_\nu)$ where
$\mathscr{E}= \sum x_\nu \otimes y_\nu$ is a way of expressing
$\mathscr{E}$ as the sum of a finite number of elements of the type
$x_\nu \otimes y_\nu , x_\nu \in L, Y_\nu \in M$. Also, $m$ if for a
pair of elements $x, y; x \in L$ and $y \in M$ we have $\mathscr{E} =
x \otimes y$, we can show that $r(\mathscr{E})=p(x) q(y)$. If $L$ and
$M$ are normed spaces, $L \otimes M$ is a normal space with the tensor
product of the norms on $L$ and $M$ as the norm. That is to say, 
$$
\parallel \mathscr{E} \parallel= \Inf \sum \parallel x_\nu \parallel
\parallel y_\nu \parallel.
$$

Suppose $L$ and $M$ are both Frechet spaces. One can show then that 
$\fohprod{L}{M}{\pi}$ is also Frechet. 

\begin{proof}
Let $L$ and $M$ be Frechet spaces. In order to show that 
$\fohprod{L}{M}{\pi}$ is a Frechet space it suffices to
show that in $\foprod{L}{M}{\pi}$ we have a countable base
for the neighbourhoods of $0$. $L$ and $M$ being metrizable, we have
countable\pageoriginale bases for the neighbourhoods of the zero
elements in the case of the two spaces $L$ and $M$. If $U_n$ and $V_n$
are countable bases for neighbourhoods of the zero elements in $L$ and
$M$ respectively, $\Gamma (U_n \otimes V_n)$ is a fundamental system
of neighbourhoods of $0$ in $\foprod{L}{M}{\pi}$. Hence 
$\foprod{L}{M}{\pi}$ is metrizable and hence its completion is a
complete metric space and as it is locally convex, it is a Frechet
space.

In what follows immediately $L$ and $M$ stand for two Frechet spaces.   
\end{proof}

\begin{prop}\label{chap13:prop13.2}
Any $\mathscr{E} \in L \hat{\otimes}M$ can be written as a convergent
infinite series $\sum \lambda_\nu x_\nu \otimes y_\nu$ in which the
$x_\nu s$ and $y_\nu s$ can be so chosen as to converge to $0$ as
$\nu \to \infty$ in $L$ and $M$ respectively and $\sum\limits_\nu
|\lambda_\nu| < \infty$ Also if $K$ is compact in
$\fohprod{L}{M}{\pi}$, any $\mathscr{E} \in K$ can be
written as $\sum\limits_\nu \lambda_\nu x_\nu \otimes y_\nu$ with
$\sum\limits_\nu |\lambda_\nu | < \infty$ and $x_\nu \in A, y_\nu \in
B$ where $A$ and $B$ are compact subsets of $L$ and $M$
respectively. That is to say $K\subset\overline{\Gamma(A \otimes B)}$. For
the proof, refer to Memoirs of the American Mathematical Society,
No.~16, Products Tensoriels Topologique et Espaces Nucleaires, by
Alexandre Grothendieck, p.~51.
\end{prop}


\setcounter{section}{13}
\setcounter{definition}{0}
\begin{definition}\label{chap13:def13.1}
An $E L C$ $E$ is said to be nuclear if for {\bf every} $E L C$, $F$
the $\pi$ and $\varepsilon$ topologies coincide on $E \otimes F$. 

We now give a criterion for a locally convex Hausdorff space $L$ to be
nuclear.  
\end{definition}

\setcounter{section}{13}
\setcounter{crite}{0}
\begin{crite}\label{chap13:crite13.1}
An $E L C$ $E$ is nuclear if and only if for every Banach space $B$,
we have $\foprod{L}{B}{\pi} =\foprod{L}{B}{\varepsilon}$. 

We shall give another criterion, which is, to some extent, better than
the above criterion. Before giving the criterion, we introduce certain
notions needed to state the criterion. 
\end{crite}

\begin{definition}\label{chap13:def13.2}
Let\pageoriginale $N$ be a complete $E L C$. Let $L$ be an $E L C$. A linear
continuous map $u : L \to N$ is called nuclear if it can be written as 
$$
u = \sum\limits_\nu \lambda_\nu (l_\nu' \otimes m_\nu)
$$
with the $l_\nu's$ contained in an equicontinuous subset of $L'$ and
$m_\nu$ contained in a bounded subset of $N$ and $\sum\limits_\nu
|\lambda_\nu| < \infty$.

To see that the definition makes sense, we consider the expression
$\mu = \sum\limits_\nu \lambda_\nu (l_\nu' \otimes m_\nu)$ with the
$l_\nu's$ lying in an equicontinuous subset of $L'$ and the $m_\nu$
lying in a bounded set of $N$ and $\sum\limits_\nu |\lambda_\nu | <
\infty$. Let $\mu (l)$ for any $l$ be defined as $\sum\limits_\nu
\lambda_\nu (l_\nu'(l).m_\nu)$. We shall show that $l \to \mu(l) \in
\hat{N} = N$ is a continuous map. Let $V$ be any neighbourhood of $0$
in $N$. Since $m_\nu s$ lie in a bounded set, there exists $\alpha >
0$ such that $m_\nu \in \alpha V$ for every $\nu$. Since the $l_\nu'$
lie in an equicontinuous set, given any $\in > 0$, we can find
a neighbourhood of $W_\in$ of $0$ in $L$ such that $|\langle l,
l_\nu' \rangle | \leq \in$ for any $l \in W_\in$. Hence 
$$
 \sum \lambda_\nu \langle l, l_\nu'\rangle m_\nu \in \wedge
 \in \alpha \overline{\Gamma(V)}
$$
where $\sum |\lambda_\nu | < \wedge$ and $\overline{\Gamma(V)}$ is the
convex, stable, closed envelope of $V$. But $V$ itself can be chosen
to be a disked neighbourhood of $0$ in $N$. Hence  $\sum \lambda_\nu
\langle l, l_\nu' \rangle m_\nu \in \wedge \in \alpha
V$. Choose $\in = \frac{1}{\wedge \alpha}$. Then
$\sum\limits_\nu \lambda_\nu \langle l, l_\nu' \rangle m_\nu \in
V$. This proves the continuity of $\mu$. That $\mu$ is linear is
obvious. If $u : L \to N$ is a nuclear map, we can express $u$ as
$\sum\limits_\nu \lambda_\nu (l_\nu' \otimes m_\nu)$ with $m_\nu \to
0$ in $N$ and $l_\nu'$ lying in an equicontinuous subset of $L'$. In
fact, since $\sum\limits_\nu |\lambda_\nu |$ converges we can find a
divergent sequence of real numbers $\{r_\nu\}$, diverging to $+
\infty$ such that the series $\sum\limits_\nu \lambda_\nu r_\nu$ still
converges absolutely. Then $\sum\limits_\nu \lambda_\nu (l_\nu'
\otimes m_\nu) = \sum \lambda_\nu r_\nu (l_\nu \otimes
\frac{m_\nu}{r_\nu})$. Since the $m_\nu s$ lie in a bounded set,
$\frac{m_\nu}{r_\nu} \to 0$ as $\nu \to \infty$ and $\sum |\lambda_\nu
r_\nu | < \infty$. 
\end{definition}

\begin{crite}\label{chap13:crite13.2}
Let\pageoriginale $U$ be any disked neighbourhood of $0$ in $L$. With $U$ we
associate a seminorm $p$. This seminorm gives a certain equivalence
relation in $L$. $x \sim y$ if and only if $p(x) = p(y)$. We put on
$L$ the coarsest topology under which the seminorm $p$ is
continuous. Then we take the quotient space $L_U$ under the
equivalence relation defined with the help of the seminorm $p$. Let
$\hat{L}_U$ be the completion of $L_U$. $\hat{L}_U$ is a Banach
space. $L$ is Nuclear if and only if the canonical map $L \to
\hat{L}_U$ is a Nuclear map for every disked neighbourhood $U.$ ($A$.
Grothendieck: Espaces Nucleares, Memoirs of the Amer. Math. Soc.,
No.~16, p.~34). We shall now prove the following
\end{crite}

\begin{prop}\label{chap13:prop13.3}
If for every disked neighbourhood $U$ of $0$ the canonical map $L \to
\hat{L}_U$ is a nuclear map, any continuous linear map $u : L \to B$,
$B$ being any Banach space is nuclear.
\end{prop}

\begin{proof}
There exists a disked neighbourhood $U$ of $0$ in $L$ such that\break
$u(U)\subset \Gamma$ where $\Gamma$ is the unit ball of $B$. The
continuous linear map $u$ can be factored into the canonical map $L
\to L_U$ and a continuous linear map of $\hat{L}_U \to B$. Since $L
\to \hat{L}_U$ is nuclear, our assertion follows.
\end{proof}

\setcounter{section}{13}
\setcounter{theorem}{0}
\begin{theorem}\label{chap13:thm13.1}
A nuclear space has the approximation property.
\end{theorem}

\begin{theorem}\label{chap13:thm13.2}
In a nuclear complete space $L$ all the bounded sets are relatively compact.
\end{theorem}

\begin{theorem}\label{chap13:thm13.3}
A nuclear Banach space is finite dimensional. 
\end{theorem}
\noindent We admit Criterion \ref{chap13:crite13.2}, and Theorem
\ref{chap13:thm13.1}. Theorem \ref{chap13:thm13.3} is an
immediate consequence of Theorem \ref{chap13:thm13.2}. We prove here
Theorem \ref{chap13:thm13.2}.

\begin{proof}
Let $B$ be a bonded set. Without loss of generality, $B$ can be
assumed to be disked, for the convex, stable envelope of a bounded set
is bounded\pageoriginale. Since $L$ is complete, to show that $B$ is
\emph{relatively} compact, it suffices to show that $B$ is precompact. For
$B$ to be precompact, it is necessary and sufficient that for any
disked neighbourhood of $0$ say $U$ the image of $B$ in $L_U$ is
precompact. Now $L \to \hat{L}_U$ is a nuclear map and hence a compact
map. Hence the image of $B$ in $\hat{L}_U$ is relatively compact. This
completes the proof of Theorem~\ref{chap13:thm13.2}.
\end{proof}

\begin{theorem}\label{chap13:thm13.4}
If $L$ and $M$ are both nuclear, $\foprod{L}{M}{\pi}=
\foprod{L}{M}{\varepsilon}$ is nuclear and $L\times M$ is 
nuclear. 
\end{theorem}
\noindent The proof is trivial in the case of $\foprod{L}{M}{\pi}$.

\begin{theorem}\label{chap13:thm13.5}
$L$ is nuclear if and only if $\hat{L}$ is nuclear.
\end{theorem}

\begin{theorem}\label{chap13:thm13.6}
If $L$ is a nuclear Frechet space, the strong dual $L'$ is a nuclear space.
\end{theorem}

\begin{theorem}\label{chap13:thm13.7}
If $L$ and $M$ are Frechet nuclear spaces, $\mathscr{L}_\delta(L, M)$
is a nuclear space.
\end{theorem}

\begin{theorem}\label{chap13:thm13.8}
If $L$ is a nuclear space, any subspace of $L$ with the induced
topology is a nuclear space. If $H$ is any closed subspace of $L$, the
quotient space $L/H$ is also nuclear.
\end{theorem}

\noindent {\bf Examples of nuclear spaces.} $\mathscr{D}_+', \mathscr{D},
\mathscr{D}', \mathscr{E}, \mathscr{E}', \mathscr{S}, \mathscr{S}',
\mathscr{O}_M, \mathscr{O}_M', \mathscr{O}_c', \mathscr{O}_c$ are
nuclear spaces. $\mathscr{E}^m$ and $L^p$ are not nuclear. In fact in
$\mathscr{E}^m$ and $L^p$ bounded sets are not relatively compact.

For the proofs of all these results, refer to Espaces Nucleaires by
A. Grothendieck, Memoirs of Amer. Math. Soc., No.16.                               


