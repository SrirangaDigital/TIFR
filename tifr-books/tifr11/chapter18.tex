
\chapter{Representations of semigroups (contd.)}\label{chap18}

\setcounter{section}{18}
\setcounter{lemma}{0}
\begin{lemma}\label{chap18:lem18.1}
If\pageoriginale $\mu$ and $\nu$ are summable measures, $\mu * \nu$ is
a summable measure and $U(\mu * \nu)= U (\mu)\circ U(\nu)$. 
\end{lemma}
\begin{proof}
We have already remarked that $\mu * \nu$ is summable and that $\int
|d \mu * \nu | = \parallel \mu * \nu \parallel \leq
\parallel\mu\parallel \parallel\nu\parallel = \int |d \mu |. \int |d \nu |$. We
shall now prove that $U(\mu * \nu) = U(\mu)\circ U(\nu)$. For any
bounded continuous function $\varphi$ with values in $C$ we know that
$\int \int \varphi(st) d \mu (s) d \nu(t) = \mu * \nu (\varphi)$. We
shall show that this formula is true for any vector valued continuous,
bounded function. 

Let $F$ be an $E L C$ and $\hat{F}$ its completion. Let
$\overrightarrow{\varphi}$ be an $F$-valued function which is
continuous and bounded. Considered as an $\hat{F}$-valued function
also $\overrightarrow{\varphi}$ is continuous and bounded. Let
$\overleftarrow{f}' \in (\hat{F})'$. Then $\langle
\overrightarrow{\varphi}, \overleftarrow{f}' \rangle$ is a continuous
bounded function with values in $C$. Hence \quad $\mu * \nu \langle
\overrightarrow{\varphi}, \overleftarrow{f}' \rangle = \int \int
\langle \overrightarrow{\varphi} (st), f' \rangle \; d \mu(s)d
\nu(t)$. Since $\overrightarrow{f} \to \langle \overrightarrow{f},
\overleftarrow{f}'\rangle$ is a continuous linear map of $\hat{F}$ 
in $C$, we have $(\mu * \nu) \langle \overrightarrow{\varphi},
\overleftarrow{f}' \rangle = \langle ( \mu * \nu)
(\overrightarrow{\varphi}), \overleftarrow{f}' \rangle$ and
$\int\int\langle \overrightarrow{\varphi}(st),
\overleftarrow{f}'\rangle
d\mu(s)d\nu(t)=\langle\int\int\overrightarrow{\varphi}(st) d\mu(s)
d\nu(t), \overleftarrow{f}'\rangle$. Hence 
\begin{align}
\langle(\mu * \nu)(\overrightarrow{\varphi}),
\overleftarrow{f}'\rangle &=
\langle\int\int\overrightarrow{\varphi}(st) d\mu(s) d\nu(t),
\overleftarrow{f}'\rangle\notag\\
\intertext{Therefore} \quad (\mu * \nu)(\overrightarrow{\varphi}) &=
\int\int \overrightarrow{\varphi}(st) d \mu (s) d
\nu(t)\tag{1}\label{chap18:eq1}
\end{align}
 
\noindent If it so happens that $(\mu * \nu)(\overrightarrow{\varphi})$ is in
$F$ itself, we have $\int\int\overrightarrow{\varphi}(st) d\mu\break(s) d
\nu(t) \in F$ because of equality \eqref{chap18:eq1}.
\end{proof}
Since $\{U(x), x \in G_+\}$ is an equicontinuous set of
linear maps, it is a bounded set, and hence $\int\int U(st) d \mu(s)
d\nu(t) =(\mu * \nu ) (U)$. But for any summable measure $\lambda$ we
have defined $U(\lambda)$ to be $\int U(x) d \lambda (x)$ or as
$\lambda(U)$.\pageoriginale Hence $(\mu * \nu)(U)=U(\mu *
\nu)=\int\int U(st) d \mu (s) d \nu (t)$. To evaluate the double
integral $\int\int U(st) d \mu (s) d \nu (t)$ we use Fubini's Theorem
for integrals of vector valued functions. We want only the following
form of Fubini's Theorem. If $\overrightarrow{\varphi}(s, t)$ is a
continuous, bounded $F$-valued function on $G_+ \times G_+$ and if
$\mu$ and $\nu$ are summable measures 
\begin{align*}
\int\limits_{G_+\times}\int\limits_{G_+} \overrightarrow{\varphi}(s, t) d
\mu(s) d\nu(t) &= \int\limits_{G_+} d \mu (s) \int\limits_{G_+}
\overrightarrow{\varphi} (s, t)d \nu (t)\\
&= \int\limits_{G_+} d \nu(t) \int\limits_{G_+}
\overrightarrow{\varphi} (s, t) d \mu (s).
\end{align*}
Now, the integrals $\int\limits_{G_+} \overrightarrow{\varphi}(s, t) d
\nu (t)$ and $\int \overrightarrow{\varphi}(s, t) d \mu (s)$ exist for
all $s$ and $t \in G_+$ and are continuous functions of $s$ and $t$
respectively. By forming the scalar product with any
$\overleftarrow{f}' \in (\hat{F})'$ and applying the theorem of Fubini
for scalar valued function we get 
\begin{align*}
\int\limits_{G_+\times}\int\limits_{G_+}\overrightarrow{\varphi} (s,
t) d \mu (s) d \nu (t) &= \int\limits_{G_+} d\mu (s) \int\limits_{G_+}
\overrightarrow{\varphi} (s, t) d \nu (t)\\
& = \int\limits_{G_+} d \nu
(t) \int\limits_{G_+} \overrightarrow{\varphi} (s,t)d\mu(s)
\end{align*}
Applying this form of Fubini's Theorem we get 
\begin{align*}
\int\limits_{B_+\times}\int\limits_{G_+}U(st)d\mu(s)d\nu(t) &=
\int\limits_{G_+} d\mu(s) \int\limits_{G_+} U(st) d\nu(t)\\
&= \int\limits_{G_+} d\mu(s) \int\limits_{G_+} U(s)\circ U(t) d \nu (t).
\end{align*}
Since $V \to U(s) \circ V$ for a fixed $s \in G_+$ is a linear
continuous map of $\mathscr{L}_s(E, E)$ in itself, we have 
\begin{align*}
\int\limits_{G_+\times}\int\limits_{G_+} U(st) d\mu(s) d\nu(t) &=
\int\limits_{G_+} d\mu(s) \left\{ U(s)\circ \int\limits_{G_+}
U(t)d\nu(t) \right\}\\
&= \int\limits_{G_+}d\mu(s)U(s)\circ U(\nu)\\
&= \left\{\int\limits_{G_+}d\mu(s)U(s)\right\}\circ U(\nu) \quad\\
&\qquad\qquad\text{since} \quad V \to V\circ U(\nu)
\end{align*}
is a continuous linear map of $\mathscr{L}_s(E, E)$ into itself. 

\noindent Hence\pageoriginale $\quad \int\limits_{G_+\times}
\int\limits_{G_+} U(st)d\mu(s)d\nu(t) = U(\mu)\circ U(\nu)$

\noindent Hence $\qquad U(\mu * \nu) = U(\mu)\circ U(\nu)$. 

Suppose $E$ is a Banach space and each $U(x)$ satisfies $\parallel
U(x)\parallel \leq 1$. Then $\parallel U(\mu)\parallel \leq \int
|d\mu|$ for any summable measure $\mu$. This follows immediately from
the definition $U(\mu)=\int\limits_{G_+}U(x)d\mu(x)$.

\setcounter{section}{18}
\setcounter{prop}{0}
\begin{prop}\label{chap18:prop18.1}
Let $E$ be a complete $E L C$. If $\int|d\mu|\to0$, then $U(\mu) \to
0$ in $\mathscr{L}_\delta(E, E)$.
\end{prop}

\begin{proof}
We have $U(\mu)\overrightarrow{e}=\int\limits_{G_+}U(x)
\overrightarrow{e} d\mu(x)$ for every $\overrightarrow{e} \in
E$. Suppose we take vectors $\overrightarrow{e}$ in a bounded set $B$
of $E$. Since the set $\{U(x), x \in G_+\}$ is an equicontinuous set
of linear maps of $E$ in $E$, the set $\Gamma =\underset{\substack{x\in
    G_+ \\ \overrightarrow{e} \in B}}{U} \{U(x)\overrightarrow{e}\}$
is a bounded set of $E$. Now $U(\mu)\overrightarrow{e} \in
\overset{\frown}{\Gamma} \int\limits_{G_+}|d\mu|$ where
$\overset{\frown}{\Gamma}$ is the convex, closed, stable envelope of
$\Gamma$. $\overset{\frown}{\Gamma}$ is bounded since $\Gamma$ is. If
$\int |d\mu|\to 0$ we see that for $\overrightarrow{e} \in B, \;
U(\mu) \overrightarrow{e} \to 0$ uniformly in $E$. Hence our
proposition.   
\end{proof}

\begin{prop}\label{chap18:prop18.2}
If $\{\mu_j\}$ is a sequence of measures tending to $0$ strictly,
$\{U(\mu_j)\}$ tends to $0$ in $\mathscr{L}_s(E, E)$.
\end{prop}
\begin{proof}
We have to prove that for every fixed $\overrightarrow{e} \in E$,
$U(\mu_j) \overrightarrow{e}\to 0$ in $E$. But $\int\limits_{G_+} U(x)
\overrightarrow{e} d\mu_j(x)=U(\mu_j)\overrightarrow{e}$. Now $U(x)
\overrightarrow{e}$ is a bounded vector valued function of $G_+$ in
$E$; this follows from the fact that $\{U(x), x \in G_+\}$ is an
equicontinuous set of operators. Hence the proposition is proved if we
prove the following more general proposition. 

For any continuous bounded $E$ valued function
$\overrightarrow{\varphi}(x)$ on $G_+$ we have $\int
\overrightarrow{\varphi}(x) d\mu_j(x) \to 0$ if $\mu_j \to 0$
strictly. Assume first, that $\overrightarrow{\varphi}$ is a
continuous function with values in $E$ having a compact support. Then
$\overrightarrow{\varphi} \in \mathscr{D}^\circ(E)$ and $\mu_j \in
\mathscr{D}_c^{\circ'}$. The integral $\int \overrightarrow{\varphi}(x)
d \mu_j(x)$ is nothing but the\pageoriginale product
$\overrightarrow{\varphi}$. $\mu_j$ extending the scalar product
defining the duality between $\mathscr{D}^\circ$ and
$\mathscr{D}_c^{\circ'}$. This product is hypocontinuous with respect
to compact subsets of $\mathscr{D}^\circ(E)$ and compact subsets of
$\mathscr{D'}_c^\circ$. Hence if $\overrightarrow{\varphi} \in
\mathscr{D}^\circ(E)$ is a fixed element and $\mu_j \to 0$ in
$\mathscr{D}_c'^\circ$ we have $\overrightarrow{\varphi}.\mu_j\to
0$. Our assumption is that $\mu_j\to 0$ strictly. If we prove that
$\mu_j \to 0$ strictly implies $\mu_j\to 0$ in $\mathscr{D}_c'^\circ$
we are through. When $\mu_j \to 0$ strictly $\{\mu_j\}$ is a bounded
set of measures and hence $\{\mu_j\}$ is an equicontinuous set of
measures. Hence the topology of compact convergence and the topology
of simple convergence induce the topology on the set
$\{\mu_j\}$. Hence $\mu_j\to 0$ in $\mathscr{D}_c^{\circ'}$. 

Now we go to the case of a continuous, bounded function
$\overrightarrow{\varphi}(x)$ with values in $E$. 

To show that $\int\overrightarrow{\varphi}(x) d \mu_j(x)\to 0$ we have
to show that given any convex neighbourhood $V$ of $0$ in $E$, there
exists a $j_V$ such that for $j\geq j_V$ we have
$\int\overrightarrow{\varphi} (x)d\mu_j(x)\in V$. Let $B$ be the
set of values $\{\overrightarrow{\varphi}(x)\}$. $B$ is bounded and
hence $\overset{\frown}{B}$ also. Hence there exists an $\varepsilon
>0$ such that $\varepsilon \overset{\frown}{B}\subset\frac{V}{2}$. Let
$K$ be a compact subset of $G_+$ such that the $\varepsilon(K)$ that
corresponds to $K$ is less than $\varepsilon$. Let $\alpha$ be a
continuous function equal to $1$ or $K$ and with compact support,
satisfying $0\leq\alpha(x)\leq 1$ for every $x \in G_+$. Then $\mu_j
(\overrightarrow{\varphi}) =\mu_j(\alpha\overrightarrow{\varphi})+
\mu_j ((1-\alpha)\overrightarrow{\varphi})$, i.e., $\int\limits_{G_+}
\overrightarrow{\varphi} (x)d\mu_j(x)=\int\limits_{G_+}\alpha(x)
\overrightarrow{\varphi} (x)d\mu_j(x)+\int\limits_{G_+}(1-\alpha(x))
\overrightarrow{\varphi} (x)d\mu_j(x)$. Since $\alpha(x)
\overrightarrow{\varphi} (x)$ is a continuous function with compact
support, we have a $j_V$ such that for $j\geq j_V$,
$$
\int\limits_{G_+}\alpha(x)\overrightarrow{\varphi}(x)d\mu_j(x) \in
\frac{V}{2}.
$$
Also\pageoriginale $\int\limits_{G_+}(1-\alpha(x))
\overrightarrow{\varphi}(x) d\mu_j(x)=\int\limits_{[K} (1-\alpha(x))
  \overrightarrow{\varphi} (x)d\mu_j(x)$
\begin{align*}
&\in \overset{\frown}{B} \int\limits_{[K} |d\mu_j|\\
&\in \varepsilon \overset{\frown}{B} \subset \frac{V}{2}.
\end{align*}
Hence $\quad \int\limits_{G_+}\overrightarrow{\varphi}(x)d\mu_j(x)\in
V$. This proves our proposition. 
\end{proof}
