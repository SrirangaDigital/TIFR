
\chapter{The Laplace transform of vector valued
  distributions}\label{chap9}

Let\pageoriginale $E$ be a complete $E L C$. Let $\overrightarrow{T}
\in \mathscr{D}_+'(E)$. As in the case of scalar distributions we have
the following three possibilities:
\begin{itemize}
\item [1)] There exists no real numbers $\mathscr{E}$ such that
  $e^{-\mathscr{E} t}\overrightarrow{T} \in \mathscr{S}'(E)$.
\item [2)] For every real $\mathscr{E}$ we have
  $e^{-\mathscr{E}t}\overrightarrow{T} \in \mathscr{S}'(E)$.
\item [3)] There exist two real numbers $\mathscr{E}_\circ'$ and
  $\mathscr{E}_1$ such that we have $e^{-{\mathscr{E}}_\circ't}\break
  \overrightarrow{T} \in \mathscr{S}' (E)$ and $e^{-{\mathscr{E}}_1 t}
  \overrightarrow{T} \notin \mathscr{S}' (E)$.
\end{itemize}

\setcounter{section}{9}
\setcounter{prop}{0}
\begin{prop}\label{chap9:prop9.1}
In case (3) there exists a real number $\mathscr{E}_\circ$ such that
for $\mathscr{E} > \mathscr{E}_\circ$ we have
$e^{-\mathscr{E}t}\overrightarrow{T} \in \mathscr{S}'(E)$ and for
$\mathscr{E} < \mathscr{E}_\circ$ we have
$e^{-\mathscr{E}t}\overrightarrow{T} \notin \mathscr{S}'(E)$.
\end{prop}

\begin{proof}
This will follow immediately if we show that $e^{-\mu
t}\overrightarrow{T} \in \mathscr{S}'(E), \mu$ real, and $\nu > \mu$
imply $e^{-\nu t} \overrightarrow{T} \in \mathscr{S}'(E)$.
$$
Now,\quad e^{-\nu t} \overrightarrow{T} = e^{-\mu
t}\overrightarrow{T} \cdot e^{-(\nu -\mu)t} \alpha (t),
$$
where $\alpha (t)$ is a $C^\infty$-function which is $1$ for $t \geq
0$, which is $0$ for $t\leq -1$, and which satisfies $0\leq\alpha
(t)\leq 1$. Now $e^{-\mu t}\overrightarrow{T} \in \mathscr{S}'(E)$ and
$e^{-(\nu -\mu)t}\alpha (t) \in \mathscr{S}$ and hence $e^{-\nu t}
\overrightarrow{T}\in\mathscr{O}_c' (E) \subset
\mathscr{S}'(E)$. Thus, we have, in case (3), a real number
$\mathscr{E}_\circ$ such that for $\mathscr{E} > \mathscr{E}_\circ$ we
have $e^{-\mathscr{E}t}\overrightarrow{T} \in \mathscr{S}'(E)$ and for
$\mathscr{E} < \mathscr{E}_\circ$, $e^{-\mathscr{E}t}
\overrightarrow{T} \notin \mathscr{S}'(E)$. As in the case of scalar
distributions, we say in case (1) the distribution
$\overrightarrow{T}$ has no Laplace transform and in cases (2) and (3)
it has a Laplace transform. In case (2) the whole of the complex plane
is defined to be the domain of existence of the Laplace transform of
$\overrightarrow{T}$ and case (3) the half-plane $Rl p >
\mathscr{E}_\circ$ is defined to be the domain of existence of the
Laplace transform of $\overrightarrow{T}$.
\end{proof}
\begin{prop}\label{chap9:prop9.2}
In\pageoriginale case (2) we have $e^{-pt}\overrightarrow{T} 
\in \mathscr{O}_c'(E)$ for every $p$, and in case (3) we have 
$e^{-pt} \overrightarrow{T} \in \mathscr{O}_c'(E)$ for every 
$p$ satisfying $Rl p > \mathscr{E}_\circ$. 
\end{prop}
\begin{proof}
\text{\bf Case(2)}. For every $\overleftarrow{e}' \in E'$,
$\langle e^{-pt}\overrightarrow{T},
\overleftarrow{e}'\rangle \in \mathscr{S}'$ and hence $\langle
e^{-pt}\overrightarrow{T}, \overleftarrow{e}'\rangle \in
\mathscr{O}_c'$ for every $p$, from what we have 
seen in lecture~\ref{chap8}.

{\bf Case(3)}. For every $\overleftarrow{e}' \in E'$ and $Rl p>
\mathscr{E}_\circ$ we have $\langle e^{-pt}\overrightarrow{T},
\overleftarrow{e}'\rangle \in \mathscr{S}'$ and hence $\langle e^{-pt}
\overrightarrow{T}, \overleftarrow{e}'\rangle \in \mathscr{O}_c'$ for
every $p$ such that $Rl p > \mathscr{E}_\circ$. Now, since
$\mathscr{O}_c'$ satisfies the $\mathcal{E}$-property we have the
required result. 
\end{proof}

\begin{prop}\label{chap9:prop9.3}
The function $p \to e^{-pt} \overrightarrow{T}$ is a holomorphic
function with values in $\mathscr{O}_c'(E)$ in case (2) and a
holomorphic function with values in $\mathscr{O}_c'(E)$, in the half
plane $Rl p > \mathscr{E}_\circ$ in case (3). 
\end{prop}

\begin{proof}
Similar to the proof of proposition \ref{chap8:prop8.5}.
\end{proof}
\setcounter{section}{9}
\setcounter{definition}{0}
\begin{definition}\label{chap9:def9.1}
Let $\overrightarrow{T} \in \mathscr{D}_+'(E)$ have a Laplace
Transform. We denote the domain of existence of the Laplace Transform
of $\overrightarrow{T}$ by $Rl p > a^.$ where either `$a$' is a
certain real number $\mathscr{E}_\circ$ or stands for the symbol
`$-\infty$'. The function $\overrightarrow{F} (p)$ defined 
in $Rl p > \mathscr{E}_\circ$ by 
$$
\overrightarrow{F}(p)=\int\limits_0^\infty e^{-pt} \overrightarrow{T}
dt = 1.e^{-pt}\overrightarrow{T}
$$
\end{definition}
with values in $E$ is called the Laplace transform of
$\overrightarrow{T}$. The scalar product $1.e^{-pt}
\overrightarrow{T}$ is the scalar product of $1 \in
(\mathscr{O}_c')_c'$ and $e^{-pt} \overrightarrow{T} \in
\mathscr{O}_c'(E)$. The function $p \to \overrightarrow{F}(p)$ is a
holomorphic function of the complex variable with values in $E$.

\begin{prop}\label{chap9:prop9.4}
If $\overrightarrow{F}(p)$ is the Laplace Transform of
$\overrightarrow{T} \in \mathscr{D}_+'(E)$, the domain of existence
being $Rl p> :a:$ and if $U \in \mathscr{D}_+'$ has $G(p)$ as
its Laplace transform, with $Rl p > :b:$ as the domain of
existence $\overrightarrow{T} * U$ has, in $Rl p> Max (a, b) =
\alpha$ the function $\overrightarrow{F}(p)$ $G(p)$ as its Laplace transform.
\end{prop}

\begin{proof}
Now\pageoriginale $\langle e^{-pt}\overrightarrow{T} * U,
\overleftarrow{e}\rangle = e^{-pt} \{\langle\overrightarrow{T},
\overleftarrow{e}'\rangle * U\}$ for every $\overleftarrow{e}' \in
E'$. Hence, for every $\overleftarrow{e}' \in E'$, $\langle e^{-pt}
\overrightarrow{T} * U, \overleftarrow{e}\rangle = (e^{-pt}\langle
\overrightarrow{T}, \overleftarrow{e}'\rangle )* e^{-pt}U$. For $Rl p
> \alpha$ we have $e^{-pt}\langle \overrightarrow{T},
\overleftarrow{e}\rangle \in \mathscr{O}_c'$ and $e^{-pt} U \in
\mathscr{O}_c'$ and hence $e^{-pt} \langle \overrightarrow{T},
\overleftarrow{e}' \rangle * e^{-pt} U \in \mathscr{O}_c'$.

Hence $\langle e^{-pt} \overrightarrow{T} * U,\overleftarrow{e}'
\rangle \in \mathscr{O}_c'$ for $Rl p > \alpha$ and this for
every $\overleftarrow{e}' \in E'$. Since $\mathscr{O}_c'$ has the
$\mathcal{E}$-property, we have $e^{-pt} \overrightarrow{T} * U \in
\mathscr{O}_c' (E) \subset \mathscr{S}'(E)$. Also the Laplace
transform of $\langle \overrightarrow{T}, \overleftarrow{e}' \rangle *
U$ is the same as $\langle \overrightarrow{F}(p), \overleftarrow{e}'
\rangle \cdot G(p)$. This proves that
$$
\overrightarrow{T} \underset{Rl p> \alpha}{* U \sqsupset
  \overrightarrow{F}}(p) G(p).
$$  
\end{proof}

We shall be only interested in the case when $E$ is a Banach space. We
shall now study the properties of the Laplace transform in this
particular case.
\noindent {\bf Notation}. In what follows $E$ is a Banach space.
\begin{prop}\label{chap9:prop9.5}
$\overrightarrow{T} \in \mathscr{D}_+'(E)$ is a distribution having
  $\overrightarrow{F}(p)$ as its Laplace transform with $Rl p > a$ as
  the domain of existence. Then if $a$ is a real number, given any
  $\varepsilon > 0$ in the half plane $Rl p \geq a + \varepsilon$ we
  have a uniform majorisation $\parallel
  \overrightarrow{F}(p)\parallel \leq A(1+|p|^2)^k$. If $a$ stands for
  the symbol $-\infty$ then for any real number $r$ we have a
  majorisation $\parallel \overrightarrow{F}(p)\parallel\leq
  A(1+|p|^2)^k$ in $Rl p \geq r, A$ being a constant $>0$.
\end{prop}

\begin{proof}
First we prove that the following two statements are equivalent:

\begin{itemize}
\item [1)] There exists an integer $k$ such that 
$$
\parallel \overrightarrow{F} (p)\parallel \leq A (1+|p|^2)^k
$$
\item [2)] For every sequence of complex numbers $p_1, p_2,
  p_3,\ldots$ such that $|p_n| \to \infty$ and for every sequence of
  real numbers $\alpha_1, \alpha_2, \ldots,\break\alpha_n, \ldots$ such
  that $\alpha_n |p_n|^k$ tends to $0$ for every integer $k$, the
  sequence $\parallel \overrightarrow{F}(p_n)\parallel \alpha_n$ is bounded.
\end{itemize}
Trivially\pageoriginale (1) implies (2). We have to only prove (2)
implies (1). Suppose (2) is satisfied and (1) is not satisfied. Given
any integer $n$ we can find $a\,p_n$ such that 
$$
\parallel \overrightarrow{F} (p_n)\parallel \geq A (1 + |p_n|^2)^n.
$$
Take for $\{\alpha_n\}$ the sequence $\frac{1}{|p_n|}n/2$. Then
obviously for every integer $k$, the sequence $\alpha_n|p_n|^k$ tends
to $0$. Hence by (2) we should have $\frac{\parallel
  \overrightarrow{F}(p_n)\parallel}{|p_n|^{n/2}}$ bounded. But
$\frac{\parallel \overrightarrow{F} (p_n)\parallel}{|p_n|^{n/2}} \geq
\frac{A(1+|p_n|^2)^n}{|p_n|^{n/2}}$. Obviously this sequence is not
bounded. Hence (2) has to imply (1).

Let $\overrightarrow{T} \underset{\mathscr{E} > a'}{\sqsupset}
\overrightarrow{F} (p)$. Then $\langle \overrightarrow{T},
\overleftarrow{e}'\rangle \underset{\mathscr{E}> a'}{\sqsupset}
\langle \overrightarrow{F} (p), \overleftarrow{e}'\rangle$ for every
fixed $\overleftarrow{e}' \in E'$. Hence there exists an integer
$\mu_e'$ such that $|\langle \overrightarrow{F} (p),
\overleftarrow{e}'\rangle | \leq A_{\overleftarrow{e}'} (1+|p|^2)^\mu
\overleftarrow{e}'$ where $A_{\overleftarrow{e}'}$ is a constant $>0$,
uniformly in $Rl p \geq a + \varepsilon$ if $a$ is real or in $R1 p
\geq r, r$ any real number if $a$ stands for the symbol `$-\infty$'.

Now suppose $\{p_n\}$ is any sequence of complex numbers with $|p_n|
\to \infty$ and suppose $\{\alpha_n\}$ is any sequence of real numbers
with $\alpha_n|p_n|^k\to 0$ for every integer $k$

\noindent 
We have $\qquad |\langle \overrightarrow{F}(p_n)
\alpha_n,\overleftarrow{e}'| \leq{A_{\overleftarrow{e}'}
(1+|p_n|^2)}^{\mu_{\overleftarrow{e}'}} |\alpha_n|$.

\noindent 
Now as $n \to \infty, |\alpha_n|
 {(1+|p_n|^2)}^{\mu_{\overleftarrow{e}'}} \to 0$, and hence
 $\langle \overrightarrow{F} (p_n)\alpha_n,
 \overleftarrow{e}'\rangle$ is a bounded sequence of points. Hence the
 set of points $\overrightarrow{F} (p_n)\alpha_n$ is a weakly bounded
 set in $E$, and hence a strongly bounded set. Hence $\parallel
 \overrightarrow{F} (p_n)\parallel |\alpha_n|$ is bounded.

This completes the proof of proposition \ref{chap9:prop9.5}.    
\end{proof}
\begin{prop}\label{chap9:prop9.6}
If $\overrightarrow{F}(p)$ is a holomorphic function of the complex
variable $p$ in $Rl p \geq \mathscr{E}$ with values in $E$ and if we
have a majorisation
$$
\parallel \overrightarrow{F} (p)\parallel \leq A {(1 + |p|^2)}^k
\quad \text{in} \quad Rl p \geq \mathscr{E},
$$\pageoriginale
$\overrightarrow{F}(p)$ is the Laplace transform of a certain unique
distribution $\overrightarrow{T} \in \mathscr{D}_+' (E)$ in their
common domain of definition.
\end{prop}

\begin{proof}
Similar to the scalar case.
\end{proof}
