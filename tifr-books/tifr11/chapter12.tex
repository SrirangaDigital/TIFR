
\chapter{Topological tensor products}\label{chap12}

Let\pageoriginale $L$ and $M$ be two vector spaces (algebraic) over $K$. 
Let $L\otimes M$ be the tensor product of $L$ and $M$ over $K$. Then for
any vector space $N$ over $K$ there exists a biunique correspondence
between the bilinear maps of $L \times M$ in $N$ and the linear maps
of $L \otimes M$ in $N$. In fact to the bilinear map $u$ of $L\times
M$ in $N$ corresponds the linear map $\tilde{u} : L \otimes M$ in $N$
which takes $l \otimes m$ into $u(l, m)$. The map $\eta : L \times M
\to L \otimes M$ defined by $\eta (l, m) = l \otimes m$ is a bilinear
map and is called the canonical bilinear map of $L \times M$ in $L
\otimes M$. We have commutativity in the following diagram:
\[
\xymatrix{
L \times M\ar[rr]^\eta \ar[dr]_{u} & & L \otimes M \ar[dl]^{\tilde{u}}\\
& N &
}
\]
Let now $L$ and $M$ be two locally convex Hausdorff vector spaces over
$C$. Let $L \otimes M$ be the algebraic tensor product of $L$ and $M$
over $C$.
\setcounter{section}{12}
\setcounter{theorem}{0}
\begin{theorem}\label{chap12:thm12.1}
There exists a unique locally convex, Hausdorff topology on $L \otimes
M$ such that under the usual correspondence between bilinear maps of
$L \times M$ in an $E L C$ $N$ and the linear maps of $L \otimes M$ in
$N$, the continuous bilinear maps of $L \times M$ in $N$ precisely
correspond to the continuous linear maps of $L \otimes M$ provided
with this topology in $N$. Moreover under the biunique correspondence
between bilinear maps of $L \times M$ in $C$ and linear maps of $L
\otimes M$ in $C$, equicontinuous sets of bilinear maps of $L \times
M$ in $C$ correspond to equicontinuous sets of linear maps of $\fibreoproduct
{L}{M}{\pi}$ in $C$ ($\fibreoproduct{L}{M}{\pi}$ being the tensor product
provided with the above topology) and conversely.
\end{theorem}

\begin{proof}
Assuming\pageoriginale the existence of at least one such topology, we will 
prove the uniqueness. Let $\pi_1$ and $\pi_2$ be two such topologies. Take
for $N$ the space $\fibreoproduct{L}{M}{\pi_1}$. The identity map
of $\fibreoproduct{L}{M}{\pi_1} \to N$ is continuous and hence the
bilinear map which corresponds to this, that is to say, $\eta : L
\times M \to N$ is continuous. Now that $\eta : L \times M \to 
\fibreoproduct{L}{M}{\pi_2}$ is continuous, we have $i : 
\fibreoproduct{L}{M}{\pi_2}\to\fibreoproduct{L}{M}{\pi_2}$
continuous. Hence $\pi_2$ is finer than $\pi_1$. Interchanging the
roles of $\pi_1$ and $\pi_2$ we see that $\pi_1$ is finer than
$\pi_2$. Hence $\pi_1 = \pi_2$.
\end{proof}

We now go to the proof of the existence of one such topology. Let
$\mathscr{B}(L, M)$ denote the set of continuous bilinear forms on $L
\times M$. The spaces $\mathscr{B}(L, M)$ and $L \otimes M$ are in
duality with respect to the scalar product which is got by restricting
the scalar product between $\mathscr{L}(L, M; C)$ and $L \otimes M$
where $\mathscr{L}(L, M; C)$ is the set of all bilinear maps of $L
\times M$ in $C$. Now the duality between $\mathscr{B}(L, M)$ and $L
\otimes M$ allows us to define a topology on $L \otimes M$, namely the
topology of uniform convergence on equicontinuous subsets of
$\mathscr{B}(L, M)$ which is a locally convex, Hausdorff
topology. This topology on $L \otimes M$ we denote by $\pi$ and
provided with this topology the space $L \otimes M$ is denoted by 
$\fibreoproduct{L}{M}{\pi}$. We shall prove that $\pi$ is a topology
having all the properties mentioned in the theorem. We shall show
first that under the biunique correspondence of bilinear maps of $L
\times M$ in $C$ and linear maps of $\fibreoproduct{L}{M}{\pi}$ in $C$,
equicontinuous sets of bilinear maps precisely correspond to
equicontinuous linear maps. Let $H \subset \mathscr{L}(L, M; C)$ be
any equicontinuous set. Then $H \subset \mathscr{B}(L, M)$
trivially. Let $\tilde{H}$ be the corresponding subset of
$\mathscr{L}(L\otimes M, C)$ (the set of all linear maps of $L\otimes
M$ in $C$). Let $\Gamma(H)$ be the stable envelope of $H$
in\pageoriginale $\mathscr{L}(L, M; C)$. $\Gamma (H)$ is an
equicontinuous set and hence $\Gamma(H)\subset \mathscr{B} L, M)$. Let
$W=\Gamma (H)^\circ$ be the polar of $\Gamma(H)$ with respect to the
duality between $\mathscr{B}(L, M)$ and $L \otimes M$. 

Since $\Gamma(H)$ is stable, 
$$W=\left\{\gamma \in L \otimes M /
|\langle r, \gamma \rangle | \leq 1,  \text{ for every }
r\in \Gamma(H) \right \}.$$ 
Now $\tilde{h}(\gamma) = |\langle h,
\gamma \rangle |$ where $\tilde{h} \in \tilde{H}$ and $h$ is the
corresponding bilinear map. For $h \in H$ we have $|\langle h, \gamma
\rangle | \leq  1$. Hence $|\tilde{h}(\nu)| \leq 1$ for $\nu \in
W$. Hence $\tilde{H}$ is an equicontinuous set of linear maps of 
$\fibreoproduct{L}{M}{\pi}$ in $C$ (as $W$ is a neighbourhood of $0$ in
$\fibreoproduct{L}{M}{\pi}$).

Now suppose $\tilde{H}$ is any equicontinuous set of linear maps of 
$\fibreoproduct{L}{M}{\pi}$ in $C$. Let $H$ be the corresponding set of
bilinear maps of $L \times M$ in $C$. Given any neighbourhood $N$ of
$0$ in $L \otimes M$ we show that there exist neighbourhoods $U$ and
$V$ of the zero elements in $L$ and $M$ respectively such that the
set $U \otimes V \subset N$. (The set $U\otimes V$ is, by definition,
the set of all elements of the form $u \otimes v$, $u \in U$, $v \in
V$). Now $N \supset W^\circ$ where $W$ is an equicontinuous subset of
$\mathscr{B}(L, M)$. We can find stable neighbourhoods $U$ and $V$ of
the zero elements such that $|w(u,v)| \leq 1\,w \in W$, $u \in U$ and
$v \in V$. The pair $(U, V)$ does what we need, for if $u \in U$ and
$v \in V$ we have $|\langle w, u \otimes v \rangle | = | w(u, v)| \leq
1$. Hence $u\otimes v \in W^\circ \subset N$. Hence $U\otimes V
\subset N$. (This incidentally proves the continuity of the map $\eta
: L \times M \to\fibreoproduct{L}{M}{\pi}$).

Now for any $h \in H$ and $(u, v) \in U \times V$ we have 
$$
|h(u, v)| = |\tilde{h}(u \otimes v)| \leq 1,
$$
since $U\otimes V \subset N$. Hence $H$ is an equicontinuous set of
bilinear maps of $L \times M$ in $C$. 

Now\pageoriginale we prove that the continuous bilinear maps of $L \times M$ 
in any $E L C$ $N$, precisely correspond to the continuous linear maps of 
$\fibreoproduct{L}{M}{\pi}$ in $N$. For this, we need the following

\begin{lemma}\label{chap12:lem12.1}
Let $\varphi$ be any continuous bilinear map of $L \times M$ in
$N$. With each $\mu \in N'$ we associate the bilinear form $\langle
\varphi, \mu \rangle$ defined by $\langle \varphi, \mu \rangle\break (l, m)
= \langle \varphi (l, m), \mu \rangle$. The bilinear form $\langle
\varphi, \mu \rangle$ is an element of $\mathscr{B}(L, M)$,
(trivially). If $W'$ is any equicontinuous subset of $N'$, the set
$\underset{w' \in W'}{\cup} \langle \varphi, w' \rangle$ is an
equicontinuous subset of $\mathscr{B}(L, M)$. 
\end{lemma}

\begin{proof}
Since $W'$ is an equicontinuous subset of $N'$ we can find a
neighbourhood $\Gamma$ of $0$ in $N$ such that 
$$
|\langle r, w'\rangle | \leq 1 \quad \text{for every} \quad r \in
\Gamma \quad \text{and} \quad w' \in W'
$$
Since $\varphi : L \times M \to N$ is continuous, there exist
neighbourhoods $U$ and $V$ of the zero elements of $L$ and $M$
respectively such that $\varphi(U \times V)\subset \Gamma$. For $(u,
v)\in U \times V$ we have
$$
|\langle \varphi, w' \rangle (u, v)|=|\langle \varphi(u, v), w'
\rangle | = | \langle r, w' \rangle | \leq 1 \quad \text{with} \quad r
\in \Gamma.
$$
Hence $\underset{w' \in W'}{\cup}\{\langle \varphi, w' \rangle \}$ is
an equicontinuous subset of $\mathscr{B}(L, M)$. 

Now, let $\varphi : L \times M \to N$ be bilinear and continuous. For
any equicontinuous set $W'$ of $N'$ we have $\underset{w' \in
  W'}{\cup}\{\langle \varphi, w' \rangle \}$ is an equicontinuous
subset of $\mathscr{B}(L, M)$. Hence if $\alpha \to 0$ in 
$\fibreoproduct{L}{M}{\pi}$, 
$$
\langle \langle \varphi, w' \rangle, \alpha \rangle
\underset{\mathscr{B}(L, M), L\otimes M}{\to 0}
$$
uniformly for $w' \in W'$. If $\tilde{\varphi}:\fibreoproduct{L}{M}{\pi}
\to N$ is the corresponding linear map, we have $\langle
\tilde{\varphi}(\alpha), w' \rangle_{N, N'} = \langle \langle \varphi,
w'\rangle, \underset{\mathscr{B}, L \times M}{\alpha \rangle}$ and
this tends to $0$ uniformly when $w' \in W'$. 
\end{proof}

Hence\pageoriginale $\tilde{\varphi}(\alpha) \to 0$ in $N$ when
$\alpha \to 0$ in $\fibreoproduct{L}{M}{\pi}$. Hence
$\tilde{\varphi}$ is continuous.

Conversely, suppose $\tilde{\varphi}:\fibreoproduct{L}{M}{\pi} \to
N$ is continuous. Let $\varphi$ be the corresponding bilinear map of
$L \times M \to N$. We have $\varphi = \tilde{\varphi}_\circ \eta$. As
has been shown already, $\eta$ is continuous. Hence $\varphi$ is
continuous. 

\begin{cor}
$\pi$ is the strongest (finest) locally convex Hausdorff topology on
$L \otimes M$ such that $\eta : L \times M \to L \otimes M$ is
continuous. In fact $\pi$ is a locally convex Hausdorff topology on
$L \otimes M$ such that $\eta : L \times M \to\fibreoproduct{L}{M}{\pi}$
is continuous. Let $\pi'$ be any locally convex Hausdorff topology
such that $\eta : L \times \to\fibreoproduct{L}{M}{\pi'}$ is
continuous. Then $i :\fibreoproduct{L}{M}{\pi}\to
\fibreoproduct{L}{M}{\pi'}$ is continuous and hence $\pi$ is finer
than $\pi'$.  
\end{cor}

\begin{cor}
The topology $\pi$ is finer than the
topology $\varphi$. 
\end{cor}

In fact, with the topology $\varepsilon, \eta : L \times M \to
\foprod{L}{M}{\varepsilon}$ is continuous, Hence $\pi$ is 
{\bf finer than $\varepsilon$.}

\setcounter{prop}{0}
\begin{prop}\label{chap12:prop12.1}
Let $u : L_1 \to L_2$ and $v : M_1 \to M_2$ be continuous linear maps
where $L_1, L_2, M_1$ and $M_2$ are locally convex, Hausdorff,
topological vector spaces. Then $u \otimes v =\fibreoproduct{L_1}
{M_1}{\pi} \to\fibreoproduct{L_2}{M_2}{\pi}$ is a
continuous linear map.
\end{prop}

\begin{proof}
To show this, it suffices to prove that the bilinear map $u \times v :
L_1 \times M_1 \to\fibreoproduct{L_2}{M_2}{\pi}$ defined by $u
\times v (l_1, m_1) = u(l_1)\otimes v(m_1)$ is continuous. If $l_1 \to
0$ in $L_1$ and $m_1 \to 0$ in $M_1$, we have $u(l_1)$ and $v(m_1)$
tending to $0$ in $L_2$ and $M_2$. Hence $u(l_1)\otimes v(m_1) \to 0$
in $\fibreoproduct{L_2}{M_2}{\pi}$ since the canonical map $L_2
\times M_2 \to \fibreoproduct{L_2}{M_2}{\pi}$ is continuous. Hence $u
\times v : L_1 \times M_1 \to\fibreoproduct{L_2}{M_2}{\pi}$ is continuous.
\end{proof}

\begin{coro*}
If\pageoriginale $u : L_1 \to L_2$ and $v:M_1 \to M_2$ are continuous
injections, $u\otimes v :\fibreoproduct{L_1}{M_1}{\pi}
\to\fibreoproduct{L_2}{M_2}{\pi}$ is a continuous injection. 

This is an immediate consequence of the fact that $C$ is a field and
of proposition~\ref{chap12:prop12.1} 
\end{coro*}

\noindent {\bf Remarks:}
\begin{itemize}
\item [1.] Though $u \otimes v :\foprod{L_1}{M_1}{\pi} \to 
\foprod{L_2}{M_2}{\pi}$ is a continuous injection, the extension
of $u \otimes v$ by continuity from $\foprod{L_1}{M_1}{\pi}
\to \foprod{L_2}{M_2}{}$
  (the completions) is a continuous linear map, but not necessarily an
  injection.

\item [2.] Let $\mathscr{H}$ be a complete space of distributions and
  $E$ a complete $E L C$. The space $\mathscr{H}(E)$ is complete. The
  map $i :\foprod{\mathscr{H}}{E}{\pi} \to\foprod{\mathscr{H}}{E}
 {\varepsilon}$ given by $i(\alpha)=\alpha$
  is a continuous linear map which extends itself
  into a continuous linear map $\hat{i} :\fohprod{\mathscr{H}}{E}{\pi}
 \to\mathscr{H}(E)$. $\hat{i}$ is in general not an injection. 

\item [3.] Let $L, M$ and $N$ be three $E L C$s over $C$. The
  canonical isomorphism of $(L\otimes M)\otimes N$ with $L \otimes
  (M\otimes N)$ is a topological isomorphism of $(\foprod{L}{M}{\pi})
  \foprod{}{N}{\pi}$ with $\foprod{L}{}{\pi}(\foprod{M}{N}{\pi})$.
  By using trilinear maps we can introduce a locally convex, Hausdorff
  topology $\pi$ on $L\otimes M \otimes N$ such that the canonical
  biunique correspondence between trilinear maps of $L\times M\times
  N$ and linear maps of $L\otimes M\otimes N$ in any $E L C$ $F$ takes
  the continuous trilinear maps precisely into the continuous linear
  maps $(L\otimes M\otimes N)_\Pi$ in $F$. The canonical isomorphism of
  $L\otimes(M\otimes N)$ with $(L\otimes M\otimes N)$ is a topological
  isomorphism of $\foprod{L}{}{\pi}(\foprod{M}{N}{\pi})$
  with $(L\otimes M \otimes N)_\pi$. That is to say, we have
$$
(L\otimes M\otimes N)_\pi \approx \foprod{L}{}{\pi}(\foprod{M}{N}
{\pi})\approx (\foprod{L}{M}{\pi})\foprod{}{N}{\pi}
$$
as topological vector spaces. 
\end{itemize}


