


\chapter{Vector valued Distributions}\label{chap1}



{\bf Notations}. $\mathscr{D}$\pageoriginale denotes the space of $C^\infty$
functions with compact supports on $R^N$. 

$\mathscr{E}$ denotes the space of all $C^\infty$ functions on
$R^N$. $\mathscr{S}$denotes the space of `rapidly decreasing'
functions on $R^N$. All these spaces are provided with their usual
topologies. (See ``Theorie des distributions'' by L. Schwartz, Vol.\ 1
and 2) We denote by $\mathscr{D}', \mathscr{E}'$ and $\mathscr{S}'$
the strong duals of $\mathscr{D},\mathscr{E}$ and $\mathscr{S}$
respectively. $\mathscr{D}'$, $\mathscr{E}'$ and $\mathscr{S}'$ are
respectively the space of distributions on $R^N$, the space of
distributions with compact support on $R^N$ and the space of
`tempered' distributions on $R^N$.

\medskip
\noindent{\textbf{Definition of a vector valued distribution:}}

Let $E$ be a locally convex Hausdorff topological vector space. We
will refer to such a space as $ELC$.

\setcounter{section}{1}
\begin{definition}\label{chap1:def1.1} 
A linear continuous map from $\mathscr{D}$ to $E$ is defined to be an
$E$-valued distribution or a distribution with values in $E$.
\end{definition}

\begin{remark*}
The space of $E$-valued distributions depends only on the\break bounded sets
of $E$.
\end{remark*}

\begin{proof}
Since $\mathscr{D}$ is bornological (Theorie des distributions, 
Tome~1,\break p.~71) a linear map from $\mathscr{D}$ to $E$ is continuous if and
only if it takes bounded sets of $\mathscr{D}$ into bounded sets of
$E$. Hence the space of $E$-valued distributions depends only on the
bounded sets of $E$. In particular if we replace the topology of $E$
by the weakened topology, due to the identity between the bounded sets
in the initial topology and the weakened topology, we have the space
of $E$-valued distributions to be the\pageoriginale same algebraically in the above
two cases.
\end{proof}

We denote by $\mathscr{D}' (E)$ the space of $E$-valued distributions.
\textit{Topology of $\mathscr{D}' (E)$}. On $\mathscr{D}' (E)$ we
put the topology of uniform convergence on bounded sets of
$\mathscr{D}$. Since bounded sets of $\mathscr{D}$ are relatively
compact, the topology that we introduce is the same as the topology of
uniform convergence on compact sets of $\mathscr{D}$. 

\medskip
\noindent{\textbf{Examples of vector valued distributions:}}

Let $T$ be a distribution on $R^N$ and $\overrightarrow{e}$ a fixed vector of
$E$. $T \overrightarrow{e}$ defined by $T \overrightarrow{e}(\varphi)
= T(\varphi) \overrightarrow{e}$
for every $\varphi \in \mathscr{D}$ is an $E$-valued distribution. $T
\overrightarrow{e}$ maps the whole of $\mathscr{D}$ either into a one-dimensional
subspace of $E$ or into zero according as $\overrightarrow{e} \neq 0, T\neq 0$ or
one of the quantities $e$ and $T$ is zero.

The map $(T, \overrightarrow{e}) \to T\overrightarrow{e}$ of $\mathscr{D}'x E$ into
$\mathscr{D}'(E)$ is a bilinear map and hence induces a linear map
$i:\mathscr{D}'\otimes E \to \mathscr{D}'(E)$. This map `$i$' is an
injection. For, let $\{\overrightarrow{e}_\nu\}$ be a basis of $E$. Any element
of $\mathscr{D}'\otimes E$ can be written as $\sum T_\nu \otimes
\overrightarrow{e}_\nu$, $T_\nu \in \mathscr{D}'$. Now, $i(\sum T_\nu
\otimes \overrightarrow{e}_\nu)=\sum T_\nu\overrightarrow{e}_\nu$.
 Hence if $i(\sum T_\nu \otimes \overrightarrow{e}_\nu) = 0$,
we have $\sum T_\nu \overrightarrow{e}_\nu = 0$, or $\sum T_\nu(\varphi)
\overrightarrow{e}_\nu = 0$ for every $\varphi \in \mathscr{D}$. The linear
independence of the $\overrightarrow{e}_\nu$'s gives $T_\nu(\varphi)=0$ for every
$\varphi \in \mathscr{D}$. Hence $\sum T_\nu \otimes \overrightarrow{e}_\nu = 0$,
which proves that $`i'$ is an injection.

It is easy to see that the image of $\mathscr{D}' \otimes E$ under
this injection is the space of continuous linear maps from
$\mathscr{D}$ into $E$ which are of finite rank, that is to say, which
map $\mathscr{D}$ into a finite dimensional subspace of $E$. When $E$
is finite dimensional, every $E$-valued distribution is of finite rank
and so $\mathscr{D}'(E)$ can be identified algebraically with\pageoriginale
$\mathscr{D}'\otimes E$. When $E$ is finite dimensional by choosing a
basis $\overrightarrow{e}_1,\ldots, \overrightarrow{e}_m$ of $E$ we see
that any  $\overrightarrow{T}\in\mathscr{D}'(E)$ can be written as 
$$
\overrightarrow{T}=T_1 \overrightarrow{e}_1+T_2 \overrightarrow{e}_2+ 
\cdots+ T_m\overrightarrow{e}_m,
$$
where $T_1,\ldots, T_m$ are uniquely determined scalar
distributions. Instead of giving $\overrightarrow{T}$ it suffices to
give the $m$-scalar distributions $T_1, T_2,\ldots,T_m$.

Now we give an example of a distribution which can have infinite rank.

If $f$ is a complex valued continuous function we know that $f$
defines a distribution, also denoted by $f$, in the following way:
$$
f(\varphi)=\int\limits_{R^N} f(x) \varphi (x) \,dx\quad \text{for every}\quad
\varphi \in \mathscr{D}. 
$$
We shall now define an analogous vector valued distribution. Let $E$
be a complete $ELC$. If $\varphi \in \mathscr{D}$ the function
$\overrightarrow{f} \varphi$ defined by $\overrightarrow{f} \varphi
(x)=\varphi (x) f(x)$ is an $E$-valued continuous function with
compact support. $E$ being complete, the integral $\int\limits_{R^N} f
(x)\varphi(x)\,dx$ (for the definition of this integral, see Bourbaki,
Integration, Chap.III $\S$ 4) is an element of $E$. 


The map $\varphi \to\overrightarrow{f}(\varphi)=\int\limits_{R^N}
\overrightarrow{f} (x) \varphi (x) \,dx$, which is evidently linear, is
an $E$-valued distribution. We have to prove the continuity of the map
$\varphi \to \overrightarrow{f} (\varphi)$ of $\mathscr{D}$ in
$E$. Suppose $\{\varphi_n\}$ is a sequence of functions all having
their supports in a fixed compact set $K$ and tending uniformly to
$0$, together with all their partial derivatives. Let $V(K)$ be the
volume of the compact set $K$. Then
$$
\int\limits_{R^N} f (x) \varphi (x) \,dx \in m V (K)
{\overrightarrow{f}(K)}
$$
(Bourbaki, Integration, Chap. III, $\S$ 4).

\noindent where \pageoriginale $m=\sup\limits_{x\in K} | \varphi (x)|$ and
$\overgroup{\overrightarrow{f} (K)}$ is the convex, closed envelope of
the compact set $\overrightarrow{f} (K)$. Since $E$ is complete
$\overgroup{\overrightarrow{f} (K)}$ is compact.

Hence
$$
\int\limits_{R^N} \overrightarrow{f} (x) \varphi_n (x) \,dx \in m_n
V(K) \overgroup{\overrightarrow{f}(K)},
$$
where $m_n=\sup\limits_{x\in K} |\varphi_n (x)|$. If $\varphi_n$ tend
to $0$ uniformly on $K$, we have $\overrightarrow{f}(\varphi_n)\to 0$ in $E$.
This proves the continuity of the map $\varphi \to \overrightarrow{f}
(\varphi)$.

\noindent{\textbf{The identity distribution}}. The identity map of
$\mathscr{D}$ into $\mathscr{D}$ is a continuous linear map of
$\mathscr{D}$ into $\mathscr{D}$, hence it is a $\mathscr{D}$-valued
distribution.
\begin{definition}\label{chap1:def1.2} 
An $E$-valued distribution $\overrightarrow{T}:\mathscr{D} \to E$ is said
to be of order $m$, $m$ an integer $\geq 0$, if
$\overrightarrow{T}$ can be extended into a continuous map from
$\mathscr{D}^m$ to $E$. (For the definition and the topology of
$\mathscr{D}^m$ refer to ``Theorie des distributions'', vol.\ 1).
\end{definition}

One knows that a scalar distribution is locally of finite order. But
the analogous result is in general false for vector valued
distributions. For example, the identity distribution (example (2.3))%%ref
is of infinite order in every open subset. In fact, if $i:\mathscr{D}
\to \mathscr{D}$ is of finite order, every $E$-valued distribution,
$E$ being a complete $ELC$, will be of finite order. For, let
$f:\mathscr{D} \to E$ be an $E$-valued distribution and
$\tilde{i}:\mathscr{D}^m \to \mathscr{D}$ be the extension of $i$ into
a continuous linear map of $\mathscr{D}^m$ in $\mathscr{D}$. Then
$\tilde{f}=f\circ\tilde{i}:\mathscr{D}^m \to E$ is an extension of $f$
into a continuous linear map of $\mathscr{D}^m$ into $E$. But given
any open set $\Omega$ there exists a distribution on $\Omega$ with
values in $C$ (field of complex numbers) which is of infinite order.

Suppose\pageoriginale $E$ and $F$ are two locally convex Hausdorff
spaces such that $E$ is a subspace of $F$ with a finer topology. It
may happen that an $E$-valued distribution $\overrightarrow{T}$ which
is of infinite order becomes a distribution of finite order considered
as a distribution with values in $F$. For example take $E=\mathscr{D}$
and $F=\mathscr{D}'$. The identity distribution of $\mathscr{D}$ with
values in $\mathscr{D}$ is of infinite order. But considered as a
$\mathscr{D}'$-valued distribution it is given by the indefinitely
differentiable $\mathscr{D}'$-valued function $\overrightarrow{f}$
defined as $\overrightarrow{f}(a)=\delta_a, \delta_a$ being the Dirac
distribution at $'a'$. It is easily seen that $\overrightarrow{f}$ is a
$C^\infty$, $\mathscr{D'}$-valued function. We now verify that the
distribution given by $\overrightarrow{f}$ is the same as the
$\mathscr{D'}$-valued distribution `$i$'.

The distribution defined by $\overrightarrow{f}$ maps any $\varphi
\in \mathscr{D}$ into the element
$\overrightarrow{f}(\varphi)=\int\limits_{R^N} f (a) \varphi (a)\,da$
of $\mathscr{D}'$. Now, I claim $\overrightarrow{f}(\varphi)$ is the
same as the element $i(\varphi)$ of $\mathscr{D}'$. $i(\varphi)$,
considered as an element of $\mathscr{D}'$ maps any $\psi \in
\mathscr{D}$ into the element $\varphi(\psi)= \int\limits_{R^N}
\varphi(a) \psi(a)\,da$ of $C$. To show that
$\overrightarrow{f}(\varphi) = i(\varphi)$ we have to show merely
$$
\langle \overrightarrow{f}(\varphi),\psi \rangle =
\varphi(\psi) \quad\text{for every} \quad \psi \in \mathscr{D}.
$$
Now,
\begin{align*}
\langle\overrightarrow{f}(\varphi),\psi \rangle &=
\langle\int\limits_{R^N} \overrightarrow{f}(a)\varphi(a)\,da, \psi \rangle =
\int\limits_{R^N} \langle \overrightarrow{f}(a)\varphi(a), \psi
\rangle\,da\\
& = \int\limits_{R^N} \langle \delta_a \psi(a), \psi \rangle\,da =
\int\limits_{R^N} \varphi(a)\langle\delta_a,\psi\rangle\,da\\
& = \int\limits_{R^N} \psi(a) \psi(a)\,da\\
& = \varphi(\psi).
\end{align*}

