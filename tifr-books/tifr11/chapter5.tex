
\chapter{The Approximation Property}\label{chap5}

\setcounter{section}{5}
\setcounter{definition}{0}
\begin{definition}\label{chap5:def5.1}
We\pageoriginale say that an $ELC$ $L$ has the approximation property
if $L' \otimes L$ is dense in $\mathscr{L}_c(L, L)$. 

Trivially if the identity map $I : L \to L$ is adherent to $L' \otimes
L$ in $\mathscr{L}_c(L, L)$, $L$ has the approximation property. In
fact, if $I$ is adherent to $L' \otimes L$ in $\mathscr{L}_c(L, L)$, we
have $L' \otimes M$ dense in $\mathscr{L}_c(L, M)$ for every $E L C$
$M$.

The spaces $\mathscr{D}, \mathscr{D}', \mathscr{E}, \mathscr{E}',
\mathscr{D}^m, L', (\mathscr{D}')^{(o)}, \mathscr{S}, \mathscr{S}',
\mathscr{O}_M$ and $\mathscr{O}_c'$ have all the approximation
property. It is not known whether $\mathscr{D}'^m$ with the strong
topology has the approximation property or not.
\end{definition}

\setcounter{section}{5}
\setcounter{prop}{0}
\begin{prop}\label{chap5:prop5.1}
If $L$ or $M$ has the approximation property, we have $L \otimes M$
dense in $L \mathcal{E} M$. If for every $E L C M$, $L \otimes M$ is
dense in $L \mathcal{E} M$, then $L$ has the approximation property.

Assume that $L$ has the approximation property. We have to prove that
continuous linear maps from $M_c' \to L$ of finite rank are dense in
$\mathscr{L}_{\mathcal{E}}(M_c', L)$. Since $L$ has the approximation property,
we can find a filter $(v_j)$ of maps of finite rank of $L$ into $L$
converging to the identity map in $\mathscr{L}_c(L; L)$, i.e., the
filter $(v_j)$ converges uniformly to $I$ on compact discs of $L$
(disc = convex, stable subset). Let $u \in \mathscr{L}_\mathcal{E}
(M_c' L) = L \mathcal{E} M$. Since every equicontinuous subset of $M'$
is contained in a compact disc of $M_c'$, the filter $(v_j \circ u)$
converges uniformly on equicontinuous subsets of $M_c'$ to $u$. Also
$v_j \circ u$ are maps of finite rank of $M_c'$ in $L$. Hence the
required result follows.

Conversely,\pageoriginale suppose $L \otimes M$ is dense in $L
\mathcal{E} M$ for every $ E L C M$. Taking for $M$ the space $L_c'$,
we have $L \otimes L_c'$ dense in $L \mathcal{E} L_c'$. The topology
of $(L_c')_c'$ is finer than that of $L$, though algebraically they
are the same. Hence $I : (L_c')_c' \to L$ is continuous and hence it
can be approximated by continuous maps of finite rank of $L$ in $L$ on
equicontinuous subsets of $(L_c')'$. Any convex, compact stable subset
of $L$ is equicontinuous in $(L_c')'$. This proves our proposition.
\end{prop}

As pointed out in the previous lecture, we have

\begin{itemize}
\item [(1)] $\mathscr{D}' (E) \approx \mathscr{D}' \mathcal{E} E \approx
  \mathscr{L}_\delta (\mathscr{D}, E) \approx \mathscr{L}_\mathcal{E}
  (E_c', \mathscr{D}')$
\item [(2)] $\mathscr{S}'(E) \approx \mathscr{S}' \mathcal{E} E
  \approx \mathscr{L}_\delta (\mathscr{S}, E) \approx
  \mathscr{L}_\mathcal{E} (E_c', \mathscr{S}')$, and 
\item [(3)] $\mathscr{E}' (E) \approx \mathscr{E}' \mathcal{E} E
  \approx \mathscr{L}_\delta (\mathscr{E}, E) \approx
  \mathscr{L}_\mathcal{E} (E_c', \mathscr{E}')$.
\end{itemize}

\begin{definition}\label{chap5:def5.2}
Let $\overrightarrow{T}$ be an $E$-valued distribution. The support of
$\overrightarrow{T}$ is, by definition, the smallest closed set
$\Omega\subset R^n$ such that if $\varphi$ is any $C^\infty$ function
with compact support whose support is contained in the complement of
$\Omega$, we have $\overrightarrow{T}(\varphi)=0$. 
\end{definition}

\begin{remark*}
An element of $\mathscr{E}' (E)$ need not have compact support. In
fact, the identity map $I : \mathscr{E} \to \mathscr{E}$ is an element
of $\mathscr{E}' (\mathscr{E})$. It does not have a compact
support. For, if it had a compact support $K$, every continuous image
of $I$ will have its support in $K$. In particular, every
scalar-valued distribution with compact support, being a continuous
image of $I$, will have its support in $K$, a fixed compact set, which
is absurd.

However, if an $E$-valued distribution $\overrightarrow{T}$ has a
compact support, $\overrightarrow{T} \in \mathscr{E}' (E)$. In fact,
${}^t \overrightarrow{T} : E_c' \to \mathscr{D}'$ maps $E_c'$ into 
 $\mathscr{E}'$.
\end{remark*}

\setcounter{prop}{2}
\begin{prop}\label{chap5:prop5.3}
If\pageoriginale $E$ has a neighbourhood of $0$ which does not contain
any straight line, then every element of $\mathscr{E}' (E)$ has a
compact support.
\end{prop}

\begin{proof}
Let $V$ be a neighbourhood of $0$ in $E$ not containing any straight
line and $\overrightarrow{T} : \mathscr{E} \to E$ a continuous linear
map. Since $\overrightarrow{T}$ is continuous, $\exists$ an integer $m
\geq 0$, a compact set $K$ and an $\varepsilon > 0$ such that for every
$\varphi \in \mathscr{E}$ with $\sup \limits_{|p|\leq m, x \in
  K} |D^p \varphi (x)| \leq \varepsilon$ we have $\overrightarrow{T} (\varphi)
\in V$. Let $\psi \in \mathscr{D}$ with support in the
complement of $K$. Then $\sup \limits_{|p|\leq m, x \in K} |D^p
\lambda \psi (x)|=0$ and hence $\lambda \overrightarrow{T} (\psi)
\in V$ for every $\lambda$. Since $V$ does not contain any
straight line, $\overrightarrow{T} (\psi)=0$ or the support of
$\overrightarrow{T}$ is contained in $K$. Therefore
$\overrightarrow{T}$ has a compact support.  
\end{proof}

\begin{coro*}
$E$ having a neighbourhood of $0$ not containing any straight line is
  equivalent to saying that there exists a continuous semi-norm on $E$
  which is a norm.

If $E$ is a normed space, then any $\overrightarrow{T} \in
\mathscr{E}'(E)$ has a compact support.
\end{coro*}

\begin{definition}\label{chap5:def5.3}
The space of $E$-valued distributions with compact support is denoted
by $\tilde{\mathcal{E}}' (E)$.

We have $\tilde{\mathscr{E}}' (E) \subset \mathscr{E}' (E)$ algebraically.
\end{definition}

\begin{definition}\label{chap5:def5.4}
$\tilde{\mathscr{E}}^m (E)$ is the space of $m$-times continuously
  differentiable functions from $R^n$ to $E$.
\end{definition}

\begin{prop}\label{chap5:prop5.4}
We have the algebraic inclusion $\mathscr{E}^m(E)
\subset\tilde{\mathscr{E}}^m(E)$. 
\end{prop}

\begin{proof}
Let $\overrightarrow{T}$ be a continuous linear map $\mathscr{E'}_c^m
\to E$. Let $\overrightarrow{f}$ be an $E$-valued function defined as
follows: $\overrightarrow{f} (a) = T (\delta_a)$. Now\pageoriginale
the map $a \to \delta_a$ is an $m$-times continuously differentiable
function of $R^n$ with values in $\mathscr{E}_c'^m$ and $T$ is a
continuous linear map. Hence $\overrightarrow{f}$ is an $m$-times
continuously differentiable function. We show that
$\overrightarrow{T}$ is the distribution defined by the function
$\overrightarrow{f}$. We have

\begin{align*}
\langle \overrightarrow{f}, \overleftarrow{e}'\rangle (a) = \langle
\overrightarrow{f} (a), \overleftarrow{e}'\rangle &= \langle
\overrightarrow{T} (\delta_a), \overleftarrow{e}'\rangle = \delta_a
(\langle \overrightarrow{T}, \overleftarrow{e}'\rangle )\\
&= \langle \overrightarrow{T}, \overleftarrow{e}'\rangle (a)\quad \text{(as
  a function)}
\end{align*}
\end{proof}
This proves our assertion.

\begin{prop}\label{chap5:prop5.5}
If $E$ is complete, $\tilde{\mathscr{E}}^m(E)=\mathscr{E}^m(E)$. 
\end{prop}

\begin{proof}
If we prove that $\tilde{\mathscr{E}}^m(E) \subset
\mathscr{E}^m(E)$. we are through, because of proposition~\ref{chap5:prop5.4}.
Let $\overrightarrow{f} \in \tilde{\mathscr{E}}^m(E)$. Since $E$ is
complete, $\overrightarrow{f}$ can be used to define a distribution
$\overrightarrow{f} (\varphi)=\int\limits_{R^n} \overrightarrow{f}(x)
\varphi (x)\,dx$. It is evident that the above distribution
$\overrightarrow{f}$ scalarly belongs to $\mathscr{E}^m$. Now
$\mathscr{E}^m$ does not satisfy the $\mathcal{E}$-property. We cannot
immediately conclude that $\overrightarrow{f} \in
\mathscr{E}^m(E)$. We have to prove that the map $E_c' \to
\mathscr{E}^m$ defined by $\overrightarrow{f}$ is continuous. Suppose
$\overleftarrow{e}' \to 0$ in $E_c'$, we have to show that $\langle
\overrightarrow{f}, \overleftarrow{e}'\rangle \to 0$ in
$\mathscr{E}^m$, i.e., $D^p\langle \overrightarrow{f},
\overleftarrow{e}'\rangle \to 0$ uniformly on every compact subset
$K\subset R^n$ for $|p|\leq m$. But $D^p\langle \overrightarrow{f},
\overleftarrow{e}'\rangle=\langle D^p f, \overleftarrow{e}'\rangle$
for $|p|\leq m$. For each $p$ with $|p|\leq m$, the set of values $D^p
\overrightarrow{f}(x)$, $x \in K$ is a compact set in $E$. Since $E$
is complete the convex, stable, closed envelope of the set
$\underset{x \in K}{\{D^p \overrightarrow{f}}(x)\}$ is compact
and so, $\langle D^p \overrightarrow{f},\overleftarrow{e}' \rangle \to 0,
|p|\leq m$ uniformly for $ x \in K$.
\end{proof}

\noindent{\bf Characterization of $\mathscr{E}^m(E)$}. In the general
case, one sees that $\mathscr{E}^m(E)$ is the set of all $m$-times
continuously differentiable functions $\overrightarrow{f}$ satisfying
the following conditions: For each $p$ with $|p|\leq m$, and for each
compact set $K \subset R^n$, the convex, stable, closed envelope of
$D^p \overrightarrow{f}(K)$ is compact in $E$. 

Let\pageoriginale $E$ be a complete $E L C$. Let $\mathfrak{H}$ denote
the space of holomorphic functions on $R^{2n}$ provided with the
canonical complex structure. We put on $\mathfrak{H}$ the topology
induced by that of $\mathscr{E}^\circ$. Let $\mathfrak{H}(E)$ denote
the space $\mathfrak{H} \varepsilon E$.

\begin{definition}\label{chap5:def5.5}
Any element $\overrightarrow{f} \in \mathfrak{H}(E)$ is called a
holomorphic function with values in $E$.
\end{definition}

\setcounter{prop}{6}
\begin{prop}\label{chap5:prop5.7}
Let $\overrightarrow{f} (z)$ be an $E$-valued function such that for
every $\overleftarrow{e}'$, the function $\varphi_\mathfrak{H}$
defined by $\varphi_{\overleftarrow{e}'} (z) = \langle
\overrightarrow{f} (z), \overleftarrow{e}'\rangle$ is in
$\mathfrak{H}$. Then $\overrightarrow{f} \in \mathfrak{H} (E)$ and we
have a formula similar to the formula of Cauchy:
$$
\overrightarrow{f} (z) = \tfrac{1}{2\pi i} \int
\tfrac{\overrightarrow{f}(\zeta)}{\zeta - z}d\zeta.
$$
\end{prop}

\begin{proof}
Since $\mathfrak{H}$ has the $\mathcal{E}$-property, if
$\overrightarrow{f}$ belongs scalarly to $\mathfrak{H},
\overrightarrow{f}$ belongs to $\mathfrak{H}(E)$. This proves the
first part. To prove the second part we see that for every
$\overleftarrow{e}' \in E', \varphi_{\overleftarrow{e}'}$ being a
scalar-valued holomorphic function, we have 
\begin{align*}
\varphi_{\overleftarrow{e}'}(z) = \langle \overrightarrow{f} (z),
\overleftarrow{e}' \rangle &= \frac{1}{2 \pi i} \int
\frac{\varphi_{\overleftarrow{e}'} (\zeta)}{\zeta - z}\,d\zeta\\
&= \frac{1}{2 \pi i} \int \frac{\langle \overrightarrow{f}(\zeta),
  \overleftarrow{e}'\rangle}{\zeta - z}d\zeta.\\
\text{That is to say,}\qquad \langle \overrightarrow{f}(z),
\overleftarrow{e}' \rangle &= \frac{1}{2 \pi i} \int \frac{\langle
  \overrightarrow{f} (\zeta), \overleftarrow{e}' \rangle}{\zeta - z}
d\zeta\\ 
&= \frac{1}{2 \pi i} \int \left\langle
\frac{\overrightarrow{f}(\zeta)}{\zeta - z},
\overleftarrow{e}'\right\rangle\, d\zeta\\
&= \left\langle \frac{1}{2 \pi i} \int
\frac{\overrightarrow{f}(\zeta)}{\zeta - z}\, d\zeta,
\overleftarrow{e}'\right\rangle.\\
\end{align*}
Hence $\overrightarrow{f}(z) = \frac{1}{2 \pi i} \int
\frac{\overrightarrow{f}(\zeta)}{\zeta - z}\, d\zeta$.

\end{proof}
