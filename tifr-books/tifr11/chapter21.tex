
\chapter[Remarks on the representation...]{Remarks on the representation of non-commutative Lie
  semi-groups}\label{chap21}

Let\pageoriginale $G$ be a Lie group of dimension $n$ and $G_+$ a sub semi-group of
$G$ such that 
\begin{itemize}
\item [i)] $G_+$ is closed in $G$;
\item [ii)] the unit element of $G$ belongs to $G_+$.
\item [iii)] $G_+$ is the closure of its interior.
\end{itemize}

We call $G_+$ a Lie semi-group.

We denote by $\mathscr{D}_{L^1}'(G_+)$ the set of distributions $T$ on
$G$ with support in $G_+$ which are of the form
$$
T=\sum\limits_{p}D_p*\mu_p \quad \text{(finite sum)}
$$
where $D_p$ are distributions\footnote{By a distribution we mean a
  continuous linear form on the space of indefinitely differentiable
  \emph{functions} on $G$ with compact supports.} 
in $G$ with support at the unit element
$e$, and $\mu_p$ are summable measures on $G$. A sequence $T_j\in
\mathscr{D}_{L^1}'(G_+)$ is said to converge to zero strictly if $T_j=
\sum\limits_p D_p*\mu_{p,j}$, where $D_p$ are fixed distributions with
support at $e$ of order at most $m$, and if for every $p$ the measures
$\mu_{p,j}$ converge to zero strictly.

Let $U$ be a representation of $G_+$, by equicontinuous operators in
$E$. For $T\in\mathscr{D}_{L^1}'(G_+)$ we shall define a linear
operator $U(T)$ on $E$. Let $\mathscr{F}=\{\alpha_j\}$ be a filter of
$C^\infty$ $n$-forms, ($n=$ dimension of $G$) of the second kind (odd
type) on $G$ such that 

\begin{itemize}
\item [i)] $\{\alpha_j\}$ have compact supports contained in $G_+$;
\item [ii)] $\alpha_j\geq 0$, for every $j$;



\item [iii)] The\pageoriginale support of $\alpha_j$ tends uniformly
  to the unit element of $G$;
\item [iv)] $\int\limits_G\alpha_j\to 1$.
\end{itemize}

By definition the domain $E_T$ of $U(T)$ will consist of those $x$ in
$E$ for which $\lim\limits_{\mathscr{F}}U(\alpha *T)x$ exists and
$U(T)x$ is defined to be this limit. (Note the order in which $\alpha$
and $T$ enter in the convolution).

\begin{remark*}
If $\mathscr{F}'$ is another filter having the same properties as
$\mathscr{F}$ and $U'(T)$ the corresponding operator, it will follow
from the results to be indicated later, that $U(T)$ and $U'(T)$ have
the same domain of definition and are equal on their common domain of
definition.

Proposition \ref{chap19:prop19.1} is true also in the non-abelian
case, if $\rho$ is a $C^\infty$ form with compact support. To uphold
this proposition in the non-abelian case we have to prove the
following: if $D$ is a distribution with support at the origin and
$\mu$ a summable measure, then $(\rho *\alpha)*(D *\mu)\to\rho
*D*\mu$ strictly as $\alpha\to\delta$ following $\mathscr{F}$. For
this, it is sufficient to prove that $\rho *\alpha *D\to\rho *D$
strictly. But this follows from the separate continuity of the
convolution map $\mathscr{D}^n\times\mathcal{E}^1\to\mathscr{D}^n$. 

Proposition 19.1' is also true, if $S$ has compact
support. Proposition \ref{chap19:prop19.2} is also true; proof is the
same.

Suppose $x$ is an element of $E$ such that $\lim\limits_{\mathscr{F}}U
(T*\alpha)x=y$ exists $(T\in\mathscr{D}_{L^1}'(G_+))$. Since
$T*\alpha\to T$ strictly we see, using Proposition
\ref{chap19:prop19.2} that $x$ belongs to the domain of $U(T)$ and
$U(T)x=y$. Thus the definition for $U(T)$ we have given using
convolution on the left by $\alpha$ gives a domain of definition for
$U(T)$ which is larger than the domain we would have obtained if we
chose to convolve on the right by $\alpha$. 
\end{remark*}
