
\chapter{The \texorpdfstring{$\mathcal{E}$}{E}-product of two locally convex Hausdorff
  spaces}\label{chap4}


Let\pageoriginale $L$ and $M$ be two locally convex Hausdorff vector spaces. We
shall define a space $L \mathcal{E} M$.

\setcounter{section}{4}
\setcounter{definition}{0}
\begin{definition}\label{chap4:def4.1}
$L \mathcal{E} M$ is the set of bilinear forms on $L_c' \times M_c'$
  hypocontinuous with respect to the equicontinuous subsets of $L'$
  and $M'$. (For the definition of hypocontinuity, see Bourbaki,
  E.V.T., Chap.~III, \S~4). $L \mathcal{E} M$ is a linear space. We
  put on $L \mathcal{E} M$ the topology of uniform convergence on
  products of equicontinuous subsets of $L'$ and $M'$.
\end{definition}

Let $\mathscr{E} \in L \mathcal{E} M, l' \in L'$ and $m' \in M'$. Write
$\mathscr{E}(l',m')=\langle l'|\mathscr{E}|m'\rangle$  For any fixed $m'\in M'$ the
mapping $l' \to \langle l'|\mathscr{E}|m'\rangle$ is a continuous linear form
on $L_c'$ and hence defines an element of $L$ which we denote by
$|\mathscr{E}|m'\rangle$. The mapping $m' \to |\mathscr{E}|m'\rangle$ is a
continuous, linear map of $M_c'$ in $L$. That it is linear is
trivial. To show that it is continuous we have to only show that if
$m' \to 0\;|\mathscr{E} |m'\rangle \to 0$ in $L$. Now $|\mathscr{E}|m'\rangle \to 0$
if for $l'$ lying in an equicontinuous subset of $L'$ we have $\langle
l'|\mathscr{E}| m'\rangle \to 0$ uniformly. But this is half of the
hypocontinuity assumption on $\mathscr{E}$. Hence $m'\to|\mathscr{E}|m'\rangle$ is a
continuous linear map of $M_c'$ into $L$. Thus $\mathscr{E}$ defines an
element of $\mathscr{L}(M_c', L)$. Similarly, using the other half of
the hypocontinuity hypothesis, we can show that $\mathscr{E}$ determines an
element of $\mathscr{L}(L_c', M)$. In fact this is nothing but the
transpose of the linear map $M_c' \to L$ that corresponds to $\mathscr{E}$.

Let\pageoriginale us denote by $\mathscr{E}'$ the continuous linear
map of $M_c'$ into $L$ that corresponds to $\mathscr{E}$. Then
$\mathscr{E}'(m')$ is that element of $L$ which satisfies
$\mathscr{E}\langle l', m'\rangle = \langle l|\mathscr{E}|m'\rangle =
\langle \mathscr{E}(m'), l'\rangle L, L'$. Conversely suppose that  
$\eta:M_c' \to L$ is a continuous linear map.

\noindent Then the bilinear map $\eta :L_c'\times M_c' \to C$ defined
by $\eta(l', m')=\langle\eta(m'), l'\rangle$ is hypocontinuous with
respect to the products of equicontinuous subsets of $L_c'$ and
$M_c'$. First we show that if $l'$ lies in an equicontinuous subset of
$L'$ and $m' \to 0, \langle \eta(m'), l'\rangle \to 0$
uniformly. Since $\eta$ is continuous, $\eta(m') \to 0$ in $L$ and
hence $\langle \eta(m'), l'\rangle \to 0$ uniformly if $l'$ lies in an
equicontinuous subset of $L'$. The transpose $t_\eta:L_c' \to M$ is
also continuous for $(M_c')_c'$ is finer than $M$. This gives the
other half of the hypocontinuity, namely, if $m'$ lies in an
equicontinuous set of $M'$ and $l' \to 0$ in $L'$,
$$
\langle \eta (m'), l'\rangle = \langle m', t_\eta(l')\rangle \to 0
$$
uniformly.

Thus we see that $L \mathcal{E} M \approx \mathscr{L}(M_c', L)$
algebraically. Similarly we have $L \mathcal{E} M \approx
\mathscr{L}(L_c', M)$ algebraically.

\noindent{\bf Topologies on $\mathscr{L}(M_c'; L)$ and
  $\mathscr{L}(L_c'; M)$}.

On both these spaces we put the $\mathcal{E}$-topology which we define
below.

\begin{definition}\label{chap4:def4.2}
Let $E$ and $F$ be two $ELC$. The $\mathcal{E}$-topology on the space
$\mathscr{L}(E_c', F)$ is the topology of uniform convergence on
equicontinuous subsets of $E'$.
\end{definition}

\setcounter{section}{4}
\setcounter{prop}{0}
\begin{prop}\label{chap4:prop4.1}
The algebraic isomorphisms between the three spaces $L \mathcal{E} M,
\mathscr{L}_\mathcal{E}(L_c', M)$ and $\mathscr{L}_\mathcal{E}(M_c',
L)$ are topological isomorphisms.

We\pageoriginale shall prove the isomorphism $L \mathcal{E} M \approx
\mathscr{L}_\mathcal{E} (M_c', L)$ is topological, the other case
being similar to this.

Let $\mathscr{E} \in L \mathcal{E} M$ and $\mathscr{E}'$ the
corresponding element in $\mathscr{L}_\mathcal{E}(M_c', L)$. We show
$\mathscr{E} \to 0$ in $L \mathcal{E} M \Longleftrightarrow \mathscr{E}'
\to 0$ in $\mathscr{L}_\mathcal{E}(M_c', L)$. Now, $\mathscr{E} \to 0$
in $L \mathcal{E} M$ if and only if for $l' \in P, m' \in Q, P$ and
$Q$ being arbitrary equicontinuous sets of $L'$ and $M'$ respectively,
we have $\mathscr{E}(l', m') \to 0$ uniformly. $\mathscr{E}'$ tends to
$0$, if and only if for $m'$ in any equicontinuous subset, say $R$ of
$M'$, $\mathscr{E}'(m') \to 0$ in $L$ or for $l'$ in any
equicontinuous set $S$ of $L'$, $\langle \mathscr{E}'(m'), l'\rangle
\to 0$ uniformly. This is precisely equivalent to $\mathscr{E}(l', m')
\to 0$ uniformly for $(l', m') \in P \times Q, P$ and $Q$ any
equicontinuous subsets of $L'$ and $M'$ respectively. Hence
$\mathscr{E} \to 0$ in $L \mathcal{E} M \Longleftrightarrow
\mathscr{E}'\to 0$ in $\mathscr{L}_\mathcal{E} (M_c', L)$. This
proves proposition \ref{chap4:prop4.1}.   
\end{prop}

\noindent{\bf Examples of the $\mathcal{E}$-product of a space of
 distributions and an $ELC$}.

\begin{itemize}
\item [1)] $\mathscr{D}' \mathcal{E} E$ $\approx
  \mathscr{L}_\mathcal{E}(E_c', \mathscr{D}') \approx
  \mathscr{L}_\mathcal{E}(\mathscr{D}; E)$ \hfill (by 
proposition~\ref{chap4:prop4.1}).

But we have $\mathscr{D}'(E) \approx
\mathscr{L}_\mathcal{E}(\mathscr{D}; E)$ topologically for in
$\mathscr{D}$, considered as the dual of $\mathscr{D}'$, bounded sets
are equicontinuous. 
\item [2)] $\mathscr{S}' \mathcal{E} E \approx
  \mathscr{L}_\mathcal{E}(\mathscr{S}; E) \approx \mathscr{L}(E_c',
  \mathscr{S}')$ and $\mathscr{S}'(E)\approx \mathscr{S}' \mathcal{E}
  E$ for $\mathscr{S}'(E) \approx \mathscr{L}_\mathcal{E}(\mathscr{S};
  E)$ topologically since $\mathscr{S}', \mathscr{S}$ are Montel\break
  spaces.
\item [3)] $\mathscr{O}_M \mathcal{E} E \approx
  \mathscr{L}_\mathcal{E} (E_c'; \mathscr{O}_M)$. 

\noindent (For the definition of the spaces $\mathscr{S},
\mathscr{S}', \mathscr{O}_M, \mathscr{O}_c, \mathscr{O}_M',
\mathscr{O}_c'$ refer: Theorie des distributions, Tome ii).
\item [4)] $\mathscr{E}'(E) \approx
  \mathscr{L}_\mathcal{E}(\mathscr{E};E)$ and $\mathscr{E}'
  \mathcal{E} E \approx \mathscr{L}_\mathcal{E}(\mathscr{E}; E)
  \approx \mathscr{L}_\mathcal{E}(E_c', \mathscr{E}')$. 
\end{itemize}

\noindent 
{\bf Covariance\pageoriginale property of the $\mathcal{E}$-product}.

Let $u : L_1 \to L_2$ and $v : M_1 \to M_2$ be continuous linear maps,
$L_1, M_1, L_2$ and $M_2$ being locally convex Hausdorff topological
vector spaces. We shall now see how with $u$ and $v$ a continuous
linear map, say $u \mathcal{E} v : L_1 \mathcal{E} M_1 \to L_2
\mathcal{E} M_2$ can be associated. There are, in fact, three ways of
defining this map according as we consider the three forms of writing
the $\mathcal{E}$-product, namely $L_1 \mathcal{E} M_1,
\mathscr{L}_\mathcal{E} (M_c', L_1)$ and
$\mathscr{L}_\mathcal{E}(L_{1c}', M_1)$.

\begin{defspl}\label{chap4:def4.3(1)}
Let $(l_2', m_2') \in L_2' \times M_2'$ and $\mathscr{E} \in L_1
\mathcal{E} M_1$. Let $(u \mathcal{E} v) (\mathscr{E})$ be the
bilinear form $\eta$ defined by 
$$
\eta(l_2', m_2') = \mathscr{E}(^tu(l_2'), {}^tv(m_2')).
$$
We now prove that $\eta \in L_2 \mathcal{E} M_2$. For this we have to
only prove hypocontinuity of $\eta$ with respect to the equicontinuous
subsets of $L_{2c}'$ and $M_{2c}'$. Suppose $l_2'$ lies in an
equicontinuous subset of $L_2'$. Then there exists a neighbourhood
$U_2$ of $0$ in $L_2$ such that $|\langle l_2, l_2'\rangle|<1$ for
$l_2 \in U_2$. Since $u$ is continuous, $U_1 = u^{-1}(U_2)$ is a
neighbourhood of $0$ in $L_1$ and for any $l_1$ in $U_1$ we have
$|\langle l_1, t_u(l_2')\rangle| = |\langle u(l_1), l_2'\rangle|$. But
$u(l_1) = l_2 \in U_2$.

Hence $|\langle l_1, t_u(l_2')\rangle |<l$ for every $l_1 \in
U_1$. Hence the set $\{{}^tu(l_2')|l_2' \in$ an equicontinuous subset of
$L_2'\}$ is an equicontinuous subset of $L_1'$. Let $m_2' \to 0$ in
$M_{2c}'$. Then since $t_v: M_{2c}' \to M_{1c}'$ is continuous,
${}^tv(m_2')^{\to 0}$ in $M_{1c}'$. Hence $\mathscr{E}{}^{(t}u(l_2');
{}^tv(m_2')^{)\to 0}$ uniformly if $l_2'$ lies in an equicontinuous
subset of $L_2'$ and $m_2'\to 0$ in $M_{2c}'$. Similarly we prove the
other part of the hypocontinuity.

Thus\pageoriginale $\eta \in L_2 \mathcal{E} M_2$. The mapping
$\mathscr{E} \to \eta$ is denoted by $u \mathcal{E} v$.   
\end{defspl}


\begin{defspl}\label{chap4:def4.3(2)}
Let $\mathscr{E} \in \mathscr{L}_\mathcal{E}(M_{1c}', L_1)$. Let $u :
L_1 \to L_2$ and $v : M_1 \to M_2$ be continuous linear maps. Then
$t_v \uarrow{M'_{1c}}{M'_{2c}}$ and $\darrow{L_{1}}{L_{2}}u$ are
continuous linear maps. $\mathscr{E} : M_{1c}' \to L_1$ is continuous
linear. The composite $\eta = u \circ \mathscr{E} \circ t_v : M_{2c}'
\to L_2$ is a continuous linear map and hence $\eta \in
\mathscr{L}(M_{2c}', L_2)$. With $\mathscr{E} \in
\mathscr{L}_\mathcal{E}(M_{1c}', L_1)$ we associate the element $\eta
\in \mathscr{L}_\mathcal{E}(M_{2c}', L_2)$. 
\end{defspl}


\begin{defspl}\label{chap4:def4.3(3)}
Let $\mathscr{E} \in \mathscr{L}_\mathcal{E}(L_{1c}', M_1)$. $t_u :
L_{1c}' \leftarrow L_{2c}', v:M_1 \to M_2$ continuous. Hence the
composite $\eta = v \circ \mathscr{E} \circ t_u : L_{2c}' \to M_2$ is
a continuous linear map. With $\mathscr{E} \in \mathscr{L}_\mathcal{E}
(L_{1c}', M_1)$ we associate the element $\eta \in
\mathscr{L}_\mathcal{E} (L_{2c}', M_2)$.
\end{defspl}

\begin{prop}\label{chap4:prop4.2}
The above three definitions give one and the same element $\eta$ of
$L_2 \mathcal{E} M_2 \approx \mathscr{L}_\mathcal{E} (M_{2c}', L_2)
\approx \mathscr{L}_\mathcal{E} (L_{2c}', M_2)$.
\end{prop}

\begin{proof}
Let us, for the sake of clarity, denote the elements got from
definitions~4.3 (\ref{chap4:def4.3(1)}), 4.3 (\ref{chap4:def4.3(2)}) and
4.3 (\ref{chap4:def4.3(3)}) by $\eta_1, \eta_2$ and
$\eta_3$. Our assertion will be proved if we show 
$$\eta _1(l_2', m_2') = \langle \eta_2(m_2'), l_2'\rangle_{L_2, L_2'} =
\langle \eta_3(l_2'), m_2'\rangle_{M_2, M_2'}.
$$
\begin{equation*}
\text{Now},\quad\eta_1(l_2',m_2') =\mathscr{E}{}^{(t}u(l_2')'\;{}^t v(m_2')^)
\tag{i}\label{c4:eqi}
\end{equation*}
\begin{align*}
\langle\eta_2(m_2'), l_2'\rangle_{L_2, L_2'}  
&= \langle u \circ \mathscr{E}\circ{}^tv(m_2'), l_2'\rangle_{L_2, L_2'}\\
&= \langle \mathscr{E}{}^ tv(m_2')'\;{}^tu(l_2')\rangle_{L_1, L_1'}\\
&= \mathscr{E}({}^tu(l_2')'\;{}^tv(m_2')^)\tag{ii}\label{c4:eqii}\\
\end{align*}
\begin{align*}
\text{Similarly}\quad \langle \eta_3(l_2'), m_2'\rangle_{M_2, M_2'} =
\mathscr{E} {}^{(t}u(l_2')'\;{}^tv(m_2')^)\tag{iii}\label{c4:eqiii} 
\end{align*}
A comparison\pageoriginale of \eqref{c4:eqi}, \eqref{c4:eqii}, and 
\eqref{c4:eqiii} gives the required proposition. 
\end{proof}

\begin{prop}\label{chap4:prop4.3}
If $u : L_1 \to L_2$ and $v : M_1 \to M_2$ are injections $u
\mathcal{E} v : L_1 \mathcal{E} M_1 \to L_2 \mathcal{E} M_2$ is also
an injection.
\end{prop}

\begin{proof}
We have to show that $(u \mathcal{E} v) \mathscr{E}_1 = 0 \Rightarrow
\mathscr{E}_1 = 0$. Now $(u \mathcal{E} v) \mathscr{E}_1 = \langle
{}^tu(l_2')|\mathscr{E}_1|{}^tv(m_2')\rangle$. Since $u$ and $v$ are
injections, $t_u(L_2')$ and $t_v(M_2')$ are dense in $L_1'$ and
$M_1$. Hence $\mathscr{E}_1$ which is a separately continuous bilinear
form, is zero on the product $t_u(L_2')\times t (M_2')$ where $t_u
(L_2')$ and $t_v (M_2')$ are dense subspaces of $L_1'$ and
$M_1'$. Hence $\mathscr{E}_1 =0$. (See Bourbaki, EVT, Chap.~III,
\S~4, No.~3). 

In fact, one can even show that if $u : L_1 \to L_2$ and $v : M_1 \to
M_2$ are monomorphisms, $u \mathcal{E} v : L_1 \mathcal{E} M_1 \to L_2
\mathcal{E} M_2$ is a monomorphism. That is to say if we assume that
$u : L_1 \to u(L_1)$ is a topological isomorphism with the topology
induced on $u(L_1)$ by $L_2$ and $v : M_1 \to v(M_1)$ is a topological
isomorphism with the topology induced by $M_2$, then $u \mathcal{E} v
: L_1 \mathcal{E} M_1 \to u \mathcal{E} v(L_1 \mathcal{E} M_1)$ is a
topological isomorphism with the induced topology.

Also we have $L \otimes M \subset L \mathcal{E} M$. The topology on $L
\otimes M$ induced by $L \mathcal{E} M$ is called the
$\mathcal{E}$-topology and provided with this topology $L \otimes M$
is denoted by $\fibreoproduct{L}{M}{\mathcal{E}}$ 
\end{proof}

\begin{prop}\label{chap4:prop4.4}
If $L$ and $M$ are complete, $L \mathcal{E} M$ is complete.
\end{prop}

\begin{proof}
Let $(\mathscr{E}_j)$ be a Cauchy filter on $L \mathcal{E} M$. This
Cauchy filter gives rise to a Cauchy filter, which also we denote by
$(\mathscr{E}_j)$, on $\mathscr{L}_{\mathcal{E}}(L_c', M)$. 
Since $M$ is complete it
follows that there exists a linear map $\mathscr{E} : L_c' \to M$ such
that $\mathscr{E}_j$ converges to $\mathscr{E}$ uniformly on every
equicontinuous subset of $L'$. Similarly $(\mathscr{E}_j)$ defines a
Cauchy\pageoriginale filter on $\mathscr{L}_\mathcal{E} (M_c', L)$
which also we denote by $(\mathscr{E}_j)$ and this defines a linear
map $\mathscr{E}' : M_c' \to L$. Also trivially $\mathscr{E}$ and
$\mathscr{E}'$ are transposes of each other. Now, every equicontinuous
subset of $L'$ is contained in a compact (for $L_c'$), convex,
equicontinuous subset of $L'$. Since the restriction of $\mathscr{E}$
to every equicontinuous subset of $L'$ is continuous, it follows that the
image of every equicontinuous subset of $L'$ by $\mathscr{E}$ is
contained in a compact convex subset of $M$. This proves that
$\mathscr{E}' : M_c' \to L$ is continuous. Similarly $\mathscr{E} :
L_c' \to M$ is continuous. Consequently $\mathscr{E} \in L \mathcal{E}M$.     
\end{proof}
