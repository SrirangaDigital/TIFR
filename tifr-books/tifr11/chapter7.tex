\chapter[Multiplicative product of a vector...]{Multiplicative product of a vector valued distribution and 
a scalar valued distribution}\label{chap7}

Let\pageoriginale $\mathscr{H}, K$ and $\mathscr{L}$ be three spaces
of distributions on $R^n$.

\setcounter{section}{7}
\setcounter{definition}{0}
\begin{definition}\label{chap7:def7.1}
A bilinear map $U$ of $\mathscr{H} \times \mathscr{K} \to \mathscr{L}$
which is separately continuous and which coincides with the
multiplication of functions on $\mathscr{D} \times \mathscr{D}$ is
called a multiplication between the elements of $\mathscr{H}$ and the
elements of $\mathscr{K}$ with values in $\mathscr{L}$. 

If $S\in\mathscr{H}, T\in\mathscr{K}$, we write $S U T$ for $U(S, T)$.
\end{definition}

\setcounter{section}{7}
\setcounter{theorem}{0}
\begin{theorem}\label{chap7:thm7.1}
Let $\mathscr{H}, \mathscr{K}$ and $\mathscr{L}$ be any three locally
convex spaces. Let $E$ be a complete $E L C$. Let $U : \mathscr{H}
\times \mathscr{K} \to \mathscr{L}$ be a bilinear map hypocontinuous
with respect to the bounded sets of $\mathscr{H}$ and
$\mathscr{K}$. Then we can define a bilinear map $\tilde{U}:
\mathscr{H}(E) \times \mathscr{K} \to \mathscr{L}(E)$ which is
separately continuous and which satisfies $S \overrightarrow{e}
\tilde{U} T = (S U T) \overrightarrow{e}$ for every $S \in
\mathscr{H}, \overrightarrow{e} \in E$ and $T \in \mathscr{K}$;
moreover it is hypocontinuous with respect to the bounded sets of
$\mathscr{H}(E)$ and $\mathscr{K}$.

If $\mathscr{H}$ satisfies the approximations property, the bilinear
map that we define is the only bilinear map which is separately
continuous and which satisfies $S \overrightarrow{e} \tilde{U} T = 
(S U T) \overrightarrow{e}$ for every $S \in \mathscr{H},
\overrightarrow{e} \in E$ and $T\in\mathscr{K}$.
\end{theorem}

\begin{proof}
Let $T$ be any element of $\mathscr{K}$. Then it defines a continuous
linear map $m_T : \mathscr{H}\to\mathscr{L}$ as follows: $m_T(S) = S
U T$ for every $S \in \mathscr{H}$. Let $I : E \to E$ be the
identify map of $E$ in $E$. The map $m_T \in I :
\mathscr{H}(E) \to \mathscr{L}(E)$ (for the definition of $m_T
\in I$ refer to Lecture~\ref{chap4}) is a continuous linear map. We
define $\overrightarrow{S} \tilde{U} T$ to be $m_T \in I
(\overrightarrow{S})$ for every $\overrightarrow{S} \in
\mathscr{H}(E)$. We shall prove that $\tilde{U}$ thus defined has all
the properties mentioned in the theorem.
\end{proof}

We\pageoriginale have $\mathscr{H}(E)\approx \mathscr{L}_\varepsilon
(E_c', \mathscr{H})\approx \mathscr{L}_\varepsilon(\mathscr{H}_c',
E)$. From definitions~4.3 (\ref{chap4:def4.3(1)}), 4.3~(\ref{chap4:def4.3(2)}) 
and 4.3~(\ref{chap4:def4.3(3)}), we have the following results.
\begin{itemize}
\item [(i)] Considered as an element of $\mathscr{L}_\varepsilon
  (E_c', \mathscr{L})$, $m_T \in I (\overrightarrow{S}) =
  \overrightarrow{S}\tilde{U}T$ is the same as the composite of the
  maps
$$
t_I : E_c' \to E_c', \overrightarrow{S} : E_c' \to \mathscr{H}, m_T :
\mathscr{H} \to \mathscr{L}
$$
where $t_T$ is the transpose of the identity mapping of $E$ in $E$; in
other words, $t_T$ is the identity mapping of $E_c'$.
\item [(ii)] Considered as an element of $\mathscr{L}_\varepsilon
  (\mathscr{L}_c', E), m_T \in I (\overrightarrow{S})$ is the
  same as the composite of the maps
$$
t_{m_T} : \mathscr{L}_c' \to \mathscr{H}_c', \overrightarrow{S} :
\mathscr{H}_c' \to E, I : E \to E,
$$
$\overrightarrow{S}$ being considered as an element of
$\mathscr{L}_\varepsilon (\mathscr{H}_c', E)$. 
\end{itemize}

First we shall show that $\tilde{U}$ is hypocontinuous with respect to
the bounded subsets of $\mathscr{H}(E)$ and $\mathscr{K}$. Let
$\overrightarrow{S}$ remain in a bounded set of $\mathscr{H}(E)$ and
$T \to 0$ in $\mathscr{K}$. To show that $\overrightarrow{S} \tilde{U}
T$ tends to $0$ in $\mathscr{L} (E) = \mathscr{L}_\varepsilon (E_c',
\mathscr{L})$, we have to prove that if $\overleftarrow{e}'$ lies in
an equicontinuous set of $E', \overrightarrow{S} U T
(\overleftarrow{e}') \to 0$ uniformly in $\mathscr{L}$. Since
$\overrightarrow{S}$ lies in a bounded set $B,
\underset{\overrightarrow{S}\varepsilon B,
  \overleftarrow{e}'\varepsilon H}{U \langle
  \overrightarrow{S},}\overleftarrow{e}'\rangle$, $H$ being an
equicontinuous subset of $E'$, is bounded in $\mathscr{H}$. Hence
$\langle \overrightarrow{S}, \overleftarrow{e}'\rangle U T$ tends to
$0$ uniformly in $\mathscr{L}$. But one sees easily that $\langle
\overrightarrow{S}, \overleftarrow{e}'\rangle UT =
\overrightarrow{S}\tilde{U} T (\overleftarrow{e}')$. Hence
$\overrightarrow{S} U T (\overleftarrow{e}') \to 0$ uniformly in
$\mathscr{L}$. In other words, $\overrightarrow{S} U T$ tends to in
$\mathscr{L} (E)$ uniformly when $\overrightarrow{S}$ remains in a
bounded set of $\mathscr{H}(E)$ and $T \to 0$ in $\mathscr{K}$.

Now suppose $\overrightarrow{S} \to 0$ in $\mathscr{H} (E)$ and $T$
remains bounded in $\mathscr{K}$. Then for $\overleftarrow{e}'$ lying
in an equicontinuous set $H$ of $E', \langle \overrightarrow{S},
\overleftarrow{e}' \rangle \to 0$ uniformly in $\mathscr{H}$ and hence
when $T$ lies in a bounded set of $\mathscr{K},\langle
\overrightarrow{S}, \overleftarrow{e}'\rangle UT =
\overrightarrow{S}\tilde{U} T (\overleftarrow{e}') \to
0$\pageoriginale uniformly in $\mathscr{L}$. Hence $\overrightarrow{S}
\tilde{U}T\to 0$ in $\mathscr{L}(E)$ uniformly.

It is trivially seen that $S \overrightarrow{e} \tilde{U} T = (S U T)
\overrightarrow{e}$ for every $S\in\mathscr{H}, \overrightarrow{e}\in
E, T \in \mathscr{K}$. 

Now suppose $\mathscr{H}$ satisfies the approximation property. Then
$\tilde{U} : \mathscr{H}(E)\times \mathscr{K} \to \mathscr{L} (E)$ is the
only bilinear map separately continuous and satisfying
$S\overrightarrow{e} \tilde{U} T = (S U T) \overrightarrow{e}$ for
$S\in\mathscr{H}, \overrightarrow{e} \in E$ and $T \in
\mathscr{K}$. For if $U'$ is another such bilinear map, we have
$\tilde{U} \left|(\mathscr{H}\otimes
E)\times\mathscr{K}=U'\right|(\mathscr{H}\otimes
E)\times\mathscr{K}$. Since $\mathscr{H}\otimes E$ is dense in
$\mathscr{H}(E)$, we have $\tilde{U}=U'$ from the separate continuity
of both $\tilde{U}$ and $U'$. This proves our theorem.

\setcounter{section}{7}
\setcounter{prop}{1}
\begin{prop}\label{chap7:prop7.2}
Let $\mathscr{H}$ and $\mathscr{L}$ be normal spaces of distributions
and $\mathscr{K}$ a locally convex Hausdorff topological vector
space. Let $U: \mathscr{H}\times\mathscr{K} \to \mathscr{L}$ be a
bilinear map hypocontinuous with respect to bounded subsets of
$\mathscr{H}$ and $\mathscr{K}$. For each $T \in \mathscr{K}$ let $m_T
: \mathscr{H} \to \mathscr{L}$ be the mapping defined by $m_T (S) = S
U T$. Then $t_{m_T} : \mathscr{L}_c' \to \mathscr{H}_c'$ is linear and
continuous. Let $\alpha \in \mathscr{L}_c'$. Let us denote by
$\dot{\mathscr{L}}$ the scalar product between $\mathscr{L}$ and
$\mathscr{L}_c'$ and by $\dot{\mathscr{H}}$ the scalar product between
$\mathscr{H}$ and $\mathscr{H}_c'$. Let us denote by the same symbols
the extensions to $\mathscr{L} (E)$ and $\mathscr{L}_c'$ and to
$\mathscr{H} (E)$ and $\mathscr{H}_c'$. Then 
$$
\overrightarrow{S}_{\dot{\mathscr{H}}} T\alpha = (\overrightarrow{S}
\tilde{U} T)_{\dot{\mathscr{L}}}\alpha , \quad \text {where}\quad T\alpha =
t_{m_T}(\alpha).  
$$ 
\end{prop}
\begin{proof}
We have to only verify that for every $\overleftarrow{e}' \in E'$,
$\langle \overrightarrow{S}_{\mathscr{H}} \cdot T \alpha,
\overleftarrow{e}'\rangle = \langle (\overrightarrow{S} \tilde{U}
T)_{\mathscr{L}} \cdot\alpha , \overleftarrow{e}'\rangle$.
\begin{align*}
\text{we have} \quad \langle \overrightarrow{S}_{\mathscr{H}} \cdot T \alpha,
\overleftarrow{e}'\rangle &= \langle \overrightarrow{S},
\overleftarrow{e}'\rangle_{\mathscr{H}} \cdot T\alpha = \langle
\overrightarrow{S}, \overleftarrow{e}'\rangle_{\mathscr{H}} \cdot t_{m_T}
(\alpha)\\
&= m_T \langle \overrightarrow{S},
\overleftarrow{e}'\rangle_{\mathscr{L}} \cdot\alpha = (\langle
\overrightarrow{S}, \overleftarrow{e}'\rangle
UT)_{\mathscr{L}} \cdot \alpha\\
&= \langle \overrightarrow{S} \tilde{U} T,
\overleftarrow{e}'\rangle_{\mathscr{L}} \cdot \alpha = \langle
\overrightarrow{S} \tilde{U} T_{\mathscr{L}} 
\cdot \alpha, \overleftarrow{e}'\rangle.
\end{align*}
\end{proof}

\eject
\noindent {\bf Examples\pageoriginale of multiplicative products.}
\begin{itemize}
\item [1)] The multiplicative product $\alpha T$ defined for $\alpha
  \in \mathscr{E}$ and $T \in \mathscr{D}'$ as $\alpha T (\varphi) =
  T(\alpha \varphi)$ for every $\varphi \in \mathscr{D}$ is a bilinear
  map $\mathscr{E} \times \mathscr{D}' \to \mathscr{D}'$ which is
  hypocontinuous with respect to bounded subsets of $\mathscr{E}$ and
  $\mathscr{D}'$. (Ref: Theorie des distributions, Tome~1, pp.~117,
  Chap.~V, \S~2, Th$\acute{e}$or$\grave{e}$me 3). If $E$ is any
  complete $E L C$ we can define bilinear maps, hypocontinuous with
  respect to bounded sets, as explained in Theorem~\ref{chap7:thm7.1} in the
  following cases:
\begin{center}
{\tabcolsep=4pt
\begin{tabular}{rlcrl}
(a) & $\mathscr{D}'(E)\times \mathscr{E}\to\mathscr{D}'(E)$ & and &
  (b) & $\mathscr{D}' \times \mathscr{E} (E) \to \mathscr{D}' (E)$.
\end{tabular}}
\end{center}
\item [2)] If $\alpha \in \mathscr{O}_M$ and $T \in \mathscr{S}'$ the
  multiplicative product $\alpha T \in \mathscr{S}'$. The mapping
  $(\alpha, T) \to \alpha T$ of $\mathscr{O}_M \times \mathscr{S}' \to
  \mathscr{S}'$ is hypocontinuous with respect to the bounded subsets
  of $\mathscr{O}_M$ and of $\mathscr{S}'$. If $E$ is any complete $E
  L C$, as explained in theorem~\ref{chap7:thm7.1}, we get a bilinear map
  which is hypocontinuous with respect to the bounded subsets, in the 
  following cases:
\begin{itemize}
\item [(a)] $\mathscr{S}' (E) \times \mathscr{O}_M \to \mathscr{S}'
  (E)$
\item [(b)] $\mathscr{S}' \times \mathscr{O}_M (E) \to \mathscr{S}'(E)$.
\end{itemize}
\end{itemize}

\noindent
{\bf The Convolution product.}
\medskip

If $U : \mathscr{H} \times \mathscr{K} \to \mathscr{L}$ is a
separately continuous bilinear map of $\mathscr{H} \times \mathscr{K}$
in $\mathscr{L}$, where $\mathscr{H}, \mathscr{K}$ and $\mathscr{L}$
are three locally convex Hausdorff spaces and if $E$ is a complete $E
L C$ we can define a bilinear map $\tilde{U} : \mathscr{H}(E)\times
\mathscr{K}\to\mathscr{L}(E)$ as explained in theorem \ref{chap7:thm7.1}.
We take any fixed $T\in\mathscr{K}$ and consider the continuous linear map
$m_T \in I :\mathscr{H}(E) \to \mathscr{L}(E)$, where $m_T :
\mathscr{H}\to\mathscr{L}$ is the continuous linear map $S \to S U
T$ and $I : E \to E$ is the identity map. This can be applied to the
product of convolution. We get then products of convolution of certain
vector valued distributions by certain scalar valued distributions. 

For\pageoriginale example, we can define convolution in the following cases:
\begin{itemize}
\item [(1)] $\mathscr{S}'(E)\times \mathscr{O}_c \to \mathscr{S}'
  (E)$.
\item [(2)] $\mathscr{S}'\times \mathscr{O}_c (E) \to \mathscr{S}' (E)$.
\end{itemize}

\noindent
{\bf Regularization:}
\medskip

\begin{definition}\label{chap7:def7.2}
The mapping $(T, \alpha) \to T^\ast \alpha$ of $\mathscr{D}'\times
\mathscr{D} \to \mathscr{E}$ is called the regularization.

If $E$ is a complete $E L C$ we can get a bilinear map $\mathscr{D}'
(E)\times \mathscr{D} \to \mathscr{E} (E)$ as explained already. This
bilinear map is called the regularization in the case of vector valued
distributions. 

As in the scalar case, we have $\overrightarrow{T}^\ast \alpha (x)=
\overrightarrow{T}_{\xi}. \alpha (x-\xi)$ where $\cdot$ denotes the extension
of the multiplicative product. In fact, this is proved by forming the
scalar product with any $\overleftarrow{e}' \in E'$.

Let $\mathscr{E}^\circ$ and $\mathscr{D}^\circ$ denote the space of
continuous functions and the space of continuous functions with
compact support with their usual topologies. For $f \in
\mathscr{E}^\circ$ and $g \in \mathscr{D}^\circ$ we have $f\ast g\in
\mathscr{E}^\circ$ and we have the formula
\begin{equation}
f \ast g (x) = \int\limits_{R^n} 
f(x - \xi) g(\xi)\,d\xi \tag{1}
\end{equation}
Now suppose $E$ is a complete $E L C$. We have a convolution
$\mathscr{E}^\circ (E)\times \mathscr{D}^\circ \to
\mathscr{E}^\circ(E)$. Suppose we take $\overrightarrow{f} \in
\mathscr{E}^\circ (E)$ and $g \in \mathscr{D}^\circ$, we have a
formula similar to (1), namely
$$
\overrightarrow{f} \ast g (x) = \int\limits_{R^n}
\overrightarrow{f}(x-\xi)\,d\xi.
$$
\end{definition}

\begin{proof}
We have, for every $\overleftarrow{e}' \in E'$
\begin{align*}
\langle \overrightarrow{f} \ast g(x), \overleftarrow{e}' \rangle &=
\langle \overrightarrow{f} \ast g, \overleftarrow{e}' \rangle (x)\\
&= \langle \overrightarrow{f}, \overleftarrow{e}' \rangle \ast g (x).
\end{align*}
$\langle\overrightarrow{f}, \overleftarrow{e}' \rangle \ast
g$\pageoriginale is the convolution of the function $\langle
\overrightarrow{f}, \overleftarrow{e}' \rangle \in 
\mathscr{E}^\circ$ and $g \varepsilon \mathscr{D}^\circ$.
We have, therefore,
\begin{align*}
\langle \overrightarrow{f}, \overleftarrow{e}' \rangle \ast g(x) &=
\int\limits_{R^n} \langle \overrightarrow{f}, \overleftarrow{e}'
\rangle (x - \xi) g(\xi)\,d\xi\\
&= \int\limits_{R^n} \langle \overrightarrow{f} (x-\xi),
\overleftarrow{e}' \rangle g (\xi)\,d\xi\\
&= \int\limits_{R^n} \langle \overrightarrow{f} (x-\xi) g (\xi),
\overleftarrow{e}' \rangle\,d\xi\\
&= \left\langle \int\limits_{R^n} \overrightarrow{f} (x-\xi) g (\xi)\,
d\xi, \overleftarrow{e}' \right\rangle.
\end{align*}
Hence $\langle \overrightarrow{f} \ast g (x), \overleftarrow{e}\rangle
= \left\langle \int\limits_{R^n} \overrightarrow{f} (x-\xi) g(\xi)\,
d\xi , \overleftarrow{e}' \right\rangle$.

\noindent This gives $\overrightarrow{f} \ast g (x) =
\int\limits_{R^n}\overrightarrow{f} (x-\xi) g (\xi)\,d\xi$.
\end{proof}


