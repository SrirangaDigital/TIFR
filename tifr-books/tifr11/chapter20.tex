
\chapter{Representations of semi-groups (contd.)}\label{chap20}

Let\pageoriginale $X$ be a tangent vector to the cone $G_+$ at the
origin. If for any $\varphi\in\mathscr{D}(G_+)$ define $X(\varphi)=$
derivative of $\varphi$ at $0$ along the direction $X$, $X$ can be
considered as a distribution. It is an element of $\mathscr{D}_{L^1}'
(G_+)$. In fact $X$ has a compact support, the point `0'. The
operator $U(X)$ is called the infinitesimal generator corresponding to
the tangent vector $X$ at $0$. The directional derivative $X(\varphi)$
is by definition
$$
\lim\limits_{t\to\circ}\frac{\varphi(t X)-\varphi(0)}{t}.
$$
Hence $\hspace{2cm} X(\varphi)=
\lim\limits_{t\to\circ}\frac{\delta_{tX}-\delta_\circ}{t}(\varphi)$.

\setcounter{section}{20}
\setcounter{prop}{0}
\begin{prop}\label{chap20:prop20.1}
$U(X)x$ exists if and only if 
\begin{align*}
&\lim\limits_{t\to\circ}\frac{U(\delta_{tX})-I}{t}x \quad
  \text{exists, and}\\
&\lim\limits_{t\to\circ}\frac{U(\delta_{tX})-I}{t}x=U(X)x.
\end{align*}
\end{prop}

\begin{proof}
Assume that $\lim\limits_{t\to\circ}U
\frac{\delta_{tX}-\delta_\circ}{t}x$ exists. Now
$\frac{\delta_{tX}-\delta_\circ}{t}\to x$ strictly as $t\to 0$. By
Proposition \ref{chap19:prop19.2}, we see that $U(X)x$ exists and is
equal to $\lim\limits_{t\to\circ}\frac{U(\delta_{tX})-I}{t}x$. 

Conversely, suppose that $U(X)x$ has a meaning. We have
$$\lim\limits_{t\to\circ} \frac{\delta(tX)-\delta(\circ)}{t}=
\lim\limits_{t\to\circ}(X*\mu_t)$$ 
where $\mu_t$ is a measure,
concentrated on the line segment joining the vectors $0$ and $tX$,
which is homogeneous and gives to the segment a total mass 1. If
$U(X)x$ exists, by Proposition 19.1', we have, as
$U(\mu_t) U(X)x$ exists, $U(\mu_t*X)x=U(\mu_t) U(X)x$. But due to
commutativity\pageoriginale $\mu_t*X=X*\mu_t$. Hence $U(X*\mu_t)x=U
(\mu_t) U(X)x$. But since $\mu_t\to\delta$ strictly, we have
\begin{align*}
\lim\limits_{t\to\circ}U(X*\mu_t)x &= U(\delta) U(X)x\\
&=U(X)x.
\end{align*}
Hence, if $U(X)x$ exists, $\lim\limits_{t\to\circ}U(X*\mu_t)x$
exists. In other words,\break $\lim\limits_{t\to\circ} \frac{U(\delta
  (tX)-I}{t}x$ exists and is equal to $U(X)x$.
\end{proof}

\begin{prop}\label{chap20:prop20.2}
Let $x\in E$. The following four properties are equivalent:
\begin{itemize}
\item [i)] $U(T)x$ exists for every $T\in\mathcal{E'}_{G_+}^1$.
\item [ii)] $U(T)x$ exists for every $T\in\mathcal{E'}_\circ^1$,
  i.e. $U(X)x$ exists for every $X\in R^n$.
\item [iii)] The function $U(\overset{\wedge}{s})\times(s\to U(s)X)$ belongs
  to $\mathcal{E}_s^1(E)$. [We shall say that a function on a cone is
    once continuously differentiable if it is once continuously
    differentiable in the interior and the derivatives have a
    continuous extension to the cone].
\item [iv)] The function $U(\overset{\wedge}{s})x$ has weak
  derivatives at the origin in every direction along the cone.
\end{itemize}
\end{prop}

\begin{proof}
Evidently i) implies ii). ii) implies i) since $T\in
\mathcal{E'}_{G_+}^1$ can be written as $T=\sum
\mu_i*\frac{\partial}{\partial x_i}$, where $\mu_i$ are summable
measures with supports in $G_+$ (Whitney regularity) and we any apply
Proposition 19.1'. ii) implies iii): Let
$X_1,\ldots,X_n$ be independent vectors tangential to the cone. If
$s_\circ$ is an interior point of the cone, using the method of proof
of Proposition \ref{chap20:prop20.1}, we see that
$U(\overset{\wedge}{s})x$ is differentiable in the direction
$X_i(i=1,\ldots,n)$ at $s_\circ$ and that the derivative is equal to
$U(s_\circ) U(X_i)x$. $U(X_i)x$ being a fixed vector in $E$, the
function $s\to U(s)U(X_i)x$ is a continuous function on the
cone. Since the $X_i$ form a base for $R^n$, it follows that
$U(\overset{\wedge}{s})x\in\mathcal{E}_s^1(E)$. iii) implies iv), as
is easily seen. iv) implies ii) by Proposition \ref{chap19:prop19.2}. 
\end{proof}
