
\chapter[Partial Differential Equations - Weak boundary...]{Partial Differential Equations - Weak boundary value
  problems}\label{chap10}

\noindent {\bf Heat\pageoriginale conduction equation:}

Let $\Omega$ be a bounded domain in $R^n$ with a smooth boundary
$\mathcal{S}$. The heat conduction problem in $\Omega$ with Neumann's
boundary condition is the following. We are given smooth functions
$F(x, t) [x \in \Omega, 0 < t < \infty], H (x, t) (x \in \Omega, 0 < t
< \infty)$ and $U_\circ (x) (x \in \Omega)$. The problem is to find a
smooth function $U(x, t)$ continuous in $\bar{\Omega} X$ $(0, \infty)$
and differentiable in $\Omega X ] 0, \infty )$ such that 
\begin{itemize}
\item [i)] $\frac{\partial U(x, t)}{\partial t}-\Delta U(x,t)=F(x,t)
 \, (t>0, x \in \Omega) [\Delta \mbox{ is the Laplacian in}\break R^n]$;
\item [ii)] for each fixed $t > 0, U(x, t)$ and $H(x, t)$ have the
  same normal derivative at every point of the boundary;
\item [iii)] $U(x, \circ) = U_\circ(x)$. 
\end{itemize}
\noindent {\bf Remark concerning condition (ii):} Actually one is
given initially a function $h(x, t), x \in S$ and it is required that
$U(x, t)$ satisfy the condition 
\begin{itemize}
\item [$ii'$)] $\frac{\partial (x, t)}{\partial n_x}= h(x, t)$ for $x
  \in S$ and every $t$.
\end{itemize}
We shall, however, assume that there exists a smooth function $H(x,
t),\break x \in \Omega$, such that $\frac{\partial H(x, t)}{\partial n}=
h(x,t), x \in S$.

Putting $U - H = u$, we are led to the following homogeneous problem:
Given $f(x, t)$ and $u_\circ(x)$ which are sufficiently differentiable
find $u(x, t)$ continuous in $\bar{\Omega} X(0, \infty)$ and differentiable
in $\bar{\Omega} X(0, \infty)$ such that
\begin{itemize}
\item[i)] $\frac{\partial u(x, t)}{\partial t}-\Delta u(x, t)=f(x, t),
  t > 0$;\pageoriginale
\item [ii)] for each fixed $t > 0, u(x, t)$ has vanishing normal
  derivative at the boundary;
\item [iii)] $u(x, t) = u_\circ(x)$.
\end{itemize}

In the frame-work of Hilbert spaces, the above problem, in a weaker
formulation, can be posed as follows. Consider the Hilbert space
$H^1(\Omega)$ and the associated space $N$ (See Lions:{\bf ``On  
Elliptic Partial Differential Equations''}, Tata Inst. of Fundamental
Research, Bombay, Lec.~6). 
\noindent {\bf Problem 1.} Given a continuous function $F(t) (t > 0)$
with values in $L^2 (\Omega)$ and a function $u_\circ \in N$, find a
function $u$, with values in $L^2(\Omega)$, once continuously
differentiable in $t > 0$ such that for $t > 0, \frac{\partial
  u}{\partial t}-\Delta u = F$ and such that $u(t) \to u_\circ$ in $N$
as $t \to 0$.

We shall transform this problem in the following way. Define
$\tilde{u}(t)$ (with values in $N$) by $\tilde{u}(t) = u(t)$ for $t >
0$ and $\tilde{u}(t) = 0, t < 0$. Consider $\tilde{u}$ as an element
of $\mathscr{D}_+'(t, N)$ (space of distributions with values in $N$
and with supports in $(0, \infty]$. Similarly define $F\in
\mathscr{D}_+'(t, L^2)$. Since we require $u(0)=u_\circ$, we must have
$\frac{\partial \tilde{u}}{\partial t}={\left( \frac{\partial
u}{\partial t}\right)}^\sim + \delta_t u_\circ$, and problem~1
reduces to 

\noindent {\bf Problem~1$'$.} To find $\tilde{u} \in \mathscr{D}_+'
(t, N)$ with $\tilde{u}$ once continuously differentiable in $t > 0$
and $= 0$ for $t < 0$ and such that
$$
(*) \qquad \frac{\partial \tilde{u}}{\partial t}- \Delta \tilde{u} =
\delta u_\circ + \tilde{F}.
$$
Finally we may abandon the requirement that $\tilde{u}$ be
differentiable in $t > 0$ and replace the right hand member of $(*)$
by an arbitrary element of $\mathscr{D}_+' (t, L^2)$. We then have 

\noindent {\bf Problem 2.}\pageoriginale Given $T$ in 
$\mathscr{D}_+'(t, L^2)$ find $u$ in $\mathscr{D}_+' (t, N)$ such that 
$$
\frac{\partial u}{\partial t}- \Delta u = T.
$$

We shall treat only problem 2. We shall show that the problem admits
of a unique solution. But what we would have solved will only be a
problem much weaker than the original problem we posed. To solve the
original problem completely one has to show that $u \in
\mathscr{D}_+'(t, N)$ that we have found is a differentiable function
in $t > 0$ and also one has to prove the regularity properties of
$u(t, x)$ as a function of $x$. 
