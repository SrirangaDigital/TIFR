
\chapter{Operations on vector valued 
  distributions (contd.)}\label{chap15}

Suppose\pageoriginale $\mathscr{H}$ is a nuclear, barrelled space with
a nuclear strong dual $\mathscr{H}_\delta'$. Let $\mathscr{H}$ and
$\mathscr{H}_\delta'$ be assumed to be complete. Let $E, F$ and $G$ be
three Banach spaces and $B$ a continuous bilinear map of $E \times F$
in $G$. Then we can get a bilinear map of $\mathscr{H}(E) \times
\mathscr{H}'(F)$ in $G$ hypocontinuous with respect to the bounded
subsets of $\mathscr{H}(E)$ and $\mathscr{H}'(F)$. Since $\mathscr{H}$
is barrelled, the scalar product defining the duality between
$\mathscr{H}$ and $\mathscr{H}'$ is a bilinear map of $\mathscr{H}
\times \mathscr{H}'$ into $C$ hypocontinuous with respect to the
bounded sets of $\mathscr{H}$ and $\mathscr{H}_\delta'$. Let us denote
the scalar product by the symbol `.'. $B$ being a continuous
bilinear map of $E \times F$ in $G$, we get, as explained in the
previous lecture, a bilinear map $\underset{B}{.} : \mathscr{H}(E)
\times \mathscr{H}'(F) \to C(G) = G$, hypocontinuous with respect to
the bounded sets of $\mathscr{H}(E)$ and $\mathscr{H}'(F)$.

Now, $\mathscr{D}$ and $\mathscr{S}$ are barrelled spaces. Hence if $B
: E \times F \to G$ is a continuous bilinear map, we get a separately
continuous bilinear map in each of the following cases, which is
further hypocontinuous with respect to the bounded sets

\begin{enumerate}
\item [1)] $\mathscr{D}(E) \times \mathscr{D}'(F) \to G$
\item [2)] $\mathscr{S}(E) \times \mathscr{S}'(F) \to G$
\end{enumerate}
If $\overrightarrow{\varphi} \in \mathscr{D}(E)$ and
$\overrightarrow{T} \in \mathscr{D}'(F)$ we denote by
$\overrightarrow{T}_{\underset{B}{.}} \overrightarrow{\varphi}$ the image in
$G$ of the element $(\overrightarrow{\varphi}, \overrightarrow{T})$ by
the bilinear map in case (1). We use similar notation in case (2)
also. 

For\pageoriginale the Banach spaces $E, F$ and $G$ we can take $E, E'$
and $C$ and for $B$ the canonical bilinear form on $E \times E'$. Then
when $\mathscr{H}$ and $\mathscr{H}'$ are complete nuclear spaces with
$\mathscr{H}$ barrelled, we get a bilinear form on $\mathscr{H}(E)
\times \mathscr{H}'(E)$ hypocontinuous with respect to the bounded
subsets of $\mathscr{H}(E)$ and $\mathscr{H}'(E')$. If
$\overrightarrow{\varphi} \in \mathscr{H}(E)$ and $\overrightarrow{T}
\in \mathscr{H}'(E')$ the image of $(\overrightarrow{\varphi},
\overrightarrow{T})$ in $C$ by the above bilinear map is denoted by
$\langle \overrightarrow{T.}, \overrightarrow{\varphi}\rangle$. 

Since $\mathscr{H}$ is nuclear and $\mathscr{H}$ and $E$ are complete,
we have $\mathscr{H}(E) = \fohprod{\mathscr{H}}{E}{\pi}$.
Let $u \in (\mathscr{H}(E))'$. $u$ is
a linear form on $\fohprod{\mathscr{H}}{E}{\pi}$ which is
continuous. Now $\mathscr{H}\otimes E$ is dense in $\fohprod{\mathscr{H}}
{E}{\pi}$, therefore the space of continuous
linear forms on $\fohprod{\mathscr{H}}{E}{\pi}$ is the
same space of continuous linear forms on $\foprod{\mathscr{H}}
{E}{\pi}$. The definition of the $\pi$-topology hence
gives $(\mathscr{H}(E))' =$ space of continuous bilinear forms on
$\mathscr{H}\times E$ algebraically. Then $u$ can be considered also as
a continuous bilinear form on $\mathscr{H} \times E$. For any fixed $h
\in \mathscr{H}, \overrightarrow{e} \to u(h, \overrightarrow{e})$ is a
continuous linear map of $E$ in $C$; hence it defines an element $u_h$
of $E'$. The mapping $h \to u_h$ of $\mathscr{H}$ in $E'$, with the
structure of a Banach space on it, is a continuous linear
map. Conversely, suppose that $\mu : \mathscr{H} \to E'$ is a
continuous linear map. Then $\tilde{\mu}$ defined by $\tilde{\mu}(h,
\overrightarrow{e}) = \langle \mu(h), \overrightarrow{e} \rangle$ for
every $\overrightarrow{e} \in E$ and $h \in \mathscr{H}$ is a
continuous bilinear form on $\mathscr{H} \times E$. Hence we have
$(\mathscr{H}(E))' = \mathscr{L}(\mathscr{H}, E')$ algebraically. Now,
consider the space $\mathscr{H}'(E')$. We have seen that we can define
one and only one bilinear form on $\mathscr{H}(E) \times
\mathscr{H}'(E')$ hypocontinuous with respect to the bounded sets
satisfying 
$$
\langle T \overrightarrow{e} ., H' \overleftarrow{e}' \rangle =
\langle T. H' \rangle_{\mathscr{H}, \mathscr{H}'} \langle
\overrightarrow{e}, \overleftarrow{e}' \rangle_{E, E'}.
$$
If $\overrightarrow{H}' \in (\mathscr{H}(E))'$ and $\overrightarrow{\varphi} \in
\mathscr{H}(E)$ the scalar product defining the duality between
$\mathscr{H}(E)$ and $(\mathscr{H}(E))'$ satisfies  
$$
\langle H \overrightarrow{e}, H' \overleftarrow{e}' \rangle = \langle
H, H' \rangle_{\mathscr{H}, \mathscr{H}'} \langle \overrightarrow{e},
\overleftarrow{e}' \rangle_{E, E'}.
$$ 
Also\pageoriginale $(\mathscr{H}(E))' = \mathscr{L} (\mathscr{H}, E')$
and $\mathscr{H}'(E') = \mathscr{L}(\mathscr{H}')_c',
E')$. $(\mathscr{H}')_c'$ is algebraically the same as $\mathscr{H}$
but, in general, has a topology finer than that of $\mathscr{H}$. In
the case of barrelled spaces $(\mathscr{H}_c')' = \mathscr{H}$
topologically.It follows from Theorem \ref{chap14:thm14.1} that the
scalar product defining the duality between $\mathscr{H}(E)$ and
$(\mathscr{H}(E))'$ is the same as the bilinear form that is defined
on $\mathscr{H}(E) \times \mathscr{H}'(E')$ by the process described
in Theorem \ref{chap14:thm14.1} 

\vspace{.3cm}
\noindent {\bf The convolution of two vector valued distributions.} 

Let $E, F$ and $G$ be three Banach spaces and $B : E \times F \to G$ a
continuous bilinear map. Let $\overrightarrow{S} \in \mathscr{S}'(E)$
and $\overrightarrow{T} \in \mathscr{O}_c'(F)$. The convolution
operation between the elements of $\mathscr{S}'$ and the elements of
$\mathscr{O}_c'$ satisfies the conditions stipulated in Theorem
\ref{chap14:thm14.1} and the spaces $\mathscr{S}', \mathscr{S},
\mathscr{O}_c', \mathscr{O}_c$ are all nuclear complete spaces. (See:
Memoirs of the Amer. Math. Soc., No. 16, Products Tensoriels
Topologique et Espaces Nucleaires, by A. Grothendieck). Hence, as
explained in Theorem \ref{chap14:thm14.1} we can define a bilinear map
$\underset{B}{*} : \mathscr{S}'(E) \times \mathscr{O}_c'(F) \to
\mathscr{S}'(G)$ which is hypocontinuous with respect to the bounded
subsets of $\mathscr{S}'(E)$ and $\mathscr{O}_c'(F)$. We call
$\overrightarrow{S} \underset{B}{*} \overrightarrow{T}$ the
convolution of $\overrightarrow{S}$ and $\overrightarrow{T}$ under
$B$.

We know that $S \in \mathscr{D}_+'$ and $T \in \mathscr{D}_+'$ implies
$S * T \in \mathscr{D}_+'$, where $\mathscr{D}_+'$ is the space of
distributions $\in \mathscr{D}'$ with supports bounded on the left (it
is the dual of $\mathscr{D}_-$). The map $(S, T) \in S * T$ of
$\mathscr{D}_+'\times\mathscr{D}_+'$ in $\mathscr{D}_+'$ also satisfies the 
conditions stipulated in Theorem \ref{chap14:thm14.1}. Hence we get a bilinear
map $\underset{B}{*} : \mathscr{D}_+' (E) \times \mathscr{D}_+' (F)
\to \mathscr{D}_+'(G)$ hypocontinuous with respect to the bounded
sets. If $E$ is a Banach Algebra, by\pageoriginale taking $E = F = G =
E$ and $B =$ the multiplication in $E$, we get a bilinear map
$\mathscr{D}_+'(E) \times \mathscr{D}_+'(E) \to \mathscr{D}_+'(E)$
hypocontinuous with respect to the bounded subsets of
$\mathscr{D}_+'(E)$ and $\mathscr{D}_+'(E)$. 

Now suppose $\overrightarrow{S} \in \mathscr{S}'(E)$ and
$\overrightarrow{T} \in \mathscr{O}_c'(F)$. Then $\overrightarrow{S}
\underset{B}{*} \overrightarrow{T} \in \mathscr{S}'(G)$, where $B : E
\times F \to G$ is a bilinear continuous map and $E, F$ and $G$ are
three Banach spaces. The Fourier transforms of $\overrightarrow{S}$
and $\overrightarrow{T}$ satisfy $\mathscr{H}(\overrightarrow{S}) \in
\mathscr{S}'(E)$ and $\mathscr{H}(\overrightarrow{T}) \in
\mathscr{O}_M(F)$. Since $\mathscr{O}_M$ and $\mathscr{O}_M'$ are
nuclear and complete (refer to: Memories of the Amer. Math. Soc.,
No16, Espaces Nucleaire, by A. Grothendieck), using the product
between the elements of $\mathscr{S}'$ and the elements of
$\mathscr{O}_M$ and using the bilinear map $B$ we define a product
$(B)$ between the elements of $\mathscr{S}'(E)$ and $\mathscr{O}_M(F)$
as explained in Theorem \ref{chap14:thm14.1}. Then we have
$\mathscr{H}(\overrightarrow{S} \underset{B}{*} \overrightarrow{T}) =
\mathscr{H} (\overrightarrow{S}) (B)
\mathscr{H}(\overrightarrow{T})$. This follows from the separate
continuity of the operations on the two sides and from the equality 

\noindent $\mathscr{H} (S \overrightarrow{e} * T \overrightarrow{f}) =
\mathscr{H}(S \overrightarrow{e})_{(B)} \mathscr{H}(T f)$ for $S \in
\mathscr{S}', T \in \mathscr{O}_c', \overrightarrow{e} \in E$ and
$\overrightarrow{f} \in F$ and from the fact that $\mathscr{S}'
\otimes E$ and $\mathscr{O}_c' \otimes F$ are dense in
$\mathscr{S}'(E)$ and $\mathscr{O}_c'(F)$ respectively.

In the case in which $\overrightarrow{T} \in \mathscr{D}'(F)$ is a
continuous function $\overrightarrow{g}$ with values in $F,
\overrightarrow{\varphi}\underset{B}{.} \overrightarrow{T}$ for
every $\overrightarrow{\varphi} \in \mathscr{D}(E)$ can be expressed
as an integral. We will show that 
$$
\overrightarrow{\varphi}\underset{B}{.} \overrightarrow{T} =
\int\limits_{R^n} B(\overrightarrow{\varphi}(x), \overrightarrow{g}(x)
dx.
$$
we know that $\underset{B}{.} :\mathscr{D}'(F) \times
\mathscr{D}(E) \to G$ is a separately continuous function. Hence
$\underset{B}{.} : \mathcal{E}^\circ (F) \times \mathscr{D}(E)
\to G$ is also separately continuous. Also the mapping
$$
(\overrightarrow{g}, \overrightarrow{\varphi}) \to \int\limits_{R^n}
B(\overrightarrow{\varphi} (x), \overrightarrow{g}(x) dx
$$
is\pageoriginale a separately continuous map of $\mathcal{E}^\circ(F)
\times \mathscr{D}(E) \to G$. Also for every $\varphi \in \mathscr{D},
g \in \mathcal{E}^\circ , \overrightarrow{e} \in E$ and
$\overrightarrow{f} \in F$, we have 
\begin{align*}
\int\limits_{R^n} B(\varphi \overrightarrow{e} (x), g
\overrightarrow{f} (x)) dx &= \int\limits_{R^n} B(\varphi (x)
\overrightarrow{e}, g(x) \overrightarrow{f}) dx\\
&= \int\limits_{R^n} \varphi(x) g(x) B (\overrightarrow{e},
\overrightarrow{f}) dx\\
&=\left\{ \int\limits_{R^n} \varphi(x) g(x) dx \right\}. B
  (\overrightarrow{e}, \overrightarrow{f}).
\end{align*}
The mappings 
$$\underset{B}{.} : \mathcal{E}^\circ(F) \times
  \mathscr{D}(E) \to G \text{ and } (\overrightarrow{g},
  \overrightarrow{\varphi}) \to \int\limits_{R^n}
  B(\overrightarrow{\varphi}(x), \overrightarrow{g}(x)) dx$$ are
  separately continuous and agree on the decomposed elements. Since
  $\mathcal{E}^\circ$ and $\mathscr{D}$ have the approximation
  property, we see that the two maps are identical. Also if $f \in
  \mathscr{D}^\circ$ and $g \in \mathcal{E}^\circ$ the convolution $f
  * g$ is an element of $\mathcal{E}^\circ$ and is given by
  $\int\limits_{R^n} f(x-\mathscr{E}) g(\mathscr{E})d\mathscr{E}$. Now
  suppose $\overrightarrow{f} \in \mathscr{D}^\circ(E)$ and
  $\overrightarrow{g} \in \mathcal{E}^\circ(F)$. Then we shall prove
  that $\overrightarrow{f} \underset{B}{*} \overrightarrow{g} \in
  \mathcal{E}^\circ(G)$ and is given by the formula
  $\overrightarrow{f} \underset{B}{*} \overrightarrow{g}(x) =
  \int\limits_{R^n} B(\overrightarrow{f} (x-\mathscr{E}),
  \overrightarrow{g} (\mathscr{E})) d \mathscr{E}$. The maps
  $\underset{B}{*} : \mathscr{D}^\circ(E) \times \mathcal{E}^\circ(F)
  \to \mathscr{D}'(G)$ and $(\overrightarrow{f}, \overrightarrow{g})
  \to \int\limits_{R^n} B(f(x-\mathscr{E}), g (\mathscr{E})) d
  \mathscr{E}$ of $\mathscr{D}^\circ(E) \times \mathcal{E}^\circ (F)
  \to \mathcal{E}^\circ(G)$ are separately continuous and agree on the
  decomposed vectors, for if $f \in \mathscr{D}^\circ, g \in
  \mathcal{E}^\circ$ and $\overrightarrow{e} \in E$ and
  $\overrightarrow{v} \in F$. We have
\begin{align*}
f \overrightarrow{e} \underset{B}{*} g \overrightarrow{v} (x) &= f * g
(x) B (\overrightarrow{e}, \overrightarrow{v})\\
&= \left(\int f(x-\mathscr{E}) g(\mathscr{E})d\mathscr{E})
B(\overrightarrow{e}, \overrightarrow{v}\right)\\
&= \int B(f (x-\mathscr{E}) \overrightarrow{e}, g(\mathscr{E})
\overrightarrow{v}) d\mathscr{E}\\
&= \int B (f \overrightarrow{e} (x-\mathscr{E}), g \overrightarrow{v}
(\mathscr{E})) d \mathscr{E} .
\end{align*}
Now since $\mathscr{D}^\circ$ and $\mathcal{E}^\circ$ have the
approximation property $\mathscr{D}^\circ \otimes E$ and
$\mathcal{E}^\circ \otimes F$ are dense in $\mathscr{D}^\circ(E)$ and
$\mathcal{E}^\circ(F)$, hence we deduce the conclusion. 
