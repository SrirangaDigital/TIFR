
\chapter{Convexity Theory (continued)}\label{chap7}

1. The\pageoriginale maximal continuation of a family of holomorphic
functions on a manifold $V$ (spread in $C^n$) was defined in
$V$. Certain analogous concepts will now be defined.

Let $V$ be a complex manifold spread in $C^n$ by $\phi$ and let
$\mathscr{C}$ be a family of holomorphic functions on $V$. 

An $N$-\textit{continuation} of $(V, \phi, \mathscr{C})$ is a
continuation $(V', \phi', \psi', \mathscr{C}')$ such that $\{f_i\}_{i
  \in I}$ is any subfamily of $\mathscr{C}$, normally convergent in
$V$ and $f'_i$ is the continuation of $f_i$, then the family
$\{f'_i\}_{i \in I}$ converges normally in $V'$. A maximal
$N$-continuation is now defined in the same way as was maximal
continuation. 

The following two concepts are defined similarly.

A \textit{maximal $U$-continuation}: the property considered is that
of convergence of sequence of functions of $\mathscr{C}$, in
$\mathscr{H}_V$.

\textit{Maximal $B$-continuation:} The property considered is that of
boundedness of subfamilies in $\mathscr{H}_V$. 

In the same way as before, one can prove the existence and uniqueness
of maximal $N-$, $U-$ and $B-$ continuations.

The proof of \ref{chap6}, Theorem \ref{chap6:thm1} on p.36 gives us
the following result.  


\begin{dashthm}\label{chap7:dashthm1}
Let $V$ be a complex manifold, $\phi$ a spread of $V$ in
  $C^n$. Let $\mathscr{C} \subset \mathscr{H}_V$ and suppose that
  $\mathscr{C}$ is stable derivation. If $(V, \phi)$ is itself the
  maximal $N-$ continuation (or $B-$ or $U-$ continuation) of $(V, \phi,
  \mathscr{C})$, then the distances of $K$ and $\hat{K}_{\mathscr{C}}$
to be boundary of $V$ are the same.
\end{dashthm}

\ref{chap6} Theorem \ref{chap6:thm2} \pageoriginale (p.37) also has an analogue:

\begin{dashthm}\label{chap7:dashthm2}
Let $V$ be a complex manifold, $\phi$ a spread of $V$ in
  $C^n$. Suppose $\mathscr{C}$ has the following properties:
\begin{itemize}
\item[$(1^\circ)$] $f \in \mathscr{C}$ implies $\lambda f \in
  \mathscr{C}$ for every complex $\lambda$;

\item[$(2^\circ)$] If $\phi(x) = \phi(y)$, there exists $f\in
  \mathscr{C}$ such that $f_z \neq f_y$;

\item[$(3^\circ)$] If $K \subset V$ is compact, and $z
  \in\hat{K}_\mathscr{C}$, the maximal polydisc about $z$ contains
  points not in $\hat{K}_\mathscr{C}$.
\end{itemize}

Finally, let $(\tilde{V}, \tilde{\phi}, \tilde{\psi},
\tilde{\mathscr{C}})$ be the maximal $N-$, $B-$, or $U$-continuation
of $(V, \phi, \mathscr{C})$. 
\end{dashthm}

Then, $\tilde{\psi}$ is an isomorphism of $V$ onto $\tilde{V}$.

\begin{proof}
As in Step \ref{chap6:step1}, in the proof of \ref{chap6} Theorem
\ref{chap6:thm2}, $\tilde{\psi}$ is one-one (into). We now construct a
sequence of functions $(f_p)$ as follows. 

Choose a countable dense set $\{z_m\}$ in $V$ and let $S_m$ be the
maximal polydisc about $z_m$ in $V$. Consider the sequence.
$$
S_1, S_2, S_1, S_2, S_3, \ldots 
$$
and denote its $p$-th term by $\sum_p$. Let $\{K_p\}$ be a sequence of
compact sets so that $K_p \subset \overset{o}{K}_{p+1}$,
$\bigcup\limits^\infty_{p=1} K_p = V$. Then, by hypothesis
$(3^\circ)$, there is a point $z^{(p)} \in \sum_p$, $(z(p) \not\in
(\hat{K}_p)_{\mathscr{C}})$ and (by the definition of
$(\hat{K}_p)_{\mathscr{C}}$) a function $f \in\mathscr{C}$ so that
$|f(z^{(p)})| > 2^{2p} ||f ||_{K_p}$. This gives rise to a sequence of
functions $\{f_p\}$  such that $||f_p||_{K_p} \leq 2^{-p}$, $|f_p
(z^{(p)})|>2^p$. The fact that $\psi$ is onto is proved by reasoning
analogous to Step \ref{chap6:step1} of \ref{chap6}, Theorem \ref{chap6:thm2}. 
\end{proof}

\begin{examples*}
(a) Let $\theta \subset C^n$ be a (univalent) Reinhardt domain and let
  $\mathscr{C}$ be the family of monomials $\lambda z^J$ ($\lambda$
  complex). It is possible to find $\hat{K}_{\mathscr{C}}$ very
  simply.\pageoriginale Let $K'$ be the set of $(t_1 z_1, \ldots, t_n
  z_n)$, where $(z_1, \ldots, z_n) \in K$ and $|t_1| \leq 1, \ldots,
  |t_n| \leq 1$. Clearly $K' \subset \hat{K}_\mathscr{C}$. Also it is
  clear that $\hat{K}_\mathscr{C}$ is completely characterised by its
  image in the  $(|z_1|, \ldots, |z_n|)$-space, and so in the
  $(\rho_1, \ldots, \rho_n)$-space $(\rho_j = \log |z_j|)$. As in II,
  let $K^{'\ast}$ be the image of $K'$, $\hat{K}^\ast_{\mathscr{C}}$
  the image of $\hat{K}_{\mathscr{C}}$ in the $(\rho_1, \ldots,
  \rho_n)$ space. $K'^\ast$ has the following propety: if $(\rho_1,
  \ldots, \rho_n) \in K'^\ast$, then $(\rho_1 -a_1, \ldots, \rho_n -
  a_n) \in K'^\ast$ if $a_j \geq 0$, $j = 1, \ldots, n$. Also
  $\hat{K}^\ast_{\mathscr{C}}$ is defined by inequalitics $j_1 \rho_1
  + \cdots + j_n \rho_n \leq \log|| z^J||_K$, so that
  $\hat{K}^\ast_{\mathscr{C}}$ is clearly convex and so contains the
  convex closure of $K'^\ast$. Since the convex closure of $K'^\ast$
  is the intersection of all closed half spaces containing $K'^\ast$,
  it follows easily that $\hat{K}^\ast_{\mathscr{C}}$ is the convex
  closure of $K'^\ast$ and this gives us the $\mathscr{C}$-envelope of
  $K$. 

This leads to a necessary and sufficient condition for a Reinhardt
domain $\mathscr{O}$, containing $0$, to be a domain of holomorphy. If
$\mathscr{O}$ is a domain of holomorphy, then by \ref{chap2} Theorem
\ref{chap2:thm1}, 
$\mathscr{O}$ is the maximal $N$-continuation domain of the family of
all monomials. If $\mathscr{O}$ is the maximal $N$-continuation domain
of the monomials, then by Theorem \ref{chap7:dashthm1}$'$ and
\ref{chap6}, Theorem \ref{chap6:thm2}, 
$\mathscr{O}$ is a domain of holomorphy (since the envelope of a
compact set with respect to the monomials is trivially larger than
that with respect to all holomorphic functions). By the above results,
this is so if and only if the image $\mathscr{O}^\ast$ of
$\mathscr{O}$ in the $(\rho_1, \ldots, \rho_n)$-space is convex and
such that if $(\rho_1, \ldots , \rho_n) \in \mathscr{O}^\ast$ then
$(\rho_1 - a_1, \ldots, \rho_n - a_n) \in\mathscr{O}^\ast$ when $a_j
\geq 0$. By \ref{chap6}, Theorem \ref{chap6:thm2}, it follows that if
$\mathscr{O}$ is a 
Reinhardt domain which is the union of polydiscs, with centre $0$, and
$\mathscr{O}^\ast$ is convex, then there is a power series such that
$\mathscr{O}$ is precisely the domain of convergence of this power
series, a result which was stated on p.14 (Converse of \ref{chap2}, Theorem
\ref{chap6:thm2}). 

\medskip

(b) Let\pageoriginale $\mathscr{O}$ be an open set in $C$, and let
$\mathscr{C} = \mathscr{H}_{\mathscr{O}}$. If $K$ is a compact subst
of $\mathscr{O}$ and $L$ is the union of the relatively compact
components of the complement of $K(in \mathscr{O})$, then
$\hat{K}_{\mathscr{H}_{\mathscr{O}}} = K \cup L$. It is easy to see
from \ref{chap6}, Theorem \ref{chap6:thm2}, that $\mathscr{O}$ is a
domain of holomophy. In fact, it can be proved that
$\hat{K}_{\mathscr{H}_{\mathscr{O}}}$ is compact. 
\end{examples*}

\medskip
\noindent{\textbf{3. Some remarks on domains of holomorphy.}}

\setcounter{proposition}{0}
\begin{proposition}%%% 1
Let $\mathscr{O}$ be a (univalent) domain in $C^n$. The following
three conditions are equivalent:
\begin{itemize}
\item[1)] $\mathscr{O}$ is a domain of holomorphy;

\item[2)] If $K$ is a compact subset of $\mathscr{O}$ and $z \in
  \hat{K}_{\mathscr{H}}$ (where $\mathscr{H} =
  \mathscr{H}_{\mathscr{O}}$), the maximal polydisc about $z$ in
  $\mathscr{O}$ contains points not in $\hat{K}_{\mathscr{H}}$;

\item[3)] If $K$ is a compact subset of $\mathscr{O}$, then
  $\hat{K}_{\mathscr{H}}$ is compact.
\end{itemize}
\end{proposition}

\begin{proof}
After \ref{chap6}, Theorems \ref{chap6:thm1} and \ref{chap6:thm2}, it
is enough to prove that 1) implies 
3). Clearly $\hat{K}_{\mathscr{H}}$ is closed in $\mathscr{O}$, since
$\mathscr{H}$ is an algebra and $d(K) =
d(\hat{K}_{\mathscr{H}})$. Also $\hat{K}_{\mathscr{H}}$ is bounded in
$C^n$: the function $z_i$ is holomorphic in $\mathscr{O}$ and, by
definition of $\hat{K}_{\mathscr{H}}$, $|z'_i| \leq ||z_i||_K$ for
every $z'\in \hat{H}_{\mathscr{H}}$. 

Suppose $\hat{K}_{\mathscr{H}}$ were not compact. There is then a
sequence $\{X_p\}$ of $X_p \in \hat{K}_{\mathscr{H}}$ having no limit
point in $\hat{K}_{\mathscr{H}}$. However $\{X_p\}$ has a limit point
$X \in C^n$ since $\hat{K}_{\mathscr{H}}$ is bounded. Now $X$ belongs
to the boundary of $\mathscr{O}$ since $\hat{K}_{\mathscr{H}}$ is
closed in $\mathscr{O}$, so that $X$ has a distance $>0$ from $K$. But
if $\mathscr{O}$ is a domain of holomorphy, this implies that $X$ has
a positive distance from $\hat{K}_{\mathscr{H}}$ which is not the
case, and $\hat{K}_{\mathscr{H}}$ is compact.
\end{proof}

\begin{defi*}
Let $V$ be a complex manifold. $V$ is called \textit{holomorph-convex}
if the $\mathscr{H}_V$-envelope of every compact set is compact. 
\end{defi*}

\begin{proposition}%% 2
Let\pageoriginale $V$ be a manifold spread in $C^n$ and suppose that
$V$ is holomorph-convex. Then $V$ is a domain of holomorphy.

Let $(\tilde{V}, \tilde{\phi}, \tilde{\psi})$ be the envelope of $(V,
\phi)$. 

The proof of Step \ref{chap6:step1} in \ref{chap6} Theorem
\ref{chap6:thm2} shows that  
\begin{itemize}
\item[$(1^\circ)$] $V$ is a covering spaces of $\tilde{V}$ under
  projection $\tilde{\psi}$. 

\item[$(2^\circ)$] Over any point of $V$ lie only finitely many points
  of $V$: by definition of $\tilde{V}, \tilde{\psi}$, all holomorphic
  functions on $V$ have the same value at all points over one point of
  $\tilde{V}$. If $X$ is any point of $\tilde{V}$, it follows that all
  points lying over $X$ belong to the $\mathscr{H}_V$-envelope of $X$
  and since $\tilde{\psi}$ is a local homeomorphism, this set cannot
  have a limit point; since $V$ is holomorph-convex, this set must be
  finite. 

\item[$(3^\circ)$] $\tilde{V}$ is holomorph-convex: this follows from
  $(2^\circ)$. 

\item[$(4^\circ)$] If $x$, $y \in \tilde{V}$, $x \neq y$, there is a
  holomorphic function $f$ on $\tilde{V}$ such that $f(x) \neq f(y)$:
  if $\tilde{\phi} (x) \neq \tilde{\phi} (y)$, this is obvious; if
  $\tilde{\phi}(x) = \tilde{\phi}(y)$, then, since $\tilde{V}$ is a
  domain of holomorphy, there exists a function $g$ such that $g_x
  \neq g_y$ and, by going to a derivative of $g$ of sufficiently large
  order, the existence of $f$ follows. 
\end{itemize}
\end{proposition}

$(1^\circ)$, $(2^\circ)$, $(3^\circ)$, and $(4^\circ)$ imply that $V$
is a domain of holomorphy by a theorem of J-P. Serre
\cite[Chap. XX]{p1:key1} which is a consequence of Theorem $A$ and $B$ of
Oka-Cartan-Serre on Stein manifolds and these will be proved later. 



\chapter*{Exercises}

1. Let\pageoriginale $V$ be a
connected complex manifold spread in 
$C^n$ by $\phi$. Suppose that there exists a $n$-parameter family
$T^n$ of analytic automorphisms of $V$, $\sigma (\alpha_1, \ldots,
\alpha_n)$ where the $\alpha$ are real numbers $\mod 2 \pi$, such
that, if $\phi(z) = (\phi_1(z), \ldots, \phi_n(z))$, then $\phi \circ
\sigma (\alpha_1 , \ldots, \alpha_n) (z) = (e^{i\alpha_1} \phi_1 (z),
\ldots\break e^{i\alpha_n} \phi_n (z))$ for all $\alpha \cdot (V, \phi)$ is
then called \textit{a Reinhardt domain}.

Prove the following two results:
\begin{itemize}
\item[(a)] If $(\tilde{V}, \tilde{\phi})$ is the envelope of
  holomorphy of $(V,\phi)$, then $(\tilde{V}, \tilde{\phi})$ is a
  Reinhardt domain.

\item[(b)] If there exists a point $z \in V$ so that $\phi(z) = 0$,
  then every holomorphic function on $V$ can be expanded in a series
  $\sum a_J \phi (z)^J$ on $V$, which converges normally in
  $V$. Deduce that $\tilde{\phi}$ is one-one, i.e., $\tilde{V}$ is
  univalent.
\end{itemize}


\noindent
2. Let $\mathscr{O}$ be an open set in $C^{n+1}$ and $(z_1, \ldots,
z_n, w)$ be a generic point of $\mathscr{O}$. $\mathscr{O}$ is called
a \textit{Hartogs domain} if $(z, w) \in \mathscr{O}$ implies $(z,
e^{i\alpha} w) \in \mathscr{O}$ for every real $\alpha$. Let now
$\mathscr{O}$ be a connected Hartogs domain such that there is a point
$(z, 0) \in \mathscr{O}$. Then, if $f(z,w)$ is holomorphic in
$\mathscr{O}$, prove that $f(z, w)$ can be expanded in a series of the
form
$$
f(z,w) = \sum\limits^\infty_{p=0} a_p (z,w) w^p, 
$$
where the $a_p(z, w)$ are holomorphic functions in $\mathscr{O}$ which
are locallyl independent of $w$, i.e., $\dfrac{\partial a_p}{\partial
  w} = 0$ in $\mathscr{O}$, and such that the series converges
normally in $\mathscr{O}$. 

\noindent
3. Let\pageoriginale $\mathscr{O}$ be the following open set in $C^2$:
$$
\begin{cases}
-3 <\mathscr{R} z < 0, \quad |w| < e^{\mathscr{R}z}, \mathfrak{J} z
\quad \text{ arbitrary}\\
0 \leq \mathscr{R} z < 3, \quad e^{-1/ \mathscr{R}z} < |w| < 1,
\mathfrak{J} z \quad \text{ arbitrary}. 
\end{cases}
$$
($\mathscr{O}$ is a Hartogs domain). Prove that the $a_p(z, w)$ of
Exercise 2 are independent of $w$ and deduce that every holomorphic
function in $\mathscr{O}$ can be continued to $\mathscr{O}'$, the
union of $\mathscr{O}$ and the set of points $(z,w)$ with $0<
\mathscr{R} z < 3$, $|2| \leq e^{-1/\mathscr{R}z}$. 

Prove also that the mapping $(z, w) \to (e^{i \pi / 2} z, w)$ spreads
$\mathscr{O}$, but cannot be continued univalently to $\mathscr{O}'$.

\begin{remark*}
Exercise 1 gives an example of a non-univalent domain whose envelope
is univalent, and Exercise 3 gives an example of a univalent domain
whose envelope is not univalent. 

In the following two exercises, $D$ will denote the closed unit disc
$|z| \leq 1$ in the $C$-plane. 
\end{remark*}

\noindent
4. If $f(z)$ is holomorphic in a neighbourhood of $D$, 
$$
\log |f(z)| \leq \frac{1}{2\pi} \int\limits^{2\pi}_0 \log
|f(e^{i\theta})| \mathscr{R} \frac{e^{i\theta} +z}{e^{i\theta}- z} d
\theta 
$$ 

\noindent
5. Let $f_p(z)$, $(p=1,2,\ldots)$ be holomorphic in neighbourhoods of
$D$ and suppose that (i)  $\overline{\lim\limits_{p \to \infty}}
\dfrac{1}{p} \log ||f_p|| D < + \infty$, \quad (ii)
$\overline{\lim\limits_{p \in \infty}} \dfrac{1}{p} \log |f_p (z)|
\leq M$, $z \in D$. Prove that $\overline{\lim\limits_{p \in\infty}}
\dfrac{1}{p} \log ||f_p||_D \leq M$.

\noindent
6. (Hartog's main theorem''). Let $\theta$ be an open set in $C$ and
let $\{f_p(z)\}$ be a sequence of holomorphic functions in
$\theta$. The domain of absolute (normal) convergence of the series
$$
\sum\limits^\infty_{p=0} f_p (z) w^p
$$
is defined\pageoriginale to be the set of points $(z_0, w_0) \in C^2$
such that $\sum\limits^\infty_{p=0} f_p(z)w^p$ converges absolutely
(normally) in a neighbourhood of $(z_0, w_0)$. Let $R(z) (\bar{R}(z))$
be the greatest number $(\geq 0)$ such that the set $\{ z \in
\mathscr{O}$, $|w| < R (z) (\bar{R}(z))$ is contained in the domain of
absolute (normal) convergence. 

Prove that if $\bar{R}(z)>0$, the $R(z) = \bar{R}(z)$ (use Exercise
5). 

\noindent
7. Let $\{P_n(z_1, z_2)\}$ be a sequence of homogeneous polynomials
($P_n$ is of degree $n$) in $z_1, z_2$. Consider the series 
$$
\sum\limits^\infty_{n=0} P_n (z_1, z_2).
$$
The domain of absolute convergence, $\delta$, is defined as the set of
$(z^{(0)}_1,\break z^{(0)}_2) \in C^2$ such tht in a neighbourhood of
$(z^{(0)}_1, z^{(0)}_2)$ the series $\sum\limits^\infty P_n (z_1,
z_2)$ converges absolutely.

Prove the following statements. 
\begin{itemize}
\item[(a)] If $(z_1, z_2) \in \Delta$ then $(tz_1, tz_2) \in \Delta$
  if $0 < |t| \leq 1$.

\item[(b)] If $\Delta$ is not empty, $(0,0) \in\Delta$ and the series
  converges normally near $(0,0)$.

\item[(c)] The series converges normally in $\Delta$. 
\end{itemize}
(Also due to Hartogs; for (b), use Baire's theorem and the maximum
principle; for (c) use Exercise 6).


\begin{thebibliography}{99}
\bibitem{p1:key1} \textit{H. Cartan}:\pageoriginale \textit{S\'eminaire E.N.S.,}
  1951/52 (especially lectures VII-X).  

\bibitem{p1:key2} \textit{H. Cartan and P. Thullen}: Zur Theorie der
  Singularit\"aten der Funktionen mehrerer komplexen Var\"anderlichen,
  Regularit\"ats - und Konvergenzbereiche, \textit{Math. Ann.,} 106 (1932),
  617-647. 

For the exercises, reference may be made to

\bibitem{p1:key3} \textit{F. Hartogs}: Zur Theorie dur analytischen Funktionen
  mehrerer unabhangiger Ver\"anderlichen, \textit{Math. Ann.,} 62
  (1906), 1-88. 

\bibitem{p1:key4} \textit{L. Schwartz}: \textit{Lectures on Complex Analytic
  Manifolds}, Tata Institute of Fundamental Research, Bombay, 1955. 
\end{thebibliography}
