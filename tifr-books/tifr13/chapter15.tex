\chapter{Stein Manifolds: preliminary results}\label{chap15}

\section{Theorems A) and B) for closed polydiscs in \texorpdfstring{$C^n$}{Cn}}\label{chap15:sec1}
\textit{Let\pageoriginale $V$ be a complex analytic manifold, $K$ a
  compact subset of $V$. We say that Theorems A) and B) are true for
  $K$ if A) every coherent analytic sheaf $\u{\mathscr{F}}$ on $K$ is
  a quotient of a sheaf $\simeq \u{\mathscr{O}}^N$ and $B) H^p (K,
 \u{\mathscr{F}}) = 0$ for $p \geq 1$.}

\setcounter{proposition}{0}
\begin{proposition}\label{chap15:prop1}
Theorems A) and B) are true for (closed) polydiscs in $C^n$.
\end{proposition}

\begin{proof}
Let $P$ be the given polydisc. $P$ has a fundamental system of
neighbourhoods each of which is analytically isomorphic to a closed
cube: $P$ has a fundamental system of neighbourhoods which are open
polydiscs $\prod$; $\prod$, being a product of open discs, is
isomorphic to an open cube $\prod_1$ and the image of $P$ in $\prod_1$
is contained in a closed cube contained in $\prod_1$ whose inverse
image in $\prod$ is a neighbourhood isomorphic to a closed cube. Since
Theorems A) and B) are true for closed cubes, it follows that Theorems
A) and B) are ture for a fundamental system of neighbourhoods of $P$
and after the extension theorem of XII it is easily seen that A) is
true for $P$, and $B)$ follows from
\end{proof}

\setcounter{lem}{0}
\begin{lem}\label{chap15:lem1}
Let $X$ be a paracompact topological space, $\u{\mathscr{F}}$ a sheaf
of abelian groups on $X$. Let $Y$ be a closed set in $X$ with a
fundamental system of closed neighbourhoods $L$. Then
$H^p(Y,\u{\mathscr{F}})$ is the direct limit of $H^p
(L\u{\mathscr{F}})$ as $L$ shrinks to $Y$. 
\end{lem}

\begin{proof}
For $p=0$, this follows from the fact that every section over $Y$ can
be extended to a section over an open neighbourhood of $Y$ and the
fact that the set of points at which two sections coincide is
open. For $p>0$, we construct an exact sequence
\begin{equation*}
0 \to \u{\mathscr{F}} \xrightarrow{i} \u{\mathscr{G}}_\circ
\xrightarrow{d_\circ} \u{\mathscr{G}}_1 \xrightarrow{d_1} \ldots
\xrightarrow{d_{k-1}} \u{\mathscr{G}}_k \xrightarrow{d_k} \ldots 
\tag{1}\label{chap15:eq1}
\end{equation*}
such\pageoriginale that $H^p (E, \u{\mathscr{G}}_1) = 0$ if $p >0$, $l
\geq 0$ for all subsets $E$ of $X$. To do this it is clearly
sufficient to construct an exact sequence
$$
0 \to \u{\mathscr{F}} \to \u{\mathscr{G}} \text{ with } H^p
(E,\u{\mathscr{G}}) = 0 \text{ for } p > 0
$$
(for the construction can be repeated with the quotient
$\u{\mathscr{G}}/ \u{\mathscr{F}}$ and the process continued). We
define $\u{\mathscr{G}}$ to be the sheaf of germs of all mappings $f :
X \to \u{\mathscr{F}}$ with $f(x) \in\mathscr{F}_x$ for $x \in
X$. Then clearly we have an exact sequence
$$
0 \to \u{\mathscr{F}} \to \u{\mathscr{G}}
$$
and $H^p (E, \u{\mathscr{G}}) =0$ for $p>0$ by the theorem in the
appendix

Having constructed the exact sequence (\ref{chap15:eq1}) we consider the associated
sequence
$$
0 \to \Gamma (E, \u{\mathscr{F}}) \xrightarrow{i^\ast} \Gamma (E,
\u{\mathscr{G}}_\circ) \xrightarrow{d^\ast_\circ}  \ldots
\xrightarrow{d^{\ast}_{k-1}} \Gamma (E, \u{\mathscr{G}}_k)
\xrightarrow{d^\ast_k} \ldots 
$$
and by the abstract de Rham theorem, we have
\begin{equation*}
H^p (E, \u{\mathscr{F}}) \simeq \text{ kernel } d^\ast_p / \text{
  image } d^{\ast}_{p-1} (p \geq 1). 
\tag{2}\label{chap15:eq2}
\end{equation*}
As in the case when $p=0$, as $L$ shrinks to $Y$, kernel $d^\ast_k$,
image $d^\ast_{k-1}$ (with $E$ replaced by $L$) have as their direct
limit, the kernel and image of the mappings
$$
\gamma(Y, \u{\mathscr{G}}_k) \to \Gamma (Y, \u{\mathscr{G}}_{k+1})
\text{ and } \Gamma (Y, \u{\mathscr{G}}_{k-1}) \to \Gamma(Y,
\u{\mathscr{G}}_k)  
$$
respectively and an application of (\ref{chap15:eq2}) with $E =Y$
establishes the lemma. 
\end{proof}

\section{Coherent analytic sheaves on an analytic submanifold}\label{chap15:sec2}

Let $X$ be a topological space, $Y$ a closed subset of $X,
\u{\mathscr{F}}$ a sheaf of abelian groups on $Y$. We define a sheaf
$\tilde{\u{\mathscr{F}}}$ on $X$ by setting $\tilde{\mathscr{F}}_a =
\mathscr{F}_a$ if $a \in Y, = $ the group 0 if $a \not\in Y$. Clearly,
this defines a  sheaf\pageoriginale on $X$. Then one has 

\begin{proposition}\label{chap15:prop2}
For $p=0,1, \ldots$
$$
H^p (Y, \u{\mathscr{F}}) \simeq H^p (X, \tilde{\u{\mathscr{F}}}). 
$$
\end{proposition}

\begin{proof}
If $\mathscr{O} = \{\mathscr{O}_i\}_{i \in I}$ is an open covering of
$X$, $\mathscr{O}'_i = \mathscr{O}_i \cap Y$, then $\{\mathscr{O}'_i\}
= \mathscr{O}'$ is an open covering of $Y$ and clearly
$$
H^p (\mathscr{O}, \tilde{\u{\mathscr{F}}}) \simeq H^p (\mathscr{O}'
\u{\mathscr{F}}). 
$$
Also given an open covering $\mathscr{O}' = \{\mathscr{O}'_i\}$ of
$Y$, if $\mathscr{O}_i$ is such that $\mathscr{O}_i\cap Y =
\mathscr{O}'_i$, then the open covering $\{\mathscr{O}_i, X - Y\}$ of
$X$ gives rise to $\mathscr{O}'$ in the above way and Proposition
\ref{chap15:prop2} follows. 
\end{proof}

\begin{defi*}
Let $V^n$ be a complex analytic manifold of complex dimension $n$. Let
$W^m$ be a closed subset of $V^n$. $W^m$ is called an 
\textit{analytic submanifold of dimension $m$}, if, for $a \in W^m$,
the local coordinates $(z_1, \ldots, z_n)$ at a on $V^n$ (in an open
set $U \subset V^n$) can be so chosen that $W^m \cap U = \{z \in U
\big| z_{m+1} = \ldots = z_n = 0\}$. 
\end{defi*}

An application of the implicit function theorem shows that if $W$ is a
complex analytic manifold of dimension $m$ and $i$ an analytic one-one
mapping of $W$ into $V^n$, $i(W)$ is an analytic submanifold of $V^n$
if and only if 
\begin{itemize}
\item[1)] $i$ is proper: the inverse image of a compact subset of
  $V^n$ is a  compact subset of $W$; 

\item[2)] $i$ has rank $m$ (i.e., the Jacobian matrix of $i$ has rank
  $m$ at every point of $W$).
\end{itemize}

Let $V$ be a complex manifold and $W$ a submanifold of $V$. Let $X
\subset V$ and $Y = X \cap W$. We shall denote by ${}_V
\u{\mathscr{O}}$, ${}_W \u{\mathscr{O}}$ the sheaves of germs of
holomorphic functions on $V$ and $W$ respectively, considered as
complex manifolds \textit{in their own rights.}

Let\pageoriginale $\u{\mathscr{F}}$ be a coherent ${}_W
\u{\mathscr{O}}$-analytic sheaf on $Y$, and $\tilde{\u{\mathscr{F}}}$
the sheaf which continues $\u{\mathscr{F}}$ to $X$ by $0$ outside
$Y$. Then $\tilde{\u{\mathscr{F}}_a}$ has a structure of ${}_V
\u{\mathscr{O}}$-analytic sheaf on $X$: if $a \not\in Y$,
$\tilde{\mathscr{F}}_a$ is an ${}_V \mathscr{O}_a$-module
($\tilde{\mathscr{F}}_a$ being 0); if $a \in Y$, $f_a \in \tilde{\mathscr{F}}_a
= \mathscr{F}$  and $h_a \in {}_V \mathscr{O}a$, $h_a f_a$ is defined
to be $h_{aW} f_a$, where $h_{aW}$ is the restriction of $h_a$ to
$W$. 

Let $\u{\mathfrak{I}}(W)$ be the subsheaf of ${}_V\u{\mathscr{O}}$
consisting of those germs which vanish on $W$. Then we have the
isomorphism
$$
{}_V \u{\mathscr{O}} / \u{\mathfrak{I}} (W) \simeq W
\tilde{\u{\mathscr{O}}} ; 
$$
by a theorem of Cartan \cite{p3:key2}, \cite[lecture XVI]{p3:key3},
$\u{\mathfrak{I}} (W)$  is coherent.

\begin{proposition}\label{chap15:prop3}
If $\u{\mathscr{F}}$ is a coherent ${}_W\u{\mathscr{O}}$-analytic
sheaf on $Y$, then $\tilde{\u{\mathscr{F}}}$ is a coherent
${}_V\mathscr{O}$-analytic sheaf on $X$. 
\end{proposition}

\begin{proof}
If $\u{\mathscr{F}}$ is the sheaf ${}_W \u{\mathscr{O}}_Y$
(restriction of ${}_W\u{\mathscr{O}}$ to $Y$), then, by what we
observed above,
$$
{}_V \u{\mathscr{O}}_X / \u{\mathfrak{I}} (W)_X \simeq {}_W
\tilde{\u{\mathscr{O}}} 
$$
and so $\tilde{\u{\mathscr{O}}}$ is a coherent ${}_V
\u{\mathscr{O}}$-analytic sheaf.

In the general case, let $\u{\mathscr{F}}$ be a sheaf on $Y$, $a \in
Y$, $\Omega$ an open neighbourhood of a in $Y$ such that 
$$
\u{\mathscr{F}}_\Omega \simeq {}_W \u{\mathscr{O}}^N_\Omega /
\u{\mathscr{R}}, 
$$
where $\u{\mathscr{R}}$ is a coherent analytic subsheaf of
${}_W\u{\mathscr{O}}^N$.

Let $\Omega$ be chosen so small that there are $N_1$ sets of $N$
holomorphic functions on $W$ which ${}_W \mathscr{O}_b$-generate
$\mathscr{R}_b$ at every point of $\Omega$. If $\Omega$ is again
sufficiently small, there is a neighbourhood $\Omega'$ of a in $V$
such that $\Omega' \cap Y = \Omega$ and these functions are
restrictions of holomorphic functions in $\Omega'$ to $\Omega$.
Let $\mathscr{R}'$ be the subsheaf of
${}_V\u{\mathscr{O}}^N_{\Omega'}$ generated by these\pageoriginale
$N_1$ elements of ${}_V\u{\mathscr{O}}^N_{\Omega'}$. Then
$\u{\mathscr{R}}'$ is a coherent analytic subsheaf of
${}_V\u{\mathscr{O}}^N_{\Omega'}$, while clearly
$\tilde{\mathscr{F}}_{\Omega' \cap X} \simeq
{}_V\u{\mathscr{O}}^N_{\Omega' \cap X} / \u{\mathscr{R}}'_X +
\u{\mathfrak{I}}^N (W)_X$ and the proposition follows.
\end{proof}

\begin{proposition}\label{chap15:prop4}
Let $P$ be a closed polydisc in $C^n$, $\prod$ an open polydisc
$\supset P$. Let $W$ be an analytic manifold which is a submanifold of
$\prod$. Then, Theorems A) and B) are true for $W \cap P$ (considered
as subset of $W$). This follows at once from Propositions
\ref{chap15:prop1}, \ref{chap15:prop2} and \ref{chap15:prop3}. 
\end{proposition}

\section{Stein Manifolds}\label{chap15:sec3}

\begin{defi*}
A complex analytic manifold $V$ of dimension $n$ which is countable at
infinity is said to be a \textit{Stein manifold} if
\begin{itemize}
\item[$(\alpha)$] $V$ is holomorph-convex (VII, 3);

\item[$(\beta)$] for any two points $a \neq b$ on $V$, there exists a 
holomorphic $f$ on $V$, such that $f(a) \neq f(b)$. 

\item[$(\gamma)$] if $a \in V$, there are $n$ functions holomorphic in
  $V$ which form a system of local coordinates at $a$.
\end{itemize}
\end{defi*}

\medskip
\noindent{\textbf{Examples of Stein manifolds.}}
\begin{enumerate}
\item Univalent domains of holomorphy in $C^n$ (see VII, Prop. 1)

\item Any open connected Riemann surface. [A Riemann surface is
  countable at infinity by Rado's theorem; $(\alpha)$, $(\beta)$ and
  $(\gamma)$ follow from Runge's theorem. For the details of proof,
  see H. Behnke and K. Stein: Entwicklung analytischer Funktionen auf
  Riemann\-schen Fl\"achen, Math. Ann., 120 (1948), 430-461, and
  B. Malgrange: Existence at approximation des solutions des
  \'equations aux d\'exiv\'ees partielles et des \'equations de
  convolution Th\`ese, Paris, 1956 (Chap.III, \S 4)]. 

\item Analytic\pageoriginale submanifolds of $C^n$. In particular, algebraic
  varieties over $C$ which have no singularities.
\end{enumerate}

\begin{lem}[on Stein manifolds]\label{chap15:lem2}
Let $V$ be a Stein manifold, $K$ a compact subset of $V$ such that $K
=\hat{K}$ ($\hat{K}$ is the $\mathscr{H}_V$-envelope of $K$; see
VI). Then $K$ has a fundamental system $\{L\}$ of compact
neighbourhoods $L$ having the following properties:

To every $L$ correspond an open set $\wedge \supset L$, a closed
polydisc $P\subset C^N$ an open polydisc $\prod \supset P$ and a
finite number $N$ of holomorphic functions $f_1, \ldots, f_N$ on $V$
such that the restrictions to $\wedge$ of the $f_i$ realize $\wedge$
as an analytic submanifold of $\prod$ and such that $\phi(L) = P \cap
\phi(\wedge)$ where $\phi$ is the mapping $(f_1, \ldots, f_N)$ of $V$
in $C^N$. 
\end{lem}

\begin{proof}
Let $\Omega$ be a relatively compact neighbourhood of $K$, and let $F$
be the boundary of $\Omega$. For every $a \in F$, there is an $f$ such
that $|f(a)|>1$, $||f||_K < 1$. Since $F$ is compact and the set of a
with $|f(a)|>1$ is open, there are a finite number, $f_1, \ldots,
f_{N'}$ of holomorphic functions such that $||f_i||_K \leq \theta <1$
while $\max\limits_i |f_i(a)|>1$ for $a \in F$. Let $\Omega'$ be the
set of $a \in \Omega$ with $|f_i(a)| <1$ for $i = 1, \ldots,
N'$. Also, the set of $a \in \Omega$ with $|f_i(a)| \leq \rho$,
$\theta < \rho < 1$ is compact since $\bar{\Omega'}$ is compact and
the closure of this set does not intersect $F$. This shows that the
mapping of $\Omega'$ in $C^{N'}$ defined by $(f_1, \ldots, f_{N'})$ is
proper. 

Set $\wedge = \Omega'$. By adjoining a finite number of functions
$f_{N'+1}, \ldots, f_N$ to $f_1, \ldots, f_{N'}$ we can ensure that
points of $\Omega$ are separated by the mapping $\phi = (f_1, \ldots,
f_N)$ and such that $\phi$ is of maximal rank \{this follows from
properties $(\beta)$, $(\gamma)$ of Stein manifolds and the
compactness of $\bar{\Omega}$\}.

If\pageoriginale $1 > \rho >\theta$ and $||f_{N'+1}||_K < A, \ldots,
||f_N||_K<A$, we thke $P$ to be the polydisc $|z_i| \leq \rho$, $i
\leq N'$, $|z_i| \leq A$, $i \geq N'+1$ in $C^N$ and $L$ to be the
inverse image in $\wedge$ of $P \cap \phi(\wedge)$ under the mapping
$\phi$. Since $K$ has a fundamental system of relatively compact
neighbourhoods $\Omega$, Lemma \ref{chap15:lem2} is proved. 
\end{proof}

\begin{theorem*}
Let $V$ be a Stein manifold, $K$ a compact set $\subset V$ such that
$K = \hat{K}$. Then 
\begin{itemize}
\item[1)] Theorems A) and B) are true for $K$.

\item[2)] Every holomorphic function on $K$ can be approximated,
  uniformly on $K$ by holomorphic funtions on $V$.
\end{itemize}
\end{theorem*}

\begin{proof}
1) follows from Proposition \ref{chap15:prop4} and Lemmas
\ref{chap15:lem1} and \ref{chap15:lem2}. To 
prove 2), let $g$ 
be a holomorphic function on $K$, and $L$ a neighbourhood of $K$
having the properties of Lemma \ref{chap15:lem2}, such that $g$ is
holomorphic on $L$.  

Now, by Proposition \ref{chap15:prop4}, we have the exact sequence
$$
{}_{C^N} \u{\mathscr{O}}_P \to {}_V\tilde{\u{\mathscr{O}}_L} \to 0
$$
($L$ is considered as a subset of $\prod$),
and if $\u{\mathfrak{I}}$ is the kernel of the first mapping, the
sequence
$$
0 \to \u{\mathfrak{I}}  \to {}_{C^N}\u{\mathscr{O}}_P \to{}_V
\tilde{\u{\mathscr{O}}}_L \to 0
$$
is exact. The associated exact cohomology sequence
$$
H^\circ (P, {}_{C^N} \u{\mathscr{O}}_P) \to H^\circ (P, {}_V 
\tilde{\u{\mathscr{O}}}_L ) \to H^1 (O, \u{\mathfrak{I}})  
$$
shows, since $H^1 (P,\u{\mathscr{I}}) = 0$ by Theorem B) for a
polydisc, that every element of $H^\circ (P, {}_V
\tilde{\u{\mathscr{O}}}_L)$ is the image of an element of $H^\circ
(P,{}_{C^N} \u{\mathscr{O}}_P)$ and $g$ is the restriction to $L$ of a
holomorphic function on $P$; hence $g$ can be expanded in a power
series in the $f_1, \ldots, f_N$ which converges uniformly on $K
\subset \overset{\circ}{L}$.\pageoriginale Since the partial sums of
this power series, being polynomials in $f_1, \ldots, f_N$ are
holomorphic on $V$, the theorem is proved.
\end{proof}

\begin{center}
\textbf{Appendix}
\end{center}

\begin{theorem*}
Let $X$ be a topological space, $\u{\mathscr{F}}$ a sheaf of abelian
groups on $X$ which is such that any section of $\u{\mathscr{F}}$ over
an open set of $X$ can be extended to a section of $\u{\mathscr{F}}$
over $X$. Then, for any open covering $\mathscr{O} =
\{\mathscr{O}_i\}_{i \in I}$ of $X$, $H^p (\mathscr{O},
\u{\mathscr{F}}) = 0$ for $p > 0$, and in particular $H^p (X,
\u{\mathscr{F}}) = 0$ for $p > 0$. 
\end{theorem*}

\begin{proof}
The proof is by induction on $p$. 
\begin{itemize}
\item[a)] $p=1$. Let $c$ be a 1-cocycle of the covering $\mathscr{O} =
  \{\mathscr{O}_i\}_{i \in I}$. Suppose $J$ is a subset of the
  indexing set $I$ such that there is a $0$-cochain $\gamma$ of
  $\mathscr{O}$ with $\gamma_i - \gamma_j = c_{ij}$ for $i$, $j \in
  J$. Let $\alpha \in I$, $\alpha \not\in J$. We define a 0-cochain
  $\gamma'$ as follows: $\gamma'_i = \gamma_i$ if $i \neq \alpha$,
  $\gamma'_{\alpha} = \gamma_i + c_{\alpha i}$ on $\mathscr{O}_{\alpha
  } \cap \mathscr{O}_i$, $i \in J$. Then on $\mathscr{O}_\alpha \cap
  \mathscr{O}_i \cap \mathscr{O}_j $, $\gamma_i + c_{\alpha i} =
  \gamma_j + c_{\alpha j}$ since $\gamma_i - \gamma_j = c_{ij} =
  c_{\alpha j} - c_{\alpha i}$ ($c$ being alternate). Hence $\gamma'$
  is defined uniquely on $\bigcup\limits_{i \in J} (\mathscr{O}_\alpha
  \cap \mathscr{O}_i)$. By hypothesis, $\gamma'_\alpha$ can be
  extended to a section of $\u{\mathscr{F}}$ over $\mathscr{O}_\alpha$
  and the cochain $\gamma'$ is defined completely. Also $\gamma'_i -
  \gamma'_j = c_{ij}$ if $i$, $j \in J \cup \{\alpha\}$.  It is clear
  that $J$ is non-empty (since $c$ is alternate) and the theorem for
  $p=1$ follows by an application of Zorn's lemma. 

\item[b)] $p>1$. Suppose the theorem true with $p$ replaced by $p-1$ for
  all spaces $X$, all coverings $\mathscr{O}$ of $X$ and all
  $(p-1)$-cocycles of $\mathscr{O}$. Let $c$ be a $p$-cocycle and let
  $J$ be a subset of $I$ such that there is a $(p-1)$-cochain $\gamma$
  with $(\delta \gamma)_{i_\circ \ldots i_p}$ for $i_\circ, \ldots,
  i_p \in J$. Let $\alpha \in I$, $\alpha \not\in J$. 
\end{itemize}

For every\pageoriginale $i_\circ, \ldots, i_{p-2} \in J$, we determine
a section $\gamma'_{i_\circ \ldots i_{p-2} \alpha} $ of
$\u{\mathscr{F}}$ over $\mathscr{O}_{i_\circ \ldots i_{p-2} \alpha}$
such that 
$$
c_{i_\circ \ldots i_{p-1}\alpha} = \sum\limits^{p-1}_{k=0} (-1)^k
\gamma'_{i_\circ \ldots \hat{i}_{k} \ldots i_{p-1} \alpha} + (-1)^p
\gamma_{i_\circ \ldots i_{p-1}}
$$
over $\mathscr{O}_{i_\circ \ldots i_{p-1} \alpha}$. This is possible:
it is easily seen, from the definition of $\gamma$, that the
$(p-1)$-cochain $c'$ defined by 
$$
c'_{i_\circ \ldots i_{p-1}} = c_{i_\circ \ldots i_{p-1} \alpha} +
(-1)^{p-1} \gamma_{i_\circ \ldots i_{p-1}}
$$
is a cocycle of the covering $\{\mathscr{O}_\alpha \cap
\mathscr{O}_i\}_{i \in J}$ of the space $Y = \bigcup\limits_{i \in J}
(\mathscr{O}_\alpha \cap \mathscr{O}_i)$ (since $c$ is a cocycle) and
the existence of the $\gamma'_{i_\circ \ldots i_{p-2}\alpha}$ follows
from inductive hypothesis.

We new define a $(p-1)$-cochain $\gamma_1$ as follows: if $i_\circ,
\ldots, i_{p-2} \in  J$, $(\gamma_1)_{i_\circ \ldots i_{p-2} \alpha} =
\gamma'_{i_\circ \ldots i_{p-2} \alpha}$; $\gamma_1$ is defined by the
condition that it is alternate for other $p$-tuples of indices of $J
\cup \{\alpha\}$ which contain $\alpha$ and $(\gamma_1)_t = \gamma_t$
if $t \in J^p$. $\gamma_1$ has the property that 
$$
(\delta \gamma_1)_{j_\circ \ldots j_p} = c_{j_\circ \ldots j_p} \text{
for } j_\circ, \ldots, j_p \in J \cup \{\alpha\}
$$
while $\gamma_1 = \gamma$ on $J^p$. If we partially order the pairs
$(J, \gamma)$ by setting $(J, \gamma) < (J', \gamma')$ if $J \subset J'$
and $\gamma' = \gamma$ on $J^p$, the theorem follows by an application
of Zorn's lemma.
\end{proof}
