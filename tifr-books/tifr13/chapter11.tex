\part{Coherent Analytic Sheaves}\label{part3}

\chapter{Sheaves}\label{chap11}

\begin{defi*}
Let\pageoriginale $X$ and $\mathscr{F}$ be topological spaces and
$\pi$ a mapping $\mathscr{F} \to X$ such that
\begin{itemize}
\item[(i)] $\pi$ is onto;

\item[(ii)] $\pi$ is a local homeomorphism.
\end{itemize}
Then we call $\mathscr{F}$ (with the mapping $\pi$) a \textit{sheaf}
on $X$. $\pi$ will be called the \textit{projection} of $\mathscr{F}$
on $X$.

A \textit{section} of the sheaf $\mathscr{F}$ (over $X$) is a
continuous mapping $s : X \to \mathscr{F}$ such that $\pi \circ s = $
identity.

If $W$ is any subset of $X$, $\pi^{-1} (W)$ is a sheaf on $W$ in a
natural way 

\textit{A section of the sheaf $\mathscr{F}$ over a subset $W$ of }
$X$ is a section of the sheaf $\pi^{-1}(W)$ over $W$. 

We shall sometimes say `section' for the image of the section $s$ in
$\mathscr{F}$. 

If $x \in X$, $\mathscr{F}_x$ will stand for $\pi^{-1} (x)$. 
\end{defi*}

\setcounter{proposition}{0}
\begin{proposition}\label{chap11:prop1}
A section over an open set $U \subset X$ is an open set in
$\mathscr{F}$. 
\end{proposition}

\begin{proof}
Let $s : U \to \mathscr{F}$ be a section over $U$. Let $a \in s(U)$
and let $\mathscr{O}$ be an open set in $s(U)$ such that $\mathscr{O}
= \mathscr{O}' \cap s (U)$ where $\mathscr{O}'$ is open in
$\mathscr{F}$ and $\pi$ restricted to $\mathscr{O}'$ is a
homeomorphism of $\mathscr{O}'$ onto an open subset of $X$. Since
$s$ is continuous, $s^{-1}(\mathscr{O})$ is open in $X$, i.e., $\pi
(\mathscr{O})$ is open in $X$ and so in $\pi(\mathscr{O}')$. Since
$\pi$ is a homeormphism in $\mathscr{O}', \mathscr{O}$ is an open set
in $\mathscr{O}'$ and so in $\mathscr{F}$. 

This\pageoriginale implies, in particular, that two sections which
coincide at a point, coincide in a neighbourhood of this point. 
\end{proof}

\begin{proposition}\label{chap11:prop2}
If $\mathscr{F}$ is a Hausdorff space, a section over a closed set is
closed. 
\end{proposition}

\begin{proof}
Let $W$ be a closed set in $X$, and let $a \in \overline{s(W)}$ ($s$ is
the given section). Let $x = \pi (a)$. Let $\Omega_a$ be an open
neighbourhood of a, homeomorphic with its projection. $a$ belongs to
the closure of $\Omega_a \cap s (W)$  in $\Omega_a$ and so $\pi (a)$
belongs to the closure of $\pi (\Omega_a \cap s (W)) \subset \pi \circ
s (W) = W$ so that, since $W$ is closed, $\pi (a) \in W$. Suppose next
that $s(x) = b \neq a$ and let $\Omega_a$, $\Omega_b$ be disjoint
neighbourhoods of $a$, $b$, homeomorphic with their projections. If
$y$ is near enough to $x$, then $s(y) \in \Omega_b$ and consequently
$\Omega_a$ does not meet $\overline{s(W)}$, a contradiction.

Proposition \ref{chap11:prop2} is not true if $\mathscr{F}$ is not a
Hausdorff space.  
\end{proof}

\medskip
\noindent{\textbf{Examples of sheaves.}}
\begin{itemize}
\item[$1^\circ)$] If $V$, $W$ are manifolds and $W$ is spread on $V$
  by a mapping onto $V$, $W$ is a sheaf on $V$.

\item[$2^\circ)$] $X$ being a topological space, $Y$ an arbitrary set,
  the set of all mappings $X \to Y$ give rise to a sheaf, the sheaf of
  germs of the mappings $X \to Y$ in the following way: two mappings
  of a neighbourhood of $x \in X$ are identified if they coincide in a
  neighbourhood of $x$. The set of equivalence classes at $x$ is
  $\mathscr{F}_x$ and $\mathscr{F} = \bigcup\limits_{x \in X}
  \mathscr{F}_x$. The topology on $\mathscr{F}$ is obtained by a
  method similar to that used in the case of the sheaf of germs of
  holomorphic function in IV. 

\item[$3^\circ)$] In the same way, $X$, $Y$, being topological spaces,
  one defines the sheaf of germs of continuous mappings of $X$ in
  $Y$. 

\item[$4^\circ)$] If\pageoriginale $V$, $W$ are complex analytic
  manifolds, we can define the sheaf of germs of analytic mappings of
  $V$ in $W$. This sheaf is a Hausdorff space, by reasoning similar to
  that used in IV when $W = C$. 
\end{itemize}

\medskip
\noindent{\textbf{Diagonal product of two sheaves.}}

Let $(\mathscr{F}, \pi)$, $(\mathscr{G}, \pi')$ be sheaves on $X$. The
set of $(f, g) \in \mathscr{F} \times \mathscr{G}$ such that $\pi(f) =
\pi'(g)$ can be topologized in a natural way. This set then forms a
sheaf on $X$ called the \textit{diagonal product} $\mathscr{F}_V
\mathscr{G}$ of the sheaves $\mathscr{F}$ and $\mathscr{G}$. 

\medskip
\noindent{\textbf{Sheaf of groups.}} Let $\mathscr{F}$ be a sheaf on
$X$. $\mathscr{F}$ is a \textit{sheaf of groups} if 
\begin{itemize}
\item[$1^\circ)$] for ever $x \in X$, $\mathscr{F}_x$ is a group;

\item[$2^\circ)$] the mapping $a \to a^{-1}$ of $\mathscr{F} (a,
  a^{-1} \in \mathscr{F}_x)$ in $\mathscr{F}$ is continuous.

\item[$3^\circ)$] the mapping $(f, g) \to fg$ of $\mathscr{F}_V
  \mathscr{F}$ to $\mathscr{F}$ is continuous.
\end{itemize}

\begin{example*}
The sheaf of germs of continuous mappings of $X$ in a topological
group is a sheaf of groups.
\end{example*}


\medskip
\noindent{\textbf{Sheaf of rings.}}

Let $\mathscr{F}$ be a sheaf on $X$. $\mathscr{F}$ is a \textit{sheaf
  of rings} if 
\begin{itemize}
\item[$1^\circ)$] for every $x \in X$, $\mathscr{F}_x$ is a ring;

\item[$2^\circ)$] $\mathscr{F}$ is a sheaf of (additive, abelian) groups;

\item[$3^\circ)$] the mapping $(f,g) \to fg $ of $\mathscr{F}_V
  \mathscr{F} \to \mathscr{F}$ is continuous. 
\end{itemize}

Let $\mathfrak{a}$ be a sheaf of rings and $\mathfrak{m}$ a sheaf on
the same space $X \cdot \mathfrak{m}$ is called a \textit{sheaf of
  $\mathfrak{a}$ modules} if
\begin{itemize}
\item[$1^\circ)$] $\mathfrak{m}$ is a sheaf of abelian groups (additive);

\item[$2^\circ)$] for every\pageoriginale $x \in X$, $\mathfrak{m}_x$ is an
  $\mathfrak{a}_x$-module;

\item[$3^\circ)$] the mapping $(\alpha, m) \to \alpha m$ of
  $\mathfrak{a}_V \mathfrak{m} \to \mathfrak{m}$ is continuous (it
  being assumed that $\mathfrak{m}_x$ is a left $\mathfrak{a}_x$
  module). 
\end{itemize}

The following is an important example.

Let $\{\mathscr{O}, c_{ij}\}$ cefine a linear bundle $E$ on the space
$V$ (the $c_{ij}$ are continuous mappings of $\mathscr{O}_i \cap
\mathscr{O}_j$ into $GL(m,C)$). Let $p$ be the projection of $E$ onto
$V$. A \textit{cross-section} of the bundle is a continuous map $s : V
\to E$ such that $p \circ s$ is the identity.

We can now define the \textit{sheaf of germs of sections of the
  bundle} in the usual way. This sheaf is a sheaf of
$\mathscr{C}$-modules, where $\mathscr{C}$ is the sheaf of germs of
continuous, complex valued functions in $V$. 

It is of importance to decide when a sheaf of modules over
$\mathscr{C}$ can be obtained from a bundle by the method given
above. 

Suppose that the sheaf $\mathscr{F}$ can be so obtained. Then, since
every $a \in V$ has a neighbourhood $U$ such that $p^{-1}(U) \simeq U
\times C^m$ (as linear bundles) locally, the contraction of
$\mathscr{F}$ to $U$ is isomorphic (as a sheaf) to $\mathscr{C}^m_U$
where $\mathscr{C}_U$ is the sheaf of germs of continuous functions
in $U$. 

Suppose, conversely, that the sheaf $(\mathscr{F}, \pi)$ has the above
property. Let $\{\mathscr{O}_i\}$ be a covering of $V$ such that
$\pi^{-1} (\mathscr{O}_i) \simeq \mathscr{C}^m_{\mathscr{O}_i}$. Then
$\mathscr{O}_i \cap \mathscr{O}_j$, being an open subset of both
$\mathscr{O}_i$ and $\mathscr{O}_j$, gives rise to an automorphism
$\phi$ of $\mathscr{C}^m_{\mathscr{O}_i \wedge \cap \mathscr{O}_j}$ as
a sheaf of modules over $\mathscr{O}_{\mathscr{O}_i \cap
  \mathscr{O}_j}$. Let $e_p \in \mathscr{C}^m_{\mathscr{O}_i \cap
  \mathscr{O}_j }$ be the element defined by $(0, \ldots, 0, 1, 0,
\ldots, 0)$ where all but the $p$-th place contain 0. Then $\phi(e_p)
= \sum a_{pq} e_{q}$, (since $\phi$ is an automorphism of the sheaf
$\mathscr{C}^m_{\mathscr{O}_i \cap \mathscr{O}_j}$ as a sheaf of
modules\pageoriginale over $\mathscr{C}_{\mathscr{O}_i \cap
  \mathscr{O}_j}$) and we can define the matrix $c_{ij} =
(a_{pq})$. Since $\phi$ is an automorphism, $c_{ij}$ is invertible. It
is easy to see that the bundle defined by $\{\mathscr{O}_i, c_{ij}\}$
gives rise to the sheaf $\mathscr{F}$. 

This leads to a one-one correspondence between classes of linear
bundles and sheaves locally isomorphic with $\mathscr{C}^m$. 

Also there is a one-one correspondence between cross-sections of a
bundle and sections of the sheaf defined by the bundle. 
