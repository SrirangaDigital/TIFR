\chapter{General properties of Coherent Analytic Sheaves}\label{chap12}

\section{Analytic Sheaves}\label{chap12:sec1}

Let\pageoriginale $\u{\mathscr{F}}$ be a sheaf on the base space
$X$. The concept of a 
\textit{subsheaf} is defined in the obvious way (as a subset of
$\u{\mathscr{F}}$ which is made into a sheaf by the restriction of the
projection to the subset). It is clear that if $\u{\mathscr{H}}$,
$\u{\mathscr{G}}$ are subsheaves of $\u{\mathscr{F}}$, so is
$\u{\mathscr{H}} \cap \u{\mathscr{G}}$. 

Let now $\u{\mathscr{F}}$, $\u{\mathscr{G}}$ be two sheaves of groups on $X$
with projections $\pi_f$, $\pi_g$. Let $\phi$ be a mapping
$\u{\mathscr{F}} \to \u{\mathscr{G}}$. $\phi$ is called a
(\textit{sheaf}) 
\textit{homomorphism} if
\begin{itemize}
\item[(i)] $\phi$ is continuous;

\item[(ii)] $\pi_f = \pi_g \circ \phi$;

\item[(iii)] the restriction of $\phi$ to $\mathscr{F}_x(=
\pi^{-1}_f (x), x \in X)$ is a homomorphism of the group
$\mathscr{F}_x$ in $\mathscr{G}_x (= \pi^{-1}_g (x))$. 
\end{itemize}

There are corresponding definitions of subsheaves of sheaves of
algebraic structures and of homomorphisms between such sheaves. The
concepts of a one-one mapping (into), mapping onto, \textit{image} of
a homomorphism, \textit{kernel} of a homomorphism are defined in the
obvious way. 

A sequence
$$
\ldots \to \u{\mathscr{F}}_p \xrightarrow{d_p} \u{\mathscr{F}}_{p+1}
\xrightarrow{d_{p+1}} \u{\mathscr{F}}_{p+2} \to \ldots
$$
of sheaves $\u{\mathscr{F}}_p$ of groups (or other algebraic structures)
and homomorphisms $d_p : \u{\mathscr{F}}_p \to \u{\mathscr{F}}_{p+1}$ is said
to be \textit{exact at} $\u{\mathscr{F}}_p$ if the kernel of $d_{p+1} = $
image of $d_p$; it is \textit{exact} if it is exact at $\u{\mathscr{F}}_p$
for all $p$. 

\medskip
\noindent{\textbf{Quotient Sheaves.}}

Let $\u{\mathscr{G}}$ be a sheaf of groups on the topological space $X$,
$\mathscr{F}$ a subsheaf of $\u{\mathscr{G}}$ such that for every $x \in
X$, $\mathscr{G}_x$ is a normal subgroup of
$\mathscr{G}_x$. Then\pageoriginale there is precisely one sheaf
$\u{\mathscr{H}}$ on $X$ such that $\mathscr{H}_x = \mathscr{G}_x /
\mathscr{F}_x$ and the mapping $\eta:\u{ \mathscr{G}} \to \u{\mathscr{H}} $
($\eta_x : \mathscr{G}_x \to \mathscr{H}_x$ is the natural
homomorphism) is a sheaf homomorphism. We have only to set
$\u{\mathscr{H}} = \bigcup\limits_{x \in X} (\mathscr{G}_x /
\mathscr{F}_x)$ and put quotient topology on $\u{\mathscr{H}}$. It is
clear the the conditions on $\u{\mathscr{H}}$ above determine uniquely
the topology on it.

\medskip
\noindent{\textbf{Analytic Sheaves.}}

Let $V$ be a complex analytic manifold, and $\u{\mathscr{O}}
=\u{\mathscr{O}}_V$ the sheaf of germs of holomorphic functions on
$V$. 

\begin{defi*}
An \textit{analytic sheaf} on $V$ is a sheaf of $\u{\mathscr{O}}$-modules.

One can then define \textit{(analytic) subsheaves} of an analytic
sheaf $\u{\mathscr{F}}$ in the obvious way. Clearly, the intersection of
analytic subsheaves of $\u{\mathscr{F}}$ is analytic.
\end{defi*}

\medskip
\noindent{\textbf{Notations.}}
In what follows, $\mathscr{O}$ will denote the ring of holomorphic
funtions on the complex manifold $V$, $\u{\mathscr{O}}$ the sheaf of germs
of holomorphic functions on $V; \mathscr{O}^m, \u{\mathscr{O}}^m$ will
stand for the $m$-th powe of $\mathscr{O}$, $\u{\mathscr{O}}$
respectively. 

If $\u{\mathscr{F}}$ is a sheaf on the space $X$, and $U$ is a subset
of $X$, $\u{\mathscr{F}}_U$ will denote the restriction of
$\u{\mathscr{F}}$ to $U$. $\mathscr{F}_U$ or $\Gamma(U,\mathscr{F})$
will stand for the \textit{sections} of $\mathscr{F}$ over $U$. 

\begin{examples*}
\begin{itemize}
\item[$1^\circ)$] $\u{\mathscr{O}^m}$ is an analytic sheaf;

\item[$2^\circ)$] Let $\mathfrak{m}$ be a finitely generated submodule
  of $\mathscr{O}^m$ (generated by $h_1, \ldots, h_p$ say). For $a \in
  V$, let $\mathfrak{m}_a$ be the submodule of $\mathscr{O}^m_a$
  generated over $\mathscr{O}_a$ by $(h_1)_a, \ldots, (h_p)_a$. Then
  $\u{\mathfrak{m}} = \bigcup\limits_{a \in V} \mathfrak{m}_a$ is an
  analytic sheaf on $V$. To prove this, one has only to show that if
  $f$ is a section of $\u{\mathscr{O}}^m$ over an open neighbourhood
  $U$ of $a \in V$, then there is a neighbourhood $U_1 \subset U$ of
  $a$ such that $f_a \in \mathfrak{m}_a$ implies $f_b \in
  \mathfrak{m}_b$ for $b \in U_1$. 

Now,\pageoriginale $f_a = \sum\limits^p_{i=1} (\lambda_i)_a (h_i)_a$,
$(\lambda_i)_a \in\mathscr{O}_a$, and there is an open neighbourhood
$U_1$ of $a$, and functions $f$, $\lambda_i$ in $U_1$ such that $f =
\sum\limits^p_{i=1} \lambda_i h_i$ in $U_1$ , which implies that
$(f)_b = \sum\limits^p_{i=1} (\lambda_i)_b (h_i)_b \in \mathfrak{m}_b$
for $b \in U_1$. 

\item[$3^\circ)$] The \textit{sheaf of relations} between $q$ elements
  of $\mathscr{O}^m$. Let $h_1, \ldots, h_q \in \mathscr{O}^m$. Let
  $\mathscr{R}_a$  be the submodule of $\mathscr{O}^q_a$ consisting of
  the $q$-tuples $((c_1)_a, \ldots, (c_q)_a)\cdot (c_i)_a \in
  \mathscr{O}_a$ such that $(c_1)_a (h_1)_a + \cdot + (c_q)_a(h_q)_a =
  0$. Then it is easily verified that $\u{\mathscr{R}} =
  \u{\mathscr{R}} (h_1, \ldots, h_q) = \bigcup\limits_{a \in V}
  \mathscr{R}_a$  is an analytic sheaf, called the \textit{sheaf of
    relations} between $h_1, \ldots, h_q$.
\end{itemize}
\end{examples*}


\section{Coherent analytic subsheaves of \texorpdfstring{$\mathscr{O}^m$}{M0m}}\label{chap12:sec2}

\begin{defi*}
Let $\u{\mathscr{F}}$ be an analytic subsheaf of
$\u{\mathscr{O}}^m$. $\u{\mathscr{F}}$ is said to be \textit{coherent}
if the following is true: for every $a \in V$ there is a neighbourhood
$U$ of $a$, and a \textit{finite number of sections of $\mathscr{F}$
  over} $U$, $f_1, \ldots, f_q$, such that for every $b \in U$,
$\mathscr{F}_b$ is $\mathscr{O}_b$-generated by $(f_1)_b, \ldots,
(f_q)_b$. 
\end{defi*}

The following important theorem holds, but we shall not prove it
here. The theorem is due to $K$. Oka 7. For the proof see Cartan
\cite{p3:key2}, \cite{p3:key3} Lecture XV.


\medskip
\noindent{\textbf{Theorem of Oka.}}
Let $h_1, \ldots, h_q \in \mathscr{O}^m$. Then, the sheaf of relations
$\u{\mathscr{R}}(h_1, \ldots, h_q)$ is a coherent analytic sheaf.

\begin{coro*}
Let $\u{\mathscr{F}}, \u{\mathscr{G}}$ be coherent analytic subsheaves
of $\u{\mathscr{O}}^p$. Then $\u{\mathscr{F}} \cap \u{\mathscr{G}}$ is
a coherent analytic sheaf.
\end{coro*}

\medskip
\noindent{\textbf{Proof of the Corollary.}}
Let $U$ be a neighbourhood of $a \in V$, $f_1, \ldots, f_k$; $g_1,
\ldots, g_m$ sections of $\u{\mathscr{F}}$, $\u{\mathscr{G}}$ over $U$
such that $(f_1)_b, \ldots, (f_k)_b$ (resp. $(g_1)_b, \ldots,
(g_m)_b$) $\mathscr{O}_b$-generate $\mathscr{F}_b$
(resp. $\mathscr{G}_b$) for every $b \in U$. Consider the sheaf
$\u{\mathscr{R}} (f_1, \ldots, f_k, \; g_1, \ldots, g_m) =
\u{\mathscr{R}}$ on $U$. 

It is\pageoriginale coherent, by Oka's theorem. We define a map $\phi$
of $\u{\mathscr{R}}$ onto $\u{\mathscr{F}}_U \cap \u{\mathscr{G}}_U$, as
follows: For $x \in U$, let $r = ((c_1)_x, \ldots, (c_k)_x, (c'_1)_x ,
\ldots, (c'_m)_x) \in \u{\mathscr{R}}$. i.e., $\sum (c_i)_x (f_i)_x +
\sum (c'_j)_x (g_j)_x = 0$. The image $\phi(r)$ of this point is, by
definition, $\sum(c_i)_x (f_i)_x \in \mathscr{F}_x$. By the above
relation, $\phi (r) \in \u{\mathscr{G}}_x$ and so $\phi(r) \in
\mathscr{F}_x \cap \mathscr{G}_x$. $\phi$ is clearly a homomorphism of
$\u{\mathscr{R}}$ in $\u{\mathscr{F}}_U \cap \u{\mathscr{G}}_U$. Also,
if $f_x = \sum (c_i)_x (f_i)_x = \sum (c_j)_x (g_j) \in\mathscr{F}_x
\cap \mathscr{G}_x$, then $\phi(r) = f_x$ where $r = ((c_1)_x, \ldots,
(c_k)_x, (-c'_1)_x, \ldots, (-c'_m)_x)$ and so $\phi$ is a
homomorphism of $\u{\mathscr{R}}$ onto $\u{\mathscr{F}}_U \cap
\u{\mathscr{G}}_U$. It follows easily that $\u{\mathscr{F}}_U \cap
\u{\mathscr{G}}_U$ is coherent and the result follows.

We give some examples of non-coherent, analytic sheaves. 
\begin{itemize}
\item[(a)] Let $\u{\mathscr{F}}$ be a sheaf of ideals on $V$, i.e.,
  $\mathscr{F}_a$ is an ideal of $\mathscr{O}_a$ (so an
  $\mathscr{O}_a$-module). Suppose that $\u{\mathscr{F}}$ is coherent
  and let, in a neighbourhood $U$ of $a$, $f_1, \ldots, f_p$ generate
  $\mathscr{F}_b$ over $\mathscr{O}_b$. Then, the necessary and
  sufficient condition that $\mathscr{F}_b = \mathscr{O}_b$ is that at
  least one $f_i(b) \neq 0$. Hence the set of $b$ with $\mathscr{F}_b
  \neq \mathscr{O}_b$ is precisely the set of common zeros of $f_1,
  \ldots, f_p$. Hence the set of $b$ wiht $\mathscr{F}_b \neq
  \mathscr{O}_b$ is an \textit{analytic subset} of $V$ [i.e., locally
    in $V$, it is the set of common zeros of a finite number of
    holomorphic functions].

The complement of an open ball, $S$, in $C^n$ is not an analytic
subset. Hence if we get $\mathscr{F}_a = \mathscr{O}_a$ for $a \in S$,
$\mathscr{F}_a = 0$ for $a \notin S$, the analytic sheaf
$\u{\mathscr{F}} = \bigcup\limits_{a \in C^n} \mathscr{F}_a$ is not
coherent. 

\item[(b)] Let $\Omega \neq V$ be an open subset of the complex
  manifold $V$. Let $\mathscr{F}_a = \mathscr{O}_a$, $a \notin\Omega$;
  $\mathscr{F}_a = 0$, $a \notin \Omega$. The sheaf $\u{\mathscr{F}}
  = \bigcup\limits_{a \in V} \mathfrak{F}_a$ is analytic but not
  coherent (the definition is violated at a point on the boundary of
  $\Omega$). 
\end{itemize}

Let $\u{\mathscr{F}}$ be a coherent analytic sheaf on a (connected)
complex manifold $V$. Then $\mathscr{F}_a \neq 0$ at any point $a \in
V$ unless $\mathscr{F}_a = 0$ for all $a \in V$: if $\mathscr{F}_a =
0$ and $(f_1)_b, \ldots, (f_p)_b \mathscr{O}_b$-generate
$\mathscr{F}_b$ for $b \in U$ where\pageoriginale $U$ is a connected
neighbourhood of $a$, then $f_1, \ldots, f_p$ are zero in a
neighbourhood of a since $\mathscr{F}_ a= 0$ and $\mathscr{F}_b = 0$
for $b \in U$ by the principle of analytic continuation. Hence the set
of a with $\mathscr{F}_a = 0$ is open. It is easily proved in the same
way that this set is closed and the result follows. 

\section[General coherent analytic sheaves on a...]{General coherent
  analytic sheaves on a complex analytic 
  manifold}\label{chap12:sec3}

\begin{defi*}
Let $V$ be a complex analytic manifold and $\u{\mathscr{F}}$ an
analytic sheaf on $V$. $\u{\mathscr{F}}$ is said to be
\textit{coherent} if every $a \in V$ has an open neighbourhood
$\Omega$ such that $\u{\mathscr{F}}_\Omega \simeq
\u{\mathscr{O}}^p_\Omega \big/ \u{\mathscr{N}}$, where $\u{\mathscr{N}}$ is
a \textit{coherent subsheaf of} $\u{\mathscr{O}}^p_\Omega$ (in the
first sense).
\end{defi*}

For subsheaves of $\u{\mathscr{O}}^m$, the two definitions of
coherence coincide. It is clear that given a subsheaf of
$\u{\mathscr{O}}^m$, which, locally, is an \textit{arbitrary} quotient
of an $\u{\mathscr{O}}^p$, there is a natural homomorphism of
$\u{\mathscr{O}}^p_\Omega$ onto $\u{\mathscr{F}}_\Omega$ ($\Omega$
being an open neighbourhood of a given point $a \in V$) and
$\u{\mathscr{F}}$ is coherent in the first definition. Converselyl,
suppose that $\u{\mathscr{F}} \subset \u{\mathscr{O}}^m$ and that to
every $a \in V$, there is a neighbourhood $\Omega$ such that in
$\Omega$, $p$ elements $g_1, \ldots, g_p$ of $\mathscr{O}^p_\Omega$
generate $\u{\mathscr{F}}$. We have a homomorphism
$$
((\phi_1)_b, \ldots, (\phi_p)_b) \{\in \mathscr{O}^p_b\} \to
(\phi_1)_b (g_1)_b + \cdot + (\phi_p)_b (g_p)_b
$$
of $\u{\mathscr{O}}^p_\Omega$ onto $\u{\mathscr{F}}_\Omega$, of which
the kernel is the sheaf of relations $\u{\mathscr{R}}_\Omega(g_1,
\ldots,\break g_p)$ which is coherent by the theorem of Oka. Hence the
condition of the second definition is fulfilled. 

An example of a coherent sheaf which is not a subsheaf of some
$\u{\mathscr{O}}^m$  is the sheaf of germs of sections of an analytic,
non-trivial, bundle. This sheaf is locally isomorphic to an
$\u{\mathscr{O}}^m$. 

\setcounter{proposition}{0}
\begin{proposition}\label{chap12:prop1}
Let\pageoriginale $\u{\mathscr{F}}$ be a coherent analytic sheaf. Let
$f_1, \ldots, f_q$ be a finite number of sections of
$\u{\mathscr{F}}$. Then the sheaf $\u{\mathscr{R}} (f_1 , \ldots,
f_q)$ of relations between $f_1, \ldots, f_q$ is coherent. 
\end{proposition}

\begin{proof}
Let $a \in V$. There are $p$ sections $g_1, \ldots, g_p$ over a
neighbourhood $\Omega$ of a such that
\begin{itemize}
\item[(i)] $(g_1)_b, \ldots, (g_p)_b \mathscr{O}_b$-generate
  $\mathscr{F}_b$ for $b \in \Omega$. 

\item[(ii)] $\bar{\mathscr{R}}_\Omega(g_1, \ldots, g_p)$ is a coherent
  subsheaf of $\mathscr{O}^p_\Omega$. 
\end{itemize}
[This is just a reformulation of the definition of coherence]. We have
$(f_i)_a = \sum\limits^p_{j=1} (\lambda^j_i)_a (g_j)_a
((\lambda^j_i)_a \in \mathscr{O}_a)$ so that there is an open
neighbourhood $\Omega' \subset \Omega$ of a such that $f_i = \sum
\lambda^j_i g_j$ in $\Omega'$. Now suppose $b \in \Omega'$ and that
$((\mu^1)_b, \ldots, (\mu^q)_b) \in\mathscr{R}_b (f_1, \ldots,
f_q)$. Then, we have $\sum\limits^q_{i=1} (\mu^i)_b (f_i)_b =0$, so
that $\sum\limits_{i,j} (\mu^i)_b (\lambda^j_i)_b(g_j)_b = 0$. This
implies that $(\sum\limits_i (\mu^i)_b(\lambda^1_i)_b, \ldots,
\sum\limits_i (\mu^i)_b\break (\lambda^q_i)_b) \in\mathscr{R}_b (g_1,
\ldots, g_p)$. Since by hypothesis $\u{\mathscr{R}}_\Omega(g_1,
\ldots, g_p)$ is coherent, there exist functions $h^j_k$, $j=1, \ldots
p$, $k =1,\ldots, r$ and a neighbourhood $\Omega'' \subset \Omega'$ of
a such that 
$$
\sum\limits_i (\mu^i)_b  (\lambda^j_i)_b = \sum\limits^r_{k=1}
(\nu^k)_b (h^j_k)_b
$$
for $b \in \Omega''$, where $(\nu^k)_b \in \mathscr{O}_b$. This
implies that 
$$
(\mu^1, \ldots, \mu^q, -\nu^1, \ldots, -\nu^r) \in
\u{\mathscr{R}}_{\Omega''} (\lambda^j_i, h^j_k). 
$$
By Oka's theorem $\u{\mathscr{R}}_{\Omega''} (\lambda^j_i, h^j_k)$ is
coherent and the systems $(\mu^1, \ldots, \mu^q) \in
\u{\mathscr{R}}_{\Omega''} (f_1, \ldots, f_q)$ form a quotient of
$\u{\mathscr{R}}_{\Omega''} (\lambda^j_i, h^j_k)$  and, this being
$\subset \u{\mathscr{O}}^q_{\Omega''}$, $\u{\mathscr{R}}_{\Omega''}
(f_1, \ldots, f_q)$ is coherent. This proves Proposition \ref{chap12:prop1}. 
\end{proof}

\setcounter{thm}{0}
\begin{thm}\label{chap12:thm1}
Let\pageoriginale $\u{\mathscr{G}}$ and $\u{\mathscr{F}}$ be two coherent
analytic sheaves. Let $\phi:\u{\mathscr{G}} \to \u{\mathscr{F}}$ be a
homomorphism (as analytic sheaves) of $\mathscr{G}$ in
$\mathscr{F}$. Then the kernel, the image, the cokernel and the
coimage of $\phi$ are coherent analytic sheaves.

(The cokernel is $\u{\mathscr{F}}/ \phi (\u{\mathscr{G}})$, the
coimage is $\u{\mathscr{G}}/ \phi^{-1} (\u{\mathscr{O}})$.)
\end{thm}

\begin{proof}
\begin{enumerate}
\renewcommand{\labelenumi}{\theenumi)}
\item \textbf{The image of $\phi$.} Let $g_1, \ldots, g_q$ be sections of
  $\u{\mathscr{G}}$  over an open set $\Omega$ such that $(g_1)_a,
  \ldots, (g_q)_a$ $\mathscr{O}_a$-generate $\mathscr{G}_a$ for $a \in
  \Omega$. Then $\phi(g_1), \ldots, \phi(g_q)$ are sections of
  $\u{\mathscr{F}}$ over $\Omega$ and they $\mathscr{O}_a$-generate
  $\phi(\u{\mathscr{G}})_\Omega$ for $a \in \Omega$. Hence on
  $\Omega$, $\phi(\u{\mathscr{G}})$ is isomorphic with
  $\u{\mathscr{O}}^q_\Omega/ \u{\mathscr{N}}$ where $\u{\mathscr{N}} =
  \u{\mathscr{R}}_\Omega (\phi(g_1), \ldots, \phi(g_q))$. The result
  follows from Proposition \ref{chap12:prop1} and the definition.

\item \textbf{The kernel of $\phi$.} Let $g_1, \ldots, g_q \in
  \mathscr{G}_\Omega$ generate $\u{\mathscr{G}}_\Omega$. Now, the
  sheaf $\u{\mathscr{R}}_\Omega (\phi(g_1), \ldots, \phi(g_q))$ is
  coherent. We define a mapping of $\mathscr{R}_\Omega$ to the kernel
  of $\phi$ as follows: if $((c_1)_a, \ldots, (c_q)_a) \in
  \mathscr{R}_a$, map this point on $(c_1)_a (g_1)_a + \cdots +
  (c_q)_a (g_q)_a$. This gives us a homomorphism of
  $\u{\mathscr{R}}_\Omega$ onto the kernel of $\phi$ (restricted to
  $\Omega$) and by 1) the kernel is coherent. 

\item and 4)\textbf{The cokernel and the coimage of $\phi$.} After 1)
  and 2) it is clearly sufficient to prove the following statment:

If $\u{\mathscr{G}}$ is a coherent analytic sheaf, $\u{\mathscr{F}}$ a
coherent analytic subsheaf of $\u{\mathscr{G}}$, then
$\u{\mathscr{G}}/ \u{\mathscr{F}}$ is coherent. To every $a \in V$
corresponds an open set $\Omega$, $a \in\Omega$ with
$\u{\mathscr{G}}_\Omega \simeq \u{\mathscr{O}}^p_\Omega/
\u{\mathscr{N}}$ where $\u{\mathscr{N}}$ is coherent. Let $f'_1,
\ldots, f'_q \in \mathscr{F}_\Omega$ generate
$\u{\mathscr{F}}_\Omega$. Since $\u{\mathscr{F}}_\Omega \subset
  \u{\mathscr{G}}_\Omega$, there are elements $f_1, \ldots, f_q \in
  \mathscr{O}^p_\Omega$ which go into $f'_1, \ldots, f'_q$. Let
  $\u{\mathscr{R}}$ be the analytic sheaf on $\Omega$ generated by
  $f_1, \ldots, f_p$; this sheaf is clearly
  coherent. One\pageoriginale has $\u{\mathscr{F}}_\Omega \simeq
  (\u{\mathscr{R}} + \u{\mathscr{N}})/\u{\mathscr{N}}$, and
  consequently $\u{\mathscr{G}}_\Omega /\u{\mathscr{F}}_\Omega \simeq
  \u{\mathscr{O}}^q_\Omega / (\u{\mathscr{R}} + \mathscr{N})$. Since
  obviously $\u{\mathscr{R}} + \u{\mathscr{N}}$ is coherent, the
  result follows. 
\end{enumerate}
\end{proof}

\section[Coherent analytic sheaves on subsets of a...]{Coherent analytic sheaves on subsets of a complex analytic
  manifold}\label{chap12:sec4}
Let $X$ be a subset of the complex manifold $V$. An \textit{analytic}
sheaf on $X$ is defined to be a sheaf of $\u{\mathscr{O}}_X$-modules
($\u{\mathscr{O}}_X$ is, of course, the restriction of
$\u{\mathscr{O}} $ to $X$). The definition of a \textit{coherent}
analytic sheaf $\u{\mathscr{F}}$ on $X$ is the same as before: if
$\u{\mathscr{F}} \subset \u{\mathscr{O}}^p_X$, then $\u{\mathscr{F}}$
is coherent if to every $a \in X$ exist a neighbourhood $U$ in $X$ and
elements $f_1, \ldots, f_p \in \mathscr{F}_U$ such that $(f_1)_b,
\ldots, (f_p)_b$ $\mathscr{O}_{Xb}$-generate $\mathscr{F}_b$ for $b
\in U$. An analytic sheaf $\u{\mathscr{F}}$ on $X$ is
\textit{coherent}, if every $a \in X$ has an open neighbourhood $U
\subset X$ such that $\u{\mathscr{F}}_U \simeq \u{\mathscr{O}}^p_U /
\u{\mathscr{N}}$ where $\u{\mathscr{N}}$ is a coherent analytic
subsheaf of $\u{\mathscr{O}}^p_U$. The result of 3 generalize to the
sheaves $\mathscr{O}^p_X$. The following theorem will be proved.

\begin{thm}\label{chap12:thm2}
Let $X$ be a compact subset of the complex manifold $V$, and
$\mathscr{F}$ a coherent analytic sheaf on $X$. Then there is an open
set $\Omega \supset X$ (open in $V$) and a coherent analytic sheaf
$\u{\mathscr{G}}$  on $\Omega$ such that $\u{\mathscr{G}}_X \simeq
\u{\mathscr{F}}$.
\end{thm}

We begin the proof with a remark, which follows at once from the fact
that a section of sheaf is an open mapping and the definition of
coherence.

\begin{remark*}
Let $\u{\mathscr{F}}$, $\u{\mathscr{G}}$ be two coherent analytic
sheaves on a subset $Y$ of the complex manifold $V$, $f$, $g$ two
homomorphisms: $\u{\mathscr{F}} \to \u{\mathscr{G}}$. Then, the set of
$y \in Y$ with $f_y = g_y$ ($f_y, g_y$ are the homomorphisms
$\mathscr{F}_y \to \mathscr{G}_y$ determined by $f$, $g$ respectively)
is open in $Y$. 
\end{remark*}

For the proof of Theorem \ref{chap12:proofofthm2}, we require two lemmas.

\setcounter{lem}{0}
\begin{lem}\label{chap12:lem1}
Let\pageoriginale $\u{\mathscr{F}}$, $\u{\mathscr{G}}$ be coherent
analytic sheaves on a subset $Y \subset V$. Let $a \in Y$ and suppose
that there is a homomorphism $\phi_a : \mathscr{F}_a \to
\mathscr{G}_a$. Then there is a neighbourhood $\Omega$ of a in $Y$
such that $\phi_a$ can be continued to a homomorphism $\phi :
\u{\mathscr{F}}_\Omega \to \u{\mathscr{G}}_\Omega$ (in an obvious
sense). 
\end{lem}

\begin{proof}
Suppose that in a neighbourhood $\Omega'$ of a, $f_1, \ldots, f_q \in
\mathscr{F}_\Omega$, generate $\u{\mathscr{F}}_{\Omega'}$; let $g_1,
\ldots, g_p \in \u{\mathscr{G}}_{\Omega'}$ define respectively the
germs $(g_1)_a = \phi_a((f_1)_a), \ldots, (g_p)_a =
\phi_a((f_p)_a)$. The sheaf of relations between $f_1, \ldots,\break f_p$ is
coherent, and so, if $\Omega'$ is small enough, is generated in
$\Omega'$ by functions $(\lambda^k_i)$ and $\sum\limits_i
(\lambda^k_i)_a(f_i)_a = 0$ so that we have $\sum\limits_i
(\lambda^k_i)_a (g_i)_a = 0$. Let $\Omega \subset \Omega'$ be such
that $\sum\limits_i \lambda^k_i f_i = 0$, $\sum\limits_i \lambda^k_i
g_i =0$ in $\Omega$. Let $ b\in \Omega$ and let $\sum\limits_i
(\mu_i)_b (f_i)_b = 0$. Then $(\mu_i)_b = \sum\limits_k (a_k)_b
(\lambda^k_i)_b$ and $\sum\limits_i  (\mu_i)_b (g_i)_b  =
\sum\limits_k (a_k)_b \sum\limits_i (\lambda^k_i)_b (g_i)_b =
0$. Hence, if $b \in \Omega$, $\sum\limits_i (\mu_i)_b (f_i)_b = 0$
implies $\sum\limits_i (\mu_i)_b (g_i)_b =0$ and the homomorphism
$\phi$ on $\Omega$ can be defined by $\phi_b ((f_i)_b) = (g_i)_b$ for
$b \in \Omega$. 
\end{proof}

\begin{lem}\label{chap12:lem2}
Let $X$ be a compact set in $V$, $\u{\mathscr{F}}$, $\u{\mathscr{G}}$,
coherent analytic sheaves on a neighbourhood of $X$. Let $\phi$ be a
homomorphism $\u{\mathscr{F}}_X \to \mathscr{G}_X$. Then $\phi$ can be
continued to a homomorphism $\u{\mathscr{F}}_U \to
\u{\mathscr{G}}_U$, $U$ being a suitable neighbourhood of $X$. 
\end{lem}

\begin{proof}
By Lemma \ref{chap12:lem1}, for every $a \in X$, $\phi_a$ can be extended to a
homomorphism $\phi_{\Omega_a} : \u{\mathscr{F}}_{\Omega_a} \to
\u{\mathscr{G}}_{\Omega_a}$ in a neighbourhood $\Omega_a$ of $a$. Now
$\phi$ and $\phi_{\Omega_a}$ determine the same homomorphism $\phi_a$
of $\mathscr{F}_a \to \mathscr{G}_a$. By the remark before the proof
of Lemma \ref{chap12:lem1}, we may suppose that $\phi = \phi_{\Omega_a}$ in $\Omega_a
\cap X$.\pageoriginale Since $X$ is compact, we obtain a finite
covering $\{\Omega_i\}$ of $X$ and homomorphisms $\psi_i :
\u{\mathscr{F}}_{\Omega_i} \to \u{\mathscr{G}}_{\Omega_i}$ such that
$\psi_i = \psi_j =\phi$ in $\Omega_i \cap \Omega_j \cap X$. Again, by
the remark before Lemma \ref{chap12:lem1} and the compactness of $X$, we may assume
that $\psi_i = \psi_j$ in $U_i \cap U_j$ where $U_i$ is an open set
containing $\Omega_i \cap X$. It is clear that there is an open set $U
\supset X$ such that $\psi_i = \psi_j$ in $\Omega_i \cap \Omega_j \cap
U$ and $\psi_i = \phi$ on $\Omega_i \cap X$. Lemma \ref{chap12:lem2} follows. 
\end{proof}

\setcounter{proofoftheorem}{1}
\begin{proofoftheorem}\label{chap12:proofofthm2}
From the definition of coherent analytic sheaves (as locally
isomorphic to quotient of $\u{\mathscr{O}}^p_X$, $\u{\mathscr{O}}_X$
being the restriction of $\u{\mathscr{O}}_V$) it follows that if $a
\in X$, there is an open set $\Omega_a$, $a \in \Omega_a$ and a
coherent analytic sheaf $\u{\mathscr{G}}^a $ on $\Omega_a$ such that
$\u{\mathscr{G}}^a_{\Omega_a \cap X} \simeq
\u{\mathscr{F}}_{\Omega_a \cap X}$. Since $X$ is compact, there are
finitely many coherent analytic sheaves $\u{\mathscr{G}}^1, \ldots ,
\u{\mathscr{G}}^r$ on open sets $\Omega_1, \ldots,\Omega_r$
respectively $(\bigcup\limits^r_{i=1} \Omega_i \supset X)$ such that
$\u{\mathscr{G}^i_{\Omega_i \cap X}} \simeq \u{\mathscr{F}}_{\Omega_i
  \cap X}$. Let this isomorphism be given by a mapping $c_i :
\u{\mathscr{G}}^i_{\Omega_i \cap X} \to \u{\mathscr{F}}_{\Omega_i \cap
X}$. On $\Omega_i \cap \Omega_j \cap X$, there is thus a homomorphism
$c_{ij} : \u{\mathscr{G}}^i_{\Omega_i \cap \Omega_j \cap X} \to
\u{\mathscr{G}}^j_{\Omega_i \cap \Omega_j \cap X}$ (where $c_{ij} =
c^{-1}_j c_i$). Also $c_{ii} = $ identity, $c_{ij} c_{jk}  c_{ki} = $
identity on $\Omega_i \cap \Omega_j \cap \Omega_k \cap X$. By Lemma
\ref{chap12:lem2}, 
there is a neighbourhood $U_{ij}$ of $\Omega_i \cap \Omega_j \cap X$
(with $U_{ij} = U_{ji}$) and a homomorphism
$$
\gamma_{ij} : \u{\mathscr{G}}^i_{\Omega_i \cap \Omega_j \cap U_{ij}}
\to \u{\mathscr{G}}^j_{\Omega_i \cap \Omega_j \cap U_{ij}}.
$$
Also, by
the remark before Lemma \ref{chap12:lem1}, we may suppose that $U_{ij}$ is such that
$\gamma_{ii} = $ identity on $\Omega_i \cap U_{ii}$, $\gamma_{ij}
\gamma_{jk} \gamma_{ki} = $ identity on $\Omega_i \cap \Omega_j \cap
\Omega_k \cap U_{ij} \cap U_{jk} \cap U_{ki}$; in particular,
$\gamma_{ij}$ is an isomorphism of $\u{\mathscr{G}}^i_{\Omega_i \cap
  \Omega_j \cap U_{ij}}$ onto $\u{\mathscr{G}}^i \Omega_i \cap
\Omega_j \cap U_{ij}$.\pageoriginale Since $X$ is compact, there is a
neighbourhood $U$ of $X$ such that $\gamma_{ij}$ is an isomorphism of 
$\u{\mathscr{G}}^i_{\Omega_i \cap \Omega_j \cap U}$ onto
$u{\mathscr{G}}^j_{\Omega_i \cap \Omega_j \cap U}$, and $\gamma_{ii} =
$ identity, $\gamma_{ij} \gamma_{jk} \gamma_{ki} = $ identity (on
$\Omega_i \cap U$, $\Omega_i \cap \Omega_j \cap \Omega_k \cap U$
respectively). These sheaves and isomorphism give rise to a coherent
analytic sheaf $\u{\mathscr{G}}$ on a neighbourhood of $X$. Also, by
the definition of the $\gamma_{ij}$ it is clear that
$\u{\mathscr{G}}_X \simeq \u{\mathscr{F}}$ and the proof of Theorem
\ref{chap12:proofofthm2} is complete.  

If the manifold $V$ is paracompact, $X$ may be replaced by any closed
set in Theorem \ref{chap12:thm2}. For the details of proof, see Cartan
\cite{p3:key4} or Dowker \cite{p3:key6}.  
\end{proofoftheorem}

