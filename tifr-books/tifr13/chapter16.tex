\chapter{Coherent analytic sheaves on a Stein manifold}\label{chap16}

1. We shall\pageoriginale prove here the fundamental theorem of
Oka-Cartan-Serre on Stain manifolds.

\medskip
\noindent{\textbf{Fundamental Theorem.}} 
Let $V$ be a Stein manifold and $\u{\mathscr{F}}$ a coherent analytic
sheaf on $V$. Then 
\begin{itemize}
\item[A)] For every $a \in V$, $H^\circ (V, \u{\mathscr{F}})$
  $\mathscr{O}_a$-generates $\mathscr{F}_a$.

\item[B)] For $p \geq 1$, $H^p (V, \u{\mathscr{F}})=0$. 
\end{itemize}

(It is clear that for compact subsets of $V$, Theorem A) as formulated
in XV is equivalent to the theorem as formulated above).

The following two results will be required, the first will not be
proved here. For the proof, see Cartan \cite{p3:key1}. 

\setcounter{thm}{0}
\begin{thm}\label{chap16:thm1}
Let $V$ be a complex analytic manifold and let $a \in V$. Let
$\mathfrak{M}$ be a submodule of $\mathscr{O}^p_a$ \{as an
$\mathscr{O}_a$-module\} and let $f = (f_1, \ldots, f_p) \in
\mathscr{O}^p$ ($\mathscr{O} = \mathscr{H}_V$ is the space of all
holomorphic funtions on $V$). Suppose that $f$ is the limit in
$\mathscr{O}^p$ of functions $f_i \in \mathscr{O}^p$ such that
$(f_i)_a \in \mathfrak{m}$. Then $f_a \in\mathfrak{m}$. 
\end{thm}

\begin{lemma*}
Let $K$ be a compact subset of the Stein manifold $V$ such that $K =
\hat{K}$ ($\hat{K}$ is the $\mathscr{H}_V$-envelope of $K$) and
$\u{\mathscr{F}}$ a coherent analytic sheaf on $K$. Let $f_1, \ldots,
f_{m} \in H^\circ (K, \u{\mathscr{F}})$ and suppose that for every $a
\in K$, $f_1, \ldots, f_m$ $\mathscr{O}_a$-generate
$\mathscr{F}_a$. Then $f_1, \ldots, f_m$ $H^\circ
(K,\u{\mathscr{O}})$-generate $H^\circ(K,\break \u{\mathscr{F}})$. 
\end{lemma*}

This lemma is proved, using Theorems A) and B) for $K$ in exactly the
same way as was the lemma in the proof of Theorems A) and B) for a
cube in $XIV$. 

\setcounter{section}{1}
\section{Topology on \texorpdfstring{$H^\circ(V, \text{\underline{$\mathscr{F}$}})$}{HV}}%%%% 2 

Let\pageoriginale $\{K_p\}$ be a sequence of compact subsets of $V$
such that $K_p \subset \overset{\circ}{K}_{p+1}$,
$\bigcup\limits^\infty_1 K_p = V$ and $K_p = \hat{K}_p$ (such a
sequence exists since $V$ is eountable at infinity and, for any
compact set $K$, $(\hat{\hat{K}}) = \hat{K}$).

For an integer $N \geq 1$, we introduce a norm in $H^\circ (K_p,
\u{\mathscr{O}}^N)$ by setting the norm of $f = (f_1, \ldots, f_N) \in
H^\circ (K_p, \u{\mathscr{O}}^N)$ to the equal to the greatest of the
suprema of $|f_1|, \ldots, |f_N|$ on $K_p$. We then introduce a
seminorm $||\ldots ||_p$ on $H^\circ (K_p, \u{\mathscr{F}})$ as
follows: by Theorem A) for $K_p$, $\u{\mathscr{F}}_{K_p} \simeq
\u{\mathscr{O}}^N /\u{\mathscr{R}}$ and $|| \ldots ||_p$ is defined to
be the quotient seminorm of the norm on $\u{\mathscr{O}}^N$. It is
easy to verify that two isomorphisms $\u{\mathscr{F}}_{K_p} \simeq
\u{\mathscr{O}}^{N_1}/ \u{\mathscr{R}}_1 \simeq \u{\mathscr{O}}^{N_2}/
\u{\mathscr{R}}_2$ give rise to equivalent seminorms. Also, for every
$p$, there is a canonical mapping $H^\circ (V, \u{\mathscr{F}})$
$H^\circ(K_p, \u{\mathscr{F}})$ (namely, restriction to $K_p$). On
$H^\circ (V, \u{\mathscr{F}})$ we put the weakest topology for which
these mappings are continuous in these seminorms (which may also be
described by saying that $f \in H^\circ (V, \u{\mathscr{F}})$ tends to
zero if $||f||_p \to 0$ for every $p$). Also, it is easily seen that
the topology induced by $||\ldots ||_{p+1}$ on $H^\circ (K_p,
\u{\mathscr{F}})$ is finer than that given by $|| \ldots ||_p$. 

The next results will show that $H^\circ (V,\u{\mathscr{F}})$ is a
Fr\'echet space. One has only to show that it is Hausdorff and
complete.

(a) If $f_{p+1} \in H^\circ (K_{p+1}, \u{\mathscr{F}})$ and
$||f_{p+1}||_{p+1} =0$, then the restriction of $f_{p+1}$ to $K_p$ is
zero. (As a consequence, the topology of $H^\circ (V,
\u{\mathscr{F}})$ is Hausdorff).

\begin{proof}
If $\phi_1, \ldots, \phi_{N_{p+1}}$ $\mathscr{O}_a$-generate
$\mathscr{F}_a$ for $a \in K_{p+1}$ (Theorem A) for $K_{p+1}$) and 
$$
f_{p+1} = \sum\limits^{N_{p+1}}_{i=1} c_i \phi_i \quad (\mbox{lemma on
p.120})
$$
(the $c_i$\pageoriginale are holomorphic functions on $K_{p+1}$) it
follows from the definition of the seminorm $\|\ldots \|_{p+1}$ and
the fact that $\|f_{p+1} \|_{p+1} = 0$ that, given $\epsilon >0$ there are
holomorphic functions $c^\epsilon_1, \ldots, c^\epsilon_{N_{p+1}}$ on $K_{p+1}$
such that 
$$
f_{p+1} = \sum\limits^{N_{p+1}}_{i=1} c^\epsilon_{i} \phi_i 
$$
and $\sup\limits_{i, a \epsilon K_{p+1}}$ $|c_i(a)| < \epsilon$. If
$\gamma^\epsilon_{i} = c_i - c^\epsilon_i$, then $(c_1, \ldots, c_{N_{p+1}})$ is
uniformly approximated on $K_{p+1}$ by $(\gamma^\epsilon_1, \ldots,
\gamma^\epsilon_{N_{p+1}})$ where $(\gamma^\epsilon_1, \ldots,
\gamma^\epsilon_{N_{p+1}})$ is an element of the sheaf $\u{\mathscr{R}}$ of
relations between $\phi_1, \ldots, \phi_{N_{p+1}}$. It follows from
Theorem \ref{chap16:thm1} (stated on page 120) that on $\overset{\circ}{K}_{p+1}
\supset K_p  (c_1, \ldots, c_{N_{p+1}}) \in \u{\mathscr{R}}$ and (a)
is proved. 

(b) If $f_1, \ldots , f_, \ldots$ is a sequence of elements of
$H^\circ (K_{p+1}, \u{\mathscr{F}})$ such that
$$
\sum\limits^\infty_{k=1} || f_k ||_{p+1} < + \infty, 
$$
then the sequence $\{\sum\limits^N_{k=1} f_k\}$ has a limit point in
$H^\circ (K_p, \u{\mathscr{F}})$. The restrictions to $K_{p-1}$ of two
such limit points coincide.
\end{proof}

\begin{proof}
Let $\phi_1, \ldots, \phi_{N_{p+1}}$ $\mathscr{O}_a$-generate
$\mathscr{F}_a$ for $a \in K_{p+1}$ and let 
$$
f_k = \sum\limits^{N_{p+1}}_{i=1} c^{(k)}_i \phi_i. 
$$
Then, since $\sum\limits^\infty_{k=1} ||f_k||_{p+1} < + \infty$, the
$c^{(k)}_i$ can be so chosen that $\sum\limits_k\max\limits_i\break
||c_i^{(k)} ||_{K_{p+1}} < + \infty$ (by the definition of $|| \ldots
||_{p+1}$, one may, for example, take the $c^{(k)}_i$ such that
$||c^{(k)}_i||_{K_{p+1}} < 2 ||f_k||_{p+1}$). Then, for each $i$,
$\sum_k c^{(k)}_i$  converges to a holomorphic function $c_i$ on
$K_p$, and it is clear that $|| \sum\limits^{N}_{k=1} f_k -
\sum\limits^{N_{p+1}}_{i=1} c_i \phi_i||_p \to 0$ as $N \to
\infty$. This proves the existence of the limit point. The uniqueness
on $K_{p-1}$ follows at once from a). 

(c) $H^\circ (V, \u{\mathscr{F}})$\pageoriginale is a Fr\'echet space.

Given a Cauchy sequence $\{s_k\}$, $||s_k - s_l||_p \to 0$ as $k$, $l
\to \infty$ for every $p$. If we choose a sequence $\{n_k\}$ of
integers such that $||s_m - s_{n_k}||_p < 1/2^k$ for $p \leq k$ and $m
\geq n_k$ (with $n_{k+1} > b_k$) then $\sum\limits_k ||s_{n_{k+1}} -
s_{n_k}||_p < + \infty$ for every $p$. It follows at once from (b)
that $\{s_k\}$ has a limit in $H^\circ (V, \u{\mathscr{F}})$ which is
unique since $H^\circ (V, \u{\mathscr{F}})$ is Hausdorff. 

(d) (Approximation property). Given $f_p \in H^\circ (K_p,
\u{\mathscr{F}})$ and $ \epsilon >0$ there is a section $f \in H^\circ (V,
\u{\mathscr{F}})$ such that $||f_p - f||_p < \epsilon$. 
\end{proof}

\begin{proof}
If $\phi_1, \ldots, \phi_{N_{p+1}} \mathscr{O}_a$-generate
$\mathscr{F}_a$ for $a \in K_{p+1}$, then their restrictions to $K_p$
clearly $\mathscr{O}_a$-generate $\mathscr{F}_a$ for $a \in
K_p$. Hence, by the lemma,
$$
f_p = \sum\limits^{N_{p+1}}_{i=1} c_i \phi_i
$$
where the $c_i$ are holomorphic on $K_p$. By the theorem of XV, the
$c_i$ can be approximated uniformly on $K_p$ by holomorphic functions
on $V$. This shows that $f_p$ can be approximated on $K_p$ (in $||
\ldots ||_p$) by a section $f_{p+1} \in H^\circ (K_{p+1},
\u{\mathscr{F}})$. Approximating $f_{p+1}$ on $K_{p+1}$ by $f_{p+2}
\in H^\circ (K_{p+2}, \u{\mathscr{F}})$ in $||\ldots ||_{p+1}$ and
continuing this process, we construct a sequence $f_{p+1}, f_{p+2},
\ldots$ such that $||f_{p+k+1} - f_{p+k}||_{p+k} \leq \epsilon_k$. If the
$\epsilon_k$ are small enough, it is seen that
$\sum\limits^\infty_{m=k}||f_{p+m+1} - f_{p+m}||_{p+m} < + \infty$
(since $||\ldots ||_{m+1}$ is finer than $||\ldots ||_m$ on $K_m$) and
so $f'_{p+k} = f_{p+k} + \sum\limits^\infty_{k}$ $(f_{p+m+1} -
f_{p+m})$ is defined uniquely in $K_{p+k-2}$. It is clear that
$f'_{p+k+1} = f'_{p+k}$ on $K_{p+k-2}$ and so there is an $f$ $H^\circ
(V, \u{\mathscr{F}})$ with $f = f'_{p+k}$ on $K_{p+k-2}$. If the
$\epsilon_k$ are small enough, $f$ approximates to $f_p$ in $|| \ldots
||_p$. 

[If we\pageoriginale say that a sequence $\{s_m\}$, where $s_m \in
  H^\circ (K_m, \u{\mathscr{F}})$, converges to $s \in H^\circ (
V, \u{\mathscr{F}})$ if $|| s- s_m||_m \to 0$ as $m \to \infty$, the
above proof may be rephrased by saying simply that $f_{p+m} \to f$ as
$m \to \infty$.].
\end{proof}

\section{Proof of the fundamental theorem}\label{chap16:sec3}

\medskip
\noindent{\textbf{Proof of Theorem A):}}

Let $a \in V$ and suppose that $a \in \overset{\circ}{K}_p$. Let
$\phi_1, \ldots , \phi_{N_p}$ $\mathscr{O}_b$-generate $\mathscr{F}_b$
for $b \in K_p$. Theorem A) asserts that the set of the $N_p$-tuples
$(a_1, \ldots,\break a_{N_p})$, where $a_i \in \mathscr{O}_a$, such that
$\sum_i \phi$ belongs to the submodule of $\mathscr{F}_a$ generated
by $H^\circ (V, \u{\mathscr{F}})$ is $\mathscr{O}^{N_p}_a$. Such
$N_p$-tuples from a submodule $\mathfrak{m}$ (over $\mathscr{O}_a$) of
$\mathscr{O}^{N_p}_a$ and after Theorem \ref{chap16:thm1}, we have
only to prove that 
on a certain fixed open nieghbourhood $U$ of a, every $N_p$-tuple of
holomorphic functions on $U$ is the uniform limit of $N_p$-tuples
$(b_1, \ldots, b_{N_p})$ such that $\sum\limits^{N_p}_1 b_i \phi_i$
induces at a an element of $\mathscr{F}_a$. But this follows at once
from the approximation property.

\medskip
\noindent{\textbf{Proof of Theorem B):}} We prove first that $H^p(V,
\mathscr{F})=0$ if $p>1$.  Let $\alpha$ be a $p$-cocycle of $V$. On
$K_m$ we have $\alpha = \delta \beta_m$, $\beta_m$ a cochain of $K_m$,
by Theorem B) for $K_m (m = 1,2, \ldots)$. Also, on $K_m$, $\delta
(\beta_{m+1} - \beta_m) = 0$, so that $\beta_{m+1} - \beta_m = \delta
\gamma'_m$ where $\gamma'_m$ is a $(p-2)$-cochain of $K_m$. By the
definition of cochain, we may suppose that $\gamma'_m$ is the
restriction to $K_m$ of a $(p-2)$-cochain $\gamma_m$ of $V$, and so
$\beta_m = \beta_{m+1} - \delta \gamma_m$ on $K_{m}$, while $\delta
(\beta_{m+1} - \delta \gamma_m) = \alpha$ on $K_{m+1}$. It is clear
that by repeating this process with $m = 1,2, \ldots$ we obtain a
$(p-1)$-cochain $\beta$ of $V$ such that $\delta \beta = \alpha$ and
so $H^p (V, \u{\mathscr{F}}) =0$. 

Finally, we turn to the proof that $H^1(V, \u{\mathscr{F}}) = 0$. Let
$\alpha$ be a 1-cocycle of $V$ and let $\alpha = \delta \beta'$ on
$K_p$, where $\beta'_p$ is a 0-cochain of $K_p$. Again, $\beta'_{p+1}
- \beta'_{p} \in H^\circ (K_p , \u{\mathscr{F}})$, i.e., is a
cocycle. 

By the\pageoriginale approximation property, there is a cocycle
$c'_{p+1} \in H^\circ (V, \u{\mathscr{F}})$ such that $||c'_{p+1} +
\beta'_{p+1} - \beta'_{p}||_p \leq \epsilon_p$, where $\epsilon_p$ can be chosen
arbitrarily small. It is clear that we find thus a cochain $\beta_p$
of $K_p$ with $\delta \beta_p = \alpha$ on $K_p$ and $||\beta_{p+1} -
\beta_p||_{p} \leq \epsilon_p$ for every  $p \geq 1$. 

If we say that a sequence of cochains, $\{\beta_p\}$, where $\beta_p$
is a cochain of $K_p$, tends to a cochain $\beta$ of $V$ if $\beta -
\beta_p \in H^\circ (K_p, \u{\mathscr{F}})$ and $||\beta - \beta_p||_p
\to 0$ as $ p \to \infty$, a repetition of the proof of the
approximation property (with trivial modifications) shows that the
sequence $\{\beta_p\}$ defined above tends to a cochain $\beta$ of $V$
if the $\epsilon_p$ are small enough and it is clear that $\delta \beta =
\alpha$. This proves Theorem B) for $p=1$ and the proof of the
fundamental theorem is complete. 



\begin{thebibliography}{99}
\bibitem{p3:key1} H. Cartan:\pageoriginale Id\'eaux de fonctions
  analytiquesde $n$ variables complexes, \textit{Annales de
    l'E.N.S.} $3^e$ serie, 61 (1944), 149-197. 

\bibitem{p3:key2} H. Cartan: Id\'eaux et modules de fonctions analytiques
  de variables complexes, \textit{Bull. Soc. Math. de France, 78
    (1950) 29-64.}

\bibitem{p3:key3} H. Cartan: \textit{S\'eminaire E.N.S.} 1951/52
  (especially Lectures XV-XIX). 

\bibitem{p3:key4} H. Cartan: Vari\'et\'es analytiques r\'eelles et
  vari\'et\'es analytiques complexes, \textit{Bull. Soc. Math. de
    Frace} 85 (1957), 77-99.

\bibitem{p3:key5} J. Dieudonn\'e: Une g\'en\'eralisation des espaces
  compacts, \textit{Jour. Math. Pure et Appl}. (9) 23 (1944), 65-76.

\bibitem{p3:key6} C. H. Dowker: \textit{Lectures on Sheaf Theory}, Tata
  Institute of Fundamental Research, Bombay, 1956.

\bibitem{p3:key7} K. Oka: Sur les fonctions analytiques des plusieurs
  variables VII. Sur quelques notions arithm\'etiques,
  \textit{Bull. Soc. Math. de France,} 78 (1950), 1-27.

\bibitem{p3:key8} J-P. Serre: Faisceaux alg\'ebriques coh\'erents,
  \textit{Annals of Math}, 61 (1955), 197-278.

\end{thebibliography}


\begin{center}
\textbf{Supplementary References}
\end{center}

In the\pageoriginale papers listed below, the reader will find
applications of the theorems proved in these lectures and several
important results that could not be treated here. For further
references, see the Scientific report on the second summer Institute:
Several Complex Variables, by W. T. Martin, S. S. Chern and
O. Zariski, Bull. Amer. Math. Soc., 62 (1956), 79-141.


\begin{thebibliography}{99}
\bibitem{key1} H. J. Bremermann - \"Uber die \"Aquivalenz der
  pseudo-konvexen Gebiete und der Holomorphiegebiete in Raum von $n$
  komplexen Var\"anderlichen, Math. Ann. 128 (1954), 63-91.

\bibitem{key2} H. Cartan - S\'eminaire E.N.S. 1951/52 (Lecture XX).

\bibitem{key3} H. Cartan - S\'eminaire E.N.S. 1953/54 (Lecture XVII). 

\bibitem{key4} H. Cartan - Espaces fibr\'es analytiques,
  vol. consacr\'e an Symposium internat. de Mexico de 1956.

\bibitem{key5} J. Frenkel - Cohomologie non ab\'elienne et espaces
  fibr\'es. Bull. Soc. Math. de France, 85 (1957), 135-220.

\bibitem{key6} H. Grauert - Charakterisierung der holomorph
  vollst\"andigen komplexen R\"aume, Math. Ann. 129 (1955), 233-259.

\bibitem{key7} H. Grauert - Approximationss\"atze f\"ur holomorphe
  Funktionen mit Werten in komplexen R\"aumen, Math. Ann. 133 (1957),
  139-159. 

\bibitem{key8} H. Grauert - Holomorphe Funktionen mit Werten in
  komplexen Lieschen Gruppen, Math. Ann. 133 (1957) 450-472. 

\bibitem{key9} H. Grauert - Analytische Faserungen \"uber
  holomorph-vollstandigen R\"auman, To appear in Math. Ann. 1958.

\bibitem{key10} F. Hirzebruch\pageoriginale - Neue topologische
  Methoden in der algebraischen Geometrie, Erg. d. Math. Springer,
  1956. 
 
\bibitem{key11} F. Norguet - Sur les domaines d'holomorphie des
  fonctions uniformes de plusieurs variables complexes. (Passage du
  local au global.) Bull. Soc. Math. de France, 82 (1954), 137-159. 

\bibitem{key12} K. Oka - Sur les fonctions analytiques de plusieurs
  variables. VI Domaines pseudoconvexes, T\^ohoku Math. J. 49 (1942),
  15-52. 

\bibitem{key13} K. Oka - Sur les fonctions analytiques de plusieurs
  variables. IX Domaines fini sans point critique int\'erieur,
  Jap. J. Math. 23 (1953/54), 97-155. 

\bibitem{key14} R. Remmart - Sur les espaces analytiques
  holomorphiquements\'e parables et holomorphiquement convexes,
  C.R.Acad. Sci. Paris, 243 (1956), 118-121.

\bibitem{key15} J-P. Serre - Quelques probl\`emes globaux relatifs aux
  vari\'et\'esde Stein, Colloque de Bruxelles sur les fonctions de
  plusieurs variables, (1953), 57-68.

\bibitem{key16} K. Stein - \"Uberlagerung holomorph-vollst\"andiger
  komplexer R\"aume, Arch. der Math. 7 (1956), 354-361. 

\end{thebibliography}
