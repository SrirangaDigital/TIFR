
\chapter{Analytic Continuation}\label{chap4}

All\pageoriginale the manifolds considered in this and the next
lecture are assumed to be connected. 

\begin{defi*}
Let $V$, $W$ be two complex analytic manifolds of complex dimension
$n$, $\phi$ a mapping of $V \to W$. We say that $\phi$ is  a
\textit{local analytic isomorphism} if every point $a \in V$ has an
open neighbourhood $\mathscr{O}$ such that $\phi$ restricted to
$\mathscr{O}$ is an analytic isomorphism.
\end{defi*}

We say that $V$ is spread in $W$ and that $\phi$ spreads $V$ in $W$ if
$\phi$ is a local analytic isomorphism of $V$ in $W$.

One may define the continuation of a holomorphic function $f$ on $V$,
to the manifold $W$ in which $V$ is spread by $\phi$ by saying that
$g$ is the continuation of $f$ to $W$ if $f = g \circ \phi$. If such a
$g$ exists, it is unique, for if $g' \circ \phi = f = g\circ \phi$,
then since $\phi$ is a local homemorphism, $g = g'$ in an open set on
$W$, and $W$ being connected, $g = g'$ on $W$. 

\begin{example*}
\begin{itemize}
\item[(i)] $V$ is an open set $\subset W$, $\phi$ is the inclusion
  map $\phi (a) = a$ for every $a \in V$. The functions in $V$ which
  can be continued to $W$ are precisely the restrictions to $V$ of
  holomorphic functions on $W$.

\item[(ii)] $V$ is a convering space of $W$ and $\phi$ the natural
  projection. The functions on $V$ which can be continued to $W$ are
  those functions which have the same value at all points which lie
  over one point of $W$.
\end{itemize}

However, this definition of continuation turns out to be too general
to be of use. To have interesting theorems, it is necessary to
restrict the\pageoriginale definition, and we introduce therefore the
following
\end{example*}

\begin{defi*}
Let $V$ be a complex analytic manifold, $\phi$ a map which\break  spreads
$V$ in $C^n$. (The necessary and sufficient condition that such a
$\phi$ exist is that there are $n$ global functions on $V$ which form
a system of local coordinates at each point of $V$). Let $(V', \phi')$
be another such pair, $\phi'$ spreading $V'$ in $C^n$. Suppose also
that there exists a map $\psi : V \to V'$ which spreads $V$ in $V'$,
such that $\phi = \phi' \circ \psi$ ($\phi, \phi', \psi$ are assumed
to be given and fixed). Let $f$ be a holomorphic function on $V$. We
say that $f'$ is the continuation of $f$ from $V$ to $V'$ if $f'$ is a
holomorphic function on $V'$ such that $f=f' \circ \psi$.
\end{defi*}

\medskip
\noindent{\textbf{Maximal continuation.}}

Let $(V,\phi)$ be a pair consisting of the ($n$ dimensional) complex
analytic manifold $V$ and a spread $\phi$ of $V$ in $C^n$. Let $f$ be
a holomorphic function on $V$. Suppose that there exists another such
pair $(\tilde{V}, \tilde{\phi})$ with the following properties:
\begin{itemize}
\item[(i)] $f$ can be continued to $(\tilde{V}, \tilde{\phi})$, i.e.,
  there exists a holomorphic function $\tilde{f}$ on $\tilde{V}$ and a
  spread $\tilde{\psi}$ of $V$ in $\tilde{V}$ such that $f = \tilde{f}
  \circ \tilde{\psi}$, $\phi = \tilde{\phi} \circ \tilde{\psi}$.

\item[(ii)] If $f'$ is a continuation of $f$ to the pair $(V', \phi')$ and
  $\psi$ is the spread of $V$ in  $V'$ such that $f = f' \circ \psi$,
  $\phi = \phi' \circ \psi$, then there exists a spread $\chi$ of $V'$
  in $\tilde{V}$ such that $f' = \tilde{f} \circ \chi$ and such that
  the mapping $\tilde{\psi} : V \to \tilde{V}$ factorises into
  $\tilde{\psi} = \chi \circ \psi$. Then we can show that we have also
  $\phi' = \tilde{\phi} \circ \psi$ and we call $(\tilde{V},
  \tilde{\phi}, \tilde{\psi}, \tilde{f})$ a \textit{maximal
    continuation} of $(V, \phi, f)$.
\end{itemize}

To consider\pageoriginale the problem of the existence and uniquencess
of a maximal continuation, we shall have to introduce the so-called
\textit{sheaf of germs of holomorphic functions}.


Let $W$ be a complex analytic manifold (of complex dimension $n$). Let
$a \in W$. Consider the set of all holomorphic functions in open sets
containing the point $a$. We introduce an equivalence relation in this
set of functions by identifying two functions $f$, $g$, if, in a
neighbourhood of $a$, $f=g$. The equivalence classes are called germs
of \textit{holomorphic functions} at $a$. $f_a$ will stand for a germ
at $a$. It is clear that $f_a(a)$ has an unambiguous meaing.

Denote by $\mathscr{O}_a$ the set of germs at $a$. \textit{The sheaf
  of germs of holomorphic functions on $W$} is defined to be
$\u{\mathscr{O}}_W = \u{\mathscr{O}} = \bigcup\limits_{a \in W}
\mathscr{O}_a$. A complex analytic structure can be put on
$\u{\mathscr{O}}$ in the following way. 

Let $a \in W$ and let $(a, f_a) \in\mathscr{O}_a \cdot f_a$ is defined
by a holomorphic function $f$ in a neighbourhood $U$ of $a$. Also, for
every $b \in U$, $f$ defines a germ $f_b$ at $b$. We define
$\bigcup\limits_{b \in U} (b, f_b)$ to be a neighbourhood of $a$. It
is easy to verify that this defines a topology on $\u{\mathscr{O}}$.

\setcounter{proposition}{0}
\begin{proposition}\label{chap4:prop1}
$\u{\mathscr{O}}$ is a Hausdorff space.
\end{proposition}

\begin{proof}
Let $(a, f_a) \neq (b, g_b)$ be two points of $\mathscr{O}$.

If $a \neq b$ we choose neighbourhoods $U_a$, $V_b$ of $a$, $b$ in $W$
such that $f_a$ is determined by $f$ in $U_a$, $g_b$ by $g$ in $V_b$
and $U_a \cap V_b = 0$. Then $\bigcup\limits_{c \in U_a} (c , f_c)$,
$\bigcup\limits_{d\in V_b} (d, g_d)$ are disjoint neighbourhoods of
$(a, f_a)$, $(b, g_b)$ in $\u{\mathscr{O}}$. 
\end{proof}

If $a = b$,\pageoriginale then $f_a \neq g_a$. Let $U$ be a connected
neighbourhood of $a$ on $W$ such that $f_a$, $g_a$ are defined by
holomorphic functions $f$, $g$ in $U$. Then the neighbourhoods
$\bigcup\limits_{c \in U} (c, f_c)$, $\bigcup\limits_{c \in U}(c,
g_c)$ of $(a, f_a)$ $(a, g_a)$ are disjoint, for if $(c, f_c) = (c,
g_c)$, then $f$, $g$ coincide in a neighbourhood of $c$, and since $U$
is connected, $f = g$ in $U$ so that $f_a = g_a$ which is not the
case. 

Let $p$ be the projection $\u{\mathscr{O}} \to W$ defined by $p(a,
f_a) =a$. This is a mapping of $\u{\mathscr{O}}$ onto $W$. It is a
local homeormorphism as follows at once from the definition of the
topology on $\u{\mathscr{O}}$. It is clear now, how $p$ can be used to
carry over the complex analytic structure from $W$ to
$\u{\mathscr{O}}$. 

Let now $V$ be an $n$ dimensional complex analytic manifold, $\phi$ a
spread of $V$ in $C^n$. Let $\u{\mathscr{O}}$ be the sheaf of germs of
holomorphic functions on $C^n$. Let $f$ be a holomorphic function on
$V$. Let $a \in V$ and $U$ a neighbourhood of a such that $\phi$,
restricted to $U$ is an analytic isomorphism. Then the holomorphic
function $f \circ \phi^{-1}$ in $\phi (U)$ defines a germ in
$\u{\mathscr{O}}$, viz., $(f \circ \phi^{-1})_{\phi (a)}$. We define a
mapping $\tilde{\psi} : V \to \u{\mathscr{O}}$ by setting
$\tilde{\psi}(a) = (\phi(a), (f \circ \phi^{-1})_{\phi (a)}) \in
\u{\mathscr{O}}$. $\tilde{\psi}$ is a local analytic isomorphism which
spreads $(V, \phi)$ in $\u{\mathscr{O}}$.

Let $\tilde{V}$ be the connected component in $\u{\mathscr{O}}$ of
$\phi(V)$, and define $\tilde{\phi}$ to be the restriction to
$\tilde{V}$ of the projection $p : \u{\mathscr{O}} \to C^n$ and define
$\tilde{f}(b, g_b)= g_b (b) $ for $(b, g_b) \in V$. Clearly $\phi =
\tilde{\phi} \circ \psi$ and $f =\tilde{f} \circ \psi$. Hence
$(\tilde{V}, \tilde{\phi}, \tilde{\psi}, \tilde{f})$ is a continuation
of $(V, \phi, f)$. We assert that it is a maximal
continuation. Suppose that $(V', \phi', f')$ is any continuation of
$(V, \phi, f)$. Let $\psi$ be a local analytic isomorphism $V \to V'$
such that $f= f' \circ \psi$, $\phi = \phi' \circ \psi$. We can apply
the above reasoning to\pageoriginale $(V', \phi', f')$ to continue it
to $(\tilde{V}', \tilde{\phi}', \tilde{f}')$. The point $\psi(a) \in
V'(a \in V)$ is mapped onto $(\phi' (\psi(a)), (f' \circ
\phi'^{-1})_{\phi' (\psi(a))})$ and this $ = (\phi (a), (f \circ
\phi^{-1})_{\phi (a)})$. Hence the images of $V$ and $V'$ have a
common point in $\u{\mathscr{O}}$ and by the definition of
$\tilde{V}$, $\tilde{V}'$ we have $\tilde{V} = \tilde{V}'$. It follows
easily that $(\tilde{V}, \tilde{\phi}, \tilde{\psi}, \tilde{f})$ is
maximal.

If now $(\tilde{V}, \tilde{\phi}, \tilde{f})$ and $(\tilde{V}',
\tilde{\phi}' , \tilde{f}')$ are both maximal continuations of
$(V,\phi, f)$, $\tilde{\psi} : V \to \tilde{V}$, $\tilde{\psi}' : V
\to \tilde{V}'$ the corresponding spreads of $V$ in $\tilde{V}$,
$\tilde{V}'$ respectively, then there exists a spread $\chi$ of
$\tilde{V}$ in $\tilde{V}'$ such that $\tilde{f} = \tilde{f}' \circ
\chi$, $\tilde{\psi}' = \chi \circ \tilde{\psi}$, $\tilde{\phi} =
\tilde{\phi}' \circ \chi$. Using the similar spread $\chi_1$ of
$\tilde{V}'$ in $\tilde{V}$ it is easily shown that $\chi$ is an
analytic isomorphism. Thus we have the following

\begin{theorem*}
Let $V$ be a complex analytic manifold, $\phi$ a spread of $V$ in
$C^n$. Let $f$ be a holomorphic function on $V$. Then there exists a
maximal continuation $(\tilde{V}, \tilde{\phi}, \tilde{\psi},
\tilde{f})$ of $(V, \phi, f)$. If $(\tilde{V}, \tilde{\phi},
\tilde{\psi}, \tilde{f})$ and $(\tilde{V}', \tilde{\phi}',
\tilde{\psi}', \tilde{f})$ are two maximal continuations, then there
exists an analytic isomorphism $\chi: V \to V'$ such that $\tilde{f} =
\tilde{f}' \circ \chi$, $\tilde{\psi}= \chi \circ \tilde{\psi}$,
$\tilde{\phi} = \tilde{\phi}' \circ \chi$.
\end{theorem*}
