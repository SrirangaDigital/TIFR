
\chapter{Coherent analytic sheaves on a cube}\label{chap14}

\section{The abstract de Rham Theorem}\label{chap14:sec1}

Let\pageoriginale $X$ be a paracompact topological space,
$\u{\mathscr{F}}$ a sheaf of abelian groups on $X$, and suppose that 
$$
0 \to \u{\mathscr{F}} \xrightarrow{i} \u{\mathscr{G}_\circ}
\xrightarrow{d_\circ} \u{\mathscr{G}_1} \xrightarrow{d_1} \ldots
\xrightarrow{d_{k-1}} \u{\mathscr{G}_k} \xrightarrow{d_k} \ldots 
$$
is an exact sequence of sheaves on $X$ and that $H^p(X,
\u{\mathscr{G}}_k) = 0$ for $p \geq 1$, $k \geq 0$. Consider the
sequence
$$
0 \to \Gamma (X, \u{\mathscr{F}}) \xrightarrow{i^\ast} \Gamma (X,
\u{\mathscr{G}}_\circ)  \xrightarrow{d^\ast_\circ} \ldots
\xrightarrow{d^\ast_{k-1}} \Gamma (X, \u{\mathscr{G}}_k)
\xrightarrow{d^\ast_k} \ldots 
$$ 
with the induced homomorphisms $d^\ast_k$ (this is not in general
exact). 

Then 
$$
H^k (X, \u{\mathscr{F}}) \simeq \text{ kernel } d^\ast_k/ \text{ image
} d^\ast_{k-1} \text{ for } k \geq 1. 
$$

\begin{proof}
Consider the exact sequence
$$
\u{\mathscr{G}}_{k-1} \xrightarrow{d_{k-1}} \u{\mathscr{G}}_k
\xrightarrow{d_k} \u{\mathscr{G}}_{k+1}
$$
and let $\u{\mathscr{H}}_k = $ kernel $d_k = $ image $d_{k-1}$. Then
we have an exact sequence 
$$
0 \to \u{\mathscr{H}}_k \to \u{\mathscr{G}}_k \to
\u{\mathscr{H}}_{k+1} \to 0
$$
and $X$ being paracompact, we obtain, for $q > 0$, the exact sequence
$$
H^q (X, \u{\mathscr{G}}_k) \to H^q (X, \u{\mathscr{H}}_{k+1}) \to
H^{q+1} (X , \u{\mathscr{H}}_k) \to H^{q+1}(X, \u{\mathscr{G}}_k) 
$$
and since $H^p (X, \u{\mathscr{G}}_k) = 0$, if $ p \geq 1$,
$$
H^q (X, \u{\mathscr{H}}_{k+1}) \simeq H^{q+1} (X, \u{\mathscr{H}_k})
$$
for $q \geq 1$. By iteration
\begin{equation*}
H^p(X, \u{\mathscr{F}}) \simeq H^{p-1} (X, \u{\mathscr{H}}_1) \simeq
\ldots \simeq H^1(X, \u{\mathscr{H}}_{p-1}).  \tag{1}
\end{equation*}
Also we have the exact sequence
$$
0 \to \u{\mathscr{H}}_{p-1} \to \u{\mathscr{G}}_{p-1} \to
\u{\mathscr{H}}_p \to 0
$$
and the\pageoriginale induced exact sequence
$$
H^\circ (X, \mathscr{G}_{p-1}) \to H^\circ (X, \u{\mathscr{H}}_p) \to
H^1 (X, \u{\mathscr{H}}_{p-1}) \to H^1 (X, \u{\mathscr{G}}_{p-1}). 
$$
Since the last term is $0$ by hypothesis,
$$
H^1 (X, \bar{H}_{p-1}) \simeq H^\circ (X, \u{\mathscr{H}}_p)/ \text{
  image } H^\circ (X, \u{\mathscr{G}}_{p-1}). 
$$
It is easy to see that $H^\circ (X,\u{\mathscr{H}}_p) \simeq $ kernel
$d^\ast_k$, while image $H^\circ (X, \mathscr{G}_{p-1}) = $ image
$d^\ast_{k-1}$ and the result follows from (1).
\end{proof}

\medskip
\noindent{\textbf{Applications.}}
\begin{itemize}
\item[a)] \textbf{de Rham's Theorem.} Let $X = V$ be a paracompact
  differentiable manifold and $\u{\mathscr{E}}^p$ the sheaf of germs
  of differentiable $p$-forms on $V$. Then we have a sequence
$$
0 \to \u{C} \xrightarrow{i} \u{\mathscr{E}}^\circ \xrightarrow{d}
\cdots \xrightarrow{d} \u{\mathscr{E}}^p \xrightarrow{d}
\u{\mathscr{E}}^{p+1} \xrightarrow{d} \cdots
$$
($C$ is a constant sheaf, $C$ being the group of complex
numbers). This sequence is exact by the local form of Poincar\'e's
theorem (VIII). By the method given in Step 1 of the solution of the
generalized first Cousin problem (VIII) it is seen that $H^k (V,
\u{\mathscr{E}}^p) = 0$ if $k \geq 1$, $p \geq 0$. Hence, the above
theorem shows that 
$$
H^p(V,\u{C}) \simeq$$ 
(group of closed
$p$-forms)/(the differentials of $(p-1)$ forms),
i.e., $H^p(V,C)$ is the same as the $p$-dimensional $d$-cohomology of
 $V$. This is the theorem of de Rham. 

\item[b)] \textbf{Dolbeault's theorem.} Let $V$: a paracompact complex
  manifold and $\u{\mathscr{E}}^{o,p}$ the sheaf of germs of forms of
  type $(0,p)$ on $V$. Then, the local form of Grothendieck's theorem
  shows that the sequence
$$
0 \to \u{\mathscr{O}} \xrightarrow{i} \u{\mathscr{E}}^{o,o}
\xrightarrow{d''} \u{\mathscr{E}}^{0,1} \xrightarrow{d''} \ldots
$$
is exact ($\u{\mathscr{O}}$ is the sheaf of germs of holomorphic
functions on $V$). Again, by the\pageoriginale method of the
generalized first Cousin problem, we prove that 
$$
H^k (V, \u{\mathscr{E}}^{o,p}) = 0, \quad p \geq 0, \quad k \geq 1,
$$
and we obtain 
$H^p(V, \u{\mathscr{O}}) \simeq$ ($d''$-closed $(0,p)$
forms)/($d''$-differenti\-als of $(0,p-1)$ forms),
i.e., $H^p (V,\u{\mathscr{O}})$ is the same as the $p$ - dimensional
$d''$-cohomology of the $(0,r)$ forms, $r=0$, $1, \ldots$ on $V$. 

Similar reasoning proves that the $p$-th cohomology of $V$ with
coefficients in the sheaf of germs of holomorphic $(q,0)$-forms is the
$p$-th $d''$-cohomology of the $(q,r)$-forms, $r=0, 1, \ldots$ on
$V$. This is a particular case of Dolbeault's theorem.

\item[c)] Let $K$ be a cube imbedded in $C^n$, i.e., $K$ is a subset of a
  fixed $C^n$, consisting of the points $z \in C^n$ with 
$$
|\mathscr{R} z_i| \leq a_i |\mathfrak{I} z_i| \leq b_i, \;  a_i,
\; b_i \geq 0.
$$
Consider the sheaf $\u{\mathscr{O}} = \u{\mathscr{O}}_K$ (in the sense
of XII, i.e., the restriction to $K$ of the sheaf of germs of
holomorphic functions in $C^n$). We define the sheaves
$\u{\mathscr{E}}^{o,p}$ in the same way as the restriction to $K$ of
the sheaf of germs of $(0,p)$-forms in $C^n$. As before we have the
exact sequence 
$$
0 \to \u{\mathscr{O}} \xrightarrow{i} \u{\mathscr{E}}^{o,o}
\xrightarrow{d''} \u{\mathscr{E}}^{o,1} \xrightarrow{d''} \ldots 
$$
and
{$H^p (K,\u{\mathscr{O}}) \simeq $ ($d''$-closed
  $(0,p)$-forms)/($d''$-differentials of $(0,\break p-1)$-forms).}
Grothendieck's theorem shows that this quotient is zero. (The theorem
was proved only when $a_i$, $b_i > 0$, but it is immediate that the
theorem holds also when some of the $a_i$, $b_i$ are $0$). This proves
the following 
\end{itemize}

\setcounter{thm}{0}
\begin{thm}\label{chap14:thm1}
If $K$ is a closed cube in $C^n$, $H^p(K,\u{\mathscr{O}}) =0$ for $p
\geq 1$. 
\end{thm}

\begin{coro*}
If\pageoriginale $p, q \geq 1$, $H^p (K, \u{\mathscr{O}}^q) = 0$.  

(One has only to apply Theorem 1 to the components of an element $c$
of $Z^p (K, \u{\mathscr{O}}^q)$ to see that $c$ is a coboundary).
\end{coro*}

It is instructive to compare the above proof with the solution of the
generalized first Cousin problem for a cube which implies that $H^1(K,\break
\u{\mathscr{O}}) = 0$. (This is exactly the generalization of that
proof to the more general setting here).


\section{Coherent analytic sheaves on a cube}\label{chap14:sec2}
Let $K$ be a cube in $C^n$ and $\u{\mathscr{F}}$ a coherent analytic
sheaf on $K$. \textit{Fundamental Theorem. (Oka-Cartan-Serre).}

A) \textbf{$\mathscr{F}$ is (globally) a quotient of a sheaf
  $\u{\mathscr{O}}^N$} (This can also be expressed by saying that
there are $N$ (global) sections $f_1, \ldots, f_N$ of
$\u{\mathscr{F}}$ which $\mathscr{O}_b$-generate $\mathscr{F}_b$ for
every $b \in K$.)  

B) For $p \geq 1$, $H^p (K, \u{\mathscr{F}}) = 0$. 

We introduce the following statement.

$A')$ $\mathscr{F}$ is (globally) a quotient of a coherent analytic
sheaf locally isomorphic to $\u{\mathscr{O}}^N$. 

The proof of the fundamental theorem now divides into two parts, the
proof that $A')$ implies A) and B) for a cube and the proof of $A')$
for a cube.

\begin{step}%%% 1
The truth of $A')$ for every $\u{\mathscr{F}}$ implies the truth of A)
and B) for every $\u{\mathscr{F}}$.
\begin{itemize}
\item[(i)] Suppose $A')$ true for all $\u{\mathscr{F}}$. Then
  $\u{\mathscr{F}}$ is a quotient of a sheaf $\u{\mathscr{G}}$ locally
  isomorphic to $\u{\mathscr{O}}^N$. Hence $\u{\mathscr{G}}$ defines a
  class of analytic bundles over a neighbourhood of $K$ (end of XI;
  the results proved there relate to topological bundles, but they
  remain valid with obvious modifications\pageoriginale for analytic
  bundles). By Theorem 2 of $X$, this class is the trivial class (on
  some neighbourhood of $K$) and so $\u{\mathscr{G}}$ is $\simeq
  \u{\mathscr{O}}^N$ and A) is proved. 

\item[(ii)] According to A), we have an exact sequence
$$
\u{\mathscr{O}}^{N_1} \to \u{\mathscr{F}} \to 0
$$
and if $\u{\mathscr{G}}_1$ is the kernel of this mapping,
$\u{\mathscr{G}}_1$ is coherent analytic by Theorem 1 of XII, and we
have the exact sequence
$$
0 \to \u{\mathscr{G}}_1 \to \u{\mathscr{O}}^{N_1} \to \u{\mathscr{F}}
\to 0.
$$
Since A) is supposed to hold for all coherent analytic sheaves, we
obtain exact sequences
\begin{equation*}
\begin{matrix}
0 \to \u{\mathscr{G}}_2 \to \u{\mathscr{O}}^{N_2} \to
\u{\mathscr{G}}_1 \to 0\\
\hdotsfor{1}\\
0 \to \u{\mathscr{G}}_k \to \u{\mathscr{O}}^{N_k} \to
\u{\mathscr{G}}_{k-1} \to 0 \\
\hdotsfor{1}
\end{matrix}
\end{equation*}
This leads to the exact sequence
$$
H^p (K, \u{\mathscr{O}}^{N_k}) \to H^p (K, \u{\mathscr{G}}_{k-1}) \to
H^{p+1} (K, \u{\mathscr{G}}_k) \to H^{p+1} (K,
\u{\mathscr{O}}^{N_k}). 
$$
If $p \geq 1$, the first and last terms are zero by Theorem 1 above
and consequently,
\begin{equation*}
H^p (K, \u{\mathscr{G}}_{k-1}) \simeq H^{p+1} (K, \u{\mathscr{G}}_k)
\tag{2} 
\end{equation*} 
Now, there is an integer $m$ such that every covering of $K$ has a
refinement in which the intersection of any $m+1$ sets (of the
refinement) is empty. [This is seen for example by subdividing $K$
  into smaller cubes by hyperplanes parallel to the coordinate
  hyperplanes. The stetement is, however,\pageoriginale essentially of
dimension theoretic character; classical dimension theory shows that
te best possible $m$ is $2n+1$ and that this is not special to
cubes. The above property is the starting point of the more modern
theory of dimension]. Hence $H^{m+p} (K, \u{\mathscr{H}}) =0$ for
every sheaf $\u{\mathscr{H}}$ over $K$ and $p \geq 1$. By iterating
(2), we obtain 
$$
H^p (K, \u{\mathscr{F}}) \simeq H^{p+1} (K, \u{\mathscr{G}}_1) \simeq
\ldots \simeq H^{m+p} (K, \u{\mathscr{G}}_m) = 0
$$
and B) is proved. 
\end{itemize}
\end{step}


\begin{step}%%% 2
\textbf{Proof of $A')$ for a cube.}
The proof will be by induction on the real dimension of the cube
$K$. If $K$ has dimension $0, A')$ is just the definition of a
coherent analytic sheaf. Suppose $A')$ true for all cubes $K'$ of real
dimension $p$ and all coherent analytic sheaves $\u{\mathscr{F}}$ on
$K'$. Then A) and B) are also true for $K'$ and $\u{\mathscr{F}}$. 
\end{step}

Let now $K$ be a cube of real dimension $p+1$. We find a coordinate
hyperplane such that the intersection of $K$ with any hyperplane
parallel to it is of dimension $p$ (if it is non-empty). The
restriction of $\u{\mathscr{F}}$ to each such intersection is coherent
analytic, and by inductive hypothesis, $\simeq$ a quotient of
$\u{\mathscr{O}}^N$. The extension theorem of XII shows that there is
a neighbourhood of each intersection in which $\u{\mathscr{F}}$
induces a coherect analytic sheaf which is isomorphic to a quotient of
same $\u{\mathscr{O}}^N$. By the Borel-Lebesgue lemma, it follows that
we have only to prove the following result:

Given two adjacent cubes $K_1$, $K_2$ of dimension $p+1$ such that $P=
K_1 \cap K_2$ is of dimension $p$, and a coherent analytic sheaf
$\u{\mathscr{F}}$ on $K_1 \cup K_2$ such that $\u{\mathscr{F}}$ is a
quotient of a sheaf $\u{\mathscr{O}}^{N_1}$ on $K_1$, and a quotient
of a sheaf $\u{\mathscr{O}}^{N_2}$ on $K_2$, then $\u{\mathscr{F}}$ is
a quotient of a sheaf locally isomorphic to
$\u{\mathscr{O}}^{N_1+N_2}$ on $K_1 \cup K_2$. 

Let\pageoriginale $(f) = \begin{pmatrix} 
f_1\\\vdots \\f_{N_1} \end{pmatrix}$, $f_1, \ldots, f_{N_1}$
being sections of $\u{\mathscr{F}}$ over $K_1$ which
$\mathscr{O}_a$-generate $\mathscr{F}_a$ for $a \in K_1$, $(g) =
 \begin{pmatrix}
g_1\\ \vdots \\ g_N\end{pmatrix}$, $g_1 , \ldots, g_{N_2}$ being
sections of $\u{\mathscr{F}}$ over $K_2$ which
$\mathscr{O}_b$-generate $\mathscr{F}_b$ for $b \in K_2$. It is easily
seen that it is enough to find a holomorphic regular matrix $c$ on $P$
such that 
$$
c\begin{pmatrix}
f\\0 
\end{pmatrix} =
\begin{pmatrix}
0 \\ g
\end{pmatrix}.
$$ 
To find $c$, we use the following lemma, which is of interest by
itself. 

\begin{lemma*}
If $\phi$ is a section of $\u{\mathscr{F}}$ over $P$, there are $N_1$
holomorphic functions $\lambda_1, \ldots, \lambda_{N_1}$ on $P$ such
that 
$$
\phi = \lambda_1 f_1 + \cdots + \lambda_{N_1} F_{N_1}. 
$$
\end{lemma*}

\medskip
\noindent{\textbf{Proof of the lemma:}}
Consider the exact sequence
$$
0 \to \u{\mathscr{G}}' \to \u{\mathscr{O}}^{N_1} \to \u{\mathscr{F}}
\to 0
$$
(these being sheaves on $P= K_1 \cap K_2$; $\u{\mathscr{G}}'$ is the
kernel of the homomorphism $\u{\mathscr{O}}^{N_1} \to
\u{\mathscr{F}}$). This gives us the exact sequence
$$
H^\circ (P, \u{\mathscr{O}}^{N_1})\to H^\circ (P, \u{\mathscr{F}}) \to
H^1 (P, \u{\mathscr{G}}')
$$
and by inductive hypothesis, $H^1 (P, \u{\mathscr{G}}') =0$ since $P$
has dimensions $p$. Hence the mapping 
$$
H^\circ (P, \u{\mathscr{O}}^{N_1}) \to H^\circ (P, \u{\mathscr{F}})
$$
is onto and the lemma follows. 


\medskip
\noindent{\textbf{Construction of c.}} The lemma proves that there is
a matrix $\gamma_1$ of $N_1$ columns and $N_2$ rows such that 
$$
\gamma_1 \begin{pmatrix}
f_1\\\vdots \\ f_{N_1} 
\end{pmatrix} 
= 
\begin{pmatrix}
g_1\\\vdots \\ g_{N_2}
\end{pmatrix}
$$
i.e., $\gamma_1(f) = (g)$.\pageoriginale In the same way there is a
matrix $\gamma_2$ of $N_2$ columns and $N_1$ rows such that
$$
\gamma_2 (g) = (f).
$$
If we set 
$$
c = \begin{pmatrix}
-I & \gamma_2\\
0 & I
\end{pmatrix} \cdot
\begin{pmatrix}
I & 0\\
\gamma_1 & I
\end{pmatrix}
$$
it is clear that $c$ is regular. Also $\begin{pmatrix}
I & 0\\ \gamma_1 & I\end{pmatrix}$ takes $\begin{pmatrix}
f\\0
\end{pmatrix}$ to $\begin{pmatrix}
f\\g\end{pmatrix}$, $\begin{pmatrix}
-I & \gamma_2 \\0 & I\end{pmatrix}$ takes $\begin{pmatrix}
f \\g
\end{pmatrix}$ to $\begin{pmatrix}
0 \\g
\end{pmatrix}$. This concludes Step 2 and with it the proof of the
fundamental theorem.

\begin{remark*}
All reference to bundles can be avoided if after the construction  of
$c$, one applies Cartan's theorem on holomorphic, regular matrices to
prove directly A) and B) without introducing $A')$.

The statement $A')$ is true for a much wider class of sets than is
A). Actually the problem of classifying the compact sets for which
$A')$ is true is still open (even in $C^n$). 
\end{remark*}
