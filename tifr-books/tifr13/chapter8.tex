\part{Differential Properties of the Cube}\label{partII}

\chapter{\texorpdfstring{$d''$}{d''}-cohomology on the cube}\label{chap8}

\section{Differential forms}\label{chap8:sec1}

Let\pageoriginale  $\mathscr{O} \subset R^n$ be an open set. The concept of
$(C^\infty)$-differential form is assumed known. A differential form
$\overset{r}{\omega}$ of degree $r$ in $\mathscr{O}$ has a
representation 
\begin{equation*}
\overset{r}{\omega} = \sum\limits_{i_1 < \ldots < i_r} a_{i_1 \ldots
  i_r} dx_{i_1} \wedge \ldots \wedge dx_{i_r}\tag{1}\label{chap8:eq1}
\end{equation*}
where $\wedge$ is the sign of exterior multiplication. The $a_{i_1
  \ldots i_r}$ are $C^\infty$ - functions. Also we define the partial
derivatives of a form (\ref{chap8:eq1}) by 
$$
\frac{\partial \overset{r}{\omega}}{\partial x_i} = \sum\limits_{i_1 <
\ldots < i_r} \frac{\partial a_{i_1 \ldots i_r}}{\partial x_i}
dx_{i_1} \wedge \ldots \wedge dx_{i_r}. 
$$

The differential $d \overset{r}{\omega}$ of the form (\ref{chap8:eq1}) is defined by
$$
d \overset{r}{\omega} = \sum\limits_{i=1}^n dx_i \wedge \frac{\partial
\overset{r}{\omega}}{\partial x_i}. 
$$

The operator $d$ has the following properties:
\begin{itemize}
\item[(a)] $d$ is a local operator: if $\omega = \omega'$ in an open
  set $U$, $d \omega = d\omega'$ in $U$. 

\item[(b)] $d$ is linear on the forms considered as a vector space
  over the complex numbers (but not as a module over
  $C^\infty$-functions). 

\item[(c)] $d(\overset{p}{\omega} \wedge \overset{q}{\omega}) = d
  \overset{p}{\omega} \wedge \overset{q}{\omega} + (-1)^p
  \overset{p}{\omega} \wedge d \overset{q}{\omega}$. 

\item[(d)] $d d = 0$. 

\item[(e)] $d$ is invariant under diffeomorphisms. 
\end{itemize}

Of course,\pageoriginale $d$ has the property that for functions $f$
(forms of degree $0$), $df = \sum \dfrac{\partial f}{\partial x_i}
dx_i$ is the ordinary differential of $f$. 

This property with (b), (c), and (d) characterize $d$ completely. The
following result, called \textit{Poincar\'e's theorem} holds:

Let $\mathscr{O}$ be an open ball in $R^n$ (or $\mathscr{O} =
R^n$). Let $\overset{p}{\omega}$ be a form of degree $p$ in
$\mathscr{O}$, such that $d \overset{p}{\omega} = 0$. Then, there
exists a form $\overset{p-1}{\pi}$ of degree $p-1$, such that 
$$
d \overset{p-1}{\pi} = \overset{p}{\omega}. 
$$

This will not a proved here. See for instance \cite{p2:key6}.


\section{The operators \texorpdfstring{$d'$}{d'} and \texorpdfstring{$d''$}{d''}}\label{chap8:sec2}

We now identify $C^n$ and $R^{2n}$, and set $z_j = x_j + i x_{n+j}$,
$j=1, \ldots, n$, where $(x_1, \ldots, x_{2n})$ are the coordinates in
$R^{2n}$. Then an $r$-form $\overset{r}{\omega}$ (form of degree $r$)
in $\mathscr{O}$ can be written uniquely in the form 
$$
\overset{r}{\omega} = \sum\limits_{\substack{p+q=r\\p,q \geq 0}}
\overset{(p,q)}{\omega} 
$$
where 
$$
\overset{(p,q)}{\omega} =\sum\limits_{\substack{0<i_1 < \ldots < i_p
    \leq n\\0<j_1<\ldots < j_q \leq n}} a_{i_1 \ldots i_p \; j_1
  \ldots j_q} dz_{i_1} \wedge \ldots \wedge dz_{i_p} \wedge
d\bar{z}_{j_1} \wedge \ldots \wedge d \bar{z}_{j_q} 
$$
$\overset{(p,q)}{\omega}$ is said to be of \textit{type} $(p,q)$. Its
degree is, of course, $p+q =r$. It is easy to verify that one has, for
every form $\omega$, 
$$
d\omega = \sum\limits^n_{j=1} dz_j \wedge \frac{\partial
  \omega}{\partial z_j} + \sum\limits^n_{j=1} d \bar{z}_j \wedge
\frac{\partial \omega}{\partial\bar{z}_j} .
$$
The first\pageoriginale sum is denoted by $d'\omega$, the second by
$d''\omega \cdot d'$, $d''$ are operators of degree $+1$, i.e., an
$r$-form goes into an $(r+1)$-form under $d'$, $d''$;  more precisely
\textit{$d'$ is of type (1,0)}, i.e., a form of type $(p,q)$ goes into
one of type $(p+1, q)$, while \textit{$d''$ is of type (0,1)},
taking forms of type $(p,q)$ into forms of type $(p, q+1)$.

The operators $d'$, $d''$ have properties similar to those of
$d$. They are the following:
\begin{itemize}
\item[(a)] $d'$, $d''$ are the local operators.

\item[(b)] $d'$, $d''$ are linear.

\item[(c)] $d'(\overset{p}{\omega} \wedge \overset{q}{\omega}) = d'
  \overset{p}{\omega} \wedge \overset{q}{\omega} + (-1)^p
  \overset{p}{\omega} \wedge d' \overset{q}{\omega}$ and similarly for
  $d''$. 

\item[(d)] $d'd' = 0$, $d'd'' + d'' d' = 0$, $d'' d'' = 0$ . 

\item[(e)] $d'$, $d''$ are invariant under analytic isomorphisms (but
  not under diffeomorphisms). A form $\omega$ is said to be
  \textit{$d'$, $(d'')$ closed} if $d'\omega =0$ $(d''\omega =0)$. 
\end{itemize}

\section{Triviality of \texorpdfstring{$d''$}{d''}-cohomology on a cube}\label{chap8:sec3}

A form $\omega$ is said to be \textit{holomorphic} if it is of type
$(p,0)$ and hte coefficient of $dz_{i_1} \ldots dz_{i_p}$ are
holomorphic for all $i_1 < \ldots < i_p$. A $(p,0)$ form $\omega$ is
holomorphic if and only if $d'' \omega =0$. In particular, a function
$f$ is holomorphic if and only if $d'' f =0$. 

For forms of type $(p,q)$ with $q \geq 1$, we prove an analogue of
Poin\-car\'e's theorem, due to A. Grothendieck.

A \textit{clsoed cube} in $C^n$ is a set in $C^n$ defined by
inequalities 
$$
|\mathscr{R} z_j| \leq a_j, \quad |\mathscr{J} z_j| \leq b_j, \quad
a_j, \quad b_j > 0. 
$$

\setcounter{thm}{0}
\begin{thm}\label{chap8:thm1}
Triviality\pageoriginale of $d''$-cohomology on a cube. Let $K$ be a
closed cube $\subset C^n$. Let $\omega$ be a form of type $(p,q)$, $q
\geq 1$ defined in a neighbourhood of $K$ and suppose that $d''\omega
= 0$. Then there exists a neighbourhood $U$ of $K$ and a form $\pi$ of
type $(p, q-1)$ in $U$ such that $d''\pi = \omega$ in $U$. 
\end{thm}

We need a lemma.

\begin{lemma*}
Let $\alpha(z, \lambda, \mu)$ be a complex function defined for $z \in
U$ ($U$ a neighbourhood of the closed unit square $\Delta$ with centre
$0$ in the $z$-plane), $\lambda \in \mathscr{O} \subset C^l$, $\mu \in
\Omega \subset R^m$. Suppose that $\lambda$ is differentiable in all
its variables, and is holomorhic in $\lambda_1, \ldots, \lambda_l$
$(\lambda = (\lambda_1, \ldots, \lambda_l))$. 
Then
$$
f(z) = f(z, \lambda, \mu) = \frac{1}{2\pi i} \iint_\Delta
\frac{\alpha(\zeta, \lambda, \mu)}{\zeta-z} d \zeta \wedge d
\bar{\zeta} 
$$
is differentiable in all its variables in $\overset{o}{\Delta} x
\mathscr{O} x \Omega$ and is a holomorphic function in $\lambda$, such
that
$$
\frac{\partial f}{\partial \bar{z}} = \alpha (z, \lambda, \mu), \; z
\in \overset{o}{\Delta}. 
$$
\end{lemma*}
 

\medskip
\noindent{\textbf{Proof of the lemma:}}
The integral exists since $\dfrac{1}{z}$ is locally summable. Let
$\delta$ be a closed square, centre $0$, $\delta \subset
\overset{o}{\Delta}$. It is sufficient to prove that $\dfrac{\partial
  f}{\partial \bar{z}} = \alpha$ for $z \in \overset{o}{\delta}$. Let
$\beta(z)$ be a $C^\infty$-function which is 1 in $\delta$ and $\beta
(z) = 0$ in a neighbourhood of the boundary of $\Delta$. (Such a
$\beta(z)$ exists). Now $\alpha = \alpha_1 + \alpha_2$, where
$\alpha_1 = \beta \alpha$, $\alpha_2 = (1-\beta) \alpha$, and we have 
$$
f(z) = f_1 (z) + f_2 (z), 
$$
where 
$$
f_1 (z) = \frac{1}{2\pi i} \iint\limits_{\Delta} \frac{\alpha_1
  (\zeta)}{\zeta - z} d \zeta \wedge d \bar{\zeta}, \quad f_2(z)  =
\frac{1}{2\pi i } \iint\limits_\Delta \frac{\alpha_2(\zeta)}{\zeta-z}
d \zeta \wedge d\bar{\zeta}. 
$$
and it\pageoriginale is obvious that $f_2$ is holomorphic in $z,
\dfrac{\partial f_2}{\partial \bar{z}} = 0$, if $z \in \delta^\circ$
and that $f_2$ is holomorphic in $\lambda$. Since $\alpha_1 (z) = 0$
in  a neighbourhood of the boundary of $\Delta$, we can define
$\alpha_1 (z) = 0$ outside $\Delta$ and write 
\begin{align*}
f_1 (z, \lambda, \mu) & = \frac{1}{2\pi i } \iint \frac{\alpha_1 (\zeta,
  \lambda, \mu)}{\zeta-z} d \zeta \wedge d \bar{\zeta}\\
& = \frac{1}{2\pi i} \iint \frac{\alpha_1 (u + z, \lambda, \mu)}{u}
\cdot du \wedge d \bar{u}, 
\end{align*}
if we substitute $u =\zeta -z$ (integrals without limits being over
the whole plane). From this form of the integral, it it clear that
$f_1(z, \lambda, \mu)$ is differentiable in all the variables and
holomorphic in $\lambda$. Also (writing $\alpha_1 (u+z)$ for $\alpha_1
(u+z, \lambda, \mu)$)
\begin{align*}
\frac{\partial f_1}{\partial \bar{z}} & = \frac{1}{2\pi i} \iint
\frac{\partial \alpha_1(u+z)}{\partial \bar{z}} \cdot \frac{1}{u}  du
\wedge d \bar{u}\\
& = \frac{1}{2 \pi i} \iint \frac{\partial \alpha_1(u+z)}{\partial
  \bar{u}} \frac{1}{u} du \wedge d\bar{u}\\
& = \lim\limits_{\epsilon\to 0} \frac{1}{2\pi i}  \iint_{|u| \geq \epsilon }
\frac{\partial \alpha_1 (u+z)}{\partial \bar{u}} \cdot \frac{1}{u}\, du
\wedge d \bar{u} \\
& = \lim\limits_{\epsilon \to 0} \frac{1}{2\pi i} \iint\limits_{|u| \geq
  \epsilon} \frac{\partial (\alpha_1(u+z) \cdot \frac{1}{u})}{\partial
  \bar{u}}  du \wedge d\bar{u}. 
\end{align*}
If $\Gamma^+_\varepsilon (\Gamma^-_{\varepsilon})$ is the positively
(negatively) oriented circle $|u| = \varepsilon$, Riemann's formula
applied to the last integral above gives
\begin{align*}
\frac{\partial f_1}{\partial\bar{z}} &= \lim\limits_{\epsilon\to 0} -
\frac{1}{2\pi i} \int\limits_{\Gamma^1_\varepsilon} \alpha_1 (u+z)
\frac{du}{u} \\
& = \lim\limits_{\epsilon \to 0} \frac{1}{2\pi i}
\int\limits_{\Gamma^+_\varepsilon} \alpha_1 (u+z) \frac{du}{u} =
\alpha_1(z) = \alpha(z),
\end{align*}
if $z \epsilon \delta^0$. This proves the lemma. 


\medskip
\noindent{\textbf{Proof of Grothendieck's Theorem:}}
The proof will be given first for forms of type $(0,1)$ to bring out
the method clearly, and then it will be given in the general case. 

The proof\pageoriginale is by induction. Consider the following
statement:

For all forms $\omega$ of type $(0,1)$ which are $d''$-closed and in
which the coefficients of $d \bar{z}_{k+1}, \ldots, d \bar{z}_n$ are
all 0, there exists an $f$ such that $d''f = \omega$.

The statement is trivially true when $k=0$ for then $\omega =0$ and we
may take $f=0$.

Suppose the statement true for all forms with $k$ replaced by
$k-1$. Suppose that in $\omega$ the coefficients of $d\bar{z}_{k+1},
\ldots, d\bar{z}_n$ are zero. Then the coefficients of $\omega$ are
holomorphic functions of $z_{k+1}, \ldots, z_n$ (for, if $\omega =
\sum\limits^k_{j=1} a_j d \bar{z}_j$, $d\bar{z}_l(l>k)$ occurs as
$\sum\limits^k_{j=1} \dfrac{\partial a_j}{\partial \bar{z}_l}
d\bar{z}_l \wedge d \bar{z}_j$ in $d''\omega$ and, since $\omega$ is
$d''$-closed $\dfrac{\partial a_j}{\partial \bar{z}_l}=0$ for
$l>k$). Now, by the lemma, there exists a function $g(z_1, \ldots,
z_n)$, differentiable in all the variables $z_1, \ldots, z_n$ and
holomorphic in $z_{k+1}, \ldots, z_n$ (in a neighbourhood of $K$) such
that $\dfrac{\partial g}{\partial \bar{z}_k} = a_k$. 

The problem is to find an $f$ so that $d'' f = \omega$. If we put $f_1
=f -g$, the problem becomes that of finding $f_1$, so that $d''f_1 =
\omega_1 = \omega - d''g$. Clearly $d'' \omega_1 = 0$, and by the
construction of $g$, the cofficient of $d\bar{z}_i$ in $\omega_1$ is
$0$ if $l \geq k$. By inductive hypothesis there is an $f_1$ such that
$d'' f_1 = \omega_1$ and the statement is true also for
$k$. Grothendieck's theorem for forms of type $(0,1)$ follows on
taking $k=n$. 

In the general case, the proof is the same.

Consider the following statement: For any $d''$-closed form $\omega$
of type $(p,q)$ in which all terms in which $d \bar{z}_{k+1}, \ldots,
d \bar{z}_n$ occur are zero, there exists a form $\pi$ of type $(p,
q-1)$ such that $d'' \pi = \omega$. 

The\pageoriginale statement is tirvially true if $k=0$. Suppose the
statement is true when $k$ is replaced by $k-1$, and let $\omega$ be a
$(p,q)$-form (form of type $(p,q)$) such that all the terms in which
$d \bar{z}_{k+1}, \ldots, d \bar{z}_n$ occur are zero. Then, in the
same way as above, it is seen that all the coefficients of $\omega$
are holomorphic functions of $z_{k+1}, \ldots, z_n$. Suppose now that
$$
\omega= d \bar{z}_k \wedge \overset{(p,q-1)}{\alpha} +
\overset{(p,q)}{\beta}. 
$$
By the lemma, there exists a form $\phi$ of type $(p, q-1)$
differentiable in all the variables, such that its coefficients are
holomorphic in $z_{k+1}, \ldots, z_n$ and such that $\dfrac{\partial
  \phi}{\partial \bar{z}_k} = \overset{(p,q-1)}{\alpha}$. (One has
merely to apply the lemma to the coefficients of
$\overset{(p,q-1)}{\alpha}$). As above, the problem reduces to finding
a form $\pi_1$ such that $d''\pi_1 = \omega_1 = \omega -
d''\phi$. Since $d''\omega_1 =0$ and the terms in which $d \bar{z}_k,
\ldots, d \bar{z}_n$ occur are zero by construction of $\phi$, the
existence of $\pi_1$ follows from inductive hypothesis and the theorem
of Grothendieck follows on taking $k=n$. 

It may be remarked that the theorem of Grothendieck is true also for
open cubes and polydiscs, but the proof necessitates a limit process,
and since this can be carried out for arbitrary ``Stein manifolds,''
these special cases are not considered here.

\section{Meromorphic functions}\label{chap8:sec4}

Let $V$ be a complex analytic manifold, and let $a \epsilon V$. Let
$\mathscr{O}_a$ denote the ring of germs of holomorphic functions at
$a$. It can be easily verified that $\mathscr{O}_a$ is an integrity
domain and we may therefore from the quotient field $\mathfrak{m}_a
\cdot \mathfrak{m}_a$ is called the set of \textit{germs of meromorphic
functions at $a$}.\pageoriginale Let $\mathfrak{m} = \bigcup\limits_{a
\in V} \mathfrak{m}_a$. A topology may be introduced on $\mathscr{m}$
as follows. Let $a \in V$ and let $\dfrac{f_a}{g_a} = m_a \in
\mathfrak{m}_a$. Let $f_a$ and $g_a$ be defined by holomorphic
functions $f$, $g$ in an open connected neighbourhood $\Omega$ of
$a$. For every point $b \in \Omega$, $m_b$ is defined to be
$\dfrac{f_b}{g_b}$. If $f'$, $g'$ are two other holomorphic functions
in $\Omega$ such that $\dfrac{f'_a}{g'_a} = m_a$, then $f'_a g_a -
g'_a f_a = 0$ and $f' g - g'f =0$ in a neighbourhood of $a$, and by
the principle of analytic continuation, $f'g -g 'f =0$ in $\Omega$ so
that $\dfrac{f'_b}{g_b} = \dfrac{f'_b}{g'_b}$ and the above definition
is unique. A neighbourhood of $(a, m_a) \in\mathfrak{m}$ is now
defined to be $\bigcup\limits_{b \in \Omega} (b, m_b)$, where $\Omega$
has the properties mentioned above. In this topology, $\mathfrak{m}$
is a sheaf over $V$.

A \textit{meromorphic function} is now simply defined to be \textit{a
  section of $\mathfrak{m}$ over} $V$, i.e, a continuous map $f: V
\to \mathfrak{m}$ such that $f(a) \in \mathfrak{m}_a$ for every $a \in
V$. 

The weak principle of analytic continuation remains valid when
holomorphic functions are replaced by meromorphic
functions. Meromorphic functions may also be defined in terms of
coverings and local quotients of holomorphic functions, with certain
obvious consistency conditions. 

\medskip
\noindent{\textbf{Principle Parts.}}

A system of \textit{principal parts} on $V$ is a section of the
quotient $\mathfrak{m}/\mathscr{O}$ ($\mathfrak{m}$ being the sheaf of
additive groups of germs of meromorphic functions, $\mathscr{O}$ the
sheaf of additive groups of germs of holomorphic functions.) But the
following alternative definition is the one that will be used in
Cousin's first problem.

A system\pageoriginale of principal parts on the complex manifold $V$
consists of an open covering $\{\Omega_i\}$ of $V$ and meromorphic
functions $f_i$ in $\Omega_i$; such that $f_i - f_j$ is holomorphic in
$\Omega_i \cap \Omega_j$. (The meaning of this last statement is
clear). Two systems $\{\Omega_i, f_i\}$, $\{\Omega'_j, f'_j\}$ define
the same principal parts if $f_i - f'_j$ is holomorphic in $\Omega_i
\cap \Omega'_j$ for every $i,j$. Here, and in what follows, properties
like the above are assumed fulfilled when the intersections in
question ar empty.

\section{The first Cousin problem}\label{chap8:sec5}

The problem is the following: Suppose given a sytem of principal parts
$\{\Omega_i, f_i \}$ on the complex manifold $V$. Then does there
exist a meromorphic function $f$ on $V$ such that $f-f_i$ is
holomorphic in $\Omega_i$, i.e., when is the system of principal parts
defined by one function?

This problem may be generalized to 


\medskip
\noindent{\textbf{The generalized first Cousin problem.}}

Let $\{\Omega_i\}$ be an open covering of the complex manifold $V$ and
suppose given a family of functions $\{c_{ij}\}$ such that $c_{ij}$ is
holomorhic in $\Omega_i \cap \Omega_j$ and having the following
properties:
\begin{equation*}
c_{ij} + c_{ji}  = 0 \text{ in } \Omega_{i} \cap \Omega_j, \;
c_{ij} + c_{jk} + c_{ki} = 0 \text{ in } \Omega_i \cap \Omega_j \cap
\Omega_k. 
\end{equation*}

Then, is it possible to find holomorphic functions $c_i$ in $\Omega_i$
such that $c_{ij} = c_i - c_j$ in $\Omega_i \cap \Omega_j$?

A solution of this problem leads to a solution of the first cousin
problem, for if we take $c_{ij} = f_i - f_j$ and $c_{ij} = c_i - c_j$,
then $f_i - c_i = f_j - c_j$ in $\Omega_i \cap \Omega_j$, and if we
define $f = f_i - c_i$ in $\Omega_i$, it is easy to see that $f$
solves the first Cousin problem. 

\medskip
\noindent{\textbf{The first Cousin Problem fo the cube.}}\pageoriginale

The following therorem will now be proved.

\begin{thm}\label{chap8:thm2}
Let $K$ be a cube in $C^n$ and let $\{\Omega_i, c_{ij}\}$ be a system
such that $\{\Omega_i\}$ is an open covering of $K$ and the $c_{ij}$
have the properties given above. Then there exists a neighbourhood $U$
of $K$ such that the generalized Cousin problem has a solution for the
system $\{U \cap \Omega_i, c_{ij}\}$. 
\end{thm}

\begin{proof}
We assume first that $\{\Omega_i\}$ is a finite covering 
\end{proof}

\setcounter{step}{0}
\begin{step}\label{chap8:step1}
There exists a neighbourhood $U_1$ of $K$ and a system $\{\gamma_i\}$
of $C^\infty$-functions $\gamma_i$ in $\Omega_i \cap U_1$ such that
$\gamma_i - \gamma_j = c_{ij}$. 
\end{step}

Let $\{\phi_i\}$ be a differentiable partition of unity relative to
the convering $\{\omega_i\}$ of $K$, i.e., $\phi_i$ is $C^\infty$ and
has compact support contained in $\Omega_i$, $\phi_i \geq 0$ and $\sum
\phi_i =1$ in a neighbourhood $U_1$ of $K$. Such a partition of unity
exists.

Consisder the following function $\gamma_i$ on $\Omega_i \cap U_1$. 

Let $z \in\Omega_i \cap U_1$; define $\gamma_i (z) = \sum\limits_{j
  \neq 1} \phi_j (z) c_{ij} (z)$. This sum is meaningful, for if
$c_{ij}(z)$ is not defined, then $z \not\in \Omega_j$ and so
$\phi_j(z) =0$, and we define $\phi_j(z) c_{ij} (z)$, for such $z$, to
be zero. It is easily seen that $\gamma_i$ isdifferentiable in
$\Omega_i \cap U_1$. Now 
$$
\gamma_i - \gamma_j = \sum\limits_{k \neq i, j} \phi_k (c_{ik} -
c_{jk}) + \phi_j c_{ij} - \phi_i c_{ji}. 
$$
Also
$$
c_{ik} - c_{jk} = c_{ik} + c_{kj} = - c_{ji} = c_{ij}. 
$$
Hence
$$
\gamma_i - \gamma_j = \sum\limits_k \phi_k c_{ij} = c_{ij} \text{ in }
\Omega_i \cap \Omega_j \cap U_1. 
$$
Step 1 is completed. 


\begin{step}\label{chap8:step2}
\textbf{Solution of the generalized first Coursin
    problem.} 

In\pageoriginale $\Omega_i \cap \Omega_j \cap  U_1$ we have
$d''\gamma_1 - d'' 
\gamma_j = d''c_{ij} = 0$ since $c_{ij}$ is holomorphic. Hence if we
define a form $\alpha$ (of type $(0,1)$) by $\alpha = d'' \gamma _i $
in $\Omega_i \cap U_1$ we have a well-defined form on $U_1$. Clearly
$d'' \alpha =0$ and by Grothendieck's theorem, there is a $(0,0)$ form
$\beta$, i.e., a function $\beta$ such that $d'' \beta = \alpha$ in a
neighbourhood $U \subset U_1$ of $K$. If we set $c_i =\gamma_i -\beta$
in $\Omega_i \cap U$, $d'' c_i = d'' \gamma_i - d'' \beta = \alpha -
\alpha =0$ so that $c_i$ is holomorphic in $\Omega_i$ while $c_i - c_j
= \gamma_i - \gamma_j = c_{ij}$ in $\Omega_j \cap \Omega_j \cap
U$. This commpletes the proof of the theorem when $\{\Omega_i\}$ is
finite. 
\end{step}

In the general case let $\Omega_1, \ldots, \Omega_p$ be a finite
covering of the cube $K$, extracted from $\{\Omega_i\}$. By passing to
suitable intersections, we may assume that each $\Omega_\alpha$ is
contained in $\Omega_1 \cup \ldots \cup \Omega_p $ while the functions
$c_1, \ldots, c_p$ are defined everywhere in $\Omega_1, \ldots,
\Omega_p$ respectively. Given $\alpha$, we define $c_\alpha = c_i +
c_{\alpha i}$ on $\Omega_\alpha \cap \Omega_i (i = 1, \ldots, p)$ . On
$\Omega_\alpha \cap \Omega_i \cap \Omega_j$, $c_i + c_{\alpha i} = c_j
+ c_{\alpha j}$, for $c_i - c_j = c_{ij} = c_{\alpha j} - c_{\alpha
  i}$. Since $\Omega_\alpha \subset \bigcup\limits^p_{i=1} \Omega_i$,
$c_{\alpha}$ is defined on $\Omega_\alpha$ and it is easily verified
that the system $\{\Omega_\alpha, c_\alpha\}$ solves the generalized
first Cousin problem. 



