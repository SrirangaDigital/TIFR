

\part{Domains of Holomorphy}\label{part1}

\chapter{Cauchy's formula and elementary consequences}\label{chap1}

1. Let\pageoriginale $C^n$ be the space of $n$ complex variables
$(z_1,\ldots, z_n)$. We write simply $z$ for $(z_1, \ldots, z_n)$. Let
$z_j = x_j + iy_j$, $j=1,\ldots, n$, and let $\Omega$ be an opne set
in $C^n$. Suppose that $f(z) = f(z_1, \ldots, z_n)$ is a complex-valued
function defined in $\Omega$ which is (once) continuously
differentiable as a function of the $2n$ real variables $x_1 , y_1;
\ldots ; x_n, y_n$. 
 
Set, by definition,
$$
\frac{\partial f}{\partial z_j} = \frac{1}{2} \left(\frac{\partial
  f}{\partial x_j} - i \frac{\partial f}{\partial y_j} \right)
$$
and
$$
\frac{\partial f}{\partial \bar{z}_j} = \frac{1}{2} \left(\frac{\partial
  f}{\partial x_j} + i \frac{\partial f}{\partial y_j} \right). 
$$

\begin{defi*}
$f(z)$ is said to be \textit{holomorphic} in $\Omega$ is
  $\dfrac{\partial f}{\partial \bar{z}_j} = 0$, $j=1,\ldots , n$, at
  every point of $\Omega$. (These equations generalize the
  Cauchy-Riemann equations to the case of functions of several
  variables). 
\end{defi*}

A definition which is equivalent to the above, is $t$ the following:
$f(z)$ is said to have a complex derivative at $a \in\Omega$ if, for
any $b \in C^n$, $\lambda$ complex, 
$$
\lim\limits_{\lambda \to 0} \frac{f(a+\lambda b) - f(a)}{\lambda}
$$
exists. $f(z)$ is said to be holomorphic in $\Omega$ if it has a
complex derivative at every $a\in\Omega$.

\begin{defi*}
A polydisc, with centre at the origin is the set $K$ of points such
that 
$$
|z_1| \leq \rho_1, |z_2| \leq \rho_2, \ldots, |z_n| \leq \rho_n,
\rho_1, \rho_2, \ldots, \rho_n > 0. 
$$
These inequalities are, for brevity, written
$$
|z| \leq \rho.
$$
\end{defi*}

Let $\Gamma$\pageoriginale denote the set of points $z\in C^n$ for
which
$$
|z_1| = \rho_1, \ldots, |z_n| = \rho_n.
$$

Consider a function $f(z)$ holomorhpic in a neighbourhood of $K$ and
denote by $c_j$ the curve $|z_j| = \rho_j$ in the $z_j$-plane. Then
the following theorem holds.

\medskip
\noindent{\textbf{Cauchy's formula.}}
If $z$ is a point with $|z| < \rho$ (i.e., $|z_j| < \rho_j$,
$j=1,\ldots, n$), then 
$$
f(z_1,\ldots, z_n) = \frac{1}{(2\pi i)^n}
\underset{\Gamma}{\int\ldots\int} \frac{f(\zeta_1, \ldots,
  \zeta_n)}{(\zeta_1 -z_1) \ldots (\zeta_n - z_n)} d \zeta_1 \ldots d
\zeta_n. 
$$

(The integral $\underset{\Gamma}{\int\ldots\int} g(\zeta_1, \ldots,
\zeta_n) d\zeta_1 \ldots d\zeta_n$, is defined, for continuous $g$, to
be $\int_{c_1} d \zeta_1 \ldots \int_{c_n} g(\zeta_1, \ldots, \zeta_n)
d\zeta_n$ and is independent of the order in which the repeated
integration is performed.) 

\begin{proof}
Repeated application of the Cauchy formula for holomorphic functions
of one complex variable gives
\begin{align*}
f(z_1,\ldots, z_n) & = \frac{1}{2\pi i} \int_{c_1} \frac{f(\zeta_1,
  z_2, \ldots, z_n)}{\zeta_1 - z_1} d \zeta_1\\
& = \frac{1}{(2\pi i)^2} \int_{c_1} \frac{d\zeta_1}{\zeta_1-z_1}
\int_{c_2} \frac{f(\zeta_1, \zeta_2, z_3, \ldots, z_n)}{\zeta_2 - z_2}
d \zeta_2 \\
& = \ldots = \frac{1}{(2\pi i)^n} \int_{c_1} \frac{d\zeta_1}{\zeta_1
  -z_1} \int_{c_2} \frac{d\zeta_2}{\zeta_2 - z_2} \ldots\\
&\qquad\qquad\int_{c_n}
\frac{f(\zeta_1, \ldots, \zeta_n)}{\zeta_n -z_n} d \zeta_n\\
& =  \frac{1}{(2\pi i)^n} \underset{\Gamma}{\int\cdots\int}
\frac{f(\zeta_1, \ldots, \zeta_n)}{(\zeta_1-z_1) \ldots (\zeta_n -
  z_n)} d \zeta_1 \ldots d \zeta_n.
\end{align*}
\end{proof}

2. If $f$ is a complex valued continuous function on a compact space
$K$, we define
$$
||f||_K = \sum\limits_{x\in K} |f(x)|.
$$

\begin{defi*}
A series\pageoriginale $\sum\limits_{m\in N^n} a_m (z)$ of complex
valued continuous functions on a compact space $K$ is said to converge
normally if 
$$
\sum\limits_{m\in N}||a_m||_K < + \infty.
$$
($N$ is the set of non-negative integers).
\end{defi*}

If $a_m(z)$ are functions defined in an open set $\Omega$, we say the
series converges normally in $\Omega$ if it converges normally on
every compact subset of $\Omega$.

The following are simple consequences of Cauchy's formula.

\begin{proposition}\label{chap1:prop1}
If $f(z)$ is holomorphic in a neighbourhood $U$ of a polydisc $K$,
then 
\begin{equation*}
f(z_1,\ldots, z_n) = \sum\limits_{(j_1,\ldots, j_n) \in N^n}
a_{j_1,\ldots, j_n} z^{j_1}_1 \ldots z^{j_n}_n \text{ for } z \in
K. \tag{1}\label{chap1:eq1} 
\end{equation*}
\end{proposition}


\medskip
\noindent{\textbf{The series converges normally on $K$.}}

The proposition is proved simply by applying Cauchy's formula to a
polydisc $K'$ wiht $K \subset \overset{o}{K'} \subset  K' \subset U$
($\overset{o}{K'}$ is the interior of $K'$). 

The coefficients $a_j = a_{j_1, \ldots, j_n}$ in (\ref{chap1:eq1}) are given by 
\begin{align*}
& a_{j_1, \ldots, j_n}  = \frac{1}{(2\pi i)^n}
\underset{\Gamma}{\int\ldots \int}
\frac{f(\zeta_1,\ldots,\zeta_n)}{\zeta_1^{j_1+1} \ldots
  \zeta_n^{j_n+1}} d \zeta_1 \ldots d \zeta_n\\
& \qquad = \frac{1}{(2\pi)^n} \int^{2\pi}_0 \ldots \int^{2\pi}_0 \frac{f(\rho_1
  e^{i\theta_1}, \ldots, \rho_n e^{i\theta_n})}{\rho_1^{j_1}\ldots
  \rho_n^{j_n}}\\
&\qquad\qquad e^{-i(j_1 \theta_1+ \cdots + j_n \theta_n)}
d\theta_{1} \ldots d\theta_n \tag{2} \label{chap1:eq2}
\end{align*}


From the expansion of $f(z)$ as a power series (\ref{chap1:eq1}), it is seen that
$f(z)$ is indefinitely differentiable and that 
\begin{equation*}
\frac{\partial^{j_1 + \ldots + j_n} f(0)}{\partial z^{j_1}_1 \ldots
  \partial z^{j_n}_n} = j _1 ! \ldots j_n ! a_{j_1,\ldots j_n} 
\end{equation*}
so that\pageoriginale the expansion is unique. By differentiation in
Cauchy's formula, we have also
{\fontsize{10}{12}\selectfont
\begin{align*}
\frac{\partial^{j_1 + \ldots + j_n} f(z)}{\partial z^{j_1}_{1} \ldots
  \partial z^{j_n}_{z_n}} &= \frac{j_1 ! \ldots j_n !}{(2\pi i)^n}
\underset{\Gamma}{\int\ldots \int} \frac{f(\zeta_1, \ldots,
  \zeta_n)}{(\zeta_1 - z_1)^{j_1+1} \ldots (\zeta_n - z_n)^{j_n+1}}  d
\zeta_1 \ldots d \zeta_n\tag{3}\label{chap1:eq3}
\end{align*}}\relax

If $f(z)$ is assumed to be holomorphic only in the interior of $K$, we
can write
$$
f(z) = \sum\limits_{J\in N^n} a_J z^J
$$
where 
$$
a_J = \frac{1}{(2\pi i)^n} \underset{\Gamma'}{\int\ldots\int} 
\frac{f(\zeta_1 \ldots, \zeta_n)}{\zeta_1^{j_1+1} \ldots
  \zeta_n^{j_n+1}}   d\zeta_1\ldots d\zeta_n, 
$$
$\Gamma'$ being a set of points of the form 
$$
|z_1| = r_1, \ldots , |z_n| = r_n, \quad 0 < r_j < \rho_j, \quad j =
1, \ldots, n.
$$
The series converges normally in the interior of $K$.

If $f(z)$ is holomorphic in a neighbourhood of $K$ and $M =
\max\limits_{z\in\Gamma}\break |f(z)| = \max\limits_{z\in K}|f(z)|$  (the
latter equality is easily established using\break Cauchy's formula), then 
$$
|a_J| \leq \frac{M}{\rho J}
$$
i.e.
$$
|a_{j_1, \ldots, j_n}| \leq M/ (\rho^{j_1}_1 \ldots \rho^{j_n}_n). 
$$
This follows at once from the expression (\ref{chap1:eq2}) for $a_J$ as an integral
in terms of $f(z)$. The inequalities are called the \textit{Cauchy
  inequalities}. 

\begin{proposition}\label{chap1:prop2}
Let $\{f_k(z)\}$ be a sequence of functions holomorphic in $\Omega$
and suppose that $\{f_k(z)\}$ converges, uniformly on every compact
subset of $\Omega$, to a function $f(z)$. Then $f(z)$ is holomorphic
in $\Omega$. 
\end{proposition}

\begin{proof}
Since\pageoriginale $f_k(z) \to f(z)$ uniformly in a neighbourhood of
any $z\in \Omega$, $f(z)$ is continuous in $\Omega$. Also, in a
polydisc $K$ about any point of $\Omega$ (lying wholly in $\Omega$),
$f_k(z)$ verifies Cauchy's formula, and $\{f_k(z)\}$ being uniformly
convergent on $K$, so does $f(z)$, from which it follows easily that
$f(z)$ is holomorphic in $\Omega$.

Also, using the integral (\ref{chap1:eq3}) for the derivatives of a holomorphic
function, one proves that the derivatives (opf all orders) of $f_k(z)$
converge to the corresponding derivatives of $f(z)$, uniformly on
every compact subset of $\Omega$. 

Another interpretation of the Proposition \ref{chap1:prop2} is as follows. If
$\mathscr{C}_{\Omega}, \mathscr{H}_{\Omega}$ denote the sets of
continuous and holomorphic functions in $\Omega$ respectively,
$\mathscr{C}_{\Omega}$  and $\mathscr{H}_\Omega$ are vector spaces
over the field of complex numbers. We recall that one may topologize
$\mathscr{C}_\Omega, \mathscr{H}_\Omega$  by putting on them the
topology of uniform convergence on compact sets, namely if $f_n \in
\mathscr{C}_\Omega$ (or $\mathscr{H}_\Omega$),  $f_n \to 0$ if
$||f_n||_K \to 0$ as $n \to \infty$ for every compact $K \subset
\Omega$. A fundamental system of neighbourhoods of the origin is given
by the sets $\mathscr{U} (K_m,1/m)$, where $\mathscr{U} (K,a)(a>0)$ is
the set of $f\in \mathscr{C}_\Omega$ (resp. $\mathscr{H}_\Omega$) for
which $||f||_K < a$, and $\{K_m\}$ is a sequence of compact sets with 
$$
K_m \subset \overset{o}{K}_{m+1}, \quad \bigcup\limits^\infty_{m=1}
\overset{o}{K}_m = \Omega.
$$
$\mathscr{C}_\Omega$ is an $(\mathscr{F})$-space, i.e., it has a
countable fundamental system of neighbourhood of 0 and is complete.

Proposition \ref{chap1:prop2} may be expressed by saying that $\mathscr{H}_\Omega$
\textit{is a closed subspace of} $\mathscr{C}_\Omega$.
\end{proof}

\begin{proposition}\label{chap1:prop3}
Every\pageoriginale bounded closed set $\Phi$ in $\mathscr{H}_\Omega$
is compact. 

(A bounded set $\Phi$ in any topological vector space is a set such
that to any neighbourhood $\mathscr{U}$ of the origin, there exists a
$\lambda > 0$ such that $\Phi \subset \lambda
\mathscr{U}$. $\mathscr{C}_\Omega $ or $\mathscr{H}_\Omega$, we may
say equivalently 
that a set $\Phi$ is bounded if $\sum\limits_{f\in\Phi}||f||_K < +
\infty$ for every compact $K\subset \Omega$.).
\end{proposition}

\begin{proof}
Let $K$ be any compact subset of $\Omega$. Let $\Phi_K$ be the set of
functions $f_K$, where $f_K$ is the restriction of $f\in \Phi$ to
$K$. We prove that $\Phi_K$ is equicontinuous from which it follows by
means of Ascoli's theorem (see e.g. Bourbaki: Topologie G\'en\'erale,
Chap.X, p.48) that $\Phi_K$ is relatively compact in $\mathscr{C}_K$. 

Choose a compact set $K'$ so that $K\subset \overset{o}{K'} \subset K'
\subset \Omega$. $K$ has positive distance from the boundary of
$K'$. Since $\sup\limits_{f\in\Phi} ||f||_{K'} < + \infty$ Cauchy's
inequlity applied to the derivatives of the $f$ in a suitable polydisc
about an arbitrary point of $K$, contained in $K'$ shows that
$$
\sup\limits_{f\in \Phi} \max\limits_j ||\frac{\partial f}{\partial 
  z_j}||_K = M_K < \infty. 
$$
Since $\dfrac{\partial f}{\partial \bar{z}_j} = 0$, it follows from
the definitions of $\dfrac{\partial f}{\partial z_j}$,
$\dfrac{\partial f}{\partial \bar{z}_j}$ that $\dfrac{\partial
  f}{\partial x_j}$, $\dfrac{\partial f}{\partial y_j}$, are bounded
uniformly on $K$, for $f\in\Phi$ (actually, $||\dfrac{\partial
  f}{\partial x_j}||_K \leq M_K$, $||\dfrac{\partial f}{\partial
  y_j}||_K \leq M_K$). The mean value theorem now shows that $\Phi_K$
is equicontinuous. 

To prove now that $\Phi$ is compact, it suffices to prove that any
sequence $\{f_m\}$, $f_m \in \Phi$ has a limit point in
$\mathscr{C}_\Omega$. We choose a sequence $\{K_m\}$ of compact sets
$K_m$ with $K_m \subset \overset{\circ}{K}_{m+1}, \bigcup^\infty_{m=1}
K_m = \Omega$. Since $\Phi_K$ is relatively compact for every compact
$K\subset\Omega$, we choose, inductively, subsequences $\{f_{m,y}\}$
of $\{f_m\}$, such that each is a subsequence of the
preceding,\pageoriginale while if $f_{m,\nu+1} = f_{m',\nu_0}$ then
$m'>m$, and $\{f_{m,\nu}\}$ converges uniformly on $K_\nu$. Since
$\bigcup \overset{\circ}{K}_m = \Omega$, the subsequence
$\{f_{\nu\nu}\}$ of $\{f_m\}$ converges to a limit in
$\mathscr{C}_\Omega$ and Proposition \ref{chap1:prop3} is proved.

Suppose given a subset $\Phi$ of $\mathscr{H}_\Omega$ such that
$\sup\limits_{f\in\Phi} |f(z)| < + \infty $ for every $z\in
\Omega$. The question arises as to what one can assert about the
boundedness of the set $\Phi$. The following result holds.
\end{proof}

\begin{proposition}\label{chap1:prop4}
There exists an open set $\Omega' \subset \Omega$, which is dense in
$\Omega$ such that $\Phi_{\Omega'}$ (the set of the restrictions to
$\Omega'$ of functions of $\Phi$ is a bounded set in
$\mathscr{H}_{\Omega'}$.) 

The proposition and its proof remain valid if $\mathscr{H}_\Omega$ is
replaced by $\mathscr{C}_\Omega$. We need the following
\end{proposition}


\medskip
\noindent{\textbf{Theorem of Baire.}}
Let $U$ be an open set $\subset \mathbb{C}^n$ and $\{\mathscr{O}\}$, $k =
1,2,\ldots$ a sequence of open sets $\subset U$, such that each
$\mathscr{O}_k$ is dense in $U$. The $\bigcap\limits^\infty_{k=1}
\mathscr{O}_k$ is dense in $U$.

\begin{proof}
Since $\mathscr{O}_1$ is dense in $U$, given any open ball $B \subset
U$, $\mathscr{O}_1 \cap B$ contains an open ball $B_1$. In the same
way, $\mathscr{O}_2 \cap \frac{1}{2} B_1$ ($\frac{1}{2} B_1$ is the
ball with the same centre as $B_1$ and half its radius) contains an
open ball $B_2$ and this process can be continued. Clearly
$\bigcap\limits^\infty_{k=1} \bar{B}_k \neq 0$ and since $\bar{B}_k \subset
B_{k-1}$, $\bigcap\limits^\infty_{k=1} B_k \neq 0$. Also $B_k \subset
\mathscr{O}_k \cap B$, so that $\bigcap\limits^\infty_{k=1}
\mathscr{O}_k \cap B \neq 0$ and $\bigcap\limits^\infty_{k=1}
\mathscr{O}_k$ is dense in $U$. 
\end{proof}

\medskip
\noindent{\textbf{Proof of the proposition.}}

Let $U$ be an arbitrary open set $\subset \Omega$. Let $\mathscr{O}_k
\subset U$ be the set of $z\in U$ for which there exists at least one
$f\in\bar{\Phi}$ for which $|f(z)|>k$. Clearly $\mathscr{O}_k$ is
open. Also since for any $z \in \Omega$,
$\sup\limits_{f\in\Phi}|f(z)|< + \infty$,
$\bigcap\limits^\infty_{k=1}\mathscr{O}_k =0$. By Baire's theorem, at
least one $\mathscr{O}_k$ is not\pageoriginale dense in $U$. Hence $U$
contains an open set $\mathscr{O}_U$ contained in the complement of
$\mathscr{O}_k$ for some $k$ and the functions of $\Phi$ are uniformly
bounded for $z \in \mathscr{O}_U$(by $k$). If we set $\Omega' =
\bigcup\limits_U \mathscr{O}_U$, $\Omega'$ is clearly open and dense
in $\Omega$. If $z \in \Omega'$, $z$ possesses a neighbourhood on
which $\Phi$ is uniformly bounded and so $\Phi$ is bounded uniformly
on every compact subset of $\Omega'$ (by the Borel-Lebesgue lemma).
