\chapter{Cohomology with coefficients in a sheaf}\label{chap13}

\section{Cohomology of a covering}\label{chap13:sec1}

Let\pageoriginale $X$ be a topological space, $\u{\mathscr{F}}$ a
sheaf of abelian groups on $X$. Let $\mathscr{O} =
\{\mathscr{O}_i\}_{i\in I}$ be an open covering of $X$. We shall
denote by $\mathscr{O}_{i_\circ, \ldots, i_p}$ the set
$\mathscr{O}_{i_\circ} \cap \ldots \cap \mathscr{O}_{i_p}$ and, $U$
being an open set in $X$, by $\Gamma(U, \u{\mathscr{F}}) =
\u{\mathscr{F}}_U$ the sections of $\u{\mathscr{F}}$ over $U$. (If $U$
is empty, we set $\Gamma (U, \u{\mathscr{F}}) = 0$).

\begin{defi*}
A $p$-\textit{cochain} of $\mathscr{O}$ is a mapping $c$ of $I^{p+1}$
such that $c_{j_\circ \ldots j_p} \in \Gamma (\mathscr{O}_{i_\circ
  \cdot i_p}, \u{\mathscr{F}})$ and which is, moreover,
\textit{alternate}, (i.e., $c_{j_\circ \ldots j_p} = \varepsilon
c_{i_\circ \ldots i_p}$ if $(j_\circ, \ldots, j_p)$ is a permutation
of $(i_\circ, \ldots, i_p)$ and $\varepsilon = \pm 1$ according as
this permutation is even or odd). 
\end{defi*}

$\mathscr{C}^p (\mathscr{O}, \u{\mathscr{F}})$ will denote the abelian
group of the $p$-cochains of $\mathscr{O}$, $\mathscr{C} (\mathscr{O},
\u{\mathscr{F}}) = \sum\limits_{ p \geq 0} \mathscr{C}^p(\mathscr{O},
\u{\mathscr{F}})$ the direct sum of the $\mathscr{C}^p (\mathscr{O},
\u{\mathscr{F}})$ for $p \geq 0$. 

The \textit{coboundary operator} $\delta^p: \mathscr{C}^p
(\mathscr{O}, \u{\mathscr{F}}) \to \mathscr{C}^{p+1} (\mathscr{O},
\u{\mathscr{F}})$  is defined as follows: if $c \in \mathscr{C}^p$,
$$
(\delta^p c)_{i_\circ \ldots i_{p+1}} = \sum\limits^{p+1}_{j=0} (-1)^j
c_{i_\circ \ldots \hat{i}_j \ldots i_{p+1}}
$$ 
where $i_\circ, \ldots \hat{i}_j \ldots i_{p+1}$ signifies that the
$j$-th index $i_j$ is to be omitted. The $\delta^p$ give rise to a
coboundary operator $\delta: \mathscr{C} (\mathscr{O},
\u{\mathscr{F}}) \to (\mathscr{O}, \u{\mathscr{F}})$. It can be
verified that 
$$
\delta^p \circ \delta^{p-1} = 0 \text{ i.e., } \delta \circ \delta = 0.
$$
Now, the elements $c$ of $\mathscr{C}^\circ (\mathscr{O},
\u{\mathscr{F}})$ such that $\delta c = 0$ are precisely those
elements with $c_{i_1} - c_{i_\circ} = 0$ in $\mathscr{O}_{i_\circ
  i_1}$ (by the definition of $\delta$) and so,\pageoriginale since
$c_{i_\circ} = c_{i_1}$ in $\mathscr{O}_{i_\circ} \cap
\mathscr{O}_{i_1}$ they correspond to sections over the whole of $X$,
i.e., to elements of $\Gamma(X, \u{\mathscr{F}})$. Hence we have the
following compelx:
$$
0 \to \Gamma (X, \u{\mathscr{F}}) \xrightarrow{i} \mathscr{C}^\circ
(\mathscr{O}, \u{\mathscr{F}})  \xrightarrow{\delta^\circ}
\mathscr{C}^1 (\mathscr{O}, \u{\mathscr{F}}) \xrightarrow{\delta^1} \ldots
$$
and $\delta^p \circ \delta^{p-1} = 0$. Setting $z^p (\mathscr{O},
\u{\mathscr{F}}) = $ kernel of $\delta^p$, $B^p (\mathscr{O},
\u{\mathscr{F}}) = $ image of $\delta^{p-1}$, we define the $p$-th
\textit{cohomology group $H^p (\mathscr{O}, \u{\mathscr{F}})$ of the
  covering $\mathscr{O}$ with coefficient sheaf $\u{\mathscr{F}}$ by}
\begin{align*}
 H^p (\mathscr{O}, \u{\mathscr{F}}) & = Z^p (\mathscr{O},
 \u{\mathscr{F}})/ B^p (\mathscr{O}, \u{\mathscr{F}}), p >0\\
 H^\circ  (\mathscr{O}, \u{\mathscr{F}}) & = \Gamma (X,
 \u{\mathscr{F}}).  
\end{align*}

\section{Cohomology of the space \texorpdfstring{$X$}{X}}\label{chap13:sec2}

Let $\mathscr{O} = \{\mathscr{O}_i\}_{i\in I}$, $\Omega =
\{\Omega_\alpha\}_{\alpha \in A}$ be two (indexed) coverings of $X$
and suppose that $\Omega$ is a refinement of $\mathscr{O}$, i.e.,
there is a mapping $\phi: A \to I$ such that $\Omega_\alpha \subset
\mathscr{O}_{\phi(\alpha)}$. (We do not consider $\phi$ as given once
for all, but merely require its existence). The mapping $\rho$ of
$\mathscr{C}^p (\mathscr{O}, \u{\mathscr{F}})$ in $\mathscr{C}^p
(\Omega,\u{\mathscr{F}})$ defined by  
$$
(\rho_c)_{\alpha_\circ \ldots \alpha_p} = \text{ restriction  of }
c_{\phi(\alpha_\circ) \ldots \phi (\alpha_p)}  \text{ to }
\Omega_{\alpha_\circ, \ldots, \alpha_p} 
$$
induces a mapping $\rho^\ast$ of $H^p(\mathscr{O}, \u{\mathscr{F}})
\to H^p (\Omega, \u{\mathscr{F}})$ (this is easy to verify). 

\setcounter{proposition}{0}
\begin{proposition}\label{chap13:prop1}
$\rho^\ast$ does not depend on $\phi$.
\end{proposition}

\begin{proof}
Let $\phi$, $\psi$ be two mappings $A \to I$ such that $\Omega_\alpha
\subset \mathscr{O}_{\phi (\alpha)} \cap
\mathscr{O}_{\psi(\alpha)}$. Suppose that $A$ is totally ordered. For
$p=0$, the result is obvious since $H^\circ (\mathscr{O},
\u{\mathscr{F}}) = \Gamma (X, \u{\mathscr{F}})$ for every
$\mathscr{O}$. If $p \geq 1$, we define a mapping $k$ (``homotopy
operator''): $\mathscr{C}^{p+1} (\mathscr{O}, \u{\mathscr{F}}) \to
\mathscr{C}^p (\Omega, \mathscr{F})$ by 
$$
(kc)_{\alpha_\circ \ldots \alpha_p} = \sum\limits^p_{j=0 } (-1)^j
c_{\phi(\alpha_\circ) \ldots \phi(\alpha_j) \psi (\alpha_j) \ldots
  \psi (\alpha_p)}, 
$$
if\pageoriginale $\alpha_\circ < \alpha_1 < \ldots < \alpha_p$ in the
total order of $A$ and then define $kc$ uniquely to be an (alternate)
cochain. If the cochains corresponding to the maps $\phi$, $\psi$ are
$c'$, $c''$ in $\mathscr{C}^p (\Omega, \mathscr{F})$ respectively, it
can be verified that 
$$
(k \delta + \delta k) c = c' - c''.
$$
Consequently, if $c$ is a cocycle (i.e., $\delta c = 0$), then $c' -
c''$ is a coboundary, $c' - c'' = \delta (kc)$ and the mappings $\phi$
and $\psi$ induce the same homomorphism of $H^p (\mathscr{O},
\u{\mathscr{F}}) \to H^p (\Omega, \u{\mathscr{F}})$ . This proves
Proposition \ref{chap13:prop1}. 

This homomorphism is denoted by $\sigma (\mathscr{O}, \Omega)$. It
satisfies certain obvious transitivity properties (as a functions of
$\mathscr{O}$, $\Omega$).

If $p=0$, $\sigma$ is always an isomorphism as observed above.
\end{proof}

\begin{proposition}\label{chap13:prop2}
If $p =1$, $\sigma$ is a monomorphism.

We have to show that if $\phi: A \to I$ is such that $\Omega_\alpha
\subset \mathscr{O}_{\phi (\alpha)}$ and $c=0$ in $H^1 (\Omega,
  \u{\mathscr{F}})$, then $c =0$ in $H^1 (\mathscr{O},
  \mathscr{F})$. Let $c$ be a cochain with $c_{\phi(\alpha)\varphi(\beta)}
  = \gamma_\beta - \gamma_\alpha$ in $\Omega_{\alpha \beta}$. For
  every $i$ and $x \in \mathscr{O}_i$ if $x \in\Omega_\alpha$, set
  $c_i(x) = \gamma_\alpha(x) + c_{\phi(\alpha)i} (x)$. If $x$ is also
  in $\Omega_\beta$, then $\gamma_\beta(x) + c_{\phi(\beta)i} (x) =
  \gamma_\alpha (x) + c_{\phi(\alpha)i}(x)$ since $\gamma_\beta -
  \gamma_a = c_{\phi(\alpha) \phi (\beta)} = c_{\phi(\alpha) i} -
  c_{\phi(\beta)i}$. Hence this defines a section $c_i$ on
  $\mathscr{O}_i$ and clearly $c_{ij} = c_i - c_j$ in
  $\mathscr{O}_{ij}$, which proves the proposition.

The homomorphism $\sigma (\mathscr{O}, \Omega)$ defined above depends
only on the coverings $\mathscr{O}, \Omega$. If $\dot{\mathscr{O}}, \Omega$
are refinements of one another, $\sigma (\mathscr{O}, \Omega)$ is an
isomorphism. Hence we identify all coverings which are two by two
refinements of one another, and consider the class of all indexed
coverings modulo this identification. It is clear that this quotient
can be put in one-one correspondence with a subclass of the power set
of $X$ and so is a set. It is clearly a directed set and we have a
directed system 
$$
\{H^p (\mathscr{O}, \u{\mathscr{F}}), \sigma (\mathscr{O},
\Omega)\}_{\mathscr{O}} 
$$\pageoriginale
for every $p$. The direct limit of this system is called the
\textit{$p$-th cohomology group of $X$ with coefficient sheaf
  $\u{\mathscr{F}}$} and is denoted $H^p (X, \u{\mathscr{F}})$. It is
obvious that $H^\circ (X, \u{\mathscr{F}}) = \Gamma(X,
\u{\mathscr{F}})$. 
\end{proposition}


\section{The exact cohomology sequence}\label{chap13:sec3}
Let 
$$
0 \to \u{\mathscr{F}} \xrightarrow{i} \u{\mathscr{G}}
\xrightarrow{\eta} \u{\mathscr{H}} \to 0
$$
be an exact sequence of sheaves. This evidently gives rise to an exact
sequence
$$
0 \to \mathscr{C} (\mathscr{O}, \u{\mathscr{F}}) \to \mathscr{C}
(\mathscr{O}, \u{\mathscr{G}}) \to \mathscr{C} (\mathscr{O},
\u{\mathscr{H}}) 
$$
but the last mapping is not in general onto. If we denote by
$\mathscr{C}_a (\mathscr{O}, \u{\mathscr{H}})$ the image $\mathscr{C}
(\mathscr{O}, \u{\mathscr{G}})$ (group of ``cocha\^ines
ascensionelles'') then we obtain the exact sequence
$$
0 \to \mathscr{C} (\mathscr{O}, \u{\mathscr{F}}) \to \mathscr{C}
(\mathscr{O}, \u{\mathscr{G}}) \to \mathscr{C}_a (\mathscr{O},
\u{\mathscr{H}}) \to 0.
$$
It $Z^p_a$ is the group of cochains $c$ in $\mathscr{C}^p_a
(\mathscr{O}, \u{\mathscr{H}})$ with $\delta c = 0$ and $B^p_a$ is the
group of cochains $\delta c$, $c \in \mathscr{C}^{p-1}_a
(\mathscr{O}, \bar{\mathscr{H}})$, we define the group $H^p_a
(\mathscr{O}, \u{\mathscr{H}})$ by
$$
H^p_a (\mathscr{O}, \u{\mathscr{H}}) = Z^p_a / B^p_a. 
$$
We now define a mapping $d^\ast : H^p_a (\mathscr{O}, \u{\mathscr{H}})
\to H^{p+1} (\mathscr{O}, \u{\mathscr{F}})$ in the following way. Let
$h \in Z^p_a$ and let $h_{i_\circ \ldots i_p} =\eta (g_{i_\circ \ldots
i_p})$ where $g \in \mathscr{C}^p (\mathscr{C}, \u{\mathscr{G}})$;
also, since clearly $\eta$ and $\delta$ commute,
$\eta\{(\delta_g)_{i_\circ \ldots i_{p+1}} \} = (\delta h)_{i_\circ
  \ldots i_{p+1}} = 0$ since $h \in Z^p_a \cdot (\delta_g)_{i_\circ,
  \ldots i_{p+1}}$ being a section over $\mathscr{O}_{i_\circ \ldots
  i_{p+1}}$ which goes to 0 under $\eta$, $\delta g \in
\mathscr{C}^{p+1} (\mathscr{O}, \u{\mathscr{F}})$ (since
$\u{\mathscr{F}}$ is the kernel of $\eta$). It is easy to see that the
class of $\delta g $ in $H^{p+1} (\mathscr{O}, \u{\mathscr{F}})$
remains unchanged if $g$\pageoriginale is replaced by another cochain
$g'$ with $\eta g' =h$ and if $h$ is replaced bya cohomologous
cocycle. This defines $d^\ast$.

It is clear that $i$, $\eta$ induce homomorphisms
$$
i^\ast , \eta^\ast : H^p (\mathscr{O}, \u{\mathscr{F}})
\xrightarrow{i^\ast} H^p (\mathscr{O}, \u{\mathscr{G}})
\xrightarrow{\eta^\ast} H^p (\mathscr{O}, \u{\mathscr{H}})
$$
and it can be verified that the following sequence is exact:
\begin{align*}
& 0 \to H^\circ (\mathscr{O}, \u{\mathscr{F}}) \xrightarrow{i^\ast}
  H^\circ (\mathscr{O}, \u{\mathscr{G}}) \xrightarrow{\eta^\ast}
  H^\circ_a(\mathscr{O}, \u{\mathscr{H}}) \xrightarrow{d^\ast} H^1
  (\mathscr{O}, \u{\mathscr{F}}) \xrightarrow{i^\ast} \ldots \\
\ldots & \xrightarrow{d^\ast} H^p(\mathscr{O}, \u{\mathscr{F}})
\xrightarrow{i^\ast} H^p(\mathscr{O}, \u{\mathscr{G}})
\xrightarrow{\eta^\ast} H^p_a (\mathscr{O}, \u{\mathscr{H}})
\xrightarrow{d^\ast}  H^{p+1} (\mathscr{O}, \u{\mathscr{F}})
\xrightarrow{i^\ast}  \ldots 
\end{align*}
We can now define the groups $H^p_a (X, \u{\mathscr{H}})$ by taking
direct limits as the covering becomes finer, as for the groups $H^p(X,
\u{\mathscr{H}})$. Also there is a canonical mapping $H^p_a
(\mathscr{O} \u{\mathscr{H}}) \to H^p (\mathscr{O}, \u{\mathscr{H}})$
and so a canonical homomorphism $H^p_a (X, \u{\mathscr{H}}) \to H^p
(X, \u{\mathscr{H}})$. Since the operation of taking direct limits
commutes with exact sequences we obtain the exact sequence
$$
\ldots \xrightarrow{d^\ast} H^p (X, \u{\mathscr{F}})
\xrightarrow{i^\ast} H^p(X,\u{\mathscr{G}}) \xrightarrow{\eta^\ast}
H^p_a (X, \u{\mathscr{H}}) \xrightarrow{d^\ast} H^{p+1} (X,
\u{\mathscr{F}}) \xrightarrow{i^\ast} \ldots 
$$
It is of interest to decide when $H^p_a (X, \u{\mathscr{H}}) = H^p
(X,\u{\mathscr{H}})$. This is so in the case when $X$ is paracompact
(i.e., a Hausdorff space in which every covering admits a locally
finite refinement).

\begin{theorem*}
If $X$ is paracompact and 
$$
0 \to \u{\mathscr{F}} \xrightarrow{i} \u{\mathscr{G}}
\xrightarrow{\eta} \u{\mathscr{H}} \to 0
$$
an exact sequence of sheaves on $X$, then the canonical homomorphism
$$
H^p_a(X, \u{\mathscr{H}}) \to H^p (X, \mathscr{H})
$$
is an isomorphism.
\end{theorem*}

The theorem follows at once from the following

\begin{lemma*}
If $\mathscr{O} = \{\mathscr{O}_i\}_{i\in I}$ is a covering of $X$ and
$c \in \mathscr{C}^p (\mathscr{O}, \u{\mathscr{H}})$, there exists a
covering $\Omega = \{\Omega_\alpha\}_{\alpha \in A}$ and a mapping
$\phi: A \to I$ with $\Omega_\alpha \subset
\mathscr{O}_{\phi(\alpha)}$\pageoriginale such that the induced
homomorphism $\phi^\ast: \mathscr{C}^p (\mathscr{O}, \u{\mathscr{H}})
\to \mathscr{C}^p (\Omega, \u{\mathscr{H}})$ take $c$ to a cochain
$\phi^\ast (c) \in \mathscr{C}^p_a (\Omega, \u{\mathscr{H}})$. 
\end{lemma*}

\medskip
\noindent{\textbf{Proof of the lemma:}}
Since $X$ is paracompact, we may suppose $\mathscr{O}$ locally
finite. Since $X$ is \textit{normal}, (see Dieudonn\'e \cite{p3:key5})
there is an open covering $\{\mathscr{O}'_i\}_{i \in I}$ such that
$\bar{\mathscr{O}}'_i \subset \mathscr{O}_i$. For every $x \in X$, we
choose an open neighbourhood $\Omega_x$ of $x$ such that
\begin{itemize}
\item[(i)] $x \in \mathscr{O}_i$ (respectively $\mathscr{O}'_i$)
  implies $\Omega_x \subset \mathscr{O}_i$ (respectively
  $\mathscr{O}'_i$). 

\item[(ii)] $\Omega_x \cap \mathscr{O}'_i \neq 0$ implies $\Omega_x
  \subset \mathscr{O}_i$. 

\item[(iii)] If $x \in \mathscr{O}_{i_\circ \ldots i_p}$, there is a
  section $S$ of $\u{\mathscr{G}}$ over $\Omega_x$ such that $\eta(s)
  = c_{i_\circ \ldots i_p}$ on $\Omega_x$. 
\end{itemize}

Since $\mathscr{O}$ is locally finite, it follows from the definition
of quotient sheaf that (iii) can be fulfilled; and (i) and (ii) are
then ensured if we choose the $\Omega_x$ small enough. This gives us a
covering $\{\Omega_x\}_{x \in X} = \Omega$ of $X$; we choose a mapping
$\phi: X \to I$ such that $\Omega_x \subset
\mathscr{O}'_{\phi(x)}$. It is then easy to verify that $\Omega$ and
$\phi$ have the property stated in the lemma.
