\thispagestyle{empty}

\begin{center}
{\Large\bf Lectures on}\\[5pt]
{\Large\bf Expansion Techniques In}\\[5pt]
{\Large\bf Algebraic Geometry}
\vfill

{\bf By}
\medskip

{\large\bf S.S. Abhyankar}
\vfill


{\bf Tata Institute Of Fundamental Research}
\medskip

{\bf Bombay}
\medskip

{\bf 1977}
\end{center}
\eject


\thispagestyle{empty}
\begin{center}
{\Large\bf Lectures on}\\[5pt]
{\Large\bf Expansion Techniques In\\[5pt] Algebraic Geometry}
\vfill

{\bf By}
\medskip

{\large\bf S.S. Abhyankar}
\vfill

{\bf Notes by}
\medskip

{\large\bf Balwant Singh}
\vfill

{\bf Tata Institute Of Fundamental Research}
\medskip

{\bf Bombay}
\medskip

{\bf 1977}
\end{center}
\eject

\thispagestyle{empty}
\begin{center}

~

\vfill

{\bf \copyright \quad Tata Institute Of Fundamental Research, 1977}
\medskip

\parbox{0.7\textwidth}{No part of this book may be reproduced
in any form by print, microfilm or any
other means without written permission
from the Tata Institute of Fundamental
Research, Colaba, Bombay 400005}
\vfill

Printed in India
\medskip

{\bf By}
\medskip

K.P. Puthran at Tata Press Limited, Bombay 400025
\medskip

and published by K.G. Ramanathan for the 
\medskip

{\bf Tata Institute of Fundamental Research,}
\medskip

{\bf Bombay} 400005
\end{center}

\eject

\chapter{Preface}


These notes are based upon my lectures at the Tata Institute from
{\small November} 1975 to March 1976 and further oral communication between me
and the note taker. 

The notes are divided into two parts. In \S 8 or Part One we prove the
Fundamental Theorem on the structure of the coordinate ring of a
meromorphic curve and its value group. We then give some applications
of the Fundamental Theorem, the principal one among them being the
Epimorphism Theorem. The proof of the Main Lemmas (\S\ 7) presented
here is a simplified version of the original proof of Abhyankar and
Moh. The process of simplification started with my lectures at Poona
University in 1975 and culminated into the present version during my
lectures at the Tata Institute. The simplification resulted mainly
from the keen and stimulating interest in my lectures shown by the
audience at these two places, especially at the Tata Institute. 

In Part Two we record some progress on the Jacobian problem, which is
as yet unsolved. The results presented here were obtained by me during
1970-71. Partial notes on these were prepared by M. van der Put and
W. Heinzer at Purdue University in 1971. However, since the notes were
not complete, they were never formally circulated. 

I wish to thank the Tata Institute for inviting me and providing me
with an opportunity to give these lectures. My special thanks go to
Balwant Singh who took over the task of recording the lectures and
preparing these notes entirely on his own even to the extent of
relieving me of the tedium of having to read and check the
manuscript. 
\vskip 1cm

\hfill{\large\bf S. S. Abhyankar}
