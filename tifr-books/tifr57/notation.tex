
\chapter{Notation}


The following notation is used in the sequel.

The set of integers (\resp non-negative integers, positive integers,
real numbers) is denoted by $\mathbb{Z}$ (\resp $\mathbb{Z}^+$,
$\mathbb{N}$, $\mathbb{R}$). We write card $(S)$ for the cardinality
of a set $S$ and we write $\inf (S)$ (\resp $\sup (S)$) for the
infimum (\resp supremum) of a subset $S$ of $\mathbb{R}$. If $T$ is a
subset of a set $S$ then $S-T$ denotes the complement of $T$ in
$S$. If $k$ is a field and $n$ is a positive integer, we denote by
$\mu_n(k)$ the group of $n$th roots of unity in $k$. For $w \in \mu_n
(k)$ we write ord$(w)$ for the order of $w$ i.e., ord$(w)$ is the
least positive integer $r$ such that $w^r=1$. 

Suppose. in a given context, $k$ is a fixed field. We then denote by
the symbol $\diameter$ a generic (i.e. unspecified) non-zero element
of $k$. Thus if $k'$ is a ring containing $k$ and $a \in k'$ then $a=
\diameter$ means that $a \in k$ and $a \neq 0$. Similarly, $b=
\diameter c$ means that $b=ac$ for some $a \in k$, $a \neq 0$. Note
that $a = \diameter$, $b= \diameter$ does not mean that $a=b$. 

