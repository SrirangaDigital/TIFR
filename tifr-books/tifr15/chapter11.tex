\chapter[Application of Jensen's and Carleman's formulae...]{Application of Jensen's and Carleman's formulae
 to closure theorems}\label{chap11}%Lect 11 

\section{ Application of Jensen's formula}\label{chap11:sec1}%Sec 1

The\pageoriginale formula of Jensen allows us to have a condition of totality of a
set $\{ e^{\lambda z} \}$ in an open set $\Omega$. The relation 
$$
[\Phi (w) = \int\limits^{e^{wz}}_{\Omega} d \mu (z), \Phi (\Lambda) = 0, d
 \mu \in \mathscr{H}' (\Omega)] \Rightarrow \Phi = 0, 
$$
implies that $\{ e^{\lambda z} \}$ is total in $\Omega$. Let us see
what happens when $\Phi \nequiv 0$. 

If the convex closure of $\Omega$ is the intersection of the half
planes $x \cos \theta - y \sin \theta \le k (\theta)$, then $h(\theta)
= \limsup \limits_{r \rightarrow \infty} \log \dfrac{| \Phi (re^{i
 \theta})|}{r} \le k(\theta)- 2 \varepsilon, \varepsilon > 0 $. We
can find $r$ such that $R > r$ implies $\log |\Phi (Re^{i \theta})| <
(h(\theta) + \varepsilon) R$. Suppose $0 \notin \Lambda$, and denote
by $n_1(r)$ the distribution function of $\{|\lambda| \}$. Taking
$F(w) = \Phi(w)$ in Jensen's formula, we have the following: 
$$
\int^R_0 \frac{n_1(r)}{r} dr \le \frac{1}{2 \pi} \int^{2 \pi}_{0}
(h(\theta) + \varepsilon ) R d \theta + 0(1). 
$$
Since $h(\theta) \le k (\theta) - 2 \varepsilon$ we have the following
relation: 
$$
\frac{1}{R} \int^R_o \frac{n_1 (r)}{r} dr = \bar{D}^.(R) < \frac{1}{2
 \pi} \int^{2 \pi}_{o} k(\theta)d \theta - \varepsilon +
0(\frac{1}{R}) 
$$
In order that $\Phi \equiv 0$ it is sufficient that this condition is
not satisfied. In other words, if $\Phi(\Lambda) = 0, 0 \notin
\Lambda$, in order that $\Phi \equiv 0$ it is sufficient to have the
inequality 
$$
\bar{D}^._1 \ge \frac{1}{2 \pi} \int^{2 \pi}_{o} k(\theta)d \theta.
$$

In this formula, $\bar{D}^._1$ is the mean upper density of the sequence
$|\lambda_n|$. If $\Lambda$ is symmetric, $\Lambda = \{ \pm \lambda_n
\}$, the formula becomes 
$$
2 \bar{D}^. \le \frac{1}{2 \pi} \int^{2 \pi}_{o} k(\theta) d \theta~ (*)
$$
In\pageoriginale particular, we have the following result:

\begin{theorem*}
 If $\Lambda$ is symmetric and $C(w)= \prod (1 -
 \dfrac{w^2}{\lambda^2_n})$ is of exponential type zero on the real
 line, the conjugate diagram of $C(w)$ reduces to the segment $[- i
 \alpha, i \alpha], \alpha = \pi \bar{D}$.	 
\end{theorem*}

To prove it, we take $\Omega$)[$-i \alpha, i \alpha$] such that
$k(\theta)=\alpha \sin \theta | + \varepsilon $ If $\alpha < \pi \bar{D}^.$, the
formula $(*)$ will be satisfied for $\varepsilon$ sufficiently small
and $\{ e^{+i \lambda_{n^z}} \}$ will be total in
$\mathscr{H}_{\Lambda}(\Omega)$, which is false. Therefore $\alpha \ge
\pi \bar{D}^.$. As we know (Lecture 10, \S \ref{chap10:sec1}) $\alpha \le \pi
\bar{D}$, we have $\alpha = \pi \bar{D}^.$. 

We shall see later (Lecture 12, \S \ref{chap12:sec1}) that if $C(u)$ is bounded on
the real axis, one has $\alpha = \pi D^. = \pi D^. = \pi D$ 

\section{Application of Carleman's formula}\label{chap11:sec2}%Sec 2

We have see the use of Jensen's formula in a theorem of closure
involving the behaviour of the sequence $\Lambda$ in the whole
plane. But when we want to use only the behaviour of the sequence
$\Lambda$ in one half plane we can use the formula of Carleman. For
simplicity we suppose $\lambda_n$ are real and positive, and
$\bar{D}^. < \infty$(if
not, $\mathscr{H}_\Lambda (\Omega) = \mathscr{H}(\Omega)$ what ever be
the open bounded set. Moreover let $\Omega$ be contained in a horizontal
strip $\alpha \le v \le \beta$. Let $\Phi(w) = \int_\Omega e^{wz} d
\mu (z)$. We have the closure theorem $\mathscr{H}_\Lambda(\Omega) =
\mathscr{H}(\Omega)$, when the following relation holds: 
$$
[\Phi (\lambda) = 0~~ \text{for every } \lambda \in \Lambda ]
\Rightarrow \equiv 0. 
$$
Let us see what happens when $\Phi \nequiv 0$. Let $h(\theta)$ be the
type of $\Phi(w)$ along $\theta$, then $h(\pi/2) < - \alpha, h(- \pi /
2) < \beta$. Applying Carleman's formula we have the following
relations: 
\begin{align*}
 \int^R_o \left(\frac{1}{r} - \frac{r}{R^2}\right) dn(r) & = \frac{1}{2 \pi}
 \int^R_o \left(\frac{1}{y^2} - \frac{1}{R^2}\right) \log | \Phi (iy) \Phi
 (-iy)| dy + 0 (1) \\ 
 \int^R_o \left(\frac{1}{r^2} - \frac{1}{R^2}\right) n(r) dr & <
 \frac{1}{2 \pi} 
 (\beta - \alpha) \int^R_o \left(\frac{1}{y^2} - \frac{1}{R^2}\right) y dy +
 0(1) \\ 
 & = \frac{\beta - \alpha}{2} \log R + 0(1) \\
 \frac{1}{\log R} \int^R_o \frac{D(r)}{r} dr & < \frac{\beta- \alpha}{2
 \pi} + 0(1). 
\end{align*}

Suppose\pageoriginale $\bar{D}^. < \infty$; then
$$
\int^R_o \frac{dn(r)}{r} = \int^R_o \frac{D(r)}{r} + 0(1)= \int^R_o
\frac{\bar{D}(r)}{r} dr + 0(1). 
$$

\begin{defi*}%Def
 \begin{align*}
   \hat{D}^. & = \limsup_{R \rightarrow \infty} \frac{1}{\log R}
   \int^R_o \frac{dn(r)}{r} = \limsup_{R \rightarrow \infty}
   \frac{1}{\log R} \int^R_o (\frac{D(r)}{r} dr \\ 
   & = \limsup_{R \rightarrow \infty} \frac{1}{\log R} \int^R_o
   \frac{\bar{D}(r)}{r} dr 
 \end{align*}
 is defined to be the {\em logarithmic upper density of $\Lambda$}
\end{defi*}

We have $\hat{D}^. \le \bar{D}^. \le D^.$, since $\int^R_o
\dfrac{{D}(r)}{r} dr \le \log R(\bar{D}^. + \varepsilon)$. In order that
$\Phi \equiv 0$ it is sufficient to have $\hat{D}^. \ge \dfrac{\beta -
 \alpha}{2 \pi}$. Thus we have the closure theorem. 

\begin{theorem*}%Thm
  $$
  \hat{D}^. \ge \frac{\beta - \alpha}{2} \Rightarrow \mathscr{H}_\Lambda
  (\Omega) = \mathscr{H}(\Omega). 
  $$
\end{theorem*}

The constants occurring in the above inequality are the best
possible. To see this we take $\Lambda = N = (1, 2, \ldots)$ and
$\Omega$ a strip of width greater than $2 \pi$. We do not have closure
in this case and $\bar{D}^.= 1 < (\beta - \alpha) /2 \pi$. 

When $\Lambda$ and $\Omega$ are given, either Jensen's or Carleman's
formula can be applied to prove $\mathscr{H}_{\Lambda} (\Omega) =
\mathscr{H}(\Omega)$. Roughly specking, Carleman's formula is better
if $\Omega$ is ``flat'' enough, or if the part of $\Lambda$ which is in
some half-plane is ``scarce''. If we try to prove
$\mathscr{H}_{\Lambda}(\Omega) \neq \mathscr{H}(\Omega)$ the methods of
the lecture \ref{chap10} have be applied. 

\section{Theorem of closure on a compact set}\label{chap11:sec3}%Sec 3

Let\pageoriginale $K$ be a compact set which does not divide the plane, and
$\mathscr{H}(K)$ be the space of functions, continuous on $K$ and
holomorphic in the interior of $K$; the polynomials form a total set
in $\mathscr{H}(K)$ (Theorem of Mergelyan, Lecture \ref{chap8}). Let
$\mathscr{H}(K)$ be the closed span of $\{ e^{\lambda z}\}_{\lambda
 \in \Lambda}$ in $\mathscr{H}(K)$. We want to find when
$\mathscr{H}_{\Lambda}(K) = \mathscr{H}(K)$. Let 
\begin{equation}
 \Phi (w) = \int_K e^{wz} d \mu (z) \tag{*}
\end{equation}
with $\Phi(\Lambda) = 0$. In order that $\mathscr{H}_{\Lambda}(K) =
\mathscr{H}(K)$ it is sufficient that $(*)$ implies $\Phi \equiv 0$
and $\mathscr{H}_{\Lambda}(K) \neq \mathscr{H}(K)$ if there exists a
$\Phi \nequiv 0$ satisfying $(*)$. Suppose $\Phi \nequiv 0$. We have the
following majorization: 
$$
|\Phi(w)| <\max_{z \in K}| e^{zw}| \int_K |d \mu|
$$

Now we apply Jensen's formula, as in \S \ref{chap11:sec1}. We have the inequality:
$$
R \bar{D}_1 (R) < \frac{1}{2 \pi} R \int^{2 \pi}_{0} k(\theta) d \theta + 0(1)
$$
where $k(\theta) = \max\limits_{Z \in K} Re(z e^{i \theta} ) =
\max\limits_{Z \in K} (x cos \theta - y sin \theta )$, and $\bar{D}_1
(R)$ is the function of mean density of the sequence $\{| \lambda|
\}_{\lambda \in \Lambda}$(then, if $\Lambda = \{ \pm \lambda_n \},
\bar{D}_1 (R) = 2 \bar{D}(R), \bar{D}(R)$ being the function of mean
density of the sequence $\{ | \lambda_n |\}$. 

\begin{theorem*}%Thm
 In order that $\{ e^{\lambda z}\}_{\lambda \in \Lambda}$ be total in
 $\mathscr{H}(K)$ it is sufficient that $\limsup\limits_R (R
 \bar{D}_1 (R) - \frac{1}{2 \pi} R \int^{2 \pi}_{0} k (\theta)
 d\theta ) = \infty$. 
\end{theorem*}

In particular, let $K = [- \pi, \,\pi ]$. Then $\mathscr{H}_{\Lambda}(K)
= \mathscr{H}(K) = \mathscr{C}(K)$ when the following condition is
satisfied: 
\begin{equation}
 \limsup (R \bar{D}(R) - 2 R) = \infty \tag{1}\label{chap11:sec3:eq1}
\end{equation}

The\pageoriginale result is a very precise one. For example, take $\Lambda = \pm n =
Z - \{ 0\}$. Then $R \bar{D}_1 (R) = 2R - \log R + 0(1)$; the
condition (\ref{chap11:sec3:eq1}) is not satisfied, and it is easily seen that
$\mathscr{C}_{\Lambda}(K) \neq \mathscr{C}(K)$. If we add one element
$\alpha \neq 0$ to $\Lambda$, we add log $R + 0(1)$ or $R
\bar{D}_1(R)$; the condition (\ref{chap11:sec3:eq1}) is not yet satisfied, and in fact
$\mathscr{C}_{\Lambda + \{ \alpha \}} (K) \neq \mathscr{C}(K)$ (this
is very easy to see if $\alpha = 0$, because the functions take equal
values at $\pi$ and $- \pi$, and still holds for $\alpha \neq 0$). But
if we add two elements $\alpha, \beta \neq 0$ to $\Lambda$
(\ref{chap11:sec3:eq1}) is
satisfied, and $\mathscr{C}_{\Lambda + \{\alpha \} + \{ \beta\}}(K) =
\mathscr{C}(K)$. 

\section{Theorem of Muntz}\label{chap11:sec4}%Sec 4

We shall see what happens if we try to apply Carleman's formula in the
case of a line segment. We can get a finer result by having a
condition of totality in an infinite interval. For simplicity we can
take the half- line $L = (- \infty, 0)$ Let $\mathscr{C}_0(L)$ be the
space of functions, continuous on $L$ and vanishing at infinity. We
take $\Lambda$ to be a sequence of positive numbers. In order that
$\{e^{\lambda z} \}$ be non-total it is necessary and sufficient that
there exists a measure $d \mu$ orthogonal to $\{e^{\lambda z} \}$ and
not orthogonal to $\mathscr{C}_o(L)$. Then $\Phi \equiv 0$ where
$\Phi(w) = \int\limits^0_{-\infty} e^{wz} d\mu(z)$ with $\Phi (\Lambda) =
0$.For $u \ge 0$ we have $|\Phi (w)| < \int |d \mu|$. Applying
Carleman's formula we have $\int^R_o \dfrac{D(r)}{r} dr = 0(1)$, where
$D(r)$ is the function of density of $\Lambda$. As $rD(r)\nearrow$ we
have 
$$
\int\limits^{\infty}_R \frac{RD(R)}{r^2} dr, < ~\int\limits^\infty_R
\frac{D(r)}{r} dr = 0(1) 
$$
then $D = 0$. Then $\int^R_0 \dfrac{dN(r)}{r} = D(R) + \int^R_o
\dfrac{D(r)}{r}dr =0 (1)$. This means $\sum \limits_{\lambda \in
 \Lambda} \dfrac{1}{\lambda} < \infty$. Therefore $\{e^{\lambda z}
\}$ is total whenever $\sum \dfrac{1}{\lambda} = \infty$. Conversely,\pageoriginale
if $\sum \dfrac{1}{\lambda} < \infty$, we do not have totality. This
is proved by taking a function $\Phi(w)$ of exponential type with $\Phi
(\Lambda) = 0$ and small at infinity. For the construction of $\Phi
(w)$, we take 

 $\Phi (w) = 	\prod \limits_{\lambda \in \Lambda} \dfrac{\sin \pi w
 / \lambda}{\pi w/ \lambda}, |\Phi (u)| = \prod \limits^{3}_{1} \prod
\limits^{\infty}_4 = 0\left(\dfrac{1}{u^2}\right) 0 (1)$ and 
 
$\Phi(w)$ is of exponential type ( we have already seen such a
constructed in Lecture $6, \S 1.$). Thus $\Phi(w)$ is the Fourier
transform of a measure, $\Phi (w) \neq 0$ and $\Phi(\Lambda) = 0$. 

Thus we have the following closure theorem.

\begin{theorem*}%Thm
  In order that $\{e^{\lambda z} \}_{\lambda \in \Lambda, \lambda >
    0}$ be total in $\mathscr{C}_0 (0, \infty)$ it is necessary and
  sufficient that $\sum \limits_{\lambda \in \Lambda}
  \frac{1}{\lambda} = \infty$. 
\end{theorem*}

This theorem gives us, by making the transformation $e^z = x,
e^{\lambda z} = x^{\lambda}$ the theorem of Muntz, viz.,
$\{x^{\lambda} \}$ is total in $\mathscr{C}_o [0,1]$ if and only if
$\sum \dfrac{1}{\lambda} = \infty. (\mathscr{C}_o [0, 1]$ being the
subspace of $\mathscr{C}[ 0,1]$ consisting of the $f$ vanishing at
zero). 

The proof of the above theorem gives also the following result:

\textit {In each closed interval} $I \{ \mathscr{C}_\Lambda (I)=
\mathscr{C}(I), \Lambda ~\text{positive}~ \} \Leftrightarrow \sum \limits
_{\lambda \varepsilon \Lambda} \frac{1}{\lambda} = \infty$ and also
$\{\mathscr{C}_{\Lambda}(I) = \mathscr{C}(I), \Lambda ~\text{negative}~ \}
\Leftrightarrow \sum \limits_{\lambda \in \Lambda}
\frac{1}{\lambda} = \infty$.

We shall complete the last result in Lecture 16 \S \ref{chap16:sec1}.
