\chapter{Introduction}\label{chap1}

\section{Scope of the lectures}\label{chap1:sec1}%Sec 1

In\pageoriginale the course of these lectures, we shall consider mean periodic
functions of one variable. They are a generalization of periodic
functions - a generalization carried out on a basis different from
that of the generalization to almost periodic functions. This
generalization enables us to consider questions about periodic
functions such as Fourier-series, harmonic analysis, and later on, the
problems of uniqueness, approximation and quasi-analyticity, as
problems on mean periodic functions. For instance, the problems posed
by $S$. Mandelbrojt (Mandelbrojt $1$) can be considered as problems
about mean periodic functions. In the two final lectures we shall
consider mean periodic functions of several variables. 

\section{Definition of mean periodic functions}\label{chap1:sec2} %Sec 2

We consider three equivalent definitions of mean periodic functions. A
periodic function $f(x)$, defined on the real line $R$ and with period
${a}$, satisfies the equation $f(x) - f(x - a)=0$. This can
be written $\int f(x - y) d \mu(y) = 0$, where $d \mu (x)$ is a
measure which is the difference of the Dirac measures at $0$ and
a. For our generalization, we consider a continuous complex valued
solution of the integral equation 
\begin{equation*}
 \int f(x - y) d \mu (y) = 0 \tag{1}\label{chap1:sec2:eq1}
\end{equation*}
where $d\mu (y)$ is a measure with compact support, $\mu \neq 0$. [The
 support of\pageoriginale a continuous function $f(x)$ is the smallest closed set
 outside which $f(x)$ vanishes. [The support of a measure $d \mu$ is
 the smallest closed set outside of which the integral of any
 continuous function $f(x)$ (integral with respect to $d \mu$) is
 zero.] 

\begin{defi}\label{chap1:sec2:def1}%defini 1.
 Mean periodic functions are continuous complex-valued solutions
 of homogeneous integral equations of convolution type (\ref{chap1:sec2:eq1}). 
\end{defi}

The introduction of mean periodic functions as solutions of
(\ref{chap1:sec2:eq1}) is
due to $J$. Delsarte (Delsarte 1). His definition differs slightly
from ours in that he takes $d \mu (y) = K(y) ~dy$, where $K(y)$ is
bounded. When $K(y) =1$ on ($0, a$), and zero otherwise, the solutions
of (\ref{chap1:sec2:eq1}) are $a$-periodic functions, whose mean value is zero. This is
the reason why Delsarte calls a function whose mean is zero, in the
sense of (\ref{chap1:sec2:eq1}), a ``mean-periodic'' function. 

We study the equation (\ref{chap1:sec2:eq1}), for a given $d \mu$.

\begin{enumerate}[(a)]
\item The solutions of (\ref{chap1:sec2:eq1}) form a vector space over the complex
 numbers $C$. If a sequence $\{ f_n \}$ of solutions of (\ref{chap1:sec2:eq1})
 converges uniformly on every compact set, to $f$, then $f$ is again
 a solution of (\ref{chap1:sec2:eq1}). For, (\ref{chap1:sec2:eq1}) can
 be written as  
 $$
 \displaylines{\hfill
 \int f(y) d \nu_x (y) = 0, d \nu_x (y) = d \mu (x-y)
 \hfill (1')\cr
 \text{and}\hfill 
 \int f_n (y) d \nu_x (y) = 0 \Longrightarrow \int f(y) d \nu(y) =
 0.\hfill }
 $$
 With $f$, each translate $f_{\alpha}(x) = f(x - \alpha)$ is also a
 solution\break of~(\ref{chap1:sec2:eq1}). Thus it is natural to consider the topological
 vector space $\mathscr{C} = \mathscr{C} (R)$ over $C$ all
 complex-valued continuous functions $f$ with the topology of compact
 convergence (uniform convergence on each compact). The solutions
 of (\ref{chap1:sec2:eq1}) form a closed subspace of $\mathscr{C}$, invariant under
 translations. 
\item Solutions\pageoriginale of (\ref{chap1:sec2:eq1}) which are of a certain special type are easy
 to characterize: viz. those of the form $f(x) = e^{i \lambda x}$
 where $ \lambda$ is a complex number. For these solutions we have 
 \begin{equation*}
 \int e^{-i \lambda y} d \mu (y) = 0. \tag{2}\label{chap1:sec2:eq2}
 \end{equation*}
\end{enumerate}

Other solutions of a similar type are of the form $f(x) = P(x) e^{i
 \lambda x}$, where $P(x)$ is a polynomial. For $P(x)e^{i \lambda
 x}$ to be a solution, we must have $\int P(x - y) e^{-i \lambda y}
d \mu (y) = 0$ for all $x$. Considering this for $n+1$ different
values of $x(n = \deg P)$, we find that, for $P(x) e^{i \lambda x}$
to be a solution of (\ref{chap1:sec2:eq1}), it is necessary and sufficient that 
\begin{equation*}
 \int y^m e^{-i \lambda y} d \mu (y) = 0 \qquad (0 \leq m \leq n =
 \deg P) \tag{3}\label{chap1:sec2:eq3} 
\end{equation*}
(a system of $n+1 $ equations).

It is easy to solve (\ref{chap1:sec2:eq2}) and (\ref{chap1:sec2:eq3}) if one considers $M(w)$, the
Fourier transform of $d \mu : M (w) = \int e^{-i wx} d \mu (x)$. $M(w)$
is an entire function. In order that $e^{i \lambda x}$ be a solution
of (\ref{chap1:sec2:eq1}) it is necessary and sufficient that $M (\lambda) =0$. In order
that $P (x) e^{i \lambda x}$ be a
solution of (\ref{chap1:sec2:eq1}) it is necessary and sufficient that $M(\lambda) =
M^{(1)} (\lambda) = \cdots = M^{(n)} (\lambda) = 0$ where $M^{(j)} (w)
= (-i^j \int x^j e^{-iwx} d \mu (x)$. These conditions follow from
(\ref{chap1:sec2:eq2}) and from (\ref{chap1:sec2:eq3}). Thus the study
of the solutions of (\ref{chap1:sec2:eq1}) reduces 
to the study of the zeros of a certain entire function $M(w)$. 

We call ``simple solutions'' the solutions $e^{i \lambda x}$ of
(\ref{chap1:sec2:eq2})
and $P(x) e^{i \lambda x}$ of (\ref{chap1:sec2:eq3}). In French, the products $P (x)
e^{i \lambda x}$ of a polynomial and an exponential are called
``exponentielles-polyn$\hat{o}$mos''; the linear combinations of $e^{i 
 \lambda x}$ are often called ``polyn$\hat{o}$mes exponential''
i.e.\pageoriginale exponential polynomials; to avoid confusion, we shall use the
term ``polynomial-exponentials'' for the translation of the word
``exponentials polynomes'' - then, the English order of the term is
better than the French one -; if $P (x)$ is a monomial, $P(x) e^{i
 \lambda x}$ will be called ``monomial exponential". We use this
terminology later. Linear combinations of simple solutions and their
limits are again solutions of (\ref{chap1:sec2:eq1}). We are thus led to another
natural definition of mean-periodic functions (as a generalization of
periodic functions). 

\begin{defi}\label{chap1:sec2:def2}%defini 2
 Mean-periodic functions are limits in $\mathscr{C}$ of linear
 combinations of polynomial exponentials $P(x) e^{i \lambda x}$ which
 are orthogonal in the sense of (\ref{chap1:sec2:eq3}) to a measure with compact
 support.
\end{defi}

A third natural definition occurs if we consider the closed
linearsubspace of $\mathscr{C}$ spanned by $ f \in \mathscr{C}$ and
its translates: call this space $\tau (f)$. 

\begin{defi}\label{chap1:sec2:def3}%defini 3
 $f$ is mean-periodic if $ \tau (f) \neq \mathscr{C}$. 
\end{defi}

 This intrinsic definition is due to L. Schwartz (Schwartz $1$). In
 order that $f$ be a solution of (\ref{chap1:sec2:eq1}) it is
 necessary that $ \tau 
 (f) \neq \mathscr{C}$. Thus (\ref{chap1:sec2:eq1}) implies
 (\ref{chap1:sec2:eq3}). Also (\ref{chap1:sec2:eq2})
 implies (\ref{chap1:sec2:eq1}). We prove the equivalence later. We
 take (\ref{chap1:sec2:eq3}) as 
 the definition of mean-periodic functions since it is intrinsic
 and allows us to pose the problems of harmonic analysis and
 synthesis in the greater generality. 

\section{Problems considered in the sequel}\label{chap1:sec3}%Sec3

We consider the following problems in the sequel.

\begin{enumerate}[(1)]
\item Equivalence of the definitions (\ref{chap1:sec2:def1}),
  (\ref{chap1:sec2:def2}) and (\ref{chap1:sec2:def3}).
\item Harmonic\pageoriginale analysis and synthesis.

 Definition \ref{chap1:sec2:def3} permits us to consider this problem and its solutions
 allows us to prove (\ref{chap1:sec2:eq1}). 
\item Spectrum of a function; the relation between the spectrum and
 the properties of the function-for example, uniqueness and
 quasi-analy\-ticity when the spectrum has sufficient gaps. 
\item Relations between mean-periodic functions and almost
  periodic\break  functions.

 There are mean-periodic functions which are not almost periodic. For
 example $e^x$ is mean-periodic since $e^{x+1} -e. e^x = 0$. Being
 unbounded it is not almost periodic. There are almost periodic
 functions which are not mean periodic. Let $f(x) = \sum a_n e^{i
 \lambda_n x}$ be an almost periodic function. We can take for $\{
 \lambda_n \}$ a sequence which has a finite limit point. Then $f(x)$
 cannot be mean-periodic, as every $e^{i \lambda_n x} (a_n \neq 0)$
 belongs to $\tau (f)$ and no function $M(w) \neq 0$ (Fourier
 transform of $d \mu$) can vanish on $\{ \lambda_n \}$. 

 $(4a)$ Are bounded mean-periodic functions almost periodic?

 $(4b)$ What are the properties of $\{ \lambda_n \} $ in order that
 the almost periodic functions with spectrum $\{ \lambda_n \} $ be
 mean-periodic? 
\item Given a set $\left\{ P(x) e^{i \lambda x } \right\}$, is it
 total in $\mathscr{C}$? 

 A set is total in $\mathscr{C}$ if its closed span is
 $\mathscr{C}$. 
 If it is not, is it possible that $\left\{P (x) e^{i \lambda x }
 \right\}$ is total in $\mathscr{C} (I)$, where $I$ is an interval?
 For sets of the type $\{ e^{i \lambda_n x}\}$ this is the problem
 posed by Paley and Wiener (Paley- Wiener). 
\item If\pageoriginale $f \in \mathscr{C} (I)$, is it possible to extend $f$ to a
 mean-periodic function? 
\end{enumerate}

Problem (3) is related to questions in (Mandelbrojt 1) while
problem (5) is related to the work of Paley-Wiener, Mandelbrojt,
Levinson and Schwartz. Problems (5) and (6) can be posed
analogously for analytic functions in an open set of the plane. This
would give a new interpretation of some classical results on
Dirichlet series and give some new results too. 
