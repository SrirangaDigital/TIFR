\chapter{Continuation on the line}\label{chap17}

\section{Some definitions about positive sequences \texorpdfstring{$\big \{ \lambda_n
 \big\} = \Lambda$}{Lambda}}\label{chap17:sec1}%Sec 1 

We\pageoriginale give a few definitions with notations. We have already defined
(lecture 10) $n (r) D (r), D^., D., \bar{D}(r), \bar{D}^., \bar{D}_.$. We
define the following expressions to make list complete. 
$$
\hat{D}(r) = \frac{1}{\log r} \int^r_o \frac{dn (t)}{t} ; \hat{D}. =
\lim\sup_{r} \hat{D}(r) ; \hat{D}. = \lim\inf_{r} \hat{D}(r). 
$$

Taking $\lambda_n \nearrow$ we shall consider $\lim \sup (\lambda_{n +
 1} - \lambda_n)$ and $\lim \inf (\lambda_{n + 1} - \lambda_n)$. 

The maximum density $D_{\max}$ (minimum density $D_{\min})$ is the
lower bound (upper bound) of densities of sequences containing
(contained in) and having a density. They are given by the following
formulas (the first one is proved in appendix $1$; the second one
follows from the remark at the end. 
\begin{align*}
 D_{\max} & = \lim_{\zeta \to 1-0} \lim\sup_{r \to \infty} \frac{n(r)
 - n(\xi r)}{r - \xi r}\\ 
 D_{\min} & = \lim_{\zeta \to 1-0} \lim_{r \to \infty} \inf \frac{n
 (r) - n (\xi r}{r - \xi r} 
\end{align*}

We call a sequence \textit{regular} if $\lim \inf (\lambda_{n+1} -
\lambda_n) > 0$, and a sequence with density $D$ \textit{well
 distributed} if $\lambda_n - nD = 0(1)$. (In general, we have only
$\lambda_n - n D = o (n))$. 

We define the \textit{upper distribution density} $\triangle^.$ (lower
distribution density $\triangle.)$ to be the lower bound (upper bound)
of densities of well distributed sequences containing (contained in)
$\Lambda$. We have the following formulae to calculate them: 
\begin{align*}
 \triangle^. & = \lim_{h \to \infty} \lim_{r \to \infty} \sup \frac{n(r+h)- n(r)}{h}\\
 \triangle. & = \lim_{h \to \infty} \lim_{r \to \infty} \inf \frac{n(r+h) - n(r)}{h}
\end{align*}

\noindent \textbf{Proof (for $\triangle^.$):} % pro 
 set\pageoriginale $\triangle_h = \lim\limits_{r \to
 \infty} \sup \dfrac{n (r+h) - n(r)}{h}$. For sufficiently small
 $\in > 0$ there exists a well distributed sequence $\Lambda^*
 \supset \Lambda$ and having a density $D_h + \in $. Therefore
 $\triangle^. \le \underset{h \to \infty}{\lim \inf} \triangle_h$. On
 the other hand, if $\Lambda^*$ has a density $D^*$, we have $n^*(r +
 h) - n^* (r) = h D^* + 0(1)$ and $D^* > \triangle_h - o(1)$. So
 $\triangle^. \ge \lim \sup\limits_{h \to \infty}
 \triangle_h$. Therefore $\triangle^* = \lim\limits_{h \to \infty}
 \triangle_h$. 

Now $\triangle^. < \infty$ means that for an arbitrary given interval
of length $h$, there are only a bounded number of $\lambda_n$ in
it. $\triangle. < \infty$ implies that the sequence is relatively
dense in the sense of Bohr, i.e. there exists a well distributed
sequence contained in it. We have the following relations between
these densities: 
\begin{align*}
 \frac{1}{\lim \sup (\lambda_{n + 1} - \lambda_n)} \le \triangle. &
 \le D_{\min} \le D. \le \bar{D}. \le \hat{D}. \le \hat{D}^. \le
 \hat{\bar{D}}^. \le D^.\\ 
 D^. & \le D_{\max} \le \triangle^. \le \frac{1}{\lim \inf
 (\lambda_{n + 1} - \lambda_n)}\\ 
 \text { One can prove } \quad D^. & \le e \bar{D}^. (\text
    {Mandelbrojt} 2, p. 53). 
\end{align*}

If $\bar{D}^. = \bar{D}., D.= D^.$ and so $D_{\min} = D_{\max}$.

If $\Lambda_1 + \Lambda_2 = \Lambda, \Lambda$ having a density,
$D(\Lambda) = D_{\max} \Lambda_1 * D_{\min} \Lambda_2$. 

The definitions of well-distributed sequences, upper and lower
distribution densities can be immediately translated in case of real
non-symmetric sequences. We shall make use of these notions in the
next paragraph. 

\section{A problem of continuation on the line}\label{chap17:sec2}%sec 2

Suppose\pageoriginale $\Lambda$ to be real, and take the space $\mathscr{E}_\Lambda
(I)$. We are interested in finding conditions on $\Lambda$ and $I$ in
order that every $f \in \mathscr{E}_\Lambda (I)$ is a
restriction of an $f \in \mathscr{E}_\Lambda$. Moreover, if
\textit{on I} $ f $ belongs to a specified class of
$C^\infty$-functions, we are interested to know the related properties
of its continuation on the line. 

It can be natural to expect that if $\big | I \big | > 2 \pi D_{\max}
$ and if $f \in \mathscr{C}_\Lambda(I)$, then $f$ has a continuation so that
$f \in \mathscr{C}_\Lambda$. We shall give an example at the end of
this lecture which shows that this is not true even when $f \in
\mathscr{E}_\Lambda (I)$. 

Suppose $\Lambda$ is a real regular sequence and let $\big | I \big |
> 2 \pi \triangle^. $. Can every $C^\infty$-function $\in
\mathscr{C}_\Lambda (I)$ be continued into a $C^\infty$ - function in
$\mathscr{C}_\Lambda$? 

\begin{defi*} % def
  The class $C_I \big \{ M_n\big\}$ for a given sequence $\big \{
  M_n\big\}$ and a given interval $I$ is defined to be the set of
  $C^\infty$- functions on $I$ which verify the conditions $\big |
  f^{(n)} \big | < K M_n$ where $K$ depends only on $f$. We define $ C
  \big \{ M_n\big\} = C_{(- \infty, \infty)} \big \{ M_n\big\}$. 
\end{defi*}

\begin{theorem*} % the
 Suppose $f \in \mathscr{C}_\Lambda (I) \cap C_I \big \{ M_n\big\}$,
 $\Lambda$ real and regular, $M_n \nearrow | I | > 2 \pi
 \triangle^.$. Then $f$ can be continued into a function belonging to
 $\mathscr{C}_\Lambda \cap C\big \{ M_{n + p}\big\}$ for an integer
 $p = p(\Lambda, I)$. 
\end{theorem*}

It is sufficient to prove theorem for well distributed sequences,
since we can find $a \Lambda' \supset \Lambda$ with $| I | > 2 \pi
\triangle^.$. We suppose $I$ to be symmetric with respect to the
origin. 

Consider the canonical product
$$
C(w) = (w - \lambda_o ) \prod^\infty_{n = 1} \left(1- \frac{w}{\lambda_n}\right)
\left( 1 - \frac{w}{\lambda_n}\right) 
$$
we\pageoriginale use the following result of $B$. Levin about $C(w)$. (Levin 1
and 2; Mandelbrojt 3). $C(w)$ is of type $\Pi, | C (u) | < K (1+ |
u |^N)$, and $\big | C' (\lambda_n) \big | > K' / (1+ |\lambda_n|^N)$
where $K, K'$ and $N$ are dependent on the given sequence. 

$C(w)$ is not the transform of a measure but of a distribution. We
want to construct a distribution whose transform $M(w)$ has the same
type as $C(w)$ and satisfies the equation $M(w) - e^{iw X} = 0$
whenever $W \in \Lambda$. In order to construct $M(w)$, we take $M(w)
= C(w) A(w), A(w)$ having a polar part $\approx \sum \big \{ e^{( i
 \lambda_n X)}/ C'(\lambda_n)~(w-\lambda_n) \big\}$. To assure normal
convergence we take $M(w) = w^q C(w) \sum\limits_{n} \dfrac{e^{i
 \lambda_n X}} { \lambda^q_n C' (\lambda_n) (w- \lambda_n)}$. This
series is normally convergent outside circles of radius $\in $
around the $\lambda_n 's $ for $q > N + 2$. We want to have a uniform
majorization for $M_X (w)$. For this take a strip around the real
axis: in this strip we have (Phragmen Lindelof $\big | C(w) \big | <
K' | 1 + w |^N$; then in the strip minus these circles $\big | M_x
(w) \big | < K'' \big | w \big |^q \big | 1 + w \big|^N$ and (Cauchy)
$\big | M_X (u) \big | < \dfrac{K''}{1 + u^2} (1 + u^p)$ for $p$
even, $ p \ge N +q+2$. Thus $M_X (u) = \mathscr{C} (T_X), T_x = d
\mu_x + \dfrac{d^p}{dx^p} dv_x$ has support in $I$, and $\int_I \big |
d \mu_x \big |$ and $\int_{I} \big | dv_x \big |$ are uniformly
bounded. 

Let us consider $f \in \mathscr{E}_I (\Lambda)$ ; we can
continue $f$ at the point $x$ by the formula 
$$
f (X) = \langle f_j, T_x \rangle = \int_I (f d \mu_x + f^{(p)} dv_x).
$$

By our standard argument, the continued function belongs to
$\mathscr{E}_\Lambda$. More over, if $M_n \nearrow$ and $f \in
\mathscr{C}_I \{M_n\}$, then the continued function $\in \mathscr{C}
\big \{ M_{n + p}\big\}$. 

For a refinement of this result, see (Kahane 2).

In\pageoriginale this type of result $\triangle^.$ is the good density to
consider. Actually we cannot get more if we replace $\triangle^.$ by
$D_{\max}$. In order to show this we construct an example of $\Lambda$
and $f \in \mathscr{C}_\Lambda(I), f a ~ C^\infty$- function such that
$f \notin \mathscr{C}_\Lambda$ and $\Lambda$ has $\triangle^. = 1$ and
$D. = D_{\max} = 0$. 

We take $\Lambda$ to be a sequence of integers which are situated in
such a manner that $2n$ of them are in an interval $I_n$, the
intervals $I_n$ being disjoint and having lengths which increase
indefinitely. To do this we take $p_n = n^k, k> 1$ and the sequence
$\Lambda$ is the union of the sets $\big\{ p_n - n, p_n - (n - 1),
\ldots, p_n + n \big\}$. It is verified easily that $D = 0$ and
$\triangle^. = 1$. We shall construct a function $f \in
\mathscr{C}_\Lambda (I), I = [- \pi + \varepsilon, \pi - \in ]$
and not continuable in $\mathscr{E}_\Lambda$, nor even in
$\mathscr{D}'_\Lambda$. 

Let $\alpha (x) = \sum\limits_{- \infty}^{\infty} C_j e^{i jx}$. We
can choose $C_j$ in such a manner that $C_o = 1$ and $\alpha (x)$
vanishes on $I$. By the theorem of Denjoy Carleman we can have $\alpha
(x)$ to be not in any quasi-analytic class (lecture 19, \S \ref{chap19:sec2}), but
in the class $C \big \{ n^{\alpha n} \big\}$ for $\alpha > 1$. Then
$\big | \alpha^{(k)} (x) \big | < k^{\alpha k}$. As $\alpha^{(k)} (x)
= (i)^k \sum C_jj^k e^{i j x} je, \big | C_j \big | < \dfrac{| \alpha (k)
 |}{j^K}$. Hence $\big | C_j \big | < \min\limits_{n}
\dfrac{n^{\alpha^n}}{|j|^n} < e^{- j^\alpha}, \alpha' < 1$. Also
$\sum\limits_{| j | \ge n} | C_j | < e^{-n^\beta}, \beta < 1$. This
majorizes $S_n (x) = \sum\limits_{-n}^n C_j e^{i j x}$ on $I$. Let
$f(x) = \sum\limits_{n = 1}^\infty a_n e^{i p} n^x S_n (x)$. If
$\sum | a_n | e^{-n^\beta} < \infty, f$ is continuous on $I$. We can
take $a_n$ such that $\sum a_n p^m_n e^{-n^\beta} < \infty$ for every
$m$ and $a_n$ increasing more rapidly than any polynomial in $p_n$. On
account of the construction of $a_n$, $f (x)$ is infinitely
differentiable in $I$, but cannot be continued either in
$\mathscr{E}_\Lambda$ or in $\mathscr{D}'_\Lambda$ (because $a_n$
would be the Fourier coefficient of order $p_n$). 
