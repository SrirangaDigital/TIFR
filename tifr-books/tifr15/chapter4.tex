\chapter[Harmonic analysis for mean periodic functions...]{Harmonic analysis for mean periodic functions on
 the real line}\label{chap4}%Lect 4 

\section{Equivalence of definitions I and III}\label{chap4:sec1}%sec 1.

The\pageoriginale study of the problem of harmonic analysis and synthesis for mean
periodic functions will allow us to prove the equivalence of our
definitions of mean periodic functions. We take definition $III$ as
the basic definition, viz., $f$ is mean periodic if $\tau (f) \neq
\mathscr{C}$. By the condition of Riesz $\tau (f) \neq \mathscr{C}$
if and only if there exists a $d \mu \in \mathscr{C}'$, $d \mu \neq
0$, orthogonal to $f_y$ for every $y$. Writing this in an equivalent
form as 

$f * d \mu = 0, d \mu \neq 0$ we see that definitions $I$ and $III$
are equivalent. 

\section{Carleman transform and spectrum of a function}\label{chap4:sec2} %sec 2

As preliminary to the study of harmonic analysis and synthesis we
introduce the Carleman transform of a mean periodic function 
\begin{alignat*}{4}
 \text{We put}\qquad & f^+ (x) = 0, \quad &f^{-}(x) & =
 f (x) & \qquad &\text{for}~ x < 0 \\ 
 & f^{+} (x) = f(x), \quad &f^{-}(x) & = 0 & &\text{for}~ x \geq 0 
\end{alignat*}
From $(1)$ we can define $g(y) = f^{-} * d \mu = -f^+ * d \mu$.

\begin{lemma*}%Lem
 Segment of support of support of $g \subset $ segment of support of $d \mu $.
\end{lemma*}

Let [$a, b$] be the segment of support of $d \mu$. Then
\begin{alignat*}{4}
 g(y) &= - \int_o^\infty f(x) d\mu (y - x), &\quad &\text{ and }\qquad y < a
 \text{ implies } ~~\quad g(y) = 0.\\ 
 g(y) &= \int_{-\infty}^{o} f(x) d\mu (y - x), &&\text{ and }
 \qquad ~y < b \text{ implies }\quad g(y) = 0 
\end{alignat*}

Let $M(w) = \int e^{-ixw} d \mu (x)$ and $G(w) = \int e^{-ixw} g(x) dx$.

\begin{defi*}%Def
 We\pageoriginale define the Carleman transform of a mean periodic function $f$ to
 be the meromorphic function $F(w) = G(w)/ M(w)$. 
\end{defi*}

This definition is independent of the measure $d\mu$. In fact suppose
$f*d\mu = 0$ and $f * d \mu_i = 0$. Let $g_1 = f^{-}* d \mu_1$. We
have $g_1 * d\mu = f^{-} * d \mu_1 * d \mu = f^{-} * d\mu * d \mu_1 =
g * d \mu_1$ and so $G(w) M(w) = G(w) M_1(w)$. 

The original definition of Carleman for functions which are not very
rapidly increasing at infinity cannot be applied to every mean
periodic function. But we shall see later that his definition
coincides with ours in the special case that $| f(x)| = 0
(e^{a|x|})$. (Lecture VI \label{chap6:sec3}). 

Is the quotient of two entire functions of exponential type the
Carleman transform of a mean periodic function? Let us look at this
question. 

If $F(w) = G(w) / M (w)$, $G(w)$ and $M(w)$ two functions of
exponential type, in order that $F(w)$ be the Carleman transform of a
function $f(x)$ it is necessary that if $a_G, b_G ; a_M, b_M$ are
the constants of condition $(2')$ (lecture \ref{chap3}) for $G$ and $M, a_M
\leq a_G \leq b_G \leq b_M$. Examples may be constructed to show that
this condition is not sufficient even to assert that $F(w)$ is the
Carleman transform of a mean periodic distribution. (A mean periodic
distribution is defined by the intrinsic property $\tau (T) \neq
\mathscr{D}'$ (lecture V ~ \ref{chap5:sec3}). 

We now proceed to relate the ``simple solutions'' to the poles of the
Carleman transform. 

\begin{lemma*}
 $$
 \displaylines{\hfill 
 f_1(x) = e^{- i \lambda x} f(x) \Rightarrow F_1 (w) = F(w + \lambda
 ).\hfill \cr
 \text{ Set } \hfill d \mu_1 = e^{-i \lambda x} d\mu, f_1 (x) = e^{-i
 \lambda x} f(x).\hfill }
 $$
\end{lemma*}

Then\pageoriginale $g_1 (x) = e^{-i \lambda x} g(x)$. For $f * d\mu =
0 \Rightarrow f e^{-i \lambda x} * e^{-i \lambda x} d \mu = 0$. So $f_1 * d \mu_1 =
0$ and  
\begin{align*}
 g_1 (x) &= f^{-}_1 * d \mu_1 = e^{-i \lambda x} f^{-} (x) * e^{-i
 \lambda x} d \mu (x) = e^{-i \lambda x} (f^{-} * d \mu )\\ 
 &= e^{-i \lambda x} g(x).
\end{align*}

Thus we have $G_1 (w) = G(w + \lambda )$ and so $F_1 (w) = F(w + \lambda )$.

Now $e^{i \lambda x} \in \tau (f) \Leftrightarrow (f * d \mu = 0
\Rightarrow e^{i \lambda x} * d \mu = 0 = M (\lambda ))$. 

$P(x) e^{i \lambda x} \in \tau (f) \Leftrightarrow P(x) \in \tau
(f_1), f_1 = e^{-i \lambda x} f(x)$. 

$P(x) \in \tau (f_1) \Leftrightarrow P(x + y) \in \tau (f_1)$ for every $y$.

Hence $P(x) \in \tau (f_1) \Leftrightarrow x^p \in \tau (f_1)$, $p =
0, 1, 2, \ldots \ldots n = $ degree of $P$. 

Now $x^p \in \tau (f_1), p = 0, 1, \ldots \ldots, n \Leftrightarrow$
for every $d \mu_1$ with $f_1 * d\mu_1 = 0$ 

we have
$$
M_1 (0) = M_1^{(1)} (0) = \ldots \ldots = M_1^{(n)} (0) = 0 (M_1 =
\mathscr{C} (d \mu_1)) 
$$
This proves
\begin{lemma*}%Lem
 \begin{gather*}
   P(x) e^{i \lambda x} \in \tau (f) \Leftrightarrow \text{for
     every } ~d \mu ~\text{with } f * d \mu = 0, \\
   M(\lambda ) = M^{(1)} = \ldots \ldots = M^{(n)} (\lambda ) = 0\\
   (M = \mathscr{C}(d \mu), n = \text{degree of } P)
 \end{gather*}
\end{lemma*}

We\pageoriginale shall prove that in this case $\lambda$ is a pole of order $\geq n$
( = \text{ degree of } $P$) of $F(w)$. 

\begin{theorem*}
  $P(x) e^{i \lambda x} \in \tau (f) \Leftrightarrow \lambda$ is a
  pole of order $\geq n$ (degree of $P$) of $F(w)$. 
\end{theorem*}

First we prove that $\lambda$ is a pole. Without loss of generality we
take $\lambda = 0$ (replacing $f$ by $f_1$, if necessary). Suppose
$M(0) = 0$. 

If $G(0) \neq 0$ there is nothing to prove. Suppose $G(0) = 0$. Then we can define
\begin{align*}
  d \mu_1 (x) &= \int_{-\infty}^x d \mu (t) -
  \int_x^\infty d \mu (t) \\ 
  g_1 (x) &= \int_{-\infty}^x g(t) dt = - \int_x^{\infty} g(t) dt\\ 
  M_1 (w) &= \int e^{-i x w} d\mu_1 (x) = \frac{-iM (w)}{w}\\
  G_1(w) &= \int e^{-i x w} g_1 (x) dx = \frac{-iG (w)}{w}
\end{align*}

Moreover we have $g_1 (x) = \int\limits_{-\infty}^0 f^{-} (t) d\mu_1
(x - t)$ because the derivatives are the same, and the functions are
$0$ when $x \rightarrow + \infty$. For a similar reason (with $x
\rightarrow -\infty$) we have $g_1 (x) = - \int f^+ (t) d \mu_1
(x-t)$. Thus we have $f * d\mu_1 = 0$ and so $M_1 (0) = 0$. If $G_1
(0) \neq 0$, then $F(w)$ has a pole at the origin. If $G_1 (0) = 0$ we
iterate this method and finally arrive at $M_m (w) = -iM_{m-1} (w)/ w,
G_m (w) = -iG_{m-1} (w)/w$ with $M_m (0) = 0$ and $G_m (0) \neq
0$. Thus we see that $\lambda$ is a pole. 

To see that $\lambda$ is a pole of order $\geq$ $n$ (degree of $P$) by
the above lemma, we have merely to replace the condition $M(0) = 0$ by
$M(0) = M^{(1)} (0) = \ldots$. $M^{(n)} (0) = 0$ in the above
construction. 

The\pageoriginale above construction gives us the following corollary.

\begin{coro*}%Corlry
  Suppose $M(\lambda) = 0$ and $\lambda$ is not a pole of $F(w)$. Then\break
  $M(w)/_{(w - \lambda )} = \mathscr{C} (p_\lambda )$ with $f *
  p_\lambda = 0$ and segment of support of $p_{\lambda} \subset$
  segment of support of $d \mu$. 
\end{coro*}

\begin{defi*}
 The spectrum $S(f)$ of a mean periodic function $f(x)$ is defined to
 be the set of poles of $F(x)$, each counted with its order of
 multiplicity. 
\end{defi*}

\section{The problem of harmonic synthesis}\label{chap4:sec3}%Sec 3

It will turn out that the only ``simple subspaces'' in $\mathscr{C}$ are
generated by the translates of a polynomial exponential. In order to
answer the problem of harmonic synthesis, we shall try to prove that
$f$ can be approximated by sums of polynomial exponentials belonging
to $\tau (f)$. 

\begin{lemma*}
 Suppose $d \nu \in \mathscr{C}'$ and $f$ mean periodic. Then $\varphi = f
 * d \nu$ is also mean periodic. 
\end{lemma*}

This results from $\varphi * d\mu = f * d\nu * d \mu = 0$.

We shall study the spectrum of $\varphi$. Since $d \nu$ has compact
support, for $| x | \geq x_0, x_0$ sufficiently large, $\varphi^{-} =
f^{-} * d\nu$ coincide. Thus we can write $\varphi^{-} = f^{-} * d\nu
+ h, h(x)$ being a function with compact support. Let $\Phi, N, H$
be the Fourier transforms of $\varphi, d \nu$ and $h$. We have the
following equations: 
\begin{align*}
  \varphi^{-} * d \mu &= (f^{-} * d\nu * d\mu ) + (h * d \mu )\\
  \Phi (w) M(w) &= G(w) N(w) + H(w) M(w) \\
  \Phi (w) &= F(w) N (w) + H(w)
\end{align*}

Thus\pageoriginale we have the following lemma :
\begin{lemma*}
 The spectrum of $\varphi = f * d\nu$ is the set of poles of $F(w) N (w)$.
\end{lemma*}

Now our problem requires for solution the result that $P_n (x) e^{i
 \lambda x} * d \nu = 0$ for every $P_n(x) e^{i \lambda x } \in \tau
(f)$ implies $f * d \nu = 0$. This is just the reformulation of the
problem using the condition of Riesz. Now $P_n (x) e^{i \lambda x} * d
\nu = 0$ for every $P_n (x) e^{i \lambda x} \in \tau (f)$ implies
$N(w)$ vanishes on the spectrum of $f$. Thus, $F(w) N(w)$ is an entire
function. By the above lemma, the solution of the problem of synthesis
will follow if we prove that a mean periodic function $f$ whose
spectrum is void is zero. 
