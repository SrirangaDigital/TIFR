\chapter{Mean Periodic Functions of Several Variables}\label{chap23}% Lecture 23

In\pageoriginale the case of several variables instead of considering continuous
function, we consider $C^\infty$-functions. It would also be possible
to consider distributions. 

We consider the space $\mathscr{E} = \mathscr{E}(R^n), f \in
\mathscr{E}, f(x) = f(x_1,\ldots, x_n), x = (x_1,\ldots, x_n) \in
R^n$. The monomial exponentials are of the form $x_1^{i_1} \cdots
x_n^{i_n}$ $e^{i (\lambda_1 x_1 + \cdots + \lambda_n x_n)}$ and the
polynomial exponentials are of the form $P(x_1,\ldots,\break x_n)$ 
$e^{i(\lambda_1x_1 + \cdots + \lambda_n x_n)}$. 

A function $f$ is mean periodic if $\tauup(f) \neq
\mathscr{E}(R^n)$. An equivalent definition is that $f$ is mean
periodic if there exists a $\mu \in \mathscr{E}', \mu \neq 0, \mu * f
= 0$. 

\begin{prob}\label{chap23:prob1}% problem 1
 If $f$ is mean periodic, can we assert that $f$ belongs to the span
 of monomial exponentials belonging to $\tauup (f)?$ 
\end{prob}

The answer is in the negative (Ehrenpreis $1$). In $R^2$, let $f = x_1
+ x_2$. The monomial exponentials in $\tauup(f)$ are constants. 

\begin{prob}\label{chap23:prob2}% problem 2
 Let $f \in \tauup \subset \mathscr{E}, \tauup $ a closed subspace
 invariant under translations. Is $f \in$ span of polynomial
 exponentials $\in \tauup?$ 
\end{prob}

Solution is not known.

\begin{prob}\label{chap23:prob3}% problem 3
 Suppose $f * \mu = 0, \mu \in \mathscr{E}'(R_n)$. Is $f \in$ span of
 polynomial exponentials $Q$ with $Q * \mu = 0?$ 
\end{prob}

This is answered in the affirmative (Malgrange $1$).

Let $V(\mu)$ denote the set of linear combinations of polynomial
exponentials verifying the equation $Q * \mu = 0$. 

Let\pageoriginale $\overset{v}{\nu}$ be the distribution symmetric to the
distribution $\nu$. If $\overset{v}{\nu} * Q = 0$ for every polynomial
exponential $Q$ verifying the equation $Q * \mu = 0$ and $f * \mu= 0$,
then is $\overset{v}{\nu} * f = 0?$ This problem is put in the
following form: 
$$
\{(\overset{v}{\nu} \in (V(\mu)^\perp), f* \mu = 0 \} \Rightarrow \nu * f = 0.
$$
\setcounter{theorem}{0}
\begin{theorem}\label{chap23:thm1}% theorem 1
 $f * \mu = 0 \Rightarrow f \in \overline{V(\mu)}$. This theorem
 follows from parts $a)$ and $e)$ of the following theorem (Malgrange
 $1$). 
\end{theorem}

\begin{theorem}\label{chap23:thm2}% theorem 2
 The following conditions are equivalent: 
 \begin{enumerate}[a)]
 \item $\overset{v}{\nu} \in [V(\mu)]^{\perp}$
 \item {\em{Every exponential polynomial $Q$ satisfying the equation
  $Q * \mu = 0$ also satisfies the equation $Q * \nu = 0$. In
  other words, $V(\mu) \subset V(\nu)$.}} 
 \item $\mathscr{C}(\mu) / \mathscr{C}(\nu)$ {\em{is an entire function}}.
 \item $\mathscr{C}(\mu)/ \mathscr{C}(\nu)$ {\em{is an entire
  function of exponential type}}. 
 \item $\nu \in \mu * \mathscr{E}'= $ {\em{ closed ideal generated by $\mu$}}.
 \end{enumerate}
\end{theorem}

We recall that $F(w)$ is of exponential type if $|F(w)| < A ~
e^{B|w|}, | w | = |w_1| + \cdots + | w_n |$. We prove the theorem in
four steps, inserting the lemmas and propositions which we require in
the four steps, inserting the lemmas and propositions which we require
in the proof of each step. 
\begin{enumerate}
\item $a) \Longleftrightarrow b)$. Indeed $\overset{\nu}{\nu} \in
 [V(\mu)]^\perp \Rightarrow \overset{v}{\nu}. Q = 0$ for every
 exponential polynomial $Q$ satisfying $Q * \mu = 0$, and so
 $\overset{\nu}{\nu} \in [V(\mu)]^\perp \Longleftrightarrow
 \overset{v}{\nu} * \tauup_a Q = 0$ for every translate $\tauup_a Q $
 of $Q$. 

 In other words, $\overset{v}{\nu} \in [V(\mu)]^\perp
 \Longleftrightarrow Q * \nu = 0$ for every exponential polynomial
 $Q$ satisfying $Q * \mu = 0$. 
\item $b \Longleftrightarrow c)$. Let $\mathscr{C}(\mu) = M(w); w =
 (w_1, \ldots, w_n)$. According to Schwartz's theory of the Fourier
 transforms of distributions, $b)$ is equivalent to the relation:
 $\{\mathscr{C} (\mu)\mathscr{C}(Q) = 0 \Rightarrow \mathscr{C}(\nu)
 \mathscr{C}(Q) = 0 \}$. So $c) \rightarrow b)$. 
\end{enumerate}

Now\pageoriginale $\mathscr{C}(Q)$ is a distribution with support at $\lambda$ and
we denote it by $D_\lambda$. $b)$ gives us that $m.D_\lambda= 0
\Rightarrow N.D_\lambda = 0$, where $M.D_\lambda$ is the product of
the distributions $M$ and $D_\lambda$ and so also $N.D_\lambda$. To
prove that $b) \Rightarrow c)$ we use the following lemma: 
\begin{lemma*}% lemma
 Let $M$ and $N$ be analytic at the origin. In order that $N/M$ be
 analytic at the origin it is necessary and sufficient that for
 every distribution $D$ with support at origin $M.D = 0 \Rightarrow
 N.D = 0$. 
\end{lemma*}

Necessity is obvious. To prove the sufficiency, consider the ring
$\mathscr{A}$ of formal power-series $\sum a_{i_1, \cdots i_n }
x^{i_1}_1 \cdots x^{i_n}_n$. It is a topological ring with the simple
convergence of the coefficients. It is locally convex and its dual
$\mathscr{A}' = \mathscr{E}'_o$ is the space of distributions with
support at the origin. The scalar product is given by 
$$
\langle D, \sum \cdots \rangle = \sum a_{i_1 \cdots i_n} \langle D,
x^i_1 \cdots x^{i_n}_n \rangle. 
$$

We use the following result proved in (Cartan $X, p.7$).
$$
\frac{N}{M} \text{ is analytic at } 0 \Longleftrightarrow \frac{N}{M}
\in \mathscr{A}. 
$$

Thus we have to prove that $N \in M \mathscr{A}$, the ideal generated
by $M$. We use the fact that this ideal is closed (Cartan $XI,
p.7$). Then it is sufficient to prove that $D \in (M
\mathscr{A})^\perp \Rightarrow \langle D, N \rangle = 0$. But $\langle
D, MK \rangle = \langle M.D, K \rangle$, so that $D \in (M\mathscr{A})
\Longleftrightarrow M.D = 0$. As we assume $M.D = 0 \Rightarrow N.D =
0$, and $\langle D, N \rangle = \langle N.D, 1 \rangle$, we have
proved $N \in M \mathscr{A}$, then $N/M$ is analytic at the origin. 

Since $b)$ gives that $M.D_\lambda = 0 \Rightarrow N.D_\Lambda = 0$ we
have, by the above lemma, $N/M$ analytic at every point $\lambda$. 

\textbf{3}. $c) \Longleftrightarrow d)$ is a consequence of the
following theorem: (Malgrange) (Ehrenpreis $1$). 

\textit{Theorem of Lindelof-Malgrange-Ehrenpreis:\pageoriginale If $F$ and $G$
 are functions of exponential type and $F/G$ is entire, then $F/G$ is
 also a function of exponential type}. 

In this proof, L. Ehrenpreis used a theorem on minimum modules. On the
other hand, Malgrange's proof is directly inspired from that of
Lindelof. 

We have to return to analytic functions of one variable.

Suppose $F(w) = w^P e^{aw} \prod\limits_{1}^\infty (1-w/\lambda_j)
e^{w/\lambda_j}$. This form of $F(w)$ need not imply that $F(w)$ is
of exponential type. For example, the inverse of the classical
$\Gamma$- function is not of exponential type. Lindelof gave a
characterization of entire functions of exponential type. Let the
zeros of $F(w)$ be arranged in such a fashion that $|\lambda_{j+1} |
\ge | \lambda_j |$. Then $F(w)$ is of exponential type if and only if
$\dfrac{n}{| \lambda_n |} = 0(1)$ and $\sum\limits_{| \lambda_i | <
 |K} 1/\lambda_i = 0(1)$ for all $K$. Using this characterization,
one proves in the case of one variable that $F(w) / G(w)$ is of
exponential type. Malgrange gave a refinement of this
characterization. 

\begin{proposition}\label{chap23:prop1}% proposition 1
  Let $F(w)$ be an entire function with $F(0) \neq 0$ and\break $| F (w) | <
  A ~ e^{B|w|}$ and let $\lambda_n$ be the zeros of $F(w)$. Then 
  \begin{gather*}
    a)~ \frac{n}{\lambda_n | }< \frac{C}{|F(0) |}, ~b)~ | a + \sum_1^n ~
    \frac{1}{\lambda_j} | < \frac{D}{|F(0)|^2};\\ 
    C = C(A,B), D = D(A,B).
  \end{gather*}
\end{proposition}

Conversely, if $\dfrac{n}{\lambda_n} < e$ and $| a + \sum\limits_1^p
\dfrac{1}{\lambda_j}| < D$ for every $n$ and $p$, then 
$$
F(w) = e^{aw} \prod_{1}^\infty \left(1- \frac{w}{\lambda_j}\right) e^{w/\lambda_j}
$$
is an entire function and 
$$
| F(w) | = | F(w)/F(0)| < A ~ e^{B|w|}, A = A(c,D) \text{ and } B = B(C,D).
$$

For\pageoriginale the first part of the proposition, $a)$ follows easily from
Jensen's formula: $b)$ is more involved. The second part follows from
some calculations (Malgrange). 

\begin{proposition}\label{chap23:prop2}% proposition 2
  Let $|F(w)| < A ~ e^{B|w|}$ with $F(0) = 1$ and let $|G(w) | < A' ~
  e^{B'|w|}$ with $G(0) = 1$. Suppose $H(w) = F(w) / G(w)$ is an
  entire function. Then $|H(w) | < A'' e^{B''|w|}$ with $A''$ and $B''$
  depending only on $A, A', B$ and $B'$. 
\end{proposition}

Proposition \ref{chap23:prop2} is an immediate consequence of
Proposition \ref{chap23:prop1}.

Now the proof of Lindelof theorem for several variables follows easily
from Proposition \ref{chap23:prop2}. 

Suppose $|F(w_1 \cdots w_n) | < A ~ e^{B(|w_1|+ \cdots | w_n|)}$,
$$
| G(w_1, \ldots, w_n) | < A' ~ e^{B'(|w_1| + \cdots + |w_n|)} \text{
 and } H(w) = F(w)/G(w) 
$$
is an entire function. We fix $w_1 \cdots w_n$ such that $|w | = |w_1
| + \cdots + | w_n | = 1$ and take $F_w(\delta) = F(\delta w_1,
\ldots, \delta w_n), G_w(\delta) = G(\delta w_1, \ldots, \delta
w_n)$. Then by Proposition \ref{chap23:prop2}, $| H_w (\delta) | < A''
e^{B''|\delta|}$.  

Hence, for any $w = (w_1, \ldots, w_n)$, we have $|H(w)|< A''
e^{B''|w|}$ (because $H(w) = H_{w/|w|}(|w|)$). 
