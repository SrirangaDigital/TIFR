
\chapter{Mean period and fourier series}\label{chap6}%Lect 6

\section{Mean period}\label{chap6:sec1}

The\pageoriginale equivalence of the definitions of mean periodic functions allows us
to define the mean period of a function, or again, the mean period
related to its spectrum $\Lambda$, in different ways. 

Definition $II$ can be put in a different form. Let
$\mathscr{C}_{\Lambda}$ be the closed subspace spanned by $\left\{x^p
e^{i \lambda x}\right\}(\lambda)_{p+1}\in \Lambda$ (We use the
notation $(\lambda)_{p+1}\in \Lambda$ if $\lambda \in \Lambda$, at
least $p+1$ times). We have either $\mathscr{C}_{\Lambda} =
\mathscr{C}$ or $\mathscr{C}_{\Lambda}\neq \mathscr{C}$. Definition
$II$ means: $f$ is mean periodic if $f \in \mathscr{C}_{\Lambda}$ and
$\mathscr{C}_{\Lambda} \neq \mathscr{C}$. Then $S (f) \subset \Lambda;
S(f)$ being the spectrum of $f$. 

We recall (lect. \ref{chap2}, ~ \ref{chap1}) that given a closed interval $I$
$\mathscr{C}(I)$ is the space of continuous functions on $I$, with the
topology of uniform convergence; $\mathscr{C}'(I)$ is the space of
measure with support in $I$. We define $\mathscr{C}_{\Lambda}(I)$ as
the closed subspace generated by $\left \{ x^p e^{i \lambda x}\right
\} (\lambda)_{p+1}\in \Lambda$. 

\medskip
\noindent{\textit{\textbf{Definition of mean period}}}

\textbf{1. \textit{Mean period of a mean periodic function.}}

We recall this definition from 2, lecture \ref{chap5}. The mean period of a
mean periodic function is defined to be the infimum of the lengths of
the segments of support of $d \mu \perp \tau (f)$. 

\medskip
\textbf{2. \textit{Mean period related to $\Lambda$.}}

\begin{enumerate}[(a)]
\item It\pageoriginale is the common mean period of functions with spectrum
 $\Lambda$ (for which $\tau(f) = \mathscr{C}_{\Lambda}$). Now we use
 the fact that $\mathscr{C}_\Lambda(I) \neq \mathscr{C}(I)$ is
 equivalent to saying that there exists a measure $d \mu$ orthogonal
 to all $x^p e^{i \lambda x}$, with support of $d \mu \subset I$. 
\item The mean period, $L$, is the infimum of $I$ where $I$ is such
 that $\mathscr{C}_{\Lambda} (I) \neq \mathscr{C} (I)$. 
\item The mean period, $L$, is the supremum of $I$ where $I$ is such
 that $\mathscr{C}_{\Lambda}(I)=\mathscr{C}(I)$. 
\item The mean period, $L$, is the infimum of lengths of segment of
 support of measure $d \mu$ orthogonal to $x^n e^{i \lambda x}, n=0,
 \ldots, p, (\lambda)_{p+1}\in \Lambda$. 
\item The mean-period related to $\Lambda$ is the infimum of the $L'$
 such that there exists an entire function $M(w)$ of exponential type
 with $M(w) = 0 (e^{\dfrac{1}{2}L'|w|}), M(\Lambda) = 0, M(u) =
 0(1)(|u| \to \infty)$. 
\item Same definition, with $M(u) = 0(|u|^N)$ for one $N = N(M(u))$,
 instead of $M(u) = 0(1)$. 
\item Same definition, with $M(u) = 0(|u|^{-n})$ for each $n$, instead
 of $M(u) = 0(|u|^N)$. 
\end{enumerate}

The equivalence between $(d)$ and $(e)$ is a simple consequence of the
Paley-Wiener theorem (lect. \ref{chap3}). The equivalence between $(e)$ and
$(f)$ is obvious, because multiplication by a polynomial does not
affect the exponential type of an entire function. The equivalence
between $(f)$ and $(g)$ depends on the following remark: given\pageoriginale $\varepsilon >
0$, the function $\Phi= (w)\prod\limits_1^{\infty}\dfrac{\sin \alpha_n
 w}{\alpha_n w}$, with $\sum\limits_1^\infty \alpha_n =
\sum\limits_1^\infty |\alpha_n| < \varepsilon$, satisfies $\Phi (w) =
0(e^{\varepsilon |w|})$ and $\Phi (u) = 0(|u|^{-n})$ for each
$n$. Then, if $M(w)$ satisfies the conditions in $(e), M(w) \Phi (w)$
satisfies those in $(g)$ (with $L'+\varepsilon$ instead of $L'$). 

The equivalence between $(e)~(f)~(g)$ shows that the mean-period
related to $\wedge$ would be the same, when, defined either from the
$C$-function or from the distribution of spectrum $\wedge$. 

Half of the mean-period has been called by Schwartz ``radius of
totality'' associated with $\Lambda$ (Schwartz $2$) (because of the
definition $(c)$). In (Paley-Wiener), (Levinson), (Schwartz $2$),
problems of closure of a set of exponentials over an interval are
considered, which naturally lead to the calculation of the mean period
related with a sequence $\wedge$. 

In this direction, the simplest facts are the following.
\begin{theorem*}%Thm
 Let $L$ be the mean-period of $\Lambda$. If $\sum\limits_{\lambda
 \in \Lambda}\dfrac{1}{|\lambda|} < \infty$ then $L = 0$, Further
 if $\lambda_n - D_n = 0 (1) ~ (n \to \pm \infty), \lambda_{-n} =
 -\lambda_n (n = 0, 1, \ldots,)$, and $\Lambda = \{\lambda_n \}$,
 then $L = 2 \pi D$. 
\end{theorem*}

\noindent \textbf{Proof of the first part:} Let $\Lambda = \Lambda_1 \cup
\Lambda_2, \Lambda_1$ finite, and $\sum\limits_{\lambda \in
 \Lambda_2}\dfrac{1}{|\lambda|}<\varepsilon$. As $(\sin 2u)^2 <
\dfrac{4u^2}{1+u^2}$, the function 
$$
M(w) = \sum_{\lambda \in \Lambda_1}\left(1-\frac{w^2}{2}\right)\sum_{\lambda \in
 \Lambda_2}\left(1-\frac{w^2}{2}\right) \left(\frac{\sin 2w / |
 \lambda |^2}{2w / | \lambda |}\right) 
$$
is\pageoriginale $0(|u|^N)$ on the real axis when $N$ is large enough of exponential
type $ < 4 \varepsilon$, and $M(\lambda) = 0$ for $\lambda \in
\Lambda$. According to definition $(f), L = 0$. 

For the second part, the proof depends on an estimate of
$\prod\limits_1^\infty (1-\dfrac{w^2}{\lambda_n^2}) = M_1 (w)$. We
shall see (lect. \ref{chap10}) that $M_1 (w) = 0(e^{\pi D |w|})$. Moreover, a
careful calculation shows $M_1(u) = 0(|u|^N)$ for $N$ large enough
(Paley-Wiener p. 93 - 94) According to definition $(f), L\geq 2 \pi
D$. From the Jensen formula we have $L \leq 23 \pi D$ (lect. \ref{chap11}). 

In lect. \ref{chap12} we shall prove $L \geq 2 \pi D_{\max}$, where $D_{\max}$
is the maximum density of Polya of the sequence $\Lambda$ (see
appendix 1). This inequality, together with the above theorem, led
$L$. Schwartz to the following hypothesis; if $\Lambda =
\{\lambda_n\}$ is real, symmetric $(\lambda_{-n} = -\lambda_n)$ and
has a density $\lim \dfrac{n}{\lambda n} = D$, then $L = 2 \pi D$. In
(Kahane $1, p. 57$) this equality is given as a consequence of the
following statement: each analytic function $f \in \mathscr{C}_{\Lambda}
(I)$ is continuable into an analytic function $f \in
\mathscr{C}(I)$. This last statement was not proved, and it is not at
all equivalent to the known results about continuation of analytic
functions (lect. \ref{chap16}, \ref{chap17}). In fact, Kahane's statement as well as the
Schwartz hypothesis are false, since we can construct a real symmetric
sequence $\wedge$ of density zero, whose mean-period is infinitely (see
appendix 2). 

\section{Fourier series of a mean periodic function}\label{chap6:sec2}%sec 3

We shall study the Carleman transform $F (w)$ of $f$ in order\pageoriginale to
define the Fourier series of $f$. 

If $\Big|f(x)\Big| < e^{a |x|}$, then we define with Carleman (Carleman)
$$
F^+ (w) = \int\limits^0_{-\infty}f(x) e^{-ixw} dx, F^- (w) =
-\int\limits_0^\infty f(x) e^{-ixw}dx. 
$$

When $w = u + iv, F^+$ is defined and holomorphic in $v>a$ and $F^-$
is defined and holomorphic in $v < -a$. We can write 
\begin{align*}
 F^+ (u + iv) &= \int\limits_\infty^0 f (x) e^{vx}e^{-iux}dx =
 \mathscr{C}(e^{vx}f^-)v \text{ fixed } v > \varepsilon \\ 
 F^- (u + iv) &= \mathscr{C} (-e^{vx}f^+ ), v \text{ fixed } v < -a
\end{align*}

Suppose $f$ is mean periodic and let $g = -f^+ * d \mu= f^- * d \mu$. Then (see
lect. \ref{chap3}, \ref{chap4}), $e^{vx} f^- * e^{vx}d \mu = e^{vx}g$;
$M(u + iv) = 
\mathscr{C}(e^{vx} d \mu); G(u + iv) = \mathscr{C}(e^{vx}g)$; $F^+ (u +
iv) M(u + iv) = G(u + iv), v>a$; $F^-(u+iv)M(u+iv)=G(u+iv), v<-a$, when
$v > a$, we have $F^+ = F$ and when $v < -a, F^-=F$. Hence the
definition of $F(w)$ is consistent with the definition of Carleman. We
shall use this interpretation of one definition of the Carleman
transform to define the Fourier series of $f$. 

Suppose $\varphi (x) =\sum A x^p e^{i \lambda x}$, with $A =
A(\lambda, p, \varphi)$ and the sum a finite sum. Then we have the
following equation: 
$$
\Phi^+(w)= \sum A \int\limits_{-\infty}^{0}x^p e^{-
 ix(w-\lambda)}dx=\sum Ap ! / (i(w- \lambda))^{p+1} 
$$

Suppose $f$ is the given mean-periodic function; there is a sequence
of finite sums $\varphi \in \tau(f)$ tending to $f$ in $\mathscr{C}$
(lect. 5 \ref{chap5:sec1}). Let $d \mu$ be a measure with the segment of support
$(-\ell, 0)$ such that\pageoriginale $f * d \mu=0, \varphi * d \mu =0$, since
$\varphi \to f$ in $\mathscr{C}, \varphi^- \to f^-$ in $\mathscr{C}$
and $\varphi^-* d \mu \to g $ in $\mathscr{C}$. Hence $\Phi(w) M(w)
\to F(w)M(w)$. 

Therefore $(*) A(\lambda, p, \varphi) \to A(\lambda, p)$ where the
term containing $(w-\lambda)^{-p-1}$ in the development of $F (w)$ is
$A (\lambda, p) p! / i^{p + 1}(w - \lambda)^{p + 1}$. 
\begin{defi*}
 We denote the polar part in the expansion of $F (w)$ by $C (f) =
 \sum A(\lambda, p)p! / i^{p+1}(w - \lambda)^{p+1}$. The \em{Fourier
 series} of $f$ is defined to be the formal sum $\mathscr{I}(f)$
 having an expansion of the following form 
 $$
 f \sim \mathscr{I}(f) = \sum_{(\lambda ) p + 1 \in \Lambda}
 A(\lambda, p) x^p e^{i \lambda x}, 
 $$
 where $A(\lambda, p)$ is the term obtained from the polar part
 $C(f)$ of the Carleman transform $F (w)$ of $f$. 
\end{defi*}

From $(*)$, taking into account the remark at the end of lect. 2
\ref{chap2:sec2}, we have:
\begin{theorem*}%Thm
 Suppose $\tau(f) \neq \mathscr{C}$. The $x^p e^{i \lambda x}\in
 \tau(f)$ form a basis of $\tau(f)$. Each $f \in
 \tau(f)$ admits a Fourier development which is the formal sum
 $\mathscr{I} (f) = \sum\limits_{\lambda^{p+1} \in \Lambda} A(\lambda, p)
 x^p e^{i \lambda x}$, with respect to this basis and the formal
 Carleman transform of this sum is the polar part $C (f)$ of $F(w),
 C(f) = \sum_i \dfrac{A(\lambda, p) p!}{i^{p+1}(w-\lambda)^{p+1}}$. 
\end{theorem*}

\begin{coro*}%Corlry
 Each simple subspace of $\mathscr{C}$ is generated by a finite
 number of monomial exponentials $x^pe^{i \lambda x}, p=0, 1, \ldots
 n$ 
\end{coro*}

\begin{remark*}%Remk
  The function $f \in \tau(f)\neq \mathscr{C}$ \em{is uniquely
    determined by its Fourier series} $\mathscr{I}(f)$. In other
  words, if all the $A(\lambda, p)$ are zero, then\pageoriginale $f \equiv
  0$. 
\end{remark*}

In (Kahane 1), $F(w)$ is called a ``Fourier-transform'' of $f$; it
is not a good definition. It is possible to define a kind of
Fourier-transform in the following manner. Consider the operation
$D_{\lambda}^p$ defined for functions analytic in the complex plane by
$\langle D_{\lambda}^p, \varphi (w) \rangle
=\varphi^{(p)}(\lambda)$. The Fourier transform $\mathscr{C}(f)$ can
be defined as a formal linear combination of $D^p_{\lambda}$ such that
$\mathscr{I}(f) = \dfrac{1}{2 \pi} \langle \mathscr{C}(f), e^{i w x}
\rangle$ and $C(f)(w) = \dfrac{1}{2 \pi i} \langle
\mathscr{C}(f)(w'), \dfrac{1}{w-w'} \rangle$; for example, if the
spectrum is simple, 
$$
\mathscr{C}(f) = 2 \pi \sum A(\lambda)D^0_{\lambda},
\mathscr{I}(f)=\sum A(\lambda)e^{i \lambda x}, C(f) = \sum
\frac{A(\lambda)}{i (w-\lambda)} 
$$

Let us find the relation between the Fourier series of $f$ and $f * d
\nu, d \nu \in \mathscr{C}'. ~C(f *d \nu) = $ polar part of $C(f)
\mathscr{C}(d \nu)$, for if $\varphi \to f$ in $\mathscr{C}$, $\varphi
*d \nu \to f * d \nu$ and $\mathscr{C}(f*d \nu ) =
\mathscr{C}(f)\mathscr{C}(d \nu) + N$. Therefore $\mathscr{I}(f * d
\nu) = \sum A (\lambda, p)(x^p e^{ i \lambda x}* d
\nu)=\mathscr{I}(f)*d \nu$. 

The definition of Fourier series of a mean periodic
$C^\infty-$functions or distributions is given in the same
fashion. Comparing $\mathscr{I}(T)$ and $\mathscr{I}(T *
\dfrac{d}{dx})$ we have: 
\begin{theorem*}%Thm
  The derivative of a mean periodic distribution $T$, is mean periodic
  and its Fourier series is obtained by deriving $\mathscr{I}(T)$
  formally. 
\end{theorem*}

We have a similar theorem for the primitive of a distribution.

\textit{The primitive $T_1$ of a mean-periodic distribution $T$ is
 mean-periodic,\pageoriginale and its Fourier series is obtained by taking a}
formal primitive of $\mathscr{I}(T)$. 

For, there exists $\varphi \in \mathscr{D}, \varphi \neq 0$, such that
$T * \varphi =0;$ then $T_1 * \varphi' = 0$, and $T_1$ is mean
periodic; the property of its Fourier series depends on the precedent
theorem. 
