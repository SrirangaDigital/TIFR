\chapter{Special algebraic extensions}\label{chap6}%chap  VI

\section{Roots of unity}\label{c6:s1}\pageoriginale %sec 1

Consider the polynomial $x^m -1$ in $k [k]$, where $k$ is a
field. Let $k$ have characteristic $p$. If $p=o$, then $x^m -1$ is a
separable polynomial over $k$, whereas if $p \neq o$, the derivative
of $x^m -1$ is $m x^{m-1}$ which is zero, if $p$ divides $m$. If $p
\nmid m$, then $x^m -1$ is a separable polynomial over $k$. Therefore,
let $m > o$ be an arbitrary positive integer, if $k$ has characteristic
zero and $m$, an integer prime to $p$, if $k$ has characteristic $p
\neq 0$. Then $x^m -1$ is a separable polynomial over $k$ and it 
has $m$ roots in $\Omega$, an algebraic closure of $k$. We call these
$m$ roots, the \textit{$m$th roots  of unity}.  

If $\rho$ and $\tau$ are $m$th roots of unity then $\rho^m = 1 =
\tau^m$. Therefore $(\rho \tau)^m = \rho^m \tau^m = 1$, $(\rho^{-1})^m =
(\rho^m)^{-1} = 1$ which shows that the $m$th roots of unity from a
group. This group $G_m$ is abelian and of order $m$.  

Let now $d/m$. Any root of $x^d -1$ in $\Omega$ is also a root of $x^m
-1$.If $\rho$ is an $m$th root of unity such that $\rho^d=1$, then
$\rho$ is  a root of $x^d-1$. Since $x^d-1$ has exactly $d$ roots in
$\Omega$, it follows that $G_m$ has the property that for every
divisor $d$ of $m$, the order of $G_m$, there exist exactly $d$
elements of $G_m$ whose orders divide $\underline{d}$. Such a group
$G_m$ is clearly a cyclic group.  Hence 

\textit{The\pageoriginale $m$th roots  of unity form a cyclic group $
  G_m $ of   order $m$.}  

There exists, therefore, a generator $\rho$ of $ G_m$. $\rho$ is
called a \textit{primitive $m$th root of unity}. Clearly, the number of
primitive $m$th roots of unity is $\varphi(m)$. All the primitive $m$th
roots of unity are given by $\rho ^a$ with $1 \le a < m$, $(a,m) =1$,
$\rho$ a fixed primitive $m$th roots of unity.   

Let $\Omega$ be an algebraically closed field. Let $m$ and $n$ be two
positive integers which are arbitrary, if $\Omega$ has characteristic
zero and, prime to $p$, if $\Omega$ has characteristic $p \neq o$. If
$\rho$ is an $m$th root of unity and $\tau$ an $n$th root of unity,
then  
$$
(\rho \tau)^{mn} = \rho^{mn} \cdot \tau^{mn} =1,
$$
so that $\rho \tau$ is an $mn$th root of unity. This shows that the
roots of unity in $\Omega$ form a group $H(\Omega)$. We now determine
the structure of $H$. 

\setcounter{thm}{0}
\begin{thm}\label{c6:thm1}%theo 1
If $\Omega$ has characteristic zero, then $H$ is isomorphic to the
 additive group of rational numbers mod $1$, whereas, if $\Omega$ has
  characteristic $p \neq o, H$ is isomorphic to the additive group of
  rational numbers $\dfrac{a}{b}$, $(a,b) = 1$, $p \nmid b$, $\mod 1$. 
\end{thm} 

\begin{proof}%proo 0
Let $R$ denote the group (additive) of rational numbers and $\nu_1 < 
\nu_2 < \nu_3 \cdots$ the sequence of natural numbers, if $\Omega$ has
characteristic zero, whereas, if $\Omega$ has characteristic $p \neq
o$, let $R$ denotes the rational numbers $\dfrac{a}{b}$, $(b,a) = 1$, $p
\nmid b$ and $\nu_1 < \nu_2 < \nu_3 \cdots$ the sequence of natural numbers
prime to $p$. Put 
$$
\mu_n = \nu_1 \cdots \nu_n.
$$\pageoriginale

Denote, by $H_n$, the group of $\mu_n$ the roots of unity in
$\Omega$. Since every integer $m$($p \nmid m$, if $\Omega$ has
characteristic $p$) divides some $\mu_n$, if follows that   
$$
H = \bigcup_n H_n.
$$

Since $H_n$ is cyclic, we can choose a generator $\rho_n$ of $H_n$ is
such a way that   
$$
\rho_n = {\rho'}^{\nu_{n+1}}_{n+1}.
$$
\end{proof}

Any $x$ in $R$ may be written as $\dfrac{a}{\mu_n}$ where $\underline{a}$
is an integer. Define the mapping $\sigma $ as follows  
$$
\sigma x = \rho^a_n,
$$
so that $\sigma$ is a function on $R$ with values in $H$. The mapping
is well defined: for if $x = \dfrac{b}{\mu_m}$, then $ \mu_m a = \mu_n
b$. Suppose $m > n$; then 
$$
b = a \nu_{n+1} \cdots \nu_m
$$
so that $\rho^b_m = \rho^{a \nu_{n+1} \cdots \nu_m}_m = \rho^a_n$ by choice
of $\rho_n$. We now verify that $\sigma$ is a homomorphism of $R$ on
$H$. If $x \dfrac{a}{\mu_n}$, $y = \dfrac{b}{\mu_m}$ are in $R$ and $m
\ge n$, then 
$$
\sigma (x+y) = \sigma (\frac{a \nu_{n+1} \cdots \nu_m +b}{\mu_m}) = \rho_m
^{a \nu_{n+1} \cdots \nu_m + b} 
$$
which equals $\rho^a_n \cdot \rho^b_m = \sigma x \cdot \sigma y$. Also, since
any root of unity is in some $H_n$, it is of the form $\rho^a_n$ so
that, for $x= \dfrac{a}{\mu_n}$, $\sigma x = \rho^a_n$. We have,
therefore, to determine the kernel of the homomorphism. It is the set
of $x$ in $R$ such that 
$$
\sigma x = 1.
$$  

If\pageoriginale $x = \dfrac{a}{\mu_n}$, then $1 = \sigma x =
\rho^a_n$ so that 
$|\mu_n|a$ and so $x$ is an integer. The converse being trivial, it
follows that the kernel is precisely the additive group of integers
and our theorem is established. 

\section{Cyclotomic extensions}\label{c6:s2}%sec 2

Let $k$ be a field and $x^m-1$ a separable polynomial in $k[x]$. This
implies, in case $k$ has characteristic $p \neq o$, that $p$ does not
divide $m$. Let $\rho$ be a primitive $m$th root of unity in $\Omega$,
an algebraic closure of $k$. Then $K = k (\rho)$ is the splitting
field of $x^m - 1$ in $\Omega$. Therefore, $K/k$ is a separable, normal
extension. Let $G$ be its galois group. If $\sigma \in G$,
$\sigma$ is determined by its effect on $\rho$. Since $\rho$ is a
primitive $m$th root of unity, so is $\sigma \rho$. For,  
$$
(\sigma \rho)^m = \sigma (\rho^m) =1 ;
$$
so $\sigma \rho$ is a root of unity. Also, if $(\sigma \rho)^t = 1$,
then $\sigma (\rho^t)= 1$. Since $\sigma$ is an automorphism, it
follows that $\rho^t = 1$ or $m/t$. Thus  
$$
\sigma \rho = \rho^{\nu}, (\nu, m) =1.
$$

If $\sigma$, $\tau$ are in $G$, let $\sigma \rho = \rho^\nu$, $(\nu ,m) = 1$
and $\tau \rho = \rho^\mu \; (\mu, m) =1$. Then 
$$
\sigma \tau (\rho) = \sigma (\tau \rho) = \sigma (\rho^\mu) =
\rho^{\mu \nu} 
$$
which shows that $\sigma \tau = \tau \sigma $ or that $G$ is abelian. 

Consider now the mapping
$$
g : \sigma \to \nu
$$
where $\sigma \rho =\rho^\nu$, $(\nu ,m) =1$. This is clearly a
homomorphism of $G$ into the multiplicative group prime residue classes
$\mod m$. The kernel of the mapping $g$ is set of $\sigma$ with
$\sigma \rho = \rho$. 

But\pageoriginale then $\sigma t = t$ for all $t \in K$, so that by,
galois theory, $\sigma$ is the identity.   

Let us call extension $K = k (\rho)$ a \textit{cyclotomic extension}
of $k$. We then have proved the  

\begin{thm}\label{c6:thm2} %theorem 2 
The Cyclotomic extension $k(\rho) / k  $ is an abelian extension whose
galois is isomorphic to a subgroup of the group of prime residue
classes $\mod m$ where $\rho^m = 1 $ and $\rho$ is a primitive $m$th
root of unity.  
\end{thm}

Let $\Gamma$ be the prime field contained in $k$. Then $\Gamma (\rho)$
is a subfield of $k(\rho)  = K$. Let $G$ be the galois group
$K/k$. 
\[
\xymatrix@R=0.5cm{
& K=k(\rho) \ar@{-}[dl] \ar@{-}[dd]\\
k\ar@{-}[dd] & \\
& \Gamma (\rho) \ar@{-}[dl] \\
\Gamma  & 
}
\]

Let $\sigma$ be in $G$ and $\bar{\sigma}$ the restriction of
$\sigma$ to $\Gamma (\rho )$. Since $\sigma$ is identity on $k$ and
$\sigma \rho$ is again a primitive root of unity, it follows that
$\bar{\sigma }$ is an automorphism of $\Gamma (\rho) / \Gamma$. It is
easy to see that the mapping $\sigma \to \bar{\sigma}$ is an
isomorphism of $G$ into the galois group of $\Gamma (\rho) / \Gamma$.  

We shall therefore confine ourselves to studying the galois group of
$\Gamma (\rho) / \Gamma$.  

First, let $\Gamma$ be the rational number field and $\rho$ a
primitive $m$th root of unity. Let $\Gamma (\rho)$ be the cyclotomic
extension. Let $f(x)$ be the primitive integral polynomial which is
irreducible in $\Gamma [x]$ an which $\rho$ satisfies. Then $f(x)$ is
a monic polynomial. For, since $f(x)$ divides $x^{m - 1}$,  
$$
x^m - 1 = f(x)\psi (x).
$$
$\psi (x)$\pageoriginale has rational coefficients and so $\psi (x)  =
\dfrac{a}{b} 
\psi_1 (x)$ where $\psi_1(x)$ is a primitive integral polynomial and
$a$ and $b$ are integers. 
From the the theorem  of Gauss on primitive polynomials it follows
that $f(x)$ is monic. 

Let $p$ be a prime not dividing $m$. Let $\varphi (x)$ be the minimum
polynomial (which is monic and integral) of $\rho^p$. We assert
that $f(x) = \varphi (x)$. For, if not, $f(x)$ and $\varphi(x)$ are coprime
and so  
$$
x^m - 1 = f(x) \cdot \varphi(x) \cdot h(x) .
$$ 
for some monic integral polynomials $h(x)$.

Consider the polynomial $\varphi(x^p)$. It has $\varphi$ as a root
and so $f(x)$ divides $\varphi (x^p)$. Hence  
$$
\varphi (x^p) = f(x) g (x), 
$$
$g(x)$, again, a monic and integral polynomial. Considering the above
$\mod p $ we get  
$$
f (x) g(x) \equiv \varphi (x^p) \equiv (\varphi (x))^p (\text{mod } p)
$$
so that $f(x)$ divides $(\varphi (x))^p (\mod p )$. If $t(x)$ is a
common factor of $f(x)$ and $(\varphi (p) (\mod p )$  then $(t(x))^2$
divides $x^m -1$ mod $p$, which is impossible, since 
$p \nmid m$ and $x^m -1$ does not have $x$ as a factor. Thus our
assumption $f(x) \neq \varphi (x) $ is false.  

This means that, for every prime $p \nmid m$, $\rho ^p$ is a root
of $f(x)$. If $(\nu, m) = 1$, then $\nu = p_1 p_2 \cdots p_1$ where
$p_1, \ldots, p_1$ are primes not dividing $m$. By using the above
fact successively, we see 
that, for every $\nu, (\nu, m) = 1$, $\rho^\nu$ is a root of
$f(x)$. Therefore  
\begin{equation*}
\rho_m (x) = \prod\limits_{ (\nu, m ) = 1 } (x - \rho^\nu
)\tag{1}\label{c6:eq1}
\end{equation*}
divides\pageoriginale $f(x)$. But $\varphi_m(x)$ is fixed under all
automorphisms of  $\Gamma(\rho) /_\Gamma$ so that $f(x)= \varphi_m
(x)$. We have proved the    

\begin{thm}\label{c6:thm3}%the 3
 If $\Gamma$  is the field of rational numbers and $\rho$
  is a primitive $m$th root of unity, then the galois  group of the
  cyclotomic extension $\Gamma(\rho)/ \Gamma$ is isomorphic to the
  group of prime  residue classes $ \mod m$.  The irreducible
  polynomial  $\varphi_m (x)$ of  $\rho$ is given by \eqref{c6:eq1}.  
\end{thm}

$\varphi_m(x)$ is called the \textit{cyclotomic polynomial} of order
$m$. Its degree is $\varphi(m)$. In order to be able to obtain an
expression  for $\varphi_m(x)$ in terms of polynomials over $\Gamma$,
we proceed thus.  

We introduce the Mobius function defined as follows: 

It is a function $\mu(n)$ defined for all positive integers $n$ such
that  
\begin{enumerate}[1)]
\item $\mu(1)=1$

\item $\mu(p_1 \cdots p_t) =(-1)^t$ where $p_1, \ldots , p_t$ are
  distinct primes. 

\item $\mu (m) =o$ if $p^2/m$, $p$ being a prime.

From  this, one deduces easily

\item $\mu(m)\cdot \mu  (n)= \mu (mn)$ for $(m,n)=1$. 

For, one has to verify it only for $m =p_1 \cdots p_t$, $n= q_1 \cdots
q_1$ where $p's$ and $q's$ are all distinct primes. Then $\mu (m) =
(-1)^t$, $\mu (n) =(-1)^1$ and $\mu (mn) =(-1)^{t+1}$.  

We now prove the following  simple formula
\item  
\begin{align*}
\sum_{d|m} \mu (d) & = o \text{ if }  m>1 \\
& = 1 \text{ if } m = 1 
\end{align*}
the\pageoriginale summation running through all divisors $d$ of $m$. 
\end{enumerate}

If $m=1$, the formula reduces to \eqref{c6:eq1}. So, let $ m >1$. Let $m =
p^{a_1}_1 \cdots p^{a_t}_t$ be the prime factor decomposition of 
$m$. Any divisor $d$ of $m$ is of the form $p_1^{b_1} \cdots
p_t^{b_t}$ where $o \le b_i \le a_i$, $i=1,\ldots,t$. In view of
\eqref{c6:eq3}, it is enough to consider divisors $d$ of $m$ for which $o \le
b_i \le 1$. In that case,  
$$
\sum_{d|m} \mu (d) = \sum_{i=o}^t (^t_i) (-1)^i = o. 
$$

Let now $f (n)$ be a function defined on positive integers, with
values in a multiplicative abelian group. Let $g(n)$ also be such a
function. We then have the M\"obius inversion formula, 
\begin{center}
\begin{fbox}
{$\prod\limits_{d|n} f (d) = g (n) \Longleftrightarrow f (n)
  =\prod\limits_{d|n} (g(d))^{\mu(\frac{n}{d})}$}
\end{fbox}
\end{center}

Suppose $\prod\limits_{d|n} f(d) = g(n)$. Then 
$$
\prod_{d|n} (g (d))^{\mu(\frac{n}{g})} = \prod\limits_{d|n}
\bigg(\prod_{d_i |\frac{n}{d}} f(d_1)\bigg)^{\mu(d)}. 
$$
 
Changing the order products, we get 
$$
\prod_{d_1|n} \big(f (d_1)\big) \sum_{d|\frac{n}{d_1}} \mu(d);
$$
using formula \eqref{c6:eq5}, we obtain the inversion formula. The converse
follows in the same way.  

Consider now the integers $\mod m$. Divide them into classes in the
following manner. Two integers $a$, $b$ are in the same class if and
only if 
$$
(a, m) = (b, m).
$$

Let\pageoriginale $d/m$ and $C_d$, the class of integers a $(\mod m)$
with $(a,m) = 
d$. Then $a$ is of the form $d \lambda$ where $(\lambda, \dfrac{m}{d})
= 1$. Thus $C_d$ has $\varphi (\dfrac{m}{d})$ elements. The classes
$C_d$, for $d|m$, exhaust the set of integers $\mod m$. If $\rho$ is a
primitive $m$th root of unity, then 
$$
x^m -1 =\prod_{t =1}^m (x - \rho^t).
$$

In view of the above remarks, we can write 
$$
x^m-1 = \prod_{d|m} \bigg(\prod_{ t \in C_d} (x-\rho^t)\bigg).
$$

But if $t \in C_d$, $\rho^t = \rho^{d \lambda}$, $(\lambda,
\dfrac{m}{d}) =1$ and so $\rho^t$ is a primitive $\dfrac{m}{d}$ th
root of unity. Using the definition of $\varphi_m (x)$, if follows
that   
$$
x^m-1 =\prod_{d|m} \varphi_d (x)
$$

By the inversion formula we get
\begin{center}
\begin{fbox}
{$\varphi_m (x) = \prod\limits_{d|m} (x^d -1)^{\mu(\frac{m}{d})}$}
\end{fbox}
\end{center}

Comparison of degrees on both sides gives the formula
$$
\varphi(m) = \sum_{d|m} d \mu (\frac{m}{d}) = m \sum_{d|m} \frac{\mu
  (d)} {d}. 
$$

We may compute $\varphi_m (x)$ for a few special values of $m$. Let $
m=p $, a prime number. Then  
$$
\varphi_p(x) \frac{x^p -1}{x-1} = x^p-1 +\cdots+ x+1.
$$

If $m=pq$, the product of two distinct primes, then
$$
\varphi_{pq} (x) = \frac{(x^{pq}-1) (x-1)} {(x^p-1) (x^q-1)}.
$$

Let us now consider the case when $\Gamma$ is the prime filed of $p$
elements. Obviously, $\Gamma (\rho) /\Gamma$ is a cyclic
extension. If we define\pageoriginale the cyclotomic polynomial as
before, it is no longer irreducible over $\Gamma [x]$. For instance,
let $p=5$ and $m=12$. Then  
$$
\varphi_{12} (x) = \frac{(x^{12}-1) (x^2-1)} {(x^6-1) (x^4-1)} = x^4-x^2 +1.
$$

Also
$$
x^4 -x^2+1 = (x^2-2x-1) (x^2+2x-1) \; ({\rm mod } \; 5).
$$

Therefore, $\Gamma(\rho) / \Gamma$ has degree $< \varphi (m)$. It is
obvious since $\Gamma (\rho) / \Gamma$ is cyclic, that, for
$\varphi_m(x)$ to be irreducible, the group of prime residue classes $\mod m$
should be cyclic. We shall prove 

\begin{thm}\label{c6:thm4}%theo 4
If $\Gamma$ is the prime filed of characteristic $p \neq o$ and
  $p \nmid m$,the cyclotomic polynomial $\varphi_m(x)$ is irreducible,
  if  and only if the group of prime residue classes mod $m$ is cyclic
  and $p$ is a generator of this cyclic group. 
 \end{thm} 

 \begin{proof}
We already know that $\Gamma (\rho) / \Gamma$ is cyclic. If $\varphi_m
(x)$ is irreducible, then $\Gamma (\rho) / \Gamma$ has order
$\varphi(m)$. Let $\sigma$ be the Frobenius automorphism of $\Gamma
(\rho) / \Gamma$. Then 
$$
\rho^\sigma = \rho^p,
$$
$p$ being the number of elements in $\Gamma$. The $\varphi(m)$
automorphisms \break $1, \sigma, \sigma ^2 ,\ldots, \sigma
^{\varphi(m)-1}$ are distinct. Hence   
$$
\rho, \rho^p, \rho^{p^2} ,\ldots, \rho^{p^{\varphi(m)-1}} 
$$
are all distinct, which means $1,p,p^2, \ldots, p^{\varphi (m)-1}$ are
distinct mod $m$. Therefore, $p$ is a generator of the multiplicative
group of prime residue classes $\mod m$.  
\end{proof} 

The converse is trivial.
 
 The\pageoriginale theorem is true, if, instead of $\Gamma$ being the
 prime filed of  $p$ elements, $\Gamma$ is a finite filed of $q$
 elements. Then $q$  has to be a generator of the group of prime
 residue classes $\mod m$.    

 If $R_m$ denotes the group of prime residue classes $\mod m$, then
 $R_m$ is cyclic, if and only if   
 $$
 m=2^a, a=1,2 \quad \text{or} \quad m =q^b \quad \text{or} \quad 2q^b,
 $$
 where $q$ is a prime. Thus $m$ has necessarily to have one of these
 forms. 
 
 We now use the irreducibility of $\varphi_m (x)$ over the rational
 number field to prove a theorem of \textit{Wedderburn}. 

 \begin{thm}\label{c6:thm5}%the 5
A division ring with a finite number of elements is a filed. 
 \end{thm} 

 \begin{proof}
Let $D$ be the division ring and $k$ its centre. Then $k$ is a
field. $D$ being finite, let $k$ have $q$ elements. If $D$ is of rank
$n$ over $k$, then $D$ has $q^n$ elements. We shall prove that $n =1$.     
 \end{proof}
  
Let $D_1$ be a subalgebra of $D$ over $k$. Let $D_1$ have rank $m$
over $k$. Then $D_1$ has $q^m$ elements. But $D^*_1$ is a sub group of
$D^*$ so that $q^m-1$ divides $q^n-1$. This means that $m|n$. For, if
$n =tm + \mu$, $o \le \mu < m $; then    
$$
q^n-1 = q^\mu (q^{tm}-1) + (q^\mu -1)
$$
so that $q^m -1 | q^\mu -1$ which cannot happen unless $\mu=0$. Thus
every subalgebra of $D$ has rank $d$ dividing $n$. 

Let $x \in D$. Consider the $y \in D$ such that
$xy=yx$. They\pageoriginale form a subalgebra over $k$. Therefore, the
number of $y$ 
in $D^*$ which commute with $x$ form a group of order $q^d-1$, for
some $d$ dividing $n$. This group is the normaliser of $x$. Hence, the
number of distinct conjugates (in the sense of group theory) of $x$ in
$D^*$ is  
$$
q^n-1| q^d-1.
$$

For the finite group $D^*$, we have
$$
D^* = k^* + \sum_x D^*_x,
$$
where $D^*_x$ is the set of all conjugates of $x$. Comparing number of
elements on both sides 
$$
q^n-1 = q-1 +\sum_d q^n-1 /q^d-1
$$ 
for some divisors $d$ of $n$.

Since $\varphi_n (x)| x^n-1$, we see that $\varphi_n (q)$, which is
an integer, divides $q^n-1$. Also $\varphi^n (x)$ divides $x^n-1 |
x^d-1$ for any $d|n$, $d \neq n$. Therefore, $\varphi_n (q)$ divides
$q-1$. But  
$$
\varphi_n (q) > (q-1)^{\varphi (n)}
$$
which shows that $ n \not > 1$.

This proof is due to \textit{Ernst Witt}.


\section{Cohomology}\label{c6:s3}%sec 3

Let $G$ be a finite group and $A$ an abelian group on which $G$ acts
as a group of left operators. Let $A$ be a multiplicative group. We
denote elements of $G$ by $\sigma, \tau, \rho ,\ldots,$ and elements
of $A$ by $a,b,c,\ldots$. We denotes by $a^\sigma$ the effect
of $\sigma$ on $a$. Then   
\begin{align*}
(ab)^\sigma & = a^\sigma b^\sigma\\
(a^\tau)^\sigma & = a^{\sigma \tau}
\end{align*}\pageoriginale

Let 1 be the unit element of $A$. Denote by $G^n$, $n \geq 1$ the
Cartesian product of $G$ with itself $n$ times. 

A function on $G^n$ with values in $A$ is said to be an $n$
\textit{dimensional Cochain} or, simply, an $n$ cochain. This function
$f(x_1,\ldots,x_n)$ has values in $A$. We denote by a $\sigma
_1,\ldots,\sigma_n$ the element in $A$ which is the value taken by the
$n$ cochain for values $\sigma_1,\ldots,\sigma_n$ of its variables. We
denote the function also, by  $a \sigma_1,\ldots,\sigma_n$. If $a
\sigma_1,\ldots,\sigma_n$ and $b \sigma_1,\ldots,\sigma_n$ are two
functions, we define their product by  
$$
o_{\sigma_1,\ldots,\sigma_n} = a_{\sigma_1,\ldots,\sigma_n} \cdot
b_{\sigma_1,\ldots,\sigma_n}.  
$$

Similarly
$$
o_{\sigma_1,\ldots,\sigma_n} = (a_{\sigma_1,\ldots,\sigma_n})^{-1}
$$
is called the inverse of $a_{\sigma_1,\ldots,\sigma_n}$. With these
definitions, the $n$ cochains form a group $C^n(G,A)$ or simply
$C^n$. 

We define $C^o(G,A)$, the zero dimensional cochains, to be the group 
of functions with constant values, that is functions $a_\sigma$ on $G$ 
such that 
$$
a_\sigma = a
$$
is the same for all $\sigma \in G$. It is then clear that $C^o$ is
isomorphic to $A$. 

We now introduce a coboundary operator $\partial$ in the following
manner. $\partial(=\partial_n)$ is a homomorphism of $C^n$ into
$C^{n+1}$ defined by 
\begin{align*}
a_{\sigma_1,\ldots,\sigma_{n+1}} & = \partial a_{\sigma_1,\ldots,
  \sigma_n} \\
& = a^{\sigma_1}_{\sigma_2,\ldots,\sigma_{n+1}}
\prod^n_{n=1} (a_{\sigma_1,\ldots,\sigma_{i-1} ,\sigma_i
  \sigma_{i+1},\ldots,\sigma_{n+1}} )^{(-1)^{i}} 
(a_{\sigma_1,\ldots,\sigma_n})^{(-1)^{n+1}}. 
\end{align*}\pageoriginale

For $n \geq o$, we define the groups $Z^n(G,A)$ and $B^{n+1}(G,A)$ in\break 
the following manner. $Z^n(G,A)$ is the kernel of the homomorphism
$C^n \overset{\partial n} \longrightarrow C^{n+1}$ and
$B^{n+1}(G,A)$ 
is the homomorphic image. The elements of $Z^n(G,A)$ are called $n$
\textit{dimensional cocycles} or $n$-cocycles and the elements of
$B^{n+1}(G,A)$ are $n+1$-\textit{dimensional coboun\-daries} or $n+1$
coboundaries. The homomorphism $\partial$ has the property 
\begin{equation*}
\partial \partial=\text{ identity } \tag{2}\label{c6:eq2}
\end{equation*}
which proves that $B^{n+1}(G,A)$ is a subgroup of $Z^{n+1}(G,A)$ and
we can form for every $n > o$ the factor group $H^n(G,A)$ the
$n$-\textit{dimensional cohomology group}. We verify \eqref{c6:eq2} only in
case $n=o$ and 1 which are the ones of use in our work. 

The coboundary of a zero cochain is a one cochain given by
\begin{equation*}
\partial a=a_\sigma = \frac{a^\sigma}{a}. \tag{3}\label{c6:eq3}
\end{equation*}

Its coboundary, by definition is
$$
\partial \partial a=a_{\sigma,
 \tau}=\frac{a_\sigma \cdot a^\sigma_\tau}{a_{\sigma \tau}}.
 $$
 
 Substituting from \eqref{c6:eq3}, we get $a_{\sigma, \tau}=1$ which
 verifies \eqref{c6:eq2} for $n=o$.  
 
 We define the zero coboundary, that is the elements in $B^o(G,A)$, to
 be the function with value 1 on all $\sigma \in G$. Thus $B^o(G,A)$
 consists only of the identity. An element in $Z^o(G,A)$ will be the
 constant function a with 
 $$
 a^\sigma =a
 $$
 for all $\sigma$. Hence
 
 $H^o(G,A)$\pageoriginale \textit{is isomorphic with the set of $a \in
   A$ with the property $a^\sigma = a$ for all $\sigma \in G$}.  
 
 A one dimensional cocycle is a function $a_\sigma$ for which
 $\partial a_\sigma =1$. But 
 $$
 \partial a_\sigma=\frac{a_\sigma \cdot a^\sigma_\tau}{a_{\sigma
     \tau}}. 
 $$
 
  Therefore, the elements in $Z'(G,A)$ are functions $a_\sigma$ with
 $$
 a_\sigma a^\sigma_\tau =a_{\sigma \tau}.
 $$
 
 A one coboundary is, already, a function $a_\sigma$ of the form
 $a^\sigma/a$. It is, certainly, a one cocycle. 
 
 For our purposes, we shall need also the \textit{additive
   cohomology}, instead of the multiplicative cohomology above. We
 regard now $A^+$ as an additive, instead of a multiplicative,
 group. Then $C^n(G,A^+)$ is an additive abelian group. We define the
 coboundary operator as 
{\fontsize{10pt}{12pt}\selectfont
 \begin{align*}
 a_{\sigma_1,\ldots,\sigma_{n+1}} & = \partial
 a_{\sigma_1,\ldots,\sigma_n} \\
& = a^{\sigma_1}_{\sigma_2
   ,\ldots,\sigma_{n+1}}+\sum^n_{i=1}(-1)^i
 a_{\sigma_i, \ldots, \sigma_{i-1}, \sigma_i,
    \tau \sigma_{i+1},\ldots,\sigma_{n+1}} + (-1)^{n+1}
 a_{\sigma_i,\ldots,\sigma_n}.   
 \end{align*}}\relax

 
 As before, a zero cochain is a constant function on $G$ and its
 coboun\-dary $a_\sigma$ is 
 $$
 \partial a = a_\sigma = a^\sigma - a.
 $$
 
 A one cochain $a_\sigma$ is a cocycle if
 $$
 \partial a_\sigma =0
 $$
 which is the same thing as 
 $$
 a_\sigma + a^\sigma_\tau =a_{\sigma \tau}.
 $$
 
 Exactly as before, we see that $H^o(G,A)$, the zero dimensional
 additive cohomology group is isomorphic to the set of elements
 $\underbar{a}$\pageoriginale in $A^+$ with $a^\sigma = a$ for all
 $\sigma$.  
 
 We now consider the case when $G$ is a cyclic group. Let $\sigma$ be
 a generator of $G$ so that $1,\sigma, \sigma^2,\ldots,\sigma^{n-1}$
 are all the elements of $G$. Taking multiplicative cohomology, if
 $a_\sigma$ is a one cocycle, then 
 $$
 a_{\tau \mu} = a_\tau a^\tau _\mu
 $$
 or $a^\tau_\mu  = a_{\tau \mu}/ a_{\tau}$. Substituting for $\tau$
 successively $1,\sigma,\ldots,\sigma^{n-1}$ we get 
 $$
 a^{1+\sigma + \cdots + \sigma^{n-1}}_{\mu}=1.  
 $$

We denote $a^{1+\sigma + \cdots + \sigma^{n-1}}_\mu$ by $N a_\mu$ and
call it the \textit{norm of} $a_\mu$. Thus, if $a_\mu$ is a cocycle,
then, 
 $$
 N a_\mu =1.
 $$
 
 Conversely, suppose $\underbar{a}$ is an element in $A$ with
 $$
 N a = a^{1+\sigma + ..+ \sigma^{n-1}=1} =1.
 $$
 
 We can define a cocycle $a_\mu$ such that $a_\sigma = a$. For let
 $\mu =\sigma^\nu$, for some $\nu$. Put 
 $$
 a_\mu =a^{1+ \sigma + \cdots +\sigma^{\nu -1}}. 
 $$

 Obviously, $a_\sigma = a$. Also, $a_\sigma $ is a cocycle. For, if
 $\tau = \sigma^{\nu'}$. 
 \begin{align*}
a_\mu a^\mu_\tau & = a^{1+\sigma + \cdots + \sigma^{\nu -1}}
(a^{1+\sigma + \cdots +\sigma^{\nu'-1}})^{\sigma^\nu}\\ 
& = a^{1+\sigma + \cdots + \sigma^{\nu -1}+\sigma^\nu + \cdots
  +\sigma^{\nu+\nu'-1}}\\ 
& = a_{\mu \tau}.
 \end{align*} 
 
 In a similar manner, for additive cohomology, we have $a_{\tau
   \mu}=a_\tau + a^\tau_\mu $ where $a_\mu$ is a cocycle. If $G$ is
 cyclic, on substituting\pageoriginale $1,\sigma,\ldots,\sigma^{n-1}$
 for $\tau$, we  get  
 $$
 S a_{\mu} =a_\mu +a^\sigma_\mu + \cdots + a^{\sigma^{n-1}}_\mu =o.
 $$
 
 We call $S a_\mu$ the \textit{spur or trace} of $a_\mu$. If
 $\underbar{a}$ is any element of $A$, with trace $S a=a+a^\sigma +
 \cdots + a^{\sigma^{n-1}}=o$, then the cocycle  
 $$
 a_\mu = a+a^\sigma +\cdots +a^{\sigma^{\nu -1}}
 $$
 where $\mu = \sigma^\nu$, has the property
 $$
 a_\sigma = a.
 $$
 
 We now apply the considerations above, in the following
 situation. $K/k$ is a finite galois extension with galois group
 $G$. Then $G$ acts on $K^*$ and also the additive group $K^+$ as a
 group of operators. We might, therefore, consider the cohomology
 groups $H^o(G,K^*)$, $H^1(G,K^*),\ldots $ and
 $H^o(G,K^+),H^1(G,K^+)\ldots $ etc. As before, $H^o(G,K^*)$ and
 $H^o(G,K^+)$ are isomorphic to subgroups of $K^*$ and $K^+$ with
 $a^\sigma = a $ for all $\sigma$. This, by galois theory, shows 

 \begin{thm}\label{c6:thm6} % the 6
$H^o (G,K^*)$ is isomorphic to $k^*$ and $H^o(G,K^+)$ is
     isomorphic to $k^+$. 
 \end{thm} 
 
 But what we are interested in, is the following important theorem due
 to \textit{Artin}. 

 \begin{thm}\label{c6:thm7} % the 7
 The group $H^1 (G,K^*)$ and $H^1 (G,K^+)$ are trivial.
 \end{thm} 

 \begin{proof} % proof
Let us first consider multiplicative cohomology. If $a_\sigma$ is a
cocycle, we have to prove that it is a coboundary. The elements
$\sigma,\tau ,\ldots$ of $G$ are independent $k$-linear mappings of
$K$ into $\Omega$, the algebraic closure. Hence, if $(a_\sigma)$ are
elements of $K^*$, 
$$
\sum_\sigma a_\sigma  \cdot \sigma 
$$\pageoriginale
is a non-trivial $k$-linear map of $K$ into $\Omega$. Therefore, there
exists a $\theta \neq o$ in $K$ such that 
$$
\sum_\sigma a_\sigma \theta^\sigma \neq o.
$$
\end{proof} 

Put $b^{-1}=\sum\limits_\sigma a_\sigma \theta^\sigma = \sum
\limits_\tau a_\tau \theta^\tau$. Then 
$$
(b^{-1})^\sigma = \sum_\tau a^\sigma_\tau \theta^{\sigma \tau}.
$$

Therefore
$$
\frac{a_\sigma}{b^\sigma} = \sum_\tau a_\sigma a^\sigma_\tau
\theta^{\sigma \tau}. 
$$

Since $(a_\sigma)$ is a cocycle, we get
$$
\frac{a_\sigma}{b^\sigma}=\sum_\tau a_{\sigma \tau }\theta^{\sigma
  \tau}=\sum_\tau a_\tau \theta^\tau = b^{-1}. 
$$

Thus
$$
a_\sigma = \frac{b^\sigma}{b} = b^{\sigma-1}
$$
which is a coboundary.

Consider now the additive cohomology. $K/k$ being finite and
separable, there exists $a \theta \in K$ such that 
$$
\sum_\sigma \theta^\sigma = S_{K/k}\theta =1.
$$

Put now
$$
-b=\sum_\sigma a_\sigma \theta^\sigma
$$
$a_\sigma$ being an additive cocycle. Then
$$
-b^\sigma =\sum_\tau  a^\sigma_\tau \theta^{\sigma \tau}.
$$

But $a_\sigma = a_\sigma \cdot 1 =\sum_\tau a_\sigma \cdot
\theta^\tau$ so that 
$$
a_\sigma -b^\sigma=\sum_\tau (a_\sigma+a^\sigma_\tau) \theta^{\sigma
  \tau }=\sum_\tau a_{\sigma \tau}\theta^{\sigma \tau }= -b 
$$
which proves that $a_\sigma = b^\sigma -b$ is a coboundary.

Our theorem is completely proved.

We\pageoriginale apply the theorem in the special case where $G$ is
cyclic. Let $\sigma$ be a generator of $G$. If $a_\sigma$ is a cocycle
then, in multiplicative theory  
$$
a^{1+\sigma + \cdots +\sigma^{n-1}}_\sigma =1
$$
or $N_{K/k} a_\sigma =1$. Similarly in additive cohomology,
$$
 a_\sigma + a^\sigma_\sigma + \cdots +a^{\sigma^{n-1}}_\sigma =o
$$
or $S_{K/k} a_\sigma =o$.

Using theorem~\ref{c6:thm7}, we obtain 'theorem 90' of
\textit{Hilbert}.  

\begin{thm}\label{c6:thm8} %the 8
If $K/k$ is a finite cyclic extension, $\sigma$, a generator of
  the galois group of $K/k$ and $a$ and $b$ two elements of $K$ with
  $N_{K/k}a=1$ and $S_{K/k}b=0$ respectively, then 
\begin{align*}
a & = c^{1-\sigma}\\
b & = d^{\sigma}-d
\end{align*}
for two elements $c$, $d$ in $K$.
\end{thm}


\section{Cyclic extensions}\label{c6:s4}%%% 4

Let $K/k$ be a cyclic extension of degree $m$. Put $m=n p^a$, $(n,p)=1$
if $p$ is the characteristic of $k$; otherwise, let $m=n$. 

Let $G$ be the galois group of $K/k$. It has only one sub-group of 
order $p^a$. Let $L$ be its fixed field. Then $K/L$ is cyclic of 
degree $p^a$ and $L/k$ is cyclic of degree $n$ prime to $p$. Let
$\rho$ be a primitive $n$th root of unity and $k(\rho)$ the cyclotomic
extension. The composite $F = Lk(\rho)$ is cyclic over $k(\rho)$ and
of degree prime to $p$. We shall see that $K$ over $L$ 
and $F$ over $k(\rho)$ can be described in a simple manner. 

We shall, therefore, consider the following case, first.

$K/k$\pageoriginale is a cyclic extension of degree $m$ and $p \not
{\mid}m$, if $k$ has characteristic $p \neq o$; otherwise, $m$ is an
arbitrary integer. Also, $k$ contains all the $m$th roots of unity. We
then have the theorem of \textit{Lagrange}.  

\begin{thm}\label{c6:thm9} %the 9
$ K = k (w)$ where  $w^m \in k$.
\end{thm}

\begin{proof} %pro
Let $\rho$ be a primitive $m$th root of unity. $\rho$ is in $k$.
\end{proof}

Hence, since $K/k$ has degree $m$,
$$
N_{K/k} \, \rho = \rho^m =1.
$$

By Hilbert's theorem, therefore, if $\sigma$ is a generator of the
galois group of $K/k$, then there is an $\omega \in K$ such that 
$$
\omega^{1-\sigma}=\rho.
$$

Since $\rho$ is a primitive $m$th root of unity,
$\omega,\omega^\sigma,\omega^{\sigma^2},\ldots$ are all distinct and
are conjugates of $\omega$. Hence our theorem. 

$\omega$ satisfies a polynomial $x^m -a$, $a \in k$. If $K=k(\omega')$,
where $\omega'$ also satisfies a polynomial $x^m -b$, $b \in k$,then
$$
\omega'^\sigma =\omega' \cdot \rho^+,
$$
where $\rho^t$ is a primitive $m$th root of unity and so $(t,m)=1$.

\noindent
Now
$$
\left(\frac
     {\omega'}{\omega^t}\right)^\sigma = \frac{\omega'\cdot\rho^t}
     {\omega^t \cdot\rho^t} = \frac{\omega'}{\omega^t}
$$
which shows that
$$
\omega' = \omega^t \cdot  c, \quad  c \in k. 
$$

We shall call $x^m - a$ a \textit{normed polynomial}. We have, then, the


\begin{coro*}  % coro
 If $k(\omega)=K = k(\omega')$, where $\omega$ and $\omega'$ are
  roots of normed polynomials, then 
$$
\omega' = \omega^t \cdot c ,
$$\pageoriginale
where $(t,m)=1$ and $c \in k $.
\end{coro*}

We consider the special case, $m=q$, a prime number. Let $K/k$ be a
cyclic extension of degree $q$. Let $K=k(\alpha)$ and let
$\alpha_1=(\alpha),\alpha_2,\ldots,\alpha_q$ be the irreducible
polynomial of $\alpha $ over $k$. Suppose $\sigma$ is a generator of
the galois group of $K/k$. 

Let notation be so chosen that
$$
\alpha^\sigma_1=\alpha_2, \alpha^\sigma_2=\alpha_3,\ldots ,
\alpha^\sigma_{q-1}=\alpha_q, \alpha^\sigma_q=\alpha_1. 
$$

Since $k$ contains the $q$th roots of unity and every $q$th root of
unity $\rho \neq 1$ is primitive, we construct the \textit{Lagrange
  Resolvent}. 
$$
\omega = \omega (\alpha, \rho) = \alpha_1 + \rho \alpha_2 + \cdots +
\rho^{q-1} \alpha_ q. 
$$

Then
$$
\omega^\sigma = \alpha_2 +\rho \alpha_3 + \cdots +\rho^{q-2} \alpha_q
+ \rho^{q-1} \alpha_1 
$$
which shows that $\omega^\sigma = \rho^{-1} \omega$. Hence
$K=k(\omega)$. Also, 
$$
(\omega^q)^{\sigma} = \omega^q
$$
which proves that $\omega^q \in k$ and $\omega$ satisfies a normed
polynomial. 

In particular, if $k$ has characteristic $\neq 2$, and $K/k$ has
degree 2, then 
$$
K = k(\surd d)
$$
for $d \in k$.

The polynomial $x^q -a$ for $a \in k$ is, thus, either irreducible
and, then, a root of it generates a cyclic extension, or else, $x^q-a$
is a product of linear factors in $k$. 

We study the corresponding situation when $K$ has characteristic $p
\neq o$ and $K/k$ is a cyclic extension of degree $p^t$, $t \geq 1$. 

We\pageoriginale first consider cyclic extensions of degree $p$. 

Let $K/k$ be a cyclic extension of degree $p$ and $\sigma, a$ generator
of the galois group of $K/k$. Let $\mu $ be a generic element of the
prime field $\Gamma$ contained in $k$. 

Introduce the operator $\mathscr{P}x=x^p-x$. Then
$$
\mathscr{P}(x+\mu)=\mathscr{P}x.
$$

Also, the only $\alpha$ in $k$ satisfying $\mathscr{P} \alpha =o$ are 
the elements of $\Gamma$ and these are all the roots of $\mathscr{P} x
= o$. 

The element $1$ in $k$ has the property
$$
S_{K/k} \, 1 =o,
$$
so that by Hilbert's theorem, there is an $\omega \in K $ such that 
$$
1 = \omega^\sigma -\omega .
$$

Therefore, since $K/k$ has degree $p$,
$$
K =k(\omega).
$$

In order to determine the polynomial of which $\omega$ is a root,
consider $\mathscr{P} \omega$. 
$$
(\mathscr{P} \omega)^\sigma = \mathscr{P} \omega^\sigma =
\mathscr{P}(\omega +1) =\mathscr{P} \omega  
$$
which shows that $\mathscr{P} \omega \in k$. If we put $\mathscr{P}
\omega = \alpha$, then $\omega$ is a root the irreducible polynomial 
$$
x^p - x - \alpha
$$
in $k[x]$. $\omega$ is a root of the polynomial and $\omega^\sigma = 
\omega + 1$. 
Since $\sigma$ is a generator of the galois group of $K/k$, the roots
of $x^p - x - \alpha$ are $\omega, \omega + 1,\ldots , \omega+p-1$.  

A polynomial of the type $x^p - x - \alpha$, $\alpha \in k$ is called a
\textit{normed polynomial}. 

Suppose\pageoriginale $K=k(\omega')$ where $\omega'$ also satisfies a
normed polynomial. Then $\omega' , \omega'+1,\ldots,\omega'+p-1$ are
the roots of this normed polynomial. So  
$$
\omega'^\sigma = \omega' + h, \quad o < h \leq p-1.
$$

Now $\omega' - h \omega $ satisfies
$$
(\omega'-h \omega)^\sigma = \omega'^\sigma - h \omega^\sigma
=\omega'+h -h \omega - h = \omega' - h \omega 
$$
which shows that $\omega'-h \omega \in k$. We have, hence, the 

\begin{thm}\label{c6:thm10} %the 10
 If $K/k$ is cyclic of degree $p$ and $\sigma$, a generator of the
  galois group of $K/k$, then $K=k(\omega)$ where $\omega$ satisfies a
  normed polynomial $x^p - x- \alpha$ in $k[x]$ and $\omega^\sigma =
  \omega + 1$. 
If $K=k(\omega')$ and $\omega'$ also satisfies a normed polynomial,
then 
$$
\omega' = \mu \omega + a,
$$
$\mu \in \Gamma$ and $a \in k$.
\end{thm}

In order to be able to construct $\omega$, we proceed thus. Let
$\alpha$ be an element in $K$ with 
$$
S_{K/k} \alpha =1.
$$

Let $\alpha_1 (= \alpha),\ldots , \alpha_p$ be the roots of the
irreducible polynomial which $\alpha$ satisfies over $k$. Construct
the resolvent 
$$
-\omega = \alpha_1 + 2 \alpha_2 + \cdots +(p-1) \alpha_{p-1} + p
\alpha_p. 
$$

Let notation be so chosen that
$$
\alpha^\sigma_1 = \alpha_2, \alpha^\sigma_2 =
\alpha_3,\ldots,\alpha^\sigma_{p-1} = \alpha_p, \alpha^\sigma_p =
\alpha_1. 
$$  

Then
$$
- \omega^\sigma = \alpha_2 + 2 \alpha_3 + \cdots  + (p-1) \alpha_p 
$$
and so $\omega^\sigma - \omega = \alpha_1 + \alpha_2 + \cdots +
\alpha_p =1$. Therefore $\omega^p - \omega =t$ for some $t$ in
$k$. This gives the normed polynomial. 

It should\pageoriginale be noticed that, here, we use additive
cohomology whereas, in case $K/k$ has degree $m$ prime to $p$, we used
multiplicative cohomology. Furthermore, in the second case, when $K/k$
is of degree $p$, the elements of $\Gamma$ serve the same purpose as
the roots of unity in the first case.  

If $k$ has characteristic 2 and $K/k$ is a separable extension of
degree 2, then 
$$
K = k(\omega)
$$
where $\omega^2-\omega \in k$.

Observe, also, that any polynomial $x^p-x-\alpha$, for $\alpha \in k$,
is either irreducible over $k$ and so generates a cyclic extension
over $k$, or splits completely into linear factors in $k$. For, if
$\omega$ is a root of $x^p - x-\alpha$ then, for $\mu \in \Gamma$,
$\omega + \mu$ is also a root. 

Just as we denote a root of $x^m - \alpha$ by $\sqrt[m]{\alpha}$, we
shall denote, for $\alpha \in k$, by $\dfrac{\alpha}{\mathscr{P}}$, a
root of $x^p-x-\alpha$. It is obvious that
$\dfrac{\alpha}{\mathscr{P}}$ is $p$ valued and if $\omega$ is one
value of $\dfrac{\alpha}{\mathscr{P}}$, all the values are 
$$
\omega, \omega + 1, \ldots,\omega + p-1.
$$

We now study the case of cyclic extension of degree $p^n$, $n \geq 1$.

Let $K/k$ be a cyclic extension of degree $p^n$ and $\sigma$, a
generator of the galois group $G$ of $K/k$. Since $G$ is cyclic of
order $p^n$, it has only one subgroup of order $p$ and hence, there
exists a unique subfield $L$ of $K$ such that $K/L$ is cyclic of
degree $p$ and $L/k$ is cyclic of degree $p^{n-1}=m$. Let us assume
that $n \geq 2$. 

$\sigma^m$\pageoriginale is of order $p$ and hence is a generator of
the galois group of $K/L$. Thus $K=L(\omega)$ where  
$$
\sigma^m \omega = \omega +1
$$
and $\omega$ satisfies $\mathscr{P} \omega = \alpha \in L$.

Put $\sigma \omega -\omega=\beta$. Then $\sigma^m \beta = \sigma
(\sigma^m \omega)- \sigma^m \omega = \beta$ which shows that $\beta
\in L$. Also, 
$$
S_{L/k} \beta = \beta + \beta^\sigma + \cdots + \beta^{\sigma^{m-1}}.
$$

Substituting the value of $\beta$, we get
\begin{gather*}
S_{L/k} \beta = (\sigma \omega - \omega) + (\sigma^2 \omega - \sigma
\omega)+ \cdots + (\sigma^m \omega - \sigma^{m-1} \omega) \\ 
= \sigma^m \omega - \omega = 1.
\end{gather*}

Hence $\beta$ has the property
$$
S_{L/k} \beta =1.
$$

It is easy to see that $\alpha$ is not $k$. For, 
$$
\mathscr{P} \beta = \sigma (\mathscr{P} \omega) - \mathscr{P} \omega=
\sigma \alpha - \alpha 
$$
and $\alpha$ in $k$ would mean $\mathscr{P} \beta = o$ or $\beta \in
\Gamma$. This means that $S_{L/k}\beta = o$. 

We now proceed in the opposite direction. Let $L$ be cyclic of degree
$p^{n-1} > 1$ over $k$. We shall construct an extension $K$ which is
cyclic \textit{over $k$} and contains $L$ as a proper subfield. Let
$\sigma$ be a generator of the galois group of $L/k$. 

Let us choose $\beta \in L$, such that
$$
S_{L/k} \beta = 1.
$$

Now $ S_{L/k}(\mathscr{P} \beta) = \mathscr{P}(S_{L/k} \beta) = o $
which shows that 
$$
\mathscr{P} \beta = \sigma \alpha - \alpha
$$
for\pageoriginale some $\alpha$ in $L$. Also, $\alpha$ is not in
$k$. We claim that for $\lambda \in k$,   
\begin{equation}
x^p - x - \alpha - \lambda \tag{4}\label{c6:eq4}
\end{equation}
is irreducible over $L [x]$. For, if it is not irreducible, it is
completely reducible in $L$. Let $\omega$ be a root of $x^p - x-
\alpha-\lambda$ in $L$. Then 
$$
\omega^p - \omega = \alpha + \lambda.
$$

This means that
$$
\mathscr{P} (\sigma \omega - \omega) = \sigma (\mathscr{P} \omega) -
\mathscr{P} \omega=\sigma \alpha - \alpha = \mathscr{P} \beta. 
$$
 
Thus
$$
\mathscr{P} (\sigma \omega - \omega - \beta) = o 
$$
or $\sigma \omega - \omega - \beta = \mu $. Since $S_{L/k}(\sigma 
\omega-\omega)=o$ and $S_{L/k}\mu =o$, it follows that
$S_{L/k}\beta=o$, which is a contradiction. Thus for $\lambda \in
k$, \eqref{c6:eq4} is irreducible in $L [x]$. Let $\omega$ be a root of this
irreducible polynomial for some $\lambda$. Then $K=L(\omega)/L$ is
cyclic of degree $p$. 

Let $\bar{\sigma}$ denote an extension of $\sigma$ to an isomorphism
of $K/k$ in $\Omega$, the algebraic closure of $k$. Then
$\bar{\sigma}$ is not identity on $L$. Since $\mathscr{P}
\omega = \alpha + \lambda$ and $\lambda \in k$, we get 
$$
\bar{\sigma}(\mathscr{P} \omega) = \sigma \alpha + \lambda.
$$
 
Now
$$
\mathscr{P}(\bar{\sigma}\omega - \omega)= \bar {\sigma}
(\mathscr{P}\omega) - \mathscr{P} \omega = \sigma \alpha - \alpha =
\mathscr{P} \beta 
$$
and, again, we have
$$
\mathscr{P}(\bar{\sigma} \omega - \omega - \beta) =o 
$$
or that $\bar{\sigma} \omega  = \omega +\beta + \mu$ which shows that
$\bar{\sigma}$ is an automorphism of $K/k$. If $t$ is any integer, 
$$
\bar{\sigma}^t \omega = \omega + \beta + \beta^\sigma + \cdots +
\beta^{\sigma^{t-1}} + t \mu. 
$$\pageoriginale

This shows that 1, $\bar{\sigma}, \bar{\sigma}^2,\ldots ,
\bar{\sigma}^{p^{n-1}}$ have all different effects on $\omega$, so
that they are distinct automorphisms of $K/k$. But, since $K/k$ has
degree $p^n$, it follows that $K/k$ is cyclic of degree $p^n$. We
have, hence, the important 

\begin{thm}\label{c6:thm11} %the 11
 If $k$ is a field of characteristic $p$ and admits a cyclic extension
 $K$ of $k$ containing $L$ and of degree $p^m$, $m>n$, for every $m$. 
\end{thm}

In fact, if $k$ is an infinite field, we may very $\lambda$ over $k$
and obtain an infinity of extensions $K$ over $k$ with the said
property. It follows theorem~\ref{c6:thm10}. 

\begin{coro*} % coro
If $k$ is a field of characteristic $p$ and admits a cyclic
  extension of degree $p^n$ for some $n \geq 1$, then its algebraic
  closure is of infinite degree over it. 
\end{coro*}

\section{Artin-Schreier theorem}\label{c6:s5}%%% 5

We had obtained, in the previous section, a sufficient condition on a
field so that its algebraic closure may be of infinite degree over
it. We would like to know if there are fields $K$ which are such that
their algebraic closures are finite over them. The complete answer to
this question is given by the following beautiful theorem due to
\textit{Artin} and \textit{Schreier}. 

\begin{thm}\label{c6:thm12} % the 12
If $\Omega$ it an algebraically closed field and $K$ is a
  subfield such that 
$$
1 < (\Omega : K) < \infty
$$
then\pageoriginale $K$ has characteristic zero and $\Omega = K(i)$,
where $i$ is a root of $x^2+1$.  
\end{thm}

\begin{proof} % pro
The proof is as follows:-
\begin{enumerate}[1)]
\item $K$ is a perfect field. For if not $K^{p^{- \infty}} \subset
  \Omega$ and $K^{p^{- \infty}}$ is of infinite degree over $K$. This
  shows that $\Omega / K$ is a finite separable extension. Since it is
  is trivially normal, it is galois extension. Let $n$ be the order of
  the galois group $G$ of $\Omega / K$. 

\item If $p$ is the characteristic of $K$, then $p\nmid n$ (if $p \neq
  o$). 
For, if $p/n$, then $G$ has a subgroup of order $p$ generated by an
element $\sigma$. Let $L$ be the fixed field of this subgroup. Then
$\Omega / L$ is a cyclic extension of degree $p$. By corollary of
theorem~\ref{c6:thm11} it follows that $\Omega / L$ has infinite
degree. This is a contradiction.  

Therefore, the order $n$ of $G$ is prime to the characteristic of $K$,
if different from zero. 

\item Let $q$ be a prime dividing $n$. Then $q \neq p$. Let $L$ now be
  the fixed field of a cyclic subgroup of $G$, of order $q$. 

Then $\Omega / L$ is cyclic, of order $q$. Now $L$ contains the
primitive $q$th root $\rho$ of unity. For, if not since $\rho$ satisfies an
irreducible polynomial of degree $q-1$, it follows that $L(\rho)/L$
has degree $q-1$. But  
$$
(L(\rho):L)|(\Omega : L)
$$
which means that $L(\rho)=L$. By theorem~\ref{c6:thm9}, therefore 
$$
\Omega= L(\omega_1)
$$
where $\omega^q_1 = \omega \in L$.

\item Any\pageoriginale irreducible polynomial over $L$ is either
  linear or of degree $q$. For, if $t$ is its degree and $\alpha$, $a$
  root of it then $t=(L(\alpha) : L)$ divides $q$. Thus, every
  polynomial in $L[x]$ splits in $L$ into product of linear factors
  and of factors of degree $q$. 

Consider, in particular, the polynomial $x^{q^2} - \omega$. In $\Omega$,
we can write  
\begin{equation*}
x^{q^2}-\omega = \pi_{\mu} (x - \mu \sqrt[q^2]{\omega}) \tag{5}\label{c6:eq5}
\end{equation*}
where $\mu$ runs through all $q^2$th roots of unity. Since
$(\Omega:L)>1$, the polynomial $x^{q^2}-\omega$ has, in $L[x]$, an
irreducible factor of degree $q$. Since this factor is formed by $q$ of
the linear factors on the right of \eqref{c6:eq5}, this factor is of the form   
$$
x^q + \cdots + \varepsilon \sqrt[\varepsilon]{\omega}, 
$$
$\varepsilon$ being a $q^2$th root of unity. We assert that this is a
primitive $q^2$th root of unity. For if not, it is either $1$ or a
primitive $q$th root of unity. Since $\varepsilon \sqrt[q]{\omega} \in L$,
it would then follow that $\sqrt[q]{\omega} \in
L$. Therefore we get  
$$
\Omega= L(\varepsilon).
$$

\item Let $\Gamma$ be the prime field contained in $L$. Consider the
  field $\Gamma(\varepsilon)$. Let $\Gamma(\varepsilon)$ contain the
  $q^\nu$th but not $q^{\nu+1}$ roots of unity. This integer certainly
  exists. For, if $L$ has characteristic zero $\nu =2$. If $L$ has
  characteristic $p$ (and this is different from $q$, from \eqref{c6:eq2}), then
  $\Gamma(\varepsilon)$ contains $p^f$ elements where $f$ is the
  smallest positive integer with $p^f \equiv 1 (\mod q^2)$. If
  $q^{\nu}$ is the largest power of $q$ dividing $p^f-1$ then
  \pageoriginale $\Gamma(\varepsilon)$ contains the $q^{\nu}$th, but
  not the $q^{\nu+1}$ the roots of unity.  

Let $\rho$ be a primitive $q^{v+1}$th root of unity. Then $\rho$
satisfies a polynomial of degree $q$(irreducible) over $L$. Thus
$\Gamma(\rho)$ is of degree $q$ over $\Gamma(\rho) \cap L$. Now,
$\rho$ being a primitive $q^{\nu+1}$th root of unity $\rho^q$ is a
$q^{\nu}$th root of unity and so is contained in
$\Gamma(\varepsilon)$. But all the roots of $x^q-\rho^q$ are 
$q^{v+1}$th roots of unity. Thus $x^q - \rho^q$ is the irreducible
polynomial of $\rho$ over $\Gamma(\varepsilon)$. Hence  
\[
\xymatrix@R=0.5cm{
& \Gamma(\rho) \ar@{-}[dl] \ar@{-}[dr] & \\
\Gamma (\varepsilon)\ar@{-}[dr] & & \Gamma(\rho) \cap L \ar@{-}[dl]\\
& \Gamma & 
}
\]
and $(\Gamma(\rho):\Gamma(\varepsilon))=q=(\Gamma(\rho):(\Gamma(\rho) 
\cap L)$.  

If $\Gamma$ is a finite field, then $(\Gamma(\rho)/{\Gamma}$ is
cyclic and it cannot have two distinct subfields
$\Gamma(\varepsilon)$ and $\Gamma(\rho) \cap L$ over which it has
the same degree. Since $\varepsilon \notin L$, it follows that
$\Gamma(\varepsilon)$ and $\Gamma(\rho) \cap L$ are distinct and so,
$\Gamma$ has characteristic zero. In this case, if $q$ is odd,
$\Gamma(\rho)/\Gamma$ is cyclic and the same thing holds. Thus
$q=2$. We know, then that $\nu =2$. If $i$ denotes a fourth root of
unity, then 
$$
\Omega =L(i). 
$$

\item Form above, it follows that $n=2^t$ for some $t$ and $K$ does
  not contain the 4th root of unity. Now $t=1$, for if not, let
  $t>1$. Let $M$ be a subfield of $L$ such that $(L:M)=2$. Then
  $(\Omega:M(i))=2$ and, by above considerations, $M(i)$ cannot
  contain $1$ which is a contradiction. So 
$$
K(i) = \Omega.
$$
\end{enumerate}

We\pageoriginale shall make use of this theorem in the next chapter. 
\end{proof}


\section{Kummer extensions}\label{c6:s6}%%%% 6

We now study the structure of finite abelian extensions of a field
$k$. We use the following notation:- 

$k$ is a field of characteristic $p$, not necessarily $>o$. 

$n$ is a positive integer not divisible by $p$ if $p \neq o$,
otherwise arbitrary. 

$\alpha$, the group of non-zero elements of $k$. A generic element of
$\alpha$ will also be denoted, following Hasse and Witt, by $\alpha$. 

$\alpha^n$ is the group of $n$th powers of elements of $\alpha$.

$\omega$, a subgroup of $\alpha$ containing $\alpha^n$ and such that 
\begin{equation*}
(\omega:\alpha^n) < \infty. \tag{6}\label{c6:eq6}
\end{equation*}

\textit{Let $k$ contain the $n$th roots of unity}. We shall establish
$a(1,1)$ correspondence between abelian extensions of $k$ of exponent
dividing $n$ and subgroups $\omega$ of $\alpha$ satisfying \eqref{c6:eq6}. 

Let $K$ be an extension field obtained from $k$ by adjoining to $k$
the $n$th roots of all the elements of $\omega$. We shall obtain some
properties of $K$. 

Since $\omega/\alpha^n$ is a finite group, let $\lambda_1, \ldots,
\lambda_t$ in $\omega$, form a system of generators of $\omega$ mod
$\alpha^n$. Then any $\omega \in \omega$ is of the form  
$$
\omega = \lambda^{a1}_{1}, \ldots, \lambda^{ai}_t \alpha^n
$$
where $a_1, \ldots, a_t$ are integers. Put $\Lambda_i =
\lambda^{1/n}_i, i=1, \ldots,t$. Then $\Lambda_i$ is uniquely
determined up to multiplication by an $n$th root of unity, which is
already in $k$. This means that  
$$
K=k(\Lambda_1, \ldots, \Lambda_t).
$$\pageoriginale

Therefore
\begin{enumerate}[1)]
\item $K/k$ \textit{is a finite extension}.

Each $\Lambda_i$ is a root of a polynomial of the form $x^n -
\lambda_i$. This polynomial, by the condition on $n$, is separable over
$k$. Also, $x^n - \lambda_i$ splits completely in $K$. Thus  

\item \textit{$K/k$ is a finite galois extension and is the splitting
  field of the polynomial 
$$
f(x) = \prod^t_{i=1} (x^n - \lambda_i)
$$
over $k[x]$.}

Let us denote by $\Lambda$ the group generated by $\Lambda_1, \ldots,
\Lambda_t$ and $\alpha$. Let $G$ denote the galois group of $K/k$. In
the first place, 

\item $\Lambda_{/\alpha}  \simeq \omega_{/\alpha} n$. \textit{(These are
  isomorphic groups)}. 

Consider the homomorphism $\Lambda \to \Lambda^n$ of $\Lambda$ into
itself. It takes $\alpha$ into $\alpha^n$. The kernels in $\Lambda$
and $\alpha$ are both the same. Since $\Lambda$ is taken to $\omega$
by this homomorphism, it follows that $\Lambda / \alpha \simeq
\omega/\alpha^n$. Incidentally, therefore, $\Lambda/\alpha$ is a
finite group. 

We shall now prove the important property,

\item $G$ \textit{is isomorphic to} $\omega/\alpha^n$.

(This proves that $K/k$ is a finite abelian extension.)

In order to prove this, consider the 'pairing', $(\tau,\Lambda)$ of
$G$ and $\Lambda$, given by   
\begin{equation*}
(\sigma,\Lambda)=\Lambda^{1-\sigma}, \tag{7}\label{c6:eq7}
\end{equation*}
\end{enumerate}
$\sigma \in G$, $\Lambda \in \Lambda$.\pageoriginale Because of
definition of  $\Lambda$, 
$$
(\sigma,\Lambda)^n = \frac{\Lambda^n}{(\Lambda^n)^{\sigma}}=1.
$$

Thus $(\sigma,\Lambda)$ is an $n$th root of unity. Also,
$$
(\sigma \tau,\Lambda)  = \frac{\Lambda}{\Lambda^{\sigma \tau}} =
\frac{\Lambda}{\Lambda^{\sigma}}= 
\left(\frac{\Lambda}{\Lambda^{\tau}} \right)^{\sigma} =
\frac{\Lambda}{\Lambda^{\sigma}} \cdot  \frac{\Lambda}{\Lambda^{\tau}} =
(\sigma,\Lambda) \cdot (\tau,\Lambda).   
$$

Furthermore,
$$
(\sigma,\Lambda\Lambda')= 
\frac{\Lambda'\Lambda'}{(\Lambda'\Lambda)^{\sigma}} = 
\frac{\Lambda}{\Lambda^{\sigma}}, \frac{\Lambda'}{\Lambda^{\sigma}} =
(\sigma,\Lambda)(\sigma,\Lambda').   
$$

Thus $(\sigma,\Lambda)$ defined by~\eqref{c6:eq7} is a pairing. 

Let $G_o$ be the subgroup of $G$ consisting of all $\sigma$ with
$(\sigma,\Lambda) =1$, for all $\Lambda$. Since $\Lambda$ is generated
by $\Lambda_1, \Lambda_2, \ldots, \Lambda_t, \alpha$ we get  
$$
\Lambda_i=\Lambda^{\sigma}_i.
$$

But $\Lambda_1, \ldots,\Lambda_t$ generate $K$. Therefore,$\beta=
\beta^{\sigma}$ for all $\beta \in K$.  


By galois theory, $\sigma =1$. Thus $G_o=(1)$.

Let $\Lambda_o$ be the subgroup of all $\Lambda$ with
$(\sigma,\Lambda)=1$, for all $\sigma$. For the same reason as before,
$\Lambda_o= \alpha$. Thus, by theorem on pairing, $G$ is isomorphic to
$\Lambda/_{\alpha}$. \eqref{c6:eq4} is thus proved.  

Observe, now, that since $\Lambda^n \subset \alpha$, it follows the $G$ is
an abelian group whose exponent divides $n$. 

We will denote $K$ symbolically by $K=k(\sqrt[\tau]{\omega})$.

We shall now prove 

\begin{thm}\label{c6:thm13} %the 13
Let $K/k$ be a finite extension with abelian galois group $G$ of
  exponent dividing $n$, then $K=k(\sqrt[n]{\omega})$ for a sub
  group $\omega \subset \alpha$ with $\omega_{/ \alpha}n$ finite. 
\end{thm}

\begin{proof} %pro 
Let\pageoriginale $\bar{\Lambda}$ denote the group of non-zero
elements of $K$ with the property $\Lambda^n \in \alpha$ for $\Lambda
\in \bar{\Lambda}$. Then, by considering the homomorphism $\Lambda \to
\Lambda^n$, it follows that   
$$
\bar{\Lambda_{/\alpha}} \simeq \omega_{/\alpha} n,
$$
where $\omega = \bar{\Lambda}^n$. We shall prove that $K=
k(\sqrt[n]{\bar{\omega}})$. In order to prove this, it suffices to prove
that  
\begin{equation*}
G \simeq \omega_{/\alpha}n. \tag{8}\label{c6:eq8}
\end{equation*}
\end{proof}

For, in that case, construct the field
$k(\sqrt[n]{\omega})$. Because of the properties of $K$, it
follows that $K \supset k(\sqrt[n]{\bar{\omega}})$. But, by the
previous results, $(k(\sqrt[n]{\omega}):k)=$ order of
$\omega_{/\alpha} n=$ order of $G$. 

Hence
$$
K= k(\sqrt[n]{\omega}).
$$

We shall, therefore, prove \eqref{c6:eq8}.

Let $G*$ denote the dual of $G$. For every $\Lambda$ in
$\bar{\Lambda}$, define the function 
$$
\chi_{\Lambda} \, (\sigma)= \Lambda^{1 - \sigma}
$$
on $G$ into $\bar{\Lambda}$. Since $\Lambda^n \in \alpha$, it follows
that $\chi_{\Lambda}(\sigma)$ is an $n$th root of unity. Also, 
$$
\chi_{\Lambda}(\sigma \tau) = \frac{\Lambda}{\Lambda^{\sigma
    \tau}}= \frac{\Lambda}{\Lambda^{\sigma
}}  \left(\frac{\Lambda}{\Lambda^{\tau}} \right)^{\sigma}
=\frac{\Lambda}{\Lambda^{\sigma }} \cdot  \frac{\Lambda}{\Lambda^{
    \tau}} =\chi_{\Lambda}(\sigma) \cdot \chi_{\Lambda}( \tau) 
$$
so that $\chi_{\Lambda}$ is a character of $G$.

Let $\chi$ be any character of $G$. Since the exponent of $G$ divides
$n$, $\chi^n(\sigma)=1$ for all $\sigma$. Therefore, $\chi(\sigma)$ is
an $n$th root of unity and hence is an element of $k$. Also, since
$\chi$ is a character of $G$, 
$$
\chi (\sigma \tau) = \chi(\sigma) \chi(\tau)
$$
which\pageoriginale equals $\chi(\sigma)\cdot
(\chi(\tau))^{\sigma}$. Thus $\chi(\sigma)$ is a one cocycle and, by
theorem~\ref{c6:thm7},  
$$
\chi(\sigma)= \beta^{1-\sigma}
$$
for $\beta \in K$. But $(\chi(\sigma))^n=1$ so that 
$$
\beta^n=(\beta^n)^{\sigma}
$$
for all $\sigma$ or $\beta^n \in \alpha$. This means, by definition of
$\bar{\Lambda}$ that $\beta \in \bar{\Lambda}$. 

Consider the mapping $\omega \to \chi_{\Lambda}$ of $\bar{\Lambda}$
into $G^*$. This is, trivially a homomorphism. By above, it is a
homomorphism onto. The kernel of the homomorphism is set of $\Lambda$
for which $\chi_{\Lambda}(\sigma) =1$ for all $\sigma$. That is  
$$
\Lambda =\Lambda ^{\sigma}
$$
for all $\sigma$. By galois theory $\Lambda  \in \alpha$. But every
element in $\alpha$ satisfies this condition. Thus $\alpha$ is
precisely the kernel, and  
$$
\bar{\Lambda }_{/ \alpha} \simeq G^*.
$$

Since $G$ is a finite abelian group, $G \simeq G^*$ and our theorem is
completely proved. 

We call a finite abelian extension $K/k$, a \textit{Kummer extension}
if  
\begin{enumerate}[1)]
\item $k$ contains the $n$th roots of unity, $p \nmid n$, if $p(\neq
  o)$ is characteristic of $k$, 

\item $K/k$ has a galois Galois group of exponent dividing $n$.What
  we have then proved is  
 \end{enumerate}

\begin{thm}\label{c6:thm14}%%%14 
The Kummer extensions of $k$ stand in $(1,1)$ correspondence with
  subgroups $\omega$ of $\alpha$ with $\omega_{/{\alpha}}n$ finite. 
\end{thm}


\section{Abelian extensions of exponent \texorpdfstring{$p$}{p}}\label{c6:s7}\pageoriginale %%% 7

We had introduced earlier the operator $\mathscr{P}$ which is defined
by 
$$
\mathscr{P} x =x^p-x. 
$$

Let $k$ be a field of characteristic $p$ and denote by $k^+$ the
additive group of $k$. Then $\alpha \to \mathscr{P} \alpha$ is a
homomorphism of $k^+$ into itself. The kernel of the homomorphism is
precisely the set of elements in $\Lambda$, the prime field of
characteristic $p$, contained in $k$. 

If $\alpha \in k$, we denote by $\dfrac{\alpha}{\mathscr{P}}$, a root
of the polynomial $x^p-x- \alpha$. Obviously
$\dfrac{\alpha}{\mathscr{P}}$ is $p$ valued. Also, 
\begin{equation*}
\frac{\alpha+ \beta}{\mathscr{P}}=\frac{\alpha}{\mathscr{P}}+\frac{
  \beta}{\mathscr{P}}. \tag{9}\label{c6:eq9} 
\end{equation*}

Let us denote by $\mathscr{P}k^+$, the subgroup of $k^+$ formed by
elements $\mathscr{P} \alpha$, $\alpha \in k^+$. Let $\omega$ be a
subgroup of $k^+$ with the properties, 
$$
k^+ \supset \omega \supset \mathscr{P} k^+
$$
$$
\omega/_{\mathscr{P} k^+} \text{ is finite}.
$$

Let $K$ be the extension field of $k$, formed by adjoining to $k$ all
the elements $\dfrac{\alpha}{\mathscr{P}}$, fro $\alpha \in
\omega$. We denote 
$$
K=k(\frac{\omega}{\mathscr{P}}).
$$

We then obtain in exactly the same way as before 
\begin{enumerate}[1)]
\item $K/k$ is a finite abelian extension. 

\item The galois group $G$ of $K/k$ is isomorphic with
  $\omega/_{\mathscr{P} k^+}$. 

 \item $G$ is an abelian group of exponent $p$. For the proofs, we
   have to use additive, instead of multiplicative, pairing and the
   property \eqref{c6:eq9}. 
\end{enumerate}

Suppose,\pageoriginale now, on the other hand, $K/k$ is a finite
abelian extension of exponent $p$, where $p$ is the characteristic of
$k$.  

Then
$$
K= k (\frac{\omega}{\mathscr{P}}),
$$
where $\omega$ is a subgroup of $k^+$ with $\omega/\mathscr{P} k^+$
finite. For the proof of this, we have to use additive instead of
multiplicative, cohomology. For, let $G$ be the galois group of $K/k$
and $G^\ast$ its character group. Denote, by $\bar{\Lambda}$, the additive
subgroup of $K^+$ formed by elements $\Lambda$ with $\mathscr{P}
\Lambda \in k^+$. Denote, by $\omega$, the group $\mathscr{P}
\bar{\Lambda}$. Then  
$$
\bar{\Lambda}_{/k^+} \simeq \omega_{/ \mathscr{P} k^+}
$$

Define $\chi_{\Lambda}(\sigma)$ by 
$$
\chi_{\Lambda}(\sigma) = \Lambda- \Lambda^{\sigma}.
$$

Then
$$
\mathscr{P}(\chi_{\Lambda}(\sigma)) =  \mathscr{P} \Lambda
-(\mathscr{P} \Lambda)^{\sigma}). 
$$

But, $\mathscr{P} \Lambda \in k^+$ by
definition. Hence, $\mathscr{P}(\chi_{\Lambda}(\sigma))=o$. 

Therefore, $\chi_{\Lambda}(\sigma)$ is an element of $\Gamma$. The rest
of the proof goes through in the same way and we have the  

\begin{thm}\label{c6:thm15} %the 15
 If $k$ has characteristic $p \neq o$, the finite abelian
  extensions $K/k$ of exponent $p$ stand in $a(1,1)$ correspondence
  with subgroups $\omega$ of $k^+$ such that $\omega/ \mathscr{P} k^+$
  is finite, and then $K = k(\frac{\omega}{\mathscr{P}})$.
\end{thm}


\section{Solvable extensions}\label{c6:s8}%%%%% 8

We propose to study now the main problem of the theory of algebraic
equations, namely the `solution' of algebraic equations by radicals  
 
$k$ will\pageoriginale denote a field of characteristic $p$,($p=o$ or
$p \neq o$), 
$\Omega$ will be its algebraic closure and $n,n_1,n_2, \ldots$ integers
$>o$ which are prime to $p$, if $k$ has characteristic $p \neq o$,
otherwise arbitrary. An element $\omega \in \Omega$ will be said to
be a \textit{simple radical} over $k$ if $\omega^n \in k$, for some
integer $n$. $k(\omega)/k$ is then said to be a \textit{simple radical
  extension}. $k(\omega)/k$ is clearly separable. If $\omega$ is a
root of $x^p-x-a$, for a $\in k, \omega$ is called a \textit{simple
  pseudo radical} over $k.k(\omega)/k$ is a \textit{pseudo radical
  extension} and is separable. In fact, it is a cyclic extension over
$k$. This situation occurs, only if $p \neq o$.  

An extension field $K/k$ is said to be a \textit{generalized radical
  extension} if it is a finite tower  
$$
k= K_o \subset K_1 \subset K_2 \subset \cdots \subset K_m=K
$$
where $K_i/ K_{i-1}$ is either a simple radical or a simple pseudo
radical extension. Every element it $K$ is called a \textit{
  generalized radical}. A typical element would be  
$$
\sqrt[n_1]{a_1 + \sqrt[n_2]{a_2 + \frac{a_3}{\mathscr{P}}+.}}. 
$$

Clearly, $K/k$ is a separable extension. A generalized radical extension
is called a \textit{radical extension} if $K_i/K_{i-1}$ is a simple
radical extension for every $i$. A typical element of $K$ would, then be  
$$
\sqrt[n_1]{a_1 + \sqrt[n_2]{a_2 + \cdots}}.
$$

Let $f(x)$ be a polynomial in $k[x]$ and let it be separable. Let $K$
be its splitting field. Then $K/k$ is a galois extension. The galois
group $G$ of $K/k$ is called the \textit{group of the polynomial}
$f(x)$. The polynomial $f(x)$ is said to be \textit{solvable by
  generalized\pageoriginale radicals}, if $K$ is a subfield of a
generalized radical 
extension. It is, then, clear that the roots of $f(x)$ are generalized
radicals. In order to prove the main theorem about solvability of a
polynomial by generalized radicals, we first prove some lemmas. 

\setcounter{lem}{0}
\begin{lem}\label{c6:lem1} % lem 1
Every generalized radical extension is a subfield of a generalized
radical extension which is galois, with a solvable galois group. 
\end{lem}


\begin{proof} % pro
Let $K/k$ be a generalized radical extension so that  
$$
k= K_o \subset K_1 \subset  \cdots \subset K_m=K,
$$
where $K_i/ K_{i-1}$ is a simple radical or simple pseudo radical
extension. Let $K_i=K_{i-1}(\omega_i)$. Then either $\omega^n_i \in
K_{i-1}$ for some $n_i$ or $\mathscr{P} \omega_i \in K_{i-1}$. Let $(K :
k) = n$. Put $N=n$,  if $k$ has characteristic zero; otherwise, let  
$$
n=N \cdot p^a,
$$
$a \geq o, (p,N)=1$. Let $\rho$ be a primitive $N$th root of unity. 
\end{proof}

Let $L_1 = K_o (\rho)$. The $L_1/K_o$ is a simple radical
extension. Since $K_1=K_o(\omega_1)$, put $L_2=K_o(\rho,
\omega_1)$. Then $L_2$ is the splitting field of the polynomial
$(x^N-1)(x^{n1}-a_1)$ or $(x^N-1)(\mathscr{P}x-a_1)$ depending on whether
$\omega_1$ is a simple radical or a simple pseudo radical. In any
case, $L_2/K_o$is a galois extension. Furthermore 
$$
L_2=L_1 (\omega_1)
$$
so that $L_2/L_1$ is cyclic, since, when $\omega_1$ is a simple
radical, $L_1$ contains the requisite roots of unity. Let
$\sigma_1,\ldots, \sigma_{\ell}$ be the distinct\pageoriginale
automorphisms of $L_2 / K_o$. Put    
$$
f(x) = \prod^\ell_{ i = 1} (x^{n2} -a^{\sigma i}_2),
$$
if $\omega_2$ is a simple radical with $\omega^{n_2}_2 = a_2$, and 
$$
f(x) = \prod^{\ell}_{i=1} (\mathscr{P} x - a^{\sigma_i}_2),
$$
if $\omega_2$ is a simple pseudo radical with $\mathscr{P} \omega_2 =
a_2$.

Then $f(x)$ is a  polynomial in $k[x]$. Let $L_3$ be its splitting
field. Then $L_3$ is galois over $k$. Also, $L_3$ is splitting field
of $f(x)$  over $L_3$ so that $L_3 / L_2$ is either a Kummer extension
or else, an abelian extension of exponent $p$. In this way, one
constructs a galois extension $T$ of $k$ such that  
$$
L_o = k \subset L_1 \subset L_2 \subset L_3 \subset \cdots \subset
L_{m-1} \subset T = L_m, 
$$
where $L_i / L_{i - 1} $ is either a kummer extension or an abelian
extension of exponent $p$. Clearly, $L_i/L_{i-1}$ and, hence, $T/k$ is
a generalized radical extension. Let $G_i, i = o, 1, 2, \ldots , m$ be
the galois group of $T/L_i$. Then 
$$
G = G_o \supset G_1 \supset \ldots \ldots \supset G_m = (e)
$$
is a normal series. Further, by our construction, $G_{ i - 1} / G_i$
is the galois group of $L_i / L_{ i-1}$ and hence, abelian. Thus, $G$
is a solvable group. The lemma is thus proved.  

\begin{lem}\label{c6:lem2} %lem 2
If $K/k$ is a finite solvable extension, then $K$ is a subfield of a
generalized extension over $k$. 
\end{lem}

\begin{proof} %pro 
Let $G$ be the galois group of $K/k$ and $G$ solvable. Let $n$ be the
order of $G$ and put  
$$
n=p^{a}N
$$\pageoriginale
with the same connotation, as before. Let $\rho$ be a primitive
$N$th root of unity and $L=k(\rho)$. Then $L/k$ is a simple radical
extension. Let $M$ be a composite of $K$ and $L$. Then $M/L$ is
a galois extension with a galois group which is isomorphic to a
subgroup of $G$ and hence, solvable. Let $G_o$ be the galois group of
$M/L$ and let it have a composition series  
$$
G_o \supset G_1 \supset \cdots \supset G_m=(e).
$$
\end{proof}

Then $G_i/G_{i+1}$ is a cyclic group of prime degree. Let $L_o=L$,
$L_1, \ldots L_m=M$ be the fixed fields of $G_o,G_1, \ldots,G_m$
respectively. Then $L_i/L_{i-1}$ is a cyclic extension of prime
degree. Since $L_{i-1}$ contains the requisite roots of unity, $L_i$
is a simple radical or a simple pseudo radical extension of
$L_{i-1}$. Then $M/k$ is a generalized radical extension and our lemma
is proved. 

We are, now, ready to prove 

\begin{thm}\label{c6:thm16} %the 16
 A separable polynomial $f(x) \in k[x]$ is solvable by generalized
 radicals if and only if its group is solvable. 
\end{thm}

\begin{proof} % pro
Let $K$ be the splitting field of $f(x)$ and $G$ the galois group of
$K/k$. Suppose $f(x)$ is solvable by generalized radicals. Then $K
\subset L$ where $L/k$ is galois and by lemma~\ref{c6:lem1}, has
solvable galois group $H$. Let $G_o$ be the galois group of $L/K$ Then
$H/G_o$ is isomorphic to $G$ and so $G$ is solvable.  
\end{proof} 

Let, conversely, $G$ be solvable. Then, by lemma~\ref{c6:lem2}, $K$
is contained in a generalized radical extension and so $f(x)$ is
solvable by generalized radicals.  

We can\pageoriginale easily prove 

\begin{coro*} % coro
A separable polynomial $f(x) \in k[x]$ is solvable by radicals if
  and only if its splitting field has a solvable galois group of order
  prime to the characteristic of $k$, if different from zero. 
\end{coro*}

Let $k$ be a field. The polynomial
$$
f(x) = x^m-x_1 x^{m-1} + x_2 x^{m-2} \cdots + (-1)^m x_m, 
$$
where $x_1, \ldots , x_m$ are algebraically independent over $k$, is
said to be the \textit{general polynomial} of the $m$th degree over
$k$. It is so called, because every monic polynomial of degree $m$
over $k$ is obtained by specializing the values of $x_1, \ldots , x_m$
to be in $k$. Let $L= k(x_1, \ldots , x_m)$. Let $y_1, \ldots , y_m$
be roots of $f(x)$ over $L$. Then 
$$
f(x) = (x-y_1) \cdots(x -y_m)
$$
and $y_1, \ldots , y_m$ are distinct. The splitting field $k(y_1,
\ldots , y_m)$ of $f(x)$ over $L$ is a galois extension whose galois
group is isomorphic to $S_m$. Hence, the general polynomial of the $m$th
degree over $k$ has a group isomorphic to the symmetric group on $m$
symbols. 

But,$S_m$ is not solvable, if $m>4$, so that in virtue of 
theorem~\ref{c6:thm16}, we have the theorem of \textit{Abel}. 

\begin{thm}\label{c6:thm17} %the 17
 The group of the general polynomial $f(x)$ of the $m$th degree is
  isomorphic to the symmetric group $S_m$ on $m$ symbols and hence,
  for $m>4,f(x)$ is not solvable by generalized radicals. 
\end{thm}

We shall now explicitly show how to obtain the roots of a polynomial
of degree $\leq 4$ in terms of generalized radicals. 

Let\pageoriginale $f(x)$ be a general polynomial of degree $m$ over
$k$ and let $K$ be the splitting field. $K/k$ has the group $S_m$. Let
$y_1, \ldots , y_m$ be the roots of $f(x)$. Put 
 $$
 D= \prod_{i < j} (y_i- y_j)^2.
 $$
 
 Then $D$ is fixed under all permutations in $S_m$ and, hence, $D \in
 k$. If we assume that the characteristic of $k$ is $\neq 2$, then
 $k(\sqrt{D})$ is a galois extension of $k$ and $K/k(\sqrt{D})$ has
 the galois group isomorphic to the alternating  group on $m$
 symbols. $D$ called the \textit{discriminant of the polynomial}
 $f(x)$. 
 
 Let us, first, consider the general polynomial of the second degree
 $$
 f(x) = (x-y_1) (x-y_2)= x^2- x_1 x+x_2.
 $$ 
 
 Then  
 $$
 y_1  +y_2 =x_1, y_1 y_2 =x_2.
 $$
 
 Suppose, now, that $k$ has characteristic $\neq 2$. Then  
 $$
 D= (y_1-y_2)^2 = (y_1+y_2)^2 -4y_1y_2=x^2_1-4x_2.
 $$
 
 Also, $y_1 +y_2 = x_1, y_1-y_2= \pm \sqrt{D}$ so that  
 $$
 y_1= \frac{x_1+ \sqrt{D}}{2},{y_2}= \frac{x_1-\sqrt{D}}{2} 
 $$
 and $K=k(\sqrt{D})$ is the splitting  field of $f(x)$ and is a
 radical extension. 
 
 Let, now, $k$ have characteristic 2. Then $x_1 \neq o$, since $f(x)$
 is separable. Put $x_1x$ instead  of $x$. Then  the polynomial $x^2-
 x+  \dfrac{x_2}{x^2_1}$ has roots $\dfrac{y_1}{x_1}$ and
 $\dfrac{y_2}{x_1}$. But this is a normal polynomial\pageoriginale so
 that if  $\lambda = \dfrac{x_2}{x_{1^2}}$, then   
 $$
 y_1 = x_1 \frac{\lambda} {\mathscr{P}}, y_2 =x_1
 \frac{\lambda}{\mathscr{P}}+x_1 
 $$
 and thus $k(\dfrac{\lambda}{\mathscr{P}})$ is a pseudo radical
 extension and is splitting field of $f(x)$. 
 
 We shall now study cubic and biquadratic polynomials.
 
 Let, first, $k$ \textit{have characteristic} $\neq 2$ or 3. Let
 $m=3$ or 4 and  
 $$
 f(x)= x^m - x_1 x^{m-1} + \cdots + (-1)^m x_m
 $$
 be the polynomial of the $m$th degree whose roots are $y_1, \ldots,
 y_m$. 
 
 If we put $x+  \dfrac{x_1}{m}$ instead of $x$, we get a polynomial
 whose roots are $y_1- \dfrac{x_1}{m}, \ldots , y_m- \dfrac{x_1}{m}$
 and which lacks the terms in $x^{m-1}$. 
 
 We shall, therefore, take the polynomial $f(x)$ in the form  
 $$
 f(x) =x^m + x_2 x^{m-2}+ \ldots + (-1)^m  x_m.
 $$
 
 If $y_1, \ldots , y_m$ are the roots, then
 $$
 y_1 +y_2 + \ldots + y_m = o.
 $$ 
 
 Also , $D_m= \prod\limits_{i < j}(y_i- y_j)^2$. A simple computation
 shown that 
 $$
 D_3 = -  4 x^3_2 - 27 x^2_3
 $$
 and 
\begin{gather*}
 D_4 =16 x^4_2x_4 -4 x^3_2 x^2_3 - 128 x^2_2 x^2_4 + 144 x_2 x_3^2 x_4\\ 
  - 27 x^4_3 + 256 x^3_4.
 \end{gather*}
 
 If $K$ is the splitting field of $f(x)$ over $L=k(x_1, \ldots, x_m)$
 then $L(\sqrt{D})$ is the fixed field of the  alternating  group
 $A_m$ and $L(\sqrt{D})/L$ is a radical extension. In order to study
 the extension $K/L(\sqrt{D})$, let us, first, take the case
 $m=3$. The symmetric group on 3 symbols, $S_3$, has the composition
 series  
 $$
 S_3 \supset A_3 \supset (e). 
 $$\pageoriginale
  $K/L(\sqrt{D})$ is thus a cyclic extension of degree 3. Let
 $\rho$ be a primitive cube root  of unity and let $M=L(\sqrt{D},
 \rho)$. Let $N=KM$ be a composite of $K$ and $M$. Then  $KM/M$ has
 degree 1 or 3, according as $\rho$ is in $K$ or not. In the
 first  case, $M=K$. In the second case, $KM/M$ is a cyclic
 extension  of degree  3 over $K$ and $M$ contains the cube roots
 of unity. Thus  $KM=M(\sqrt[3]{\omega})$, for some  $\omega \in M$. In
 order  of determine  this $\omega$, we use Lagrange's method.  
 
 $KM$ is the splitting field over $M$ of the polynomial $x^3+
 x_2x-x_3$. Let $y_1$, $y_2$, $y_3$ be roots of this polynomial. Put 
 \begin{align*}
\omega &= y_1 + \rho y_2 +\rho^2 y_3\\
\omega' &= y_1 + \rho^2 y_2 + \rho y_3.
   \end{align*}   
   
   Then $\omega^3 = \dfrac{27}{2} x_3  +3 \sqrt{D}(\rho-
   \frac{1}{2})$. Changing $\rho$ into $\rho^2$ we get
   $\omega'^3$. Hence 
   $$
   \omega = \rho^a \sqrt[3]{\frac{27}{2} x_3 +3
     \sqrt{D}(\rho- \frac{1}{2})} 
   $$
   and we have a similar expression for $\omega'$. Here $a \ge o$. In
   order to determine $\underbar{a}$, we use the fact that $\omega
   \omega'=-3x_2$ and so, choosing  the root of unity $\rho^a$ for
   $\omega$ arbitrarily, the root of unity in the expression for
   $\omega '$ is uniquely determine. Now 
   \begin{align*}
y_1 &+y_2 + y_3 =o\\
y_1 &+\rho y_2 +\rho^2 y_3 =\omega \\
y_1 &+\rho^2 y_2 + \rho y_3 =\omega'
   \end{align*}    
      and since the matrix
$$
\begin{pmatrix}
1 & 1 &1\\
1& \rho &\rho^2\\
1& \rho^2 &\rho
\end{pmatrix}
$$   
is\pageoriginale non-singular, the values of $y_1$, $y_2$, $y_3$ are
uniquely determined $K \cdot M$ is a radical of $L$ and contains $K$.  

Consider, now, the polynomial of the fourth degree 
$$
f(x) =x^4 +x_2 x^2 -x_3 x+ x_4
$$   
whose  roots  are $y_1$, $y_2$, $y_3$, $y_4$ with $y_1 +y_2
+y_3+y_4=o$. The galois group of the splitting field $K/k$ is
$S_4$. This has the composition series  
$$
S_4 \supset A_4 \supset B_4 \supset C_4 \supset (e). 
$$ 

Let $K_1$, $K_b$, $K_c$ be the fixed fields of $A_4$, $B_4$ and $C_4$
respectively. Now $A_4$ is the alternating group, $B_4$ the group
consisting of the  permutations 
$$
(1), (12)(34), (13) (24), (14)(23)
$$
and $C_4$ is the group of order 2 formed by 
$$
(1), (12)(34).
$$

$K_a = k(\sqrt{D})$ and is of degree 2 over $k$. Now  $K_b/ K_a$ is
of degree 3 and is cyclic. Hence $K_b= K_a (\theta)$ where $\theta
\in K$ is an  element fixed by $B_4$ but not by $A_4$. Such as
elements, for instance, is  
$$
\theta_1 = (y_1 +y_2) (y_3 +y_4).
$$ 
$\theta_1$ has 3 conjugates $\theta_1$, $\theta_2$, $\theta_3$
obtained  from $\theta_1$ by operating on $\theta_1$, by
representatives of cosets  of $A_4/B_4$. Thus 
\begin{align*}
\theta_2 &= (y_1 +y_3)(y_2+y_4)\\
\theta_3 &= (y_1 +y_4)(y_2+y_3).
 \end{align*} 
 
 Consider\pageoriginale the polynomial
 $$
 \varphi(x)= (x- \theta_1) (x- \theta_2)(x- \theta_3). 
 $$
 
 It is fixed under $A_4$ and so, its coefficients are in $K_a$. A
 simple computation shows that 
 $$
 \varphi(x) =x^3-2 x_2x^2 +(x_2^2- 4 x_4)x+x_3^2.
 $$ 
 
$\varphi(x)$ is called the \textit{reducing  cubic} or the
 \textit{cubic resolvent}. By  the method adopted for the solution  of
 the  cubic, if $\rho$ is a primitive cube root of unity and $M=K_a
 (\rho)$, then  $K_bM$, the composite, is a radical extension of
 $k$ in which $\theta_1$, $\theta_2$, $\theta_3$ lie. 

$K_c/K_b$ is of degree 2 and so $K_c =K_b(\alpha)$ where $\alpha$
 is fixed under $B_4$, but not by $C_4$. such as element is 
$$
\alpha = y_1+y_2.
$$ 

Now $\alpha^2 =(y_1+y_2)^2=-(y_1 +y_2)(y_3 +y_4)=- \theta_1$. Thus
$K_c= K_b (\sqrt{- \theta_1})$. Hence, if $K_bM= K_d$, then
$K_d(\alpha)$ is a radical extension of $k$ containing $\alpha$. Put
now 
$$
T=K_d(\alpha, \beta)
$$
where $\beta= y_1+y_3= \sqrt{- \theta_3}$. Then  $T$ is a radical
extension of $k$ containing $K$. The indeterminacy signs in taking the
square roots of $- \theta_1$ and $- \theta_3$ can be  fixed by
observing that   
$$
(y_1 +y_2)(y_1 + y_3)(y_1+y_4)= x_3.
$$
we now have 
\begin{align*}
y_1 +y_2 &=\sqrt{- \theta_1}, y_3 +y_4 =-\sqrt{- \theta_1} \\
y_1 +y_3 &=\sqrt{- \theta_2}, y_2+y_4 =-\sqrt{- \theta_2} \\
y_1 +y_4 &=\sqrt{- \theta_3}, y_2 +y_3 =-\sqrt{- \theta_3}. 
\end{align*}
for\pageoriginale which $y_1$, $y_2$, $y_3$, $y_4$ can be
obtained. We have, hence, proved  

\begin{thm}\label{c6:thm18} % Theorem  18 
If $K$ has characteristic $\neq 2$ or 3, then  the cubic and
  biquadratic polynomials over $k$ can be radicals. 
 \end{thm} 
 
 Let us, now, assume that $k$ \textit{has characteristic} 3. Let $x^3
 +  x_1 x^2 +x_2 x+x_3$ be a cubic polynomial and $K$, its splitting
 field. $K/k$ has the galois group $S_3$. Let $L$ be the fixed
 field. $K/k$ has the galois  group $S_3$. Let  $L$ be the  fixed
 field  of $A_3$. Then  
 $$
 L=k (\sqrt{D})
 $$
 where 
 $$
 D= (y_1- y_2)^2 (y_2- y_3)^2 (y_3- y_1)^2.
 $$
 
 $K/L$ is now a cyclic extension of degree 3 and since $k$ has
 characteristic 3, $K=L(\omega)$, where 
 $$
 \omega^3- \omega- \alpha=o
 $$
 for some  $\alpha \in L$. Thus $K$ is a generalized  radical
 extension. We shall now determine $\omega$ and $\alpha$ and
 therefrom, $y_1$, $y_2$ and $y_3$. 
 
 In order to do this, we have  to consider  two cases,  $x_1=o$, and
 $x_1 \neq o$. 
 
 Let, first, $x_1=o$. Then $y_1 +y_2 +y_3=o$. Let $\sigma$ be a
 generator  of the galois  group  of $K/L$ and let  notation be so
 chosen that  
 $$
 y^\sigma_1 =y_2, y^\sigma_2 =  y_3 , y^\sigma_3 = y_1.
 $$
 
 Since $S_{K/L} y_1= y_1+y_1^\sigma +y_1^{\sigma^2}=o$, by Hilbert's
 theorem, there is a $\lambda$ in $K$ such that  
 $$
 y_1 = \lambda^\sigma -\lambda.
 $$ 
  
 Also\pageoriginale
 $$
 x_2 = y_1 y_2+y_2y_3+y_3y_1 =\sum_{\sigma}(\lambda^\sigma-
 \lambda)(\lambda^{\sigma^2}- \lambda^\sigma) =-
 (\lambda+\lambda^\sigma + \lambda^{\sigma^{2}})^2. 
 $$
 
 Since $x_2 \neq o$ (otherwise, polynomials is not separable), we see
 that $x_2=- t^2$, for some $t \in L$. The  polynomial, therefore, has
 the form  
 $$
 x^3- t^2 x+x_3. 
 $$
 
 Put now $tx$ for $x$. Then  the polynomial $x^3-x+x_3/t^3$ has roots
 $\dfrac{y_1}{t}$, $\dfrac{y_2}{t}$, $\dfrac{y_3}{t}$. Let $\omega= x_3/
 t^3$. Then  
 $$
  y_1 = t \frac{\omega}{\mathscr{P}}, \quad y_2 = t
  \frac{\omega}{\mathscr{P}}+t, \quad  y_3 = t \frac{\omega}{\mathscr{P}}+ 2t 
  $$
 and thus the roots are all obtained.
 
 The indeterminacy in the sign  of $t$ does not cause  any
 difficulty; for, if we use $(-t)$ instead of $t$, then, observing that
 $- \dfrac{\omega}{\mathscr{P}}= \dfrac{-\omega}{\mathscr{P}}$, we see
 that  
 $$
 y_1 = - t \frac{-\omega}{\mathscr{P}}, y_3 = - t \frac{-
   \omega}{\mathscr{P}} -t, y_2 =-t \frac{- \omega}{\mathscr{P}}+t 
 $$
 so that $y_2$ and $y_3$ get  interchanged.
 
 We now consider the case $x_1 \neq o$.
 
 Put $x+a$ instead of $x$. Then  the new polynomial is 
 $$
 x^3 + x_1 x^2 +x(x_2 +2 x_1 a)+  x_3 +x_2 a +a^2 +a^3.
 $$
 
 Choose $\underbar{a}$ so that $x_2 +2x_1 a=o$. For this  value  of
 $\underbar{a}$, $\mu  = x_3 +x_2 a+ a^2+a^3 \neq o$; for, otherwise,
 the  polynomial will be reducible and the roots are $o$, $o$, $-x_1$. The
 roots  of this  new polynomial are $y_1- a$, $y_2 -a$, $y_3-a$. 
 
 Let now $\dfrac{1}{x}$ be written  for $x$, then the polynomial is
 reduced to $x^3 +\dfrac{x_1}{\mu} x+ \dfrac{1}{\mu}$. We are in the
 previous case. The roots, now, are $\dfrac{1}{y_1-a}$,
 $\dfrac{1}{y_2-a}$, $\dfrac{1}{y_3-a}$. If $-t^2= \dfrac{x_1}{\mu}, t
 \in  L$ and $\nu = \dfrac{1}{\mu t^3}$,\pageoriginale then
 $K=L(\dfrac{\nu}{\mathscr{P}})$ and the roots $y_1$, $y_2$, $y_3$ are
 given by  
 $$
 y_1=a +\frac{1}{t \frac{\nu}{\mathscr{P}}}, y_2 = a +  \frac{1}{t+t
   \frac{\nu}{\mathscr{P}}}, y_3 =a + \frac{1}{2t+t
   \frac{\nu}{\mathscr{P}}} 
 $$
 
 Since the characteristic is 3, the biquadratic polynomial can be
 taken in the form 
 $$
 x^4 +x_2 x^2+x_3 x+x_4.
 $$
 
 The proof is similar to the old one except that $K_b/ K_a$ is a
 cyclic extension of degree 3 and so $K_b= K_a
 \left(\dfrac{\omega}{\mathscr{P}} \right)$ for a suitable  $\omega$
 in $K_a$. To 
 find $\omega$, we use the foregoing method. One finds that $K$ is  a
 generalized radical extension. 
 
 
 We now consider the case where  $k$ \textit{has characteristic}
 2. In this case, the cubic  polynomial cab be taken in the form,
 $x^3+x_2 x+x_3$. Let $y_1$, $y_2$, $y_3$ be the roots. Then 
 $$
 y_1 +y_2+y_3=o
 $$ 
 and therefore $y_1^2+y_2^2 =y_3^2$ and so on. Put
 \begin{align*}
\omega &= \frac{y_1}{y_2}+ \frac{y_2}{y_3}+\frac{y_3}{y_1},\\
\omega' &= \frac{y_2}{y_1}+ \frac{y_3}{y_2} + \frac{y_1}{y_3}.
 \end{align*} 

 Then
 $$
 \omega + \omega'= \frac{y_1 (y^2_2+y^2_3)+y_2 (y_3^2+ y_1^2)+
   y_2(y^2_1+y^2_2)}{y_1 y_2 y_3}=1, 
 $$
 which shows that $\omega \notin k$, because ,then, it will be
 symmetric and equal to $\omega'$. Thus $k(\omega)$ is a quadratic
 subfield of $K$. A simple computation shows that $\omega$ and
 $\omega' = \omega +1$ are roots of 
 $$
 x^2- x + x^3_2.
 $$  
 
 $K/k(\omega)$\pageoriginale is cyclic of degree 3 and one uses the
 method of Lagrange to obtain a generalized radical extension.  
 
 Suppose now that $f(x)$ is a polynomial of degree 4. Let  $f(x)=
 x^4 +x_1 x^3  +x_2 x^2+x_3 x + x_4$. We have to consider two
 cases. Let, first, $x_1=o$. Then  the roots $y_1$, $y_2$, $y_3$, $y_4$
 satisfy 
 $$
 y_1+y_2+y_3+y_4=o. 
 $$
 
 As before, put
 $$
 \theta_1= (y_1+y_2)(y_3+y_4), \quad \theta_2= \ldots , \theta_3 = \cdots
 $$
 
 The reducing cubic is then 
 $$
 \varphi(x)= x^3 + x^2_2 x+ x_3. 
 $$ 
 
 Furthermore $\theta_1 +  \theta_2+  \theta_3=o$. Put now 
 $$
 \omega= \frac{\theta_1}{\theta_2} + 
 \frac{\theta_2}{\theta_3}+\frac{\theta_3}{\theta_1}.  
 $$
 
 Then $\omega$ is fixed under $A_4$ but not under $S_4$. Also
 $K_a=k(\omega)$ is a simple pseudo radical extension. 
 
 Now $K_b/K_a$ is a cyclic extension of degree 3 and has to be
 solved by Lagrange's methods. Further, $K_c/K_b$ is of degree 2. Put
 now  
 $$
 \omega_1 = \frac{y_3}{y_1y_2y_4},  \quad \omega'_1 = \frac{y_4}{y_1 
   y_2 y_3}. 
 $$
 
 Then
 $$
 \omega_1 +\omega'_1 = \frac{y^2_3 + y^2_4}{y_1y_2y_3y_4}=
 \frac{(y_1+y_2)(y_3+y_4)}{y_1 y_2y_3y_4} = \frac{\theta_1}{x_4}. 
 $$
 
 We assume $x_3 \neq o$. Then $\theta_1 \neq o$. If we put 
 $$ 
 \omega_2 = \frac{x_4 \omega_1}{\theta_1}, \quad \omega'_1= \frac{x_4
   \omega'_1}{\theta_1},  
 $$
 then\pageoriginale $\omega_2+  \omega'_2=1$ and $K_c =K_b(\omega_2)$.  
 
 In similar manner, $K=K_c(\omega_3)$ where 
 $$
 \omega_3 = \frac{x_4}{\theta_2}, \quad  \frac{y_1}{y_2y_3y_4}.
 $$
 
 Suppose, now, that  $x_1 \neq o$. Then, as before, we construct the
 field $K_b$. In order to exhibit $K_c$ as a pseudo radical extension
 of $K_b$, observe that $y_1 +y_2$ is fixed under $C_4$ but not under
 $B_4$. Also 
 $$
 (y_1 +y_2)^2= (y_1+y_2)(y_3+y_4+x_1) = \theta_1 +  x_1(y_1+y_2)
 $$ 
which shows that 
$$
K_c=K_b \left( \frac{y_1+y_2}{x_1} \right).
$$

Similarly, $K=K_c \left(\dfrac{y_1+y_3}{x_1} \right)$.

Our contentions are completely established.


