\chapter{Valuated fields}%Chap VIII

\section{Valuations}\pageoriginale %\section 1.

Let $k$ be a field. A \textit{valuation} on $ k $ is a function 
$| \, |$ on $k$ with values\break in the real number field satisfying   
\begin{enumerate}[(1)]
\item $ \mid 0\mid= 0 $

\item $ \mid a \mid > 0 $, if $ a \neq 0 $

\item $ \mid ab \mid =  \mid a \mid  \cdot \mid b \mid $

\item $ \mid a + b \mid  \leq  \mid a \mid   + \mid b \mid $
\end{enumerate}
where $ \underbar{a} $ and $ \underbar{b}$ are elements in $k$. It
follows that $ a \rightarrow \mid a \mid $  is a homomorphisms of $ k^*
$ into the multiplicative group of positive real numbers. If we denote
by $1$ the unit element of $ k $, then  
$$
\mid 1 \mid = \mid 1^2 \mid = \mid 1 \mid^2 = \mid 1 \mid \cdot  \mid 1
\mid  
$$
 so that $ \mid 1 \mid  = 1 $. If $ \zeta $ is a root of unity, say an
 $n$ say an $n$th root of unity, then  
 $$
 1 = \mid \rho^n \mid = \mid \rho \mid^n
 $$
 and so $ \mid \rho \mid =1 $. Thus means that $ \mid -1 \mid = 1 $
 and so, for $ a \in k $, 
 $$
 \mid -a \mid = \mid a \mid .
 $$

 Also, since $ a = a - b + b $, we get 
 $$
 \mid a \mid - \mid b \mid \leq a -b \mid .
 $$

 A valuation is said to be trivial if $ \mid a \mid = 1 $ for all $ a
 \neq 0 $. 
 
 Two valuations $ \mid \, \mid_1 $ and $ \mid \, \mid_2 $ are said to
 be  \textit{equivalent} if for every $ a \neq 0 $ in $ k $,   
 \begin{align*}
 \mid a \mid_1 < 1 &\Rightarrow  \mid a \mid_2 < 1 ,\\
\mid a \mid_1 = 1 &\Rightarrow  \mid a \mid_2 =  1.
\end{align*}\pageoriginale

It is obvious that the  above relation between valuations is  an
equivalence relation. All valuations equivalent to a given valuation
form an equivalence class of valuations. 

If $ \mid \, \mid $ is a valuation then, for $ 0 \leq c \leq 1$, $\mid 1
\mid^c $ is also a valuation. We shall now prove   

\setcounter{thm}{0}
\begin{thm}% Thm1
If $ \mid \mid_1 $  and $ \mid \mid_2 $ are equivalent
  valuations, there exists a real number $ c > 0 $ such that 
$$
\mid a \mid_1 =  \mid a \mid^c_2
$$
for all $ \underbar{a} \in k $.
\end{thm}

\begin{proof}
 Let us assume that $ \mid \mid_1 $ is non-trivial. Then $ \mid \mid_2
 $ is also non-trivial. Also, there exists a $b \in k $ such
 that $ \mid b \mid_1 > 1 $, $ \mid b \mid_2  > 1 $. Let $ a \in k $,
 $ a \neq 0 $. Then $ \mid a \mid_1 $ and $ \mid b \mid_1 $ being
 positive real numbers.  
 $$
 \mid a  \mid_1 = \mid b \mid^\lambda_1
 $$
 where $ \lambda = \dfrac{ \log \mid a \mid_1}{\log \mid b \mid_1} $. 
\end{proof}

We approximate to the real number $ \lambda $ from below and from
above by means of rational numbers. Let $ \dfrac{m}{n} < \lambda
$. Then  
$$
\mid a \mid_1 > \mid b \mid_1^{m/n}
$$
which means that  $ \mid  \dfrac{a^n}{b^m} \mid_1 > 1 $. Since $ \mid
\mid_1 $ and $ \mid \mid_2 $ are equivalent, this means that  
$$
\mid a \mid_2 > \mid b \mid^{m/n}_2 .
$$

In a similar manner, if $ p /q  > \lambda $, then
$$
\mid a \mid_2 < \mid b \mid^{p/q}_2 .
$$\pageoriginale

This means that if $ \dfrac{m}{n} \rightarrow \lambda $ and $
\dfrac{p}{q} \rightarrow \lambda $,then  
$$
\mid a \mid_2 = \mid b \mid^\lambda_2 .
$$

Therefore $ \lambda = \dfrac{\log \mid a \mid_2}{\log \mid b \mid_2}
$. This  shows that $ \dfrac{\log \mid a \mid_1}{\log \mid a \mid_2} =
\dfrac{\log \mid b \mid_2}{\log \mid b \mid_2} $. Putting $ c =
\dfrac{\log \mid b \mid_1}{\log \mid b \mid_2} $ and observing that $
c > 0 $, our theorem follows. 


\section{Classification of valuations}%section 2.

 A valuation is  said to be \textit{archimedian} if  for every $
 a \in k $, there exists an integer $ \underbar{n} = n (a) $
(that is $ne$, if $e$  is the unit  element of $k$) such that 
 $$
 \mid a \mid < \mid n \mid .
 $$

 (Compare this with archimedian axiom in ordered  fields).

 
 A valuation of $k$ which is not archimedian is said to be
\textit{non-archi\-median}. We shall deduce a few simple consequences of
these definitions. 
  \begin{enumerate}[1)]
\item $ \mid \mid $ \textit{is archimedian $ \Longleftrightarrow$
  there exists an integer $n$ in $k$ such that $\mid n \mid  > 1$.} 

 \begin{proof}
If $ \mid \mid $ is archimedian, there exists  $ \underbar{a} \in k $
with $ \mid a \mid > 1 $ and an integer $n$ with  $ \mid n \mid > \mid
a \mid $. This means that  
$$
 \mid n \mid > 1.
$$
 \end{proof} 
 
 Let $ \mid \mid $  be a valuation and $n$, a rational integer so
 that $ \mid n \mid > 1$. Let $\underbar{a}$ any element of $k$. If
 $\mid a \mid \leq 1$, then clearly $ \mid a \mid < |n \mid $. Let $
 \mid a \mid > 1 $. Since archimedian axiom holds in real number
 fields, we have an integer $m$ with 
 $$
 \mid a \mid < \mid n \mid \cdot m . 
 $$\pageoriginale
 
 If $ m = 1 $, there is nothing  to prove. So let $m > 1 $. Then $ m
 = \mid n \mid^\lambda$  $( \lambda = \dfrac{\log m }{\log \mid n
   \mid } > 0)$. Let $\mu$ be a positive integer greater than $\lambda$.  
 Then $ m < \mid n \mid^\mu = \mid n^\mu \mid $.  Therefore
 $$
 \mid a \mid < \mid n \mid \cdot  m < \mid n  \mid^{\mu +1}
 $$ 
 and our assertion is proved.
 
 We deduce 

 \item \textit{$\mid \mid$ non-archimedian $\Longleftrightarrow
   \mid n \mid \leq 1$, for every integer  $n$ in $k$.} 

This shows at once that 

\item \textit{All the valuations of a field of characteristic $p$ are
  non-archimedian.} 

We now prove the  important property

\item \textit{$\mid \mid$ is an non-archimedian valuation if and
  only if for every $a$, $b$ in $k$} 
$$
 \mid a + b \mid \leq Max ( \mid a \mid , \mid b \mid ) .
$$

\begin{proof}
If $ \mid \mid $ is  a non-archimedian valuation, then for every
integer $n,  \mid n \mid  \leq 1 $. Let $m$ be any positive
integer. Then  
$$
( a + b )^m = a^m + ( ^m_1 ) a^{m-1} b + \ldots + b^m
$$
so that
\begin{gather*}
 \mid a + b \mid^m \leq  \mid a \mid^m + \mid a \mid^{m-1} \mid b \mid
 + \ldots + \mid b \mid^m \\ 
 \leq ( m + 1 ) Max ( \mid a \mid^m,   \mid b \mid^m ) . 
\end{gather*}

Taking $m$th roots and making $ m \rightarrow \infty$ we get 
$$
\mid a + b \mid \leq Max ( \mid a \mid, \mid b \mid ) .
$$

The converse is trivial, since $ \mid n \mid = \mid 1 + \dots + 1 \mid
\leq 1 $. 
\end{proof}

We\pageoriginale deduce easily

\item \textit{If $ \mid \mid $ is non-archimedian and  $ \mid a \mid
  \neq  \mid b \mid $, then} 
 $$
 \mid a + b \mid = Max ( \mid a \mid, \mid b \mid ) .
 $$
 
 \begin{proof}
Let, for instance, $ \mid a \mid > \mid b \mid $. Then  
$$
\mid a  + b \mid  \leq Max ( \mid a \mid ,  \mid b \mid ) = \mid a
\mid 
$$

Also $ a = a + b -b $, so that 
$$
\mid a \mid \leq Max ( \mid a + b \mid , \mid b \mid ).
$$

But, since $ \mid a \mid > \mid b \mid ,  \mid a + b \mid \ge \mid b
\mid $. Thus $ \mid a \mid \leq \mid a + b \mid $ and our contention
is proved. 
 \end{proof}
  \end{enumerate}

 More generally, we have, if $ \mid a_1 \mid > \mid a_j \mid $, $ j
 \neq 1 $, then  
 $$
 \mid a_1 + a_2 + \ldots a_n  \mid = \mid a_1 \mid .
 $$

 
 In the case of non-archimedian valuations, many times, the so-called
 \textit{exponential valuation} is used. It is defined thus: If 
 $\mid \mid_0 $ is an non-archi\-median valuation, define the  function 
 $\mid \mid $ by  
 $$
 \mid a \mid  = - \log  \mid a \mid_0,   \quad a \neq 0 . 
 $$
 
 This has a meaning since $ \mid a \mid_0 > 0 $  for  $ a \neq 0 $. We
 introduce a quantity $ \infty $ which has the property  
  \begin{align*}
\infty + \infty &= \infty \\
\infty +  a &=  \infty \\
 \end{align*} 
  for any real number ${a}$ and 
 $$
 \frac{a}{\infty} = 0 
 $$
 for any real number $a$. Then $ \mid \mid $ satisfies 
  \begin{enumerate}[1')]
\item $ \mid 0 \mid = \infty $

\item $ \mid a \mid $  is a real number 

\item $ \mid ab \mid  = \mid a \mid + \mid b \mid $

\item $ \mid a + b \mid \ge  \,  \Min \, (\mid a \mid ,  \mid b \mid)$.
 \end{enumerate} 
 
  Then\pageoriginale $|1| = 0$, $|\rho| = 0$ for a root of unity
  $\rho$ and the  valuation is trivial if $|a| = 0$ for $a \ne 0$. For
  two valuations  $| |_1$ and $| |_2$ which are equivalent  
$$
|a|_1 = c |a|_2 ,
$$
$c \rangle 0$ being a real number.



\section{Examples}%Sec 3

First, let $\Gamma$ be the finite field of $q$ elements. Every element
of $\Gamma^*$ satisfies the polynomial $x^{q-1}-1$. Therefore, any
valuation $|\, |$ on $\Gamma$ is trivial. 

Let now $\Gamma$ be the field of rational numbers.

Let $|\, |$ be an archimedian valuation on $\Gamma$. It is enough to
determine its effect on the set of integers in $\Gamma$. There is an
integer $n$ such that $|n| > 1$. Let $m$ be any positive integer. 
Then
$$
n = a_0 + a_1 m + \cdots + a_t m^t ,
$$
where $t \le \left[ \dfrac {\log n} {\log m} \right]$, $0 \le a_i <
m$. Therefore $|a_i| \le a_i < m$ so that  
$$
|n| \le m (t + 1) \max (1, |m|^t).
$$

Replace $n$ by $n^r$, where $r$ is a positive integer. Then again, we
have 
$$
|n| \le m^{\dfrac{1}{r}} (r \frac{\log n}{\log m} + 1)^{\dfrac{1}{r}}
\max (1, |m| ^t). 
$$

Making $r \to \infty$, we get
$$
|n| \le \max (1, |m|^{\frac{\log n}{\log m}}). 
$$

This proves that, since $|n| > 1$,
$$
|n| \le |m|^{\dfrac{\log n}{\log m}}
$$
and\pageoriginale $|m| > 1$.

Now we can repeat the argument with $m$ and $n$ interchanged and thus
obtain 
$$
|m| \le |n|^{\dfrac{\log n}{\log m}}
$$
 
Combining the two inequalities we get
$$
\frac{\log |n|}{\log n} = \frac{\log |m|}{\log m}.
$$

Since $m$ is arbitrary, it follows that 
$$
|m| = m^C
$$
where $c > 0$ is a constant. Obviously, the valuation is determined by
its effect on positive integers. From the definition of equivalence,
$| |$ is equivalent to the ordinary absolute value induced by the
unique order in $\Gamma$. 

Let now $| |$ be a non-trivial non-archimedian valuation. It is enough
to determine its effect on $Z$, the ring of integers. 

Since $|\, |$ is non-trivial, consider the set $\mathscr{Y}$ of
$\underline{a} \in Z$ with $|a| < 1$. $\mathscr{Y}$ is an ideal. For, 
$$
|a| < 1, |b| < 1 \Rightarrow |a+b| \le \Max (|a|, |b|) < 1. 
$$

Also, if $|a| < 1$ and $b \in Z$, then
$$
|ab| = |a| \, |b| < 1 .
$$

Furthermore, if $ab \in \mathscr{Y}$, then $|ab| < 1$. But $|ab| = |a|
|b|$ and $|a| \le 1$, $|b| \le 1$, since valuation is
non-archimedian. Hence $|a| < 1$ which means that $\mathscr{Y}$ is a
prime ideal. Thus $g = (p)$ generated by a prime $p$. Since, by
definition of $\mathscr{Y}$, $M < 1 \Leftrightarrow p/n$, we have, if $n
= p^{\lambda} \cdot n_1$, $(n_1 , p) = 1$, 
$$
|n| = |p|^\lambda .
$$\pageoriginale

If we denote $|p|$ by $c$, $0 < c < 1$, then for any rational number
$\dfrac{a}{b}$, the value is  
$$
\left|\frac{a}{b} \right| = c^\lambda
$$
where $\dfrac{a}{b}= p^\lambda \dfrac{a'}{b'}$ where $(a', p) = 1 =
(b', p)$ and $\lambda$ a rational integer. This valuations is called
the \textit{$p$-adic valuation}. 

Thus with every non-archimedian valuation, there is associated a prime
number. Conversely, let $p$ be any prime number and let $n$ be any
integer, $n = p^\lambda \cdot  n_1$, $\lambda \ge 0$, where $(n_1, p)
= 1$.  
Put
$$
|n| = |p|^\lambda \qquad , 0 < |p| < 1 .
$$

Then $|\, |$ determines a non-archimedian valuation on
$\Gamma$. Further, if $p$ and $q$ are distinct primes, then the
associated valuations are inequivalent. For, if $|\, |_p$ and $| \, |_q$
are the valuations, then 
$$
|q|_p = 1 , \quad |q|_q < 1 .
$$

We shall denote the valuation associated with a prime $p$, by $|
|_p$. Then we have the  

\begin{thm}\label{chap8:thm2}%Thm 2
 The ordinary absolute valuation and the $p$-adic valuation by
  means of primes $p$ form a complete system of in-equivalent
  valuations of the rational number field. 
\end{thm}

We shall denote the ordinary absolute valuation by $| |_\infty$. 

Let us consider the case of a function field $K$ over a ground field
$k$ and let $L$ be the algebraic closure of $k$ in $K$. $L$ is called
the \textit{field of constants} of the function field $K$. The\pageoriginale
valuations on $K$ that we shall consider shall always be such that 
$$
|a| = o
$$
for $a \in k$. (We consider exponential valuation). Hence valuations
of function fields are always non-archimedian, since the prime field
is contained in $k$. 

Let, now, $\alpha$ be in $L$. Then $\alpha$ satisfies the equation 
$$
\alpha^n + a_1 \alpha^{n-1} + \cdots + a_n = o
$$
where $a_1 , \ldots , a_n \in k$. If $||$ is a valuation of $K$ then 
$$
|\alpha|^n \ge \min (|\alpha|^{n-1}, \ldots , |1|) .
$$

From this, it follows that $|\alpha| \ge 0$. Also, since $1/ \alpha$
is algebraic, $|\alpha| \le 0$. Hence for all $\alpha \in L$  
$$
|\alpha| = 0 .
$$

Thus the valuation is trivial on the field of constants. 

We shall consider the simple case where $K = k(x), x$ transcendental
over $k$. Here $L = k$. Let $| |$ be a valuation and let $|x| <
0$. Let $f(x) = a_0 x^n + \cdots + a_n$ be in $k[x]$. Then 
$$
f(x) \ge \min (|x|^n , |x|^{n-1}, \ldots , |1|).
$$

Therefore
$$
f(x) = n |x| .
$$

This means that, if $R(x) = \dfrac{g(x)} {h(x)}$ is an element of $K$,
then  
$$
|R(x)| = (\deg h(x) - \deg g(x)) (- |x|) .
$$

We denote this valuation by $| |_\infty$.

Suppose now that $|x| \ge 0$. As in the case of rational integers,
consider the subset $\mathscr{Y}$ of $k[x]$ consisting of polynomials
$f(x)$ with $|f(x)| > 0$. Then, since for every $\phi(x)$ in $k[x]$,
$|\varphi(x)| \ge 0$, it follows that $\mathscr{Y}$ is a maximal ideal
generated by an\pageoriginale irreducible polynomial $p(x)$. As in the
case of the rational number field  
$$
|R(x)| = \lambda |p(x)| 
$$
where $R(x) = \{ p(x)\}^\lambda \dfrac{A(x)} {B(x)}$ where $A(x)$ and
$B(x)$ are prime to $p(x)$ and $\lambda$ is a rational integer. If we
denote, by $| |_{p(x)}$, this valuation and put $\lambda = ord_{p(x)}
R(x)$, then 
$$
|R(x)| = c \ord_{p(x)} R(x)
$$
where $c = |p(x)|_{p(x)} > 0$. Every irreducible polynomial also gives
rise to a valuation of this type. Hence 

\begin{thm}\label{chap8:thm3}%thm 3
 A complete system of inequivalent valuations of $k(x)$ is given
  by the valuations induced by irreducible polynomials in $k[x]$ and
  the valuation given by the difference of degrees of numerator and
  denominator of $f(x)$ in $k(x)$. 
\end{thm}


\section{Complete fields}%Sec 4

Let $k$ be a field and $| |$ a valuation of it. The valuation function
defines on $k$ a metric and one can complete $k$ under this
metric. The method is the same as in the previous chapter and we give
here the results without proofs. 

A sequence $(a_1, a_2, \ldots)$ of elements of $k$ is said to be a
Cauchy sequence if for every $\varepsilon > 0$, there exists an
integer $n = n(\varepsilon)$ such that 
$$
|a_{n_1} - a_{n_2}| < \varepsilon \qquad , n_1 , n_2 > n . 
$$

It is a null sequence if for every $\varepsilon > 0$, there is an
integer $n$ such that
$$
|a_m| < \varepsilon , \qquad m > n .
$$\pageoriginale

The Cauchy sequences form a commutative ring $R$ and the null
sequences a maximal ideal $\mathscr{Y}$, therein. The quotient
$\bar{k} = R/\mathscr{Y}$ is called the completion of $k$ under $|
|$. The mapping $a \to (a, a, a, \ldots)$ is an isomorphism of $k$ in
$\bar{k}$ and we identify this isomorphic image with $k$ itself. 

We extend to $\bar{k}$ the valuation $| |$ in $k$, in the following
manner. The real number field $\bar{\Gamma}$ is complete under the
valuation induced by the unique order on it. So, if $a \in \bar{k}$,
then $a = (a_1, a_2, \ldots)$, a Cauchy sequence of elements of
$k$. Put  
$$
|a| = \lim_{n \to \infty} |a_n|
$$

That this is a valuation follows from the properties of limits in
$\bar{\Gamma}$. Also, the extend valuation is archimedian or
non-archimedian, according as the valuation in $k$ is archimedian or
non-archimedian.

For instance, if $| |$ on $k$ is non-archimedian and
\begin{align*}
a & = (a_1 , a_2 , \ldots)\\
b & = (b_1 , b_2 , \ldots)
\end{align*}
are two elements of $\bar{k}$, then
$$
a + b = (a_1 + b_1 , a_2 , + b_2 , \ldots)
$$
and
$$
|a + b| - \Max ( |a| , |b| ) = \lim_{n \to \infty} (|a_n + b_n| - \Max
(|a_n|, |b_n|)) 
$$
which is certainly $\le 0$,

$\bar{k}$ \textit{is thus a complete valuated field}. 

It\pageoriginale may also be seen that the elements of $k$ are dense
in $\bar{k}$ in the topology induced by the metric.  

Let $c_1 + c_2 + c_3 +  \cdots $ be a series in $\bar{k}$. We denote
by $S_n$ the partial sum 
$$
S_n = c_1 + c_2 + \cdots + c_n .
$$

We say that $c_1 + c_2 + \cdots + c_n + \cdots$ is \textit{convergent}
if and only if the sequence of partial sums $S_1 , S_2 , \ldots , 
S_n , \ldots$ converges. This means that, for every $\varepsilon > 0$,
there exists an integer $n = n(\varepsilon)$ such that 
$$
|S_{m'} - S_m| = |c_{m+1} + \cdots + c_{m'}| < \varepsilon , m, m' > n .
$$

Obviously $c_1 , c_2 , \ldots $ is a null sequence in $\bar{k}$.

In case the valuation is non-archimedian, we have the following
property:- 

1) \textit{The series $c_1 + c_2 + \cdots + c_n + \cdots$ is
  convergent if and only if $c_1 , c_2 , \ldots$ is a null
  sequence}. 

\begin{proof}%proof
We have only to prove the sufficiency of the condition. 
Suppose that $c_n \to 0$; then, for large $n$ and $m$,
$$
|S_n - S_m| = |c_{m+1} + c_{m+2} + \cdots + c_n|
$$
$$
\le \max (|c_{m+1}| , \ldots , |c_n|)
$$
and so tends to zero. This proves the contention.
\end{proof}

Note that this theorem is false, in case the valuation is archimedian.

Let $k$ be a field and $| |$ a valuation on it. $a \to |a|$ is a
homomorphism of $k^*$ into multiplicative group of positive real
numbers. Let $G(k)$ denote this homomorphic image. This is a
group\pageoriginale which we call the \textit{value group} of $k$ for
the valuation. If $\bar{k}$ is the completion of $k$ by the valuation
in $k$, then $G(\bar{k})$ is value group of $\bar{k}$.  

Suppose $| |$ is an archimedian valuation. Then $k$ has characteristic
zero and contains $\Gamma$, the rational number field as a
subfield. On $\Gamma, | |$ is the ordinary absolute value. Since
$\bar{k}$ contains $\bar{\Gamma}$, the field of real numbers, it
follows that
\begin{enumerate}[1)]
\item \textit{$G(\bar{k})$ is the multiplicative group of all positive
  real numbers.} 

Also, because of the definition of the extended valuation, it follows
that $G(k)$ is dense in the group $G(\bar{k})$. 

We shall now assume that $| |$ is a non-archimedian valuation. We
consider the exponential valuation. Then $G(k)$ is a subgroup of the
additive group of all real numbers. We shall now prove 

\item[2)] $G(\bar{k}) = G(k)$.

 \begin{proof}%proof
For, let $0 \neq a \in \bar{k}$. Then $a = (a_1 , \ldots)$ is a Cauchy
sequence, not a null sequence in $k$. By definition, 
$$
|a| = \lim_n |a_n|
$$

Now $a_n = a_n - a + a$ so that 
$$
|a_n| \ge \min (|a_n - a| , |a| ).
$$
But, for $n$ large, $|a_n - a| > |a|$ so that $|a_n| = |a|$ and our
contention is established. 
 \end{proof} 
 
 $G(k)$ being an additive subgroup of the real number field is either
 dense or discrete. The valuation is then called dense or discrete
 accordingly. In the second case, there exists $\pi$ in $k$ with\pageoriginale
 smallest positive value $|\pi| \cdot |\pi|$ is then the generator of the
 infinite cyclic group $G(k) \cdot \pi$ is called a \textit{uniformising
   parameter}. It is clear that $\pi$ is not unique. For, if $u \in k$
 with $|u| = 0$, then $|u \pi| = |\pi|$ and $u \pi$ is also a
 uniformising parameter. 
 
 Consider in $k$ the set $\mathscr{O}$ of elements $a$  with $|a| \ge
 o. \mathscr{O}$ is then an integrity domain. For, 
$$
|a| \ge 0, |b| \ge 0 \Rightarrow |a + b| \ge \Min (|a| , |b|) \ge 0. 
$$ 

Also $|ab| = |a| + |b| \ge 0$. We call $\mathscr{O}$ the \textit{ring
  of integers} of the valuation. Consider the set $\mathscr{Y}$ of
elements $a \in k$ with $|a| > 0$. Then $\mathscr{Y}$ is a subset of
$\mathscr{O}$ and is a maximal ideal in $\mathscr{O}$. For, if $a \in
r$ and $b \in \mathscr{Y}$, then $|ab| = |a| + |b| > 0$. Also, if $a
\in r$ but not in $\mathscr{Y}$, then $|a| = 0$ and $|a^{-1}| = 0$ so
that if $\mathscr{U}$ is an ideal in $\mathscr{O}$ containing
$\mathscr{Y}$, then $\mathscr{U} = \mathscr{Y}$ or $\mathscr{U} =
r$. We call $\mathscr{Y}$, the \textit{prime divisor of the
  valuation}. Since $\mathscr{Y}$ is maximal, $r/\mathscr{Y}$ is a
field. We call $\pi/ \mathscr{Y}$  \textit{the residue class field}. 
 
 Exactly the same notions can be defined for $\bar{k}$. We denote by
 $\bar{\mathscr{O}}$ the ring of integers of the valuation so that
 $\bar{\mathscr{O}}$  is the set of $\underline{a} \in \bar{k}$ with
 $|a| \ge 0. \bar{\mathscr{Y}}$ is the maximal ideal in
 $\bar{\mathscr{O}}$, hence the set of  $\underline{a} \in \bar{k}$
 with $|a| > 0$. Also $\bar{\mathscr{O}}/ \bar{\mathscr{Y}}$ is the
 residue class field. Clearly 
 $$
 \mathscr{Y} = \bar{\mathscr{Y}} \cap r
 $$
 
 We now have

 \item Every $\underline{a} \in k (\bar{k})$ has the property; $a
   \in r (\bar{r})$ or $a^{-1} \in \mathscr{Y} (\bar{\mathscr{Y}})$. 
 
 This\pageoriginale is evident since if $a \notin \mathscr{O}
 (\bar{\mathscr{O}})$,  $|a| < 0$ so that $|a^{-1}| > 0$ and hence
 $a^{-1} \in \mathscr{Y}  (\bar{\mathscr{Y}})$.  

 \item $\mathscr{O} (\bar{\mathscr{O}})$ \textit{is integrally closed
   in $k (\bar{k})$}. 

\begin{proof}%proof
We should prove that every $\alpha$ in $k (\bar{k})$ which is a root
of a polynomial of the type $x^n + a_1 x^{n-1} + \cdots + a_n , a_1 ,
\ldots, a_n \in \mathscr{O} (\bar{\mathscr{O}})$, is already in
$\mathscr{O} (\bar{\mathscr{O}})$. For, suppose 
$$
\alpha^n + a_1 \alpha^{n-1} + \cdots + a_n = o
$$
and $\alpha \notin \mathscr{O} (\bar{\mathscr{O}})$. Then $\alpha^{-1}
\in \mathscr{Y} (\bar{\mathscr{Y}})$. Therefore 
$$
1 = -(a_1 \alpha^{-1} + a_2 \alpha^{-2} + \cdots + a_n \alpha^{-n}) 
$$
and so 
$$
o = |1| \ge \min (|a_1 \alpha^{-1}|, \ldots | a_n \alpha^{-n}) > o 
$$
which is absurd
\end{proof}

\item \textit{Every element in $\bar{\mathscr{O}}$ is the limit of a
  sequence of elements in $\mathscr{O}$ and conversely.} 

\begin{proof}%proof
Let $a = (a_1, \ldots , a_n , \ldots)$ be in $\bar{\mathscr{O}}, a_1,
\ldots , a_n , \ldots$ in $k$. Then, as we saw earlier, for
sufficiently large $n$ 
$$
|a_n| = |a|.
$$
But $|a| \ge 0$ so that $|a_n| \ge 0$. Thus, for sufficiently large
$n$, all $a_n 's$ are in $\mathscr{O}$. Converse is trivial. 

We now prove
\end{proof}

\item \textit{There is a natural isomorphism of
  $\mathscr{O}/\mathscr{Y}$ on $\bar{\mathscr{O}}/
  \bar{\mathscr{Y}}$.} 

\begin{proof}%proof
The elements of $\mathscr{O}/\mathscr{Y}$ are residue classes $a +
\mathscr{Y}$, $a \in r$. We now make correspond to $a + \mathscr{Y}$ the
residue class $a + \bar{\mathscr{Y}}$. Then $a + \bar{\mathscr{Y}}$ is
an element of $\bar{\mathscr{O}}/ \bar{\mathscr{Y}}$. The mapping $a +
\mathscr{Y} \to a + \bar{\mathscr{Y}}$\pageoriginale is a homomorphism
(non-trivial) of $\mathscr{O}/\mathscr{Y}$ into $\bar{\mho}/
\bar{\mathscr{Y}}$. Since both are fields, this is an isomorphism
into. In order to see that it is an isomorphism of $\mathscr{O}/
\mathscr{Y}$ onto $\bar{\mathscr{O}}/ \bar{\mathscr{Y}}$, let $\bar{a}
+ \bar{\mathscr{Y}}$ be any residue class in $\bar{\mathscr{O}}/
\bar{\mathscr{Y}}$. By (4) therefore, given any $N > 0$, there
exists $a \in \mho$ with 
$$
|\bar{a} - a| > N > 0
$$
or $\bar{a} - a \in \bar{\mathscr{Y}}$. If we then take $a +
\mathscr{Y}$, then $\bar{a} + \bar{\mathscr{Y}}$ is the image of $a +
\mathscr{Y}$. This proves that the mapping is an isomorphism onto. 
\end{proof}

(5) enables us to choose, in $k$ itself, a set of representatives of
 $\bar{\mho} /\bar{\mathscr{Y}}$, the residue class field. We
 shall denote this set by $\mathscr{R}$ and assume that it contains
 the zero element of $\mathscr{O}$. Also it has to be observed that,
 in general, $\mathscr{R}$ is \textit{not a field}. It is not even an
 additive group. 


We now assume that $| \, |$ is a \textit{discrete valuation}. Let $\pi$
in $\mathscr{O}$ be a uniformising parameter. Then clearly 
$$
\bar{\mathscr{Y}} = (\pi) 
$$
is a principal ideal. Let $a \in \bar{k}$. Then $|a|/ |\pi|$ is a
rational integer, since $G(k)$ is an infinite cyclic group. Put $|a|/
|\pi| = t$. Then $|a \pi^{-t}| = 0$. If we call elements $u$ in
$\bar{k}$ with $|u| = 0$, units, then 
$$
a = \pi^t u
$$
where $u$ is a unit.

Let $\mathscr{R}$ be the set defined above and $a \in
\bar{\mho}$. Then 
$$
a \equiv a_0 (\mod \bar{\mathscr{Y}})
$$
with\pageoriginale $a_0 \in \mathscr{R}$. This means that $(a-a_0)
\pi^{-1}$ is an integer (in $\bar{\Omega}$).  
$$
(a-a_0) \pi^{-1} \equiv a_1 ({\rm mod} \bar{\mathscr{Y}}),
$$
where $a_1 \in \mathscr{R}$. Then
$$
a \equiv a_0 + a_1 \pi ({\rm mod} \bar{\mathscr{Y}}^2)
$$

In this way, one proves by induction that
$$
a \equiv a_0 + a_1 \pi + \cdots + a_m \pi^m ({\rm mod})
\bar{\mathscr{Y}}^{m+1}) 
$$
where $a_0 , a_1 , \ldots , a_m \in \mathscr{R} $. Put $b_m = a_0 +
a_1 \pi + \cdots + a_m \pi^m$.  

Then $a \equiv b_m ({\rm mod} \bar{\mathscr{Y}}^{m+1})$ which means that 
$$
|a - b_m| \ge m+ 1
$$
 
Consider the series
$$
b_0 +(b_1 - b_0) + (b_2 - b_1) + \cdots
$$

Then, since $b_{m+1} - b_m = a_{m+1} \pi^{m+1}$, we see that $|b_{m+1}
- b_m|$ increases indefinitely. Hence the above series
converges. Also, since its elements are integers, 
$$
b = b_0 + (b_1 - b_0) + (b_2 - b_1) + \cdots
$$
is an element of $\bar{\mathscr{O}}$.

Since $ b_0 + (b_1 - b_0) + \cdots + (b_m - b_{m-1})= b_m$, it follows
that  
$$
b = \lim_m b_m .
$$

Thus we have
$$
a = \lim_{m \to \infty} b_m = a_0 + a_1 \pi + a_2 \pi^2 + \cdots 
$$

By the very method of construction this expression for $\underbar{a}$
is unique, once we have chosen $\mathscr{R}$ and $\pi$. 

If\pageoriginale $ a \in \bar{k}$, $a \pi^t \in \bar{\mathscr{O}}$ for
some rational integer $t$. Hence we have   

\item \textit{Every element $\underbar{a}$ in $\bar{k}$ has the
  unique expression} 
$$
a = \sum_{n =- t}^{\infty} a_n \pi^n
$$
\textit{$a_n \in \mathscr{R}$ and $\pi$ in $\bar{\mathscr{O}}$ is a
  generator of $\mathscr{Y}$. t is a rational integer.} 

If $t > 0$, we shall call
$$
h(a) = \sum_{n = -t}^{ -1} a_n \pi^n
$$
the \textit{principal part} of $a$. If $t \le 0$ we put $h(a) =
0$. Clearly, $h(a)$ is an element in $k$. Also $a - h(a) \in
\bar{\mathscr{O}}$. 

We now study the two important examples of the rational number field
and the rational function field of one variable. 

Let $\Gamma$ be the field of rational numbers. We shall denote by $|
|_\infty$ the ordinary absolute value and by $| |_p$ the $p$-adic
value for $p$, $a$ prime. If $\underbar{a}$ is an integer, $a =
p^\lambda n_1$, $(p, n_1) = 1$. 

We put
$$
|a|_p = \lambda \log p
$$
and
$$
|a|_\infty = \text{ absolute value of a.}
$$

It is then clear that each of the non-archimedian valuations of
$\Gamma$ is discrete. Let us denote by $\Gamma_\infty$ the completion
of $\Gamma$ by the archimedian valuation and by $\Gamma_p$ the
completion of $\Gamma$ by the $p$-adic valuation. $\Gamma_\infty$ is
clearly the real number field. 

If $\mathscr{O}_p$ denotes the set of integers of $\Gamma_p$ and
$\mathscr{Y}$ the prime ideal of the valuation then  
$$
\mathscr{Y} = (p)
$$

A set\pageoriginale of representatives of $\mathscr{O}_p \mod
\mathscr{Y}$ is given by the integers $0, 1, 2,\break \ldots , p-1$ as can
be easily seen. Hence, by (6),  

\item \textit{Every $a \in \Gamma_p$ is expressed uniquely in the form}
$$
a = \sum_{n = -t}^{\infty} a_n p^n
$$
where $a_i = 0, 1, 2, \ldots , p-1$.

The elements of $\Gamma_p$ are called the \textit{$p$-adic numbers of
  Hensel}. 

As before, we denote, by $h_p (a)$, the principal part of
$\underbar{a}$ at $p$. Clearly $a-h _p (a)$ is a $p$-adic integer. 

Let $\underbar{a}$ be a rational number, $a \in \mho_p \Leftrightarrow
|a|_p \ge 0$, that is $a = \dfrac{b}{c}$, $(b, c) = 1$ and $c$ is prime
to $p$. Since only finitely many primes divide $b$ and $c$, it follows
that $a \in \mho_p$ for almost all $p$, that is except for a
finite number of $p$. Hence $h_p(a) = 0$ for all except a finite
number of primes. Hence for any $\underbar{a}$ 
$$
\sum_p h_p (a)
$$ 
has a meaning. Also, $a - h_p (a)$ is a rational number whose
denominator is prime to $p$. Hence $a - \sum \limits_p h_p(a)$ is a
rational number whose denominator is prime to every rational
integer. Hence 

\item \textit{For every rational number $ \underbar{a}$}
$$
a - \sum_p h_p (a) \equiv 0 ({\rm mod} 1).
$$

This is the so-called \textit{partial fraction decomposition} of a
rational number. 

Let now $k$ be an  algebraically closed field and $K = k(x)$ the field
of rational functions of one variable $x$. All the valuations
are\pageoriginale non-archimedian. Every irreducible polynomial of $k
[x]$ is linear and of the form $x - a$. With every $a \in k$ there is
the valuation $| |_a$ associated, which is defined by   
$$
|f(x)|_a = \lambda , 
$$
where $f(x) \in k [x]$, $f(x) = (x - a)^\lambda \varphi(x)$, $\varphi (a)
\neq 0$. If we denote by $| |_\infty$ the valuation by degree of
$f(x)$, then, for $f(x) \in k[x]$, 
$$
|f(x)|_\infty = - \deg f(x)
$$

Let $K_a$ and $K_\infty$ denote the completions, respectively, of $K$
at $| |_a$ and $| |_\infty$. If $\mathscr{O}_a$ and
$\mathscr{O}_\infty$ are the set of integers of $K_a$ and $K_\infty$,
$\mathscr{Y}_a$ and $\mathscr{Y}_\infty$ the respective prime divisors,
then 
$$
\mathscr{Y}_a = \{ x - a\} , \mathscr{Y}_\infty = \{ \frac{1}{x}\} .
$$

It is clear, then, that $\mho_a / \mathscr{Y}_a$ and
$\mho_\infty / \mathscr{Y}_\infty$ are both isomorphic to $k$
and since $K$ contains $k$, we may take $k$ itself as a set of
representatives of the residue class field. Any element $f$ in $K_a$
is uniquely of the form 
$$
f = \sum_{n = -t}^{\infty} a_n (x - a)^n ,
$$
$a_n \in k$. Similarly, if $\varphi \in K_\infty$,
$$
\varphi = \sum_{n =- t}^{\infty} b_n x^{-n}.
$$

As before, if we denote by $h_a (f)$ and $h_\infty(f)$ the principal
parts of $f \in K$ for the two valuations, then 
$$
\sum_a h_a(f) + h_\infty(f)
$$
has a meaning since $h_a(f) = 0$ for all but a finite number of
$\underbar{a}$ 

If\pageoriginale we define $\varphi \in K$ to be regular ar
$\underline{a} (\infty)$ if $\varphi \in \mho_a(\mho_{\infty})$, then
for $f \in K$,   
$$
f- \sum_a h_a (f)
$$
where $\underline{a}$ may be infinity also, is regular at all, $a \in
k$ and also for the valuation $||_\infty$. Such an element, clearly,
is a constant. Hence 

\item \textit{If $ f \in K $ then}
$$
f - \sum_a h _a (f) = \text{ constant }
$$

Conversely, it is easy to see that there exists, up to an additive
constant, only one $f \in K$ which is regular for all $a \in k$ except
$a_1, \ldots, a_n$ (one of which may be $\infty$  also) and with
prescribed principal parts at these $a_i$. 9) gives the partial
fraction decomposition of the rational function $f$.  
\end{enumerate}

\section[Extension of the valuation of a complete\ldots]{Extension of the valuation of a complete\hfill\break non-archimedian 
  valuated field}%Sec 5 

We shall study the following problem. Suppose $k$ is compute under a
valuation $||$. Can this valuation be extended, and if so in how many
ways, to a finite algebraic extension $K$ of $k$ ?   

We prove, first 

\setcounter{lem}{0}
\begin{lem} %lemma 1
Let $k$ be a field complete under a valuation $||$, and $K$ a
  finite algebraic extension of $k$. Let $\omega _1 , \ldots ,
  \omega_n $ be a basis of $K/k$. Let $||$ have an extension to $K$
  and  
$$
\alpha _\nu = \sum_ { i = 1}^ n  a_{ i v} \omega_i, a_{iv}  \in k, 
$$

$v = 1 ,2, 3, \ldots$\pageoriginale  be a Cauchy sequence in $K$. Then 
  $a_{iv}$ $ i = 1, 2, \ldots , n $ are Cauchy sequence in $K$. 
\end{lem}

\begin{proof}
We consider the Cauchy sequence $\{ \alpha _v \}$, 
$$
\alpha _v = \sum_ { i = 1}^m a_{iv} \omega_i , a_{iv} \in , k 
$$
$1 \leq m \leq n$ and we shall prove that the $\{ a_{iv}\}$ are Cauchy
sequences in $k$. We use induction on $m$. Clearly, if $m = 1$,  
$$
\alpha _v = a_{ i v} \omega_1 
$$
and $\{ \alpha_v\}$ is a Cauchy sequence in $K$ if and only if $\{
a_{iv}\}$ is a sequence in $k$. 
\end{proof}

Suppose we have proved our statement for $m - 1 \geq 1$, instead of
$m$. Write  
$$
\alpha _v = \sum_{i =1}^{m -1} a_{iv}\omega_i + a_{mv} \omega_m . 
$$

If $\{a_{mv}\}$ is a Cauchy in $k$, then $\{ \alpha_v - a_{mv}
\omega_m \}$ is a Cauchy sequences in $K$ and induction hypothesis
works. Let us assume that $a_{mv}$ is not a Cauchy sequence in
$k$. This means that there exists a $\lambda >0$ and for every $v$, an
integer $\mu_v$ such that  
$$
\mu_\nu > v 
$$
and 
$$
| a_m \mu _v - a_{mv} | > \lambda . 
$$

Consider now the sequence $\{ \beta _v\}$ in $K$ with 
$$
\beta_v = \frac{\alpha_{\mu_v}- \alpha_v }{ a_{m \mu_v} - a_{m v}}.
$$

Because of the above property, we see that $\{\beta _v\}$ is a null
sequence in $K$.  

Now 
$$
\beta_v - \omega_m = \sum^{m-1}_{ i = 1} \left(\frac{a_i \mu _v -
  a_{iv}}{a_m \mu _v - a_{mv}}\right )\omega_i 
$$\pageoriginale
and $\{\beta _v - \omega_m \}$  is a Cauchy sequence in
$K$. Induction hypothesis now works and so, if  
$$
\lim_{v \to \infty} \left(\frac{a_{i \mu_v} - a_{iv}}{a_{m\mu_v} -
  a_{mv}} \right)=b_i, 
$$
then
$$
- \omega_m = \sum_{i = 1} b_i \omega_i,\qquad b_i \in k.
$$

This is impossible because $\omega_1, \ldots , \omega_m$ are linearly
independent over $k$. Therefore $\{ a_{ m v}\}$ is a Cauchy sequence
and our Lemma is thereby proved.  

We\pageoriginale shall now prove the following theorem concerning
extension of valuation. 
 
\begin{thm}\label{chap8:thm4} %Theorem 4
 If the valuation $||$ of a complete field $k$ can be extended to
  a finite extension $K$, then this extension is unique and $K$ is
 complete under the extended valuation. 
\end{thm}


\begin{proof}
That $K$ is complete under the extended valuation is easy to see. For,
if $\{ \alpha _v\}$ is a Cauchy sequence in $K$ and  
$$
\alpha _ v = \sum_ i a_{iv} \omega _i ,\quad a_{iv} \in k .,
$$
by the lemma, the $\{a_{iv} \}$'s are Cauchy sequences. So, if 
$$
\lim_{ v \to \infty} a_{iv} = b_i \in k , 
$$
then 
$$
\lim_{ v\to \infty} \alpha_{\gamma} = \sum_i \omega_1 \lim_{ v \to \infty}
a_{iv} = \sum_i b_i \omega_i  
$$
which is again in $K$. 
\end{proof}

We shall now prove that the extended valuation is unique.

From the lemma, it follows that if $\{ \alpha _v\}$ is a  null
sequence in $K$ and $\alpha _v = \sum\limits_i a_{iv} \omega_i$, $a_{iv}
\in k $, then the $ a_{iv}$'s are null sequences in $k$.  

In particular, if $\alpha \in K$ and $|\alpha| < 1$ in some extension
the valuation $||$ in $k$, then $\alpha, \alpha^2, \alpha^3 , \ldots$
is a null sequence in $K$. If $\alpha^m=\sum\limits_1
a_1^{(m)}\omega_i$, $a_i^{(m)}\in k$, then $a_i^{(m)}, i=1,\ldots, n$
are null sequence in $k$. 


If $\alpha = \sum x_i \omega_i $ is a general element of $ x_1, \ldots
, x_n $. Put  
$$
\alpha ^t = \sum_{ i = 1} x_i ^{(t)} \omega_i ;  
$$
then $N_{K/k} \alpha^t$ is the same polynomial in $x_t ^{(t)}$, 
$i = 1 , \ldots , n $ as $N_{K/k} \alpha^t $ is in $x_1, \ldots ,
x_n$. If now $|\alpha | < 1$, is an extended valuation, then the $\{
x_i^(t)\}$ are null sequences. Hence $N\alpha$, $N\alpha^2 $ is a null
sequences in $k$, But $N \alpha ^t  = (N \alpha )^t $ so that $(N
\alpha ), (N \alpha )^2 , \ldots  $,  is s a null sequence in $k$.
   This means that $|N \alpha | < 1$. We have, thus, proved that if
   $\alpha$ in $K$ is such that $|\alpha|<1$ is an extended valuation,
   then $|N \alpha| < 1$ in $k$.  

In a similar manner, if $  |\alpha | > 1$, then $|N \alpha | >
1$. Thus we get $|N\alpha| = 1 \Rightarrow |\alpha | = 1$. 

Let now $\beta$ be in $K$ and write $\beta = \dfrac{\alpha ^n}{N
  \alpha}$ where $ n = (K : k )$  

Then $N \beta = \dfrac{(N \alpha )^n}{(N \alpha )^n}$. Thus 
$$
|N \beta | = 1 . 
$$

By the above, it means that $|\beta| = 1 $ in the valuation. Hence  
$$
|\alpha| = \sqrt[n]{|N \alpha |} 
$$
showing\pageoriginale that the value of $\alpha$ in the extended
valuation is unique fixed.   

Our theorem is thus completely proved. 


In order to prove that an extension of the valuation is possible, we
shall consider the case where $k$ is complete under a \textit{discrete}\break
non-archimedian valuation. Let $\mathscr{O}$ be the  ring of 
integers $\mathscr{Y}$, the prime divisor of the valuation and
$\mathscr{O}/\mathscr{Y}$, the residue class field. We shall now
prove the celebrated lemma due \textit{Hensel}  

\begin{lem}\label{chap8:lem2} % Lemma 2
Let $f(x)$ be a polynomial of degree $m$ in $\mathscr{O}[x]$, $g_0
(x)$,  a monic polynomial of degree $r \geq 1$ and $h_0 (x)$, a
polynomial of degree $\leq m - r$ both with coefficients in
$\mathscr{O}$ such that  
\begin{enumerate} [ 1)]
\item $f(x) \equiv g_o (x) h_o (x) \, ({\rm mod} \mathscr{Y})$

\item $g_o (x)$ and  $h_o (x) $ are coprime $\mod \mathscr{Y}$
\end{enumerate}

Then there exists polynomials $g(x)$ and $h(x)$ in
  $\mathscr{O} [x]$ such that
\begin{equation*}
\left.
\begin{aligned}
g(x)  & \equiv g_o (x)  \\
h(x)  & \equiv h_o (x) 
\end{aligned}
\right \} \pmod{\mathscr{Y}},
\end{equation*}
$g(x)$ has the same degree as $g_o (x)$ and $f(x) =
g(x) \cdot  h(x)$. 
\end{lem}

\begin{proof}
We shall now construct two sequences of polynomials \break $g_o (x), g_1,
(x), \ldots $ and $h_o (x), h_1, (x), \ldots $ satisfying  
\begin{align*}
g_n (x) & \equiv g_{n - 1}(x) (\mod \mathscr{Y}^n )\\
h_n (x) & \equiv h_{n - 1}(x) (\mod \mathscr{Y}^n )\\
f_n (x) & \equiv g_{n}(x) h_{n }(x) (\mod \mathscr{Y}^{n + 1} )
\end{align*}
$g_n (x)$ is a monic polynomial of degree $r$ and $h_n (x)$ of degree
$ \leq m - r$. All the polynomials have coefficients in $\mathscr{O}$.  
\end{proof}

The\pageoriginale polynomials are constructed inductively. For $ n =
0$, $ g_o (x)$ and $h_o (x)$ are already given and satisfy the
conditions. Assume now that $g_ o (x) , \ldots , g_{n - 1} (x)$ and
$h_ o (x) , \ldots , h_{n - 1} (x)$ have been constructed so as
satisfy the requisite conditions.  

Since  $g_{ n - 1}(x) \equiv g_o (x)({\rm mod} \, \mathscr{Y})$ and $h_{ n -
  1}(x) \equiv h_o (x)({\rm mod} \, \mathscr{Y})$ and $g_0 (x)$ and
$h_o (x)$ are 
coprime$\mod \mathscr{Y}$, there exists, for any polynomial $f_n
(x)$ in $\mathscr{O} [x]$, two polynomials $L(x)$ and $M(x)$ with  
$$
f_n (x) \equiv L(x) g_{n-1} (x) + M(x) h_{ n -1} (x) \, ({\rm mod } \, 
\mathscr{Y}).  
$$
$L(x) $ and $M(x)$ are clearly not uniquely determined. We can replace
$L(x)$ by $L(x)+ \lambda (x) h_{ n- 1} (x)$ and $M(x)$  by $M(x) +
\lambda (x) g_{n- 1} (x)$.  

Let $\pi$ be a generator of the principal ideal $\mathscr{Y}$. By
induction hypothesis,  
$$
f_n (x) = \pi^{ -n }(f(x) - g_{ n-1} (x) h_{n-1}(x))
$$
is an integral polynomial, so in $\mathscr{O} [x]$. Since $g_{n -1}
(x)$ has degree $r$ and is monic and $h_{ n-1}(x)$ degree $\leq m - r
$, it is possible to choose $M(x)$ and $L(x)$ so that $M(x)$ has
degree $< r$ and $L(x)$ degree $\leq m - r$. Put now  
\begin{align*}
g_n (x) & = g_{n - 1} (x) + \pi ^n M(x), \\
h_n (x) & = h_{n - 1} (x) + \pi ^n L(x).
\end{align*}

Then $g_n (x)$ is monic and of degree $r$, since $M(x)$ has degree $ <
r. h_n (x)$ has degree $\leq m - r$. Now $f(x) - g_n(x) h_n (x) = f(x)
- g_{ n-1}(x) h_{ n-1}(x) - \pi^n (g_{ n-1}(x) L(x) + h_{ n-1}(x) M(x) )
(\mod \mathscr{Y}^{ n+1})$ 
By choice of $L(x)$ and\break $M(x)$, it follows that 
$$
f (x) \equiv g_n (x) h_n (x) ({\rm mod}\, \mathscr{Y}^{ n+1}). 
$$

We\pageoriginale have thus constructed the two sequences of
functions. Put now  
\begin{align*}
g(x) = g_o (x) + (g_1 (x) - g_o (x) )+ \cdots + (g_n (x) - g_{
  n-1}(x)) + \cdots \\ 
h(x) = h_o (x) + (h_1 (x) - h_o (x) )+ \cdots + (h_n (x) - h_{
  n-1}(x)) + \cdots 
\end{align*}

 Since $g_n (x) - g_{ n-1}(x) = 0 ({\rm mod} \pi ^n)$, it follows that the
 corresponding coefficients of the sequence of polynomials $g_0 (x),
 g_1 (x) , \ldots $ form Cauchy sequences in $\mathscr{O}$. Since $k$
 is complete and  
$$
g(x) = \lim_{ n \to \infty } g_n (x), 
$$
 it follows that $g(x) \in \mho [x]$. It is monic and is of
 degree $r$. In a similar way, $b(x) \in \mho [x]$ and has
 degree $\leq m -r $.  


Also since $f(x) - g_n (x) h_n (x) \equiv 0 ({\rm mod} y^{n +1})$, it
follows that the coefficients of $(f(x) - g_ n (x) h_n (x))$ form
null sequences. Hence 
$$
f(x) = \lim _ n g_n (x) h_n (x) = h_n (x) = g(x) h(x)
$$
and our lemma is proved. 

We now deduce the following important . 

\begin{lem}\label{chap8:lem3} % lemma 3
 Let $f(x) = x^n + a_1 x^{ n -1} + \cdots + a_n$ be an
  irreducible polynomial $k[x]$ , $k$ satisfying hypothesis of lemma
  \ref{chap8:lem2}. Then $f(x) \in \mathscr{O} [x] $ if and only if $ a_n \in
  \mathscr{O} $.  
\end{lem}

\begin{proof}
It is clearly enough to prove the sufficiency of the condition. Let
$a_n \in \mho $, and if $a_1, \ldots, a_{n-1}$ (some or all of them)
are not in $\mho$, then there is a smallest power $\pi^a$, $a >
0$, of $\pi$ such that   
$$
\pi^a f(x) = b_o x^n + b_1 x^{ n-1} + \cdots + b_n  
$$
is a\pageoriginale primitive polynomial in $ \mathscr{O}
[x]$. Also, now $b_n \equiv 0 (\mod \mathscr{Y})$. and at least one of
$b_0,\ldots,b_{n-1}$ is not divisible by $\mathscr{Y}$. Let $b_r$ be
the first coefficient from the right not divisible by $\mathscr{Y}$. Then  
$$
\pi^a \equiv (b_ 0 x^r + \cdots b_r) x^{ n - r} ({\rm mod} \, \mathscr{Y})
$$

Since $b_r \not \equiv 0 (\mod \mathscr{Y})$, Hensel's lemma can be
applied and we see that $\pi ^a f(x)$ is reducible in $\mathscr{O}[x]
$. Thus $f(x)$ is reducible in $k [x]$ which contradicts the 
hypothesis. The lemma is therefore established.  
\end{proof}

We are now ready to prove the important theorem concerning extension of
discrete non-archimedian valuations, namely  

\begin{thm} %theorem 5
 Let $k$ be complete under a discrete non-archimedian valuation
  $||$ and $K$ a finite algebraic extension over $k$. Then $||$ can be
  extended uniquely to $K$ and then for any $\alpha$  in $K$,  
$$
|\alpha | = \frac{1}{(K : k)} |N_{ K/k}\alpha|.
$$ 
\end{thm}

\begin{proof}
Because of Theorem \ref{chap8:thm4}, it is enough to prove that the
function defined on $K$ by   
$$
|\alpha | = \frac{1}{(K : k)}|N_{K/k}\alpha | 
$$
is a valuation function. Clearly , $|0| = \infty$; $ |\alpha| $ is a real
number for $\alpha \neq 0$. Also,  
$$
|\alpha \beta | = |\alpha| + |\beta |. 
$$

We shall now prove that 
$$
|\alpha + \beta | \geq \min (|\alpha| , |\beta|). 
$$

If $\alpha$ or $\beta $ is zero, then the above is trivial. So let
$\alpha \neq 0$, $\beta \neq 0$.\pageoriginale  Since $
\left|\dfrac{\alpha}{\beta} \right|$ or $\left|\dfrac{\beta}{\alpha}
\right|$ is $\geq 0$, it is enough to prove that if $|\lambda| \geq
0$, $| 1 +\lambda| \geq 0 $.   
\end{proof}

Let $f(x) =x^m + a_1 x^{ m - 1} + \cdots +a_m $ be the minimum
polynomial of $\lambda$ in $K$ over $k$, Then  
$$
N \lambda = ( (-1)^m a_m)^{(K : k (\lambda))}.
$$

Also, $N(1 + \lambda) = (-1)^n ( 1 \pm a_1 \pm \cdots +a_m  ) ^{(K :
  k(\lambda))}$.  
If $|N\lambda \geq 0|$, then $|a_m |\geq 0$ which , by lemma \ref{chap8:lem3},
means that $|a_1|, \ldots , |a_m|$. Hence $|N( 1 + \lambda)| \geq
0$. Our theorem is proved.  

Incidentally it shows that the extended valuation  is discrete
also.  


\section{Fields complete under archimedian valuations}%Sec 6

Suppose $k$ is complete under an archimedian valuation. Then $k$ has
characteristic zero and contains, as a subfield, the completion
$\bar{\Gamma}$ of the rational number field. $bar{\Gamma} (i)$ is,
then, the complex number field. Every complex number is to the form $a
+ ib$, $a$, $b \in \bar{\Gamma}$. On $\bar{\Gamma}$, we have the ordinary
absolute value. Define in $\bar{\Gamma} (i)$ the function  
$$
|z| = (a^2 + b^2 )^{\frac{1}{2}}
$$
where $z = a + ib $. It is, then, easy to verify that $| |$ is a
valuation on $ \bar{\Gamma} (i)$ which extends the valuation in
$\bar{\Gamma}$. Also, by theorem \ref{chap8:thm4}, this is the only
extension of the 
ordinary absolute value. We consider the case $k \supset \bar{\Gamma
}(i)$ and prove the theorem of \textit{A. Ostrowski.} 

\begin{thm}\label{chap8:thm6} %theorem 6
 Let $k \supset \bar{\Gamma} (i)$ be the complex number field and
  let $k$ be a field with archimedian valuation and containing $\bar{
    \Gamma} (i)$. If the valuation in $k$ is an extension of the
  valuation in $\bar{\Gamma}(i)$, then $k = \bar{\Gamma}(i)$.  
\end{thm}

\begin{proof}
If\pageoriginale $k \neq \bar{\Gamma}(i)$, let $a \in k$ but not in
$\bar{\Gamma}(i)$. Denote by $||$ the valuation in $k$. Consider $|a
- z |$ for all $ k= \bar{\Gamma} (i)$. Since $|a - z | 1 \geq
0$, we have  
$$
\rho = g  \cdot \frac{1}{2} \cdot  b |a - z| \leq 0. 
$$

There exists, therefore, a sequences $z_1, \ldots , z_n, ..$ of
complex numbers such that  
$$
\lim_{ n \to \infty} | a - z_n| = \rho. 
$$
\end{proof}

But $z_n = z_n - a+ a$ and so $|z_n| \leq |z_n - a| + |a|$ which shows
that the $|z_n|$, for large $n$, are bounded. We may therefore, choose
a subsequence $z_{ i _1} , z_{ i _2}, \ldots $ converging to a limit
point $z_o$ such that  
$$
\rho = \lim_{ n \to \infty} |a - z_{ i_n}|  = |a - z_o  |
$$

We have thus proved the existence of a $z_0$ in $\bar{\Gamma} (i)$
such that $b = a - z_0$ has $|b| = \rho $. Since, by assumption, $a
\notin \bar{\Gamma}(i)$ we have  
$$
|b| = \rho > 0.
$$
$\rho$, by definition, being $g.l.b.$, it follows that  
$$
| b - z| \geq \rho 
$$
for $z \in \bar{\Gamma}(i)$.

Consider the set of complex numbers $z$  with $|z| < \rho$.  
Let $n > 0$ be an arbitrary rational integer and $\varepsilon$, a
primitive $n$th root of unity. Then  
$$
b^n - z^n = ( b - z) (b - \varepsilon z) \ldots (b - \varepsilon ^{ n
  -1}z).  
$$

Therefore\pageoriginale 
\begin{gather*}
|b - z | \rho^{ n - 1} \leq | b - z| | b - \varepsilon z|  \ldots | b
-\varepsilon^{ n - 1}  z |= | b^n - z^n | \\ 
\leq |b|^n + |z|^n = \rho^n ( 1+ (\frac{|z|}{\rho})^n ). 
\end{gather*}

Hence
$$
|b - z |\leq \rho \left( 1+ \frac{|z|^n}{\rho^n} \right). 
$$

But $|z| < \rho$ and as $n$ is arbitrary, it follows that $ |b - z |
\leq \rho$. 

We therefore have 
$$
|z| < \rho \Rightarrow |b - z| = \rho.  
$$

We now prove that for every integer $m > 0$, $|b - mz| = \rho $  if $|z|
< \rho$. For, suppose we have proved this for $m - 1$ instead of $m$,
then we can carry through the above analysis with $b- (m - 1)z$ instead
of $b$, then we can carry through the above analysis with $ b -
(m-1)z$ instead of $b$ and then we obtain $|b - mz| = \rho$. 

Suppose now that $z'$ is \textit{any} complex number. Then there is
an integer $m > 0$ such that $|\dfrac{z'}{m}|< \rho$.  Therefore 
$$
\left|b-m \frac{z'}{m} \right|=|b-z'|=\rho.
$$
Now $z ' = z' - b + b$ and so 
$$
|z'| \leq |b - z'| + |b| \leq 2 \rho  
$$
which shows that all complex numbers are bounded in absolute
value. This is a contradiction. Hence our assumption that $a \notin
\bar{\Gamma} (i)$  is false .  

The theorem is thereby proved,  

Before proving theorem \ref{chap8:thm7} which gives a complete
characterization of 
all complete fields with archimedian valuation, we shall prove a
couple of lemmas.  

\begin{lem}\label{chap8:lem4} % lemma 4
 Let\pageoriginale $k$ be complete under an archimedian valuation $||$ and
  $\lambda$ in $k$  such that $x^2 + \lambda$ is irreducible in
  $k[x]$. Then $| 1 + \lambda | \geq 1$.  
\end{lem}

\begin{proof}
If possible, let $| 1 + \lambda| < 1$. We construct, by recurrence, the
sequence $c_0 , c_1, c_2 , \ldots , $ in $k$, defined as follows: -  
\begin{gather*}
c_0 = 1 \\
c_{n + 1} = - 2- \frac{1 + \lambda}{c_n} \qquad n = 0, 1, 2, \ldots 
\end{gather*}

It, then, follows that $|c_n |\geq 1$. For, if we have proved it upto
$c_{n -1}$, then  
$$
|c_n| \geq 2 - \frac{|1+\lambda|}{|c_{n-1}|} \geq 1. 
$$

Thus $c_n$ does not vanish for any $n$. Also, 
$$
|c_{n +1} - c_n| = \frac{|1+\lambda | |c_n - c_{n-1}|}{|c_n||c_{n-1}|}
\leq \rho |c_n-c_{n-1}| 
$$
where $\rho = |1+\lambda| < 1$. This means that the series  
$$
c_0 + (c_1 - c_0) + (c_2 - c_1) + \cdots 
$$  
converges in $k$. Let it converge to $c$ in $k$. Then  
$$
c= \lim_{ n \to \infty} c_o + \cdots + (c_n - o_{ n -1}) = \lim_{ n 
  \to \infty} c_n. 
$$ 
 
Therefore, by definition of $c_n$, we get $c = -2-
\dfrac{1+\lambda}{c}$. But this means that $- \lambda =c^2 + 2c + 1 =
(c + 1)^2$ which contradicts the fact $x^2 + \lambda$ is irreducible
in $k[x]$.  

We now prove the 
\end{proof}

\begin{lem}\label{chap8:lem5} % lemma 5
 If $k$ is complete under an archimedian valuation, $||$, then this
  valuation can be extended to $k(i)$.  
\end{lem}

\begin{proof}
If $i \in k$, there is nothing to prove. Let $ i \notin k$. Then every
element of $k(i)$ is of the form $a + ib$, $a$, $b \in k$.  

The\pageoriginale norm from $k(i)$ to $k$ of $\alpha  = a + ib $ is   
$$
N \alpha = a^2 + b^2 . 
$$

By theorem \ref{chap8:thm4}, therefore, it is enough to prove that  
$$
|\alpha| = |(a^2 + b^2)^{ \frac{1}{2}}|
$$
is a valuation on $k(i)$. By putting $\lambda = \dfrac{b^2}{a^2} $ in
the lemma \ref{chap8:lem4}, we see that $|a^2 + b^2 |\geq a^2 $. Therefore  
$$
|(1+a)^2 + b^2| \leq 1 +| a^2 + b^2 | + 2 | a | 
$$ 
\begin{align*}
& \leq 1 + |a^ 2 + b^2 | + 2 \sqrt{ |a^2 + b^2|}\\ 
& = (1 + \sqrt{ |a^2 + b^2 |})^2. 
\end{align*}

This shows that 
$$
| 1 + \alpha | \leq 1 + | \alpha |
$$
and our lemma is proved. 
\end{proof}

We now obtain a complete characterization of complete archimedian
fields, namely,  

\begin{thm}\label{chap8:thm7} %theorem 7
The only fields complete under an archimedian valuation are the
  real and complex numbers fields. 
\end{thm}

\begin{proof}
$k$ has characteristic zero and since it is complete, it contains the
  field $\bar{\Gamma}$ of real numbers. If $k$ contains $\bar{\Gamma}$
  properly, then we assert that $k$ contains $i$. For, $k(i)$, by
  lemma \ref{chap8:lem5}, is complete and $k(i)$ contains
  $\bar{\Gamma}(i)$ and, by   theorem \ref{chap8:thm6},  
$$
k(i) = \bar{\Gamma}(i). 
$$
\end{proof}

Therefore
$$
\bar{\Gamma}(i) = k(i) \supset k \supset \bar{\Gamma}. 
$$

But $(\bar{\Gamma}(i) : \bar{\Gamma}) = 2$ so that $ k = k(i) =
\bar{\Gamma}(i)$. 

We have thus found all complete fields with archimedian valuation. 


\section{Extension of valuation of an incomplete field}%%% 7

Suppose\pageoriginale $k$ is a complete field under a valuation $||$
and let  $\Omega$ be its algebraic closure. Then $||$ can be extended
to $\Omega$ by the prescription  
$$
| \alpha | = | N \alpha |^{\frac{1}{n}}, 
$$
where Norm is takes form $k(\alpha)$ over $k$ and $n = (k (\alpha):
k)$. It is clear that it defines a valuation function. For, of $K$ is 
a subfields of $\Omega$ and $K/k$ is finite and $K$ contains $\alpha$ then,
by properties of norms,  
$$
|\alpha| = | N_{ K/k}\alpha|^{\frac{1}{m}},  
$$
where $m = (K : k)$. So, if $\alpha$ and $\beta$ are in $\Omega$, we
may take for $K$ a field containing $\alpha$ and $\beta$ and with $(K
: k)$ finite.  

Furthermore, defined as such, the valuation on $\Omega$ is dense
because, if $|\alpha| > 1$, then $|\alpha|^{ 1/n}$ has value as near
1 as one wishes, by increasing $n$ sufficiently. Also, for every
$n$, $\alpha^{1/n}$ is in $\Omega$.  

Also, let $\sigma$ be an automorphism of $\Omega / k$, and $\alpha$ in
$\Omega$. Then, by definition of norm,  
$$
N \alpha = N (\sigma \alpha) 
$$
so that $ |\alpha| = |\sigma \alpha | $. Thus all conjugates of an
element have the same value.  

We shall now study how one can extend a valuation of an incomplete
field to an algebraic extension.  

Let $k$ be a field and $K$ a finite algebraic extension of it. Let
$||$ be a valuation of $k$ and $\bar{k}$ the completion of $k$
under\pageoriginale this valuation. Let $k \neq \bar{k}$.   
\[
\xymatrix@R=0.5cm{
& K\bar{k}\ar@{-}[dd]\ar@{-}[dl]\\
K \ar@{-}[dd] & \\
& \bar{k} \ar@{-}[dl]\\
k & 
}
\]

Suppose it is possible to extend $||$ to $K$. Let $\bar{K}$ be the
completion of $K$ under this extended valuation. Since $\bar{K}
\supset K \supset k$, it follows that $\bar{K}$ contains $K$ and
$\bar{k}$ and therefore the composite $K \bar{k}$. Thus  
$$
\bar{K} \supset K\bar{k}.
$$

On the other hand,  $K \bar{k}/ \bar{k}$ is a finite extension, since
$K/k$ is finite. Since $\bar{k}$ is complete, $K \bar{k}$ is complete
also. $K \bar{k}$ contains $K$ and hence its completion $\bar{K}$
under this extended valuation. Thus  
$$
\bar{K} = K \bar{k}. 
$$

Thus if the valuation can be extended, then the completion of $K$ by
this extended valuation is a composite extension of $K$  and
$\bar{k}$.  

Suppose now that $\Omega$ is an algebraic closure of
$\bar{k}$. $\Omega$, then, contains an algebraic closure of $k$. We
have seen above that the given valuation of $k$ can be extended to
$\Omega$. Let $\sigma$ be an isomorphism of $K/k$ into $\Omega$. The
valuation in $\bar{k}$ can be extended to $\sigma K \cdot \bar{k}$ which
is a subfield of $\Omega$. Therefore, there is a valuation on $\sigma
K$. Define now, for $\alpha$ in $K$, 
$$
| \alpha |_o = | \sigma \alpha | 
$$ 
where $||$ is the extension of $||$ on $k$ to $\sigma K \cdot
\bar{k}$, which extension is unique. It is now trivial to see that
$||_o$ is a valuation on $K$ and extends $||$ on $k$.  

Hence every isomorphism of $K$ into $\Omega$ which is trivial on $k$,
gives rise to a valuation of $K$.  

We\pageoriginale now investigate when two isomorphisms give rise to
the same valuation on $K$. Let $\sigma$ and $\tau$ be two isomorphisms
of $K/k$ into $\Omega$ giving the same valuation on $K$. $\sigma K$
and $\tau K$ are subfields of $\Omega$ and they have the same 
valuation. Thus $\mu = \sigma \tau^{-1}$ is an isomorphism of $\tau K$
onto $\sigma K$ which preserves the valuation on $\tau K$. Now $\mu$
is identity on $k$ and so on $\bar{k}$. Since $\sigma K\cdot \bar{k}$ is
the completion of $\sigma K$, it follows that $\mu$ is an isomorphism
of the composite extensions $\sigma K \bar{k}$ and $\tau K
\bar{k}$. Hence, if $\sigma$ and $\tau$ give rise to the same
valuation on $K$, the corresponding composite extensions are
equivalent.  

Suppose now that $\sigma$ and $\tau$ are isomorphisms of $K$ into
$\Omega$ such that the composite extensions $\sigma K \bar{k}$ and
$\tau K \bar{k}$ are equivalent. There exists then a mapping $\mu$ of
$\tau K \bar{k}$ on $\sigma K \bar{k}$ which is identity on $\bar{k}$
and such that  
$$
\mu \tau = \sigma 
$$ 
$\sigma$ induces a valuation $| |_1$ on $K$ such that $|\alpha|_1 =
|\sigma \alpha|$ and $\tau$ induces a valuation $| |_2$ on $K$ such
that $|\alpha|_2 = |\tau \alpha |$. But $\mu$ is such that $\mu \tau
\alpha = \sigma \alpha$ or $\sigma \alpha$ and $\tau \alpha$ are
conjugates over $\bar{k}$ in $\Omega$. Hence  
$$
| \sigma \alpha| = |\tau \alpha|
$$
or $||_1 = | |_2$ which shows that $\sigma$, $\tau$ give the same
valuation on $K$. 

We have hence the 

\begin{thm}\label{chap8:thm8}%them 8
A valuation $||$ of $k$ can be extended to a finite extension $K$ of
$k$ only in a finite number of ways. The number of\pageoriginale these
extensions of $||$ to $K$ stand in a (1, 1) correspondence with the
classes of composite extensions of $K$ and $\bar{k}$  
\end{thm}

From what we have already seen, the number of distinct composite
extensions of $K$ and $\bar{k}$ is at most $(K : k)$. 

We apply these to the case where $k = \Gamma$, rational number field
and $K /\Gamma$ finite so that $K$ is an algebraic number filed. From
theorem \ref{chap8:thm2}, it therefore follows that $K$ has at most
$(K : \Gamma)$ 
distinct archimedian valuations and that all the non-archimedian
valuations of $K$, which are countable in number, are discrete. 

In a similar manner, if $K$ is an algebraic function field of one
variable over a constant filed $k$, then all the valuations of $K$ are
non-archimedian and discrete. This can be seen from the fact that if
$x \in K$ is transcendental over $k$, then $K /k(x)$ is algebraic and
one has only to apply theorems \ref{chap8:thm3} and \ref{chap8:thm8}.   



