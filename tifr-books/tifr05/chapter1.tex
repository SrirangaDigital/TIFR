\chapter{General extension fields}\label{chap1}

\section{Extensions}\label{c1:s1}%sec 1.

A field\pageoriginale has characteristic either zero or a prime number
$p$. 

Let $K$ and $k$ be two fields such that $K \supset k$. We shall say
that $K$ is an \textit{extension field} of $k$ and $k$ a
\textit{subfield} of $K$. Any field $T$ such that $K \supset T \supset
k$ is called an \textit{intermediary} field, intermediate between $K$
and $k$. 

If $K$ and $K'$ are two fields, then any homomorphism of $K$ into $K'$
is either trivial or it is an isomorphism. This stems from the fact
that only ideals in $K$ are $(o)$ and $K$. Let $K$ have characteristic
$p \neq o$. Then the mapping $a \to a^p$ of $K$ into itself is an
isomorphism. For, 
\begin{align*}
(a \pm b)^p & = a^p \pm b^p\\
(ab)^p & = a^p \cdot b^p
\end{align*}
and $a^p = b^p \Longrightarrow (a-b)^p = o \Longrightarrow a = b$. In
fact for any integer $e \geq 1$, $a \to a^{p^e}$ is also an isomorphism
of $K$ into itself. 

Let now $Z$ be the ring of rational integers and $K$ a field whose
unit element we denote by $e$. The mapping $m \to $ me  of $Z$ into
$K$ obviously a homomorphism of the ring $Z$ into $K$. The kernel of
the homomorphism is the set of $m$ is $Z$ such that $me =0$ in
$K$. This is an ideal in $Z$ and as $Z$ is a principal ideal domain,
this ideal is generated by integer say $p$. Now $p$ is either zero or
else is a prime. In the first case it means that $K$ contains a
subring isomorphic to $Z$ and $K$ has characteristic zero. Therefore
$K$ contains a subfield isomorphic to the field of rational numbers. In  the
second case $K$ has characteristic $p$ and since $Z/ (p)$ is a finite
\pageoriginale field of $p$ elements, $K$ contains a subfield
isomorphic to $Z / (p)$. Hence the 

\begin{thm}\label{c1:thm1}%%% 1
 A field of characteristic zero has a subfield isomorphic to the
  field of rational numbers and a field of characteristic $p > o$ has
  a subfield isomorphic to the finite of $p$ residue classes of $Z$
  modulo $p$. 
\end{thm}

The rational number filed and the finite field of $p$ elements are
called \textit{prime fields}. We shall denote them by $\Gamma$. When
necessary we shall denote the finite field of $p$ elements by $\Gamma p$. 

Let $K/k$ be an extension field of $k$. We shall identity the
elements of $K$ and $k$ and denote the common unit element by
$1$. Similarly for the zero element. $K$ has over $k$ the structure of
a vector space. For, $\alpha$, $\beta \in K$, $\lambda \in k
\Longrightarrow \alpha + \beta \in K$, $\lambda \alpha \in
K$. Therefore $K$ $\delta$ has over $k$ a base $\{\alpha_\lambda
\}$ in the sense that every $\alpha \in K$ can be uniquely written in
the form 
$$
\alpha = \sum_{\lambda} a_{\lambda} \alpha_{\lambda} \quad 
       {a_{\lambda}} \in k 
$$
and $a_{\lambda} = 0 $ for almost all $\lambda$. If the base $\{
\alpha_{\lambda}\}$ consists only of a finite number of elements we
say that $K$ has a finite base over $k$. The extension $K/k$ is called
a \textit{finite} or \textit{infinite extension} of $k$ according as
$K$ has over $k$ a finite or an infinite base. The number of basis
elements we call the \textit{degree} of $K$ over $k$ and denote it by
$(K : k)$. If $(K : k) =n$ then there exist $n$ elements $\omega ,
\ldots \omega_n$ in $K$ which are linearly independent over $k$ and
every $n +1$ elements of $K$ linearly dependent over $k$. 

Let $K$ be a finite field of $q$ elements. Obviously $K$ has
characteristic $p \neq o$. Therefore $K$ contains a subfield
isomorphic to $\Gamma _p$. Call it also $\Gamma_p$. $K$ is a finite
dimensional vector space over $\Gamma_p$. Let $(K : \Gamma_p)=n$ Then
obviously $K$ has $p^n$ elements.\pageoriginale Thus  


\begin{thm}\label{c1:thm2}%%% 2 
The number of elements $q$ in a finite field is a power of the
  characteristic. 
\end{thm}

Let $K \supset T \supset k$ be a tower of fields. $K/ T$ has a base
$\{\alpha _{\lambda}\}$ and $T/K$ has a base
$\{\beta_{\nu}\}$. This means that for $\alpha \in K$  
$$
\alpha = \sum_{\lambda} t_{\lambda} \alpha_{\lambda}
$$
$t_ \lambda \in T$ and $ t_\lambda =0$ for almost all $\lambda$. Also
$t_{\lambda}$ being in $T$ we have 
$$ 
t_{\lambda}= \sum_{\mu} a_{\lambda \mu} \beta_{\mu}
$$ 
$a_{\lambda \mu} =0$ for almost all $\mu$. Thus
$$
\alpha = \sum_{\lambda \mu} a_{\lambda \mu} (\alpha_{\lambda} 
\beta_{\mu}) 
$$

Thus every element $\alpha$ of $K$ can be expressed linearly in terms
of $\{ \alpha_{\lambda} \beta_{\mu}\}$. On the other hand let 
$$
\sum_{\lambda \mu} (\alpha_{\lambda} \beta_{\mu}) = 0
$$
$a_{\lambda\mu} \in k$ and $a_{\lambda \mu} =0$ for almost all $\lambda ,
\mu$. Then  
$$
0 = \sum_{\lambda}(\sum_{\mu} a_{\lambda \mu} \beta_{\mu})
\alpha_{\lambda} 
$$

But $\sum_{\mu} a_{\lambda \mu} \beta_{\mu} \in T$ and since the
$\{\alpha _{\lambda}\}$ from a base of $K/T$ we have  
$$
\sum_{\lambda} a_{\lambda \mu} \beta_{\mu} =0 \text{ for all } \lambda .
$$

But $\{ \beta_{\mu}\}$ form a base of $T/ k$ so that $a_{\lambda \mu}
=0$ for all $\lambda, \mu$. We have thus proved that
$\{\alpha_{\lambda} \beta_{\mu}\}$ is a base of $K/k$. In particular
if $(K : k)$ is finite then $(K : T)$ and $(T : k)$ are finite and  
$$
(K:k) = (K:T) (T:k)
$$

As special  cases, $(K:k) = (T: k) \Longrightarrow K =T $ ($T$ is an
intermediary field of $K$ and $k$). $(K:k) = (K:T) \Longrightarrow T=
k$. 

\section{Adjunctions}\label{c1:s2}%%% 2

Let $K/k$ be an extension filed and $K_{\alpha}$ a family of
intermediary extension fields. Then $\bigcap\limits_{\alpha}
K_{\alpha}$ is again an intermediary field\pageoriginale but, in general,
$\bigcup\limits_{\alpha}K_{\alpha}$ is not a field. We shall define
for \textit{any subset} $S$ of $K/k$ the field $k(S)$ is called the
\textit{field generated by}$S$ over $k$. It is trivial to see that 
$$
k(S) = \bigcap_{T \supset S} T
$$
i.e., it is the intersection of all intermediary fields $T$ containing
$S$. $k(S)$ is said to be got from k by \textit{adjunction} of $S$ to
$k$. If $S$ contains a finite number of elements, the adjunction is
said to be \textit{finite} otherwise \textit{infinite}. In the former
case $k(S)$ is said to be finitely generated over $k$. If $(K : k) <
\infty$ then obviously $K$ is finitely generated over $k$ but the
converse is not true. 

Obviously $k (S U S') = k (S) (S')$ because a rational function of $ S
U S'$ is a rational function of $S'$ over $k(S)$. 

Let $K/k$ be an extension field and $\alpha \in K$. Consider the ring
$k[x]$ of polynomials in $x$ over $k$. For any $f(x) \in k [x]$, $f(x)$
is an element of $K$. Consider the set $\mathscr{G}$ of polynomials
$f(x) \in k[x]$ for which $f(\alpha) =o$. $\mathscr{G}$ is obviously a
prime ideal. There are now two possibilities, $\mathscr{G} = (0)$,
$\mathscr{G} \neq (o)$. In the former case the infinite set of elements
$1, \alpha , \alpha^2, \ldots$ are all linearly  independent over
$k$. We call such an element $\alpha$ of $K$, \textit{transcendental}
over $k$. In the second case $\mathscr{G} \neq (o)$ and so
$\mathscr{G}$  is a principal ideal generated by an irreducible
polynomial $\varphi (x)$. Thus $1, \alpha , \alpha^2 , \ldots$ are
linearly dependent. We call an element $\alpha$ of this type
\textit{algebraic} over $k$. We make therefore the  

\begin{defi*}
 Let $K/k$ be an extension field. $\alpha \in K$ is said to be
  algebraic over $k$ if $\alpha$ is root of a non zero polynomial in
  $k[x]$. Otherwise it is said to be transcendental. 
  
If $\alpha$ is algebraic, the ideal $\mathscr{G}$ defined above is
called\pageoriginale the \textit{ideal of} $\alpha$ over $k$ and the
irreducible 
polynomial $\varphi(x)$ which  is a generator of $\mathscr{G}$ is
called the \textit{irreducible polynomial} of $\alpha$ over
$k$. $\varphi(x)$ may be made by multiplying by a suitable element of
$k$. This monic polynomial we shall call the minimum polynomial of
$\alpha$. 
\end{defi*}

\section{Algebraic extensions}\label{c1:s3}%%% sec 3

Suppose $\alpha \in K$ is algebraic over $k$ and $\varphi (x)$ its
minimum polynomial over $k$. Let $f(x) \in k[x]$ and $f(\alpha) \neq
o$. $f(x)$ and $\varphi (x)$ are then coprime and so there exist
polynomials $g(x)$, $h(x)$ in $k[x]$ such that  
$$
f(x) g(x) =1 + \varphi (x) h(x)
$$
which means that $(f (\alpha))^{-1} = g(\alpha) \in k[\alpha]$. Thus
$k[\alpha] = k(\alpha)$. On the other hand suppose $\alpha \in K$ such
that $k[\alpha] = k(\alpha)$ then there is a $g(\alpha)$ in
$k[\alpha]$ such that $\alpha g(\alpha) =1$ or that $\alpha$ satisfies
$x g(x) -1$ in $k[x]$ so that $\alpha$ is algebraic. Hence 
\begin{enumerate}[1)]
\item  $\alpha \in K$ \textit{algebraic over} $k \Longleftrightarrow
  k[\alpha]$ \textit{is a field}. 

We now define an extension $K/k$ to be algebraic over $k$ if every of
$K$ is algebraic over $k$. In the contrary case $K$ is said to be
transcendental extension of $k$ 

We deduce immediately

\item $K/k$ \textit{algebraic $\Longleftrightarrow$ every ring $R$
  with} $k \subset R \subset K$ \textit{is a field} 

If $R$ is a ring and $\alpha$ in $R$ then $k[\alpha] \subset R$ then
$k[\alpha] \subset R$. But $\alpha $ is algebraic so that
$\alpha^{-1}\in k[\alpha] \subset R$ so that $R$ is a field. The
converse follows from $(1)$. 

\item $(K : k) < \infty \Longrightarrow K/k$ \textit{algebraic.} 

For let $(K : k) =n$ then for any for $\alpha \in K$, the $n+1$
elements $1, \alpha , \alpha^2 , \ldots \alpha^n$ are linearly
dependant over $k$ so that $\alpha$\pageoriginale is algebraic. 

The converse is not true

Let $K/k$ be an extension field and $\alpha \in K$ algebraic over
$k$. Let $\varphi (x)$ be the minimum polynomial of $\alpha$ over $k$
and let degree of $\varphi (x)$ be $n$. Then $1, \alpha, \alpha^2 ,
\ldots , \alpha^{n-1}$ are linearly independent over $k$ so that 
$$
(k (\alpha) : k ) \geq n.
$$

On the other hand any $\beta$ in $k (\alpha)$ is a polynomial in
$\alpha$ over $k$. Let $\beta = b_o + b_1 \alpha + \cdots + b_m
\alpha^m $. Put  $\psi (x) = b_o + b_1 x + \cdots + b_m x^m$. 

Then
$$
\psi (x) = \varphi (x) h(x) + R(x)
$$
where $R(x) =0$ or deg $R(x) < n$. Hence $\psi (\alpha) = \beta =
R(\alpha)$ and so every $\beta$ cab be expressed linearly in terms of
$1, \alpha, \ldots, \alpha^{n-1}$.  

Thus
$$
\bigg( k(\alpha) : k \bigg) \le n.
$$

We have hence

\item \textit{If $\alpha \in K $ is algebraic over $K, k(\alpha) / k$
  is an algebraic extension of degree equal to the degree of the
  minimum polynomial of $\alpha$ over $k$}. 

We shall call $\bigg( k(\alpha) : k \bigg)$ \textit{the degree of
  $\alpha$ over $k$} 

\item \textit{If $\alpha$ is algebraic over $k$ then for any $L, k
  \subset L \subset K, \alpha$ is algebraic over $L$}. 

For, the ideal of $\alpha$ over $k$  (which is $\neq (0) $ since
$\alpha$ is algebraic) is contained in the ideal of $\alpha$ in $L [X]
\supset k [x]$. 

Therefore
$$
\bigg( k(\alpha) : k \bigg) \geq (L (\alpha) : L)
$$

Note\pageoriginale that the converse is not true. For let $z$ be
transcendental over 
$k$ and consider the field $k(z)$ of rational functions of $z. \bigg(
k(z) : k \bigg)$ is not finite. But $\bigg( k(z) : k  (z^2)\bigg)$ is
finite as $z$ is a root of $x^2 - z^2$ over $k(z^2)$. 

\item \textit{If $\alpha_1, \ldots, \alpha_n$ in $K$ are algebraic
  over $k$ then $k(\alpha_1, \ldots, \alpha_n)$ is algebraic over
  $k$}. 

For, put $K_o = k$, $K_i = k (\alpha_1, \ldots, \alpha_i)$, 

\noindent
$K_n = k (\alpha_1, \ldots, \alpha_n)$

Then $K_i/K_{i-1}$ is algebraic and is a finite extension. Now  
$$
(K_n : k) = \pi_i (K_i : K_{i-1})
$$
which is also finite. Hence $K_n$ is algebraic over $k$.

We deduce immediately

\item $K/T$ \textit{algebraic, $T/k$ algebraic $\Longrightarrow  K/k$ 
  algebraic.} 
\end{enumerate}

For if $\alpha \in K$, $\alpha$ is a root of $\varphi (x) = X^n + a_1
x^{n+1}+ \cdots + a_n$ in $T[x]$. Thus $\alpha$ is algebraic over $k
(a_1 \ldots , a_n)$. Hence   
\begin{align*}
\big(k(a_1, a_2, \ldots a_n, \alpha) & : k (a_1, a_2, \ldots, a_n)
\big) < \infty.\\
& \big(k(a_1, a_2, \ldots, a_n) : k \big) < \infty\\
& \big(k(a_1, a_2, \ldots, a_n, \alpha) : k \big) < \infty
\end{align*}
which proves the contention.

If follows that if $K/k$ is any extension, then the set $L$ of elements
$\alpha$ of $K$ algebraic over $k$ is a field $L$ which is algebraic
over $k$. $L$ is called the \textit{algebraic closure of $k$ in $K$} 

We shall now show how it is possible to construct algebraic extensions
of a field. 

If $k$ is a field and $\varphi (x)$ a polynomial in $k[x]$, an element
$\alpha$ of an extension field $K$ is said to be \textit{root} of
$\varphi(x) $ if $\varphi (\alpha) = o$. It then follows that
$\varphi(x)$, has in $K$ at most $n$ roots,\pageoriginale $n$ being
degree of $\varphi (x)$.  

Let $f(x)$ be an irreducible polynomial in $k[x]$

The ideal generated by $f(x)$ in $k[x]$ is a maximal ideal since
$f(x)$ is 
irreducible. Therefore the residue class ring $K$ of $k[x]/ (f(x))$ is
a field. Let $\sigma$ denote the  natural homomorphism of $k[x]$ onto
$K$. $\sigma$ then maps $k$ onto a subfield of $K$. We shall identify
this subfield with $k$ itself (note that $k[x]$ and $(f(x))$ are
vector spaces over $k$). Let $\xi$ in $K$ be the element into which
$\underline{x}$ goes by $\sigma$ 
$$
\xi = \sigma x
$$

Then $K = k(\xi)$. In the first place $k (\xi) \subset K$. Any element
in $K$ is the image, by $\sigma$, of an element say $\varphi (x)$ in
$k[x]$. But  
$$
\varphi (x) = h(x) f(x) + \psi(x)
$$ 

So $\varphi (\xi) \in K$ and $\varphi (\xi) = \psi (\xi)$. But $\psi
(x)$ above has degree $\leq $ degree of $f$. Thus  
$$
k(\xi) \subset K \subset k[\xi] \subset k (\xi)
$$

This shows that $K = k(\xi)$ and that $(K : k)$ is equal to the degree
of $f(x)$. Also $\xi$ in $K$ satisfies $f(\xi) =o$. We have thus
proved  that for every irreducible polynomial  $f(x)$ in $k[x]$ there
exists  an extension field in which  $f(x)$ has a root. 

Let now  $g(x)$ be any polynomial in $k[x]$ and  $f(x)$ an irreducible
factor of  $g(x)$ in $k[x]$. Let $K$ be an extension of $k$ in which
$f(x)$ has a root $\xi$. Let in $K$ 
$$
g(x) = (x - \xi)^{\lambda}\psi (x).
$$

Then $\psi (x) \in k[x]$. We again take an irreducible factor of $\psi
(x)$ and construct $K'$ in which $\psi(x)$ has a root. After  finite
number of steps we arrive at a field $L$ which is an extension of $k$
and in which   $g(x)$ splits completely into linear factors. Let
$\alpha _1 , \ldots, \alpha _n$  be the\pageoriginale distinct roots
of  $g(x)$ in $L$. We call $k(\alpha_1 ,\ldots , \alpha_n)$
\textit{the splitting   field of  $g(x)$ in $L$}.  

Obviously $(k(\alpha_1 ,\ldots , \alpha_n):k) \leq n !$

We have therefore the important

\begin{thm}\label{c1:thm3}%%% 3
If $k$ is a field and $f(x) \in k[x]$ then $f(x)$ 
  has a splitting field $K$ and $(K:k) \leq n !, n$ being degree of
  $f(x)$. 
\end{thm}

It must be noted however that a polynomial might have several
splitting fields. For instance if $D$ is the quaternion algebra over
the rational number field $\Gamma$, generated by $1, i, j,k$ then
$\Gamma (i)$, $\Gamma(j)$, $\Gamma (f)$, $\Gamma (k)$ are all
splitting fields of $x^2 
+1 $ in $\Gamma [x]$. These splitting fields are all distinct. 

Suppose $k$ and $k'$ are two fields which are isomorphic by means of
an isomorphism $\sigma$. Then $\sigma$ can be extended into an
isomorphism $\bar{\sigma}$ of $k[x]$ on $k'[x]$ by the following
prescription   
$$
\bar{\sigma}(\sum a_i x^i) = \sum (\sigma a_i )x^i \quad  a_i \in k
\quad \sigma a_i \in k'  
$$

Let now $f(x)$  be a polynomial in $k[x]$ which is irreducible. Denote
by $f^{\bar{\sigma}} (x)$, its image in $k'[x]$ by means of the
isomorphism $\bar{\sigma}$. Then $f^{\bar{\sigma}}(x)$ is again
irreducible in $k'[x]$; for if not one can by means of
$\bar{\sigma}^{-1}$ obtain a nontrivial factorization of $f(x)$ in
$k[x]$. 

Let now $\alpha$ be a root of $f(x)$ over $k$ and $\beta$ a root of
$f^{\bar{\sigma}}(x)$ over $k'$. Then 
$$
k(\alpha ) \simeq k[x] / (f(x)) , k' (\beta) \simeq k' [x] /
(f^{\bar{\sigma}}(x)) 
$$

Let $\tau$ be the natural homomorphism of $k'[x]$ on $k'[x] /
(f^{\overline{\sigma}}(x))$. Consider the mapping $\tau \cdot
\overline{\sigma}$ on 
$k[x]$. Since $\overline{\sigma}$ is an isomorphism, it follows that
$\tau \cdot \overline{\sigma}$ is a homomorphism of $k[x]$ on $k'[x] /
(F^{\overline{\sigma}} (x))$. The kernel of the homomorphism is the set of
$\varphi (x)$ in $k[x]$ such that 
$$
\varphi^{\overline{\sigma}} ( x )\in  \bigg( f^{\overline{\sigma}} ( x
) \bigg). 
$$\pageoriginale

This set is precisely  $ ( f ( x) ) $. Thus 
$$
k [ x ] /  ( f (x) ) \simeq k' [ x ]/ ( f^{\bar{\sigma}} (x) ) 
$$ 

By our identification, the above fields contain $k$ and $ k'$
respectively as subfields so that there is an isomorphism $ \mu $ of
$ k ( \alpha ) $ on $ k' ( \beta ) $ and  the restriction of $\mu$ to
$k$ is $ \sigma $. 

In particular if $k = k'$, then $k( \alpha )$  and  $ k ( \beta ) $
are $k -$ isomorphic i.e., they are isomorphic by means of an
isomorphism which  is identity on $k$. We have therefore  

\begin{thm}\label{c1:thm4}%them 4
If $ f (x) \in  k [ x ] $ is irreducible and $ \alpha$ and $
  \beta $  are two roots of it (either in the same extension field of
  $ k $ or in different extension fields), $ k ( \alpha ) $ and $ k
  ( \beta ) $ are $k-$ isomorphic.  
\end{thm}

Note that the above theorem is false if $f (x)$ is not
irreducible in $k[x]$. 

\section{Algebraic Closure}\label{c1:s4}%section 4

We have proved  that every polynomial over $k$  has a splitting
field. For a given  polynomial this field might very well coincide
with $k$ itself. Suppose  $k$ has the property that every  polynomial
in $k$ has a root in $k$. Then it follows that  the only irreducible
polynomials over $k$ are linear polynomials. We make now the  

 \begin{defi*}
A field $\Omega$ is  \textit{algebraically closed} if  the only
irreducible  polynomials in $\Omega [ x ]$ are linear polynomials.  
 \end{defi*} 
 
 We had already defined the algebraic closure of a field $k$ contained
 in a field $K$. Let us now make the   

 \begin{defi*}
A field $ \Omega /k $ is said  to be an algebraic closure of $k$ if  
\begin{enumerate}[1)]
\item $ \Omega $ is  algebraically closed 

\item  $ \Omega / k $\pageoriginale is algebraic.
\end{enumerate}
 \end{defi*} 
 
 We now prove the important 

\begin{thm}\label{c1:thm5}%them 5.
Every field  $k$ admits, upto $k$-isomorphism, one and only one
  algebraic  closure. 
\end{thm} 

\begin{proof}
\begin{enumerate}[1)]
\item  \textit{Existence.}  Let $M$ be the family of algebraic
  extensions $K_{\alpha}$ of  $k$. Partially order $M$ by
  inclusion. Let $ \{ K_\alpha \}$ be a totally ordered subfamily of
  $M$. Put $ K = \bigcup\limits_{\alpha} K_\alpha $ for  $K_\alpha $
  in  this totally ordered family. Now $K$ is a field; for $\beta_1
  \in K$, $ \beta_2 \in K$  means $\beta_1 \in K_\alpha$ for some
  $\alpha$ and $\beta_2 \in K_\beta$ for some $\beta$. Therefore
  $\beta_1$, $\beta_2$ in  $K_\alpha $ or $K_\beta $ whichever is
  larger  so that $ \beta_1 + \beta_2 \in K $. Similarly $ \beta_1
  \beta^{-1}_2  \in K $.  Now $ K /k $ is algebraic since every  $
  \lambda \in K $ is in some $ K_\alpha $ and so algebraic over $
  k$. Thus $K \in M $ and so we can apply Zorn's
  lemma. This proves that $ M $ has a maximal element  $\Omega
  . \Omega $ is algebraically closed; for if not let  $f(x)$ be an
  irreducible  polynomial in $\Omega [ x ]$ and $\rho$ a root of $f
  (x)$ in an extension $\Omega ( \rho)$ of $\Omega $. Then since
  $\Omega /k $  is algebraic. $ \Omega ( \rho ) $ is an element of
  $M$. This  contradicts  maximality of $ \Omega $. Thus $ \Omega $ is
  an algebraic  closure of $k$. 

\item \textit{Uniqueness}. Let $k$ and  $k'$ be two fields  which are
  isomorphic by means of an isomorphism $\sigma $. Consider the family
  $M$ of triplets  $ \{ ( K, K', \sigma  )_\alpha \} $ with the
  property   $ 1) $ $K_\alpha $ is an algebraic extension of $k,
  K'_\alpha $ of $ k'$, $ 2) $ $ \sigma_{\alpha} $ is  an
  isomorphism of $K_\alpha $ on $ K'_\alpha $ extending $ \sigma
  $. By theorem~\ref{c1:thm4}, $M$ is not empty. We partially order
  $M$ in the   following manner  
 \end{enumerate}
 \end{proof} 
 $$
 (K, K', \sigma )_\alpha < ( K, K', \sigma )_\beta 
 $$
 
 If  1) $ K_\alpha \subset K_\beta $, $ K'_\alpha \subset K'_\beta
 $, 2)  $ \sigma_\beta$ is  an extension of $\sigma_\alpha$. Let
 $\{ K, K', \sigma )_\alpha \} $ be a simply ordered subfamily. Put $
 K = \bigcup\limits_{\alpha} K_\alpha, K'  = \bigcup\limits_{\alpha}
 K'_\alpha$, 
 
These are then algebraic over $k$ and $k'$ respectively.
 
  Define\pageoriginale $ \bar{\sigma} $ on $K$ by 
  $$
  \bar{\sigma} x = \sigma_\alpha x
  $$
where  $ x \in K_\alpha $. (Note that every $x \in K$ is in some
$K_\alpha$ in the simply ordered subfamily). It is easy to see that
$\bar{\sigma}$ is well - defined. Suppose $ x \in K_\beta $ and
$K_\beta \subset K_\alpha$ then $\sigma_\alpha $ is an extension of
$\sigma_\rho$ and so $\sigma_\alpha x = \sigma_\beta x$. This
proves  that $\bar{\sigma}$ is an isomorphism of  $K$ on $ K'$ and
extends $\sigma$. Thus the triplet  $ ( K, K', \bar{\sigma}) $ is in
$M$ and is an upper bound of the  subfamily. By Zorn's lemma there
exists a maximal triplet $(\Omega, \Omega', \tau) $. We assert that $
\Omega $ is algebraically closed; for if  not let $\rho$ be a  root
of an irreducible polynomial $ f (x) \in  \Omega [ x ] $. Then
$f^\zeta (x) \in  \Omega' [x]$  is also irreducible. Let $ \rho' $ be
a root of $ f^\tau (x) $. Then $\tau$ can be extended to an
isomorphism $ \bar{\tau}$ of  $ \Omega ( \rho )$ on  $ \Omega'
(\rho)$. Now  $(\Omega (\rho), \bar{\tau} $  is in $M$ and hence
leads to a contradiction. Thus $\Omega$ is an algebraic closure of
$k$, $\Omega'$ of $k'$ and  $ \tau $ an isomorphism of $ \Omega$ on $
\Omega' $  extending $ \sigma $. 

In particular if $ k = k' $ and $ \sigma $  the  identity isomorphism,
then $ \Omega $ and $ \Omega' $ are  two algebraic closures of $k$ and
$ \tau $ is then a k-isomorphism. 

Out theorem is  completely demonstrated.

Let $ f(X) $ be a polynomial in $ k [ x ] $ and  $K = k ( \alpha_1,
\ldots , \alpha_n $, a splitting field  of $ f (x) $, so that
$\alpha_1, \ldots , \alpha_n $ are the distinct roots  of $f(x) $ in
$K$. Let $K'$  be any other splitting field and $ \beta_1, \ldots
\beta_m $  the distinct roots  of $ f (x) $ in $K'$. Let  $\Omega$  be
an algebraic  closure of $K$ and $\Omega' $ of $K'$. Then $ \Omega $
and  $ \Omega'$ are two algebraic  closures  of  $k$. There  exists
therefore an isomorphism $\sigma$ of $\Omega $ on $ \Omega' $  which
is identity on $k$. Let $\sigma K = K_1 $. Then $ K_1 = k (
\sigma_{\alpha_1}, \ldots, \sigma_{\alpha_{n}} )$. Since $\alpha_1
,\ldots \alpha_n$ are distinct $\sigma\alpha_1 \ldots, \sigma\alpha_n$
are distinct and are roots of $f(x)$. Thus $K_1$ is a
splitting\pageoriginale field of $ f (x) $ in $ \Omega' $. This proves
that   
$$
K' = K_1 .
$$

$ \beta_1, \ldots , \beta_m $ are distinct and are roots of  $ f (x) $
in $ \Omega'$. We have  $ m = n $ and $ \beta_i = \sigma \alpha_2 $
in some order. Therefore the restriction of  $ \sigma$ to $K$ is an
isomorphism of $K$ on $K'$. We have  

\begin{thm}\label{c1:thm6}%them 6
Any two splitting fields $K$, $K'$ of a polynomial $ f (x) $ in $
  k [ x ] $ are $ k-$ isomorphic. 
\end{thm}

Let $K$ be a finite field of  $q$  elements. Then $ q= P^n $ where
$n$  is an integer $\geq 1 $ and $ p $ is the characteristic of
$K$. Also $n = ( K: \Gamma )$, $\Gamma$ being the prime field. Let
$K^*$  denote  the abelian group of non-zero elements of $K$. Then $
K^* $ being a finite group of order $q - 1$, 
$$
\alpha^{q-1} = 1 
$$ 
for all $ \alpha \in K^* $. Thus $K$ is the  splitting field of the
polynomial  
$$
x^q - x
$$
in $ \Gamma [ x ] $. It therefore follows 

\begin{thm}\label{c1:thm7}%them 7
Any two finite fields  with the  same number of elements are isomorphic.
\end{thm}

A finite field cannot be algebraically closed; for if $K$ is a  finite
field of $q$  elements  and $ \underbar{a} \in K^*$ the polynomial  
$$
f (x) = x \prod_{b \in K^*}( X-b )  + a
$$
is in $ K [ x ] $ and  has no root in $K$.


\section{Transcendental extensions}\label{c1:s5} %\section 5
 
 We had already defined a transcendental extension as one which contains
 at least on transcendental element. 
 
 Let $K/k$ be a transcendental extension and $ Z_1, \ldots , Z_n $ any
 $n$  elements of $K$. Consider the ring $ R = k [ x_1, \ldots x_n ] $
 of polynomials over $k$ in $n$  variables. Let $ \mathscr{Y} $ be
 the subset of $R$  consisting\pageoriginale of those polynomials  $ f
 ( x_1, \ldots  x_n ) $  for which   
 $$
 f (Z_1, \ldots Z_n ) = 0 .
 $$
 $\mathscr{Y}$ is obviously an ideal of $R$. If $ \mathscr{Y} = (o) $
 we say that $ Z_1, \ldots Z_n $ are \textit{algebraically
   independent} over $k$. If $ \mathscr{Y} \neq (o) $, they are said to
 be algebraically dependent. Any element of $K$  which is algebraic
 over $ k ( Z_1,\ldots , Z_n )$ is therefore algebraically dependent
 on $ Z_1, \ldots Z_n $. 

We now define a subset $S$ of $K$  to be algebraically independent
over $k$ if every finite subset of $S$ is algebraically independent
over $k$. If $ K/k $  is transcendental there  is at least one  such
non empty set $S$. 

Let $ K/k $ be a transcendental extension and let $S$, $S'$ be two
subset of $K$  with the properties  
\begin{enumerate}[i)]
\item  $S$  algebraically independent over $k$
\item $S'$ algebraically independent over $k (s) $
\end{enumerate}

Then $S$ and $S'$  are disjoint subsets  of $K$ and  $ S U S'$  are
algebraically independent over $k$. That $S$ and $S'$  are disjoint is
trivially seen. Let now $ Z_1, \ldots Z_m \in S$ and $ Z'_1, \ldots
Z'_n \in S' $  be algebraically dependent. This will mean that there
is a polynomial $f$, 
$$
f = f ( x_1, \ldots , x_{m + n } )
$$
in $m +n$  variables with coefficients  in $k$, such that  
$$
f( Z_1,\ldots, Z_m, Z'_1 , \ldots  Z'_n ) = 0.
$$

Now $f$ can be regarded as a polynomial in $x_{m+1} , \ldots , x_{m +
  n} $ with coefficients in $k ( x_1, \ldots , x_m ) $. If all these
coefficients are zero then $ Z_1 ,\ldots Z_m$, $Z'_1, \ldots Z'_n  $
are algebraically independent over $k$.  If  some coefficient is $
\neq 0$, then $f(Z_1, \ldots Z_m $, $ x_{m+1}, \ldots , x_{m + n})$
is a non zero polynomial over $k (S)$ which vanishes for $x_{m+1} =
Z'_1, \ldots x_{m+n} = Z'_n $  which contradicts the fact that $S'$ is
algebraically independent over $k (S)$. Thus $ f = o $  identically
and our contention is proved.\pageoriginale 
 
The converse of the above statement is easily proved. 

An extension field  $K / k $ is said to be generated  by a subset
$M$ of $K$  if  $ K/ k (M) $ is algebraic. Obviously $K$ itself  is a
set of  generators. $A$ subset $B$ of $K$  is said to be a
\textit{transcendence base} of $K$ if 
\begin{enumerate}[1)]
\item $B$ is a set of generators of $ K / k$
\item $B$  algebraically independent over $k$.
\end{enumerate}

If $ K /k $ is transcendental, then, it contains algebraically
independent elements. We shall prove that $K$  has  a transcendence
base. Actually much more can be proved as in  

\begin{thm}\label{c1:thm8}%them 8
Let $ K/k $ be a transcendental extension generated by $S$ and
   $A$  a set of algebraically independent elements contained in
   $S$. Then there is a transcendence base $B$ of $K$  with  
 $$
 A  \subset B \subset S
 $$
\end{thm}

\begin{proof}
 Since $S$ is a set of generators of $K$, $ K/k  (S ) $ is
 algebraic. Let $M$  be the family of subsets $ A_\alpha$  of $K$
 with  
 \begin{enumerate}[1)]
\item $A \subset  A_\alpha \subset S$
\item $A_\alpha$ algebraically independent over $k$.
\end{enumerate} 
\end{proof}

The set $M$  is not empty since  $A$ is in  $M$. Partially order $M$
by inclusion. Let $ \{ A_\alpha \} $  be  a totally ordered
subfamily. Put $ B_0 = {\bigcup\limits_\alpha} A_\alpha $. Then $
B_o \subset S $. Any finite  subset of $ B_o $  will be in some
$A_\alpha $ for  large $ \alpha$  and so $ B_o $  satisfies  2)
also. Thus using  Zorn's lemma  there exists a maximal element $B$
in $M$. Every element $x$  of $S$  depends algebraically on $B$ for
otherwise  $ B U x $  will be in $M$  and  will be larger than
$B$. Thus $ k (S)/k (B) $ is  algebraic. Since $ K / k (S) $ is
algebraic, it follows that  $B$   satisfies the  conditions of the
theorem. 

The\pageoriginale importance  of the theorem is two fold; firstly that
every set of elements algebraically independent can be  completed into a
transcendence base of $K$  and further more every set of generators
contains  a base. 

 We make the following  simple observation. Let $K /k$  be  an
 extension, $Z_1, \ldots Z_m, m $  elements  of $K$  which have the
 property that $ K /k ( Z_1 , \ldots Z_m ) $  is algebraic, i.e.,
 that $ Z_1 , \ldots , Z_m $ is a set  of generators. If $Z \in K $
 then $Z$  depends algebraically on $ Z_1 , \ldots , Z_m $ i.e., 
$ k ( Z, Z_1, \ldots ,Z_m ) / k ( Z_1 , \ldots , Z_m ) $  is
 algebraic. We may also remark  that if in the  algebraic relation
 connecting $ Z, Z_1, \ldots Z_m, Z_1 $ occurs then we can say that  
 $$
 k ( Z, Z_1, \ldots Z_m ) /k ( Z, Z_2 , \ldots , Z_m ) 
 $$
 is algebraic which means that $ Z,Z_2, \ldots Z_m $  is again a set
 of generators. 
 
 We now  prove the 

\begin{thm}\label{c1:thm9}%them 9
If $ K/k $  has a transcendence base consisting  of a finite
  number $n$  of elements, every transcendence base has $ n $
  elements. 
\end{thm}   

\begin{proof}
 Let $ Z_1 , \ldots , Z_n $  and $ Z'_1 , \ldots , Z'_m $  be two
 transcendence  bases consisting  of $n$  and $m$ elements
 respectively. If $  n \neq m $  let  $ n < m $. Now $ K / k ( Z_1 ,
 \ldots , Z_n ) $ is algebraic. $Z'_1$  is transcendental over $k$
 and depends algebraically on $ Z_1 , \ldots , Z_n $  so  that if
 $Z_1$  appears in  the algebraic relation, by the remark above, $
 Z'_1, Z_2 , \ldots , Z_n $ is again a set of generators, $ Z'_2 $
 depends algebraically on $ Z'_1 , \ldots , Z_n $. In this  algebraic
 relation at least one  of $ Z_2 , \ldots , Z_n $ has to appear since
 $ Z'_1, Z'_2 $, are algebraically independent. If $Z_2 $ appears then
 $ Z'_2, Z'_1, Z_3 , \ldots , Z'_n $  is a set of generators. We
 repeat this process $n$  times, and find, that $ Z'_1, Z'_2, Z'_3,
 \ldots , Z'_n $ is  a set of generators  which\pageoriginale means
 that $ Z'_{n + 1  } , \ldots , Z'_m $  depend algebraically on $ Z'_1
 , \ldots , Z'_n  $. This is a contradiction. So $n \geq m$. We
 interchange $n$  and  $m$  and repeat the argument and get $n \leq
 m$. This proves that $n  = m$.  
 
 The unique  integer $\underbar{n}$ will be called the
 \textit{dimension} of  $K/k$.   
$$
n = \dim_k K 
$$
\end{proof} 

It is also called the  \textit{transcendence  degree}.

A similar  theorem is true even if $K$  has infinite  transcendence
base but we don't prove it. 
 
 Let $ k \subset L \subset K $  be a  tower of extensions and let   $
 B_1 $  be  a transcendence base of $ L / k $ and $ B_2 $  that over $
 K / L $. We assert that $ B_1  U B_2 $  is a transcendence base of $
 K/k$. In the first place $ B_1 U B_2 $  is algebraically independent
 over $k$. Now $k ( B_1 U B_2 )$  is a subfield of $ L(B_2)$. Every
 element in $ L(B_2) $ is a ratio of two polynomials in $B_2$  with
 coefficients in $L$. The elements of $L$  are algebraic over $K
 (B_1)$. Thus $L (B_2)$ is  algebraic over $k ( B_1 U B_2 )$. But $ K
 /L (B_2) $  is algebraic. Thus $ K/k (B_1 U B_2 ) $  is
 algebraic. This proves our  assertion. In particular it proves  

\begin{thm}\label{c1:thm10}%them 10
If $ k \subset L \subset K $ then
 $$
 \dim_k K = \dim_k L + \dim_L K.
 $$
 \end{thm} 
 
 A transcendental extension  $ K /k $ is  said to be \textit{purely
   transcendental} if  there exists a base $B$  with $K = k (B)$. Note
 that this does not mean that every base has this property. For
 instance  if $k(x)$ is the field of rational functions of $x$  then
 $x^2$  is also transcendental over $x$ but $k (x^2)$  is a proper
 subfield of $k(x)$. 
 
 Let $K = k (x_1, \ldots x_n)$ and $ K' = k ( x'_1, \ldots x'_n ) $
 be two purely transcendental extensions of dimension $n$. Consider
 the  homomorphism\pageoriginale $\sigma$ defined by   
$$
\sigma f (x_1 , \ldots x_n ) = f ( x'_1 , \ldots , x'_n)
$$

Where $ f ( x_1 ,\ldots , x_n )  \in k [x_1 ,\ldots , x_n ] $. It is 
then easy to see that this is an  isomorphism of $K$ on $K'$. This
proves  

\begin{thm}%them 11.
Two purely transcendental extensions of the same dimension $n$
 over $k$  $k$-isomorphic. 
\end{thm}

This theorem is true even if the  dimension is infinite. 


