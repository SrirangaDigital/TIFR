\chapter{Algebraic function fields}\label{chap3}%cha 3

\section{F.K. Schmidt's theorem}\label{c3:s1}%sec 1

Let\pageoriginale $K/k$ be an extension field and $x_1, \ldots ,
x_{n+1}$ any $n+1$ elements of $K$. Let $R=k[z_1,\ldots ,z_{n+1}]$ be
the ring of polynomials in $n+1$ variables over $k$. Let $\mathscr{Y}$
be the ideal in $R$ of polynomials $f(z_1,\ldots,z_{n+1})$ with the
property  
$$
f(x_1,\ldots,x_{n+1})=0 .
$$

Then clearly $\mathscr{Y}$ is a prime ideal of $R$. Also since $R$ is
a Noetherian ring, $\mathscr{Y}$ is finitely generated. Note that
$\mathscr{Y}=(o)$ if and only if $x_1,\ldots,\break x_{n+1}$ are
algebraically independent over $k$. $\mathscr{Y}$ is called the
{\textit ideal of the set} $x_1,\ldots,x_{n+1}$. 

We shall consider the case where the set $x_1,\ldots , x_{n+1}$ has
dimension $\underbar{n}$ over $k$, that is that
$k(x_1,\ldots,x_{n+1})$ is of transcendence degree $n$ over $k$. We
prove 

\setcounter{thm}{0}
\begin{thm}\label{c3:thm1}%the 1
$\mathscr{Y}$ is a principal ideal generated by an irreducible
  polynomial. 
\end{thm}

\begin{proof}
Without loss in generality we may assume that $x_1,\ldots,x_n$ are
algebraically independent over $k$ and that $x_{n+1}$ is algebraic
over\break
 $k(x_1,\ldots,x_n)$, so that $\mathscr{Y} \neq (o)$. 
\end{proof}

Consider the degrees of the polynomials $f(z_1,\ldots,z_{n+1})$ (in
$\mathscr{Y}$) in the variable $z_{n+1}$. These degrees have a minimum
greater than zero since $\mathscr{Y} \neq (o)$, and $x_1,\ldots , x_n$
are algebraically independent over $k$. Let $\varphi$ be a polynomial
in $\mathscr{Y}$ of smallest degree in $z_{n+1}$. Put 
$$
\varphi =A_o z^\lambda_{n+1}+A_1 z^{\lambda-1}_{n}+ \ldots+A_\lambda
$$
where\pageoriginale $A_0,A_1,\ldots,A_\lambda$ are polynomials in
$z_1,\ldots z_n$ 
with coefficients in $k$. We may assume that \; $A_0,A_1,\ldots
,A_\lambda$ \; have no common factor in \break $k[z_1,\ldots ,z_n]$. For if
$A(z_1,\ldots,z_n)$ is a common factor of $A_0,\ldots,A_\lambda$ then 
$$
\varphi(z_1,\ldots,z_{n+1})=A(z_1,\ldots,z_n) \, \varphi_1(z_1,\ldots,z_{n+1}) 
$$
and so, since $x_1,\ldots,x_n$ are algebraically independent,
$$
\varphi_1(x_1,\ldots,x_{n+1})=o
$$
and $\varphi_1$ will serve our purpose. So we can take $\varphi$ to be
a primitive polynomial in $z_{n+1}$ over $R'=k[z_1,\ldots,z_n]$. 

Clearly $\varphi$ is irreducible in $R'$. For, if
$$
\varphi=g_1(z_1,\ldots,z_{n+1}) \; g_2(z_1,\ldots,z_{n+1})
$$
then $g_i(x_1,\ldots ,x_{n+1})=0$ for $i=1$ or 2, so that either
$g_1$ or $g_2$ is in $\mathscr{Y}$. Both cannot have a term in
$z_{n+1}$ with non zero coefficient. For then the degrees in $z_{n+1}$
of $g_1$ of $g_2$ will both be less that of $\varphi$ in $z_{n+1}$
contradicting the definition of $\phi$. So one $g_1$, $g_2$ say
$g_1$ is independent of $z_{n+1}$. But this means that $\varphi$ is not
a primitive polynomial. 

Thus we have chosen in $\mathscr{Y}$  a polynomial $\varphi$ which is
irreducible, of the smallest degree in $z_{n+1}$ and primitive in
$R'[z_{n+1}]$. 

Let $\psi(z_1,\ldots,z_{n+1})$ be any other polynomial in $\mathscr{Y}$ .

Since $F=k(z_1,\ldots,z_n)$ is a field, $F[z_{n+1}]$ is a Euclidean
ring so that in $F[z_{n+1}]$ we have  
$$
\psi(z_1,\ldots, z_{n+1}) = A(z_1,\ldots,z_{n+1}) \; 
\varphi(z_1,\ldots,z_{n+1}) + L(z_1,\ldots,z_{n+1}) 
$$
where\pageoriginale $A$ and $L$ are polynomials in $z_{n+1}$over
$F$. Here either 
$L=o$ or degree of $L$ in $z_{n+1}$ is less than that of $\varphi$. If
$L \neq o$, then we may multiply both sides of the above equation by a
suitable polynomial in $z_1,\ldots,z_n$ over $k$ so that 
$$
B(z_1,\ldots,z_n) \psi =
C(z_1,\ldots,z_{n+1})\varphi(z_1,\ldots,z_{n+1})+L_1(z_1,\ldots,z_{n+1}) 
$$
$L_1$ having in $z_{n+1}$ the same degree as $L$. Since $\varphi$ and
$\psi$ are in $\mathscr{Y}$, it follows that $L_1 \in
\mathscr{Y}$. Because of degree of $L_1$, it follows that 
$$
L_1 = L=0.
$$
 
Thus
$$
B(z_1,\ldots,z_n) \psi=A(z_1,\ldots,z_{n+1})\varphi(z_1,\ldots,z_{n+1})
$$ 

Since $\varphi$ is a primitive polynomial, it follows that $\varphi$
divides $\psi$ and our theorem is proved. 

We call $\varphi$ \textit{the irreducible polynomial of $x_1,\ldots
  x_{n+1}$ over $k$}. 

Note that since $x_1,\ldots x_n$ are algebraically independent over
$k$, the polynomial 
$$
\varphi_1(z_{n+1})=\varphi(x_1,\ldots,x_n,z_{n+1})
$$
over $k(x_1,\ldots,x_n)$ is irreducible in $z_{n+1}$.

Let $x_1,\ldots,x_{n+1}$be of dimension $n$ over $k$ and $\varphi$ the
irreducible polynomial of $x_1,\ldots,x_{n+1}$ over $k$. Let $\varphi$
be a polynomial in $z_1,\ldots,z_{i+1}$ but not in
$z_{i+2},\ldots,z_{n+1}$, that is it does not involve
$z_{i+2},\ldots,z_{n+1}$ in its expression. Consider the field
$L=k(x_1,\ldots,x_{i+1})$. because $x_1,\ldots,x_{i+1}$ are
algebraically\pageoriginale dependent $(\varphi(x_1,\ldots,x_{i+1})=0)$, 
$$
\dim_k L \leq i.
$$

But $k(x_1,\ldots,x_{n+1})=L(x_{i+2},\ldots,x_{n+1})$ so that
$$
\dim_L k(x_1,\ldots,x_{n+1})\leq n-i . 
$$

Since dimensions are additive, we have
$$ 
n=\dim_k L+ \dim_L k(x_1,\ldots,x_{n+1})\leq i+n-i=n .
$$

Thus $L$ has over $k$ the dimension $i$. Since $\varphi$ is a
polynomial in $z_1,\ldots,z_{i+1}$ every one of these variables
occurring, with non zero coefficients, we get the 

\begin{coro*}
 If $x_1,\ldots,x_{n+1}$ be $n+1$ elements of $K/k$ and have
  dimension $n$, there exist among them $i+1$ elements, $i \leq n$,
  say $x_1,\ldots ,x_{i+1}$ (in some order) such that $k(x_1,\ldots
  x_{i+1})$ has dimension $i$ over $k$ and every $i$ of them are
  algebraically independent. 
\end{coro*}

Let $K/k$ be a transcendental extension with a transcendence base $B$
over $k$. Then $K/k(B)$ is algebraic. We call $K/k$ an
\textit{algebraic function field} if  
\begin{enumerate}
\renewcommand{\labelenumi}{(\theenumi)}
\item $B$ is a finite set
\item $K/k(B)$ is finite algebraic.
\end{enumerate}

Let $\dim_k K=n$. There exist $x_1,\ldots,x_n$ in $K$ which form a
transcendence base of $K/k$. If $K$ is an algebraic function field then
$K/k(x_1,\break \ldots,x_n)$ is finite algebraic. Hence
$K=k(x_1,\ldots,x_m), m \geq n$, is finitely generated. This shows that
algebraic function fields are identical with finitely generated
extensions. 

An algebraic\pageoriginale function field $K/k$ is said to be
\textit{separably 
  generated} if there exists a transcendence base \, $x_1,\ldots,x_n$ \, of
\, $K/k$ \, such that \break $K/k (x_1,\ldots,x_n)$ is a separable algebraic
extension of finite degree. \break $x_1,\ldots,x_n$ is then said to be a
\textit{separating base}. Clearly every purely transcendental
extension is separably generated. Also, if $k$ has characteristic zero
and $K$ is an algebraic function field, it is separably generated. In
this case every transcendence base is a separating base. This is no
longer true if $k$ has characteristic $p\neq o$. 

For example, let $K=k(x,y)$ be a function field of transcendence
degree one and let 
$$
x^2-y^p=o.
$$

Let $k$ have characteristic $p \neq 2$. Obviously $x$ and $y$ are both
transcendental over $k$. But $K/k(x)$ is a simple extension generated
by $y$ which is a root of  
$$
z^p-x^2
$$
over $k(x)[z]$ and since $k$ has characteristic $p$, $K/k(x)$ is purely
inseparable. On the other hand $K/k(y)$ is separable since $x$
satisfies over $k(y)$ the polynomial 
$$
x^2-y^p.
$$

Thus $y$ is separating but not $x$.

An algebraic function field which is not separably generated is said
to be \textit{inseparably generated}. This means that for every base
$B$ of $K/k$. $K/k(B)$ is inseparably algebraic. 

In algebraic\pageoriginale geometry and in algebraic function theory,
it is of 
importance to know when an algebraic function field is separably
generated. An important theorem in this regard is 
theorem~\ref{c3:thm2} due to F.K. Schmidt. We shall first prove a  

\begin{lemma*}
Let $k$ be a perfect field of characteristic $p \neq o$ and
  $K=k(x_1,\ldots,x_{n+1})$ an extension field of dimension $n$. Then
  $K$ is separably generated. 
\end{lemma*}

\begin{proof}
Let $\varphi$ be the irreducible polynomial of $x_1,\ldots,x_{n+1}$
and let it be a polynomial in $z_1,\ldots,z_{i+1}$ but in
$z_{i+2},\ldots,z_{n+1}$. Then 
$$
\varphi(x_1,\ldots,z_t,\ldots,x_{i+1})
$$
is irreducible over $k(x_1,\ldots,x_{t-1},x_{t+1},\ldots,x_{i+1})$ for
every $t,1 \leq t \leq i+1$. At least for one $t,
\varphi(z_1,\ldots,z_t,\ldots,z_{i+1})$ is a separable polynomial in
$z_t$ over $k(z_1,\ldots,z_{t-1},\ldots,z_{i+1})$. For, if it is
inseparable in every $z_t$, then 
$$
\varphi(z_1,\ldots,z_{i+1})\in k[z^p_1,\ldots,z^p_{i+1}]
$$
and since $k$ is perfect, this will mean that
$\varphi(z_1,\ldots,z_{i+1})$ is the $p^{\text{th}}$ power of a polynomial in
$k[z_1,\ldots,z_{i+1}]$ which contradicts irreducibility of
$\varphi$. So, for some $z_t$, say $z_1$, we have
$\varphi(z_1,x_2,\ldots,x_{i+1})$ is a separable polynomial. Hence
$x_1$ is separable over $k(x_2,\ldots,x_{i+1})$ and so over
$k(x_2,\break \ldots,x_{n+1})$. But $x_2,\ldots,x_{n+1}$ has dimension $n$
and our lemma is proved. 
\end{proof}

\begin{coro*}
Under the conditions of the lemma, a separating base of $n$
  elements may be chosen from among $x_1,\ldots,x_{n+1}$. 
\end{coro*}

We are\pageoriginale now ready to prove the theorem of \textit
{F.K.Schmidt}. 

\begin{thm}\label{c3:thm2}%%% 2
 Every algebraic function field $K$ over a perfect field $k$ is
 separably generated. 
\end{thm}

\begin{proof}
Obviously, the theorem is true if $k$ has characteristic zero. So let
$k$ have characteristic $p \neq o$. Let $K=k(x_1,\ldots,x_m)$ and let
$n$ be the dimension of $K/k$. Then $m \geq n$. If $m=n$ there is
nothing to prove. Let $m=n+q$. If $q=1$ then lemma~1 proves the
theorem. So let us assume theorem proved for $q-1$ instead of
$q>1$. We may assume, without loss in generality that $x_1,\ldots,x_n$
is a transcendence base of $K/k$. Consider the fields
$L=k(x_1,\ldots,x_n,x_{n+1})$. It satisfies the conditions of the
lemma. Hence there exist $n$ elements among $x_1,\ldots,x_{n+1}$ say
$x_1,\ldots,x_{t-1},x_{t+1},\ldots x_{n+1}$ which from a separating
base of $L/k$. Thus $x_t$ is separable over
$k(x_1,\ldots,x_{t-1},\ldots,x_{n+1})$ and hence over 
$$
M=k(x_1,\ldots,x_{t-1},x_{t+1},x_{t+2},\ldots,x_m)
$$  

$M/k$ now satisfies the induction hypothesis and so is separably
generated. Since $K/M$ is separable, it follows that $K$ is
separably generated. 
\end{proof}

We could prove even more if we assume as induction hypothesis the fact
that among $x_1,\ldots,x_m$ there exists a separating base. 


\section{Derivations}\label{c3:s2}%sec 2

Let $R$ be a commutative ring with unit element $\underbar{e}$. A
mapping $D$ of $R$ into itself is said to be a \textit{derivation}
of $R$ if 
\begin{enumerate}
\renewcommand{\labelenumi}{(\theenumi)}
\item $D(a+b)=  Da+Db$\pageoriginale
\item $D(ab) = aDb+bDa$
\end{enumerate}
for $a,b \in R$. It is said to be a derivation \textit{over} a
subring $R'$ if for every $a \in R'$, $Da=o$. It then follows that for
$a \in R'$ and $x \in R$ 
$$
Dax = aDx
$$

The set $R_o$ of $ a \in R$ with $Da=o$ is a subring of $R$ and
contains $\underbar{e}$. For, 
$$
De=De^2=De. \quad  e+e. \quad De=2 De \,;
$$
so $De=o$. If $Da=o$, $Db=o$, then
\begin{align*}
D(a+b)= & Da+Db=o \\
D(ab)= a & Db+bDa=o 
\end{align*}

Thus $R_o$ is a subring. We call $R_o$ the \textit{ring of constants
  of the derivation} $D$. 

If $R$ is a field, then $R_o$ also is a field. For, if $x \in R_o$ and
$x \neq o$, then 
$$
o=De=Dx.x^{-1}=Dx.x^{-1}+x.Dx^{-1}
$$
Thus
$$
Dx^{-1}=o
$$
so that $x^{-1}\in R_o$.

$D$ is said to be a \textit{non-trivial} derivation of $R$ if there is
an $x \in R$ with $Dx \neq o$. It follows from above that 

\begin{thm}\label{c3:thm3}%the 3
A prime field has no non-trivial derivations.
\end{thm}

A Derivation $\bar{D}$ of $R$ is said to be an \textit{extension} of a
derivation $D$ of a subring $R'$ of $R$ if $\bar{D}a=Da$ for $a \in R'$. We
now\pageoriginale prove the   

\begin{thm}\label{c3:thm4}%the 4
 If $K$ is the quotient field of an integrity domain $R$, then a
 derivation $D$ of $R$ can be uniquely extended to $K$. 
\end{thm}

\begin{proof}
Every element $c$ in $K$ can be expressed in the form
$c=\dfrac{a}{b}$, $a$, $b \in R$. If an extension $\bar{D}$ of $D$ exists,
then 
$$
\bar{D}a =Da
$$

But $a=bc$ so that
$$
Da=\bar{D}a=\bar{D}bc=b\bar{D}c+c Db
$$

Therefore
$$
\bar{D}c=\frac{Da-c Db}{b}=\frac{b Da-a Db}{b^2}
$$

If $c$ is expressed in the form $\dfrac{a'}{b'}$, $a'$, $b' \in R$ then
$ab'=ba'$and so 
$$
Da.b'+a.Db'=Db.a'+b Da'
$$
or that
$$
\frac{Da'-c Db'}{b'}= \frac{Da-c Db}{b}
$$
which proves that $\bar{D}c$ does not depend on the way $c$ is expressed as
the ratio of two elements from $R$. We have therefore only to prove
the existence of $\bar{D}$. In order to prove this, put for $c=\dfrac{a}{b}$ 
$$
\bar{D}c =\frac{b Da-a Db}{b^2};
$$
we should verify that it is a derivation, is independent of the way
$c$ is expressed as ratio of elements in $R$ and that it coincides
with $D$ on $R$. These are very simple. 
\end{proof}

Let\pageoriginale $D_1$, $D_2$ be two deviations of $R$. Define
$D=D_1+D_2$ by $Da=D_1 a+D_2 a$ for $a \in R$. Then it is easy to
verify that $D$ is a derivation of $R$. Furthermore if $a \in R$
define $aD$ by   
$$
(aD)x =a.Dx
$$

By this means, the derivations of $R$ from an $R$ module, Suppose
$D_1,\ldots,D_r$ from a basis of the module of derivations of $R$. 

Then every derivation $D$ of $R$ is of the form
$$
D=\sum_i a_i D_i,\quad a_i \in R.
$$

Let $R$ be an integrity domain and $\bar{D}_1,\ldots,\bar{D}_r$ the
unique extensions of $D_1,\ldots,D_r $ respectively to $K$, the
quotient, field of $R$. Then $\bar{D}_1,\ldots,\bar{D}_r$ are linearly
independent over $K$. For, if 
$$
\sum_i 1_i \bar{D}_i = o ,\qquad 1_i\in K
$$
then write $1_i = \dfrac{a_i}{b_i}$, $a_i$, $b_i \in R$. We get
$$
\sum_i \frac{ai}{bi}\bar{D}_i = o.
$$

Multiplying throughout by $b_1,\ldots ,b_r$ which is not zero, we get
\break $(\sum \limits_i \lambda \bar{D}_i)t=o$ for every $t \in R$. Therefore
$\sum \limits_i \lambda_i D_i=o$ which implies that $\lambda_i=o$ or
$a_i=o$. 

Also, since every derivation $D$ of $K$ is an extension of a
derivation of $R$, it follows that the derivations of $K$ from an
$r$-dimensional vector space over $K$. 

Let us now consider the case where $R=k[x_1,\ldots,x_n]$ is the ring
of polynomials in $n$ variables $x_1,\ldots,x_n$. The $n$ mappings  
$$
D_i: \qquad a \to \frac{\partial a}{\partial x_i}, i=1,\ldots,n 
$$\pageoriginale
are clearly derivations of $R$ over $k$. They form a base of
derivations of $R$ which are trivial on $k$. For, if  
$$
\sum_i a_i D_i =o, \qquad a_i \in R
$$
then, since $D_i(x_j)=\delta_{ij}$, we get
$$
o =(\sum_i a_i D_i) x_j =a_j,j=1,\ldots,n.
$$

Also, if $D$ is any derivation of $R$ which is trivial on $k$, then
let $Dx_i=a_i$. Put 
$$
\bar{D}=D-\sum_i a_i D_i.
$$

Then
 $$
 \bar{D}x_j=Dx_j -(\sum_i a_i D_i)x_j=o
 $$
 which shows that since $x_1,\ldots,x_n$ generate $R,\bar{D}=o$. 
 
 Suppose $D$ is any derivation of $k$ and let $\bar{D}$ be an
 extension of $D$ to $R$. Let $\bar{D}x_i=a_i$. Let $D_o$ be another
 extension of $D$ which has the property $D_ox_i=a_i, i=1,\ldots,n$. 
 
 Then $D-D_o$ is a derivation of $R$ which is trivial on $k$. But since
 $$
 (D-D_o)x_i=o,\quad i=1,\ldots,n
 $$
 it follows that $D=D_o$. This gives us the
 
\begin{thm}\label{c3:thm5}%the 5
 The derivations of $K=k(x_1,\ldots,x_n)$, the field of rational
  functions of $n$ variables over $k$ which are trivial on $k$ from a
  vector space of dimension $n$ over $K$. A basis of this space of
  derivations is given by the $n$ partial derivations 
$$
D_i=\frac{\partial}{\partial x_i}, \quad i=1,\ldots,n.
$$
 Furthermore\pageoriginale if $D$ is any derivation of $k$, there
 exists only   one extension $\bar{D}$ of $D$ to $K$ for which  
$$
\bar{D}x_i = a_i,\quad i=1,\ldots,n
$$
 where $a_1,\ldots,a_n$ are any $n$ quantities of $K$ arbitrarily given.
 \end{thm} 
 
 We now consider derivations of algebraic function fields.
 
 Let $K=k(x_1,\ldots,x_m)$ be a finitely generated extension of
 $k$. Put $T=k[x_1,\ldots,x_m]$. In order to determine all the
 derivations of $K$, it is enough to determine the derivations of $T$
 since $K$ is the quotient field of $T$. Let $D$ be a derivation of
 $k$; we wish to find extensions $\bar{D}$ of $D$ to $K$. 
 
 Let $R$ denote the ring of polynomials $k[z_1,\ldots,z_m]$ in $m$
 independent variables. For any polynomial $f(x_1,\ldots,x_m)$ in $T$,
 denote by $\dfrac{\partial f}{\partial x_i}$ the polynomial obtained
 by substituting $z_i=x_i,i=1,\ldots, m$ in $\dfrac{\partial
   \bar{f}}{\partial z_i}$ where $\bar{f}=f(z_1,\ldots,z_m)$ is in
 $R$. 
 
 If
 $$
 f=\sum_\lambda a_{\lambda_1},\ldots,\lambda_m  x^{\lambda_1}_1\cdots 
 x^{\lambda_m}_m 
 $$
 $a \lambda_1,\ldots,\lambda_m \in k$, put 
 $$
 f^D =\sum_\lambda(D a_{\lambda_1},\ldots,\lambda_m)x^{\lambda_1}_1  
 \cdot x^{\lambda_m}_m. 
 $$
 
 Obviously $f^D$ is a polynomial in $T$. If $\bar{D}$ is an extension
 of the derivation $D$, then clearly 
 $$
 \bar{D}f = f^D+\sum^{m}_{i=1}\frac{\partial f}{\partial 
   x_i}\bar{D}x_i 
 $$
for any\pageoriginale $f \in T$. Also $\bar{D}$ is determined uniquely
by its values on $x_1, \ldots , x_m$ which generate $T$. Now
$\bar{D}x_i$ cannot be arbitrary elements of $T$. For, let
$\mathscr{Y}$ be the ideal in $R$ of the set $x_1, \ldots , x_m$.   

Then for $f(z_1, \ldots,z_m)$ in $\mathscr{Y}$,
$$
f(x_1, \ldots,x_m)=o.
$$

Therefore, since $\bar{D} o =0$, the $\bar{D} x_i$ would have to
satisfy the infinity of equations  
$$
0=f^D+ \sum^m_{i=1} \frac{\partial f}{\partial x_i}\bar{D} x_i
$$
for every $f$ in $\mathscr{Y}$.

Conversely, suppose $u_1, \ldots u_m$ are $m$ elements in $T$
satisfying  
$$
f^D+ \sum^m_{i=1} \frac{\partial f}{\partial x_i} u_i = 0
$$
for every $f$ in $\mathscr{Y}$. For any $\varphi$ in $T$ define
$\bar{D}$ by 
$$
\bar{D}\varphi = \varphi^D + f^D+ \sum^m_{i=1} \frac{\partial
  \varphi}{\partial x_i}\bar{D} x_i 
$$
where $\bar{D} x_i = u_i$. Then clearly $\bar{D}$ is a derivation of
$T$ and it coincides with $D$ on $k$. Furthermore $\bar{D}$ does not
depend on the way $\varphi$ is expressed as a polynomial in $x_1,
\ldots, x_m$. For, if $\varphi=a(x_1, \ldots, x_m) = b(x_1, \ldots,
x_m)$ then, since $a-b \in \mathscr{Y}$ have 
$$
a^D - b^D + \sum_i \big(\frac{\partial a}{\partial x_i}u_i - 
\frac{\partial b}{\partial x_i} u_i\bigg) = o 
$$
which proves our contention. Hence  

\begin{thm}\label{c3:thm6}%the 6
Let\pageoriginale $K=k(x_1, \ldots,x_m)$ and $D$ a derivation of $k$. Let
  $\mathscr{Y}$ be ideal in $k[z_1, \ldots,z_m]$ of the set $x_1,
  \ldots,x_m$. Let $u_1, \ldots,u_m$ be any elements of $K$. There
  exists a derivation $\bar{D}$ and only one satisfying   
$$
\bar{D}x_i = u_i,  i= 1, \ldots, m 
$$
and extending the derivation $D$ in $k$, if and only if, for every $f
\in \mathscr{Y}$ 
$$
f^D + \sum^m_{i=1} \frac{\partial f}{\partial x_i}u_i =  o.
$$
and then for every $\varphi$ in $K$,
$$
\bar{D} \varphi = \varphi^D + \sum^m_{i=1} \frac{\partial
  \varphi}{\partial x_i}u_i. 
$$
\end{thm} 

The infinite number of conditions above can be reduced to a finite
number in the following manner. Since $R=k[z_1, \ldots , z_m]$
is a noetherian ring, the ideal $\mathscr{Y}$ has a finite set $f_1,
\ldots , f_s$ of generators so that $f \in \mathscr{Y}$ may be written  
$$
f = \sum^s_{i=1} A_i f_i, \qquad A_i \in R
$$

Suppose $f_1, \ldots , f_s$ satisfy the above conditions, then since  
\begin{align*}
f^D(x) & = \sum_i A^D_i f_i + \sum_i f^D_i A_i \\
\frac{\partial f}{\partial x_i} & = \sum_j A_j \frac{\partial
  f_j}{\partial x_i}+ \sum_j f_j \frac{\partial A_j}{\partial x_i}, 
\end{align*}
we get 
$$
f^D(x) + \sum^m_{i=1} \frac{\partial f}{\partial x_i}u_i = o. 
$$

We may therefore replace the above by the finitely many conditions  
$$
f^D_i + \sum^m_{j=1} \frac{\partial f}{\partial x_j} u_j= o , i= 1,
\ldots,s. 
$$

We now\pageoriginale consider a few special cases.

Let $K=k(x)$ be a simple extension of $k$. Let $D$ be a derivation of
$k$. 	We will study extensions $\bar{D}$ of $D$ into $K$. 
\begin{enumerate}
\renewcommand{\labelenumi}{(\theenumi)}
\item First let $x$ be transcendental over $k$. The ideal of $x$ in $k
  [z]$ is zero. This means that we can prescribe $\bar{D}x$
  arbitrarily. Thus for every $u \in K$ there exists one and only
  extension $\bar{D}$ with  
$$
\bar{D}x = u
$$

\item Let now $x$ be algebraic over $k$. Suppose $x$ is inseparable over
  $k$. Let $f(z)$ be the minimum polynomial of $x$ in $k[z]$. Then
  $f(z)$ generates the ideal of $x$ in $k[z]$. But $x$ being
  inseparable, $f'(x)=o$. This means that $D$ has to satisfy  
$$
f^D=o.
$$

Also, $\bar{D}$ is uniquely fixed as soon as we assign a value $u$ to
$\bar{D}x$. This can be done arbitrarily as can be easily seen. Thus
there exist an infinity of extensions $\bar{D}$. 

\item Finally, let $X$ be separable. Then $f(z)$, the irreducible
  polynomial of $x$ over $k$ is such that  
$$
f'(x) \neq o.
$$
\end{enumerate}

Since $f(z)$ generates $\mathscr{Y}$ we must have 
$$
f^D(x)+f'(x) \bar{D} x = o,
$$
or that $\bar{D}x$ is uniquely fixed by $D$. 
\begin{equation*}
-\bar{D}x = \frac{f^D(x)}{f'(x)} \tag{*}
\end{equation*}

There is thus only one extension of $D$ to $K$ and it is given by $(*)$.

In\pageoriginale particular, $K$ has no derivations, except the
trivial one, over $k$. 

We shall now prove 
 
\begin{thm}\label{c3:thm7}%the 7
In order that a finitely generated extension $K =k (x_1 ,\break \ldots,
  x_n)$ be separably algebraic over $k$, it is necessary and
  sufficient that $K$ have no non-trivial derivations over~$k$. 
\end{thm}

\begin{proof}
If $K/k$ is algebraically separable, then since $K$ is finitely
generated over $k$, it follows that $K= k(x)$ for some $x$ and the
last of the considerations above shows that $K$ has no nontrivial
derivations over~$k$. 
\end{proof}

Suppose now $K/k$ has non-trivial derivations. In case $n=1$, our
considerations above show that $K/k$ is separable. Let now $n>1$ and
assume that theorem is proved for $n-1$ instead of $n$. 

Put
$$
K=K_1(x_n) \qquad K_1 = k(x_1, \ldots, x_{n-1}). 
$$

Then $x_n$ is separably algebraic over $K_1$. For, if not, let $x_1$ be
inseparable over  $K_1$ or transcendental over $K_1$. In both cases
the zero derivation in $K_1$ can be extended into a non-trivial
derivation of $K$ contradictions hypothesis over $K$. 

Thus $x_n$ is separable over $K_1$. This implies, since $K$ has no
derivations over $k$ that $K_1$ and our theorem is proved. 

Note that in the theorem above, the fact that $K/k$ is finitely
generated is essential. For instance, if $k$ is an imperfect field and
$K=k^{p^{-\infty}}$ then $K/k$ is infinite. Also if $a \in k$ then
\pageoriginale $a=b^p$ for some $b \in K$. If $D$ is a derivation of $K$, then  
$$
D a = Db^p = p b^{p-1}Db = o
$$

This proves, in particular, that a perfect field of characteristic $p
\neq o$, has only the trivial derivation. 

If we take $K$ to be the algebraic closure of the rational number
filed then $K$ has only the trivial derivation. 

Let $K=k(x,y)$ be an algebraic function field of one variable. Let us
assume that $x$ is a separating variable and  $\varphi(X,Y)$ the
irreducible polynomial of $x$, $y$ over $k$. Then if $D$ is a derivation
of $K$ over $k$, 
$$
\frac{\partial \varphi}{\partial x} Dx + \frac{\partial
  \varphi}{\partial y} Dy = o 
$$
so that if we assume that $y$ is separable over $k(x)$, then
$\dfrac{\partial \varphi}{\partial y} \neq o$ and hence  
$$
Dy=\frac{-{\frac{\partial \varphi}{\partial x}}}{\frac{\partial
    \varphi}{\partial y}} Dx 
$$

This shows that the ratio $Dy/Dx$ is independent of $D$.

Also, for any rational function $\psi(x,y)$ of $x,y$
$$
D \psi = \frac{\partial \psi}{\partial x} Dx + \frac{\partial
  \psi}{\partial y} Dy 
$$
which gives,if $Dx \neq o$
$$
\frac{D \psi}{D x}= \frac{\partial \psi}{\partial x}-\frac{\partial
  \psi}{\partial y} \Bigg(\frac{{\frac{\partial \varphi}{\partial
      x}}}{\frac{\partial \varphi}{\partial y}} \Bigg) 
$$
which is a well known formula in elementary calculus.

We shall now obtain a generalisation of theorem~\ref{c3:thm7} to algebraic
function fields. 

Let $K= k(x_1, \ldots, x_m)$ be an algebraic function field of
dimension $n$, so that $o \leq n \leq m$. Let $f_1, \ldots, f_s$ be a
system\pageoriginale of generators of the ideal $\mathscr{Y}$ of
polynomials $f(z_1, 
\ldots, z_m)$ in $k [x_1, \ldots, x_m]$ which vanish for $x_1, \ldots,
x_m$. Let $\dfrac{\partial
  f}{\partial x_i}$ for $f$ in $k [z_1, \ldots, z_m]$ have the same
meaning as before. 

Denote by $M$ the matrix
\begin{align*}
M = \Bigg(\frac{\partial f}{\partial x_i} \Bigg)  \qquad &i = 1,
\ldots ,m\\ 
\qquad &j=1, \ldots,s
\end{align*}
where $i$ is the row index and $j$ the column index. We denote by
$\underbar{t}$ the rank of the matrix $M$ which is a matrix over $K$. 

Let $V_K(D)$ denote the vector space of derivations of $K$ which are
trivial on $k$. This is a vector space over $K$. Denote by $l$ the
dimension of  $V_K(D)$ over $K$. We then have  

\begin{thm}\label{c3:thm8}%the 8
\fbox{ l+ t =m.}
\end{thm}

\begin{proof}
For any integer $p$, denote by $W_p$ the vector space over $K$ of
dimension $p$, generated by $p$-tuples $(\beta_1, \ldots, \beta_p),
\beta_i \in K$. 
\end{proof}

Let $\sigma$ denote the mapping 
$$
\sigma D=(Dx_1, \ldots , Dx_m)
$$
of $V_k(D)$ into $W_m$. This is clearly a homomorphism of $V_k(D)$
into $W_m$. The kernel of the homomorphism is the set of $D$ for which
$Dx_i =o; i = 1, \ldots,m$. But since $K$ is generated by $x_1,
\ldots, x_m$, this implies that $D=o$. Thus $V_k(D)$ is isomorphic to
the subspace of $W_m$ formed the vectors  
$$
(Dx_1, \ldots , Dx_m).
$$

Consider now the vector space $W_s$ and let $\tau$ be the mapping \break
$(\tau(\alpha_1, \ldots, \alpha_m) = ( \alpha_1, \ldots, \alpha_m)M$
of $W_m$\pageoriginale into $W_s$. Put  
$$
\beta_j= \sum^m_{j=1} \alpha_j \frac{\partial f_i}{\partial x_j}; i=
1, \ldots,s 
$$
so that 
$$
(\beta_1, \ldots, \beta_s) =(\alpha_1, \ldots, \alpha_m)M.
$$
 
The rank of the mapping $\tau$ is clearly $y$ equal to the rank $t$ of
the matrix $M$. It is the dimension of the image by $\tau$ of $W_m$
into $W_s$. The kernel of the mapping $\tau$ is the set of $(\alpha_1,
\ldots, \alpha_m)$with  
$$
\beta_i = o; i= 1, \ldots ,s.
$$
which, by theorem~\ref{c3:thm6} is clearly isomorphic to the
subspace of $W_m$ formed by vectors $(Dx_1, \ldots,Dx_m)$, $D \in
V_K(D)$. This, by previous considerations, proves the theorem.  

We shall now prove the 

\begin{thm}\label{c3:thm9}%the 9
 With the same notations as before, there exist $\ell$ elements, say
  $x_1, \ldots, x_1$ of dimension $n$ over $k$ such that $K/k(x_1,
  \ldots, x_1)$ is a separably algebraic extension. 
\end{thm}

\begin{proof}
Since the matrix $M$ has rank $t$, there exists a submatrix of $M$ of
$t$ rows and which is non-singular. Choose notation in such a way,
that this matrix is  
$$
P=\left(\frac{\partial f_j}{\partial x_i} \right),
\qquad 
\begin{matrix}
i=m-t+1, \ldots,m\\[5pt]
j=s-t+1, \ldots,s
\end{matrix}
$$

Note that \, $t \leq $ \, Min \, $(s,m)$. Let \, $L= k(x_1, \ldots,
x_1)$. \, Then\\
 $K= L(x_{1+p}, \ldots, x_m)$. Let $D$ be a derivation of $K$ over
$L$. Then since $f_j(x_1, \ldots, x_m) = o$, we must have, by
theorem~\ref{c3:thm6} 
$$
\sum^{m}_{i=1} \frac{\partial f_j}{\partial x_i} Dx_i =o.
$$

But\pageoriginale since $D$ is zero on $L$,
$$
Dx_i=o, \;\; i=1, \ldots, l.
$$

Thus
$$
\sum^{m}_{i=l+1} \frac{\partial f_j}{\partial x_i} Dx_i =o. 
$$

This means that 
\begin{equation*}
P
\begin{pmatrix}
Dx_{l+1}\\
\vdots\\
Dx_m
\end{pmatrix}
=
\begin{pmatrix}
o\\
\vdots\\
o
\end{pmatrix}
\end{equation*}

But since $|P| \neq o$, it follows that $Dx_{l+1} =o \ldots, Dx_m=o$
which shows that $D=o$ 

But $K=L(x_{l+1}, \ldots, x_m)$ is finitely generated over $L$. Using
theorem~\ref{c3:thm7}, it follows that $K/L$ is algebraic and
separable. We get incidentally  
$$
n \leq 1.
$$

We now prove the important 
\end{proof}

\begin{thm}\label{c3:thm10}%the 10 
 Let $K=k(x_1, \ldots, x_m)$ be of dimension $n$. Then $K$ is
  separably generated over $k$, if and only if $\dim V_K(D)=n$. In
  that case there exists, among $x_1, \ldots, x_m$, a separating base
  of $n$ elements. 
\end{thm}

\begin{proof}
If $V_K(D)$ has dimension $n$ then theorem~\ref{c3:thm9} shows that
there exist $n$ elements $x_1, \ldots, x_n$ among $x_1, \ldots, x_m$
such $K/k(x_1, \ldots, x_n)$ is separably algebraic.  
\end{proof}

Suppose now that $K/k$ is separably generated. Let $y_1, \ldots,y_n$
be a separating base so that $K/k(y_1, \ldots, y_n)$ is separably
algebraic. $k(y_1,\break \ldots, y_n)$ has $n$ linearly independent
derivations $D_1,\ldots,D_n$ over $k$ defined by 
\begin{equation*}
D_iy_j=
\begin{cases}
o, \text{ if } i \neq j\\
1, \text{ if } i = j
\end{cases}
\end{equation*}\pageoriginale 

Since $K/k(y_1, \ldots, y_n)$ is separably algebraic and finite, it
follows that each of $D_1,\ldots,D_n$ has a unique extension
$\bar{D}_1,\ldots,\bar{D}_n$ to $K$. Now
$\bar{D}_1,\ldots,\bar{D}_n$ are linearly independent over
$K$. For, if  
$$
\sum_i a_i \bar{D}_i = o , a_i \in K,
$$
then 
$$
\sum_i a_i \bar{D}_i(y_j) = o \text{ for all } j.
$$

Hence $a_j =o$ for $j=1, \ldots,n$. Let now $D$ be a derivation of
$K/k$ and let $D y_i = a_i$. Put  
$$
\bar{D} = D-\sum_i a_i \bar{D}_i.
$$

Then $\bar{D} y_i = o$ for all $i$. Therefore since $K/k(y_1, \ldots,
y_n)$ is separable, $\bar{D}=o$.This proves that  
$$
\dim V_K(D) =n.
$$
and our theorem is completely established. 

Let $K=k(x_1, \ldots, x_n,y)$ where $k$ is of characteristic $p \neq
o$ and $k$ is an imperfect field. Let $y$ be algebraic over $k$ and be
a root of  
$$
z^p-t
$$
$t \in k$. Then $K/k$ is an inseparably generated extension and  
$$
\dim V_K(D) = n+1
$$
where $n$ is the dimension of $K/k$.

\section{Rational function fields}\label{c3:s3}\pageoriginale %sec 3

Let us now consider the field $K=k(x)$ of rational functions of one
variable. Let $y$ be any element of $K$. Hence $y=\dfrac{f(x)}{g(x)}$
where $f$ and $g$ are polynomials in $x$ over $k$. Also $K$ is the
quotient field of the ring $L=k[x]$ of polynomials in $x$. 

Assume that $(f(x)$, $g(x))=1$, that is that they have no factor in
common. Let $n$ be defined by  
$$
n=\max(\deg f(x), \deg g(x)).
$$

If $n=o$, then clearly $f \in k$, $g \in k$ and so $y \in k$. Let us
assume that $n \neq o$ so that at least at least one of $f$ and $g$ is
a non-constant polynomial. $n$ is called the \text{degree} of $y$. 

Let $F=k(y)$ be the field generated over $k$ by $y$. 

Then $x$ satisfies over $F$ the polynomial
$$
\varphi(z) = f(z) -yg(z) .
$$

$\varphi(z)$ is not a constant polynomial over $k(y)$. For, let 
\begin{equation*}
\left.
\begin{aligned}
f(z)&= \sum^1_{i=o} a_i z^i\\
g(z)& = \sum^m_{i=o} b_i z^i
\end{aligned}
\right\}
a_i,b_i \in k .
\end{equation*}

Then $n= \max (1,m)$. The coefficient of $z^n$ in $\varphi(z)$ is 
\begin{equation*}
\begin{cases}
a_1 & \text{ if} \quad 1 >m\\
a_1 -yb_1 & \text{ if}\quad  1=m\\
-yb_m & \text{ if} \quad 1<m
\end{cases}
\end{equation*}

In every case, it follows that since $y$ is not in $k, \varphi(z)$ is
a non-constant polynomial. Since $\varphi(z)$ has degree $n$ in $z$,
it follows that  
$$
(K:F) \leq n.
$$

We\pageoriginale assert that $\varphi(z)$ is irreducible over
$F$. For, if it is 
reducible over $F[z]$, then since $F=k(y)$, it will be reducible over
$k[y,z]$. So let $\varphi(z) = \psi_1(y,z)$  $\psi_2(y,z)$ in
$k[y,z]$. Since $\varphi(z)$ is linear in $y$ it follows that  one of
$\psi_1$or $\psi_2$ has to be independent of $y$. But then since
$(f(z)$, $g(z))=1$, $\varphi(z)$ is a primitive polynomial in $y$ over
$k[z]$. Therefore $ \varphi(z)$ is irreducible. This means that  
$$
(K:F)=n
$$

It proves, in particular that $y$ is transcendental over $k$. Hence  

\begin{thm}\label{c3:thm11}%the 11
$k$ is algebraically closed in $k(x)$.
\end{thm} 

We can extend it to the case where $K=k(x_1, \ldots,x_n)$ is a purely
transcendental extension of dimension $\underbar{n}$. We use induction
on $\underbar{n}$. Theorem~\ref{c3:thm11} proves that $n=1$, $k$ is
algebraically 
closed in $k(x)$. Let, for $n-1$ instead of $n$, instead of $n$, it be
proved that $k$ is algebraically closed in the purely transcendental
extension $k(x_1, \ldots,x_{n-1})$. Let $K=k(x_1, \ldots,x_n)$ be of
dimension $\underbar{n}$. 

Let $z$ in $K$ be algebraic over $k$. Then $z \in K =
K_1(x_n)$, $K_1=k(x_1, \ldots,x_{n-1})$. Therefore by 
theorem~\ref{c3:thm11} since 
$z$ is algebraic over $K_1$, $z \in K_1$. By induction hypothesis, $z
\in k$.  

Thus 

\begin{coro*}
 If $K=k(x_1, \ldots,x_n)$ has dimension $n$ over $k$, then $k$ is
  algebraically closed in $K$ 
\end{coro*}

It is easy to extend this to the case where $K$ is a purely
transcendental extension of any transcendence degree. 

Since\pageoriginale $ K = k (x) $, we call $x$ a \textit{generator} of
$K$ over $k$. Let $y$ also be a generator so that $ K = k (y) $. Then   
$$
(k (x) : k (y) ) =1
$$
which shows by our considerations leading to theorem~\ref{c3:thm11} 
that  
$$
y = \frac{a(x)}{b(x)}
$$
where $a(x)$ and $b(x)$ are coprime and have at most the degree
1 in $x$. Thus 
$$
y = \frac{\lambda x+ \mu}{\nu x + \rho}
$$
where $ \lambda$, $\mu$, $\nu$, $\rho$ are in $k$ and since $y$ is
transcendental over $k$, 
$$
\lambda \rho - \mu \nu \neq 0 .
$$

An automorphism of $K$  which is identity on  $k$, is uniquely fixed
by its  effect on $x$. If it takes $x$  into $y$ then $x$ and $y$ are
related as above. If $x$ and $y$  are related as above, then the
mapping which assigns to $x$ the element $y$ is an automorphism.  

If we consider the group of two rowed non-singular matrices, with
elements in $k$, then each matrix gives rise to an automorphism of $
K/k $. Obviously two matrices $ \sigma $ and $ \tau $ give rise to
the same automorphism if and only if $ \sigma = \lambda \tau $ for
some $ \lambda \neq o $ in $k$. Hence 

\begin{thm}\label{c3:thm12}%\theorem.12
 The group of automorphisms of $K = k (x) $ over $ k $ is
  isomorphic to  the factor group of the group of two  rowed matrices
  over $k$ modulo the group of matrices  $ \lambda E$, $\lambda \neq 0
  \in k $ and $E$ is  the unit matrix of order 2. 
\end{thm}

We shall call this group $ P_2 $.

From\pageoriginale theorem~\ref{c3:thm11}, it follows that if $L$ is an
intermediary field between $K$ and $k$ then $L$ is transcendental over
$k$. But much more is true as in shown by the following theorem of
Luroth.  

\begin{thm}\label{c3:thm13}%thm 13
If $ K = k (x) $ is a simple transcendental extension of $k$  and
  $k \subset L \subset K $, $ L= K ( \omega ) $ for some $ \omega \in
  K $.  
\end{thm}

\begin{proof}
We shall assume that $ L \neq k $  so that $L$ is transcendental over
$k$ and contains an element $t$, transcendental over $k$ and by
considerations leading to theorem~\ref{c3:thm11}, we have $ K/k (t)
$ is finite algebraic. Since $ L \supset k (t) $, it follows that   
$$
( K : L ) < \infty 
$$

Let $x$ satisfy over $L$  the irreducible polynomial 
$$
f (z) = z^n + a_1 z^{n-1} + \cdots + a_n 
$$
where $ a_1 , \ldots , a_n \in L $ and $ n = ( K:L ) $. At least one $
a_i $ is not in $k$ since $x$ is transcendental over $k$. The $a_i's$
are rational functions of $x$. So we may write  


$ b_0 (x)  f(z)   =  f ( x,z )  = b_0 (x) z^n + b_1 (x) z^{n-1}+
\cdots + b_n (x) $ where \break $ b_0 (x), \ldots , b_n (x) $ are polynomials
in $x$ and $ f ( x, z ) $ is  a primitive polynomial in $z$ over $ k [
  x ] $. Let $ m $ be the maximum of the degrees of $ b_0 (x),\break \ldots
b_n (x) $. Let $a_i$ be not in $k$. Then 
$$
a_i = \frac{b_i(x)}{b_0(x)}
$$
so that degree $ a_i \leq m $. Since $ a_i \in L $ and $ L \supset k
(a_i) $, it follows that 
$$
n \leq m
$$

Let us write $w$ instead of $ a_i $. Then $ w = \dfrac{ b_i (x)}{b_0
  (x)} $. 
\end{proof}

Let\pageoriginale $w=h(x)/g(x)$ where $(h(x),g(x))=1$. Then $L \supset
k(w)$.  

Also $x$ satisfies over $k(w)$ the polynomial
$$
h(z) - w g(z)
$$
so that since $f(z)$ is irreducible, it follows that $f(z)$ divides
$h(z)- wg(z)$, which is a polynomial of degree $\leq m$ in $z$. Let us
therefore write  
$$
h(z)-wg(z)= \frac{f(x,z)}{b_o(x)} \frac{\varphi(x,z)c_1(x)}{c_o(x)} 
$$
where $\varphi(x,z)$ is a primitive polynomial in $z$ over $k[x]$. We
therefore get on substituting the $w= \dfrac{h(x)}{g(x)}$, 
$$
h(z)g(x)-g(z)h(x)= \frac{g(x)c_1(x)}{b_o(x)c_o(x)}. f(x,z)
\varphi(x,z). 
$$

The left hand side being a polynomial in $x$ and $z$, $f$ and $\varphi$
being primitive polynomials in $z$ over $k[x]$, it follows that  
$$
h(z)g(x)-g(z)h(x)= f(x,z) \varphi_1(x,z)
$$
where $\varphi_1(x,z) \in k[x,z]$. We now compare degrees in $x$ and
$z$ on both sides of the above identity. On the right hand side the
degree in $x$ is $\geq m$ since one of $b_o(x), \ldots,b_n(x)$ has
degree $m$. Therefore the left side has degree in $x \geq m$. But the
degree in $x$ equals degree of $w \leq m$. Thus degree of $w
=m$. Since the left side is symmetrical in $z$ and $x$,it follows that
it has degree $m$ in $z$. Therefore $\varphi_1$ has to be independent
of $x$. 

Hence $h(z) - w g(z) =f(z) \varphi(z)$, $\varphi(z)$ being independent
of $x$. This can happen only if $\varphi(z)$ is a constant. This
proves that  
$$
n=m.
$$

Now\pageoriginale $(K:k(w))=m=(K:L)$ and $L \supset k(w)$. Thus 
$$
L = k(w)
$$
and our theorem is proved.

The analogue of Luroth's theorem for $K=k(x_1, \ldots, x_n)$ is not
known for $n >1$. 

Let $K=k(x)$ and let $G$ be a finite granite group of automorphism of
$K/k$. If $L$ is the fixed field of $G$, then $K/L$ is a finite
extension of degree equal to order of $G$. By Luroth's theorem $L=k(y)$
for some $y$. Thus 
$$
\text{ degree } y = \text{ order of } G.
$$

For instance, let $G$ be the finite group of automorphisms of $K=
k(x)$ defined by  
$$
x \to x, x\to 1-x,x \to \frac{1}{x}, x \to 1-\frac{1}{x},x \to
\frac{1}{1-x},x\to \frac{x}{x-1} 
$$

This is a group of order 6 and the fixed field will be $k(y)$ where
$k(y)$ consists of all rational functions $f(x)$ of $x$ with  
$$
f(x)= f(1-x)= f(\frac{1}{x}) = f(1-\frac{1}{x}) = f(\frac{1}{1-x}) =
f(\frac{x}{1-x}). 
$$

We have only to find a rational function of degree 6 which satisfies
the above conditions. The function  
$$
f(x)= \frac{(x^2-x+1)^3}{x^2(x-1)^2}
$$
satisfies the above conditions and so $y=f(x)$.

\begin{thm}\label{c3:thm14}%the 14
If $G$ is any finite subgroup of $P_2$ of linear transformations 
\end{thm}
$$
x \to \frac{ax+b}{cx+d}
$$
$a$, $b$, $c$, $d \in k $, $ad - bc \neq 0$\pageoriginale \textit{then
  there exists a rational function $f(x)$ such that every function
  $\varphi(x)$ which is invariant under $G$ is a rational function
  of $f(x)$. $f(x)$ is uniquely determined up to a linear
  transformation}  
\begin{gather*}
\frac{\lambda f (x) + \mu }{ V f (x) + \rho}\\
\lambda, \mu, \nu  \rho \in k, \quad  \lambda \rho - \mu \nu \neq 0.
\end{gather*}

We now consider the case of a rational function field  $ K = k ( x_1,\break
\ldots x_n ) $ of $n$  variables. Let $G$ be a finite group of
automorphisms of $K$ which are trivial on  $k$ and let $L$ be the
fixed  field of $G$. Clearly $L$ has transcendence degree $n$ over
$k$. It is \textit{not known}, except in simple  cases, whether $L$
is  a purely transcendental extension of $k$ or not. We shall,
however, consider the  the case where $G$ is the symmetric group on
$n$ symbols. So $ G \simeq S_n $. Let $G$ operate on $K$ in the
following manner. If $ \sigma $ is an element of $ S_n$, then $ \sigma
$ is a permutation  
$$
 \sigma = \begin{pmatrix} 1,2,3,\ldots , n\\ \sigma_1, \sigma_2,
   \sigma_3, \ldots, \sigma_n \\  \end{pmatrix}. 
$$

We define $ \sigma$ on $K$ by
$$
\sigma x_i = x_{\sigma_{i}}  ; \; 1, \ldots , n .
$$

We then obtain a faithful representation of $S_n$  on $K$  and we
denote this  group again by $S_n$. An element of $k$ which is fixed
under $S_n$ and which therefore is in $L$ is called a
\textit{symmetric function} of $ x_1 , \ldots, x_n$. Obviously  
$$
( K :L ) =n!
$$
and\pageoriginale the galois group of $ K / L $ is  $S_n $.

Consider the polynomial 
$$
f (z) = (z- x_1) \ldots (z -x_n ).
$$

Since every permutation in $ S_n $  leaves  $f (z) $ fixed, it follows
that $f(z) \in L [ z ] $. Let us write  
$$
f (z) = z^n- s_1 z^{n-1} + s_2 z^{n-2} - \cdots + ( 1 )^n s_n
$$ 
where 
$$
s_i = \sum_{1 \leq t_1 < t_2 < \ldots < t_i \leq n} x_{t_{1}} \cdots x_{t_{i}}.
$$

The quantities $s_1 , \ldots , s_n$ are called the
\textit{elementary symmetric functions} of $x_1 , \ldots x_n$. Put
$L_1 = k ( s_1 ,\ldots , s_n )$. Then  $ L_1 \subset L $. Also $f
(z)$ is  a polynomial in $ L_1 \{ z \} $ and is irreducible over
it. $f (z)$ is separable and $K$ is the splitting field of $ f(z)$
over $L_1$. Thus $ K/L_1 $ is galois. Since $f(z)$ is of degree
$n$ 
$$
(K : L_1) \leq n! 
$$

Since $L \supset L_1 $, it follows that $ L = L_1 $ and 
$$
 L = k(s_1 ,\ldots , s_n).
$$

We have therefore the 

\begin{thm}\label{c3:thm15}%\theorem  15
 Every rational symmetric function of $ x_1 , \ldots , x_n $ is a
 rational function over $k$ of the elementary symmetric functions $
 s_1 , \ldots , s_n $.
\end{thm}

Incidentally since $L/k$  has dimension $n$, the elementary
symmetric functions $s_1 , \ldots , s_n$ are algebraically
independent over $k$. 

