
\chapter{The completion functors}\label{sec2}

In\pageoriginale this section we collect the definitions and the main
results on completion functors. For omitted proofs consult \S \ref{sec3} of
\cite{key15} or \cite{key33}.

Let $\mathfrak{a}$ denote the Lie algebra $sl(2,\mathbb{C})$. Choose a
basis $H$, $X$, $Y$ of $\mathfrak{a}$ with $[H,X] = 2X$, $[H,Y] = -
2Y$ and $[X,Y] = H$. Such a basis of $\mathfrak{a}$ is called a
standard basis. Put $\mathfrak{h} = \mathbb{C} \cdot H$ and
$\mathfrak{b} = \mathfrak{h} \oplus \mathbb{C} \cdot X$. Then
$\mathfrak{h}$ is a CSA of $\mathfrak{a}$ and $\mathfrak{b}$ is a
Borel subalgebra.

We now define Verma modules for $\mathfrak{a}$. For $\lambda \in
\mathfrak{h}^*$, let $\mathbb{C}_\lambda$ denote the one dimensional
$\mathfrak{b}$-module where $X$ acts by zero and $H$ acts by
multiplication by $\lambda(H)$. Put $V_\lambda = U(\mathfrak{a})
\bigotimes\limits_{U(\mathfrak{b})} \mathbb{C}_\lambda \cdot
V_\lambda$ is the Verma with highest weight $\lambda$. It will be
convenient to identify  $\mathfrak{h}^*$ and $\mathbb{C}$ by $\lambda
\rightleftarrows \lambda (H)$. 


\begin{prop}\label{chap2:prop2.1}%%% 2.1
(i) For $\lambda \in \mathbb{C}$, $V_\lambda$ is irreducible unless
  $\lambda \in \mathbb{N}$. 

\noindent
(ii) For $n \in \mathbb{N}$, we have an inclusion $V_{-n-2}
  \hookrightarrow V_n \cdot F_n = V_n / V_{-n-2}$ is the irreducible
  finite dimensional module with highest weight $n$. 
\end{prop}

\begin{definition}\label{chap2:def2.2}%%% 2.2
Let $\mathscr{I} = \mathscr{I}(\mathfrak{a})$ denote the category of
$\mathfrak{a}$-modules $A$ which satisfy: (i) $H$ acts semisimply on
$A$ with integral eigenvalues; i.e., $A = \bigoplus\limits_{n \in
  \mathbb{Z}} A_n$ with $A_n  = \{a \in A | H \cdot a = na\}$, (ii)
$X$ acts locally nilpotently on $A$ and (iii) as a module over
$U(\mathbb{C} \cdot Y)$, $A$ is torsion free. 
\end{definition}

Since\pageoriginale $U(\mathbb{C} \cdot Y)$ is a $p.i.d$., if $A \in
\mathscr{I}$ is finitely generated then $A$ is a free module over
$U(\mathbb{C} \cdot Y)$. For any $A \in \mathscr{I}$, let $A^X$ equal
the subspace of $A$ annihilated by $X$, $A_n$ equal the subspace where
$H$ acts by eigenvalue $n$ and $A^X_n = A^X \cap A_n$. 

\begin{definition}\label{chap2:def2.3}%%% 2.3
For $A \in\mathscr{I}$, we say $A$ is complete if for all $n \in
\mathbb{N}$ and all $\mathfrak{a}$-module maps $\varphi: V_{-n-2} \to
A$ there exists a unique $\mathfrak{a}$-module map $\bar{\varphi}:V_n
\to A$ such that the following diagram is commutative:
\[
\xymatrix{
V_n \ar[dr]^{\bar{\varphi}} & \\
& A \\ 
V_{-n-2} \ar[ur]_\varphi \ar@{^{(}->}[uu] & 
}
\]
\end{definition}

We leave as an exercise the verification that (\ref{chap2:def2.3}) is
equivalent to the alternate definition:

\begin{definition}\label{chap2:def2.4}%%% 2.4
For $A \in \mathscr{I}$, $A$ is complete if for all $n \in
\mathbb{N}$, $Y^{n+1}$: $A^X_n \xrightarrow{\sim} A^X_{-n-2}$ is an
isomorphism.
\end{definition}

\begin{definition}\label{chap2:def2.5}%% 2.5
For $A, B \in\mathscr{I}$, $B$ is called the completion of $A$ if
(i) $B$ is complete, (ii) there is an injection $i: A \hookrightarrow B$
with $B/iA$ locally $U(\mathfrak{a})$-finite.
\end{definition}

\begin{prop}\label{chap2:prop2.6}%%% 2.6
For $A \in \mathscr{I}$, $A$ admits a unique completion which we
denote by $C(A)$. Moreover, for any finite dimensional
$\mathfrak{a}$-module $F$, $F\otimes C(A) \simeq C(F \otimes A)$. 
\end{prop}

\begin{theorem}\label{chap2:thm2.7}
The\pageoriginale assignment $A \mapsto C(A)$ is a covariant functor
on $\mathscr{I}$.
\end{theorem}

We let $C (\cdot)$ denote the functor given by (\ref{chap2:thm2.7})
and we call this the completion functor associated with the standard
basis $H,X,Y$ of $\mathfrak{a}$. 

Next we consider the case of an ambient Lie algebra. Let
$\mathfrak{g}$ be a Lie algebra with $\mathfrak{a} \subseteq
\mathfrak{g}$.

\begin{definition}\label{chap2:def2.8}%% 2.8
Let $\mathscr{I}_{\mathfrak{g}} = \mathscr{I}_\mathfrak{g}
(\mathfrak{a})$ denote the category of $\mathfrak{g}$-modules whose
underlying $\mathfrak{a}$-modules lie in $\mathscr{I}$.
\end{definition}

For any $\mathfrak{g}$-module $A$, let $A_\mathfrak{a}$ denote the
underlying $\mathfrak{a}$-module.


\begin{prop}\label{chap2:prop2.9}
For $A \in \mathscr{I}_\mathfrak{g}$, $C(A_\mathfrak{a})$ admits
unique $\mathfrak{g}$-module structure such that $A \hookrightarrow
C(A_\mathfrak{a})$ is an inclusion of $\mathfrak{g}$-modules.
\end{prop}

For $A \in \mathscr{I}_\mathfrak{g}$, let $C(A)$ denote the unique
$\mathfrak{g}$-module given by (\ref{chap2:prop2.9}). Then
(\ref{chap2:thm2.7}) generalizes to:

\begin{theorem}\label{chap2:thm2.10}%%% 2.10
The assignment $A \to C(A)$ is a covariant functor on
$\mathscr{I}_\mathfrak{g}$. 
\end{theorem}


