
\chapter[Unitary representations and relative...]{Unitary
  representations and relative Lie algebra cohomology}\label{sec15} 

Although\pageoriginale in the preceding section the question of
unitarizability of $Z(\u{\lambda})$ is reduced to a question of
pairings for highest weight modules, the determination of the
unitarizable irreducible admissible $(\mathfrak{g},
\mathfrak{t})$-module remains an open question. However, with certain
restrictions on either the group or the class of representations,
solutions do exist. In this section we summarize without proof these
partial results. 

In the first such result we restrict the infinitesimal character of
the module. 

\begin{theorem}\label{chap15:thm15.1}
Let $Z$ be an irreducible admissible
$(\mathfrak{g},\mathfrak{t})$-module having regular integral
infinitesimal character. Then $Z$ is unitarizable if and only if there
exists a parabolic subgroup $Q$ of $G$ and a one dimensional unitary
representation $\sigma$ of $Q$ with $Z$ equivalent to the
$\mathfrak{g}$-module of $K$-finite vectors in the representation of
$G$ unitarily induced from $\sigma$ and $Q$ to $G$. 
\end{theorem}

It is somewhat surprising that the proof of (\ref{chap15:thm15.1})
follows by only a 
computation in $\mathfrak{h}^*$ and application of the Dirac operator
inequality. A proof of (\ref{chap15:thm15.1}) is given in
\cite{key18}. We have two 
corollaries to (\ref{chap15:thm15.1}). The first involves the relative
Lie algebra 
cohomology groups  for the pair $(\mathfrak{g}, \mathfrak{t})$. If $F$
is an irreducible finite dimensional $\mathfrak{g}$-module and $Z$ is
an irreducible admissible $(\mathfrak{g}, \mathfrak{t})$-module then
we let $H^*(\mathfrak{g}, \mathfrak{t}; F \otimes Z)$ denote the
relative Lie algebra cohomology groups with coefficients in $F \otimes
Z$ as defined in \cite{key4}. Chapter I. A version of Shapiro's lemma
due to P. Delorme (cf. \cite{key4} III \ref{chap3:lem3.3}) gives a formula for
computing the relative cohomology groups with coefficients in
induced\pageoriginale representations. By combining this formula and
(\ref{chap15:thm15.1}), we obtain the following vanishing theorem:

\begin{theorem}\label{chap15:thm15.2}
Assume $G$ is simple and $Z$ is unitarizable and nontrivial. Then $H^r
(\mathfrak{g}, \mathfrak{t}; F \otimes Z) = 0$ for all integers $r <
r_G$ where $r_G$ is given in Table 1 below. 
\begin{center}
{\bf Table 1.}

\medskip

\renewcommand{\arraystretch}{1.2}
\tabcolsep=12pt
  \begin{tabular}{c|c||c|c}
    \hline
Type of \ $G$ & $r_G$ & Type of\ $G$ & $r_G$\\
    \hline
$A_{\ell}$ & $\ell$ & $E_6$ & $16$\\
$B_{\ell}$ & $2\ell-1$ & $E_7$ & $27$\\
$C_{\ell}$ & $2\ell-1$ & $E_8$ & $57$\\
$D_{\ell}$ & $2\ell-2$ & $F_4$ & $15$\\
& & $G_2$ & $5$\\
    \hline 
  \end{tabular}
\end{center}
\end{theorem}

As corollaries to (\ref{chap15:thm15.2}) we obtain results for a
co-compact discrete 
subgroup $\Gamma$ of $G$. For any representation of $\Gamma$ on $A$,
let $H^*(\Gamma, A)$ denote the cohomology groups of $\Gamma$ with
values in $A$. 

\begin{coro}\label{chap15:coro15.3}
For any finite dimensional $\mathfrak{g}$-module $F$ and integer
$r<r_G$, there is a natural isomorphism
$$
H^r(\mathfrak{g}, \mathfrak{t}; F) \xrightarrow{\sim} H^r (\Gamma,
F). 
$$
Let $\mathbb{C}$ denote the trivial representation of $\mathfrak{g}$. 
\end{coro}

\begin{coro}\label{chap15:coro15.4}
Assume $\Gamma$ has no elements of finite order different from the
identity: Let $X_u$ be the simply connected compact symmetric space
dual to $G/K$. For integers $p, 0\leq p < r_G$, the $p^{\rm th}$ Betti
numbers of $\Gamma \backslash G/K$ and $X_u$ are equal and equal to $\dim H^p
(\mathfrak{g}, \mathfrak{t}; \mathbb{C})$. 
\end{coro} 

For\pageoriginale a more detailed discussion of the results above the
reader should consult \cite{key18} and \cite{key4}. 

The vanishing theorem (\ref{chap15:thm15.2}) has been proved in the
general setting of 
a real semisimple Lie group \cite{key4}, \cite{key39}. In the general
setting the constant $r_G$ is replaced by the split rank of $G$. For a
type of generalization of (\ref{chap15:thm15.1}) to a general real
semisimple Lie group 
the reader should consult the recent article of Vogan and Zuckerman
\cite{key35}.  

As a second corollary to (\ref{chap15:thm15.1}) we note that if $Z$
satisfies the 
hypotheses of (\ref{chap15:thm15.1}) then the distribution character of $Z$ is
completely determined. It is the character of a representation induced
from a one dimensional representation of a parabolic subgroup. 

Lastly we consider general results for special cases. For the cases
where the rank of $\mathfrak{g}_0$ is one or two or $\mathfrak{g}_0
\simeq s\ell (4)$ or $s\ell(5)$, Duflo \cite{key10}, \cite{key11},
has determined the unitarizable admissible irreducible $(\mathfrak{g},
\mathfrak{t})$ - modules. This work for $s\ell(n)$, $n = 2,3,4,5$, 
suggests a generalization of (\ref{chap15:thm15.1}). If
$\mathfrak{g}_0$ is of type 
$A_\ell$ then we might expect that every unitarizable irreducible
admissible $(\mathfrak{g}, \mathfrak{t})$-module is equivalent to a
representation induced from a quasi character of a parabolic subgroup
of $G$. For the other types the description in \cite{key10} already
shows that the picture will be more complicated. There are unitary
representations which are not induced. 
 

