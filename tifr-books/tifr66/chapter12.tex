
\chapter{Concepts from homological algebra}\label{sec12}

In this\pageoriginale section, let $\mathscr{O}$ denote the category
$\mathscr{O}$ for the data $(\mathfrak{t}, \mathfrak{t},
P_\mathfrak{t})$. Let $\mathfrak{n}$ denote the category of
$\mathfrak{g}$-modules whose underlying $\mathfrak{t}$-modules lie in
$\mathscr{O}$. Using standard concepts from homological algebra we
will define a collection of left derived functors on
$\mathfrak{n}$. In the next section, we will relate these to an
inverse for the lattice functor $\tau$. 

\begin{lemma}\label{chap12:lem12.1}
If $P$ is a projective object in $\mathscr{O}$, then $U(\mathfrak{g})
\bigotimes\limits_{U(\mathfrak{t})} P$ is projective in
$\mathfrak{n}$. 
\end{lemma}

The proof of (\ref{chap12:lem12.1}) is elementary. 

For $\mu \in \mathfrak{t}^*$, let $M(\mu)$ be the
$\mathfrak{t}$-Verma module with highest weight $\mu
-\delta_\mathfrak{t}$ and put $U(\mu) = U(\mathfrak{g})
\bigotimes\limits_{U(\mathfrak{t})} M(\mu)$. 

\begin{coro}\label{chap12:coro12.2}
If $\mu \in \mathfrak{t}^*$ is dominant, then $U(\mu)$ is projective
in $\mathfrak{n}$. 
\end{coro}

\begin{proof}
$U(\mu) = U(\mathfrak{g}) \bigotimes\limits_{U(\mathfrak{t})} M(\mu)$
  and, since $\mu$ is dominant, $M(\mu)$ is projective (cf. Lemma 7
  \cite{key12}). 
\end{proof}

Fix a set $\psi \subseteq \mathfrak{t}^*$ of dominant integral
elements. For any $\mathfrak{t}$-module $L$, put $L'=\sum
L^{n_\mathfrak{t}}_\mu$ with the sum over $\mu$ in $\psi$. 


\begin{definition}\label{chap12:def12.3}
\begin{itemize}
\item[{\rm (a)}]  A complex in the category $\mathfrak{n}, \ldots A_i \to  \ldots
  \to A_0 \to A \to 0 $, is called a $\psi$-resolution of $A$ if (i)
  $A_i$ is projective, $i \in \mathbb{N}$, (ii) for $i \in
  \mathbb{N}$, $A_i$ is generated over $U(\mathfrak{g})$ by the
  subspace $A'_i$ and (iii) $\ldots A'_i \to \ldots \to A'_0 \to A'
  \to 0$ is exact. 

\item[{\rm (b)}] A $\psi$-resolution\pageoriginale is called special if each $A_i$
  is the formal direct sum of $\mathfrak{g}$-modules isomorphic to
  $U(\mu)$ for $\mu \in \psi$. 
\end{itemize}
\end{definition}

\begin{lemma}\label{chap12:lem12.4}
Every object $A$ in $\mathfrak{n}$ admits a special
$\psi$-resolution. 
\end{lemma}

\begin{proof}
Let $A_0 = U(\mathfrak{g})
\bigotimes\limits_{U(\mathfrak{b}_\mathfrak{t})} A'$ and let $d_0$ be
the unique $\mathfrak{g}$-module map of $A_0$ to $A$ which extends the
inclusion $A' \hookrightarrow A$. Assume $d_i : A_i \to A_{i-1}$ has
been defined (here $A_{-1} = A$). Let $K_i =$ kernal $d_i$ and put
$A_{i+1} = U(\mathfrak{g})
\bigotimes\limits_{U(\mathfrak{b}_\mathfrak{t})} K'_i$. As above let
$d_{i+1}$ be the unique $\mathfrak{g}$-module map of $A_{i+1}$ to
$A_i$ which extends the inclusion $K_i \hookrightarrow A_i$. Then
$d_{i+1}: A_{i+1} \to K_i$ and $A'_{i+1} \to K'_i \to 0$ is exact. By
induction we have constructed a special $\psi$-resolution of $A$. 
\end{proof}

Let $A,B$ be $\mathfrak{g}$-modules in $\mathfrak{n}$ and $\varphi: A
\to B$ a $\mathfrak{g}$-module map. Let $A_* \to A$ and $B_* \to B$ be 
$\psi$-resolutions of $A$ and $B$ respectively. For any
$\mathfrak{g}$-module let $C^\sim$ be the $\mathfrak{g}$-submodule of
$C$ generated by $C'$. Note that $\varphi A^\sim \subseteq B^\sim$. We
now show that $\varphi$ induces a map $\varphi_*$. 

\setcounter{section}{12}
\setcounter{subsection}{4}
\subsection{}\label{chap12:subsec12.5}
$\varphi$ induces a map $\varphi_*: A_* \to B_*$ which commutes with
the boundary maps. 

The map $B_0 \to B^\sim$ is surjective and since $A_0$ is projective
there is a map $\varphi_0$ which makes the following diagram
commutative. 
\begin{equation*}
\xymatrix{
B_0 \ar[r] & B^\sim  \ar[r] & 0\\
A_0 \ar[r] \ar[u]_{\varphi_0} & A^\sim \ar[u]_\varphi & 
}\tag{12.6}\label{eq12.6}
\end{equation*}

Let\pageoriginale $L_i$ equal the kernel of $d_i : B_i \to B_{i-1}$
and $K_i$ equal the kernel $d_i : A_i \to A_{i-1}$. Note that by
(\ref{eq12.6}), $\varphi_0 K_0 \subseteq L_0$. Assume for $i \in
\mathbb{N}^*$, $\varphi_i$ has been defined where $\varphi_i : A_i \to
B_i$ and $\varphi_i K_i \subseteq L_i$. The map $B_{i + 1} \to L^\sim
_i$ is surjective by (\ref{chap12:def12.3}) (iii); and so, by projectivity of
$A_{i+1}$, there exists a map $\varphi_{i+1}$ which makes the
following diagram commutative. 
\begin{equation*}
\xymatrix{
B_{i+1} \ar[r] & L^\sim_i  \ar[r] & 0\\
A_{i+1} \ar[r] \ar[u]_{\varphi_{i+1}} & K^\sim_i \ar[u]_{\varphi_i} & 
}\tag{12.7}\label{eq12.7}
\end{equation*}

Note that (\ref{eq12.7}) implies $\varphi_{i+1}: K_{i+1} \to L_{i+1}$. By
induction we obtain the map $\varphi_*$ of complexes, $\varphi_* : A_*
\to B_*$. This proves (\ref{chap12:subsec12.5}). 

We next recall the standard notion of homotopy. A map of complexes,
$\Sigma_* : A_* \to B_*$, is said to be of degree $j$ if $\Sigma_i :
A_i \to B_{i+j}$, $i \in \mathbb{N}$. Unless stated otherwise a chain
map will be a map $\varphi_*$ of complexes of degree zero. For two
chain maps $\varphi_*$ and $\psi_*$, we say these maps are homotopic
if there exists a degree one chain map $\Sigma_*: A_* \to B_*$ with
$\varphi_i - \psi_i = d_{i+1} \circ \Sigma_i - \Sigma_{i-1} \circ
d_i$, $i \in \mathbb{N}$. The degree one map $\Sigma_*$ is called the
homotopy. 

\setcounter{prop}{7}
\begin{lemma}\label{chap12:lem12.8}
Let $A$ and $B$ be $\mathfrak{g}$-modules  in $\mathfrak{n}$,
$\varphi$ a map $\varphi: A \to B$ and let $A_* \to A$, $B_* \to B$ be
$\psi$-resolutions. Let $\varphi_*$ and $\psi_*$ be any two chain maps
induced by $\varphi$. Then $\varphi_*$ and $\psi_*$ are homotopic. 
\end{lemma}

For projective resolutions, (\ref{chap12:lem12.8}) is a standard result. In our case
of $\psi$-resolutions, the standard proof works (cf. Theorem \ref{chap4:def4.1}
\cite{key23}) with modules\pageoriginale $C$ replaced by submodules
$C^\sim$ as in the construction of $\varphi_*$ given above.

Let $\mathfrak{U}$ be a category of modules with enough
projectives. Let $T$ be an additive functor defined on the full
subcategory $\mathscr{P}$ whose objects are the projective objects in
$\mathfrak{U}$. We now define a set of functors $T_i$, $i \in
\mathbb{N}$, on $\mathfrak{U}$ called the left derived functors of the
functor $T$ on $\mathscr{P}$. For an object $A$ in $\mathfrak{U}$, let
$A_* \to A$ be a projective resolution of $A$. Then $TA_*$ is a
complex and we define $T_i A$ to be the $i^{\rm th}$ homology group of
the complex; i.e., $T_i A = $ kernel $Td_i/ \text{image }
Td_{i+1}$. Since any two projective resolutions are homotopic and
since $T$ is additive, the module $T_i A$ is independent of the choice
of the projective resolution. For $i \in \mathbb{N}$, $T_i$ is a
covariant functor on $\mathfrak{U}$. 

We would like to apply this technique to our setting. To do this we
begin by defining a functor $\sigma$ on a special full subcategory of
$\mathfrak{n}$. Let $\mathscr{S}$ be the full subcategory of
$\mathfrak{n}$ whose objects are formal direct sums of modules
$U(\mu)$, $\mu \in \psi$. Let $t_0$ denote the maximal element of
$\mathfrak{w}_\mathfrak{t}$. Then, for $\mu \in \psi$, $M(t_0 \mu)$ is
the unique irreducible submodule of the $\mathfrak{t}$-Verma module
$M(\mu)$. Let $\sigma U(\mu)$ be the submodule of $U(\mu)$ given by
the inclusion $U(\mathfrak{g}) \bigotimes\limits_{U(\mathfrak{t})}
M(t_0 \mu) \hookrightarrow U(\mathfrak{g})
\bigotimes\limits_{U(\mathfrak{t})} M(\mu) = U(\mu)$. Extend $\sigma$
linearly to a map on objects in $\mathscr{S}$. Next we prove:
\begin{equation*}
\sigma \text{ is an additive functor on }\mathscr{S}. 
 \tag{12.9}\label{eq12.9}
\end{equation*}

\begin{proof}
For objects $A$  and $B$ in $\mathscr{S}$ and for $\varphi \in \Hom
(A,B)$, we define $\sigma \varphi$ to be the restriction of $\varphi$
to the  submodule $\sigma A \subset A$. Assume $\varphi(\sigma A)
\subset \sigma B$. Then clearly $\sigma$ takes the identity map to the
identity map and also $\sigma$ commutes with compositions. Thus to
prove (\ref{eq12.9}) we need only check\pageoriginale $\varphi (\sigma A)
\subset \sigma B$. From the definition of $\mathscr{S}$ we need only
check, for $\mu, \nu \in \psi$. 
\begin{equation*}
\text{if } \varphi : U(\mu) \to U(\nu) \text{ then } \varphi (\sigma
U(\mu)) \subset \sigma U(\nu). \tag{12.10} \label{eq12.10}
\end{equation*}
\end{proof}

For $s \in \mathfrak{w}_\mathfrak{t}$, put $U_s = U(\mathfrak{g})
\bigotimes\limits_{U(\mathfrak{t})} M(s\mu)$, $V_s = U(\mathfrak{g})
\bigotimes\limits_{U(\mathfrak{t})} M(s\nu)$. Now $\varphi:U_1 \to
V_1$. Assume $\varphi:U_s \to V_s$ and assume $\alpha$ is a simple
root in $P_\mathfrak{t}$ with $(s\mu)_\alpha \in \mathbb{N}^*$. Put
$\mathfrak{t}^\alpha$ equal to the reductive subalgebra of
$\mathfrak{t}$, $\mathfrak{t}^\alpha = t \oplus \mathfrak{t}_\alpha
\oplus \mathfrak{t}_{-\alpha}$. The module $M(s\nu)/M(s_\alpha s \nu)$
is locally $U(\mathfrak{t}^\alpha)$-finite; and so, $V_s/V_{s_\alpha
  s}$ is also. The cyclic vector for $U_{s_\alpha s}$ generates under
$U(\mathfrak{t}^\alpha)$ an irreducible Verma
$\mathfrak{t}^\alpha$-module. Therefore the induced map $U_{s_\alpha
  s} \to V_s/V_{s_\alpha s}$ must be zero. This implies $\varphi:
U_{s_\alpha s} \to V_{s_\alpha s}$. By induction we have $\varphi: U_r
\to V_r$ for all $r \in \mathfrak{w}_\mathfrak{t}$. For $r = t_0$, we
have proved (\ref{eq12.10}) which completes the proof of (\ref{eq12.9}).


For convenience, we list a corollary to the proof of (\ref{eq12.10}):

\setcounter{prop}{10}
\begin{coro}\label{chap12:coro12.11}
Let $A_s$ and $B_s$, $s \in \mathfrak{w}_\mathfrak{t}$, be lattices of
$\mathfrak{g}$-modules and $\varphi: A_1 \to B_1$ a
$\mathfrak{g}$-module map. Then, for each $s \in
\mathfrak{w}_\mathfrak{t}$, $\varphi A_s \subseteq B_s$. 
\end{coro}

\begin{definition}\label{chap12:def12.12}
Let $A$ be an object in $\mathfrak{n}$ and by (\ref{chap12:lem12.4}) let $A_* \to A$
be a special $\psi$-resolution of $A$. Define $\sigma_i A$ to be the
$i^{\rm th}$ homology group of the complex $\sigma A_*$. By (\ref{eq12.9}),
$\sigma$ is an additive functor on $\mathscr{S}$; and so, each
$\sigma_i$ is a covariant functor on $\mathfrak{n}$. Also, using
(\ref{chap12:lem12.8}), $\sigma_i A$ is independent of the choice of $\psi$-resolution
$A_*$ of $A$. We call the covariant functor $\sigma_i$ the $i^{\rm
  th}$ left derived functor of $\sigma$. 
\end{definition}

We conclude\pageoriginale this section by showing that the $\sigma_i$,
$i \in \mathbb{N}$, preserve central character. 

\begin{prop}\label{chap12:prop12.13}
Let $A \in \mathfrak{n}$ and let $\chi$ be a character of
$Z(\mathfrak{g})$. Assume for some $n \in \mathbb{N}^*$, $a(z) =
(1-\chi (z))^n$ annihilates $A$ for $z \in Z(\mathfrak{g})$. Then, for
$i \in \mathbb{N}$ and $z \in Z(\mathfrak{g})$, $a(z)$ annihilates
$\sigma_i (A)$. 
\end{prop}

\begin{proof}
Let $A_* \to A$ be a special $\psi$-resolution of $A$. Then
multiplication by $a(z)$ and by zero on $A_*$ are homotopic. For $T
\in\Hom (A_*, A_*)$, $\sigma T$ is restriction to $\sigma
A_*$. Therefore, $\sigma(a(z)) = a(z)$; and so, multiplication by
$a(z)$ and by zero are homotopic on $\sigma A_*$. Thus, they induce
the same maps on homology groups. This proves (\ref{chap12:prop12.13}). 
\end{proof}
