
\chapter[Resolutions of irreducible admissible...]{Resolutions of
  irreducible admissible $(\mathfrak{g},  
  \mathfrak{t})$-modules and $\mathfrak{t}$-multiplicity
  formulae}\label{sec7} 

In\pageoriginale \cite{key1} Bernstein, Gel'fand and Gel'fand have
given a resolution of finite dimensional modules in terms of sums of
Verma modules. These resolutions yield at once the character formulae
as well as the weight space structure of the finite dimensional
modules. This section will give the corresponding results for the
irreducible $(\mathfrak{g}, \mathfrak{t})$-modules $Z(\u{\lambda})$
defined in \S\ \ref{sec6}.

Recall the category $\mathscr{O} \otimes M(\lambda)$ from \S \ref{sec3}. Let
$\ell(\cdot)$ denote the length function on the Weyl group
$\mathfrak{w}_\mathfrak{t}$. 

\begin{prop}\label{chap7:prop7.1}
Let $B$ be an object in $\mathscr{O} \otimes M(\lambda)$ with integral
$\mathfrak{t}$-weights and assume $M(\lambda)$  is irreducible. Let
$B_s$, $s \in \mathfrak{w}_\mathfrak{t}$, be a lattice above $B$ and,
for $0 \leq i \leq d = \mid P_\mathfrak{t}\mid$, put $\beta_i =
\sum\limits_{\ell(s) = i} B_s$. Then there is a resolution:
$$
0 \to \beta_d \to \ldots \to \beta_0 \to \tau B \to 0.
$$
\end{prop}

To proof will use the lemma:

\begin{lemma}\label{chap7:lem7.2}
Let $C$ be any finite dimensional $\mathfrak{b}$-module which is a
weight module for $\mathfrak{h}$. Then there exists $\nu \in
\mathfrak{h}^*$ with $\nu_\alpha << 0$, $\forall \alpha \in P$, a
finite dimensional $\mathfrak{g}$-module $F$ and an injective
$\mathfrak{b}$-module map $C \hookrightarrow F \otimes
\mathbb{C}_\nu$. 
\end{lemma}

This lemma can be found as Lemma \ref{chap4:def4.7} in \cite{key15}. 

\begin{theorem}\label{chap7:thm7.3}
Let\pageoriginale $\mu \in \mathfrak{t}^*$ be regular and
-$P_\mathfrak{t}$-dominant integral. Let $V_s$, $s \in
\mathfrak{w}_\mathfrak{t}$, be the lattice above the
$\mathfrak{t}$-Verma module $M(\mu)$ and put $\mathcal{C}_i =
\sum\limits_{\ell(s) = i} V_s$. Then there is a resolution
$$
0 \to \mathcal{C}_d \to \ldots \to \mathcal{C}_0 \tau M(\mu) \to 0. 
$$
Moreover, $\tau M(\mu)$ is the irreducible finite dimensional
$\mathfrak{t}$-module with extreme weight $\mu +
\delta_\mathfrak{t}$. Each $V_s$ is a Verma module and the maps
$\mathcal{C}_i \to \mathcal{C}_{i-1}$, $1 \leq i \leq d$, are linear
combinations of the inclusion maps among Verma modules. 
\end{theorem}

This theorem is a reformulation of the Bernstein Gel'fand Gel'fand
result \cite{key1}. 

For a character $\chi$ of $Z(\mathfrak{t})$, let a superscript $\chi$
denote the generalized eigensubspace for $\chi$. Since $B$ is the
direct sum of subspaces $B^\chi$ it is sufficient to give a
resolution:
\begin{equation*}
0 \to \beta^{\chi}_d \to \ldots \to \beta^{\chi}_0
\tag{7.4}\label{eq7.4}
\end{equation*}

By assumption we may write $B = A \otimes M(\lambda)$. Then by
(\ref{chap3:lem3.10}), we have a $\mathfrak{t}$-module isomorphism $B \simeq
U(\mathfrak{t}) \bigotimes\limits_{U(b_\mathfrak{t})} (A \otimes
\mathbb{C}_{\lambda - \delta_0})$. Since $A \in \mathscr{O}$, there
exists a finite dimensional $\mathfrak{b}$-module $C \subseteq A$ with
$B^\chi$ contained in the submodule $U = U(\mathfrak{t})
\bigotimes\limits_{U(\mathfrak{b}_\mathfrak{t})} (C \otimes
\mathbb{C}_{\lambda - \delta_0})$. Now applying (\ref{chap7:lem7.2}), we obtain an
injection $B^\chi \hookrightarrow F \otimes M (\nu + \lambda)$. By
Proposition \ref{chap3:prop3.9}, $B$ and hence $B^\chi$ admits a nondegenerate
$\mathfrak{t}$-invariant form. Clearly $B^\chi = U^\chi$ and so
$B^\chi$ is a summand of $U$. Thus we may extend the
$\mathfrak{t}$-invariant form on $B^\chi$ to $U$ and then by
Proposition \ref{chap3:prop3.6}\pageoriginale to a form on $F \otimes M(\nu +
\lambda)$. Taking the complement of $B^\chi$ we find that $B^\chi$ is
a summand of $F \otimes M(\nu+ \lambda)$. Applying (\ref{chap7:thm7.3}) with $\mu =
\nu + \lambda$ and tensoring by $F$, we obtain a resolution:
\begin{equation*}
0 \to \mathscr{D}_d \to \ldots \to \mathscr{D}_0 \to \tau (F \times
M(\nu + \lambda)) \to 0 \tag{7.5}\label{eq7.5}
\end{equation*}
where the maps $\mathscr{D}_i \to \mathscr{D}_{i-1}$ are linear
combinations of inclusion maps. Since $B^\chi$ is a summand of $F
\otimes M(\nu + \lambda)$, the resolution (\ref{eq7.5}) is the direct sum of
two resolutions with one resolution being the resolution (\ref{eq7.4}). This
proves (\ref{chap7:prop7.1}). 

\setcounter{prop}{5}
\begin{coro}\label{chap7:coro7.6}
{\bf to (\ref{chap7:prop7.1}).} Let $\u{\lambda} \in \mathfrak{L}$ and assume
$M(\lambda')$ is irreducible. Let $B_s$, $s \in
\mathfrak{w}_\mathfrak{t}$, be a lattice above $L(\u{\lambda})$ and,
for $0 \leq i \leq d$, put $\beta_i = \sum\limits_{\ell(s) = i }
B_s$. Then there is a resolution 
$$
0 \to \beta_d \to \ldots \to \beta_0 \to \tau L(\u{\lambda}) \to 0.
$$

As preparation for a $\mathfrak{t}$-multiplicity formula, we need:
\end{coro}

\begin{lemma}\label{chap7:lem7.7}
Let $A \in \mathscr{O}$ and assume that $M(\lambda)$ is
irreducible. For $\mu \in \mathfrak{h}^*_0$, put $\mu_1 = (\mu, 0)
\mid_t$ and $B = A \otimes M (\lambda)$. Then we have the following
dimension formula for the weight spaces of the
$\mathfrak{n}_\mathfrak{t}$-invariants in $B$:
$$
\dim B^{\mathfrak{n}_\mathfrak{t}}_{\mu_1} = \dim A_{\mu  - \lambda +
  \delta_0}. 
$$
\end{lemma}

\begin{proof}
Since $B$ admits a nondegenerate $\mathfrak{t}$-invariant form
(cf. Proposition \ref{chap3:prop3.9}), Lemma (\ref{chap7:lem7.7})
follows from the decomposition $B = 
A \otimes 1 \oplus n^-_\mathfrak{t} B$ and the equivalences
$n^+_\mathfrak{t} v = 0 \Leftrightarrow \langle n^+_\mathfrak{t} v, B
\rangle = 0 \Leftrightarrow \langle v, n^-_\mathfrak{t} B \rangle =
0$.  
\end{proof}

\begin{prop}\label{chap7:prop7.8}
Assume\pageoriginale $M(\lambda)$ is irreducible. Let $B = A \otimes
M(\lambda) \in \mathscr{O} \otimes (\lambda)$. For $\mu \in
\mathfrak{h}^*_0$ $P_0$-dominant integral, put $\mu_1 = (\mu, 0)
\mid_\mathfrak{t}$ and let $F$ be the irreducible finite dimensional
$\mathfrak{t}$-module with extreme weight $\mu_1$. Then the
multiplicity of $F$ in the $(\mathfrak{g}, \mathfrak{t})$-module $\tau
B$ is given by:
$$
\dim \Hom_\mathfrak{t} (F, \tau B) = \sum\limits_{s \in
  \mathfrak{w}_0} (-1)^{d-\ell(s)} \dim A_{s(\mu +
  \delta_0)-\lambda}. 
$$
\end{prop}

\begin{proof}
By assumption $\nu$ is $P_0$-dominant, so $\mu_1$ is
$P_1$-dominant. Since $M(\mu_1+ \delta_\mathfrak{t})$ is projective
(cf. Lemma 7 \cite{key12}) we obtain from (\ref{chap7:prop7.1}) the formula
\begin{equation*}
\dim \Hom_{\mathfrak{t}} (F, \tau B) = \sum\limits_{0 \leq i \leq d}
(-1)^i \dim (\beta^{\mathfrak{n}_\mathfrak{t}}_i)_{\mu_1}. \tag{7.9}\label{eq7.9}
\end{equation*}
\end{proof}

From Proposition \ref{chap4:prop4.11} \cite{key15} we have the isomorphisms:
\begin{equation*} 
(B^{\mathfrak{n}_\mathfrak{t}}_s)_{\mu_1} \sim
  (B^{\mathfrak{n}_\mathfrak{t}})_{t_0 s^{-1} (\mu_1 +
    \delta_\mathfrak{t})-\delta_\mathfrak{t}} , \quad \forall s \in
  \mathfrak{w}_\mathfrak{t}, \tag{7.10}\label{eq7.10}
\end{equation*}
where $t_0$ is the unique element of maximal length in
$\mathfrak{w}_\mathfrak{t}$. Combining (\ref{eq7.9}) and (\ref{eq7.10}), we have:
\begin{equation*}
\dim \Hom_\mathfrak{t} (F, \tau B) = \sum\limits_{s \in
  \mathfrak{w}_\mathfrak{t}} (-1)^{\ell(s)} \dim
B^{\mathfrak{n}_\mathfrak{t}}_{t_0 s^{-1}
  (\mu_l+\delta_\mathfrak{t})-\delta_\mathfrak{t}}. \tag{7.11}\label{eq7.11}
\end{equation*}

Now replacing $t_0 s^{-1}$ by $s$ and using (\ref{chap7:lem7.7}), we reach the final
form of the multiplicity formula, (\ref{chap7:prop7.8}). 

\setcounter{prop}{11}
\begin{coro}\label{chap7:coro7.12}
Let\pageoriginale $\u{\lambda} = (\lambda, \lambda') \in \mathfrak{L}$
and assume $M(\lambda')$ is irreducible. Then 
$$
\dim \Hom_\mathfrak{t} (F, \tau L(\u{\lambda})) = \sum\limits_{s \in
  \mathscr{w}_0}  (-1)^{d-\ell(s)} \dim L(\lambda)_{s(\mu + \delta_0)
  - \lambda'}. 
$$
\end{coro}

\begin{coro}\label{chap7:coro7.13}
Let $\alpha \in P_0$ be simple and assume $\lambda'_\alpha =0$ and
$\lambda_\alpha \in \mathbb{N}^*$. Then $\tau L(\u{\lambda}) = 0$. 
\end{coro}

Note: See (\ref{chap9:prop9.14}) for a sharper result. 

\begin{proof}
$\ell(s_\alpha s) = \ell(s) \pm 1$ and the dimensions of weight spaces
  of $L(\lambda)$ are invariant under multiplication by
  $S_\alpha$. Therefore by (\ref{chap7:coro7.12}). $\Hom_\mathfrak{t}
  (F,\break \tau   L(\u{\lambda})) = 0$ for all finite dimensional 
  $\mathfrak{t}$-modules $F$. This implies $\tau L(\u{\lambda}) = 0$. 
\end{proof}

\begin{coro}\label{chap7:coro7.14}
Let $\u{\lambda} = (\lambda, \lambda') \in \mathfrak{L}$ and assume
$M(\lambda')$ is irreducible. Put $\nu = (\u{\lambda})
\mid_\mathfrak{t}$. Then, for any finite dimensional
$\mathfrak{t}$-module $F$,
$$
\dim \Hom_\mathfrak{t} (F, \tau M(\u{\mathfrak{\lambda}})) = \dim
\Hom_\mathfrak{t} (\mathbb{C}_\nu, F). 
$$
\end{coro}

\begin{proof}
We may assume $F$ is irreducible with -$P_\mathfrak{t}$ highest weight
$\mu_1 = (\mu, 0) \mid_t$, $\mu \in \mathfrak{h}^*0$. Using the BGG
resolution of $F$ (\ref{chap7:thm7.3}), we have:
\begin{equation*}
\dim F_{t_0 \nu} = \dim \Hom_\mathfrak{t} (\mathbb{C}_{t_0 \nu}, F) =
\sum\limits_{s \in \mathfrak{w}_\mathfrak{t}} (-1)^{\ell(s)} \dim M
(st_0 (\mu_1 - \delta_\mathfrak{t}))_{t_0 \nu}. 
\tag{7.15}\label{eq7.15}
\end{equation*}
\end{proof}

Now replacing $\mu_1$ by $t_0 \mu_1$ in (\ref{chap7:prop7.8}), we obtain
\begin{equation*}
\dim  \Hom_\mathfrak{t} (F, \tau M(\u{\lambda})) = \sum\limits_{s \in
  \mathfrak{w}_0} (-1)^{\ell(s)} \dim M(\lambda)_{s (\mu - \delta_0)
  -\lambda'}. \tag{7.16}\label{eq7.16}
\end{equation*}\pageoriginale

Next define the partition function on $\mathfrak{h}^*_0$ as
follows. Let $Q$ be a positive system of roots for $\Delta_0$. If $Q =
\{\alpha_1, \ldots \alpha_r\}$ and $\xi \in \mathfrak{h}^*_0$ define
$\mathscr{P}_Q(\xi)$ to be the number of $\mathfrak{n}$-tuples $(a_1,
\ldots, a_n)$ with $a_i \in \mathbb{N}$ and $\xi = a_1 \alpha_1 +
\cdots + a_n \alpha_n$. From the elementary properties of Verma
modules, we have:
\begin{align*}
\dim M(\lambda)_{s(\mu - \delta_0)-\lambda'} & = \mathscr{P}_{-P_0} (s
(\mu - \delta_0)- \lambda' - \lambda + \delta_0)\\
\dim M(st_0 (\mu_l - \delta))_{t_0 \nu} & = \mathscr{P}_{-P_0} (t_0
(\lambda + \lambda') -st_0 (\mu - \delta_0) + \delta_0)\\
& = \mathscr{P}_{P_0} (\lambda + \lambda' - t_0 st_0 (\mu - \delta_0) -
\delta_0)\\
& \qquad \text{ since } t_0 P_0 = - P_0\\
& = \mathscr{P}_{-P_0} (t_0 st_0 (\mu - \delta_0) - \lambda - \lambda'
+ \delta_0). 
\end{align*}

Since $\ell(t_0 st_0) = \ell(s)$ and $\dim F_\nu = \dim F_{t_0 \nu}$,
these formulae yield, when inserted into (\ref{eq7.15}) and (\ref{eq7.16}), the
equivalence of dimensions (\ref{chap7:coro7.14}). 

Corollary \ref{chap7:coro7.14} is a reciprocity formula for the $(\mathfrak{g},
\mathfrak{t})$ modules $\tau M(\u{\lambda})$. In the context of
induced representations this same formula is called the Frobenius
Reciprocity Theorem (cf. (\ref{chap8:subsec8.3})). 
