
\chapter{Translation functors}\label{sec5}


In\pageoriginale this section we described the Zuckerman functors
which allow the translation from modules with one infinitesimal
character to modules with another. Let $\mathfrak{g}_0$ be a real
semisimple Lie algebra with complexification $\mathfrak{g}$. With our
standard notation $\mathfrak{g}$ has a CSA $\mathfrak{h}$, roots
$\Delta$, positive system $P$ and Weyl group $\mathfrak{w}$. For $\nu
\in \mathfrak{h}^*$, let $\chi_\nu$ denote the $Z(\mathfrak{g})$
character which is the infinitesimal character of the Verma module
$M(\nu)$. By the Chevalley restriction theorem we know that every
character of $Z(\mathfrak{g})$ has the form $\chi_\nu$ and, for $\nu,
\nu' \in \mathfrak{h}^*$, $\chi_\nu = \chi_{\nu'}$ precisely when
$\nu$ and $\nu'$ lie in the same $\mathfrak{w}$-orbit. For $\nu \in
\mathfrak{h}^*$ and any $\mathfrak{g}$-module $A$, let $P_\nu A$ be
the maximal $\mathfrak{g}$-submodule of $A$ where elements $z -
\chi_\nu(z)$, $z \in Z(\mathfrak{g})$, are locally nilpotent. $P_\nu
A$ is called the generalized eigen subspace of $A$ for $\chi_\nu$.

Let $\mathfrak{g} = \mathfrak{t} \oplus \mathfrak{p}$ be the
complexification of a Cartan decomposition of $\mathfrak{g}_0$ and let
$\mathfrak{U}$ denote the category of admissible $(\mathfrak{g},
\mathfrak{t})$-modules. Let $\mathscr{O}$ denote the BGG category
$\mathscr{O}$ for $\mathfrak{g}$. For $A \in \mathfrak{U}$
(resp. $\mathscr{O}$), $A$ is the direct sum of its subspaces $P_\nu
A$, $v \in \mathfrak{h}^*$. For $\nu \in \mathfrak{h}^*$, let
$\mathfrak{U}_\nu$ (resp. $\mathscr{O}_\nu$) denote the full
subcategory of $\mathfrak{U}$ with all objects $A$ in $\mathfrak{U}$
(resp. $\mathscr{O}$), such that $P_\nu A = A$. 

Fix $\lambda \in \mathfrak{h}^*$ and let $\Delta_\lambda$, $P_\lambda$
and $\mathfrak{w}_\lambda$ be as in \S\ \ref{sec1}. Let $F_\mu$ be the
irreducible finite dimensional $\mathfrak{g}$-module with extreme
weight $\mu$. Assume, for $\alpha \in \Delta_\lambda$, $\lambda_\alpha
\mu_\alpha \in \mathbb{N}$. Then define functors
$\varphi^{\lambda}_{\lambda + \mu}$ and $\psi^{\lambda+\mu}_{\lambda}$ by:
\begin{equation*}
\varphi^{\lambda}_{\lambda + \mu} A = P_{\lambda + \mu} (F_\mu \otimes
P_\lambda (A)), \; \psi^{\lambda + \mu}_\lambda (A) = P_\lambda
(F^*_\mu \otimes P_{\lambda + \mu} (A)). \tag{5.1}\label{eq5.1}
\end{equation*}
We\pageoriginale say that $\lambda$ and $\lambda + \mu$ are
equisingular if they have the same stabilizer in $\mathfrak{w}$. For
convenience we write $\varphi = \varphi^\lambda_{\lambda + \mu}$ and
$\psi = \psi^{\lambda + \mu}_\lambda$.

\setcounter{prop}{1}
\begin{theorem}\label{chap5:thm5.2}
\begin{itemize}
\item[{\rm (i)}] $\varphi$ is injective on both $\mathfrak{U}_\lambda$ and
  $\mathscr{O}_\lambda$. 

\item[{\rm (ii)}] If $\lambda$ and $\lambda + \mu$ are equisingular, then we have
  natural equivalences: $\varphi:\mathfrak{U}_\lambda
  \xrightarrow{\sim} \mathfrak{U}_{\lambda + \mu}$, $\varphi:
  \mathscr{O}_\lambda \xrightarrow{\sim} \mathscr{O}_{\lambda +
    \mu}$. Moreover, $\psi$ is the natural inverse to $\varphi$ in
  both cases.

\item[{\rm (iii)}] If $A \in \mathfrak{U}$ or $\mathscr{O}$ and is irreducible,
  then $\psi A$ is either zero or irreducible.

\item[{\rm (iv)}]
If $A \in \mathfrak{U}_\lambda$ (resp. $\mathscr{O}_\lambda$) and
  is irreducible, then there is an irreducible $B \in
  \mathfrak{U}_{\lambda + \mu}$ (resp. $\mathscr{O}_{\lambda + \mu}$)
  with $\psi B \simeq A$.
\end{itemize}
\end{theorem}

Assertions (i) and (ii) are due to Zuckerman \cite{key38}; while (iii)
and (iv) have been established for the category $\mathscr{O}$ by
Jantzen \cite{key24} and for the category $\mathfrak{U}$ by Vogan
\cite{key34}. For the category $\mathscr{O}$, assertions (iii) and
(iv) are implied by the following proposition. For the category of
admissible modules for a complex Lie group, we give a proof of (iii)
and (iv) in (\ref{chap10:prop10.9}). 

Recall the notation for Verma modules and their irreducible
quotients. 

\begin{prop}\label{chap5:prop5.3}
With notation as in (\ref{chap5:thm5.2}), we have:
\begin{itemize}
\item[{\rm(i)}] $\psi M(\lambda + \mu) = M(\lambda)$

\item[{\rm(ii)}] $\psi L(\lambda + \mu) = L(\lambda)$ or zero

\item[{\rm(iii)}] Let $Q = P \cap \{\alpha \mid \lambda_\alpha = 0
  \}$. Then $\psi L(\lambda + \mu) = L(\lambda)$ if and only if
  $\lambda + \mu$ is $-Q$-dominant.

\item[{\rm(iv)}] If\pageoriginale $\lambda$ and $\lambda + \mu$ are
  equisingular,   then 
$$
\varphi M(\lambda) = M(\lambda + \mu), \quad \varphi L(\lambda) = L
(\lambda + \mu).
$$
\end{itemize}
\end{prop}

\begin{proof}
$F^*_\mu \otimes M(\lambda + \mu) \simeq U(\mathfrak{g})
  \bigotimes\limits_{U(\mathfrak{b})} (F^*_\mu \otimes
  \mathbb{C}_{\lambda + \mu -\delta})$. Let $0 = L_0 \subset \ldots
  \subset L_r = F^*_\mu \otimes \mathbb{C}_{\lambda + \mu - \delta}$
  be a flag of $\mathfrak{b}$-modules for $L_r$ with $L_i/L_{i-1}
  \simeq \mathbb{C}_{\nu_i - \delta}$, $1 \leq i \leq r$. Then by
  inducing form $U(\mathfrak{b})$ to $U(\mathfrak{g})$, $F^*_\mu
  \otimes M(\lambda + \mu)$ has a flag of submodules  $0= N_0 \subset
  \ldots \subset N_r$ with $N_i/N_{i-1} \simeq M(\nu_i)$. By
  assumption, $\lambda$ and $\mu$ lie in the same Weyl chamber for
  $\Delta_\lambda$; and so, the only element $\nu_i$ in the
  $\mathfrak{w}$-orbit of $\lambda$ is $\lambda$ itself. This proves
  (i).
\end{proof}

The functor $\psi$ is exact. So by (i), $\psi L(\lambda + \mu)$ is a
quotient of $M(\lambda)$. However, $L(\lambda +\mu)$ and $F^*_\mu$
both admit nondegenerate invariant forms (cf. (\ref{chap3:prop3.5})); and so, $\psi
L(\lambda + \mu)$ admits a nondegenerate invariant form. Therefore by
(\ref{chap3:prop3.5}), $\psi L(\lambda + \mu)$ is either zero or $L(\lambda)$.

The functor $\psi$ is exact; and so, by (i) and (ii), for some
$L(\xi)$ occurring in $M(\lambda + \mu)$, $\psi L(\xi ) =
L(\lambda)$. Choose $s \in \mathfrak{w}$ with $\xi = s (\lambda +
\mu)$. Then, again by exactness of $\psi$, $L(\lambda)$ must occur as
a quotient of $\psi M(\xi) = M(s\lambda)$. Thus $s\lambda = \lambda$
and then if $\lambda + \mu$ is -$Q$-dominant, $\xi = \lambda +
\mu$. This proves half of (iii). Now assume $\lambda + \mu$ is not
-$Q$-dominant and choose $\alpha \in Q$ with $(\lambda + \mu)_\alpha
\in \mathbb{N}^*$. Put $\xi = s_\alpha (\lambda + \mu)$. Now $M(\xi)
\subset M(\lambda + \mu)$ and $\psi M(\xi) = M(\lambda) = \psi
M(\lambda) = M(\lambda) = \psi M(\lambda + \mu)$. But $\psi$ is exact;
and so, $\psi A =0$ for any component of $M(\lambda +
\mu)/M(\xi)$. Thus $\psi L(\lambda + \mu) =0$, completing the proof of
(iii). Statement (iv) follows from (i), (ii) and (iii) using (\ref{chap5:thm5.2})
(ii). 

\begin{prop}\label{chap5:prop5.4}
Let $\mathfrak{t} \subset \mathfrak{g}$ be a subalgebra and let $\tau$
be the functor defined on $\mathscr{I}_\mathfrak{g}(\mathfrak{t})$ as
in \S\ \ref{sec4}. Then $\tau$ commutes with both $\varphi$ and $\psi$. 
\end{prop}

\begin{proof} 
By  (\ref{chap4:thm4.8}), $\tau$ commutes with tensoring by finite dimensional
$\mathfrak{g}$-modules. Also, completions preserve generalized
infinitesimal character; so, $P_\nu$ commutes with completions, $\nu
\in \mathfrak{h}^*$. These two facts give (\ref{chap5:prop5.4}).
\end{proof}

