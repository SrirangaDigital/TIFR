
\chapter{Determination of the irreducible admissible $(\mathfrak{g},
  \mathfrak{t})$-modules}\label{sec10}

Here\pageoriginale we combine the results of the last four sections to
describe the equivalence classes of irreducible admissible
$(\mathfrak{g}, \mathfrak{t})$-modules. In turn this will give a
classification of the infinitesimal equivalence classes of
topologically completely irreducible representations of $G$. 

\begin{lemma}\label{chap10:lem10.1}
Let $\lambda, \nu, \mu \in \mathfrak{h}^*_0$. Assume $\re \nu$ is
-$P_\nu$-dominant, that $\lambda + \nu$ and $\mu + \nu$ are integral,
and that $\mu$ lies below $\lambda$ in the Bruhat ordering. Let $S$ be
the stabilizer of $\nu$ in $\mathfrak{w}_0$. Then $||\mu + \nu|| \geq
||\lambda + \nu ||$ and equality holds if and only if $\mu$ and
$\lambda$ lie in the same $S$-orbit. 
\end{lemma}

\begin{proof}
By assumption choose $d \in \mathbb{N}$ and elements $\alpha_0 ,\ldots
, \alpha_{d-1} \in P_0$ with $\mu_{i+1} = s_{\alpha_i} \mu_i$, $0
\leq i \leq d -1$, and $\mu = \mu_d < \ldots < \mu_0 = \lambda$. Now
$||\mu_{i+1} + \nu|| = ||\mu_i + \nu - m \alpha_i||$ with $m =
(\mu_i)_{\alpha_i} \in \mathbb{N}^*$. Then $||\mu_{i+1} + \nu||^2 =
||\mu_i + \nu||^2 + \langle \mu_i + \nu - \dfrac{1}{2} m \alpha_i, - 2
m \alpha_i \rangle = ||\mu_i + \nu||^2 + \langle \re \nu, -2 m
\alpha_i\rangle $. Since $\mu_i + \nu$ is integral, $\nu_{\alpha_i}
\in \mathbb{N}$ and we may replace $\re \nu $ by $\nu$. Also since
$\re \nu$ is -$P_\nu$-dominant, $||\mu_{i+l} + \nu || \geq ||\mu_i +
\nu||$. Moreover, equality holds if and only if $s_{\alpha_i} \in
S$. So, $||\mu + \nu || \geq ||\lambda + \nu||$ and $\lambda$ and
$\mu$ lie in the same $S$-orbit if and only if equality holds. 
 \end{proof}

\begin{coro}\label{chap10:coro10.2}
Let $\u{\lambda} = (\lambda, \lambda') \in \mathfrak{L}$ and assume
$\u{\lambda}$ satisfies (\ref{chap6:subsec6.5}). Assume $\mu$ lies below $\lambda$ in 
the Bruhat ordering. Then $||\mu + \lambda'|| > || \lambda +
\lambda'||$. 
\end{coro}

\begin{proof}
By (\ref{chap6:subsec6.5}), if $Q = P_0 \cap \{\alpha \mid \lambda'_\alpha = 0\}$ then
$\lambda$ is -$Q$-dominant. $S$ is generated by the $s_\alpha$,
$\alpha \in Q$, and so, with notation as in the proof of
(\ref{chap6:subsec6.1})\pageoriginale with  $\lambda' = \nu$, $\lambda_{\alpha_0} \in
\mathbb{N}^*$ implies $\alpha_0 \not\in Q$. Then $||\mu + \lambda'||
\geq ||\mu_1 + \lambda'|| > ||\lambda + \lambda'||$. This proves
(\ref{chap10:coro10.2}). 
\end{proof}

\begin{definition}\label{chap10:def10.3}
For $\u{\lambda} \in \mathfrak{L}$, let $\hat{X} (\u{\lambda})$ denote
the unique subquotient of $X(\u{\lambda})$ which contains the unique
irreducible $\mathfrak{t}$-submodule with extreme weight
$\u{\lambda}|_\mathfrak{t}$. 
\end{definition}

Note that for $\u{\lambda} = (\lambda, \lambda')$, the norm of
$\u{\lambda}\mid_\mathfrak{t}$ is equal to $||\lambda + \lambda'||$. 

\begin{lemma}\label{chap10:lem10.4}
For all $s \in \mathfrak{w}$, $\hat{X}(\u{\lambda})$ and $\hat{X}(s\u{\lambda})$ are  isomorphic. 
\end{lemma}

\begin{proof}
By (\ref{chap8:prop8.7}), $X(\u{\lambda})$ and $X(s\u{\lambda})$ have the same
Jordan-H\"older factors. 
\end{proof}

\begin{prop}\label{chap10:prop10.5}
Assume $\u{\lambda} \in \mathfrak{L}$ and satisfies (\ref{chap6:subsec6.5}). Then
$Z(\u{\lambda})$ and $\hat{X} (\u{\lambda})$ are isomorphic. Moreover, if
$\u{\lambda}$ and $\u{\mu}$ lie in the same
$\mathfrak{w}_\mathfrak{t}$-orbit and both satisfy (\ref{chap6:subsec6.5}), then
$Z(\u{\lambda})$ and $Z(\u{\lambda})$  are isomorphic. 
\end{prop}

\begin{proof}
The second assertion follows from the first by  (\ref{chap10:lem10.4}). Using Theorem
\ref{chap9:thm9.1} we now prove the first assertion by induction on the length of a
Jordan-H\"older series for $M(\lambda)$. If $M(\lambda)$ is
irreducible then by (\ref{chap9:thm9.1}), $\hat{X}(\u{\lambda}) = X(\u{\lambda})$ and
the proof is complete. Assume $M(\u{\lambda})$ is reducible. By (\ref{chap9:thm9.1}),
$Z(\u{\lambda})$ occurs as a Jordan-H\"older factor of
$X(\u{\lambda})$; and so, we need only show that $Z(\u{\lambda})$
contains the $\mathfrak{t}$-submodule with extreme weight
$\u{\lambda}\mid_\mathfrak{t}$.  
\end{proof}

Any Jordan-H\"older factor of $X(\u{\lambda})$ has by (\ref{chap9:thm9.1}) the form
$\tau L (\mu, \lambda')$ where $L(\mu)$ is a factor of
$M(\lambda)$. If $\mu \neq \lambda$, then by (\ref{chap10:coro10.2}), $||\mu +
\lambda'|| > ||\lambda + \lambda'||$. Since $||\mu + \lambda'||$ is
the smallest norm of an extreme weight of any $\mathfrak{t}$-submodule
of $X(\mu, \lambda')$ (cf. (\ref{chap8:coro8.4})), the $\mathfrak{t}$-submodule with
extreme weight\pageoriginale $\u{\lambda} \mid_\mathfrak{t}$ does not
occur in $\tau M (\mu, \lambda')$ or in $\tau L(\mu,
\lambda')$. Therefore, this $\mathfrak{t}$-submodule occurs in
$Z(\u{\lambda})$ and the proof is complete. 

\begin{prop}\label{chap10:prop10.6}
Let $\u{\lambda} = (\lambda, \lambda')$ and assume $M(\lambda')$ is
irreducible. Then $\tau L(\u{\lambda})$ is irreducible or zero
according as $\u{\lambda}$ satisfies (\ref{chap6:subsec6.5}) or not. 
\end{prop}

\begin{proof}
If $\u{\lambda}$ satisfies (\ref{chap6:subsec6.5}), then by
(\ref{chap10:prop10.5}), $\tau 
L(\u{\lambda})$ is irreducible. If $\u{\lambda}$ does not satisfy
(\ref{chap6:subsec6.5}) then by (\ref{chap9:prop9.14}), $\tau
L(\u{\lambda})$ is zero.  
\end{proof}

\begin{prop}\label{chap10:prop10.7}
Every irreducible admissible $(\mathfrak{g}, \mathfrak{t})$-module is
isomorphic to some $Z(\u{\lambda})$ where $\u{\lambda} \in
\mathfrak{L}$ and $\u{\lambda}$ satisfies (\ref{chap6:subsec6.5}). 
\end{prop}

\begin{proof}
By the subquotient theorem (\ref{chap8:thm8.6}) combined with
(\ref{chap8:prop8.7}) and (\ref{chap9:thm9.1}), if $A$ 
is an irreducible admissible $(\mathfrak{g}, \mathfrak{t})$-module
then $A$ is isomorphic to $\tau L(\u{\lambda})$ for some $\u{\lambda}$
with $M(\lambda')$ irreducible. Lemma (\ref{chap10:prop10.6}) implies that
$\u{\lambda}$ satisfies (\ref{chap6:subsec6.5}). 
\end{proof}

Recall Definition (\ref{chap6:def6.6}). 

\begin{theorem}\label{chap10:thm10.8}
The map $\u{\lambda} \mapsto Z(\u{\lambda})$ induces a bijection of
$\mathfrak{w}_\mathfrak{t}$-orbits in $\mathfrak{L}$ and the set of
equivalence classes of irreducible admissible $(\mathfrak{g},
\mathfrak{t})$-modules. 
\end{theorem}

\begin{proof}
By convention (cf. \S\ \ref{sec6}) we choose any element of the
$\mathfrak{w}_\mathfrak{t}$-orbit of $\u{\lambda}$ which satisfies
(\ref{chap6:subsec6.5}). Then by (\ref{chap10:prop10.5}), the
equivalence class of $Z(\u{\lambda})$ is 
independent of this choice. Moreover, by (\ref{chap10:prop10.7}), the induced map in
the theorem is surjective. We now prove injectivity of the map. 

Let\pageoriginale $\u{\mu}, \u{\lambda} \in\mathfrak{L} $ both
satisfying (\ref{chap6:subsec6.5}) and assume $Z(\u{\lambda})$ and $Z(\u{\mu})$ are
isomorphic. They must have the same infinitesimal character so there
are $t, s \in \mathfrak{w}_0$ with $\mu = t \lambda$ and $\mu' =
s\lambda'$. Let $L_1 = \{\alpha \in \Delta_0 \mid \re \lambda'_\alpha
< 0\}$, $L_0 = \{\alpha \in\Delta_0 \mid \re \lambda'_\alpha =
0\}$. Then $L = L_1 \cup L_0$ is the set of roots of a parabolic
subalgebra and any positive root system $L^+_0$ of $L_0$ gives a
positive system $L_1 \cup L^+_0$ of $\Delta_0$. Using this, let $R$ be
any positive system of $\Delta_0$ with $\alpha \in R \Rightarrow \re \lambda'_\alpha  \leq 0$, and $\lambda'_\alpha
=0\Rightarrow \Iim \lambda'_\alpha \leq 0$. Let $N_0 = \{\alpha \in
\Delta_0 \mid \lambda'_\alpha = 0\}$ and $N_1 = \{\alpha \in R \mid
\alpha \not\in N_0\}$. Then $N = N_1 \cup N_0$ is the set of roots of
another parabolic subalgebra of $\mathfrak{g}_0$ and any positive
root system $N^+_0$ of $N_0$ gives a positive system $N_1 \cup N^+_0$
of $\Delta_0$. Let $B$ denote the set of simple roots of $R$ and write
$B = B' \cup B^\sim$ where $B'$ is the set of simple roots of $R \cap
N_0$. Let $B^\sim = \{\alpha_1 , \ldots, \alpha_t\}$ and let
$\{\gamma_1, \ldots , \gamma_t\}$ be the set of dual roots to $B^\sim$
orthogonal to $B'$; i.e., $\langle \gamma_i, \alpha_j\rangle =
\delta_{ij}$, $1 \leq i$, $j \leq t$, and $\langle \gamma_i, B'\rangle
= 0$. Put $\lambda^\sim = \lambda' - a \sum\limits_{1 \leq i \leq t}
\gamma_i$, $a \in \mathbb{N}$, $a >> 0$. Since $\lambda'$ and
$\lambda^\sim$ have the same stabilizer in $\mathfrak{w}_0$, the
Zuckerman functors applied to $Z(\u{\lambda})$ and $Z(\u{\mu})$ give
the isomorphism $Z(\lambda, \lambda^\sim) \simeq Z(\mu, s
\lambda^\sim)$. 

Let $T$ be the positive system $-N_1 \cup (N_0 \cap -P_0)$. Then by
(\ref{chap6:subsec6.5}) and the choice of $\lambda^\sim$, $\lambda + \lambda^\sim$ is
$T$-dominant. Also, by (\ref{chap6:subsec6.5}), $\mu$ is $(sN_0 \cap -P_0)$-dominant;
and so, if $T'= - N_1 \cup (N_0 \cap - s^{-1}P_0)$ then $s^{-1} \mu +
\lambda^\sim$ is $T'$-dominant. However, both $\lambda + \lambda^\sim$
and $\mu + s \lambda^\sim$ are extreme weights of the minimal
$\mathfrak{t}$-submodule of $Z(\lambda, \lambda^\sim)$. Therefore,
they lie in the same $\mathfrak{w}_0$-orbit; i.e., $\lambda+
\lambda^\sim = r(\mu + s\lambda^\sim)$. Thus we have $\lambda +
\lambda^\sim = rs (s^{-1} \mu + \lambda^\sim)$. However $T$ and $T'$
are positive system contained in $-N$ and $\lambda + \lambda^\sim$ is
$T$-dominant while $s^{-1} \mu + \lambda^\sim$ is
$T'$-dominant. Therefore $rs$ must be an element of the Weyl group of
$N_0$. Thus $rs\lambda' = \lambda'$; and so, $\lambda+ \lambda' = r\mu
+ \lambda'$. Then\pageoriginale $\lambda = r\mu$ and, from above, $\mu
= t\lambda$. So $r^{-1} \lambda = t\lambda$ and $r^{-1} \lambda' =
s\lambda'$. This proves $\u{\lambda}$ and $\u{\mu}$ lie in the
same $\mathfrak{w}_\mathfrak{t}$-orbit; and this completes the proof. 
\end{proof}

For our special case of $(\mathfrak{g}, \mathfrak{t})$-modules, we can
now give a proof of Theorem \ref{chap5:thm5.2} (iii) and (iv) for this category
$\mathfrak{U}$ of admissible $(\mathfrak{g},
\mathfrak{t})$-modules. Let notation be as in (\ref{chap5:thm5.2}) except that we
shall replace $\lambda$ by $\u{\lambda}$, $\mu$ by $\u{\mu}$ and write
$\u{\lambda} = (\lambda, \lambda')$, $\u{\mu} = (\mu, \mu')$. 

\begin{prop}\label{chap10:prop10.9}
Assume $\u{\lambda} + \u{\mu}$ satisfies (\ref{chap6:subsec6.5}). The conditions
\begin{itemize}
\item[{\rm (i)}] $\u{\lambda}$ satisfies (\ref{chap6:subsec6.5})

\item[{\rm (ii)}] If $Q = \{\alpha \in P_0 \mid \lambda_\alpha = 0\}$
  then $\lambda + \mu$ is -$Q$-dominant
\end{itemize}
are necessary and sufficient for $Z(\u{\lambda})$ to be nonzero and to
have $\psi Z(\u{\lambda} + \u{\mu}) = Z(\u{\lambda})$. If either
condition does not hold then $\psi Z(\u{\lambda} + \u{\mu}) = 0$. 
\end{prop}

\begin{proof}
Put $A = \psi Z(\u{\lambda} + \u{\mu})$. If (ii) does not hold then by
(\ref{chap5:prop5.3}), $\psi L(\u{\lambda} + \u{\mu}) = 0$. Since $\psi$ and $\tau$
commute, $A = \tau (0) = 0$. If (ii) holds but (i) does not then, by
(\ref{chap5:prop5.3}), $A \simeq \tau L(\u{\lambda})$ and by
(\ref{chap9:prop9.14}) (whose proof does 
not rely on (\ref{chap5:thm5.2}) (iii) or (iv)), $\psi L(\u{\lambda}) = 0$. This
shows $A =0$ unless both conditions hold. 
\end{proof}

Now assume both conditions. Then $A \simeq \tau L(\u{\lambda})$, by
(\ref{chap5:prop5.3}). To complete the proof we shall show that $\tau L(\u{\lambda})$
contains with multiplicity one the $\mathfrak{t}$-submodule with
extreme weight $\u{\lambda}\mid_\mathfrak{t}$. We know $\tau
M(\u{\lambda}) = X(\u{\lambda})$ and by (\ref{chap8:coro8.4}), $X(\u{\lambda})$
contains this $\mathfrak{t}$-module with multiplicity one. Using the
exactness of $\tau$ on $\mathscr{O} \otimes M(\lambda')$, for some
$\u{\xi}$ below or equal to $\u{\lambda}$ in the Bruhat ordering,
$\tau L(\u{\xi})$ contains this $\mathfrak{t}$-module. But $\tau
L(\u{\xi})$ is a quotient of $X(\u{\xi})$; and so, by (\ref{chap8:coro8.4}), $||\xi +
\xi'|| \leq ||\lambda + \lambda'||$, where $\u{\xi} = (\xi,
\xi')$. Since $\u{\lambda}$ satisfies (\ref{chap6:subsec6.5}) and $\u{\xi}$ is below or
equal to $\u{\lambda}$, $\xi' =\lambda'$. Then applying (\ref{chap10:coro10.2}), we
have $\u{\xi} = \u{\lambda}$. This proves $\tau L(\u{\lambda})$ is not
zero and completes the proof. 

\begin{prop}\label{chap10:prop10.10}
\begin{itemize}
\item[{\rm(i)}] For\pageoriginale irreducible $A \in \mathfrak{U}$, $\psi A$ is
  either irreduci\-ble or zero. 

\item[{\rm(ii)}] If $A \in \mathfrak{U}_\lambda$ is irreducible then
  there exists $B \in \mathfrak{U}_{\lambda+ \mu} $ irreducible with
  $\psi B = A$. 
\end{itemize}
\end{prop}

\begin{proof}
To avoid a circular argument we shall make use of the preceding
results only for regular parameter. First, we assume $\u{\lambda} +
\u{\mu}$ is regular. We may assume that $\u{\lambda} + \u{\mu}$
satisfies (\ref{chap6:subsec6.5}) and also, by
(\ref{chap10:prop10.7}), that $A \simeq 
Z(\u{\lambda}+\u{\mu})$. Now by the proof of (\ref{chap10:prop10.9}), $\psi A$ is
either zero or contains a $\mathfrak{t}$-submodule with multiplicity
one. From the Zuckerman article \cite{key38}, $\psi A$ is primary and
thus $\psi A$ is irreducible. For $B \in \mathfrak{U}_\lambda$ and
irreducible, by  \cite{key38}, $\psi \varphi B$ is primary with
irreducible factors isomorphic to $B$. $\psi$ is exact so choose any
Jordan-H\"older factor $A$ of $\varphi B$ such that $\psi A \neq
0$. Then by (i), $\psi A \simeq B$. This completes the proof for
$\u{\lambda} + \u{\mu}$ regular. This case of $\u{\lambda} + \u{\mu}$
regular is sufficient for all applications of (\ref{chap10:prop10.10})
in the previous 
sections of these notes. Now, for $\u{\lambda} + \u{\mu}$ general, we
may apply (\ref{chap10:prop10.7}) in the above argument. This argument now proves
(\ref{chap10:prop10.10}) without restriction on $\u{\lambda} + \u{\mu}$. 
\end{proof}

\begin{remark}\label{chap10:rem10.11}
By combining (\ref{chap10:thm10.8}) and (\ref{chap10:prop10.9}) we
have actually stre\-ngthened the 
results asserted in (\ref{chap5:thm5.2}). For all parameters $\u{\lambda}$ and
$\u{\lambda} + \u{\lambda}$, (\ref{chap10:prop10.9}) describes the image $\psi (A)$
with $A$ irreducible. 
\end{remark}

