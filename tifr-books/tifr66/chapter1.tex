
\chapter{Introduction and summary of results}\label{sec1}

\setcounter{pageoriginal}{1}
Within\pageoriginale the theory of representations of Lie groups, the
case of compact groups is distinguished by the simplicity and
completeness of the theory. The fundamental results here are classical
and go back to the work of H. Cartan and H. Weyl. For the noncompact
case, the representation theory is far from complete although in this
case an elaborate theory does exist. The foundations of this theory
were developed by Harish-Chandra and the contributions to this field
during the last twenty-five years have been extensive. The subcase of
the noncompact case of complex Lie groups is of interest due to the
special simplicity of the theory for these groups and the unusual
parallels which exist between this theory and the theory for the
compact case. The representation theory of complex semisimple Lie
groups has a long history beginning in 1950 with the fundamental work
of Gel'fand and Naimark \cite{key20}.  Aspects of this history are
described in the introduction to the expository article \cite{key9} by
Duflo and also the survey article \cite{key37} by Zelobenko.

Let $G$ be a connected complex semisimple Lie group. One of the
fundamental questions of representation theory is the description of\break
$\varepsilon (G)$, the infinitesimal equivalence classes of
irreducible representations of $G$. Although the parametrization of
$\varepsilon(G)$ is essentially the same as in the compact case the
proof of this fact lies much deeper in this case. The first result on
the parametrization of $\varepsilon (G)$ was established by
Parthasarathy, Ranga Rao and Varadarajan \cite{key30} for the subset
of $\varepsilon(G)$ of spherical classes. Zelobenko \cite{key36} in
1969 gave the first parametrization of $\varepsilon (G)$ and in 1973
Langlands \cite{key27} described a program for general $G$ which in
this case specialized to the Zelobenko classification.

There\pageoriginale are four questions related to any complete
description of $\varepsilon (G)$ which naturally arise.

\smallskip
\noindent{\textbf{Question A:}} For any element $\pi \in \varepsilon
(G)$, what is the formula for the (distribution) character of the
class $\pi$ ?

\smallskip
\noindent{\textbf{Question B:}} For any element $\pi \in \varepsilon
(G)$,  the restriction of $\pi$ to a maximal compact subgroup of $G$
splits into the direct sum of irreducible finite dimensional
representations. What are the components and multiplicities of this
direct sum ?

\smallskip
\noindent{\textbf{Question C:}} For compact groups all indecomposable
representations are irreducible; for complex groups this is not
so. What is the category of all representations of $G$ ?

\smallskip
\noindent{\textbf{Question D:}} What is the subset of $\varepsilon
(G)$ of unitarizable classes ?


Although $\varepsilon (G)$ has been determined by Zelobenko, his
description does not give a solution to any of these questions. In
these notes we give an alternate description of $\varepsilon (G)$
which will at the same time answer the above questions in terms of
data associated with highest weight modules. The method applied will
be to use functor introduced in \cite{key15} to relate irreducible
highest weight modules to elements of $\varepsilon (G)$.

We now describe the results of these notes in some detail. In section
two the definition and basic properties of the completion functors are
summarized. The application of completion functors is the fundamental
technique used here to construct modules for a Lie algebra. In section
three invariant pairings and forms are introduced. The main result in
this section is a splitting theorem which we now describe. Let
$\mathfrak{L}$ be a Lie algebra with  subalgebra\pageoriginale
$\mathfrak{t}$ and let $\mathcal{C}$ be a category of
$\mathfrak{L}$-modules. The category $\mathcal{C}$ is called
$\mathfrak{t}$-semisimple if every short exact sequence of
$\mathfrak{L}$-modules splits as a $\mathfrak{t}$-module sequence. Let
$m$ be a semisimple Lie algebra over $\mathbb{C}$, the complex
numbers, and let $\mathfrak{h}$ be a Cartan subalgebra (CSA) and $\mathfrak{b}$ a
Borel subalgebra of $m$ with $\mathfrak{h} \subset \mathfrak{b}$. For any Lie
algebra $\mathfrak{g}$, let $U(\mathfrak{g})$ denote the universal
enveloping algebra. By the category $\mathscr{O}$ for $m$ we mean the
full subcategory of $m$-modules which are (i) finitely generated, (ii)
weight modules for $\mathfrak{h}$ and (iii) $U(\mathfrak{b})$-locally
finite. This category was introduced by Bernstein, Gel'fand and
Gel'fand \cite{key2} in their study of irreducible highest weight
modules and Verma modules. Let $\mathfrak{h}^*$ denote the algebraic
dual of $\mathfrak{h}$ and for $\lambda \in \mathfrak{h}^*$, let
$\mathbb{C}_\lambda$ denote the one dimensional $\mathfrak{b}$-module
where $\mathfrak{h}$ acts by $\lambda$. Let $\delta$ denote half the
sum of the roots of $\mathfrak{b}$ and let $M(\lambda)$ denote the
Verma module with highest weight $\lambda-\delta$; i.e., $M(\lambda)
\simeq U(m) \bigotimes\limits_{U(\mathfrak{b})} \mathbb{C}_{\lambda -
  \delta}$. Following the notation in Dixmier \cite{key8}, let
$L(\lambda)$ denote the unique irreducible quotient of
$M(\lambda)$. Now fix an irreducible Verma $m$-module $M$, and let
$\mathscr{O} \otimes M$ denote the category of $m \times m$-modules of
the form $A \otimes M$, where $A$ is an  object in
$\mathscr{O}$. Using the theory of invariant forms we obtain:
Proposition \ref{chap3:prop3.9}: If $\mathfrak{t}$ denotes the
diagonal subalgebra of 
$m \times m$, then $\mathscr{O} \otimes M$ is a
$\mathfrak{t}$-semisimple category of $m \times m$-modules. This
proposition will play a central role in the remaining sections of
these notes. 

In section four, the definitions of a lattice of modules and the
functor $\tau$ are recalled from \cite{key15}, and their basic
properties are summarized. Section five includes a description of the
Zuckerman functors which allow the translation from modules with one
infinitesimal character to modules with another. 


Section\pageoriginale six begins the study of modules for a complex
semisimple Lie algebra. We now introduce the notation necessary to
complete the description of the main results.


Let $\mathfrak{g}_0$ be a complex semisimple Lie algebra with CSA
$\mathfrak{h}_0$, roots $\Delta_0$ and positive system of roots
$P_0$. Let $\mathfrak{w}_0$ denote the Weyl group of $\Delta_0$ and
let $n_0$ (resp. $n^-_0$) denote the nilpotent Lie subalgebra spanned
by the positive (resp. negative) root spaces of $\mathfrak{g}_0$. The
subalgebra $\mathfrak{b}_0 = \mathfrak{h}_0 \oplus n_0$ is a Borel
subalgebra of $\mathfrak{g}_0$. Let $\langle \cdot , \cdot \rangle$
denote the Killing form on $\mathfrak{h}^*_0$ and for $\lambda \in
\mathfrak{h}^*$, $\alpha \in \Delta_0$, let $\lambda_\alpha = 2
\langle \lambda, \alpha \rangle / \langle \alpha, \alpha \rangle$. Let
$\mathbb{Z}$ denote the integers and $\Delta_\lambda = \{\alpha \in
\Delta_0 \mid    \lambda_\alpha \in \mathbb{Z}\}$. $\Delta_\lambda$ is a
root system and $P_\lambda = P_0 \cap \Delta_\lambda$ is a positive
system or roots for $\Delta_\lambda$. Let $\mathfrak{w}_\lambda$ be
the Weyl group for $\Delta_\lambda$. Let $\delta_0$ equal half the sum
of the roots $P_0$. If $\lambda_\alpha \geq 0$ for all $\alpha \in
P_0$, we call $\lambda$ $P_0$-dominant.

By deleting the subscript $0$, we denote the product of the algebra
with itself; i.e., $\mathfrak{g}=\mathfrak{g}_0 \times \mathfrak{g}_0$,
$\mathfrak{h} = \mathfrak{h}_0 \times \mathfrak{h}_0$, etc. Now
$\mathfrak{h}^*$ and $\mathfrak{h}^*_0 \times \mathfrak{h}^*_0$ are
identified by the formula: $(\lambda, \lambda') (H, H') = \lambda (H)
+ \lambda' (H')$, $\forall \lambda$, $\lambda' \in
\mathfrak{h}^*_0$. $H, H' \in \mathfrak{h}_0$. Then $P= (P_0 \times 0)
\cup (0 \times P_0)$ is a positive system of roots for $\Delta =
(\Delta_0 \times 0) \cup (0 \times \Delta_0)$. The Weyl group
$\mathfrak{w}$ of $\Delta$ is the product $\mathfrak{w}_0 \times
\mathfrak{w}_0$. Let $\mathfrak{L}$ denote the subset of
$\mathfrak{h}^*$ of elements $(\lambda, \lambda')$ with $\lambda +
\lambda'$ integral (i.e., $\lambda_\alpha + \lambda'_\alpha \in
\mathbb{Z}$ for all $\alpha \in P_0$). Let $\delta = (\delta_0,
\delta_0)$. 

Let $\mathfrak{t}$ denote the diagonal subalgebra of $\mathfrak{g}$
and for any subalgebra of $\mathfrak{g}$, let a subscript
$\mathfrak{t}$ denote the intersection with $\mathfrak{t}$;
e.g. $\mathfrak{h}_{\mathfrak{t}} = \mathfrak{h} \cap \mathfrak{t}$,
$n_{\mathfrak{t}} = n \cap \mathfrak{t}$. For convenience we also
write $\mathfrak{t} = \mathfrak{h}_{\mathfrak{t}}$. Let
$\Delta_{\mathfrak{t}}$ equal the roots of $(\mathfrak{t},
\mathfrak{t})$ and let $P_{\mathfrak{t}} = \{(\alpha, 0) \mid_{\mathfrak{t}} \;
\mid \alpha \in P_0\}$ (here $\mid_{\mathfrak{t}}$ denotes the
restriction to $\mathfrak{t}$). One checks easily that
$P_{\mathfrak{t}}$ is a positive system of roots\pageoriginale for
$\Delta_{\mathfrak{t}}$. Let $\mathfrak{w}_{\mathfrak{t}}$ denote the 
diagonal subgroup of $\mathfrak{w}$. $\mathfrak{w}_\mathfrak{t}$ acts
on $\mathfrak{t}^*$ and can be identified with the Weyl group of
$\Delta_\mathfrak{t}$. The subset $\mathfrak{L}$ of $\mathfrak{h}^*$
is stable under $\mathfrak{w}_\mathfrak{t}$; and so, for $(\lambda,
\lambda') \in \mathfrak{h}^*$ we let $[\lambda, \lambda']$ denote the
$\mathfrak{w}_\mathfrak{t}$-orbit of $(\lambda, \lambda')$. We write
$\mathfrak{L}/\mathfrak{w}_\mathfrak{t}$ for the set of
$\mathfrak{w}_\mathfrak{t}$-orbits in $\mathfrak{L}$.

A $\mathfrak{g}$-module $M$ will be called an admissible
$(\mathfrak{g}, \mathfrak{t})$-module of (i) $M$ is finitely
generated, (ii) $M$ is $U(\mathfrak{t})$-locally finite, and (iii) for
each $\mathfrak{t}$-module $E$, $\Hom_{\mathfrak{t}}(E,M)$ is finite
dimensional. If $G$ is the complex connected and simply connected Lie
group with Lie algebra $\mathfrak{g}_0$, then the theory of
representations for $G$ is equivalent to the theory of admissible
$(\mathfrak{g}, \mathfrak{t})$ modules. For irreducible
representations this equivalence is implied by Harish-Chandra's
subquotient theorem. Refinements of this theorem have established the
equivalence for all (quasi-simple) representations of $G$
(cf. \cite{key29}). With this equivalence in mind, we offer solutions
to the questions above rephrased in the category of admissible
$(\mathfrak{g}, \mathfrak{t})$-modules. We begin with a description of
the irreducible admissible $(\mathfrak{g}, \mathfrak{t})$-modules.

Let $\tau$ denote the lattice functor for the algebra $\mathfrak{t}$
determined by the positive system $P_\mathfrak{t}$ (cf. \S
\;\ref{sec4}). For $\u{\lambda} = (\lambda, \lambda') \in
\mathfrak{L}$, we say $\u{\lambda}$ satisfies (\ref{chap6:subsec6.5})
if (i) $\lambda'$ is -$P_\lambda$-dominant and (ii) if $\alpha \in
P_\lambda$ and $\lambda'_\alpha =0$ then $\lambda_\alpha \in -
\mathbb{N}$ ($\mathbb{N} = \{0,1,2, \ldots\}$). Each orbit $[\lambda,
  \lambda']$ contains one or more elements which satisfy
(\ref{chap6:subsec6.5}). Now for $\u{\lambda} \in \mathfrak{L}$
satisfying (\ref{chap6:subsec6.5}) define a $\mathfrak{g}$-module $Z
(\u{\lambda})$ to be the image under $\tau$ of the irreducible highest
weight module $L(\u{\lambda})$; i.e., $Z(\u{\lambda}) = \tau
L(\u{\lambda})$. The basic result of these notes is:


\medskip
\noindent{\textbf{Theorem 10.8.}}
\textit{The\pageoriginale map $\u{\lambda} \mapsto Z(\u{\lambda})$ induces a
bijection of $\mathfrak{L}/\mathfrak{w}_\mathfrak{t}$ onto the
equivalence classes of irreducible admissible $(\mathfrak{g},
\mathfrak{t})$-modules}.

The starting point for both the proof of Theorem 10.8 as well as a
detailed description of the modules $Z(\u{\lambda})$ is the connection
between Verma modules and principal series modules of
$\mathfrak{g}$. Let $Q = (P_0 \times 0) \cup (0 \times - P_0)$. Then
$Q$ is a positive system for $\Delta$. Let $\delta_Q$ equal half the
sum of the roots in $Q$ and let $\bar{M}(\lambda)$ denote the
$\mathfrak{g}$-Verma module with $Q$-highest weight $\lambda -
\delta_Q$, $\lambda \in \mathfrak{h}^*$. For $\lambda \in
\mathfrak{h}^*$, let $X(\lambda)$ denote the submodule of
$U(\mathfrak{t})$-locally finite vectors in the algebraic dual of
$\bar{M}(-\lambda)$. The $\mathfrak{g}$-modules $X(\lambda)$, $\lambda
\in \mathfrak{L}$, are called the principal series modules of
$\mathfrak{g}$. These modules are isomorphic to the
$\mathfrak{g}$-modules of $K$-finite vectors of the principal series
of $G$ (here $K$ is the analytic subgroup of $G$ with complexified Lie
algebra $\mathfrak{t}$). Section eight includes the definition and
basic properties of the principal series modules. The correspondence
between Verma modules and principal series modules is given as:

\medskip
\noindent{\textbf{Theorem 9.1. }}
\textit{Let $\u{\lambda} = (\lambda, \lambda') \in \mathfrak{L}$ and
  assume $M(\lambda')$ is an irreducible Verma module. Then $\tau M
  (\u{\lambda})$ and $X(\u{\lambda})$ are isomorphic
  $\mathfrak{g}$-modules.} 

By our remarks above regarding Proposition \ref{chap3:prop3.9}, $\mathscr{O} \otimes
M(\lambda')$ is a $\mathfrak{t}$-semisimple category of
$\mathfrak{g}$-modules; and so, $\tau$ is exact on this category. This
exactness and Theorem 9.1 give character formulae for the modules $Z
(\u{\lambda})$. Let $ch$ $A$ denote the formal character of the
$\mathfrak{g}_0$-module $A$ in $\mathscr{O}$ and let $E(\u{\mu})$
denote the distribution character of the principal series module
$X(\u{\mu})$, $\u{\mu} \in \mathfrak{L}$, and $\Theta (\u{\mu})$ the
distribution character of $Z(\u{\mu})$.

\medskip
\noindent{\textbf{Proposition 9.15.}}
\textit{Let \pageoriginale $\u{\lambda} = (\lambda, \lambda') \in
\mathfrak{L}$ and assume $\u{\lambda}$ satisfies
(\ref{chap6:subsec6.5}). Fix integers $m(s\lambda)$, $s \in
\mathfrak{w}_\lambda$, such that $chL(\lambda) = \sum\limits_{s
  \lambda \in \mathfrak{w}_\lambda \cdot \lambda} m(s\lambda) $
$chM(s\lambda)$. Then} 
$$
\Theta (\u{\lambda}) = \sum\limits_{s \lambda \in \mathfrak{w}_\lambda
\cdot \lambda} m (s\lambda)  E (s\lambda,\lambda').
$$

This is the solution to Question $A$ given in terms of data associated
with the category $\mathscr{O}$. Using this result and the known
$\mathfrak{t}$-module structure of the principal series modules
(cf. (\ref{chap8:subsec8.3})), Proposition 9.15 also gives an answer
to Question B. However, we will obtain an answer to Question B by
specializing a somewhat stronger result. 

In the classical case of compact semisimple Lie groups, the weight
space structure of the irreducible representations can be obtained
directly from the Bernstein, Gel'fand and Gel'fand resolution of
finite dimensional modules by sums of Verma modules \cite{key1}. We
approach the $\mathfrak{t}$-structure of $Z(\u{\lambda})$ from a
similar point of view. A resolution of $Z(\u{\lambda})$ is given in
terms of the modules in the lattice above $L(\u{\lambda})$.

The simple reflections generate the Weyl group
$\mathfrak{w}_\mathfrak{t}$. An expression of  $s$ as a product of
simple reflections is called reduced if the number of simple
reflections is a minimum. This minimum number is called the length of
$s$ and is denoted $\ell(s)$. If $d = \card  P_\mathfrak{t}$ and $s
\in \mathfrak{w}_\mathfrak{t}$, then $0 \leq \ell (s) \leq
d$. Moreover, $\ell(s) = 0$ implies $s = 1$ and $\ell(s) =d$ implies
$s$ is the unique element with $sP_\mathfrak{t} = -
P_\mathfrak{t}$. The resolution of $Z(\u{\lambda})$ is included as a
special case of the general result: 

\medskip
\noindent{\textbf{Proposition 7.1. }}
Let\pageoriginale $\lambda \in \mathfrak{h}^*_0$ and assume the
$\mathfrak{g}_0$-Verma module $M(\lambda)$ is irreducible. Let $B$ be
a $\mathfrak{g}$-module in $\mathscr{O} \otimes M(\lambda)$ with
integral $\mathfrak{t}$-weights. Let $B_s$, $s \in
\mathfrak{w}_\mathfrak{t}$, be a lattice of modules above $B$ and for
$0 \leq i \leq d$, put $\mathscr{B}_i = \sum\limits_{\ell(s)=i}
B_s$. Then there is a resolution
$$
0 \to \mathscr{B}_d \to \ldots \to \mathscr{B}_0 \to \tau B \to 0.
$$

From this resolution and standard properties of lattices we obtain a
$\mathfrak{t}$-multiplicity formula for the modules
$Z(\u{\lambda})$. Let $\mu \in \mathfrak{h}^*_0$ be $P_0$-dominant
integral and put $\mu_1 = (\mu,0) \mid_t$. Let $F$ denote the
irreducible finite dimensional $ \mathfrak{t}$-module with extreme
weight $\mu_1$. For any $\nu \in  \mathfrak{h}^*_0$, let a subscript
$\nu$ denote the weight space of weight $\nu$. By specializing
Corollary \ref{chap7:coro7.12} we have:

\begin{coro*}
Let $\u{\lambda} = (\lambda, \lambda') \in  \mathfrak{L}$ and assume
$\u{\lambda}$ satisfies (\ref{chap6:subsec6.5}). Then
$$
\dim \Hom_\mathfrak{t} (F, Z(\u{\lambda})) =\sum\limits_{s \in
  \mathfrak{w}_0} (-1)^{d-\ell(s)} \dim L(\lambda)_{ s(\mu + \delta_0)
- \lambda'}. 
$$
\end{coro*}

So far our theme has been the reduction of questions for admissible $(
\mathfrak{g},  \mathfrak{t})$-modules to related questions for highest
weight modules. In section eleven we invert this theme and use the
correspondence between highest weight and admissible $( \mathfrak{g},
\mathfrak{t})$-modules to obtain certain skew-symmetry properties for
characters of irreducible highest weight modules. Fix $\lambda \in
\mathfrak{h}^*_0$ and write $ch$ $L(\lambda) = \sum m (s\lambda) ch M
(s\lambda)$ with the sum over the $ \mathfrak{w}_\lambda$ orbit of
$\lambda$ ($ \mathfrak{w}_\lambda$ is the Weyl group of
$\Delta_\lambda$). In Proposition 11.2 we assert the following\pageoriginale
skew-symmetry property: Let $\alpha$ a simple root of $P_\lambda$ and
assume $\lambda_\alpha \in \mathbb{N}^*$ (positive integers). Then
$m(s_\alpha s \lambda) + m (s\lambda)=0$ for all $s\lambda$. This
skew-symmetry property follows easily from a determination of those
$\u{\lambda} \in  \mathfrak{L}$ for which $\tau L(\u{\lambda})$ equals
zero (cf. Proposition 10.6).

Section twelve contains a review of some standard concepts from
homological algebra, as well as, a somewhat special definition of
certain derived functors. Let $\mathscr{O}_{ \mathfrak{t}}$ denote the
category $\mathscr{O}$ for the data $( \mathfrak{t},  \mathfrak{t}, P_
\mathfrak{t})$ and let $n$ denote the category of $
\mathfrak{g}$-modules whose underlying $ \mathfrak{t}$-modules lie in
$ \mathscr{O}_\mathfrak{t}$. Fix a set $\psi \subseteq t^*$ of
dominant integral elements. For any $ \mathfrak{t}$-module $L$, put
$L'$ equal to the span of the $n_ \mathfrak{t}$-invariants of weight
$\mu$, $\mu \in \psi$. Then, a complex
$$
\ldots A_i \to \ldots \to A_0 \to A \to 0
$$
in the category $n$ is called a $\psi$-resolution if (i) $A_i$ is
projective, $i \in \mathbb{N}$ (ii) $A_i$ is generated as a $
\mathfrak{g}$-module by $A'_i$, $i \in \mathbb{N}$, and (iii) $0 \to
A'_i \to \ldots \to A'_0 \to A' \to 0$ is exact. For $\mu \in
\mathfrak{t}^*$ and $P_ \mathfrak{t}$-dominant integral, let $M(\mu)$
denote the $ \mathfrak{t}$-Verma module. If $r \in  \mathfrak{w}_
\mathfrak{t}$ is the element of maximal length, then $M(r\mu)
\subseteq M(\mu)$; and moreover, this subspace is unique. For $\nu \in
 \mathfrak{t}^*$, put $U(\nu) = U( \mathfrak{g})
 \bigotimes\limits_{U( \mathfrak{t})} M(\nu)$.  The inclusion
 $M(r\mu) \subseteq M(\mu)$ induces the inclusion $U(r\mu) \subseteq
 U(\mu)$, $\mu \in \psi$. The task of section twelve is to give a
 definition of an additive covariant functor $\sigma_0$ on $n$ having
 the property that, for $\mu \in \psi$, $\sigma_0 U(\mu) =
 U(r\mu)$. Next with projective resolutions replaced with
 $\psi$-projective resolutions, we definite the additive covariant
 functors $\sigma_i$, $i \in \mathbb{N}$, as the derived\pageoriginale
 functors of $\sigma_0$. The setting of this section is quite general
 with $ \mathfrak{t}$ a reductive Lie algebra and $ \mathfrak{g}$ an
 arbitrary Lie algebra.

In section thirteen we return to the setting of complex Lie algebras
and attempt to compute the values of $\sigma_i$ on the subcategory of
$n$ of admissible $( \mathfrak{g},  \mathfrak{t})$-modules. Fix
$\u{\lambda} = (\lambda, \lambda') \in  \mathfrak{L}$ and assume $\re
\lambda'_\alpha << 0 $, $\alpha \in P_0$. Let $\wedge \mathfrak{t}$
denote the exterior algebra of $\mathfrak{t}$ and let $r$ be the
element of $\mathfrak{w}_\mathfrak{t}$ of maximal length. Put $\psi =
\{r(s\lambda, \lambda')\mid_t + \xi | s\in \mathfrak{w}_0$ and $\xi$
is a weight of $\wedge \mathfrak{t}\}$. Let $\sigma_i$ be the derived
functors associated with $\psi$ as in section twelve. Let $\chi$ denote
the $Z(\mathfrak{g})$ character parametrized by the
$\mathfrak{w}$-orbit of $\u{\lambda}$ and let $\mathfrak{U}$ denote
the category of admissible $(\mathfrak{g}, \mathfrak{t})$-modules
having generalized infinitesimal character $\chi$ (i.e., $z- \chi (z)$
is locally nilpotent for all $z \in Z(\mathfrak{g})$). Let
$\mathfrak{B}$ denote the full subcategory of the category of
$\mathfrak{g}$-modules with objects which are finitely generated, have
precisely $L(s\lambda, \lambda')$, $s \in \mathfrak{w}_0$, as
irreducible objects, and are weight modules for $\mathfrak{t}$. The
basic result in this section is:

\medskip
\noindent{\textbf{Proposition 13.13.~}}
\textit{
Assume $\re \lambda'_\alpha << 0$, $\alpha \in P_0$. Then the lattice
functor $\tau$ gives a natural equivalence of categories; $\tau :
\mathfrak{B} \xrightarrow{\sim} \mathfrak{U}$. When restricted to
$\mathfrak{U}$, $\sigma_i$ is exact for $i \in \mathbb{N}$; and
$\sigma_i \equiv  0$ for $i \in \mathbb{N}^*$. Moreover, $\sigma_0 :
\mathfrak{U} \xrightarrow{\sim} \mathfrak{B}$ and is a natural inverse
to $\tau$.}


Using translation functors we obtain:

\medskip
\noindent{\textbf{Theorem 13.2.~}}
\textit{Let $\u{\lambda} = (\lambda, \lambda') \in \mathfrak{L}$ and
  assume $(\lambda')$ is irreducible and $\lambda'$ is regular. Then
  the lattice functor $\tau$ gives a natural equivalence;}
$$
\tau : \mathfrak{B} \xrightarrow{\sim} \mathfrak{U} .
$$\pageoriginale
The map $B \mapsto B/(0, n^-_0) \cdot B$ induces a natural equivalence
of $\mathfrak{B}$ onto the category of finitely generated
$\mathfrak{g}_0$-modules which are $U(\mathfrak{b}_0)$-locally finite,
have generalized $Z(\mathfrak{g}_0)$-character with orbit
$\mathfrak{w}_0 \cdot \lambda$ and have generalized
$\mathfrak{g}_0$-weights $\nu$ with $\nu + \lambda'$
integral. Therefore Theorem 13.2 gives an equivalence of the category
$\mathfrak{U}$ and a category of highest weight
$\mathfrak{g}_0$-modules. The reader may wish to compare this with a
similar equivalence of categories established by Bernstein, Gel'fand
and Gel'fand in \cite{key3}. The category studied in \cite{key3} is
the subcategory of $\mathfrak{U}$ of objects which have $1 \otimes
Z(\mathfrak{g}_0)$ characters instead of generalized characters and
the results in \cite{key3} hold without the restriction that
$\lambda'$ is regular.

In section fourteen we begin a study of the question of
unitarizability of the modules $Z(\u{\lambda})$, $\u{\lambda} \in
\mathfrak{L}$. The main result of this section gives a necessary and
sufficient condition for $Z(\u{\lambda})$ to be unitarizable in terms
of certain invariant Hermitian pairings of highest weight modules. 

In the last two sections we summarize several additional results on
complex groups as well as another point of view in analyzing the
admissible $(\mathfrak{g}, \mathfrak{t})$-modules. Section fifteen
includes a number of partial results on the question of
unitarizability obtained by a variety of techniques. The main results
here are the description of the unitarizable representations with
regular integral infinitesimal character as unitarily induced
representations of the group and the resulting description of the
representations with relative Lie algebra cohomology. In section
sixteen, we describe a technique of constructing admissible
$(\mathfrak{g}, \mathfrak{t})$-modules by derived functors introduced
by G. Zuckerman.\pageoriginale We then describe how the first part of
the program followed by these notes could equally well have been
completed with the lattice functor $\tau$ replaced by the derived
functor in the ``middle'' dimension. The section ends with an example
which shows the lattice functor and the ``middle'' dimension functor
are not equivalent.

As mentioned earlier, the first description of the irreducible
admissible $(\mathfrak{g}, \mathfrak{t})$-modules was given by
Zelobenko. We now describe the Zelobenko classification, compare it
with ours and match the parameters. Let $\u{\lambda} \in \mathfrak{L}$
and put $\mu = \u{\lambda}\mid_t$. By Frobenius reciprocity the
irreducible $\mathfrak{t}$-module with extreme weight $\mu$ occurs
with multiplicity one in $X(\u{\lambda})$ (cf. \S \ref{sec8}); and so, we let
$X(\u{\lambda})$ denote the ($\mathfrak{g}$-module) subquotient of
$X(\u{\lambda})$ which contains this $\mathfrak{t}$-module. These
$\mathfrak{g}$-modules were first studied by Parthasarathy, Ranga Rao
and Varadarajan \cite{key30} for the case $\mu = 0$ (the spherical
case). The classification of Zelobenko \cite{key36} can be expressed
as follows:

\begin{theorem*}
The map $\u{\lambda} \mapsto \hat{X}(\u{\lambda})$ induces a bijection
of $\mathfrak{L}/\mathfrak{w}_\mathfrak{t}$ onto the equivalence
classes of irreducible admissible $(\mathfrak{g},
\mathfrak{t})$-modules.
\end{theorem*}

The connection with our parameters is given as:

\medskip
\noindent{\textbf{Proposition 10.5.~}}
\textit{Let $\u{\lambda} \in \mathfrak{L}$ and assume $\u{\lambda}$
  satisfies (\ref{chap6:subsec6.5}). Then $Z(\u{\lambda})$ and
  $\hat{X} (\u{\lambda})$ are isomorphic}.

There has been some recent work on the subject of these notes. The
Bernstein Gel'fand article \cite{key3} offers a different and
especially beautiful approach to Questions A, B and C. Their results
for complex groups are\pageoriginale essentially the same as those
described here. The question of the decomposition of the principal
series modules can be interpreted as a question of primitive ideals in
the enveloping algebra. From this point of view, A. Joseph
\cite{key25} has obtained results for Questions A and B which are
essentially the same as those described above. The conjectures of
Kazhdan and Lustig \cite{key26} offer a precise formula for the
integers $m(s\lambda)$ given in the character formulae above. A
positive resolution of this conjecture would offer a very satisfying
description of the character theory of complex semisimple Lie groups. 

Lastly we list a few conventions which will remain in force throughout
the notes. We denote the integers (resp. nonnegative integers,
positive integers) by $\mathbb{Z}$ (resp. $\mathbb{N}$,
$\mathbb{N}^*$). The symbol $\Leftrightarrow$ will be used in place of
the term if and only if. For a Lie algebra $\mathfrak{g}$, let
$U(\mathfrak{g})$ denote the universal enveloping algebra of
$\mathfrak{g}$ and let $Z(\mathfrak{g})$ denote the center of
$U(\mathfrak{g})$. 
