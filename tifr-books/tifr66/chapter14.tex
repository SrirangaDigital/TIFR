
\chapter{Preunitary pairings}\label{sec14}

Let\pageoriginale $\sigma$ denote the conjugate linear
antiautomorphism of $U(\mathfrak{g})$ which\break equals $(-1)\cdot$
identity on the real form $i(\mathfrak{g}_0)$ of
$\mathfrak{g}$. Recall remark (\ref{chap3:coro3.13}) and note that all invariant
sesquilinear pairings and forms will be given with respect to this
choice of $\sigma$. In this section we begin the study of which
irreducible admissible $(\mathfrak{g}, \mathfrak{t})$-modules admit
invariant Hermitian forms with definite signature. 

\begin{definition}\label{chap14:def14.1}
An irreducible admissible $(\mathfrak{g}, \mathfrak{t})$-module $A$
will be called unitarizable if $A$ admits a nonzero invariant
Hermitian form which is positive definite. 
\end{definition}

$T'$ definition is suggested by the fact: A satisfies (\ref{chap14:def14.1}) if and
only if $A$ is isomorphic to the set of $K$-finite vectors in an
irreducible unitary representation of $G$. 

From the definition of $\sigma$, $\sigma$ equals the identity on
$\mathfrak{t} \cap \mathfrak{h}_\mathbb{R}$ and $(-1)$ identity on
$\mathfrak{p} \cap \mathfrak{h}_\mathbb{R}$. Let $^-$ denote
conjugation  on $\mathfrak{h}$ (resp. $\mathfrak{h}^*$) with respect
to the real form $\mathfrak{h}_\mathbb{R}$
(resp. $\mathfrak{h}^*_\mathbb{R}$). Then 
\begin{equation*}
(H, H')^{\sigma} = \overline{(H', H)}, \;(\lambda, \lambda')^\sigma
  = \overline{(\lambda' ,  \lambda)}, \; H, H' \in \mathfrak{h}_0,
  \lambda, \lambda' \in \mathfrak{h}^*_0.  \tag{14.2}\label{eq14.2}
\end{equation*}
From (\ref{eq14.2}), we have:
\begin{equation*}
\sigma (n^{\pm}) = n^{\mp}.  \tag{14.3}\label{eq14.3}
\end{equation*}

For an admissible $(\mathfrak{g}, \mathfrak{t})$-module $A$, let
$A^\sigma$ denote the $\mathfrak{h}$-locally finite subspace of the
algebraic dual of $A$. Give $A^\sigma$ the conjugate complex structure
and define the action of $\mathfrak{g}$ on $A^\sigma$ by: 
\begin{equation*}
(x \cdot \varphi) (a) = \varphi (x^\sigma \cdot a), \; x \in
  \mathfrak{g}, \; \varphi \in  A^\sigma, \; a \in A.  \tag{14.4}\label{eq14.4}
\end{equation*}\pageoriginale

Then $A^\sigma$ is a $\mathfrak{g}$-module and is called the conjugate
dual to $A$. 

\setcounter{prop}{4}
\begin{lemma}\label{chap14:lem14.5}
Let $\u{\lambda}$, $\u{\mu} \in \mathfrak{h}^*$. Then
$L(\u{\lambda})$ and $L(\u{\mu})$ admit a nonzero invariant
sesquilinear pairing if and only if $\u{\mu} =
\u{\lambda}^\sigma$. Moreover, if $\u{\mu} = \u{\lambda}^\sigma$ then
the space of invariant sesquilinear pairings is one dimensional. 
\end{lemma}

\begin{proof}
By (\ref{eq14.2}), $\mathbb{C}_{\u{\lambda}}$ and $\mathbb{C}_{\u{\mu}}$ admit
an $\mathfrak{h}$-invariant pairing if and only if $\u{\mu} =
\u{\lambda}^\sigma$. By Proposition (\ref{chap3:prop3.6}), we obtain
(\ref{chap14:lem14.5}) for 
modules $M(\u{\lambda})$ and $M(\u{\mu})$. However by cyclicity of the
highest weight space it follows that every pairing of $M(\u{\lambda})$
and $M(\u{\lambda})$ is the pull back of a pairing of $L(\u{\lambda})$
and $L(\u{\mu})$. This proves (\ref{chap14:lem14.5}). 
\end{proof}

\begin{lemma}\label{chap14:lem14.6}
Let $A$ be an irreducible $(\mathfrak{g}, \mathfrak{t})$-module and
let $I(A)$ be the vector space of invariant sesquilinear forms on
$A$. Then $\dim_\mathbb{C} I(A) \leq 1$. Also, if this dimension is
one then there exists a Hermitian form $\varphi$ on $A$.  
\end{lemma}

\begin{proof}
Let $B$ be an irreducible admissible $(\mathfrak{g},
\mathfrak{t})$-module and let $B^\sigma$ denote the
$\mathfrak{g}$-module of $U(\mathfrak{t})$-locally finite vectors in
the algebraic dual of $B$. Give $B^\sigma$ the conjugate complex
structure and action (\ref{eq14.4}). Then $B^\sigma$ is an irreducible
$\mathfrak{g}$-module and any pairing $\varphi$ of $A$ and $B$
corresponds to a $\mathfrak{g}$-module map $\bar{\varphi}: A \mapsto
B^\sigma$ where $\bar{\varphi} (a)(b) = \varphi(a, b)$, $a\in A$, $b
\in B$. Now the first part of (\ref{chap14:lem14.6}) follows by Schur's lemma. 

For any invariant form $\varphi$, put $\varphi_1 (a,b) = \varphi(a,b)
+ \overline{\varphi(b,a)}$, $\varphi_2 (a,b) = \varphi(a,b) -
\overline{\varphi(b, a)}$ $a, b \in A$. Then $\varphi_1$ and
$\varphi_2$ are also invariant. Moreover, $\varphi_1$ and $\sqrt{-1}
\varphi_2$ are Hermitian. This completes the proof of (\ref{chap14:lem14.6}). 
\end{proof}

We\pageoriginale now give a necessary and sufficient condition for an
irreducible admissible $(\mathfrak{g}, \mathfrak{t})$-module to admit
an invariant Hermitian form. 

\begin{prop}\label{chap14:prop14.7}
Let $\u{\lambda} \in \mathfrak{L}$ and assume $\u{\lambda}$ satisfies
(\ref{chap6:subsec6.5}). Then $Z(\u{\lambda})$ admits an invariant Hermitian form if and
only if $\u{\lambda}$ and $\u{\lambda}^\sigma$ lie in the same
$\mathfrak{w}_\mathfrak{t}$-orbit. 
\end{prop}

\begin{proof}
Let $(6.5)'$ denote the condition (\ref{chap6:subsec6.5}) with the roles of $\mu$ and
$\mu'$ interchanged. Note that if $\u{\lambda}$ satisfies (\ref{chap6:subsec6.5}) then
$\u{\lambda}^\sigma$ satisfies $(6.5)'$. The proof of (\ref{chap10:prop10.5}) applies
equally well under the hypothesis $(6.5)'$ in place of
(\ref{chap6:subsec6.5}). Therefore, 
\begin{equation*}
\tau L (\u{\lambda}^\sigma) \text{ is isomorphic to }
\hat{X}(\u{\lambda}^\sigma).  \tag{14.8}\label{eq14.8} 
\end{equation*}

By (\ref{eq14.3}), let $\varphi$ be a nonzero invariant sesquilinear pairing
of $L(\u{\lambda})$ and $L(\u{\lambda}^\sigma)$. Applying $\tau$,
$\tau \varphi$ is an invariant sesquilinear pairing of
$Z(\u{\lambda})$ and $\hat{X} (\u{\lambda}^\sigma)$. Both of these
modules are irreducible; and so, $Z(\u{\lambda})$ admits an invariant
sesquilinear form $\Leftrightarrow Z(\u{\lambda})$ and
$\hat{X}(\u{\lambda}^\sigma)$ are isomorphic
$\Leftrightarrow\u{\lambda}$ and $\u{\lambda}^{\sigma}$ lie in the same
$\mathfrak{w}_\mathfrak{t}$ -orbit (cf. (\ref{chap10:lem10.4}),
(\ref{chap10:prop10.5}), (\ref{chap10:thm10.8})). Lemma 
(\ref{chap14:lem14.6}) completes the proof. 
\end{proof}

The question of positive definiteness of Hermitian forms is much
deeper than that resolved by (\ref{chap14:prop14.7}). In the remaining portion of this
section we relate the question of definiteness of invariant forms on
modules $Z(\u{\lambda})$ to a property of pairing of $L(\u{\lambda})$
and $L(\u{\lambda}^\sigma)$. 

For $\mu \in \mathfrak{t}^*$, let $U(\mu) =
U(\mathfrak{g})\bigotimes\limits_{U(\mathfrak{t})} M(\mu)$. If $\mu,
\nu \in \mathfrak{t}^*$ and $M(\nu)$ is a sub Verma module of $M(\mu)$
then the inclusion $M(\nu) \hookrightarrow M(\mu)$ induces an
inclusion $U(\nu) \hookrightarrow U(\mu)$. 

\setcounter{prop}{8}
\begin{definition}\label{chap14:def14.9}
Let $A$\pageoriginale and $B$ be $\mathfrak{g}$-modules and assume
that $\varphi$ is an invariant sesquilinear pairing of $A$ and $B$. We
call $\varphi$ a preunitary pairing of $A$ and $B$ if the following
properties hold:
\begin{itemize}
\item[{\rm (i)}] there exists $\mu \in \mathfrak{t}^*$ which is
  -$P_\mathfrak{t}$-dominant integral and regular and maps $S: U(\mu)
  \to A$, $T : U(\mu) \to B$. 

\item[{\rm (ii)}] if $\bar{\varphi}$ is the ``pull back'' form on
  $U(\mu)$ given by $\bar{\varphi} = (S(~), T(~))$, then $\bar{\phi}$
  has a nonzero positive semidefinite restriction to the span of the
  subspaces of $n_\mathfrak{t}$-invariants of weight $\xi$ with
  $\xi+2\delta_\mathfrak{t}$ -$P_\mathfrak{t}$-dominant. 
\end{itemize}
\end{definition}

\begin{prop}\label{chap14:prop14.10}
Let $\u{\lambda} \in \mathfrak{L}$ and assume $\u{\lambda}$ satisfies
(\ref{chap6:subsec6.5}) and $Z(\u{\lambda})$ admits an invariant Hermitian form. Then
$Z(\u{\lambda})$ is unitarizable if and only if $L(\u{\lambda})$ and
$L(\u{\lambda}^{\sigma})$ admit a preunitary pairing. 
\end{prop}

\begin{proof}
For any $\mathfrak{t}$-module $A$, let $A'$ (resp. $A^\sim$) denote
the span of the subspaces of $n_\mathfrak{t}$-invariants of weight
$\xi$ with $\xi $ $P_\mathfrak{t}$-dominant integral (resp. $\xi + 2
\delta_\mathfrak{t}$ -$P_\mathfrak{t}$ dominant integral). Let
$\mathfrak{g} = \mathfrak{t} \oplus \mathfrak{p}$  be a Cartan
decomposition for $\mathfrak{g}$ and let $S(\mathfrak{p})$ denote the
symmetric tensor algebra of $\mathfrak{p}$. Then for $\mu \in
\mathfrak{t}^*$, $S(\mathfrak{p}) \otimes M(\mu)$ and $U(\mu)$ are
isomorphic as $\mathfrak{t}$-modules. Let $t_0$ be the element of
$\mathfrak{w}_\mathfrak{t}$ of maximal length and let $\mu \in
\mathfrak{t}^*$ be $P_\mathfrak{t}$-dominant integral. Then from
Proposition 4.13 \cite{key15}, there is an isomorphism:
\begin{equation*}
U(\mu)' \xrightarrow{\sim} U(t_0 \mu)^\sim. \tag{14.11}\label{eq14.11}
\end{equation*}
Moreover, if $\psi$ is any invariant sesquilinear form on $U(t_0 \mu)$
then by successive completions we obtain a form $\psi_1$ on
$U(\mu)$. From (\ref{eq14.11}) and (\ref{chap3:thm3.14}) we find that: 
\end{proof}


\setcounter{section}{14}
\setcounter{subsection}{11}
\subsection{}\label{chap14:subsec14.12}
$\psi$\pageoriginale restricted to $U(t_0 \mu)^\sim$ is positive
semidefinite (positive definite) if and only if $\psi_1$ restricted to
$U(\mu)'$ is positive semidefinite (positive definite). 

First assume $Z(\u{\lambda})$ is unitarizable. Then by (\ref{chap14:prop14.7}),
$\u{\lambda}$ and $\u{\lambda}^\sigma$ lie in the same
$\mathfrak{w}_\mathfrak{t}$-orbit. Let $\varphi$ be a nonzero
invariant sesquilinear pairing of $L(\u{\lambda})$ and
$L(\u{\lambda}^\sigma)$ given by (\ref{chap14:lem14.5}). Then $\tau\varphi$ is an
invariant sesquilinear pairing of $Z(\u{\lambda})$ and $\tau L
(\u{\lambda}^\sigma)$. By (\ref{eq14.8}) and (\ref{chap14:prop14.7}), these modules are
isomorphic; and so, $\tau \varphi$ is an invariant sesquilinear form
on $Z(\lambda)$. By (\ref{chap14:lem14.6}), we may assume that $\tau \varphi$ is
Hermitian and positive definite since $Z(\u{\lambda})$ is
unitarizable. Let $L_s$ (resp. $L^\sigma_s$), $s \in
\mathfrak{w}_\mathfrak{t}$, be a lattice above $L(\u{\lambda})$
(resp. $L(\u{\lambda}^\sigma)$). Let $z$ be any nonzero
$\mathfrak{t}$-highest weight vector in $Z(\u{\lambda})$. Since
$Z(\u{\lambda})$ is an admissible $(\mathfrak{g},
\mathfrak{t})$-module the weight $\mu - \delta_\mathfrak{t}$ of $z$ is
dominant integral. Choose maps $S : U(\mu) \to L_1$, $T:U(\mu) \to
L^\sigma_1$ which when composed with the projections onto
$Z(\u{\lambda})$, map the canonical cyclic vector of weight $\mu$ onto
$z$. We now claim: 
\begin{equation*}
S: U(t_0\mu) \to L(\u{\lambda}), \quad T: U(t_0 \mu) \to
L(\u{\lambda}^\sigma). \tag{14.13}\label{eq14.13}
\end{equation*}

Let $U_s$, $s \in \mathfrak{w}_\mathfrak{t}$, be a lattice above
$U(t_0 \mu)$. Then $U_1 = U(\mu)$. We now prove $S:U_s \to L_s$ for
all $s \in \mathfrak{w}_\mathfrak{t}$ by induction on the length of
$s$. For $s = 1$, this is (\ref{eq14.13}). Assume $s \neq 1$ and choose
$\alpha \in P_\mathfrak{t}$ simple with $\ell(s_\alpha s) = \ell(s) -
1$. Then by the induction hypothesis, $S: U_{s_\alpha s} \to
L_{s_\alpha s} $. Consider the induced map of $U_s$ into $L_{s_\alpha
  s}/L_s$. The former module is generated by an irreducible infinite
dimensional Verma $\mathfrak{a}^{(\alpha)}$-module (cf. \S\  \ref{sec4}) while
the latter module is $U(\mathfrak{a}^{\alpha})$-locally finite. This
implies that the induced map is zero; and so, $S:U_s \to L_s$. This
proves the first part of (\ref{eq14.13}). The same argument applies to prove
the second. 

Let\pageoriginale $\psi$ denote the pull back of $\varphi$ to a form
on $U(t_0 \mu)$; i.e., $\psi = \varphi (S(~), T(~))$. By (\ref{eq14.13}),
$\psi$ is defined. Moreover, if $\psi_1$ is the pull back of $\tau
\varphi$ to $U(\mu)$ then one checks easily that $\psi_1$ can be
defined by successive completions beginning with $\psi$. Since
$Z(\u{\lambda})$ is unitarizable, $\tau \varphi$ is positive definite
and thus $\psi_1$ is positive semidefinite and nonzero when restricted
to $U(\mu)$. Then (\ref{chap14:subsec14.12}) states that $\psi$ is
positive semidefinite and nonzero when restricted to $U(t_0
\mu)^{\sim}$. This means that $\varphi$ is a preunitary pairing of
$L(\u{\lambda})$ and $L(\u{\lambda}^\sigma)$.  

We now prove the converse. Assume $\varphi$ is a perunitary pairing of
$L(\u{\lambda})$ and $L(\u{\lambda}^\sigma)$. From (\ref{chap14:def14.9}) we let $\mu
\in \mathfrak{t}^*$ be $P_\mathfrak{t}$-dominant integral and regular
and let $S: U(t_0 \mu) \to L(\u{\lambda})$, $T: U(t_0 \mu) \to
L(\u{\lambda}^\sigma)$ be given by (\ref{chap14:def14.9}). To conform with the
notation above we let $S$ and $T$ also denote the extensions of $S$
and $T$ to the lattice above $U(t_0)$ obtained by applying completion
functors. If $\psi$ is the pull back of $\varphi$ to $U(t_0 \mu)$ and
if $\psi_1$ is the pull back of $\tau\varphi$ to $U(\mu)$, then
$\psi_1$ is obtained from $\psi$ by successive completions. So by
(\ref{chap14:subsec14.12}), $\psi_1$ is positive semidefinite and
nonzero when restricted  to $U(\mu)'$. But then $\tau \varphi$ is
nonzero and positive semidefinite; and since $Z(\u{\lambda})$ is
irreducible, $\tau \varphi$ must be positive definite. Therefore
$Z(\u{\lambda})$ is unitarizable.  
