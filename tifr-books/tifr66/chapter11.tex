
\chapter{An application to highest weight modules}\label{sec11}

Our\pageoriginale main purpose in these notes is to describe the
category of admissible $(\mathfrak{g}, \mathfrak{t})$-modules by
defining a correspondence between highest weight modules and
admissible $(\mathfrak{g}, \mathfrak{t})$-modules. The working
hypothesis here is that the category of highest weight modules is more
tractable than the category of admissible $(\mathfrak{g},
\mathfrak{t})$-modules. Usually, we expect results for highest weight
modules to give new results for admissible $(\mathfrak{g},
\mathfrak{t})$-modules. The present section is an exception to this
rule. Here we use the vanishing theorem $\tau L(\u{\lambda})=0$ for
certain $\u{\lambda}$ to conclude various skew-symmetry properties for
coefficients of the character of $L(\lambda)$, $\u{\lambda} =
(\lambda, \lambda')$. 


For any module $A \in \mathscr{O}$, let $ch A$ denote the formal
character of $A$ (cf. \cite{key24}). Fix $\lambda \in
\mathfrak{h}^*_0$ and write 
\begin{equation*}
ch \; L(\lambda) = \sum m (s\lambda) ch \; M (s\lambda) \tag{11.1}\label{eq11.1}
\end{equation*}
with the sum taken over the $\mathfrak{w}_\lambda$ orbit of
$\lambda$. 

\setcounter{prop}{1}
\begin{prop}\label{chap11:prop11.2}
Let $\alpha$ be a simple root of $P_\lambda$. Assume $\lambda_\alpha
\in \mathbb{N}^*$. Then $m(s_\alpha s\lambda) + m (s\lambda) = 0$. 
\end{prop}

The reader should compare (\ref{chap11:prop11.2}) with Satz 2.12 in \cite{key24} which
gives for regular $\lambda$ and which an additional hypothesis on
$\alpha$, the identity $m(ss_\alpha \lambda) + m(s\lambda) = 0$. 

\begin{proof}
Choose $\lambda'$ with $\lambda + \lambda'$ integral, $\lambda'$
-$P_\lambda$-dominant and such that $\lambda'_\beta = 0$, $\beta \in
P_0 \Rightarrow \beta = \alpha$. Set $\u{\lambda} = (\lambda,
\lambda')$ and note that $P_{\lambda'} = P_{\lambda}$. For $\u{\nu}
\in \mathfrak{L}$, let $E(\u{\nu})$ denote the global distribution
character of $G$ for the principal\pageoriginale series representation
with $\mathfrak{g}$-module of $K$-finite vectors isomorphic to
$X(\u{\nu})$. Let $\Theta (\u{\lambda})$ denote the distribution
character of any admissible representation of $G$ with
subspace of $K$-finite vectors $\mathfrak{g}$-isomorphic to $\tau
L(\u{\lambda})$. Since the category $\mathscr{O} \otimes M(\lambda')$
is $\mathfrak{t}$-semisimple (Proposition \ref{chap3:prop3.9}), $\tau$ is exact on
$\mathscr{O} \otimes M(\lambda')$. This exactness of $\tau$ implies
that $\tau$ induces a linear map of Grothendieck groups. So the
formula (\ref{eq11.1}) becomes under $\tau$:
\begin{equation*}
\Theta (\u{\lambda}) = \sum m (s\lambda) E (s\lambda, \lambda')
\tag{11.3}\label{eq11.3} 
\end{equation*}
By our choice of $\lambda'$, $\{\pm \alpha\}$ is the set of roots
orthogonal to $\lambda'$; and so, $E(s_\alpha s \lambda, \lambda') =
E(s\lambda, \lambda')$, $\forall s \in
\mathfrak{w}_\lambda$. Moreover, these identities generate all the
linear relations among the $E(s\lambda, \lambda')$. By assumption
$\lambda'_\alpha \in \mathbb{N}^*$; and so, $\u{\lambda}$ does not satisfy
(\ref{chap6:subsec6.5}). This implies by (\ref{chap10:prop10.6}),
$\Theta (\u{\lambda}) = 0$; and thus, 
$m(s_\alpha s \lambda) + m (s\lambda) = 0$. 
\end{proof}

Let $B_\lambda$ denote the simple roots of $P_\alpha$. Choose a subset
$B^0_\lambda \subset B_\lambda$, be the positive system generated by
$B^0_\lambda$ and let $\mathfrak{w}^0_\lambda$ be the Weyl group of
$B^0_\lambda$. 


\setcounter{prop}{3}
\begin{prop}\label{chap11:prop11.4}
Let $\lambda, \mu, \nu \in \mathfrak{h}^*_0$. Assume $\mu$ is
-$P^0_\lambda$-dominant and assume $\lambda = w\mu$ for some $w \in
\mathfrak{w}^0_\lambda$. If $\nu$ is -$P^0_\lambda$-dominant, then
$L(\nu)$ does not occur as a subquotient of $M(\lambda) / M (\mu)$. 
\end{prop}

\begin{proof}
 Choose $\lambda'$ with $\lambda + \lambda'$ integral, $\lambda'$
 -$P_\lambda$-dominant and with $B^0_\lambda = \{\alpha \in B_\lambda
 \mid \lambda'_\alpha = 0\}$. Put $\u{\lambda} = (\lambda, \lambda')$
 and $\u{\mu} = (\mu, \lambda' )$. Now, by assumption on $\lambda$,
 $\u{\lambda}$ and $\u{\mu}$  lie in the same $\mathfrak{w}$-orbit;
 and so, $\tau M (\u{\lambda}) \simeq \tau M (\u{\lambda})$. By the
 exactness of $\tau$ on $\mathscr{O} \otimes M(\lambda')$, $\tau
 (M(\u{\lambda})) = 0$ and, for any subquotient $A$ of\pageoriginale
 $M(\u{\lambda}) / M(\u{\mu})$, $\tau A = 0$. If $L(\nu)$ occurs as a
 subquotient of $M(\lambda) \backslash M(\mu)$, then $\tau L(\nu, \lambda')
 =0$. By (\ref{chap10:prop10.6}), $(\nu, \lambda')$ must not satisfy
 (\ref{chap6:subsec6.5}); and so, 
 $\nu$ is not -$P^0_\lambda$-dominant. 

The reader should consult section \ref{chap4:coro4.5} in \cite{key3} as well as Satz
2.12 in \cite{key24} for identities relating multiplicities of
irreducible highest weight modules in Verma modules. 
\end{proof}
