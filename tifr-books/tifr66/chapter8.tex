 
\chapter{The principal series modules}\label{sec8}

In this\pageoriginale section we give the definition of the principal
series modules and list some of their basic properties. Our notation
will differ from that in \cite{key8} (cf. \S\ \S\ 9.3, 9.4) by a slight 
shift in the parameter.


We retain the notation of \S\ \ref{sec6}. Put $Q = P_0 \times \{0\} \cup \{0\}
\times (-P_0)$. Then $Q$ is a positive system for $\Delta$ and $\theta
Q = - Q$. As usual let $\delta_Q = \dfrac{1}{2} \sum\limits_{\alpha
  \in Q} \alpha$. If $n^{\pm}_Q$ and $\mathfrak{b}_Q$ denote the
associated nilpotent and Borel subalgebras, then $n^\pm_Q = n^{\pm}_0
\times n^{\mp}_0$ and $\mathfrak{b}_Q = \mathfrak{h} \oplus
n^+_Q$. Let $\mathfrak{n}_{Q,0}$ be the real subalgebra of
$\mathfrak{g}_0$ such that $i(\mathfrak{n}_{q,0}) = \mathfrak{n}_Q
\cap i (\mathfrak{g}_0)$. One  may check that $\mathfrak{g}_0 =
\mathfrak{t}_0 \oplus \mathfrak{a}_0 \oplus \mathfrak{n}_{Q,0}$ is an
Iwasawa decomposition of $\mathfrak{g}_0$. For $\u{\lambda} \in
\mathfrak{h}^*$, let $\bar{M}(\u{\lambda})$ denote the Verma module
with $Q$-highest weight $\u{\lambda} - \delta_Q$. 

\begin{definition}\label{chap8:def8.1}
For $\u{\lambda} \in \mathfrak{h}^*$, define $X(\u{\lambda})$ to be
the submodule of $U(\mathfrak{t})$-locally finite vectors in the
algebraic dual of $\bar{M}(-\u{\lambda})$. $X(\u{\lambda})$ is called
the principal series module with parameter $\u{\lambda}$. 
\end{definition}


\begin{remarks}\label{chap8:rem8.2}
\begin{itemize}
\item[{\rm (i)}]  $X(\u{\lambda})$ is naturally isomorphic to the set
  of $U(\mathfrak{t})$ - locally finite vectors in coinduced
  $\mathfrak{g}$-module $\Hom_{U(\mathfrak{b}_Q)} (U(\mathfrak{g}),\break
  \mathbb{C}_{\u{\lambda} + \delta_Q})$.

\item[{\rm (ii)}] $X(\u{\lambda})$ admits an infinitesimal character
  $\chi: Z(\mathfrak{g}) \to \mathbb{C}$ parametrized by the
  $\mathfrak{w}$-orbit of $\mathfrak{\lambda}$. 
\end{itemize}
\end{remarks}

\setcounter{section}{8}
\setcounter{subsection}{2}
\subsection{(Frobenius Reciprocity)}\label{chap8:subsec8.3}
 Let $F$ be a finite dimensional
$\mathfrak{t}$-module. Then, if $\nu = \u{\lambda}\mid_{\mathfrak{t}}$
\begin{itemize}
\item[{\rm(i)}] $\dim \Hom_\mathfrak{t} (F,X(\u{\lambda})) = \dim
  \Hom_\mathfrak{t} (\mathbb{C}_\nu, F)$. 

\item[{\rm (ii)}] $X(\u{\lambda}) = 0$ if $\nu$ is not integral. 
\end{itemize}

\setcounter{prop}{3}
\begin{coro}\label{chap8:coro8.4}
If\pageoriginale $F$ is irreducible with extreme weight $\nu$, then
$F$ occurs in $X(\u{\lambda})$ with multiplicity one and all other
$\mathfrak{t}$-modules in $X(\u{\lambda})$ have extreme weights with
norms strictly greater than the norm of $\nu$.
\end{coro}



Recall now the translation functors of \S\ \ref{sec5}. Let $\u{\lambda} =
(\lambda, \lambda')$, $\u{\mu} = (\mu, \mu')$ be elements of
$\mathfrak{h}^*$ with $\u{\lambda}$ integral. Let $\varphi =
\varphi^{\u{\lambda}}_{\u{\lambda} + \u{\mu}}$ and $\psi =
\psi^{\u{\lambda} + \u{\mu}}_{\u{\lambda}}$ be the Zuckerman
translation functors. From Proposition \ref{chap5:prop5.3} on Verma
modules we obtain by duality:

\setcounter{subsection}{4}
\subsection{}\label{chap8:subsec8.5}
\begin{itemize}
\item[{\rm (i)}] $\psi(X(\u{\lambda} + \u{\mu})) = X(\u{\lambda})$

\item[{\rm (ii)}] if $\u{\lambda}$ and $\u{\lambda} + \u{\mu}$ are
  equisingular, then $\varphi (X(\u{\lambda})) = X(\u{\lambda} +
  \u{\mu})$. 
\end{itemize}

The next two properties of principal series modules are much deeper
results from the theory of semisimple Lie algebras. The first is the
fundamental result of Harish-Chandra:

\setcounter{prop}{5}
\begin{theorem}[The subquotient theorem]\label{chap8:thm8.6}
Each irreducible admissible $(\mathfrak{g}, \mathfrak{t})$-module is
isomorphic to a Jordan-H\"older factor of some principal series module.
\end{theorem}

\begin{prop}\label{chap8:prop8.7}
Let $s \in \mathfrak{w}_0$, $(\lambda, \lambda') \in
\mathfrak{h}^*$. Then $X(\lambda, \lambda')$ and $X(s\lambda,
s\lambda')$ have the same Jordan-H\"older factors occurring with the
same multiplicity.
\end{prop}

The easiest proof of (\ref{chap8:prop8.7}) is given by first checking that
$X(\lambda,\lambda')$ is isomorphic to the set of
$\mathfrak{t}$-finite vectors for a principal series representation of
$G$, say $\bar{X}(\lambda,\lambda')$, and then showing that
$\bar{X}(\lambda, \lambda')$ and $\bar{X}(s\lambda,
s\lambda')$\pageoriginale have the same distribution character. For
more details the reader may consult \cite{key31} or
\cite{key32}. There is also a Lie algebraic proof of
(\ref{chap8:prop8.7}) given in \cite{key28}. 
