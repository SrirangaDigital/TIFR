\chapter[Application of the Comparison Theorem....]{Application of the Comparison Theorem an Exact Sequence for
  Fundamental Groups}\label{chap6}


\section{}\label{chap6-sec6.1}\pageoriginale
As usual we make the convention that the preschemes considered are
locally noetherian and the morphisms are of finite type (with the same
remark as in the beginning of Ch. \ref{chap3}). 

\begin{defin}\label{defin6.1.1}
\begin{itemize}
\item[(a)] A morphism $X\to \Spec k$, $k$ a field, is said to be {\em
  separable} if, for any extension field $K/k$, the prescheme
  $\foprod{X}{K}{k}$ is {\em reduced.}

\item[(b)] A morphism $X\xrightarrow{f}Y$ is separable if $f$ is flat and
  for any $y\in Y$, $\foprod{X}{\Spec k(y)}{Y}$ is separable over $k(y)$.
\end{itemize}

We shall now state a few results which we will need for our next main
theorem. Proofs can be found in EGA.
\end{defin}

\setcounter{theorem}{1}
\begin{theorem}\label{chap6-thm6.1.2}
Let $f:X\to Y$ be a proper morphism and $\mathscr{F}$ a coherent
$\mathscr{O}_{X}$-Module. If $Y_{1}\xrightarrow{q}Y$ is a flat
base-change
\[
\xymatrix@=1.2cm{
X\ar[d]^{f} & \ar[l]_-{q_{1}} X_{1}=\fprod{X}{Y_{1}}{Y}\ar[d]^{f_{1}}\\
Y & \ar[l]_-{q} Y_{1} 
}
\]
then we have the isomorphisms
$$
R^{n}f_{1\ast}(q^{\ast}_{1}(\mathscr{F}))\xleftarrow{\sim}\foprod{R^{n}f_{\ast}(\mathscr{F})}{\mathscr{O}_{Y_{1}}}{\mathscr{O}_{Y}} 
$$
for any $n\in \mathbb{Z}^{+}$. (Prop. (1.4.15), EGA, Ch. III).
\end{theorem}

\setcounter{subsection}{2}
\subsection{}\label{chap6-sec6.1.3}
Suppose\pageoriginale $Y$ {\em is a noetherian prescheme and $f:X\to
  Y$ a proper morphism.} Let $Y'\hookrightarrow Y$ be a closed
subscheme of $Y$, defined by a coherent Ideal $\mathscr{T}$ of
$\mathscr{O}_{Y}$. The ``inverse image'' of $Y'$ by $f$, namely the
fibre-product $\fprod{X}{Y'}{Y}=X'$ is then a closed subscheme of $X$,
defined by the $\mathscr{O}_{X}$-Ideal
$\mathscr{T}=f^{\ast}(\mathscr{T})\mathscr{O}_{X}$.

Let $\mathscr{F}$ be any {\em coherent} $\mathscr{O}_{X}$-Module; for
$n\in \mathbb{Z}^{+}$, consider
$\mathscr{F}_{n}=\foprod{\mathscr{F}}{\mathscr{O}_{X}/\mathscr{T}^{n+1}}{\mathscr{O}_{X}}$
(this is a coherent $\mathscr{O}_{X}$-Module, concentrated on the
pre\-scheme $X_{n}=(X',\mathscr{O}_{X}/\mathscr{T}^{n+1})$ and may also
be considered as an $\mathscr{O}_{X_{n}}$-Module). Consider
$R^{q}f_{\ast}(\mathscr{F}_{n})$; this is a coherent
$\mathscr{O}_{Y}$-Module (finiteness theorem), is concentrated on the
prescheme $Y_{n}=(Y',\mathscr{O}_{Y}/\mathscr{T}^{n+1})$ and is in
fact an $\mathscr{O}_{Y_{n}}$-Module. From the homomorphism
$\mathscr{F}\to \mathscr{F}_{n}$ we obtain a homomorphism
$R^{q}f_{\ast}(\mathscr{F})\to R^{q}f_{\ast}(\mathscr{F}_{n})$ and
since the latter is an $\mathscr{O}_{Y_{n}}$-Module, we get a natural
homomorphism: 
\begin{equation*}
\foprod{R^{q}f_{\ast}(\mathscr{F})}{\mathscr{O}_{Y}/\mathscr{T}^{n+1}}{\mathscr{O}_{Y}}\to
R^{q}f_{\ast}(\mathscr{F}_{n}).\tag{*} 
\end{equation*}

As $n$ varies, we get a projective system of homomorphisms. With the
assumptions we have made about $X$, $Y$, $f$, the {\em comparison
  theorem} (EGA, Ch. III, Theorem (4.1.5)) states that in the limit,
this gives an {\em isomorphism}:
\begin{equation*}
\foprod{\varprojlim_{n}R^{q}f_{\ast}(\mathscr{F})}{\mathscr{O}_{Y}/\mathscr{T}^{n+1}}{\mathscr{O}_{Y}}\xrightarrow{\sim}\varprojlim_{n}R^{q}f_{\ast}(\mathscr{F}_{n}). \tag{**}
\end{equation*}
({\bf Note:}\pageoriginale Both sides of (**) are concentrated on $Y'$
and they are also equal to
$R^{q}\widehat{f}_{\ast}(\varprojlim_{n}\foprod{\mathscr{F}}{\mathscr{O}_{X}/\mathscr{T}^{n+1}}{\mathscr{O}_{X}})$
where $\widehat{f}$ is the morphism of the {\em ringed spaces}
$\widehat{X}=(X',\varprojlim_{n}\mathscr{O}_{X}/\mathscr{T}^{n+1})\to
\widehat{Y}=(Y',\varprojlim_{n}\mathscr{O}_{Y}/\mathscr{T}^{n+1})$
obtained from the morphisms $f_{n}:X_{n}\to Y_{n}$, induced by $f$).

We shall now ``specialise'' the above comparison theorem to the case
$Y$ affine, say $Y=\Spec A$. Then $Y'$ is defined by an ideal $I$ of
$A$. The first member of (**) corresponds to
$\varprojlim_{n}H^{q}(X,\mathscr{F})\otimes_{A}\frac{A}{I^{n+1}}$
which is precisely the completion of $H^{q}(X,\mathscr{F})$ under the
$I$-adic topology, while the second member of (**) corresponds to
$\varprojlim_{n}H^{q}(X,\mathscr{F}_{n})$ and thus: 
\begin{equation*}
H^{q}(X,\mathscr{F})\sphat
\xrightarrow{\sim}\varprojlim_{n}H^{q}(X,\mathscr{F}_{n}). \tag{{*}{**}}
\end{equation*}

\section{The Stein-factorisation}\label{chap6-sec6.2}

Let $X\xrightarrow{f}Y$ be a proper morphism. Then the coherent
$\mathscr{O}_{Y}$-Algebra $f_{\ast}(\mathscr{O}_{X})$ (finiteness
theorem) defines a $Y$-prescheme $Y'\xrightarrow{q}T$, {\em finite} on
$Y$. To the identity $\mathscr{O}_{Y}$-morphism
$q_{\ast}(\mathscr{O}_{Y'})=f_{\ast}(\mathscr{O}_{X})\to
f_{\ast}(\mathscr{O}_{X})$ corresponds a $Y$-morphism $f':X\to Y'$,
i.e., we have a commutative diagram
\[
\xymatrix@=1.2cm{
X\ar[dr]_{f}\ar[rr]^{f'} & & Y'=\Spec f_{\ast}(\mathscr{O}_{X})\ar[dl]^{q}\\
 & Y &
}
\]

The\pageoriginale morphism $f'$ is again proper. This factorisation
$f=q\circ f'$ is known as the {\em Stein-factorisation} of $f$. For
details the reader is referred to EGA Ch. III.

We shall now prove a theorem, which is of great importance to us.

\begin{theorem}\label{chap6-thm6.2.1}
Let $f:X\to Y$ be a separable, proper morphism. Let
$X\xrightarrow{f'}Y'=\Spec f_{\ast}(\mathscr{O}_{X})\xrightarrow{q}Y$
be the Stein-factorisation of $f$. Then $Y'\xrightarrow{q}Y$ is an
\'etale covering.
\end{theorem}

\begin{proof}
We have only to show that $q$ is \'etale; this is a purely local
problem and we may thus assume that $Y$ is affine, say, $Y=\Spec A$.

We shall make a few simplifications to start with. Suppose $X_{1}\to
Y$ is a flat base-change; then from \ref{chap6-thm6.1.2} one gets:
$$
f_{(Y_{1})_{\ast}}(\foprod{\mathscr{O}_{X}}{\mathscr{O}_{Y_{1}}}{\mathscr{O}_{Y}})\simeq 
f_{\ast}\foprod{(\mathscr{O}_{X})}{\mathscr{O}_{Y_{1}}}{\mathscr{O}_{Y}} 
$$

This means that in the commutative diagram
\[
\xymatrix{
X\ar[d]^{f'} & X_{1}=\fprod{X}{Y_{1}}{Y}\ar[l]\ar[d]_{f'_{(Y_{1})}}\\
\Spec f_{\ast}(\mathscr{O}_{X})=Y'\ar[d]^{q} &
\ar[l]\ar[d]^{q_{(Y_{1})}} Y'_{1}=\fprod{Y'}{Y_{1}}{Y}=\Spec
f_{(Y_{1})_{\ast}}(\foprod{\mathscr{O}_{X}}{\mathscr{O}_{Y_{1}}}{\mathscr{O}_{Y}})\\ 
Y &\ar[l] Y_{1} 
}
\]
the second vertical sequence is the Stein-factorisation of $f_{(Y_{1})}$.

In\pageoriginale view of this and the fact that it is enough to look
at $\Spec \mathscr{O}_{y,Y}$ for \'etaleness over $y\in Y$ we may make
the base change $Y\leftarrow \Spec \mathscr{O}_{y,Y}$ and assume that
$A$ is a local ring. Again, in view of the same remark and Lemma
\ref{chap4-sec4.4.1} we may assume that $A$ is {\em complete}.

Consider now the functor $T$ on the category of finite type
$A$-modules $M$, defined by $M\mapsto
T(M)=\Gamma(X,\mathscr{O}_{X}\otimes_{A}M)$. We shall show that {\em
  the assumption} ``$T$ {\em is right-exact}'' implies the theorem.

We remark that the {\em right-exactness of $T$ is equivalent to the
  assumption that the natural map}
\begin{equation*}
\foprod{\Gamma(X,\mathscr{O}_{X})}{M}{A}\to
\Gamma(X,\foprod{\mathscr{O}_{X}}{M}{A}) \tag{+}
\end{equation*}
{\em is an isomorphism}. Denote by $T'$ the right-exact functor
$$
M\mapsto \foprod{\Gamma(X,\mathscr{O}_{X})}{M}{A}.
$$ 
Suppose $T$ is
right exact. Then for any exact sequence $A^{n}\to A^{m}\to M\to 0$ we
have a commutative diagram of exact sequences:
\[
\xymatrix@=1.2cm{
T'(A^{n})\ar[r]\ar[d]^{\wr} & T'(A^{m})\ar[r]\ar[d]^{\wr} &
T'(M)\ar[r]\ar[d] & 0\\
T(A^{n})\ar[r] & T(A^{m})\ar[r] & T(M)\ar[r] & 0
}
\]
the\pageoriginale first two vertical maps are isomorphisms; hence the
third is also an isomorphism. Conversely, if (+) is an isomorphism,
clearly $T$ is right-exact.

Now from the flatness of $f$ and the assumed right exactness of $T$ it
follows that $\Gamma(X,\foprod{\mathscr{O}_{X}}{M}{A})$ is exact in
$M$ and thus [from the above remark] $\Gamma(X,\mathscr{O}_{X})$ is
$A$-flat. But $Y'=\Spec\Gamma(X,\mathscr{O}_{X})$ and therefore
$q:Y'\to Y$ is flat. It remains to show that $q$ is unramified.

Let $y\in Y$ be the unique closed point of $Y$; denote by $k$ the
residue field $A/\mathscr{M}=k(y)$. Then
$T(k)=\Gamma(X,\foprod{\mathscr{O}_{X}}{k}{A})=\Gamma(X_{y},\mathscr{O}_{X_{y}})$
where $X_{y}$ is the fibre $\foprod{X}{k(y)}{Y}$. But as $X$ is
$Y$-proper, $T(k)=\Gamma(X,\foprod{\mathscr{O}_{X}}{k}{A})$ is an
artinian $k$-algebra which, by the separability of $X$ over $Y$, is
radical-free for any base-change $K/k$, $K$ an extension field of
$k$. Hence $T(k)=\bigoplus\limits^{r}_{i=1}K_{i}$ where the $K_{i}$
are finite separable field extensions of $k$. Also, by our assumption,
$T(k)=\foprod{\Gamma(X,\mathscr{O}_{X})}{k}{A}=\Gamma(X,\mathscr{O}_{X})/\mathscr{M}\Gamma(X,\mathscr{O}_{X})$. Let
$y'\in Y'$ be a point above $y$. From the equality
$\Gamma(X,\mathscr{O}_{X})/\mathscr{M}\Gamma(X,\mathscr{O}_{X})=\bigoplus\limits^{r}_{i=1}K_{i}$,
($K_{i}/k$ are separable) it follows that the maximal ideal
$\mathscr{M}$ of $\mathscr{O}_{y}$. generates the maximal ideal of
$\mathscr{O}_{y'}$ and moreover that $\mathscr{O}_{y'}$ is unramified
over $\mathscr{O}_{y}$. 
\end{proof}

The\pageoriginale proof of the theorem is thus complete modulo the
assumption that $T$ is right-exact. Before proceeding to prove this we
may make some more simplifications. First we may assume that
$T(k)=\Gamma(X_{y},\mathscr{O}_{X_{y}})=\bigoplus\limits^{r}_{i=1}k$
(this can be done as in Lemma \ref{chap4-lem4.1.2} by making a faithfully
flat base-change which ``kills'' the extensions $K_{i/k}$). Then
finally we may assume $\Gamma(X_{y},\mathscr{O}_{X_{y}})=k$, i.e.,
$X_{y}$ connected, because the connected components of $X$ over $y$
correspond bijectively with the connected components of $X_{y}$ and
it is enough to prove the theorem separately for each component over
$y$.

\medskip
{\bf Case (a).~Assume {\boldmath$A$} artinian.}
\smallskip

Our aim is to show that for any finite type $A$-module $M$,
$T'(M)\xrightarrow{\sim}T(M)$. We shall first show that $T'(M)\to
T(M)$ is surjective.

For $n$ large, one has $\mathscr{M}^{n}M=0$ so that
$T'(\mathscr{M}^{n}M)\to T(\mathscr{M}^{n}M)$ is onto. Assume now that
$T'(\mathscr{M}^{i+1}M)\to T(\mathscr{M}^{i+1}M)$ is onto, and
consider the exact sequence: $0\to \mathscr{M}^{i+1}M\to
\mathscr{M}^{i}M\to \frac{\mathscr{M}^{i}M}{\mathscr{M}^{i+1}M}\to 0$
of $A$-modules.

We get a commutative diagram
\[
\xymatrix@=1.2cm{
T'(\mathscr{M}^{i+1}M)\ar[r]\ar[d] & T'(\mathscr{M}^{i}M)\ar[d]\ar[r]
&
T'\left(\frac{\mathscr{M}^{i}M}{\mathscr{M}^{i+1}M}\right)\ar[d]\ar[r]
&0 \\
T(\mathscr{M}^{i+1}M)\ar[r] & T(\mathscr{M}^{i}M)\ar[r] &
T\left(\frac{\mathscr{M}^{i}M}{\mathscr{M}^{i+1}M}\right) & 
}
\]
of\pageoriginale exact sequences (the second row is exact at the spot
$T(\mathscr{M}^{i}M)$ because $T$ is semi-exact, i.e., for $0\to M'\to
M\to M''\to 0$ exact, the sequence $T(M')\to T(M)\to T(M'')$ is exact
at the spot $T(M)$). The first vertical map is a surjection by
assumption, the last is a surjection since
$\dfrac{\mathscr{M}^{i}M}{\mathscr{M}^{i+1}M}$ is a finite direct sum
of copies of $k$ and by our assumption
$\Gamma(X_{y},\mathscr{O}_{X_{y}})\simeq k$, we have
$T'(k)\xrightarrow{\sim}T(k)$. It follows that $T'(M)\to T(M)$ is
onto, by a downward induction. Finally consider an exact sequence
$0\to R\to A^{p}\to M\to 0$. We then obtain a commutative diagram 
\[
\xymatrix@=1.2cm{
T'(R)\ar[r]\ar[d]^{\text{onto}} & T'(A^{p})\ar[d]^{\rotatebox{-90}{$\simeq$}}\ar[r] &
T'(M)\ar[d]\ar[r] & 0\\
T(R)\ar[r] & T(A^{p})\ar[r] & T(M) &
}
\]
of exact sequences. If follows easily that $T'(M)\to T(M)$ is also
injection. Thus, the theorem is completely proved in the case (a).

\medskip
{\bf Case (b).~The general case}
\smallskip

Let $A$ be a complete noetherian local ring. Denote by $Y'_{1}$ the
closed subscheme of $Y$ defined by the maximal ideal $\mathscr{M}$. As
in the comparison theorem we may define the functors
\begin{align*}
T_{n}(M_{n}) &=
\Gamma(X_{n},\foprod{\mathscr{O}_{X}}{\foprod{M}{A/\mathscr{M}^{n+1}}{A}}{A})\\[5pt] 
&= \Gamma(X,\foprod{\mathscr{O}_{X}}{\foprod{M}{A/\mathscr{M}^{n+1}}{A}}{A})
\end{align*}
where\pageoriginale $M_{n}=\foprod{M}{A/\mathscr{M}^{n+1}}{A}$. If $M$
is an $A$-module of finite type, so is
$\Gamma(X,\foprod{\mathscr{O}_{X}}{M}{A})$; as $A$ is complete under
the $\mathscr{M}$-adic topology, so is
$\Gamma(X,\foprod{\mathscr{O}_{X}}{M}{A})$ and from \ref{chap6-sec6.1.3} we
obtain: 
$$
\Gamma(X,\foprod{\mathscr{O}_{X}}{M}{A})\xrightarrow{\sim}\varprojlim_{n}T_{n}(M_{n}). 
$$

To show that $T$ is right-exact it is enough to show that for any
exact sequence $M\xrightarrow{u}N\to 0$ of finite type $A$-modules,
$T(M)\xrightarrow{T(u)}T(N)\to 0$ is again exact. But as each
$A/\mathscr{M}^{n+1}$ is a complete artinian local ring, it follows
from case (a) that $T_{n}(M_{n})\xrightarrow{T_{n}(u)}T_{n}(N_{n})\to
0$ is exact; also for each $n$, $\ker T_{n}(u)$ is a module of finite
length over $A$. Thus, it is enough now to prove the

\setcounter{lemma}{1}
\begin{lemma}\label{chap6-lem6.2.2}
Let $(K_{n},\varphi_{nm})$, $(M_{n},\psi_{nm})$,
$(N_{n},\theta_{nm})$, $n\in \mathbb{Z}^{+}$, be projective systems of
abelian groups and $u=(u_{n})$, $v=(v_{n})$ be morphisms such that,
for every $n\in \mathbb{Z}^{+}$ $0\to
K_{n}\xrightarrow{u_{n}}M_{n}\xrightarrow{v_{n}}N_{n}\to 0$ is
exact. Assume, in addition, that for each $n$, $\exists
m_{0}=m_{0}(n)$ such that
$\varphi_{nm}(K_{m})=\varphi_{nm_{0}}(K_{m_{0}})\ \forall\ m\geq
m_{0}(n)>n$ (this is the so-called Mittag-Leffler (ML) condition; it
is certainly satisfied if the $\ker u_{n}=K_{n}$ are of finite
length). Then the sequence of projective limits is also exact.
\end{lemma}

\begin{proof}
The\pageoriginale only difficult point is to show that
$\varprojlim_{n}v_{n}$ is onto. By hypothesis, for each $n$, $\exists
m_{0}(n)>n$ with
$\varphi_{nm}(K_{m})=\varphi_{nm_{0}}(K_{m_{0}})\ \forall\ m\geq
m_{0}(n)$. By passing to a cofinal subsystem we may suppose, that, for
any $n\in \mathbb{Z}^{+}$ 
$$
\varphi_{nm}(K_{m})=\varphi_{n,n+1}(K_{n+1})\ \forall\ m\geq n+1.
$$

Let now $(y_{n})\in\varprojlim_{n}N_{n}$. Choose $x'_{0}\in M_{0}$
with $v_{0}(x'_{0})=y_{0}$. Assume inductively that we have chosen
$(x_{0},x_{1},\ldots,x_{n-1},x'_{n})$ with (i)
$\psi_{r,r+1}(x_{r+1})\break =x_{r}$ for $0\leq r\leq n-2$, and
  $\psi_{n-1,n}(x'_{n})=x_{n-1}$. (ii) $v_{r}(x_{r})=y_{r}$ for $0\leq
  r\leq n-1$, and $v_{n}(x'_{n})=y_{n}$.

Our aim now is to find $(x_{0},x_{1},\ldots,x_{n-1},x_{n},x'_{n+1})$
for which the above properties (i), (ii) hold when $n$ is replaced by
$n+1$. Choose $x''_{n+1}\in M_{n+1}$ such that
$v_{n+1}(x''_{n+1})=y_{n+1}$; then,
$v_{n}(\psi_{n,n+1}(x''_{n+1})-x'_{n})=y_{n}-y_{n}=0$ i.e. to say,
$\psi_{n,n+1}(x''_{n+1})-x'_{n}\in K_{n}$. By assumption, we can find
$z_{n+1}\in K_{n+1}$ such that:
\begin{align*}
\psi_{n-1,n+1}(z_{n+1}) &= \varphi_{n-1,n+1}(z_{n+1})\\
&= \varphi_{n-1,n}(\psi_{n,n+1}(x''_{n+1})-x'_{n})\\
&= \psi_{n-1,n+1}(x''_{n+1})-\psi_{n-1,n}(x'_{n}).
\end{align*}

We now set $x'_{n+1}=(x''_{n+1}-z_{n+1})$ and
$x_{n}=\psi_{n,n+1}(x'_{n+1})$. 

We\pageoriginale then have:
\begin{align*}
\psi_{n-1,n}(x_{n}) &=
\psi_{n-1,n+1}(x''_{n+1})-\psi_{n-1,n+1}(z_{n+1})\\
&= \psi_{n-1,n}(x'_{n})=x_{n-1}.\\
v_{n+1}(x'_{n+1}) &= v_{n+1}(x''_{n+1})=y_{n+1}
\end{align*}
and finally $v_{n}(x_{n})=v_{n}(\psi_{n,n+1}(x'_{n+1}))=y_{n}$.\hfill Q.E.D.
\end{proof}


\section{The first homotopy exact sequence}\label{chap6-sec6.3}

\subsection{Some properties of the Stein-factorisation of a proper
  morphism.}\label{chap6-sec6.3.1}

Let $Y$ be locally noetherian and $f:X\to Y$ be {\em proper}. Suppose
$X\xrightarrow{f'}Y'\xrightarrow{q}Y$ is the Stein factorisation of
$f$. By using the comparison theorem one can prove the

\subsubsection{}\label{chap6-sec6.3.1.1}
({\em Zariski's connection theorem}).

The morphism $f':X\to Y'$ is also proper. And for any $y'\in Y'$ the
fibre ${f'}^{-1}(y')$ is {\em non-void and geometrically connected}
(i.e., for any field $k'\supset k(y')$ the prescheme $\fprod{X}{\Spec
  k'}{Y'}$ is connected).

(For a proof see EGA Ch. III \S\ (4.3)).

One can then easily draw the following corollaries.


\setcounter{subcoro}{1}
\begin{subcoro}\label{chap6-coro6.3.1.2}
For any $y\in Y$, the connected components of the fibre $f^{-1}(y)$
are in a $(1-1)$ correspondence with the set of points of the fibre
$q^{-1}(y)$ (which is finite and discrete).
\end{subcoro}

\begin{subcoro}\label{chap6-coro6.3.1.3}
For\pageoriginale any $y\in Y$, let $\overline{k(y)}$ be the algebraic
closure of $k(y)$ and $\overline{X}_{y}$ be the prescheme
$\fprod{X}{\overline{k(y)}}{Y}$. The connected components of
$\overline{X}_{y}$ (known as the geometric components of the fibre
over $y$) are in $(1-1)$ correspondence with the geometric points of
$Y'$ over $y$. 
\end{subcoro}

\subsection{}\label{chap6-sec6.3.2}
Now, in addition, suppose that
\begin{itemize}
\item[(i)] $Y$ is connected.

\item[(ii)] $f$ is separable

\item[(iii)] $f_{\ast}(\mathscr{O}_{X})\xleftarrow{\sim}\mathscr{O}_{Y}$.
\end{itemize}

Assumption (iii) implies that $X\xrightarrow{f}Y$ is its own
Stein-factorisation. From \ref{chap5-sec5.5.1}, it follows that the fibres
$f^{-1}(y)$ are (geometrically) connected. $f$ is a closed map and
therefore $X$ is also connected. Similarly, for any $y\in Y$,
$\overline{X}_{y}$ is also connected.

Fix $y\in Y$. We have a commutative diagram:
\[
\xymatrix@=1.2cm{
a\in X\ar[d]_{f} & \overline{X}_{y}\ni
\overline{a}\ar[l]\ar[d]^{\overline{f}}\\
y\in Y & \overline{k(y)}\ar[l] 
}
\]

Let $\Omega$ be an algebraically closed field $\supset k(y)$ and let
$\overline{a}\in\overline{X}_{y}$ be a geometric point over $\Omega$;
let $a\in X$ be the image of $\overline{a}$ in $X$. We have the
fundamental groups $\pi_{1}(\overline{X}_{y},\overline{a})$,
$\pi_{1}(X,a)$,\pageoriginale $\pi_{1}(Y,y)$ and continuous homomorphisms:
$\pi_{1}(\overline{X}_{y},\overline{a})\xrightarrow{\varphi}\pi_{1}(X,a)$,
$\pi_{1}(X,a)\xrightarrow{\psi}\pi_{1}(Y,y)$ (See \ref{chap5-sec5.1}). We
now have the following:

\setcounter{subsection}{2}
\begin{subtheorem}\label{chap6-thm6.3.2.1}
The sequence
$$
\pi_{1}(\overline{X}_{y},\overline{a})\xrightarrow{\varphi}\pi_{1}(X,a)\xrightarrow{\psi}\pi_{1}(Y,y)\to
0
$$
is {\em exact}.
\end{subtheorem}

\begin{proof}
\begin{itemize}
\item[{\bf (a)}] {\bf {\boldmath$\psi$} is surjective.}

In view of \ref{chap5-sec5.2.1} it is enough to show that, if $Y'/Y$ is
connected \'etale covering, then the \'etale covering
$X'=\fprod{X}{Y'}{Y}$ over $X$ is connected.
\[
\xymatrix@=1.2cm{
X\ar[d]_{f} & \fprod{X}{Y'}{Y}=X'\ar[l]\ar[d]^{f'\ \text{(proper)}}\\
Y & Y'\ \text{(connected)}\ar[l]
}
\]

But $\mathscr{O}_{Y}=f_{\ast}(\mathscr{O}_{X})$ and therefore, we have
(from \ref{chap6-thm6.1.2})
$$
f'_{\ast}(\mathscr{O}_{X'})=f'_{\ast}(\foprod{\mathscr{O}_{X}}{\mathscr{O}_{Y'}}{\mathscr{O}_{Y}})=
\foprod{f_{\ast}(\mathscr{O_X})}{\mathscr{O}_{Y'}}{\mathscr{O}_{Y}}=\mathscr{O}_{Y'}.
$$

It follows that the fibres of $f'$ are connected, and hence
$X'=\fprod{X}{Y'}{Y}$ is also connected.

\item[{\bf(b)}] {\bf {\boldmath$\psi\circ\varphi$} is trivial.}

In view of \ref{chap5-sec5.2.3} it is enough to show that if $Y'/Y$ is any
\'etale covering, the \'etale covering
$(\fprod{Y'}{\overline{X}_{y}}{Y})/\overline{X}_{y}$ is completely
decomposed.\pageoriginale But
$\fprod{Y'}{\overline{k(y)}}{Y}=\coprod\limits_{\text{finite}}
\overline{k(y)}$ and hence: $\fprod{Y'}{\overline{X}_{y}}{Y}\cong
\fprod{X}{(\fprod{Y'}{\overline{k(y)}}{\gamma})}=\coprod\limits_{\text{finite}}\overline{X}_{y}$. 

\item[{\bf(c)}] {\boldmath$\Iim\varphi\supset\ker \psi$.}

In view of \ref{chap5-sec5.2.4} it is enough to prove that: Suppose
$X'\xrightarrow{g}X$ is a connected \'etale covering of $X$ and
$\overline{X}'_{y}\xrightarrow{\overline{g}}\overline{X}_{y}$ admits a section
$\sigma$ (over $\overline{X}_{y}$). Then $\exists$ a connected \'etale
covering $Y'/Y$ such that $X'\xrightarrow{\sim}\fprod{X}{Y'}{Y}$. We
need, for proving this, the following
\end{itemize}
\end{proof}

\setcounter{sublemma}{1}
\begin{sublemma}\label{chap6-lem6.3.2.2}
The composite $h:X'\xrightarrow{g}X\xrightarrow{f}Y$ is proper and
separable. 
\end{sublemma}


\begin{proof}
$h$ is obviously proper and flat. We have only to show that the fibres
  of $h$ are reduced, and remain reduced after any base-change
  $Y\leftarrow \Spec K$, $K$ a field. For this, it is enough to prove
  that:
$$
X'/X\text{\ \'etale covering, $X$ reduced $\Rightarrow X'$ reduced.}
$$

We may assume $X=\Spec A$, $X'=\Spec A'$; we have $A=A_{\red}$ and we
want to show that, $A'=A'_{\red}$. 

Let $(\mathscr{P}_{i})^{n}_{i=1}$ be the minimal prime ideals of
$A$. By assumption, the natural map $A\to
\prod\limits^{n}_{i=1}(A/\mathscr{P}_{i})$ is an
injection. So,\pageoriginale $A'\to
\prod\limits^{n}_{i=1}(A'/\mathscr{P}_{i}A')$ is also injective and it
is enough then to show that each $A'/\mathscr{P}_{i}A'$ is reduced. By
making the base-change $A\to A/\mathscr{P}_{i}$, we may then assume
that $A$ is an integralldomain. Let $a\in X=\Spec A$ be the generic
point of $X$; then $k(a)=K$ the field of fractions of $A$. The fibre
over $a$ is $=\Spec(\foprod{A'}{K}{A})$ and since this is non-ramified
over $k(a)=K$, we have $\foprod{A'}{K}{A}=\sum\limits^{r}_{i=1}K_{i}$,
$K_{i/K}$ finite separable field extensions. In particular,
$\foprod{A'}{K}{A}$ is reduced and hence for each $x'\in X'$,
$\foprod{\mathscr{O}_{x'}}{K}{\mathscr{O}_{g(x')}}$ is reduced and
hence so is $\mathscr{O}_{x'}\subset
\foprod{\mathscr{O}_{x'}}{K}{g(x')}$. It follows that $A'=A'_{\red}$.

Coming back to the proof of the assertion, let now
$X'\xrightarrow{h'}Y'\to Y$ be the Stein-factorisation of $h:X'\to
Y$. From the above lemma \ref{chap6-lem6.3.2.2} and Theorem
\ref{chap6-thm6.2.1} it 
follows that $Y'\to Y$ is an \'etale covering. We have a commutative
diagram 
\[
\xymatrix@=1.2cm{
 & X'\ar[d]^{\alpha} \ar@/^2.5pc/[dd]^{h'}\ar[dl]\\
X\ar[d]_{f} & X''=\fprod{X}{Y'}{Y}\ar[l]^-{p_{1}}\ar[d]_{f'}\\
Y & Y'=\Spec h_{\ast}(\mathscr{O}_{X'})\ar[l]
}
\]

Our\pageoriginale assertion will follow if we show that $\alpha:X'\to
X''=\fprod{X}{Y'}{Y}$ is an isomorphism. We do this by showing that:
\begin{itemize}
\item[(i)] $\alpha$ is an \'etale covering.

\item[(ii)] $X''$ is connected.

\item[(iii)] rank of $\alpha$ is $1$ at {\em some} point of $X''$.
\end{itemize}

(see the remark at the end of Ch. \ref{chap3}).

\begin{itemize}
\item[(i)] $Y'\to Y$ is an \'etale covering and so $X''\to X$ is an
  \'etale covering. The composite $X'\to X''\to X$ is the \'etale
  covering $g$ and so $\alpha$ is an \'etale covering.

\item[(ii)] We know that $h'$ is onto (\ref{chap6-sec6.3.1.1}) and $X'$ is
  connected (hypothesis). Hence $Y'$ is connected and (ii) follows now
  from (a).

\item[(iii)] We make the base-change $Y\leftarrow \overline{k(y)}$ and
  obtain: 
\[
\xymatrix@=1.2cm{
 & &
  \overline{X'}_{y}\ar[dl]_{\overline{\alpha}}\ar[ddl]^{\overline{h}'}\ar[dll]_{\overline{g}}\\ 
\overline{X}_{y}\ar[d]^{\overline{f}}\ar@/^2pc/[urr]^{\sigma} &
\overline{X}''_{y}\ar[l]^{\overline{p}_{1}}\ar[d]  &\\
\overline{k(y)} & \overline{Y}'_{y}\ar[l] &
}
\]
\end{itemize}

It\pageoriginale is enough to show that rank $\overline{\alpha}=1$ at
{\em some} point of $\overline{X}''_{y}$. Since $Y'/Y$ is an \'etale
covering, $\overline{Y}'_{y}=\coprod\limits^{n}_{i=1}\overline{k}_{i}$
each $\overline{k}_{i}=\overline{k(y)}$ and so
$\overline{X}''_{y}=\fprod{\overline{X}_{y}}{\overline{Y}'_{y}}{\overline{k(y)}}=\coprod\limits^{n}_{i=1}\overline{X}_{y_{i}}$
each $\overline{X}_{y_{i}}=\overline{X}_{y}$. The section
$\sigma:\overline{X}_{y}\to \overline{X}'_{y}$ is an \'etale covering
and $\overline{X}_{y}$ is connected and hence
$\sigma(\overline{X}_{y})$ is a component $Z$ of $\overline{X}'_{y}$;
$\overline{\alpha}(Z)$ must then be some $\overline{X}_{y_{i}}$. Also
the projection $\overline{p}_{1}$ is an isomorphism from
$\overline{X}_{y_{i}}$ to $\overline{X}_{y}$; it follows that 
$\xymatrix{Z\ar@<.2em>[r]^{\overline{\alpha}} &
  \overline{X}_{y_{i}}\ar@<.2em>[l]^{(\sigma\circ \overline{p}_{1})}}$
are inverses of one another. Also we know that the number of
components of $\overline{X}'_{y}$ is equal to the number of geometric
points of $Y'$ over $y$, i.e., is $n$. It follows that the number of
connected components of $\overline{X}'_{y}$ and $\overline{X}''_{y}$
are the some. $\alpha$ is surjective because it is both open and
closed and $X''$ is connected, therefore $\overline{\alpha}$ is
surjective and \'etale and we get rank $\overline{\alpha}=1$ at every
pt. of $\overline{X}''_{y}$.\hfill Q.E.D.
\end{proof}

\setcounter{subremark}{2}
\begin{subremark}\label{chap6-rem6.3.2.3}
One may drop the assumptions
$f_{\ast}(\mathscr{O}_{X})=\mathscr{O}_{Y}$ and $Y$ connected, in the
above theorem; the assertion of the theorem will then be:

Denote by $\pi_{0}(\overline{X}_{y},\overline{a})$, $\pi_{0}(X,a)$,
$\pi_{0}(Y,y)$ the (pointed) sets of (connected) components of the
preschemes $\overline{X}_{y}$, $X$, $Y$. Then, if\pageoriginale $f$ is
proper and separable, we have the following exact sequence:
{\fontsize{10pt}{12pt}\selectfont
\begin{equation*}
\pi_{1}(\overline{X}_{y},\overline{a})\to \pi_{1}(X,a)\to
\pi_{1}(Y,y)\to \pi_{0}(\overline{X}_{y},\overline{a})\to
\pi_{0}(X,a)\to \pi_{0}(Y,y)\to (1)
\end{equation*}}\relax
\end{subremark}

\begin{proof}
Assume to start with that $X$, $Y$ are connected, dropping only the
assumption
$f_{\ast}(\mathscr{O}_{X})\xleftarrow{\sim}\mathscr{O}_{Y}$. We have
then the Stein-factorisation:
\[
\xymatrix@=1.2cm{
X\ar[rr]^{f}\ar[dr]_{f'} && Y\\
  & Y'=\Spec f_{\ast}(\mathscr{O}_{X}).\ar[ur]_{q}
}
\]

Applying the above theorem to $f'$ we obtain an exact sequence:
$\pi_{1}(\overline{X}_{y},\overline{a})\to \pi_{1}(X,a)\to
\pi_{1}(Y',y')\to (e)$ where $y'\in Y'$ is the image of $a\in X$. We
know then that $\pi_{1}(Y',y')\to \pi_{1}(Y,y)$ is an injection
(\ref{chap5-sec5.2.6}) and the quotient $\pi_{1}(Y,y)/\pi_{1}(Y',y')$ (set
of left cosets mod $\pi_{1}(Y',y')$) is isomorphic to the set of
geometric points of $Y'$ over $y$. By corollary \ref{chap6-coro6.3.1.3} this
is isomorphic to $\pi_{0}(\overline{X}_{y},\overline{a})$. We thus
obtain the exact sequence:
$$
\pi(\overline{X}_{y},\overline{a})\to \pi_{1}(X,a)\to \pi_{1}(Y,y)\to
\pi_{0}(\overline{X}_{y},\overline{a})\to (1).
$$

Now the assumptions about the connectivity of $X$ and $Y$ are dropped
in turn to get the general exact sequence.

The procedure is obvious and the proof is omitted.
\end{proof}
