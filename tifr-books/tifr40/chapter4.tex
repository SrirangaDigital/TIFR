\chapter{The Fundamental Group}\label{chap4}

Throughout\pageoriginale this chapter, we shall denote by $S$ a
locally noetherian {\em connected} prescheme and by
$\mathscr{C}=(\mathscr{E}t/S)$ the category of \'etale coverings of
$S$. We note that the morphisms of $\mathscr{C}$ will all be \'etale
coverings. 

(See \ref{chap3-sec3.3.3} and the note at the end of Ch. \ref{chap3}).


\section{Properties of the category
  {$\mathscr{C}$}}\label{chap4-sec4.1} 

\begin{itemize}
\item[($\mathscr{C}_{0}$)] $\mathscr{C}$ has an initial object
  $\emptyset$ (the empty prescheme) and a final object $S$.

\item[($\mathscr{C}_{1}$)] Finite fibre-products exist in
  $\mathscr{C}$, i.e., if $X\to Z$ and $Y\to Z$ are morphisms in
  $\mathscr{C}$, then $\fprod{X}{Y}{Z}$ exists in $\mathscr{C}$ (see
  (\ref{chap3-sec3.3.3})) 

\item[($\mathscr{C}_{2}$)] If $X$, $Y\in\mathscr{C}$, then the
  disjoint union $X\coprod Y\in\mathscr{C}$ (obvious).

\item[($\mathscr{C}_{3}$)] Any morphism $u:X\to Y$ in $\mathscr{C}$
  admits a factorisation of the form
\[
\xymatrix@=1.2cm{
X\ar[rr]^{u}\ar[dr]_{u_{i}} & & Y\\
 & Y_{1}\ar@{^{(}->}[ur]_{j}
}
\]
where $u_{1}$ is an effective epimorphism, $j$ is a monomorphism and
$Y=Y_{1}\coprod Y_{2}$, $Y_{2}\in\mathscr{C}$.
\end{itemize}

In\pageoriginale fact, $u$ is an \'etale covering so $u$ is both open
and closed; if we write $u(X)=Y_{1}$ we have $Y=Y_{1}\coprod Y_{2}$
and $u=u_{1}:X\to Y_{1}$ is then an effective epimorphism

(see (\ref{chap3-prop3.4.2.1})).

Further, this factorisation of $u$ into an epimorphism and a
monomorphism is {\em essentially unique} in the sense that if
\[
\xymatrix@=1.2cm{
X\ar[rr]^{u}\ar[dr]_{u'_{1}} & & Y\\
 & Y_{1}\ar@{^{(}->}[ur]_{j'}
}
\]
is another such factorisation, then there exists an isomorphism
$\omega:Y_{1}\to Y'_{1}$ such that $u'_{1}=\omega\circ u_{1}$ and
$j=j'\circ \omega$.

(Because a factorisation into a product of an effective epimorphism
and a monomorphism is unique).

\begin{itemize}
\item[($\mathscr{C}_{4}$)] If $X\in \mathscr{C}$ and $\mathfrak{g}$ is
  a {\em finite} group of automorphisms of $X$ acting, say, to the
  right on $X$, then the quotient $X/\mathfrak{g}$ of $X$ by
  $\mathfrak{g}$ exists in $\mathscr{C}$ and the natural morphism
  $X\xrightarrow{\eta}X/\mathfrak{g}$ is an effective epimorphism.
\end{itemize}

The quotient, if it exists, is evidently unique upto a canonical
isomorphism; also $X$ is affine over $S$ and therefore the existence
of $X/\mathfrak{g}$ has only to be proved in the case $X$, $S$ affine,
say $X=\Spec A$, $S=\Spec B$ and $G=$ the group of $B$-automorphisms
of $A$ corresponding to $\mathfrak{g}$. Then $\Spec A^{G}$ ($A^{G}$ is
the ring of $G$-invariants of $A$) is the quotient we are looking
for. (Compare with Serre, Groupes alg\'ebriques et corps de classes,
p. 57). Our aim now is to show that $X/\mathfrak{g}$ is actually in
$\mathscr{C}$. The question is again local and we may assume $X=\Spec
A$, $S=\Spec B$, with\pageoriginale $B$ noetherian, and
$X/\mathfrak{g}=\Spec A^{G}$ as above. $X\to S$ is finite and so
$X/\mathfrak{g}\to S$ is also finite. It remains to show that this
morphism is \'etale. In order to do this, we first make some
simplifications. 

Suppose that $S'\to S$ is a flat affine base-change. We have a
commutative diagram:
\[
\xymatrix{
X'=\fprod{X}{S'}{S}\ar[d]\ar[r] & Y'=\fprod{Y}{S'}{S}\ar[d]\ar[r] &
\Spec B'=S'\ar[d]\\
\Spec A=X \ar[r] & X/\mathfrak{g}=\Spec A^{G}=Y\ar[r] & \Spec B=S.
}
\]
$\mathfrak{g}$ acts on $X'$ in the obvious way, as a group of
$S'$-automorphisms of $X'$. We assert that $Y'=\fprod{Y}{S'}{S}$ is
the quotient of $X'$ with respect to this action. Indeed, we have an
exact sequence of $B$-algebras: $0\to A^{G}\to A\to
\bigoplus\limits_{\sigma\in G}A$, where $A\to
\bigoplus\limits_{\sigma\in G}A$ is the map given by $A\mapsto
\sum\limits_{\sigma\in G}(a-a^{\sigma})$. Since $B'$ is $B$-flat, we
get an exact sequence: $0\to \foprod{A^{G}}{B'}{B}\to
\foprod{A}{B'}{B}\to \bigoplus\limits_{\sigma \in
  G}(\foprod{A}{B'}{B})$ and this proves that the subring of
invariants of $\foprod{A}{B'}{B}$ is $\foprod{A^{G}}{B'}{B}$; hence
our assertion.

Let $y\in Y=X/\mathfrak{g}$ and $s\in S$ be its image. Take for $B'$
the local ring $\mathscr{O}_{s,S}$. Then there is a unique point
$y'\in Y'=\fprod{Y}{S'}{S}$ over $y$ and one has
$\mathscr{O}_{y',Y'}=\mathscr{O}_{y,Y}$; hence\pageoriginale $Y\to S$
is \'etale at $Y\Leftrightarrow Y'\to S'$ is \'etale at $y'$. We may
thus assume that $S=\Spec B$, $B$ a noetherian, local ring. In view of
the following lemma, we may assume $B$ {\em complete}.

\begin{lemma}\label{chap4-lem4.1.1}
Let $X\xrightarrow{f}S$ be a morphism and $S'\xrightarrow{\varphi}S$ a
faithfully flat base-change. Then $f$ is \'etale $\Leftrightarrow
f_{(S')}$ is \'etale.
\end{lemma}


\begin{proof}
$\Rightarrow$: is clear.

$\Leftarrow$: flatness of $f$ is straightforward. To prove
  non-ramification one observes that in view of (5)
  (\ref{chap3-sec3.3.0}), one has 
  $\Omega_{X'/S'}=\varphi^{\ast}(\Omega_{X/S})$; but $\varphi$ being
  faithfully flat, $\Omega_{X'/S'}=0\Leftrightarrow \Omega_{X/S}=0$;
  one now applies proposition \ref{chap3-prop3.3.2}.\hfill Q.E.D.
\end{proof}

Let $x_{1},\ldots,x_{n}$ be the points of $X$ over $s$. By hypothesis
each $k(x_{i})/\break k(s)$ is a finite separable extension. We choose a
sufficiently large finite galois extension $K$ of $k(s)$ such that
each $k(x_{i})$ is imbedded in $K$. We now need the 

\begin{lemma}\label{chap4-lem4.1.2}
Let $B$ be a noetherian local ring with maximal ideal $\mathscr{M}$
and residue field $k$. Let $K$ be an extension field of $k$. Then
$\exists$ a noetherian local ring $C$ and a local homomorphism
$\varphi:B\to C$ such that (i) $\varphi$ is $B$-flat and (ii)
$C/\mathscr{M}C\cong K$. (see EGA. Ch. $\text{0}_{\text{III}}$,
Prop. (10.3.1)).
\end{lemma}

In\pageoriginale addition if $[K:k]<\infty$, we can choose $C$ to be a
finite $B$-algebra. (EGA. Ch. $0_{\text{III}}$, Cor. (10.3.2)).

By making such a base-change $B\xrightarrow{\varphi}C$ we may assume
that each $k(x_{i})$ is trivial over $k(s)$. Under these assumptions
we get $A=\bigoplus\limits^{r}_{i=1}B$, a finite direct sum of copies
of $B$. Under the action of $\mathfrak{g}$, the set
$\{x_{1},\ldots,x_{n}\}$ splits into disjoint subsets
$\{x_{1},\ldots,x_{l}\}$, $\{x_{l+1},\ldots,x_{m}\},\ldots$, on each
of which $\mathfrak{g}$ acts transitively. The corresponding
decomposition of $A$ will then be given by
$A=(\bigoplus\limits^{l}_{i=1}B)\bigoplus
(\bigoplus\limits^{m}_{i=l+1}B)\bigoplus\ldots$. The action of $G$ on
each block, for instance, on a $(b_{1},\ldots,b_{l})\in
\bigoplus\limits^{l}_{i=1}B$, will then be just a permutation. The
subring $A^{G}$ will then be the direct sum
$\Delta_{1}\oplus\ldots\oplus \Delta_{\alpha}\oplus\ldots$ where
$\Delta_{1}$ is the diagonal of the block
$\bigoplus\limits^{l}_{i=1}B$ and so on. Each $\Delta$ is evidently
isomorphic to $B$. Our assertion that $X/\mathfrak{g}\to S$ is \'etale
is now clear.

Thus $X/\mathfrak{g}\in\mathscr{C}$; the natural morphism
$X\xrightarrow{\eta}X/\mathfrak{g}$ is also then an \'etale
covering. Therefore $\eta$ will be an open map and thus if $\eta$ is
not surjective one could replace $Y$ by the image of $\eta$; this is
clearly impossible. Hence $\eta$ is surjective and thus an effective
epimorphism \ref{chap3-prop3.4.2.1}.

\section{}\label{chap4-sec4.2}

We shall\pageoriginale now define a covariant functor $F$ from
$\mathscr{C}(\mathscr{E}t/S)$ to the category of finite sets. We shall
fix once and for all a point $s\in S$ and an algebraically closed
field $\Omega\supset k(s)$.

For any $X\in\mathscr{C}$, $F(X)$, by definition, will be the set of
geometric points of $X$ over $s\in S$, with values in $\Omega$, i.e.,
is the set of all $S$-morphisms $\Spec \Omega\to X$ for which the
diagram is commutative.
\[
\xymatrix{
\Spec \Omega\ar[r]\ar[d] & X\ar[d]\\
\Spec k(s)\ar[r] & S
}
\] 

We observe that if $x\in X$ sits above $s\in S$, then giving an
$S$-morphism $\Spec \Omega\to X$ whose image is $x\in X$ is equivalent
to giving a $k(s)$-monomorphism of $k(x)$ into $\Omega$. Also note
that for any $X\in \mathscr{C}$, $F(X)$ is a finite set whose
cardinality equals the rank of $X$ over $S$.

\medskip
\noindent
{\bf Properties of the functor {\boldmath$F$}.}
\smallskip

\begin{itemize}
\item[$(F_{0})$] $F(X)=\emptyset\Leftrightarrow X=\emptyset$.

\item[$(F_{1})$] $F(S)=$ a set with one element;

$F(\fprod{X}{Y}{Z})=\fprod{F(X)}{F(Y)}{F(Z)}$, $\forall X$, $Y$, $Z\in
  \mathscr{C}$. 

\item[$(F_{2})$] $F(X_{1}\coprod X_{2})=F(X_{1})\coprod F(X_{2})$. 

\item[$(F_{3})$] If\pageoriginale $X\xrightarrow{u}Y$ is an effective
  epimorphism in $\mathscr{C}$, the map $F(u):F(X)\to F(Y)$ is {\em onto}.

In fact, if $y\in Y$ is a point above $s\in S$ and $y=u(x)$, $x\in X$,
then any $k(s)$-monomorphism of $k(y)$ to $\Omega$ extends to a
$k(s)$-monomorphism of $k(x)$ to $\Omega$.

\item[$(F_{4})$] Let $X\in\mathscr{C}$ and $\mathfrak{g}$ a finite
  group of $S$-automorphisms of $X$ (acting to the right on $X$). Then
  $\mathfrak{g}$ acts in a natural way (again to the right) on $F(X)$
  as expressed by
$$
\Spec \Omega\to X\xrightarrow{\sigma\in\mathfrak{g}}X.
$$

The natural map $\eta:X\to X/\mathfrak{g}$ (see ($\mathscr{C}_{3}$))
defines a surjection $F(\eta):F(X)\to F(X/\mathfrak{g})$ (see
($F_{3}$)). In view of the commutativity of the diagram
\[
\xymatrix{
\Spec \Omega \ar[r] & X\ar[dr]_{\eta} \ar[rr]^{\sigma\in \mathfrak{g}}
& & X\ar[dl]_{\eta}\\ 
 & & X/\mathfrak{g} & &
}
\]
it follows that $F(\eta)$ descends to a surjection
$\widetilde{\eta}:F(X)/\mathfrak{g}\to F(X/\mathfrak{g})$. We claim
that $\widetilde{\eta}$ {\em is actually a bijection.} This follows
immediately from the
\end{itemize}

\begin{lemma}\label{chap4-lem4.2.1}
\begin{itemize}
\item[\rm(i)] $\mathfrak{g}$ acts transitively on the fibres of $\eta$. 

\item[\rm(ii)] Suppose\pageoriginale $y\in Y=X/\mathfrak{g}$ and $x\in
  \eta^{-1}(y)$. Let $\mathfrak{g}_{d}(x)$ be the subgroup
  $\{\sigma\in\mathfrak{g}:\sigma(x)=x\}$ (called the {\em
    decomposition group of $x$}). Then we have:
\begin{itemize}
\item[\rm(a)] $k(x)/k(y)$ is a galois extension.

\item[\rm(b)] the natural map $\mathfrak{g}_{d}(x)\to$ the galois group
  $G({}^{k(x)}/k(y))$ is {\em onto}.
\end{itemize}
\end{itemize}
\end{lemma}

\begin{proof}
\begin{itemize}
\item[\rm(i)] We may assume $X=\Spec A$, $Y=X/\mathfrak{g}=\Spec A^{G}$
  as before (both noetherian); also one knows that $A$ is finite on
  $A^{G}$. Let $\mathscr{P}$, $\mathscr{P}_{1}$ be prime ideals of $A$
  (i.e., points of $X$) such that $\mathscr{P}_{1}\neq
  \sigma\mathscr{P}\ \forall\ \sigma \in G$, while $\mathscr{P}\cap
  A^{G}=\mathscr{P}_{1}\cap A^{G}=\underline{p}$. We may assume
  $\mathscr{P}$ and $\mathscr{P}_{1}$ maximal (otherwise apply the
  flat base change $Y\leftarrow \Spec \mathscr{O}_{y,Y}$, where $y\in
  Y$ corresponds to $\underline{p}$). Then there is an $a\in
  \mathscr{P}_{1}$ such that $a\not\in \sigma \mathscr{P}\forall
  \sigma$ (Chinese Remainder Theorem). Thus
  $b=\prod\limits_{\sigma}\sigma(a)\not\in \mathscr{P}$; but $b\in
  A^{G}$, so $b\in A^{G}\cap \mathscr{P}_{1}=A^{G}\cap
  \mathscr{P}$-contradiction. 

\item[\rm(ii)] We have a diagram:
\[
\xymatrix{
A\ar[r] & A/\mathscr{P}\simeq k(x)\\
A^{G}\ar[u]\ar[r] & A^{G}/\ub{p}\simeq k(y),\ar[u]
}
\]
and\pageoriginale we know that $k(x)/k(y)$ is a finite, separable
extension. Let $\theta\in A$ be such that
$k(x)=k(y)(\overline{\theta})$. The polynomial
$f=\prod\limits_{\sigma}(T-\sigma \theta)$ is in $A^{G}[T]$, has
$\theta$ as a root and splits completely in $A[T]$. The reduction
$\overline{f}$ of $f$ $\mod \ub{p}$ is in $k(y)[T]$, has
$\overline{\theta}$ as a root and splits completely in $k(x)[T]$. It
follows that $k(x)/k(y)$ is normal, hence galois. 

Consider now the subgroup $G_{d}(\mathscr{P})$ of $G$ corresponding to
$\mathfrak{g}_{d}(x)$. We have $\mathscr{P}\neq
\sigma^{-1}\mathscr{P}\ \forall\ \sigma \not\in G_{d}(\mathscr{P})$ and
by the Chinese Remainder Theorem we can choose a $\theta_{1}\in A$
such that $\theta_{1}\equiv \theta(\text{mod}\mathscr{P})$ and
$\theta_{1}\equiv 0(\text{mod} \sigma^{-1}\mathscr{P})$, $\forall
\sigma\not\in G_{d}(\mathscr{P})$. 

We have $k(x)=k(y)(\overline{\theta}_{1})$. Consider now the
polynomial
$\overline{g}=\prod\limits_{\sigma}(T-\overline{\sigma}\theta_{1})\in
  k(y)[T]$. As $\overline{\theta}_{1}$ is a root of $\overline{g}$,
  for every $\varphi\in G(^{k(x)}/k(y))$,
  $\varphi(\overline{\theta}_{1})$ is also a root of $\overline{g}$;
  hence
  $\varphi(\overline{\theta}_{1})=\overline{\sigma}(\overline{\theta}_{1})$
  for some $\sigma \in G$. But $\varphi(\overline{\theta}_{1})\neq 0$
  and, by the choice of $\theta_{1}$,
  $\overline{\sigma}(\overline{\theta}_{1})=0$ if $\sigma\not\in
  G_{d}(\mathscr{P})$; hence
  $\varphi(\overline{\theta}_{1})=\overline{\sigma}(\overline{\theta}_{1})$
  for some $\sigma\in G_{d}(\mathscr{P})$, i.e.,
  $\varphi=\overline{\sigma}$ for some $\sigma \in
  G_{d}(\mathscr{P})$.\hfill Q.E.D.
\end{itemize}
\end{proof}

\begin{itemize}
\item[$(F_{5})$] If $u:X\to Y$ is a morphism in $\mathscr{C}$ such
  that $F(u):F(X)\to F(Y)$ is a bijection, then $u$ is an isomorphism
\end{itemize}

From\pageoriginale the fact that $F(u)$ is a bijection it follows (see
the remark at the end of Ch. \ref{chap3}) that the rank of $u:X\to Y$ is $1$
at every $y\in Y$, hence (again by the same remark) $u$ is an
isomorphism.

A category which has the properties
$(\mathscr{C}_{0}),\ldots,(\mathscr{C}_{4})$ of \ref{chap4-sec4.1} and from
which there is given a functor $F$ into finite sets with the above
properties $(F_{0}),\ldots, (F_{5})$ is called a {\em galois
  category}; the functor $F$ itself is known as a {\em fundamental
  functor}. 

\section{}\label{chap4-sec4.3}
Before we start our construction of the fundamental group of a galois
category we motivate our procedure by two examples.

\begin{exam}\label{chap4-exam1}
Let $S$ be a connected, locally arcwise connected, locally simply
connected topological space and $\mathscr{C}$ the category of
connected coverings of $S$; the morphisms of $\mathscr{C}$ are
covering maps. Let $X\xrightarrow{p}S$ be such a covering. Fix a point
$s\in S$; we define $F(X)=p^{-1}(s)$. Then $(X,p)\mapsto p^{-1}(s)$ is
a covariant functor $F:\mathscr{C}\to \calEns$.
\end{exam}

Each member of $\mathscr{C}$ determines (upto conjugacy) a sub-group
of the fundamental group $\Pi_{1}(S,s)$; and to each subgroup $H$ of
$\Pi_{1}(S,s)$ there corresponds a member of $\mathscr{C}$
determining $H$. To the subgroup $\{e\}$ corresponds, what is known
as, the universal covering $\widetilde{S}$ of $S$; and $\Pi_{1}(S,s)$
is isomorphic to the group of $S$-automorphisms of $\widetilde{S}$,
i.e., to the group of covering transformations of $\widetilde{S}$ over
$S$. Further we have the isomorphism (in $\calEns$):
$$
\Hom_{\mathscr{C}}(\widetilde{S},X)\xrightarrow{\sim}F(X),\quad
\forall X\in \mathscr{C}. 
$$

The\pageoriginale functor $F:\mathscr{C}\to \calEns$ is thus {\em
  representable} in the following sense.

\begin{defin}\label{chap4-defi4.3.1}
A covariant functor $\mathscr{G}$ from a category $\mathscr{C}$ to
$\calEns$ is {\em representable} if $\exists$ an object
$Y\in\mathscr{C}$ such that:
$$
\Hom_{\mathscr{C}}(Y,X)\xrightarrow{\sim}\mathscr{G}(X)\ \forall\
X\in\mathscr{C}.
$$
\end{defin}

\begin{exam}\label{chap4-exam2}
Let $k$ be a field and $\Omega$ a fixed algebraically closed field
extension of $k$. Set $S=\Spec k$ and $\mathscr{C}=$ the category of
connected \'etale coverings of $S$; any member of $\mathscr{C}$ is of
the form $\Spec K$, where $K/k$ is a finite separable field
extension. For any $X\in\mathscr{C}$ we define $F(X)=$ the set of
geometric points of $X$ with values in $\Omega$. Then, $F(X)\simeq
\Hom_{k}(K,\Omega_{s})$ is $X=\Spec K$, where $\Omega_{s}$ is the
separable closure of $k$ in $\Omega$. If $\Omega_{s}$ is finite over
$k$, we can further write $F(X)\simeq \Hom_{\mathscr{C}}(\Spec
\Omega_{s},X)$ and the functor $F:\mathscr{C}\to$ (finite sets),
defined above, will be representable. However this is {\em not} the
case in general; out we can find an indexed, filtered family
$(N_{i})_{i\in I}$ of finite galois extensions of $k$, (namely, the
set of finite galois extensions contained in $\Omega_{s}$) such that
for any $X\in\mathscr{C}$, we can find an $i_{0}=i_{0}(X)$ such that
$F(X)\simeq \Hom_{\mathscr{C}}(\Spec N_{i},X)$, $\forall i\geq
i_{0}(X)$. In other words, we may write
$$
F(X)\simeq \varinjlim_{i\in I} \Hom_{\mathscr{C}}(\Spec
  N_{i},X),\quad \forall X\in \mathscr{C}
$$
and\pageoriginale the family $(\Spec N_{i})_{i\in I}$ is in fact a
projective family of objects in $\mathscr{C}$.
\end{exam}

Suppose now that $\mathscr{C}$ is any category and
$\mathscr{G}:\mathscr{C}\to \calEns$ is a covariant functor. If $X\in
\mathscr{C}$ and $\xi\in \mathscr{G}(X)$, we write, as a matter of
notation, $\mathscr{G}\xrightarrow{\xi}X$. If
$\mathscr{G}\xrightarrow{\xi}X$, and $\mathscr{G}\xrightarrow{\eta}Y$
and $X\xrightarrow{u}Y$ is a $\mathscr{C}$-morphism, we say that the
diagram
\[
\xymatrix{
\mathscr{G}\ar[rr]^{\xi}\ar[dr]^{\eta} && X\ar[dl]^{u}\\
 & Y &
}
\]
is commutative if $\mathscr{G}(u)(\xi)=\eta$.

If $\mathscr{G}\xrightarrow{\xi}X$, then for any $Z\in\mathscr{C}$, we
have a natural map $\Hom_{\mathscr{C}}(X,Z)\break \to \mathscr{G}(Z)$ defined
by $u\mapsto \mathscr{G}(u)(\xi)$. 

\begin{defin}\label{chap4-defi4.3.2}
We say that $\mathscr{G}$ is {\em pro-representable} if $\exists$ a
projective system $(S_{i},\varphi_{ij})_{i\in I}$ of objects of
$\mathscr{C}$ and elements $\tau_{i}\in\mathscr{G}(S_{i})$ (called the
canonical elements of $\mathscr{G}(S_{i})$) such that 
\begin{itemize}
\item[\rm(i)] the diagrams
\[
\xymatrix{
\mathscr{G}\ar[rr]^{\tau_{i}}\ar[dr]_{\tau_{j}} && S_{i}\\
 & S_{j}\ar[ur]_{\varphi_{ij}(j\geq i)} & 
}
\]
are commutative.

\item[\rm(ii)] for\pageoriginale any $Z\in\mathscr{C}$, the (natural) map
$$
\varinjlim_{i\in I} \Hom_{\mathscr{C}}(S_{i},Z)\to F(Z)
$$
is {\em bijective.}
\end{itemize}
\end{defin}

In addition, if the $\varphi_{ij}$ are {\em epimorphisms} of
$\mathscr{C}$, we say that $\mathscr{G}$ is {\em strictly
  pro-representable.}

Thus, our functor $F$ in Example \ref{chap4-exam2} is
pro-representable. Example \ref{chap4-exam1} and \ref{chap4-exam2} show that
representable and pro-representable functors arise naturally in the
consideration of the fundamental group.

\section{Construction of the Fundamental group}\label{chap4-sec4.4}

\subsection{Main theorem}\label{chap4-sec4.4.1}

\begin{enumerate}
\renewcommand{\labelenumi}{(\theenumi)}
\item Let $\mathscr{C}$ be a galois category with a fundamental
  functor $F$. Then there exists a pro-finite group $\pi$ (i.e., a
  group $\pi$ which is a projective limit of finite discrete groups
  provided with the limit topology) such that $F$ is an equivalence
  between $\mathscr{C}$ and the category $\mathscr{C}(\pi)$ of finite
  sets on which $\pi$ acts continuously.

\item If $\mathscr{C}\xrightarrow{F'}\mathscr{C}(\pi')$ is another
  such equivalence, then $\pi'$ is continuously isomorphic to $\pi$
  and this isomorphism between $\pi$ and $\pi'$ is canonically
  determined upto an inner automorphism of $\pi$.
\end{enumerate}

The\pageoriginale profinite group $\pi$, whose existence is envisaged
in assertion (1) above will be called the {\em fundamental group} of
the galois category $\mathscr{C}$.

The theorem is a consequence of the following series of lemmas.

\setcounter{subsubsection}{1}
\setcounter{subdefin}{0}
\begin{subdefin}\label{chap4-defi4.4.1.1}
A category $\mathscr{C}$ is {\em artinian} if any ``decreasing''
sequence
$$
T_{1}\hookleftarrow_{j_{1}}T_{2}\hookleftarrow_{j_{2}}T_{3}\hookleftarrow_{j_{3}}\ldots 
$$
of monomorphisms in $\mathscr{C}$ is stationary, i.e., the $j_{r}$ are
isomorphisms for large $r$.
\end{subdefin}

A (covariant) functor $F:\mathscr{C}\to \calEns$ is {\em left-exact}
if it commutes with finite products i.e., if $F(X\times Y)=F(X)\times
F(Y)$ and if, for every {\em exact} sequence $\xymatrix{X\ar[r]^{u} &
  Y\ar@<.2em>[r]^{u_{1}}\ar@<-.2em>[r]_{u_{2}} & Z}$ in $\mathscr{C}$,
the sequence  
\[
\xymatrix{F(X)\ar[r]^{F(u)} &
  F(Y)\ar@<.2em>[r]^{F(u_{1})}\ar@<-.2em>[r]_{F(u_{2})} & F(Z)}
\] 
is exact as a sequence of sets.

If $\xymatrix{Y\ar@<.2em>[r]^{u_{1}}\ar@<-.2em>[r]_{u_{2}} & Z}$
 are morphisms in $\mathscr{C}$, a {\em kernel} for $u_{1}$, $u_{2}$
 in $\mathscr{C}$ is a pair $(X,u)$ with $X\in \mathscr{C}$ and
 $u:X\to Y$ in $\mathscr{C}$ such that $\xymatrix{X\ar[r]^{u} &
  Y\ar@<.2em>[r]^{u_{1}}\ar@<-.2em>[r]_{u_{2}} & Z}$ is exact in
 $\mathscr{C}$. Clearly a kernal is determined uniquely upto an
 isomorphism in $\mathscr{C}$. 

\setcounter{sublemma}{1}
\begin{sublemma}\label{chap4-lem4.4.1.2}
Let\pageoriginale $\mathscr{C}$ be a category in which finite products
exists. Then finite fibre-products exist in
$\mathscr{C}\Leftrightarrow$ kernels exist in $\mathscr{C}$.
\end{sublemma}

\begin{proof}
$\Rightarrow$~: Let
  $\xymatrix{Y\ar@<.2em>[r]^{u_{1}}\ar@<-.2em>[r]_{u_{2}} & Z}$ be
  morphisms in $\mathscr{C}$. We have a commutative diagram:
\[
\xymatrix{
 & Z & \\
Y\ar[ur]^{u_{1}}  & & Y\ar[ul]_{u_{2}}\\
 &
\fprod{Y}{Y}{Z}\ar[ul]^{p_{1}}\ar[ur]_{p_{2}}\ar[dr]^{(p_{1},p_{2})}
&\\
\fprod{(\fprod{Y}{Y}{Z})}{Y}{(Y\times Y)}\ar[ur]\ar[dr] & & Y\times
Y\\
 & Y\ar[ur]_{\text{diagonal}} &  
}
\]
and it easily follows that $\fprod{(\fprod{Y}{Y}{Z}}{Y}{(Y\times Y)}$
is a solution for the kernal of $u_{1}$ and $u_{2}$.

$\Leftarrow$~: Suppose $X\xrightarrow{f}Z$ and $Y\xrightarrow{g}Z$ are
morphisms in $\mathscr{c}$. If $p$ and $q$ are the canonical
projections $X\times Y\xrightarrow{p}X$, $X\times Y\xrightarrow{q}Y$,
we have an exact sequence:
\[
\xymatrix{
\ker (fp,gq)\ar[r] & X\times Y\ar@<.2em>[r]^-{f_{p}}\ar@<-.2em>[r]_-{gq}
& Z. 
}
\]

It\pageoriginale follows that $\ker(fp,gq)$ is a solution for the
fibre-product $\fprod{X}{Y}{Z}$.\hfill Q.E.D.
\end{proof}

In fact, we have shown that finite fibre products and kernels can be
expressed in terms of each other. Hence, $F$ commutes with finite
fibre-products $\Leftrightarrow$ it is left-exact.

\begin{coro*}
A fundamental functor is left-exact (see ($F_{1}$))
\end{coro*}

\begin{sublemma}\label{chap4-lem4.4.1.3}
A galois category is artinian.
\end{sublemma}

\begin{proof}
Let 
$$
T_{1}\hookleftarrow_{j_{1}}T_{2}\hookleftarrow_{j_{2}}\ldots
\hookleftarrow_{j_{r-1}}T_{r}\hookleftarrow_{j_{r}}T_{r+1}\ldots
$$
be a decreasing sequence of monomorphisms in $\mathscr{C}$. We have
then:
\begin{align*}
& T_{r+1}\xrightarrow{j_{r}}T_{r}\quad\text{is a monomorphism}\\
\Leftrightarrow\ &
T_{r+1}\xrightarrow[\Delta]{\sim}\fprod{T_{r+1}}{T_{r+1}}{T_{r}}\\
\Rightarrow\ &
F(T_{r+1})\xrightarrow[\Delta]{\sim}\fprod{F(T_{r+1})}{F(T_{r+1})}{F(T_{r})}\quad
(\text{by\ } (F_{1}), (F_{5}))\\
\Leftrightarrow\ &
F(T_{r+1})\xrightarrow{F(j_{r})}F(T_{r})\quad\text{is a monomorphism.}  
\end{align*}

Since the $F(T_{r})$ are finite, this implies that the $F(j_{r})$ are
isomorphisms for large $r$; we are through by $(F_{5})$.\hfill Q.E.D.
\end{proof}


\begin{sublemma}\label{chap4-lem4.4.1.4}
Let\pageoriginale $\mathscr{C}$ be a galois category with a
fundamental functor $F$. Then $F$ is strictly pro-representable.
\end{sublemma}

\begin{proof}
With the notations of \ref{chap4-defi4.3.2}, consider the set
$\mathscr{E}$ of pairs $(X,\xi)$ with $F\xrightarrow{\xi}X$. We order
$\mathscr{E}$ as follows: 
$$
(X,\xi)\geq (X',\xi')\Leftrightarrow \exists\quad\text{a commutative diagram:}
$$
\[
\xymatrix{
F\ar[rr]^{\xi}\ar[dr]_{\xi'} & & X\ar[dl]\\
 & X' &
}
\]

We claim that $\mathscr{E}$ is {\em filtered} for this ordering; in
fact, if $(X,\xi)$, $(X',\xi')\break \in\mathscr{E}$, in view of $(F_{1})$ we
get a commutative diagram:
\[
\xymatrix@C=1.4cm@R=1.5cm{
 & & X\\
F\ar[urr]^{\xi}\ar[r]^{(\xi,\xi')}\ar[drr]^{\xi'} & X\times
X'\ar[dr]^{p'}\ar[ur]_{p} & \\ 
 & & X'
}
\]
where $p$ and $p'$ are the natural projections.

We\pageoriginale say that a pair $(X,\xi)\in\mathscr{E}$ is {\em
  minimal} in $\mathscr{E}$ if for any commutative diagram
\[
\xymatrix@=1.2cm{
F\ar[rr]^{\xi}\ar[dr]_{\eta} & & X\\
 & Y\ar@{^{(}->}[ur]_{j} & 
}
\]
with a monomorphism $j$, one necessarily has that $j$ is an
isomorphism.
\begin{itemize}
\item[(*)] Every pair in $\mathscr{E}$ is dominated, in this ordering,
  by a minimal pair in $\mathscr{E}$.

Observe that $\mathscr{C}$ is artinian (Lemma \ref{chap4-lem4.4.1.3}).

\item[(**)] If $(X,\xi)\in\mathscr{E}$ is minimal and
  $(Y,\eta)\in\mathscr{E}$ then a $u\in \Hom_{\mathscr{C}}(X,Y)$ in a
  commutative diagram
\[
\xymatrix@=1.2cm{
F\ar[rr]^{\xi}\ar[dr]_{\eta} & & X\ar[dl]^{u}\\
 & Y & 
}
\]
is uniquely determined.
\end{itemize}

In fact, if $u_{1}$, $u_{2}\in \Hom_{\mathscr{C}}(X,Y)$ such that
the\pageoriginale diagrams
\[
\vcenter{\xymatrix{
F\ar[rr]^{\eta}\ar[dr]_{\xi} & & Y\\
 & X\ar[ur]_{u_{1}}
}}
\quad\text{and}\quad 
\vcenter{\xymatrix{
F\ar[rr]^{\eta}\ar[dr]_{\xi} & & Y\\
 & X\ar[ur]_{u_{2}}
}}
\]
are commutative then by $(\mathscr{C}_{1})$ and Lemma \ref{chap4-lem4.4.1.2}
$\ker (u_{1},u_{2})$ exists; since $F$ is left exact we get a
commutative diagram
\[
\xymatrix@=1.5cm{
F \ar[r]^{\eta}\ar[d]_{\xi}\ar[dr]_{\xi} & Y\\
\ker(u_{1},u_{2})\ar@{^{(}->}[r]_{j} & X\ar@<.2em>[u]^{u_{1}}\ar@<-.2em>[u]_{u_{2}}
}
\]
with a monomorphism $j$. As $(X,\xi)$ is minimal $j$ must be an
isomorphism, i.e., $\underline{u_{1}=u_{2}}$.

From (*), (**) it follows that the system $I$ of minimal pairs of
$\mathscr{E}$ is {\em directed.}

If $(X,\xi)\in I$, $(Y,\eta)\in\mathscr{E}$ and
$u\in\Hom_{\mathscr{C}}(Y,X)$ appears in a commutative diagram
\[
\xymatrix@=1.2cm{
F\ar[rr]^{\xi}\ar[dr]_{\eta} && X\\
 & Y\ar[ur]_{u} &
}
\]
then\pageoriginale $u$ must be an effective epimorphism.

In fact, be $(\mathscr{C}_{3})$ we get a factorisation
\[
\xymatrix@=1.2cm{
Y\ar[rr]^{u}\ar[dr]_{u_{1}} & & X_{1}\coprod X_{2}=X\\
 & X_{1}\ar@{^{(}->}[ur]_{j}
}
\]
with an effective epimorphism $u_{1}$ and a monomorphism $j$. By
$(F_{2})$ and $(F_{3})$ we then obtain a commutative diagram:
\[
\xymatrix{
F\ar[rrr]^{\xi}\ar[drr]^{\xi}\ar[ddr]_{\eta} & & & X=X_{1}\coprod X_{2}\\
 & & X_{1}\ar@{^{(}->}[ur]_{j} && \\ 
& Y\ar[ur]_{u_{1}} & &
}
\]

By minimality of $(X,\xi)$, it follows that $j$ is an isomorphism;
thus $u$ is an effective epimorphism. In particular:
\begin{itemize}
\item[{*}{**}] The structure morphisms occurring in the projective
  family $I$ are effective epimorphisms.
\end{itemize}

Consider now the natural map
$$
\varinjlim_{i\in I}\Hom_{\mathscr{C}}(S_{i},X)\to F(X),\quad X\in \mathscr{C}.
$$

By\pageoriginale (*) this is onto; by (**) it is injective. From
({*}{**}) it thus follows that $F$ is strictly
pro-representable.\hfill Q.E.D.
\end{proof}

\setcounter{subdefin}{4}
\begin{subdefin}\label{chap4-defi4.4.1.5}
Let $\mathscr{C}$ be a category with zero $(\emptyset)$ in which
disjoint unions exist in $\mathscr{C}$. An $X\in\mathscr{C}$ is {\em
  connected} in $\mathscr{C}\Leftrightarrow X\neq X_{1}\coprod X_{2}$
in $\mathscr{C}$ with $X_{1}$, $X_{2}\neq \emptyset$.
\end{subdefin}

\begin{note*}
In $(\mathscr{E}t/S)$, a prescheme is connected $\Leftrightarrow$ it
is connected as a topological space.
\end{note*}

With the notations of the preceding lemma, we have:

\setcounter{sublemma}{5}
\begin{sublemma}\label{chap4-lem4.4.1.6}
\begin{itemize}
\item[\rm(i)] $(X,\xi)\in\mathscr{E}$ is minimal $\Leftrightarrow X$ is
  connected in $\mathscr{C}$.

\item[\rm(ii)] If $X$ is connected in $\mathscr{C}$, then any $u\in
  \Hom_{\mathscr{C}}(X,X)$ is an automorphism.

\item[\rm(iii)] For any $X\in\mathscr{C}$, $\Aut X$ acts on $F(X)$ as
  follows:

$F\xrightarrow{\xi}X\xrightarrow{\sigma \in \Aut X}X$. It
  $X\in\mathscr{C}$ is connected, then for any $\xi \in F(X)$ the map
  $\Aut X\to F(X)$ defined by $u\mapsto F(u)(\xi)=u\circ \xi$, is an
  {\em injection}.
\end{itemize}
\end{sublemma}

\begin{proof}
\begin{itemize}
\item[(i)] Suppose $X=X_{1}\coprod X_{2}$ in $\mathscr{C}$, $X_{1}$,
  $X_{2}\neq \emptyset$ and that $(X,\xi)\in \mathscr{E}$; then $\xi
  \in F(X)=F(X_{1})\coprod F(X_{2})$ say, $\xi \in F(X_{1})$. 

We then have a commutative diagram
\[
\xymatrix@=1.2cm{
F\ar[rr]^{\xi}\ar[dr]_{\xi} & & X\\
 & X_{1}\ar@{^{(}->}[ur]_{j} &
}
\]
with\pageoriginale a monomorphism $j$ which is {\em not} an
isomorphism. Thus $(X,\xi)$ is {\em not} minimal.


On the other hand, let $X\in\mathscr{C}$ be connected and
$(X,\xi)\in\mathscr{E}$. Suppose we have a commutative diagram:
\[
\xymatrix@=1.2cm{
F\ar[rr]^{\xi}\ar[dr]_{\eta} & & X\\
 & Y\ar@{^{(}->}[ur]_{j} &
}
\]
with a monomorphism $j$. By $(\mathscr{C}_{3})$ we get a
factorisation:
\[
\xymatrix@=1.2cm{
Y\ar[rr]^{j}\ar[dr]_{j_{1}} & & X=X_{1}\coprod X_{2}\\
 & X_{1}\ar@{^{(}->}[ur]_{j_{2}} &
}
\]
with an effective epimorphism $j_{1}$ and a monomorphism $j_{2}$. As
$j$ is a monomorphism, so is $j_{1}$ and thus $j_{1}$ is an
isomorphism; since $X$ is connected, $X_{2}=\emptyset$ and one gets
that $j=j_{2}\circ j_{1}$ is an isomorphism.

\item[(ii)] As $X$ is connected, it follows by $(\mathscr{C}_{3})$
  that $u$ is an effective epimorphism; by $(F_{3})$, $F(u):F(X)\to
  F(X)$ is onto and thus is a bijection ($F(X)$ finite). By $(F_{5})$
  it follows that $u\in \Aut X$.

\item[(iii)] Let\pageoriginale $u_{1}$, $u_{2}\in \Aut X$ such that
  $F(u_{1})(\xi)=F(u_{2})(\xi)$, i.e., $\xi\in \ker
  (F(u_{1}),F(u_{2}))=F(\ker(u_{1},u_{2}))$ ($F$ is left-exact). We
  thus have a commutative diagram
\[
\xymatrix@R=1.2cm{
F\ar[rr]^{\xi}\ar[dr]_{\xi'} & & X\ar@<.2em>[r]^{u_{1}}\ar@<-.2em>[r]_{u_{2}} &X\\
 & \ker (u_{1},u_{2})\ar@{^{(}->}[ur]_{j} & & 
}
\]
with a monomorphism $j$; as $(X,\xi)$ is minimal by (i), $j$ is an
isomorphism, in other words, $\underline{u_{1}=u_{2}}$.\hfill Q.E.D.
\end{itemize}
\end{proof}

We briefly recall now the example \ref{chap4-exam2} of
\ref{chap4-sec4.3}. Assertion (iii) of the above lemma simply says in this
case that if $K/k$ is a finite separable extension field and if
$\xi\in \Hom_{k}(K,\Omega)$, then the map $\Aut K\to
\Hom_{k}(K,\Omega)$ given by $u\mapsto \xi\circ u$ is injective. We
know that $K/k$ is galois $\Leftrightarrow$ this map is also
onto. Following this, we now make the

\setcounter{subdefin}{6}
\begin{subdefin}\label{chap4-defi4.4.1.7}
A connected-object $X\in\mathscr{C}$ is {\em galois} if for any
$\xi\in F(X)$, the map $\Aut X\to F(X)$ defined by $u\mapsto u\circ
\xi$ is a {\em bijection}. Note that this is equivalent to saying that
the action of $\Aut X$ on $F(X)$ is {\em transitive}. Also observe
that this definition is independent of $F$ because the cardinality of
$F(X)$ is the degree of the covering $X$ over $S$; the action is
already effective since $X$ is connected\pageoriginale (by (iii),
Lemma \ref{chap4-lem4.4.1.6}). 
\end{subdefin}

\setcounter{sublemma}{7}
\begin{sublemma}\label{lem4.4.1.8}
If $F\xrightarrow{\eta}Y$, then there is a galois object
$X\in\mathscr{C}$, a $\xi\in F(X)$ and a $u\in
\Hom_{\mathscr{C}}(X,Y)$ such that the diagram
\[
\xymatrix@=1.2cm{
F\ar[rr]^{\xi}\ar[dr]^{\eta} & & X\ar[dl]^{u}\\
 & Y &
}
\]
is commutative. In other words, the system $I_{1}$ of galois pairs of
$\mathscr{E}$ is {\em cofinal} in $\mathscr{E}$.
\end{sublemma}

\begin{proof}
Let $(S_{i})_{i\in I}$ be a projective system of minimal objects of
$\mathscr{C}$ such that
$$
F\xleftarrow{\sim} \varprojlim_{i\in I}\Hom_{\mathscr{C}}(S_{i},*).
$$

Let $\eta_{1},\ldots,\eta_{r}$ be the elements of $F(Y)$. We can
choose $i$ large enough such that as $u$ varies over
$\Hom_{\mathscr{C}}(S_{i},Y)$, the $u\circ \tau_{i}$ give all the
$\eta's$ ($\tau_{i}$ is the canonical element of $F(S_{i})$). We then
get:
$$
F\xrightarrow{\tau_{i}}S_{i}\xrightarrow{\alpha}Y^{r}=\underbrace{Y\times\ldots\times
  Y}_{r\text{\ times}}\xrightarrow{p_{j}}Y 
$$
where $p_{j}$ is the $j^{\text{th}}$ canonical projection $Y^{r}\to
Y$; the elements $p_{j}\circ\alpha\circ \tau_{i}$, $1\leq j\leq r$,
are precisely the elements $\eta_{1},\ldots,\eta_{r}$ of\pageoriginale
$F(Y)$. By $(\mathscr{C}_{3})$ we get a factorisation:
\[
\xymatrix@=1.2cm{
S_{i}\ar[rr]^{\alpha}\ar[dr]_{\alpha_{1}} & & Y^{r}\\
 & X\ar@{^{(}->}[ur]_{\beta} &
}
\]
with a monomorphism $\beta$ and an effective epimorphism
$\alpha_{1}$. We claim that {\em $X$ is galois.}
\begin{itemize}
\item[(i)] {\bf {\boldmath$X$} is connected.}

Suppose $X=X_{1}\coprod X_{2}$, $X_{1}$, $X_{2}$ in $\mathscr{C}$,
$\neq \emptyset$; the element $\alpha_{1}\circ \tau_{i}\in F(X_{1})$,
say. We can then choose a $j$ large enough for us to get a commutative
diagram: 
\[
\xymatrix{
 & S_{i}\ar[rr]^{\alpha_{1}} & & X=X_{1}\coprod X_{2}\\
F\ar[ur]^{\tau_{i}}\ar[dr]_{\tau_{j}} & &  X_{1}
\ar@{^{(}->}[ur]_{\alpha'} &\\
 & S_{j}\ar[ur]_{\alpha'}\ar[uu]_{\varphi_{ij}} & & 
}
\]

Hence $\beta'\circ\alpha'=\alpha_{1}\circ\varphi_{ij}$ is an
epimorphism, which is absurd. 

\item[(ii)] Set\pageoriginale $\xi=\alpha_{1}\circ\tau_{i}\in
  F(X)$. We shall prove that the map $\Aut X\to F(X)$ defined by
  $u\mapsto u\circ \xi$ is {\em onto}.

Let $\xi'\in F(X)$; we may assume $i$ is so large that we get a
commutative diagram:
\[
\xymatrix@C=1.4cm@R=1.5cm{
 & & X\\
F\ar[urr]^{\xi}\ar[r]^{\tau_{i}}\ar[drr]_{\xi'} & 
S_{i}\ar[dr]^{\alpha'_{1}}\ar[ur]_{\alpha_{1}} & \\ 
 & & X
}
\]

Our aim is to find a $\sigma \in\Aut X$ such that
$\alpha'_{1}=\sigma\circ \alpha_{1}$. Since $X$ is connected,
$\alpha'_{1}$ is also an effective epimorphism. Since the manner in
which a morphism in $\mathscr{C}$ is expressed as the composite of an
effective epimorphism and a monomorphism is essentially unique, we
will be through if we find a $\rho\in\Aut Y^{r}$ such that the diagram
\[
\xymatrix@C=1.4cm@R=1.5cm{
 & & X\ar@{^{(}->}[r]^{\beta} & Y^{r}\ar[dd]^{\rho}\\
F\ar[urr]^{\xi}\ar[r]^{\tau_{i}}\ar[drr]_{\xi'} & 
S_{i}\ar[dr]^{\alpha'_{1}}\ar[ur]_{\alpha_{1}} & & \\ 
 & & X\ar@{^{(}->}[r]^{\beta} & Y^{r}
}
\]
is\pageoriginale commutative. By assumption the elements $p_{1}\circ
\beta\circ \xi$, $1\leq j\leq r$, are all the distinct elements of
$F(Y)$; so the morphisms $p_{j}\circ \beta\circ\alpha_{1}$ are all
distinct. This means that the $p_{j}\circ\beta$ are all distinct;
since $\alpha'_{1}$ is an effective epimorphism the $p_{j}\circ
\beta\circ\alpha'_{1}$ are all distinct; and as $X$ is connected and
$\beta$ is a monomorphism, it follows that the $p_{j}\circ \beta\circ
\xi'$ are all distinct and therefore form the set $F(Y)$.
\end{itemize}

If we set $p_{j}\circ \beta\circ \xi =\eta_{j}$, $1\leq j\leq r$, and
$p_{j}\circ \beta\circ \xi'=\eta_{\rho(j)}$, $1\leq j\leq r$, we get a
permutation $\rho$ of the set $\{1,2,\ldots,r\}$; this permutation
determines an automorphism of $Y^{r}$ with the required
property.\hfill Q.E.D.
\end{proof}

By this lemma we may clearly assume now that $F$ is strictly
pro-represented by a projective system $(S_{i})$ of galois objects of
$\mathscr{C}$.

Let $\mathfrak{g}_{i}=\Aut S_{i}$ and $\theta_{i}$ be the bijection
$\mathfrak{g}_{i}\to F(S_{i})$ defined by $u\mapsto u\circ
\mathscr{C}_{i}$ where $\mathscr{C}_{i}$ is the canonical element of
$F(S_{i})$. For $j\geq i$, we define $\psi_{ij}:\mathfrak{g}_{j}\to
\mathfrak{g}_{i}$ as the composite 
$$
\mathfrak{g}_{j}\xrightarrow{\theta_{j}}F(S_{j})\xrightarrow{F(\varphi_{ij})}F(S_{i})\xrightarrow{\theta^{-1}_{i}}\mathfrak{g}_{i}. 
$$

For any $u\in \mathfrak{g}_{j}$, $\psi_{ij}(u)$ is the uniquely
determined automorphism of $S_{i}$ which makes either one (and hence
also the other) of the diagrams\pageoriginale
\[
\vcenter{\xymatrix@C=1.2cm@R=.5cm{
 & S_{i}\ar[r]^{\psi_{ij}(u)} & S_{i}\\
F\ar[ur]^{\tau_{i}}\ar[dr]_{\tau_{j}} & &\\
 & S_{j}\ar[r]_{u} & S_{j}\ar[uu]_{\varphi_{ij}}
}}
\qquad\qquad\qquad 
\vcenter{\xymatrix@=1.5cm{
S_{i}\ar[r]^{\psi_{ij}(u)} & S_{i}\\
S_{j}\ar[u]_{\varphi_{ij}}\ar[r]_{u} & S_{j}\ar[u]_{\varphi_{ij}}
}}
\]
commutative. It follows from this easily that the $\psi_{ij}$ are
group homomorphisms.

We thus obtain a projective system
$\{\mathfrak{g}_{i},\psi_{ij}\}_{i\in I}$ of finite groups with each
$\psi_{ij}$ surjective. Denote by $\{\pi_{i},\psi_{ij}\}_{i\in I}$ the
projective system of the opposite groups. The group
$\pi=\varprojlim_{i\in I}\pi_{i}$ with the limit topology is
pro-finite and we shall prove that it is the fundamental group of the
galois category $\mathscr{C}$; we denote it by $\pi_{1}(S,s)$ when
$\mathscr{C}$ and $F$ are as in \ref{chap4-sec4.1} and \ref{chap4-sec4.2}.

$\pi_{i}$ acts on $\Hom_{\mathscr{C}}(S_{i},X)$ to the left and hence
$\pi$ acts continuously on the set
$\varinjlim_{i}\Hom_{\mathscr{C}}(S_{i},X)\xrightarrow{\sim}F(X)$, to
the left. Since $F(X)$ is finite, the action of $\pi$ on $F(X)$ comes
from the action of some $\pi_{i}$ on $F(X)$.

\setcounter{subsubsection}{8}
\subsubsection{}\label{chap4-sec4.4.1.9}
We\pageoriginale shall find it convenient now to introduce informally
the notion of the procategory Pro $\mathscr{C}$ of $\mathscr{C}$. An
object of Pro $\mathscr{C}=$ (called a pro-object of $\mathscr{C}$)
will be a projective system $\widetilde{p}=(P_{i})_{i\in I}$ in
$\mathscr{C}$. If $\widetilde{P}$, $\widetilde{P}'=(P'_{j})_{j\in J}$
are pro-objects of $\mathscr{C}$, we define
$\Hom(\widetilde{P},\widetilde{P}')$ as the double limit
$\varprojlim_{j\in J}\varinjlim_{i\in
  I}\Hom_{\mathscr{C}}(P_{i},P'_{j})$. An object of $\mathscr{C}$ will
be considered an object of Pro $\mathscr{C}$ in a natural way.

We may look at a pro-representable functor on $\mathscr{C}$, as a
functor ``represented'' in a sense by a pro-object of
$\mathscr{C}$. For instance, in the case of \ref{chap4-lem4.4.1.4}, we have:
\begin{align*}
F(X) & \xleftarrow{\sim} \varinjlim_{i\in
  I}\Hom_{\mathscr{C}}(S_{i},X),\quad \forall X\in\mathscr{C}\\
&\simeq \Hom_{\Pro\mathscr{C}}(\widetilde{S},X)
\end{align*}
where $\widetilde{S}$ is the pro-object $(S_{i})_{i\in I}$ of
$\mathscr{C}$.

Also for any $i\in I$,
$\Hom_{\Pro\mathscr{C}}(\widetilde{S},S_{i})\simeq
\Hom_{\mathscr{C}}(S_{i},S_{i})=\mathfrak{g}_{i}$ and we may then
write:
\begin{align*}
\mathfrak{g}=\varprojlim_{i}\mathfrak{g}_{i} &=
\varprojlim_{i}\Hom_{\mathscr{C}}(S_{i},S_{i})\\
&=
\varprojlim_{i}\Hom_{\Pro\mathscr{C}}(\widetilde{S},S_{i})=\Hom_{\Pro\mathscr{C}}(\widetilde{S},\widetilde{S}) 
\end{align*}
and hence $=\Aut_{\Pro\mathscr{C}}\widetilde{S}$.


If\pageoriginale we call $\widetilde{S}$ a pro-representative of $F$
(in the case $\mathscr{C}=(\mathscr{E}t/S)$ we call it a {\em
  universal covering} of $S$) then $\pi$ is the opposite of the group
of automorphisms of $\widetilde{S}$.

\setcounter{sublemma}{9}
\begin{sublemma}\label{lem4.4.1.10}
Let $E\in\mathscr{C}(\pi)$; then $\exists$ an object
$G(E)\in\mathscr{C}$, and a $\mathscr{C}(\pi)$-isomorphism
$\gamma_{E}:E\to FG(E)$ such that the map
$\Hom_{\mathscr{C}}(G(E),X)\break \to \Hom_{\mathscr{C(\pi)}}(E,F(X))$ given
by $u\mapsto F(u)\circ\gamma_{E}$ is a bijection for all
$X\in\mathscr{C}$. The assignment $E\mapsto G(E)$ can be extended to a
functor $G:\mathscr{C}(\pi)\to\mathscr{C}$ such that $F$ and $G$
establish an equivalence of $\mathscr{C}$ and $\mathscr{C}(\pi)$.
\end{sublemma}

\begin{proof}
If $E=\coprod E_{i}$ is a decomposition of $E$ into connected sets in
$\mathscr{C}(\pi)$ and if $G(E_{i})$ are defined we may define
$G(E)=\coprod G(E_{i})$. We may thus assume that $\pi$ acts
transitively on $E$. Fix an element $\mathscr{E}\in E$ and consider
the surjection $\pi\to E$ defined by $\sigma\mapsto
\sigma\cdot\mathscr{E}$. As $E$ is finite, there is an $i$ such that
the diagram
\[
\xymatrix@=1.2cm{
\pi \ar[rr]\ar[dr] && E\\
& \pi_{i} \ar[ur]
}
\]
($\pi\to \pi_{i}$ is the natural projection and $\pi_{i}\to E$ is the
map $\sigma\mapsto \sigma\cdot \mathscr{E}$) is commutative. Let
$H_{i}\subset \pi_{i}$ be the isotropy group of $\mathscr{E}$ in
$\pi_{i}$. It is easily proved that the set $\pi_{i}/H_{i}$ of
left-cosets of $\pi_{i}\text{mod}\cdot H_{i}$ is
$\mathscr{C}(\pi)$-isomorphic to $E$. We\pageoriginale then define:
$G(E)=G(\pi_{i}/H_{i})=S_{i}/H^{0}_{i}$, the quotient of $S_{i}$ by
the opposite $H^{0}_{i}$ of $H_{i}$ (remark: $H^{0}_{i}\subset
\mathfrak{g}_{i}\subset \Aut S_{i}$). By $(F_{4})$ we have:
$F(G(E))=F(S_{i}/H^{0}_{i})\xleftarrow{\sim}F(S_{i})/H^{0}_{i}\simeq
\pi_{i}/H_{i}\simeq E$, and hence a $\mathscr{C}(\pi)$-isomorphism
$\gamma_{E}:E\to FG(E)$. If $j\geq i$ and $H_{j}\subset \pi_{j}$ is
the isotropy group of $\mathscr{E}$ in $\pi_{j}$, then we have a
$\mathscr{C}$-morphism $S_{j}/H^{0}_{j}\to S_{i}/H^{0}_{i}$; since
$F(S_{j}/H^{0}_{j})\to F(S_{i}/H^{0}_{i})$ is a
$\mathscr{C}(\pi)$-isomorphism, it follows from $(F_{5})$ that
$S_{j}/H^{0}_{j}\xrightarrow{\sim}S_{i}/H^{0}_{i}$ and that $G(E)$ is
independent of the choice of $i$ (upto a $\mathscr{C}$-isomorphism).

Let $X\in\mathscr{C}$. Consider the map
\begin{gather*}
\omega:\Hom_{\mathscr{C}}(G(E),X)\to \Hom_{\mathscr{C(\pi)}}(E,F(X))\\
u\mapsto F(u)\circ \gamma_{E}
\end{gather*}
\begin{itemize}
\item[{\bf(i)}] {\bf {\boldmath$\omega$} is an injection:}

Let $u_{1}$, $u_{2}\in \Hom_{\mathscr{C}}(G(E),X)$ be such that
$F(u_{1})\circ \gamma_{E}=F(u_{2})\circ \gamma_{E}$. But
$\gamma_{E}:E\to FG(E)$ is an isomorphism and so
$\ker(F(u_{1}),F(u_{2}))\break {\displaystyle{\mathop{\hookrightarrow}_{\cong}}}FG(E)$, 
i.e.,
$F(\ker(u_{1},u_{2})){\displaystyle{\mathop{\hookrightarrow}_{\simeq}}}FG(E)$ 
($F$ is left-exact). It follows from $(F_{5})$ that
$\ker(u_{1},u_{2}){\displaystyle{\mathop{\hookrightarrow}_{\simeq}}}G(E)$,
in other words, $u_{1}=u_{2}$.   

\item[{\bf(ii)}] {\bf {\boldmath$\omega$}\pageoriginale is a surjection.}

Let $E\xrightarrow{\alpha}F(X)$ be any $\mathscr{C}(\pi)$-morphism;
put $\delta=\alpha(\mathscr{E})\in F(X)$. The
$\Pro\mathscr{C}(\pi)$-morphism $\delta:\widetilde{S}\to X$ can be
factored through some $S_{i}$:
\[
\xymatrix@=1.2cm{
\widetilde{S}\ar[rr]^{\delta}\ar[dr]_{\tau_{i}} & & X\\
 & S_{i}\ar[ur]_{\delta_{i}} &
}
\]

Let $H'_{i}$ be the isotropy group of $\delta_{i}$ in $\pi_{i}$ then
$H'_{i}\supset H_{i}$ where $H_{i}$ is as before (take $i$ large
enough). By the construction of $G(E)=S_{i}/H^{0}_{i}$, we have a
morphism $\dfrac{S_{i}}{H^{0}_{i}}\to X$ making the diagram
\[
\xymatrix@=1.2cm{
\widetilde{S}\ar[r]^{\tau_{i}} & S_{i}\ar[rr]^{\delta_{i}}\ar[dr] &&
X\\
 & & S_{i}/H^{0}_{i}\ar[ur] &
}
\]
commutative; one easily checks that this morphism goes to $\alpha$
under $\omega$. It only remains to show that the assignment $E\mapsto
G(E)$ can be extended to a functor. Let $E$, $E'\in\mathscr{C}(\pi)$ and
$\theta\in \Hom_{\mathscr{C}(\pi)}\break (E,E')$. To\pageoriginale the
composite $\gamma_{E'}\circ \theta:E\to FG(E')$ there corresponds a
{\em unique} $\overline{\theta}\in \Hom_{\mathscr{C}}(G(E),G(E'))$
such that the diagram
\[
\xymatrix@=1.2cm{
E\ar[r]^{\theta}\ar[d]^{\gamma_{E}} & E'\ar[d]^{\gamma_{E'}}\\
FG(E)\ar[r]_{F(\overline{\theta})} & FG(E')
}
\]
is commutative. We set $G(\theta)=\overline{\theta}$. It follows
easily that $G$ is a covariant functor from $\mathscr{C}(\pi)$ to
$\mathscr{C}$. 
\end{itemize}

One now checks that there are functorial isomorphisms
\begin{align*}
&\Phi: I_{\mathscr{C}}\to G\circ F\\
\text{and}\qquad & \Psi: I_{\mathscr{C(\pi)}}\to F\circ G,
\end{align*}
such that, for any $X\in \mathscr{C}$ and $E\in\mathscr{C}(\pi)$,
\begin{align*}
& F(X)\xrightarrow{F(\Phi(X))}FGF(X)\xrightarrow{\Psi^{-1}(F(X))}F(X)\\
\text{and}\qquad
& G(E)\xrightarrow{G(\Psi(E))}GFG(E)\xrightarrow{\Phi^{-1}(G(E))}G(E)
\end{align*}
and the identity maps.

This completes the proof of assertion (1) of Theorem (\ref{chap4-sec4.4.1}).
\end{proof}

\begin{sublemma}\label{chap4-lem4.4.1.11}
Let\pageoriginale $\mathscr{C}$ be a galois category and $F$, $F'$ be
two fundamental functors $\mathscr{C}\to$ (finite sets). Suppose
$\pi$, $\pi'$ are the profinite groups defined by $F$, $F'$
respectively as above; then $\pi$ and $\pi'$ are continuously
isomorphic and this isomorphism is canonically determined upto an
inner automorphism of $\pi$.
\end{sublemma}

\begin{proof}
We know that $F:\mathscr{C}\to \mathscr{C}(\pi)$ is an
equivalence. Replacing $F'$ by $F'\circ G$ (with $G$ as before) we can
assume that $\mathscr{C}=\mathscr{C}(\pi)$, that $F$ is the trivial
functor identifying an object of $\mathscr{C}(\pi)$ with its
underlying set and $\pi$ itself is the pro-object pro-representing
this functor. Let $\widetilde{T}\in\Pro \mathscr{C}(\pi)$
pro-represent $F'$; first we show that
$\pi\xrightarrow{\sim}\widetilde{T}$ in $\Pro\mathscr{C}$.

In order to do this, let $(T_{j})_{j\in J}$ be a projective family of
galois objects (with respect to $F'$) of $\mathscr{C}$ such that
$\widetilde{T}=\varprojlim_{j}T_{j}$; we denote the canonical maps
$T_{j}\to T_{i}$ by $q_{ij}$ and $\widetilde{T}\to T_{j}$ by
$q_{j}$. Let $t_{j}\in T_{j}$ be a coherent system of points and
consider the continuous maps $\alpha_{j}:\pi\in T_{j}$ determined by
$\alpha_{j}(e)=t_{j}$; there exists a continuous $\alpha:\pi\to
\widetilde{T}$ such that
\[
\xymatrix@=1.2cm{
\pi\ar[rr]^{\alpha}\ar[dr]_{\alpha_{j}} & & \widetilde{T}\ar[dl]^{q_{j}}\\
& T_{j} & 
}
\]
is commutative. We have $\alpha(e)=\widetilde{t}=(t_{j})\in
\widetilde{T}$. The $T_{j}$ are connected, hence\pageoriginale
$\alpha_{j}$ transitive and as $\pi$ is compact it follows that
$\alpha$ is onto. Let $\mathfrak{H}$ be the isotropy group of
$\widetilde{t}$, then we have
$\pi/\mathfrak{H}\xrightarrow{\sim}\widetilde{T}$. ($\pi/\mathfrak{H}$
is the set of left-cosets of $\pi\text{\,mod.\,}\mathfrak{H}$) and
this is an isomorphism of topological spaces (it is a continuous map
of compact Hausdorff spaces). However
$F'=\Hom_{\Pro\mathscr{C}}(\widetilde{T},*)$ and then clearly
$F'(X)=X^{\mathfrak{H}}$ (the points of $X$ invariant under
$\mathfrak{H}$). We know that $F'(X)=\emptyset\Leftrightarrow
X=\emptyset$; from this it follows that $\mathfrak{H}=(e)$ (take for
$X$ the $\pi_{i}$ for larger and larger $i$). Hence
$\alpha:\pi\xrightarrow{\sim}\widetilde{T}$ is an isomorphism of
topological spaces. Our aim is to show that this is an isomorphism in
$\Pro \mathscr{C}$.

For this it is enough to show the following: if
$\beta:\widetilde{T}\to \pi$ is $\alpha^{-1}$ and if $p_{i}:\pi\to
\pi_{i}$ are the canonical maps then every $\beta_{i}=p_{i}\circ
\beta:\widetilde{T}\to \pi_{i}$ must factor through some $T_{j}$. In
other words, we must find some morphism $T_{j}\to \pi_{i}$ making the
diagram 
\[
\xymatrix@=1.2cm{
\widetilde{T}\ar[dr]^{\beta_{i}}\ar[r]^{\beta}\ar[d]_{q_{j}} &
\pi\ar[d]^{p_{i}}\\ 
T_{j}\ar[r] & \pi_{i}
}
\]
commutative. For this we must show that given $i$, $\exists j\in J$
such\pageoriginale that for every $t\in T_{j}$, $\exists$ an
$s\in\pi_{i}$ such that $q^{-1}_{j}(t)\subset
\beta^{-1}_{i}(s)$. Since the sets $\beta^{-1}_{i}(s)$ are open we can
find for every point $x\in\beta^{-1}_{i}(s)$ an open neighbourhood
$U_{x}$ of the form $q^{-1}_{j}(t_{x})$ for some $j_{x}\in J$ and
$t_{x}\in T_{j_{x}}$, such that $U_{x}\subset
\beta^{-1}_{i}(s)$. Since $\widetilde{T}$ is compact, $\exists$ a
finite covering $U_{x_{1}},\ldots,U_{x_{N}}$ of $\widetilde{T}$ of
this type and $j>\max(j_{x_{1}},\ldots,j_{x_{N}})$ satisfies our
requirements. 

This shows that $\pi\xrightarrow{\sim}\widetilde{T}$ in
$\Pro\mathscr{C}$, and hence the map
\begin{gather*}
{\pi'}^{0}=\Aut_{\Pro \mathscr{C}}\widetilde{T}\to
\Aut_{\Pro\mathscr{C}}\pi=\pi^{0}\\
\rho\mapsto \beta\circ\rho\circ \alpha
\end{gather*}
is a group isomorphism; it remains to be shown that this map is
continuous and hence a homeomorphism. Take a fixed
$\gamma^{0}=\beta\circ \rho^{0}\circ \alpha$ in $\pi$, some $i\in I$
and consider the commutative diagram
\[
\xymatrix@=1.2cm{
\pi\ar[d]^{p_{i}}\ar[r]^{\gamma^{0}} & \pi\ar[d]^{p_{i}}\\
\pi_{i}\ar[r]^{\gamma^{0}_{i}} & \pi_{i}
}
\]

Let\pageoriginale $U$ be the neighbourhood of $\gamma^{0}$ consisting
of all $\gamma\in \pi$ such that $p_{i}\circ
\gamma=\gamma^{0}_{i}\circ p_{i}$. We want to find a neighbourhood $V$
of $\rho^{0}$ such that $\beta\circ \rho\circ \alpha\in U\ \forall\
\rho\in V$. There is an index $j\in J$ and $l\in I$ and morphisms
making all the following diagrams
\[
\xymatrix@=1.2cm{
\pi\ar[r]^{\alpha}\ar[d]^{p_{1}} &
\widetilde{T}\ar[r]^{\rho^{0}}\ar[d]^{q_{j}} &
\widetilde{T}\ar[d]^{q_{j}}\ar[r]^{\beta} & \pi\ar[dd]^{p_{i}}\\
\pi_{l}\ar[d]^{p_{l}}\ar[r] & T_{j}\ar[r]^{\rho^{0}_{j}} &
  T_{j}\ar[dr] & \\
\pi_{i}\ar[rrr]^{\gamma^{0}_{i}} & & & \pi_{i}
}
\]
commutative.

Consider all $\rho:\widetilde{T}\to \widetilde{T}$ such that
$q_{j}\circ\rho=\rho^{0}_{j}\circ q_{j}$; these $\rho$ form a
neighbourhood $V$ of $\rho^{0}$ and $\beta\circ\rho\circ\alpha\in
U\ \forall\ \rho\in V$. Hence we are through.

Finally we observe that the isomorphism $\pi\to \pi'$ is fixed as soon
as $\alpha:\pi\to \widetilde{T}$ is fixed and this is in turn fixed by
the choice of $\widetilde{t}=(t_{j})\in\widetilde{T}$. By a different
choice of $\widetilde{t}$ we obtain an isomorphism $\pi\to \pi'$ which
differs from the first one by an inner automorphism of $\pi$ (or
$\pi'$).\hfill Q.E.D. 
\end{proof}

\setcounter{subremark}{11}
\begin{subremark}\label{chap4-rem4.4.1.12}
One\pageoriginale can, in fact, show that $\Pro\mathscr{C}(\pi)$ is
precisely the category of compact, totally disconnected (Hausdorff)
topological spa\-ces on which $\pi$ acts continuously.
\end{subremark}
