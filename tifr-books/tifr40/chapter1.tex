\chapter{Affine Schemes}\label{chap1}


\section{}\pageoriginale\label{chap1-sec1.1}

Let $A$ be a commutative ring with 1 and $S$ a multiplicatively closed
set in $A$, containing 1. We then form fractions $\dfrac{a}{s}$, $a\in
A$, $s\in S$; two fractions $\dfrac{a_{1}}{s_{1}}$,
$\dfrac{a_{2}}{s_{2}}$ are considered equal if there is an $s_{3}\in
S$ such that $s_{3}(a_{1}s_{2}-a_{2}s_{1})=0$. When  addition and
multiplication are defined in the obvious way, these fractions form a
ring, denoted by $S^{-1}A$ and called the ring of fractions of $A$
with respect to the multiplicatively closed set $S$. There is a
natural ring-homomorphism $A\to S^{-1}A$ given by $a\mapsto a/1$. This
induces $a(1-1)$ correspondence between prime ideals of $A$ not
intersecting $S$ and prime ideals of $S^{-1}A$, which is
lattice-preserving. If $f\in A$ and $S_{f}$ is the multiplicatively
closed set $\{1,f,f^{2},\ldots\}$, the ring of quotients $S^{-1}_{f}A$
is denoted by $A_{f}$. If $p$ is a prime ideal of $A$ and $S=A-p$, the
ring of quotients $S^{-1}A$ is denoted by $A_{p}$; $A_{p}$ is a local
ring. 

\section{}\label{chap1-sec1.2}
Let $A$ be a commutative ring with 1 and $X$ the set of prime ideals
of $A$. For any $E\subset A$, we define $V(E)$ as the subset
$\{\underline{p}:\underline{p}$ a prime ideal $\supset E\}$ of
$X$. Then the following properties are easily verified:
\begin{itemize}
\item[(i)] $\bigcap\limits_{\alpha}V(E_{\alpha})=V(\bigcup\limits_{\alpha}E_{\alpha})$\pageoriginale

\item[(ii)] $V(E_{1})\cup V(E_{2})=V(E_{1}\cdot E_{2})$

\item[(iii)] $V(1)=\emptyset$

\item[(iv)] $V(0)=X$.
\end{itemize}

Thus, the sets $V(E)$ satisfy the axioms for closed sets in a topology
on $X$. The topology thus defined is called the {\em Zariski topology}
on $X$; the topological space $X$ is known as $\Spec A$.

\begin{note*}
$\Spec A$ is a generalisation of the classical notion of an affine
  algebraic variety.
\end{note*}

Suppose $k$ is an algebraically closed field and let
$k[X_{1},\ldots,X_{n}]=k[X]$ be the polynomial ring in $n$ variables
over $k$. Let $\mathfrak{a}$ be an ideal of $k[X]$ and $V$ be the set
in $k^{n}$ defined by
$V=\{(\alpha_{1},\ldots,\alpha_{n}):f(\alpha_{1},\ldots,\alpha_{n})\break
=0\forall
f\in \mathfrak{a}\}$. Then $V$ is said to be an affine algebraic
variety and the Hilbert's zero theorem says that the elements of $V$
are in $(1-1)$ correspondence with the maximal ideals of
$k[X]/\mathfrak{a}$.

\setcounter{chapremarks}{2}
\begin{chapremarks}\label{chap1-rems1.3}% 1.3
\begin{itemize}
\item[(a)] If $\mathfrak{a}(E)$ is the ideal generated by $E$ in
  $A$, then we have: $V(E)=V(\mathfrak{a}(E))$

\item[(b)] For $f\in A$, define $X_{f}=X-V(f)$; then the $X_{f}$
  form a basis for the Zariski topology on $X$. In fact,
  $X-V(E)=\bigcup\limits_{f\in E}X_{f}$ by (i).

\item[(c)] $X$\pageoriginale is not in general Hausdorff; however, it
  is $T_{0}$. 

\item[(d)] $X_{f}\subset
  \bigcup\limits_{\alpha}X_{f_{\alpha}}\Longleftrightarrow \exists$ an
  $n\in\mathbb{Z}^{+}$ such that $f^{n}$ is in the ideal generated by
  the $f'_{\alpha}s$.

For any ideal $\mathfrak{a}$ of $A$, define
$\sqrt{\mathfrak{a}}=\{a\in A:a^{n}\in\mathfrak{a}$ for some
$n\in\mathbb{Z}^{+}\}$. $\sqrt{\mathfrak{a}}$ is an ideal of $A$ and
we assert that $\sqrt{\mathfrak{a}}=\cap
\{\underline{p}:\underline{p}\text{ a prime ideal }\supset
\mathfrak{a}\}$. To prove this, we assume, (as we may, by passing to
$A/\mathfrak{a}$) that $\mathfrak{a}=(0)$. Clearly, $\sqrt{(0)}\subset
\cap \underline{p}$. On the other hand, if $a\in A$ is such that
$a^{n}\neq 0\forall n\in\mathbb{Z}^{+}$, then $S^{-1}_{a}A=A_{a}$ is a
non-zero ring and so contains a proper prime ideal; the lift $p$ of
this prime ideal in $A$ is such that $a\not\in p$.

It is thus seen that $V(\mathfrak{a})=V(\sqrt{a})$ and $V(f)\supset
V(\mathfrak{a})\Longleftrightarrow f\in\sqrt{\mathfrak{a}}$. Hence:
$$
X_{f}\subset \bigcup_{\alpha}X_{f_{\alpha}}\Longleftrightarrow
V(f)\supset
\bigcap_{\alpha}V(f_{\alpha})=V(\bigcup_{\alpha}\{f_{\alpha}\}) 
$$
and this proves (d).

\item[(e)] The open sets $X_{f}$ are quasi-compact.

In view of (b), it is enough to consider coverings by $X'_{g}s$; thus,
if $X_{f}\subset \bigcup\limits_{\alpha}X_{f_{\alpha}}$, then by (d),
we have $f^{n}=\sum\limits^{r}_{i=1}a_{i}f_{\alpha_{i}}$ (say); again
by (d), we obtain $X_{f}=X_{f^{n}}\subset
\bigcup\limits^{r}_{i=1}X_{f_{\alpha_{i}}}$. 

\item[(f)] There\pageoriginale is a $(1-1)$ correspondence between
  closed sets of 
  $X$ and roots of ideals of $A$; in this correspondence, closed
  irreducible sets of $X$ go to prime ideals of $A$ and
  conversely. Every closed irreducible set of $X$ is of the form
  $(\overline{x})$ for some $x\in X$; such an $x$ is called a {\em
    generic point} of that set, and is uniquely determined.

\item[(g)] If $A$ is noetherian, $\Spec A$ is a noetherian space
  (i.e. satisfies the minimum condition for closed sets).
\end{itemize}
\end{chapremarks}

\setcounter{section}{3}
\section{The Sheaf associated to {$\mathbf{\text{Spec\,}
      A}$}}\label{chap1-sec1.4} 

We shall define a presheaf of rings on $\Spec A$. It is enough to
define the presheaf on a basis for the topology on $X$, namely, on the
$X'_{f}s$; we set $\mathscr{F}(X_{f})=A_{f}$. If $X_{g}\subset X_{f}$,
$V(g)\supset V(f)$ and so $g^{n}=a_{0}\cdot f$ for some
$n\in\mathbb{Z}^{+}$ and $a_{0}\in A$. The homomorphism $A_{f}\to
A_{g}$ given by $\frac{a}{f}q\mapsto \dfrac{a\cdot
  a^{q}_{0}}{g^{qn}}$ is independent of the way $g$ is expressed in
terms of $f$ and thus defines a natural map
$\rho^{f}_{g}:\mathscr{F}(X_{f})\to \mathscr{F}(X_{g})$ for
$X_{g}\subset X_{f}$. The transitivity conditions are readily verified
and we have a presheaf of rings on $X$. This defines a sheaf
$\widetilde{A}=\mathscr{O}_{X}$ of rings on $X$. It is easy to check
that the stalk $\mathscr{O}_{\underline{p},X}$ of $\mathscr{O}_{X}$ at
a point $\underline{p}$ of $X$ is the local ring $A_{\underline{p}}$.

If $M$ is an $A$-module and if we define the presheaf $X_{f}\mapsto
M_{f}=M\otimes_{A}A_{f}$, we get a sheaf $\widetilde{M}$ of
$\widetilde{A}$-modules (in short, an $\widetilde{A}$-Module
$\widetilde{M}$), whose stalk at $\underline{p}\in X$ is
$M\otimes_{A}A_{\underline{p}}=M_{P}$. 

\begin{remark*}
The\pageoriginale presheaf $X_{f}\mapsto M_{f}$ is a sheaf, i.e. it
satisfies the axioms $(F_{1})$ and $(F_{2})$ of Godement, Th\'eorie
des faisceaux, p.~109.
\end{remark*}

\section{}\label{chap1-sec1.5}%1.5
In this section, we briefly recall certain sheaf-theoretic notions.

\subsection{}\label{chap1-sec1.5.1}%1.5.1
Let $f:X\to Y$ be a continuous map of topological spaces. Suppose
$\mathscr{F}$ is a sheaf of abelian groups on $X$; we define a
presheaf of abelian groups on $Y$ by $U\mapsto
\Gamma(f^{-1}(U),\mathscr{F})$ for any open $U\subset Y$; for
$V\subset U$, open in $Y$, the restriction maps of this presheaf will
be the restriction homomorphisms $\Gamma(f^{-1}(U),\mathscr{F})\to
\Gamma(f^{-1}(V),\mathscr{F})$. This presheaf is already a sheaf. The sheaf
defined by this presheaf is called the {\em direct image}
$f_{\ast}(\mathscr{F})$ of $\mathscr{F}$ under $f$.

If $U$ is any neighbourhood of $f(x)$ in $Y$, the natural homomorphism
$\Gamma(f^{-1}(U),\mathscr{F})\to\mathscr{F}_{x}$ given a homomorphism
$\Gamma(U,f_{\ast}(\mathscr{F}))\to \mathscr{F}_{x}$; by passing to
the inductive limit as ``$U$ shrinks down to $f(x)$'', we obtain a
natural homomorphism:
$$
f_{x}:f_{\ast}(\mathscr{F})\xrightarrow[f(x)]{}\mathscr{F}_{x}.
$$

If $\mathscr{O}_{X}$ is a sheaf of rings on $X$,
$f_{\ast}(\mathscr{O}_{X})$ has a natural structure of a sheaf of
rings on $Y$. If $\mathscr{F}$ is an $\mathscr{O}_{X}$-Module,
$f_{\ast}(\mathscr{F})$ has a natural structure of an
$f_{\ast}(\mathscr{O}_{X})$-Module. 

The direct image $f_{\ast}(\mathscr{F})$ is a covariant functor on
$\mathscr{F}$. 

\subsection{}\label{chap1-sec1.5.2}% 1.5.2
Let\pageoriginale $f:X\to Y$ be a continuous map of topological spaces
and $\mathfrak{g}$ be a sheaf of abelian groups on $Y$. Then it can be
shown that there is a unique sheaf $\mathscr{F}$ of abelian groups on
$X$ such that:
\begin{itemize}
\item[(a)] there is a natural homomorphism of sheaves of abelian
  groups
$$
\rho=\rho_{\mathfrak{g}}:\mathfrak{g}\to f_{\ast}(\mathscr{F})
$$
and

\item[(b)] for any sheaf $\mathscr{H}$ of abelian groups on $X$, the
  homomorphism $\Hom_{X}(\mathscr{F},\mathscr{H})\to
  \Hom_{Y}(\mathfrak{g},f_{\ast}(\mathscr{H}))$ given by $\rho\mapsto
  f_{\ast}(\varphi)\circ \rho_{\mathfrak{g}}$ is an isomorphism. The
  unique sheaf $\mathscr{F}$ of abelian groups on $X$ with these
  properties is called the {\em inverse image} $f^{-1}(\mathfrak{g})$
  of $\mathfrak{g}$ under $f$.
\end{itemize}

It can be shown that canonical homomorphism
\begin{equation*}
f_{x}\circ\rho_{f(x)}:\mathfrak{g}_{f(x)}\to f^{-1}(\mathfrak{g})_{x}\tag{*}
\end{equation*}
is an isomorphism, for every $x\in X$.

The inverse image $f^{-1}(\mathfrak{g})$ is a covariant functor on
$\mathfrak{g}$ and the isomorphism $(\ast)$ shows that it is an exact
functor.

If $\mathscr{O}_{Y}$ is a sheaf of rings on $Y$,
$f^{-1}(\mathscr{O}_{Y})$ has a natural structure of a sheaf of rings
on $X$. If $\mathfrak{g}$ is an $\mathscr{O}_{Y}$-Module,
$f^{-1}(\mathfrak{g})$ has a natural structure of an
$f^{-1}(\mathscr{O}_{Y})$-Module. 

\subsection{}\label{chap1-sec1.5.3}%1.5.3

A\pageoriginale {\em ringed space} is a pair $(X,\mathscr{O}_{X})$
where $X$ is a topological space and $\mathscr{O}_{X}$ is a sheaf of
rings on $X$, called the structure sheaf of $(X,\mathscr{O}_{X})$. A
morphism $\Phi:(X,\mathscr{O}_{X})\to (Y,\mathscr{O}_{Y})$ of ringed
spaces is a pair $(f,\varphi)$ such that
\begin{itemize}
\item[(i)] $f:X\to Y$ is a continuous map of topological spaces, and

\item[(ii)] $\varphi:\mathscr{O}_{Y}\to f_{\ast}(\mathscr{O}_{X})$ is
  a morphism of sheaves of rings on $Y$.
\end{itemize}

Ringed spaces, with morphisms so defined, form a category. Observe
that condition (ii) is equivalent to giving a morphism
$f^{-1}(\mathscr{O}_{Y})\to \mathscr{O}_{X}$ of sheaves of rings on
$X$ (see (1.5.2)).

If $\mathscr{F}$ is a sheaf of $\mathscr{O}_{X}$-modules, we denote by
$\Phi_{\ast}(\mathscr{F})$ the sheaf $f_{\ast}(\mathscr{F})$,
considered as an $\mathscr{O}_{Y}$-Module through $\varphi$. If
$\mathfrak{g}$ is an $\mathscr{O}_{Y}$-Module, $f^{-1}(\mathfrak{g})$
is an $f^{-1}(\mathscr{O}_{Y})$-Module and the morphism
$f^{-1}(\mathscr{O}_{Y})\to \mathscr{O}_{X}$, defined by $\varphi$,
gives an $\mathscr{O}_{X}$-Module
$f^{-1}(\mathfrak{g})_{f^{-1}}\otimes_{(\mathscr{O}_{Y})}\mathscr{O}_{X}$;
the stalks of this $\mathscr{O}_{X}$-Module are isomorphic to
$\mathfrak{g}_{f(x)}\otimes_{\mathscr{O}_{f(x)}}\mathscr{O}_{x}$,
under the identification $f^{-1}(\mathfrak{g})_{x}\simeq
\mathfrak{g}_{f(x)}$. We denote this $\mathscr{O}_{X}$-Module by
$\Phi^{\ast}(\mathfrak{g})$. In general, $\Phi^{\ast}$ is {\em not} an
exact functor on $\mathfrak{g}$. 

\section{Affine Schemes}\label{chap1-sec1.6}%1.6

A ringed space of the form $(\Spec A,\widetilde{A})$, $A$ a ring,
defined in (1.4) is called an {\em affine scheme.}

\subsection{}\label{chap1-sec1.6.1}%1.6.1

Let\pageoriginale $\varphi:B\to A$ be a ring-homomorphism; $\varphi$
defines a map
\begin{gather*}
f={}^{a}\varphi:X=\Spec A\to \Spec B=Y\\
\underline{p}\mapsto \varphi^{-1}(\underline{p}).
\end{gather*}

Since ${}^{a}\varphi^{-1}(V(E))=V(\varphi(E))$ for any $E\subset B$,
${}^{a}\varphi$ is a continuous map.

Let $s\in B$; $\varphi$ defines, in a natural way, a homomorphism
$$
\varphi_{s}:B_{s}\to A_{\varphi(s)}.
$$

In view of the remark at the end of (\ref{chap1-sec1.4}), this gives
us a homomorphism:
$$
B_{s}=\Gamma(Y_{s},\widetilde{B})\to
{}^{A}\varphi(s)=\Gamma(X_{\varphi(s)},\widetilde{A})=\Gamma(Y_{s},f_{\ast}(\widetilde{A})) 
$$
and hence a homomorphism $\widetilde{\varphi}:\widetilde{B}\to
f_{\ast}(\widetilde{A})$. If $x\in X$ the stalk map defined by
$\widetilde{\varphi}$, namely
$$
\widetilde{\varphi}_{x}:\mathscr{O}_{f(x)}\simeq B_{f(x)}\to
\mathscr{O}_{x}\simeq A_{x}
$$
is a local homomorphism (i.e. the image of the maximal ideal in
$B_{f(x)}$ is contained in the maximal ideal of $A_{x}$).

\setcounter{defin}{1}
\begin{defin}\label{chap1-def1.6.2}
A\pageoriginale morphism $\Phi:(\Spec A,\widetilde{A})\to (\Spec
B,\widetilde{B})$ of two affine schemes, is a morphism of ringed
spaces with the additional property that $\Phi$ is of the form
$({}^{a}\varphi,\widetilde{\varphi})$ for a homomorphism $\varphi:B\to
A$ of rings.
\end{defin}

It can be shown that a morphism $\Phi=(f,\varphi)$ of ringed spaces is
a morphism of affine schemes $(\spec A,\widetilde{A})(\spec
B,\widetilde{B})$ if and only if the stalk-maps
$$
\mathscr{O}_{f(x)}\to \mathscr{O}_{x}\quad\text{defined by}\quad
\Phi(\text{rather, by }\varphi)
$$
are local homomorphisms.

\setcounter{remarks}{3}
\begin{remarks}\label{chap1-rems1.6.4}
\begin{enumerate}
\renewcommand{\theenumi}{\alph{enumi}}
\renewcommand{\labelenumi}{\rm(\theenumi)}
\item If $M$ is an $A$-module, $\widetilde{M}$ is an exact covariant
  functor on $M$.

\item For any $A$-modules $M$, $N$,
  $\Hom_{\widetilde{A}}(\widetilde{M},\widetilde{N})$ is canonically
  isomorphic to $\Hom_{A}(M,N)$.

\item If $(X,\mathscr{O}_{X})=(\spec A,\widetilde{A})$,
  $(Y,\mathscr{O}_{Y})=(\spec B,\widetilde{B})$ are affine, there is a
  natural bijection from the set $\Hom(X,Y)$ of morphisms of affine
  schemes $X\to Y$ onto the set $\Hom(B,A)$ of ring-homomorphisms
  $B\to A$.

\item Let $(X,\mathscr{O}_{X})=(\Spec A,\widetilde{A})$ be an affine
  scheme and $\mathscr{F}$ an $\mathscr{O}_{X}$-module. Then one can
  show that $\mathscr{F}$ is quasi-coherent (i.e. for every $x\in X$,
  $\exists$ an open neighbourhood $U$ of $x$ and an exact sequence
  $(\mathscr{O}_{X}|U)\xrightarrow{(I)} (\mathscr{O}_{X}|U)\xrightarrow{(J)}
  \mathscr{F}|U\to 0)\Leftrightarrow \mathscr{F}\simeq \widetilde{M}$
  for an $A$-module $M$. If we assume that $A$ is {\em noetherian},
  one sees that $\mathscr{F}$ is coherent $\Leftrightarrow
  \mathscr{F}$ is $\simeq$ to $\widetilde{M}$ for a finite type
  $A$-module $M$.

\item Let\pageoriginale $X\xrightarrow{\Phi}Y$ be a morphism of affine
  schemes and $\mathscr{F}$ (resp. $\mathfrak{g}$) be a quasicoherent
  $\mathscr{O}_{X}$-Module (resp. $\mathscr{O}_{Y}$-Module). Then one
  can define a quasi-coherent $\mathscr{O}_{Y}$-Module
  (resp. $\mathscr{O}_{X}$-Module) denoted by
  $\Phi_{\ast}(\mathscr{F})$ (resp. $\Phi^{\ast}(\mathfrak{g})$) just
  as in (1.5.3); if $X=\Spec A$, $Y=\Spec B$ and
  $\Phi=({}^{a}\varphi,\widetilde{\varphi})$ for a $\varphi:B\to A$
  and if $\mathscr{F}=\widetilde{M}$, $M$ an $A$-module
  (resp. $\mathfrak{g}=\widetilde{N}$, $N$ a $B$-module) then
  $\Phi_{\ast}(\mathscr{F})$ (resp. $\Phi^{\ast}(\mathfrak{g})$) is
  canonically identified with $\widetilde{[\varphi]^{M}}$, where
  $[\varphi]^{M}$ is the abelian group $M$ considered as a $B$-module
  through $\varphi$ (resp. $\widetilde{M\otimes_{B}A}$).
\end{enumerate}
\end{remarks}

(For proofs see EGA Ch. I)


