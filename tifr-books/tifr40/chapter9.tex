\chapter[The Homomorphism of Specialisation.....]{The Homomorphism of Specialisation of the Fundamental
  Group}\label{chap9} 

\section{}\label{chap9-sec9.1}\pageoriginale
With the notations and assumptions of Chapter \ref{chap8}, let $s_{1}$ be an
arbitrary point of $S=\Spec A$ and
$\overline{X}_{1}=\fprod{X}{\overline{k(s_{1})}}{S}$. If
$\overline{a}_{1}$ is a geometric point of $\overline{X}_{1}$ and
$a_{1}$ its image in $X$ and if $\overline{X}_{1}$ is connected we get
a sequence
$$
\pi_{1}(\overline{X}_{1},\overline{a}_{1})\to \pi_{1}(X,a_{1})\to
\pi_{1}(S,s_{1})\to (e)
$$
such that the composites are trivial. We have continuous isomorphisms
$\beta:\pi_{1}(X,a_{1})\to \pi_{1}(X,a_{0})$ and
$\alpha:\pi_{1}(S,s_{1})\to \pi_{1}(S,s_{0})$. Consider now the
diagram:
\[
\xymatrix@=1.1cm{
& \pi_{1}(\overline{X}_{1},\overline{a}_{1})\ar[r] &
\pi_{1}(X,a_{1})\ar[d]^{\beta} \ar[r] &
\pi_{1}(S,s_{1})\ar[d]^{\alpha}\ar[r] & (e)\\
(e)\ar[r] & \pi_{1}(\overline{X}_{0},\overline{a}_{0})\ar[r] &
\pi_{1}(X,a_{0})\ar[r] & \pi_{1}(S,s_{0})\ar[r] & (e).
}
\]

This is {\em not} necessarily commutative; but it is {\em commutative
  upto an inner automorphism of} $\pi_{1}(S,s_{0})$. (It is readily
seen in the example of section \ref{chap5-sec5.1} that, with the notation
therein, the continuous homomorphism there, is determined upto an
inner automorphism of $\pi_{1}(S,s)$ if we take for $s$ a point
different from $\varphi(s')$-see Ch. \ref{chap4}.)

Now the lower row is {\em exact}. This implies that there is a
continuous homomorphism $\pi_{1}(\overline{X}_{1},\overline{a}_{1})\to
\pi_{1}(\overline{X}_{0},\overline{a}_{0})$ which is
clearly\pageoriginale determined upto an inner automorphism of
$\pi_{1}(X)$. This is called the {\em homomorphism of specialisation
  of the fundamental group.}

\section{}\label{chap9-sec9.2}
With the same assumptions as above further suppose $X/S$
separable. Since we have assumed $\overline{X}_{0}$ connected, the
condition $f_{\ast}(\mathscr{O}_{X})=\mathscr{O}_{S}$ is automatically
satisfied and then the upper sequence in the diagram of \ref{chap9-sec9.1}
is also exact. In this case the homomorphism of specialisation is {\em
  surjective.} 

\section{}\label{chap9-sec9.3}
Let $Y$ be locally noetherian and $X\to Y$ be a separable, proper
morphism with fibres universally connected. Suppose $y_{0}$, $y_{1}\in
Y$ with $y_{0}\in (\overline{y}_{1})$. Let $\overline{a}_{0}$,
$\overline{a}_{1}$ be geometric points of
$\overline{X}_{0}=\fprod{X}{\overline{k(y_{0})}}{Y}$ and
  $\overline{X}_{1}=\fprod{X}{\overline{k(y_{1})}}{Y}$
  respectively. Then there is a (natural) homomorphism of
  specialisation $\pi_{1}(\overline{X}_{1},\overline{a}_{1})\to
  \pi_{1}(\overline{X}_{0},\overline{a}_{0})$ which is {\em
    surjective.}

In fact, in view of Proposition \ref{chap7-prop7.3.2} we may make the
base-change $Y\leftarrow \Spec \widehat{\mathscr{O}}_{y_{0},Y}$ and apply the
above considerations to points above $y_{0}$ and $y_{1}$.

This is the so-called {\em semi-continuity of the fundamental group.}

\vfill\eject

\medskip

\addcontentsline{toc}{chapter}{\protect{Appendix to Chapter IX}}

\begin{center}
{\Large\bf Appendix to Chapter IX}
\bigskip

{\LARGE\bf The Fundamental Group of an Algebraic Curve}
\medskip

{\LARGE\bf Over an Algebraically Closed Field}
\end{center}

Let\pageoriginale $k$ be an algebraically closed field and $X$ a
proper, nonsingular, connected algebraic curve over $k$ (i.e., a
$k$-scheme of dimension 1).

\medskip
\noindent
{\bf Case 1.}~{\em Charac. $k=0$.}
\smallskip

We compute $\pi_{1}(X)$ in this case by assuming results of
transcendental geometry.

In view of Lemma \ref{chap7-lem7.2.1.4} and
Proposition \ref{chap7-prop7.3.2} one 
can assume that $k=\mathbb{C}$, the field of complex numbers. Then one
has an analytic structure on $X$ and on every $X'\in (\mathscr{E}t/X)$
(see GAGA - J.P. Serre); one can then show that the natural functor
$$
(\mathscr{E}t/X)\to \left\{
\begin{array}{l}
\text{Finite topological coverings of the}\\
\text{analytic space $X^{h}$ defined by $X$}
\end{array}
\right\}
$$
is an equivalence [cf. GAGA and Espaces fibr\'es alg\'ebriques -
\'Seminair Chevally 1958, (Anneaux de Chow) - See especially Prop. 19,
Cor. to Prop. 20 and Cor. to Theorem  3].

One thus obtains:

$\pi_{1}(X,a)\simeq \pi_{1}^{\widehat{\text{top}}}(X^{h},a)$, the
completion of the topological fundamental group
$\pi_{1}^{\text{top}}(X^{h},a)$ with respect to subgroups of finite
index. This group $\pi^{\text{top}}_{1}(X^{h},a)$ is ``known'' if $X$ is a
nonsingular, proper, connected curve of genus $g$. It is a group with
$2g$ generators $u_{i}$, $v_{i}(i=1,\ldots,g)$ subject to one
relation, namely 
$$
(u_{1}v_{1}u^{-1}_{1}v^{-1}_{1})(u_{2}v_{2}u^{-1}_{2}v^{-1}_{2})\ldots
(u_{g}v_{g}u^{-1}_{g}v^{-1}_{g})=1. 
$$

Before\pageoriginale going to the case charac. $k=p\neq 0$, we shall
briefly recall some evaluation-theoretic
results. (Ref. S. Lang-Algebraic Numbers or J.P. Serre-Corps Locaux).

Let $V$ be a discrete valuation ring with maximal ideal $\mathscr{M}$
and field of fractions $K$. Let $K'/K$ be a finite galois extension
and $V'$ the integral closure of $V$ in $K'$. Choose any maximal ideal
$\mathscr{M}'$ of $V'$ and consider the decomposition group of
$\mathscr{M}'$, i.e., the subgroup $\mathfrak{g}_{d}(\mathscr{M}')$ of
$G(K'/K)$ of elements leaving $\mathscr{M}'$ stable. Then there is a
natural homomorphism
$\mathfrak{g}_{d}(\mathscr{M}')\to
G(k(\mathscr{M}')/k(\mathscr{M}))$; the kernel
$\mathfrak{g}_{i}(\mathscr{M}')$ of this homomorphism is called the
{\em inertial group} of $\mathscr{M}'$. The maximal ideals of $V'$ are
transformed into each other by the action of the galois group $G(K'/K)$
and the various inertial groups are conjugates to each other (Compare with
Lemma \ref{chap4-lem4.2.1}). For the following definition it is irrelevant
which of the $\mathscr{M}'$ we take and therefore we shall write
$\mathfrak{g}_{i}$, or sometimes $\mathfrak{g}_{i}(K'/K)$ instead of
$\mathfrak{g}_{i}(\mathscr{M}')$. We say that $K'$ is {\em tamely
ramified} over $V$ if the order of $\mathfrak{g}_{i}$ is prime to the
characteristic of $k(\mathscr{M})$ and that $K'$ is {\em unramified}
over $V$ if $\mathfrak{g}_{i}$ is trivial.

One has then the following:
\begin{enumerate}
\renewcommand{\labelenumi}{(\theenumi)}
\item If $K'/V$ is tamely ramified then the inertial group
$\mathfrak{g}_{i}(K'/K)$ is {\em cyclic.}

\item If $K''\supset K'\supset K$ is a tower of finite galois
extensions and if $V'$ is a localisation of the integral closure of
$V$ in $K'$ and if $K'/V$, $K''/V'$ are tamely ramified then $K''/V$
is tamely ramified.

In\pageoriginale addition, one has an exact sequence:
$$
(e)\to \mathfrak{g}_{i}(K''/K')\to \mathfrak{g}_{i}(K''/K)\to \mathfrak{g}_{i}(K'/K)\to
(e) 
$$
(cf. Corps Locaux, Ch. I, Propostion 22).

\item Let $\tau\in \mathscr{M}$ be a uniformising parameter and
$n\in \mathbb{Z}^{+}$ be prime to charac. $k(\mathscr{M})$. If $K$
contains the $n^{\text{th}}$ roots of unity, then
$K'=\dfrac{K[X]}{(X^{n}-\tau)}$ is a finite galois extension, tamely
ramified, with galois group
$=\mathfrak{g}_{i}(K'/K)\simeq \mathbb{Z}/n\mathbb{Z}$. 
\end{enumerate}

\begin{lemma*}[(Abhyankar)]
Let $L$, $K'$ be finite galois extensions of $K$, tamely ramified over
$V$ with order $\mathfrak{g}_{i}(L/K)$ $(=n)$ dividing order
$\mathfrak{g}_{i}(K'/K)(=m)$. Let $L'$ be the composite extension of
$L$ and $K'$. Then $L'$ is unramified over the localisations of the
integral closure $V'$ of $V$ in $K'$.
\end{lemma*}

\begin{proof}
One checks that $\exists$ a monomorphism
$\mathfrak{g}_{i}(L'/K)\to \mathfrak{g}_{i}(K'/K)\times \mathfrak{g}_{i}(L/K)$
such that the projections
$\mathfrak{g}_{i}(L'/K)\to \mathfrak{g}_{i}(K'/K)$ and
$\mathfrak{g}_{i}(L'/K)\to \mathfrak{g}_{i}(L/K)$ are onto (see
2)). Since the orders of $\mathfrak{g}_{i}(L/K)$ and
$\mathfrak{g}_{i}(K'/K)$ are prime to $p=$ charac. $k(\mathscr{M})$,
so is the order of $\mathfrak{g}_{i}(L'/K)$, i.e., $L'/V$ is tamely
ramified; hence $\mathfrak{g}_{i}(L'/K)$ is cyclic (1)). As $n|m$, it
follows that each element of $\mathfrak{g}_{i}(L'/K)$ has order
dividing $m$. But $\mathfrak{g}_{i}(L'/K)\to \mathfrak{g}_{i}(K'/K)$
is onto. Thus,
$\mathfrak{g}_{i}(L'/K)\xrightarrow{\sim}\mathfrak{g}_{i}(K'/K)$. From
(2) the kernel of this map is $\mathfrak{g}_{i}(L'/K')$ which is
therefore trivial.\hfill Q.E.D.
\end{proof}

\medskip
\noindent
{\bf Case 2.}~{\em Charac. $k=p\neq 0$.}\pageoriginale 
\smallskip

\begin{defi*}
We say that a $Y$-prescheme $X$ is {\em smooth} at a point $x\in X$
(or $X\to Y$ is {\em simple} at $x$) if $\exists$ an open neighborhood
$U$ of $x$ such that the natural morphism $U\to Y$ admits a
factorisation of the form
$$
U\xrightarrow{\text{\'etale}}\Spec \mathscr{O}_{Y}\left[T_{1},\ldots,T_{n}\right]\to Y
$$
(the $T_{i}$ are indeterminates).
\end{defi*}

\begin{note*}
Smoothness is stable under base-change.
\end{note*}

\begin{defi*}
For a $Y$-prescheme $X$, the {\em sheaf of derivations} of $X$ over
$Y$ is the dual of the $\mathscr{O}$-Module $\Omega_{X\nmid Y}$; it is
denoted by $\mathfrak{g}_{X\nmid Y}$.
\end{defi*}

We shall assume the following

\begin{thm}
(SGA, 1960, III, Theorem (7.3)).
\end{thm}

Let $A$ be a complete noetherian local ring with residue field $k$. If
$X_{0}/k$ is a projective, smooth scheme such that
\begin{align*}
 & H^{2}(X_{0},\mathfrak{g}_{X_{0}|k})=(0)\\
\text{and}\quad & H^{2}(X_{0},\mathscr{O}_{X_{0}})=(0);
\end{align*}
then $\exists$ a projective, smooth $A$-scheme $X$ such that
$\foprod{X}{k}{A}\simeq X_{0}$. 

Let\pageoriginale $k$ be an algebraically closed field of
charac. $p\neq 0$ and $X_{0}$ a nonsingular (= smooth in this case)
connected curve, proper (hence, as is well-known projective) over
$k$. Consider the ring $A=W(k)$ of Witt-vectors; this is a complete,
discrete valuation ring with residue field $k$; if $K$ is the field of
fractions of $A$, then charac. $K=0$, and $K$ is complete for the
valuation defined by $A$. The conditions of Theorem 1 are satisfied by
$X_{0}$. Therefore $\exists$ a projective smooth $A$-scheme $X$ such
that $X_{0}\xleftarrow{\sim}\foprod{X}{k}{A}$. The $X_{0}$ is
universally connected; but then it follows that the generic fibre is
also universally connected (Use Stein-factorisation; Note that
$\Spec A$ has only two points). Furthermore, as $X$ is smooth over
$A$, the local rings of $X$ are regular (one has to show that if $X\to
Y$ is \'etale and $Y$ is regular then $X$ is regular). Finally we
mention the important fact that by Proposition \ref{chap8-prop8.1.3}, we
have: $\pi_{1}(X)\xleftarrow{\sim}\pi_{1}(X_{0})$. 

We may thus replace $X_{0}$ by $X$. Let $a_{0}$ be the closed point of
$\Spec A$ and $a_{1}$ the generic point; set $X_{K}=\foprod{X}{K}{A}$,
$\overline{X}_{K}=\foprod{X}{\overline{K}}{A}$, where as usual
$\overline{K}$ is the algebraic closure of $K$. If
$\overline{a}_{0}\in X_{k}=\foprod{X}{k}{A}$ is a geometric point over
$a_{0}$ and $\overline{a}_{1}\in\overline{X}_{K}$, a geometric point
over $a_{1}$ then we have the homomorphism of specialisation 
$$
\pi_{1}(\overline{X}_{K},\overline{a}_{1})\to \pi_{1}(X_{k},\overline{a}_{0})\simeq \pi_{1}(X)
$$
which is surjective (\ref{chap9-sec9.2}).

From\pageoriginale case 1 one already ``knows'' about
$\pi_{1}(\overline{X}_{K},\overline{a}_{1})$
(charac. $\overline{K}=0$). Thus it only remains to study the kernel
of the above epimorphism. This amounts by \ref{chap5-sec5.2.4} to studying
the following question: Given a connected \'etale covering
$\overline{Z}$ of $\overline{X}_{K}$ which is {\em galois}, when does
there exist a $Z\in(\mathscr{E}t/X)$ such that
$Z\xleftarrow{\sim}\foprod{Z}{\overline{K}}{A}$. [Remember: a
connected \'etale covering is galois if the degree of the covering
equals the number of automorphisms. Also note: we have integral
schemes because, for instance, our schemes are connected and
regular. Therefore we may consider the function fields
$R(\overline{Z})$ and $R(\overline{X}_{K})$ of $\overline{Z}$ and
$\overline{X}_{K}$. Clearly if $\overline{Z}$ is galois over
$\overline{X}_{K}$ then $R(\overline{Z})$ is a galois extension of
$R(\overline{X}_{K})$.]. 

By Lemma \ref{chap7-lem7.2.1.4}, $\exists$ a finite subextension $K_{1}$ of
$K$ and a $Z_{K_{1}}\in(\mathscr{E}t/X_{K_{1}}\break =\foprod{X}{K_{1}}{A})$
such that
$\overline{Z}\xleftarrow{\sim}\foprod{Z_{K_{1}}}{\overline{K}}{K_{1}}$. 
Then the above question takes the form: Given a connected
$\overline{Z}\in (\mathscr{E}t/\overline{X}_{K})$, which is galois
when does there exist a $Z\in(\mathscr{E}t/X)$ and a finite extension
$K_{2}$ of $K_{1}$ such that
$Z_{K_{2}}=\foprod{Z_{K_{1}}}{K_{2}}{K_{1}}\xleftarrow{\sim} \foprod{Z}{K_{2}}{A}$? 

Now, for any finite extension $K'$ of $K$, let $A'$ be the integral
closure of $A$ in $K'$; then by a corollary to Hensel's lemma it
follows that $A'$ is again a discrete valuation ring whose residue
field is again $k$ (recall that $k$ is algebraically closed). Thus
$\pi(X)\simeq \pi_{1}(X_{k})\simeq\pi_{1}(X_{A'})$ where
$X_{A'}=\foprod{X}{A'}{A}$, again by Proposition \ref{chap8-prop8.1.3}. Thus
the question becomes: Given\pageoriginale $Z_{K_{1}}\in
(\mathscr{E}t/X_{K_{1}})$, universally connected and galois, when does
there exist a finite extension $K_{2}$ of $K_{1}$ such that
$Z_{K_{2}}=\foprod{Z_{K_{1}}}{K_{2}}{K_{1}}$ comes from a $Z_{A_{2}}$
in $(\mathscr{E}t_{/X_{A_{2}}}=\foprod{X}{A_{2}}{A})$ (where $A_{2}$ is
the integral closure of $A$ in $K_{2}$)?

Consider any finite extension $K'$ of $K_{1}$ and let $A'$ be the
integral closure of $A$ in $K'$. We are given a situation of the form:


$\begin{smallmatrix}\text{\mycirc{$Z_{K'}$}}\\[4pt] \text{\mycirc{$X_{K'}$}}\end{smallmatrix}  
+ p'$- generic point of the fibre over the closed
point. 

\smallskip

$\displaystyle{\mathop{\sim\Spec K'}^{\text{generic
point}}}-\underbrace{0\quad \cdot}_{\Spec A'}$-closed point (residue
field $k$).

The $Z_{K_{1}}$ (resp. $Z_{K'}=\foprod{Z_{K_{1}}}{K'}{K_{1}}$) are
connected and hence, as we have already remarked, integral. Let
$R(Z_{K_{1}})$, $R(Z_{K'})$, $R(X_{K_{1}})$ and $R(X_{K'})$ be the
function fields of $Z_{K_{1}}$, $Z_{K'}$, $X_{K_{1}}$ and
$X_{K'}$. Then $R(Z_{K_{1}})$ is a finite galois extension of
$R(X_{K_{1}})$ and
$$
R(Z_{K'})=\foprod{R(Z_{K_{1}})}{K'}{K_{1}}=\foprod{R(Z_{K_{1}})}{R(X_{K'})}{R(X_{K_{1}})}.
$$ 
Our
aim now is to choose, if possible, the field $K'\supset K_{1}$ such
that $R(Z_{K'})$ is {\em unramified} over $X_{A'}$. By this we mean
the following: consider the normalisation $Z'$ of $X'=X_{A'}$ in
$R(Z_{K'})$; we want $Z'$ to be unramified\pageoriginale over $X'$.


({\bf Note:}~Since $Z_{K'}$ is regular, so certainly normal, we have
that $\foprod{Z'}{K'}{A'}\break \xrightarrow{\sim}Z_{K'}$, because the
process of normalisation is unique. -- See EGA, II, \S\ (6.3) and also
S. Lang, Introduction to Algebraic Geometry, Ch. V). We can also
express this as follows: we want $Z'$ to be unramified over the whole
of $X'$ and not merely on the open subscheme
$X_{K'}=\foprod{X'}{K'}{A'}$. 

Consider the integral closure $A_{1}$ of $A$ in $K_{1}$. Let $p$ be
the generic point of the fibre in $X_{A_{1}}=\foprod{X}{A_{1}}{A}$
over the closed point (as a space, this is nothing but $X_{0}$). Then
$p$ is a point of codimension 1 in $X_{A_{1}}$; furthermore
$X_{A_{1}}/A_{1}$ is smooth and therefore the local ring
$\mathscr{O}_{1}$ of $p$ (as a point in $X_{A_{1}}$) is
regular. Therefore $\mathscr{O}_{1}$ is a discrete valuation ring in
$R(X_{A_{1}})=R(X_{K_{1}})$. Similarly define $\mathscr{O}'$ in
$R(X_{K'})=R(X_{A'})$ for any $K'\supset K_{1}$. It is easily checked
that $\mathscr{O}'$ is the integral closure of $\mathscr{O}_{1}$ in
$R(X_{K'})$. Now, any open set in $X_{A'}$, containing $X_{K'}$ and
the generic point $p'$ of the fibre over the closed point of $\Spec
A'$ is such that its complement is of codimension $\geq 2$ in
$X'_{A}$. We have the following theorem depending on the so-called
{\em purity of the branch locus}:

\begin{thm}
(SGA, 1960-1961 expose X, Cor (3.3))
\end{thm}

Let $P$ be a locally noetherian regular prescheme and $U$ the
complement of a closed set of codimension $\geq 2$ in $P$. Then
the\pageoriginale natural functor
$(\mathscr{E}t/P)\xrightarrow{\Phi}(\mathscr{E}t/U)$ is an
equivalence. (For a proof see SGA, X, 1962).

[For the special case of the theorem which we need, there is a direct
proof in SGA, X, 1961, p.~16. However even for that proof one needs
the following result (which we have {\em not} proved in these
lectures). An unramified covering of a normal prescheme
is \'etale. (SGA, I, 1960-1961, Theorem (9.5)).]

Thus, it is enough to prove that $\exists K'\supset K_{1}$ such that
$R(Z')$ is unramified over $X'$ at the point $p'$ because $Z'$ is
clearly flat over $X'=X_{A'}$ at the point $p$ (the local ring
$\mathscr{O}'$ of $p'$ in $X'$ is a discrete valuation ring).

If $\tau$ is a uniformising parameter of $A_{1}$, it is also a
parameter of $\mathscr{O}_{1}$. Let then $n\in \mathbb{Z}^{+}$ be
such that $(n,p)=1$ and set $K'=K_{1}[T]/(T^{n}-\tau)$. Then
$K'/K_{1}$ is a finite galois extension and $R(X_{K'})\simeq
R(X_{K_{1}})[T]/(T^{n}-e)$. Hence $R(X_{K'})$ is tamely ramified
over $\mathscr{O}_{1}$, and has an inertial group of order $n$.

Assume now that the degree of the galois covering $\overline{Z}$ over
$\overline{X}_{K}$ is prime to $p=$ charac.$k$. Then one may take $n$
equal to this degree. By Abhyankar's lemma one then has that $R(Z')$
is unramified over $X'$ above the point $p'$. Thus one has ``proved''
the 

\setcounter{proposition}{2}
\begin{proposition}
The kernel of the (surjective) homomorphism of specialisation
$\pi_{1}(\overline{X}_{K},\overline{a}_{1})\to \pi_{1}(X_{0},\overline{a}_{0})$
is contained in the inter-section of the kernels of continuous
homomorphisms from $\pi_{1}(\overline{X}_{K})$ to finite groups of
order prime to $p$.
\end{proposition}

Therefore\pageoriginale if, for a profinite group $\mathscr{G}$, we
denote by $\mathscr{G}(p)$ the profinite group
$\varprojlim \mathscr{G}_{i}$ where the limit is taken over all
quotients $\mathscr{G}_{i}$ of $\mathscr{G}$ which are finite of order
prime to $p$, then we have:
$$
\pi^{(p)}_{1}(\overline{X}_{K},\overline{a}_{1})\xrightarrow{\sim}\pi^{(p)}_{1}(X_{0},\overline{a}_{0}). 
$$
(One can also say: $\mathscr{G}^{(p)}$ is the quotient of
$\mathscr{G}$ by the closed normal subgroup generated by the $p$-Sylow
subgroups of $\mathscr{G}$). Since we ``know'', by topological
methods, the group $\pi_{1}(\overline{X}_{K},\overline{a}_{1})$ we
obtain: 

\setcounter{thm}{3}
\begin{thm}
If $X$ is a nonsingular, connected, proper curve of genus $g$ over an
algebraically closed field $k$ of charac. $p\neq 0$, then
$\pi^{(p)}_{1}(X)\xrightarrow{\sim}\mathscr{G}^{(p)}$ where
$\mathscr{G}$ is the completion with respect to subgroups of finite
index of a group with $2g$ generators $u$, $u_{i}$, $v_{i}
(i=1,2,\ldots,g)$ with one relation:
$$
(u_{1}v_{1}u^{-1}v^{-1}_{1})(u_{2}v_{2}u^{-1}_{2}v^{-1}_{2})\ldots
(u_{g}v_{g}u^{-1}_{g}v^{-1}_{g})=1. 
$$
\end{thm}

\begin{thebibliography}{99}
\bibitem{key1} {\bf Bourbaki, N.} Alg\'ebre Commutatif (Hermann,
Paris)

\bibitem{key2} {\bf Grothendieck, A.} \'El\'ements de G\'eom\'etrie
Alg\'ebrique (EGA) - Volumes I, II, III, IV. (I.H.E.S. Publication)

\bibitem{key3} {\bf -- } S\'eminaire de G\'eom\'etrie Alg\'ebrique
(SGA) - Volumes I, II, (1960-61) I, II (1962)

\bibitem{key4} {\bf Lang, S.} Introduction to Algebraic Geometry
(Interscience Publication)

\bibitem{key5} {\bf Serre, J.P.} Corps locaus (Hermann, Paris)

\bibitem{key6} {\bf -- } Groupes alg\'ebriques et corps de classes
(Hermann, Paris)

\bibitem{key7} {\bf --} Espaces Fibr\'es alg\'ebriques (Expos\'e 1,
S\'eminaire Chevalley (1958) -- Anneaux de Chow et applications)

\bibitem{key8} {\bf -- } G\'eom\'etrie Alg\'ebrique et G\'eom\'etrie
Analytique (Annales de l'Institut Fourier, t.6, 1955-56, pp.1-42)

\end{thebibliography}

\label{addchap9}
