\chapter{Galois Categories and Morphisms of Profinite Groups}\label{chap5}

\section{}\label{chap5-sec5.1}\pageoriginale

Suppose $\pi$ and $\pi'$ are profinite groups and $u:\pi'\to \pi$ a
continuous homomorphism. Then $u$ defines, in a natural way, a functor
$H_{u}:\mathscr{C}(\pi)\to \mathscr{C}(\pi')$; and $H_{u}$ being the
identity functor on the underlying sets is fundamental.

On the other hand, let $\mathscr{C}$, $\mathscr{C}'$ be galois
categories, $\pi'$ a profinite group and $H:\mathscr{C}\to
\mathscr{C}'$, $F':\mathscr{C}'\to\mathscr{C}(\pi')$ be functors such
that $F=F'\circ H$ is fundamental. Then we can choose a pro-object
$\widetilde{S}=\{S_{i},\varphi_{ij}\}_{i\in I}$ of $\mathscr{C}$ such
that, $\forall X\in \mathscr{C}$,
$F(X)\xleftarrow{\sim}\Hom_{\Pro\mathscr{C}}(\widetilde{S},X)$. Moreover
we may assume that the $S_{i}$ are galois objects of $\mathscr{C}$,
therefore the $F(S_{i})$ are principal homogeneous spaces under the
action of the $\pi_{i}$ (on the right) (notations from Ch. \ref{chap4}) and
if we identify $F(S_{i})$, by means of the canonical element
$\tau_{i}$, with $\pi_{i}$ then the maps
$F(\varphi_{ij})=\psi_{ij}:F(S_{j})\to F(S_{i})$ are group
homomorphisms (see Ch. \ref{chap4}). However, in the present situation the
$F(S_{i})$ are not merely sets but are objects of $\mathscr{C}(\pi')$;
as such, the group $\pi'$ acts continuously upon the sets to the left,
and this action commutes with the right-action of the $\pi_{i}$. This
gives a continuous homomorphism $u_{i}:\pi'\to \pi_{i}$ determined by
the condition that for $\sigma'\in\pi'$ the $u_{i}(\sigma')$ is the
unique element of $\pi_{i}$ such\pageoriginale that $\sigma'\cdot
\tau_{i}=\tau_{i}\cdot u_{i}(\sigma')$ ($\tau_{i}$ is the canonical
element of $F(S_{i})$); for $j\geq i$ we clearly have: $\psi_{ij}\circ
u_{j}=u_{i}$; thus, we obtain a continuous homomorphism thus, we
obtain a continuous homomorphism $u:\pi'\to
\varprojlim_{i}\pi_{i}=\pi$ and $u$ corresponds to $H$ if we identify
$\mathscr{C}$ with $\mathscr{C}(\pi)$.

\begin{example*}
We shall apply the above to the particular case
$\mathscr{C}=(\mathscr{E}t/S)$, $\mathscr{C}'=(\mathscr{E}t/S')$ where
$S$, $S'$ are, as usual locally noetherian, connected
preschemes. Suppose $\varphi:S'\to S$ is a morphism of finite type and
$s'\in S'$, $s=\varphi(s')\in S$. Let $\Omega$ be an algebraically
closed field containing $k(s')$. We define functors
$F:\mathscr{C}\to\{\text{Finite sets}\}$ and
$F':\mathscr{C}'\to\{\text{Finite sets}\}$ by defining:
\begin{align*}
& F(X) =\Hom_{S}(\Spec \Omega,X),X\in \mathscr{C},\\
& F'(X')=\Hom_{S'}(\Spec \Omega,X'),X'\in\mathscr{C}'. 
\end{align*}
\end{example*}

Denote by $\pi$ and $\pi'$ the fundamental groups $\pi_{1}(S,s)$ and
$\pi_{1}(S',s')$. 

Corresponding to the morphism $\varphi:S'\to S$, we obtain a functor
$\Phi:\mathscr{C}\to \mathscr{C}'$ given by $X\mapsto
\fprod{X}{S'}{S}$. We then have:
\begin{align*}
F'(\fprod{X}{S'}{S}) &= \Hom_{s'}(\Spec \Omega, \fprod{X}{S'}{S})\\
&\simeq \Hom_{s}(\Spec \Omega,X)=F(X).
\end{align*}

Thus\pageoriginale $F=F'\circ\Phi$; in view of the fact that $F$ is
fundamental and the equivalences
$\mathscr{C}{\displaystyle{\mathop{\sim}^{F}}}\mathscr{C}(\pi)$,
$\mathscr{C}'{\displaystyle{\mathop{\sim}^{F}}}\mathscr{C}(\pi')$, we
obtain a continuous homomorphism $\pi'\to \pi$.

\section{}\label{chap5-sec5.2}
In this section, we shall correlate the properties of a homomorphism
$u:\pi'\to \pi$ and those of the corresponding functor
$H_{u}:\mathscr{C}(\pi)\to \mathscr{C}(\pi')$. 

\subsection{}\label{chap5-sec5.2.1}
Suppose $u:\pi'\to \pi$ is onto; for any connected object $X$ of
$\mathscr{C}(\pi)$ (i.e., $\pi$ acts transitively on $X$) any
$\pi$-morphism $\pi\to X$ defined by, say, $e\mapsto x$ is onto and
therefore so is the map $\pi'\to X$ defined by $e'\mapsto x$; in other
words, $H_{u}(X)$ is a connected object of $\mathscr{C}(\pi')$.

Conversely, suppose that for any connected $X\in\mathscr{C}(\pi)$, the
object $H_{u}(X)$ is again connected in $\mathscr{C}(\pi')$. Write
$\pi=\varprojlim_{i}\pi_{i}$ where the $\pi_{i}$ are finite groups and
the structure-homomorphisms $\pi_{j}\to \pi_{i}$, $j\geq i$, are all
onto; this implies that the $\pi\to \pi_{i}$ are all onto and by our
assumption then all the $\pi'\to \pi_{i}$ are onto. Since $\pi$ and
$\pi'$ are both profinite it follows that $u:\pi'\to \pi$ is
onto. Thus: $u:\pi'\to \pi$ {\em is onto} $\Leftrightarrow$ {\em for
  any connected} $X\in \mathscr{C}(\pi)$, {\em the object} $H_{u}(X)$
{\em is connected in} $\mathscr{C}(\pi')$. 

\subsection{}\label{chap5-sec5.2.2}
A\pageoriginale {\em pointed object} 
of $\mathscr{C}(\pi)$ is, by definition, a pair $(X,x)$ with
$X\in\mathscr{C}(\pi)$ and $x\in X$. By the definition of the topology
on $\pi$, it is clear that giving a pointed, connected object of
$\mathscr{C}(\pi)$ is equivalent to giving an open subgroup $H$ of
$\pi$; the object is $\pi/H=$ set of left-cosets of
$\pi\text{\,mod\,}H$ and the point is the class $H$. A final object of
$\mathscr{C}(\pi)$ is a point $e_{c}$ on which $\pi$ acts
trivially. We say that an $X\in\mathscr{C}(\pi)$ has a section if
there is a $\mathscr{C}(\pi)$-morphism from a final object $e_{c}$ to
$X$; giving a section of $X$ is equivalent to giving a point of $X$,
invariant under the action of $\pi$. A pointed object $(X,x)$ of
$\mathscr{C}(\pi)$ admits a pointed section (i.e., $e_{c}$ is mapped
onto $x$) $\Leftrightarrow x$ is invariant under $\pi$.

Suppose $u:\pi'\to \pi$ is a homomorphism and $H$ is an open subgroup
of $\pi$ such that $u(\pi')\subset H$. Let $(X,x)$ be the pointed,
connected object of $\mathscr{C}(\pi)$ determined by $H$. Then, in the
action of $\pi'$ on $H_{u}(X)$, $x$ remains invariant, i.e., to say,
the pointed object $(H_{u}(X),x)$ of $\mathscr{C}(\pi')$ admits a
pointed section. The converse situation is clear. Thus:

{\em For an open subgroup $H$ of $\pi$, one has $u(\pi')\subset
  H\Leftrightarrow H_{u}(\pi/H)$ admits a pointed section in
  $\mathscr{C}(\pi')$.} 

\subsection{}\label{chap5-sec5.2.3}
We say that an $X\in\mathscr{C}(\pi)$ is {\em completely decomposed}
if $X$ is a finite disjoint sum of final objects of
$\mathscr{C}(\pi)$, i.e., if the action of $\pi$ on $X$ is trivial.

Suppose\pageoriginale $u:\pi'\to \pi$ is trivial, then for any
$X\in\mathscr{C}(\pi)$, $H_{u}(X)$ is completely decomposed in
$\mathscr{C}(\pi')$. Conversely, assume that for any
$X\in\mathscr{C}(\pi)H_{u}(X)$ is completely decomposed in
$\mathscr{C}(\pi')$. Write $\pi=\varprojlim_{i}\pi_{i}$ as usual; by
assumption, each composite $\pi'\to \pi\to \pi_{i}$ is trivial. Hence
$u:\pi'\to \pi$ is also trivial. Thus:

{\em $u:\pi'\to \pi$ is trivial $\Leftrightarrow$ for any
  $X\in\mathscr{C}(\pi)$, $H_{u}(X)$ is completely decomposed in
  $\mathscr{C}(\pi')$.} 

\subsection{}\label{chap5-sec5.2.4}
Let $H'$ be an open subgroup of $\pi'$ and $X'\in\mathscr{C}(\pi')$
the connected, pointed object defined by $H'$. Assume that $\ker
u\subset H'$; then $u(\pi')/u(H')\simeq \pi'/H'$ in
$\mathscr{C}(\pi')$. This means that $u(H')$ is a subgroup of finite
index of the pro-finite group $u(\pi')$ and hence is open in
$u(\pi')$. Since $\pi'$ is compact and $\pi$ Hausdorff we can find an
open subgroup $H$ of $\pi$ such that $H\cap u(\pi')\subset u(H')$.

Consider now the connected, pointed object $X=\pi/H$ of
$\mathscr{C}(\pi)$. Denote by $H_{u}(X)_{0}$ the
$\mathscr{C}(\pi')$-component of the pointed object $H_{u}(X)$ of
$\mathscr{C}(\pi')$, containing the distinguished point of
$H_{u}(X)$. There exists then an open subgroup $H'_{1}$ of $\pi'$ such
that $H_{u}(X)_{0}\simeq \pi'/H'_{1}$ in $\mathscr{C}(\pi')$. We claim
that $H'_{1}\subset H'$; in fact, $u(H'_{1})\subset H$ and so
$u(H'_{1})\subset H\cap u(\pi')\subset u(H')$, hence $H'_{1}\subset
u^{-1}(u(H'_{1}))\subset u^{-1}(u(H'))=H'$, since, by assumption, $H'$
is saturated under $u$.

Thus,\pageoriginale there is a pointed $\mathscr{C}(\pi')$-morphism
$H_{u}(X)_{0}\simeq \pi'/H'_{1}\to \pi'/H'\break \simeq X'$. If, on the other
hand, we assume that $\exists$ a pointed, connected object $X$ of
$\mathscr{C}(\pi)$ such that we have a pointed
$\mathscr{C}(\pi')$-morphism $H_{u}(X)_{0}\simeq \pi'/H'_{1}\to
X'\simeq \pi'/H'$, then, we must have $H'_{1}\subset H'$ and hence
$\ker u\subset H'_{1}\subset H'$. If $u$ is surjective, then we can
say that $H_{u}(X)\simeq X'$ (see \ref{chap5-sec5.2.1}). Also $\ker u\subset
H'$ is a relation independent of the choice of the distinguished point
in $X'\simeq \pi'/H'$. Thus:

{\em $\ker u\subset H'\Leftrightarrow \exists$ a connected object $X$
  of $\mathscr{C}(\pi)$ and a $\mathscr{C}(\pi')$-mor\-phism of a
  connected $\mathscr{C}(\pi')$-component of $H_{u}(X)$ to
  $X'=\pi'/H'$. If $u$ is onto then $X'=\pi'/H'\simeq H_{u}(X)$ for a
  connected object $X\in\mathscr{C}(\pi)$.}

In particular:

{\em $u$ is injective $\Leftrightarrow$ for every connected $X'\in
  \mathscr{C}(\pi')$, there is a connected $X\in\mathscr{C}(\pi)$ and
  a $\mathscr{C}(\pi')$-morphism from a $\mathscr{C}(\pi')$-component
  of $H_{u}(X)$ to $X'$.}

\subsection{}\label{chap5-sec5.2.5}
Let $\pi'\xrightarrow{u'}\pi\xrightarrow{u''}\pi''$ be a sequence of
morphisms of profinite groups. From \ref{chap5-sec5.2.3} and
\ref{chap5-sec5.2.4} 
we obtain the following necessary and sufficient conditions for the
sequence to be exact:
\begin{itemize}
\item[(a)] $u''\circ u'$\pageoriginale is trivial $\Leftrightarrow$
  {\em for any} $X''\in \mathscr{C}(\pi'')$, $H_{u'}\circ
  H_{u''}(X'')$ is {\em completely decomposed in} $\mathscr{C}(\pi')$.

\item[(b)] $\Iim u'\supset \ker u''\Leftrightarrow$ for any open
  subgroup $H$ of $\pi$, with $H\supset \Iim u'$, we also have
  $H\supset \ker u''\Leftrightarrow$ {\em for any connected pointed
    object $X$ of $\mathscr{C}(\pi)$ such that $H_{u'}(X)$ admits a
    pointed section in $\mathscr{C}(\pi')$, there is a connected
    object $X''\in\mathscr{C}(\pi'')$} and a $\mathscr{C}(\pi)$-{\em
    morphism of a $\mathscr{C}(\pi)$-component of $H_{u''}(X'')$ to $X$.}
\end{itemize}

\subsection{}\label{chap5-sec5.2.6}
Let $\mathscr{C}$ be a galois category with a fundamental functor
$F$. Let $\widetilde{S}=(S_{i})_{i\in I}$ be a pro-object of
$\mathscr{C}$ with usual properties (in particular, $S_{i}$ are
galois) pro-representing $F$. Let $\pi$ be the fundamental group of
$\mathscr{C}$ determined by $F$. We know then that
$\mathscr{C}{\displaystyle{\mathop{\sim}^{F}}}\mathscr{C}(\pi)$. 

Let now $T\in\mathscr{C}$ be a connected object and $t\in F(T)$ be
fixed for our considerations. We form the category
$\mathscr{C}'=\mathscr{C}|T$; it is then readily checked that
$\mathscr{C}'$ satisfies the axioms
$(\mathscr{C}_{0}),\ldots,(\mathscr{C}_{4})$ of Ch. IV. We have an
exact functor $\mathscr{P}:\mathscr{C}\to \mathscr{C}'$ defined by
$X\mapsto X\times T$. We now define a functor
$F':\mathscr{C}'\to\{\text{Finite sets}\}$ by setting, for any
$X\in\mathscr{C}'$, $F'(X)=$ inverse image of $t$ under the map
$F(X)\to F(T)$. Again it is easily checked that $\mathscr{C}'$,
equipped with $F'$, is galois. A cofinal subsystem $\widetilde{S'}$ of
$\widetilde{S}$ is defined by the condition:
$S_{i}\in\widetilde{S'}\Leftrightarrow (S_{i},\tau_{i})$ dominates
$(T,t)$ in the sense of Ch. \ref{chap4}. An\pageoriginale
$S_{i}\in\widetilde{S'}$ can be considered in the obvious way as an
object of $\mathscr{C}'$; it is then easily shown that they are galois
in $\mathscr{C}'$ and the pro-object $\widetilde{S'}$ of
$\mathscr{C}'$ pro-represents $F'$ (To do the checking one may
identify $\mathscr{C}$ and $\mathscr{C}(\pi)$). 

Let $H$ be the isotropy group of $t\in F(T)$ in $\pi$; let also
$N_{i}$ be the isotropy groups of $\tau_{i}\in F(S_{i})$,
$S_{i}\in\widetilde{S'}$. We have a diagram of the form
\[
\xymatrix@=1.2cm{
\pi \ar[rr]\ar[dr] & & F(T)\ni t\\
 & \tau_{i}\in F(S_{i})\ar[ur] &
}
\]
which is commutative and thus $N_{i}\subset H$, $\forall i$. Since the
$F(S_{i})$ and $F(T)$ are all connected objects of $\mathscr{C}(\pi)$,
we have $\frac{\pi}{N_{i}}\simeq F(S_{i})$ and $\pi/H\simeq F(T)$. The
maps $F(S_{i})\to F(T)$ are then the natural maps $\pi/N_{i}\to \pi/H$
and it follows that $F'(S_{i})\simeq H/N_{i}$ in
$\mathscr{C}(\pi)$. But as we have already remarked the $S_{i}$ in
$\widetilde{S'}$ form a system of galois objects with respect to $F'$,
pro-representing $F'$ and one thus obtains:
$$
\pi'\simeq
\varprojlim_{S_{i}\in\widetilde{S'}}F'(S_{i})=\varprojlim_{N_{i}\subset
  H}H/N_{i}\approx H.
$$

Finally we remark that the composite functor $F'\circ \mathscr{P}$ is
isomorphic with $F$ and therefore fundamental; following the procedure
of \ref{chap5-sec5.1} we see that the corresponding continuous homomorphism
$u:\pi'\to \pi$ is nothing but the canonical inclusion
$H\hookrightarrow \pi$. 
