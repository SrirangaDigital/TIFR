\chapter{An Application of the Existence Theorem}\label{chap8}

\section{The second homotopy exact
  sequence}\label{chap8-sec8.1}\pageoriginale 

Let $A$ be a {\em complete noetherian local ring} and $S=\Spec A$; let
$s_{0}\in S$ be the closed point of $S$. Let $X$ be a {\em proper
  $S$-scheme} and set
$\overline{X}_{0}=\fprod{X}{\overline{k(s_{0})}}{S}$. 

Let $\overline{a}_{0}\in\overline{X}_{0}$ be a geometric point of
$\overline{X}_{0}$ (over some fixed algebraically closed field
$\Omega\supset k(s_{0})$) and $a_{0}\in X$ be its image in $X$. Assume
that $\overline{X}_{0}$ is {\em connected}.
\[
\xymatrix@=1.2cm{
a_{0}\in X\ar[d]^{f}_{\text{proper}} & \overline{X}_{0}\ni
\overline{a}_{0}\ar[l]\ar[d]\\
s_{0}\in\Spec A & \overline{k(s_{0})}\ar[l]  
}
\]

\begin{theorem}\label{chap8-thm8.1.1}
The sequence
$$
e\to \pi_{1}(\overline{X}_{0},\overline{a}_{0})\to \pi_{1}(X,a_{0})\to
\pi_{1}(S,s_{0})\to e
$$
is {\em exact}; and we have the isomorphism
$$
\pi_{1}(S,s_{0})\cong G(\overline{k(s_{0})}/k(s_{0})).
$$
\end{theorem}

(Note: Compare with Theorem \ref{chap6-thm6.3.2.1}).

The\pageoriginale proof of the theorem is a consequence of the
following results.

\setcounter{prop}{1}
\begin{prop}\label{chap8-prop8.1.2}
In the above Theorem \ref{chap8-thm8.1.1} assume $A$ is an artinian local
ring. With the same notation, the same assertions hold.
\end{prop}

\begin{proof}
Since the introduction of nilpotent elements does not affect the
fundamental groups (Theorem \ref{chap7-sec7.2.1}). We may assume that
$\underline{A=k(s_{0})}$. In this case, the assertion
$\pi_{1}(S,s_{0})\cong G(\overline{k(s_{0})}/k(s_{0}))$ is
clear. Next, if characteristic $k(s_{0})=p$ and if
$k'=(k(s_{0}))^{p^{-\infty}}$, then the morphism $\Spec
k(s_{0})\leftarrow \Spec k'$ is faithfully flat, quasi-compact and
radiciel and hence by Proposition \ref{chap7-prop7.2.2}, we may replace
$k(s_{0})$ by $(k(s_{0}))^{p^{-\infty}}$, i.e., we may assume
$k(s_{0})$ is {\em perfect}. In this case, $\overline{k(s_{0})}$ is
the inductive limit $\varinjlim_{i\in I}k_{i}$ of finite galois
extensions $k_{i}$ of $k(s_{0})$; set
$X_{i}=\fprod{X}{k_{i}}{k(s_{0})}$ and $a_{i}=$ image of
$\overline{a}_{0}$ in $X_{i}$. 
\[
\xymatrix@=1.2cm{
a_{0}\in X\ar[d] & X_{i}\ni a_{i}\ar[l]\ar[d] & \overline{X}_{0}\ni
\overline{a}_{0}\ar[d]\ar[l]\\
k(s_{0}) & k_{i}\ar[l] & \overline{k(s_{0})}\ar[l]
}
\] 

By\pageoriginale Lemma \ref{chap7-lem7.2.1.4} an \'etale covering of
$\overline{X}_{0}$ is 
determined by an \'etale covering of some $X_{i}$ and the latter is
uniquely determined modulo passage to $X_{j}$, $j\geq i$.

One thus gets the isomorphism:
$$
\pi_{1}(\overline{X}_{0},\overline{\underline{a}}_{0})\xrightarrow{\sim}
\varinjlim_{i}\pi_{1}(X_{i},a_{i}) 
$$
[The injectivity follows from the fact that for any open subgroup $H$
  of $\pi=\pi_{1}(\overline{X}_{0},\overline{a}_{0})$ there exists, by
  Lemma \ref{chap7-lem7.2.1.4} an index $i$ and an open subgroup $H^{(i)}$
  of $\pi^{(i)}=\pi_{1}(X_{i},a_{i})$ such that $\pi/H\simeq
  \pi^{(i)}/H^{(i)}$. The surjectivity follows because otherwise there
  would exist a set $E\in\mathscr{C}(\pi^{(i)})$ with two points $a$
  and $b$ in $\dot{E}$ such that $a$ and $b$ are in the same connected
  component of $E$ with respect to the action of all $\pi^{(j)}(j\geq
  i)$ but $a$ and $b$ lie in different components with respect to the
  action of $\pi$; again by \ref{chap7-lem7.2.1.4} this is impossible because
  a connected component of $E$ in $\mathscr{C}(\pi)$ can be realised
  in some $\mathscr{C}(\pi^{(j)})$].

On the other hand, we assert that each $X_{i}/X$ is galois and
$G(k_{i}/k(s_{0}))\break \xrightarrow{\sim}\Aut (X_{i}/X)$. 

In fact, suppose we have a situation of the following type:
\[
\xymatrix@=1.2cm{
a_{0}\in X\ar[d] & X'=\fprod{X}{k'\ni a'}{k}\ar[l]_-{\varphi}\ar[d]\\
k & k'\ar[l]
}
\] 
with\pageoriginale $X$ universally connected over $k$. Then $X'/X$ is
a connected 
\'etale covering. Also, we have: $\deg(X'/X)=\rank \varphi=$ number of
geometric points in the fibre $\varphi^{-1}(a_{0})=\deg (k'/k)=$
number of automorphisms of $k'/k\leq$ number of automorphisms of
$X'/X\leq$ number of geometric points in the fibre
$\varphi^{-1}(a_{0})$ (because $X$ is {\em connected} - see the proof
of Lemma \ref{chap4-lem4.4.1.6}). Hence
$\Aut(k'/k)\xrightarrow{\sim}\Aut(X'/X)$ and $X'/X$ is galois. Now for
every $i\in I$ we have an exact sequence:
$$
(e)\to \pi_{1}(X_{i},a_{i})\to \pi_{1}(X,a_{0})\to \Aut (X_{i}/X)\to (e)
$$
(see \ref{chap5-sec5.2.6}). Since each $\pi_{1}$ is pro-finite, by passing
to the projective limit we obtain an {\em exact} sequence:
$$
(e)\to \pi_{1}(\overline{X}_{0},\overline{a}_{0})\to
\pi_{1}(X,a_{0})\to
G(\overline{k(s_{0})}/k(s_{0}))=\pi_{1}(S,s_{0})\to (e)
$$
\end{proof}



\begin{prop}\label{chap8-prop8.1.3}
Let $A$ be a complete, noetherian local ring and $S=\Spec A$. Let $X$
be a {\em proper} $S$-scheme such that if $s_{0}\in S$ is the closed
point of $S$ the fibre $X_{0}=\fprod{X}{k(s_{0})}{S}$ is universally
connected. Let $a'_{0}$ be a geometric point of $X_{0}$ with values in
some algebraically closed $\Omega\supset k(s_{0})$. If $a_{0}\in X$ is
the image of $a'_{0}$ then canonically
$\pi_{1}(X,a_{0})\xleftarrow{\sim}\pi_{1}(X_{0},a'_{0})$. 
\end{prop}

\begin{proof}
This will come from the fact that the natural functor
$(\mathscr{E}t/X)\xrightarrow{\Phi}(\mathscr{E}t/X_{0})$ is an
equivalence.
\begin{itemize}
\item[(a)] If $Z$, $Z'\in(\mathscr{E}t/X)$ then
$$
\Hom_{X}(Z,Z')\xrightarrow{\sim}\Hom_{X_{0}}(\fprod{Z}{X_{0}}{X},\fprod{Z'}{X_{0}}{X}). 
$$
\end{itemize}

In\pageoriginale fact, let $\mathscr{A}(Z)$, $\mathscr{A}(Z')$ be the
coherent, locally free $\mathscr{O}_{X}$-Algebras defining $Z$, $Z'$
over $X$. If $\mathscr{M}$ is the maximal ideal of $A$, set
$A_{n}=A/\mathscr{M}^{n+1}$,
$X_{n}=\fprod{X}{A}{A}/\mathscr{M}^{n+1}$, $Z_{n}=\fprod{Z}{X_{n}}{X}$
and so on, $n\in\mathbb{Z}^{+}$. We then have:
$$
\Hom_{X}(Z,Z')\xrightarrow{\sim}\Hom_{\mathscr{O}_{X}-\text{Alg.}}(\mathscr{A}(Z'),\mathscr{A}(Z)). 
$$

Now for the $\mathscr{O}_{X}$-Modules, we have:
\begin{align*}
& \Hom_{\mathscr{O}_{X}}(\mathscr{A}(Z'),\mathscr{A}(Z))=\Gamma(X,\Hhom_{\mathscr{O}_{X}}(\mathscr{A}(Z'),\mathscr{A}(Z)))\\ 
&\xrightarrow{\sim}\varprojlim_{n}\Gamma(X_{n},\Hhom_{\mathscr{O}_{X}}
\foprod{(\mathscr{A}(Z'),\mathscr{A}(Z))}{A/\mathscr{M}^{n+1}}{A})
\end{align*}
by the Comparison theorem, (where $\Hhom_{\mathscr{M}_{X}}$ is the
sheaf of germs of $\mathscr{O}_{X}$-homomorphisms); since the
$\mathscr{A}(Z')$, $\mathscr{A}(Z)$ are coherent and locally free, we
have:
$$
\Hhom_{\mathscr{O}_{X}}(\mathscr{A}(Z'),\mathscr{A}(Z))\mathop{\otimes}_{A}A/\mathscr{M}^{n+1}=\Hhom_{\mathscr{O}_{X_{n}}}(\mathscr{A}(Z'_{n}),\mathscr{A}(Z_{n}));
$$
therefore,
$$
\Hom_{\mathscr{O}_{X}}(\mathscr{A}(Z'),\mathscr{A}(Z))\xrightarrow{\sim}\varprojlim_{n}\Hom_{\mathscr{O}_{X_{n}}}(\mathscr{A}(Z'_{n}),\mathscr{A}(Z_{n})).
$$
However,
this holds also for homomorphisms of $\mathscr{O}_{X}$-Algebras,
because the condition for an $\mathscr{O}_{X}$-Module homomorphism to
be an $\mathscr{O}_{X}$-Algebra homomorphism can be expressed by means
of commutativity in diagrams of $\mathscr{O}_{X}$-Modules. Therefore,
$\Hom_{X}(Z,Z')\xrightarrow{\sim}\varprojlim_{n}\Hom_{X_{n}}(Z_{n},Z'_{n})$. But
by Theorem \ref{chap7-sec7.2.1}, the natural map
$\Hom_{X_{n}}(Z_{n},Z'_{n})\to \Hom_{X_{0}}(Z_{0},Z'_{0})$
is\pageoriginale an isomorphism; therefore one obtains:
$$
\varprojlim_{n}\Hom_{X_{n}}(Z_{n},Z'_{n})\xrightarrow{\sim}\Hom_{X_{0}}(Z_{0},Z'_{0}). 
$$
\hfill Q.E.D.
\end{proof}

\begin{itemize}
\item[(b)] If $Z_{0}$ is an \'etale covering over $X_{0}$, then there
  exists an \'etale covering $Z/X$ such that
  $Z_{0}\xleftarrow{\sim}\fprod{Z}{X_{0}}{X}$. 
\end{itemize}

For proving this we need another powerfull theorem from the EGA.

\setcounter{subsection}{3}
\subsection{The Existence Theorem for proper
  morphisms}\label{chap8-sec8.1.4}

Let $A$ be a noe\-therian ring and $I$ and ideal of $A$ such that $A$ is
complete for the $I$-adic topology. Let $Y=\Spec A$ and $f:X\to Y$ be
a {\em proper} morphism. Set $A_{n}=A/I^{n+1}$, $n\in \mathbb{Z}^{+}$,
and $Y_{n}=\Spec A_{n}$, $X_{n}=\fprod{X}{Y_{n}}{Y}$. Suppose, for
every $n$, $\mathscr{F}_{n}$ is a coherent
$\mathscr{O}_{X_{n}}$-Module such that $\mathscr{F}_{n-1}\simeq
\displaystyle\mathscr{F}_{n}\mathop{\otimes}_{\mathscr{O}_{X_{n}}}\mathscr{O}_{X_{n-1}}$. Then
$\exists$ a coherent $\mathscr{O}_{X}$-Module $\mathscr{F}$ such that,
for each $n$,
$\mathscr{F}_{n}\xleftarrow{\sim}\foprod{\mathscr{F}}{\mathscr{O}_{X_{n}}}{\mathscr{O}_{X}}$. 

(For a proof see EGA, Ch. III, (5.1.4).)

Coming back to the proof of (b), the \'etale covering $Z_{0}\to X_{0}$
is defined by a coherent locally free $\mathscr{O}_{X_{0}}$-Algebra
$\mathscr{B}_{0}$. By Theorem \ref{chap7-sec7.2.1}, for each $n$, we get a
coherent locally free $\mathscr{O}_{X_{n}}$-Algebra such that
$\mathscr{B}_{n-1}\xleftarrow{\sim}
\foprod{\mathscr{B}_{n}}{\mathscr{O}_{X_{n-1}}}{\mathscr{O}_{X_{n}}}$. By
the existence theorem, $\exists$ a coherent $\mathscr{O}_{X}$-Algebra
$\mathscr{B}$ such that
$\mathscr{B}_{n}\xleftarrow{\sim}\foprod{\mathscr{B}}{\mathscr{O}_{X_{n}}}{\mathscr{O}_{X}}$. Set\pageoriginale
$Z=\Spec\mathscr{B}$. We claim that $Z/X$ is the \'etale covering we
are looking for. It is clear that
$\fprod{Z}{X_{0}}{X}\xrightarrow{\sim}Z_{0}$. It remains to show that
$Z/X$ is \'etale. 

We first observe that since $A$ is a local ring and $f$ is closed any
(open) neighbourhood of the fibre $f^{-1}(s_{0})$ is the whole of
$X$. Also, if $Z/X$ is \'etale over the points of the fibre
$X_{0}=f^{-1}(s_{0})$, then $Z$ is \'etale over $X$ at points of an
open neighbourhood of $X_{0}$; for, if $x_{0}\in X_{0}$ then
$\mathscr{B}_{x_{0}}$ is free over $\mathscr{O}_{x_{0},X}$ and hence
$\mathscr{B}$ is a free $\mathscr{O}_{X}$-Module in a neighbourhood of
$x_{0}$; and similarly for non-ramification (use Ch. \ref{chap3}, Proposition
\ref{chap3-prop3.3.2}). Therefore it is enough to prove that $Z\to X$ is
\'etale at points of $Z$ lying over the fibre $X_{0}=f^{-1}(s_{0})$.

Let then $x_{0}\in X_{0}$; choose an affine open neighbourhood $U$ of
$x_{0}$ in $X$ such that $\mathscr{B}_{0}$ is
$\mathscr{O}_{X_{0}}$-free over $U_{0}=U\cap X_{0}$. Set
$\Gamma(U,\mathscr{O}_{X})=C$ and $J=\mathscr{M}C$, the ideal
generated by $\mathscr{M}$; then, for $n\in\mathbb{Z}^{+}$,
$C_{n}=C/J^{n+1}\xrightarrow{\sim}\Gamma(U,\mathscr{O}_{X_{n}})$. Similarly,
if $M=\Gamma(U,\mathscr{B})$ then
$$
M_{n}=M/J^{n+1}M\xrightarrow{\sim}\Gamma(U,\mathscr{B}_{n}). 
$$

We know that
$M_{0}=\Gamma(U,\mathscr{B}_{0})\xrightarrow{\sim}\Gamma(U_{0},\mathscr{B}_{0})$
is free as a $C_{0}=\Gamma(U_{0},\mathscr{O}_{X_{0}})$-module. Choose
a basis for $M_{0}$ over $C_{0}$. Lift this basis to $M$ and let $L$
be the free $C$-module on these elements and $\varphi:L\to M$ be the
natural $C$-linear map; then we have an exact sequence of $C$-modules:
$$
0\to \ker\varphi\to L\to M\to \coker\varphi\to 0.
$$

For\pageoriginale $n\in\mathbb{Z}^{+}$, set $L_{n}=L/J^{n+1}L$ and let
$\varphi_{n}$ be the $C_{n}$-linear map
$L_{n}\xrightarrow{\varphi_{n}}M_{n}$ defined by $\varphi$. We claim
that each $\varphi_{n}$ is an isomorphism.

That $\varphi_{0}$ is an isomorphism is clear from the
construction. Assume that $\varphi_{0},\ldots,\varphi_{n-1}$ are all
isomorphisms and consider the exact sequence
$$
0\to {J^{n}}/J^{n+1}\to C_{n}\to C_{n-1}\to 0
$$
of $C_{n}$-modules. Since $L_{n}$, $M_{n}$ are $C_{n}$-flat (recall
that $\mathscr{B}_{n}$ is $\mathscr{O}_{X_{n}}$-flat) we get the
following commutative diagram of exact sequences of $C_{n}$-modules:
\[
\xymatrix@=1.2cm{
0\ar[r] & \foprod{J^{n}/J^{n+1}}{M_{n}}{C_{n}}\ar[r] & M_{n}\ar[r] &
M_{n-1}\ar[r] & 0\\
0\ar[r] & \foprod{J^{n}/J^{n+1}}{L_{n}}{C_{n}}\ar[r]\ar[u] &
L_{n}\ar[r]\ar[u]_{\varphi_{n}} & L_{n-1}\ar[r]\ar[u]_{\varphi_{n-1}} & 0.
}
\]

But
\begin{align*}
& J^{n}/J^{n+1}\simeq
  \frac{\mathscr{M}^{n}}{\mathscr{M}^{n+1}}\mathop{\otimes}_{A}C\cong
  \bigoplus^{r}_{i=1}\Gamma(U,\mathscr{O}_{X_{0}}),\\[4pt]
& J^{n}/J^{n+1}\mathop{\otimes}_{C_{n}}M_{n}\simeq
  \bigoplus^{r}_{i=1}\Gamma(U,\mathscr{B}_{0})=\bigoplus^{r}_{i=1}M_{0},\\[4pt]
\text{and}\quad & J^{n}/J^{n+1}\mathop{\otimes}_{C_{n}}L_{n}\simeq
\bigoplus^{r}_{i=1}L_{0}.
\end{align*}

It\pageoriginale follows that the first vertical map is an
isomorphism; so is $\varphi_{n-1}$ by inductive assumption. Hence
$\varphi_{n}$ is an isomorphism.

If $\widehat{C}$, $\widehat{L}$, $\widehat{M}$ are the $J$-adic
completions of $C$, $L$, $M$ and if $\widehat{\varphi}$ is the
$\widehat{C}$-linear map defined by $\varphi$, it follows that
$\widehat{\varphi}$ is an isomorphism. Thus, one obtains: (coker
$\varphi$)$\sphat=0$ and $\ker\varphi\subset \ker(L\to
\widehat{L})$. It follows then that $\mathscr{B}$ is locally free at
$x_{0}\in X_{0}$, i.e., $Z\to X$ is flat at points over $X_{0}$. It is
also unramified at points above $X_{0}$ since $Z_{0}\to X_{0}$ is
unramified and for non-ramification it suffices to look at the fibre.

The proof of Proposition \ref{chap8-prop8.1.3} is complete.

\medskip
\noindent
{\bf Proof of Theorem \ref{chap8-thm8.1.1}.}~The isomorphism
$\pi_{1}(S,s_{0})\xleftarrow{\sim}G(\overline{k(s_{0})}/k(s_{0}))$
follows from Proposition \ref{chap8-prop8.1.3} when $X=S$.

Consider now the diagram
\[
\xymatrix@=1.2cm{
a_{0}\in X\ar[d] & X_{0}\ni a'_{0}\ar[l]\ar[d] & \overline{X}_{0}\ni
\overline{a}_{0}\ar[l]\ar[d]\\
s_{0}\in S & k(s_{0})\ar[l] & \overline{k(s_{0})}\ar[l] 
}
\]

By Proposition \ref{chap8-prop8.1.2} we get the exact sequence:
$$
(e)\to \pi_{1}(\overline{X}_{0},\overline{a}_{0})\to
\pi_{1}(X_{0},a'_{0})\to G(\overline{k(s_{0})}/k(s_{0}))\to (e).
$$

One now uses the isomorphisms
$\pi_{1}(X_{0},a'_{0})\xrightarrow{\sim}\pi_{1}(X,a_{0})$ (Proposition
\ref{chap8-prop8.1.3}) and
$G(\overline{k(s_{0})}/k(s_{0}))\xrightarrow{\sim}\pi_{1}(S,s_{0})$ to
get he required exact sequence.\hfill Q.E.D.
