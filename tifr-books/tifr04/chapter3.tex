\chapter{Lecture 3}

\section*{``The Tensor Bundles''}\pageoriginale

We have different kinds of tensor's attached at each point of a
$C^{\infty}$ manifold $V^{N}$. We shall organize the system of tensors
of a specified type into a new manifold.

We first consider the set of all the tangent vectors at all points of
$V$. We denote this set by $T(V)$. We define on $T(V)$ first a
topological structure and then a differentiable structure so that
$T(V)$ becomes a $2N$-dimensional $C^{\infty}$-manifold. We do this by
means of the fundamental system of maps defining the manifold
structure on $V$. We choose a map $(U,\varphi)$. Let
$x_{1},\ldots,x_{N}$ be the coordinate functions corresponding to this
map. Let $a\in U$. In terms of the canonical basis
$\left\{\dfrac{\partial}{\partial x_{i}}\right\}$ a tangent vector $L$
at $\ub{a}$ has the representation
$$
L=\sum \xi_{j}\left(\frac{\partial}{\partial x_{j}}\right)_{a}
$$
Let $\varphi(a)=(a_{1},\ldots,a_{N})\in R^{N}$. We can represent $L$
by the point ($a_{1}$, $\ldots$, $a_{N},\xi_{1},\ldots,\xi_{N})$ in
$R^{2N}$. The set of all tangent vectors at points $a\in U$ are in
$(1,1)$ correspondence with the space $\varphi(U)\times R^{N}$. We
carry over the topology in $\varphi(U)\times R^{N}$ to this set (by
requiring the map $L\to (a_{1},\ldots,a_{N},\xi_{1},\ldots,\xi_{N})$
to be a homeomorphism). This we do for every map $(U,\varphi)$. We
have now to verify that the topology we introduced is consistent on
the overlaps.

Let $(W,\psi)$ be another map and $y_{1},\ldots,y_{N}$ the
corresponding coordinate functions. Let $a\in U\cap W$ and
$\psi(a)=(b_{1},\ldots,b_{N})$. Let\pageoriginale the components of
$L$ with respect to the canonical basis corresponding to the map
$(W,\psi)$ be $(\eta_{1},\ldots,\eta_{N})$. $b_{1},\ldots,b_{N}$ are
$C^{\infty}$ functions of $(a_{1},\ldots,a_{N})$. Further
\begin{align*}
L &= \sum\eta_{i}(\frac{\partial}{\partial y_{i}})_a\\
 &= \sum\xi_{j}\left(\frac{\partial}{\partial x_{j}}\right)_{a}\\
&= \sum_{j}\xi_{j}\sum_{i}\left(\frac{\partial y_{i}}{\partial
  x_{j}}\right)_{a}\left(\frac{\partial}{\partial y_{i}}\right)_{a} 
\end{align*}
so that
$$
\eta_{i}=\sum_{j}\xi_{j}\left(\frac{\partial y_{i}}{\partial x_{j}}\right)_{a}
$$
So the $\eta$ are $C^{\infty}$ functions of
$(a_{1},\ldots,a_{N},\xi_{1},\ldots,\xi_{N})$. Hence the map
$$
(a_{1},\ldots,a_{N},\xi_{1},\ldots,\xi_{N})\to
(b_{1},\ldots,b_{N},\eta_{1},\ldots,\eta_{N}) 
$$
is a $C^{\infty}$ map; likewise the map
$$
(b_{1},\ldots,b_{N},\eta_{1},\ldots,\eta_{N})\to
(a_{1},\ldots,a_{N},\xi_{1},\ldots,\xi_{N}) 
$$
is a $C^{\infty}$ map. In particular the maps are continuous and this
proves that the topology we defined is consistent on the
overlaps. Now the above reasoning shows also that we have in fact
defined a $C^{\infty}$ structure on $T(V)$. Of course the atlas we
have given is not complete. This is a very special atlas; in fact any
map of $V$ gives rise to a map in this atlas.

If we have a $C^{k}$-manifold $V$ we can put on $T(V)$ only a $(k-1)$
times differentiable structure; for, the expressions for the
$\eta_{i}$ in terms of the $\xi_{i}$ involve the first partial
derivatives which are only\pageoriginale $(k-1)$ times
differentiable. In particular if $V$ is a $C^1$-manifold $T(V)$ will be
only a topological manifold.

The set of all $p$-times contra variant and $q$-times covariant
tensors on a $C^{\infty}$ manifold $V$ can be made similarly into a
$C^{\infty}$ manifold
$$
\left(\bigotimes^{p}T(V)\right)\otimes
\left(\bigotimes^{q}T^{\ast}(V)\right)
$$
We also have the manifold of $p$-covectors
$\overset{p}{\Lambda}T^{\ast}(V)$ which is of dimension
$N+\binom{N}{P}$. The set of all tangent covectors, not-necessarily
homogeneous, (\iec the union of $\Lambda T^{\ast}_{a}(V)$, $a\in V$)
can also be endowed with a $C^{\infty}$ structure: this manifold,
$\Lambda T^{\ast}(V)$, is of dimension $N+2^{N}$.

All these manifolds built from $V$ have quite a special structure;
they have the structure of a vector fibre bundle. We shall not define
here precisely the concept of a fibre bundle.

Let us consider, for example, $T(V)$. If $a\in V$, $T_{a}(V)$ is
called the fibre over the point $\ub{a}$. $T(V)$ is the collection of
the fibres, each fibre being a vector space of dimension $N$. Actually
$T(V)$ is partitioned into the fibres. $T(V)$ is called the tangent
bundle of $V$. $T(V)$ is the bundle space and $V$ the base
space. There are two important maps associated with $T(V)$. One is the
canonical projection $\pi:T(V)\to V$ which associates to every tangent
vector its origin: if $L\in T_{a}(V)$, $\pi(L)=a$. (This map is the
projection of the bundle onto the base). The other map gives the
canonical imbedding of $V$ in $T(V)$; this map $V\to T(V)$ assigns to
every point $\ub{a}\in V$ the zero tangent vector at $\ub{a}$. We can
now consider $V$ as a submanifold of $T(V)$. (This\pageoriginale
canonical imbedding 
is possible in any fibre bundle in which the fibre is a vector space).

A section of a fibre bundle is a map which associates to every point
$\ub{a}$ in the base space a point in the fibre over $\ub{a}$. A
section of $T(V)$ is a vector field. A section of
$\overset{p}{\Lambda}T^{\ast}(V)$ is a differential form of degree
$p$. A tensor field of a definite type is a section of the
corresponding tensor bundle.

\section*{$C^{\infty}$ Tensor fields and $C^{\infty}$ Differential
  forms}

A $C^{\infty}$ vector field is an indefinitely differentiable section
of $T(V)$ \ie it is a $C^{\infty}\map \theta:V\to T(V)$ such that for
$a\in V$, $\theta(a)\in T_{a}(V)$. A $C^{\infty}$ differential form is
a $C^{\infty}\map \omega:V\to \Lambda T^{\ast}(V)$ such that
$\omega(a)\in \Lambda T^{\ast}_{a}(V)$ for $a\in V$. A $C^{\infty}$
tensor field is defined similarly.

It is good to come back to the coordinate system and see how a
$C^{\infty}$ vector field looks. Let $\theta$ be a $C^{\infty}$ vector
field. Let $(U,\varphi)$ be a map of $V$. Let $a\in U$ and
$\varphi(a)=(a_{1},\ldots,a_{N})\in R^{N}$. Let the components of
$\theta(a)$ in terms of the canonical basis (with respect to
$(U,\varphi)$) at $\ub{a}$ be $\xi_{1},\ldots,\xi_{N}$.
$$
\xi_{i}=\xi_{i}(a_{1},\ldots,a_{N})
$$
are functions of $(a_{1},\ldots,a_{N})$. Thus exactly as in $R^{N}$
the vector field is given by $N$-functions of the coordinates of the
origin of the vector. If $\theta$ is a $C^{\infty}$ vector field the
map
$$
(a_{1},\ldots,a_{N})\to (a_{1},\ldots,a_{N},\xi_{1},\ldots,\xi_{N})
$$
is\pageoriginale a $C^{\infty}$ map and hence the $\xi_{i}$'s are $C^{\infty}$
functions of $(a_{1},\ldots,a_{N})$. Conversely if for every choice of
the map $(U,\varphi)$ the $\xi_{i}$'s are $C^{\infty}$ functions of
$(a_{1},\ldots,a_{N})$, $\theta$ is a $C^{\infty}$ vector field, since
the $a_{i}$'s are always $C^{\infty}$ functions of
$(a_{1},\ldots,a_{N})$.

It will be useful in particular to know how to recognize by means of
the coordinate systems whether a given differential form is
$C^{\infty}$. Let us consider, for simplicity, a differential form of
degree two, $\overset{2}{\omega}$. Let $(U,\varphi)$ be a map and
$x_{1},\ldots,x_{N}$ the corresponding coordinate functions. If $a\in
U$, $(dx_{1})_{a},\ldots,(dx_{N})_{a}$ is the canonical basis for
$T^{\ast}_{a}(V)$. Once we know the canonical basis for
$T^{\ast}_{a}(V)$ we also know the canonical basis for its second
exterior power $\overset{2}{\Lambda}T^{\ast}_{a}(V)$. The canonical
basis for $\overset{2}{\Lambda}T^{\ast}_{a}(V)$ is 
$$
\left\{(dx_{i})_{a}\Lambda (dx_{j})_{a}\right\},i<j.
$$
In terms of this basis we write
$$
\overset{2}{\omega}(a)=\sum_{i<j}\omega_{ij}(a)\,(dx_{i})_{a}\Lambda
(dx_{j})_{a}. 
$$
$\omega_{ij}(a)$ are functions of $\ub{a}\cdot \overset{2}{\omega}$ is
a $C^{\infty}$ differential form if and only if for every choice of
the map $(U,\varphi)$ the functions $\omega_{ij}$ are $C^{\infty}$
functions.

In terms of the canonical basis $(dx_{i})_{a}$, we can write $\omega$
as follows:
$$
\omega=\sum_{i<j}\omega_{ij} \, dx_{i}\Lambda dx_{j}
$$
This is not `abuse of language'; the expression has a correct
meaning. $\omega_{ij}$ is a function \iec a differential form of
degree zero. $dx_{i}$ is the differential of the function $x_{i}$ \iec
the differential form of degree\pageoriginale one which assigns to the
point $\ub{a}$ the differential $(dx_{i})_{a}$ and we have already
defined the exterior multiplication of two differential forms.

It is quite remarkable that we can define a manifold structure on the
set of all tangent vectors; a priori there was no relation between
tangent spaces defined abstractly. Also the notion of a vector varying
continuously in a vector space which itself varies is a priori
extraordinary.

\section*{$C^{\infty}$ Differential forms}

By a form we shall always mean a $C^{\infty}$ differential form. When
we consider $C^{k}$ differential forms we will state it explicitly. We
recall some fundamental properties of the forms: Let
$\overset{p}{\omega}$ be a $p$-form. When we have a map we have a
representation of $\overset{p}{\omega}$ in terms of the canonical
basis:
$$
\overset{p}{\omega}=\sum_{j_{1}<\ldots<j_{p}}\omega_{j_{1}\ldots
j_{p}}dx_{j1}\Lambda\ldots\Lambda dx_{j_{p}}.
$$
This is only a local representation; we do not, in general, have a
global representation. The following are the principal properties of
the forms.

\begin{enumerate}
\item {\bf Differentiability.}

\item {\bf The linear structure:} We can add two $p$-forms and
  multiply a form by a scalar. All $p$-forms form a vector space.

\item {\bf Algebra structure:} We can multiply two forms $\omega$ and
  $\ob{\omega}$ and obtain the form $\omega\Lambda\ob{\omega}$. The
  multiplication satisfies the anti-commuta\-ti\-vity rule.

\item {\bf The reciprocal image of a form:}\pageoriginale

This is a very important notion. Let $U$ and $V$ be two $C^{\infty}$
manifolds, and $\Phi:U\to V$ a $C^{\infty}$ map. Suppose
$\overset{p}{\omega}$ is a differential form of degree $p$ on
$V$. $\overset{p}{\omega}$ gives rise to a differential form of degree
$p$ on $U$, $\Phi^{-1}(\overset{p}{\omega})$, which we call the
reciprocal image of $\omega$ by $\Phi$. Let $\ub{a}\in U$ and
$\Phi(a)=b$. The value of $\omega$ at $b$ is a $p$-covector,
$\omega(b)$, at $b$. $\Phi^{-1}(\overset{p}{\omega})$ is the
differential form which assigns to every point $a\in U$ the
$p$-covector $\Phi^{-1}(\overset{p}{\omega}(b))$. It is seen easily
that if $\omega$ is a $C^{\infty}$ form $\Phi^{-1}(\omega)$ is also a
$C^{\infty}$ form.


Any kind of covariant tensor field has a reciprocal image. However it
is in general impossible to define the direct image of contravariant
tensor field. For it may be that a point $b\in V$ is the image of no
point of $U$ or the image of an infinity of points of $U$. Of course
we can define the direct image of a contravariant tensor field when
$\Phi$ is a diffeomorphism.

One of the reasons for the utility of differential forms is that they
have a reciprocal image. Another is the possibility of exterior
differentiation. 

\item {\bf Exterior differentiation:}

To a given $C^{k}$ differential form $\overset{p}{\omega}$ of degree
$p$ we associate a differential form $d\overset{p}{\omega}$ of degree
$p+1$ which is of class $C^{k-1}$; $d\overset{p}{\omega}$ is called
the exterior differential or the coboundary of
$\overset{p}{\omega}$.\pageoriginale The important point to be noticed
is that the exterior differential of a differential form of degree $p$
is of degree $p+1$.
\end{enumerate}

We define the exterior differentiation by axiomatic properties. We
restrict our attention to $C^{\infty}$ forms only. Let
$\overset{p}{\mathscr{E}}$ denote the space of all $C^{\infty}p$-forms
on the manifold (if $p\neq
0,1,\ldots,N,\overset{p}{\mathscr{E}}=0$). Then
$d:\overset{p}{\mathscr{E}}\to
\overset{p+1}{\mathscr{E}}(p=0,1,\ldots,N)$ is a map which satisfies
the following properties:
\begin{enumerate}
\renewcommand{\labelenumi}{\theenumi)}
\item The operation $d$ is purely local: if two forms $\omega$ and
  $\ob{\omega}$ coincide on an open set $U$, then
  $d\omega= -d\ob{\omega}$ on $U$.

\item $d$ is a linear operation:
\begin{align*}
& d(\omega+\ob{\omega})=d\omega+d\ob{\omega}\\
& d(\lambda\omega)=\lambda d\omega,\lambda \text{ a constant.}
\end{align*}

\item With respect to the structure of algebra $d$ has the following
  property:
$$
d(\overset{p}{\omega}\wedge
\overset{\ub{q}}{\omega})=d\overset{p}{\omega}\wedge
\overset{\ub{q}}{\omega}+(-1)^{p}\overset{p}{\omega}\wedge
d\overset{\ub{q}}{\omega} 
$$
(The sign $(-1)^{p}$ in the second term is to be expected as no symbol
can pass over a $p$-form without taking the sign $(-1)^{p}$).

\item $d^{2}=0$; \iec $d(d\omega)=0$.

\item If $f$ is a form of degree zero \iec a scalar function, then
  $df$ is the ordinary differential of the function which associates
  to every point $\ub{a}$ the differential of $f$ at $\ub{a}$.

\end{enumerate}


