\chapter{Lecture 12}

\section*{The operator $\Delta$ on functions}\pageoriginale

Let $x_{1},\ldots,x_{N}$ be a local coordinate system such that
$dx_{1}\wedge\ldots\wedge dx_{N}>0$; let
$g_{ij}=\left(\dfrac{\partial}{\partial
  x_{i}},\dfrac{\partial}{\partial x_{j}}\right)$ and $(g^{ij})$ the
matrix inverse to the matrix $(g_{ij})$. Let $\sum \omega_{j}dx_{j}$
be a $1$-form. We calculate
$\partial(\sum\limits_{j}\omega_{j}dx_{j})$
$$
\partial\left(\sum_{j}\omega_{j}dx_{j}\right)=
-\overset{-1}{\ast}d\ast \left(\sum_{j}\omega_{j}dx_{j}\right)  
$$
Now $\ast\omega_{j}dx_{j}=\omega_{j}\ast dx_{j}$. Suppose
$$
\ast dx_{j}=\sum_{i}\widetilde{\omega}_{ij}dx_{1}\wedge\ldots\wedge
d/x_{i}\wedge\ldots\wedge dx_{N};
$$
from the relation
$$
(dx_{k},dx_{j})\tau=dx_{k}\wedge\ast dx_{j}
$$
we obtain
$$
g^{kj}\sqrt{g}dx_{1}\wedge\ldots\wedge
dx_{N}=(-1)^{k-1}\widetilde{\omega}_{kj}dx_{1}\wedge\ldots\wedge
dx_{N}
$$
so that $\widetilde{\omega}_{kj}=(-1)^{k-1}g^{kj}\sqrt{g}$.
\begin{align*}
d\left(\ast \left(\sum_{j}\omega_{j}dx_{j}\right)\right) &=
d \left(\sum_{i,j}(-1)^{i-1}\omega_{j}g^{ij}\sqrt{g}dx_{1}\wedge\ldots\wedge
d/x_{i}\wedge\ldots\wedge dx_{N}\right)\\
&= \sum_{i,j}(-1)^{i-1}\frac{\partial}{\partial
  x_{i}}(\omega_{j}g^{ij}\sqrt{g})dx_{i}\\
&\wedge
dx_{1}\wedge\ldots\wedge d/x_{i}\wedge\ldots\wedge dx_{N}\\
&= \sum_{i,j}\frac{\partial}{\partial
  x_{i}}(\omega_{j}g^{ij}\sqrt{g})dx_{1}\wedge\ldots\wedge dx_{N}\\
&= \frac{1}{\sqrt{g}}\sum_{i,j}\frac{\partial}{\partial
  x_{i}}(\omega_{j}g^{ij}\sqrt{g})\tau 
\end{align*}
For\pageoriginale $N$ forms $\overset{-1}{\ast}=\ast$. Hence
\begin{align*}
\overset{-1}{\ast}d\ast \left(\sum_{j}\omega_{j}dx_{j}\right) &=
\frac{1}{\sqrt{g}}\sum_{i,j}\frac{\partial}{\partial
  x_{i}}(\omega_{j}g^{ij}\sqrt{g})\tau\\
&= \frac{1}{\sqrt{g}}\sum_{i,j}\frac{\partial}{\partial
  x_{i}}(\omega_{j}g^{ij}\sqrt{g}) 
\end{align*}
Finally
$$
\partial
\left(\sum_{j}\omega_{j}dx_{j}\right)=-\frac{1}{\sqrt{g}}\sum_{i,j}\frac{\partial}{\partial
  x_{i}}(\omega_{j}g^{ij}\sqrt{g}) 
$$

We shall now calculate $\Delta u$ where $u$ is a $0$-form. We have
$$
\partial u=0, \Delta u=-\partial du\quad\text{and}\quad
du=\sum\dfrac{\partial u}{\partial x_{j}}dx_{j}.
$$
By the calculation made above we find that
$$
\Delta u=\sum_{i,j} \frac{1}{\sqrt{g}}\frac{\partial}{\partial
  x_{i}}\left[g^{ij}\sqrt{g}\frac{\partial u}{\partial x_{j}}\right].
$$
This is the well-known Laplace operator on functions. In particular if
$V^{N}=R^{N}$ with the natural metric and the natural orientation, the
matrix $(g_{ij})$ is the unit matrix and
$$
\Delta u=\sum_{i}\frac{\partial^{2}u}{\partial x^{2}_{i}}.
$$

\section*{The elliptic character of $\Delta$. Harmonic forms}

In the case of a partial differential equation of the second order an
elliptic operator is usually defined by considering the nature of the
quadratic form given by the coefficients of the derivatives of the
second order. However it is found more convenient to define an
elliptic operator by intrinsic properties of the operator.
This\pageoriginale definition is valid for systems of differential
equations and also for differential equations of higher order.

A local or differential operator $D$ is defined to be a linear
continuous operator on currents $(D:\mathscr{D}'\to \mathscr{D}')$
taking forms into forms\footnote{Actually this condition is
  superfluous; it can be proved that it is a consequence of the other
  conditions.} and having the local character: $DT$ ($T$ a current) is
known in an open set $\Omega$ if $T$ is known in $\Omega$. A
differential operator $D$ is called an elliptic operator\footnote{In
  current literature such differential operators are referred to as
  hypoelliptic operators.} if
the following condition is satisfied: If $T$ is a current such that
$DT=\alpha$ is a $C^{\infty}$ form, in an open set $\Omega$ then $T$
itself is a $C^{\infty}$ form in $\Omega$. If $D$ is elliptic every
solution of the homogeneous equation $DT=0$ is a $C^{\infty}$
form. The operator
$$
D=\left(\frac{d}{dx}\right)^{m}+a_{1}\left(\frac{d}{dx}\right)^{m-1} +
\cdots+a_{m},a_{i}\in\mathscr{E} 
$$
on the distributions in $R$ is elliptic. In $R^{2}$ the operator
$\dfrac{\partial^{2}}{\partial x^{2}}+\dfrac{\partial^{2}}{\partial
  y^{2}}$ is elliptic while the wave operator
$\dfrac{\partial^{2}}{\partial x^{2}}-\dfrac{\partial^{2}}{\partial
  y^{2}}$ is not elliptic. The function $\psi(x,y)=f(x+y)+g(x-y)$
where $f$ and $g$ are continuous is a solution of the wave equation
(as a distribution), even though the functions may not be
differentiable.

We shall admit without proof the important theorem which states that
the operator $\Delta$ is elliptic.

From the elliptic character of $\Delta$ we can deduce that $d$ is
elliptic\pageoriginale on $\overset{0}{\mathscr{D}'}\partial =0$ on
$\overset{0}{\mathscr{D}'}$. So $-\Delta=\partial d$ on
$\overset{0}{\mathscr{D}'}$. Now if $D_{1}$ and $D_{2}$ are
differential operators and $D_{1}D_{2}$ is elliptic then $D_{2}$ is an
elliptic operator (but not necessarily $D_{1}$). For let
$D_{2}T=\alpha$ be a $C^{\infty}$ form; since
$D_{1}D_{2}T=D_{1}\alpha$, $D_{1}\alpha$ is a $C^{\infty}$ form and
$D_{1}D_{2}$ is elliptic it follows that $T$ is a $C^{\infty}$
form. Since $\Delta=-d\partial$ is elliptic, $d$ is elliptic on
$\overset{0}{\mathscr{D}'}$. 

A form $\omega$ which satisfies the equation $\Delta\omega=0$ is
called a harmonic form.

\section*{Compact Riemannian manifolds}

We shall assume henceforth that $V$ is a compact, oriented Riemannian
manifold.

If $V$ is compact every harmonic form is closed and $\ast$
closed. (This result is false when $V$ is not compact; for example, a
closed $0$-form in $R^{N}$ is a constant while there exist
non-constant harmonic functions). For let $\omega$ be a harmonic form
\begin{align*}
(\Delta \omega,\omega) &= (d\partial \omega,\omega)+(\partial d\omega,
  \omega)\\
&= (\partial \omega,\partial \omega)+(d\omega,d\omega).
\end{align*}
Since $(\partial \omega,\partial\omega)\geq 0$, $(d\omega,d\omega)\geq
0$ and $(\Delta\omega,\omega)=0$ it follows that $(\partial\omega,
\partial\omega)=0$ and $(d\omega,d\omega)=0$. But if $f$ is a
continuous non-negative function such that $\int\limits_{V}f\tau=0$
then $f\equiv 0$. Since the Riemannian scalar product is positive
definite it follows that 
$$
d\omega=0\quad\text{and}\quad \partial\omega=0.
$$

\section*{The Hilbert space of square summable forms}\pageoriginale

We now define the Hilbert space $\mathscr{H}^{p}$ of square summable
differential forms of degree $p$.

A form $\omega$ is said to be measurable if its coefficients are
measurable on every map.

An element of $\mathscr{H}$ is a class, a class consisting of all
measurable forms which are equal almost every-where to a form $\omega$
for which $(\omega,\omega)\break =\int\limits_{V}(\omega,\omega)_{a}\tau$ is
finite. If $\omega$ and $\widetilde{\omega}$ are elements of
$\mathscr{H}$ then $(\omega,\widetilde{\omega})$ is defined and this
defines a positive definite scalar product in $\mathscr{H}$. We can
prove that $\mathscr{H}$ is complete with respect to the norm given by
the scalar product. So $\mathscr{H}$ is a Hilbert space.

