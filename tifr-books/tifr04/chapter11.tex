\chapter{Lecture 11}

\section*{The star operator}\pageoriginale

The star operator is defined for a Euclidean {\em oriented} vector
space; this operator associates to every $p$-vector an $(N-p)$ vector.

Let us consider $T^{\ast}_{a}(V)$ with the canonical Euclidean
structure given by the Riemannian structure on $V$. Let
$T^{\ast}_{a}(V)$ be oriented. We first define the star $(\ast)$
operator on $0$ - vectors \iec
scalars. $\overset{N}{\Lambda}T^{\ast}_{a}(V)$ is a one dimensional
space in which the class of positive vectors has been
chosen. $\overset{N}{\Lambda}T^{\ast}_{a}(V)$ has a canonical
Euclidean structure. Let $\tau\in \overset{N}{\Lambda}T^{\ast}_{a}(V)$
be the unique positive vector of length $1$. We define $\ast
1=\tau$. For $\overset{p}{\beta}\in\overset{p}{\Lambda}T^{\ast}_{a}$
we define $\ast\beta$ as the $(N-p)$ vector which satisfies the
relation
$$
(\overset{p}{\alpha},\overset{p}{\beta})=\overset{p}{\alpha}\wedge
(\ast\beta^{p}). 
$$
for every $\overset{p}{\alpha}\in
\overset{p}{\Lambda}T^{\ast}_{a}$. There exists one and only one
element with this property. We choose an orthonormal basis
$e_{1},\ldots,e_{N}$ in $T^{\ast}_{a}$ such that
$e_{1}\wedge\ldots\wedge e_{N}>0$ and define $\ast$ on the basis
elements $e_{i_{1}}\wedge\ldots\wedge e_{i_{p}}(i_{1}<\ldots<i_{p})$
of $\overset{p}{\Lambda}T^{\ast}_{a}$ by:
$$
\ast(e_{i_{1}}\wedge\ldots\wedge e_{i_{p}})=\epsilon
e_{k_{1}}\wedge\ldots\wedge e_{k_{N-p}}
$$
where $k_{1},\ldots,k_{N-p}$ are the indices complementary to
$i_{1},\ldots,i_{p}$ and $\epsilon$ is the sign of the permutation
$$
\begin{pmatrix}
1 & 2\ldots\ldots\ldots N\\
i_{1} & i_{2}\ldots i_{p}k_{1}\ldots k_{n-p}
\end{pmatrix}
$$
we\pageoriginale define $\ast$ on the whole of $\overset{p}{\Lambda}
T^{\ast}_{a}$ by linearity. It is immediate that this is the only
operation having the property
$$
(\overset{p}{\alpha},\overset{p}{\beta})=\overset{p}{\alpha}\wedge
(\ast\overset{p}{\beta}) 
$$
We write $\ast^{-1}\beta =(-1)^{p(N-p)}\ast\beta$. It is easily
verified that
$$
\ast\ast \overset{p}{\beta}=(-1)^{p(N-p)}\beta.
$$

The $\ast$ operator gives an isomorphism of
$\overset{p}{\Lambda}T^{\ast}_{a}$ onto
$\overset{N-p}{\Lambda}T^{\ast}_{a}$. This isomorphism carries an
orthonormal basis into an orthonormal basis and hence preserves scalar
products. 

If $A$ and $B$ are two vectors in $R^{3}$ with the natural
orientation, $\ast(A\wedge B)$ ($\ast$ operation with respect to the
natural Riemannian structure in $R^{3}$) is what is usually called the
vector product of $A$ and $B$. In $R^{2}$ the star operation for
vector is essentially rotation through an angle $\pi/2$.

\section*{The star operator on differential forms}

We suppose that $V^{N}$ is a oriented Riemannian manifold. The $\ast$
operation is then defined on each
$\overset{p}{\Lambda}T^{\ast}_{a}(V)$.

Suppose now that $\omega$ is a differential form of degree $p$. By
taking at every point $\ub{a}$ the $N-p$ covector $\ast\omega(\alpha)$
we get a differential form of degree $N-p$, which we denote by
$\ast\omega$. If $\omega$ is a $C^{\infty}$ $p$-form $\ast\omega$ is a
$C^{\infty}(N-p)$ form. In particular we have the $N$-form $\ast
1$. This $N$ form, denoted by $\tau$, defines the volume element on
the Riemannian manifold. If $x_{1},\ldots,x_{N}$ is a local
coordinate\pageoriginale system for which $dx_{1}\wedge\ldots\wedge
dx_{N}>0$ then 
$$
\tau=\sqrt{g}\cdot dx_{1}\wedge\ldots\wedge dx_{N},
$$
where $\sqrt{g}$ is the positive square root of the determinant $g$ of
the matrix $(g_{ij})$.

The star operator on differential forms, defined above, gives an
isomorphism, called the star isomorphism, between
$\overset{p}{\mathscr{E}}$ and $\overset{N-p}{\mathscr{E}}$ and also
between $\overset{p}{\mathscr{D}}$ and $\overset{N-p}{\mathscr{D}}$. 

\section*{The global scalar product of two $C^{\infty}$-forms}

We define the scalar product of two $p$-forms $\alpha$ and $\beta$
when the support of either $\alpha$ or $\beta$ is compact by:
$$
(\alpha,\beta)=\int\limits_{V}(\alpha,\beta)_{a}\tau
$$
($(\alpha,\beta)_{a}$ is the scalar product of the $p$-covectors
$\alpha(a)$ and $\beta(a)$ at $a$; $\tau$ is the volume element of the
Riemannian manifold). We have
\begin{align*}
(\alpha,\beta) &= \int\limits_{V}\alpha\wedge\ast \beta\\
 &= \langle \alpha,\ast\beta\rangle\\
 &= \langle\ast^{-1}\alpha,\beta\rangle
\end{align*}

\section*{The $\ast$ operator on currents}

Let $T$ be a current. We define $\ast T$ by:
$$
\langle \ast T,\varphi\rangle=\langle
T,\overset{-1}{\ast}\varphi\rangle.
$$
We\pageoriginale then have
$$
\langle \overset{-1}{\ast}T,\varphi\rangle=\langle
T,\ast\varphi\rangle.
$$
The star operator on currents is the transpose of the operator
$\overset{-1}{\ast}$ defined on forms with compact support.

\section*{The Riemannian scalar product of a $p$-current and a
  $p$-form}

We define the scalar product of a current $T$ of degree $p$ and a
$p$-form $\varphi$ by:
$$
(T,\varphi)=\langle T,\ast\varphi\rangle
$$
if the support of either $T$ or $\varphi$ is compact. We shall call
this scalar product the Riemannian scalar product between $T$ and
$\varphi$. 

\section*{The star coboundary operator $\partial$}

We define the operator $\partial$ (``del'') on currents. If $T$ is a
current of degree $p$ we define $\partial T$, a current of degree
$p-1$, by:
$$
(\partial T,\overset{p-1}{\varphi})=(T,d\overset{p-1}{\varphi})
$$
$\partial$ is the adjoint of the operator $d$ with respect to the
Hilbertian structure defined by the Riemannian scalar product. We have
\begin{align*}
(\partial T^{p},\varphi) &= (T,d\varphi)\\
&= \langle \overset{-1}{\ast}T,d\varphi\rangle\\
&= (-1)^{N-p-1}\langle d\overset{-1}{\ast}T,\varphi\rangle\\
&= (-1)^{N-p-1}\langle \overset{-1}{\ast}\ast
  d\overset{-1}{\ast}T,\varphi\rangle\\
&= (-1)^{N-p+1}(\ast d\overset{-1}{\ast}T,\varphi)
\end{align*}
so\pageoriginale that
$$
\partial \overset{p}{T}=(-1)^{N-p+1}\ast d\overset{-1}{\ast}T
$$
and 
$$
\partial \overset{p}{T}=(-1)^{p}\ast^{-1}d\ast T.
$$
From $d^{2}=0$ it follows that $\partial^{2}=0$. The operator
$\partial$ defines a new differential graded structure in
$\mathscr{D}'$. The operator is also defined for the spaces
$\mathscr{E}$, $\mathscr{E}'$ and $\mathscr{D}$.

We now have a new cohomology, the $\ast$ cohomology.

This cohomology is not different from the cohomology that we already
have. Consider, for instance, $\mathscr{E}$. We define an isomorphism
between $\underset{\partial}{H^{N-p}}(\mathscr{E})$ and
$H^{p}_{d}(\mathscr{E})$. Suppose $\omega$ is an $N-p$ form with
$\partial\omega=0$. Then $\pm \ast
d\overset{-1}{\ast}\omega=0$. Since $\ast$ is an isomorphism
$d\overset{-1}{\ast}\omega=0$ or $d\ast \omega=0$ \iec $\ast\omega$ is
closed. The mapping $\omega\to \omega^{\ast}$ induces an isomorphism
between $\underset{\partial}{H^{N-p}}(\mathscr{E})$ and
$\underset{d}{H^{p}}(\mathscr{E})$. 

\section*{The Laplacian $\Delta$}

We define the operator $\Delta$ by:
$$
-\Delta=d\partial +\partial d.
$$

This is a differential operator of the second order. $\Delta$
preserves degrees. Since $d\partial $ and $\partial d$ are
self-adjoint, $\Delta$ is self-adjoint:
$$
(\Delta T,\varphi)=(T,\Delta\varphi).
$$
and $ \Delta d$ commute: $\Delta d=d\partial d=d\Delta$. $\Delta$ and
$\partial$ also commute. $\Delta$ also commutes with $\ast$:
$$
\ast\Delta=\Delta\ast.
$$


