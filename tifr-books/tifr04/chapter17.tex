\chapter{Lecture 17}

\section*{The operator $J$}\pageoriginale


We now pass on to some intrinsic properties of a complex analytic
manifold $V^{(n)}$ of complex dimension $n$. The (real) tangent space
to $V$ at $a$, $T_{a}(V)$, is a vector space of dimension $2n$ over
$R$. We shall now introduce on $T_{a}(V)$ an intrinsic
$J$-structure. Take a map at $\ub{a}$ into $R^{2n}$. This map gives an
isomorphism between $T_{a}(V)$ and $R^{2n}$. In $R^{2n}$ we have a $J$
corresponding to the canonical complex structure in $R^{2n}$; by the
canonical isomorphism between $T_{a}(V)$ and $R^{2n}$ (given by the
map) we also have a $J$ in $T_{a}(V)$. We shall now prove that this
$J$ in $T_{a}(V)$ is intrinsic. If $J_{z_{1},\ldots,z_{n}}$ and
$\mathscr{J}_{\zeta_{1},\ldots,\zeta_{n}}$ are the $J$ operators on
$T_{a}(V)$ corresponding to the local coordinates
$(z_{1},\ldots,z_{n})$ and $(\zeta_{1},\ldots,\zeta_{n})$ we have to
prove that $J$ and $\mathscr{YJ}$ are the same. To prove this we
  consider the complexification of $T_{a}(V)$,
  $T_{a}(V)+iT_{a}(V)$. We extend $J$ and $\mathscr{J}$ to
  $T_{a}(V)+iT_{a}(V)$; $J$ 
  and $\mathscr{J}$ are operators in $T_{a}(V)+iT_{a}(V)$ with
  eigenvalues $\pm i$. To prove that $J$ and $\mathscr{J}$ are the
  same it is sufficient to show that the corresponding eigen spaces
  are the same. From the relations
$$
J\left(\frac{\partial}{\partial
  z_{k}}\right)=i\frac{\partial}{\partial z_{k}},\quad
J\left(\frac{\partial}{\partial
  \ob{z}_{k}}\right)=-i\frac{\partial}{\partial \ob{z}_{k}}
$$
it follows that for $J$ the eigenspace corresponding to the eigenvalue
$i$ is the space spanned by 
$$
\frac{\partial}{\partial z_{1}},\ldots,\frac{\partial}{\partial z_{n}}
$$\pageoriginale
and the eigen-space corresponding to $-i$ is the space spanned by
$$
\frac{\partial}{\partial \ob{z}_{1}},\ldots,\frac{\partial}{\partial
  \ob{z}_{n}};
$$
for $\mathscr{J}$ the corresponding spaces are spanned by
$$
\frac{\partial}{\partial \zeta_{1}},\ldots,\frac{\partial}{\partial
  \zeta_{n}}\quad\text{and}\quad
\frac{\partial}{\partial\ob{\zeta}_{1}},\ldots,\frac{\partial}{\partial\ob{\zeta}_{n}}, 
$$
respectively. Since the maps are related on the overlaps by
holomorphic functions we have the relations
\begin{align*}
\frac{\partial}{\partial z_{j}} &= \sum_{k}\frac{\partial
  \zeta_{k}}{\partial z_{j}}\frac{\partial}{\partial \zeta_{k}},\\
\frac{\partial}{\partial \zeta_{j}} &= \sum_{k}\frac{\partial
  z_{k}}{\partial \zeta_{j}}\frac{\partial}{\partial z_{k}},\\
\frac{\partial}{\partial \ob{z}_{j}} &= \sum_{k}\frac{\partial
  \ob{\zeta}_{k}}{\partial \ob{z}_{j}}\frac{\partial}{\partial
  \ob{\zeta}_{k}},\\
\frac{\partial}{\partial \ob{\zeta}_{j}} &= \sum_{k}\frac{\partial
  \ob{z}_{k}}{\partial \ob{\zeta}_{j}}\frac{\partial}{\partial \ob{z}_{k}};
\end{align*}
these relations show that the eigen space of $J$ and $\mathscr{J}$
corresponding to the eigen values $i$ and $-i$ are the same.

$J$ operates on the space of tangent covectors real or complex; we
have the relations $Jdz_{k}=-idz_{k}$
$Jd\ob{z}_{k}=id\ob{z}_{k}$. $J$\pageoriginale\ also operates on the
space of tangent $p$-covectors (real or complex) at $\ub{a}$;
consequently $J$ operates on the space differential forms. In fact $J$
is a real operator \iec $J$ takes real differential forms into real
differential forms. For forms of degree $P$ we have
$$
J^{2}=(-1)^{p}I
$$

\section*{Bigradation for differential forms}

A differential form $\omega$ is said to be of bidegree or of type
$(p,q)$ if in every map, $\omega$ has the form
$$
\omega=\sum_{\substack{j_{1}<\ldots<j_{p}\\ k_{1}<\ldots<k_{q}}} 
\omega_{j_{1}\ldots j_{p}k_{1}\ldots 
  k_{q}}dz_{j_{1}}\wedge\ldots\wedge dz_{j_p} \wedge
d\ob{z}_{k_{1}}\wedge\ldots\wedge d\ob{z}_{k_{q}}.
$$
This definition is correct since if two maps overlap and a form
defined in the overlap be of type $(p,q)$ in one of the maps it is
also of type $(p,q)$ in the other map. (This follows easily from the
fact that the maps are related by holomorphic functions on the
overlaps). $p$ is called the $z$ degree and $q$ the $\ob{z}$ degree of
$\omega$.

Any differential form $\omega$ of total degree $r$ can be uniquely
written in the form
$$
\omega=\sum_{p+q=r}\overset{(p,q)}{\widetilde{\omega}}
$$
where $\overset{(p,q)}{\widetilde{\omega}}$ is a form of bidegree
$(p,q)$. 

\section*{Holomorphic functions and forms}

\pageoriginale
A (complex valued) function $f$ on $V^{(n)}$ is said to be holomorphic
if $f$ is holomorphic on every map; this definition is correct as a
holomorphic function of $n$ holomorphic functions on $C^{n}$ is again
holomorphic. A holomorphic differential form of degree $p$ is a form
of type $(p,0)$ whose coefficients in every map are holomorphic.

\section*{The operators $d_{z}$ and $d_{\ob{z}}$}

Suppose $\omega$ is a form of bidegree $(p,q)$. A priori $d\omega$ is
the sum of forms of all bidgree $(r,s)$ with $r+s=p+q+1$. However we
shall show that $d\omega$ is the sum of a form of bidegree $(p+1,q)$
and a form of bidegree $(p,q+1)$. Let 
$$
\omega=\sum_{J,K}\omega_{J,K}dz_{J}\wedge d\ob{z}_{K}
$$
in a map. $J=(j_{1},\ldots,j_{p})$, $K=(k_{1},\ldots,k_{q})$ denote a
system of indices in 
\begin{align*}
d\omega &= \sum d\omega_{J,K} dz_{J}\wedge d\ob{z}_{K}\\
&= \sum\frac{\partial \omega_{J,K}}{\partial z_{1}}dz_{1}\wedge
dz_{J}\wedge d\ob{z}_{K}\\
& + \sum\frac{\partial \omega_{J,K}}{\partial
  \ob{z}_{1}}d\ob{z}_{1}\wedge dz_{J}\wedge d\ob{z}_{J} 
\end{align*}
Here the first form is of bidegree $(p+1,q)$ and the second of
bidegree $(p,q+1)$; therefore these two forms have an intrinsic
meaning. Thus\pageoriginale
$$
d\overset{(p,q)}{\omega}=\overset{(p+1q)}{\alpha}+\overset{(p,q+1)}{\beta}
$$
intrinsically, where $\alpha$ and $\beta$ are forms of bidegree
$(p+1,q)$ and $(p,q+1)$ respectively. We now define the operators
$d_{z}$ and $d_{\ob{z}}$ by:
$$
d_{z}\omega=\alpha,\quad d_{\ob{z}}\omega=\beta.
$$
We have
$$
d\omega=d_{2}\omega+d_{\ob{z}}\omega.
$$
We observe that, in a map, $d_{z}\omega$ involves only the partial
derivatives with respect to $z$ while $d_{\ob{z}}\omega$ involves only
the partial derivatives with respect to $\ob{z}\cdot d_{z}$ increases
the degree corresponding to $z$ by one while $d_{\ob{z}}$ increases
the degree corresponding to $\ob{z}$ by one.

We extend the operators $d_{z}$ and $d_{\ob{z}}$ to all forms by
linearity.

We shall now consider some properties of $d_{z}$ and $d_{\ob{z}}\cdot
d_{z}$ and $d_{\ob{z}}$ are complex operators (If $\omega$ is real
$d_{z}\omega$ is complex). $d_{z}$ is of type $(1,0)$ and $d_{\ob{z}}$
is of type $(0,1)$. [An operator is said to be of type $(r,s)$ if it
  takes a form of bigradation $(p,q)$ into a form of bigradation
  $(p+r,q+s)$]. These operators are local operators; they are
linear. If $\omega$ is a form of degree $r$ we have the formulae:
\begin{align*}
d_{z}(\omega\wedge \widetilde{\omega}) &=
d_{z}\omega\wedge\widetilde{\omega}+(-1)^{r}\omega\wedge
d_{z}\widetilde{\omega},\\
d_{\ob{z}}(\omega\wedge\widetilde{\omega}) &=
d_{\ob{z}}\omega\wedge\widetilde{\omega} +(-1)^{r}\omega\wedge
d_{\ob{z}}\widetilde{\omega}. 
\end{align*}\pageoriginale

It is enough to prove this for homogeneous forms. We take
$d(\omega\wedge\widetilde{\omega})$ and decompose it:
\begin{align*}
d(\overset{p,q}{\omega}\wedge \overset{s,t}{\widetilde{\omega}}) &=
d\omega\wedge\widetilde{\omega}+(-1)^{p+q}\omega \wedge
d\widetilde{\omega}\\
&= (d_{z}\omega\wedge\widetilde{\omega}+(-1)^{p+q}\omega\wedge
d_{z}\widetilde{\omega})\\
&+(d_{\ob{z}}\omega\wedge\widetilde{\omega}+(-1)^{p+q}\omega\wedge
d_{\ob{z}}\widetilde{\omega}); 
\end{align*}
the first term is of bidegree $(p+s+1,q+t)$ while the second is of
bidegree $(p+s,q+t+1)$ and this proves the result.

Further we have the relations
$$
d_{z}d_{z}=0, d_{\ob{z}}d_{\ob{z}}=0\quad\text{and}\quad
d_{z}d_{\ob{z}}+d_{\ob{z}}d_{z}=0.
$$
For, from the relations
$$
(d_{z}+d_{\ob{z}})(d_{z}+d_{\ob{z}})=d^{2}=0
$$
we obtain 
$$
d_{z}d_{z}+(d_{z}d_{\ob{z}}+d_{\ob{z}}d_{z})+d_{\ob{z}}d_{\ob{z}}=0.
$$
To conclude the above relations from this we have only to observe that
for any form $\overset{p,q}{\omega}$ the forms $d_{z}d_{z}\omega$,
$(d_{z}d_{\ob{z}}\omega+d_{\ob{z}}d_{z}\omega)$ and
$d_{\ob{z}}d_{\ob{z}} \omega $ are of different bidegrees namely of the
bidegrees $(p+2,q)$, $(p+1,q+1)$ and $(p,q+2)$ respectively.

\section*{$z$ and $\ob{z}$ cohomologies}\pageoriginale

We have now two new coboundary operators $d_{z}$ and $d_{\ob{z}}$ and
hence two new cohomologies, $z$ and $\ob{z}$ cohomologies. We shall
confine our attention to the $\ob{z}$ cohomology; this will give all
the information about holomorphic forms. We shall see that the
construction of holomorphic forms depends on the $\ob{z}$
cohomology. We shall denote by $H^{p,q}_{\ob{z}}(\mathscr{E}(V))$ the
space $Z^{p,q}/B^{p,q}$ where $Z^{p,q}$ and $B^{p,q}$ are the
subspaces of $\mathscr{E}^{p,q}(V)$ (the space of forms of bidegree
$(p,q)$) consisting of $\ob{z}$ cocycles and $\ob{z}$ coboundaries
respectively. 

Similarly we have the space $H^{p,q}_{\ob{z}}(\mathscr{D}(V))$.

\section*{Intrinsic characterization of holomorphic forms}

Let $\omega$ be a $C^{\infty}$ form of type $(p,0)$; a necessary and
sufficient condition for $\omega$ to be holomorphic is that
$$
d_{\ob{z}}\omega=0.
$$
Let
$$
\omega=\sum_{K}\omega_{K}dz_{K}
$$
in a map.
$$
d_{\ob{z}}\omega=\sum_{K,1}\frac{\partial \omega_{K}}{\partial
  \ob{z}_{1}}d\ob{z}_{1}\wedge dz_{K} 
$$
In $\omega$ is holomorphic, $\omega_{K}$ are holomorphic; hence
$\dfrac{\partial \omega_{K}}{\partial \ob{z}_{1}}=0$ and
$d_{\ob{z}}\omega=0$. If $d_{\ob{z}}\omega=0$
$$
\sum_{K,1}\frac{\partial \omega_{K}}{\partial
  \ob{z}_{1}}d\ob{z}_{1}\wedge dz_{K}=0;
$$\pageoriginale
in the sum on the left side all the terms $d\ob{z}_{1}\wedge dz_{K}$
are different (\iec if $(K,1)\neq (K',l')$, $d\ob{z}_{1}\wedge
dz_{K}\neq d\ob{z}_{l'}\wedge dz_{K'}$). Consequently $\dfrac{\partial
  \omega_{K}}{\partial \ob{z}_{1}}=0$. This proves that the
$\omega_{K}$ are all holomorphic.

\section*{Holomorphic forms and $\ob{z}$ cohomology}

Let us now consider the space $H^{p,0}_{\ob{z}}(\mathscr{E}(V))\cdot
H^{p,0}_{\ob{z}}(\mathscr{E}(V))$ is just the space of $(p,0)$ forms
which are $\ob{z}$ cocycles (since a $(p,0)$ from which is $\ob{z}$
coboundary is trivial). \iec $H^{p,0}_{\ob{z}}(\mathscr{E}(V))$ is the
space of holomorphic $p$-forms. Similarly
$H^{p,0}_{\ob{z}}(\mathscr{D}(V))$ is the space of holomorphic forms
with compact support. A holomorphic form with compact support is zero
on each non-compact connected component.

Consider the space $\mathscr{E}^{,q}=\sum\limits_{p}\mathscr{E}^{p,q}$
of $C^{\infty}$ forms of $\ob{z}$ degree
$q$. $\sum\limits_{q}\mathscr{E}^{,q}$ is a complex with the
differential operator $d_{\ob{z}}$. This gives rise to the cohomology
groups $H_{\ob{z}}^{,q}$. $H^{,0}_{\ob{z}}$ is the space holomorphic
forms.



