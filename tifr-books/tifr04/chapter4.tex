\chapter{Lecture 4}

\section*{Existence and uniqueness of the exterior
  differentiation. The DGA $\mathscr{E}(V)$.}\pageoriginale

We give a sketch of the proof of the existence and uniqueness of
$d$. First we assume the existence and prove uniqueness. Since $d$ is
a local operation it is enough to reason on an open subset $U$ of
$R^{N}$. Let $\omega$ be a $p$ form on $U$ and $x_{1},\ldots,x_{N}$
the coordinate functions in $R^{N}$. Then
$$
\omega = \sum_{j_{1}<\ldots<j_{p}}\omega_{j_{1}\ldots
j_{p}}dx_{j_{1}}\wedge\ldots\wedge dx_{j_{p}}.
$$
Let $d'$ be an operation satisfying conditions [(1) - (5)]. Since $d'$
is linear it is sufficient to consider the form
$$
\omega=\omega_{j_{1}\ldots j_{p}} dx_{j_{1}}\wedge\ldots\wedge
dx_{j_{p}}
$$
By the product rule
\begin{align*}
d'\omega &=d'(\omega_{j_{1}\ldots j_{p}} dx_{j_{1}}\wedge\ldots\wedge
dx_{j_{p}})\\
&= d'\omega_{j_{1}\ldots j_{p}} dx_{j_{1}}\wedge\ldots\wedge
dx_{j_{p}}\\
&+\omega_{j_{1}\ldots j_{p}}d'(dx_{j_{1}}\wedge\ldots\wedge dx_{j_{p}})
\end{align*}
By property $5$, $df=d'f$ for a function $f$. So
$$
d'(dx_{j_{1}}\wedge\ldots\wedge
dx_{j_{p}})=d'(d'x_{j_{1}}\wedge\ldots\wedge d'x_{j_{p}}).
$$
Using the product rule it is seen by induction that
\begin{align*}
& d'(d'x_{j_{1}}\wedge\ldots\wedge d'x_{j_{p}})\\
& =\sum^{p}_{i=1}(-1)^{i-1}d'x_{j_{1}}\wedge\ldots\wedge
  d'x_{j_{i-1}}\wedge d'(d'x_{j_{i}})\wedge
  d'x_{j_{i}+1}\wedge\ldots\wedge d'x_{j_{p}}
\end{align*}\pageoriginale
Since ${d'}^{2}=0$, this sum is zero. So
$$
d'\omega=d\omega_{j_{1}\ldots j_{p}}\wedge
dx_{j_{1}}\wedge\ldots\wedge dx_{j_{p}}.
$$
This proves that $d'$ is unique. This formula also shows how we should
try to define the operation $d$ to prove the existence. We define $d$
locally by this formula.

Let $(U,\varphi)$ be a map and $x_{1},\ldots,x_{N}$ the corresponding
coordinate functions. Let $\omega$ be a $p$ form. In $U$, $\omega$ can
be written as
$$
\omega=\sum_{j<\ldots<j_{p}}\omega_{j_{1}\ldots
j_{p}} dx_{j_{1}}\wedge\ldots\wedge dx_{j_{p}}
$$
In $U$ we define
$$
d\omega=\sum_{j_{1}<\ldots<j_{p}}d\omega_{j_{1}\ldots j_{p}}\wedge
dx_{j_{1}}\wedge\ldots\wedge dx_{j_{p}}.
$$
It can be verified that $d$ has the properties 1 - 5 in $U$. It
follows from this and the uniqueness theorem we proved above that $d$
is defined intrinsically on the whole manifold (If $U_{i}$ and $U_{j}$
are of two maps and $d_{i}$, $d_{j}$ the exterior differentiations
defined in $U_{i}$, $U_{j}$ respectively by the above formula, then
$d_{i}=d_{j}$ in $U_{i}\cap U_{j}$ by the uniqueness theorem).

Let us consider some examples. In $R^{3}$ we have $0$, $1$, $2$ and
$3$ forms. If $f$ is a zero form,
$$
df=\frac{\partial f}{\partial x}dx+\frac{\partial f}{\partial
  y}dy+\frac{\partial f}{\partial z}dz.
$$\pageoriginale
In $R^{3}$ the canonical basis for two forms is usually taken to be
$dy\wedge dz$, $dz\wedge dx$ and $dx\wedge dy$. Let $A\, dx+B
\ dy+C \ dz$ be a $1$-form.
\begin{align*}
& d(A\ dx+B\ dy+C\ dz)=\left(\frac{\partial c}{\partial
    y}-\frac{\partial B}{\partial z}\right)dy\wedge dz\\
&\qquad +\left(\frac{\partial A}{\partial z}-\frac{\partial
    C}{\partial x}\right)dx\wedge dx+\left(\frac{\partial B}{\partial
    x}-\frac{\partial A}{\partial y}\right)dx\wedge dy
\end{align*}
If $A\ dy\wedge dz+B\ dz\wedge dx+C\ dx\wedge dy$ be a two form, then
$$
dA\wedge dy\wedge dz=\frac{\partial A}{\partial x}dx\wedge dy\wedge
dz.
$$
and
$$
d(A\ dy\wedge dz+B\ dz\wedge dx+C\ dx\wedge dy)=\left(\frac{\partial
  A}{\partial x}+\frac{\partial B}{\partial y}+\frac{\partial
  C}{\partial Z}\right)dx\wedge dy\wedge dz.
$$
The exterior derivatives of $0$, $1$ and $2$ forms correspond
respectively to the notions of the gradient of a function, and curl
and divergence of a vector field. The formula $d^{2}=0$ corresponds to
curl grad $=0$ and $\text{div\,} curl =0$.

We make some remarks on the space $\overset{p}{\mathscr{E}}(V)$ of $p$ forms on
$V$. $\overset{p}{\mathscr{E}}(V)$ is an infinite dimensional vector
space. For instance $\overset{0}{\mathscr{E}}(V)$ is infinite
dimensional since we can construct a $C^{\infty}$ function taking
prescribed values at arbitrary finite number of points. The space
$$
\mathscr{E}(V)=\sum^{N}_{0}\overset{p}{\mathscr{E}}(V)
$$
is an algebra. (The product of two $C^{\infty}$ forms is
$C^{\infty}$). We shall later put on $\mathscr{E}(V)$ a topological
structure so that it becomes a topological\pageoriginale vector
space. $\mathscr{E}(V)$ is a graded algebra: it is decomposed into the
homogeneous pieces $\overset{p}{\mathscr{E}}(V)$ and the
multiplication obeys the anti-commutativity
rule. $\mathscr{E}(V)$ has an internal operation
$d:\mathscr{E}(V)\to \mathscr{E}(V)$ by which a homogeneous element of
order $p$ is taken into a homogeneous element of order $p+1$ with the
properties $d^{2}=0$ and
$$
d(\overset{p}{\omega}\wedge
\overset{\ub{q}}{\omega})=d\overset{p}{\omega}\wedge
\overset{\ub{q}}{\omega}+(-1)^{p}\overset{p}{\omega}\wedge
d\overset{\ub{q}}{\omega}.
$$
$\mathscr{E}(V)$ has the structure of what is called a graded algebra
with a differential operator (DGA).

We consider the behaviour of $d$ with respect to mappings of
manifolds. Let $\Phi:U\to V$ be a $C^{\infty}$ map of the two
manifolds $U$ and $V$. Then we have a mapping
$\Phi^{-1}:\mathscr{E}(V)\to \mathscr{E}(U)\cdot \Phi^{-1}$ is a
homomorphism of the graded algebra $\mathscr{E}(V)$ into the graded
algebra $\mathscr{E}(U)$. We shall now prove that $\Phi^{-1}$ is a
homomorphism of the DGA's. We have to prove that $\Phi^{-1}$ commutes
with the coboundary operator $d$: 
$$
d\Phi^{-1}(\omega)=\Phi^{-1}(d\omega)
$$
where $\omega$ is a form on $V$. (Strictly speaking the `$d$'s are
different). We shall prove this with the minimum possible
calculation. Since $d$ and $\Phi^{-1}$ have a local character we may
assume $U$ and $V$ to be open subsets of Euclidean spaces. Since
$\Phi^{-1}:\mathscr{E}(V)\to \mathscr{E}(U)$ is a homomorphism of the
algebras it is sufficient to prove the result for a system of elements
which generate $\mathscr{E}(V)$. If $x_{1},\ldots,x_{N}$ are the
coordinate functions in $V$, $dx_{1},\ldots,dx_{N}$ and the
$C^{\infty}$ functions on $V$ generate $\mathscr{E}(V)$. So we have
only to prove in the case when $\omega$ is a $0$-form
and\pageoriginale when $\omega$ is a $1$-form which is the
differential of a function. Let $f$ be a $0$-form; the result we wish
to prove is just the definition of the reciprocal image:
\begin{align*}
\Phi^{-1}(df) &= d(f\circ \Phi)\quad\text{by definition}\\
 &= d(\Phi^{-1}(f)).
\end{align*}
Let $\omega=df$, $f$ being a function. $d\omega=0$, so
$\Phi^{-1}(d\omega)=0$. We have to prove that $d\Phi^{-1}(df)=0$; but
we have just proved that 
$$
\Phi^{-1}(df)=d(\Phi^{-1}(f)),\text{ \  so that \ }
d\Phi^{-1}(df)=d(d\Phi^{-1}(f))=0. 
$$

\section*{Change of Variables.}

The reciprocal image of a map gives a good method of obtaining the
formula for change of variables. Let us consider for instance polar
coordinates in $R^{3}:x=r\Sin \theta\Cos \varphi$, $y=r\Sin
\theta\Sin\varphi$, $z=r\Cos \varphi$. We wish to express $A\ dx
\wedge dy$ in terms of polar coordinates. We have a map $\Phi:R^{3}\to
R^{3}:\Phi(r,\theta,\varphi)=(x,y,z)$. We have to find the reciprocal
image of $A\ dx\wedge dy$ with respect to $\Phi$. Since $\Phi^{-1}$
preserves products,
\begin{align*}
\Phi^{-1}(A\ dx\wedge dy) &=
\Phi^{-1}(A)\Phi^{-1}(dx)\wedge\Phi^{-1}(dy)\\
 &= A(r\Sin\theta\Cos\varphi, r\Sin\theta \Sin\varphi, r\Cos
\varphi)\\
&= (\Sin\theta\Cos\varphi dr+r\Cos\theta\Cos\varphi
d\theta-r\Sin\theta\Sin\varphi d\varphi)\\
&= (\Sin\theta\Sin\varphi dr+r\Cos \theta\Sin\varphi
d\theta+r\Sin\theta\Cos\varphi d\varphi)\\
&= A(r^{2}\Sin\theta\Cos\theta d\theta\wedge d\varphi-r\Sin^{2}\theta
d\varphi\wedge dr) 
\end{align*}
(We use $dr\wedge dr=0$, $-dr\wedge d\theta=d\theta\wedge dr$ etc).

\section*{Poincar\'e's Theorem on differential forms in $R^{N}$.}

We mention here without proof a theorem of Poincar\'e. We say
that\pageoriginale a $p$-form $\overset{p}{\omega}$ is exact if
$\overset{p}{\omega}=d\overset{p-1}{\widetilde{\omega}}$ where
$\overset{p-1}{\widetilde{\omega}}$ is a $(p-1)$ form. For $p=0$, this
implies, by convention, that $\overset{p}{\omega}=0$. Since $d^{2}=0$,
a necessary condition for $\overset{p}{\omega}$ to be exact is that
$d\overset{p}{\omega}=0$. Poincar\'e's theorem is that, in $R^{N}$,
this condition is also sufficient for $\overset{p}{\omega}$ to be
exact, provided $p\geq 1$. Thus a necessary and sufficient condition
for a $p$-form $(p\geq 1)$ in $R^{N}$ to be exact is that its exterior
derivative is zero.


