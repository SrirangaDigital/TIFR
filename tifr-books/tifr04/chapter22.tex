\chapter{Lecture 22}

\section*{The identity between $d$, $z$ and $\ob{z}$ cohomologies: de
  Rham theorem (Compact K\"ahlerian manifolds)}\pageoriginale

We shall now prove that the cohomology spaces defined by the operators
$d$, $d_{z}$, $d_{\ob{z}}$ and $\widetilde{d}$ are canonically
isomorphic. That the ordinary cohomology and the $\ob{z}$ cohomology
are the same will be of importance in the study of holomorphic and
meromorphic forms on a compact K\"ahlerian manifold.

We shall give the proof in the case of the ordinary cohomology and the
$\ob{z}$ cohomology. Let $H$ be a cohomology space with respect to $d$
and $H_{\ob{z}}$ the corresponding $\ob{z}$ cohomology space. We have
a canonical mapping from $H$ to $H_{\ob{z}}$. In each cohomology class
$\overset{0}{\omega}$ (with respect to $d$) we choose a form,
$\omega$, which is $\ob{z}$ closed and map the class
$\overset{0}{\omega}$ to the $\ob{z}$ cohomology class determined by
$\omega$. In each $d$-cohomology class such a form exists, the
harmonic form belonging to the class. We have to verify that if a
form, $\alpha$, which is a $d$-coboundary is $\ob{z}$ closed then
$\alpha$ is a $\ob{z}$ coboundary. Since $\alpha$ is a $d$-coboundary
$\pi \alpha_1=0$. By the decomposition formula
\begin{align*}
\alpha &=
\pi_{1}\alpha+2d_{\ob{z}}\partial_{\ob{z}}G\alpha+2\partial_{\ob{z}}d_{\ob{z}}G\alpha\\ 
&= 2d_{\ob{z}}\partial_{\ob{z}}\alpha
\end{align*}
so that $\omega$ is a $\ob{z}$ coboundary. The mapping so defined is
actually an isomorphism. For the space $H_{\ob{z}}$ can be identified
with the space of harmonic\pageoriginale forms and we know that the
space $H$ also can be identified with the space of harmonic forms. We
have thus a canonical isomorphism between $d$ and $\ob{z}$
cohomologies, which is independent of the K\"ahlerian metric.

Thus the $\ob{z}$ cohomology for a compact complex analytic manifold
with a K\"ahlerian metric is just the ordinary cohomology. The first
and third parts of de Rham's theorem for $\ob{z}$ cohomology for such
manifolds follow immediately. The projections $\pi_{1}$,
$\pi_{2,\ob{z}}=2d_{\ob{z}}-\partial_{\ob{z}}G,\pi_{3,\ob{z}}$ are
continuous and the image spaces corresponding to these projections are
closed because they are kernels. This proves the second part of de
Rham's theorem for $\ob{z}$ cohomology.

\section*{de Rham theorems for $\ob{z}$ cohomology of an arbitrary
  complex analytic manifold}

The first part of de Rham's theorem for $\ob{z}$ cohomology in the
case of an arbitrary manifold was proved only recently. The third part
of de Rham's theorem is also true: the $\ob{z}$ Betti-numbers are
finite for a compact manifold. However it is not true in general that
the spaces of $\ob{z}$ coboundaries are closed. On the other hand if
we assume that all the Betti-numbers are finite this theorem can be
restored.

\section*{The complex projective space}

Let $PC^{n}$ denote the complex projective space of $n$ dimensions
$PC^{n}$ is a compact K\"ahlerian manifold (see the
appendix). Algebraic manifolds imbedded without singularities in
$PC^{n}$ are also compact K\"ahlerian\pageoriginale\ manifolds. (In
general it is true that a complex analytic manifold regularly imbedded
in a K\"ahlerian manifold is a K\"ahlerian manifold). For $PC^{n}$ we
have $b^{p}=1$ for even $p$ and $b^{p}=0$ for odd $p$. The form
$\Omega^{k}$ ($\Omega$ is the exterior $2$-form associated with the
K\"ahlerian metric) gives an element $\neq 0$ of the $p$th cohomology
space.

In this connection we shall state another part of de Rham's theorem
(arbitrary $C^{\infty}$ manifold, $d$-cohomology). This part of de Rham
theorem states that each cohomology class of currents contains a cycle
(closed chain) and that a cycle which is the coboundary of a current
is the boundary of a chain. We shall also make use of de Rham's
correspondence between cohomology of forms and homology of chains in
which exterior products of cohomology classes correspond to algebraic
intersections of homology classes.

Let $PC^{n-1}$ be a hyperplane of $PC^{n}\cdot PC^{n-1}$ is a
cycle. $PC^{n-1}$ determines an element $(\neq 0)$ of the $2$nd
cohomology class (of currents). So $\Omega$ is homologous (in the
sense of the currents) to $k\cdot PC^{n-1}$ where $k$ is a real
number: $\Omega\approx k\cdot PC^{n-1}$. Actually $k>0$. For,
$$
\int\limits_{PC^{n}}\Omega^{n}=\langle
\Omega,\Omega^{n-1}\rangle=k\int\limits_{PC^{n-1}}\Omega^{n-1} 
$$
Since $PC^{n}$ and $PC^{n-1}$ are K\"ahler manifolds (with the
associated exterior $2$-form $\Omega$),
$$
\int\limits_{PC^{n}}\Omega^{n}>0\quad\text{and}\quad
\int\limits_{PC^{n-1}}\Omega^{n-1}>0\quad\text{so that}\quad k>0.
$$\pageoriginale
Considering $n$ hyperplanes in general position (whose intersection is
a point $PC^{0}$) we find that
\begin{gather*}
\Omega^{n}\approx k^{n}PC^{0}\\
\approx k^{n}x\text{ \  point (as a chain or current).}
\end{gather*}
Therefore
$$
\int\limits_{PC^{n}}\Omega^{n}=k^{n}\langle PC^{0},1\rangle =k^{n}.
$$
So
$$
k=\sqrt[n]{\int\limits_{PC^{n}}\Omega^{n}}.
$$
If we choose the K\"ahlerian metric whose associated $2$-form is
$$
\Omega'=\Omega/\sqrt[n]{\int\limits_{PC^{n}}\Omega^{n}}
$$
we have ${\Omega'}^{p}\approx PC^{n-p}$. In other words if the
associated $2$-form $\Omega$ of a K\"ahlerian metric satisfies the
relation
$$
\int\limits_{PC^{n}}\Omega^{n}=1
$$
then
$$
\Omega^{p}\approx PC^{n-p}.
$$
The volume of $PC^{n}$ with respect to the volume element given by
such a metric is $n!$.

The Betti numbers of $PC^{n}$ verify, of course, all the properties of
the Betti numbers of a compact K\"ahlerian manifold. We have 
$$
b^{0,0}=b^{1,1}=\ldots=b^{n,n}=1.
$$\pageoriginale
and all the other double Betti-numbers are zero. In particular
$b^{p,0}=0$ for $p\neq 0$. So on the complex projective space there
are no holomorphic differentials of any degree except the degree zero
(in which case the holomorphic forms are constant functions).

\section*{Example of a compact complex manifold which is not
  K\"ahlerian}

Let us consider in $C^{n}$ the shell between the two spheres of radii
$1$ and $2:(1\leq |z|\leq 2)$. Identify the two points on the spheres
which are on the same radius.

Let us denote by $V$ the space obtained after this identification. Let
$G$ denote the properly discontinuous group of analytic automorphisms
of $(C^{n}-0)$ consisting of the homothetic transformations. 
$$
(z_{1},\ldots,z_{n})\to (2^{k}z_{1},\ldots,2^{k}z_{n})
$$
$k$ running over all integers, positive, negative or zero. $V$ is a
fundamental domain for this group. Since we can always introduce a
complex analytic structure in the quotient space obtained from a
properly discontinuous group, $V$ can be endowed with a natural
analytic structure. $V$ is compact. $V$ is not K\"ahlerian for
$n>1$. It can be proved that, for $n>1$,
\begin{align*}
b^{0} &= b^{1}=1,\\
b^{i} &= 0\text{ \ for \ } 2\leq i\leq 2(n-1).\\
b^{2n-1} &= 1, b^{2n}=1.
\end{align*}\pageoriginale 
For $n\geq 3$, the Betti-numbers $b^{p}$ are not greater than $1$ for
even $p$. For $n\geq 2$ odd dimensional Betti numbers are not even.

If $P(z_{1},\ldots,z_{n})$ and $Q(z_{1},\ldots,z_{n})$ are homogeneous
polynomials of the same degree
$P(z_{1},\ldots,z_{n})/Q(z_{1},\ldots,z_{n})$ defines a meromorphic
function on $V$. In this connection we may remark that there exist
compact complex analytic manifolds which admit of no non-constant
meromorphic function. Examples are provided by certain torii.




