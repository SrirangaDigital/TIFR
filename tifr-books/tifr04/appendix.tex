\chapter{Appendix}

\section*{K\"ahlerian structure on the complex\protect\hfil\protect\break projective
  space}\pageoriginale

The unit sphere $S^{2n+1}$ in $C^{n}$ can be considered in a natural
way as a fibre bundle over $PC^{n}$ with circles as fibres (if $z$ is
a point of $S^{2n+1}$ the points $e^{i\theta}{z}$, $\theta$ real,
will constitute the fibre through $z$).

We shall first determine how $J$ operates on the tangent spaces of
$PC^{n}$. Let $\pi:S^{2n+1}\to PC^{n}$ denote the projection map. We
shall denote the differential of $\pi$ also by $\pi$. The differential
map $\pi$ maps the tangent bundle of $S^{2n+1}$ onto that of
$PC^{n}$. Let $X=\pi U$, $U$ tangent to $S^{2n+1}$, be a vector
tangent to $PC^{n}$, say at $a$. $J$ is uniquely determined by:
\begin{align*}
\langle JX, df\rangle &= i\langle X,df\rangle\\
\langle JX, \ob{df}\rangle &=-i \langle X,\ob{df}\rangle
\end{align*}
for every $f$ holomorphic in a neighbourhood of $\ub{a}$. If
$z_{n+1}\neq 0$, $z_{k}/z_{n+1}$ is a local coordinate system; we may
suppose that $\ub{a}$ belongs to the domain of this coordinate
system. $J$ is uniquely determined by:
\begin{align*}
\langle JX, d(z_{k}/z_{n+1})\rangle &=i \langle
X,d(z_{k}/z_{n+1})\rangle\\
\langle JX, \ob{d(z_{k}/z_{n+1})}\rangle &=-i\langle X,
\ob{d(z_{k}/z_{n+1})}\rangle. 
\end{align*}
$(u_{1},\ldots,u_{n})$ being the coordinates of $U$, let $(iU)$ denote
the vector whose coordinates are
$(iu_{1},\ldots,iu_{n})$. The\pageoriginale vector $\pi(iU)$ has the
properties
$$
\langle \pi (iU),d(z_{k}/z_{n+1})\rangle =\langle iU,
d(z_{k}/z_{n+1})\rangle =i\langle X,d(z_{k}/z_{n+1})\rangle;
$$
and $\langle\pi(iU),\ob{d(z_{k}/z_{n+1})}\rangle=-i\langle
X,\ob{d(z_{k}/z_{n+1})}\rangle$.

Therefore
$$
\pi(iU)=JX
$$
So, to find $JX$ we multiply $U$ by $i$ and take its image by $\pi$. 

Consider now the form
$$
\omega=\sum \ob{z}_{k}dz_{k}
$$
of bidegree $(1,0)$ in $C^{n+1}$. Put
$$
H=\sum dz_{k}\ob{dz_{k}}-\omega\ob{\omega}.
$$
This Hermitian form on $C^{n+1}$ induces on $S^{2n+1}$ a semi-definite
Hermitian form. To prove that this form is semi-definite on
$S^{2n+1}$, let $U=(u_{1},\ldots,u_{n})$ be a vector tangent to
$S^{2n+1}$ at $z\in S^{2n+1}$, $z=(z_{1},\ldots,z_{n})$. By Schwarz's
inequality
$$
|\sum \ob{z}_{k}u_{k}|\leq |\ob{z}|\;|U|=|U|
$$
(since $z$ is on the unit sphere). Similarly
$$
|\sum z_{k}\ob{u}_{k}|\leq |U|.
$$
So
$$
H(U,U)=\sum u_{k}\ob{u}_{k}-\left(\sum\ob{z}_{k}\ob{u}_{k}\right)
\left(\sum z_{k}\ob{u}_{k}\right)\geq 0.
$$
Moreover $H(U,U)=0$ if and only if $u_{k}=\mu z_{k}$, $\mu$
complex. Actually $\mu=i\lambda$, where $\lambda$ is real. For, since
$z$ is on $S^{2n+1}$ and $U$ is tangent to the unit sphere at $z$ 
$$
\sum(\ob{z}_{k}u_{k}+z_{k}\ob{u}_{k})=0
$$\pageoriginale
or
$$
\left(\sum z_{k}\ob{z}_{k}\right)(\mu+\ob{\mu})=0\quad\text{or}\quad Rl\mu=0.
$$
Thus $H(U,U)=0$ if and only if $U=i\lambda z$, $\lambda$ real \iec if
and only if $U$ is tangent to the fibre, that is if $\pi(U)=0$.

If $U$ is tangent to the fibre at $z$ and $V=(v_{1},\ldots,v_{n})$ is
any vector tangent to $S^{2n+1}$ at $z$, $H(U,V)=0$. For by Schwarz's
inequality
$$
|H(U,V)|\leq \sqrt{H(U,U)H(V,V)}=0
$$
Moreover $H$ is invariant under the operations $z\to e^{i\theta}z$,
$\theta$ real. These two facts imply that $H$ defines quadratic forms
$\widetilde{H}$ on tangent spaces of $PC^{n}$ (if $X$ and $Y$
are tangent to $PC^{n}$ at $\ub{a}$ we choose vectors $U$ and $V$
tangent to $S^{2n+1}$ at some point in $\pi^{-1}(a)$ such that $\pi
U=X$ and $\pi V=Y$ and define $\widetilde{H}(X,Y)=H(U,V)$. The two
properties proved above imply that this definition defines
$\widetilde{H}$ invariantly).

$\widetilde{H}$ are $J$ Hermitian forms.
\begin{itemize}
\item[i)] Evidently $\widetilde{H}$ is $R$-bilinear

\item[ii)] $\widetilde{H}(JX,Y)=-\widetilde{H}(X,JY)=i\widetilde{H}(X,Y)$

For,
$$
\widetilde{H}(JX,Y)=H(iU,V)=i\widetilde{H}(X,Y)
$$
as we have seen that $JX=\pi(iU)$. Similarly
$$
\widetilde{H}(X,JY)=-i\widetilde{H}(X,Y)
$$

\item[iii)] $\tilde{H}(X,X)>0$\pageoriginale for $X\neq 0$. This follows from
  the fact that $H(U,U)=0$ if and only if $U$ is tangent to the
  fibre. It remains to prove that the exterior form
  $\widetilde{\Omega}$ associated with $\widetilde{H}$ is closed. If
  $\Omega$ is the form associated with $H$,
\begin{align*}
-2i\Omega &= \sum dz_{k}\wedge \ob{dz_{k}}-\omega\wedge \ob{\omega}\\
          &= -d\omega-\omega\wedge \ob{\omega}
\end{align*}
$\Omega=\pi^{-1}\widetilde{\Omega}$. But on $S^{2n+1}$,
$\ob{\omega}=-\omega$; for, the relation 
$$
\sum \ob{z}_{k}z_{k}=1
$$
yields
$$
\sum(\ob{d}z_{k}z_{k}+\ob{z}_{k}dz)=0
$$
Consequently $-2i\Omega=d\omega$; so $\Omega$ is closed.
\end{itemize}

Now $d\Omega$ is the reciprocal image of $d\widetilde{\Omega}$;
therefore $\widetilde{\Omega}$ is also closed. This proves that
$PC^{n}$ is K\"ahlerian.

It may be remarked that even though $\Omega$ is a coboundary,
$\widetilde{\Omega}$ is not a coboundary, (as $PC^{n}$ is
K\"ahlerian!); $\omega$ is not a reciprocal image.

Another method to prove that $PC^{n}$ is K\"ahlerian would be to
consider $PC^{n}$ as symmetric space with respect to the unitary
group, $U^{n+1}$. Since $U^{n+1}$ is compact we may construct an
invariant Hermitian metric by the averaging process. The associated
$2$-form $\Omega$ will be an invariant form. But in a symmetric space
any invariant form is closed.

\vskip 1cm

\begin{center}
{\bf BIBLIOGRAPHY}
\end{center}

We\pageoriginale do not give here an exhaustive bibliography on the
subject. We mention some articles and books which may help the reader
of these notes to obtain some additional information on the subject or
which contain other methods of exposition.

Differentiable manifolds may be studied in (7), (18); differential
forms, exterior differentiation and Stokes' formula, in (7), (15),
(18); Poincar\'e's theorem on differential forms in $R^{N}$, in (3.\@
p.72), (16), (18); Currents and distributions, in (18), (19); de
Rham's theorems and\break Poincar\'e's duality theorem, in (5-a), (16),
(18), (27); harmonic forms, in (10), (11), (17), (18); elliptic
character of $\Delta$, in (18); Hahn-Banach theorem and Banach's
closed graph theorem, in (2); K\"ahlerian manifolds in (5-b), (11),
(12), (17), (24), (26), (28); the osculating Euclidean metric in (4.\@
p.\@ 90), (5-b); Cousin's problem, in (5-b), (17), (21), (22), (25);
$\ob{z}$ cohomology, in (9), (12), (13), (21), (22), (24); Riemann
surfaces and the Riemann-Roch theorem, in (8), (14), (21), (29).

\begin{thebibliography}{99}
\bibitem{1} {\em S.\@ Bochner\pageoriginale and W.\@ T.\@ Martin:} Several Complex
  Variables, Princeton University Press, 1948.

\bibitem{2} {\em N.\@ Bourbaki:} Espaces vectoriels topologiques,
  Paris, Hermann, 1953.

\bibitem{3} {\em E.\@ Cartan:} Le\c{c}ons sur les invariants int\'egraux,
  Paris, Gauthiers-Villars, 1922.

\bibitem{4} {\em E.\@ Cartan:} Le\c{c}ons sur la g\'eom\'etrie des
  espaces de Riemann, Paris, Gauthiers-Villars, 1946.

\bibitem{5} {\em H.\@ Cartan:} S\'eminaire de l'Ecole Normale
  Superieure:
\begin{itemize}
\item[(a)] 1950-51

\item[(b)] 1951-52

\item[(c)] 1953-54
\end{itemize}

\bibitem{6} {\em H.\@ Cartan and J.\@ P.\@ Serre:} Un th\'eor\`eme de
  finitude concernant les vari\'et\'es analytiques compactes, Comptes
  Rendus de l'Acad\'emie des Sciences (Paris), 237(1953) pp.\@
  128-130.

\bibitem{7} {\em C.\@ Chevalley:} Theory of Lie Groups, Princeton
  University Press, 1946.

\bibitem{8} {\em C.\@ Chevalley:} Introduction to the theory of
  algebraic functions of one variable, Americal Mathematical Society, 1951.

\bibitem{9} {\em P.\@ Dolbeault:} Sur la cohomologie des vari\'et\'es
  analytiques complexes, Comptes Rendus de l'Acad\'emie des Sciences,
  Paris, 236 (1935), pp.\@ 175-177; pp.\@ 2203-2205.

\bibitem{10} {\em G.\@ F.\@ D.\@ Duff and D.\@ C.\@ Spencer:} Harmonic
  tensors on Riemannian manifolds with boundary, Annals of
  Mathematics, 56(1952) pp.\@ 128-156.

\bibitem{11} {\em W.\@ V.\@ D.\@ Hodge:}\pageoriginale The theory and
  applications 
  of harmonic integrals, Cambridge University Press, 1952.

\bibitem{12} {\em W.\@ V.\@ D.\@ Hodge:} Differential forms on
  K\"ahler manifolds, Proceedings of the Cambride Philosophical
  Society, 47 (1951), pp.\@ 504-517.

\bibitem{13} {\em K.\@ Kodaira:} On a differential geometric method in
  the theory of analytic stacks, Proceedings of the National Academy
  of Sciences (Washington), 39 (1953), pp.\@ 1268-73.

\bibitem{14} {\em K.\@ Kodaira:} Some results in the transcendental
  theory of algebraic varieties, Annals of Mathematics, 59 (1954),
  pp.\@ 86-134.

\bibitem{15} {\em A.\@ Lichnerovicz:} Alg\`ebre et Analyse
  Lin\'eaires, Paris, Masson, 1947.

\bibitem{16} {\em G.\@ de Rham:} Sur l'analysis situs des vari\'et\'es
  \`a $n$ dimensions, Journal de Math\'ematiques pures et appliqu\'ees,
  10 (1931), 115-120.

\bibitem{17} {\em G.\@ de Rham and K.\@ Kodaira:} Harmonic Integrals,
  Institute for Advanced Study, Princeton, 1950.

\bibitem{18} {\em G.\@ de Rham:} Vari\'et\'es diff\'erentiables
  (Formes, courants, formes harmoniques), Paris, Hermann, 1955.

\bibitem{19} {\em L.\@ Schwartz:} Theorie des Distributions, Vol.\@ I,
  II, Paris, Hermann, 1950-51.

\bibitem{20} {\em L.\@ Schwartz:} Homomorphismes et applications
  compl\`etement continues, Comptes Rendus de l'Acad\'emie des
  Sciences (Paris), 236 (1953), pp.\@ 2472-74.

\bibitem{21} {\em L.\@ Schwartz:}\pageoriginale Courant attach\'e \`a
  une forme diff\'erentielle meromorphe sur une vari\'et\'e analytique
  complexe, Colloque de G\'eom\'etrie Diff\'erentielle, Strasbourg,
  1953, pp.\@ 185-195.

\bibitem{22} {\em J.\@ P.\@ Serre:} Quelques probl\`emes globaux
  relatifs aux vari\'et\'es de Stein, Colloque sur les functions
  plusieurs variables complexes, Bruxelles, 1953, pp.\@ 57-68.

\bibitem{23} {\em J.\@ P.\@ Serre:} Un th\'eor\`eme de dualit\'e,
  Commentarii Mathematici Helvetici, 29 (1955), pp.\@ 9-26.

\bibitem{24} {\em D.\@ C.\@ Spencer:} Real and complex operators on
  Manifolds, Contributions to the theory of Riemann surfaces,
  Princeton University Press, 1951, pp.\@ 203-227.

\bibitem{25} {\em K.\@ Stein:} Analytische Funktionen mehrerer
  kumplexer Veranderlichen zu vorgegebenen Periodizit\"atsmoduln und
  das zweite Cousinsche Problem, Mathematische Annalen, 123 (1951),
  pp.\@ 201-222.

\bibitem{26} {\em A.\@ Weil:} Sur la th\'eorie des formes
  diff\'erentialles attach\'ees \`a une vari\'et\'e analytique
  complexe. Commentarii Mathematici Helvetici, 20 (1947), pp.\@ 110-116.

\bibitem{27} {\em A.\@ Weil:} Sur les th\'eor\`emes de de Rham,
  Commentarii Mathematici Helvetici, 26 (1952), pp.\@ 119-145.

\bibitem{28} {\em A.\@ Weil:} Theorie der K\"ahlerschen
  Mannigfaltigkeiten, Universit\"at G\"ottingen, 1953.

\bibitem{29} {\em H.\@ Weyl:} Die Idee der Riemannschen Fl\"ache,
  Leipzig, 1913 and 1923; Stuttgart, 1955.

\end{thebibliography}
