\chapter{Lecture 25}

\section*{Some applications}\pageoriginale

Before proving the Riemann-Roch theorem we shall make some
applications of the existence theorems proved above.

We shall first prove that every compact connected Riemann Surface,
$V$, of genus zero is analytically homeomorphic to the Riemann sphere,
$S^{2}$. Since the genus of $V$ is zero, there are no compatibility
conditions on any Cousin's problem for meromorphic functions. So we
can construct a meromorphic function, $f$, having a simple pole at a
point $\ub{a}$ and regular elsewhere. Every meromorphic function on a
compact (connected) Riemann surface assumes every value in $S^{2}$ the
same number of times. So $f$ assumes every value in $S^{2}$ exactly
once and maps $V$ conformally onto $S^{2}$.

In the case of a torus $(g=1)$ there is no distinction between
meromorphic functions and meromorphic forms of degree $1$ (because of
the existence of $dz$). Evidently, the problem of finding an elliptic
function with prescribed periods and singularities is a problem of
finding a meromorphic function or a meromorphic differential of degree
$1$ with prescribed singularities on a torus. Therefore our theorem
yields the existence of elliptic functions for which singularities are
prescribed in the fundamental parallelogram with the restriction that
the sum of the residues at the singularities is zero. The function is
determined uniquely upto an additive constant by the singularities. In
particular if we prescribe the principal part\pageoriginale $1/z^{2}$
at the origin and choose the constant term in the Laurent development
at the origin to be zero, we obtain the Weierstrass
$\mathscr{P}$-function.

\section*{The Riemann-Roch Theorem}

Let $V^{(1)}$ be a compact connected Riemann Surface of genus $g$. A
divisor $D$ on $V$ is a formal linear combination of points of $V$
with integer coefficients such that all but a finite number of
coefficients are zero:
$$
D=\sum\alpha_{p}p-\sum\beta_{q}q;
$$
$p$, $q$ are points of $V$, $\alpha_{p}$, $\beta_{q}$ integers,
$\alpha_{p}>0$, $\beta_{q}>0$. The degree of the divisor $D$ is
defined to be the integer $(\sum\alpha_{p}-\sum\beta_{q})$. We write
$A=\sum\alpha_{p}$, $B=\sum\beta_{q}$. Meromorphic function $f$ on $V$
is said to be a multiple of the divisor $D$ if at every point $\ub{p}$
$f$ has a zero of order $\geq \alpha_{p}$ and at every point $q$, $f$
has a pole of order $\leq \beta_{q}$. For example in $C^{1}$ a
function $f$ is a multiple of the divisor $D$ if and only if
$$
f=h\prod (z-p)^{\alpha_{p}}(z-q)^{-\beta}q
$$
where $h$ is an entire function. To find a meromorphic function which
is a multiple of a given divisor is to find a meromorphic function for
which the maximum number of poles with the maximum order at each pole
and the minimum number of zeros with the minimum order at each zero
are prescribed. Since the order of a zero or a pole of a\pageoriginale
meromorphic differential form of degree $1$ has an invariant meaning
we can similarly define what it means to say that a differential form
is a multiple of $D$. (The order is defined by means of a map). Let
$\overset{0}{m}(D)$ denote the dimension of the vector space of the
meromorphic function which are multiples of $D$ and
$\overset{1}{m}(D)$ that of the vector space of the meromorphic
differential forms which are multiples of $D$.

The Riemann-Roch theorem asserts that
$$
\boxed{\overset{0}{m}(-D)-\overset{1}{m}(D)=d-g+1}
$$
($d$ is the degree of the divisor $D$; $-D$ is the inverse of $D$).

To prove the Riemann-Roch theorem let us first consider {\em the
  meromorphic functions which are multiples of $-D$.} The singular
part of such a function at a point $\ub{p}$ can be written in a map as
$$
f_{p}=\sum_{j\leq \alpha_{p}}C_{j,p}1/z^{j}_{p}
$$
($z_{p}$ is the function defining the map). The singular parts $f_{p}$
give a Cousin's datum. If we denote by $[\dfrac{1}{z^{j}_{p}}]$ a
pseudo-solution (which depends on the choice of $U_{i}$ and the
metric) associated with the Cousin datum $\dfrac{1}{z^{j}_{p}}$, a
pseudo-solution associated with the Cousin datum $f_{p}$ is
$$
\omega=\sum_{p,j\leq \alpha_{p}}C_{j,p}[\frac{1}{z^{j}_{p}}]+C
$$
and\pageoriginale this gives the general (true) solution when the
Cousin problem has a solution.

As conditions for the Cousin datum to be compatible we obtain
$$
-\sum_{p,j\leq\alpha_{p}}C_{j,p}\Res_{p}(h_{\nu}/z^{j}_{p})=0,(\nu=1,2,\ldots,g)
$$
where $\{h_{\nu}\}$ is a basis for the space of holomorphic forms of
degree $1$. (The minus sign on the left side is introduced for later
convenience). To write the condition that the solution has a zero of
order $\geq \beta_{q}$ at $q$ it is sufficient to write that the
product of $\omega$ by $dz_{q}/z^{k}_{q}$ has zero residue at $q$ for
$k\leq \beta_{q}$. This residue exists if $\omega$ is a meromorphic
form, otherwise it is defined by
$$
\frac{1}{2i\pi}\int\limits_{bU_{q}}\omega \frac{dz_{q}}{z^{k}_{q}}
$$
So we obtain the equations
$$
\sum_{p,j\leq
  \alpha_{p}}C_{j,p}\frac{1}{2i\pi}\int\limits_{bU_{q}}\frac{dz_{q}}{z^{k}_{q}}[\frac{1}{z^{j}_{p}}]+\frac{C}{2i\pi}\int\limits_{bU_{q}}\frac{dz_{q}}{z^{k}_{q}}=0 
$$
for every $q$ and for every $k\leq\beta_{q}$. We have here $A+1$
unknowns, $C_{j,p}$ and $C$, and $g+B$ equations:
\begin{gather*}
g(\nu=1,2,\ldots,g)\cdot B(q,k\leq \beta_{q}).\\
\left(A=\sum\alpha_{p}B=\sum\beta_{q}\right)
\end{gather*}

Next\pageoriginale let us consider {\em the meromorphic forms of
  degree $1$ which are multiples of $D$.} The singular part of such a
differential in a map at $q$ can be written as
$$
\sum_{q,k\leq \beta_{q}}d_{k,q}\frac{dz_{q}}{z^{k}_{q}}
$$
A general pseudo-solution of the associated Cousin problem is
$$
\sum_{q,k\leq \beta_{q}}d_{k,q}\left[\frac{dz}{z^{k}_{q}}\right]+\sum
e_{\nu}h_{\nu} 
$$
If we write down the condition for the solution of the Cousin problem
to exist, we obtain
$$
\sum_{q,k\leq
  \beta_{q}}d_{k,q}\Res_{q}\left[\frac{dz_{q}}{z^{k}_{q}}\right]=0. 
$$
That the solution should have a zero of order $\geq \alpha_{p}$ at
$\ub{p}$ gives the conditions
$$
-\left\{\sum_{q,k}d_{k,q}\frac{1}{2i\pi}\int\limits_{bU_{p}}\frac{1}{z^{j}_{p}}\left[\frac{dz_{q}}{z^{k}_{q}}\right]+\sum
e_{\nu}\Res_{p}(h_{\nu}/z^{j}_{p})\right\}=0. 
$$
In this case we have $B+g$ unknowns, $d_{k,q}$ and $e_{\nu}$, and
$A+1$ equations (compatibility condition and $A(p,j\leq \alpha_{p})$
equations). We shall now show that the systems of equations 
$$
(I)
\begin{cases}
  -\sum C_{j,p}\frac{1}{2\pi
    i}\Res_{p}(h_{\nu}/z^{j}_{p})=0\,(\nu=1,\ldots,g),\\
  \sum\limits_{p,j}C_{j,p}\frac{1}{2\pi i} \int\limits_{bU_{q}}
  \frac{dz_{q}}{z^{k}_{p}} 
  \left[\frac{1}{z^{j}_{p}}\right]\frac{C}{2\pi i} 
  \int\limits_{bU_{q}}\frac{dz_{q}}{z^{k}_{q}}=0
\end{cases}
$$\pageoriginale
and
$$
(II)
\begin{cases}
\sum d_{k,q}\Res_{q}\left[\frac{dz_{q}}{z^{k}_{q}}\right]=0\\
-\left\{\sum\limits_{q,k\leq
  \beta_{q}}d_{k,q}\frac{1}{2i\pi}\int\limits_{bU_{p}}\frac{1}{z^{j}_{p}}
\left[\frac{dz_{q}}{z^{k}_{q}}\right]+\sum
  e_{\nu}\Res_{p}\left(\frac{h_{\nu}}{z^{j}_{p}}\right)\right\}=0 
\end{cases}
$$
are transposes of each other. We have only to verify (the rest of the
verification being trivial) that
$$
-\int\limits_{bU_{p}}\frac{1}{z^{j}_{p}}\left[\frac{dz_{q}}{z^{k}_{q}}\right]=\int\limits_{bU_{q}}\frac{dz_{q}}{z^{k}_{q}}\left[\frac{1}{z^{j}_{p}}\right] 
$$
Now,
\begin{align*}
\left[\frac{1}{z^{j}_{p}}\right] &=
\frac{\widetilde{1}}{z^{j}_{p}}+2\partial_{\ob{z}}G\left(\frac{bU_{p}}{z^{j}_{p}}\right)\\ 
\left[\frac{dz_{q}}{z^{k}_{q}}\right] &=
\frac{\widetilde{dz}}{z^{k}_{q}}+2\partial_{\ob{z}}G\left(\frac{bU_{q}\wedge
  dz_{q}}{z^{k}_{q}}\right) 
\end{align*}
{\em (Refer to the last lecture).}

Since $\dfrac{\widetilde{1}}{z^{j}_{p}}=0$ on $bU_{q}$ and
$\dfrac{\widetilde{dz}}{z^{k}_{p}}=0$ on $bU_{p}$ is remains only to
verify\pageoriginale that
\begin{align*}
& -\int\limits_{bU_{p}}\frac{1}{z_{p}^{j}}2\partial_{\ob{z}} G\frac{(bU_{q}\wedge
  dz_{q})}{z_{q}^{k}}\\
&=\int\limits_{bU_{q}}\frac{dz_{q}}{zq^{k}}2\partial_{\ob{z}}G\left(bU_{p}\frac{1}{z_{p}^{j}}\right) 
\end{align*}
or
\begin{align*}
& -\langle (bU_{p})\frac{1}{z_{p}^{j}},
2\partial_{\ob{z}}G\left(bU_{q}\wedge\frac{dz_q}{z^{k}_{q}}\right)\rangle\\
&=\langle 2\partial_{\ob{z}}G\left(bU_{p}\cdot
\frac{1}{z_{p}^{j}}\right), bU_{q}\wedge \frac{dz_{q}}{z^{k}_{q}}\rangle.
\end{align*}
If we put
$$
S=bU_{p}\cdot \frac{1}{z_{p}^{j}}\quad\text{and}\quad T=bU_{q}\wedge
\frac{dz_{q}}{z^{k}_{q}}.
$$
We have to verify that
$$
-\langle
\overset{1}{S},\partial_{\ob{z}}\overset{1}{G}T\rangle=\langle
\overset{2}{T},\partial_{\ob{z}}\overset{0}{G}S\rangle
$$
Write
$$
2\partial_{\ob{z}}GS=U,\quad 2\partial_{\ob{z}}GT=V
$$
We have then
$$
S=\pi_{1}S+d_{\ob{z}}U, T=\pi_{1}T+d_{\ob{z}}V.
$$
(as $S$ and $T$ are $\ob{z}$ closed). So we have to verify that
$$
-\langle\pi_{1}S+d_{\ob{z}}U,V\rangle=\langle
U,\pi_{1}T+d_{\ob{z}}V\rangle. 
$$
Now\pageoriginale $\langle\pi_{1}S,V\rangle=0$ as $\pi_{1}S$ is
harmonic and $V$ is a $\partial_{\ob{z}}$ coboundary. Similarly
$\langle U,\pi_{1}T\rangle=0$. It remains to show that
$$
-\langle d_{\ob{z}}U,V\rangle=\langle d_{\ob{z}}V,U\rangle.
$$

We define the set of singularities of a current $T$ to be the smallest
closed subset of the manifold in the (open) complement of which $T$ is
an indefinitely differentiable form (such a set exists, for, if $T$,
is a form in a family of open subsets then it is a form in their
union). Now let $S$ and $T$ be two currents whose sets of
singularities have no common point; then it is possible to give a
meaning to $\langle S,T\rangle$. It is possible to find
decompositions. 
$$
S=S_{1}+S_{2}, T=T_{1}+T_{2}
$$
where $S_{2}$ and $T_{2}$ are $C^{\infty}$ forms and $S_{1}$ and
$T_{1}$ are currents whose supports have no common point. (For
example, let $1=\alpha+\beta$ be a partition of unity subordinate to
$\complement F$, $\complement G$ where $F$ and $G$ are the supports of
$S$ and $T$ respectively. We may take $S_{1}=\beta S$,  $S_{2}=\alpha
S$, $T_{1}=T_{2}=T$). {\em By definition} we put:
$$
\langle S,T\rangle=\langle S_{1},T_{2}\rangle+\langle
S_{2},T_{1}\rangle+\langle S_{2},T_{2}\rangle;
$$
each bracket on the right side has a meaning since $S_{2}$ and $T_{2}$
are $C^{\infty}$ forms ($\langle S_{1},T_{1}\rangle$ is not written;
we take it by definition to be zero which is natural because the
supports of $S$ and $T$ have no common point). This definition is
correct provided we prove i) it\pageoriginale is independent of the
choice of the choice of the decompositions of $S$ and $T$ and ii) it
gives the usual result if $S$ or $T$ is a form. Suppose we have
another decomposition $S'_{i}$, $T'_{i}(i=1,2)$. It is sufficient to
prove that the decompositions $S_{i}$, $T_{i}$ and $S_{i}$, $T'_{i}$
give the same result; for then the decompositions $S_{i}$, $T'_{i}$
and $S'_{i}$, $T'_{i}$ will also give the same result for an analogous
reason. We have to show that
$$
\langle S_{1},T'_{2}-T_{2}\rangle+\langle
S_{2},T'_{1}-T_{1}\rangle+\langle S_{2},T'_{2}-T_{2}\rangle
$$
is zero. The last two brackets can be added by linearity because
$S_{2}$ is a $C^{\infty}$ form and their sum is $\langle
S_{2},0\rangle=0$. It remains to prove that $\langle
S_{1},T'_{2}-T_{2}\rangle=0$. We shall add to $\langle S_{1},
T'_{2}-T_{2}\rangle$ the expression $\langle
S_{1},T'_{1}-T_{1}\rangle$ which has a meaning because $T'_{1}-T_{1}$
is a form ($(T'_{1}-T_{1})+(T'_{2}-T_{2})$ is the $0$ form and
$T'_{2}-T_{2}$ is a form) and which is equal to zero since the
supports of $S_{1}$ and $T'_{1}-T_{1}$ have no common point. But now
the sum $\langle S_{1},T'_{1}-T_{1}\rangle+\langle
S_{1},T'_{2}-T_{2}\rangle$ is equal, by linearity, to $\langle
S,0\rangle=0$. So $\langle S_{1},T'_{2}-T_{2}\rangle=0$. This proves
(i). (ii) is trivial because if $S$ is a form we may take the
decomposition $S=0+S$; $T=T+0$. Now we have proved the existence of
the expression when the sets of singularities of $S$ and $T$ have no
common point. (We did not pay any attention to the question of
supports, assuming the manifold to be compact; if the manifold is not
compact, we have to assume further that the intersection of the
supports of\pageoriginale $S$ and $T$ is compact). That the sets of
singularities of $S$ and $T$ have no common point can be expressed in
the following way: every point of the manifold has an open
neighbourhood in which one at least of the currents $S$ and $T$ (not
necessarily the same one) is a form. In this case the properties of
differentiation such as the formula
$$
\langle d_{\ob{z}}S,T\rangle =(-1)^{p+1}\langle S,d_{\ob{z}}T\rangle
$$
($S$ is of total degree $p$) remain obviously valid as is seen
immediately by means of a decomposition $S_{i}$, $T_{i}$ of $S$,
$T$. In our case since the sets of singularities of $U$ and $V$ have
no common point 
\begin{align*}
& -\langle d_{\ob{z}}U,V\rangle=\langle d_{\ob{z}}V,U\rangle\\
\text{or}\qquad &-\langle S,\partial_{\ob{z}}GT\rangle=\langle
T,\partial_{\ob{z}}GS\rangle 
\end{align*}
Thus the systems (I) and (II) are transposes of each other. Two
systems which are transposes of each other have the same rank $\rho$,
$\rho=$ number of unknowns - degree of indeterminacy. So we have
$$
(A+1)-\overset{0}{m}(-D)=(B+g)-\overset{1}{m}(D)
$$
or
\begin{align*}
\overset{0}{m}(-D)-\overset{1}{m}(D) &= (A-B)-g+1\\
 &= d-g+1.
\end{align*}
This completes the proof of the Riemann-Roch theorem.

Let $D$ be a positive divisor, $D=\sum\alpha_{p}p$, $\alpha_{p}\geq
0$. All the differentials of degree $1$ which are multiples of $D$ are
holomorphic. Now an holomorphic differential form of degree $1$ has
exactly $2g-2$ zeros. (This can be proved by using Poincar\'e's
theorem on\pageoriginale vector fields or deduced from the Riemann
Roch theorem). Consequently if $\sum \alpha_{p}>2g-2$,
$\overset{1}{m}(D)=0$ and
$$
\overset{0}{m}(-D)=d-g+1\,(d>2g-2)
$$
\iec if we give a sufficient number of poles the $g$ conditions of
compatibility are all independent.


