\chapter{Lecture 9}

\section*{Topology on $\mathscr{E}$, $\mathscr{E}'$, $\mathscr{D}$,
  $\mathscr{D}'$}\pageoriginale

On $\mathscr{E}'$ we introduce the weak topology. The space
$\mathscr{D}$ and $\mathscr{D}'$ are in duality. In each of the spaces
$\mathscr{D}$ and $\mathscr{D}'$ we introduce the weak topology (which
is a locally convex topology) defined by the other; the topologies on
$\mathscr{D}$ and $\mathscr{D}'$ are Hausdorff.

We remark that if $F$ is a topological vector space and $F'$ its dual
with the weak topology, then the dual of $F'$ is $F$.

\section*{de Rham's Theorem}

\noindent
{\bf The first part of de Rham's theorem.}
\smallskip

The first part of de Rham's theorem gives canonical isomorphisms
between the cohomology vector spaces of $\mathscr{E}$,
$\widetilde{\mathscr{E}}^{m}$, $\mathscr{D}'$ and
$\widetilde{\mathscr{D}'}^{m}$ and also canonical isomorphisms between
the cohomology vector space of $\mathscr{D}$,
$\widetilde{\mathscr{D}}^{m}$, $\mathscr{E}'$ and
$\widetilde{\mathscr{E}'}^{m}$. 

For instance, let us consider $\mathscr{E}(V)$ and
$\mathscr{D}'(V)$. We define the canonical isomorphism between
$H^{p}(\mathscr{E}(V))$ and $H^{p}(\mathscr{D}'(V))$. If
$\overset{p}{\omega_{1}}\in \overset{p}{\mathscr{E}}(V)$ with
$d\overset{p}{\omega_{1}}=0$ it determines a closed current,
$\overset{p}{\omega_{1}}$ of degree $p$ (as the coboundary operator
for currents is an extension of the coboundary operator for forms). If
$\overset{p}{\omega_{1}}$ and $\overset{p}{\omega_{2}}$ are
cohomologous (\iec there is a $p-1$ form
${}^{p}\ob{\widetilde{\omega}}^{1}$ with
$\overset{p}{\omega_{1}}-\overset{p}{\omega_{2}}=d\overset{p-1}{\omega}$)
the currents $\overset{p}{\omega_{1}}$ and $\overset{p}{\omega_{2}}$
are also cohomologous. So we have in fact a linear map of $H^{p}$
($\mathscr{E}(V)$ into $H^{p}(\mathscr{D}'(V))$). The first part of de
Rham's theorem asserts that this mapping\pageoriginale is an
isomorphism. That this mapping is $(1,1)$ means that if a form
considered as a current is the coboundary of a current it is also the
coboundary of a differential form. That the map is onto means that in
each cohomology class of currents there exists a current defined by a
differential form (that is, we have no other cohomology classes of
currents than the ones given by closed differential forms).

We call the cohomology spaces given by any one of the complexes
$\mathscr{E}$, $\widetilde{\mathscr{E}}^{m}$, $\mathscr{D}'$,
$\widetilde{\mathscr{D}'}^{m}$ the cohomology spaces with arbitrary
supports. The cohomology spaces given by any one of the complexes
$\mathscr{D}$, $\widetilde{\mathscr{D}}^{m}$, $\mathscr{E}'$,
$\widetilde{\mathscr{E}'}^{m}$ are called the cohomology spaces with
compact supports. We shall denote by $H^{p}(V)$ and $H^{p}_{c}(V)$ the
$p$th cohomology spaces with arbitrary supports and compact supports
respectively and by $b^{p}$ and $b^{p}_{c}$ the dimensions of $H^{p}$
and $H^{p}_{c}$.

In this connection we shall give an example of a natural homomorphism
which is not an isomorphism in general. We have an obvious linear map
from $H^{p}(\mathscr{D}(V))$ to $H^{p}(\mathscr{E}(V))$. In general
this map is neither $(1,1)$ nor onto. It may happen that a $p$-form
$\overset{p}{\omega}$ with compact support is the coboundary of some
$p-1$ form but not the coboundary of a $p-1$ form with compact support
(as in the case of $R^{N}$, $p=N$); and it may happen that there are
cohomology classes of $p$-forms which contain no forms with compact
support.

\medskip
\noindent
{\bf The second part of de Rham's theorem: the theorem of closure.}
\smallskip

In each of the spaces $\mathscr{D}$, $\mathscr{E}$ etc., the space of
co-cycles is closed: for if $\omega_{j}\to \omega$ then
$d\omega_{j}\to d\omega$. The theorem of closure asserts that in each
of these spaces the space of coboundaries is also closed.

\medskip
\noindent
{\bf Some consequences of the theorem of closure}\pageoriginale
\smallskip

Let $F$ be a locally convex topological vector space and $F'$ its dual
endowed with the weak topology. Suppose $G$ is a linear sub-space of
$F$. Let $G^{\circ}$ be the subspace of $F'$ orthogonal to $G$ (An element
of $G^{0}$ is a linear form $T$ on $F$ such that $\langle
T,\varphi\rangle=0$ for every $\varphi\in G$). Let $G^{\circ\circ}$ be the
subspace of $F$ orthogonal to $G^{0}$. $G^{00}$ is the biorthogonal of
$G$. It is a simple consequence of Hahn-Banach theorem that
$G^{00}=\ob{G}$. Similarly the biorthogonal of a sub-space $G$ of $F'$
is $\bar{G}$.

Let further $H$ and $G$ be subspaces of $F$ such that $H\subset G$;
suppose $H$ is closed in $F$. It can be shown that $H^{0}/G^{0}$ is
canonically the topological dual of $G/H$ and conversely. ($H^{0}$ and
$G^{0}$ are subspaces of $F'$ orthogonal to $H$ and $G$ respectively).

These general considerations along with the closure theorem lead to
two interesting consequences.
\begin{enumerate}
\renewcommand{\theenumi}{\roman{enumi}}
\renewcommand{\labelenumi}{\theenumi)}
\item {\bf Orthogonality relation}

Let $Z$ be the space of cocycles and $B$ the space of coboundaries in
$\mathscr{D}^{N-p}$. Let further $Z'$ be the space of cocycles and
$B'$ the space of coboundaries in ${\mathscr{D}'}^{p}$. It is trivial
to see that $Z'$ is the orthogonal space of $B$. For if $dT=0$ then 
$$
\langle T,d\varphi\rangle=\pm \langle dT,\varphi\rangle =0,
$$
conversely if $\langle T,d\varphi\rangle=0$ for every $\varphi\in
\overset{N-p-1}{\mathscr{D}}$, 
$$
\langle dT,\varphi\rangle=\pm \langle T,d\varphi\rangle =0
$$
for\pageoriginale every $\varphi\in \overset{N-p-1}{\mathscr{D}}$ so
that $dT=0$. It is also true that the orthogonal space of $Z$ is
$B'$. To prove this we first notice that the orthogonal space of $B'$
is $Z$. For, if $\varphi\in Z$ 
$$
\langle dS,\varphi\rangle=\pm \langle S,d\varphi\rangle
$$
as $d\varphi=0$; conversely if $\langle dS,\varphi\rangle=0$ for every
$S\in \overset{p-1}{\mathscr{D}'}$ then $\langle S,d\varphi\rangle=0$
for every $S\in\overset{p-1}{\mathscr{D}'}$ so that $d\varphi=0$. So
the biorthogonal of $B'$ is the orthogonal space of $Z$. But the
biorthogonal of $B'$ is the closure of $B'$ and by the closure theorem
$\ob{B}'=B'$. So the orthogonal space of $Z$ is $B'$.

Thus the orthogonal space of the space of cocycles is the space of
coboundaries and the orthogonal space of the space of coboundaries is
the space of cocycles (in $\mathscr{D}$ and $\mathscr{D}'$ and so on).

\item {\bf Poincar\'e's duality theorem}

By the general result on the topological vector spaces stated above it
follows that $H^{p}=Z'/B'$ is canonically the dual of
$H^{N-p}_{c}=Z/B$. Thus the $p$th cohomology vector space with
arbitrary supports is canonically the (topological) dual of the
$(N-p)$ th cohomology vector space with compact supports. This is the
Poincar\'e duality theorem.

Suppose $H^{N-p}(\mathscr{D}(V))=H^{N-p}_{c}$ is finite
dimensional. Since the topology on $H^{N-p}_{c}$ is Hausdorff it is
the usual topology on $R^{b_{c}^{N-p}}$. So the topological dual and
the algebraic dual of $H^{N-p}_{c}$ are the same. So $H^{p}$ is the
algebraic dual of $H^{N-p}_{c}$. Consequently
$$
b^{p}=b^{N-p}_{c}.
$$
It is to be remarked that $H^{p}$ and $H^{N-p}_{c}$ are not
canonically isomorphic but only canonically dual.
\end{enumerate}
