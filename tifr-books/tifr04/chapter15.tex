\chapter{Lecture 15}

\section*{Decomposition of $\mathscr{D}'$}\pageoriginale

We shall now extend the operator $G$ to $\mathscr{D}'$. For a current
$T$ we define $GT$ by:
$$
(GT,\varphi)=(T,G\varphi).
$$
(This definition is consistent) Since $G$ is continuous on
$\mathscr{D}$, it is easy to verify that $GT$ is continuous on
$\mathscr{D}$. For if $\varphi_{j}\to 0$ in $\mathscr{D}$,
$G\varphi_{j}\to 0$ and $(GT,\varphi_{j})=(T,G\varphi_{j})\to 0$. The
operator $G$ is continuous on $\mathscr{D}'$ endowed with the weak
topology. For if $T_{j}\to 0$ in the weak topology, for a fixed
$\varphi\in\mathscr{D}$ $(GT_{j},\varphi)=(T_{j},G\varphi)\to 0$. We
define the operators $\pi_{1}$, $\pi_{2}$ and $\pi_{3}$ on
$\mathscr{D}'$ by:
\begin{gather*}
\pi_{1}=I+\Delta G\\
\pi_{2}=d\partial G,\pi_{3}=\partial dG.
\end{gather*}
The operators $\pi_{1}$, $\pi_{2}$ and $\pi_{3}$ verify the relations:
\begin{gather*}
\pi_{i}\pi_{j}=0\text{ \  for \ } i \neq j\\
\pi^{2}_{i}=\pi_{i}\\
I = \pi_{1}+\pi_{2}+\pi_{3}.
\end{gather*}
$\mathscr{D}'$ is the direct sum of $\mathscr{D}'_{1}$ (the space of
harmonic forms), $\mathscr{D}'_{2}=d\mathscr{D}'$ and
$\mathscr{D}'_{3}=\partial \mathscr{D}'$. For a current $T$ we have
the decomposition formula 
$$
T=\pi_{1}T+d\partial GT+\partial dGT.
$$

\section*{Commutativity of an operator with $\Delta$, $\pi_{1}$ and
  $G$}\pageoriginale

If an operator $A:\mathscr{D}\to \mathscr{D}$ commutes with $\Delta$
it also commutes with $\pi_{1}$ and $G$. If $A\Delta\omega=\Delta
A\omega$ we have to show that $\pi_{1}A\omega=A\pi_{1}\omega$ and
$GA\omega=AG\omega$. If $\pi_{1}A\omega=\widetilde{\omega}$,
$\widetilde{\omega}$ is characterised by the properties:
\begin{itemize}
\item[i)] $\Delta\widetilde{\omega}=0$

\item[ii)] $A\omega-\widetilde{\omega}=\Delta \eta$ for some
  $\eta\in\mathscr{D}$. 
\end{itemize}
We shall verify that $\widetilde{\omega}=A\pi_{1}\omega$ also
possesses these properties.
$$
\Delta\widetilde{\omega}=\Delta A\pi_{1}\omega=A\Delta \pi_{1}\omega=0
$$
as $\pi_{1}\omega$ is a harmonic form;
\begin{align*}
A\omega -\widetilde{\omega} &= A\omega-A\pi_{1}\omega\\
&= A(\omega-\pi_{1}\omega)\\
&= A(\Delta\eta')\\
&= \Delta(A\eta').
\end{align*}
So $\pi_{1}A\omega=A\pi_{1}\omega\cdot\omega_{1}=GA\omega$ is
characterised by:
\begin{itemize}
\item[i)] $-\Delta\omega_{1}=A\omega-\pi_{1}A\omega$

\item[ii)] $\pi_{1}\omega_{1}=0$
\end{itemize}
We shall show that $\omega_{1}=AG\omega$ has these properties.
\begin{align*}
-\Delta AG\omega &= -A\Delta G\omega\\
&= A(\omega-\pi_{1}\omega)\\
&= (A\omega-\pi_{1}A\omega)~(\text{as } A \text{ and } \pi_{1}\text{ commute})
\end{align*}\pageoriginale
and 
\begin{align*}
\pi_{1}AG\omega &= A\pi_{1}G\omega\\
&= 0.
\end{align*}

The operator $\Delta$ commutes with each of the operators $d$,
$\partial$, $\Delta$, $\ast$, $\pi_{1}$, $\pi_{2}$, $\pi_{3}$,
$G$. Consequently $\pi_{1}$ and $G$ also commute with each of these
operators.

\section*{Operators on currents as operators on cohomology spaces}

Suppose $A$ is an operator on currents with the following properties:
\begin{enumerate}
\renewcommand{\theenumi}{\roman{enumi}}
\renewcommand{\labelenumi}{\theenumi)}
\item for every cohomology class $\overset{0}{\alpha}$ there exists at
  least one element of the class such that $A\alpha$ is closed.

\item If $\alpha$ is a coboundary and $A\alpha$ is closed, then
  $A\alpha$ is a coboundary. Then we can define $A$ on the cohomology
  vector spaces intrinsically: for each cohomology class
  $\overset{0}{\alpha}$ we choose a representative $\alpha$ such that
  $A\alpha$ is closed and map $\overset{0}{\alpha}$ onto the
  cohomology class determined by $A\alpha$. Of course, this definition
  makes no use of any particular metric.

If there exists at least one Riemannian metric on the manifold for
which $A$ and $\Delta$ commute then the conditions i) and ii) are
verified. In a cohomology class $\overset{0}{\alpha}$ we choose the
harmonic form $\alpha$. Then $A\alpha$ is closed, in fact,
harmonic. $\Delta A\alpha=A\Delta\alpha=0$. To verify the second
condition we notice that a necessary and sufficient condition for a
closed element $\omega$ to be a coboundary is that
$\pi_{1}\omega=0$. If $\alpha$ is a coboundary and $A\,\alpha$ is closed,
\begin{align*}
\pi_{1}A\alpha &= A\pi_{1}\omega~(\text{ as } A\text{ and }
\pi_{1}\text{ commute})\\
&= 0.
\end{align*}\pageoriginale
So $A$ operates {\em intrinsically} on the cohomology spaces.
\end{enumerate}

[We may simply identify the cohomology space with the space of
  harmonic forms and let $A$ operate on the space of harmonic
  forms. (A operates on the space of harmonic forms as $A$ and
  $\Delta$ commute).]

\section*{Complex differential forms on a manifold}

Let $V^{N}$ be a $C^{\infty}$ manifold. Just as we considered real
valued $C^{\infty}$ functions on $V^{N}$ we may also consider
complex-valued $C^{\infty}$ functions on $V^{N}$; a complex valued
$C^{\infty}$ function is of the form $\varphi=f+ig$ where $f$ and $g$
are real valued $C^{\infty}$ functions.

At a point $\ub{a}$ of $V^{N}$ we can define the space of
differentials $T^{\ast}_{a}(V^{N})$ with respect to the complex valued
functions, just the same way we did in the case of real valued
functions. The space $T^{\ast}_{a}(V^{N})$ is of complex dimension
$N$. (Henceforth, $T^{\ast}_{a}(V^{N})$ will always denote the space
of complex differentials at $\ub{a}$. However $T_{a}(V^{N})$ will only
denote the real tangent space at $\ub{a}$). Now if $F$ is a vector
space over the reals, the space $F+iF$ is called the complexification
of $F$; the complex dimension of $F+iF$ is equal to the real dimension
of $F$. We shall always consider $T^{\ast}_{a}(V^{N})$ as the
complexification of the space of real differentials at
$\ub{a}$. Similarly the space $\overset{p}{\Lambda}T^{\ast}_{a}(V)$
will be considered as the complexification of the real tangent
$p$-covectors at $\ub{a}$. Thus an element $\omega\in T^*_{a}(V)$ 
has\pageoriginale the canonical decomposition
$\omega=\omega_{1}+i\omega_{2}$ where $\omega_{1}$ and $\omega_{2}$
are real tangent $p$-covectors at $\ub{a}$.

If a complex vector space is obtained as the complexification of a
real vector space, we have the notion of complex conjugate in this
space. For if $G=F+iF$ is the complexification of the real vector
space $F$ any element $\omega\in G$ has the canonical decomposition
$\omega=\omega_{1}+i\omega_{2}$ where $\omega_{1}$, $\omega_{2}\in F$;
the complex conjugate of $\omega$ is the element
$\omega_{1}-i\omega_{2}$.

The manifold $\overset{p}{\Lambda}T^{\ast}_{N}(V)$ of all
$\overset{p}{\Lambda}T^{\ast}_{a}(V)$ is a $C^{\infty}$ manifold with
real dimension $N+2\binom{N}{p}$.

We extend the operators $d$, $\partial$, $\Delta$ to the complex
differential forms by linearity
$d(\omega_{1}+i\omega_{2})=d\omega_{1}+id\omega_{2}$ and a scalar
product in the space of real differentials at $\ub{a}$ canonically to
a Hermitian scalar product in its complexification \iec in
$T^{\ast}_{a}(V)$. We extend $\ast$ by anti-linearity:
$$
\ast(\omega_{1}+i\omega_{2})=\ast\omega_{1}-i\ast\omega_{2}
$$
so that the relation
$$
(\alpha,\beta)\tau=\alpha\Lambda\ast\beta
$$
is preserved. The space of (complex) square summable forms becomes a
Hilbert space over complex numbers.

