\chapter{Lecture 1}

\section*{Introduction}\pageoriginale

We shall first give a brief account of the problems we shall be considering.

We wish to define a complex analytic manifold, $V^{(n)}$, of complex
dimension $n$ and study how numerous (in a sense to be clarified
later) are the holomorphic functions and differential forms on this
manifold. If $V^{(n)}$ is compact, like the Riemann sphere, we find by
the maximum principle that there are no non-constant holomorphic
functions on $V^{(n)}$. On the contrary if $V^{(n)}=C^{n}$, the space
of $n$ complex variables, there are many non-constant holomorphic
functions. On a compact complex analytic manifold the problem of the
existence of holomorphic functions is trivial, as we have remarked;
but not so the problem of holomorphic differential forms. On a compact
complex manifold there exist a finite number, say $h^{p}$, of linearly
independent holomorphic differential forms of degree $p$. We have to
study the relation between these forms and the algebraic cohomology
groups of the manifold \iec the relation between $h^{p}$ and the
$p^{\text{th}}$ Betti-number $b^{p}$ of the manifold.
It is necessary not only to study holomorphic functions and forms but
also meromorphic functions and meromorphic forms. For instance, it is
necessary to go into Cousin's problem which is a generalization of the
problem of Mittag-Leffler in the plane. [Mittag-Leffler's problem is
  to find in the plane meromorphic functions with
  prescribed\pageoriginale polar singularities and polar
  developments]. We may look for the same problem on manifolds and the
relation between this problem and the topological properties of the
manifold. In the case of Stein manifolds the problem always admits of
a solution as in the case of the complex plane.

We thus observe the great variety of problems which can be
examined. The study of these problems can be divided into three
fundamental parts:
\begin{enumerate}
\renewcommand{\labelenumi}{\theenumi)}
\item Local study of functions, which is essentially the study of
  functions on $C^{n}$. Immediately afterwards one can pose some
  general problems for all complex analytic manifolds.

\item The study of compact complex manifolds; in particular, the study
  of compact K\"ahlerian manifolds.

\item The study of Stein manifolds.
\end{enumerate}

We shall examine some properties of compact K\"ahlerian manifolds in
detail. Compact Riemann surfaces will appear as special case of these
manifolds. We shall prove the Riemann-Roch theorem in the case of a
compact Riemann surface. We shall not concern ourselves with the study
of Stein manifolds.

\section*{Differentiable Manifolds}\pageoriginale

To start with, we shall work with real manifolds and examine later the
situation on a complex manifold. We shall look upon a complex analytic
manifold of complex dimension $n$ as a $2n$-dimensional real manifold
having certain additional properties.

We define an indefinitely differentiable $(C^{\infty})$ real manifold
of dimension $N$. (We reserve the symbol $\ub{n}$ for the complex
dimension). It is first of all a locally compact topological space
which is denumerable at infinity (\iec a countable union of compact
sets). On this space we are given a family of `maps', exactly in the
sense of a geographical map. Each map is a homeomorphism of an open
set of the space onto an open set in $R^{N}$, the $N$-dimensional
Euclidean space. [For instance, if $N=2$, we can imagine the manifold
  to be the surface of the earth and the image of a portion of the
  surface to be a region on the two dimensional page on which the map
  is drawn]. We require the domains of these maps to cover the
manifold so that every point of the manifold is represented by at
least one point in the `atlas'. We also impose a condition of
coherence in the overlap of the originals of two maps. Suppose that
the originals of the maps have a non-empty intersection; we get a
correspondence in $R^{N}$ between the images of this intersection if
we make correspond to each other the points which are images of the
same point of the manifold. We now have a correspondence between two
open subsets of $R^{N}$ and we demand that this correspondence should
be indefinitely differentiable \iec defined\pageoriginale by means of
$N$ indefinitely differentiable functions of $N$ variables.

Thus an {\em $N$-dimensional $C^{\infty}$ manifold} is a locally
compact space denumerable at infinity for which is given a covering by
open sets $U_{j}$, and for each $U_{j}$ a homeomorphism $\varphi_{j}$
of $U_{j}$ onto an open set in $R^{N}$ such that the map
$$
\varphi_{j}\circ \varphi^{-1}_{i}:\varphi_{i}(U_{i}\cap U_{j})\to
\varphi_{j}(U_{i}\cap U_{j}) 
$$
is indefinitely differentiable.

We call the pair $(U_{j},\varphi_{j})$ a map, and we say that the
family of maps given above defines a differentiable structure on the
manifold. We call the family of maps an atlas. If $(U,\varphi)$ is a
map, by composing $\varphi$ with the coordinate functions on $R^{N}$
we get $N$-functions $x_{1},\ldots,x_{N}$ on $U$; the functions
$x_{i}$ are called the coordinate functions and form the local
coordinate system defined by the map $(U,\varphi)$.

From a theoretical point of view it is better to assume that the atlas
we have is a maximal or complete atlas, in the sense that we cannot
add more maps to the family still preserving the compatibility
conditions on the overlaps. From any atlas we can obtain a unique
complete atlas containing the given atlas; we say that two atlases are
equivalent or define the same differentiable structure on the manifold
if they give rise to the same complete atlas.

We have defined a $C^{\infty}$ manifold by requiring certain maps from
$R^{N}$ to $R^{N}$ to be $C^{\infty}$ maps; it is clear how we should
define real analytic or quasi-analytic or $k$-times differentiable
$(C^{k})$ manifolds. 

We\pageoriginale shall be able to put on a $C^{\infty}$ manifold all
the notions in $R^{N}$ that are invariant under $C^{\infty}$
transformations. For instance we have the notion of a $C^{\infty}$
function on a $C^{\infty}$ manifold $V^{N}$. Let $f$ be a real valued
function defined on $V^{N}$. Let $(U,\varphi)$ be a map; by
restriction of $f$ to $U$ we get a function on $U$; by transportation
we get a function on an open subset of $R^{N}$, namely the function
$f\circ \varphi^{-1}$ on $\varphi(U)$ and we know what it means to say
that such a function is a $C^{\infty}$ function. We define $f$ to be a
$C^{\infty}$ function if and only if for every choice of the map
$(U,\varphi)$ the function $f\circ \varphi^{-1}$ defined on
$\varphi(U)$ is a $C^{\infty}$ function. We are sure that this notion
is correct, since the notion of a $C^{\infty}$ function on $R^{N}$ is
invariant under $C^{\infty}$ transformation. [We can define similarly
  $C^{\infty}$ functions on open subsets of $V^{N}$ (which are also
  manifolds!)].

If $x_{1},\ldots,x_{N}$ are coordinate functions corresponding to
$(U,\varphi)$ and $a\in U$ we define $\left(\dfrac{\partial f}{\partial
  x_{1}}\right)_{a},\ldots,\left(\dfrac{\partial f}{\partial
  x_{N}}\right)_{a}$ to be the partial derivatives of $f\circ
\varphi^{-1}$ at $\varphi(a)$. 

On a $C^{k}$ manifold we can define the notion of a $C^{p}$ function
for $p\leq k$.

\noindent
{\bf The space of differentials and the tangent space at a point.}

We have now to define the notions of tangent vector and differential
of a function at a point of the manifold.

We define the differential of a $C^{1}$ function at a point $\ub{a}$
of $V^{N}$. For a $C^{1}$ function $f$ on $R^{N}$ the differential at
a point $\ub{a}$ is the datum of the system of values
$\left(\dfrac{\partial f}{\partial
  x_{1}}\right)_{a},\ldots,\left(\dfrac{\partial f}{\partial
  x_{N}}\right)_{a}$. We\pageoriginale make an abstraction of this in
the case of a manifold. Let $U$ be a fixed open neighbourhood
containing $\ub{a}$. All $C^{1}$ functions defined on this
neighbourhood form a vector space, in fact an algebra, denoted by
$E_{a,U}$. We say that a function $f\in E_{a,U}$ is stationary at
$\ub{a}$ (in the sense of maxima and minima!) if all the first partial
derivatives vanish at the point $\ub{a}$. The notion of a function
being stationary at $\ub{a}$ has an intrinsic meaning; for if the
partial derivatives at $\ub{a}$ vanish in one coordinate system they
will do so in any other coordinate system. Let $S_{a,U}$ denote the
subspace of functions stationary at $a$. We define the space of
differentials at $\ub{a}$ to be the quotient space
$E_{a,U}/S_{a,U}$. If $f\in E_{a,U}$ its differential at $\ub{a}$,
$(df)_{a}$, is defined to be its canonical image in the quotient
space. We can prove two trivial properties: its independence of the
neighbourhood $U$ chosen and the fact that the space of differentials
is of dimension $N$. If we choose a coordinate system
$(x_{1},\ldots,x_{N})$ at $\ub{a}$ we have the canonical basis
$(dx_{1})_{a},\ldots,(dx_{N})_{a}$ for the space of differentials; in
terms of this basis the differential of a function $f$ has the
representation
$$
(df)_{a}=\sum\left(\frac{\partial f}{\partial
  x_{i}}\right)_{a}(dx_{i})_{a}.
$$

We now proceed to define the tangent space at $\ub{a}$. If $x$ is a
vector in $R^{N}$ we have the notion of derivation along $x$. The best
way to define this is to define the derivative of a function $f$ along
$x$ as $\lim\limits_{t\to 0}\dfrac{f(a+tx)-f(a)}{t}$, if it exists. If
$x$ is the unit vector along the $x_{i}$-axis we have the partial
derivation $\dfrac{\partial}{\partial x_{i}}$. 
Thus\pageoriginale in $R^{N}$ a vector defines a derivation. In a
manifold, on the other hand, we may define a tangent vector as a
derivation. We may define a tangent vector as a derivation of the
algebra $E_{a,U}$ with values in $R^{1}$, \iec as a linear map
$L:E_{a,U}\to R^{1}$ which has the differentiation property: $L(f\cdot
g)=L(f)g(a)+f(a)L(g)$. But there will be some difficulty and we prefer
to give a definition which is related directly to the space of
differentials defined above. When we consider only $C^{\infty}$
functions these two definitions coincide.

We define a tangent vector or a derivation at $\ub{a}$ to be a linear
function $E_{a,U}\to R'$ which is zero on stationary functions. Once
again this definition is independent of $U$; moreover the tangent
space is $N$-dimensional. If $(x_{1},\ldots,x_{N})$ is a coordinate
system at $\ub{a}$, $\left(\dfrac{\partial}{\partial
  x_{i}}\right)_{a}$ are tangent vectors at $\ub{a}$. These are
obviously linearly independent. Let $L$ be any tangent vector at
$a$. We may write any $C^{1}$ function $f$ as
$$
f=f(a)+\sum(x_{i}-a_{i})\left(\dfrac{\partial f}{\partial
  x_{i}}\right)_{a}+g 
$$
where $g$ is stationary at $\ub{a}$ so that we have
$$
L(f)=\sum L(x_{i})\left(\dfrac{\partial f}{\partial
  x_{i}}\right)_{a}\quad\text{or}\quad
L=\sum L(x_{i})\left(\dfrac{\partial}{\partial x_{i}}\right)_{a} 
$$
Thus corresponding to a coordinate system $(x_{1},\ldots,x_{N})$ we
have the canonical base $\left(\left(\dfrac{\partial}{\partial
  x_{1}}\right),\ldots,\left(\dfrac{\partial}{\partial
  x_{N}}\right)\right)$ of the tangent space at $\ub{a}$.

Thus at a point $\ub{a}$ on the manifold $V^{N}$ we have two
$N$-dimensional vector spaces: the tangent space at $\ub{a}$,
$T_{a}(V)$ and the space of differentials at $\ub{a}$,
$T^{\ast}_{a}(V)$. They are duals of each other and the\pageoriginale
duality is given by the scalar product
$$
\langle L,(df)_{a}\rangle=L(f),\quad L\in T_{a}(V),\quad (df)_{a}\in
T^{\ast}_{a}(V).
$$
As it is, we first defined the space of differentials at a point and
then the tangent space at that point; as we shall see, it is better to
think of the tangent space as the original space and the space of
differentials as its dual space.



