\chapter{Lecture 2}

\section*{$C^{\infty}$ maps, diffeomorphisms. Effect of a
  map}\pageoriginale 

We define a $C^{\infty}$ map from a $C^{\infty}$ manifold $U$ of
dimension $p$ to a $C^{\infty}$ manifold $V$ of dimension $q$ ($p$ and
$q$ need not be equal). Let $\Phi$ be a continuous map of $U$ into
$V$. We say that $\Phi$ is a $C^{\infty}$ map if, for every choice of
maps $(u,\varphi)$ in $U$ and $(v,\psi)$ in $V$ such that $u\cap
\Phi^{-1}(v)$ is non empty, the map
$$
\psi\circ \Phi\circ \varphi^{-1}:\varphi(u\cap \Phi^{-1}(v))\to
\psi(v)
$$
which is a map of an open subset of $R^{p}$ into one in $R^{q}$, is a
$C^{\infty}$ map.

We can define a $C^{k}$ map of one $C^{n}$ manifold into another if
$k<n$. If $U$, $V$, $W$ are $C^{\infty}$ manifolds and $\Phi$; $U\to
V$ and $\Psi:V\to W$ are $C^{\infty}$ maps then the map $\Psi\circ
\Phi:U\to W$ is also a $C^{\infty}$ map. A map $\Phi:U\to V$ ($U$, $V$
$C^{\infty}$ manifolds) is said to be a $C^{\infty}$ isomorphism or a
diffeomorphism if $\Phi$ is $(1,1)$ and both $\Phi$ and $\Phi'$ are
$C^{\infty}$ maps.

Let $\Phi$ be a $C^{\infty}$ map of a $C^{\infty}$ manifold $U^{p}$
into another $C^{\infty}$ manifold $V^{q}$. Let $\ub{a}$ be a point of
$U$ and $b=\Phi(\ub{a})$. We shall now describe how $\Phi$ gives rice
to a linear map of $T_{a}(U)$ into $T_{b}(V)$ and linear map of
$T^{\ast}_{b}(V)$ into $T^{\ast}_{a}(U)$. Let us choose an open
neighbourhood $A$ of $\ub{a}$ and an open neighbourhood $B$ of $b$
such that $\Phi(A)\subset B$. If $f$ is a $C^{k}$ function on an open
subset of $V$, $f\circ\Phi$ is also a $C^{k}$ function on an open
subset of $U$. We call $f\circ \Phi$ the reciprocal image of $f$ with
respect to $\Phi$ and denote it by $\Phi^{-1}(f)$ or
$\Phi^{\ast}(f)$. Now\pageoriginale let $f\in E_{b,B}$; the
restriction of $\Phi^{-1}(f)$ to $A$ belongs to $E_{a;A}$. We thus
have a natural linear map $\Phi^{-1}:E_{b,B}\to E_{a,A}$. By this map
functions stationary at $b$ go over into functions stationary at
$a:\Phi^{-1}:S_{b,B}\to S_{a,A}$. So $\Phi$ induces a linear map of
$E_{b,B}/S_{b,B}$ into $E_{a,A}/S_{a,A}$ \iec a linear map of
$T^{\ast}_{b}(V)$ into $T^{\ast}_{a}(U)$. We denote this map also by
$\Phi^{-1}$. 

We now give this map in terms of coordinate functions
$x_{1},\ldots,x_{p}$ at $\ub{a}$ and $y_{1},\ldots,y_{q}$ at
$\ub{b}$. Suppose the map is given by
$y_{j}=\Phi_{j}(x_{1},\ldots,x_{p})$. $\Phi_{j}$ are $C^{\infty}$
functions of $(x_{1},\ldots,x_{p})$. The reciprocal image of the
function $f(y_{1},\ldots,y_{q})$ is the function
$$
(x_{1},\ldots,x_{p})\to
f(\Phi_{1}(x_{1},\ldots,x_{p}),\ldots,\Phi_{q}(x_{1},\ldots,x_{p})). 
$$
Let $(df)_{b}=\sum\limits^{q}_{j=1}\left(\dfrac{\partial f}{\partial
  y_{j}}\right)_{b}(dy_{j})_{b}$ be a differential at $b$; then
$$
\left(\Phi^{-1}(df)_{b}\right)=\sum^{q}_{i=1}\left(\frac{\partial
  f}{\partial
  y_{i}}\right)_{b}\left(\sum^{p}_{j=1}\left(\frac{\partial
  \Phi_{i}}{\partial x_{j}}\right)_{a}(dx_{j})_{a}\right).
$$
In particular,
$$
\Phi^{-1}(dy_{j})_{b}=\sum_{i}\left(\dfrac{\partial \Phi_{j}}{\partial
  x_{i}}\right)_{a}(dx_{i})_{a} 
$$
So in terms of the canonical basis $(dy_{1})_{b},\ldots,(dy_{q})_{b}$
for $T^{\ast}_{b}(V)$ and the canonical basis
$(dx_{1})_{a},\ldots,(dx_{p})_{a}$ for $T_{a}^{\ast}(U)$ the linear
transformation $\Phi^{-1}:T_{b}^{\ast}(V)\to T^{\ast}_{a}(U)$ is given
by the Jacobian matrix
$$
\left(\left(\frac{\partial \Phi_{j}}{\partial x_{i}}\right)_{a}\right)
$$
We have a Jacobian matrix only if we choose coordinate systems at
$\ub{a}$ and $\ub{b}$.

The\pageoriginale map $\Phi^{-1}$ goes in the direction opposite to
that of the map $\Phi$; we now give a direct
transformation. Associated with the linear map
$\Phi^{-1}:T^{\ast}_{b}(V)\to T^{\ast}_{a}(U)$ we have the transpose
of this map $\Phi:T_{a}(U)\to T_{b}(V)$. This mapping is called the
differential of the mapping $\Phi$ at $\ub{a}$. By the definition of
the transpose we have
$$
\langle \Phi(L), (df)_{b}\rangle=\langle L,(d(f\circ\Phi)_{a}\rangle
$$
where $L\in T_{a}(U)$; that is, we have $\Phi(L)(f)=L(f\circ \Phi)$
and this gives a direct description of the $\map \Phi:T_{a}(U)\to
T_{b}(V)$, because if $L$ is a derivation at $\ub{a}$, the above
formula defines $\Phi(L)$ as a linear form on $E_{b;B}$ which is
obviously a derivation. We shall hereafter refer to a differential as
a tangent co-vector.

\noindent
{\bf Invariance of dimension}

We now prove the theorem of invariance of dimension: tow diffeomorphic
manifolds have the same dimension. If $\Phi$ is a diffeomorphism of
$U$ onto $V$, and $\Phi(a)=b$, $a\in U$, we have the linear maps
$\Phi:T_{a}(U)\to T_{b}(V)$ and $\Psi:T_{b}(V)\to T_{a}(U)$ where
$\Psi$ denotes the inverse of the map $\Phi:U\to V$. Since $\Phi\circ
\Psi=$ identity and $\Psi\circ \Phi=$ identity, the same relations
hold for the associated linear transformations $\Phi$ and
$\Psi$. Consequently $T_{a}(U)$ and $T_{b}(V)$ are isomorphic; so $U$
and $V$ have the same dimension.

\medskip
\noindent
{\bf Tensor fields and differential forms}

We proceed to examine the question of tensor fields and differential
forms. Let $\ub{a}$ be a point on the $C^{\infty}$ manifold $V$. An
element of\pageoriginale $T_{a}(V)$ is called a contravariant vector
at $\ub{a}$; an element of $T^{\ast}_{a}(V)$ is called a covariant
vector at $\ub{a}$. A tensor at $\ub{a}$ of contravariant order $p$
and covariant order $q$ is an element of the tensor product
$$
\left(\bigotimes^{p}T_{a}(V)\right)\otimes
\left(\bigotimes^{q} \wedge T^{\ast}_{a}(V)\right).
$$
Thus any tensor of any kind that can be defined on a vector space can
be put at a point on the manifold.

We call an element of the $p$ th exterior power
$\overset{p}{\Lambda}\, T_{a}(V)$, a $p$-vector at $\ub{a}$ and an
element of $\overset{p}{\Lambda} \,T^{\ast}_{a}(V)$ a $p$-covector at
$\ub{a}$. If $\overset{p}{\omega}$ is a $p$-covector 
$\overset{p}{\omega}$ and $\overset{p}{\omega} \wedge
\frac{q}{\omega}$ a $q$-covector
$\overset{p}{\omega}\Lambda \ub{q}$ (the exterior product 
of $\overset{p}{\omega}$ and $\frac{q}{\omega}$) is a $p+q$ covector
and we have
$$
\overset{p}{\omega}\Lambda
\frac{q}{\omega}=(-1)^{pq}\frac{q}{\omega}\Lambda \overset{p}{\omega}.
$$
The exterior algebra,
$$
T^{\ast}_{a}(V)=\sum^{N}_{0}\overset{p}{\Lambda}T^{\ast}_{a}(V),
(\overset{0}{\Lambda}T^{\ast}_{a}(V)=R)   
$$
over $\wedge T^{\ast}_{a}(V)$ will be of particular interest in the sequel.

All this we have at a point of the manifold. Now we consider the whole
manifold. A vector field (contravariant) on $V$ is a map which assigns
to every point $\ub{a}$ of the manifold $V$ a tangent vector
$\theta(a)$ at $\ub{a}$;  $\theta$ is a map of $V$ into the set of all tangent
vectors of $V$ such that $\theta(a)\in T_{a}(V)$. Similarly a $p$
times contravariant, $q$ times covariant tensor field on $V$ is a map
which assigns to every point $\ub{a}$ of $V$ a tensor at $\ub{a}$ of
contravariant order $p$ and covariant order $q$. A scalar field is an
ordinary real valued function on $V$. If we assign to every point
$a\in V$ a $p$-covector $\omega(a)$ at $\ub{a}$ we obtain a
differential form $\omega$ of degree $p$ on $V$. $\omega(a)$ is called
the value of\pageoriginale the differential form at the point
$\ub{a}$. Associated with a differentiable function $f$ we have a
differential form of degree $1$ which assigns to every point $\ub{a}$
the differential of $f$ at $a$, $(df)_{a}$.

We state some trivial properties of tensor fields and differential\break
forms. All tensor fields of a particular kind (of contravariant order
$p$ and covariant order $q$) form a vector space. We can add tensors
at each point and multiply the tensor at each point by a scalar).
The differential forms of degree $p(p=0,1,\ldots,N)$ form a vector
space. We can multiply a differential form of degree $p$,
$\overset{p}{\omega}$, and a differential form of degree $q$,
$\dfrac{q}{\omega}$ and obtain a differential form,
$\overset{p}{\omega}\Lambda \dfrac{q}{\omega}$ of degree $p+q$; we
have only to take at every point $\ub{a}$ the exterior product
$\overset{p}{\omega}(a)\Lambda\dfrac{q}{\omega}(a)$. Thus the space of
all differential forms is an algebra.



