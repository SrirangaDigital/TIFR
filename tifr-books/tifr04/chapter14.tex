\chapter{Lecture 14}

\section*{Green's Operator $G$}\pageoriginale

Suppose $A\in \mathscr{H}_{2}+\mathscr{H}_{3}$. We know that there
exists a unique $X\in\mathscr{H}_{2}+\mathscr{H}_{3}$ such that
$-\Delta X=A$. We write $X=GA$; we then have an operator $G:A\to GA$
on $\mathscr{H}_{2}+\mathscr{H}_{3}$ with the property $-\Delta G=I$
on $H_{2}+\mathscr{H}_{3}$. ($I$ is the identity operator). If
$\alpha\in \mathscr{H}_{2}+\mathscr{H}_{3}$ and
$\Delta\alpha\in\mathscr{H}_{2}+\mathscr{H}_{3}$ we have
$-G\Delta\alpha=\alpha$. We extend $G$ to the whole space
$\mathscr{H}$ by putting $G\alpha=0$ for
$\alpha\in\mathscr{H}_{1}$. $G:\mathscr{H}\to \mathscr{H}$ is an
operator which is zero on $\mathscr{H}_{1}$ and which leaves the
spaces $\mathscr{H}_{2}$ and $\mathscr{H}_{3}$ invariant. We shall
prove a little later that $G$ is continuous. $G$ is called the Green's
operator.

Let $\omega\in \mathscr{H}$. We can write $\omega$ uniquely as
$$
\omega=\pi_{1}\omega+\widetilde{\omega},
\widetilde{\omega}\in\mathscr{H}_{2}+\mathscr{H}_{3}.
$$
But $\widetilde{\mathscr{\omega}}=-\Delta G\widetilde{\omega}=-\Delta
G\omega$ as $\widetilde{\omega}\in\mathscr{H}_{2}+\mathscr{H}_{3}$ and
$G=0$ on $\mathscr{H}_{1}$. So
$$
\omega=\pi_{1}\omega-\Delta G\omega.
$$
Thus we have the formula
\begin{align*}
I &= \pi_{1}-\Delta G\\
 &= \pi_{1}+d\partial +\partial dG.
\end{align*}
Since $I=\pi_{1}+\pi_{2}+\pi_{3}$, it follows that $\pi_{2}=d\partial
G$ and $\pi_{3}=\partial dG$.

\section*{Decomposition of $\mathscr{D}$}\pageoriginale

$\mathscr{H}$ has certain fundamental defects; $d$, $\partial$ and
$\Delta$ do not operate on $\mathscr{H}$. But these operators operate
on $\mathscr{D}$. We now consider $\mathscr{D}$.

$G$ operates on $\mathscr{D}$. For let $\omega$ be a $C^{\infty}$
form. $\omega=\pi_{1}\omega-\Delta G\omega$ or $\Delta
G\omega=\pi_{1}\omega-\omega$. Since $\pi_{1}\omega-\omega$ is a
$C^{\infty}$ form it follows from the elliptic character of $\Delta$
that $G\omega$ is a $C^{\infty}$ form. $\pi_{1}$ evidently operates on
$\mathscr{D}$. Since $\pi_{2}=d\partial G$ and $\pi_{3}=\partial dG$,
$\pi_{2}$ and $\pi_{3}$ also operate on $\mathscr{D}$.

Let
$$
\mathscr{D}_{1}=\mathscr{D}\cap \mathscr{H}_{1},
\mathscr{D}_{2}=\mathscr{D}\cap \mathscr{H}_{2},
\mathscr{D}_{3}=\mathscr{D}\cap \mathscr{H}_{3}
$$
$\mathscr{D}_{1}$ is the space of all harmonic
forms. $\mathscr{D}_{2}=d\mathscr{D}$ and $\mathscr{D}_{3}=\partial
\mathscr{D}$ by de Rham's theorem. $\mathscr{D}_{1}$,
$\mathscr{D}_{2}$ and $\mathscr{D}_{3}$ re closed subspaces of
$\mathscr{D}$. ($\mathscr{D}=\mathscr{E}$ has a genuine topology). The
linear maps $\pi_{1}$, $\pi_{2}$ and $\pi_{3}$ from $\mathscr{D}$ onto
the spaces $\mathscr{D}_{1}$, $\mathscr{D}_{2}$, $\mathscr{D}_{3}$
respectively have the following properties:
\begin{align*}
\pi_{i}\pi_{j} &= 0\text{ \  for \ } i\neq j\\
 \pi^{2}_{i} &= \pi_{i}\\
 I &= \pi_{1}+\pi_{2}+\pi_{3},
\end{align*}
Consequently $\mathscr{D}$ is the direct sum of the closed subspaces
$\mathscr{D}_{1}$,\break $\mathscr{D}_{2}$ and\pageoriginale
$\mathscr{D}_{3}$. For any element $\omega\in\mathscr{D}$ we have the
decomposition formula:
$$
\omega=\pi_{1}\omega+d\partial G\omega+\partial dG \omega.
$$

\section*{Continuity of $G$}

Let $F$ be a Fr\'echet space \iec a complete topological vector space
with a denumerable basis of neighbourhoods of $0$. Banach's closed
graph theorem states that if a linear map $G:F\to F$ is discontinuous
then there exists a sequence of elements $\varphi_{j}\to 0$ such that
$G\varphi_{j}$ tends to a non-zero element $\theta$. So in order to
prove that a linear map $G$ of a Fr\'echet space into itself is
continuous it is sufficient to show that $\varphi_{j}\to 0$ and
$G\varphi_{j}\to \theta$ together imply that $\theta=0$. [We can give
  an example of a normed vector space in which the closed graph
  theorem is not true. Let $E$ be the space of polynomials in the
  closed interval $(0,1)$ with the topology of uniform convergence in
  $(0,1)$ (Norm $f=\Max\limits_{x\in(0,1)}|f(x)|$). The operator
  $\dfrac{d}{dx}$ is a discontinuous operator on this space, as a
  sequence of polynomials may tend uniformly to zero in $(0,1)$ while
  their derivatives may not. However the closed graph theorem is not
  true for this operator; for if a sequence of polynomials $P_{j}\to 0$
  uniformly in $(0,1)$ and $\dfrac{d P_{j}}{dx}\to\theta$ uniformly in
  $(0,1)$, we must have $\theta=0$. Here the space is not complete. In
  fact the completion of $E$ in the norm defined above is the space of
  continuous functions in $(0,1)$ (Weierstrass approximation theorem)
  and $\dfrac{d}{dx}$ can not be extended to this space].

We shall now use the closed graph theorem to prove the continuity of
$G$ in $\mathscr{D}$. $\mathscr{D}$ is a Fr\'echet space. (In general
$\mathscr{E}$ is a\pageoriginale Fr\'echet space; here since the
manifold is compact $\mathscr{D}=\mathscr{E}$). Let $\{\varphi_{j}\}$
be a sequence of elements of $\mathscr{D}$ such that $\varphi_{j}\to
0$ and $G\varphi_{j}\to \theta$; we have to show that
$\theta=0$. Write
$$
\varphi_{j}=\pi_{1}\varphi_{j}-\Delta G\varphi_{j}
$$
$\pi_{1}$ is continuous in $\mathscr{D}$. For if
$\omega\in\mathscr{D}$
$$
\pi_{1}\omega=\sum_{k}(\omega,\theta_{k})\theta_{k}
$$
where $\theta_{k}$ is an orthonormal base for $\mathscr{H}_{1}$. If
$\varphi_{j}\to \varphi$ in $\mathscr{D}(\varphi_{j},\theta_{k})\to
(\varphi,\theta_{k})$. Therefore if $\varphi_{j}\to 0$ in
$\mathscr{D}$, $\pi_{1}\varphi_{j}\to 0$ in $\mathscr{D}$. Now since
$G\varphi_{j}\to \theta$ and $\Delta$ is continuous $\Delta
G\varphi_{j}\to \Delta\theta$. So $\Delta\theta=0$ or $\theta\in
\mathscr{H}_{1}$; already $\theta\in\mathscr{H}_{2}+\mathscr{H}_{3}$;
consequently $\theta=0$.

Similarly it can be proved that $G$ is continuous in $\mathscr{H}$.

\section*{Self-adjointness of $G$}

We shall now show that $G$ is self-adjoint in $\mathscr{H}$;
$$
(G\varphi,\psi)=(\varphi,G\psi),\varphi,\psi\in \mathscr{H}
$$
If we put $G\varphi=\alpha$ and $G\psi=\beta$ we have
$$
\psi=\pi_{1}\psi-\Delta\beta\text{ \ and
  \ }\varphi=\pi_{1}\varphi-\Delta\alpha
$$
so that
\begin{align*}
(G\varphi,\psi) &= (\alpha, \psi)\\
&= (\alpha,\pi_{1}\psi)-(\alpha,\Delta\beta)
\end{align*}
and\pageoriginale
\begin{align*}
(\varphi,G\psi) &= (\varphi,\beta)\\
&= (\pi_{1}\varphi,\beta)-(\Delta\alpha,\beta)
\end{align*}
Since $(\alpha,\pi_{1}\psi)=(\pi_{1}\varphi,\beta)=0$
($\alpha,\beta\in\mathscr{H}_{2}+\mathscr{H}_{3}$ while $\pi_{1}\psi$,
$\pi_{1}\varphi\in\mathscr{H}_{1}$) we have only to prove that
$(\alpha,\Delta\beta)=(\Delta\alpha,\beta)$. But this is evident when
$\varphi$ or $\psi$ (and hence $\alpha$ or $\beta$) is a $C^{\infty}$
form. So we have the relation $(G\varphi,\psi)=(\varphi,G\psi)$ when
$\varphi$ or $\psi$ is a $C^{\infty}$ form. Since $\mathscr{D}$ is
dense in $\mathscr{H}$ and $G$ is continuous we obtain
$$
(G\varphi,\psi)=(\varphi,G\psi),\varphi,\psi\in\mathscr{H}.
$$
$G$ is hermitian positive $(G\varphi,\varphi)\geq 0$ and
$(G\varphi,\varphi)=0$ if and only if $\varphi$ is harmonic. For if
$G\varphi=\alpha$ 
\begin{align*}
(G\varphi,\varphi) &= (\alpha,\pi_{1}\varphi-\Delta\alpha)\\
 &=(\alpha,-\Delta\alpha)\\
 &\geq 0
\end{align*}
and $(G\varphi,\varphi)=0$ if and only if $\Delta\alpha=0$ or
$\Delta\varphi=0$. 
