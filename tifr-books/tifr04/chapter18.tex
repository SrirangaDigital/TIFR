\chapter{Lecture 18}

\section*{The canonical orientation of a complex
  manifold}\pageoriginale

We shall now show that a complex analytic manifold $V^{(n)}$
(considered as a $2n$-dimensional real manifold) is orientable and has
a canonical orientation. The orientation on $V^{2n}$ is determined by
the maps giving the complex analytic structure on $V^{2n}$; we have to
verify that two such maps, considered as real coordinate systems, have
a positive Jacobian on the overlaps. To prove this, let
$(z_{1},\ldots,z_{n})\to (\zeta_{1},\ldots,\zeta_{n})$ be a
holomorphic map of $C^{n}$ to $C^{n}$. Let $D$ be the Jacobian of
$\zeta_{1},\ldots,\zeta_{n}$ with respect to $z_{1},\ldots,z_{n}$; let
further $z_{i}=x_{i}+iy_{i}$, $\zeta_{i}=\xi_{i}+i\eta_{i}$ and $J$
the Jacobian of the functions $(\xi_1,\eta_{1},\ldots,\xi_{n},\eta_{n})$
with respect to $(x_{1},y_{1},\ldots,x_{n},y_{n})$. We shall prove
that $J=|D|^{2}$. Now $d\zeta_{i}=d\xi_{i}+\id\eta_{i}$ and
$d\ob{\zeta}_{i}=d\xi_{i}-\id\eta_{i}$ so that
$$
d\xi_{i}\wedge d\eta_{i}=-\frac{1}{2i}d\zeta\wedge d\ob{\zeta}_{i};
$$
Similarly
$$
dx_{i}\wedge dy_{i}=\frac{-1}{2i}dz_{i}\wedge d\ob{z}_{i}i_{1}
$$
We have
\begin{align*}
J=\frac{d\xi_{1}\wedge d\eta_{1}\wedge\ldots\wedge d\xi_{n}\wedge
  d\eta_{n}}{dx_{1}\wedge dy_{1}\wedge\ldots\wedge dx_{n}\wedge
  dy_{n}} &=
\frac{d\zeta_{1}\wedge d\ob{\zeta}_{1}\wedge\ldots\wedge
  d\zeta_{n}\wedge d\ob{\zeta}_{n}}{dz_{1}\wedge
  d\ob{z}_{1}\wedge\ldots\wedge dz_{n}\wedge d\ob{z}_{n}}\\
&= \frac{d\zeta_{1}\wedge\ldots\wedge d\zeta_{n}\wedge
  d\ob{\zeta}_{1}\wedge\ldots\wedge
  d\ob{\zeta}_{n}}{dz_{1}\wedge\ldots\wedge dz_{n}\wedge
  d\ob{z}_{1}\wedge\ldots\wedge d\ob{z}_{n}}\\
&= D\ob{D}=|D|^{2}
\end{align*}\pageoriginale
as
$$
\frac{d\zeta_{1}\wedge\ldots\wedge
  d\zeta_{n}}{dz_{1}\wedge\ldots\wedge dz_{n}}=D,\quad
\frac{d\ob{\zeta}_{1}\wedge\ldots\wedge
  d\ob{\zeta}_{n}}{d\ob{z}_{1}\wedge\ldots\wedge d\ob{z}_{n}}=\ob{D}. 
$$
In the case of our maps $D\neq 0$ and therefore $J>0$.

\section*{Currents}

We shall now define bigradation for currents and the operators $J$,
$d_{z}$ and $d_{\ob{z}}$ on currents.

A current $T$ is said to be of bidegree $(p,q)$ if
$$
\langle T,\overset{r,s}{\varphi}\rangle =0
$$
whenever $(p,q)\neq (n-r,n-s)$. ($\overset{r,s}{\varphi}$ is a form of
bidegree $(r,s)$). A current $\overset{p,q}{T}$ of bidegree $(p,q)$
can be considered as a continuous linear functional on
$\overset{n-p,n-q}{\mathscr{D}}$, the space of forms of bidegree
$(n-p,n-q)$ with compact support.

We define the operator $J$ on currents by:
$$
\langle JT,\varphi\rangle=\langle T,J^{-1}\varphi\rangle.
$$
We do this because, when $\omega$ is a form we have
$$
\langle J\omega,\varphi\rangle=\langle\omega,J^{-1}\varphi\rangle.
$$
Writing $J^{-1}\varphi=\psi$, we have to prove that 
$$
\int J\omega\wedge J\psi=\int\omega\wedge \psi\quad\text{or}\quad \int
J(\omega\wedge\psi)=\int\omega\wedge\psi 
$$\pageoriginale
but
$$
J(\omega\wedge\psi)=\omega\wedge\psi
$$
since $\omega\wedge\psi$ is a form of degree $2n$ and $J=I$ on a form
of degree $2n$. (In general
$$
J\overset{p,q}{\omega}=(-i)^{p}i^{q}\omega=i^{q-p}\omega).
$$

We define $d_{z}$ and $d_{\ob{z}}$ on currents by:
\begin{align*}
\langle
d_{z}\overset{p,q}{T},\overset{n-p-1,n-q}{\varphi}\rangle=(-1)^{p+q+1}\langle
T,d_{\ob{z}}\varphi\rangle\\
\langle d_{\ob{z}}\overset{p,q}{T},\overset{n-p,n-q-1}{\varphi}\rangle
=(-1)^{p+q+1}\langle T,d_{\ob{z}}\varphi\rangle. 
\end{align*}

These relations are true for forms. In fact when $\omega$ is a form 
of total degree $p$ we have
$$
\int\limits_{V}d_{z}\omega\wedge
\varphi=(-1)^{p+1}\int\limits_{V}\omega\wedge d_{z}\varphi.
$$
For, $\varphi$ is of type $(n-1,n)$ so that
$$
d_{\ob{z}}(\omega\wedge\varphi)=0\quad\text{and}\quad
d_{z}(\omega\wedge\varphi)=d(\omega\wedge \varphi)
$$
By Stokes' formula $\int\limits_{V}d(\omega\wedge\varphi)=0$; hence
$\int\limits_{V}d_{z}(\omega\wedge \varphi)=0$ or 
$$
\int\limits_{V}(d_{z}\omega\wedge\varphi)+(-1)^{p}\omega\wedge
d_{z}\varphi)=0).
$$

We\pageoriginale now have some more cohomologies and there is need to
prove some kind of de Rham's theorems. It can be proved that the
$\ob{z}$ cohomologies of $\mathscr{D}'$ and $\mathscr{E}$ are the same
and those of $\mathscr{D}$ and $\mathscr{E}'$ are the same.

\section*{Ellipticity of the system $\partial/\partial \ob{z}k$}

We shall show that the system $\dfrac{\partial}{\partial \ob{z}_{k}}$
in $C^{n}$ is elliptic \iec if
$$
\frac{\partial T}{\partial \ob{z}_{k}}=\alpha_{k}~(k=1,2,\ldots,n)
$$
where $T$ is a current and $\alpha_{k}$ are $C^{\infty}$ forms then
$T$ is a $C^{\infty}$ form. We have
\begin{align*}
& \frac{\partial}{\partial
  z_{k}}=\frac{1}{2}\left(\frac{\partial}{\partial
  x_{k}}-i\frac{\partial}{\partial y_{k}}\right)\\
& \frac{\partial}{\partial
    \ob{z}_{k}}=\frac{1}{2}\left(\frac{\partial}{\partial
    x_{k}}+i\frac{\partial}{\partial y_{k}}\right)\\
& \frac{\partial}{\partial z_{k}}\frac{\partial}{\partial
    \ob{z}_{k}}=\frac{1}{4}\left(\frac{\partial^{2}}{\partial
    x^{2}_{k}}+\frac{\partial^{2}}{\partial y^{2}_{k}}\right) 
\end{align*}
so that
$$
\sum_{k}\frac{\partial}{\partial z_{k}}\frac{\partial}{\partial
  \ob{z}_{k}}=\frac{\Delta}{4} 
$$
where $\Delta$ is the usual Laplacian in $R^{2n}$.

Since $\dfrac{\partial T}{\partial
  \ob{z}_{k}}=\alpha_{k}\in\mathscr{E}$, $\Delta T\in\mathscr{E}$. By
the elliptic character of $\Delta$, $T$ is $C^{\infty}$.

It\pageoriginale follows in particular that a distribution $T$ on
$C^{n}$ which satisfies the Cauchy relations
$$
\frac{\partial T}{\partial \ob{z}_{k}}=0\,(k=1,2,\ldots,n)
$$
is a holomorphic function. Similarly if a current $T$ of bidegree
$(p,0)$ on $C^{n}$ satisfies the system of partial differential
equations
$$
\frac{\partial T}{\partial \ob{z}_{k}}=0\, (k=1,2,\ldots,n)
$$
then $T$ is a holomorphic form of degree $p$.

\section*{Ellipticity of $d_{\ob{z}}$ on $\mathscr{D}^{0'}$}

Let $V^{(n)}$ be a complex analytic manifold. If $T$ is a current of
degree {\em zero} and $d_{\ob{z}}T=\alpha$ is a $C^{\infty}$ form,
then $T$ is a $C^{\infty}$ function. For, in a map, the relation
$d_{\ob{z}}T=\alpha$ implies
$$
\frac{\partial T}{\partial \ob{z}_{k}}=\alpha_{k}
$$
where $\alpha_{k}$ are $C^{\infty}$ functions. By the result proved
earlier $T$ is a $C^{\infty}$ function.

As a consequence we obtain that if $\overset{p,0}{T}$ is a current of
bidegree $(p,0)$ such that $d_{\ob{z}}\overset{p,q}{T}=0$ then
$\overset{p,0}{T}$ is a holomorphic form. 

\section*{$J$-Hermitian forms}\pageoriginale

Let $G^{2n}$ be a vector space over $R$ with a $J$-structure. A
positive definite $J$-Hermitian form on $G$ is a map $H$ of $G\times
G$ into the complex numbers with the following properties:
\begin{enumerate}
\renewcommand{\labelenumi}{\theenumi)}
\item $H$ is $R$-bilinear.

\item $H(JX,Y)=-H(X,JY)=iH(X,Y)$ for $X$, $Y\in G$ (Hence
  $H(JX,\break JY) =H(X,Y)$).

\item $H(X,X)>0$ for $X\neq 0$. 
\end{enumerate}

The real part of the positive definite Hermitian form, $(X,Y)=Rl\break
  H(X,Y)$, defines a Euclidean structure in $G$. Since $H(X,Y)$ is
invariant under $J$, $(X,Y)$ is also invariant under $J$:
$$
(JX,JY)=(X,Y)
$$
Since $(JX,X)=0$ the vectors $X$ and $JX$ are orthogonal with respect
to the Euclidean structure. From the relation $H(JX,Y)=iH(X,Y)$, we
have
$$
(JX,Y)=-\Iim H(X,Y).
$$
Hence
\begin{align*}
H(X,Y) &= (X,Y)+i\Iim H(X,Y)\\
&= (X,Y)-i(JX,Y)\\
&= (X,Y)+i(X,JY).
\end{align*}
This shows that $H$ is completely determined by its real part. If
$(X,Y)$ is a positive definite quadratic form on $G$ which is
invariant under\pageoriginale $J$ this determines a positive definite
$J$-Hermitian form on $G$ if we set
$$
H(X,Y)=(X,Y)+i(X,JY).
$$
Since $(JX,X)=0$, $(\quad)(JX,Y)$ is an anti-symmetric bilinear form
on $G$. This defines an element $\Omega$ of $\overset{2}{\wedge}
G^{\ast}$ ($G^{\ast}$ is the dual of $G$) by the formula
$$
\langle\Omega,X\wedge Y\rangle=(JX,Y),X,Y\in G.
$$

$G$ can be considered (canonically) as a vector space over complex
numbers, as we have already remarked. If $(u_{1},\ldots,u_{n})$,
$(v_{1},\ldots,v_{n})$ are the coordinates of the vectors $U$ and $V$
of $G$ with respect to a complex basis $(e_{1},\ldots,e_{n})$.,
$$
H(U,V)=\sum_{j,k}g_{jk}u_{j}\ob{v}_{k},g_{jk}=\ob{g}_{kj},(g_{ij})>0.
$$
Let us compute $\Omega$ in terms of this basis. $\Omega$ is given by
\begin{align*}
\langle \Omega, U\wedge V\rangle &= -\frac{H(U,V)-H(V,U)}{2i}\\
&= -\frac{H(U,V)-\ob{H(U,V)}}{2i}\\
&=-\frac{1}{2i}\sum_{j,k}g_{jk}(u_{j}\ob{v}_{k}-\ob{u}_{k}v_{j}).
\end{align*}
If $(e^{\ast}_{1},\ldots,e^{\ast}_{n})$ is the dual basis of
$(e_{1},\ldots,e_{n})$ 
$$
\langle e^{\ast}_{j}\wedge\ob{e}^{\ast}_{k}, U\wedge V\rangle=
\begin{vmatrix}
u_{j},\ob{u}_{k}\\
v_{j},\ob{v}_{k}
\end{vmatrix}
$$\pageoriginale
so that
$$
\Omega=-\frac{1}{2i}\sum g_{jk}e^{\ast}_{j}\wedge \ob{e}_{k}^{\ast}.
$$
If we choose an orthonormal basis $(e_{1},\ldots,e_{n})$ for the
hermitian form,
$$
\Omega=-\frac{1}{2i}\sum e^{\ast}_{j}\wedge \ob{e}^{\ast}_{k}.
$$

\section*{Hermitian Manifolds}

Let $V^{(n)}$ be a complex analytic manifold. At each tangent space
$T_{a}(V)$ we have a canonical $J$-structure. $V^{(n)}$ will be called
a Hermitian manifold if on each $T_{a}(V)$ we have a positive definite
$J$-Hermitian form such that the twice covariant tensor field defined
by these forms is a $C^{\infty}$ tensor field.

On every complex analytic manifold we can put a Hermitian structure,
just the same way we introduced a Riemannian structure on a
$C^{\infty}$ manifold.

In a Hermitian manifold $V^{(n)}$ the real part of the $J$-Hermitian
form on each $T_{a}(V)$ gives rise to a Riemannian structure on the
manifold; the imaginary part gives rise to a real $C^{\infty}$
differential form $\Omega$ of bidegree $(1,1)$.

If in a coordinate system $(z_{1},\ldots,z_{n})$ the Hermitian form is
given by 
$$
\sum g_{jk}dz_{j}d\ob{z}_{k},
$$
then\pageoriginale in this map,
$$
\Omega=-\frac{1}{2i}\sum g_{jk}dz_{j}\wedge d\ob{z}_{k}.
$$
If $\tau$ is the volume element associated with the Riemannian
structure we have the relation $\Omega^{n}=n!\tau$. For, if
$(z_{1},\ldots,z_{n})$, $z_{j}=x_{j}+iy_{j}$, is a local coordinate
system at $a\in V$ in which the Hermitian form at $\ub{a}$ is given by
$\sum (dz_{k})_{a}\ob{(dz_{k})}_{a}$, we have,
\begin{align*}
\Omega_{a} &= -\frac{1}{2i}\sum_{k}(dz_{k})_{a}\wedge
(d\ob{z}_{k})_{a}\\
&= \sum (dx_{k})_{a}\wedge (dy_{k})_{a};\\
\Omega^{n}_{a} &= (dx_{1})_{a}\wedge (dy_{1})_{a}\wedge\ldots\wedge
(dx_{n})_{a}\wedge (dy_{n})_{a}\wedge n!\\
&= n!\tau_{a}.
\end{align*}

\section*{Kahlerian Manifolds}

A Hermitian manifold is called a Kahlerian manifold if $d\Omega=0$.

There exist manifolds with an infinity of Hermitian structures but
with no Kahlerian structure.

For a compact Kahlerian manifold $V$,
$$
b^{2p}\geq 1.
$$
To prove this we notice that
$$
\int\limits_{V}\Omega^{n}=\int\limits_{V}n!\tau>0.
$$\pageoriginale
$\Omega^{n}$ is a cocycle but not a coboundary, since
$\int\limits_{V}\Omega^{n}\neq 0$. It follows that the forms
$\Omega^{p}(1\leq p\leq n)$, which are cocycles, are not
coboundaries. Hence the $2p$-th cohomology groups are different form
zero or $b^{2p}\geq 1$.

Since $b^{4}(S^{6})=0$, if $S^{6}$ is a complex analytic manifold it
is not Kahlerian.

Every complex analytic manifold of complex dimension $1$ is Kahlerian
because and $2$-form is closed.



