\chapter{Lecture 20}

\section*{Proof of the formula
  $[\Delta,d_{z}]=i\partial_{\ob{z}}$}\pageoriginale 

To prove the bracket relation
$$
[\Lambda,d_{z}]=i\partial_{\ob{z}}
$$
in the case of a K\"ahlerian manifold, we first verify this relation
in the case of $C^{n}$ with the canonical K\"ahlerian metric
$$
\sum^{n}_{k=1}dz_{k}d\ob{z}_{k}.
$$
In this case
$$
\Omega=-\frac{1}{2i}\sum_{k}dz_{k}\wedge d\ob{z}_{k}.
$$
The real part of the Hermitian form is given by
$$
\sum_{k}(dx^{2}_{k}+dy^{2}_{k}).
$$
The Euclidean structure given by the metric on the real tangent space
at a point $\ub{a}$ induces a Euclidean structure on the real
co-tangent space at $\ub{a}$. We extend this Euclidean structure to a
Hermitian structure in the complex co-tangent vector space. The $2n$
vectors $dx_{1}$, $dy_{1},\ldots,dx_{n}$, $dy_{n}$ form an orthonormal
basis for the real cotangent vector space; the vectors $dz_{1}$,
$d\ob{z}_{1},\ldots,dz_{n}$, $d\ob{z}_{n}$, form an orthogonal basis
for the complex co-tangent vector space; each of these
vectors\pageoriginale is of length $\sqrt{2}$ as
$dz_{k}=dx_{k}+idy_{k}$ and $d\ob{z}_{k}=dx_{k}-idy_{k}$. The vector
$dz_{J}\wedge d\ob{z}_{K}$ is of length $\sqrt{2^{j+k}}$ where $j$ and
$k$ are the number of elements in $J$ and $K$.

We shall now introduce some elementary operators in $C^{n}$ and
express the operators $\Lambda$, $d_{z}$, $\partial_{z}$ etc.\@ in
terms of these operators. The operator
$$
e_{k}=dz_{k}\Lambda
$$
operates on forms by multiplying every form on the left by
$dz_{k}\cdot e_{k}$ is an operator of the type $(1,0)$. $\ob{e}_{k}$
is the operator
$$
d\ob{z}_{k}\Lambda.
$$
$\ob{e}_{k}$ is an operator of the type $(0,1)$. $i_{k}$ and
$\ob{i}_{k}$ are defined to be the adjoints of $e_{k}$ and
$\ob{e}_{k}$ respectively. $i_{k}$ is an operator of the type $(-1,0)$
while $\ob{i}_{k}$ is an operator of the type $(0,-1)$. We shall prove
that the linear operator $i_{k}$ is given by the formula
\begin{align*}
& i_{k}[\omega dz_{J}\wedge d\ob{z}_{K}]=0\\
& i_{k}[\omega\wedge dz_{k}\wedge dz_{J'}\wedge d\ob{z}_{K}]=2\omega
  \wedge dz_{J'}\wedge d\ob{z}_{K}
\end{align*}
($J'$ is a set of indices without the index $k$). (this amounts
essentially to the suppression of $dz_{k}$). We shall verify that the
operator defined by these formulae is the adjoint of $e_{k}$. For any
two forms $\alpha$ and $\beta$ we shall verify that 
$$
(e_{k}\alpha,\beta)_{a}=(\alpha,i_{k}\beta)_{a}.
$$\pageoriginale
It is sufficient to verify these for the elementary forms \iec to
verify that
$$
(e_{k}\alpha'dz_{J}\wedge d\ob{z}_{K},\beta'dz_{L}\wedge
  d\ob{z}_{M})_{a}=(\alpha'dz_{J}\wedge
  d\ob{z}_{K},i_{k}\beta'dz_{L}\wedge d\ob{z}_{M})_{a}
$$
or
$$
(e_{k}dz_{J}\wedge d\ob{z}_{K}, dz_{L}\wedge
d\ob{z}_{M})_{a}=(dz_{J}\wedge d\ob{z}_{K},i_{k}dz_{L}\wedge
d\ob{z}_{M})_{a}.
$$
The elements $\{dZ_{J}\wedge d\ob{z}_{K}\}$ are orthogonal. The right
and left sides both vanish except when $L=k+J$ and $M=K$. In this case
both the sides are equal to $2^{1+r+s}$ where $r$ and $s$ are the
number of indices in $J$ and $K$ respectively.

We shall now introduce two more elementary operators:
$$
\partial_{k}=\frac{\partial}{\partial
  z_{k}},\ob{\partial}_{k}=\frac{\partial}{\partial \ob{z}_{k}}.
$$
We can prove that the adjoint of $\partial_{k}$ is
$-\ob{\partial}_{k}$ and the adjoint of $\ob{\partial}_{k}$ is
$-\partial_{k}$ in a similar way. We can express all the operators in
the K\"ahlerian structure by means of these operators. We have 
\begin{align*}
L &=-\frac{1}{2i}\sum e_{k}\ob{e}_{k}\\
\Lambda &= \frac{1}{2i}\sum \ob{i}_{k}i_{k}\quad \text{(taking the
  adjoint).}\\
d_{z} &= \sum\partial_{k}e_{k}=\sum e_{k}\partial_{k}(e_{k}\text{
  \  and \ } \partial_{k}\text{ \ commute})\\
d_{\ob{z}} &= \sum\ob{\partial}_{k}\ob{e}_{k}\\
\partial_{\ob{z}} &= -\sum\partial_{k}\ob{i}_{k}.
\end{align*}\pageoriginale

Now we can prove the bracket relation. We have
\begin{align*}
\Lambda d_{z} &=
\sum_{k,l}\frac{1}{2i}\ob{i}_{k}i_{k}\partial_{l}e_{l},\\
d_{z}\Lambda &= \sum_{k,l}\frac{1}{2i}\partial_{1}e_{1}\ob{i}_{k}i_{k}.
\end{align*}
Since $\partial_{1}$ commutes with $i_{k}$ and $\ob{i}_{k}$,
$$
\Lambda d_{z}=\sum_{k,l}\frac{1}{2i}\partial_{1}\ob{i}_{k}i_{k}e_{1}.
$$
$e_{1}$ and $i_{k}$ do not commute. For $k\neq
1\ i_{k}e_{1}=-e_{1}i_{k}$ so that
\begin{align*}
\sum_{\substack{k,1\\ k\neq
    1}}\frac{1}{2i}\partial_{1}\ob{i}_{k}i_{k}e_{1} &=
\sum_{\substack{k,1\\ k=\neq
    1}}-\frac{1}{2i}\partial_{1}\ob{i}_{k}e_{1}i_{k}\\
&= \sum_{\substack{k,1\\ k\neq
    1}}\frac{1}{2i}\partial_{1}e_{1}\ob{i}_{k}i_{k}. 
\end{align*}
Now\pageoriginale
$$
i_{k}e_{k}+e_{k}i_{k}=2
$$
(as $i_{k}e_{k}$ is zero for a term which contains $dz_{k}$ as factor
while $e_{k}i_{k}$ is zero in the contrary case). So
\begin{align*}
\frac{1}{2i}\partial_{k}\ob{i}_{k}i_{k}e_{k} &=
-\frac{1}{2i}\partial_{k}\ob{i}_{k}e_{k}i_{k}+\frac{2}{2i}\partial_{k}i_{k}\\
&= \frac{1}{2i}\partial_{k}e_{k}\ob{i}_{k}i_{k}-i\partial_{k}i_{k}.
\end{align*}
Consequently,
\begin{align*}
\Lambda d_{z} &=
\sum_{k,1}\frac{1}{2i}\partial_{1}\ob{i}_{k}i_{k}e_{1}\\
&=
\sum_{k,1}\frac{1}{2i}\partial_{1}e_{1}\ob{i}_{k}i_{k}-i\sum\partial_{k}\ob{i}_{k}\\
&= d_{z}\Lambda+i\partial_{\ob{z}}
\end{align*}
which proves the bracket relation.

We shall now derive the bracket relation in the case of an arbitrary
K\"ahlerian manifold by using a theorem of differential geometry.

Suppose we have two $C^{\infty}$ tensor fields $\oplus$ and $\oplus'$
of the same kind on a $C^{\infty}$ manifold $V$. We shall say that
$\oplus$, $\oplus'$ coincide upto the order $m$ at a point $\ub{a}\in
V$ if the coefficients as well as the\pageoriginale partial
derivatives upto order $m$ of the coefficients of $\oplus$ and
$\oplus'$ coincide at $\ub{a}$. This has an intrinsic meaning: for if
the property is true for one map at $\ub{a}$ it is true for any other
map at $\ub{a}$. Let now $V$ be a Riemannian manifold with the field
of positive-definite quadratic forms $Q$. Let $\ub{a}$ be a point of
$V$. According to a theorem of Riemannian geometry we can find another
$C^{\infty}$ field $Q'$ of positive definite quadratic forms defined
in a neighbourhood of $\ub{a}$ such that $Q$ and $Q'$ coincide upto
the first order at $\ub{a}$ and such that $Q'$ gives the Euclidean
structure in a neighbourhood of $\ub{a}$ (\iec there exists a map in
which
$$
Q'=\sum \delta_{ij}dx_{i}dx_{j}
$$
where $\delta_{ij}$ is the Kronecker Symbol). $Q'$ is said to be an
osculating Euclidean structure for $Q$ at $\ub{a}$.

We now consider the analogous question for Hermitian
manifolds. Suppose a $C^{\infty}$ manifold $V^{2n}$ has two different
complex structures, (1) and (2). We shall say that the complex
structures (1) and (2) coincide at $\ub{a}\in V^{2n}$ upto order $m$
if for every function $f$ holomorphic in (1) the field
$(d_{\ob{z}})_{a}f$ is zero upto order $m$ at $\ub{a}$. Equivalently,
we may say that the structures (1) and (2) coincide upto order $m$ at
$\ub{a}$ if the intrinsic $J$-operators $J_{1}$ and $J_{2}$ coincide
upto order $m$ on the tangent space at $\ub{a}$. Let $V^{(n)}$ be a
Hermitian manifold. Let $H$ denote the field of Hermitian forms giving
the Hermitian structure and $\Omega$ the associated exterior $2$-form.
Let\pageoriginale $\ub{a}$ be a point of $V^{(n)}$. Is it possible to
find a Hermitian structure, $H'$ on $V$ (with the same underlying
$C^{\infty}$ structure as $V^{(n)}$) such that
\begin{enumerate}
\renewcommand{\theenumi}{\roman{enumi}}
\renewcommand{\labelenumi}{\theenumi)}
\item the two analytic structures coincide upto order 1 at $\ub{a}$ 

\item the fields of Hermitian forms $H$ and $H'$ coincide upto order
  $1$ at $\ub{a}$.

\item $H'$ is the canonical hermitian structure on $C^{n}$ for some
  map at $\ub{a}$?. There is one trivial necessary condition. If
  $\Omega'$ is the exterior $2$-form corresponding to $H'$, $\Omega$
  and $\Omega'$ coincide upto order $1$ at $\ub{a}$ so that $d\Omega$
  and $d\Omega'$ coincide at $\ub{a}$; but $d\Omega'=0$ at
  $\ub{a}$. Therefore, $d\Omega(a)=0$ is a necessary condition. This
  condition can also be proved to be sufficient; this is the difficult
  part. In particular in a K\"ahlerian manifold there exists an
  osculating Hermitian structure at every point.

Let $V^{(n)}$ be a Kahlerian manifold. At $\ub{a}\in V^{(n)}$ we
choose an osculating Hermitian structure. The operators $J$
corresponding to the two structures coincide at $\ub{a}$ up to order
$1$. The operators $\widetilde{d}=JdJ^{-1}$ coincide at
$\ub{a}$. Since $d_{z}$ is a linear combination of $d$ and
$\widetilde{d}$ the operators $d_{z}$ coincide at $a$; similarly the
operators $d_{\ob{z}}$ coincide at $\ub{a}$. The operators $\Lambda$
coincide upto order $1$ at $\ub{a}$ and the operators
$[\Lambda,d_{z}]$ coincide at $\ub{a}$. The operators
$i\partial_{\ob{z}}$ also coincide at $\ub{a}$. Since we have proved
the relation
$$
[\Lambda,d_{z}]=i\partial_{\ob{z}}
$$
for\pageoriginale the canonical Hermitian structure in $C^{n}$ the
same relation holds also for any K\"ahlerian manifold.
\end{enumerate}

The relation
$$
[\Lambda,d_{\ob{z}}]=-i\partial_{z}
$$
is proved similarly.

