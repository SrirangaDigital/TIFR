\chapter{Normal Solvability}\label{chap9}

{\bf Introduction.}~ Let\pageoriginale $X$ and $Y$ be Banach
spaces. Let $f:X\to Y$ be a given function. This section is devoted to
questions relative to the solvability of the equation $f(x)=y$, when
$y\in Y$ is given. The function $f$ is supposed to be G\^ateaux
differentiable and we would like to have sufficient conditions for the
solvability of the above equation stated in terms of the properties of
the G\^ateaux derivative $Df_{x}$. Parallelling the Fredholm theory
for compact operators these conditions will naturally involve
$Df^{*}_{x}$. For the sake of later referencing we start by recalling
some results from the theory of linear operators. Then we go to the
so-called normal solvability results of F. E. Browder and
S. I. Poho\u{z}aev. Some results of W. O. Ray and I. Ekeland are also
discussed. We close the section with a comparative study of the
results here with the classical inverse mapping theorem.

\medskip
\noindent
{\bf What is Normal Solvability?} Let $L:X\to Y$ be a bounded linear
operator from a Banach space $X$ to a Banach space $Y$. The equation
$Lx=y$ is said to be {\em normally solvable} (in the sense of
Hausdorff) if
\begin{equation*}
y\in N(L^{*})^{\perp}\Rightarrow y\in
\mathcal{R}(L).\tag{9.1}\label{chap9-eq9.1} 
\end{equation*}

Here $L^{*}:Y^{*}\to X^{*}$ is the adjoint operator defined as follows
for each $\mu\in Y^{*}$, $L^{*}\mu \in X^{*}$ is given by $\la
L^{*}\mu,x\ra=\la\mu,Lx\ra$ for all $x\in X$. The other notations in
\eqref{chap9-eq9.1} are: (i) $N(T)$ to denote the kernel of an
operator $T:X\to Y$, i.e.\@ $N(T)=\{x\in X:Tx=0\}$; (ii) the range of
$T$, $\mathcal{R}(T)=\{y\in Y:\exists x\in X \text{~ s.t.~ }
Tx=y\}$,\pageoriginale and the (right) polar of a subspace $B\subset
X^{*}$, $B^{\perp}=\{x\in X:\la \mu,x\ra=0,\forall \mu\in B\}$. Later
on we also use the (left) polar of a subspace $A\subset X$ defined as
$$
A^{\perp}=\{\mu\in X^{*}:\la \mu,x\ra=0\q \forall x\in A\}.
$$

The well-known Fredholm alternative for compact linear operators
$T:X\to X$ in a Banach space $X$ gives that the operator $L=I-T$ is
normally solvable. For a general bounded linear operator $L:X\to Y$,
one can prove that
\begin{equation*}
N(L^{*})^{\perp}=\overline{\mathcal{R}(L)}.\tag{9.2}\label{chap9-eq9.2} 
\end{equation*}

So all operators $L$ with a closed range are normally solvable.

\medskip
\noindent
{\bf Some Results From The Linear Theory.}~ We now recall some
theorems from the theory of bounded linear operators in Banach
spaces. The reader can find the proofs in many standard texts in
Functional Analysis, see for instance Yosida \cite{key79}, Br\'ezis
\cite{key15}. 

\begin{theorem}[Closed range Theorem]\label{chap9-thm9.1}
Let $L:X\to Y$ be a bounded linear operator, $X$ and $Y$ Banach
spaces. The following properties are equivalent: {\rm(i)}
$\mathcal{R}(L)$ is closed, {\rm(ii)} $\mathcal{R}(L^{*})$ is closed,
{\rm(iii)}~ $\mathcal{R}(L)=N(L^{*})^{\perp}$, {\rm(iv)}~
$\mathcal{R}(L^*)=N(L)^{\perp}$. 
\end{theorem}


\begin{theorem}[Surjectivity Theorem]\label{chap9-thm9.2}
Let $L:X\to Y$ be a bounded linear operator. The following conditions
are equivalent {\rm(i)} $\mathcal{R}(L)=Y$, {\rm(ii)} $N(L^{*})=\{0\}$
and $\mathcal{R}(L^{*})$ is closed, {\rm(iii)} there exists a constant
$C>0$ such that $||y^{*}||\leq C||L^{*}y^{*}||$. 
\end{theorem}


\begin{theorem}[Surjectivity Theorem for the adjoint]\label{chap9-thm9.3}
Let $L:X\to Y$ be a bounded linear operator. The following conditions
are equivalent: {\rm(i)} $\mathcal{R}(L^{*})=X$, {\rm(ii)}
$N(L)=\{0\}$ and $\mathcal{R}(L)$ is closed, {\rm(iii)} there exists a
constant $C>0$ such that $||x||\leq C||Lx||$
\end{theorem}

\medskip
\noindent
{\bf Normal Solvability of Nonlinear Operators.}~ Now we describe a
nonlinear\pageoriginale analogue of Theorem \ref{chap9-thm9.2}. 
This result was proved by Poho\u{z}aev \cite{key63} (see also
\cite{key64}, \cite{key65}) for reflexive Banach spaces and $f(X)$
weakly closed, and by Browder \cite{key20}, \cite{key21}, for general
Banach spaces. The Browder papers mentionned above contain much more
material on normal solvability besides the simple results presented
here.

\begin{theorem}\label{chap9-thm9.4}
Let $f:X\to Y$ be a G\^ateaux differentiable function between Banach
spaces $X$ and $Y$. Assume that $f(X)$ is closed. Let us use the
notation $Df_{x}$ for the Gateaux derivative at a point $x\in
X$. Assume that $N(Df^{*}_{x})=\{0\}$ for all $x\in X$. Then $f$ is
surjective. 
\end{theorem}

The above result follows from a more general one, (namely Theorem
\ref{chap9-thm9.5}) due also to Browder. The proof below follows the
same spirit of Browder's original proof. However it uses a more direct
approach [directness is a function of the arrangement one sets in
  one's presentation!] through the Drop Theorem (Theorem
\ref{chap7-thm7.3}), proved in Chapter \ref{chap7} via the Ekeland
Variational Principle. 


\begin{theorem}\label{chap9-thm9.5}
Let $X$ and $Y$ be Banach spaces, and $f:X\to Y$ a G\^ateaux
differentiable function. Assume that $f(X)$ is closed. Let $y\in Y$ be
given and suppose that there are real numbers $\rho>0$ and $0\leq p<1$
such that
\begin{gather*}
f^{-1}(B_{\rho}(y))\neq \emptyset\tag{9.3}\label{chap9-eq9.3}\\
\Inf\{||y-f(x)-z||:z\in \overline{\mathcal{R}(Df_{x})}\}\leq
p||y-f(x)||,\tag{9.4}\label{chap9-eq9.4} 
\end{gather*}
for all $x\in f^{-1}(B_{\rho}(y))$. Then $y\in f(X)$.
\end{theorem}

\setcounter{remark}{0}
\begin{remark}\label{chap9-rem1}
If \eqref{chap9-eq9.3} and \eqref{chap9-eq9.4} holds simultaneously
for each $y\in Y$, then $f$ is surjective. Observe that a large $\rho$
gives \eqref{chap9-eq9.3}, but then \eqref{chap9-eq9.4} is harder to
be attained.
\end{remark}

\begin{remark}\label{chap9-rem2}
Proof of Theorem \ref{chap9-thm9.4} $N(Df^{*}_{x})=\{0\}$ implies, by
\eqref{chap9-eq9.2}, that \eqref{chap9-eq9.4} is attained with $p=0$
and arbitrary $\rho$. So, for each given $y$, take $\rho$ such that
$\dist(y,f(X))<\rho$, and take $p=0$. Therefore Theorem
\ref{chap9-thm9.5} implies Theorem \ref{chap9-thm9.4}.
\end{remark}

\begin{remark}\label{chap9-rem3}
The thesis of Theorem \ref{chap9-thm9.4} still holds if in the
hypotheses we replace $N(Df^{*}_{x})=\{0\}$ by $\mathcal{R}(Df_{x})$
dense in $Y$. Theorem \ref{chap9-thm9.4} contains a result of
Kacurovskii \cite{key35}, who considered continuously Frechet
differentiable mappings\pageoriginale $f$ and assumed that
$\mathcal{R}(Df_{x})=Y$ for all $x\in X$. The proof uses a
Newton-Kantorovich method of successive approximations. See Remark
\ref{chap9-rem1} after Theorem \ref{chap9-thm9.8}.
\end{remark}

\begin{remark}\label{chap9-rem4}
A G\^ateaux differentiable mapping $f:X\to Y$ is said to be a {\em
  Fredholm mapping} if $Df_{x}:X\to Y$ is a Fredholm (linear) operator
for each $x\in X$. We recall that a bounded linear operator $L:X\to Y$
is Fredholm if $N(L)$ is finite dimensional and $\mathcal{R}(L)$ is
closed and has finite codimension. The index $i(L)$ is defined as
$i(L)=\dim N(L)-\codim \mathcal{R}(L)$. We observe that $i(Df_{x})$
for a Fredholm mapping $f$ is locally constant. Since $X$ is connected
we can then define
$$
i(f)=i(Df_{x})\q\text{for some}\q x\in X
$$
since the right side is independent of $x$. Now if in Theorem
\ref{chap9-thm9.4} we assume that $f$ is a Fredholm mapping of index
$0$, then condition $N(Df^{*}_{x})=\{0\}$ can be replaced by
$N(Df_{x})=\{0\}$. 
\end{remark}

\noindent
{\bf Proof of Theorem \ref{chap9-thm9.5}.}~ Let $S=f(X)$. Suppose by
contradiction that $y\not\in S$. Let $R=\dist (y,S)$ and choose $r$,
$\rho>0$ such that $r<R<\rho$ and $p\rho<r$. Observe that if
\eqref{chap9-eq9.3} and \eqref{chap9-eq9.4} hold for some $\rho_{0}$
then it also holds for any other $\rho$, with $R<\rho\leq
\rho_{0}$. Then use the Drop Theorem: there exists $u_{0}\in S$
\begin{equation*}
||u_{0}-y||<\rho\q\text{and}\q S\cap
D(y,r;u_{0})=\{u_{0}\}.\tag{9.5}\label{chap9-eq9.5} 
\end{equation*}

Now let $x_{0}\in X$ be such that $f(x_{0})=u_{0}$. Then
\eqref{chap9-eq9.4} implies
$$
\Inf\{||y-f(x_{0})-z||:z\in \overline{\mathcal{R}(Df_{x_{0}})}\}\leq
p||y=f(x_{0})||<r. 
$$

So there exists $x\in X$ such that
\begin{equation*}
||y-f(x_{0})-Df_{x_{0}}(x)||<r,\tag{9.6}\label{chap9-eq9.6}
\end{equation*}
and approximating the G\^ateaux derivative by the Newton quotient one
has for small $t>0$:
$$
||w_{t}||\equiv ||y-f(x_{0})-\frac{f(x_{0}+tx)-f(x_{0})}{t}||<r.
$$

Thus the vector $y-w_{t}\in D(y,r;u_{0})$, and the same is true for
$(1-t)u_{0}+t(y-w_{t})$ with $0<t<1$ and $t$ small. But this last
statement simply says that
\begin{equation*}
f(x_{0}+tx)\in D(y,r;u_{0}),\q \forall
t>0\q\text{small}.\tag{9.7}\label{chap9-eq9.7} 
\end{equation*}

The\pageoriginale second assertion in \eqref{chap9-eq9.5} and
\eqref{chap9-eq9.7} imply that
$$
f(x_{0}+tx)=u_{0}\q \forall t>0\q\text{small}.
$$
which gives $Df_{x_{0}}(x)=0$. Going back to \eqref{chap9-eq9.6} we
get $||y-f(x_{0})||<r$, which is impossible.\hfill$\Box$

\medskip
\noindent
{\bf Some Surjectivity Results.}~ Both theorems \ref{chap9-thm9.4} and
\ref{chap9-thm9.5} have as hypothesis the statement that $f(X)$ is a
closed set. This is a global assumption whose verification may cause
difficulties when applying those theorems. It would be preferable to
have local assumptions instead. That it is the contents of the next
result which is due to Ekeland, see Bates-Ekeland \cite{key7}; see
also Ray-Rosenholtz \cite{key69} for a slightly more general
result. Observe that the function $f$ is assumed to be continuous in
the next theorem. This implies that the graph of $f$ closed, but
asserts nothing like that about $f(X)$.

\begin{theorem}\label{chap9-thm9.6}
Let $X$ and $Y$ be Banach spaces and $f:X\to Y$ a continuous mapping,
which is G\^ateaux differentiable. Assume:
\begin{gather*}
\mathcal{R}(Df_{x})=Y,\q \forall x\in X\tag{9.8}\label{chap9-eq9.8}\\
\exists k>0\text{~ s.t.~ }\forall x\in X,\q \forall y\in Y,\q \exists
z\in (Df_{x})^{-1}(y)\tag{9.9}\label{chap9-eq9.9}
\end{gather*}
with the property:
$$
||z||\leq k||y||.
$$

Then $f$ is surjective.
\end{theorem}

\begin{proof}
If suffices to prove that $0\in f(X)$. Define the functional
$\Phi:X\to \mathbb{R}$ by $\Phi(x)=||f(x)||$. Clearly $\Phi$ satisfies
the conditions for the applicability of the Ekeland Variational
Principle. So given $\epsilon>0$ there exists $x_{\epsilon}\in X$ such
that
\begin{gather*}
||f(x_{\epsilon})||\leq
\Inf_{X}||f(x)||+\epsilon\tag{9.10}\label{chap9-eq9.10}\\ 
||f(x_{\epsilon})||<||f(x)||+\epsilon||x-x_{\epsilon}||\q \forall
x\neq x_{\epsilon}.\tag{9.11}\label{chap9-eq9.11}
\end{gather*}

Take in \eqref{chap9-eq9.11} $x=x_{\epsilon}+tv$, where $t>0$ and
$v\in X$ are arbitrary. Let $\mu_{t}\in Y^{*}$ such that
\begin{equation*}
||\mu_{t}||=1,\q ||f(x_{\epsilon}+tv)||=\la
\mu_{t},f(x_{\epsilon}+tv)\ra;\tag{9.12}\label{chap9-eq9.12}
\end{equation*}
see Remark \ref{chap8-rem1} after the proof of Proposition
\ref{chap8-prop8.4}: $\mu_{t}\in
J(f(x_{\epsilon}+tv)/\break ||f(x_{\epsilon}+tv||)$. We observe that
$||f(x_{\epsilon})||\geq \la \mu_{t},f(x_{\epsilon}\ra$. Altogether,
we can write \eqref{chap9-eq9.11} as
\begin{equation*}
\frac{\la
  \mu_{t},f(x_{\epsilon}+tv)\ra-\la\mu_{t},f(x_{\epsilon})\ra}{t}\geq
-\epsilon||v||.\tag{9.13}\label{chap9-eq9.13} 
\end{equation*}

By\pageoriginale the Banach-Alaoglu theorem (i.e., the
$w^{*}$-compactness of the unit ball in $Y^{*}$) and the fact that
$$
\frac{1}{t}[f(x_{\epsilon}+tv)-f(x_{\epsilon})]\to
Df_{x_{\epsilon}}(v)\q\text{(strongly) in }~ Y
$$
we can pass to the limit as $t\to 0$ in \eqref{chap9-eq9.12} and
\eqref{chap9-eq9.13} and obtain
\begin{gather*}
||\mu_{0}||=1,\q ||f(x_{\epsilon})||=\la
\mu_{0},f(x_{\epsilon})\ra\tag{9.14}\label{chap9-eq9.14}\\ 
\la \mu_{0},Df_{x_{\epsilon}}(v)\ra\geq -\epsilon||v||\q \text{for
  all}\q v\in X.\tag{9.15}\label{chap9-eq9.15}
\end{gather*}

Now using hypothesis \eqref{chap9-eq9.8} and \eqref{chap9-eq9.9} we
can select a $v\in X$ such that
$Df_{x_{\epsilon}}(v)=-f(x_{\epsilon})$ and $||v||\leq
k||f(x_{\epsilon})||$. All this gives 
$$
\la \mu_{0},f(x_{\epsilon})\ra\leq \epsilon k||f(x_{\epsilon})||.
$$

So if we start with an $\epsilon$ such that $\epsilon k<1$, the last
inequality contradicts \eqref{chap9-eq9.14}, unless $f(x_{\epsilon})=0$.
\end{proof}

\setcounter{remark}{0}
\begin{remark}\label{chap9-addrem1}
The passage to the limit in the above proof requires a word of
caution. If $X$ is separable then the $w^{*}$-topology of the unit
ball in $X^{*}$ is metrizable. So in this case we can use sequences in
the limiting questions. Otherwise we should use filters. She
Dunford-Schwartz \cite[p. 426]{key35}.
\end{remark}

\begin{remark}\label{chap9-addrem2}
Let $L:X\to Y$ be a bounded linear operator with closed range. Then
there exists a constant $k>0$ such that for each $y\in \mathcal{R}(T)$
there is an $x\in X$ with properties that $y=Lx$ and $||x||\leq
k||y||$. This is a classical result of Banach and it can be proved
from the Open Mapping Theorem in a straightforward way: consider the
operator $\widetilde{T}:X/N(T)\to \mathcal{R}(T)$. In this set-up it
is contained in Theorem \ref{chap9-thm9.3} above. Now let us see which
implications this has to Theorem \ref{chap9-thm9.6} above. Condition
\eqref{chap9-eq9.8} implies that the inequality in \eqref{chap9-eq9.9}
holds with a $k$ depending on $x$. Viewing a generalization of Theorem
\ref{chap9-thm9.6} let us define a functional $k:X\to \mathbb{R}$ as
follows. Assume that $f:X\to Y$ has a G\^ateaux derivative with the
property that $\mathcal{R}(Df_{x})$ is the whole of $Y$. For each
$x\in X$, $k(x)$ is defined as a constant that has the property
\begin{equation*}
||z||\leq k(x)||y||\q\forall y\in Y\q\text{and some}\q z\in
(Df_{x})^{-1}y.\tag{9.16}\label{chap9-eq9.16} 
\end{equation*}

We remark that for each $x\in X$, the smallest value possible for
$k(x)$ is the norm of the $T^{-1}$ where $T:X/N(Df_{x})\to Y$. 
\end{remark}

\begin{theorem}\label{chap9-thm9.7}
Let\pageoriginale $X$ and $Y$ be Banach spaces and $f:X\to Y$ a
continuous mapping 
which is G\^ateaux differentiable. Assume
\begin{gather*}
\mathcal{R}(Df_{x})=Y,\q \forall x\in
X\tag{9.17}\label{chap9-eq9.17}\\
\forall R>0\ \exists c=c(R)\text{~s.t.~}k(x)\leq c,\q \forall
||x||\leq R.\tag{9.18}\label{chap9-eq9.18}\\
||f(x)||\to \infty\q\text{as}\q ||x||\to
\infty.\tag{9.19}\label{chap9-eq9.19} 
\end{gather*}

Then $f$ is surjective.
\end{theorem}

\begin{proof}
It suffices to prove that $0\in f(X)$. Define $\Phi:X\to \mathbb{R}$
by $\Phi(x)=||f(x)||$. Let $\rho=||f(0)||$. It follows from
\eqref{chap9-eq9.19} that there exists $R>0$ such that
\begin{equation*}
||f(x)||\geq \frac{3}{2}\rho\q \text{if}\q ||x||\geq
R.\tag{9.20}\label{chap9-eq9.20} 
\end{equation*}

Choose an $\epsilon>0$ such that $\epsilon c(R)<1$ and $\epsilon\leq
\rho/2$. By the Ekeland Variational Principle there exists
$x_{\epsilon}\in X$ such that
\begin{gather*}
||f(x_{\epsilon})||\leq \Inf_{X}\Phi+\epsilon\leq \rho+\epsilon\leq
3\rho/2.\tag{9.21}\label{chap9-eq9.21}\\
||f(x_{\epsilon})||<||f(x)||+\epsilon||x-x_{\epsilon}||,\q\forall
x\neq x_{\epsilon}\tag{9.22}\label{chap9-eq9.22}
\end{gather*}

It follows from \eqref{chap9-eq9.20} and \eqref{chap9-eq9.21} that
$||x_{\epsilon}||\leq R$. Now we preceed as in the proof of Theorem
\ref{chap9-thm9.6} and conclude that $f(x_{\epsilon})=0$.
\end{proof}

\setcounter{remark}{0}
\begin{remark}\label{chap9-2rem1}
If $X=Y$ and $f=$ identity $+$ compact is a continuously Fr\'echet
differentiable operator, the surjectivity of $f$ has been established
by Ka\u{c}urovskii \cite{key50} under hypothesis \eqref{chap9-eq9.19}
and $N(Df_{x})=\{0\}$ for all $x\in X$. Since $Df_{x}$ is also of the
form identity $+$ compact, such a condition is equivalent to
\eqref{chap9-eq9.17}; this is a special case of the situation
described in Remark \ref{chap9-rem4} after the statement of Theorem
\ref{chap9-thm9.5}. So Ka\u{c}urovskii result would be contained in
Theorem \ref{chap9-thm9.7} provided one could prove that in his case
condition \eqref{chap9-eq9.18} holds. Is it possible to do that? In
the hypotheses of Ka\u{c}urovskii theorem, Krasnoselskii \cite{key54}
observed that $f$ is also injective.
\end{remark}

\begin{remark}\label{chap9-2rem2}
Local versions of Theorem \ref{chap9-thm9.7} have been studied by
Cramer and Ray \cite{key28}, Ray and Walker \cite{key69}.
\end{remark}

\noindent
{\bf Comparison with the Inverse Mapping Theorem.}~ The classical
inverse mapping theorem states: ``Let $X$ and $Y$ be Banach spaces,
$U$ an open neighborhood of $x_{0}$ in $X$, and $f:U\to Y$ a $C^{1}$
function. Assume that $Df_{x_{0}}:X\to Y$ is an isomorphism (i.e., a
linear bounded injective operator from\pageoriginale $X$ onto $Y$, and
then necessarily with a bounded inverse). Then there exists an open
neighborhood $V$ of $x_{0}$, $V\subset U$, such that $f|_{V}:V\to
f(V)$ is a diffeormorphism''. The injectivity hypothesis can be
withdrawn from the theorem just stated provided the thesis is replaced
by $f$ being an open mapping in a neighborhood of $x_{0}$. More
precisely we have the following result due to Graves \cite{key48}. If
you have the book by Lang \cite{key56}, the result is proved there.

\begin{theorem}\label{chap9-thm9.8}
Let $X$ and $Y$ be Banach spaces, $U$ an open neighborhood of $x_{0}$
in $X$, and $f:U\to Y$ a $C^{1}$ function. Assume that
$Df_{x_{0}}:X\to Y$ is surjective. Then there exists a neighborhood
$V$ of $x_{0}$, $V\subset U$, with the property that for every open
ball $B(x)\subset V$, centered at $x$, $f(V)$ contains an open
neighborhood of $f(x)$.
\end{theorem}

\setcounter{remark}{0}
\begin{remark}\label{chap9-3rem1}
If the mapping $f:X\to Y$ is defined in the whole of $X$, and it is
$C^{1}$ with $\mathcal{R}(Df_{x})=Y$ for all $x\in X$, Graves theorem
says that $f(X)$ is open in $Y$. If we have as an additional
hypothesis that $f(X)$ is closed, it follows then that $f(X)=Y$, in
view of the connectedness of $Y$. Now go back and read the statement
of Theorem \ref{chap9-thm9.4}. What we have just proved also follows
from Theorem \ref{chap9-thm9.4}, using relation
\eqref{chap9-eq9.2}. Observe that $\mathcal{R}(Df_{x})=Y$ is much
stronger a condition that $N(Df^{*}_{x})=\{0\}$. The latter will be
satisfied if $\mathcal{R}(Df_{x})$ is just dense in $Y$. We remark
that the proof of Graves theorem via an iteration scheme uses the fact
that $\mathcal{R}(Df_{x})$ is the whole of $Y$. We do not know if a
similar proof can go through just with hypothesis that
$\mathcal{R}(Df_{x})$ is dense in $Y$.
\end{remark}

\begin{remark}\label{chap9-3rem2}
Graves theorem, Theorem \ref{chap9-thm9.8} above, can be proved using
Ekeland Variational Principle. Since few seconds are left to close the
set, we leave it to the interested reader.
\end{remark}

The following global version of the inverse mapping theorem is due to
Hadamard in the finite dimensional case. See a proof in M. S. Berger
\cite{key10} or in J. T. Schwartz $NYU$ Lecture Notes
\cite{key73}. More general results in Browder \cite{key19}.

\begin{theorem}\label{chap9-thm9.9}
Let $X$ and $Y$ be Banach spaces and $f:X\to Y$ a $C^{1}$
function. Suppose that $Df_{x}:X\to \mathbb{R}$ is an isomorphism. For
each $R>0$, let
$$
\zeta(R)=\Sup\{||Df_{x})^{-1}||:||x||\leq R\}.
$$

Assume\pageoriginale that
$$
\int^{\infty}\frac{dr}{\zeta(r)}=\infty.
$$
[In particular this is case if there exists, constant $k>0$ such that
  $||(Df_{x})^{-1}||\leq k$ for all $x\in X$]. Then $f$ is a
diffeomorphism of $X$ onto $Y$.
\end{theorem}

\begin{remark*}
Go back and read the statement of Theorem \ref{chap9-thm9.6}. The
ontoness of the above theorem, at least in the particular case, is
contained there. 
\end{remark*}

\begin{thebibliography}{}
\bibitem{key1} H. Amann\pageoriginale and P. Hess --- A multiplicity
  result for a class of elliptic boundary value problems --
  Proc. Royal Soc. Edinburgh 84A (1979), 145--151.

\bibitem{key2} A. Ambrosetti and G. Prodi --- Analisi non lineare, I
  Quaderno -- Pisa (1973).

\bibitem{key3} A. Ambrosetti and G. Prodi --- On the inversion of some
  differentiable mappings with singularities between Banach spaces --
  Ann. Mat. Pura ed Appl. 93 (1972), 231--246.

\bibitem{key4} A. Ambrosetti and P. H. Rabinowitz --- Dual variational
  methods in critical point theory and applications -- J. Functional
  Anal 14 (1973), 349--381.

\bibitem{key5} J. Appell --- The superposition operator in function
  spaces -- a survey, Report No. 141 -- Institut f\"ur mathematik
  Augsburg (1987).

\bibitem{key6} J.-P. Aubin and I. Ekeland --- Applied Nonlinear
  Analysis -- John Wiley and Sons (1984).

\bibitem{key7} P. W. Bates and I. Ekeland --- A saddle point theorem --
  Differential Equations. Academic Press (1980), 123--126.

\bibitem{key8} B. Beauzamy --- Introduction to Banach spaces and their
  geometry -- Notas de Matem\'atica, 86, North Holland (1982).

\bibitem{key9} H. Berestycki --- Le nombre de solutions de certains
  probl\`emes semilineaires -- J. Fctl. Anal.

\bibitem{key10} M. S. Berger --- Nonlinearity and Functional Analysis
  -- Academic Press (1977).

\bibitem{key11} M. Berger and E. Podolak --- On the solutions of a
  nonlinear Dirichlet problem -- Indiana Univ. Math. J. 24 (1975), 837--846.

\bibitem{key12} E. Bishop and R. R. Phelps --- The support functionals
  of a convex set -- Proc. Symp. Pure Math. Amer. Math. Soc., Vol 7
  (1962), 27--35. 

\bibitem{key13} H. Brezis\pageoriginale --- The Ambrosetti-Rabinowitz
  $MPL$ via Ekeland's minimization principle. (Manuscript).

\bibitem{key14} H. Brezis --- Op\'erateurs maximaux monotones et
  semigroups de contractions dans les espaces de Hilbert -- Notas de
  Matematica No. 50. (North-Holland) (1973).

\bibitem{key15} H. Brezis --- Analyse fonctionelle -- Masson (1983).

\bibitem{key16} H. Brezis and F. E. Browder --- A general principle on
  ordered sets in nonlinear functional analysis -- Adv. Math. 21
  (1976), 355--364.

\bibitem{key17} A. Br{\o}ndsted --- On a lemma of Bishop and Phelps --
  Pac. J. Math. 55 (1974), 335--341.

\bibitem{key18} A. Br{\o}ndsted and R. T. Rockafellar --- On the
  subdifferentiability of convex functions -- Proc. AMS 16 (1965), 605--611.

\bibitem{key19} F. E. Browder --- Nonlinear operators and nonlinear
  equations of evolution in Banach spaces -- Proc. Symp. Pure
  Math. Amer. Math. Soc. Vol XVIII, Part 2 (1976).

\bibitem{key20} F. E. Browder --- Normal solvability for nonlinear
  mappings into Banach spaces -- Bull Amer. Math. Soc. 77 (1971), 73--77.

\bibitem{key21} F. E. Browder --- Normal solvability and the Fredholm
  alternative for mappings in infinite dimensional manifolds --
  J. Fctl. Anal. 8 (1971), 250--274.

\bibitem{key22} J. Caristi --- Fixed point theorems for mappings
  satisfying unwardness conditions -- Trans. Amer. math. Soc. 215
  (1976), 241--251.

\bibitem{key23} J. Caristi and W. A. Kirk --- Mapping Theorems in
  Metric and Banach spaces -- Bull. Acad. Pol. Sci. XXIII (1975),
  891--894. 

\bibitem{key24} K. C. Chang --- Solutions of asymptotically linear
  operator equations via Morse theory -- Comm. Pure App. Math. XXXIV
  (1981), 693--712.

\bibitem{key25} K. C. Chang -- Variational methods and sub -- and
  supersolutions -- Scientia Sinica A XXVI (1983), 1256--1265.

\bibitem{key26} J. A. Clarkson --- Uniformly convex spaces --
  Transactions AMS 40 (1936), 396--414.

\bibitem{key27} D. G. Costa, D. G. de Figueiredo and
  J. V. A. Gon\c{c}alves -- On the uniqueness of solution for a class
  of semilinear elliptic problems -- J. Math. Anal. Appl. 123 (1987),
  170--180. 

\bibitem{key28} W. J. Cramer Jr. and W. O. Ray --- Solvability of
  nonlinear operador equations -- Pac. J. Math. 95 (1981), 37--50. 

\bibitem{key29} E. N. Dancer\pageoriginale --- On the ranges of
  certain weakly nonlinear elliptic partial differential equations --
  J. Math. Pures et Appl. 57 (1978), 351--366.

\bibitem{key30} J. Dane\u{s} --- A geometric theorem useful in
  nonlinear functional analysis -- Boll. Un. Mat. Ital. 6 (1972),
  369--375. 

\bibitem{key31} J. Dane\u{s} -- Equivalence of some geometric and
  related results of nonlinear functional analysis --
  Comm. Math. Univ. Carolinae 26 (1985), 443--454.

\bibitem{key32} J. Diestel --- Geometry of Banach spaces -- Lecture
  Notes in Mathematics 485, Springer Verlag (1975).

\bibitem{key33} C. L. Dolph --- Nonlinear integral equations of
  Hammerstein type -- Trans. AMS 66 (1949), 289--307.

\bibitem{key34} J. Dungundji --- Topology -- Allyn \& Bacon (1966).

\bibitem{key35} N. Dunford and J. T. Schwartz --- Linear Operators,
  Part I: General Theory, Interscience Publishers, Ine. New York
  (1957). 

\bibitem{key36} I. Ekeland and R. Temam --- Convex Analysis and
  Variational Problems -- North-Holland Publishing Company (1976).

\bibitem{key37} I. Ekeland --- Sur les probl\`emes variationnels, CR
  Acad. Sci. Paris 275 (1972), 1057--1059.

\bibitem{key38} I. Ekeland --- On the variational principle,
  J. Math. Anal. Appl. 47 (1974), 324--353.

\bibitem{key39} I. Ekeland --- Nonconvex minimization problems,
  Bull. Amer. Math. Soc. 1 (1979), 443--474.

\bibitem{key40} K. Fan and I. Glicksberg --- Some geometrical
  properties of the spheres in a normed linear space -- Duke
  Math. J. 25 (1958), 553--568.

\bibitem{key41} D. G. de Figueiredo --- On the superlinear
  Ambrosetti-Prodi problem -- Nonl. Anal. TMA 8 (1984), 655-665.

\bibitem{key42} D. G. de Figueiredo and J.-P. Gossez --- Conditions de
  non-resonance pour certains probl\`emes elliptiques semilin\'eaires
  -- C.R.A.S. Paris 302 (1986), 543--545.

\bibitem{key43} D. G. de Figueiredo and S. Solimini -- A variational
  approach to superlinear elliptic problems -- Comm. in PDE 9 (1984),
  699--717. 

\bibitem{key44} S. Fu\u{c}ik --- Remarks on a result by A. Ambrosetti
  and G. Prodi -- Boll. Un. Mat. Ital. 11 (1975), 259--267.

\bibitem{key45} T. Gallouet\pageoriginale and O. Kavian --- Resultats
  d'existence et de non-existence pour certains probl\`emes
  demi-lin\'eaires a l'infini -- Ann. Fac. Sci. Toulouse, III (1981),
  201--246. 

\bibitem{key46} D. Gilbarg and N. S. Trudinger --- Elliptic Partial
  Differential Equations -- Springer Verlag (1977).

\bibitem{key47} J.-P. Gossez --- Existence of optimal controls for
  some nonlinear processes -- J. Optim. Th. and Appl. 3 (1969), 89--97.

\bibitem{key48} L. M. Graves --- Some mapping theorems -- Duke
  Math. J. 17 (1950), 111--114.

\bibitem{key49} R. C. James --- Weak compactness and reflexivity --
  Israel J. Math. 2 (1964), 101--119.

\bibitem{key50} R. J. Ka\u{c}urovskii --- Generalization of the
  Fredholm theorems and of theorems on linear operators with closed
  range to some classes of nonlinear operators --
  Sov. Math. Dokl. Vol. 12 (1971), 487--491.

\bibitem{key51} R. Kannan and R. Ortega --- Superlinear elliptic
  boundary value problems -- Czech Math. J.

\bibitem{key52} J. Kazdan and F. Warner --- Remarks on some
  quasilinear elliptic equations -- Comm. Pure Appl. Math. XXVIII
  (1975), 367--397.

\bibitem{key53} M. A. Krasnoselskii --- On several new fixed point
  principles -- Sov. Math. Dokl. Vol. 14 (1973), 259--261.

\bibitem{key54} M. A. Krasnoselsii -- Topological methods in the
  theory of nonlinear integral equations -- MacMillan, New York (1964).

\bibitem{key55} M. A. Krasnoselskii and P. P. Zabreiro -- On the
  solvability of nonlinear operator equations -- Funkcional. Anal. i
  Prilo\u{z}en. 5 (1971), 42--44.

\bibitem{key56} S. Lang --- Real Analysis -- Addison-Wesley Publ. C. (1969).

\bibitem{key57} A. Lazer and P. J. McKenna --- On multiple solutions
  of a nonlinear Dirichlet problem -- Differential Equations, Acadmic
  Press (1980), 199--214.

\bibitem{key58} A. R. Lovaglia --- Locally uniformly convex Banach
  spaces -- Transactions AMS 78 (1955), 255--238.

\bibitem{key59} C. A. Magalh\^aes --- Doctoral dissertation
  (University of Brasilia), 1988.

\bibitem{key60} J. Mawhin, J. R. Ward Jr. and M. Willem ---
  Variational methods and semilinear elliptic equations,
  Sem. Math. Louvain, Rap. 32 (1983).

\bibitem{key61} J.-P. Penot --- The drop theorem, the petal theorem
  and Ekeland's Variational Principle -- Nonlinear Analysis TMA 10
  (1986), 813--822.

\bibitem{key62} R. R. Phelps\pageoriginale --- Support cones in Banach
  spaces and 
  their applications -- Adv. Math. 13 (1974), 1--19.

\bibitem{key63} S. I. Poho\u{z}aev --- Normal solvability of nonlinear
  equations -- Soviet Mat. Dokl. 10 (1969), 35--38.

\bibitem{key64} S. I. Poho\u{z}aev --- On nonlinear operators having
  weakly closed range and quasilinear elliptic equations -- Math. USSR
  Sbornik 7 (1969), 227--250.

\bibitem{key65} S. I. Poho\u{z}aev --- Normal solvability Banach
  spaces -- Funkcional Anal. i Prilo\u{z}en 3 (1969), 80--84.

\bibitem{key66} P. H. Rabinowitz --- Some minimax theorems and
  applications to nonlinear partial differential equations --
  Nonlinear Analysis: a collection of papers in honor of E. H. Rothe,
  (L. Cesari, R. Kannan and H. F. Weinberger, editors) Academic Press
  (1978), 161--177.

\bibitem{key67} P. H. Rabinowitz --- Some critical point theorems and
  applications to semilinear elliptic partial differential equations
  -- Ann. Scuola Norm Sup. Pisa, Ser. IV, 5 (1978), 215--223.

\bibitem{key68} P. H. Rabinowitz --- Variational methods for nonlinear
  eigenvalue problems, (The Varenna Lectures) -- Eigenvalues of
  nonlinear problems (G. Prodi, editor) C.I.M.E., Edizioni Cremonese
  Roma (1974), 140--195.

\bibitem{key69} W. O. Ray and A. M. Walker --- Mapping theorems for
  G\^ateaux differentiable and accretive operators -- Nonlinear
  Anal. TMA 6 (1982), 423--433.

\bibitem{key70} I. Rosenheltz and W. O. Ray --- Mapping theorems for
  differentiable operators -- Bull. Acad. Polon. Sci.

\bibitem{key71} B. Ruf --- On nonlinear problems with jumping
  nonlinearities -- Ann. Math. Pura App. 28 (1981), 133--151.

\bibitem{key72} L. Schwartz --- Th\'eorie des distributions, Vol. I --
  Hermann, Paris (1957). 

\bibitem{key73} J. T. Schwartz --- Nonlinear functional analysis --
  NYU Lecture Notes (1963--1964).

\bibitem{key74} G. Scorza -- Drogoni --- Un teorema sulle funzioni
  continue rispetto ad una e misurabili rispetto at un'altra variabile
  -- Rend. Sem. Mat. Univ. Padova 17 (1948), 102--106.

\bibitem{key75} V. L. \u{S}mulian --- Sur la derivatilit\'e de la
  norme dans I'espace de Banach Dokl. Akad. Nauk 27 (1940), 643--648.

\bibitem{key76} S. Solimini --- Existence of a third solution for a
  class of b.v.p. with jumping nonlinearities -- J. Nonlinear Anal. 7
  (1983), 917--927.

\bibitem{key77} G. Stampacchia\pageoriginale --- Le probl\`eme de
  Dirichlet pour les \'equations elliptiques du second ordre \`a
  coeficients descontinus -- Ann. Inst. Fourier 15 (1965), 189--258.

\bibitem{key78} M. M. Vainberg --- Variational methods in the study of
  nonlinear operators -- Holden Day (1964).

\bibitem{key79} K. Yosida --- Functional Analysis -- Springer Verlag (1974).

\end{thebibliography}
