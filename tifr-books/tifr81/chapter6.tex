\chapter[Semilinear Elliptic Equations II]{Semilinear
  Elliptic\hfill\break  Equations II}\label{chap6}

{\bf Introduction.}~ In\pageoriginale  this chapter we continue the
study of the Dirichlet problem:
\begin{equation*}
-\Delta u=f(x,u)\q\text{in}\q \Omega,\q u=0\q\text{on}\q \p
\Omega\tag{6.1}\label{chap6-eq6.1} 
\end{equation*}
where $\Omega$ is a bounded smooth domain in $\mathbb{R}^{N}$, $N\geq
2$ and $\p \Omega$ denotes its boundary. In order to minimize
technicalities, we assume all along this chapter the following minimal
assumption on the nonlinearity:
$$
f:\overline{\Omega}\times\mathbb{R}\to \mathbb{R}\q\text{is a
  continuous function of both variables.}
$$

As in Chapter \ref{chap3} we search the critical points of the
functional
\begin{equation*}
\Phi(u)=\frac{1}{2}\int|\nabla u|^{2}-\int
F(x,u).\tag{6.2}\label{chap6-eq6.2} 
\end{equation*}

If not stated on the contrary all integrals are taken over the whole
of $\Omega$. We assume the following additional condition of $f$: 
\begin{equation*}
|f(x,s)|\leq c|s|^{p-1}+b(x)\tag{6.3}\label{chap6-eq6.3} 
\end{equation*}
where $c>0$ is a constant, $b(x)\in L^{p'}(\Omega)$ with
$(1/p)+(1/p')=1$, and $1\leq p<\infty$ if $N=2$, or $1\leq p\leq
2N/(N-2)$ if $N\geq 3$. As we proved in Chapter \ref{chap3}, under
this hypothesis the functional $\Phi:H^{1}_{0}\to \mathbb{R}$ defined
in \eqref{chap6-eq6.3} is continuously Fr\'echet differentiable. In
Chapter \ref{chap3} a further condition was required on $F$ (see
\eqref{chap3-eq3.6} there) which was sufficient to guarantee that
$\Phi$ is bounded\pageoriginale below. Here we are interested in the
cases when $\Phi$ is not bounded below any longer. We assume the
following condition
\begin{equation*}
\mathop{\lim\inf}_{s\to+\infty}\frac{f(x,s)}{s}>\lambda_{1}\q\text{uniformly
  in}\q \overline{\Omega}\tag{6.4}\label{chap6-eq6.4}
\end{equation*}
where $\lambda_{1}$ is the first eigenvalue of $(-\Delta,H^{1}_{0})$.

\begin{lemma}\label{chap6-lem6.1}
Under \eqref{chap6-eq6.4} the functional $\Phi$ is unbounded below.
\end{lemma}

\begin{proof}
It follows from \eqref{chap6-eq6.4} that there are constants
$\mu>\lambda_{1}$ and $c$ such that $f(x,s)\geq \mu s-c$, for all
$s>0$. Therefore we can find constants $\mu'$ and $c'$ with
$\mu>\mu'>\lambda_{1}$ such that $F(x,s)\geq \frac{1}{2}\mu's^{2}-c'$,
for all $s>0$. Thus for $t>0$ we have
$$
\Phi(t\varphi_{1})\leq \frac{1}{2}\lambda_{1}t^{2}\int
\varphi^{2}_{1}-\frac{1}{2}\mu't^{2}\int\varphi^{2}_{1}+c'|\Omega|. 
$$
\end{proof}

Viewing the future applications of the variational theorems of Chapter
\ref{chap5} we now state conditions which insure the $(PS)$ condition
for $\Phi$.

\begin{lemma}\label{chap6-lem6.2}
Assume condition \eqref{chap6-eq6.3} with $1\leq p<2N/(N-2)$ if $N\geq
3$ and $1\leq p<\infty$ if $N=2$. Then $\Phi$ satisfies the $(PS)$
condition if every sequence $(u_{n})$ in $H^{1}_{0}$, such that
\begin{equation*}
|\Phi(u_{n})|\leq \text{~const,}\q \Phi'(u_{n})\to
0\tag{6.5}\label{chap6-eq6.5} 
\end{equation*}
is bounded.
\end{lemma}

\begin{proof}
All we have to prove is that $(u_{n})$ contains a subsequence which
converges in the norm of $H^{1}_{0}$. Since $(u_{n})$ is bounded,
there is a subsequence $(u_{n_{j}})$ converging weakly in $H^{1}_{0}$
to some $u_{0}$ and strongly in any $L^{p}$ to the same $u_{0}$, with
$1\leq p<2N/(N-2)$ if $N\geq 3$ and $1\leq p<\infty$ if $N=2$. On the
other hand the second assertion in \eqref{chap6-eq6.5} means that
\begin{equation*}
||\int \nabla u_{n}\nabla v-\int f(x,u_{n})v||\leq
\epsilon_{n}||v||_{H^{1}},\q \forall v\in
H^{1}_{0}\tag{6.6}\label{chap6-eq6.6}  
\end{equation*}
where $\epsilon_{n}\to 0$. Put $v=u_{n}-u_{0}$, and taking limits over
the subsequence we obtain that
$$
\int \nabla u_{n_{j}}\nabla (u_{n_{j}}-u_{0})\to 0.
$$
[Here we have used the continuity properties of the Nemytskii
  mappings, see Chapter \ref{chap2}]. So $||u_{n_{j}}||_{H^{1}}\to
||u_{0}||_{H^{1}}$. This together with the fact that
$u_{n_{j}}\rightharpoonup u_{0}$\pageoriginale (weakly) in $H^{1}_{0}$, gives that
$u_{n_{j}}\to u_{0}$ (strongly) in $H^{1}_{0}$. 
\end{proof}

\noindent
{\bf Palais-Smale Condition for Asymptotically Linear Problems.}~
Assume that the limits below exist as $L^{\infty}$ functions 
\begin{equation*}
\lim\limits_{s\to -\infty}\frac{f(x,s)}{s}=\alpha (x)\q\text{and}\q
\lim\limits_{s\to
  +\infty}\frac{f(x,s)}{s}=\beta(x).\tag{6.7}\label{chap6-eq6.7} 
\end{equation*}

It follows then that there are positive constants $c_{1}$ and $c_{2}$
such that
\begin{equation*}
|f(x,s)|\leq c_{1}|s|+c_{2},\q \forall s\in \mathbb{R},\q \forall x\in
\Omega.\tag{6.8}\label{chap6-eq6.8} 
\end{equation*}

\begin{lemma}\label{chap6-lem6.3}
Assume \eqref{chap6-eq6.7} above. In addition suppose that the problem
below has only the solution $v\equiv 0$:
\begin{equation*}
-\Delta v=\beta(x)v^{+}-\alpha(x)v^{-}\q\text{in}\q
\Omega\q\text{and}\q v=0\q\text{on}\q
\p\Omega.\tag{6.9}\label{chap6-eq6.9} 
\end{equation*}

Then the functional $\Phi$ satisfies $(PS)$ condition. Here
$v^{+}=\max(v,0)$ and $v^{-}=v^{+}-v$. 
\end{lemma}

\setcounter{remark}{0}
\begin{remark}\label{chap6-rem1}
It suffices to consider \eqref{chap6-eq6.9} in the $H^{1}_{0}$
sense. That is
$$
\int\nabla v_{0}\nabla v=\int
           [\beta(x)v^{+}_{0}-\alpha(x)v^{-}_{0}]v,\q \forall v\in H^{1}_{0}
$$
and $v_{0}\in H^{1}_{0}$. We will use without further mentionning a
result of Stampachia \cite{key1}: if $v\in H^{1}_{0}$ then
$v^{+}_{0}$, $v^{-}_{0}$ and in general $G(v)$, where $G:\mathbb{R}\to
\mathbb{R}$ is a Lipschitz continuous function, are all $H^{1}_{0}$
functions. 
\end{remark}

\begin{remark}\label{chap6-rem2}
If $\alpha$ and $\beta$ are constants, then the pairs $(\alpha,\beta)$
such that problem \eqref{chap6-eq6.9} has non-trivial solutions
constitute the so-called singular set $\sum$. In the case of $N=1$
this set has been completely characterized by Fu\v{c}ik
\cite{key44}. It is not known a similar result for $N\geq 2$. However
some information about $\sum$ has been obtained, see Dancer
\cite{key29}, Gallouet-Kavian \cite{key45} and Magalh\~aes
\cite{key59}. 
\end{remark}

\noindent
{\bf Proof for Lemma \ref{chap6-lem6.3}.}~ We use Lemma
\ref{chap6-lem6.2}. Suppose by contradiction that there is a sequence
$(u_{n})$ in $H^{1}_{0}$ such that
\begin{gather*}
|\Phi(u_{n})|=|\frac{1}{2}\int|\nabla u_{n}|^{2}-\int F(x,u_{n})|\leq
\text{Const}\tag{6.10}\label{chap6-eq6.10}\\ 
|\la\Phi'(u_{n}),v\ra|=|\int \nabla u_{n}\nabla v-\int
f(x,u_{n})v|\leq \epsilon_{n}||v||_{H^{1}},\q \forall v\in
H^{1}_{0}\tag{6.11}\label{chap6-eq6.11}\\
||u_{n}||_{H^{1}}\to \infty,\q \epsilon_{n}\to
0.\tag{6.12}\label{chap6-eq6.12} 
\end{gather*}\pageoriginale

Let $v_{n}=u_{n}/||u_{n}||_{H^{1}}$, and (passing to a subsequence if
necessary) assume that a $v_{0}\in H^{1}_{0}$ can be found such that
$v_{n}\rightharpoonup v_{0}$ (weakly) in $H^{1}_{0}$, $v_{n}\to v_{0}$
(strongly) in $L^{2}$, $v_{n}\to v_{0}$ a.e.\@ and $|v_{n}|\leq h$ for
  some $L^{2}$-function $h$. Now we claim that
\begin{equation*}
||u_{n}||^{-1}_{H_{1}}f(x,u_{n}(x))\to
\beta(x)v^{+}_{0}-\alpha(x)v^{-}_{0}\q \text{in}\q
L^{2}.\tag{6.13}\label{chap6-eq6.13} 
\end{equation*}

We prove that using the argumetn in Costa-de Figueiredo-Gon\c{c}alves
\cite{key27}. Let us denote $\alpha_{n}\equiv ||u_{n}||$ and
$\ell(x)\equiv \beta(x)v^{+}_{0}-\alpha(x)v^{-}_{0}$. It suffices to
show that every subsequence of $f_{n}(x)\equiv
\alpha^{-1}_{n}f(x,\alpha_{n}v_{n}(x))$ possesses a further
subsequence which converges to $\ell(x)$ in $L^{2}$. Using
\eqref{chap6-eq6.8} 
\begin{equation*}
|f_{n}(x)|\leq \alpha^{-1}_{n}[c_{1}\alpha_{n}|v_{n}(x)|+c_{2}]\leq
c_{1}h(x)+\alpha^{-1}_{n}c_{2}.\tag{6.14}\label{chap6-eq6.14} 
\end{equation*}

In the set $A=\{x:v_{0}(x)\neq 0\}$, $f_{n}(x)\to \ell (x)$ a.e.. So
by the Lebesgue Dominated Convergence theorem $f_{n}\chi_{A}\to \ell$
in $L^{2}$. In the set $B=\{x:v_{0}(x)=0\}$ it follows from
\eqref{chap6-eq6.14} that $f_{n}(x)\to 0$ a.e. So similary
$f_{n}\chi_{B}\to 0$ in $L^{2}$. So the claim in \eqref{chap6-eq6.13}
is proved. Now dividing (11) by $||u_{n}||_{H^{1}}$ and passing to the
limit we obtain
$$
\int\nabla v_{0}\nabla
v-\int[\beta(x)v^{+}_{0}-\alpha(x)v^{-}_{0}]v=0\q\forall v\in
H^{1}_{0}.
$$

In view of \eqref{chap6-eq6.9} it follows that $v_{0}=0$. Next use
\eqref{chap6-eq6.11} again with $v=v_{n}$, divide it through by
$||u_{n}||_{H^{1}}$ to obtain
$$
|\int|\nabla v_{n}|^{2}-\int
\frac{f(x,u_{n})}{||u_{n}||_{H^{1}}}v_{n}|\leq
\frac{\epsilon_{n}}{||u_{n}||_{H^{1}}}||v_{n}||_{H^{1}}. 
$$

In the above inequality the first term is equal to 1 and the other two
converge to zero, impossible! 
\medskip

\noindent
{\bf Palais-Smale Condition for Superlinear Problems.}~ Assume that
\begin{equation*}
\mathop{\lim\inf}_{|s|\to
  \infty}\frac{f(x,s)}{s}=+\infty,\tag{6.15}\label{chap6-eq6.15} 
\end{equation*}
that\pageoriginale is, the problem is superlinear at both $+\infty$
and $-\infty$. In this case the following lemma provides sufficient
conditions for $(PS)$.

\begin{lemma}\label{chap6-lem6.4}
$\Phi$ satisfies $(PS)$ condition if one assumes: {\rm(i)} condition
  \eqref{chap6-eq6.3} with $1\leq p<2N/(N-2)$ in the case $N\geq 3$
  and $1\leq p<\infty$ in the case $N=2$, and {\rm(ii)} the following
  condition introduced by Ambrosetti and Rabinowitz \cite{key7}: there
  is a $\theta>2$ and $s_{0}>0$ such that
\begin{equation*}
0<\theta F(x,s)\leq sf(x,s)\q \forall x\in \overline{\Omega}\q \forall
|s|\geq s_{0}.\tag{6.16}\label{chap6-eq6.16} 
\end{equation*}
\end{lemma}

\begin{proof}
Let $(u_{n})$ be a sequence in $H^{1}_{0}$ satisfying conditions
\eqref{chap6-eq6.10} and \eqref{chap6-eq6.11} above. Replace $v$ by
$u_{n}$ in \eqref{chap6-eq6.11}. Multiply \eqref{chap6-eq6.10} by
$\theta$ and subtract \eqref{chap6-eq6.11} from the expression
obtained: 
$$
\left(\frac{\theta}{2}-1\right)\int |\nabla u_{n}|^{2}\leq \int
     [\theta F(x,u_{n})-u_{n}f(x,u_{n})]+\epsilon_{n}||u_{n}||_{H^{1}}+C
$$

Using \eqref{chap6-eq6.16} we obtain that $||u_{n}||_{H^{1}}\leq
C$. The proof is completed using Lemma \ref{chap6-lem6.2}.
\end{proof}

\begin{remark*}
Condition \eqref{chap6-eq6.16} implies that $F$ is
superquadratic. Indeed, from \eqref{chap6-eq6.16}: $\theta/s\leq
f(x,s)/F(x,s)$. Integrating from $s_{0}$ to $s:\theta[\ell n|s|-\ell
  ns_{0}]\leq \ell nF(x,s)-\ell nF(x,s_{0})$, which implies
$F(x,s)\geq F(x,s_{0})s^{-\theta}_{0}|s|^{\theta}$, for $|s|\geq
s_{0}$. Using \eqref{chap6-eq6.16} again we obtain $f(x,s)\geq \theta
F(x,s_{0})s^{-\theta}_{0}|s|^{\theta-1}$. Observe that this inequality
is stronger than the requirement put in\break \eqref{chap6-eq6.15}. So there
is some room between \eqref{chap6-eq6.15} and
\eqref{chap6-eq6.16}. Thus, how about Palais-Smale in the case when
$f$ satisfies \eqref{chap6-eq6.15} but not \eqref{chap6-eq6.16}? There
is a partial answer to this question in \cite{key42}.
\end{remark*}

\noindent
{\bf Palais-Smale Condition for Problems of the Ambrosetti-Prodi
  Type.} Now we assume
\begin{equation*}
\mathop{\lim\sup}\limits_{s\to
  -\infty}\frac{f(x,s)}{s}<\lambda_{1}\q\text{and}\q
\mathop{\lim\inf}\limits_{s\to
  +\infty}\frac{f(x,s)}{s}>\lambda_{1}\tag{6.17}\label{chap6-eq6.17} 
\end{equation*}
where the conditions above are to hold uniformly for $x\in
\overline{\Omega}$. The first limit could be $-\infty$ and the second
could be $+\infty$.

\begin{lemma}\label{chap6-lem6.5}
Assume \eqref{chap6-eq6.3} and the first assertion in
\eqref{chap6-eq6.17}. Let $(u_{n})$ be a sequence in
$H^{1}_{0}(\Omega)$ such that
\begin{equation*}
|\int\nabla u_{n}\nabla v-\int f(x,u_{n})v|\leq
\epsilon_{n}||v||_{H^{1}}\q \forall v\in
H^{1}_{0}\tag{6.18}\label{chap6-eq6.18} 
\end{equation*}
where\pageoriginale $\epsilon_{n}\to 0$. {\em [We may visualize the
    $u_{n}$'s as ``almost'' critical points of $\Phi$, or as
    ``approximate'' solutions of \eqref{chap6-eq6.1}, vaguely
    speaking!]} Then there exists a constant $M>0$ such that
$||u^{-}_{n}||_{H^{1}}\leq M$. 
\end{lemma}

\begin{proof}
If follows from the assumption that there exists $0<\mu<\lambda_{1}$
and a constant $c$ such that
\begin{equation*}
f(x,s)>\mu s-c\q\text{for}\q s\leq 0.\tag{6.19}\label{chap6-eq6.19} 
\end{equation*}

Replacing $v$ by $u^{-}_{n}$ in \eqref{chap6-eq6.18} we can estimate
$$
\int |\nabla u^{-}_{n}|^{2}\leq -\int
f(x,u_{n})u^{-}_{n}+\epsilon_{n}||u^{-}_{n}||_{H^{1}}. 
$$

Using \eqref{chap6-eq6.19} we obtain
$$
\int |\nabla u^{-}_{n}|^{2}\leq \mu \int (u^{-}_{n})^{2}+c\int
u^{-}_{n}+\epsilon_{n}||u^{-}_{n}||_{H^{1}}. 
$$

Finally using Poincar\'e and Schwarz inequalities we complete the proof.
\end{proof}

\begin{lemma}\label{chap6-lem6.6}
Assume \eqref{chap6-eq6.17} and that $f$ has linear growth, {\em i.e.}
\begin{equation*}
|f(x,s)|\leq c_{1}|s|+c_{2}\q \forall x\in \overline{\Omega},\q
\forall s\in \mathbb{R}\tag{6.20}\label{chap6-eq6.20} 
\end{equation*}
where $c_{1}$, $c_{2}$ are given positive constants. Then the
functional $\Phi$ satisfies $(PS)$ condition. 
\end{lemma}

\begin{proof}
Assume by contradiction that there exists a sequence $(u_{n})$ in
$H^{1}_{0}$ satisfying conditions \eqref{chap6-eq6.10},
\eqref{chap6-eq6.11} and \eqref{chap6-eq6.12} above. As in Lemma
\ref{chap6-lem6.3}, let $v_{n}=u_{n}/||u_{n}||_{H^{1}}$ and assume
that $v_{n}\rightharpoonup v_{0}$ in $H^{1}_{0}$, $v_{n}\to v_{0}$ in
$L^{2}$ and a.e.\@ $(\Omega)$, and that there is an $L^{2}$-function
$h$ such that $|v_{n}(x)|\leq h(x)$. It follows from Lemma
\ref{chap6-lem6.5} that $v^{-}_{n}\to 0$ in $H^{1}_{0}$, and we may
assume that $v^{-}_{n}\to 0$ a.e.\@ $(\Omega)$. So $v_{0}\geq 0$ in
$\Omega$. First we claim that the sequence
$$
g_{n}\equiv \chi_{n}\frac{f(x,u_{n})}{||u_{n}||}\to 0\q\text{in}\q L^{2}
$$
where $\chi_{n}$ is the characteristic function of the set
$\{x:u_{n}(x)\leq 0\}$. Indeed this follows easily from the Lebesgue
Dominated Convergence Theorem, observing that \eqref{chap6-eq6.20}
implies 
\begin{equation*}
|g_{n}|\leq
c_{1}\frac{u^{-}_{n}}{||u_{n}||_{H^{1}}}+\frac{c^{2}}{||u_{n}||_{H^{1}}}\to
0\q\text{a.e.}\tag{6.21}\label{chap6-eq6.21} 
\end{equation*}
and\pageoriginale $|g_{n}|\leq c_{1}h+c_{2}/||u_{n}||_{H^{1}}$. On the
other hand, the sequence (or passing to a subsequence of it):
\begin{equation*}
\gamma_{n}\equiv
(1-\chi_{n})\frac{f(x,u_{n})}{||u_{n}||_{H^{1}}}\rightharpoonup
\gamma\q\text{in}\q L^{2}\tag{6.22}\label{chap6-eq6.22} 
\end{equation*}
where $\gamma$ is some $L^{2}$ function and $\gamma\geq 0$. Indeed
using \eqref{chap6-eq6.20} we have that
$$
|\gamma_{n}|\leq c_{1}h+c_{2}/||u_{n}||_{H^{1}}\leq c_{1}h+1\in L^{2}.
$$

The positiveness of $\gamma$ comes from the following
consideration. From the second assertion in \eqref{chap6-eq6.17} there
exists $r>0$ such that $f(x,s)\geq 0$ for $s\geq r$. Let $\xi_{n}$ be
characteristic function of the set $\{x\in \Omega:u_{n}(x)\geq
r\}$. Clearly $\xi_{n}\gamma_{n}\rightharpoonup \gamma$ in
$L^{2}$. And the assertion that $\gamma\geq 0$ follows from the fact
that $\xi_{n}\gamma_{n}$ is in the cone of non-negative functions of
$L^{2}$, which is closed and convex. Now go back to
\eqref{chap6-eq6.11}, divide it through by $||u_{n}||_{H^{1}}$ and
pass to the limit using \eqref{chap6-eq6.21} and
\eqref{chap6-eq6.22}. We obtain 
\begin{equation*}
\int\nabla v_{0}\nabla v-\int \gamma v=0\q \forall v\in
H^{1}_{0}.\tag{6.23}\label{chap6-eq6.23} 
\end{equation*}

It follows from the second assertion in \eqref{chap6-eq6.17} that
there are constants $\mu>\lambda_{1}$ and $c>0$ such that $f(x,s)\geq
\mu s-c$ for $x\in \Omega$ and $s\geq 0$. So $\gamma_{n}\geq \mu
v^{+}_{n}-c||u_{n}||^{-1}_{H^{1}}$. Passing to the limit we obtain
$\gamma\geq \mu v_{0}$. Using \eqref{chap6-eq6.23} with
$v=\varphi_{1}$, where $\varphi_{1}>0$ in $\Omega$ is a first
eigenfunction of $(-\Delta,H^{1}_{0}(\Omega))$, we obtain
$$
\lambda_{1}\int v_{0}\varphi_{1}=\int \nabla v_{0}\nabla
\varphi_{1}=\int \gamma\varphi_{1}\geq \mu\int v_{0}\varphi_{1}. 
$$

Since $\mu>\lambda_{1}$ we conclude that $v_{0}\equiv 0$. So from
\eqref{chap6-eq6.23} $\gamma\equiv 0$. Finally use
\eqref{chap6-eq6.11} again, divided through by $||u_{n}||_{H^{1}}$ and
with $v=v_{n}$:
$$
|\int|\nabla v_{n}|^{2}-\int \frac{f(x,u_{n})}{||u_{n}||}v_{n}|\leq
\epsilon_{n}||u_{n}||_{H^{1}}=\epsilon_{n}. 
$$

The first term is equal to $1$ and the other two go to zero, impossible!
\end{proof}

\begin{lemma}\label{chap6-lem6.7}
Assume \eqref{chap6-eq6.3} and \eqref{chap6-eq6.17} and suppose that
there are constants $\theta>2$ and $s_{0}>0$ such that
\begin{equation*}
0<\theta F(x,s)\leq sf(x,s),\q \forall x\in \Omega,\q \forall s\geq
s_{0}.\tag{6.24}\label{chap6-eq6.24} 
\end{equation*}

Then $\Phi$ satisfies $(PS)$ condition.
\end{lemma}

\begin{proof}
Let $(u_{n})$ be a sequence in $H^{1}_{0}$ for which
\eqref{chap6-eq6.10} and \eqref{chap6-eq6.11} holds. By
Lemma\pageoriginale \ref{chap6-lem6.2} we should show that
$||u_{n}||_{H^{1}}\leq $ const. We know from Lemma \ref{chap6-lem6.5}
that $||u^{-}_{n}||_{H^{1}}\leq$ const. Using the first assertion in
\eqref{chap6-eq6.17} we see that there are constants
$0<\mu<\lambda_{1}$ and $c>0$ such that $F(x,s)\leq
\frac{\mu}{2}s^{2}-cs$, for $x\in \Omega$ and $s\leq 0$.
$$
\int F(x,-u^{-}_{n})\leq \frac{\mu}{2}\int (u^{-}_{n})^{2}-c\int
u^{-}_{n}\leq \text{~const.}
$$

So from \eqref{chap6-eq6.10} we obtain
\begin{equation*}
\frac{1}{2}\int |\nabla u^{+}_{n}|^{2}-\int F(x,u^{+}_{n})\leq
\text{~const.}\tag{6.25}\label{chap6-eq6.25} 
\end{equation*}

Using \eqref{chap6-eq6.11} with $v=u^{+}_{n}$ we obtain
\begin{equation*}
|\int |\nabla u^{+}_{n}|^{2}-\int f(x,u^{+}_{n})u^{+}_{n}|\leq
\epsilon_{n}||u^{+}_{n}||_{H^{1}}.\tag{6.26}\label{chap6-eq6.26} 
\end{equation*}

Multiplying \eqref{chap6-eq6.25} by $\theta$ and subtracting
\eqref{chap6-eq6.26} from it we get
$$
\left(\frac{\theta}{2}-1\right)\int |\nabla u^{+}_{n}|^{2}\leq \int
[\theta
  F(x,u_{n}+)-f(x,u^{+}_{n})u^{+}_{n}]+\epsilon_{n}||u^{+}_{n}||_{H^{1}}+\text{~const.} 
$$

Using \eqref{chap6-eq6.24} we conclude that
$||u^{+}_{n}||_{H^{1}}\leq$ const.
\end{proof}

\begin{remark*}
If $f(x,s)\geq 0$ for $x\in \Omega$ and $s\leq 0$, then the eventual
solutions of \eqref{chap6-eq6.1} are $\geq 0$. Indeed, let $u\in
H^{1}_{0}$ be a solution of \eqref{chap6-eq6.1}, that is:
$$
\int_{\Omega}\nabla u \nabla v=\int_{\Omega}f(x,u)v\q \forall v\in
H^{1}_{0}. 
$$

Let $v=u^{-}$. Then
$$
-\int |\nabla u^{-}|^{2}=\int f(x,u)u^{-1}\geq 0\Rightarrow \int
|\nabla u^{-}|^{2}=0. 
$$

So $u^{-}=0$, proving the claim. Observe that under this hypothesis on
$f$, the first assertion in \eqref{chap6-eq6.17} holds. So Lemmas
\ref{chap6-lem6.6} and \ref{chap6-lem6.7} provide sufficient
conditions for $(PS)$ on a class of semilinear elliptic equations with
positive solutions. For example: (i) $f=|u|^{p}$ for any $1\leq
p<\infty$ if $N=2$ or $1\leq p<(N+2)/(N-2)$ if $N\geq 3$. (ii)
$f=(u^{+})^{p}$ with the same restrictions on $p$. 
\end{remark*}

\noindent
{\bf Existence results for (1).}~ To illustrate the use of the
theorems proved in Chapter \ref{chap5} we now consider some examples. 

\begin{example}\label{chap6-ex1}
Consider\pageoriginale the following Dirichlet problem
\begin{equation*}
-\Delta u=f(u)+h(x)\q \text{in}\q \Omega\q u=0\q \text{on}\q \p
\Omega\tag{6.27}\label{chap6-eq6.27} 
\end{equation*}
where $h\in C(\overline{\Omega})$ [a weaker condition would suffice]
and $f:\mathbb{R}\to \mathbb{R}$ is a continuous function such that
\begin{equation*}
\lim\limits_{s\to-\infty}\frac{f(s)}{s}=\alpha\q \text{and}\q
\lim\limits_{s\to
  +\infty}\frac{f(s)}{s}=\beta.\tag{6.28}\label{chap6-eq6.28} 
\end{equation*}

Let $0<\lambda_{1}<\lambda_{2}\leq \lambda_{3}\leq\ldots$ be the
eigenvalues of $(-\Delta,H^{1}_{0}(\Omega))$. 
\end{example}

\begin{theorem}[Dolph \cite{key33}]\label{chap6-thm6.8}
If $\lambda_{k}<\alpha$, $\beta<\lambda_{k+1}$, then problem
\eqref{chap6-eq6.27} has a solution for every $h$.
\end{theorem}

\begin{proof}
We look for critical points of the functional $\Phi:H^{1}_{0}\to
\mathbb{R}$ defined by
$$
\Phi(u)=\frac{1}{2}\int |\nabla u|^{2}-\int F(u)-\int hu
$$
where $F(s)=\int^{s}_{0}f$. It is easy to see that $(\alpha,\beta)$
does not belong to the singular set $\sum$. So by Lemma
\ref{chap6-lem6.3}, $\Phi$ satisfies $(PS)$. Let $V$ be the finite
dimensional subspace generated by the first $k$ eigenfunctions of
$(-\Delta,H^{1}_{0})$, and $W=V^{\perp}$. Let $\mu$ and
$\overline{\mu}$ be such that $\lambda_{k}<\mu<\alpha$,
$\beta<\overline{\mu}<\lambda_{k+1}$. It follows from
\eqref{chap6-eq6.28} that there exists $s_{0}>0$ such taht
$\mu<s^{-1}f(s)<\overline{\mu}$ for $|s|\geq s_{0}$. A straightforward
computations shows that there exist constants $C$ and $\overline{C}$
such that
\begin{equation*}
\frac{1}{2}\mu s^{2}-C\leq F(s)\leq
\frac{1}{2}\overline{\mu}s^{2}+\overline{C},\q \forall
s.\tag{6.29}\label{chap6-eq6.29} 
\end{equation*}

Now if $v\in V$ we estimate
$$
\Phi(v)\leq \frac{1}{2}\int |\nabla v|^{2}-\frac{\mu}{2}\int
v^{2}+C|\Omega|+||h||_{L^{2}}||v||_{L^{2}} 
$$
and using the inequality $\int|\nabla v|^{2}\leq \lambda_{k}\int
v^{2}$, for $v\in V$, and the Poincar\'e inequality we obtain
$$
\Phi(v)\leq \frac{1}{2}\left(1-\frac{\mu}{\lambda_{k}}\right)\int
|\nabla v|^{2}+C|\Omega|+||h||_{L^{2}}\lambda^{-1}_{1}||\nabla
v||_{L^{2}}. 
$$

So $\Phi(v)\to -\infty$ as $||v||\to \infty$ with $v\in V$. On the
other hand if $w\in W$ we estimate
$$
\Phi(w)\geq \frac{1}{2}\int |\nabla
w|^{2}-\frac{\overline{\mu}}{2}\int
w^{2}-\overline{C}|\Omega|-||h||_{L^{2}}||w||_{L^{2}} 
$$
and\pageoriginale using the inequality $\int|\nabla w|^{2}\geq
\lambda_{k+1}\int w^{2}$ for $\omega\in W$, we obtain
$$
\Phi(w)\geq
\frac{1}{2}\left(1-\frac{\overline{\mu}}{\lambda_{k+1}}\right)\int
|\nabla
w|^{2}-\overline{C}|\Omega|-||h||_{L^{2}}\lambda^{-1}_{1}||\nabla
w||_{L^{2}}. 
$$

So $\Phi(w)\to +\infty$ as $||w||\to \infty$ with $w\in W$. Therefore
the result follows from the Saddle Point Theorem, Theorem
\ref{chap5-thm5.8}. 
\end{proof}

\begin{remark*}
Condition \eqref{chap6-eq6.29} suffices to having the functional with
the ``shape'' of the Saddle Point Theorem. However we do not know if
it will give $(PS)$. Observe that $(PS)$ was obtained above with
hypothesis on the derivative of $F$ with respect to $s$, namely
$f$. Of course we are willing to assume growth conditions on $f$ to
assure the differentiability of $\Phi$, see Chapter \ref{chap2}. But
even so, it is not known if $(PS)$ holds. 
\end{remark*}

As a second example we consider problem (1) again and prove

\begin{theorem}[Ambrosetti-Rabinowitz \cite{key4}]\label{chap6-thm6.9}
Supposse that $f$ satisfies the conditions of Lemma
\ref{chap6-lem6.4}. In addition, assume
\begin{equation*}
\lim\limits_{s\to
  0}s^{-1}f(x,s)<\lambda_{1}.\tag{6.30}\label{chap6-eq6.30} 
\end{equation*}

Then problem $(1)$ has a nontrivial solution.
\end{theorem}

\begin{proof}
It follows from \eqref{chap6-eq6.30} that $f(x,0)=0$, and therefore
$u\equiv 0$ is a solution of \eqref{chap6-eq6.1}. Observe that
condition \eqref{chap6-eq6.16} implies (see Remark after the proof of
Lemma \ref{chap6-lem6.4}) that \eqref{chap6-eq6.15} holds. So Lemma
\ref{chap6-lem6.1} implies that $\Phi$ is not bounded below. Also
$(PS)$ holds, by Lemma \ref{chap6-lem6.4}. We plan to apply the
Mountain Pass Theorem, Theorem \ref{chap5-thm5.7}. For that matter we
study the functional $\Phi$ near $u=0$. Given $0<\mu<\lambda_{1}$, it
follows from \eqref{chap6-eq6.30} that there exists $\delta>0$ such
that $|f(x,s)|\leq \mu|s|$ for $|s|\leq \delta$. Therefore
$$
|F(x,s)|\leq \frac{\mu}{2}|s|^{2}\q \forall |s|\leq \delta.
$$

On the other hand using \eqref{chap6-eq6.3} we can find a constant
$k>0$ such that
$$
|F(x,s)|\leq k|s|^{p}\q \forall |s|>\delta.
$$

Here without loss of generality we may suppose $p>2$. Therefore adding
the two previous inequalities:
$$
|F(x,s)|\leq \frac{\mu}{2}|s|^{2}+k|s|^{p}\q \forall s\in \mathbb{R}.
$$

Thus\pageoriginale $\Phi$ can be estimated as follows
$$
\Phi(u)\geq \frac{1}{2}\int |\nabla u|^{2}-\frac{\mu}{2}\int
u^{2}-k\int |u|^{p}
$$
and using Poincar\'e inequality and Sobolev imbedding we obtain
$$
\Phi(u)\geq \frac{1}{2}\left(1-\frac{\mu}{\lambda_{1}}\right)\int
|\nabla u|^{2}-k\left(\int|\nabla u|^{2}\right)^{p/2}. 
$$

Since $p>2$ we see that there exists $r>0$ such that if
$||u||_{H^{1}}=r$ then $\Phi(u)\geq a$ for some constant $a>0$. So the
result follows immediately from Theorem \ref{chap5-thm5.7}. 
\end{proof}

\noindent
{\bf A Problem of the Ambrosetti-Prodi Type.}~ As a third example we
consider the Dirichlet problem
\begin{equation*}
-\Delta u=f(x,u)+t\varphi_{1}+h\q\text{in}\q \Omega,\q u=0\q
\text{on}\q \p \Omega,\tag{6.31}\label{chap6-eq6.31} 
\end{equation*}
where $t$ is a real parameter, $\varphi_{1}>0$ is a first
eigenfunction of $(-\Delta,H^{1}_{0})$ and $h\in
C^{\alpha}(\overline{\Omega})$, with $\int h\varphi_{1}=0$, is
fixed. Assume that $f$ is locally lipschitzian in
$\overline{\Omega}\times \mathbb{R}$. Then we prove.

\begin{theorem}\label{chap6-thm6.10}
Assume \eqref{chap6-eq6.17} and \eqref{chap6-eq6.20} [or
  \eqref{chap6-eq6.3}, \eqref{chap6-eq6.17},
  \eqref{chap6-eq6.24}]. Then it follows that there exists $t_{0}\in
\mathbb{R}$ such that for all $t\leq t_{0}$, problem $(31)$ has at
least two solutions in $C^{2,\alpha}(\overline{\Omega})$. 
\end{theorem}

\begin{remark*}
There is an extensive literature on problems of the Ambrosetti-Prodi
type, starting with work of Ambrosetti-Prodi \cite{key3}. We mention
Kazdan-Warner \cite{key52}, Amann-hess \cite{key1}, Berger-Podolak
\cite{key11}, H. Berestycki \cite{key9}, McKenna \cite{key57}, Ruf
\cite{key71} Solimini \cite{key76},\ldots There has been several
recent papers by Lazer-McKenna which we don't
survey them here. The result above and the proof next are due to de
Figueiredo-Solimini \cite{key43}. See a similar result by K.\@
C. Chang \cite{key25}.  
\end{remark*}

\begin{proof}
It follows from either Lemma \ref{chap6-lem6.6} [or Lemma
  \ref{chap6-lem6.7}] that the functional below satisfies $(PS)$: 
\begin{equation*}
\Phi(u)=\frac{1}{2}\int |\nabla u|^{2}-\int
F(x,u)-\int(t\varphi_{1}+h)u.\tag{6.32}\label{chap6-eq6.32} 
\end{equation*}

Also\pageoriginale the second assertion in \eqref{chap6-eq6.17}
implies by Lemma \ref{chap6-lem6.1} that $\Phi$ is not bounded
below. In the previous example we also proved the existence of two
solutions; however there we had the first solution $(u\equiv 0)$ to
start. Here we have first to obtain a solution of \eqref{chap6-eq6.31}
which is a local minimum, and then apply the Mountain Pass Theorem. We
break some steps of the proof in a series of lemmas.
\end{proof}

\begin{remark*}
For the next lemma we recall some definitions. We present them in the
particular framework of the problem considered here. We refer to
Gilbarg-Trudinger \cite{key46} for more general definitions.
\begin{itemize}
\item[(i)] A function $\omega\in H^{1}_{0}$ is said to be a {\em weak
  subsolution} of the Dirichlet problem

\item[(*)] $-\Delta u+Mu=g(x)$ in $\Omega$, $u=0$ on $\p \Omega$,
  where $g\in L^{2}$ and $M$ is a real constant, if
$$
\int \nabla\omega \nabla \psi+M\int \omega\psi\leq \int g\psi,\q
\forall \psi\in H^{1}_{0},\q \psi\geq 0.
$$

\item[(ii)] A function $\omega\in C^{2,\alpha}(\overline{\Omega})$ is
  a classical subsolution of (*) if 
$$
-\Delta \omega+M\omega\leq g\q \text{in}\q \Omega\q
\omega=0\q\text{on}\q \p\Omega.
$$

\item[(iii)] Weak supersolution and classical supersolution are
  defined likewise by reverting the inequalities in the equations
  above.

\item[(iv)] Every classical subsolution [supersolution] is also weak
  subsolution [supersolution].
\end{itemize}
\end{remark*}

\begin{lemma}\label{chap6-lem6.11}
Assume \eqref{chap6-eq6.17}. Then there exist constants
$0<\mu<\lambda_{1}<\overline{\mu}$ and $C>0$ such that
\begin{equation*}
f(x,s)>\underline{\mu}s-C\q\text{and}\q
f(x,s)>\overline{\mu}s-C\tag{6.33}\label{chap6-eq6.33} 
\end{equation*}
for all $x\in \Omega$ and all $s\in \mathbb{R}$.
\end{lemma}

\begin{proof}
The first assertion in \eqref{chap6-eq6.17} gives
$0<\underline{\mu}<\lambda_{1}$ and $C_{1}>0$ such that
$f(x,s)>\underline{\mu}s-c_{1}$ for all $x\in \Omega$ and $s\leq
0$. The second assertion in \eqref{chap6-eq6.17} gives
$\lambda_{1}<\overline{\mu}$ and $C_{2}>0$ such that
$f(x,s)>\overline{\mu}s-C_{2}$ for all $x\in \Omega$ and $s\geq
0$. Let $C=\max\{C_{1},C_{2}\}$ and \eqref{chap6-eq6.33} follows.
\end{proof}

\begin{lemma}\label{chap6-lem6.12}
Assume\pageoriginale \eqref{chap6-eq6.17}. Then for each
$t\in\mathbb{R}$ problem \eqref{chap6-eq6.31} has a classical
subsolution $\omega_{t}$. Moreover given any classical supersolution
$W_{t}$ of \eqref{chap6-eq6.31} [or in particular any solution of
  \eqref{chap6-eq6.31}] one has $\omega_{t}(x)<W_{t}(x)$, $\forall
x\in \Omega$, and
$$
\frac{\p \omega_{t}}{\p \nu}(x)>\frac{\p W_{t}}{\p \nu}(x),\q \forall
x\in \p \Omega.
$$
\end{lemma}

\begin{proof}
Let $M_{t}=\Sup_{\Omega}|t\varphi_{1}(x)+h(x)|$. The Dirichlet problem
$$
-\Delta u=\underline{\mu}u-C-M_{t}\q\text{in}\q \Omega\q
u=0\q\text{on}\q \p\Omega
$$
with $\underline{\mu}$ and $C$ as in \eqref{chap6-eq6.33} above, has a
unique solution $\omega_{t}\in C^{2,\alpha}(\overline{\Omega})$, which
is $<0$ in $\Omega$ and $\frac{\p \omega_{t}}{\p \nu}<0$ in
$\p\Omega$, see Lemma \ref{chap6-lem6.13} below. It follows from the
inequality in \eqref{chap6-eq6.33} that $\omega_{t}$ is a classical
subsolution. To prove the second statement, use \eqref{chap6-eq6.33}
again
$$
-\Delta (W_{t}-\omega_{t})\geq
f(x,W_{t})+t\varphi_{1}+h-\underline{\mu}\omega_{t}-C-M_{t}>\underline{\mu}(W_{t}-\omega_{t}) 
$$
and apply Lemma \ref{chap6-lem6.13}.
\end{proof}

\begin{lemma}\label{chap6-lem6.13}
Let $\alpha(x)$ be an $L^{\infty}$ function such that
$\Sup_{\Omega}\alpha(x)<\lambda_{1}$. Let $u\in H^{1}_{0}$ be such
that
\begin{equation*}
-\Delta u\geq \alpha(x)u\q \text{in}\q
H^{1}_{0}-\text{sense}.\tag{6.34}\label{chap6-eq6.34} 
\end{equation*}

Then $u\geq 0$. Moreover, if $u\in C^{2,\alpha}(\overline{\Omega})$
then $u>0$ in $\Omega$ and $\frac{\p u}{\p \nu}<0$ on $\p \Omega$. 
\end{lemma}

\begin{proof}
Expression \eqref{chap6-eq6.34} means
$$
\int \nabla u\nabla \psi\geq \int \alpha(x)u\psi,\q \forall \psi \in
H^{1}_{0},\q \psi\geq 0.
$$

Take $\psi=u^{-}$ and use Poincar\'e's inequality
$$
-\lambda_{1}\int (u^{-})^{2}\geq -\int |\nabla u^{-}|^{2}\geq -\int
\alpha(x)(u^{-})^{2}>-\lambda_{1}\int (u^{-})^{2}
$$
which implies $u^{-}=0$. So $u=u^{+}$. If $u\in
C^{2,\alpha}(\overline{\Omega})$ then we use the classical maximum
principle to
$$
-\Delta u+\alpha^{-}u\geq \alpha^{+}u,
$$
where the right side is $\geq 0$ by the first part of this lemma.
\end{proof}

\begin{lemma}\label{chap6-lem6.14}
Assume\pageoriginale \eqref{chap6-eq6.17}. Then there exists
$at_{0}\in \mathbb{R}$ such that for all $t\leq t_{0}$
\eqref{chap6-eq6.31} has a classical supersolution $W_{t}$.
\end{lemma}

\begin{proof}
Let $k=\int f(x,0)\varphi_{1}$ and $f_{1}(x)=f(x,0)-k\varphi_{1}$. So
the equation in \eqref{chap6-eq6.31} may be rewritten as
\begin{equation*}
-\Delta
u=f(x,u)-f(x,0)+(k+t)\varphi_{1}+h+f_{1}.\tag{6.35}\label{chap6-eq6.35} 
\end{equation*}

Now let $W_{t}$ be the solution of the Dirichlet problem
$$
-\Delta u=h+f_{1}\q \text{in}\q \Omega\q u=0\q \text{on}\q \p \Omega.
$$

We see that $W_{t}$ is a supersolution of \eqref{chap6-eq6.31}
provided $t$ is such that
$$
f(x,W_{t})-f(x,0)+(k+t)\varphi_{1}\leq 0
$$
or
$$
t\leq -k-\Sup_{\Omega}\frac{f(x,W_{t})-f(x,0)}{\varphi_{1}}. 
$$

So it remains to prove that the above $\Sup$ is $<+\infty$. It
suffices then to show that the function $g(x)\equiv
[f(x,W(x))-f(x,0)]/\varphi_{1}(x)$ is bounded in a neighborhood of $\p
\Omega$. Boundedness in any compact subset of $\Omega$ follows from
$\varphi_{1}(x)>0$ there. If $x_{0}$ is a point on $\p \Omega$, we use
the Lipschitz condition on $f$ to estimate
$$
|g(x)|\leq K|\frac{W_{t}(x)}{\varphi_{1}(x)}|
$$
where $K$ is a local lipschitz constant in a neighborhood $N$ of
$x_{0}$. The function $W_{t}(x)/\varphi_{1}(x)$ is bounded in $N:$ at
the points $x\in N\cap \p \Omega$, $W_{t}(x)/\varphi_{t}(x)=|\nabla
W_{t}(x)|/|\nabla \varphi_{1}(x)|$ by L'H\^ospital rule.
\end{proof}

\begin{remark*}
The above proof is taken from Kannan and Ortega \cite{key51}. Another
proof of existence of a supersolution for this class of problems can
be seen in Kazdan and Warner \cite{key52}.
\end{remark*}

\noindent
{\bf Proof of Theorem \ref{chap6-thm6.10} Continued.}~ From now on we
fix $t\leq t_{0}$ determined by Lemma \ref{chap6-lem6.14}. So by Lemma
\ref{chap6-lem6.12}, $w_{t}\leq W_{t}$, and in fact $w_{t}<W_{t}$ in
$\Omega$. Let
\begin{equation*}
C=\{u\in H^{1}_{0}:w_{t}\leq u\leq W_{t}\}.\tag{6.36}\label{chap6-eq6.36}
\end{equation*}
which is closed convex subset of $H^{1}_{0}$. {\em Plan of action:}
(i) restrict $\Phi$ to $C$ and show that it has a minimum $u_{0}$ in
$C$ which is a critical point of $\Phi$: (ii) show that $u_{0}$ is
indeed a local minimum of $\Phi$ in $H^{1}_{0}$; (iii) obtain a
$2^{nd}$ solution of \eqref{chap6-eq6.31}\pageoriginale using the
Mountain Pass Theorem, Proposition \ref{chap5-prop5.11}. To accomplish
the first step we need the following result.

\begin{proposition}\label{chap6-prop6.15}
Let $\Phi:X\to \mathbb{R}$ be a $C^{1}$ functional defined in a
Hilbert space $X$. Let $C$ be a closed convex subset of $X$. Suppose
that {\rm(i)} $K\equiv I-\Phi'$ maps $C$ into $C$, {\rm(ii)} $\Phi$ is
bounded below in $C$ and {\rm(iii)} $\Phi$ satisfies $(PS)$ in
$C$. Then there exists $u_{0}\in C$ such that $\Phi'(u_{0})=0$ and
$\Inf_{C}\Phi=\Phi(u_{0})$. 
\end{proposition}

\begin{proof}
Apply the Ekeland variational principle to $\Phi:C\to \mathbb{R}$. So
given $\epsilon>0$ there is $u_{\epsilon}\in C$ such that
$\Phi(u_{\epsilon})\leq \Inf_{C}\Phi+\epsilon$ and 
\begin{equation*}
\Phi(u_{\epsilon})\leq \Phi(u)+\epsilon||u-u_{\epsilon}||\q \forall
u\in C\tag{6.37}\label{chap6-eq6.37}
\end{equation*}

Put in \eqref{chap6-eq6.37} $u=(1-t)u_{\epsilon}+tKu_{\epsilon}$ with
$0<t<1$ and use Taylor's formula to expand
$\Phi(u_{\epsilon}+t(Ku_{\epsilon}-u_{\epsilon}))$ about
$u_{\epsilon}$. We obtain
$$
t||\Phi'(u_{\epsilon})||^{2}\leq \epsilon t||\Phi'(u_{\epsilon}||+o(t)
$$
which implies $||\Phi'(u_{\epsilon}||<\epsilon$. We then use $(PS)$ to
conclude. 
\end{proof}

\noindent
{\bf Back to the Proof of Theorem \ref{chap6-thm6.10}.}~ The idea now
is to apply Proposition \ref{chap6-prop6.15} to the functional $\Phi$
defined in \eqref{chap6-eq6.31} and $C$ defined in
\eqref{chap6-eq6.36}. However a difficulty appears in the verification
of condition (i). The way we see to solve this question is to change
the norm in $H^{1}_{0}$ as follows. We choose $M>0$ such that the
function
\begin{equation*}
s\mapsto g(x,s)\equiv
f(x,s)+Ms+t\varphi_{1}(x)+h(x)\tag{6.38}\label{chap6-eq6.38} 
\end{equation*}
is increasing in $s\in [a,b]$, for each $x\in \overline{\Omega}$
fixed, where $a=\min w_{t}$ and $b=\max W_{t}$. The norms in
$H^{1}_{0}$ given by
$$
||u||^{2}_{H^{1}}=\int |\nabla u|^{2}\q\text{and}\q ||u||^{2}=\int
|\nabla u|^{2}+M\int u^{2} 
$$
are equivalent. Let us denote by $\langle , \rangle$ the inner product
in $H^{1}_{0}$ corresponding to the second norm. Next we rewrite
\eqref{chap6-eq6.31} 
\begin{equation*}
-\Delta u+Mu=g(x,u)\q\text{in}\q \Omega\q u=0\q\text{on}\q \p
\Omega\tag{6.39}\label{chap6-eq6.39} 
\end{equation*}

The functional associated to \eqref{chap6-eq6.39} is
$$
\Psi (u)=\frac{1}{2}\langle u,u\rangle -\int G(x,u),\q
G(x,s)=\int^{s}_{0}g(x,\xi)d\xi, 
$$
which\pageoriginale is also $C^{1}$, it satisfies $(PS)$ and it has
the same critical points as the original functional $\Phi$. Now we
show that for such a functional, condition (i) of Proposition
\ref{chap6-prop6.15} holds. Indeed, let $u\in C$ and let
$v=(I-\Psi')u$. This means
$$
\la v,\psi\ra =\la u,\psi\ra -\la u,\psi\ra +\int g(x,u)\psi
$$
for all $\psi\in H^{1}_{0}$. Then
$$
\la v-w_{t},\psi\ra \geq \int [g(x,u)-g(x,w_{t})]\psi,\q \forall \psi
\in H^{1}_{0},\q \psi\geq 0
$$
and we obtain $v\geq w_{t}$ using the weak maximum
principle. Similarly $v\leq W_{t}$. Now we apply Proposition
\ref{chap6-prop6.15} and get a critical point $u_{0}$ of
$\Phi$. However the proposition insures only that $u_{0}$ is a minimum
of $\Phi$ restricted to $C$. To apply Proposition
\ref{chap5-prop5.11}, in order to obtain a second solution, we should
now prove that $u_{0}$ is indeed a local minimum. [This is not trivial
  since $C$ has empty interior in $H^{1}_{0}$]. Observe that $u_{0}\in
H^{1}_{0}$ is a solution \eqref{chap6-eq6.31}. It follows then from
the $L^{p}$ regularity theory of elliptic equations that $u_{0}\in
C^{2,\alpha}(\overline{\Omega})$. This proved by a standard bootstrap
argument. Suppose now that $u_{0}$ is {\em not} a local minimum of
$\Phi$. This means that for every $\epsilon>0$ there exists
$u_{\epsilon}\in B_{\epsilon}\equiv \overline{B_{\epsilon}(u_{0})}$
such that $\Psi(u_{\epsilon})<\Psi(u_{0})$. Now consider the
functional $\Phi$ restricted to $B_{\epsilon}$ and use Theorem
\ref{chap3-thm3.1}: there exists $v_{\epsilon}\in B_{\epsilon}$ and
$\lambda_{\epsilon}\leq 0$ such that
\begin{gather*}
\Psi(v_{\epsilon})=\Inf_{B_{\epsilon}}\Psi\leq
\Psi(u_{\epsilon})<\Psi(u_{0})\tag{6.40}\label{chap6-eq6.40}\\
\Psi'(v_{\epsilon})=\lambda_{\epsilon}(v_{\epsilon}-u_{0}).\tag{6.41}\label{chap6-eq6.41} 
\end{gather*}

Using again a bootstrap argument as above we conclude that
$v_{\epsilon}\in
C^{2,\alpha}(\overline{\Omega})$. \eqref{chap6-eq6.41} means
\begin{equation*}
\la v_{\epsilon},\psi\ra -\int
g(x,v_{\epsilon})\psi=\lambda_{\epsilon}\la
v_{\epsilon}-u_{0},\psi\ra,\q \forall \psi\in
H^{1}_{0}.\tag{6.42}\label{chap6-eq6.42} 
\end{equation*}

Clearly $v_{\epsilon}\to u_{0}$ in $H^{1}_{0}$ as $\epsilon\to 0$. We
have seen that $u_{0}$ and the $v_{\epsilon}$'s are $C^{2,\alpha}$
functions. Now we show that $v_{\epsilon}\to u_{0}$ in the norm of
$C^{1,\alpha}(\overline{\Omega})$. From \eqref{chap6-eq6.42} we obtain
\begin{equation*}
(1-\lambda_{\epsilon})\la v_{\epsilon}-u_{0},\psi\ra =\int
  [g(x,v_{\epsilon})-g(x,u_{0})]\psi\q \forall \psi \in
  H^{1}_{0}.\tag{6.43}\label{chap6-eq6.43} 
\end{equation*}

Again a bootstrap in the equation \eqref{chap6-eq6.43} gives the
claimed convergence. On the other hand, it follows from Lemma
\ref{chap6-lem6.12} that $w_{t}<u_{0}$ in $\Omega$ and $\frac{\p
  w_{t}}{\p \nu}>\frac{\p u_{0}}{\p\nu}$ in $\p\Omega$. Therefore by
the above convergence we have similar inequalities for
$v_{\epsilon}$\pageoriginale in place of $u_{0}$. A similar argument
with $W_{t}$. Thus $v_{\epsilon}\in C$ and we have a contradiction!

\medskip
\noindent
{\bf Final Remark.}~ As said before, problems of the Ambrosetti-Prodi
have been extensively studied in the literature. A direction not
touched in these notes is the question of obtain more than two
solutions. A remarkable progress has been made by H. Hofer and
S. Solimini through a delicate analysis of the nature of the critical
points. 

