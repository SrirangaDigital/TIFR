\chapter{Convex Lower Semicontinuous Functionals}\label{chap8}

{\bf Introduction.}~ In\pageoriginale Chapter \ref{chap5} we
introduced the concept of subdifferential of a convex lower
semicontinuous function $\Phi$. We observed that $\dom \p \Phi\subset
\dom \Phi$. Here we prove a result of Br{\o}ndsted and Rockafellar
\cite{key18} which show that $\dom \p \Phi$ is dense in $\dom
\Phi$. The main ingredients in the proof are the Ekeland Variational
Principle again and a calculus of subdifferentials. The proof here
follows Aubin-Ekeland \cite{key6} and differs from the original one
and also from that in Ekeland-Temam \cite{key36}. 

\begin{proposition}\label{chap8-prop8.1}
Let $\Phi$, $\Psi:X\to \mathbb{R}\cup \{+\infty\}$ be two convex lower
semicontinuous functionals defined in a Banach space $X$ and such that
$\Phi\equiv +\infty$ and $\Psi\equiv +\infty$. Then
\begin{gather*}
\p(t\Phi)(x)=t\p\Phi(x),\q \forall t>0\q \forall x\in
X\tag{8.1}\label{chap8-eq8.1}\\
\p(\Phi+\Psi)(x)\subset \p\Phi(x)+\p\Psi(x)\q \forall x\in
X.\tag{8.2}\label{chap8-eq8.2} 
\end{gather*}

Moreover if there is $\overline{x}\in \dom \Phi\cap \dom \Psi$ where
one of the functionals is continuous then there is equality in
\eqref{chap8-eq8.2}. 
\end{proposition}

\begin{proof}
\eqref{chap8-eq8.1} and \eqref{chap8-eq8.2} are straightforward. The
last assertion is proved using the Hahn Banach theorem; see the
details in Ekeland-Temam \cite{key36}.
\end{proof}

\noindent
{\bf Subdifferential and Differentiability.}~ The reader has surely
observed that these two notions are akin. Indeed, subdifferentials
were introduced to go through situations when one does not have a
differential. This was precisely\pageoriginale what occurred in
Chapter \ref{chap5}. We have the following result

\begin{proposition}\label{chap8-prop8.2}
Let $\Phi:X\to \mathbb{R}\cup\{+\infty\}$ be a convex lower
semicontinuous function. Suppose that $\Phi$ is Goteaux differentiable
at a point $x_{0}\in \dom \Phi$, i.e., there exists an element of
$X^{*}$, denoted by $D\Phi(x_{0})$, such that
\begin{equation*}
\Phi(x_{0}+tv)=\Phi(x_{0})+t\la D\Phi(x_{0}),v\ra+o(t),\q \forall v\in
X.\tag{8.3}\label{chap8-eq8.3} 
\end{equation*}
[The above expression is to hold for small $t$; how small is $t$
  depends on $v$].

Then
\begin{equation*}
x_{0}\in \dom \p \Phi\q\text{and}\q D\Phi(x_{0})\in \p
\Phi(x_{0}).\tag{8.4}\label{chap8-eq8.4} 
\end{equation*}

Moreover
\begin{equation*}
\p \Phi(x_{0})=D\Phi(x_{0}).\tag{8.5}\label{chap8-eq8.5}
\end{equation*}
\end{proposition}

\begin{proof}
To prove \eqref{chap8-eq8.4} we have to show that
\begin{equation*}
\Phi(y)\geq \Phi(x_{0})+\la D\Phi(x_{0}),y-x_{0}\ra,\q \forall y\in
X.\tag{8.6}\label{chap8-eq8.6} 
\end{equation*}

It $\Phi(y)=+\infty$ there is nothing to do. Assume that $y\in
\dom\Phi$. So the whole segment connecting $x_{0}$ to $y$ is in $\dom
\Phi$, and we have
$$
\Phi(x_{0}+t(y-x_{0}))\leq (1-t)\Phi(x_{0})+t\Phi(y).
$$

Using \eqref{chap8-eq8.3} we get, for small $t$:
$$
\Phi(x_{0})+t\la D\Phi(x_{0}),y-x_{0}\ra +o(t)\leq
(1-t)\Phi(x_{0})+t\Phi(y) 
$$
which implies \eqref{chap8-eq8.4} readily. Next, let us suppose that
$\overline{\mu}\in \p \Phi(x_{0})$:
\begin{equation*}
\Phi(y)\geq \Phi(x_{0})+\la\overline{\mu},y-x_{0}\ra\q y\in
X.\tag{8.7}\label{chap8-eq8.7} 
\end{equation*}

Now given $v\in X$ we know by assumption \eqref{chap8-eq8.3} that
$x_{0}+tv$, for small $t$, is in $\dom \Phi$. So from
\eqref{chap8-eq8.7} 
$$
\Phi(x_{0}+tv)-\Phi(x_{0})\geq t\la \overline{\mu},v\ra.
$$

Assuming $t>0$, dividing through by $t$ and passing to the limit we
obtain $\la D\Phi(x_{0}),v\ra\geq \la \overline{\mu},v\ra$ for all
$v\in X$, which implies $\overline{\mu}=D\Phi(x_{0})$.
\end{proof}

\begin{theorem}[Br{\o}ndsted-Rockafellar \cite{key18}]\label{chap8-thm8.3}
Let $X$ be a Banach space and $\Phi:X\to \mathbb{R}\cup\{+\infty\}$ a
convex lower semicontinuous function, such that\pageoriginale
$\Phi\equiv +\infty$. Then $\dom \p\Phi$ is dense in $\dom \Phi$. More
precisely, for any $\overline{x}\in \dom\Phi$, there exists a sequence
$(x_{k})$ in $X$ such that
\begin{gather*}
||x_{k}-\overline{x}||\leq 1/k\tag{8.8}\label{chap8-eq8.8}\\
\Phi(x_{k})\to \Phi(\overline{x})\tag{8.9}\label{chap8-eq8.9}\\
\p\Phi(x_{k})\neq \emptyset\q\text{for all}\q
k.\tag{8.10}\label{chap8-eq8.10} 
\end{gather*}
\end{theorem}

\begin{proof}
The set
$$
E=\{(x,a)\in X\times \mathbb{R}:\Phi(x)\leq a\}
$$
[called the {\em epigraph} of $\Phi$] is closed and convex. (Prove!)
Since $E\neq X$ we can take $(x_{0},a_{0})\not\in E$ and use Hahn
Banach theorem: there exists $\mu\in X^{*}$ and $\alpha\in \mathbb{R}$
such that
\begin{equation*}
\Psi(x)\equiv \Phi(x)-\la \mu, x\ra -\alpha>0\q\text{for all}\q x\in
X.\tag{8.11}\label{chap8-eq8.11} 
\end{equation*}
[there is a small step to get \eqref{chap8-eq8.11} from Hahn-Banach;
  see a similar situation in the proof of Proposition
  \ref{chap5-prop5.4}]. Now let us apply Theorem \ref{chap4-thm4.2} to
$\Psi$ with $\epsilon/2=\Psi(\overline{x})-\Inf_{X}\Psi$. So for each
$\lambda=1/k$, $k\in\mathbb{N}$, we obtain $x_{k}$ such that
\eqref{chap8-eq8.8} holds, and moreover
\begin{gather*}
\Psi(x_{k})\leq \Psi(\overline{x})\tag{8.12}\label{chap8-eq8.12}\\
\Psi(x_{k})<\Psi(x)+\epsilon k||x_{k}-x||\q\forall x\neq
x_{k}.\tag{8.13}\label{chap8-eq8.13} 
\end{gather*}

Next consider the functional $\Theta:X\to \mathbb{R}\cup \{+\infty\}$
defined by
$$
\Theta(x)=\Psi(x)+\epsilon k||x-x_{k}||,
$$
which is convex and lower semicontinuous. From \eqref{chap8-eq8.13} it
follows that $x_{k}$ is the (unique) global minimum of $\Theta$. So
$0\in \p \Theta(x_{k})$. Let $\Gamma:X\to \mathbb{R}$ be the convex
lower semicontinuous functional defined by $\Gamma(x)=||x-x_{k}||$. By
\ref{chap8-prop8.1} it follows that there are $z^{*}\in \p
\Psi(x_{k})$ and $w^{*}\in \p\Gamma(x_{k})$ such that
$0=z^{*}+\epsilon kw^{*}$. Using Proposition \ref{chap8-prop8.1} once
more and Proposition \ref{chap8-prop8.2} we see that there exists
$x^{*}\in \p \Phi(x_{k})$ such that $z^{*}=x^{*}-\mu$, proving
\eqref{chap8-eq8.10}. To prove \eqref{chap8-eq8.9} we rewrite
\eqref{chap8-eq8.12} in terms of $\Phi$:
\begin{equation*}
\Phi(x_{k})\leq \Phi(\overline{x})+\la
\mu,x_{k}-\overline{x}\ra.\tag{8.14}\label{chap8-eq8.14} 
\end{equation*}

Using \eqref{chap8-eq8.8} and \eqref{chap8-eq8.14} we have lim\,sup
$\Phi(x_{k})\leq \Phi(\overline{x})$. On the other hand, since $\Phi$
is weakly lower semicontinuous we obtain $\Phi(\overline{x})\leq
\lim\inf\Phi(x_{k})$. There two last inequalities prove
\eqref{chap8-eq8.12}. 
\end{proof}

\noindent
{\bf The Duality Mapping.}~ Let\pageoriginale $X$ be a Banach
space. In the theory of monotone operators $T:X\to 2^{X^{*}}$ a very
important role is played by the so-called duality mapping. It
essentially does in Banach space the job done by the identity in
Hilbert spaces. The {\em duality mapping} $J:X\to 2^{X^{*}}$ is
defined for each $x\in X$ by $J0=0$ and
\begin{equation*}
Jx=\{\mu\in X^{*}:\la
\mu,x\ra=||x||^{2},||\mu||=||x||\},\q\text{for}\q x\neq
0,\tag{8.15}\label{chap8-eq8.15} 
\end{equation*}
where we use the same notation for norms in both $X$ and
$X^{*}$. Expression \eqref{chap8-eq8.15} is equivalent to.
$$
Jx=\{\mu\in X^{*}:\la \mu, x\ra\geq ||x||^{2}, ||\mu||\leq ||x||\}.
$$

We can show directly from the definition, using Hahn-Banach, that $Jx$
is a non-empty $w^{*}$-closed convex subset of $X^{*}$, for each $x\in
X$. However, we will prove that and much more using the results on
subdifferentials proved in Chapter \ref{chap5}, after we prove the
following proposition.

\begin{proposition}\label{chap8-prop8.4}
Let $X$ be a Banach space, and $\Phi:X\to \mathbb{R}$ the functional
defined by $\Phi(x)=\frac{1}{2}||x||^{2}$. Then
\begin{equation*}
\p\Phi=J.\tag{8.16}\label{chap8-eq8.16}
\end{equation*}
\end{proposition}

\begin{remark*}
A study of generalized duality mappings can be found in Browder
\cite{key23}. 
\end{remark*}

\begin{proof}
Let $\mu\in \p \Phi(x)$. It suffices to consider $x\neq 0$, since
$\p\Phi(0)=0$ and $J0=0$. Then
\begin{equation*}
\frac{1}{2}||y||^{2}\geq \frac{1}{2}||x||^{2}+\la \mu,y-x\ra\q\forall
\in X.\tag{8.17}\label{chap8-eq8.17} 
\end{equation*}

Let $t>0$ and $v\in X$ be arbitrary, and replace in
\eqref{chap8-eq8.17} $y$ be $x+tv$:
$$
t\la \mu,v\ra \leq \frac{1}{2}||x+tv||^{2}-\frac{1}{2}||x||^{2}\leq
t||x||~||v||+\frac{1}{2}t^{2}||v||^{2}. 
$$

Dividing through by $t$ and passing to the limit
$$
\la \mu, v\ra \leq ||x||~||v||\q\forall v\in X.
$$

This implies $||\mu||\leq ||x||$. On the other hand, let $y=tx$ in
\eqref{chap8-eq8.17}: 
\begin{equation*}
\frac{1}{2}(t^{2}-1)||x||^{2}\geq (t-1)\la
\mu,x\ra.\tag{8.18}\label{chap8-eq8.18} 
\end{equation*}

For\pageoriginale $0<t<1$ we obtain from \eqref{chap8-eq8.18}
$\frac{1}{2}(t+1)||x||^2\leq \la \mu,x\ra$. Letting $t\to 1$ we have
$\la \mu,x\ra\geq ||x||^{2}$, completing the proof that $\mu\in
Jx$. The other way around, assume now that $\mu\in Jx$, and we claim
that \eqref{chap8-eq8.17} holds. Let us estimate the right side of
\eqref{chap8-eq8.17} using the properties of $\mu$:
$$
\la \mu,y\ra -\la \mu,x\ra+\frac{1}{2}||x||^{2}\leq
||x||~||y||-||x||^{2}+\frac{1}{2}||x||^{2}, 
$$
and clearly the right side of the above inequality is $\leq
\frac{1}{2}||y||^{2}$. 
\end{proof}

\setcounter{remark}{0}
\begin{remark}\label{chap8-rem1}
If $\mu\in J(x_{0})$ it follows from the above that $\mu$ {\em
  supports the ball} of radius $||x_{0}||$ around $0$ at the point
$x_{0}$. Indeed if $||y||\leq ||x_{0}||$, then $\la \mu,y\ra\leq \la
\mu,x_{0}\ra$. I.e. $\Sup\{\la\mu,y\ra:||y||\leq
||x_{0}||\}=\la\mu,x_{0}\ra$. 

Conversely, if a functional $\mu$ supports the unit ball at a point
$x_{0}$, then $\mu\in J(||\mu||x_{0})$. Indeed, from
$$
\la \mu,x_{0}\ra=\Sup_{||y||\leq 1}\la \mu,y\ra
$$
we obtain $\la \mu,x_{0}\ra=||\mu||$, which proves the claim.
\end{remark}

\begin{remark}\label{chap8-rem2}
Let us call $\mathcal{R}(J)=\cup \{Jx:x\in X\}$. If $X$ is a reflexive
Banach space, the above remark says that $\mathcal{R}(J)=X^{*}$. If
$X$ is not reflexive there is a result of R. C. James \cite{key49}
which says that there are functionals which do not support the unit
ball. So for non reflexive Banach spaces $\mathcal{R}(J)\neq
X^{*}$. However the Bishop Phelps theorem proved in Chapter
\ref{chap7} says that $\mathcal{R}(J)$ is {\em dense} in $X^{*}$, for
all Banach spaces.
\end{remark}

\noindent
{\bf G\^ateaux Differentiability of the Norm.}~ As in the previous
section let $\Phi(x)=\frac{1}{2}||x||^{2}$. Since $\Phi:X\to
\mathbb{R}$ is continuous and convex we have by Proposition
\ref{chap5-prop5.4} that
\begin{equation*}
\lim\limits_{t\downarrow 0}\frac{\Phi(x+ty)-\Phi(x)}{t}=\Max_{\mu\in
  J_{x}}\la \mu,y\ra,\q t>0.\tag{8.19}\label{chap8-eq8.19} 
\end{equation*}

It follows from \eqref{chap8-eq8.19} that
\begin{equation*}
\lim\limits_{t\uparrow 0}\frac{\Phi(x+ty)-\Phi(x)}{t}=\Min_{\mu\in
  Jx}\la \mu,y\ra,\q t<0.\tag{8.20}\label{chap8-eq8.20} 
\end{equation*}

The\pageoriginale functional $\Phi$ is G\^ateaux differentiable if and
only if the two limits in \eqref{chap8-eq8.19} and
\eqref{chap8-eq8.20} are equal and in fact there is a continuous
linear functional, noted by $D\Phi(x)$ and called the G\^ateaux
derivative at $x$, such there limits are equal to $\la
D\Phi(x),y\ra$. So the existence of the G\^ateaux derivative of $\Phi$
at a certain point $x$ implies that
$$
\Max_{\mu\in Jx}\la \mu,y\ra=\Min_{\mu\in Jx}\la \mu,y\ra\q\text{for
  all}\q y\in Y.
$$

Clearly this implies that $Jx$ is a singleton, and $Jx=D\Phi(x)$. The
converse is clearly true: if $Jx$ is singleton then
\eqref{chap8-eq8.19} and \eqref{chap8-eq8.20} imply that $\Phi$ is
G\^ateaux differentiable at $x$ and $D\Phi(x)=Jx$. So we have proved.

\begin{proposition}\label{chap8-prop8.5}
Let $X$ be a Banach space. $\Phi(x)=\frac{1}{2}||x||^{2}$ is G\^ateaux
differentiable at a point $x$ if and only if $Jx$ is a
singleton. Moreover $Jx=D\Phi(x)$. In particular, the duality mapping
is singlevalued, $J:X\to X^{*}$, if and only if
$\Phi(x)=\frac{1}{2}||x||^{2}$ is G\^ateaux differentiable at all
points $x\in X$.
\end{proposition}


\setcounter{remark}{0}
\begin{remark}\label{chap8-addrem1}
Which geometric properties of the Banach space $X$ give a singlevaled
duality mapping? For the definitions below let $B_{1}=\{x\in
X:||x||\leq 1\}$ and $\p B_{1}=\{x\in X:||x||=1\}$. In the terminology
introduced in Chapter \ref{chap7}, we see from the Hahn Banach theorem
that all points of $\p B_{1}$ are support points, i.e.\@ given $x\in
\p B_{1}$ there exists a functional $\mu\in X^{*}$ such that
$$
\Sup_{B_{1}}\mu=\la \mu,x\ra.
$$

In geometric terms we say that the ball $B_{1}$ has a hyperplane of
support at each of its boundary points. A Banach space $X$ is said to
be {\em strictly convex} if given $x_{1}$, $x_{2}\in \p B_{1}$, with
$x_{1}\neq x_{2}$, then $||tx_{1}+(1-t)x_{2}||<1$ for all
$0<t<1$. Another way of saying that it is: each hyperplane of support
touches $\p B_{1}$ at a unique point. (Or still $\p B_{1}$ contains no
line segments). A Banach space $X$ is said to be {\em smooth} if each
point $x\in \p B_{1}$ possesses only one hyperplane of
support. Examples in $\mathbb{R}^{2}$ with different norms: (i) the
Euclidean norm $||x||^{2}=x^{2}_{1}+x^{2}_{2}$ is both strictly convex
and smooth; (ii) the sup norm $||x||=\Sup\{|x_{1}|,|x_{2}|\}$ is
neither strictly convex non smooth; (iii) the norm whose unit ball is
$\{(x_{1},x_{2}):-1\leq x_{1}\leq 1,x^{2}_{1}-1\leq x_{2}\leq
1-x^{2}_{1}\}$ is strictly convex but not smooth: (iv) the norm whose
unit ball is the union of the three sets next is smooth but not
strictly convex:
$$
\{(x_{1},x_{2}):-1\leq x_{1},x_{2}\leq 1\},\{(x_{1},x_{2}):x_{1}\geq
1,(x_{1}-1)^{2}+x^{2}_{2}\leq 1\}
$$
and
$$
\{(x_{1},x_{2}):x_{1}\leq -1,(x_{1}+1)^{2}+x^{2}_{2}\leq 1\}.
$$
\end{remark}

\begin{remark}\label{chap8-addrem2}
With\pageoriginale the terminology of the previous remark, Proposition
\ref{chap8-prop8.5} states: ``$\Phi(x)=\frac{1}{2}||x||^{2}$\,{\em is
  G\^ateaux differentiable if and if $X$ is smooth}''. No condition on
reflexivity is asked from $X$.
\end{remark}

\begin{remark}\label{chap8-addrem3}
If $X^{*}$ is strictly convex then $J$ is singlevalued. Indeed, for
each $x_{0}\in X$, $Jx_{0}$ is a convex subset of the set $\{\mu\in
X^{*}:||\mu||=||x_{0}||\}$, which is a singleton in the case when
$X^{*}$ is strictly convex. So $X^{*}$ strictly convex implies that
$X$ is smooth, in view of Proposition \ref{chap8-prop8.5} and Remark
\ref{chap8-addrem2} above.
\end{remark}

\begin{remark}\label{chap8-addrem4}
There is a duality between strict convexity and smoothness in finite
dimensional Banach spaces: $X$ is strictly convex [resp.\@ smooth] if
and only if $X^{*}$ is smooth [resp.\@ strictly convex]. Such a result
does not extend to all Banach spaces, see Beauzamy \cite[p. 186]{key8}
for an example. However this is true for reflexive Banach spaces. This
was first proved by \u{S}mulian \cite{key75}, and it follows readily
from Remark \ref{chap8-addrem3} above and Remark \ref{chap8-addrem5}
below. 
\end{remark}

\begin{remark}\label{chap8-addrem5}
$X^{*}$ smooth implies $X$ strictly convex. Indeed suppose that $X$ is
  not strictly. Then there are $x_{1}$, $x_{2}\in \p B_{1}$ such that
  $\overline{x}=\frac{1}{2}(x_{1}+x_{2})\in \p
  B_{1}$. $J\overline{x}\in \p B^{*}_{1}$ where $\p B^{*}_{1}=\{\mu\in
  X^{*}:||\mu||=1\}$. We now claim that $x_{1}$ and $x_{2}$ viewed as
  elements of $X^{**}$ are two support functionals of $B^{*}_{1}$ at
  $J\overline{x}$, contradicting the smoothness of $X^{*}$. In fact
\begin{equation*}
1=\la J\overline{x},\overline{x}\ra=\frac{1}{2}\la
J\overline{x},x_{1}\ra+\frac{1}{2}\la
J\overline{x},x_{2}\ra\tag{8.21}\label{chap8-eq8.21} 
\end{equation*}
implies $\la J\overline{x},x_{1}\ra=1$ and $\la
J\overline{x},x_{2}\ra=1$, proving the claim.
\end{remark}

\noindent
{\bf Frechet Differentiability.}~ We know that a functional $\Phi:X\to
\mathbb{R}$ which is G\^ateaux differentiable is not in general
Fr\'echet differentiable. However

\begin{proposition}\label{chap8-prop8.6}
Let $X$ be a Banach space, A functional $\Phi:X\to \mathbb{R}$ is
continuously Fr\'echet differentiable (i.e.\@ $C^{1}$) if and only if
it is continuously G\^ateaux differentiable.
\end{proposition}

\begin{proof}
One of the implications is obvious. Let us assume that $\Phi$ is
G\^ateaux differentiable in a neighborhood $V$ of point $u_{0}\in X$
and that the mapping $x\in V\mapsto D\Phi(x)\in X^{*}$ is
continuous. We claim that $D\Phi(u_{0})$ is the Fr\'echet
derivative\pageoriginale at $u_{0}$, and, indeed:
\begin{equation*}
\Phi(u_{0}+v)-\Phi(u_{0})-\la
D\Phi(u_{0}),v\ra=o(v).\tag{8.22}\label{chap8-eq8.22} 
\end{equation*}

The real-valued function $t\in [0,1]\mapsto \Phi(u_{0}+tv)$ is
differentiable for small $v$. So by the mean value theorem
$\Phi(u_{0}+v)-\Phi(u_{0})=\la D\Phi(u_{0}+\tau v),v\ra$, which holds
for small $v$ and some $\tau\in [0,1]$. So we could estimate the left
side of \eqref{chap8-eq8.22} by $|D\Phi(u_{0}+\tau v)-D\Phi(u_{0})|$,
and using the continuity of the G\^ateaux derivative we finish.
\end{proof}

It is clear that a functional $\Phi:X\to \mathbb{R}$ could be Frechet
differentiable without being $C^{1}$. However this is not the case if
$\Phi(x)=\frac{1}{2}||x||^{2}$. 

\begin{proposition}\label{chap8-prop8.7}
Let $X$ be a Banach space. If $\Phi(x)=\frac{1}{2}||x||^{2}$ is
Fr\'echet differentiable [which implies Gateaux differentiable and $\p
  \Phi=J$, where $J$ is singlevalued] then $J:X\to X^{*}$ is a
continuous mapping.
\end{proposition}

\begin{remark*}
Without additional assumptions on the Banach space, a single-valued
duality mapping $J:X\to X^{*}$ is continuous from the strong topology
of $X$ to the $w^{*}$-topology of $X^{*}$. For simplicity let us
sketch the proof using sequences. Since the $w^{*}$-topology needs not
to be metrizable, filters should be used, see Beauzamy
\cite[p. 177]{key24}. Let $x_{n}\to x$ in $X$. Since
$||Jx_{n}||_{X^{*}}\leq$ const, there is $\mu\in X^{*}$ such that
$$
Jx_{n}\displaystyle{\mathop{\rightharpoonup}^{w^{*}}}\mu. 
$$

We claim that $\mu=Jx$. First, from $\la
Jx_{n},x_{n}\ra=||x_{n}||^{2}$ we conclude $\la
\mu,x\ra=||x||^{2}$. Next given $\epsilon>0$ there exists $u\in X$
with $||u||=1$ such that $||\mu||\leq \la \mu,u\ra+\epsilon$. So
$||\mu||\leq \la Jx_{n},u\ra+\epsilon\leq ||x_{n}||+\epsilon$ for
large $n$. Passing to the limit, and since $\epsilon>0$ is arbitrary
$||\mu||\leq ||x||$.\hfill$\Box$
\end{remark*}

\noindent
{\bf Proof of Proposition \ref{chap8-prop8.7}.}~ Suppose by
contradiction that there is a sequence $x_{n}\to x$ and $r>0$ such
that $||Jx_{n}-Jx||>2r$, for all $n$. So for each $n$ there exists
$y_{n}\in X$, $||y_{n}||=1$ such that
\begin{equation*}
\la Jx_{n}-Jx,y_{n}\ra>2r.\tag{8.23}\label{chap8-eq8.23}
\end{equation*}

Using the Frechet differentiability of $\Phi$ we can find $\delta>0$
such that 
\begin{equation*}
|\Phi(x+y)-\Phi(x)-\la Jx,y\ra|\leq r||y||\q\text{for}\q ||y||\leq
\delta.\tag{8.24}\label{chap8-eq8.24} 
\end{equation*}

On\pageoriginale the other hand we have
\begin{equation*}
\Phi(x+\delta y_{n})-\Phi(x_{n})\geq \la Jx_{n},x+\delta
y_{n}-x_{n}\ra.\tag{8.25}\label{chap8-eq8.25} 
\end{equation*}

Now using \eqref{chap8-eq8.24} we estimate
$$
\la Jx_{n}-Jx,\delta y_{n}\ra\leq \Phi(x+\delta y_{n})-\Phi(x_{n})+\la
Jx_{n},x_{n}-x\ra-\la Jx,\delta y_{n}\ra 
$$
which is equal to
\begin{equation*}
\Phi(x+\delta y_{n})-\Phi(x)-\la Jx,\delta
y_{n}\ra+\Phi(x)-\Phi(x_{n})+\la
Jx_{n},x_{n}-x\ra.\tag{8.26}\label{chap8-eq8.26} 
\end{equation*}

The first three terms in \eqref{chap8-eq8.26} we estimate using
\eqref{chap8-eq8.24}. So starting with \eqref{chap8-eq8.23} we get
$$
2r\delta<r\delta+\Phi(x)-\Phi(x_{n})+||x_{n}||~||x_{n}-x||
$$
which implies that $\Phi(x_{n})$ does not converge to $\Phi(x)$. This
contradicts the continuity of $\Phi$.\hfill$\Box$ 

As a consequence of the previous propositions we have the following
characterization of Fr\'echet differentiability of the norm:

\begin{proposition}\label{chap8-prop8.8}
Let $X$ be a Banach space. $\Phi(x)=\frac{1}{2}||x||^{2}$ is Frechet
differentiable if and only if the duality mappings is singlevalued and
continuous. 
\end{proposition}

\begin{remark*}
Which geometric properties of the Banach space $X$ give a continuous
singlevalued duality mapping? We start with a condition introduced by
\u{S}mulian \cite{key75}. $X^{*}$ satisfies {\em condition} $(S)$ if
for each $x\in \p B_{1}$ we have
\begin{equation*}
\lim\limits_{\delta\to 0}\diam
A_{x}(\delta)=0,\tag{8.27}\label{chap8-eq8.27} 
\end{equation*}
where $A_{x}(\delta)=\{\mu\in X^{*}:\la \mu,x\ra\geq 1-\delta\}\cap
B^{*}_{1}$. 
\end{remark*}

\begin{proposition}\label{chap8-prop8.9}
Let $X$ be a Banach space. The duality mapping is singlevalued and
continuous if and only if $X^{*}$ satisfies $(S)$.
\end{proposition}

\begin{proof}
(i) First assume $(S)$. Suppose that there exists $x\in \p B_{1}$ such
  that $Jx$ contains $\mu_{1}\neq \mu_{2}$. Clearly $\mu_{1}$,
  $\mu_{2}\in A_{x}(\delta)$, for all $\delta>0$, and
  $||\mu_{1}-\mu_{2}||>0$ negates \eqref{chap8-eq8.27}. To show the
  continuity of $J$, it suffices to prove that if $x_{n}\to x$, with
  $||x_{n}||=1$, then $Jx_{n}\to Jx$. We know that
$$
Jx_{n}\displaystyle{\mathop{\rightharpoonup}^{w^{*}}}Jx,
$$
so\pageoriginale it is enough to show that $(Jx_{n})$ is a Cauchy
sequence. Given $\epsilon>0$, choose $\delta>0$ such that $\diam
A_{x}(\delta)\leq \epsilon$. We know that $\la Jx_{n},x\ra\to \la
Jx,x\ra=1$. So there exists $n_{0}$ such that $Jx_{n}\in
A_{x}(\delta)$ for all $n\geq n_{0}$. Using condition $(S)$ we
conclude that $||Jx_{n}-Jx_{m}||\leq \epsilon$ for all $n$, $m\geq
n_{0}$.

(ii)~ Conversely, assume by contradiction that $(S)$ does not hold,
for some $x\in \p B_{1}$. So there exists $\epsilon_{0}>0$ such that
for every $n\in \mathbb{N}$ we can find $\mu_{n}$, $\mu'_{n}$ in
$B^{*}_{1}$ with the properties 
$$
\la \mu_{n},x\ra\geq 1-\frac{1}{n},\q \la \mu'_{n},x\ra \geq
1-\frac{1}{n},\q ||\mu_{n}-\mu'_{n}||\geq \epsilon_{0}. 
$$

We have seen, Remark \ref{chap8-rem2}, that $\mathcal{R}(J)$ is dense
in $X^{*}$. So we can find $x_{n}$, $y_{n}$ in $X$ such that
\begin{gather*}
\la Jx_{n},x\ra\geq 1-\frac{2}{n},\q \la Jy_{n},x\ra\geq
1-\frac{2}{n},\q ||Jx_{n}-Jy_{n}||\geq
\frac{\epsilon_{0}}{2},\tag{8.28}\label{chap8-eq8.28}\\ 
||Jx_{n}||\leq 1+\frac{1}{n},\q ||Jy_{n}||\leq
1+\frac{1}{n}.\tag{8.29}\label{chap8-eq8.29} 
\end{gather*}

Observe that the sequences $(Jx_{n})$, $(Jy_{n})$ are bounded in
$X^{*}$. By the Banach Alaoglu Theorem, there exists $\mu$ and $\mu'$
in $X^{*}$ such that
$$
Jx_{n}\displaystyle{\mathop{\rightharpoonup}^{w^{*}}}\mu,\q
Jy_{n}\displaystyle{\mathop{\rightharpoonup}^{w^{*}}}\mu'. 
$$
(As usual take subsequences if necessary). Passing to the limit in
\eqref{chap8-eq8.28} we obtain $\la \mu,x\ra\geq 1$, $\la
\mu',x\ra\geq 1$. From \eqref{chap8-eq8.29} it follows that in fact we
have $\la \mu,x\ra=\la \mu',x\ra=1$, which implies $\mu=\mu'=Jx$. On
the other hand, from the last assertion in \eqref{chap8-eq8.28} we can
find $z\in B_{1}$ such that
$$
\la Jx_{n}-Jy_{n},z\ra \geq \frac{\epsilon_{0}}{4}.
$$

Passing to the limit in this inequality we come to a contradiction.  
\end{proof}

\begin{remark*}
Now we give a geometric condition which is sufficient to having the
continuity and singlevaluedness of $J$. Some definitions. A Banach
space $X$ is said to be {\em uniformly convex} (Clarkson \cite{key26})
if given $\epsilon>0$ there exists $\delta=\delta(\epsilon)>0$ such
that if $x$, $y\in \p B_{1}$ and $||\frac{1}{2}(x+y)||\geq 1-\delta$
then $||x-y||\leq \epsilon$. A Banach space $X$ is said to be {\em
  locally uniformly convex} (Lovaglia \cite{key58}) if given
$\epsilon>0$ and $x_{0}\in\p B_{1}$ there exists
$\delta=\delta(\epsilon,x_{0})$ such that if $x\in B_{1}$ and
$||\frac{1}{2}(x+x_{0})||\geq 1-\delta$ then $||x-x_{0}||\leq
\epsilon$. The previous two definitions can be given in terms of
sequences as follows. $X$ is {\em uniformly convex} if given
and\pageoriginale two sequences $(x_{n})$ and $(y_{n})$ in $B_{1}$
such that $||\frac{1}{2}(x_{n}+y_{n})||\to 1$ then $||x_{n}-y_{n}||\to
0$. $X$ is locally uniformly convex if given any point $x_{0}\in \p
B_{1}$ and any sequence $(x_{n})$ in $B_{1}$ such that
$||\frac{1}{2}(x_{0}+x_{n})||\to 1$ then $x_{n}\to x_{0}$. A Banach
space $X$ is said to satisfy {\em Property} $H$ (Fan-Glicksberg
\cite{key40}) if $X$ is strictly convex and
\begin{equation*}
x_{n}\rightharpoonup x_{0},\q ||x_{n}||\to ||x_{0}||\Rightarrow
x_{n}\to x_{0}.\tag{8.30}\label{chap8-eq8.30} 
\end{equation*}

Hilbert spaces are uniformly convex. A uniformly convex Banach space
is locally uniformly convex. A locally uniformly convex Banach space
satisfies Property $(H)$. The first assertion is easily verified using
the fact that $x+y$ is orthogonal to $x-y$, for $x$, $y\in \p
B_{1}$. The second is trivial. And the third is proved as follows. It
is clear that a locally uniformly convex Banach space is strictly
convex. To prove \eqref{chap8-eq8.30} we may assume that
$||x_{n}||\leq 1$ and $||x_{0}||=1$. We claim that
$x_{n}\rightharpoonup x_{0}$ and $||x_{n}||\to ||x_{0}||$ implies that
$||\frac{1}{2}(x_{n}+x_{0})||\to 1$. Once this is done, the fact that
$X$ us supposed locally uniformly convex implies that $x_{n}\to
x_{0}$. Suppose by contradiction that a subsequence of $(x_{n})$,
denoted by $(x_{n})$ again, is such that
$||\frac{1}{2}(x_{n}+x_{0})||\leq t<1$, for all $n$. Let $\mu\in
Jx_{0}$. Then
$$
\la \mu,\frac{1}{2}(x_{0}+x_{n})\ra \leq \la \mu,tx_{0}\ra
$$
which implies $\frac{1}{2}+\frac{1}{2}\la \mu ,x_{n}\ra\leq
t$. Passing to the limit we come to a contradiction.
\end{remark*}

\begin{proposition}\label{chap8-prop8.10}
Let $X$ be a Banach space and suppose that $X^{*}$ is locally
uniformly convex. Then the duality mapping is singlevalued and
continuous. 
\end{proposition}

\begin{proof}
Singlevaluedness is clear. Now let $x_{n}\to x_{0}$ in $X$. As in
Proposition \ref{chap8-prop8.9} we may suppose $||x_{n}||=1$. We know
that
$$
Jx_{n}\displaystyle{\mathop{\rightharpoonup}^{w^{*}}}Jx_{0}.
$$

Suppose by contradiction (passing to a subsequence) that
$\frac{1}{2}||Jx_{n}+Jx_{0}||\leq t<1$. Then 
$$
(Jx_{n}+Jx_{0},x_{n}+x_{0})\leq 4t.
$$

On the other hand the left side of the above inequality is equal to
$$
(Jx_{n},x_{n})+(Jx_{0},x_{n})+(Jx_{n},x_{0})+(Jx_{0},x_{0})=2+(Jx_{0},x_{n})+(Jx_{n},x_{0}) 
$$ 
which converges to 4. Impossible!
\end{proof}

\begin{proposition}\label{chap8-prop8.11}
Let\pageoriginale $X$ be a reflexive Banach space. Then $J$ is
single-valued and continuous if and only if $X^{*}$ satisfies Property
$H$. 
\end{proposition}

\begin{proof}
It can easily be seen that, in the case of reflexive spaces, Property
$H$ on $X^{*}$ implies the said properties on $J$. To prove the
converse, we use Proposition \ref{chap8-prop8.9} and show that
Property $S$ implies Property $H$ on $X^{*}$; of course reflexivity is
used again. Suppose by contradiction that there exists a sequence
$$
\mu_{n}\rightharpoonup \mu_{0},\q ||\mu_{n}||\to ||\mu_{0}||,\q
\mu\nrightarrow \mu_{0}.
$$

We may suppose without loss of generality that $||\mu_{n}||\leq 1$ and
$||\mu_{0}||=1$. So there exist $\epsilon_{0}>0$ and sequences
$(j_{n})$ and $(k_{n})$ going to $\infty$ such that
\begin{equation*}
||\mu_{j_{n}}-\mu_{k_{n}}||\geq
\epsilon_{0}.\tag{8.31}\label{chap8-eq8.31} 
\end{equation*}

Since $J$ is singlevalued continuous and onto, we can find $x_{0}$ in
$X$ with $||x_{0}||=1$ such that $\mu_{0}=Jx_{0}$. Consider the set
$A_{x_{0}}(\delta)$ defined in \eqref{chap8-eq8.27} and choose
$\delta>0$ such that $\diam A_{x_{0}}(\delta)<\epsilon_{0}/2$. Clearly
$\mu_{n}\in A_{x_{0}}(\delta)$ for large $n$. But this contradicts
\eqref{chap8-eq8.31}. 
\end{proof}

\noindent
{\bf Uniform Fr\'echet Differentiability of the Norm.}~ In this
seciton we limit ourselves to Proposition \ref{chap8-prop8.12}
below. We refer to the books of Beauzamy \cite{key8} and Diestel
\cite{key32} for more on this subject. A concept of uniform smoothness
can be introduced and be shown to enjoy a duality with uniform
convexity, just like smoothness and strict convexity do.

\begin{proposition}\label{chap8-prop8.12}
The duality mapping $J$ is singlevalued and uniformly continuous on
bounded subsets of $X$ if and only if $X^{*}$ is uniformly convex. 
\end{proposition}

\begin{proof}
(i) We first prove that $X^{*}$ uniformly convex implies the said
  properties on $J$. From Proposition \ref{chap8-prop8.10} it follows
  that $J$ is singlevalued and continuous. To prove uniform continuity
  we preceed by contradiction. Assume that the exists $\epsilon_{0}>0$
  and sequences $(x_{n})$, $(y_{n})$ in some fixed bounded subset of
  $X$ such that
$$
||x_{n}-y_{n}||<\frac{1}{n}\q \text{and}\q ||Jx_{n}-Jy_{n}||\geq
\epsilon_{0}. 
$$

We may assume that $||x_{n}||=||y_{n}||=1$. Now we claim that
$\frac{1}{2}||Jx_{n}+Jy_{n}||\to 1$ as $n\to \infty$, arriving then at
a contradiction through the use of the uniform convexity\pageoriginale
of $X^{*}$. To prove the claim just look at the identity
$$
\la Jx_{n}+Jy_{n},x_{n}\ra=\la Jx_{n},x_{n}\ra+\la Jy_{n},y_{n}\ra+\la
Jy_{n},x_{n}-y_{n}\ra 
$$
and estimate to obtain
$$
2\geq \la Jx_{n}+Jy_{n},x_{n}\ra\geq 2-2||x_{n}-y_{n}||.
$$

Now we assume that $J$ is singlevalued and uniformly continuous on
bounded sets. By the fact that $\mathcal{R}(J)$ is dense in $X^{*}$ it
suffices to show that given $\epsilon>0$ there exists a $\delta>0$
such that
\begin{equation*}
||Jx-Jy||\geq \epsilon\Rightarrow \frac{1}{2}||Jx+Jy||\leq
1-\delta\tag{8.32}\label{chap8-eq8.32} 
\end{equation*}
for $x$, $y\in \p B_{1}$. First we write an identity for $u$, $v\in \p
B_{1}$ 
\begin{equation*}
\la Jx+Jy,u\ra=-\la Jx-Jy,v\ra+\la Jx,u+v\ra+\la Jy,
u-v\ra.\tag{8.33}\label{chap8-eq8.33} 
\end{equation*}

Observe that the sup of the left side with respect to $u\in \p B_{1}$
is the norm of $Jx+Jy$, and the sup of the first term in the right
side with respect to $v\in \p B_{1}$ gives the norm of $Jx-Jy$. Next
let $0<\epsilon'<\epsilon$ and choose $\overline{v}\in \p B_{1}$ such
that
\begin{equation*}
\la Jx-Jy,\overline{v}\ra\geq \epsilon'.\tag{8.34}\label{chap8-eq8.34}
\end{equation*}

Now choose $0<\xi<\epsilon'$. By the uniform continuity of $J$ there
is $\eta>0$ such that
\begin{equation*}
||Jz_{1}-Jz_{2}||\leq \xi\q \text{if}\q ||z_{1}-z_{2}||\leq
4\eta.\tag{8.35}\label{chap8-eq8.35}
\end{equation*}

Now take $v=\eta\overline{v}$ in \eqref{chap8-eq8.33} and let us
estimate separately the three terms in the right side of
\eqref{chap8-eq8.33}. The first is trivially estimated by
$-\epsilon'\eta$. The other two are estimated as follows:
$$
\la Jx,u+v\ra+\la Jy,u-v\ra\leq ||u+v||+||u-v||.
$$

Let $s=(u+v)/||u+v||$ and $d=(u-v)/||u-v||$, and write the
identity 
\begin{equation*}
||u+v||+||u-v||=(Js,u)+(Jd,u)+(Js-Jd,v).\tag{8.36}\label{chap8-eq8.36} 
\end{equation*}

Then estimate \eqref{chap8-eq8.36} using \eqref{chap8-eq8.35} and get
$$
||u+v||+||u-v||\leq 2+\xi||v||\leq 2+\xi \eta.
$$

Finally\pageoriginale \eqref{chap8-eq8.33} is estimated by
$-\epsilon'\eta+2+\xi\eta$, for all $u\in \p B_{1}$. So
$$
\frac{1}{2}||Jx+Jy||\leq 1-\frac{\epsilon'-\xi}{2}\eta.
$$
and \eqref{chap8-eq8.32} is proved.
\end{proof}

\begin{remark*}
The functional $\Phi$ is said to be {\em uniformly Frechet
  differentiable} if it is Frechet differentiable and if given
$\epsilon>0$ there exists $\delta>0$ such that
$$
|\Phi(x+u)-\Phi(x)-\la \Phi'(x),u\ra|\leq \epsilon||u||
$$
for all $x\in B_{1}$ and all $||u||\leq \delta$. It is easy to see
that the uniform differentiability of $\Phi$ is equivalent to $J$
being singlevalued and uniformly continuous on bounded sets. So the
norm of a Banach space is uniformely Frechet differentiable iff
$X^{*}$ is uniformly convex. 
\end{remark*}












