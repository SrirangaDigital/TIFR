
\chapter{Structure Theory}\label{c3}

In\pageoriginale this chapter we shall develop the structure theory of quadratic
Jordan algebras which is analogous to and is intimately connected with
the structure with the structure theory of semi-simple Artinion
associative rings. We consider first the theory of the radical of  a
quadratic Jordan algebra which was given recently by McCrimmon in
\cite{McCrimmon1}. McCrimmon's definition of the radical is analogous to that
of the Jacobson radical in the associative case and is an important
new notion even for Jordan algebras. We shall determine the structure
of the quadratic Jordan algebras which are semi-simple (that is have
$0$ radical) and satisfy the minimum condition for principal inner
ideals. Such an algebra is a direct sum of simple ones satisfying the
same minimum condition. The simple quadratic Jordan algebras
satisfying the minimum condition are either division algebras, outer
ideals containing $1$ in algebras $\mathscr{H}(\mathfrak{a},J)$ where
$(\mathfrak{a},J)$ is simple Artinian with involution, outer ideals
containing $1$ in quadratic Jordan algebras of quadratic forms with
base points (\S $1.7$), or certain isotopes
$\mathscr{H}(\mathscr{O}_3,J_c)$ of algebras
$\mathscr{H}(\mathscr{O}_3)$, $\mathscr{O}$ an actonion algebra over a
field (\S $1.8, 1.9$). The only algebras in this list which are of
capacity $\geqq 4$ (definition in \S 6) are outer ideals cintaining
$1$ in $\mathscr{H}(\mathfrak{a},J),(\mathfrak{a},J)$ simple Artinian
with involution. In this sense the general case in the classification
of simple quadratic Jordan algebras satisfying the
minimum\pageoriginale condition is constituted by the algebras defined
by the $(\mathfrak{a},J)$ we determined in Chapter$0$.

\newpage

These results reduce in the case of (linear) Jordan algebras to those
given in Chapter IV of the author's book [2].

\section{The radical of a quadratic Jordan algebra}\label{c3:sec1}

As in the associative ring theory the notation of the radical will be
based on the Quasi-invertiblity which we define as follows:

\setcounter{defn}{0}
\begin{defn}\label{c3:defn1}
An element $z$ of a quadratic Jordan algebra $\mathscr{J}$ is called
{\em quasi-invertible} if $1-z$ is invertible. If the inverse of $1-z$
is denoted as $1-w$ then $w$ is called {\em the quasi-inverse} of $z$.

The condition that $z$ be quasi-invertible with quasi-inverse $w$ are
\begin{equation*}
(1-w)U_{1-z}=1-z,(1-w)^{2}U_{1-z}=1.\tag{1}\label{c3:eq1}
\end{equation*}

Since $U_{1-z}=1+U_z-V_z$ the conditions are $1+z^{2}-2z-w-wU_z+
w\circ z=1-z, 1+z^{2}-2z-2w-2w U_z+2w\circ z +
w^{2}+w^{2}U_z-w^{2}\circ z=1$. These reduce to
\begin{gather*}
  w+z-z^{2}-w\circ z+wU_z=0\tag{2}\label{c3:eq2}\\
  2w-z^{2}+2wU_z-w^{2}U_z+2z-2w\circ z-w^{2}+w^{2}\circ
  z=0.\tag{3}\label{c3:eq3} 
\end{gather*}

The quasi-inverse of $z$ is
\begin{equation*}
  w=(z^{2}-z)U^{-1}_{1-z}\tag{4}\label{c3:eq4}
\end{equation*}
since $(1-z)^{-1}=(1-z)U^{-1}_{1-z}=(1-z)^{2}U^{-1}_{1-z} - (z^{2}-z)
U^{-1}_{1-z} =1-(z^{2}-z)U^{-1}_{1-z}$. 
\end{defn}

Also\pageoriginale $(1-z)\circ(1-w)=2$ which gives
\begin{equation*}
  z\circ w=2(z+w)\tag{5}\label{c3:eq5}
\end{equation*}

Then $1-z=(1-w)U_{1-z}=1-w+(1-w)U_z-(1-w)\circ
z=1-w+(1-w)U_z-2z+2z+2w=1+w+(1-w)U_z$. Thus
\begin{equation*}
  w+z+(1-w)U_z=0\tag{6}\label{c3:eq6}
\end{equation*}

An immediate consequence of this is
\setcounter{lemma}{0}
\begin{lemma}\label{c3:lem1}
  If $Z$ is an inner ideal and $z\epsilon Z$ is
  quasi-invertible then the quasi-inverse $w$ of $z$ is in
$Z$.
\end{lemma}

We prove next
\begin{lemma}\label{c3:lem2}
  If $z^{n}$ is quasi-invertible then $z$ is quasi-invertible.
\end{lemma}

\begin{proof}
  We have
  $1-\lambda^{n}=(1-\lambda)(1+\lambda+\ldots+\lambda^{n-1})$. Hence
  $U_{1-z^{n}}=U_{1-z}U_y=U_yU_{1-z}$ where $y=1+z+\ldots+z^{n-1}$ $(QJ
  37)$. Since $z^{n}$ is quasi-invertible, $U_{1-z^{n}}$ is
  invertible. Hence $1-z$ is invertible and $z$ is quasi-invertible.
\end{proof}

\begin{remarks*}
  The argument shows also that $-(z+z^{2}+\ldots+z^{n-1})$ is
  quasi-invertibel. The converse of lemma \ref{c3:lem2} is false since $-1$ is
  quasi-invertible in any (linear) Jordan algebra. On the other hand,
  $1=(-1)^{2}$ is not quasi-invertible.
  
  An immediate corollary of Lemma \ref{c3:lem2} is: Any nilpontent element is
  quasi-invertible. By a {\em nilpotent} $z$ we mean an element such
  that $z^{n}=0$ for some $n$. Then $z^{n}=0$ for all $m>2n$. Since no
  idempotent\pageoriginale $\neq 1$ is invertibel and $c$ idempotent
  implies $1-e$ idempotent it is clear that no idempotent $\neq 0$ is
  quasi-invertible. 
\end{remarks*}

\begin{defn}\label{c3:defn2}
  An ideal (inner ideal, outer ideal) $Z$ is called{\em
    quasi - invertible} if very $z\epsilon Z$ is
  quasi-invertible $Z$ is called {\em nil} if every $z\epsilon
  Z$ is nilpotent.
  
  The foregoing result shows that if $Z$ is nil then
  $Z$ is Quasi-invertible.
\end{defn}

\begin{lemma}\label{c3:lem3}
If $Z$ is a quasi-invertible ideal and $u\epsilon
\mathscr{J}$ is invertible then $u-z$ is invertible for every $z$.
\end{lemma}

\begin{proof}
$u-z$ is invertible if and only if $(u-z)^{2}U^{-1}_u$ is
  invertible. We have $(u-z)^{2}U^{-1}_u=(u^{2}-u\circ z
  +z^{2})U^{-1}_u=1-w$ where $w=(u\circ z-z^{2})U^{-1}_u\epsilon
  Z$. Then $w$ is quasi-invertible and $u-z$ is invertible.
\end{proof}

\begin{lemma}\label{c3:lem4}
If $Z$ and $\mathfrak{L}$ are quasi-invertible ideals then
$Z+\mathfrak{c}$ is a quasi-invertible ideal.
\end{lemma}

\begin{proof}
Let $x\epsilon Z, y\epsilon \mathfrak{c}$. Then
$1-(x+y)=(1-x)-y$ is invcertible by lemma \ref{c3:lem3} since $1-x$ is
invertible and $y\epsilon \mathfrak{c}$. Hence $x+y$ is
quasi-invertible. Thus every element of $Z+\mathfrak{c}$ is quasi-invertible.
\end{proof}

\begin{lemma}\label{c3:lem5}
If $z$ is quasi-invertible in $\mathscr{J}$ and $\eta$ is a
homomorphism of $\mathscr{J}$ into $\mathscr{J}'$ then $z^{\eta}$ is
quasi-invertible in $\mathscr{J}'$. If $Z$ is a
quasi-invertible ideal and $\ob{z}=z+Z$ is quasi invertible
in $\ob{\mathscr{J}}=\mathscr{J}/Z$ then $z$ is quasi invertible
in $\mathscr{J}$.
\end{lemma}

\begin{proof}
The first statement is clear since invertible elements are mapped into
invertible elements by a homomorphism. To prove the second result
consider $1-z$ where $\ob{z}$ where $\ob{z}$ is quasi-invertible in
$\ob{\mathscr{J}}=\mathscr{J}/Z$. We have a\pageoriginale
$w\epsilon \mathscr{J}$ such that $(1-w)^{2}U_{1-z}=1-y,y\epsilon
Z$. Since $Z$ is quasi-invertible, $1-y$ is
invertible. Hence $1-z$ is invertible (Theorem on inverses). Thus $z$
is quasi-invertible. 

We are now in position to prove our first main result.
\end{proof}

\setcounter{thm}{0}
\begin{thm}\label{c3:thm1}
There exists a unique maximal quasi-invertible ideal $\mathfrak{R}$ in
$\mathscr{J}.\mathfrak{R}$ contains every quasi-invertible ideal and
$\ob{\mathscr{J}}=\mathscr{J}/\mathfrak{R}$ contains no
quasi-invertible ideal $\neq 0$. 
\end{thm}

\begin{proof}
Let $\{Z_{\alpha}\}$ be the collection of quasi-invertible
ideals of $\mathscr{J}$ and put
$\mathfrak{R}=\cup Z_{\alpha}$. If $x,y\epsilon
\mathfrak{R}$, $x\epsilon Z_{\alpha},y\epsilon
Z_{\beta}$ for some $\alpha, \beta$. By lemma \ref{c3:lem4},
$Z_{\alpha}+Z_{\beta}=Z_{\gamma}$. Hence
$x+y\epsilon \mathfrak{R}$ and $x+y$ is quasi-invertible. It follows
that $\mathfrak{R}$ is a quasi-invertible ideal. Clearly
$\mathfrak{R}$ contains every quasi-invertible ideal so $\mathfrak{R}$
is the unique maximal quasi-invertible ideal. Now let
$\ob{Z}$ be a quasi-invertible ideal of
$\ob{\mathscr{J}}=\mathscr{J}/\mathfrak{R}$. Then
$\ob{Z}=Z/\mathfrak{R}$ where $Z$ is
an ideal of $\mathscr{J}$ containing $\mathfrak{R}$. Let $z\epsilon
Z$. Then $\ob{z}=z+\mathfrak{R}$ is quasi invertible in
$\ob{\mathscr{J}}$. Hence by Lemma \ref{c3:lem5}, $z$ is quasi-invertible in
$\mathscr{J}$. Hence $Z$ is a quasi-invertible ideal of
$\mathscr{J}$. Then $Z\subseteq \mathfrak{R},
Z=\mathfrak{R}$ and
$Z=Z/\mathfrak{R}=0$.

The ideal $\mathfrak{R}$ is called the ({\em Jacobson}){\em radical}
of $\mathscr{J}$ and will be denoted also as $\rad
\mathscr{J}\cdot\mathscr{J}$ is called {\em semi-simple} if $\rad
\mathscr{J}=0$. The second statement of Theorem $1$ is
$\ob{\mathscr{J}}=\mathscr{J}/\rad \mathscr{J}$ is semi-simple. Since
nil ideals are quasi-invertible $\rad \mathscr{J}$ contains every nil
ideal.
\end{proof}

\section{Properties of the radical}\label{c3:sec2}

We show first that $\rad \mathscr{J}$ is independent of the base ring
$\Phi$. This is a consequence of

\begin{thm}\label{c3:thm2}
Let\pageoriginale $\mathscr{J}$ be a quadratic Jordan algebra over
$\Phi,\mathfrak{R}=\rad\mathscr{J}$ and $\gamma$ the radical of
$\mathscr{J}$ regarded as a quadratic Jordan algebra over
$\mathbb{Z}$. Then $\mathfrak{R}=\gamma$.
\end{thm}

\begin{proof}
It is clear that $\mathfrak{R}\subseteq \gamma$. The reverse
inequality will follow if we can show that $\Phi\gamma$ the
$\Phi$-submodule generated by $\gamma$ is a quasi-invertible
ideal. The elements of $\Phi\gamma$ have the form
$\sum\alpha_iz_i,\alpha_i\epsilon \Phi$, $z_i\epsilon \gamma$. If
$a\epsilon \mathscr{J}$ then $(\sum
\alpha_iz_i)U_a=\sum\alpha_i(z_iU_a)\epsilon\Phi\gamma$ since
$z_iU_a\epsilon \gamma$. Also
$aU_{\sum\alpha_iz_i}=\sum\limits_{i}\alpha^{2}_iaU_{z_{i}}+\sum\limits_{i<j}\alpha_i
\alpha_j(aU_{z_i,z_j})$ since $aU_{z_{i}},aU_{z_{i},z_{j}}\epsilon
\gamma$ for any $a\epsilon\mathscr{J}$. Next we show that
$z=\sum\alpha_iz_i$ is quasi-invertible. The result just proved show
that $z^{3}= zU_z\epsilon \gamma$. Hence this is quasi-invertible and
$z$ is quasi-invertible by lemma $2$. Hence $\Phi\gamma$ is a
quasi-invertible ideal, which completes the proof.
\end{proof}

\begin{defn}\label{c3:defn3}
An element $a\epsilon \mathscr{J}$ is called {\em regular} if
$a\epsilon \mathscr{J}U_a\mathscr{J}$ is called {\em regular} if every
$a\epsilon \mathscr{J}$ is regular.

If $\Phi$ is a field and $a$ is an algebraic element of $\mathscr{J}$
then $\Phi[a]$ is finite dimensional (\S 1.10). Then
$\Phi[a]\Phi[a]U_a\supseteq
\Phi[a]U_{a^{2}}(U_{a^{2}}=U^{2}_a)\supseteq\Phi[a]U_{a^{3}}\ldots$
Hence we have an $n$ such that
$\Phi[a]U_{a^{n}}=\Phi[a]U_{a^{n+1}}=\ldots$ Then $a^{2n}\epsilon
\Phi[a]U_{a^{n}}=\Phi[a]U_{a^{2n}}$. Hence $a^{2n}$ is regular. Thus
if $a$ is algebraic ($\Phi$ a field) then there exists  a power of a
which is regular.
\end{defn}

\begin{thm}\label{c3:thm3}
(1) $\rad \mathscr{J}$ contains no non-zero regular elements. (2) If
  $\Phi$ is a field and $z\epsilon \rad \mathscr{J}$ is algebraic then
  $z$ is nilpotent.
\end{thm}

\begin{proof}
(i) Let\pageoriginale $z\epsilon \rad \mathscr{J}$ be regular, so
  $z=xU_z$ for some $x\epsilon \mathscr{J}$. Suppose first that $x$ is
  invertible. Since $zU_x\epsilon \rad \mathscr{J}, x-z U_x$ is
  invertible by lemma $3$. Hence
  $U_{x-zU_{x}}=U_x-U_{x,zU_{x}}+U_xU_zU_x$ is invertible in End
  $\mathscr{J}$. Now
\begin{align*}
zU_{x-zU_{x}}&=zU_x-zU_{x,zU_{x}}+zU_xU_zU_x\\
&=zU_x-zV_{x,z}U_x+zU_xU_zU_x\quad (QJ 4')\\
&=zU_x-xU_{z,z}U_x+zU_xU_zU_x\\
&=zU_x-2zU_x+xU_zU_x
\end{align*}

Since $xU_2=z$ this is $0$ and since $U_{x-zU_{x}}$ is invertible,
$z=0$. Now let $x$ be arbitrary. Since
$z=xU_z=xU_{xU_{z}}=xU_zU_xU_z$. We may replace $x$ by $xU_zU_x$ and
thus assume $x \in \rad \mathscr{J}$. Now $z^{2}(xU_z)^{2}=z^{2}U_xU_z(QJ
22)=yU_z$ where $y=z^{2}U_x\epsilon \rad \mathscr{J}$. Then $u=1+x-y$
is invertible by lemma \ref{c3:lem3}. Also $u U_z=z^{2}+z-z^{2}=z$. Hence $z=0$
since $u$ is invertible by the first case. (2) If $z\epsilon \rad
\mathscr{J}$ is algebraic then we have seen that $z^{2n}$ is regular
for some $n$. Since $z^{2n}\epsilon \rad \mathscr{J}$, (1) implied that
$z^{2n}=0$. Hence $z$ is nilpotent.
\end{proof}

\begin{thm}\label{c3:thm4}
  If $u$ is an invertible element of $\mathscr{J}$ then $\rad
  \mathscr{J}^{(u)}=\rad\mathscr{J}$. 
\end{thm}

\begin{proof}
Since $\rad \mathscr{J}$ is an ideal in $\mathscr{J}$ it is ideal in
the isotope $\mathscr{J}^{(u)}$. If $z\in  \rad \mathscr{J}$ then
$u^{-1}-z$ is invertible by lemma \ref{c3:lem3}. Then $u^{-1}-z$ is invertible
in $\mathscr{J}^{(u)}$. Since $u^{-1}$ is the unit of
$\mathscr{J}^{(u)}$ this states that $z$ is quasi-invertible in
$\mathscr{J}^{(u)}$. Thus $\rad \mathscr{J}$ is a quasi-invertible ideal of
$\mathscr{J}^{(u)}$. Hence $\rad \mathscr{J}\subseteq \rad
\mathscr{J}^{(u)}$. By symmetry $\rad \mathscr{J}=\rad
\mathscr{J}^{(u)}$.\pageoriginale 
\end{proof}

\section{Absolute zero divisors.}\label{c3:sec3}

We recall that $z$ is an absolute zero divisor (\S $1.10$) if
$U_z=0$. We shall now show that such a $z$ generates a nil ideal of
$\mathscr{J}$. For the proof we shall need some information on the
ideal (inner ideal, outer ideal) generated by a non-vacuous subset $S$
of $\mathscr{J}$. Clearly the outer ideal generated by $S$ is the
smallest submodule containing $S$ which is stable under all
$U_x,x\epsilon \mathscr{J}$. This is the set of $\Phi$-linear
combinations of the elements of the form $sU_{x_{1}}U_{x_{2}}\cdots
U_{x_{k}} s\in S, x_i \in \mathscr{J}$. We have seen also that the linear ideal
generated by a single element $s$ is $\Phi s+\mathscr{J}U_s$ (\S
$1.5$). Beyond this we have no information on the inner ideal
generated by a subset. We now prove

\begin{lemma}\label{c3:lem6}
If $Z$ is an inner ideal then the outer ideal
$\mathfrak{c}$ generated by $Z$ is an ideal.
\end{lemma}

\begin{proof}
The elements of $\mathfrak{c}$ are sums of elments of the form
$bU_{x1}U_{x2}\ldots\break U_{xk}$, $b\epsilon Z, x_i\epsilon
\mathscr{J}$. We have to show that $\mathfrak{c}$ is an inner
ideal. For this it suffices to prove that if $a,x_i\epsilon
\mathscr{J}$, $b\epsilon Z$ then
$aU_{bU_{x1}}\ldots U_{x_{k}}\epsilon \mathfrak{c}$ and for
$c,d\epsilon \mathfrak{c}, aU_{c,d}\epsilon \mathfrak{c}$. Since
$aU_{c,d}=\{cad\}$ the second is clear since $c\epsilon \mathfrak{c}$
and $\mathfrak{c}$ is an outer ideal (\S 1.5). For the first we use
$$
aU_{bU_{x_{1}}\ldots U_{x_{k}}}=aU_x\ldots U_{x_{1}}U_bU_{x_{1}}\ldots U_x.
$$
Since $Z$ is an inner ideal $aU_x\cdots
U_{x_{1}}U_b\epsilon Z$. Hence
$aU_{bU_{x_{1}}\cdots U_x}\epsilon \mathfrak{c}$.

Let $z$ be an absolute zero divisor. Then the inner ideal generated by
$z$ is $\Phi z$. Hence the last result shows that the ideal generated
by $z$ is the set of $\Phi$ linear combinations of elments of
the\pageoriginale form $zU_{x_{1}}U_{x_{2}}\cdots
U_{x_{k}},x_i\epsilon \mathscr{J}$. Since the ideal generated by a set
of elements is the sum of the ideals generated by the individual
elements we see that the ideal $\mathfrak{M}=zer \mathscr{J}$
generated by all the absolute zero divisors is the set of $\Phi$
-linear combinations of elements $zU_{x_{1}}U_{x_{2}}\cdots U_x, z$
an absolute zero divisor, $x_i\epsilon \mathscr{J}$. Since for $\alpha
\epsilon \Phi, \alpha z U_{x_{1}}U_{x_{2}}\cdots U_{x_k}$ is an absolute
zero divisor we have the following 
\end{proof}

\begin{lemma}\label{c3:lem7}
The ideal $\mathfrak{M}=zer \mathscr{J}$ generated by the absolute
zero divisors is the set of sums $z_1+z_2+\cdots + z_k$ where the
$z_i$ are absolute zero divisors
\end{lemma}
We prove next
\begin{lemma}\label{c3:lem8}
If $z$ is an absolute zero divisor and $y$ is nilpotent then $x=y+z$
is nilpotent. 
\end{lemma}

\begin{proof}
We show first that for $n=0,1,2,\ldots$
\begin{equation*}
  x^{n}=y^{n}+zM_{n-1}\tag{7}\label{c3:eq7}
\end{equation*}
where
\begin{equation*}
  M_{-1}=0, M_o=1,M_{n+1}=V_{y^{n+1}}+M_{n-1}U_y, n\geqq 0.\tag{8}\label{c3:eq8}
\end{equation*}

We note that by \eqref{c3:eq8} we have $M_1=V_y,M_2=V_{y^{2}}+U_y$ and in
  general
\begin{align*}
  M_{2k}&=V_{y^{2k}}+V_{y^{2k-2}}U_y+V_{y^{2k+1}}U_{y^{2}}+\cdots +
  U_{y^{n}},\geq 1\tag{9}\label{c3:eq9}\\
M_{2k+1}&=V_{y^{2k+1}}+V_{y^{2k-1}}U_y+\cdots + V_yU_{y^{k}},k\geqq 1
\end{align*}

Now\pageoriginale \eqref{c3:eq7} is clear if $n=0,1$. Assume it for $n$. Then
$x^{n+2}=x^{n}U_x=x^{n}(U_y+U_{y,z})$ (since
$U_z=0$)$=y^{n}U_y+zM_{n-1}U_y+y^{n}U_{y,z}+zM_{n-1}U_{y,z}$. Now
$y^{n}U_y=y^{n+2}$ and $y^{n}
U_{y,z}=\{zy^{n}y\}=zV_{y^{n},y}=zV_{y^{n+1}}$ ($QJ 38$). Hence
$$
x^{n+2}=y^{n+2}+z(M_{n-1}U_y+V_{y^{n+1}})+zM_{n-1}U_{y,z}
$$

Thus we shall have \eqref{c3:eq7} by induction if we can show that
$zM_{n-1}U_{y,z}=0$. By \eqref{c3:eq9}, this will follow if we can show that
$zV_{y^{i}}U_{y^{j}}U_{y,z}=0$ for $i>0$, $j\geqq 0$ and $z
U_{y^{j}}U_{y,z}=0$ for $j\geqq 0$. For the second of these we use
$zU_{y^{j}}U_{y,z}=yzU_{y^{j}}z=yV_{zU_{y^{j},z}}=yV_{y^{j},y^{j}U_z}$
$(QJ 31)=0$ since $U_z=0$. For the first we use the bilinearization of
$QJ 31$ relative to $a$: $V_{bU_{a,c,b}}=V_{aU_{b},c}+V_{cU_b,a}$ in
  $zV_{y^{i}}U_{y^{j}}U_{y,z}=yzV_{y^{i}}U_{y^{j}}z=
  yzU_{y^{j},y^{i+j}}z$ $(QJ
  39)=yV_{zU_{y^{j},y^{i,j,z}}}=y(V_{y^{j}U_{z},y^{i+j}}+V_{y^{i+j}U_z,y^{i}})=0$.
This proves \eqref{c3:eq7}. Since $y$ is nilpotent it is clear from
\eqref{c3:eq9} that
$y^{n}=0$ and $M_n=0$ for sufficiently large $n$. Hence $x^{n}=0$ by
\eqref{c3:eq7}.

Repeated application of Lemma \ref{c3:lem8} shows that if the $z_i$ are absolute
zero sivisors then $z_1+z_2+\cdots+z_k$ is nilpotent. Hence, by lemma
\ref{c3:lem7}, we have
\end{proof}

\begin{thm}\label{c3:thm5}
The ideal $\ker \mathscr{J}$ generated by the absolute zero divisors is
a nil ideal.
\end{thm}

It is clear from this that $\ker \mathscr{J}\subseteq \rad
\mathscr{J}$. It clear also that a simple quadratic Jordan algebra
contains no absolute zero divisors$\neq0$.

\section{Minimal inner ideals}\label{c3:sec4}

An\pageoriginale inner ideal $Z$ in $\mathscr{J}$ is called
{\em minimal (maximal)} if $Z \neq 0 (Z \neq
\mathscr{J}$) there exists no inner ideal $\mathfrak{c}$ in
$\mathscr{J}$ such that $Z\supset\mathfrak{c}\supset
0(Z\subset\mathfrak{c}\subset\mathscr{J})$. If
$Z$ is a minimal inner ideal then $\mathscr{J}U_b=0$ or
$\mathscr{J}U_b=Z$ for every $b\epsilon Z$ since
$\mathscr{J} u_b$ is an inner ideal contained in
$Z$. Similarly, either $ZU_b=0$ or
$Z U_b=Z$, $b\epsilon Z$. We shall
now prove a key result on minimal inner ideals which will serve as the
starting point of the structure theory. As a preliminary to the proof
we note the following

\begin{lemma*}
Let $a,b\epsilon \mathscr{J}$ satisfy $aU_b=b$. Then $E=U_aU_b$ and
$F=U_bU_a$ are idempotent elements of End $\mathscr{J}$ and if
$d=bU_a$ then $b$ and $d$ are {\em related} in the sense that $dU_b=b$
and $bU_d=d$.
\end{lemma*}

\begin{proof}
Since $aU_b=b,U_bU_aU_b=U_b$. Then $U_aU_bU_aU_b=U_aU_b$ and
$U_bU_aU_bU_a=U_bU_a$ so $E$ and $F$ are idempotents. If $d=bU_a$ then
$dU_a=bU_aU_b=aU_a U_aU_b=aU_b=aU_b=b$ and
$bU_d=aU_bU_aU_bU_a=aU_bU_a=bU_a=d$.
We shall now prove the following
\end{proof}
\noindent
{\textbf{Theorem on Minimal Inner Ideals.}}
Any minimal inner ideal $Z$ of $\mathscr{J}$ is of one of
the following types: $IZ=\Phi z$ where $z$ is a non-zero
absolute zero divisor, II $Z=\mathscr{J}U_b$ for every
$b\neq 0$ in $Z$ but $Z U_b=0$ and $b^{2}=0$ for
every $b\in Z$, III $Z$ is a Pierce inner ideal $\mathscr{J}
U_e,e^{2}=e$, such that $(Z,U,e)$ is  a division
algebra. Moreover, if $\mathscr{J}$ contains no idempotent $\neq 0,1$
and contains a minimal inner ideal $Z$ of type II then
$2\mathscr{J}=0$ and for every $b\neq 0$ in $Z$ there
exists an element $d\epsilon \mathscr{J}$ such that
\begin{itemize}
\item[(i)] $dU_b=b, bU_d=d, b^{2}=0=d^{2}, b\circ d=1,
  \mathscr{O}=\mathscr{J}U_d$ is a minimal\pageoriginale inner ideal of type II.

\item[(ii)] $c=b+d$ satisfies $c^{2}=1$, $c^{-1}=c$ and in the isotope
  $\mathscr{J}^{(c)}$, $b$ and $d$ are supplementary strongly
  connected orthogonal idempotents such that the Pierce inner ideals
  $\mathscr{J}U_b^{(c)}Z, \mathscr{J} U_d^{(c)}= \mathscr{O}$ are
  minimal of type III. 
\end{itemize}

\begin{proof}
Suppose first that $Z$ contains an absolute zero divisor
$z\neq 0$. Then $\Phi z$ is a non-zero inner ideal contained in
$Z$ so $Z=\Phi z$, by the minimality of
$Z$. From now on we assume that $Z$ contains no
absolute zero divisor $\neq 0$. Then $Z=\mathscr{J}U_b$ for
every $b\neq 0$ in $Z$. Also $Z U_b=0$ or
$Z U_b=Z$. Suppose $Z$ contains
$ab\neq 0$ such that $Z U_b=0$ and let $y\epsilon
Z$. Then there exists an $a\epsilon \mathscr{J}$ such that
$aU_b=y$. Then $Z U_y=Z U_bU_aU_b=0$. Thus either
$ZU_b=0$ for every $b\epsilon Z$ or
$ZU_b=Z$ for every $b\neq 0$ in
$Z$. In the first case $\mathscr{J}
U_bi=\mathscr{J}U^{2}_b=\mathscr{J}U_bU_b\subset Z U_b=0$,
$b\epsilon Z$ and since $b^{2}=1U_b\epsilon Z$
and $Z$ contains no absolute zero divisors $\neq 0$ we have
$b^{2}=0$, $b\epsilon Z$. Thus $Z$ is of type
II. Now assume $Z U_b=Z$ for every $b\neq 0$ in
$Z$. Let $b$ be such an element  and let $a\epsilon
Z$ satisfy $aU_b=b$. By the lemma, $b$ and $d=bU_a$ are
related. Also $d\epsilon Z$ and $E=U_bU_d$ and $F=U_d U_b$
are idempotent operators. We have $\mathscr{J}
E=\mathscr{J}U_bU_d=Z$ and $\mathscr{J} F=Z$ so
the restrictions $\ob{E}$ and $\ob{F}$ of $E$ and $F$ to
$Z$ are $Z$ the identiy on $Z$. Put
$e=d^{2}U_b\epsilon Z$, $f=b^{2}U_d\epsilon
Z$. Then $e\neq 0$ since $\mathscr{J}
U_e=\mathscr{J}U_bU^{2}_dU_b=Z$ and similarly
$f\neq 0$. Also
$e^{2}=(d^{2}U_b)^{2}=b^{2}U^{2}_dU_b=b^{2}U_d\ob{F}=b^{2}U_d=f$. Similarly,
$f^{2}=e$. Then $e^{2}=(f^{2})^{2}=f^{2}U_f(QJ 23)
=f^{2}U_dU^{2}_bU_d=f^{2}FE=f^{2}=e$. Then $e$ is a non-zero
idempotent in $Z,Z=\mathscr{J}U_e$ and
this\pageoriginale is a division algebra since $Z
U_b=Z$ for every non-zero $b$ hence $Z$ is type
III.

Now suppose $\mathscr{J}$ contains no idempotents $\neq 0,1$ and
contains the minimal inner ideal $Z$ of type II. Let
$b\neq  0$ in $Z$ and let $a\epsilon \mathscr{J}$ satisfy
$aU_b=b$. We claim that $c=a^{2}U_b=0$. Otherwise $c$ is a non-zero
element of $Z$ and there exist $b_o,c_o\epsilon
\mathscr{J}$ suchthat $b_oU_c=b$, $c_oU_c=c$. Put $e=c_oU_bU_a$. Then
$e^{2}=(c_oU_bU_a)^{2}=a^{2}U_{c_OU_b}U_a=a^{2}U_bU_{c_{o}}U_bU_a=cU_{c_o}U_bU_a=cU_{c_o}
U_cU_{b_o}U_cU_a=cU_{b_o}U_cU_a$
(since $cU_{c_o}U_c=c$ by the Lemma)
$=c_oU_cU_{b_o}U_cU_a=c_oU_{b_oU_c}U_a=c_oU_bU_a=e$. Since
$eU_aU_b=c_oU_bU^{2}_aU_b=c_oU_bU_{a^{2}}U_b=c_oU_{a^{2}U_b}=c_oU_c=c\neq0$. More
over, if $e=1$ then
$0=b^{2}U_a=1U_bU_a=eU_bU_a=c_oU_bU_aU_bU_a=c_oU_bU_a$
(Lemma)$=e$. Hence $e$ is an idempotent $\neq 0,1$ contrary to
hypothesis. Thus we have shown that $a^{2}U_b=0$ for every $a\epsilon
\mathscr{J}$ such that $aU_b=b$. Put $d=bU_a$. Then $bU_d=d$, $dU_b=b$
and $d^{2}U_b=0$. Then $d^{2}=(bU_d)^{2}=d^{2}U_bU_d=0$. By $QJ 30$,
$(b\circ d)^{2}=b^{2}U_d+d^{2}U_b+d^{2}U_b+bU_d\circ b= b\circ d$. By
$QJ 17$, $bU_{b\circ
  d}=-bU_{d,b}+bU_dU_b+bU_bU_d+bV_aU_bV_d=-b^{2}\circ d+
b+0+dV_bU_bV_d=b$ (since $dV_bU_b=dU_bV_b=b\circ b=0$). Hence $b\circ
d$ is an idempotent $\neq 0$ and so $b\circ d=1$. We have now
established all the relations on $b,d$ in (i). Now put $c=b+d$. Then
$c^{2}=b^{2}+b\circ d+d^{2}=1$. Hence $U_c$ is an automorphism such
that $U^{2}_c=1$ and $bU_c=bU_b+bU_{b,d}+ bU_d=d$. Thus $U_c$ maps the
minimal inner ideal $Z=\mathscr{J}U_b$ onto the minimal
inner ideal $\mathscr{O}=\mathscr{J} U_d$. Next we consider
$(d+1)U_b=dU_b+b^{2}=b$. As before, this implies $(d+1)^{2}U_b=0$ so
$0=2dU_b=2b$. Then $4Z=4\mathscr{J} U_b=\mathscr{J}
U_{2b}=0$. Since $2Z$ is an inner ideal contained in
$Z$, which\pageoriginale is minimal, this implies
$2Z=0$. Applying the automorphism $U_c$ shows that
$2\mathscr{O}=2ZU_c=0$. Also $2b=0$ implies
$2\mathscr{J}U_{b,d}=U_{2b,d}=0$. We now have
$2\mathscr{J}=2\mathscr{J}U_c=2\mathscr{J}U_b+2\mathscr{J}U_{b,d}+2\mathscr{J}U_d=0$.

Next we consider the isotope $\mathscr{J}^{(c)}$. We have
$c^{-1}=cU_c=(b+d)U_{b+d}=bU_b+bU_{b,d}+bU_d+dU_bdU_{b,d}+dU_d=b+d=c$. Hence
$c$ is the unit of $\mathscr{J}^{(c)}$. Since
$cU^{(c)}_b=cU_cU_b=cU_b=(b+d)U_b=b,b$ is idempotent in
$\mathscr{J}^{(c)}$. Hence $d=c-b$ is an idempotent orthogonal to $b$
in $\mathscr{J}^{(c)}$. We have $cU^{(c)}_1=cU_cU_1=c$ and $1=b\circ
d=1U_{b,d}\epsilon \mathscr{J}U_{b,d}^{(c)}$. Hence $b$ and $d$ are
strongly connected by $1$ in $\mathscr{J}^{(c)}$. Finally,
$\mathscr{J}U^{(c)}_b=\mathscr{J}U_cU_b=\mathscr{J}U_b=Z$
and $\mathscr{J}U^{(c)}_d=\mathscr{O}$ , so $Z$ and
$\mathscr{O}$ are the Pierce inner ideals determined by the idempotents
$b$ and $d$ in $\mathscr{J}^{(c)}$. Since $Z$ and
$\mathscr{O}$ are minimal inner ideals of $\mathscr{J}$ they are
minimal inner ideals of $\mathscr{J}^{(c)}$. Clearly they are of type
III in $\ob{\mathscr{J}}$.

All the possibilities indicated in the theorem can occur. To see this
consider $\Phi^{(q)}_2$ the special quadratic Jordan algebra of
$2\times 2$ matrices over a field $\Phi$. Then $\Phi e_{11}$ is a
minimal inner ideal of type III and $\Phi e_{12}$ is a minimal iner
ideal of type II in $\mathscr{J}=\Phi^{(q)}_2$. Next let
$\mathscr{J}=\Phi 1+\Phi_{e_{12}}$. This is a subalgebra of
$\Phi_2^{(q)}$ and $e_{12}$ is an absolute zero divisor in
$\mathscr{J}$. Hence $\Phi e_{12}$ is a minimal inner ideal of type
$I$. Finaly, assume $\Phi$ has characteristic two. Then
$\mathscr{J}=\Phi 1+\Phi e_{12}+\Phi e_{21}$ is a subalgebra of
$\Phi^{(q)}_2$ since $x=\alpha 1+\beta e_{12}+\gamma e_{21},\alpha,
\beta, \gamma \epsilon \Phi$, then $x^{2}=(\alpha^{2}+\beta \gamma)1$
so $x^{2}$ and $xy+yx\epsilon \Phi 1$ for $x,y\epsilon
\mathscr{J}$. Then $xyx+yx^{2}\epsilon \Phi x$ and $xyx\epsilon \Phi
x+\Phi y$. The formula for $x^{2}$ shows that $\mathscr{J}$ contains
no idempotents\pageoriginale $\neq 0,1$. Also $Z=\Phi
e_{12},\mathscr{O}=\Phi e_{21}$ are minimal inner ideals of type II in
$\mathscr{J}$ and $b=e_{12},d=e_{21}$ satisfy (i).
\end{proof}

\section{Axioms for the structure theory.}\label{c3:sec5}

We shall determine the structure of the quadratic Jordan algebras
which satisfy the following two conditions: (1) strong non-degeneracy
(=non-existance of absolute zero divisors $\neq 0$), (2) the
descending chain condition (D C C) for principal inner ideals.The
latter is equivalent to the minimum condition for principal inner
ideals. We have called a quadratic Jordan algebra $\mathscr{J}$
regular if for every $a\epsilon \mathscr{J}$ there exists $x\epsilon
\mathscr{J}$ such that $xU_a=a$ (\S 1.10, Definition $3$ of $\S
2$). Clearly this implies strong non-degeneracy.

\setcounter{lemma}{0}
\begin{lemma}\label{c3:sec5:lem1}
  If $\mathscr{J}$ is strongly non-degenerate and satisfies the DCC for
  principal inner ideals then every non-zero inner ideal $Z$
  of $\mathscr{J}$ contains a minimal inner ideal of $\mathscr{J}$.
\end{lemma}

\begin{proof}
If $b\neq 0$ is in $Z$ then $\mathscr{J}U_b$ is a principal
inner ideal contained in $Z$ and $\mathscr{J}U_b\neq 0$ by
the strong non-degeneracy. By the minimum condition (= D C C) for
principal inner ideals contained in contains a minimal element
$\mathfrak{K}$. We claim that $\mathfrak{K}$ is a minimal inner ideal
of $\mathscr{J}$. Otherwise, we have an inner ideal $Z$
such that $\mathfrak{K}\supset Z\supset 0$. The argument
used for $Z$ shows that $Z$ contains a non-zero
principal inner ideal $\mathscr{J}U_c$ and $\mathfrak{K}\supset
\mathscr{J} U_{c}$ contrary to the choice of $\mathfrak{K}$.

As a first application of this result we note that the foregoing
conditions (1) and (2) are equivalent to $:(1)$ and $(2')$
$\mathscr{J}$ is semi-simple. For, we have
\end{proof}

\begin{thm}\label{c3:thm6}
If\pageoriginale a quadratic Jordan algebra $\mathscr{J}$ satisfies
the DCC for principal inner ideals then $\mathscr{J}$ is semi-simple
if and only if $\mathscr{J}$ is strongly non-degenerate.
\end{thm}

\begin{proof}
If $z$ is an absolute zero divisor then the ideal generated by $z$ is
nil and so is contained in $\rad\mathscr{J}$. Hence is $\mathscr{J}$ is
semi-simple, so $\rad\mathscr{J}=0$, then $\mathscr{J}$ contains no
absolute zero divisors $\neq 0$. Conversely, suppose $rad
\mathscr{J}\neq 0$. Then we claim that $\mathscr{J}$ contains
non-zero absolute xero divisors. Otherwise, we can apply lemma $1$ to
conclude that $\rad \mathscr{J}$ contains a minimal inner ideal
$\mathfrak{K}$ of $\mathscr{J}$. $\mathfrak{K}$ is not of type III
since $\rad \mathscr{J}$ contains no non-zero idempotents. Also
$\mathfrak{K}$ is not of type II since in this case the Theorem on
Minimal Inner Ideals shows that every element of $\mathfrak{K}$ is
regular. Hence $\mathfrak{K}$ is os type $I$ and $\mathscr{J}$
contains an absolute zero divisor $\neq 0$, contrary to hypothesis.

It is immediate that if $\mathscr{J}$ satisfies the DCC for principal
inner ideals, or is strongly non-degenerate, or is regular then the
same condition holds for every isotope $\mathscr{J}^{(c)}$. The same
is true of the quadratic Jordan algebra $(\mathscr{J}U_e,U,e)$ if $e$
is an idempotent in $\mathscr{J}$. This follows from
\end{proof}

\begin{lemma}\label{c3:sec5:lem2}
Let $e$ be an idempotent in $\mathscr{J}$. Then any inner (principal
inner) ideal of $(\mathscr{J}U_e,U,e)$ is an inner(principal inner)
ideal of $\mathscr{J}$ and any absolute zero divisor of
$\mathscr{J}U_e$ is an absolute zero divisor of
$\mathscr{J}$. Moreover, if $\mathscr{J}$ is regular then
$\mathscr{J}U_e$ is regular.
\end{lemma}

\begin{proof}
If $Z$ is an inner ideal of $\mathscr{J} U_e$ and
$b\in Z$ then $b=bU_e$. Hence
$\mathscr{J}U_b=\mathscr{J}U_{bU_e}=(\mathscr{J}U_e)U_bU_e\subseteq
Z U_e=Z$. Hence $Z$ is an inner ideal
of $\mathscr{J}$. If $Z$ is principal in
$\mathscr{J}U_e,Z=\mathscr{J}U_eU_b,b\epsilon
\mathscr{J}U_e$. Then $b=bU_e$\pageoriginale so
$Z=\mathscr{J}
U_eU_eU_bU_e=\mathscr{J}U_eU_bU_e=\mathscr{J}U_b$ is a principal inner
ideal of $\mathscr{J}$. Let $z\epsilon \mathscr{J} U_e$ be an absolute
zero-divisor in $\mathscr{J}U_e$. Then $z=zU_e$. Hence if $x\epsilon
\mathscr{J}$ then $xU_z=(xU_e)U_zU_e=0$. Thus $z$ is an absolute zero
divisor in $\mathscr{J}$. FInally, suppose $\mathscr{J}$ is regular
and let $a=aU_e\epsilon \mathscr{J} U_e$. Then $a=xU_a$ for some
$x\epsilon \mathscr{J}$. Hence $a=xU_a=xU_eU_aU_e=(\gamma
U_e)U_eU_aU_e=x(U_e)U_a$. Since $xU_e\epsilon \mathscr{J} U_e$ this
shows that $\mathscr{J}U_e$ is regular.
\end{proof}

We shall now give the principal examples of quadratic Jordan algebras
which are strongly non-degenerate and satisfy the DCC for principal
inner ideals.

\begin{examples*}
\begin{enumerate}
\item If $\mathfrak{a}$ is asemi-simple right Artinian algebra then it
 is well-known that $\mathfrak{a}$ is left Artinian and every right
 (left) ideal of $\mathfrak{a}$ has a complementary right (left)
 ideal. Let $a\epsilon \mathfrak{a}$. Then $\mathfrak{a}$ a is a left
 ideal so there exists a left ideal $\mathfrak{J}$ such that
 $\mathfrak{a}=\mathfrak{a}a\oplus \mathfrak{J}$. Then $1=e+e'$ where
 $e\epsilon \mathfrak{a} a, e'\epsilon \mathfrak{J}$. It follows that
 $\mathfrak{a}a=\mathfrak{a} e$ and $e^{2}=e$. Then $ae=a$ and
 $e=xa$. Hence $ax a=a$ so $a\epsilon \mathfrak{a}^{(q)}U_a$. Thus
 $\mathfrak{a}^{(q)}$ is regular and consequently strongly
 non-degenerate. We note that
 $a\mathfrak{a}a=a\mathfrak{a}\cap\mathfrak{a}a$. Clearly
 $a\mathfrak{a}a\subseteq a \mathfrak{a} \cap a\mathfrak{a}_a$. On the
 other hand, if 
 $x=au=va$ then, by regularity, $x=xyx$, $y\epsilon \mathfrak{a}$ so
 $x=auyva \epsilon a\mathfrak{a} a$. Hence $a \mathfrak{a}\cap
   \mathfrak{a}a \subseteq a \mathfrak{a} a$. If $a\mathfrak{a} a
 \supseteq b\mathfrak{a} b$ then $b=bzb=awa\epsilon a\mathfrak{a}
 a$. Then $a\mathfrak{a}\supseteq b \mathfrak{a}$ and $\mathfrak{a} a
 \supseteq \mathfrak{a}b$. Since $\mathfrak{a}$ satisfies the
 descending chain condition on both left and right ideals it is now
 clear that $\mathfrak{a}^{(q)}$ satisfies the descending chain
 condition for principal inner ideals.

\item Let\pageoriginale $(\mathfrak{a}, J)$ be an associative algebra
  with involution. Suppose $\mathfrak{a}^{(q)}$ is regular. Then
  $\mathscr{H}(\mathfrak{a}, J)$ is regular. For if $h\epsilon
  \mathscr{H}(\mathfrak{a}, J)$ there exists an $a\epsilon
  \mathfrak{a}$ such that $hah=a$. Then $ha^{J}h=h$ so
  $h=ha(ha^{J}h)=h(aha^{J})h$ and $aha^{J}\epsilon \mathscr{H}$. Hence
  $\mathscr{H}$ is regular. We note next that if $\mathfrak{a}^{(q)}$
  is regular and satisifies the descending chain condition for
  principle inner ideals then $\mathscr{H}$ satisfies these
  conditions. We have seen that $\mathscr{H}$ is regular. Now suppose
  $\mathscr{H}U_{b_1}\supseteq \mathscr{H}
  U_{b_{2}}\supseteq\mathscr{H}\break U_{b_{3}}\ldots$ where $b_i\epsilon
  \mathscr{H}$. Then $b_{i+1}\in \mathscr{H}U_{b_{i+1}}$ by
  regularity so $b_{i+1}\in \mathscr{H}U_{b_{i}}$ and so
  $b_{i+1}=b_ih_ib_i,h_i\in \mathscr{H}$. Then
  $\mathfrak{a}U_{b_i}U_{b_i}U_{b_i}\subseteq \mathfrak{a}b_i$. Hence
  $\mathfrak{a}U_{b_1}\supseteq\mathfrak{a} U_{b_2}\supseteq\ldots$ is
  a descending chain of principal inner ideals of
  $\mathfrak{a}(q)$. Hence we have an $m$ such that
  $\mathfrak{a}U_{b_m}=\mathfrak{a}U_{b_{m+1}}=\ldots$ By regularity,
  $b_i\epsilon \mathfrak{a}U_{b_i}$ so if $n\geqq m$,
  $b_n=b_{n+1}a_{n+1}b_{n+1},a_{n+1}\epsilon \mathfrak{a}$. Then
  $b_n=b_{n+1}a_{n+1}^j b_{n+1}$. Also, by regularity of $\mathscr{H}$,
  $b_n=b_nk_nb_n,\break k_n\epsilon \mathscr{H}$. Then
  $b_n=(b_{n+1}a_{n+1}b_{n+1})k_n(b_{n+1}a_{n+1}^jb_{n+1})=b_{n+1}l_{n+1}\break
  b_{n+1}$
  where $l_{n+1}=a_{n+1}b_{n+1}k_nb_{n+1}a_{n+1}^j \in
  \mathscr{H}$. Hence $b_n\epsilon \mathscr{H}U_{b_{n+1}}$ and
  $\mathscr{H}U_{b_{n}}$ $\subseteq \mathscr{H}U_{b_{n+1}}$. Thus
  $\mathscr{H}U_{b_m}=\mathscr{H}U_{b_{m+1}}=\ldots$ and $\mathscr{H}$
  satisfies the DCC for principal inner ideals.

It is clear from (1) and the foregoing results that if
$(\mathfrak{a},J)$ is semi-simple Artinian with involution then
$\mathscr{H}(\mathfrak{a},J)$ is regular and satisfies the DCCfor
principal inner ideals.

\item If $\mathscr{J}$ is a quadratic Jordan algebra over $\Gamma$ and
  $\Phi$ is a subring $\Gamma$ then $\mathscr{J}/\Phi$ and
  $\mathscr{J}/\Gamma$ have the same principal inner ideals. Hence
  $\mathscr{J}/\Phi$ has DCC on these if and only if this holds for
  $\mathscr{J}/\Gamma$. It is clear also that $\mathscr{J}/\Phi$ is
  regular if and only if $\mathscr{J}/\Gamma$ is, and is\pageoriginale
  strongly non-degenerate if and only if $\mathscr{J}/\Gamma$ is. Now
  let $\Gamma$ be a field and let $\mathscr{J}=$ Jord $(Q,1),Q$ a
  quadratic form on $\mathscr{J}/\Gamma$ with vase point 1 (cf. \S
  1.7). We have the formulas
  $yU_x=Q(x,\ob{y})x-Q(x)\ob{y},\ob{x}=T(x)1-x,T(x)=Q(x,1)$ in $\mathscr{J}$. If
  $Q(x)\neq 0$ then $x$ is invertible and
  $\mathscr{J}U_x=\mathscr{J}$(\S 1.7). If $Q(x)=0$ the $Q(x)=0$ the
  formula for $U_x$ shows that $\mathscr{J}U_X\subseteq\Gamma x$. This
  implies that $\mathscr{J}/\Gamma$, hence $\mathscr{J}/\Phi$,
  satisfies the DCC for principal inner ideals. If $x\epsilon
  \mathscr{J}$ satisfies $Q(x)=0$, $Q(x,y)=0$, $y\epsilon \mathscr{J}$,
  then $U_x=0$. On the other hand, suppose $Q$ is non-degenerate. Then
  for $x\epsilon \mathscr{J}$ either $Q(x)\neq 0$ there exists a $y$
  such that $Q(x,y)\neq 0$. In either case $x\epsilon
  \mathscr{J}U_x$. Hence $\mathscr{J}=$ Jord $(Q,1)$ has non-zero
  absolute zero divisors or is regular according as $Q$ is degenerate
  or not.

We remark that the formula for $U$ shows that if $\mathfrak{K}$ is a
subspace such that $Q(k)=0$, $k\epsilon \mathfrak{K}$, then
$\mathfrak{K}$ is an inner ideal. This can be used to construct
examples of algebras which are regular with DCC on principal inner
ideal but not all inner ideals.

\item Let $\mathcal{O}$ be an octonion algebra over a field $\Gamma$,
$\Phi$ a subring of $\Gamma$. We consider
$\mathscr{H}(\mathcal{O}_3)$ as quadratic Jordan algebra over
$\Phi$(cf. \S\S 1.8, 1.9).since $\mathscr{H}(\mathcal{O}_3)$ is a
finite dimensional vector space over $\Gamma$, $\mathscr{H}/\Gamma$
satisfies the DCC for principal inner ideals. Hence
$\mathscr{H}/\Phi$ satisfies this condition. We proceed to show that
$\mathscr{H}$ is strong non-degenerate. Let $e_i=1[ii]$,
$f_i=1-e_i$ (notations as in \S 1.7). Then
$\mathscr{H}U_{f_{3}}=\mathscr{H}_{11}\oplus \mathscr{H}_{12}\oplus
\mathscr{H}_{22}=\{\alpha[11]+\beta[22]+a[12]|\alpha ,\beta\epsilon
\Phi, a\epsilon \mathcal{O}\}$. If $x\alpha[11]+\beta[22]+a[12]$ the
Hamilton-Cayley theorem in $\mathscr{H}(\mathcal{O}_3)$ shows that
$x^{3}-T(x)x^{2}+S(x)x=0$ (since $N(x)=0$, see \S
1.9). Here\pageoriginale $T(x)=\alpha+\beta$ and
$S(x)=T(x^{\sharp})=\alpha\beta-n(a)$ by direct calculation. Also,
direct calculation using the usual matrix square shows that
$x^{2}-T(x)x+S(x)f_3=0$. Hence $(\mathscr{H}U_{f_{3}},U,f_3)=$Jord $(S,
f_3)$ (see \S 1.7). Since the symmetric form $n(a,b)$ of the norm form
$n(a)$ of $\mathcal{O}$ is non-degenerate the same is trur of the
symmetric bilinear form of $S(x)=\alpha\beta-n(a)$. Hence $S(x)$ is
non-degenerate so $\mathscr{H}U_{f_3}=$ Jord $(S,f_3)$ is strongly
non-degenerate. By symmetry, $\mathscr{H}U_{f_i}$ is strongly
non-degenerate for $i=1,2$ also. Now let $\epsilon \mathscr{H}$ be an
absolute zero divisor in $\mathscr{H}$. Then $z U_{f_i}$ is an absolute
zero divisor in $\mathscr{H}U_{f_i}$ so $zU_{f_i}=0$,
$i=1,2,3$. Clearly this implies $z=0$ so $\mathscr{H}$ is strongly
non-degenerate.

\item It is not difficult to shows by an argument similar to that used
in  $2$ that if $\mathscr{J}$ is regular then any ideal $Z$
and $\mathscr{J}$ contianing $1$ is regular and if $\mathscr{J}$ is
regular and satisfies the minimum condion then the same is true of
$Z$. We leave the proofs to the reader.
\end{enumerate}
\end{examples*}

\section{Capacity}\label{c3:sec6}

An Idempotent $e\epsilon \mathscr{J}$ is called {\em primitive} If
$e\neq 0$ and $e$ is the only non-zero idempotent of
$\mathscr{J}U_e$. If $e$ is  not primitive and $e'$ is an idempotent
$\neq 0$, $e$ in $\mathscr{J}U_e$ then $e=e'+e''$ where $e'$ and $e''$
are orthogonal idempotents $\neq 0$. Conversely if $e=e'+e''$ where
$e'$ and $e''$ are orthogonal idempotents then $e',e''\epsilon
\mathscr{J} U_e$ (cf. \S 2.1) so $e$ is not primitive. Hence $e$ is
primitive if and only if it is impossible to write $e=e'+e''$ where
$e'$ and $e''$ are non-zero orthogonal idempotents. An idempotent $e$
is called {\em completely primitive} if $(\mathscr{J}U_e,U,e)$ is a
division algebra. Since a division algebra contains no idempotents
$\neq 0,1$ it is clear that if $e$ is completely primitive then $e$ is
primitive.

\setcounter{lemma}{0}
\begin{lemma}\label{c3:sec6:lem1}
If\pageoriginale $\mathscr{J}\neq 0$ satisfies the DCC for Pierce
inner ideals then $\mathscr{J}$ contains a (finite) supplementary set
of orthogonal primitive idempotents.
\end{lemma}

\begin{proof}
Consider the set of non-zero Pierce inner ideals of $\mathscr{J}$. By
the DCC on these there exists a minimal element $\mathscr{J}U_{e_{1}}$
in the set. Clearly $e_1$ is primitive. If $e_1=1$ we are
done. Otherwise, put $f_1=1-e_1$ so $f_1\neq0$ and consider
$\mathscr{J}U_{f_{1}}$. Since $f_1$ is an idempotent the hypothesis
carries over to 
$\mathscr{J}U_{f_1}$. Hence $\mathscr{J}U_{f_1}$ contains a primitive
idempotent $e_2$ and this is orthogonal to $e_1$. If $1=e_1+e_2$ we
are done. Otherwise, put $f_2=1 -e_1-e_2$ and apply the argument to
obtain a primitive idempotent $e_3$ in $\mathscr{J}U_{f_2}$. Also
$\mathscr{J}U_{f_1}\supseteq \mathscr{J}U_{f_2}$ since $f_1=e_2+f_2,
e_2\neq 0$. Now $e_3$ is orthogonal to $e_1$ and $e_2$ so if
$1=e_1+e_2+e_3$ we are done. Otherwise, we repeat the argument with
$f_3=1-e_1-e_2-e_3$. Then $\mathscr{J} U_{f_{2}}\supset
U_{f_2}\supset U_{f_3}\supset\ldots$ Since the DCC holds for Pierce
inner ideals this process terminates with a supplementary set of
orthogonal primitive idempotents.
\end{proof}

\begin{defn}\label{c3:defn4}
A quadratic Jordan algebra $\mathscr{J}$ is said to {\em have a
  capacity} if it contains a supplementary set of orthogonal
completely primitive idempotents. Then the minimum number of elements
in such a set is called {\em the capacity} of $\mathscr{J}$.
\end{defn}

\begin{thm}\label{c3:thm7}
If $\mathscr{J}$ is strongly non-degenerate and satisfies the DCC for
principal inner ideals then $\mathscr{J}$ has an isotope
$\mathscr{J}^{(c)}$ which has a capacity. If $\mathscr{J}$ has no two
torsion then $\mathscr{J}$ itself has a capacity.
\end{thm}

\begin{proof}
Let\pageoriginale $\{e_i\}$ be a supplementary set of orthogonal
primitive idempotents in $\mathscr{J}$ (Lemma
\ref{c3:sec6:lem1}). Suppose for some 
$i,\{e_i\}$ is not completely primitive. Since the hypothesis carry
over to $\mathscr{J} U_{e_i},U_{e_i}$ contains a minimal inner ideal
$Z$ (Lemmas \ref{c3:lem1}, \ref{c3:lem2} of \S \ref{c3:sec5}). Since
$\mathscr{J}U_{e_i}$ is 
not a division algebra $Z\subset\mathscr{J}U_{e_i}$ Now
$Z$ is not of type $I$ by the strong non-degeneracy and it
is not of type III since $Z\subset \mathscr{J}U_{e_i}$ and
$e_i$ is primitive. Hence $Z$ is of tyoe II. Also since
$\mathscr{J}U_{e_i}$ contains no idempotent $\neq 0$, $e_i$ the
Theorem on Minimal Inner Ideals implies that $2\mathscr{J}U_{e_i}=0$
and $\mathscr{J}U_{e_i}$ contains an element $c_i$ such that
$c^{2}_i=e_i$, $c^{-1}_i=c_i$ (in $\mathscr{J}U_{e_i}$) and in the
isotope $(\mathscr{J}U_{e_{i}})^{(c_i)}$, $c_i=b_i+d_i$ where
$b_i,d_i$ are orthogonal idempotents such that the corresponding
pierce inner ideals are minimal of type III. Let $c_1=e_j$ if $e_j$ is
completely primitive; otherwise let $c_j$ be as just indicated. Put
$c=\sum c_j$. Then $c$ is invertible and it is clear that
$\mathscr{J}^{(c)}$ has a capacity.

It is clear from the definition that $\mathscr{J}$ has capacity $1$ is
and only if $\mathscr{J}$ is a division algebra. We consider next the
algebras of  capacity two and we shall prove the following usefull
lemma for these
\end{proof}

\begin{lemma}\label{c3:sec6:lem2}
Let $\mathscr{J}$ have capacity two, so $1=e_1+e_2$ where the $e_i$
are orthogonal completely primitive idempotents,
$\mathscr{J}=\mathscr{J}_{11}\oplus \mathscr{J}_{12}\oplus
\mathscr{J}_{22}$ the corresponding Pierce decomposition. If $x\epsilon
\mathscr{J}_{12}$ either $x^{2}=0$ or $x$ invertible. The set of
absolute zero divisors of $\mathscr{J}$ is the set of $x\epsilon
\mathscr{J}_{12}$ such that $x^{2}=0$ and $x\circ y=0$, $y\epsilon
\mathscr{J}_{12}$ and this set is an ideal. Either $e_1$ and $e_2$ are
connected or every element of $\mathscr{J}_{12}$ is an\pageoriginale
absolute zero divisor. $\mathscr{J}$ is simple if and only if
$\mathscr{J}_{12}\neq 0$ and $\mathscr{J}$ is strongly
non-degenerate. If $\mathscr{J}$ is simple $e_1$ and $e_2$ are
connected and every outer ideal containing $1$ in $\mathscr{J}$ is
simple of capacity two.
\end{lemma}

\begin{proof}
If $x\epsilon \mathscr{J}_{12},x^{2}=x_1+x_2,x_i\epsilon
\mathscr{J}_{ii}$. Since $\mathscr{J}_{ii}$ is a division algebra,
either $x_i=0$ or $x_i$ is invertible in $\mathscr{J}_{ii}$. Clearly
if $x\neq 0$ and $x_2\neq 0$ then $x^{2}$ and hence $x$ is
invertible. Suppose $x_1=0$ so $x^{2}=x_2\epsilon
\mathscr{J}_{22}$. Then since $e_1\circ x=x$, and
$V_xV_{x^{2}}=V_{x^{2}}V_x$ we have $x_2\circ x=x^{2}\circ (e_1\circ
x)=(x^{2}\circ e_1)\circ x=(x_2\circ e_1)\circ x=0$. By PD $5$, if
$a_2\epsilon \mathscr{J}_{22}$ the mapping $a_2\to \ob{V}_{a_2}$ the
restriction of $V_{a_2}$ to $\mathscr{J}_{12}$ is a homomorphism of
$(\mathscr{J}_{22},U,e_{2})$ into (End
$\mathscr{J}_{12}$)$^{(q)}$. Since $\mathscr{J}_{22}$ is a division
algebra this is a monomorphism and  the image is a division subalgebra
of (End $\mathscr{J}_{12}$)$^{(q)}$. We recall also that invertibility
in (End $\mathscr{J}_{12}$)$^{(q)}$ is equivalent to invertibility in
End $\mathscr{J}_{12}$. Since we had $xV_{x_2}=0$ it now follows that
either $x_2=0$ or $x=0$. In either case $x^{2}=x_2=0$. Thus $x_1=0$
implies $x^{2}=0$ and, by symmetry, $x_2=0$ implies $x^{2}=0$. It is
now clear that either $x^{2}=0$ or $x$ is invertible.

Let $x\epsilon \mathscr{J}_{12}$ satisfy $x^{2}=0$, $x\circ y=0$ for
all $y\epsilon \mathscr{J}_{12}$. Let $a\epsilon
\mathscr{J}_{11}$. Then $aU_x\epsilon \mathscr{J}_{22}$ and
$(aU_x)^{2}=x^{2}U_aU_x=0$. Since $\mathscr{J}_{22}$ is a division
algebra this implies that $aU_x=0$. Similarly $bU_x=0$ if $b\epsilon
\mathscr{J}_{22}$. By $QJ 17$, $U_x=U_{x\circ
  e_1}=U_xU_{e_1}+U_{e_1}U_x+V_sxU_{e_1}V_x-U_{e_1}V_x-U_{e_{1}U_{x},e_{1}}=
U_xU_{e_1}+U_{e_1}U_x +V_xU_{e_1} V_x$ since $e_1U_x=0$ by the PD
theorem. If $y\epsilon \mathscr{J}_{12}$ we have
$yU_{e_{1}}=0=y_{12}V_x$. Hence $yU_x=yU_xU_{e_{1}}\epsilon
\mathscr{J}_{11}$. By symmetry $yU_x\epsilon \mathscr{J}_{22}$ so
$yU_x=0$ Thus $U_x=0$ and  $x$ is an absolute zero\pageoriginale
divisor. Conversely suppose $x$ is an absolute zero divisor. Then
$xU_{e_i}$ is an absolute zero divisor in the division algebra
$\mathscr{J}_{ii}$ so $xU_{e_{i}}=0$. Then $x=xU_{e_1,e_2}\epsilon
\mathscr{J}_{12}$. ALso $x^{2}=1U_x=0$ and if $y\epsilon
\mathscr{J}_{12}$ then $y\circ x\epsilon
\mathscr{J}_{11}+\mathscr{J}_{22}$ and $(y\circ
x)^{2}=y^{2}U_x+x^{2}U_y+yU_x\circ y(QJ 30)=0$. As before, this
implies that $y\circ x=0$. Hence the set of absolute zero divisors
coincides with the set of $x\epsilon \mathscr{J}_{12}$ such that
$x^{2}=0$ and $x\circ y=0,y\epsilon \mathscr{J}_{12}$. To see that
this set is an ideal it is enough to prove that it is closed under
addition. This is immediate.

Suppose $e_1$ and $e_2$ are not connected. Then $x^{2}=0$ for all
$x\epsilon \mathscr{J}_{12}$. Then $x\circ
y=(x+y)^{2}-x^{2}-y^{2}=0$. for all $x,y\epsilon
\mathscr{J}_{12}$. Then the preceding result shows that every
$x\epsilon \mathscr{J}_{12}$ is an absolute zero divisor.

Now suppose $\mathscr{J}_{12}\neq 0$ and $\mathscr{J}$ is strongly
non-degenerate. Let $Z$ be an ideal $\neq 0$ in
$\mathscr{J}$. We have $Z=ZU_{e_1}\oplus
Z U_{e_2}\oplus Z U_{e_1,e_2}$ and
$Z_{ii}\equiv Z U_{e_i}=Z \cap
\mathscr{J}_{ii},Z_{12}\equiv Z
U_{e_1,e_2}=Z\cap \mathscr{J}_{12}$. Since
$\mathscr{J}_{ii}$ is a division algebra and $Z_{ii}$ is an
ideal in $\mathscr{J}_{ii}$ either $Z_{ii}=0$ or
$Z_{ii}=\mathscr{J}_{ii}$. Since $Z \neq 0$,
$Z_{22}\neq 0$. If $Z_{11}\neq 0$ so
$Z_{11}=\mathscr{J}_{11}$ then $Z\supseteq
e_1\circ \mathscr{J}_{12}=\mathscr{J}_{12}$. Similarly, if
$Z_{22}\neq 0$ then $Z\supseteq
\mathscr{J}_{12}$. Next suppose $Z_{12}\neq 0$ and let
$x\neq 0$ in $Z_{12}$. Since $x$ is not an absolute zero
divisor either $x^{2}\neq 0$ or there exists $a y\epsilon
\mathscr{J}_{12}$ such that $x\circ y\neq 0$. In either case, since
$x^{2}$ and $x\circ y\epsilon \mathscr{J}_{11}+\mathscr{J}_{22}$ we
obtain either $Z_{11}\neq 0$ or $Z_{22}\neq
0$. Then, as before, $Z_{12}=\mathscr{J}_{12}$. Thus
$Z\supseteq \mathscr{J}_{12}$. Since $\mathscr{J}_{12}\neq
0$ and $\mathscr{J}$ is strongly non-degenerate $\mathscr{J}_{12}$
contains as invertible element. Then $Z$ contains an
invertible element and so $Z=\mathscr{J}$. Hence
$\mathscr{J}$ is simple. If $\mathscr{J}_{12}=0$ then
$\mathscr{J}=\mathscr{J}_{11}\oplus \mathscr{J}_{22}$ and
the\pageoriginale $\mathscr{J}_{ii}$ are ideals. Hence in this case
$\mathscr{J}$ is not simple. Also if $\mathscr{J}$ contains absolute
zero divisors$\neq 0$ then the set of these is an ideals and
$\mathscr{J}$ is not simple. Hence simplicity of $\mathscr{J}$ implies
$\mathscr{J}_{12}\neq 0$ and $\mathscr{J}$ is strongly non-degenerate.

If $\mathscr{J}$ is simple $\mathscr{J}_{12}$ contains an invertible
element. Then $e_1$ and $e_2$ are connected. If $Z$ is an
outer ideal containing $1$ then $\mathscr{b}$ contains the $e_i$ and
$\mathscr{J}_{12}=\mathscr{J}_{12}\circ e_i$ (of. the proof of Theorem
2.2) Clearly, this and the previous results imply that $Z$
is simple of capacity two.
\end{proof}

\section{First structure theorem}\label{c3:sec7}

The results of the last section have put us into position to prove
rather quickly the 

\noindent
{\textbf{First structure Theorem.}}
Let $\mathscr{J}$ be a strongly non-degenerate qua\-dra\-tic Jordan
algebra satisfying the DCC for principal inner ideals (equivalently,
by Theorem \ref{c3:thm6} $\mathscr{J}$ is semi-simple with DCC for principal
inner ideals). Then $\mathscr{J}$ is a direct sum of ideals which are
simple quadratic Jordan algebras satisfying the DCC on principal inner
ideals. Conversely, if $\mathscr{J}=\mathscr{J}_1\oplus\ldots\oplus
\mathscr{J}_s$ where the $\mathscr{J}_i$ are ideals which are simple
quadratic Jordan algebras with DCC on principal inner ideals then
$\mathscr{J}$ is strongly non-degenerate with DCC on prinicipal inner ideals.

\begin{proof}
By Theorem 7, $\mathscr{J}$ has an isotope
$\tilde{\mathscr{J}}=\mathscr{J}^{(c)}$ whose unit $c^{-1}=c$ is a sum of
completely primitive orthogonal idempotenets $e_i$. Let
$\mathscr{J}\sum\tilde{\mathscr{J}}_{ij}$ be the corresponding Pierce
decompostion. It is clear that $\tilde{\mathscr{J}}$ and hence every
Pierce inner ideal of $\mathscr{J}$ is strongly non-degenerate. If
$c=e_1$ so $\tilde{\mathscr{J}}=\mathscr{J}_{11}$ then $\mathscr{J}$ is  a
division algebra and the result is clear. Hence assume\pageoriginale
the number of $e_i$ is $>1$. Let $i\neq j$ and consider the Pierce
inner ideal
$\mathscr{J}U_{e_i+e_j}=\mathscr{J}_{ii}+\tilde{\mathscr{J}}_{ii}+\tilde{\mathscr{J}}_{ij}+\tilde{\mathscr{J}}_{jj}$. By
lemma \ref{c3:sec6:lem2} of the proceding section, either
$\tilde{\mathscr{J}}_{ij}=0$ 
or $e_i$ and $e_j$ are connected and
$\tilde{\mathscr{J}}_{ii}+\tilde{\mathscr{J}}_{ij}+\tilde{\mathscr{J}}_{jj}$
is simple. Since connectedness of orthogonal idempotents is a
transitive relation (\S 2.3) we may decompose the set of indices $i$
into non over-lapping subsets $I_1, I_2,\ldots,I_s$ such that if
$i,j\epsilon I_k$, $i\neq j$, the $e_i$ and $e_j$ are connected but if
$i\epsilon I_k$ and $j\epsilon I_l$, $k\neq l$, then $e_i$ and $e_j$
are not connected so $\mathscr{J}_{ij}=0$. Put
$1_k=\sum\limits_{i\epsilon I_k}e_i, \tilde{\mathscr{J}}_k
=\tilde{\mathscr{J}}U_{1_k}^{(1)}$. 
$\tilde{\mathscr{J}}_{k}=\tilde{\mathscr{J}}_1\oplus\ldots \oplus
\tilde{\mathscr{J}}_s$ and the Pierce relations show that
$\tilde{\mathscr{J}}_k$ is an ideal. We claim that $\mathscr{J}_k$ is
simple. We may suppose $I_k=\{1,2,\ldots,m\},m>1$. Then
$\tilde{\mathscr{J}}_k=\sum\limits_{i\geqq j=1}^{m}\tilde{\mathscr{J}}_{ij}$
and $e_i$ and $e_j$ are connected if $i\neq j\epsilon \{1,\ldots,
m\}$. Also
$\tilde{\mathscr{J}}_{ii}+\tilde{\mathscr{J}}_{ij}+\tilde{\mathscr{J}}_{jj}$ is
simple. Let $Z$ be a non-zero ideal in
$\tilde{\mathscr{J}}_k$. Then, as before, $Z=\sum
Z_{ij}$ where $Z_{ij}=\tilde{\mathscr{J}}_{ij}\cap
Z$. Since
$\tilde{\mathscr{J}}_{ii}+\tilde{\mathscr{J}}_{ij}+\tilde{\mathscr{J}}_{jj}$ is
simple either $Z$ contains this or
$Z_{ii}=Z_{ij}=Z_{jj}=0$. Clearly
$Z\neq 0$ implies that for some $i\neq j$ we have
$Z_{ii},Z_{jj}$ or $Z_{ij}\neq
0$. Then $Z\supseteq
\tilde{\mathscr{J}}_{ii},\tilde{\mathscr{J}}_{jj}$ and consequently
$Z\supseteq\tilde{\mathscr{J}}_{ll}, \tilde{\mathscr{J}}_{lr}$
for all $l,r\epsilon\{1,\ldots, m\}$. Then
$Z=\tilde{\mathscr{J}}_{k}$ and $\tilde{\mathscr{J}}_{k}$ is
simple. Since $\mathscr{J}$ is an isotope of $\tilde{\mathscr{J}}$ we
have $\mathscr{J}=\mathscr{J}_1\oplus\ldots \oplus \mathscr{J}_s$
where $\mathscr{J}_i=\tilde{\mathscr{J}}_{i}$ as module, id an ideal of
$\mathscr{J}$ which is a simple algebra (since any ideal of
$\mathscr{J}_i$ is an ideal of $\mathscr{J}$ because of the direct
decomposition) Since $\mathscr{J}_i$ is a Pierce inner ideals of
$\mathscr{J}$ it satisfies the DCC for principal inner ideals.

Conversely, suppose
$\mathscr{J}=\mathscr{J}_1\oplus\mathscr{J}_2\oplus\ldots\oplus
\mathscr{J}_s$ where $\mathscr{J}$ is an ideal and is a simple
quadratic Jordan algebra with unit $1_i$, satisfying the\pageoriginale
DCC in principal inner ideals. Since the absolute zero divisors
generate a nil ideal $\mathscr{J}_i$ is strongly non-degenerate. If
$z$ is an absolute zero divisor in $\mathscr{J}$ in $\mathscr{J}$ then
$z=\sum z_i,z_i=zU_{1_i}$, and $z_i$ is an absolute zero divisor of
$\mathscr{J}_i$. Hence $z_i=0$, $i=1,2,\ldots,s$ and $z=0$. Thus
$\mathscr{J}$ is strongly non-degenerate. Let $a\epsilon \mathscr{J}$
and write $a=\sum a_i,a_i= aU_{1_i}$, then it is immediate that
$\mathscr{J}U_a=\sum\mathscr{J}U_{a_i}=\sum\mathscr{J}_iU_{a_i}$. Also
if $b=\sum b_i, b_i=bU_{1_i}$ then $\mathscr{J}U_a\supseteq
\mathscr{J}U_b$ if and only if $\mathscr{J}U_{a_i}\supseteq
\mathscr{J}U_{b_i}$, $i=1,2$. It follows from this that the minimum
condition for principal inner ideals carries over from the
$\mathscr{J}_i$ to $\mathscr{J}$.

It is easy to show that if $Z$ is an ideal of $\mathscr{J}$
then $Z=\mathscr{J}_{i_1}+\mathscr{J}_{i_2}+\ldots
+\mathscr{J}_{i_k}$ for some subset $\{i_1,\ldots, i_k\}$ of the index
set. Clearly this implies that the decompostion
$\mathscr{J}=\mathscr{J}_1\oplus \mathscr{J}_2\oplus\ldots \oplus
\mathscr{J}_s$ into simple ideals is unique. We shall call the
$\mathscr{J}_i$ {\em the simple components} of $\mathscr{J}$.
\end{proof}

\section{A theorem on alternative algebras with involution.}\label{c3:sec8}

Our next task is to determine the simple quadratic Jordan algebras
which satisfy the DCC for principal inner ideals. By passing to an
isotope we may assume $\mathscr{J}$ has a capacity. If the capacity is
$1,\mathscr{J}$ is a division algebra. We shall have nothing further
to say about this. The case of capacity two will be treated in the
next section by a rather lengthy direct analysis. The determination
for capacity $3$ will be based on  the Strong coordinatization Theorem
supplemented by information by information on the coordinate
algebra. Both for this and for the study of the capacity two case we
shall need to determine the coordinate algebra $(\mathscr{O},
j,\mathscr{O}_o)$ (\S 2.4 for the definition) such that\pageoriginale
$(\mathscr{O},j)$ is simple and the non-zero elements of
$\mathscr{O}_o$ are invertible. We now consider this problem.

We define an {\em absolute zero divisor} in an altenative algebra
$\mathscr{O}$ to be an element $z$ such that $zaz=0$ for all
$a\epsilon \mathscr{O}$.

\begin{defn}\label{c3:defn5}
An alternative algebra with involution $(\mathscr{O},j)$ is called a
{\em composition algebra} if $1$ $\mathscr{O}$ has no absolute zero
divisors $\neq 0$ and $2)$ for any $x\epsilon
\mathscr{O},Q(x)=x\ob{x}\epsilon \Phi 1$.

A complete determination of these algebras over a field is given in
the following.
\end{defn}

\begin{thm}\label{c3:thm8}
  Let $(\mathscr{O},j)$ be a composition algebra over a field
  $\Phi$. Then $(\mathscr{O},j)$ is of one of the following types: $I$ a
  purely is separable field $P/\Phi$ of exponent one and charateristic
  two, $j=1$. II $(\mathscr{O},j)=(\Phi,1)$. III $(\mathscr{O},j)$ a
  quadratic algebra with standard involtion
  (\S1.8).IV. $(\mathscr{O},j)=(\mathfrak{a},j)$, $\mathfrak{a}$ a
  quaternion, $j$ the standard involution. V. $(\mathscr{O},j)$ an
  octonion algebra with standard involution.
\end{thm}

\begin{proof}
We have $x\ob{x}=Q(x)\epsilon \Phi$, from which it is immediate that
$Q$ is a quadratic form on $\mathscr{O}/\Phi$ whose associated
bilinear form satisfies $Q(x,y)1$ $=x\ob{y}+y\ob{x}$. Hence
\begin{equation*}
  T(x)\equiv x+\ob{x}=Q(x,1)1\epsilon \Phi 1.\tag{10}\label{c3:eq10}
\end{equation*}

Also $[x,\ob{x},y]=[x;x+\ob{x},y]-[x,x,y]=0$ for all $y\epsilon
\mathscr{O}$ and $Q(\ob{x})x=xQ(\ob{x})=x(\ob{x} x)=(x
\ob{x})x=Q(x)x$. Hence
\begin{equation*}
  Q(x)=Q(\ob{x})Q(x,y)=Q(\ob{x}, \ob{y})\tag{11}\label{c3:eq11}
\end{equation*}

Next\pageoriginale we note that
$Q(xz,y)-Q(x,y\ob{z})=(xz)\ob{y}+y(\ob{z}\, \ob{x})-x(z\ob{y})-(y\ob{z})\ob{x}=[x,z,\ob{y}]-[y,\ob{z},\ob{x}]=-[x,y,z]-[y,z,x]=[x,y,z]-[x,y,z]=0$
(by $x+\ob{x}\epsilon N(\mathscr{O})$ and the alternating character of
$[x,y,z]$. Hence $Q(xz,y)=Q(x,y\ob{z})$ and
$Q(xz,y)=Q(\ob{zx},\ob{y})=Q(\ob{z},\ob{y}x)=Q(z,\ob{x}y))$. Thus
\begin{equation*}
  Q(xz,y)=Q(x,y\ob{z})=Q(z,\ob{x}y).\tag{12}\label{c3:eq12}
\end{equation*}
We have $x(\ob{x}y)=Q(x)y=(yx)\ob{x}$ so by bilinearization we have
\begin{equation*}
x(\ob{z}y)+z(\ob{x}y)=Q(x,z)y=(yz)\ob{x}+(yx)\ob{z}\tag{1}\label{c3:eq13}
\end{equation*}

We suppose first that $Q(x,y)$ is generate which means that we have a
non-zero $z$ such that $Q(x,z)=0$ for all $x$. Then
$x\ob{z}+z\ob{x}=0$ and $z+\ob{z}=0$ so $xz=z\ob{x}$. Also
$z^{2}=-z\ob{z}=-Q(z)1$. Hence $zxz=-Q(z)\ob{x}, x\epsilon
\mathscr{O}$. If $Q(z)=0,z$ is an absolute zero divisor contrary to
hypothesis. Hence $Q(z)\neq 0$ and $\ob{x}=\alpha zxz$,
$\alpha=-Q(z)^{-1},x\epsilon \mathscr{O}$. Then
$xy=\ob{\ob{\ob{y}\ob{x}}}=\alpha z(\ob{y}\,\ob{x})z=\alpha
(z\ob{y}(\ob{x}z)=\alpha(yz)(zx)$. In particular,
$xz=\alpha z^{2}(zx)=zx$ and consequently  $\ob{x}=\alpha zxz=\alpha
z^{2}x=x$. Thus $j=1$ and consequently $\ob{xy}=\ob{y}\, \ob{x}$ given
$xy=yx$ so $\mathscr{O}$ is commutative. Also $z+\ob{z}=0$ gives $2z=0$
and $x=\alpha z x z$ gives $2x=0$. Hence $2\mathscr{O}=0$. This
implies that $\mathscr{O}$ has no $3$ torsion. Then
\begin{align*}
3[x,y,z]&=[x,y,z]+[y,z,x]+[y,z,x]+[z,x,y]\\
&=(xy)z-x(yz)+(yz)x-y(zx)+(zx)y-z(xy)\\
&=[xy, z]+[yz,x]+[zx,y]=0
\end{align*}
and\pageoriginale commutativity imply that $\mathscr{O}$ is
associative. Since $x^{2}=x\ob{x}=Q(x)1$ and $xyx=Q(x)y$ it is clear
that $Q(x)\neq 0$ if $x\neq 0$ so $\mathscr{O}$ is a purely
inseparable extension field of exponent one over $\Phi$.

Now assume $Q(x,y)$ is non-degenerate. If $\mathscr{O}=\Phi 1$ we have
type II. Hence assume $\mathscr{O}\supset\Phi 1$. If $x\epsilon
\mathscr{O}$ we have $x^{2}-T(x)x+x\ob{x}=x^{2}-(x+\ob{x})x+x\ob{x}=0$
so $x^{2}-Q(x,1)x+Q(x)1=0$. If $\Phi$ has characteristic two then
$Q(1)=1$ and $Q(1,1)=2=0$. Hence we can choose a $u\epsilon
\mathscr{O}$ such that $Q(1,u)=1$. Then $u^{2}u+\rho 1$ and $4\rho
+1=1\neq 0$  so $\Phi[u]$ is a quadratic algebra. If $\Phi$ has
characteristic $\neq 2$ then $Q(1,1)\neq 0$ and we can choose
$q\epsilon \mathscr{O}$ such that $Q(1,q)=0$ and $Q(q)=\beta\neq
0$. Put $u=q+\frac{1}{2}1$. Then $T(u)=1$ and
$Q(u)=\frac{1}{4}+\beta$,
$u^{2}=u+\rho_1,\rho=-\beta-\frac{1}{4}$. Since $4\rho +1=-4\beta \neq
0$, $\Phi[u]$ is a quadartic algebra. Hence in both cases we obtian a
quadratic subalgebra $\Phi[q]$ which is a subalgebra of
$(\mathscr{O},j)$ since $\ob{u}=1-u$. Thus the induced involution is
the standard one in $\Phi[q]$. It is clear also that $\Phi[u]$ is
non-isotropic as a subspace relative to $Q(x,y)$. Now let
$Z$ be any  finite deminsional non-isotropic subalgebra of
$(\mathscr{O},j)$ and assume $Z\subset \mathscr{O}$. As is
well-known, $\mathscr{O}=Z\oplus Z^{\bot}$ and
$Z^{\bot}\neq 0$ is non-isotropic. Hence there exists
$a\,v\epsilon Z^{\bot}$ such that $Q(v)=-\sigma\neq 0$. Since
$1\epsilon Z,Q(1,v)=0$ so $\ob{v}=-v$ and $v^{2}=\sigma
1$. If $a\epsilon Z$ then
$Q(a,v)=a\ob{v}+v\ob{a}=-av+v\ob{a}=0$ so 
\begin{equation*}
  av=v\ob{a},a\epsilon \mathscr{Z}\tag{14}\label{c3:eq14}
\end{equation*}

If $a,b\epsilon Z, Q(av,b)=Q(v,\ob{a}b)$(by
$(12))=0$. Hence $\mathscr{Z}v=\{xv|x\epsilon \mathscr{Z}\}$ $\subseteq
\mathscr{Z}^{\bot}$ and
$\mathfrak{L}=\mathscr{Z}+\mathscr{Z}v=\mathscr{Z}\oplus
\mathscr{Z}v$\pageoriginale has dimensionality$=2dim
\mathscr{Z}$. Also $Q(av, bv)=Q((av)\ob{v},b)=Q(a(v\ob{v}),b)=-\sigma
Q(a,b)$. It follows that $x\to xv$ is a linear isomorphism of
$\mathscr{Z}$ onto $\mathscr{Z}v$ and $\mathscr{Z}v$ and
$\mathfrak{L}$ are non-isotropic. By (13) with $z=v$ we obtain
\begin{equation*}
  a(bv)=(ba)v,(av)b=(a\ob{b})v,a,b,\epsilon
  \mathscr{Z}.\tag{15}\label{c3:eq15} 
\end{equation*}

Also $(av)(bv)=v(\ob{a}b)v$ (Moufang identity)$=(\ob{b}a)v^{2}$. Hence
\begin{equation*}
  (av)(bv)=\sigma \ob{b}a,a,b\epsilon \mathscr{Z}\tag{16}.\label{c3:eq16}
\end{equation*}

We have $\ob{a+bv}=\ob{a}-v\ob{b}=\ob{a}-bv$. We apply these
considerations to the quadratic subalgebra $\Phi[u]$. If
$\mathscr{O}=\Phi[u]$ we have case III. Otherwise, we take
$\mathscr{Z}=\Phi[u]$ and obtian the quaternion algebra
$\mathfrak{a}=\Phi [u]+\Phi[u]v$. If $\mathscr{O}=\mathfrak{a}$ we
have caseIV. Otherwise, we take $\mathscr{Z}=\mathfrak{a}$ and repeat
the argument. Then $\mathfrak{L}=\mathcal{O}$ is an octonion
algebra. We now claim that $\mathscr{O}=\mathcal{O}$ so we have case
V. Otherwise, we can apply the construction to
$\mathscr{Z}=\mathcal{O}$ and obtain
$\mathfrak{L}=\mathscr{Z}+\mathscr{Z}v$ such that \eqref{c3:eq15} and
\eqref{c3:eq16} 
hold. Put $x=a+bv,y=dv,a,b,d\epsilon \mathcal{O}$. Then we have
$[\ob{x},x,y]=0$ and $\ob{x}(xy)=(\ob{a}-bv)(\sigma
\ob{d}b+(da)v)$. Since $(\ob{x} y)y$ is a multiple of $y=dv$ this
implies that $\ob{a}(\ob{d}b)=(\ob{a}\ob{d})b$. Since this holds for
all $a,b,d\epsilon \mathcal{O}$ we see that $\mathcal{O}$ we see
that $\mathcal{O}$ must be associative. Since it is readily verified
that it is not we have a contradiction. This completes the proof that
only $I-V$ can occur.

It is readily seen that the algebras with involution $I-V$ are
composition algebras.
We\pageoriginale prove next
\end{proof}

\begin{thm}[Herstein-Kleinfeld-Osborn-McCrimmon]\label{c3:thm9} 
  Let $(\mathscr{O},j,\break\mathscr{O}_o)$ be a coordinate algebra (over any
  $\Phi$) such that $(\mathscr{O},j)$ is simple and every non-zero
  element of $\mathscr{O}_o$ is invertible in $\mathscr{O}_o$. Then we
  have one of the following alternatives:
  \begin{itemize}
  \item[I.] $\mathscr{O}=\Delta \oplus \Delta^{\circ},\Delta$ an
    associative division algebra $j$ the exchange involution,
    $\mathscr{O}_o=\mathscr{H}(\mathscr{O},j)$.
  \item[II.] an associative division algebra with involution.
  \item[III.] a split quaternion algebra $\Gamma_2$ over its center
    $\Gamma $ which is a field over $\Phi$, standard involution,
    $\mathscr{O}_o=\Gamma$. 
  \item[IV.] an algebra of octonions over its center $\Gamma$ which is a
    field over $\Phi$ standard involution, $\mathscr{O}_o=\Gamma$.
  \end{itemize}
\end{thm}

\begin{proof}
We recall that the hypothesis that $(\mathscr{O},j\mathscr{O}_o)$ is a
coordinate algebra means that $(\mathscr{O},j)$ is an alternative
algebra with involution, $\mathscr{O}_o$ is a $\Phi$ submodule of
$\mathscr{O}$ contianed in
$\mathscr{H}(\mathscr{O},j)\cap N(\mathscr(O))$ and containing $1$ and
every $x d\ob{x}, d\epsilon \mathscr{O}_o,x\epsilon
\mathscr{O}$. Hence $\mathscr{O}_o$ contains all the norms $x\ob{x}$
and all the traces $x+\ob{x}$. Then $[x,\ob{x},y]=0,x,y\epsilon
  \mathscr{O}$. We recall also the following realtion in any
  alternativfe algebra
\begin{equation*}
n[x,y,z]=[nx,y,z]=[xn,y,z]=[x,y,z]n\tag{17}\label{c3:eq17}
\end{equation*} 
for $n\epsilon N(\mathscr{O}),x,y,z\epsilon \mathscr{O}$ (see the
author's book pp. 18-19).

Suppose first that $\mathscr{O}$ is not simple. Then
$\mathscr{O}=\Delta\oplus \ob{\Delta}(\ob{\Delta}=\Delta^{j})$ where
$\Delta$ is an ideal. The elements of $\mathscr{H}(\mathscr{O},j)$ are
the elements $a+\ob{a},a\epsilon \Delta$. Hence
$\mathscr{H}(\mathscr{O}, j)= \mathscr{O}_o$. Since these are in
$N(\mathscr{O})$ every 
$a\epsilon N(\mathscr{O}) $ so $\Delta$ and $\ob{\Delta}\subseteq
N(\mathscr{O})$. Then $\mathscr{O}=N (\mathscr{O})$ is
associative. Also if\pageoriginale $a\neq 0$ is in 
$\Delta$ then $a+\ob{a}$ is invertible which implies $a$ is
invertible. Hence $\Delta$ is a division algebra and we have case I. 

From now on we assume $\mathscr{O}$ simple. Then its center
$C(\mathscr{O})$(defined as the subset of $N(\mathscr{O})$ of elements
which commute with every $x\epsilon \mathscr{O}$) is a field over
$\Phi$ (see the author's book p.207). It follows that
$\Gamma=\mathscr{H}(\mathscr{O},j)\cap C(\mathscr{O})$ is a field over
$\Phi$. We can regard $\mathscr{O}$ as an algebra over $\Gamma$ when
we wish to do so. We note first that the following conditions on
$a\epsilon \mathscr{O}$ are equvalent: (i)$a\ob{a}\neq 0$, (ii) as has
a right inverse (iii) $\mathfrak{a}a\neq 0$, (iv) a has a left inverse,
Asssume (i). Then $a{\ob{a}}$ is invertible in $N(\mathscr{O})$ so
we have $ab$ such that $(a \ob{a})b=1$. Then $a(\ob{a}b)=1$ and a has a
right inverse. Hecne $(i)\Rightarrow (ii)$. Next assume
$\ob{a}a=0$. Then $0=$($\ob{a},a)b=\ob{a}(ab)$ and $ab\neq1$.Hence
$(ii)\to(iii)$. By symmetry, $(iii)\to (iv)\to$ and $(iv)\to (i)$. Let
$z\epsilon \mathfrak{z},a\epsilon \mathscr{O}$. Then
$(az)(\ob{az})=(az)(\ob{z}\,\ob{a})=(az)(\ob{z}(a+\ob{a}))-(az)(\ob{z}a)=((az)\ob{z})(a+\ob{a})-a(z\ob{z})a=0$. Hence
$az\epsilon \mathfrak{z}$. Also $\ob{\mathfrak{z}}=\mathfrak{z}$ so
$za\epsilon \mathfrak{z}$ . Moreover, $\mathfrak{z}$ is closed under
multiplication by elements of $\Phi$ and $1\notin \mathfrak{z}$. Hence if
$\mathfrak{z}$ is closed under addition it is an ideal $\neq
\mathscr{O}$, of $(\mathscr{O},j)$ and so $\mathfrak{z}=0$. Then every
non-zero element of $\mathscr{O}$ has a left and a right inverse in
$\mathscr{O}$. 

Suppose $\mathfrak{z}=0$. If $\mathscr{O}$ is associative
$(\mathscr{O},j)$ is an associative algebra with involution and we
have case II. Next assume $\mathscr{O}=N(\mathscr{O})$. We claim that
in this case $N(\mathscr{O})=C(\mathscr{O})$. By (17), if $n\epsilon
N(\mathscr{O}),x\epsilon \mathscr{O}$, $[nx]=nx-xn\epsilon
N(\mathscr{O})$ and $n$ commutes with all associators Direct
verification shows that if $x,y\epsilon \mathscr{O},n\epsilon
N(\mathscr{O})$ then $[xy,n]=[xn]y+x[yn]$ where $[ab]=ab-ba$ and
$x[x,y,z]=[x^{2},y,z]-[x,xy,z]$. The last implies\pageoriginale that
$0=[x[x,y,z],n]=[xn][x,,y,z]$. Hence we have
\begin{equation*}
  [xn][x,y,z]=0,n\epsilon N(\mathscr{O}),x,y,z,\epsilon
  \mathscr{O}\tag{18}\label{c3:eq18}
\end{equation*}
Bilinearization of this gives
\begin{equation*}
  [xn][w,y,z]+[wn][x,y,z]=0\tag{19}\label{c3:eq19}
\end{equation*}

Suppose $x\notin N(\mathscr{O})$. Then we can choose $y,z$ such that
$[x,y,z]\neq0$ so this has a right inverse. Since $[x,n]\epsilon
N(\mathscr{O})$ this and $(18)$ imply $[xn]=0$. If $x\epsilon
N(\quad)$, $(19)$ gives $[x,n][w,y,z]=0$. Since $N(\mathscr{O})\neq
\mathscr{O}$ we can choose $[w,y,z]\neq 0$ and again conclude
$[xn]=0$. Hence $[xn]=0$ for all $x$ and $N(\mathscr{O}=C(\mathscr{O})$
if $\mathfrak{z}=0$ and $\mathscr{O}$ is not associative. In this case
$x\ob{x}\epsilon C(\mathscr{O})\cap
\mathscr{H}(\mathscr{O},j)=\Gamma$. Also we have no absolute zero
divisors since $\mathscr{O}$ is a division algebra. Treating
$(\mathscr{O},j)$ as a algebra over $\Gamma$ we have a composition
algebra. Since $\mathscr{O}$ is not associative we have the octonion
case and we shall have case IV if we can show that
$\mathscr{O}_o=\Gamma$. Since $N(\mathscr{O})\subseteq \Gamma$ for an
actonion algebra over a field, $\mathscr{O}_o\subseteq \Gamma$. To
prove the opposite inequality it is enough to show that every element
of $\Gamma$ is a trace. Now $\mathcal{O}$ contains a quadratic
algebra $\Phi[u]$ in whixh $u^{2}u+\rho_1$ and $\ob{u}=1-u$. Thus
$u+\ob{u}=1$ and if $\gamma \epsilon \Gamma$ then $\gamma =\gamma
u+\ob{\gamma u}$ is a trace.

It remains to consider the situation in which $\mathfrak{z}$ is not
closed under addition. Then we have $z_1,z_2\epsilon \mathfrak{z}$
such that $z_1+z_2=u$ is invertible. Hence $e_1+e_2=1$,
$e_i=z_iu^{-1}\epsilon \mathfrak{z}$ and $\ob{e_1}+\ob{e_2}=1$. Also
$\ob{e_i}e_i=0$ and
$\ob{e_1}=\ob{e_1}(e_1+e_2)=\ob{e_1}e_2=(\ob{e_1}+\ob{e_2})e_2=e_2$,
Then $e_2e_1=0,e_1+e_2=1$\pageoriginale so the $e_i$ are orthongonal
idempotents and $\ob{e_1}=e_2,\ob{e_2}=e_1$. Let
$\mathscr{O}=\mathscr{O}_{11}\oplus \mathscr{O}_{12}\oplus
\mathscr{O}_{21}\oplus \mathscr{O}_{22}$ be the corresponding Pierce
decomposition (see shafer [1] pp. 35-37 and the author's [2'
  pp. (165-166). Since $\mathscr{O}$ is simple
$\mathscr{O}_{12}+\mathscr{O}_{21}\neq 0$ and Since
$\mathscr{O}_{12}\mathscr{O}_{21}+\mathscr{O}_{12}+\mathscr{O}_{21}+\mathscr{O}_{21}\mathscr{O}_{12}$
is an ideal, $\mathscr{O}_{12}\mathscr{O}_{21}=\mathscr{O}_{11}$,
$\mathscr{O}_{21}\mathscr{O}_{12}=\mathscr{O}_{22}$. Also since
$\ob{e_1}=e_2$, $\ob{e_2}=e_1$ we have
$\mathscr{O}_{11}=\mathscr{O}_{22}$,
$\ob{\mathscr{O}}_{22}=\mathscr{O}_{11}$,
$\ob{\mathscr{O}}_{12}=\mathscr{O}_{12}$,
$\ob{\mathscr{O}}_{21}=\mathscr{O}_{21}$. Let
$x=x_{11}+x_{12}+x_{21}+x_{22}$ where $x_{ij}\epsilon
\mathscr{O}_{ij}$. Then the Pierce relations give
$$
x_{11}\ob{x}_{11}=x_{22}\ob{x}_{22}=x_{11}\ob{x}_{21}=x_{22}\ob{x}_{12}=x_{12}\ob{x}_{22}=x_{21}\ob{x}_{11}=0.
$$

Also $x_{12}=e_1xe_2\epsilon \mathfrak{z}$ since $e_i\epsilon
\mathfrak{z}$ so $x_{12}\ob{x}_{12}=0$. Similarly,
  $x_{21}\ob{x}_{21}=0$. If $y\epsilon \mathscr{O}_{12}$,
  $y+\ob{y}\epsilon\mathscr{O}_{12}\cup
  \mathscr{O}_o\subseteq\mathfrak{z}\cap\mathscr{O}_o=0$. Similarly,
  if $y\epsilon \mathscr{O}_{21},y+\ob{y}=0$. Hence
  $x_{11}\ob{x}_{12}+x_{12}\ob{x}_{11}=0=x_{22}\ob{x}_{21}+x_{21}\ob{x}_{22}$.
Combining we see that 
$$
x\ob{x}=x_{11}\ob{x}_{22}+x_{22}\ob{x}_{11}+x_{12}\ob{x}_{21}+x_{21}\ob{x}_{12}=y+\ob{y}
$$
where $y=x_{11}\ob{x}_{22}+x_{12}\ob{x}_{21}\epsilon
\mathscr{O}_{11}$. We show next that if $y\epsilon \mathscr{O}_{11}$
then $y=\ob{y}\epsilon \Gamma$. Since $y+\ob{y}\epsilon
\mathscr{O}_o\subseteq N(\mathscr{O})$ it is enough to show that
$y+\ob{y}$ commutes with every $x\epsilon \mathscr{O}$. We have seen
that if $x_{ij}\epsilon \mathscr{O}_{ij}$ then $x_{ij}=\ob{-x}_{ij}$
if $i\neq j$ so
$x_{ii}x_{ij}=\ob{-x}_{ij}\ob{x}_{ii}=x_{ij}\ob{x}_{ii}$. Hence if
$y\epsilon \mathscr{O}_{11}$ then
$(y+\ob{y})x_{ij}=yx_{ij}+\ob{y}x_{ij}= x_{ij}\ob{y}+x_{ij}y= x_{ij}
(y+\ob{y})$. Also  
\begin{align*}
(y+\ob{y})(x_{12}x_{21})&=((y+\ob{y})x_{12})x_{21}=(x_{12}(y+\ob{y}))x_{21}\\
&=x_{12}((y+\ob{y})x_{21})=x_{12}(x_{12}(x_{21}(y+\ob{y}))\\
&=x_{12}x_{21}(y+\ob{y}).
\end{align*}
Similarly,\pageoriginale $[y+\ob{y},x_{21}x_{12}]=0$ so $y+\ob{y}$ commutes with
every $x$ and $y+\ob{y}\epsilon \Gamma$. Thus we have $x\ob{x}\epsilon
\Gamma$. We claim that if $Q(x,y)=x\ob{y}+y\ob{x}\epsilon\Gamma$ then
this is non-degenerate. The formulas (12) show that the set of $z$
such that $Q(z,x)=0$ for all $x\epsilon\mathscr{O}$ is an ideal of
$(\mathscr{O},j)$. Hence if this is not $0$ it contains $1$. But
$Q(1,e_1)=e_1+\ob{e}_1=1$. Hence $Q(x,y)$ is non-degenerate. Then the
proof of Theorem $7$ shows that we have one of cases II-V of that
theorem, one sees easily that the only possibilities allowed here are
$(\mathscr{O},j)$ is split quaternion or split octoion over
$\Gamma$. As before, we have $\mathscr{O}_o=\Gamma$ in the octonion
case and we are in case IV. In the split quaternion case,
$\mathscr{O}=\Gamma_2$, the argument used before shows that
$\mathscr{O}_o\supseteq\Gamma$. If the characteristic is $\neq 2\Gamma
=\mathscr{H}(\mathscr{O},j)$ , hence, $\mathscr{O}_o=\Gamma$. If the
characteristic is two then it is easily seen that we have a base of
matrix units $e_{ij}$ such that
$\ob{e}_{11}=e_{22},\ob{e}_{22}=c_{11}$,
$\ob{e}_{12}=e_{12},\ob{e}_{21}=e_{21}$. If 
$a\epsilon \mathscr{O}_{o}$, $a\epsilon\mathscr{H}(\mathscr{O},j)$ so
  $a=\alpha 1+\beta e_{12}+\gamma e_{21},\alpha,\beta, \gamma \epsilon
  \Gamma$. Since $e_{12}a\ob{e}_{12}=\gamma e_{21}\epsilon
  \mathscr{O}_0$ and the non-zero elements of $\mathscr{O}_o$ are
  invertible, $\gamma=0$. Similarly $\beta=0$ so again
  $\mathscr{O}_o=\Gamma$. Thus we have case III.
\end{proof}

\section[Simple quadratic Jordan algebras of capacity two]{Simple
  quadratic Jordan algebras of capacity\hfil\break two}\label{c3:sec9} 

Let $\mathscr{J}$ be of capacity two, so $1=e_1+e_2$ where the $e_i$
are completely primitive orthogonal idempotents,
$\mathscr{J}=\mathscr{J}_{11}\oplus
\mathscr{J}_{12}\oplus\mathscr{J}_{22}$ the corresponding Pierce
decomposition. Then $\mathscr{J}_{ii}$ is a division algebra. Put
$\mathfrak{m}=\mathscr{J}_{12}$. If  $x_i\epsilon \mathscr{J}_{ii}$
then $\nu_i:x_i\to \ob{V}_{x_{i}}$ the restricition of $V_{x_i}$ to
$\mathfrak{m}$ is a homomorphism of $\mathscr{J}_{ii}$ into (End
$\mathfrak{m}$)$^{(q)}$. Since $\mathscr{J}_{ii}$ is a division
algebra, $\nu_i$ is a monomorphism. Hence $\mathscr{J}_{ii}$ is
special so this can be indentified\pageoriginale with a subalgebra of
$S(\mathscr{J}_{ii})^{(q)}$ where $S(\mathscr{J}_{ii})$ is the special
universal envelope of $\mathscr{J}_{ii}$ (see \S 1.6). If $\pi$ is the
main involution of $S(\mathscr{J}_{ii})$ then
$\mathscr{J}_{ii}\subseteq \mathscr{H}(S(\mathscr{J}_{ii}),\pi)$ and
$s(\mathscr{J}_{ii})$ is generated by $\mathscr{J}_{ii}$. The
homomorphism $\nu_i$ has a unique extension to a homomorphism of
$S(\mathscr{J}_{ii})$. The latter permits us to consider $\mathfrak{m}$
as a right $s(\mathscr{J}_{ii})$ module in the natural way. If
$m\epsilon \mathfrak{m}$ and $x_i\epsilon \mathscr{J}_{ii}$ then the
definitions give $mx_i=mV_{x_i}=m\circ x_i$ and if
$x_i,y_i,\ldots,z_i\epsilon \mathscr{J}_{ii}$ then
\begin{equation*}
  m(x_iy_i\ldots z_i)=(\ldots((m\circ x_i)\circ y_i)\circ\ldots\circ
  z_i)\tag{20}\label{c3:eq20} 
\end{equation*}

Also by the associativity conseqences of the PD theorem $(m\circ
x_1)\circ x_2=(m\circ x_2)\circ x_1$ from which follows
\begin{equation*}
  (ma_1)a_2=(ma_2)a_1,a_i\epsilon
  S(\mathscr{J}_{ii})\tag{21}.\label{c3:eq21} 
\end{equation*}

We recall that if $m\epsilon \mathfrak{m}$, either $m^{2}=0$ or $m$ is
invertible (Lemma \ref{c3:sec6:lem2} of \S \ref{c3:sec6}). Suppose $m^{2}=0$. Then $x_iU_m=0$ for
$x_i\epsilon \mathscr{J}_{Iii}$ (proof of Lemma \ref{c3:sec6:lem2}
\S \ref{c3:sec6}). Then
$(x_i\circ m)^{2}=x^{2}_iU_m+m^{2}U_{x_i}+x_iU_m\circ x_i(QJ
30)=0$. This implies that if $m$ is invertible and $x\neq 0$ then
$mx_i$ is invertible. Otherwise $(m x_2)^{2}=0$ and $(m x_iy_i)^{2}=(m
x_i\circ y)^{2}=0$ for all $y_i\epsilon \mathscr{J}_{ii}$. If we
choose $y_i$ to be the inverse of $x_i$ and $\mathscr{J}_{ii}$ we
obtain the contradiction $m^{2}=0$. If $m$ is invertible and, as in
Lemma \ref{c2:lem2} of \S 2.3. we put $u=c_1+m^{2}U_{e_2},\cdot
v=e_1+m^{-2}U_{e_2}$ then we have seen that in the isotope
$\tilde{\mathscr{J}}=\mathscr{J}^{(v)},u_1=e_1$ and $u_2=m^{2}U_{e_2}$ are
supplementary orthogonal idempotents which are strongly connected by
$m$. The Pierce submodule $\mathscr{J}_{ij}$ relative to the
$u_i$\pageoriginale coincides with $\mathscr{J}_{ij}$. Moreover,
$\mathscr{J}_{11}=\tilde{\mathscr{J}}_{11}$ as quadratic Jordan algebras,
and for $m\epsilon
\mathfrak{m}=\mathscr{J}_{12}=\ob{\mathfrak{m}}=\tilde{\mathscr{J}}_{12}$
and $x_1\epsilon \mathscr{J}_{11}$ we have
$mV_{x_1}=m\ob V_{x_1}$. Hence the $S(\mathscr{J}_{11})$ module
structure on $\mathfrak{m}$ is unchanged in passing from
$\mathscr{J}$ to $\tilde{\mathscr{J}}$. Also $\mathscr{J}_{22}$ and
$\tilde{\mathscr{J}}_{22}$ are isotopic so $\tilde{\mathscr{J}}_{22}$ is a
division algebra and $u_1$ and $u_2$ are completely primitive in
$\tilde{\mathscr{J}}$. Clearly, $\mathscr{J}$ is of capactiy two. Since
the isotope $\tilde{\mathscr{J}}$ is determined by the choice of the
invertible element $m$ it will be convenient to denote this as
$\mathscr{J}_m$.

We shall now assume $\mathscr{J}$ simple and we shall prove the
following structure theorem which is due to Osborn \cite{Osborn1} in the linear
case and to McCrommon in the quadratic case.

\begin{thm}\label{c3:thm10}
Let $\mathscr{J}$ be a simple quadratic Jordan algebra of capacity
two. Then either $\mathscr{J}$ is isomorphic to an outer ideal $\ni 1$
of  a quadratic Joradan algebra of a non-degenerate quadratic form on
a vector space over a field $P/\Phi$ or $\mathscr{J}$ is isomorphic to
an outer ideal $\ni 1$ of an algebra $\mathscr{H}(\mathscr{O}_2,J_H)$
where $(\mathscr{O},J)$ is either an associative division algebra with
involution or $\mathscr{O}=\Delta\oplus \Delta^{j},\Delta$ an
associative division algebra and $J_H$ is the involution $X\to
H^{-1}X^{-1}H$, $H\epsilon\mathscr{H}(\mathscr{O}_2)$. 

We have seen in \S 1.11 that $\mathscr{H}(\mathscr{O}_2,J_H)$ is
isomorphic to the $H$-isotope of $\mathscr{H}(\mathscr{O}_2)$. Now
consider Jord $(Q,1)$ the quadratic Jordan algebra of a quadartic form
$Q $ with base point $1$ on a vector space over a field $P$. Let $u$
be an invertible elements so $Q(u)\neq 0$. Then $Q'=Q(u)Q$ is a
quadratic form which has the base point $u^{-1}=Q(u)^{-1}\ob{u}$ since
$Q(u)Q(u^{-1})=Q(Q(u)^{-1}\ob{u})=Q(Q(u,1)1-u)=Q(u)^{-1}Q(u)=1$. Now
consider Jord $(Q(u)Q,u^{-1})$. Put
$x'=Q(u)Q(x,u^{-1})u^{-1}-x$\pageoriginale and let $U'$ denote the
$U$-operator in this algebra. A straight forward calculation shows
that $xU'_a=Q(a,\ob{xU}_u)a-Q(a)\ob{xU}_u=xU_vU_a$. It follows that
Jord $(Q(u)Q,u^{-1})$ is identical with the $u$-isotope of Jord
$(Q,1)$. These remarks show that to prove Theorem $10$ it suffices to
show that there exists an isotope of $\mathscr{J}$ which is isomorphic
to an outer ideal containing $1$ in a Jord $(Q,1)$ with non-degenerate
Q(over a field) or to an outer ideal containing $1$ in an
$H(\mathscr{O}_2)$. By passing to an isotope we may assume at the
start that $1=e_1+e_2$ where the $e_i$ are orthogonal completely
primitive and are strongly connected by an element $u\epsilon
\mathscr{J}_{12}$. The proof will be divided into a series of
lemmas. An important point in the argument will be that except for
trivial cases $\mathfrak{m}=\mathscr{J}_{12}$ is spanned by invertible
elements. This fact is contained in 
\end{thm}

\setcounter{lemma}{0}
\begin{lemma}\label{c3:sec9:lem1}
  Suppose $\mathscr{J}_{11}\neq \{0, \pm e_1\}$ (that is,
  $\mathscr{J}_{11}\neq \mathbb{Z}_2$ or $\mathbb{Z}_3$). Let
  $m,n\epsilon \gamma \mathfrak{m}$, $m$ invertible. Then there exist
  $x_1\neq 0$, $y_1\neq 0$ in $\mathscr{J}_{11}$ such that $x_1\circ
  m+y_1\circ n$ is invertible. Any element of $\mathfrak{m}$ is a sum of
  invertible elements.
\end{lemma}

\begin{proof}
Since $x\epsilon \mathfrak{m}$ is either invertible or $x^{2}=0$, if
the result is false, then $x_1\circ m+y_1\circ n)^{2}=0$ for all
$x_1\neq 0$, $y_1\neq 0$ in $\mathscr{J}_{11}$. By $QJ 30$ and the PD
theorem the component in $\mathscr{J}_{22}$ of this element is 
\begin{equation*}
  x^{2}_1U_m+y^{2}_1U_n\{x_1\circ m, e_1, y_1\circ m\}=0\tag{22}\label{c3:eq22}
\end{equation*}

Take $(x_1,y_1)= (x_1,e_1),(e_1,y_1),(e_1,e_1),(x_1,y_1)$ and add the
first two equations thus obtained to the negative of the last
two. This gives
\begin{equation*}
  \{z_1\circ me_1w_1\circ n\},z_1=x_1-e_1,w_1=y_1-e_1\tag{23}\label{c3:eq23}
\end{equation*}\pageoriginale

Using this and \eqref{c3:eq22} we obtain
\begin{equation*}
  z^{2}_1U_m+w^{2}_1U_n=0\quad{\text{if}}\quad z_1,w_1\neq 0,
  -e_1\quad\text{in}\quad \mathscr{J}_{11}.\tag{24}\label{c3:eq24}
\end{equation*}

In particular, $w^{2}_1U_m+w^{2}_1U_n=0$, so, by \eqref{c3:eq24}
$(z^{2}_1-w^{2}_1)U_m=0$ and since $m$ is invertible,
$z^{2}_1=w^{2}_1$ is $z_1,w_1\neq 0$, $-e_1$. Let $z_1\neq 0$, $\pm
e_1$ so $z_1-e_1\neq 0$, $-e_1$ and so
$z^{2}_1=(z_1-e_1)^{2}=z^{2}_1-2z_1+e_1$. Hence $2z_1=e_1$. Also since
$-z_1\neq 0$, $\pm e_1$ we have also $-2z_1=e_1$. Then $4z_1=0$ and
since $\mathscr{J}_{11}$ is a quadratic Jordan division algebra,
$2z_1=0$. This gives $e_1=0$ contrary to $\mathscr{J}_{11}\neq
0$. Hence the first statement holds. For the second we note that
$\mathfrak{m}$ contains an invertible element $m$ and if $n$ is any
element of $\mathfrak{m}$ then there exist $x_1,y_1\neq 0$ in
$\mathscr{J}_{11}$ such that $p=x_1\circ m+y_1\circ n$ is
invertible. Then if $z_1$ is the inverse of $y_1$ in
$\mathscr{J}_{11},n=z_1\circ p-z_1\circ (x_1\circ m)$ and $z_1\circ p$
and $-z_1\circ (x_1\circ m)$ are invertible elements of
$\mathfrak{m}$. 

We are assuming that $e_1$ and $e_2$ are strongly
corrected by $u\epsilon \mathfrak{m}$. Then $\eta=U_u$ is an
automorphism of period two in $\mathscr{J},\eta$ maps $\mathfrak{m}$
onto itself and exchange $\mathscr{J}_{11}$ and
$\mathscr{J}_{22}$. Hence $\eta$ defines an isomorphism of
$\mathscr{J}_{11}$ onto $\mathscr{J}_{22}$. This extends uniquely to
an isomorphism $\eta$ of $(S(\mathscr{J}_{11}),\pi)$ onto
$(S(\mathscr{J}_{22}),\pi)$. We have $u^{3}=u(\S 2.3)$. Hence
$u^{\eta}=uU_u=u^{3}=u$. We shall now derive a number of results in
which $u$ plays a distinguished roles. These will be applied later to
any invertible $m\epsilon \mathfrak{m}$ by passing to the isotope
$\mathscr{J}_m(=\mathscr{J}^{(v)}$ as above). Let $x\epsilon
\mathscr{J}$. Then $(x\circ n)^{\eta}=x^{\eta}\circ u^{\eta}\circ
u$. Also $x^{\eta}\circ u=xU_uV_u=xU_{u,u^{2}}=xU_{u,1}=xV_u=x\circ
u$. Thus we have
\begin{equation*}
  (x\circ u)^{\eta}=x^{\eta}\circ u=x\circ u, x\epsilon
  \mathscr{J}\tag{25}\label{c3:eq25} 
\end{equation*}

We\pageoriginale prove next
\end{proof}

\begin{lemma}\label{c3:sec9:lem2}
Let $m\epsilon \mathfrak{m}$. Then $m+m^{\eta}=ux_1,x_1=(u\circ
m)U_{e_1}$. If $m+m^{\eta}=0$ then $u\circ m=0$.
\end{lemma}

\begin{proof}
Let $m,n\epsilon \mathfrak{m}$. Then
$mU_n=mU_{e_1,e_2}U_n=e_jV_{m,e_i}U_n=-e_jV_{n,e_i}$ $U_{m,n}+e_jU_nV_{e_i},m+e_1U_{m,n}V_{e_{i}},n$
(by $QJ 9)=-\{nnm\}+\{n^{2}U_{e_i}e_im\}+\{(m\circ n)U_{e_i}e_in\}$
(PD $6$ and its bilinearization). Hence $mU_n=-n^{2}\circ
m+n^{2}U_{e_i}\circ m+(m\circ n)U_{e_i}\circ m$ (replacing $e_i$ by
$e_i+e_j$). Since $1=U_{e_i}+U_{e_j}+U_{e_i,e_j}$ this gives
\begin{equation*}
  mU_n=-n^{2}U_{e_j}\circ m+(m\circ n)U_{e_i}\circ m, n, \in
  \mathfrak{m}\tag{26}\label{c3:eq26}  
\end{equation*}

Taking $n=u$ we get $m^{\eta}=mU_n=-n\mu_n=-m+ux_1,x_1=(m\circ
u)U_{e_1}$. This is the first statment of the lemma. If $m+m^{\eta}=0$
we have $x_1=(u\circ m)U_{e_1}=0$. Applying $\eta$ gives $(u\circ
m^{\eta})U_{e_2}=0=(u\circ m)U_{e_2}$, by $(25)$. Since $u\circ
m\epsilon \mathscr{J}_{11}+\mathscr{J}_{22}$ the realtions $(u\circ
m)U_{e_i}=0$, $i=1,2,$ imply $u\circ m=0$.
\end{proof}

\begin{lemma}\label{c3:sec9:lem3} 
  If $m\epsilon \mathfrak{m}$ satisfies $mx_1=mx_1$ for all $x_1\epsilon
  \mathscr{J}_{11}$ then $ma^{\pi}=ma^{\eta}, a\epsilon S(\mathscr{J}_{11})$.
\end{lemma}

\begin{proof}
Since $\mathscr{J}_{11}$ generates $S(\mathscr{J}_{11})$ it suffices
to prove the conclusion for $a=x_1,x_2\ldots x_k$, $x_i \in
\mathscr{J}_{11}$. We use induction on $k$. Assume $m(x_1,x_2\ldots
x_k)^{\pi}=m(x_1\ldots x_k)^{\eta}$. Then $m(x_1\ldots
x_{k+1})^{\pi}=mx_{k+1}^{\eta}$ $(x_1\ldots x_k)^{\pi}=m(x_1\ldots
x_k)^{\pi}x^{n}_{k+1}$ (by \eqref{c3:eq21})$=m(x_1\ldots x_k)^{\eta}
x^{\eta}_{n+1}=m$ \break $(x_1\ldots x_{k+1})^{\eta}$ which proves the inductive
step. We\pageoriginale have $ux_1=ux^{\eta}_1, x_1\epsilon$
 $\mathscr{J}_{11}$, by \eqref{c3:eq25}. Hence Lemma \ref{c3:sec9:lem3}
and \eqref{c3:eq25} give 
\begin{equation*}
  ua^{\pi}=ua^{\eta}=(ua)^{\eta},a\epsilon
  S(\mathscr{J}_{11})\tag{27}.\label{c3:eq27} 
\end{equation*}

Now suppose $ua=0$ for an $a\epsilon S(\mathscr{J}_{11})$. Then for
$b\epsilon S(\mathscr{J}_{11})$, $uba=u b^{\eta\pi}a=ub^{\pi
\eta}a=uab^{\pi \eta}$ (by \eqref{c3:eq21})$=0$. Hence we have
\end{proof}

\begin{lemma}\label{c3:sec9:lem4} 
  If $ua=0$ for $a\epsilon S(\mathscr{J}_{11})$ then $uba=0$ for all
  $b\epsilon S(\mathscr{J}_{11})$.
  We prove next
\end{lemma}

\begin{lemma}\label{c3:sec9:lem5} 
  Let $n\epsilon \mathfrak{n}=uS(\mathscr{J}_{11})$, $a\epsilon
  S(\mathscr{J}_{1})$. Then $n(a+a^{\pi})=nx_1,x_1=(u\circ
  ua)U_{e_1}$. Also if $y_1\epsilon \mathscr{J}_{11}$ then
  $n(a^{\pi}y,a)=nz_1$ where $z_1=y^{\eta}_{1}U_{ua} \mathscr{J}_{11}$.
\end{lemma}

\begin{proof}
We have $ua^{\pi}+ua=(ua)^{\eta}+(ua)$ (by (5))$=ux_1,x_1=(u\circ
ua)U_{e_1}$ (by Lemma \ref{c3:sec9:lem1}). Hence $u(a^{\pi}+a-x_1)=0$
so, by Lemma \ref{c3:sec9:lem4}, $n(a^{\pi}+a-x_1)=0$ for all $n\epsilon
uS(\mathscr{J}_{11})$. This proves the first statment. To prove the
second it suffices to show that $n(a^{\pi}y_1a)=n(y_1^{\eta}U_{ua})$
and $n(a^{\pi}y_1b+b^{\pi}y_1a)=n(y^{\eta}U_{ua,ub})$ for $a=t_1\ldots
z_1,y_1,t_1,\ldots, z_1\epsilon \mathscr{J}_{11}$, $b\epsilon
S(\mathscr{J}_{11}$, $b\epsilon (\mathscr{J}_{11})$. For $m\epsilon
\mathfrak{m}$, we have $y^{\eta}_1 U_{mt_1}=y^{\eta}_1U_{m\circ
  t_1}=y^{\eta}_1U_mU_{t_1}$ by $QJ 17$ and the  PD theorem. Iteration
of this gives $y_1U_{ua}=y^{\eta}_1U_{ut_1}\ldots
z_1=y^{\eta}_1U_uu_{t_1}\ldots U_{z_1}=y_1U_{t_1}\ldots
U_{z_1}=z_1\ldots t_1y_1t_1\ldots z_1$ (in
$A(\mathscr{J}_{11}))=a^{\pi}y_1a$. Next we use the first statement of
the lemma to obtain
\begin{equation*}
n(a^{\pi}uy_1b+b^{\pi}y_1a)=n((u\circ u a^{\pi}y_1b)U_{e_1},n\epsilon
\mathfrak{n}\tag{28} \label{c3:eq28} 
\end{equation*}

If $m,n\epsilon^{\mathfrak{m}}, y_1\epsilon \mathscr{J}_{11}$ then
$\{my_1n\}=((m\circ y_1)\circ n)U_{e_2}$ (PD theorem)$= ((m\circ
y_1)\circ (n\circ e_1)U_{e_2}=\{m\circ y_1e_1n\}$. Since $\{my_1 n\}$
is symmetric in $m$ and\pageoriginale $n$ we have $\{m\circ
y_1e_1,n\}=\{me_1n\circ y_1\}$. If we take $a=t_1\ldots z_1$,
$t_1,\ldots,z_1\epsilon \mathscr{J}_{11}$ then we can iterate this to
obtain
\begin{equation*}
\{my_1 ae_1 n\}=\{m, y_1,na^{\pi}\}=\{my_1e_1na\}\tag{29}\label{c3:eq29}
\end{equation*}

Since $\mathscr{J}_{11}$ generates $S(\mathscr{J}_{11})$ this holds
for all $\epsilon S(\mathscr{J}_{11})$. In particular, we have
\begin{equation*}
  \{u.e_1ub^{\pi}y_1a\}=\{ua^{\pi}y_1ub^{\pi}\}\tag{30}\label{c3:eq30} 
\end{equation*}

Now $(u\circ u a^{\pi}y_1 b)U_{e_1}=\{u e_1ua^{\pi}y_1b\}U_{e_1}+\{u
e_2 u^{\pi}_ay_1b\}U_{e_1}=\{u e_2ua^{\pi}$ $y_1b\}U_{e_1}$ (PD theorem)
$=\{u e_2ua^{\pi}y_1b\}=\{u e_1 ub^{\pi}y_1a\}^{\eta}$ (by
\eqref{c3:eq27})= \break $\{ua^{\pi}y_1ub\}^{\eta}$ (by
\eqref{c3:eq30})$=\{uay_1 ub\}$ (by 
\eqref{c3:eq27}). Going back to \eqref{c3:eq28} we obtain
$n(a^{\pi}y_1b+b^{\pi}y_1a)=n\{ua 
y^{\eta}_1ub\}=ny^{\eta}_1 U_{ua,ub}$ as required. This completes the
proof.
We obtain next an important corollary of Lemma \ref{c3:sec9:lem5} namley
\end{proof}

\begin{lemma}\label{c3:sec9:lem6}
Let $\mathfrak{n}$ be as in Lemma \ref{c3:sec9:lem5} and let $ua$ be invertible,
$a\epsilon S(\mathscr{J}_{11})$. Then there exists $a b\epsilon
S(_{11})$. Such that $nab=n=nba,n$.
\end{lemma}

\begin{proof}
The hypothesis implies that $U_{ua}$ is a invertible. Hence
$z_1=e_2U_{ua}\neq 0$ in $\mathscr{J}_{11}$. Then $z^{-1}_1$ exits in
$S(\mathscr{J}_{11})$. Applying the second part of Lemma \ref{c3:sec9:lem5} to
$y_1=e_1$ shows that $na^{\pi}a=nz_1$ holds for all $n\epsilon
\mathfrak{n}$. Hence replacing $n$ by $nz^{-1}_1$ gives $nba=n$ for
$b=z^{-1}_1a$. Also $(ua)^{\eta}=ua^{\pi}$ is invertible so
$w_1=e_2U_{ua}\pi$ invertible in $S(\mathscr{J}_{11})$\pageoriginale
and $naa^{\pi}=nw_1, n\epsilon \mathfrak{n}$. Multiplying by
$w^{-1}_1$ on the right gives $nac=n$, $c=a^{\pi} w^{-1}_1$. It now
follows that $nb=na$ and $nab=n=nba$, $n\epsilon\mathfrak{n}$. 
\end{proof}

\begin{lemma}\label{c3:sec9:lem7}
  If $x_2\epsilon \mathscr{J}_{22}$ and $m\epsilon \mathfrak{m}$ is
  invertible then $mx_2=mx_1m_1$, $x_1=x_2U^{-1}_m, m_1=m^{2}U_{e_2}$.
\end{lemma}

\begin{proof}
  We have $x_2=x_1U_m$, where $x_1=x_2U^{-1}_m$. Then $mx_2=x_2\circ m=
  x_1U_mV_m=x_1U_{m,m^{2}}=\{m x_1m^{2}\}=\{mxx_1m^{2}U_{e_1}\}=\{m
  x_1m_1\}=(m\circ x_1)\circ m_1=mx_1 m_1$. 
  We prove next
\end{proof}

\begin{lemma}\label{c3:sec9:lem8}
If $\mathfrak{m}_o=\{m\epsilon \mathfrak{m}|m^{\eta}=-m\}$,
$\mathfrak{m}^{\ast}=\{m\epsilon
\mathfrak{m}|mx_1=mx^{\eta}_1,x_1\epsilon \mathscr{J}_{11}\}$ then
$\mathfrak{m}_o\subseteq
\mathfrak{m}^{\ast}\cup\mathfrak{n},\mathfrak{n}=u \in s(\mathscr{J}_{11})$. 
\end{lemma}

\begin{proof}
Let $m\epsilon \mathfrak{m}_o$ and assume $m\epsilon
\mathfrak{m}^{\ast}$. Then we have an $x_1\epsilon \mathscr{J}_{11}$
such that $n=mx_1-mx_1^{\eta}\neq 0$ and we have to show that
$m\epsilon \mathfrak{n}$. Since
$m^{\eta}=-m,n=mx_1+(mx_1)^{\eta}=uy_1, y_1=(u \circ mx_1)U_{e_1}$, by
Lemma \ref{c3:sec9:lem2}. Now $m$ is invertible since otherwise, $m^{2}=0$ and hence
$xU_m=0$ and $x^{2}U_m=0$ for $x=x_1-x_1^{\eta}\epsilon
\mathscr{J}_{11}+\mathscr{J}_{22}$. Then $n^{2}=(m\circ x)^{2}=0$ by
QJ 30. However, $n=uy_1$ and Since $y_1\epsilon \mathscr{J}_{11}$ is
$\neq 0$ $u$ is invertible, $n$ is invertible. This contradiction
proves $m$ invertible. We have
$n=mx_1-mx_1^{\eta}=mx_1-m(x^{\eta}_1U^{-1}_m)(m^{2}U_{e_1})$ (by
Lemma \ref{c3:sec9:lem7})$=a=x_1-(x^{\eta}_{1}U^{-1}_m)(m^{2}U_{e_1}\epsilon 
  (\mathscr{J}_{11})$. Thus
\begin{equation*}
  n=uy_1=ma,y_1\neq 0\quad {\text{in}}\quad \mathscr{J}_{11},a\epsilon
  su(\mathscr{J}_{11})\tag{31}\label{c3:eq31} 
\end{equation*}

We now apply lemma \ref{c3:sec9:lem6} to $m$(replacing $u$) in the isotope
$\mathscr{J}_m$. Since $ma=n$ is invertible in $\mathscr{J}$, hence in
$\mathscr{J}_m$, and since the $S(\mathscr{J}_{11})$ module structure
on $\mathfrak{m}$ is unchanged in passing from $\mathscr{J}$ to
$\mathscr{J}_m$ it follows\pageoriginale from Lemma \ref{c3:sec9:lem6}
that there 
exists $ab\in  S(\mathscr{J}_{11})$ such that $mab=m$. Then
$m=nb=uy_1b\epsilon \mathfrak{n}$ as requried.

As before, Let $\nu_1$ be the monomorphism $x_1\to \ob{V}_{x_1}$ of
$\mathscr{J}_{11}$ into (End $\mathfrak{m}^{(q)}$. Also let $\nu_1$
denote the (unique) extension of this to a homomorphism of
$s(\mathscr{J}_{11})$ into End $\mathfrak{m}$ and let
$\mathcal{E}_1=S(\mathscr{J}_{11})^{\nu_1}$. Then
$S(\mathscr{J}_{11})^{\nu_1}$ is the algebra of endomorphisms
generated by the $\ob{V}_{x_1,x_1}\epsilon \mathscr{J}_{11}$.

We shall now prove the following important result on $\mathcal{E}_1$.
\end{proof}

\begin{lemma}\label{c3:sec9:lem9}
The involution $\pi$ in $S(\mathscr{J}_{11})$ induces an involution
$\pi$ in $\mathcal{E}_1$. If we identity $\mathscr{J}_{11}$ with its
image $\mathscr{J}_{11}^{\nu_1}$ in $\mathcal{E}_1$ then
$\mathscr{J}_{11}\subseteq \mathscr{H}(\mathcal{E}_1\pi)$,
$\mathscr{J}_{11}$ contians $1$ and every $a^{\pi}x_1a$, $x_1\epsilon
\mathscr{J}_{11}, a\epsilon\mathcal{E}_1$. Also $(\mathcal{E}_1,\pi)$ is
simple and the nonzero elements of $\mathscr{J}_{11}(\subseteq
\mathcal{E}_1)$ are invertible.
\end{lemma}

\begin{proof}
If $\mathscr{J}_{11}=\{0,\pm e_1\}$ then $\mathcal{E}_1=\mathbb{Z}_2$
or $\mathbb{Z}_3$ and the result is clear. From now on we assume
$\mathscr{J}_{11}\neq \{0, \pm e_1\}$ so lemma \ref{c3:sec9:lem1} is applicable. In
particular, $\mathfrak{m}$ is spanned by invertible elements. To show
that $\pi$ induces an involution in $\mathcal{E}_1$ we have to show
that (ker $\nu_1$)$^{\pi}\subseteq$ ker $\nu_1$ and for this it
suffices to show that if $k\epsilon$ ker $\nu_1$ and $m\epsilon
\mathfrak{m}$ is invertible then $mk^{\pi}=0$. Let $\mathscr{J}_m$ be
the isotope of $\mathscr{J}$ defined by $m$ as before. By \eqref{c3:eq27}
applied to $m$ (in place of $u$) we have
$mk^{\pi}=(mk)^{\eta}=0$. Hence the first statement is proved. It is
clear that $\mathscr{J}_{11}\subseteq \mathscr{H}(\mathcal{E}_1,\pi)$
and $1\epsilon \mathscr{J}_{11}$. Let $x_1\epsilon
\mathscr{J}_{11},a\epsilon S(\mathscr{J}_{11})$, $m$ an invertible
element of $\mathfrak{m}$. Then, by Lemma \ref{c3:sec9:lem5}, there
exists an element 
$y_m\epsilon \mathscr{J}_{11}$ such that $xa^{\pi}x_1a=xy_m$ for all
$x$ in $mS(\mathscr{J}_{11})$. Let $n$ be a second invertible element
of and let $y_n\epsilon \mathscr{J}_{11}$ satisfy
$xa^{\pi}x_1a=xy_n,x\epsilon n S(\mathscr{J}_{11})$. As in Lemma let
$u_1\neq 0$, $v_1\neq 0$ be elements of $\mathscr{J}_{11}$ such that
$p=mu_1+nv_1$ is invertible and let $y_p\epsilon \mathscr{J}_{11}$
satisfy $xa^{\pi} x_1a=xy_p, x\epsilon p S(\mathscr{J}_{11})$. Suppose
$y_m\neq y_n$. Then $d_1=y_m-y_n\neq 0$ is invertible in
$\mathscr{J}_{11}$ with inverse $d^{-1}_1$. Then\pageoriginale $n
v_1d^{-1}_1(a^{\pi}x_1a-y_n)=0$ so
$pd^{-1}_1(a^{\pi}x_1a-y_n)=mu_1d^{-1}_1(a^{\pi}x_1a-y_n)=mu_1d^{-1}_1(y_m-y_n)=mu_1$. Hence
$m\epsilon pS(\mathscr{J}_{11})$ and, similarly, $n\epsilon
pS(\mathscr{J}_{11})$. Then $ma^{\pi}x_1a=my_p$ and $na^{\pi}y_1a=
ny_p$. This implies $y_p=y_m=y_n$ contradicting $y_m\neq y_n$. Hence
there exists an element $y_1\epsilon \mathscr{J}_{11}$ such that
$ma^{\pi}x_1a=my_1$ for all invertible $m\epsilon \mathfrak{m}$. Since
$\mathfrak{m}$ is spanned by invertible elements this gives the second
statement of the Lemma. We note next that the non-zero elements of
$\mathscr{J}_{11}$ are invertible in $\mathcal{E}_1$ since
$\mathscr{J}_{11}$ is a division subalgebra of
$\mathcal{E}_{1}^{(q)}$.

It remains to show that $(\mathcal{E}_1,\pi)$ is simple. Let
$a\epsilon \mathcal{E}_1$ then $a^{\pi} a$, $aa^{\pi}\epsilon\break
\mathscr{J}_{11}$ and the non-zero element of $\mathscr{J}_{11}$ are
invertible; the proof of the theorem of
Herstein-Kleinfeld. Osbon-McCrimmon shows that either
$aa^{\pi}=0=a^{\pi}a$ or $a$ is invertible. Let $u$ be an element
strongly connecting $e_1$ and $e_2$, as before. By Lemma \ref{c3:sec9:lem5}, we have
$ua^{\pi}a=ue_1^{\eta}U_{ua}=ue_2U_{ua}=u((ua)^{2}U_{e_1})$. Hence
$a^{\pi}a=0$ implies $(ua)^{2}U_{e_1}=0$. Since $(ua)^{2}\epsilon
\mathscr{J}_{11}+\mathscr{J}_{22}$ this implies $(ua)^{2}$ not
invertible. Then $ua$ is not invertible and $(ua)^{2}=0$. Thus we see
that if $a\epsilon \mathcal{E}_1$ then either $a^{\pi}a$ and
$aa^{\pi}$ are invertible on $(ua)^{2}=0$.

Now let $Z$  be a proper ideal of $(\mathcal{E}_1,\pi)$ and
let $z\epsilon Z$. Then $z+z^{\pi},z^{\pi}z\epsilon
Z\cap \mathscr{J}_{11}$. Since $Z$ contains no
invertible elements, we have $z+z^{\pi}=0=z z^{\pi}$. Hence $z^{2}=0$
and $(uz)^{2}=0$. By Lemma \ref{c3:sec9:lem2}, if $m\epsilon\mathfrak{m}$,
$m+m^{\eta}=ux_1,x_1=(u\circ m)U_{e_1}$. Then $f(m)=x_1=(u\circ
m)U_{e_1}$ defines a $\Phi$-homomorphism of $\mathfrak{m}$ into
$\mathscr{J}_{11}$. Since $\eta$ is an automorphism of $\mathscr{J}$
mapping $\mathfrak{m}$ onto $\mathfrak{m}$ and $\mathscr{J}_{11}$
onto $\mathscr{J}_{22}$ it is clear that $\eta$ defines\pageoriginale
an isomorphism of $\mathcal{E}_1$ onto the subalgebra $\mathcal{E}_2$
of End $\mathfrak{m}$ generated by $\mathscr{J}_{22}^{\nu_2}$
extending the isomorphism of $\mathscr{J}_{11}$ onto
$\mathscr{J}_{22}$. Moreover, we have $(ma)^{\eta}=m^{\eta}a^{\eta}$,
$m\epsilon \mathfrak{m}, a\epsilon \mathcal{E}_1$. Iteration of the
result of Lemma \ref{c3:sec9:lem7} shows that if $m$ is an invertible element of
$\mathfrak{m}$ then for any $a\epsilon \mathcal{E}_1$ there exists
$ab\epsilon \mathcal{E}_1$ such that $ma^{\eta}=mb$. Then
$uf(ma)=ma+(ma)^{\eta}=ma+m^{\eta}a^{\eta}=m(a-a^{\eta})+(m+m^{\eta})+(m+m^{\eta})a^{\eta}=m(a-b)+uf(m)a^{\eta}=m(a-b)+ua^{\eta}f(m)=m(a-b)+ua^{\pi}f(m)$ 
(by \eqref{c3:eq27}). Hence
\begin{equation*}
  m(a-b)=u(f(ma)-a^{\pi}f(m))\tag{32}\label{c3:eq32}
\end{equation*}

In particular, taking $a=z\epsilon \mathscr{Z}$ we obtain $w$ so that
$mz^{\eta}=mw$ and $m(z-w)=ur$, $r=f(mz)+zf(m)\epsilon
\mathcal{E}_1$. Since $wz\epsilon \mathscr{Z}$ we have
$w^{\pi}z+z^{\pi}w=0$. Also
$mw^{2}=mz^{\eta}w=mwz^{\eta}=m(z^{\eta})^{2}=0$ since
$z^{2}=0$. Hence $w$ is not invertible and consequently
$w^{\pi}w=0$. Then
$(z-w)^{\pi}(z-w)=z^{\pi}z-z^{\pi}w-w^{\pi}z+w^{\pi}w=0$. This
relation and the second part of Lemma \ref{c3:sec9:lem5} applied to the isotope
$\mathscr{J}_n$ imply that $m(z-w)$ is not invertible in this
isotope. Hence $m(z-w)$ is not invertible in $\mathscr{J}$ and
consequently $(ur)^{2}=(m(z-w))^{2}=0$. Then a reversal of the
argument shows that $r^{\pi}r=0$. Then $r$ is not invertible and since
$r=f(mz)+zf(m)$ and $z$ is in $\mathscr{Z}$ which is a  nil ideal,
$f(mz)$ is not invertible. Since $f(mz)\epsilon \mathscr{J}_{11}$ it
follows that $f(mz)=0$. Hence we have $mz+(mz)^{\eta}=0$. By the
second part of Lemma \ref{c3:sec9:lem2}, this implies $mz\circ u=0$. Then $0=(mz\circ
u)U_{e_2}=\{u e_1 mz\}$ (by linearization of
$x^{2}U_{e_2}=e_1U_x,x\epsilon \mathfrak{m}$)$=\{uz^{\pi}e_1m\}$ (by
$(29)$)$=-\{uz e_1 m\}=-(uz\circ m)U_{e_2}$. If we replace $m$ by
$m^{\eta}$ in this and apply $\eta$ we obtian $(u z\circ
m)U_{e_1}=0$. Hence we have proved that $uz\circ m=0$ for all
invertible $m$. It follows that this holds for all $m\epsilon
\mathfrak{m}$ and since\pageoriginale $(uz)^{2}=0$, Lemma
\ref{c3:sec9:lem2} of \S 
$6$, shows that this is an absolute zero divisor. Since $\mathscr{J}$
is simple we have $uz=0$. On passing to the isotpe $_m$ we can replace
$u$ by any invertible $m\epsilon \mathfrak{m}$. Then $mz=0$ for all
invertible $m\epsilon \mathfrak{m}$ so $z=0$. Hence $Z=0$
and $(\mathcal{E}_1,\pi)$ is simple.

Lemma \ref{c3:sec9:lem9} shows that $(\mathcal{E}_1,\pi,
\mathscr{J}_{11})$ is an 
associative coordinate algebra satisfying the hypotheses of the
Herstein-Kleinfeld-Osborn-\break McCrimmon theorem. Also in the present case
$\mathscr{J}_{11}$ generates $\mathcal{E}_1$. This excludes case III
given in that theorem so we have only the possibilities I and II
given in the theorem. It is convenient to separate the cases in
which $\mathcal{E}_1$ is a division algebra into the subcases: $\pi$
non-trivial and $\pi=1$  in which case $\mathcal{E}_1$ is field.

Accordingly, the list of possibilities for $(\mathcal{E}_1,\pi,
\mathscr{J}_{11})$ we 
\begin{itemize}
\item[I] $\mathcal{E}_1=\Delta \oplus \Delta^{\pi},\Delta$ an
  assoiative division algebra,
  $\mathscr{H}(\mathcal{E}_1,\pi)=\mathscr{J}_{11}$.
\item[II] $\mathcal{E}_1$ an associative division algebra, $\pi\neq 1$
\item[III] $\mathcal{E}_1$ a field, $\pi=1$.
\end{itemize}
\end{proof}

\begin{lemma}\label{c3:sec9:lem10}
If $\mathcal{E}_1$ is of thye I or II then
$\mathfrak{m}=u\mathcal{E}_1(=uS(\mathscr{J}_{11}))$ and if
$\mathcal{E}_1$ is of type III then
$\mathfrak{m}=\mathfrak{m}^{\ast}=\{m\epsilon \mathfrak{m}|mX_1=m
X_1^{\eta},X_1\epsilon \mathscr{J}_{11}\}$ (as in Lemma \ref{c3:sec9:lem7}).
\end{lemma}

\begin{proof}
We show first that in types I and II any $m\epsilon \mathfrak{m}$ such
that $m^{\eta}=-m$ as contained in $u\mathcal{E}_1$. By Lemma
\ref{c3:sec9:lem7}, it
is enough to show this for for $m$ with $m^{\eta}=-m$ and
$mx^{\eta}_1=mx_1,x_{1_{11}}$. Then, by Lemma \ref{c3:sec9:lem3},
$ma^{\pi}=ma^{\eta}=ma$, $a\epsilon S(\mathscr{J}_{11})$. Now in
types I and II there\pageoriginale exists an invertible element $a$ in
$\mathcal{E}_1$ such that $a=+b-b^{\pi}$. In the case I we choose $b$
invertible in $\Delta$ then $b^{\pi}$ is invertible in $\Delta^{\pi}$
and $a=b-b^{\pi}$, is invertible in
$\mathcal{E}_i=\Delta\oplus\Delta^{\pi}$. in case II we choose an
element $b$ is the division algebra $\mathscr{E}_1$ such that
$b^{\pi}\neq b$. This can be done since $\pi\neq 1$. Then
$a=b-b^{\pi}\neq 0$ is invertible. Now let $m$ be as indicated
$(m^{\eta}=-m, ma^{\pi}=ma^{\eta}$, $a\epsilon
S(\mathscr{J}_{11}))$. Then
$ma=mb-mb^{\pi}=mb-mb^{\eta}=mb+m^{\eta}b^{\eta}=ma+(ma)^{\eta}=ux_1,x_1\epsilon
\mathscr{J}_{11}$. Then $m=ux_1a^{-1}\in u S(\mathscr{J}_{11})$.

Suppose we have type I and let $m\epsilon \mathfrak{m}$. Then
$m+m^{\eta}=ux_1, x_1\epsilon \mathscr{J}_{11}$. Since the type is $I,
x_1=a+a^{\pi}, a\epsilon S(\mathscr{J}_{11})$. Then
$m+m^{\eta}=ua+ua^{\pi}=ua+(ua)^{\eta}$ (by \eqref{c3:eq27}). Hence
$(m-ua)^{\eta}=m^{\eta}-(ua)^{\eta}=ua-m$. Then $m-ua \epsilon
uS(\mathscr{J}_{11}$) and $m\epsilon
US(\mathscr{J}_{11})=u\mathcal{E}_1$.

Suppose we have type II. Then $\mathscr{J}_{11}\neq \{0,\pm e_1\}$ so
$\mathfrak{m}$ is spanned by invertible elements. Hence it suffices to
show that if $m\epsilon \mathfrak{m}$ is invertible then $m\epsilon
u\mathcal{E}_1$. By Lemma \ref{c3:sec9:lem2}, $m+m^{\eta}=uf(m)$ where $f(m)\epsilon
\mathscr{J}_{11}$. By Lemma \ref{c3:sec9:lem6}, if $a\epsilon \mathcal{E}_1$ there
exists $a\, b\epsilon \mathcal{E}_1$ such that $ma^{\eta}=mb$. By
\eqref{c3:eq31}, $m(a-b)=u(f(ma)-a^{\pi}f(m))$ where $f(m), f(ma)\epsilon
\mathscr{J}_{11}$. If $a-b$ is invertible for some $a$ this implies
$m\epsilon u\mathcal{E}_1$. Otherwise, since $\mathcal{E}_1 u$  a
division algebra, $a=b$ for all $a$. Then $f(ma)=a^{\pi}f(m)$ and
applying II, $f(ma)^{\pi}=f(ma)=f(m)a$. Hence $a^{\pi}f(m)=f(m)a$. In
particular, $x_1f(m)=f(m)x_1, x_1\epsilon \mathscr{J}_{11}$ and since
$\mathscr{J}_{11}$ generates $\mathcal{E}_{1, f(m)}$ is in the center
of $\mathcal{E}_1$. Then $(a^{\pi}-1)f(m)=0$. Since we can choose $a$ so
that $a^{\pi}-a$ is inverible, $f(m)=0$. Then $m+m^{\pi}=0$. and
$m\epsilon u\mathcal{E}_1$ by the result proved before.

Now\pageoriginale suppose we have type III. The case
$\mathscr{J}_{11}=\{0,\pm e_1\}$ is trivial so we may assume
$\mathfrak{m}$ is spanned by invertible elements. It suffices to show
that if $m\epsilon \mathfrak{m}$ is invertible then $mx_1=mx_1^{\eta},
x_1\epsilon \mathscr{J}_{11}$. As in the last case, $m+m^{\eta}=uf(m)$
and if $a\epsilon\mathcal{E}_1$ then there exists $b\epsilon
\mathcal{E}_1$ such that $m(a-b)=u(f(ma)-af(m))$ (since $\pi=1$). If
$a-b\neq 0$, $m=uc$, $c\epsilon \mathcal{E}_1$ and $mx_1=u c
x_1=ux_1c$ (by commutativity of $\mathcal{E}_1$)$= ux^{\eta}_1c=uc
x^{\eta}_1=mx_1^{\eta}$ (by \eqref{c3:eq27} and \eqref{c3:eq21}). Hence the result holds in
this case. It remains to conisder the case in which $a=b$ for all
$a$. Then $f(ma)=af(m)=f(m)a, a\epsilon \mathcal{E}_1$. Then
$mx_1+(mx_1)^{\eta}=uf(mx_1)=uf(m)x_1= (m+m^{\eta})x_1$. Then
$(mx_1)^{\eta}=m^{\eta}x_1$ and $mx_1=mx_1^{\eta}$, $x_1\epsilon
\mathscr{J}_{11}$ as required.
We can now complete the 
\end{proof}

\noindent
{Proof of Theorem 10.} Suppose first $(\mathcal{E}_1,\pi
\mathscr{J}_{11})$ is of type I or II. Since
$\mathfrak{m}=u\mathcal{E}_1$ and
$\mathscr{J}_{11}^{\eta}=\mathscr{J}_{22}$ any element of
$\mathscr{J}$ can be written in the form $x_1+y_1^{\eta}+ua,
x_1,y_1\epsilon \mathscr{J}_{11}, a\epsilon \mathcal{E}_1$. Also $a$
is unique since $ua=0$ implies $\mathfrak{m}_a=(u\mathcal{E}_1)a=0$
(Lemma \ref{c3:sec9:lem4}). Hence the mapaping
\begin{equation*}
  \zeta: x_1+y_1^{\eta}+ua\to x_1[11]+y_1[22]+a[21]\tag{33}\label{c3:eq33}
\end{equation*}
is a module isomorphism of $\mathscr{J}$ onto
$\mathscr{H}((\mathcal{E}_1)_2, \mathscr{J}_{11})$. It is clear that
the mapping $\eta':X\to 1[12]X 1[12]=X_{U_1}[12]$ is an automorphism
in $\mathscr{H}((\mathcal{E}_1)_2,\mathscr{J}_{11})$ and by inspection
we have $(X^{\eta})^{\zeta}=(x^{\zeta})^{{\eta}'}$. We shall now show
that $\zeta$ is an algebra isomorphism. Because of the properties of
the Pierce decomposition, the relation between $\eta$ and $\eta'$ and
the quadratic Jordan matrix algebra properties $QN1- QM6$ this will
follow if we can establish\pageoriginale the following formulas:
\begin{itemize}
\item[(i)] $(x_1U_{y_1})^{\zeta}=Y_1x_1y_1[11]$

\item[(ii)] $(x_1 U_{ua})^{\zeta}=(a^{\pi}x_1a)[11]$

\item[(iii)] $((ua)U_{ub})^{\zeta}=ba^{\pi}b[21]$

\item[(iv)] $\{x_1ua^{\pi}ub\}^{\zeta}=(X_1ab+1(X_1 ab)^{\pi})[11]$

\item[(v)] $\{x_1y_1 ua\}^{\zeta}=(ay_1x_1)[21]$

\item[(vi)] $\{y^{\eta}_1 ua x_1\}^{\xi}=y_1ax_1[21]$. 
\end{itemize}

Since $x_1U_{y_1}=y_1x_1y_1$ in $\mathcal{E}_1$, (i) is clear. For
(ii) we use Lemmas \ref{c3:sec9:lem5} and \ref{c3:sec9:lem10} to obtain
$a^{\pi}x_1a=x^{\eta}_1U_{ua}$. For (iii) we have 
\begin{align*}
  (ua)U_{ab}&=-(ub)^{2}U_{e_1}\circ ua +(ua\circ)ubU_{e_2}\circ ub\quad
  (by \eqref{c3:eq26})\\
  &=-e_1U_{ub}\circ ua +e_1U_{ua,ub}\circ ub\quad (PD)\\
  &=-e_1^{\eta}U_{ub}\circ ua+((e^{\eta}_1U_{ua^{\pi},ub^{\pi}})\circ
  ub^{\pi})^{\eta}\\
  &=-uab^{\pi}b+(u(b^{\pi}ab^{\pi}+b^{\pi}ba^{\pi}))^{\eta}\\
  &=-uab^{\pi}b+ub a^{\pi} b+uab^{\pi}b\\
  &=uba^{\pi}b.
\end{align*}

This implies (iii). For (iv), we use $\{x_1ua^{\pi}ub\}=((x_1\circ
ua^{\pi})\circ ub))U_{e_1}=(ua^{\pi}x_1\circ
ub)U_{e_1}=e^{\eta}_1U_{ua^{\pi}x_{1},ub}=x_1 ab+b^{\pi} a^{\pi}
x_1$. This gives (iv). For (v) we have $\{x_1y_1ua\}=(ua \circ
y_1)\circ x_1=u ay_1x_1$. (vi) follows from $\{y^{\eta}_1ua
x_1\}=(ua\circ y^{\eta}_1)\circ x_1=uy^{\eta}_1a x_1=u y_1ax_1$ This
completes the proof of the first part. Now\pageoriginale suppose we
have type III. Then $mx_1=mx_1^{\eta}, m\epsilon \mathfrak{m},
x_1\epsilon \mathscr{J}_{11}$. Also  $\mathcal{E}_1$ is a field and
$\pi=1$. Hence, by Lemma \ref{c3:sec9:lem3}, $ma=ma^{\eta}$, $a\epsilon
\mathcal{E}_1$. We consider the mapping $Q:m\to -m^{2}U_{e_1}$ of
$\mathfrak{m}$ into $\mathcal{E}_1$. We claim that $Q$ is quadratic
mapping of $\mathfrak{m}$ as $\mathcal{E}_1$ module into
$\mathcal{E}_1$. If $m\epsilon \mathfrak{m}, x_1\epsilon
\mathscr{J}_{11}$ then $(mx_1)^{2}U_{e_1}=(m\circ
x_1)^{2}U_{e_1}=m^{2}U_{x_1}U_{e_1}=m^{2}U_{e_1}U_{x_1}=x_1(m^{2}U_{e_1})x^{2}_1$
since $\mathcal{E}_1$ is commutative. It follows that for $a=x_1\ldots
z_1$, $x,\ldots z_1\epsilon \mathscr{J}_{11}$, we have
$(ma)^{2}U_{e_1}=(m^{2}U_{e_1})a^{2}$. We show next that if
$m,n\epsilon\mathfrak{m}$ and $y_1\epsilon \mathscr{J}_{11}$ then
$(mx_1\circ n)U_{e_1}=((mon)U_{e_1}x_1$. Since this is clear for $n=0$
and both sides are in $\mathcal{E}_1$ it suffices to show that $(m
x_1\circ n)U_{e_1}\circ n=(((m\circ n)U_{e_1})x_1)\circ n$. Since
$\mathcal{E}_1$ is commutative $(((m\circ n)U_{e_1})x_1)\circ
n=(n\circ x_1)\circ ((m\circ n)U_{e_1})=\{nx_1(m\circ n)U_{e_1}\}$ (PD
theorem) $=\{nx_1m\circ n\}=x_1U_{n,m\circ n}=x_1(U_nV_m+V_mU_n)$ $(QJ
19)= m(x_1U_n)^{\eta}+(mx_1)U_n$. By $QJ 33$, we have
$(mx_1)U_n=(mx_1)v_{e_1}U_nV_{e_1}=(mx_1)V_n
U_{e_1}V_n-(mx_1)V_{e_1U_n}=(m
x_1)V_nU_{e_1}V_n-(mx_1)V_{n^{2}U_{e_2}}$ (PD $6$)$=((mx_1\circ
n)U_{e_1})\circ n -(mx_1)V_{n^{2}U_{e_2}}$. This and the following
relation give
\begin{align*}
  (((m\circ n)&U_{e_1})x_1)\circ n-(mx_1\circ n)U_{e_1})\circ
  n\tag{34}\label{c3:eq34}\\ 
  &=m(x_1U-n)^{\eta}-(mx_1)V_{n^{2}U_{e_2}}
\end{align*}

The left hand side is a multiple (in $\mathcal{E}_1$) of $n$ and the
right hand side is a multiple of $m$. Hence if $m$ and $n$ are
$\mathcal{E}_1$ independent then we obtian $(((m\circ
n)U_{e_1})x_1)\circ n=(((mx_1)U_{e_1})x_1)\circ n$ which gives the
required relation. Now suppose $m=na$, $a\epsilon \mathcal{E}_1$. Then
again \eqref{c3:eq34} will yield the result provided we can prove that
$n(x_1U_n)^{\eta}=(nx_1)V_{n^{2}U_{e_2}}$. This\pageoriginale follows
since
$n(x_1U_n)^{\eta}=n(x_1U_n)=x_1U_nV_n=x_1U_{n,n^{2}}=\{nx_1n^{2}\}=\{n
x_1n^{2}U_{e_2}\}=(n\circ x_1)\circ n^{2}
U_{e_2}=(nx_1)V_{n^{2}U_{e_2}}$. 

We have now proved that for $Q(m)=-m^{2}U_{e_1}$ we have
$Q(ma)=a^{2}Q(m)$ for all $a=x_1\ldots z_1,x_1,\ldots,z_1\epsilon
\mathscr{J}_{11}$ and $Q(mx_1, n)=x_1Q(m,n)$. The latter implies that
$Q(ma,n)=aQ(m,n)\, m,n\epsilon \mathfrak{m}$, $a\epsilon
\mathcal{E}_1$. This and the first result imply that $Q(m)=a^{2}Q(m)$
for all $a\epsilon \mathcal{E}_1$. Then $Q$ is a quadratic mapping. We
note next that $m^{2}U_{e_2}=(m^{2}U_{e_1})^{\eta}$. Since
$m(m^{2}U_{e_1})=m(m^{2}U{e_1})^{\eta}$ it suffices to show that
$m(m^{2}U_{e_1})=m(m^{2}U_{e_2})$. We have
$m(m^{2}U_{e_1})=m^{2}U_{e_1}\circ m
=\{m^{2}U_{e_1}e_1m\}=\{m^{2}e_1m\}=e_1U_{m,m^{2}}=e_1U_mV_m=m^{2}U_{e_2}V_m=m\circ
m^{2}U_{e_2}=mm^{2}U_{e_2}=mm^{2}U_{e_2}$. Thus
$m^{2}=m^{2}U_{e_1}+m^{2}U_{e_2}=m^{2}U_{e_1}+(m^{2}U_{e_1})^{\eta}$. Since
the elements $m\epsilon \mathfrak{m}$ such that $m^{2}=0$ and $m\circ
n=0$, $n\epsilon \mathfrak{n}$, are absolute zero divisors it now
follows that $Q$ is non-degenerate.

We now introduce $\mathfrak{K}=f_1\epsilon_1\oplus f_2\epsilon_1 \oplus
\mathfrak{m}_a$ direct sum of $\mathfrak{m}$ and two one dimensional
(right) vector spaces over $\mathcal{E}_1$ and extend $Q$ to
$\mathfrak{K}$ by defining $Q(f_1a+f_2b+m)=ab+Q(m),a,b\in
\mathcal{E}_1$. Then $Q$ is a non-degenerate quadratic form on
$\mathfrak{K}$ and $Q(f)=1$ for $f+f_1+f_2$. Hence we can form Jord
$(Q,f)$. It is immediate that $\mathscr{J}'\equiv
f_1\mathscr{J}_{11}+f_2\mathscr{J}_{11}+\mathfrak{m}$ is an outer
ideal containing $f$ in $\mathfrak{K}=$ Jord $(Q,f)$. We now define the
mapping $\zeta$ of $=\mathscr{J}$ onto $\mathfrak{K}$ by
$x_1+y_1^{\eta}+m\to f_1 x_1+f_1y_1+m$. It is easy to check that this
is a monomorphism.

\section{Second structure theorem.}\label{c3:sec10}

Let $\mathscr{J}$ be a simple quadratic Jordan algebra satisfying the
minimum condition (for principal inner ideals). Then $\mathscr{J}$
contains no absolute zero divisors $\neq 0$ since these generate a nil
ideal (Theorem \ref{c3:thm5}). By Theorem \ref{c3:thm7}. $\mathscr{J}$
has an isotope 
$\tilde{\mathscr{J}}$ which has a capacity.\pageoriginale If
the capacity is one then $\tilde{\mathscr{J}}$, hence $\mathscr{J}$, is a
division algebra. If the capacity is two then the structure of
$\ob{\mathscr{J}}$ is given by Theorem \ref{c3:thm10}. This implies that
$\mathscr{J}$ itself has the form given in Theorem
\ref{c3:thm10}. Now assume the capacity of $\tilde{\mathscr{J}}$ is
$n\geqq 3$ and 
let $\tilde{1}=\sum\limits_{1}^{n}f_i$ be decomposition of the unit
$\tilde{1}$ of $\tilde{\mathscr{J}}$ into orthogonal completely primitive
idempotents. Since $\tilde{\mathscr{J}}$ is simple the proof of the First
structure Theorem shows that every $f_j, j>1$, is connected to
$f_1$. Since we can replace $\tilde{\mathscr{J}}$ by an isotope, by Lemma
\ref{c2:lem2} of \S 2.3, we may assume the connectedness is strong. Then we can
apply the strong coordinatization Theorem to conclude that
$\tilde{\mathscr{J}}$ is isomorphic to an algebra
$\mathscr{H}(\mathcal{O}_n,\mathcal{O}_o)$ with the coordinate algebra
$(\mathcal{O},j,\mathcal{O}_o)$. By Theorem 2.2,
$\mathscr{H}(\mathcal{O}_n,\mathcal{O}_o)$ is an outer ideal in
$\mathscr{H}(\mathcal{O}_n)$ and the simplicity of
$\mathscr{H}(\mathcal{O}_n,\mathcal{O}_o)$ implies that
$(\mathcal{O},j)$ is simple. The Pierce inner ideal determined by the
idempotent $1[11]$ in $\mathscr{H}=\mathscr{H}(\mathcal{O}_n,
\mathcal{O}_o)$ is the set of elements $\alpha[11], \alpha \epsilon
\mathcal{O}_0\subseteq N(\mathcal{O})$. Since this Pierce inner ideal
is a division algebra it follows that every non-zero element of
$\mathcal{O}_o$ is invertible in $N(\mathcal{O})$. Hence
$(\mathcal{O},j,\mathcal{O}_o)$ satisfies the hypothesis of the
Herstein-Kleinfeld-Osborn-McCrimmon theorem. Hence
$(\mathcal{O},j,\mathcal{O}_o)$ has one of the types $I-V$ given in the
$H-K-O-M$ theorem. If the type is I-IV then the consideration of
chapter 0 show that $\mathcal{O}_n$ with its standard involution $J_1$
is a simple Aritinian algebra with involution. Since
$\mathscr{H}(\mathcal{O}_n,\mathcal{O}_o)$ is an outer ideal
containing $1$ in $\mathscr{H}(\mathcal{O}_n)$ it follows that
$\mathscr{H}$ is isomorphic to an outer ideal containing $1$ in
an $\mathscr{H}(\mathfrak{a},J),(\mathfrak{a},J)$ simple Artinian
with involution. Then $\mathscr{J}$ also has this form. The remaining
type of coordinate algebra allowed in the $H-K-O-M$\pageoriginale
theorem is an octoion algebra with standard involution over a field
$\Gamma$ with $\Gamma =\mathcal{O}$. In this case we must have
$n=3$. Thus if we take into account the previous results we see that
$\mathscr{J}$ is of one of the following types. 1) a division algebra,
2) an outer ideal containing $1$ in a Jord $(Q,1)$ where $(Q,1)$ is a
non-degenerate quadratic form with base point on a vector space, 3)an
outer ideal containing $1$ in an $\mathscr{H}(\mathfrak{a}.J)$ where
$(\mathfrak{a},J)$ is simple Artinian with involution, 4)an isotope of
an algebra $\mathscr{H}(\mathcal{O}_3)$ where $\mathcal{O}$ is an
octonian algebra over a field $\Gamma/\Phi$ with standard involution.

We now consider the last possibility in greater detail. Let
$c\sum\gamma_i[ii]$, $\gamma_i\neq 0$ in $\Gamma$. Since $c$ is an
invertible element of $N(\mathcal{O}_3)$ it determines an involution
$J_c:X\to c^{-1}X^{t}c$ in $\mathcal{O}_3$. Let
$\mathscr{H}(\mathcal{O}_3,J_x)$ denote the set of matrices in
$\mathcal{O}_3$ which are symmetric under $J_c$ and have diagonal
elements in $\Gamma$. If $\alpha\epsilon \Gamma$ we put
$\alpha\{ii\}=\alpha[ii]=\alpha e_{ii}$ and if $a\epsilon \mathcal{O}$
we put $a\{ij\}=ae_{ij}+\gamma_{j}^{-1}\gamma_i\ob{a}e_{ji}$, $i\neq
j$. Then $\alpha[ii], a\{ij\}\epsilon \mathscr{H}(\mathcal{O}_3,J_c)$
and every element of $\mathscr{H}(\mathcal{O}_3, J_c)$ is a sum of
$\alpha[ii]$ and $a\{ij\}$. If $X\epsilon \mathscr{H}(\mathcal{O}_3)$
then $X c\epsilon \mathscr{H}(\mathcal{O}_3J_c)$ since
$c^{-1}(\ob{Xc})^{t}c=Xc$. In view of the situation for algebras
$\mathscr{H}(\mathfrak{a},J)$, $\mathfrak{a}$ associative, it is
natural to introduce a quadratic Jordan structure in
$\mathscr{H}(\mathcal{O}_3, J_c)$ so that bijective mapping $X\to Xc$
of $\mathscr{H}(\mathcal{O}_3)^{(c)}$ onto
$\mathscr{H}(\mathcal{O}_3,J_c)$ becomes an isomorphism of quadratic
Jordan algebras. We shall call $\mathscr{H}(\mathcal{O}_3,J_c)$
endowed with this structure a cononical quadratic Jordan matrix
algrbra. It is easy to check that the elements $e_i=1\{ii\}$ are
orthogonal idempotents in $\mathscr{H}(\mathcal{O}_3,J_c)$
and\pageoriginale $\sum e_i$ is the unit of
$\mathscr{H}(\mathcal{O}_3,J_c)$. The pierce spaces relative to these
are $\mathscr{H}(\mathcal{O}_3,J_c)_{ii}=\{\alpha\{ii\}|\epsilon
\Gamma, \mathscr{H}(\mathcal{O}_3,J_c)_{ij}=\{a\{ij\}|a\epsilon
\mathcal{O}\}, i\neq j$. It is easy to verify that the formulas for
the $U$-operator for elements in these submodules are identical with
(i)-(x) of \S 1.8 with the exceptions that $(ii)$ and $(iii)$ become
\begin{align*}
  &\alpha\{ii\}U_{a\{ij\}}=\gamma^{-1}_j\gamma_i\ob{a}\gamma
  a\{ij\}\tag*{$(ii)'$}\\
  &b\{ij\}U_{a\{ij\}}=\gamma^{-1}_j\gamma_ia\ob{b}a\{ij\}\tag*{$(iii)'$} 
\end{align*}

It is clear from these formulas that if $\rho \neq 0$ in $\Gamma$ then
$\mathscr{H}(\mathcal{O}_3,J_{\rho
  c})=\mathscr{H}(\mathcal{O}_3,J_c)$. We note next that if
$a_i\epsilon \mathcal{O}$, $n(u_i)\neq 0$, $\delta_i=n(u_i)\gamma_i$
and $d=$diag$\{\gamma_1,\gamma_2,\gamma_3\}$ then there exists an
isomorphism of $\mathscr{H}(\mathcal{O}_3,J_c)$ onto
$\mathscr{H}(\mathcal{O}_3,J_d)$ fixing the $e_{ii}$. First, one can
verify directly that if $u\epsilon \mathcal{O}$, $n(u)\neq 0$, then
the $\Gamma$ -linear mapping of $\mathscr{H}(\mathcal{O}_3,J_c)$ onto
$\mathscr{H}(\mathcal{O}_3,J_d),d=$ diag$\{\gamma_1,
n(u)\gamma_2,n(u)\gamma_3\}$ such that $e_{ii}\to e_{ii}$, $a\{12\}\to
au\{12\}'$, $a\{23\}\to u^{-1}a \ob{u}\{23\}'$, $a\{13\}\to
a\ob{u}\{13\}'$ where
$a\{ij\}'=ae_{ij}+\delta_j^{-1}\delta_i\ob{a}e_{ji},\delta_1=\gamma_1,
\delta_i=n(u)\gamma_i, i=2,3$, is an isomorphism. (Because of the Pierce
relations it is sufficent to verify $QM(ii)]$, $(iii)'$ and $M4$, \S
  2.2). Similarly, one can define isomorphism of
  $\mathscr{H}(\mathcal{O}_3,J_c)$ onto
  $\mathscr{H}(\mathcal{O}_3,J_d),d=$diag$\{n(u)$ $\gamma_1,\gamma_2,
  n(u)\gamma_3\}$ or diag $\{n(u), \gamma_1 n(u)\gamma_2, \gamma_3\}$. Combining
  these and taking account the fact that
  $\mathscr{H}(\mathcal{O}_3,J_{\rho_c})=\mathscr{H}(\mathcal{O}_3,J_c)$
  we obtain an isomorphism of $\mathscr{H}(\mathcal{O}_3,J_c)$ onto
  $\mathscr{H}(\mathcal{O}_3,_d)$ fixing the $e_{ii},d$ as above.
We shall now prove the following

\begin{lemma*}
Let\pageoriginale $\mathscr{J}$ be a quadratic Jordan algebra which is
an isotope of $\mathscr{H}(\mathcal{O}'_3)$, $\mathcal{O}'$ octonian
over a field $\Gamma$. Then $\mathscr{J}$ is isomorphic to a canonica
Jordan matrix algebra $\mathscr{H}(\mathcal{O}_3,J_c)$ where
$\mathcal{O}$ is an octonian algebra.
\end{lemma*}

\begin{proof}
It is easily seen that if $(\mathcal{O}',j)$ is an octonion algebra
with standard involution then $(\mathcal{O}',j)$ is simple. Hence, by
Theorem $2.2$, $\mathscr{H}(\mathcal{O}'_3)$ and every isotope
$\mathscr{J}$ of $\mathscr{H}(\mathcal{O}'_3)$ is simple. Let
$e_1,e_2,\ldots,e_k$ be a supplementary set of primitive orthogonal
idempotents in $\mathscr{J}$ (Lemma $1$ of \S 6). We show first that
$k=3$ and the $e_i$ are completely primitive. Since $\mathscr{J}$ is
not a division algebra $1$ is not completely primitive. Hence if $1$
is primitive then the Theorem on Minimal Inner Ideals shows that an
isotope of $\mathscr{J}$ has capacity two. Since this is simple it
follows. from Theorem 10 that this algebra is special. Since an
isotope of a special quadratic Jordan algebra is special this implies
that $\mathscr{H}(\mathcal{O}'_3)$ is special. Since this is not the
case (\S $1.8$), $1$ is not primitive, so $k>1$. If $k>3$ or $k=3$ and
one the $e_0$ is not completely primitive then the Minimal Inner Ideal
Theorem implies that $\mathscr{J}$ has an isotope containing $l>3$
supplementary orthogonal completely primitive idempotents. Then the
foregoing results show that this isotope, hence
$\mathscr{H}(\mathcal{O}'_3)$ is special. Since this is ruled out we
see tjhat $k=2$ or $3$ and if $k=3$ then the $e_i$ are completely
primitive. It remains to exclude the possibility $k=2$. In this case
the arguments just used show that we may assume $e_1$ completely
primituve, $e_2$ not. By the $MII$ theorem and Lemma $2$ of \S 2.3 we
have an isotope $\tilde{\mathscr{J}}=\mathscr{J}^{(v)}$ where
$v=e_1+v_2,v_2\epsilon \mathscr{J}U_{e_2}$ such that the unit of
$\tilde{\mathscr{J}}$ is $e_1+u_2$, $u_2\epsilon \mathscr{J} U_{e_2}$ and
$u_2$\pageoriginale is a sum of two completley primitive strongly
connected orthogonal idempotents in $\tilde{\mathscr{J}}$. Then
$\mathscr{J}\ob{U}_{u_2}=\mathscr{J} U_{e_{1}+v_2}U_{u_2}=\mathscr{J}
U_{e_2}$ and $\tilde{\mathscr{J}} \ob{U_{u_2}}$ is an isotope of
$\mathscr{J} U_{e_2}$. Since $\tilde{\mathscr{J}}$ is exceptional the
foregoing results show that we can identify $\tilde{\mathscr{J}}$ with an
algebra $\mathscr{H}(\mathcal{O}''_3)$ where $\mathcal{O}''$ is an
octonion algebra. Moreover, we can identify $e_1$ with $1[11]$. Then,
as we saw in \S 5, $\tilde{\mathscr{J}}\tilde{U}_{u_2}$ is the quadratic
Jordan algebra of a quadratic form $S$ with base point such that the
associated symmetric bilinear form is non-degenerate. Since
$\mathscr{J}U_{e_2}$ is an isotope of $\tilde{\mathscr{J}} \tilde{U}_{u_2}$
  it is the quadratic Jordan algebra of a quadratic form $Q$ with base
  point such that $Q(x,y)$ is non-degenerate. Moreover, $\mathscr{J}
  U_{e_2}$ is not a division algebra since $e_2$ is not completely
  primitive. Hence we can choose $x\neq 0$ in $\mathscr{J}U_{e_2}$
  such that $Q(x)=0$. Then $x^{2}=T(x)x$, $T(x)=Q(x,1)$, and if
  $T(x)\neq 0$, $e=T(x)^{-1}x$ is an idempotent $\neq 0$, $e_2$,
  contrary to the primitivity of $e_2$. Hence $T(x)=0$. Since $Q$ is
  non-degenerate there exists $a \,y$ in $\mathscr{J} U_{e_2}$ such that
  $Q(x,y)=1$ and we may assume also that $Q(y)=0$. Then, as for $x$,
  we have $T(y)=0$. Since $Q(a,b)$ is non-degenerate there exists a
  $w$ such that $T(w)=Q(w,1)=1$. Then $z=w-Q(x,w)y-Q(y,w)x$ sarisfies
  $Q(x,z)=0=Q(y,z)$, $T(z)=1$. Put $e=z+x-Q(z)y$. Then $T(e)=1$ and
  $Q(e)=0$. Hence $e$ is an idempotent $\neq 0$, $e_2$. This
  contradiction proves our assertion on the idempotents.

Now let $e_1,e_2,e_3$ he supplementary completely primitive orthogonal
idempotents in $\mathscr{J}$. These are connected so we have an
isotope $\mathscr{J}=\mathscr{J}^{(v)}$ where $v=e_1+v_2+v_3$,
$v_i\epsilon \mathscr{J} U_{e_i}$ and the unit $u$ is a sum of three
strongly connected primitive orthogonal idempotents.

As\pageoriginale before, we can identify $\tilde{\mathscr{J}}$ with an
$\mathscr{H}(\mathcal{O}_3)$. $\mathcal{O}$ an octonian algebra over a
field, $e_1$ with $1[11]$. Then $\mathscr{J}$ is the isotope of
$\mathscr{H}(\mathcal{O}_3)$ determined by an element of the form
$e_1+e_2+e_3, e_i\epsilon \mathscr{H}(\mathcal{O}_3)U_1[ii]$. Then
$e_i=\gamma_i[ii]$. Then $\mathscr{J}$ is isomorphic to
$\mathscr{H}(\mathcal{O}_3, J_c)$.

The foregoing lemma and previous results prove the direct part of the
\end{proof}

\noindent
{\text{Second structure Theorem.} Let $\mathscr{J}$ be  a simple
  quadratic Jordan algebra satisfying DCC for principal inner
  ideals. Then $\mathscr{J}$ is of one of the follwing types: 1) a
  quadratic Jordan division algebra, 2) an outer ideal containing $1$
  in a quadratic Jordan algebra of a non-degenerate quadratic form
  with base point over a field $\Gamma/\Phi$, $3$ ) an outer ideal
  containing in $\mathscr{H}(\mathfrak{a},J)$ where $(\mathfrak{a},J)$
  is simple associative Artinian $\mathcal{O}$ with involution, 4) a
  canonical Jordan matrix algebra $\mathscr{H}(\mathcal{O}_3,J_c)$
  where is an octonion algebra over a field $\Gamma/\Phi$ and
  $c=$diag$\{1,\gamma_2,\gamma_3\}, \gamma_i\neq 0$ in
  $\Gamma$. Conversely, any algebra of one of the types 1)-4)
  satisfies the DCC for principal inner ideals and all of these are
  simple with the exception of certian algebras of type 2) which are
  direct sums of two division algebras isomorphic to outer ideals of
  $\Omega^{(q)}$.

We consider the exceptional case indicated in the foregoing
statement. Let $\mathscr{J}$=Jord$(Q,1)$ where $Q$ is a non-degenerate
quadratic form on $\mathscr{J}/\Gamma$ with base point $1$. We have
seen in \S 5 that $\mathscr{J}$ satisfies the DCC for principal inner
ideals and $\mathscr{J}$ is regular, hence, strongly
non-degenerate. Hence $\mathscr{J}=\mathscr{J}_1\oplus
\mathscr{J}_2\oplus\ldots \oplus \mathscr{J}_s$ where $\mathscr{J}_i$
is an ideal and is simple with unit $1$. If $e$ is an idempotent $\neq
0, 1$ in $\mathscr{J}$ then $Q(e)=0$\pageoriginale since $e$ is not invertible, and
$T(e)=1$ since $e^{2}-T(e)e+Q(e)=0$. Then the formula
$yU_x=Q(x,\ob{y})x-Q(x)\ob{y}$ in shows that $\mathscr{J} U_e=\Omega
e$. In particular $\mathscr{J}_i=\mathscr{J}U_{1_i}=\Omega 1_i$. If
$s> 1$ we put $u=1_1-1_2$. Then $T(u)=T(1_1)-T(1_2)=0$ so
$u^{2}+Q(u)=0$. But $u^{2}=1_1+1_2$. Hence $Q(u)=-1$ and
$1_1+1_2=1=\sum1_i$, which implies that $s=2$. Thus either
$\mathscr{J}=$Jord$(Q,1)$, $Q$ non-degenerate, is  simple or
$\mathscr{J}=\Omega 1_1\oplus \Omega 1_2$. Suppose $\mathscr{J}$ is
simple and not a division algebra. Suppose first that $\mathscr{J}$
contians an idempotent $e\neq 0,1$. Then $\mathscr{J}U_e=\Omega e$ so
$e$ is completely primitive. The same is true of $1-e$. Hence
$\mathscr{J}$ is of capcity two. Next suppose $\mathscr{J}$ contains
no idempotent $\neq 0,1$ (and is simple and not a division
algebra). Then the Theorem on Minimal Inner Ideals shows that there
exists an isotope of $\mathscr{J}$ which is of capacity two. Thus we
have the following possibilities for $\mathscr{J}=$Jord $(Q,1)$, $Q$
non-degenerate I $\mathscr{J} =\Omega 1_1\oplus \Omega 1_2$, II $\mathscr{J}$ is a
division algebra III $\mathscr{J}$ is simple and has an isotope of
capacity two. Now let $\mathfrak{K}$ be an outer ideal in
$\mathscr{J}$ containing $1$. In case $I$ it is immediate that
$\mathfrak{K}=\Omega_11_1\oplus \Omega_21_2$ where $\Omega_i$ is an
outer ideal containing $1_i$ in $\Omega$ so $\Omega_i$ is a division
algebra. In case II $\mathfrak{K}$ is a division algebra. In case III
$\mathfrak{K}$ is simple by Lemma $2$ of \S $6$.


If $\mathscr{J}$ is of types 1), 3) or 4) then we have seen in \S $5$
that satisfies the DCC for pricipal inner ideals. Also in there cases
it follows from Theorem $2.2$ and lemma $2$ of \S $6$ that
$\mathscr{J}$ is simple. This completes the proof of the second
statement of the second structure Theorem.

We\pageoriginale now consider a special case of this theorem, namely,
that in which $\mathscr{J}$ is finite dimensional over algebraically
closed field $\Phi$. The only finite dimensional quadratic Jordan
division algebra over $\Phi$ is $\Phi$ itself (see \S $1.10$). It is
clear also that the field $\Gamma$ in the statement of the theorem is
finite dimensional over $\Phi$, so $\Gamma=\Phi$. The simple algebra
with involution over $\Phi$ are: $\Phi_n\oplus \Phi_n$ with exchange
involution, $\Phi_n$ with standard involution, $\Phi_{2m}$ with the
involution $J:X\to S^{-1} X^{t}S$ where
$S=$diag$\{Q,Q\ldots,Q\}$,$Q=
\begin{pmatrix}
0 & 1\\
-1 & 0
\end{pmatrix}$. In all cases it is easy to check that any outer ideals
of $\mathscr{H})(\mathfrak{a},J)$ containing 1 in
$\mathscr{H}(\mathfrak{a},J)$ coincides with
$\mathscr{H}(\mathfrak{a},J)$. The same is true of Jord $(Q,1)$ for a
non-degenerate $Q$. There is only one algebra of octonion's
$\mathcal{O}$ over $\Phi$ (the split one). Since the norm form for
this represents every $\rho\neq 0$ in $\Phi$ it is clear that there is
only one exceptional simple quadratic Jordan algebra over $\Phi$,
namely, $\mathscr{H}(\mathcal{O}_3)$.

The determination of the simple quadratic Jordan algebras of capacity
two, which was so arduous in the general case, can be done quickly for
finite dimensional algebras over an algebraically closed field. In
this case $\mathscr{J}=\Phi e_1\oplus \Phi_{e_2}\oplus \mathfrak{m}$
where the $e_i$ are supplementary orthogonal idempotents and
$\mathfrak{m}=\mathscr{J}_{12}$. If $m\epsilon \mathfrak{m}$,
$m^{2}=\mu e_1+\nu e_2, \mu, \nu \epsilon \Phi$. As before,
$\mu m=\mu e_i \circ m=\{\Gamma
e_1e_1m\}=\{m^{2}e_1m\}=e_1U_{m.m^{2}}=e_1U_mV_m=m^{2}U_{e_2}V_m=\nu
m$. Hence $\mu=\nu$ and $m^{2}=\mu 1 =-Q(m)1$ where $Q$ is a
quadratic form on $\mathfrak{m}$. We extend this to $\mathscr{J}$ by
defining $Q(\alpha e_1+\beta e_2+m)=\alpha\beta+Q(m)$. It is easy to
check that if $x=\alpha e_1+\beta_{e_2}+m $ and $T(x)=Q(x,1)$ then
$x^{2}=T(x)x+Q(x)1=0$\pageoriginale and
$x^{3}-T(x)x^{2}+Q(x)=0$. Hence $\mathscr{J}=$Jord $(Q,1)$. Since
$\mathscr{J}$ is simple, $Q$ is non-degenerate.

There remains the problem of isomorphism of simple quadratic Jordan
algebras with DCC for principal inner ideals. This can be discussed as
in the linear case considered on pp $183-187$ and $378-381$ of
Jacobson's book [2].

\begin{thebibliography}{99}
\bibitem{Cohn1}{Cohn, P.M.} {\em Universal Algebra}, Harper and Row,
  New York, 1965.
\bibitem{Jacobson1}{Jacobson, N.}{\em Structure theory for a class of
  Jordan algebras}, Proc. Nat. Acad. Sci. U.S.A 55(1966),
  243-251. 
\bibitem{Jacobson2}: {\em Structure and Reprsebtations of Jordan Algebras},
  A.M.S. colloq. Publ. Vol. XXXIX, 1968.
\bibitem{JacobsonMcCrimmon1}{Jacobson N. and McCrimmon, R.} {\em
  Quadratic Jordan  algebras of quadratic forms}, forthcoming.
\bibitem{McCrimmon1}{McCrimmon, K.}{\em A general theory of Jordan rings},
  Proc. Nat. Acad. Sci. U.S.A. 56(1966), 1072-1079. 
\bibitem{McCrimmon2}: {\em A
    general theory of Jordan} rings, unpublished article giving the
  proofs of results announced in \cite{McCrimmon1}.
\bibitem{McCrimmon3}: {\em Macdonald's theorems
      for Jordan algebras}, to appear.
\bibitem{McCrimmon4}: {\em The Freudenthal-Springer-Tits constructions
  of exceptional 
  Jordan algebras}, Trans. Amer. Math. Soc. 139 (1969), 495-510.
\bibitem{McCrimmon5}:
  {\em The Freudenthal-Springer-Tits constructions revisited}, to appear
  in Trans.
\bibitem{AmerMathSoc6}{Amer. Math. Soc.}{\em The radical of a Jordan algebra},
  Proc. Nat. Acad. Sci. U.S.A. 62 (1969), 671-678.
\bibitem{Osborn1}{Osborn. J. M.}{\em Jordan algebras of capacity two},
  Proc. Nat. Acad. Sci. U.S.A. 57 (1967), 582-589.
\bibitem{Schafer1}{Schafer, R. D.}{\em An introduction to non-associative
  algebras}, Academic Press, New york, 1966.
\end{thebibliography}
