
\thispagestyle{empty}

\begin{center}
{\Large\bf Lectures On}\\[5pt] 
{\Large\bf Quadratic Jordan Algebras}
\vfill

{\bf  By}
\medskip

{\large\bf  N. Jacobson}
\vfill

{\bf Tata Institute of Fundamental Research, Bombay}

{\bf  1969}
\end{center}

\eject

\thispagestyle{empty}
\begin{center}
{\Large\bf Lectures on Quadratic Jordan Algebras}
\vfill



{\bf  By}
\medskip

{\large\bf  N. Jacobson}
\vfill



\parbox{0.7\textwidth}{No part of this book may be reproduced in any from by print,
microfilm or any other means without written permission from the
Tata Institute of Fundamental Research, Colaba, Bombay 5}
\vfill

{\bf  Tata Institure of Fundamental Reasearch,}

{\bf  Bombay}

{\bf  1969}
\end{center}



\chapter{Preface}

Until recently the structure theory of Jordan algebras dealt
exclusively with finite dimensional algebras over fields of
characteristic $\neq 2$. In 1966 the present author succeeded in
developing a structure theory for Jordan algebras of characteristic
$\neq 2$ which was an analogue of the Wedderburn-Artin structure
theory of semi-simple associate rings with minimum condition on left
or right ideals. In this the role of the left or right ideals of the
associative thory was played by quadratic ideals (new called inner
ideals) which were defined as subspaces $Z$ of the Jordan algebra
$\mathscr{J}$ such that $\mathscr{J} U_b \subseteq Z$ where $U_b = 2
R_b^2 -m R_{b^2}, R_a$ the multiplication by $b$ in $\mathscr{J}$. The
operator $U_b$ ($P(b)$ in the notation  of Braun and Kocher), which in
an associative algebra is $x \to b x b$, was introduced into abstract
Jordan algebras by the author in 1955 and it has playedan increasingly
important role in the theory and lits applications. It has been fairly
clear for some time that an extension of the structure theory which
was to encompass the characteristic two case would have to be
``quadratic'' in character, that is, would have to be based on the
composition $y U_x$ rather than the usual $x, y$ (which is
$\frac{1}{2}(xy + yx)$ in associative algebras). The first indication
of this appeared already in 1947 in a paper of Kaplansky's which
extended a result of Ancocheas's on Jordan homomorphisms (then called
semi-homomorphisms) of  associative algebras to the characteristic two
case by redefining Jordan homomorphisms using the product $xyx$ in
place of $xy+ yx$.

a completely satisfactory extension of the author's structure theory
which include characteristic two or more precisely algebras over an
arbitrary commutative ring has been given by McCrimmon in
\cite{McCrimmon1} and \cite{McCrimmon2}, McCrimmon's theory begins
with a simple and beautiful axiomatization of the composition
$YU_x$. In addition to the quadratic character of the mapping $x \o
U_x$ of into its algebra of endomorphisms and the existence of unit 1
such that $U_1=1$ one has to assume only the so-called ``fundamental
formula'' $U_x U_y U_x = U_{yU_x}$, one additional  indentity, and the
linearizations of these. Instead of assuming the linearizations it is
equivalent and neater to assume that the two identities carry over on
extension of the coefficient ring. If the coefficient ring $\Phi$
contains $\frac{1}{2}$ then the notion of a quadratic Jordan algebra
is equivalent to the classical notion of a (linear) Jordan algebra
there is a canonical way of passing from the operator $U$ to the usual
multiplication $R$ and back. Based on these fundations one can carry
over the fundamental notions (inverses, isotopy, powers) of the linear
theory to the quadratic case and extend the Artin like structure
theory to quadratic Jordan algebras. In particular, one obtains for
the first time a satisfactory Jordan structure theory for finite
dimensional algebras over a field of characteristic two.

In these lectures we shall detailded and self-contained exposition of
McCrimmon's  structure theiry including his recently developed theory
of radicals and absolute zero divisors which constitute an important
addition even to the classical linear theory. In our treatment we
restrict attention to algebras with unit. This effects a sybstantial
simplication. However, it should be noted that McCrimmon has also
given an axiomatization for quadratic Jordan algebras withour unit and
has  developted the structure theory also for these. Perhaps the
reader should be warned at the outsetthat he may find two (hopefully
no more) parts of the exposition somewhat heavynamely, the derivation
of the long list of identities in \S 1.3 and the proof of Osborn's
thorem on algebras of capacity two. The first of these could have been
avoided by proving a general theorem in identities due to
Macdonald. However, time did not permit this. The simplification of
the proof of Osborn's theorem remains an open problem. We shall see at
the end of our exposition that this difficulty evaporates in the
important special case of finite dimensional quadratic Jordan algebras
over an algebraically closed field.
\bigskip

\hfill  {\large\bf Nathan Jacobson}

