
\chapter{General Properties of Finite Elements}\label{chap5}

IT\pageoriginale WOULD HAVE been observed that upto now we have not
defined finite elements in a precise manner. Various polygons like
triangles, rectangles, etc.\@ were loosely called finite elements. We
rectify this omission and make precise the ideas expressed in the
previous sections.

\begin{definition}\label{chap5-defi5.1}
A finite element is a triple $(K,\Sigma,P)$ such that
\begin{itemize}
\item[(i)] $K\subset \mathbb{R}^{n}$ with a Lipschitz continuous
  boundary $\p K$ and $\Int K\neq \phi$.

\item[(ii)] $\Sigma$ is a finite set of linear forms over
  $C^{\infty}(K)$. The set $\Sigma$ is said to be the set of degrees
  of freedom of the finite element.

\item[(iii)] $P$ is a finite dimensional space of real-valued
  functions over $K$ such that $\Sigma$ is $P$-unisolvent: i.e.\@ if
  $\Sigma=\{\varphi_{i}\}^{N}_{i=1}$ and $\alpha_{i}$, $1\leq i\leq N$
  are any scalars, then there exists a unique function $p\in P$ such
  that 
\begin{equation*}
\varphi_{i}(p)=\alpha_{i},\ 1\leq i\leq N.\tag{5.1}\label{chap5-eq5.1}
\end{equation*}
\end{itemize}
\end{definition}

Condition (iii) of definition \eqref{chap5-eq5.1} is equivalent to the
conditions that $\dim P=N=\card\Sigma$ and that there exists a set of
functions $\{p_{j}\}^{N}_{j=1}$ with
$\varphi_{i}(p_{j})=\delta_{ij}(1\leq i,j\leq N)$, which forms a basis
of $P$ over $\mathbb{R}$. Given any $p\in P$ we may write
\begin{equation*}
p=\sum^{N}_{i=1}\varphi_{i}(p)p_{i}.\tag{5.2}\label{chap5-eq5.2}
\end{equation*}

Instead of $(K, \sum, P)$ one writes at times $(K, \sum_K, P_K)$ for the finite element. 

In the various examples we cited in Sec.~\ref{chap4} our set of
degrees of freedom for a finite element $K$ (which was an $n$-simplex
or hyper-rectangle) had elements\pageoriginale of the following type:

\begin{description}
\item[Type~1:] $\varphi^{0}_{i}$ given by $p\mapsto p(a^{0}_{i})$. The
  points $\{a^{0}_{i}\}$ were the vertices, the mid-points of sides,
  etc...

\item[Type~2:] $\varphi^{1}_{i,k}$ given by $p\mapsto
  \Dp(a^{1}_{i})(\xi^{1}_{i,k})$. For instance, in the Hermite
  triangle of type (3) (cf.\@ Example~\ref{chap4-exam4.6}), we had
  $a^{1}_{i}=a_{i}$, $\xi^{1}_{i,k}=a_{i}-a_{k}$, where $a_{1}$,
  $a_{2}$, $a_{3}$ were the vertices.

\item[Type~3:] $\varphi^{2}_{i,kl}$ given by $p\mapsto
  D^{2}p(a^{2}_{i})(\xi^{2}_{i,k},\xi^{2}_{i,l})$. For example, in the
  18-degree-of-freedom triangle, $a^{2}_{i}=a_{i}$,
  $\xi^{2}_{i,k}=e_{1}=\xi^{2}_{i,1}$, the unit vector in the
  $x_{1}$-direction so that we have
  $D^{2}p(a_{i})(e_{1},e_{1})=\dfrac{\p^{2}p}{\p x^{2}_{1}}(a_{i})$ as
  a degree of freedom. (cf.\@ Exercise~\ref{chap4-exer4.7}).
\end{description}

In all these cases the points $\{a^{s}_{i}\}$ for $s=0,1$ and $2$, are
points of $K$ and are called the {\em nodes of the finite element.}

\begin{definition}\label{chap5-defi5.2}
A finite element is called a Lagrange finite element if its degrees of
freedom are only of Type~1. Otherwise it is called a Hermite finite
element. (cf.\@ Remark~\ref{chap4-rem4.2})
\end{definition}

Let $(K,\Sigma,P)$ be a finite element and $v:K\to \mathbb{R}$ be a
``smooth'' function on $K$. Then by virtue of the $P$-unisolvency of
$\Sigma$, there exists a unique element, say, $\pi v\in P$ such that
$\varphi_{i}(\pi v)=\varphi_{i}(v)$ for all $1\leq i\leq N$, where
$\Sigma=\{\varphi_{i}\}^{N}_{i=1}$. The function $\pi v$ is called the
$P$-{\em interpolate function} of $v$ and the operator
$\pi:C^{\infty}(K)\to P$ is called the $P$-{\em interpolation}
operator. If $\{p_{j}\}^{N}_{j=1}$ is a basis for $P$ satisfying
$\varphi_{i}(p_{j})=\delta_{ij}$ for $1\leq i$, $j\leq N$ then we have
the explicit expression
\begin{equation*}
\pi(\cdot)=\sum^{N}_{i=1}\varphi_{i}(\cdot)p_{i}.\tag{5.3}\label{chap5-eq5.3}
\end{equation*}

\begin{example}\label{chap5-exam5.1}
In the triangle of type (1) (see Example~\ref{chap4-exam4.1}),
$P=P_{1}$, $\Sigma=\{\varphi_{i};\varphi_{i}(p)=p(a_{i}),1\leq i\leq
3\}$ and $p_{i}=\lambda_{i}$,\pageoriginale the barycentric coordinate
functions. Thus we also have
\begin{equation*}
\pi v=\sum^{3}_{i=1}v(a_{i})\lambda_{i}.\tag{5.4}
\end{equation*}
\end{example}

\begin{exercise}\label{chap5-exer5.1}
Let $K$ be a triangle with vertices $a_{1}$, $a_{2}$ and $a_{3}$. Let
$a_{ij}(i<j)$ be the mid-point of the side joining $a_{i}$ and
$a_{j}$. Define $\Sigma_{K}=\{p\mapsto p(a_{ij}),1\leq i\leq j\leq
3\}$. Show that $\Sigma$ is $P_{1}$-unisolvent and that in general
$V_{h}\not\subset C^{0}(\overline{\Omega})$ for a triangulation made
up of such finite elements.
\end{exercise}

\begin{exercise}\label{chap5-exer5.2}
Let $K$ be a rectangle in $\mathbb{R}^{2}$ with vertices
$a_{1},a_{2},a_{3},a_{4}$. Let $a_{5},a_{6},a_{7},a_{8}$ be the
midpoints of the sides as in Fig.~\ref{chap4-fig4.5}. If
$\Sigma=\{p\mapsto p(a_{i}), 5\leq i\leq 8\}$, show that $\Sigma$ is
not $Q_{1}$-unisolvent.
\end{exercise}

Let us now consider a {\em family} of finite elements of a given
type. To be more specific, we will consider for instance a family of
triangles of type (2) (see Example~\ref{chap4-exam4.2}), but our
subsequent descriptions extend to all types of finite elements in all
dimensions. 

Pick, in particular, a triangle $\hat{K}$ with vertices
$\{\hat{a}_{1},\hat{a}_{2},\hat{a}_{3}\}$ from this family. Let the
mid-points of the sides be
$\{\hat{a}_{12},\hat{a}_{23},\hat{a}_{13}\}$. Set
$\hat{P}=P_{\hat{K}}=P_{2}$ and define accordingly the associated set
of degrees of freedom for $\hat{K}$ as
$$
\hat{\sum}=\sum_{\hat{K}}=\{p\mapsto p(\hat{a}_{i}),1\leq i\leq
3;\ p\mapsto p(\hat{a}_{ij}), 1\leq i<j\leq 3\}.
$$

In as much as we consider the finite element
$(\hat{K},\hat{\Sigma},\hat{P})$ as fixed in the sequel, it will be
called the {\em reference finite element} of the family.

Given any finite element $K$ with vertices $a_{1}$, $a_{2}$, $a_{3}$
in this family, there exists a unique invertible affine transformation
of $\mathbb{R}^{2}$ i.e.\@ of the form
$F_{K}(\hat{x})=B_{K}\hat{x}+b_{K}$, where $B_{K}$ is an invertible
$2\times 2$ matrix and $b_{K}\in \mathbb{R}^{2}$, such\pageoriginale 
that $F_{K}(\hat{K})=K$ and $F_{K}(\hat{a}_{i})=a_{i}$, $1\leq i\leq
3$. It is easily verified that $F_{K}(\hat{a}_{ij})=a_{ij}$ for $1\leq
i<j\leq 3$. Also, the space $\{p:K\to \mathbb{R}; p=\hat{p}\circ
F^{-1}_{K},\hat{p}\in\hat{P}\}$ is precisely the space
$P_{K}=P_{2}$. Hence the family $\{(K,\Sigma,P)\}$ is equivalently
defined by means of the following data:
\begin{itemize}
\item[(i)] A reference finite element
  $(\hat{K},\hat{\Sigma},\hat{P})$,

\item[(ii)] A family of affine mappings $\{F_{K}\}$ such that
  $F_{K}(\hat{K})=K$, $a_{i}=F_{K}(\hat{a}_{i})$,
\begin{align*}
1\leq i\leq 3, a_{ij} &= F_{K}(\hat{a}_{ij}), 1\leq i<j\leq
3,\quad\text{and}\\
\Sigma_{K} &= \{p\mapsto p(F_{K}(\hat{a}_{i})); p\mapsto
p(F_{K}(\hat{a}_{ij}))\},\\ 
P_{K} &= \{p:K\to \mathbb{R};\ p=\hat{p}\circ F^{-1}_{K},\ \hat{p}\in
\hat{P}\}. 
\end{align*}
\end{itemize}

This special case leads to the following general definition.

\begin{definition}\label{chap5-defi5.3}
Two finite elements $(\hat{K},\hat{\Sigma},\hat{P})$ and
$(K,\Sigma,P)$ are affine equivalent if there exists an affine
transformation $F$ on $\mathbb{R}^{n}$ such that 
\begin{itemize}
\item[(i)] $F(\hat{x})=B\hat{x}+b$, $b\in\mathbb{R}^{n}$, $B$ an
  invertible $n\times n$ matrix,

\item[(ii)] $K=F(\hat{K})$,

\item[(iii)] $a^{s}_{i}=F(\hat{a}^{s}_{i})$, $s=0,1,2$,

\item[(iv)] $\xi^{1}_{i,k}=B\hat{\xi}^{1}_{i,k}$,
  $\xi^{2}_{i,k}=B\hat{\xi}^{2}_{i,k}$,
  $\xi^{2}_{i,l}=B\hat{\xi}^{2}_{i,l'}$ 

and

\item[(v)] $P\{p:K\mapsto \mathbb{R};\ p=\hat{p}\circ
  F^{-1},\ \hat{p}\in\hat{P}\}$. 
\end{itemize}
\end{definition}

This leads to the next definition.

\begin{definition}\label{chap5-defi5.4}
A family $\{(K,\Sigma_{K},P_{K})\}$ of finite elements is called an
affine family if all the finite elements $(K,\Sigma_{K},P_{K})$ are
equivalent to a single reference finite element
$(\hat{K},\hat{\Sigma},\hat{P})$. 
\end{definition}

Let\pageoriginale us see why the relations given by (iv) must be
precisely of that form. We have by (v), $p(x)=\hat{p}(\hat{x})$. This
must be valid when we use the basis functions as well. We have:
\begin{align*}
\hat{p}(\hat{x}) &=
\sum_{i}\hat{p}(\hat{a}^{0}_{i})\hat{p}^{0}_{i}(\hat{x})+\sum_{i,k}D\hat{p}(\hat{a}^{1}_{i})(\hat{\xi}^{1}_{i,k})\hat{p}^{1}_{i,k}(\hat{x})\\ 
&\quad
+\sum_{i,k,l}D^{2}\hat{p}(\hat{a}^{2}_{i})(\hat{\xi}^{2}_{i,k},\hat{\xi}^{2}_{i,l})\hat{p}^{2}_{i,kl}(\hat{x}). 
\end{align*}

Now
$D\hat{p}(\hat{a}^{1}_{i})(\hat{\xi}^{1}_{i,k})=Dp(a^{1}_{i})B\hat{\xi}^{1}_{i,k}$
by a simple application of the chain rule and therefore
$$
D\hat{p}(\hat{a}^{1}_{i})(\hat{\xi}^{1}_{i,k})=Dp(a^{1}_{i})\xi^{1}_{i,k},\quad\text{by
  (iv)}. 
$$

By a similar treatment of the second derivative term, we get
\begin{align*}
\hat{p}(\hat{x}) &=
\sum_{i}p(a^{0}_{i})p^{0}_{i}(x)+\sum_{i,k}Dp(a^{1}_{i})(\xi^{1}_{i})(\xi^{1}_{i,k})p_{i,k}(x)\\ 
&\quad + \sum_{i,k,l}D^{2}p(a^{2}_{i})
(\xi^{2}_{ik},\xi^{2}_{il})p_{ikl}(x)=p(x). 
\end{align*}

Thus the relations (iv) and (v) are compatible.

\begin{theorem}\label{chap5-thm5.1}
Let $(K,\Sigma,P)$ and $(\hat{K},\hat{\Sigma},\hat{P})$ be affine
equivalent with $F_{K}$ as the {\em affine transformation. If $v:K\to
\mathbb{R}$ induces $\hat{v}:\hat{K}\to \mathbb{R}$ by
$\hat{v}(\hat{x})=v(x)$ for $\hat{x}\in\hat{K}$, $(x=F_{K}(\hat{x}))$,
then $\widehat{\pi v}=\hat{\pi}\hat{v}$.}
\end{theorem}

\begin{proof}
Let
$\hat{\Sigma}=\{\hat{\varphi}_{i}\}^{N}_{i=1},\Sigma=\{\varphi_{i}\}^{N}_{i=1}$. By
definition, 
\begin{align*}
& \hat{\varphi}_{i}(\widehat{\pi v})=\varphi_{i}(\pi
v)=\varphi_{i}(v),\ 1\leq i\leq N.
&
\hat{\varphi}_{i}(\hat{\pi}\hat{v})=\hat{\varphi}_{i}(\hat{v})=\varphi_{i}(v),\ 1\leq
i\leq N.
\end{align*}

Thus,
$\hat{\varphi}_{i}(\hat{\pi}\hat{v})=\hat{\varphi}_{i}(\widehat{\pi
  v})$ for $1\leq i\leq N$. Hence $\hat{\pi}\hat{v}=\widehat{\pi v}$
by uniqueness of the function\pageoriginale $\hat{\pi}\hat{v}$.
\end{proof}

Let us consider a polygonal domain $\Omega$ with a triangulation
$\mathfrak{t}_{h}$. Suppose to each $K\in \mathfrak{t}_{h}$ is
associated a finite element, $(K,\Sigma_{K},P_{K})$, $\Sigma_{K}$
being the set of degrees of freedom, and $P_{K}$ the finite
dimensional space such that $\Sigma_{K}$ is $P_{K}$-unisolvent. Then
we have defined the interpolation operator $\pi_{K}$. All these make
sense {\em locally} i.e.\@ at a particular finite element $K$. We now
define the {\em global} counterparts of these terms. The comparison is
given in the following table.
\begin{center}
{\bf Table 5.1.}
\bigskip

\renewcommand{\arraystretch}{1.2}
\begin{tabular}{rp{4.2cm}|rp{4.5cm}}
\hline
 & \multicolumn{1}{c|}{Local definition}  && \multicolumn{1}{c}{Global
definition}\\ 
\hline
1. & Finite element $K$. & 1. & The set
$\overline{\Omega}=\bigcup\limits_{K=\mathfrak{t}_{n}}\overline{K}$\\
2. & The boundary of $K$, $\p K$. & 2. & The boundary of $\Omega$, $\p
\Omega=\Gamma$.\\
3. & The space $P_{K}$ of functions $K\to \mathbb{R}$, which is
finite-dimensional. & 3. & The space $V_{h}$ of functions $\Omega\to
\mathbb{R}$, which is also finite-dimensional.\\
4. & The set $\sum_{K}=\{\varphi_{i,K}\}^{N}_{i=1}$ of degrees of
freedom of $K$. & 4. & The set of degrees of freedom
$\sum_{h}=\{\varphi_{i}\}^{N}_{i=1}$, where
$\varphi_{i}(p|K)=\varphi_{i,K}(p|K)$.\\
5. & Basis functions of $P_{K}$ are $\{p_{_{i,K}}\}^{N}_{i=1}$. & 5. &
Basis function of $V_{h}$ are $\{w_{j}\}$.\\
6. & The nodes of $K$ are
$\{a_{i}^{0},a^{1}_{i},a^{2}_{i},\ldots\}$. & 6. & The nodes of
$\mathfrak{t}_{h}$ are by definition,
$\bigcup\limits_{K\in\mathfrak{t}_{h}}\{\text{Nodes of } K\}=\cup
\{b_{j}\}$ say. \\
7. & $\pi_{K}$ is the $P_{K}$-interpolation operator, defined by
$\varphi_{i,K}(\pi_{K}v)=\varphi_{i,K}(v)$, for all
$\varphi_{i,K}\in\sum_{K}$. & 7. & The $V_{h}$ interpolation operator
$\pi_{h}$ is defined by $\pi_{h}v\in V_{h}$ such that,
$\varphi_{i,K}(\pi_{h}v|K)=\varphi_{i,K}(v|K)$ for all
$\varphi_{i,K}\in\sum_{K}$.\\
\hline 
\end{tabular}
\end{center}\pageoriginale 


Notice that, by definition,
\begin{equation*}
(\pi_{h}v)|K=\pi_{K}(v|K)\quad\text{for all}\quad
  K\in\mathfrak{t}_{h}.\tag{5.5}\label{chap5-eq5.5} 
\end{equation*}

{\em It is this property and the conclusion of theorem
  \ref{chap5-thm5.1} that will be essential in our future error
  analysis.}

\begin{definition}\label{chap5-defi5.5}
We say that {\em a finite element of a given type is of class
  $C^{0}$,} resp. {\em of class} $C^{1}$, if, whenever it is the
generic finite element of a triangulation, the associated space
$V_{h}$ satisfies the inclusion $V_{h}\subset
C^{0}(\overline{\Omega})$, resp.\@ $V_{h}\subset
C^{1}(\overline{\Omega})$. By extension, a {\em triangulation is of
  class} $C^{0}$, resp.\@ {\em of class} $C^{1}$ if it is made up of
finite elements of class $C^{0}$, resp.\@ of class $C^{1}$.
\end{definition}

\noindent
{\bf Reference:}~A forthcoming book of Ciarlet and Raviart
\cite{key5}. 
