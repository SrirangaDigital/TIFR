\part{Representation by squares}\label{part4}

\chapter{Lecture}\label{part4:lec33} %%%% 33
\markboth{\thechapter. Lecture}{\thechapter. Lecture}

We\pageoriginale wish to begin the study of the representation of a
number as the sum of squares:
$$
n= n^2_1+ n^2_2 + \cdots + n^2_r
$$

We shall develop in this connection the Hardy-Littlewood circle
method. Historically it is an off shoot of the Hardly-Ramanujan method
in partition-theory, though we did not develop the latter in its
original form in our treatment. The circle method has been applied to
very many cases, and the problem of squares is a very instructive one
for finding out the general thread. We shall later replace the problem
by that of the representation of $n$ by a positive quadratic
form. This would involve only the general Poisson summation
formula. In the case of  representation as the sum of squares there is
some simplification, because the generating function is the $r^{th}$
power of a simple $\mathscr{V}$ function. We shall deal with the asymptotic
theory. Later we may go into Siegel's theory of quadratic forms. 

Let us write
$$
\Theta (x) = \sum^\infty_{n=- \infty} x^{n^2} = 1 + 2
\sum^\infty_{n=1} x^{n^2},
$$
$|x|<1$. For $r$ at least equal to 4, we consider
\begin{align*}
  \Theta^r (x) & = \left(\sum^\infty_{n=- \infty}x^{n^2} \right)^r =
  \sum^\infty_{n_j=- \infty} x^{n_1^2+ n^2_2 + \cdot + n_r^2}\\
  & = \sum^\infty_{n=0} A_r (n) x^n,
\end{align*}
on\pageoriginale collecting the terms with exponent $n$, where $A_r(n)$ is
the number of times $n$ appears as the sum of $r$ square:
$$
A_r (n) = \sum_{n_1^2 + \cdots + n_r^2=n}1
$$

It is clear that $n_i$ can be positive or negative. The more serious
thing is that we have to count the representations differently when
the 
summands are interchanged, in contradiction to the situation in the case
of partitions. The problem of partition into squares would be a more
complicated problem; the generating function would be more complicated,
and what is worse, all the help one gets in partition theory from the
theory of modular forms would break down here.

$A_n(n)$ is the $n^{\text{th}}$ coefficient of a power-series;
$$
A_r(n) = \frac{1}{2 \pi i} \int_C \frac{\Theta^r (x)}{x^{n+1}} dx
$$
where $C$ is a suitable circle inside and close to the unit
circle. The trick of Hardy and Littlewood was to break the circle
$|x|=e^{- 2\pi \delta_N}$ where $N$ is the order of a certain Farey
dissection, into Farey arcs and write
$$
A_r (n) = \frac{1}{2 \pi i} \sideset{}{'}\sum_{o \leq h < k \leq N} \int_{\xi_{hk}}
\frac{\Theta^n(x)}{x^{n+1}} dx,
$$
where $\xi_{hk}$ are the arcs over which one integration piecemeal the
prime denoting that $(h, k)=1$. Consider on each piece $\xi_{hk}$\pageoriginale the
neighbourhood of a root of unity: 
$$
x= e^{2 \pi i \frac{h}{k}- 2 \pi \xi} 
$$
$\mathscr{R}_e \mathfrak{z}<0$, and set $\mathfrak{z}= \delta_N- i \varphi$,
so chat we have a little freedom along both real and imaginary axes.
$$
x= e^{2 \pi i \frac{h}{k} - 2 \pi \delta_N + 2 \pi i \varphi}.
$$

The choice of the little arc $\varphi$ is also
classical. $\frac{h}{k}$ is a certain Farey fraction, with adjacents
$\frac{h_1}{k_1}$ and $\frac{h_2}{k_2}$, say. $\frac{h_1}{k_1} <
\frac{h}{k} < \frac{h_2}{k_2}$. We limit $\varphi$ on the separate
  arcs. Introduce the mediants:
$$
\frac{h_1}{k_1} < \frac{h_1+h}{k_1 +k} < \frac{h}{k} <
\frac{h_2+h}{k_2+k} < \frac{h_2}{k_2},
$$ 

So that the interval $\left( \frac{h_1 +h}{k_1+k} , \frac{h_2+
  h}{k_2+2}\right)$ gives the movement of $\frac{h}{k} + \varphi$. So $\varphi$
runs between
\begin{align*}
  - \mathscr{V}'_{hk} & = \frac{h_1 +h}{k_1+ k} - \frac{h}{k} \leq
  \varphi \leq \frac{h_2+ h}{h_2+k}- \frac{h}{k}= \mathscr{V}''_{hk}\\
  - \mathscr{V}'_{hk} & = - \frac{1}{(k_1 + k)k}; \mathscr{V}''_{hk}=
  \frac{1}{(k_2 +k)k} ;
\end{align*}
and since $2N > \begin{smallmatrix} k_1+ k\\ k_2+k \end{smallmatrix}>
N$, we have necessarily
$$
\frac{1}{2 Nk} \leq |\mathscr{V}_{hk}| \leq \frac{1}{Nk}
$$

Now changing the variable of integration to $\varphi$, we can write
$$
A_r(n) = \sideset{}{'}\sum_{0 \leq h < k \leq N} e^{-2 \pi i \frac{h}{k} n}
\int\limits^{\mathscr{V}''_{hk}}_{-\mathscr{V}'_{hk}} \Theta^r
\left(e^{2 \pi i \frac{h}{k} - 2 \pi \mathfrak{z}} \right) e^{2 \pi n
  \mathfrak{z}} d \varphi
$$\pageoriginale

The trick is to overcome the difficulty in the integral by replacing
on each arc the highly transcendental function by a simpler
function. Here we stop for a moment to see what we can do with the
integrand.
\begin{align*}
  \Theta \cdot \left(e^{2 \pi i \frac{h}{k} - 2 \pi \mathfrak{z}}
  \right) & = \sum^\infty_{n=- \infty} e^{\left(2 \pi i \frac{h}{k} -
    2 \pi \mathfrak{z} \right)n^2}\\
  & = \sum^{k-1}_{j=0} e^{2 \pi i \frac{h}{k} j^2} \sum_{n \equiv j
    \pmod{k}} e^{- 2 \pi \mathfrak{z}n^2}\\
  & = \sum^{k-1}_{j=0} e^{2 \pi i \frac{h}{k} j^2}
  \sum^\infty_{q=-\infty} e^{2 \pi \mathfrak{z} k^2 (q + \frac{j}{k})^2},
\end{align*}
where we have written $n=k q + j$. We can now handle this from our
$\mathscr{V}$-series formula. We proved (Lecture \ref{part2:lec12}) that
$$
\displaylines{\frac{C(\tau)}{i} \mathscr{V}_3(\mathscr{V}/-
  \frac{1}{\tau}) = e^{\pi i \tau \mathscr{V}^2
    \mathscr{V}_3(\mathscr{V} \tau/\tau)}\cr
  \text{and} \hfill \frac{C(\tau)}{i} = \sqrt{\frac{i}{\tau}},
  Im \tau> 0. \hfill }
$$
Since\pageoriginale 
$$
\mathscr{V}_3 (\mathscr{V}/ \tau) = \sum^\infty_{n=- \infty} e^{\pi i
  n^2 \tau} e^{2 \pi i n \mathscr{V}},
$$
writing $\tau= i t$, $\mathscr{R}_e t> 0$, we have from the above,
\begin{align*}
  \frac{1}{\sqrt{t}} \sum^\infty_{n=- \infty} e^{- \pi \frac{n^2}{t}}
  e^{2 \pi i n \mathscr{V}}& = e^{- \pi t \mathscr{V}^2}
  \sum^\infty_{n=- \infty} e^{- \pi n^2 t} e^{- 2 \pi \mathscr{V}n
    t}\\
  & = \sum^\infty_{n=- \infty} e^{- \pi t (n+ \mathscr{V})^2}
\end{align*}

Replacing $n$ by $q$, $\mathscr{V}$ by $\frac{j}{k}$ and $t$ by $2
\mathfrak{z} k^2$, we have
\begin{align*}
  \Theta \left( e^{2 \pi i \frac{h}{k}- 2 \pi \mathfrak{z}}\right) & =
  \sum^{k-1}_{j=0} e^{2 \pi i \frac{h}{k} j^2} \frac{i}{\sqrt{2
      \mathfrak{z} k^2}} \sum^\infty_{q=- \infty} e^{- \frac{\pi
      q^2}{2 \mathfrak{z} k^2}} e^{2 \pi i q \frac{j}{k}}\hspace{2cm}\\
  & = \frac{1}{k \sqrt{2 \mathfrak{z}}} \sum^\infty_{q=- \infty} e^{-
    \frac{\pi q^2}{2 \mathfrak{z} k^2}} T_q (h, k)\\
  \text{where} \hspace{1cm}T_q (h, k) & = \sum^{k-1}_{j=0} e^{2 \pi i
    \frac{hj^2+ qj}{k}} 
\end{align*}

This is already a good reduction. $T_q (h, k)$ depends on $q$ modulo
$k$, so it is periodic. We shall approximate to it in general. 

One\pageoriginale special case, however, is of interest: for $q=0$,
$$
T_0(h, k)= \sum^{k-1}_{j=0} e^{2 \pi i \frac{h}{k} j^2}= G(h, k),
$$
where $G(h, k)$ are the so-called Gaussian sums which we shall study
in detail. They are sums of roots of unity raised to a square power,
$\Theta$ is actually a $\mathscr{V}_3$, and when we evaluate $T_q$ we
get some other $\mathscr{V}$.

We now write
$$
\displaylines{\Theta \left( e^{2 \pi i \frac{h}{k} - 2 \pi
    \mathfrak{z}}\right) = \frac{1}{k\sqrt{2 \mathfrak{z}}} \left\{G
  (h, k) + H(h, k; \mathfrak{z}) \right\}\cr
  \text{where} \hfill H (h, k; \mathfrak{z}) =
  \sum^\infty_{\substack{q=- \infty\\q\neq 0}} T_q (h, k) e^{-
    \frac{\pi q^2}{2 k^2 \mathfrak{z}}}\hfill }
$$

We shall throw $H$ into the error term. Let us appraise $T_q(h, k)$,
not explicitly: that will take us into Gaussian sums.
\begin{align*}
  T_q (h, k) & = \sum^{k-1}_{j=0} e^{2 \pi i \frac{h j^2 + qj}{k}}\\
  |T_q (h, k)|^2 &= \sum^{k-1}_{j=0} \sum^{k-1}_{\ell=0} e^{\frac{2
      \pi i}{k} (hj^2 + q j)} e^{- 2 \pi \frac{i}{k} (h\ell^2 + q
    \ell)}\\
  & = \sum_{j \mod k}~ \sum_{\ell \mod k} e^{2 \pi \frac{i}{k} (h
    (\ell j^2- \ell^2) + q(j-\ell))}\\
  & = \sum_{j \mod k}~\sum_{\ell \mod k} e^{2 \pi \frac{i}{k} (j-
    \ell)(h(j+\ell) + q)},
\end{align*}
which,\pageoriginale an rearranging according to the difference
$j-\ell$, becomes 
\begin{align*}
  & = \sum_{a \mod k} ~\sum_{j - \ell \equiv a \pmod{k}} e^{2 \pi i
    \frac{a}{k} (h (j+ \ell)+ q)}\\
  & = \sum_{a \mod k} ~\sum_{\ell \mod k} e^{ 2 \pi i \frac{a}{k} (h
    (a+ 2 \ell) + q)}\\
  & = \sum_{a \mod k} e^{2 \pi \frac{i}{k} (ha^2 + a q)} \sum_{\ell
    \mod k} e^{4 \pi i a \frac{h}{k}\ell}
\end{align*}

The inner sum is a sum of the roots of unity. Two cases arises,
according as $k\mid 2a$ or $k \nmid 2a$. $k$ odd implies that $a=0$ and $k$
even implies that $a=0$ or $k\mid 2$. In case $k\mid 2a$, the sum is zero. We
then have 
\begin{align*}
  |T_q (h, k)|^2 & = k, ~\text{if}~ k ~\text{is odd}; k\left( 1+ e^{2
    \pi \frac{i}{k} \left( h \frac{k^2}{4} +
    \frac{k}{2}q\right)}\right), ~\text{if}~ k ~\text{is even}\\
  & = k \left(1+ e^{\pi i \left(\frac{hk}{2} + q \right)}
  \right),\quad   ~\text{if}~ k ~\text{is even} \\
  & = o ~\text{or}~ 2k \qquad \text{if}~ k ~\text{is even}
\end{align*}

It is of interest to notice that $T_q=0$ only if $k$ is even and
$\frac{hk}{2}+ q$ is an odd integer. In any case,
$$
|T_q (h, k)| \leq \sqrt{2k},
$$
and this cannot be improved. We then have
\begin{align*}
  |H (h, k; k; \mathfrak{z})| & \leq 2 \sum^\infty_{q=1} \sqrt{2k} e^{-
    \frac{\pi q^2}{2k^2} \mathscr{R} \frac{1}{\mathfrak{z}}} (q=0
  ~\text{is not involved here}).\\
  & = 2 \sqrt{2k} e^{- \frac{\pi}{2 k^2} \mathscr{R}
    \frac{1}{\mathfrak{z}}} \sum^\infty_{q=1} e^{- \pi
    \frac{(q^2-1)}{2k^2} \mathscr{R} \frac{1}{\mathfrak{z}}}\\
    & = 2 \sqrt{2k} e^{- \frac{\pi}{2k^2} \mathscr{R}
    \frac{1}{\mathfrak{z}}} \sum^\infty_{m=0} e^{- \frac{3 \pi m}{2
      k^2} \mathscr{R} \frac{1}{\mathfrak{z}}}\\
  & = 2 \sqrt{2k} e^{- \frac{\pi}{2k^2} \mathscr{R}
    \frac{1}{\mathfrak{z}}} \frac{1}{1- e^{- \frac{3 \pi}{2k^2}
      \mathscr{R} \frac{1}{\mathfrak{z}}}}
\end{align*}

Since\pageoriginale $\mathfrak{z} = \delta_N - i \varphi$,
\begin{align*}
  \frac{1}{k^2} \mathscr{R} \frac{1}{\mathfrak{z}} & = \mathscr{R}
  \frac{1}{k^2 \mathfrak{z}} = \mathscr{R} \frac{1}{k^2 (\delta_N - i
    \varphi)} = \frac{\delta_N}{k^2 (\delta^2_N + \varphi^2)}\\
    \therefore \quad \frac{1}{k^2} \mathscr{R} \frac{1}{\mathfrak{z}}
    & \geq \frac{\delta_N}{k^2 \delta^2_N + \frac{1}{N^2}}, \quad
      \text{since}~ |\mathscr{V}_{hk}| \leq \frac{1}{kN},\\
      & \geq \frac{\delta_N}{N^2 \delta^2_N + \frac{1}{N^2}}=
      \frac{1}{N^2 \delta_N + \frac{1}{N^2 \delta_N}}
\end{align*}

We want to make this keep away from 0 as far as possible. This gives a
desirable choice of $\delta_N$. Make the denominator as small as
possible. Since $x+ \frac{1}{x}$ is minimised when $x=1$, we have 
$$
\frac{1}{k^2} \mathscr{R} \frac{1}{\mathfrak{z}} \geq \frac{1}{2}, 
$$
this minimum corresponding to $N^2 \delta_N =1$. So if we choose the
radius of the circle in terms of the Farey order, we shall have
secured the best that we can:
$$
|H (h, k; \mathfrak{z})| \leq 2 \sqrt{2k} e^{- \frac{\pi}{2k^2}
  \mathscr{R} \frac{1}{\mathfrak{z}}} \times C
$$

It would be unwise to appraise the remaining exponential now. 
