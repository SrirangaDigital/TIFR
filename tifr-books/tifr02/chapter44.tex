\chapter{Lecture}\label{part4:lec44} %%% 44
\markboth{\thechapter. Lecture}{\thechapter. Lecture}

We\pageoriginale had
$$
\displaylines{F_r (e^{2 \pi i \frac{h}{k} - 2 \pi \mathfrak{z}}) =
  \frac{1}{k^r (2 \mathfrak{z})^{r/2} D^{1/2}} \sum_{\underline{m}}
  e^{- \frac{\pi}{2 \mathfrak{z} k^2} \underline{m}' S^{-1}
    \underline{m}} T_k (h, \underline{m}), \cr
  \text{and} \hfill T_k (h, m) = \sum_{\ell \mod k} e^{\frac{2 \pi
      i}{k} (h \underline{\ell}' S \underline{\ell} + \underline{m}'
    \underline{\ell})}\hfill }
$$

The common denominator in $\underline{m}' S^{-1} \underline{m}$ will
be at most $D$, the determinant; define $k^*$ and $D_k$ by
$$
k D = k \cdot (k, D)\cdot D_k = k^* D_k, (D_k, k)=1,
$$
so that $D_k$ is $D$ stripped of all its common divisors with
$k$. Suppose first that $k$ is odd. Let $\rho$ be a solution of the
congruence
\begin{align*}
  4h D_k \rho & \equiv 1 \pmod{k^*}\\
  T_k (h, \underline{m}) & = \sum_{\underline{\ell} \mod k} e^{2 \pi i
  \frac{h}{k} (\ell' S\underline{\ell + 4 D_k \rho \underline{m}'
    \underline{\ell}})}\\
  & = \sum_{\underline{\ell} \mod k} e^{2 \pi i \frac{h}{k}
    (\underline{\ell}'+ 2 D_k \rho \underline{m}' S^{-1})
    S(\underline{\ell}+ 2 D_k \rho S^{-1} \underline{m})} e^{- (4 D_k
    ^2 \rho^2 \underline{m}' S^{-1} \underline{m}) 2 \pi i
    \frac{h}{k}}.\\
  & = e^{-2 \pi i \frac{h}{k} \cdot 4 D^2_k \rho^2 \underline{m}'
    S^{-1} \underline{m}} \sum_{\underline{\ell} \mod k} e^{2 \pi i
    \frac{h}{k} (\underline{\ell}' + 2 D_k \rho \underline{m}'S^{-1})
    S(\underline{\ell}+ 2 D_k \rho S^{-1}\underline{m})}\\
  & = e^{- 2\pi \frac{D_k \rho}{k}\underline{m}' S^{-1} \underline{m}} \mathscr{U}_k,
\end{align*}
say,\pageoriginale (using the definition of $\rho$), where
$\mathscr{U}_k= \mathscr{U}_k (h, \underline{m})$ is periodic in
$\underline{m}$ with period $(k, D)$; it is enough if we take this
period to be $D$ itself. So
$$
  F_r (e^{2 \pi i \frac{h}{k}- 2 \pi \mathfrak{z}}) = \frac{1}{k^r (2
    \mathfrak{z})^{r/2} D^{1/2}} \sum_{\underline{s} \mod D}
  \mathscr{U}_k (h, \underline{s})
  \sum_{\underline{m} \equiv \underline{s} \pmod{D}} e^{-
    \left(\frac{\pi}{2 \mathfrak{z} k^2} + 2 \pi i \frac{D_{k}\rho}{k}
    \right) \underline{m}' S^{-1} \underline{m}}
$$

This is a linear combination of finitely many $\mathscr{V}$-series of
the form 
$$
\sum_{\underline{m}\equiv \underline{s} \pmod{D}}
x^{\underline{m}'S^{-1} \underline{m}}
$$

The power series goes in powers of $\frac{x}{D}$ because $\frac{1}{D}$
remains silent inside. This is for $k$ odd.

For $k$ even, define. $\sigma$ by 
$$
h D_k \sigma \equiv 1 \pmod{4k^*}
$$
\begin{align*}
  T_k (h, m)& = \sum_{\underline{\ell} \mod k}e^{2 \pi i \frac{h}{4k}
    (4 \underline{\ell}' S \underline{\ell}+ 4 D_k \sigma
    \underline{m}' \underline{\ell})}\\
  & = e^{-2 \pi i \frac{h}{4k} D^2_k \sigma^2 \underline{m}'
    S^{-1}\underline{m}} \sum_{\underline{\ell} \mod k} e^{2 \pi i
    \frac{h}{4k} (2 \ell' + D_k \sigma \underline{m}' S^{-1})S (2
    \underline{\ell} + D_k \sigma S^{-1} \underline{m})}\\
  & = e^{- 2 \pi i \frac{D_k}{4k}\sigma \underline{m}' S^{-1}
    \underline{m}} \mathscr{U}_k (h, \underline{m}),
\end{align*}
where\pageoriginale $\mathscr{U}_k$ again has a certain periodicity; we can take the
period to be $2D$ and forget about the refinement. So
$$
F_r \left(e^{2\pi i \frac{h}{k} - 2 \pi \mathfrak{z}}\right) =
\frac{1}{k^r (2 \mathfrak{z})^{r/2} D^{1/2}} \sum_{\underline{s} \mod
  2D} \mathscr{U}_k (h, \underline{s}) \sum_{\underline{m} \equiv
  \underline{s} \pmod{2D}} e^{- \left(\frac{\pi}{2 \mathfrak{z} k^2} +
  \frac{2 \pi i}{4k} D_k \sigma\right) \underline{m}' S^{-1} \underline{m}} 
$$
which is again a linear combination of theta-series with coefficients
$\mathscr{U}_k$. Observe that $T_k$ and $\mathscr{U}_k$ differ only by
a purely imaginary quantity:
$$
|T_k (h, \underline{m})| = |\mathscr{U}_k (h, \underline{m})|,
$$ 
and for $\underline{m}=\underline{0}$, $T_k (h, \underline{0}) =
\mathscr{U}_k (h, \underline{0})$.

We shall use as essential only those theta-series which are congruent
to zero modulo $D$ or $2D$; and the rest will be thrown into the error
term. Only these corresponding to $\underline{o}$ have a constant
term. The general shape in both cases is 
$$
\sum_{\underline{s} \mod 2D} \mathscr{U}_k (h, \underline{s})=
\sum_{\underline{m}\equiv \underline{s}\pmod{2D}} x^{\underline{m}'
  S^{-1} \underline{m}} 
$$
