\chapter{Lecture}\label{part3:lec28} %% 28
\markboth{\thechapter. Lecture}{\thechapter. Lecture}

We\pageoriginale had
$$
  F_n (x) =- \frac{1}{4 ix} \cot h \pi N x \cot \frac{ \pi N
    x}{\mathfrak{z}}
    + \sum^{k-1}_{\mu=1} \frac{1}{x} \cdot \frac{e^{2 \pi \mu N
        x/k}}{1- e^{2 \pi N x}} \times \frac{e^{- 2 \pi i \mu^*
        Nx/k\mathfrak{z}}}{1- e^{- 2 \pi i N x}/\mathfrak{z}},
$$
$N= n+ \frac{1}{2}$, $n$ integer $>0$, $\mu^* \equiv h \mu \pmod{k}$
    and $1\leq \mu^* \leq k-1$. At the triple pole $x=0$ the residue
    from the first summand $=- \frac{1}{12i} \left( \mathfrak{z}-
    \frac{1}{\mathfrak{z}} \right)$- Let us find the residues from the
    more interesting pieces of the sum. The general term on the right
    has in the neighbourhood of $x=0$ the expansion
\begin{align*}
  \frac{1}{x} & \left\{1+ \frac{2 \pi \mu N x}{k} + \frac{(2 \pi \mu
    N_2/k)^2}{2!} + \cdots \right\}\\ 
  & \qquad \times \frac{-1}{2 \pi Nx} 
  \left\{
  1+ \frac{2 \pi N x}{2} + \frac{(2 \pi N x)^2}{6} + \cdots
  \right\}^{-1}\\
  & \qquad \times \left\{1- \frac{2 \pi i \mu^* Nx}{k\mathfrak{z}}- \frac{(2
    \pi \mu^* Nx/k\mathfrak{z})^2}{2} + \cdots  \right\}\\
  &\qquad \times
  \frac{1}{2 \pi i N x /\mathfrak{z}} \left\{ 1- \frac{2 \pi i
    Nx}{2\mathfrak{z}} - \frac{(2 \pi N x/\mathfrak{z})^2}{6} + \cdots
  \right\}^{-1}\\
  & = \frac{- \mathfrak{z}}{4 \pi^2 i N^2 x^3} \left\{1+ \frac{2 \pi \mu
    Nx}{k} + \frac{1}{2} \left( \frac{2 \pi \mu N x}{k}\right)^2-
  \cdots \right\}\\
  & \qquad \times \left\{ 1- \left(\frac{2 \pi N x}{2} + \frac{(2 \pi N
    x)^2}{6}+ \cdots \right)+ (\cdots)^2 + \cdots \right\}\\
  & \qquad \times \left\{1- \frac{2 \pi i \mu^* Nx}{k\mathfrak{z}} - \frac{1}{2}
  \left( \frac{2 \pi \mu^* N x}{k\mathfrak{z}}\right)^2 + \cdots
  \right\}\\
  & \qquad \qquad \times \left\{1+ \frac{2 \pi i N z}{\mathfrak{z} } +
  \frac{(2 \pi N^2/\mathfrak{z})^2}{2}+ \cdots \right\}
\end{align*}\pageoriginale

Fishing out the term in $\dfrac{1}{x}$, the residue at $x=0$ from this
summand becomes
\begin{align*}
  \frac{i \mathfrak{z}}{ 4\pi^2 N^2} & \left\{ \frac{1}{2} \left(
  \frac{2 \pi \mu N}{k}\right)^2 + \frac{1}{12} (2 \pi N)^2 -
  \frac{1}{2} \left( \frac{2 \pi \mu^* N}{k
    \mathfrak{z}}\right)^2 - \frac{1}{12} \left( \frac{2 \pi
    N}{\mathfrak{z}}\right)^2  - 2 \pi
  \mu \frac{N}{k} \pi N \right.\\ 
  & \hspace{1.4cm}\left.- 4 \pi^2 \mu \mu^* \frac{N^2 i}{k^2
    \mathfrak{z}} + \frac{2 \pi ^2i \mu N^2}{k \mathfrak{z}} + \frac{2
    \pi^2 i \mu^* 
    N^2}{k \mathfrak{z}} - \frac{\pi^2 i N^2}{\mathfrak{z}}+ \frac{2 \pi^2
  \mu^* N^2}{k \mathfrak{z}^2}\right\}\\
  = \frac{i\mathfrak{z}}{4} &\left\{ \frac{2}{\mu^2}{k^2} +
 \frac{1}{3} - \frac{2 \mu}{k}\right\}+ \frac{i}{4 \mathfrak{z}}
 \left\{\frac{- 2 \mu^{*^2}}{k^2}- \frac{1}{3} + \frac{2 \mu^*}{k}
 \right\}\\
 & \hspace{3.8cm}+ \frac{i}{4} \left\{ - \frac{4 i \mu \mu^*}{k^2} +
 \frac{2 i \mu}{k} + \frac{2 i \mu^*}{k} - i\right\}\\
 = i \mathfrak{z} & \left\{ \frac{\mu^2}{2 k^2} - \frac{\mu}{2k} +
 \frac{1}{12} \right\}+ \frac{1}{i\mathfrak{z}} \left\{
 \frac{\mu^{*^2}}{2k^2} - \frac{\mu^*}{2k} + \frac{1}{12}\right\}+
 \left( \frac{\mu}{k} - \frac{1}{2} \right) \left( \frac{\mu^*}{k} - 
 \frac{1}{2}\right) \tag{*}\label{part3:lec28:eq*}
\end{align*}

We\pageoriginale have to sum this up from $\mu =1$ to $\mu =
k-1$. Let us prepare a few things.

Let us remark that 
$$
\sum^{k-1}_{\mu=1} \mu = \frac{(k-1)k}{2}; \quad \sum^{k-1}_{\mu=1}
\mu^2 = \frac{(k-1)k (2k-1)}{6}
$$

Also if $\mu$ runs through a full system of residues, so would $\mu^*$
because $(h, k)=1$. Further $0 < \frac{\mu^*}{k}<1$, and
$\frac{\mu^*}{k}$ and $\frac{h \mu}{k}$ differ only by an integer, so
that $\frac{\mu^*}{k} = \frac{h \mu}{k} - \left[ \frac{h
    \mu}{k}\right]$. Hence summing up the last expression $(*)$ from
$\mu=1$ to $\mu=k-1$, we have
\begin{multline*}
  i \mathfrak{z}  \left\{ \frac{(k-1)(2k-1)}{12k} - \frac{k-1}{4} +
  \frac{k-1}{12} \right\}\\ 
  + \frac{1}{i \mathfrak{z}} \left\{
  \frac{(k-1)(2k-1)}{12k} - \frac{k-1}{4} + \frac{k-1}{12} \right\}
  + \sum^{k-1}_{\mu=1} \left( \frac{\mu}{k}- \frac{1}{2}\right)
  \left( \frac{h \mu}{k} - \left[ \frac{h \mu}{k}\right]-
  \frac{1}{2}\right)\\ 
  = (k-1) \left( \frac{2k-1}{12k} -
  \frac{1}{6}\right) \left(i \mathfrak{z} + \frac{1}{i \mathfrak{z}}
  \right) + s(h, k)
\end{multline*}
where $s(h, k)$ stands for the arithmetical sum 
$$
\sum^{k-1}_{\mu=1} \left( \frac{\mu}{k} - \frac{1}{2}\right) \left(
\frac{h \mu}{k} - \left[ \frac{h \mu}{k}\right]- \frac{1}{2}\right)
$$
which appears here very simply as a sum of residues. The last expression becomes
$$
- \frac{k-1}{12k} \left( i \mathfrak{z} + \frac{1}{i
  \mathfrak{z}}\right) + s (h, k)
$$

So\pageoriginale the total residue at $x=0$ is 
$$
\frac{1}{12} \left( i \mathfrak{z} + \frac{1}{i \mathfrak{z}}\right) -
\frac{k-1}{12k} \left( i \mathfrak{z} + \frac{1}{i\mathfrak{z}}\right) + s(h,
k) = \frac{1}{12k} \left( i\mathfrak{z}
+\frac{1}{i\mathfrak{z}}\right) + s(h, k)
$$

Next, we consider the simple poles of $F_n (x)$ at the points $x=
\frac{i r}{N} (r \neq 0)$. The $\coth$ factor is periodic and so the
residue at any of these poles is the same as that at the origin, which
is $\frac{1}{\pi}$. Hence  the residue of $F_n (x)$ at $x= \frac{i
  r}{N} (r \neq 0)$ becomes 
$$
\frac{N}{4r} \cdot \frac{1}{ \pi N} \cot \frac{\pi i r}{\mathfrak{z}}
+ \sum^{k-1}_{\mu=1}  \frac{N}{i r} \frac{-1}{2 \pi N} e^{2 \pi i \mu
\frac{r}{k}} \frac{e^{2 \pi \mu^* r/k\mathfrak{z}}}{1- e^{2 \pi r/\mathfrak{z}}}
$$

(There is a very interesting juxtaposition of an arithmetical term and
a function theoretic term in the last part; this gets reversed for the
next set of poles)
$$
= \frac{1}{4 \pi i r} \coth \frac{\pi r}{\mathfrak{z}} - \frac{1}{2
  \pi i} \sum^{k-1}_{\mu=1} \frac{1}{r} e^{2 \pi i \frac{\mu r}{k}}
\frac{e^{2 \pi \mu^* r/k\mathfrak{z}}}{1-e^{2 \pi r/\mathfrak{z}}}
$$
$x$ remains between $\pm i$ on the imaginary axis. So $\left|
\frac{r}{N}\right|< 1$; so we need consider only $r= \pm 1$, $\pm 2,
\ldots, \pm n$. Again,
\begin{align*}
  \coth ~y & = \frac{e^y+ e^{-y}}{e^y - e^{-y}} = 1 +
  \frac{2e^{-y}}{e^y - e^{-y}}\\
  & = 1+ \frac{2 e^{-2y}}{1-e^{-2y}}
\end{align*}
$\coth~ y$\pageoriginale is an odd function so that $\frac{1}{y}\cot
h ~y$ is even. Hence summing up over all the poles corresponding to
$r= \pm 1, \ldots, \pm n$, we get the sum of the residues
\begin{multline*}
  = \frac{1}{2 \pi i} \sum^{n}_{r=1} \frac{1}{r} \left\{1 + \frac{2
    e^{-2 \pi r/\mathfrak{z}}}{1- e^{- 2 \pi r/\mathfrak{z}}}
  \right\}+ \frac{1}{2 \pi i} \sum^{k-1}_{\mu^* =1} \sum^n_{r=1}
  \frac{1}{r} e^{2 \pi i h' \mu^* r/k} \frac{e^{-2 \pi \mu^*
      r/k\mathfrak{z}}}{1- e^{-2 \pi r/\mathfrak{z}}}\\
  - \frac{1}{2 \pi i} \sum^{k-1}_{\mu^* =1} \sum^n_{r=1} \frac{1}{r}
  e^{2 \pi i h' (k-\mu^*) r/k} \frac{e^{2 \pi \mu^*
      r/k\mathfrak{z}}}{1- e^{2 \pi r/\mathfrak{z}}},
\end{multline*}
where we have made use of the fact that $hh'\equiv -1 \pmod{k}$, so
$h' \mu^* \equiv hh' \mu\equiv - \mu \pmod{k}$, or $\mu \equiv - h'
\mu^* \pmod{k}$. In the last sum replace $\mu^*$ by $k- \mu^*$; then
the previous sum is duplicated and we get
\begin{align*}
  \frac{1}{2 \pi i} \sum^n_{r=1} \frac{1}{r} & \left\{1+ \frac{2 e^{-2
      \pi r/\mathfrak{z}}}{1- e^{-2 \pi r/\mathfrak{z}}}+ \frac{1}{\pi
    i} \sum^{k-1}_{\mu^*=1} \sum^n_{r=1} \frac{1}{r} e^{2 \pi i h'
    \mu^* r/k} \frac{e^{- 2 \pi \mu^* r/k\mathfrak{z}}}{1- e^{-2 \pi
      r/\mathfrak{z}}} \right\}\\
  & = \frac{1}{2 \pi i} \sum^n_{r=1} \frac{1}{r} + \frac{1}{\pi i}
  \sum^k_{\nu=1} \sum^n_{r=1} \frac{1}{r} e^{2 \pi i h' \nu r/k}
  \frac{e^{- 2 \pi \nu n/k\mathfrak{z}}}{1- e^{-2 \pi r/\mathfrak{z}}}
\end{align*}

This\pageoriginale accounts for all the poles on the imaginary axis
(except the origin which has been considered separately before).

Finally we have poles $x= \frac{rz}{N} (e \neq 0)$ on the other
diagonal of the parallelogram. The same calculation goes through
verbatim and we get the sum of the residues at these poles to be 
$$
\frac{i}{2 \pi} \sum^n_{r=1} \frac{1}{r} + \frac{i}{\pi}
\sum^k_{\nu=1} \sum^n_{r=1} \frac{1}{r} e^{2 \pi i h \nu r/k}
\frac{e^{-2 \pi \nu r \mathfrak{z}/k}}{1- e^{- 2 \pi r \mathfrak{z}}}
$$
