\chapter{Lecture}\label{part1:lec8}
\markboth{\thechapter. Lecture}{\thechapter. Lecture}

Last\pageoriginale time we obtained the two fundamental formulae for
$\beta_{2l}$, $\beta_{2l+1}$, from which we deduced the recurrence relations:
\begin{equation*}
  \begin{aligned}
    \beta_{2m+1} & = x^{2m+1} (1-x^{2(2m+1)}) \beta_{2m},\\
    \beta_{2m+2} & = x^{2m+1} \frac{1-x^{2(2m+2)}}{1-x^{2(m+1)}} \beta_{2m+1}
  \end{aligned}\tag{1}\label{part1:lec8:eq1}
\end{equation*}
$\beta_{2m}$ came from $B_{2m}$ by a substitution which was not
yet plausible. Let us calculate the first few $\beta's$ explicitly. By
definition
\begin{alignat*}{4}
 B_0 &=1 = \beta_0&\quad &\\
  \beta_1 & = x(1-x^2) && \beta_0  = x(1-x^2)\\
  \beta_2 & = x \frac{1-x^4}{1-x^2} && \beta_1  = x^2 (1-x^4)\\
  \beta_3 & = x^3 (1-x^6) && \beta_2  = x^5 (1-x^4) (1-x^6)\\
  \beta_4 & = x^3 \frac{1-x^8}{1-x^4} && \beta_3  = x^8 (1-x^6)(1-x^8);
\end{alignat*}
and in general,
\begin{align*}
  \beta_{2m} & = x^{2m^2} (1-x^{2m+2}) (1-x^{2m+4}) \cdots
  (1-x^{4m})\\
  & = X^{m^2} \frac{X_{2m}!}{X_m!} (\text{with}~ X=x^2);
  \tag{2}\label{part1:lec8:eq2} 
\end{align*}
and similarly,
\begin{align*}
  \beta_{2m+1} & = x^{2m^2+2m+1} (1-x^{2m+2}) (1-x^{2m+4}) \cdots
  (1-x^{4m+2})\\
  & = X^{m^2+m} x\cdot  \frac{X_{2m+1}!}{X_m!}  \tag{3}\label{part1:lec8:eq3}
\end{align*}

This\pageoriginale is a very appealing result. We got the $\beta's$
in the attempt of ours to utilise the Jacobi formula. We actually had
$$
\frac{\sum\limits^\infty_{l=0} (-)^{l} x^{5l^2
    -l}}{\prod\limits^\infty_{m=1} (1-x^{2m})} = \sum^\infty_{m=0}
\frac{\beta_{2m}}{X_{2m}!}, 
$$
so that by (\ref{part1:lec8:eq2})
\begin{equation*}
  \frac{\sum\limits^\infty_{l=0} (-)^{l} X^{l(5l
      -l)/2}}{\prod\limits^\infty_{m=1} (1-x^{m})} = \sum^\infty_{m=0}
  \frac{X^{m^2}}{X_m!} \tag{4}\label{part1:lec8:eq4}
\end{equation*}

Similarly we had 
\begin{equation*}
  \frac{\sum\limits^\infty_{l=0} (-)^{l} x^{5l^2
      +3l+1}}{\prod\limits^\infty_{m=1} (1-x^{2m})} = \sum^\infty_{m=0}
  \frac{\beta_{2m+1}}{X_{2m+1}!}, 
\end{equation*}
so that by (\ref{part1:lec8:eq3})
\begin{equation*}
  \frac{\sum\limits^\infty_{l=0} (-)^{l} X^{l(5l
      +3)/2}}{\prod\limits^\infty_{m=1} (1-x^{m})} = \sum^\infty_{m=0}
  \frac{X^{m(m+1)}}{X_m!} \tag{5}\label{part1:lec8:eq5}
\end{equation*}

Now the right side in the Rogers-Ramanujan formula is 
$$
\frac{1}{\prod\limits^\infty_{m=1} (1-x^{5m-1})(1-x^{5m-4})}=
\frac{\prod\limits^\infty_{m=1}
  (1-x^{5m})(1-x^{5m-2})(1-x^{5m-3})}{\prod\limits^\infty_{m=1} (1-x^m)}
$$
which\pageoriginale becomes, on replacing $x$ by $x^2$,
$$
\frac{\prod\limits^\infty_{m=1} (1-x^{10m})(1-x^{10m-4})(1-x^{10m-6})}
{\prod\limits^\infty_{m=1}(1-x^{2m})}
$$

The numerator is the same as the left side of Jacobi's triple product
formula: 
$$
\prod^\infty_{m=1}
(1-x^{2m})(1-\mathfrak{z}x^{2m-1})(1-\mathfrak{z}^{-1}x^{2m-1})=
\sum^\infty_{l=-\infty} (-)^l \mathfrak{z}^l x^{l^2},
$$
with $x$ replaced by $x^5$ and $z$ by $x$. Hence
\begin{align*}
  \frac{\prod\limits^\infty_{l=-\infty} (1-x^{10mm})
    (1-x^{10m-4})(1-x^{10m-6})}{\prod\limits^\infty_{m=1} (1-x^{2m})} =
  \frac{\sum\limits^\infty_{l=-\infty}(-)^l X^{5l^2
      +l}}{\prod\limits^\infty_{m=1} (1-x^{2m})} =
  \frac{\sum\limits^\infty_{l=-\infty} (-)^l X^{(5l^2 +
      l)/2}}{\prod\limits^\infty_{m=1} (1-X^m)}  
\end{align*}
now
\begin{align*}
  \frac{\sum\limits^\infty_{l=-\infty} (-)^l x^{5l^2 +
      l}}{\prod\limits^\infty_{m=1} (1-x^{2m})}& = 
  \frac{\sum\limits^\infty_{k=-\infty} (-)^k \mathfrak{z}^k x^{k^2}}
      {\prod\limits^\infty_{m=1} (1-x^{2m})}
       =\sum^\infty_{n=0} \frac{B_n(\mathfrak{z}, x)}{X_n!}
       = \frac{\sum\limits^\infty_{l=-\infty} (-)^l x^{l^2+l}
        x^{(2l)^2}}{\prod\limits^\infty_{m=1}(1-x^{2m})},  
\end{align*}
on\pageoriginale replacing $\mathfrak{z}^{2l}$ by $(-)^l x^{l(l+1)}$, and this we
can do because of linearity. Hence
$$
\frac{\sum\limits^\infty_{l=-\infty} (-)^l X^{l
    (5l-1)/2}}{\prod\limits^\infty_{m=1} (1-X^m)}=
\frac{1}{\prod\limits^\infty_{m=1}(1-x^{5m-1}) (1-x^{5m-4})}  
$$

Similarly,
\begin{align*}
  \frac{1}{\prod\limits^\infty_{m=1} (1-X^{5m-2})(1-X^{5m-3})} & =
  \frac{\prod\limits^\infty_{m=1}(1-x^{10m})
    (1-x^{10m-2})(1-x^{10m-8})}{\prod\limits^\infty_{m=1} (1-X^m)}\\
    & = \frac{\sum\limits^\infty_{l=-\infty}(-)^l
      x^{5l^2+3l}}{\prod\limits^\infty_{m=1}(1-X^m)}. 
\end{align*}

This time we have to replace $\mathfrak{z}^{2k+1}$ by
$(-)^{k}x^{k(k-1)}$. Then
$$
  \frac{1}{\prod\limits^\infty_{m=1} (1-X^{5m-2})(1-X^{5m-3})} =
  \frac{\sum\limits^\infty_{l=-\infty}(-)^l X^{l(5l+3)/2}}
    {\prod\limits^\infty_{m=1} (1-X^m)}
$$

These formulae are of extreme beauty. The present proof has at least
to do with things that we had already handled. The pleasant surprise
is that these things do come out. The other proofs by Watson,
Ramanujan\pageoriginale and other use completely unplausible
combinations from the very start. Our proof is substantially that by
Rogers given in Hardy's Ramanujan, pp.96-98, though one may not
recognize it as such. The proof there contains completely foreign
elements, trigonometric functions which are altogether irrelevant
here.

We now give up formal power series and enter into an entirely
different chapter - Analysis.
