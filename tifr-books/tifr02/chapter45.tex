\chapter{Lecture}\label{part4:lec45} %% 45
\markboth{\thechapter. Lecture}{\thechapter. Lecture}

We\pageoriginale have to get a clear picture of that we are aiming
at. We are discussing the function under the integral sign. We get it as
\begin{multline*}
  F_r \left(e^{2 \pi i \frac{h}{k} - 2 \pi \mathfrak{z}} \right) =
  \frac{1}{k^r (2\mathfrak{z})^{r/2} D^{1/2}} \sum_{\underline{s} \mod
  2D} \mathscr{U}_k (h, \underline{s})\\
  \sum_{\underline{m}\equiv \underline{s} \pmod{2D}} e^{-
    \left(\frac{\pi}{2k^2 \mathfrak{z}} + 2 \pi i \frac{D_k
      \sigma}{4k} \right)}  {m}' S^{-1} {m}
\end{multline*}
where $k \cdot D = k^* D_k, (k, D_k) =1$. $k$ is even; if $k$ is odd
the formula looks finitely many different values. This most important
fact we formulate as a lemma.

\begin{lem}\label{part4:lec45:lem1}
  For $k$ even, $\mathscr{U}_k (h, \underline{s})$ depends only on $h$
  modulo 2D.
\end{lem}

This depends on a theorem on the behaviour of quadratic forms the
equivalence of quadratic forms modulo a given number. This is a lemma
of Siegel's (Annals of Mathematics, 1935, 527-606).

Let us recall that for $k$ even
\begin{align*}
  T_k(h, m)& = \sum_{\underline{\ell} \mod k} e^{2 \pi \frac{i}{k} (h
    \el' S\el +
    \underline{m}'\el)}\\
  & = e^{- 2 \pi i\frac{h}{4k}  D_k \sigma \underline{m}' S^{-1}
    \underline{m}} \mathscr{U}_k (h, \underline{m})
\end{align*}

\begin{lem}\label{part4:lec45:lem2}
  $$
  |T_k (h, \underline{m})| \leq C k^{r/2}
  $$
  We have 
  \begin{align*}
    |T_k (h, \underline{m})|^2 & = \sum_{\ell \mod k} e^{2 \pi i
      \frac{h}{k} (\ell S \ell + \sigma
      \underline{m}' \underline{\ell})} \sum_{\lambda \mod k} e^{- 2
      \pi i \frac{h}{k} (\lm ' S\lm +
      \sigma \underline{m}' \el)}\\
    & = \sum_{\el, \lm } e^{2 \pi i
      \frac{h}{k} (\el' S \el-
      \lm ' S \lm  + \sigma
      \underline{m}' (\el- \lm ))},
  \end{align*}
  and\pageoriginale since
  \begin{align*}
    \el' S \el - \lm' S \lm & = (\el' - \lm') S (\el + \lm) + \lm' S
    \el - \el' S \lm\\
    & = (\el' - \lm') S (\el + \lm) + \el' S \el - \el' S \lm\\
    & = (\el' - \lm') S (\el + \lm),
  \end{align*}
  this is equal to 
  \begin{align*}
    \sum_{\el, \lm} & e^{2 \pi i\frac{h}{k} (\el' - \lm') (S(\el + \lm)+
      \sigma \underline{m})}\\
    & = \sum_{\underline{\alpha} \mod k} ~\sum_{\el - \lm \equiv
      \underline{\alpha} \mod k} e^{2 \pi i \frac{h}{k}
      \underline{\alpha}' (S (\el + \lm) + \sigma \underline{m})}\\
    & = \sum_{\underline{\alpha} \mod k} \sum_{\el
      -\underline{\lambda} \equiv  \mod k} e^{2 \pi i \frac{h}{k}
      \underline{\alpha}' (S (2\lm + \underline{\alpha}) + \sigma
      \underline{m})}\\ 
    & = \sum_{\underline{\alpha} \mod k} e^{2 \pi i \frac{h}{k}
      \underline{\alpha}' (S \underline{\alpha}+ \sigma
      \underline{m})} \sum_{\lm \mod k} e^{2 \pi \frac{h}{k} 2
      \underline{\alpha}' S \lm}
  \end{align*}
\end{lem}

If\pageoriginale we write $2 \underline{\alpha}' S =
\underline{\beta}'$, the inner sum is 
\begin{align*}
\sum_{\lambda_1, \ldots \lambda_r \mod k} e^{2 \pi i \frac{h}{k}
  (\beta_1 \lambda_1 + \cdots + \beta_r \lambda_r)} & = k^2, ~\text{if}~
k\mid \beta_1, \ldots , k\mid\beta_r;\\ 
&\quad  0 ~\text{otherwise}
\end{align*}

So $|T_k (h, \underline{m})|^2=0$ if at least one $\beta$ is not
divisible by $k$; otherwise it is equal to 
$$
k^r \sum_{\underline{\alpha} \mod k} e^{2 \pi i \frac{h}{k}
  \underline{\alpha}' (S \underline{\alpha}+ \sigma \underline{m})}
$$

Writing $S= (s_{j k})$, the system of congruences
\begin{align*}
  2 \alpha_1 s_{11} + 2 \alpha_2 s_{21} + \cdots + 2 \alpha_r s_{r1} &
  \equiv 0 \pmod{k}\\
  \cdot\quad \cdot \quad \cdot \quad \cdot \quad \cdot &\\
  \cdot\quad \cdot \quad \cdot \quad \cdot \quad \cdot &\\
  2 \alpha_1 s_{1r} + 2 \alpha_2 s_{2 r} + \cdots + 2 \alpha_r s_{rr} &
  \equiv 0 \pmod{k} 
\end{align*}
has at most $2^r |S|^r$ solutions, and thus
$$
\displaylines{|T_k (h, \underline{m})|^2 \leq 2^r |S|^r k^r,\cr
  \text{i.e.,} \hfill |T_k (h, \underline{m})| \leq 2^{r/2} |S|^{r/2}
  k^{r/2}.\hfill} 
$$

We\pageoriginale shall now outline the main argument a little more
skillfully, putting the thing back on its track. $A_r (n)$ is the sum
of integrals over the finer-prepared Farey arcs of Kloosterman:
\begin{multline*}
  A_r (n) = \frac{1}{2^{r/2} D^{1/2}} \mathop{\textstyle{\sum'}}_{0
    \leq h < k \leq N} e^{- 2 \pi i \frac{h}{k} n}
  \int\limits^{\frac{1}{k(k+N)}}_{- \frac{1}{k(k+N)}} \frac{e^{2 \pi n
    \mathfrak{z}}}{\mathfrak{z}^{r/2}} \\
  \sum_{\underline{s} \mod 2D} \mathscr{U}_k (h, \underline{s})
  \Theta_s \left(e^{- \frac{\pi}{2k^2 \mathfrak{z}}+ \frac{2 \pi i}{4k} D_k
    \sigma}\right) d \varphi  + \frac{1}{2^{r/2} D^{1/2}}\\ 
  \sum_{h, k} e^{- 2 \pi i \frac{h}{k} n } \sum^{N-1}_{\ell =k_2}
  \int\limits^{\frac{1}{k(k+\ell)}}_{\frac{1}{k(k+ \ell+ 1)}} +
  \frac{1}{2^{r/2} D^{1/2}} \sum_{h, k} e^{- 2 \pi i \frac{h}{k} n}
    \sum^{N-1}_{\ell= k_1} \int\limits^{- \frac{1}{k(k+\ell+1)}}_{-
      \frac{1}{k(k+\ell)}}\cdots, 
\end{multline*}
where 
\begin{align*}
  \Theta_s (x) & = \sum_{\underline{m} \equiv \underline{s} \pmod{2D}}
  x^{\underline{m}' S^{-1} \underline{m}};\\
  & = S_0+ S_2 + S_1, ~\text{say},\\
  & = \left(S_{00} + \sum_{\underline{m}\neq 0} S_{0m}\right) +
  \left(S_{20} + \sum_{\underline{m}\neq 0} S_{2\underline{m}} \right)
  + \left(S_{10} + \sum_{\underline{m} \neq 0} S_{1\underline{m}} \right)
\end{align*}
in an obvious notation. Now treat the things separately. By inspection
of $\mathscr{U}_k (h, \underline{m})$ we find how it depends on $h$, it
is only modulo 4 $k^*$. We have to reconcile Lemma \ref{part4:lec45:lem1} with this. This
actual period therefore is neither $2D$ nor $4 k^*$ but the greatest
common divisor
\begin{align*}
  (2D, 4 k^*) & = 2(D, 2k^*) = 2 \left(d, \frac{kD}{D_k}\right)\\
  & = \frac{2}{D_k} (DD_k, 2 kD) = \frac{2D}{D_k} (D_k, 2k)\\
  & = \frac{2D}{D_k} ~\text{or}~ \frac{4D}{D_k}
\end{align*}

So\pageoriginale we have

\noindent \textbf{Corollary of Lemma 1.} $\mathscr{U}_k (h, m)$ for
$k$ even depends on $h$ only modulo $\frac{2D}{D_k}= \wedge$, say.
$$
S_{00} = \frac{1}{2^{r/2} D^{1/2}} \sum_{0 \leq h < k \leq N} e^{-2
  \pi i \frac{h}{k} n} T_k (h, \underline{o})
\int\limits^{\frac{1}{l(k+N)}}_{- \frac{1}{k(k+N)}} \frac{e^{2 \pi n
    \mathfrak{z}}}{\mathfrak{z}^{r/2}} d \varphi 
$$

This goes into the principal term. We shall make it a little more
explicit later. 
\begin{align*}
S_{o \underline{m}} & = \frac{1}{2^{r/2} D^{1/2}}
  \mathop{\textstyle{\sum'}}_{0 \leq h < k\leq N} e^{- 2 \pi i
    \frac{h}{k} n} \mathscr{U}_k (h, \underline{m}) e^{2 \pi i
    \frac{D_k}{4k} \sigma \underline{m}' S^{-1} \underline{m}}
  \int\limits^{\frac{1}{k(k+N)}}_{- \frac{1}{k(k+N)}}
  \frac{e^{-\frac{\pi}{2k^2 \mathfrak{z}}} \underline{m}' S^{-1}
    \underline{m}}{\mathfrak{z}^{r/2}} d \varphi\\
  & = \frac{1}{2^{r/2} D^{1/2}} \sum^N_{k=1} K_k (n, \underline{m})
  \int\limits^{\frac{1}{k(k+N)}}_{- \frac{1}{k(k+N)}} \cdots, 
\end{align*}
where\pageoriginale
\begin{align*}
  K_k (n, \underline{m}) & = \mathop{\textstyle{\sum'}}_{h \mod k} e^{- 2
  \pi i \frac{h}{k}} \mathscr{U}_k (h, \underline{m}) e^{2 \pi
    \frac{i}{4k}D_k \sigma \underline{m}' S^{-1} \underline{m}}\\
  & = \frac{1}{a} \sum_{\lambda \mod \wedge} \mathscr{U}_k (\lambda,
  \underline{m}) \sum_{\substack{h \equiv \lambda \pmod{\wedge}\\ h
      \mod 4k^*}} e^{-\frac{2 \pi i a h n .4 + 2 \pi i \mathscr{V} \sigma}{4k^*}},
\end{align*}
where $4k^*=ak$, $a \leq 4D$, and $\mathscr{V}= \frac{k^*}{k} D_k
\underline{m}^{-1} S^{-1} \underline{m}$

We defined $\sigma$ by
$$
h D_k \sigma \equiv 1 \pmod{4k^*}
$$ 

Let 
$$
D \bar{D}_k \equiv 1 \pmod{4k^*}, h \bar{H} \equiv 1 \pmod{4k^*}
$$

Then 
\begin{align*}
  K_k (n, \underline{m}) & = \frac{1}{a} \sum_{\lambda \mod \wedge}
  \mathscr{U}_k (\lambda, \underline{m}) \sum_{\substack{h \equiv
      \lambda \pmod{\wedge}\\h \mod k^*, (h, k^*) =1}} e^{\frac{- 2
      \pi i a hn + 2 \pi i \mathscr{V} \sigma}{4k^*}} D_k \bar{D}k\\
  & = \frac{1}{a} \sum_{\lambda \mod \wedge} \mathscr{U}_k (\lambda,
  \underline{m}) \mathop{\textstyle{\sum'}}_{\substack{h \equiv
      \lambda \pmod{\wedge}\\h \mod k^*}} e^{\frac{2 \pi
      i}{4k}\left(-4a nh + \mathscr{V} \bar{D}_k \bar{h} \right)}
\end{align*}

The\pageoriginale inner sum here is a Kloosterman sum. It has
essentially $4k^*$ terms. A trivial estimate of this would be
$O(k)$, and this is what we had tacitly assumed in the older
method. The advantage here is, however, that they can be appraised
letter. We shall not estimate them here but only quote the result as

\begin{lem}\label{part4:lec45:lem3}
  \begin{align*}
    K_k (u, \nu) & = \sum_{\substack{h \equiv \lambda \pmod{\wedge}\\h
    \mod k}} e^{2 \pi \frac{i}{k} (u h+ \nu \bar{h})}, \wedge\mid k, h
    \bar{h} \equiv 1 \pmod{k}\\
    & = O \left(k^{1- \alpha + \epsilon}(u, k)^\alpha \right)
  \end{align*}
\end{lem}

There has been a lot of discussion about the size of the $\alpha$ in
this formula. Kloosterman and Esternann proved that $\alpha =
\frac{1}{4}$ (Hamb, Ab. 1930), Salie' that $\alpha=1/3$ and A.Weil
that $\alpha = \frac{1}{2}$ (P.N.A.S', 48) Weil's was a very
complicated and deep method going into the zeta-functions of Artin
type and the Riemann hypothesis for these functions.

We thus save a good deal in the order of magnitude. The further S's
will be nearly similar; the complete Kloosterman sums will be replaced
by sums with certain conditions.
$$
|S_{o \underline{m}}| \leq C \sum^N_{k=1} k^{1- \alpha +
\epsilon}  \frac{(n, k)^\alpha}{k^{r/2}} e^{- \frac{\pi}{4}
  \left(\underline{m}'S^{-1} \underline{m}- \frac{1}{D}\right)}
\int\limits^{\frac{1}{k(k+N)}}_{-\frac{1}{k(k+N)}}
\frac{e^{-\mathscr{R} \left(\frac{\pi}{2k^2 \mathfrak{z}}\cdot
    \frac{1}{D} \right)}}{|\mathfrak{z}|^{r/2}} d \varphi 
$$

Since $\mathscr{R} \frac{1}{k^2} \mathfrak{z} \geq \frac{1}{k}$ on the
  Farey arc, the integrand is majorised by
\begin{align*}
  e^{- \frac{\pi}{2D} \frac{\delta_N}{k^2 | \delta_N^2+ \varphi^2}1}
  |k^2 (\delta_N^2 + \varphi^2)|^{-r/4} & = \delta_N^{- r/4} \left(
  \frac{\delta_N}{k^2 (\delta_N^2 + \varphi^2)}\right)^{r/4} 
  e^{-\frac{\pi}{2D} \frac{\delta_N}{k^2 \left(\delta_N^2+ \varphi^2\right)}}\\
  & = O (n^{r/4})\\
  |S_{o\underline{m}}| & \leq C n^{r/4} \sum^{\sqrt{n}}_{k=1} k^{1-
    \alpha + \epsilon} (n, k)^{\alpha} e^{-\frac{\pi}{4} \underline{m}'
    S^{-1} \underline{m}} \frac{1}{k\sqrt{n}} ,
\end{align*}
since\pageoriginale the path of integration has a length of the order
$1/k\sqrt{n}$. Now summing over all $\underline{m}\neq 0$,
\begin{align*}
  \left|\sum_{\underline{m} \neq 0} S_{o \underline{m}}\right| & \leq
  C n^{\frac{r}{4}- \frac{1}{2}} \sum^{\sqrt{n}}_{k=1} k^{- \alpha +
    \epsilon} (n, k)^\alpha\\
  & < Cn^{\frac{r}{4} - \frac{1}{2}} \sum_{d\mid n} d^\alpha \sum_{dt \leq
  \sqrt{n}} (dt)^{- \alpha + \epsilon}\\
  & = C n^{\frac{r}{4}- \frac{1}{2}} \sum_{d\mid n} d^\epsilon \sum_{t \leq
    \frac{\sqrt{n}}{d}} t^{- \alpha + \epsilon}\\
  & < C n^{\frac{r}{4} - \frac{1}{2}} \sum_{d\mid n} d^\epsilon
  \left(\frac{\sqrt{n}}{d}\right)^{1- \alpha + \epsilon}\\
  & = C n^{\frac{r}{4} - \frac{\alpha}{2} + \frac{\epsilon}{2}}
  \sum_{d\mid n}  d^{\alpha -1},
\end{align*}
and\pageoriginale since the number of divisors of $n$ is $O
(n^{\epsilon/2})$. This is 
\begin{align*}
  & = C n^{\frac{r}{4} - \frac{\alpha}{2} + \frac{\epsilon}{2} +
    \frac{\epsilon}{2}}\\
  & = C n^{\frac{r}{4} - \frac{\alpha}{2}+ \epsilon}
\end{align*}

Improving $\alpha$ has been the feature of many investigations.
