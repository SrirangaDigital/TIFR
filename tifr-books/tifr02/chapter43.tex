\chapter{Lecture}\label{part4:lec43} %% 43
\markboth{\thechapter. Lecture}{\thechapter. Lecture}

Let\pageoriginale us return to the generalised theta-formula:
\begin{align*}
  \Theta (t; \alpha_1, \ldots \alpha_r) & = \sum_{\underline{n}} e^{-
    \pi t (\underline{n}' +\underline{\alpha})S (\underline{n} +
    \underline{\alpha})}\\
  & = \sum_{\underline{m}} c(\underline{m}) e^{2 \pi i \underline{m}' \underline{\alpha}}
\end{align*}
where
$$
c(\underline{m}) = \mathop{\int\cdots\int}_{- \infty}^\infty e^{- \pi
  t \mathscr{V}' S \mathscr{V}} e^{-2 \pi i
  \underline{m}'\underline{\mathscr{V}}} d \mathscr{V}_1 \ldots d
\underline{\mathscr{V}_r}  
$$

To get this into shape, consider the quadratic complement 
$$
- \frac{\pi}{t} (t \underline{\mathscr{V}'}+ i \underline{m}' S^{-1})
S (t \underline{\mathscr{V}}+ i S^{-1}\underline{m}) =- \pi t
\underline{\mathscr{V}'} S \underline{\mathscr{V}}- \pi i
\underline{\mathscr{V}'} \underline{m}- \pi i
\underline{m}'\underline{\mathscr{V}} + \frac{\pi}{t} \underline{m}'
S^{-1} \underline{m}  
$$

Since $m' \underline{\mathscr{V}}= \underline{\mathscr{V}'}
\underline{m}$,
\begin{align*}
  c(m) & = \mathop{\int \cdots \int}^\infty_{- \infty} e^{-
    \frac{\pi}{t} \underline{m}' S^{-1} \underline{m}} e^{-
    \frac{\pi}{t}(t \underline{\mathscr{V}'} + i\underline{m}'
    S^{-1})S (t \underline{\mathscr{V}}+ i S^{-1} \underline{m})d
    \mathscr{V}_1 \ldots d \mathscr{V}_r}\\
  & = e^{- \frac{\pi}{t}\underline{m}' S^{-1} \underline{m}}
  \mathop{\int\cdots\int}_{- \infty}^\infty e^{- \pi (\sqrt{t}
    \underline{\mathscr{V}'}+ \frac{i}{\sqrt{t}} \underline{m}'
    S^{-1}) S(\sqrt{t} \underline{\mathscr{V}} + \frac{i}{\sqrt{t}}
    S^{-1} \underline{m})} d \mathscr{V}_1 \cdots d \mathscr{V}_r 
\end{align*}

Put\pageoriginale $\sqrt{t} \underline{\mathscr{V}}= \underline{w}$
and $\underline{\mu}= \frac{1}{\sqrt{t}} \underline{m}' S^{-1}$. Then 
$$
c(\underline{m}) = \frac{e^{- \frac{\pi}{t} \underline{m}' S^{-1}
    \underline{m}}}{(\sqrt{t})^r} \mathop{\int\cdots \int}_{-
  \infty}^\infty e^{- \pi (\underline{w}' + i \underline{\mu}')
  S(\underline{w} + i \underline{\mu})} d w_1 \cdots d w_r
$$

Since every positive definite quadratic form may be turned into $a$
sum of squares, we can put $S=A'A$, so that the exponent in the
integrand become $- \pi (\underline{w}' A' + i\underline{\mu} A')(A
\underline{w}+ i A \underline{\mu})$; and writing $A \underline{w}=
\underline{\mathfrak{z}}$, we have
$$
c (\underline{m}) = \frac{e^{-\frac{\pi}{t} \underline{m}' S^{-1}
    \underline{m}}}{(\sqrt{t})^r} \mathop{\int\cdots \int}_{-
  \infty}^\infty e^{- \pi (\underline{\mathfrak{z}'}+ i
  \underline{\mathscr{V}'})  (\underline{\mathfrak{z}}+ i
  \underline{\mathscr{V}})} \frac{d_{\mathfrak{z}_1}\ldots d_{\mathfrak{z}_r}}{|A|} 
$$
where $\underline{\mathscr{V}}= \underline{\mu} A$, and $|A|=$
determinant of $A$. Let $D= |A|^2 = |S|$, $\underline{\mathfrak{z}'}=
(\mathfrak{z}_1,\ldots , \mathfrak{z}_r)$. Then
\begin{align*}
  c(\underline{m}) & = \frac{e^{-\frac{\pi}{t} \underline{m}' S^{-1}
      \underline{m}}}{D^{1/2} t^{r/2}} \prod^r_{j=1} \int^\infty_{-
    \infty} e^{- \pi (\mathfrak{z}_j + i \mathscr{V}_j)^2} d
  \mathfrak{z}_j\\
  & = \frac{e^{- \frac{\pi}{t} \underline{m}' S^{-1}
      \underline{m}}}{D^{1/2}t^{r/2}} \left(\int^\infty_{- \infty}
  e^{- \pi \mathfrak{z}^2} d \mathfrak{z} \right)^r\\
  & = \frac{e^{- \frac{\pi}{t}} \underline{m}' S^{-1}
    \underline{m}}{D^{1/2} t^{r/2}},
\end{align*}
the last factor being unity. So we have ultimately
$$
\Theta (t; \alpha_1, \ldots , \alpha_r) = \frac{1}{D^{1/2} t^{r/2}}
\sum_{\underline{m}} e^{- \frac{\pi}{t} \underline{m}' S^{-1}
  \underline{m}} e^{2 \pi i \underline{m}' \underline{\alpha} }
$$

Let\pageoriginale us now we back to our study of $A_r(n)$. We had integrals with
now limits which were the special feature of the Kloosterman method.
$$
A_r (n) = \mathop{\textstyle{\sum'}}_{0 \leq h < k \leq N} e^{-2 \pi i
\frac{h}{k}n} \int\limits^{\frac{1}{k(k+N)}}_{-\frac{1}{k(k+N)}} F_r
\left(e^{2 \pi i \frac{h}{k} - 2 \pi \mathfrak{z}} \right) e^{2 \pi n \mathfrak{z}} d
\mathfrak{z}+ \sum^{N-1}_{\ell =0} \cdots + \sum^{N-1}_{\ell=0} \cdots  
$$

Now 
\begin{align*}
  F_r (x) & = \sum_{\underline{n}} x^{\underline{n}' S\underline{n}}=
  1+ \sum^\infty_{n=1} A_r (n) x^n\\
  F_r \left(e^{2 \pi i \frac{h}{k} - 2 \pi \mathfrak{z}} \right) & =
  \sum_{\underline{n}} e^{(2 \pi i \frac{h}{k} - 2 \pi \mathfrak{z})}
  \underline{n}' S \underline{n}\\
  & = \sum_{\underline{n}} e^{2 \pi i \frac{h}{k} \underline{n}' S
    \underline{n}} e^{- 2 \pi \mathfrak{z} \underline{n}' S \underline{n}}
\end{align*}
$\underline{n}$ is of interest only modulo $k$, so put
$$
\underline{n} = k q + \ell, \ell = (\ell_1, \ldots , \ell_n), 0 \leq
\ell_j < k.
$$

So dismissing multiples of $k$, 
$$
F_r (e^{2 \pi i \frac{h}{k} - 2 \pi \mathfrak{z}})  =
\sum_{\underline{\ell} \mod k} e^{2 \pi i \frac{h}{k}
  \underline{\ell}' S \ell'} \sum_{q} e^{- 2 \pi \mathfrak{z} k^2 (q'
  + \frac{\underline{\ell'}}{k}) S \left(q+ \frac{\underline{\ell}}{k} \right)},
$$
and applying the transformation formula we derived earlier, with $t= 2
\mathfrak{z} k^2$ and $\underline{\alpha}= \frac{1}{k}
\underline{\ell}$, this becomes
\begin{align*}
  & \frac{1}{\sqrt{D} k^r e^{r/2} \mathfrak{z}^{r/2}}
  \sum_{\underline{\ell}} e^{2 \pi i \frac{h}{k}\underline{\ell}' S
    \underline{\ell} } \sum_{\underline{m}} e^{- \frac{\pi}{2
      \mathfrak{z} k^2} \underline{m}' S^{-1} \underline{m}} e^{2 \pi
    i \underline{m}' \frac{\underline{\ell}'}{k}}\\
  & = \frac{1}{\sqrt{D} k^r (2 \mathfrak{z})^{r/2}}
  \sum_{\underline{m}} e^{- \frac{\pi}{2 \mathfrak{z} k^2}
    \underline{m}' S^{-1} \underline{m}} T_k (h, m),
\end{align*}\pageoriginale 
on exchanging summations, where
$$
T_k (h, m) = \sum_{\underline{\ell}} e^{2 \frac{\pi i}{k}(h
  \underline{\ell}' S\underline{\ell}+ \underline{m}' \underline{\ell})}
$$

$T_k (h, 0)$ will be the most important; the others we only
estimate. We require a little more number theory for this. We cannot
tolerate the presence of a both a quadratic form and a linear form in
the exponent. There will be a common denominator in $\underline{m}'
S^{-1} \underline{m}$ and that will have to be discussed.

