\chapter{Lecture}\label{part3:lec19} %%% 19
\markboth{\thechapter. Lecture}{\thechapter. Lecture}

The\pageoriginale formula for $p(n)$ looked like this:
$$
p(n) = \frac{1}{i} \left( \frac{\pi}{12}\right)^{3/2}
\sum^\infty_{k=1} A_k (n) k^{- 5/2} \int\limits^{(0+)}_{- \infty}
s^{- 5/2} e^{s+ \frac{1}{s} (\frac{\pi}{12k})^2 (24n-1)} ds
$$

We discussed the loop integral
$$
L_v (\nu) = \frac{1}{2 \pi i} \int\limits^{(0+)}_{ -\infty} s^{-v -1}
e^{s + \frac{v}{s}}ds, \mathscr{R} \nu > 0.
$$

We can differentiate under the integral sign and obtain
$$
L'_v (\nu) = \frac{1}{2 \pi i}\int\limits^{(0+)}_{ -\infty} s^{-v -2}
e^{s + \frac{v}{s}}ds= L_{v+1} (\nu)
$$

This integral is again of the same sort as before; so we can repeat
differentiation under the integral sign. Clearly then $L_v (\nu)$ is
an entire function of $\nu \cdot L_v (\nu)$ has the expansion in a
Taylor series:
$$
L_v (\nu) = \sum^\infty_{r=0} \frac{L_v^{(r)} (0)}{r!} \nu^r
$$

$L_v^{(r)} (\nu)$ can be foreseen and is clearly
$$
\frac{1}{2 \pi i} \int\limits^{(0+)}_{- \infty} s^{-v -1-r} e^{s +
  \frac{v}{s}} ds
$$

So\pageoriginale  
$$
L_v (\nu) = \sum^\infty_{r=0} \frac{\nu^r}{r!} \frac{1}{2 \pi i}
\int\limits^{(0+)}_{- \infty} s^{-v -1-r} e^s ds
$$

We now utilise a famous formula for the $\Gamma$-function - Hankel's
formula, viz,
$$
\frac{1}{\Gamma (\mu)} = \frac{1}{2 \pi i} \int\limits^{(0+)}_{-
  \infty} s^{- \mu} e^{s} ds.
$$

This is proved by means of the formula $\Gamma (s) \Gamma (1-s)=
\dfrac{\pi}{\sin \pi s}$ and the Euler integral. Using the Hankel
formula we get $L$ explicitly:
$$
L_v (\nu) = \sum^\infty_{r=0} \frac{\nu^r}{r! \Gamma (v +r+1)}
$$

What we have obtained is something which we could have guessed
earlier. Expanding $e^{v/s}$ as a power series, we have
\begin{align*}
  L_v (\nu) & = \frac{1}{2 \pi i} \int\limits^{(0+)}_{- \infty} s^{-
    v-1} e^s \sum^\infty_{r=0} \frac{(v/s)^r}{r!} ds\\
  & = \frac{1}{2 \pi i} \int\limits^{(0+)}_{-\infty}
  \sum^\infty_{r=0} \frac{v^r}{r!} e^s s^{-v-1-r} ds,
\end{align*}
and what we have proved therefore is that we can interchange the
integration and summation. We have
$$
L'_v (\nu) = L_{v+1} (\nu).
$$

Having this under control we can put it back into our formula and get
a final statement about $p(n)$.

$$
p(n) = 2 \pi \left( \frac{\pi}{12}\right)^{3/2} \sum^\infty_{k=1} A_k
(n) k^{- 5/2} L_{3/2} \left( \left( \frac{\pi}{12k}\right)^2(24n-1) \right) 
$$

This\pageoriginale is not yet the classical formula of Hardy and
Ramanujan. One trick one adopts is to replace the index. Remembering
that $L'_v (\nu) = L_{v+1} (\nu)$, we have
\begin{align*}
  L_{3/2} \left(\left( \frac{\pi}{12k}\right)^2 (24n-1) \right) & =
  L'_{1/2} \left( \left( \frac{\pi}{12k}\right)^2 (24n-1)\right)\\
  & = \frac{6k^2}{\pi^2} \frac{d}{dn} L_{1/2} \left( \left(
  \frac{\pi}{12k}\right)^2 (24n-1)\right)
\end{align*}

Let us write the formula for further preparation closer to the Hardy
Ramanujan notation:
$$
p(n) = \left( \frac{\pi}{12}\right)^{1/2} \sum^\infty_{k=1} A_k (n)
k^{-1/2} \frac{d}{dn} L_{1/2} \left( \left( \frac{\pi}{12k}\right)^2
(24n-1)\right) 
$$

Now it turns out that the $L$-functions for the subscript
$\frac{1}{2}$ are elementary functions. We introduce the classical
Bessel function
$$
\mathcal{J}_v (\mathfrak{z}) = \sum^\infty_{r=0} \frac{(-)^r
  (\mathfrak{z}/2)^{2r+v}}{r! \Gamma (v +r+1)}
$$
and the hyperbolic Bessel function (or the `Bessel function with
imaginary argument')
$$
\mathcal{I}_v (\mathfrak{z}) = \sum^\infty_{r=0}
\frac{(\mathfrak{z}/2)^{2r+v}}{r! \Gamma (v+r+1)}
$$

How\pageoriginale do they belong together? We have
\begin{align*}
  L_v \left( \frac{\mathfrak{z}^2}{4}\right) & = \mathcal{I}_v
  (\mathfrak{z}) \left(\frac{\mathfrak{z}}{2} \right)^{-v},\\
  L_v \left( -\frac{\mathfrak{z}^2}{4}\right) & = \mathcal{J}_v
  (\mathfrak{z}) \left(\frac{\mathfrak{z}}{2} \right)^{-v},  
\end{align*}
connecting our function with the classical functions. In our case
therefore we could write in particular
$$
L_{1/2} \left( \left(\frac{\pi}{12 k}\right)^2 (24n-1)\right)=
\mathcal{I}_{1/2} \left(\frac{\pi}{6k}\sqrt{24n-1}\right) \left(
\frac{\pi}{12k} \sqrt{24 n-1}\right)^{-1/2}
$$

This is always good, but we would come into trouble if we have
$24n-1\leq 0$. It is better to make a case distinction; the above
holds for $n \geq 1$, and for $n \leq 0$, $n=- m$, we have
\begin{align*}
  L_{1/2} \left( \left( \frac{\pi}{12k}\right)^2(24n-1)\right) & =
  L_{1/2} \left(- \left(\frac{\pi}{12k} \right)^2(24m+1)\right)\\
  & = J_{1/2} \left( \frac{\pi}{6k} \sqrt{24 m+1}\right)
  \left( \frac{\pi}{12k} \sqrt{24m+1}\right)^{-1/2} 
\end{align*}

So we have: \qquad $n \geq 1$.
$$
p(n) = \left(\frac{\pi}{12} \right)^{1/2} \sum^\infty_{k=1} A_k (n)
k^{-1/2} \frac{d}{dn} \left(\frac{\mathcal{I}_{1/2} \left(
  \frac{\pi}{6k} \sqrt{24 n-1}\right)}{\left( \frac{\pi}{12k} \sqrt{24
    n-1}\right)^{1/2}} \right)
$$
$n= -m \leq 0$\pageoriginale
$$
p(n) = p(-m)= - \left( \frac{\pi}{12}\right)^{1/2} \sum^\infty_{k=1}
A_k (-m) k^{-1/2} \cdot \frac{d}{dm} \left( \frac{J_{1/2}
  \left( \frac{\pi}{6k} \sqrt{24m+1}\right)}{\left( \frac{\pi}{12k}
  \sqrt{24 m+1}\right)^{1/2}}\right) 
$$

We are not yet quite satisfied. It is interesting to note that the
last expression is 1 for $n=0$ and 0 for $n< 0$. We shall pursue this
later.

We have now more or less standardised functions. We can even look up
tables and compute the Bessel function. However $\mathcal{I}_{1/2}$
and $J_{1/2}$ are more elementary functions.
\begin{align*}
  J_{1/2} (\mathfrak{z}) & = \sum^\infty_{r=0} \frac{(-)^r
    (\mathfrak{z}/2)^{2r+ 1/2}}{r! \Gamma (r + \frac{3}{2})}\\
  & = \sum^\infty_{r=0} \frac{(-)^r (\mathfrak{z}/2)^{2r +
      \frac{1}{2}}}{r! \left( r + \frac{1}{2}\right) \left( r -
    \frac{1}{2} \right) \cdots \frac{1}{2} \Gamma\left(
    \frac{1}{2}\right)}\\
    & = \left( \frac{2}{\pi \mathfrak{z}}\right)^{1/2}
    \sum^\infty_{r=0} \frac{(-)^r \mathfrak{z}^{2r+1}}{(2r+1)!}\\
    & = \left( \frac{2}{\pi \mathfrak{z}}\right)^{1/2} \sin \mathfrak{z}.
\end{align*}

Similarly\pageoriginale if we has abolished $(-)^r$ we should have
\begin{align*}
  I_{\frac{1}{2}} (\mathfrak{z})& = \left( \frac{2}{\pi
    \mathfrak{z}}\right)^{1/2} \sinh \mathfrak{z}\\
  \frac{I_{1/2} (\mathfrak{z})}{(\mathfrak{z}/2)^{1/2}} & = I_{1/2}
  (\mathfrak{z}) \left( \frac{2}{\mathfrak{z}}\right)^{1/2} =
  \frac{2}{\sqrt{2}} = \frac{\sinh \mathfrak{z}}{\mathfrak{z}}\\
  \frac{J_{1/2} (\mathfrak{z})}{(\mathfrak{z}/2)^{1/2}} & =
  \frac{2}{\sqrt{\pi}}  \frac{\sin \mathfrak{z}}{\mathfrak{z}}
\end{align*}

We are now at the final step in the deduction of our formula: 

\heading{$n \geq1$}

$$
\displaylines{\hfill p(n) = \frac{1}{\sqrt{3}} \sum^\infty_{k=1} A_k
  (n) k^{-1/2} \frac{d}{dn} \left( \frac{\sinh \frac{\pi}{6k}
    \sqrt{24n-1}}{\frac{\pi}{6k} \sqrt{24 n-1}} \right) \hfill \cr
  \text{or with} \hfill \frac{\pi}{6k} \sqrt{24 n-1} = \frac{\pi}{k}
  \sqrt{\frac{2}{3} (n- \frac{1}{24})} = \frac{c}{k} \sqrt{n-
    \frac{1}{24}}, C= \pi \sqrt{\frac{2}{3}}, \hfill \cr
  \hfill p(n) = \frac{1}{\pi \sqrt{2}} \sum^\infty_{k=1} A_k (n) k^{1/2}
  \frac{d}{dn} \left( \frac{\sinh \frac{c}{k} \sqrt{n-
      \frac{1}{24}}}{\sqrt{n- \frac{1}{24}}}\right)\hfill }
$$

\heading{$n=- m\leq 0$}
 
$$
p(n) = p(-m) =- \frac{1}{\pi \sqrt{2}} \sum^\infty_{k=1} A_k (-m)
k^{\frac{1}{2}} \frac{d}{dm} \left( \frac{\sin \frac{c}{k} \sqrt{m+
    \frac{1}{24}}}{\sqrt{m+ \frac{1}{24}}}\right)
$$

This\pageoriginale is the final shape of our formula -$a$ convergent series for
$p(n)$.

The formula can be used for independent computation of $p(n)$. The
terms become small. It is of interest to find what one gets if one
breaks the series off, say at $k=N$
$$
p(n) = \frac{\pi^{5/2}}{12\sqrt{3}} \sum^N_{k=1} \cdots + R_N
$$

Let us appraise $R_N \cdot |A_k (n)| \leq k$, because there are only
$\varphi(k)$ roots of unity. We want an estimate for $L_{3/2}$. For $n
\geq 1$,
\begin{align*}
  L_{3/2} \left( \left( \frac{\pi}{12k}\right)^2(24n-1)\right) & \leq
  \sum^\infty_{r=0} \frac{\left( \frac{\pi^2}{6k^2}n\right)^r}{r!
    \Gamma \left( r+ \frac{5}{2}\right)}\\
  & \leq \sum^\infty_{r=0} \frac{\left(
    \frac{\pi^2}{6(N+1)^2}n\right)^r}{r! \Gamma \left(
    \frac{1}{2}\right) \cdot \frac{1}{2} \cdot \left(r + \frac{3}{2}
    \right)}\\ 
  (\text{since} k > N ~\text{in}~ R_N) & = \frac{1}{\sqrt{\pi}} \sum^\infty_{r=0}
  \frac{\left(\frac{\pi^2}{6(N+1)^2} n  \right)^r \cdot 2^{2r+1}}{(2
    r+1)! (r+ \frac{3}{2})}\\
  & \leq \frac{2}{\sqrt{\pi}} \sum^\infty_{r=0} \frac{\left(\frac{2
      \pi^2}{3(N+1)^2}n  \right)^r}{(2 r+1)!}\\
  & \leq \frac{2}{3} \cdot \frac{2}{\sqrt{\pi}} \sum^\infty_{r=0}
  \frac{\left( \frac{2 \pi^2}{3(N+1)^2}n\right)^r}{(2r)!}\\
  & < \frac{4}{3 \sqrt{\pi}} e^{\frac{\pi}{N+1} \sqrt{\frac{3n}{3}}}
\end{align*}

\begin{align*}
  \therefore \qquad |R_N| &\leq \frac{\pi^2}{9 \sqrt{3}}
  e^{\frac{\pi}{N+1} \sqrt{\frac{2n}{3}}} \sum^\infty_{k=N+1}
  \frac{1}{k^{3/2}}\\
  & \leq \frac{\pi^2}{9 \sqrt{3}} e^{\frac{\pi}{N+1}
    \sqrt{\frac{2n}{3}}} \int\limits^{\infty}_{N} \frac{dk}{k^{3/2}}\\
  \therefore \qquad |R_N| & < \frac{2\pi^2}{9 \sqrt{3}}
  e^{\frac{\pi}{N+1}\sqrt{2n}{3}}\frac{1}{N^{1/2}} 
\end{align*}

This\pageoriginale tells us what we have in mind. Make $N$ suitably
large. Then one gets something of interest. Put $N= [\alpha
  \sqrt{n}]$, $\alpha$ constant. Then
$$
R_N = O (n^{-\frac{1}{4}})
$$

And this is what Hardy and Ramanujan did. Their work still looks
different. They did not have infinite series. They had replaced the
hyperbolic sine by the most important part of it, the exponential. The
series converges in our case since $\sinh x \sim x$ as $x \to 0$, so
that $\sinh \left( \frac{c}{k} \sqrt{n- \frac{1}{24}}\right)$ behaves
roughly like $\frac{c}{k}$. On differentiation we have $\frac{1}{k^2}$
so that along with $k^{1/2}$ in the numerator we get $k^{-3/2}$ and we
have convergence. In the Hardy-Ramanujan paper they had
$$
p(n) = \frac{1}{2 \pi \sqrt{2}} \sum^{[\sqrt{n}]}_{k=1} A_k (n)
k^{1/2} \frac{d}{dn} \left( \frac{e^{\frac{c}{k} \sqrt{n-
      \frac{1}{24}}}}{\sqrt{n- \frac{1}{24}}}\right)+ O 
(n^{-\frac{1}{4}})+ R^*_N 
$$

$\sinh$\pageoriginale was replaced by $\exp$.; so they neglected
\begin{align*}
  R^* & = \frac{1}{2 \pi \sqrt{2}} \sum^{[\sqrt{n}]}_{k=1} A_k (n)
  k^{1/2} \frac{d}{dn} \left(\frac{e^{- \frac{c}{k} \sqrt{n-
        \frac{1}{24}}}}{\sqrt{n- \frac{1}{24}}} \right)\\
  |R^*| & = O  \left(\sum^{[\sqrt{n}]}_{k=1} k^{3/2} \left(
  \frac{e^{- \frac{c}{k} \sqrt{n- \frac{1}{24}}}}{n- \frac{1}{24}}
  \cdot \frac{c}{k} + \frac{e^{-\frac{c}{k} \sqrt{n-
        \frac{1}{24}}}}{(n- \frac{1}{24})}\right) \right)
\end{align*}

The exponential is strongly negative if $k$ is small; so it is best
for $k=1$. Hence 
\begin{align*}
  |R^*| & = O  \left( \frac{1}{n} \left( \sum^{[\sqrt{n}]}_{k=1}
  k^{1/2} + \frac{1}{\sqrt{n}} \sum^{[\sqrt{n}]}_{k=1}
  k^{3/2}\right)\right)\\ 
  \sum^{N}_{k=1} k^{1/2} & = O  (N^{3/2})\\
  \sum^{N}_{k=1} k^{3/2} & = O  (N^{5/2}) 
\end{align*}

So
\begin{align*}
  |R^*| & = O  \left( \frac{1}{n} \left( n^{3/4} +
  \frac{1}{\sqrt{n}} n^{5/4}\right)\right)\\
  & = O  \left(n^{-\frac{1}{4}} \right)
\end{align*}

The constants in the $O$-term were not known at that time so
that numerical\pageoriginale computation was difficult. If the series
was broken off at some other place the terms might have
increased. Hardy and Ramanujan with good instinct broke off at the
right place.

We shall next resume our function-theoretic discussion and cast a
glance at the generating function for $p(n)$ about which we know a
good deal more now. 
 
