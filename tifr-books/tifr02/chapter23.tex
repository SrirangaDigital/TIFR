\chapter{Lecture}\label{part3:lec23} %%% 23
\markboth{\thechapter. Lecture}{\thechapter. Lecture}

Last\pageoriginale time we obtained the formula
\begin{multline*}
  \epsilon ^{-1} = \frac{1}{2 \sqrt{3k}} e^{\frac{\pi i (h- h')}{12k}} e^{-
    \frac{\pi i}{6k}} \sum_{j \mod 2k} e^{\frac{\pi i}{k} \left(3 hj^2
    _ j(k- h+1)\right)}\\
  + \frac{1}{2\sqrt{3k}} e^{\frac{\pi i (h- h')}{12k}} e^{\frac{\pi
      i}{6k}} \sum_{j \mod 2k} e^{\frac{\pi i}{k} (3 hj^2 + j(k-h-1))}
\end{multline*}
$\omega_{h, k}$ was defined by means of the equation
$$
f\left(e^{2 \pi i \frac{h+ i\mathfrak{z}}{k}} \right)= \omega_{hk}
  \sqrt{\mathfrak{z}} e^{\frac{\pi}{12k} \left(\frac{1}{\mathfrak{z}} -
    \mathfrak{z}\right)} f\left( e^{2 \pi i \frac{h'+ i /\mathfrak{z}}{k}}\right)
$$
$\omega_{hk}$ came from the $\epsilon $ in the transformation formula
$$
\eta \left( \frac{a \tau+b}{c \tau+d}\right)= \epsilon  \sqrt{\frac{c \tau
    +d}{i}} \eta ( \tau)
$$

In particular, 
\begin{align*}
  \eta \left( \frac{h'+ 1/\mathfrak{z}}{k}\right) & = \epsilon 
  \sqrt{\mathfrak{z}} \eta \left( \frac{h+ i\mathfrak{z}}{k}\right),\\
  f\left(e^{2 \pi i \tau} \right)= e^{\pi i \tau/12} (\eta (\tau))^{-1}
\end{align*}

Substituting in the previous formula,
$$
e^{\frac{\pi i}{12}} ~\frac{h+ i \mathfrak{z}}{k} \left\{ \eta
\left(\frac{h+ i \mathfrak{z}}{k} \right) \right\}^{-1} = \omega_{hk}
  \sqrt{\mathfrak{z}} e^{\frac{\pi}{12k} \left(\frac{1}{3}
    -\mathfrak{z} \right)} e^{\frac{\pi i}{12} ~\frac{h' +
      i\mathfrak{z}}{k}} \left\{ \eta \left( \frac{h' + i
      /\mathfrak{z}}{k}\right) \right\}^{-1}
$$
\begin{alignat*}{4}
  \text{i.e.,} &\hspace{2cm}& \eta\left(\frac{h'+ i/\mathfrak{z}}{k}
  \right) & = 
  \omega_{hk} \sqrt{\mathfrak{z}} e^{\frac{\pi i}{12k} (h' -h)} \eta
  \left( \frac{h+ i\mathfrak{z}}{k}\right)\hspace{2cm}\\
  \therefore~~ && \epsilon  & = \omega_{hk} e^{\frac{\pi i}{12k} (h'-h) }\\
  \text{or}~~~ && \omega_{hk} & = \epsilon  e^{- \frac{\pi}{12k} (h'-h)}
\end{alignat*}\pageoriginale

In the first formula we have obtained an expression for
$\epsilon^{-1}$. However, we could make a detour and act $\epsilon $ directly
instead of $\epsilon^{-1}$. Even otherwise this could be fixed up, for
after all it is a root of unity. We have $\epsilon  \bar{\epsilon} =1$ or $\epsilon  =
\bar{\epsilon}^{-1}$. So consistently changing the sign in the exponents, we
have 
\begin{multline*}
  \omega_{hk}= \bar{\epsilon}^{-1} e^{\frac{\pi i}{12 k} (h- h')} =
  \frac{1}{2\sqrt{3k}} e^{\frac{\pi i}{6k}} \sum_{j \mod 2k}
  e^{-\frac{\pi i}{k}(3 h j^2 + j (k-h+1)) }\\
  + \frac{1}{2\sqrt{3k}} e^{-\frac{\pi i}{6k}} \sum_{j \mod 2k} e^{-
    \frac{\pi i}{k} (3h j^2 + j (k-h-1))}
\end{multline*}

We now have the $\omega_{hk}$ that we need. But the $\omega_{hk}$ are
only of passing interest; we put them back into $A_k (n)$;
$$
A_k(n) = \sideset{}{'}\sum_{h \mod k} \omega_{hk} e^{-2 \pi i n h/k}
$$

This formula has one unpleasant feature, viz. $(h, k)=1$. But this
would not do any harm. We can use a lemma from an unpublished paper by
Whiteman which status that if $(h, k)=d> 1$, then 
$$
\sum_{j \mod 2k} e^{- \frac{\pi i}{k} (3 h j^2 + j (k-h \pm 1))}=0
$$ 

For\pageoriginale proving Whiteman status put $h=dh^*$, $k= dk^*$ and
$j= 2k^* l+r$, $0 \leq 1 \leq d-1$, $0\leq r \leq 2k^* -1$. Then 
\begin{align*}
  \sum_{j \mod 2k} e^{- \frac{\pi i}{k} (3 h j^2 + j (k-h \pm 1))} & =
  \sum_{\ell =0}^{d-1} \sum^{2k^*-1}_{r=0} e^{- \frac{\pi i}{dk^*} (3
    dh^* (2k^*1+ r)^2 + (2k^* \ell + r)(dk^*- dh^* \pm 1) ))}\\
  & = \sum^{2k^*-1}_{r=0} e^{- \frac{\pi i}{k}(3h r^2 + r(k-h \pm 1))}
  \sum^{d-1}_{\ell =0} e^{\mp 2 \pi i \ell/d},
\end{align*}
and the inner sum $=0$ because it is a full sum of roots of unity and
$d \neq 1$.

This simplifies the matter considerably. We can now write
\begin{multline*}
  A_k(n) = \frac{1}{2\sqrt{3k}} e^{\frac{\pi i}{6k}} \sum_{h \mod k}~
  \sum_{j \mod 2k} e^{- \frac{\pi i}{k} (3h j^2 + j (k-h+1))} e^{-2
    \pi i n \frac{h}{k}} \\
  + \frac{1}{2 \sqrt{3k}} e^{- \frac{\pi i}{6k}} \sum_{h \mod k}~
  \sum_{j \mod 2k} e^{- \frac{\pi i}{k} (3 h j^2 + j (k-h-1))} e^{- 2
    \pi i n \frac{h}{k}}
\end{multline*}

Rearranging,\pageoriginale this gives
\begin{multline*}
  A_k(n) = \frac{1}{2\sqrt{3k}} e^{\frac{\pi i}{6k}} \sum_{j \mod 2k}
  e^{- \frac{\pi i}{k} (k+1) j} \sum_{h \mod k} e^{-\frac{2
    \pi i}{k}( n+  \frac{j(3j-1)}{k})h} \\
  + \frac{1}{2 \sqrt{3k}} e^{- \frac{\pi i}{6k}} \sum_{j \mod 2k}
  e^{- \frac{\pi i}{k} (k-1)j} \sum_{h \mod k}  e^{- \frac{2
    \pi i}{k} (n+ \frac{j(3j-1)}{2})h}
\end{multline*}

The inner sum is equal to the sum of the $k^{th}$ roots of unity,
which is 0 or $k$, $k$ if all the summands are separately one, i.e.,
if 
$$
n+ \frac{j(3j-1)}{2} \equiv 0 \pmod{k}
$$

Hence 
$$
A_k (n) = \frac{1}{2} \sqrt{\frac{k}{3}} e^{\frac{\pi i}{6k}}
\sum_{\substack{j \mod 2k\\ \frac{j(3j-1)}{2} \equiv -n \pmod{k}}}
(-)^j e^{- \frac{\pi ij}{k}}+ \frac{1}{2} \sqrt{\frac{k}{3}}
e^{-\frac{\pi i}{6k}} \sum_{\substack{j \mod 2k\\ \frac{j(3j-1)}{2}
    \equiv -n \pmod{k}}} (-)^j e^{\frac{\pi i j}{k}}
$$

In the summation here we first take all $j'$s modulo $2k$ (this is the
first sieving out), and then retain only those $j$ which satisfy the
second condition modulo $k$ (this is the second sieving
out). Combining the terms,
\begin{align*}
  A_k (n) & = \frac{1}{2} \sqrt{\frac{k}{3}} \sum_{\substack{j \mod
      2k\\\frac{j(3j-1)}{2} \equiv -n \pmod{k}}} (-)^j \left\{ e^{-
    \frac{\pi i}{6k} (6j-1)} + e^{\frac{\pi i}{6k} (6j-1)}\right\}\\
  & = \sqrt{\frac{k}{3}} \sum_{\substack{j \mod
      2k\\\frac{j(3j-1)}{2} \equiv -n \pmod{k}}} (-)^j \cos \frac{\pi
    (6j-1)}{6k} 
\end{align*}\pageoriginale

This formula is due to A.Selberg. It is remarkable how simple it
is. We shall change it a little, so that it could be easily
computed. We shall show that the $A_k(n)$ have a certain
multiplicative property, so that they can be broken up into prime
parts which can be computed separately. Let us rewrite the summation
condition in the following way.
\begin{alignat*}{4}
  &\hspace{2cm}& 12 j (3j-1) & \equiv -24 n \pmod{24 k}\hspace{3cm}\\
  \text{i.e.,} && 36j^2 - 12 j + 1 & \equiv 1 - 24 n \pmod{24 k}\\
  \text{i.e.,} && (6 j-1)^2 & \equiv \nu \pmod{24 k}
\end{alignat*}
where we have written $\nu=1- 24n$. In the formula
$$
A_k (n) = \frac{1}{2} \sqrt{\frac{k}{3}} \sum_{\substack{j \mod
      2k\\\frac{j(3j-1)}{2} \equiv -n \pmod{k}}} (-)^j \left\{
e^{-\frac{\pi i}{6k} (6 j-1)} + e^{\frac{\pi i}{6k}(6 j-1) } \right\}
$$
replace $j$ by $2k-j$ in the first term (where $j$ runs through a
full system of residues, so does $2k-j$). Further, observe
that\pageoriginale we have now
\begin{alignat*}{4}
  && (12 k - 6j-1)^2 & \equiv \pmod{24 k}\\
  \text{i.e.,} & \hspace{2cm} & (6j+1)^2 & \equiv \pmod{24 k} \hspace{3cm}
\end{alignat*}

Then 
$$
  A_k (n) = \frac{1}{2} \sqrt{\frac{k}{3}} \left\{ \mathop{\sum
    ~~(-)^j}_{\substack{j \mod 2k\\ (6j-1)^2 \equiv \nu \pmod{ 24 k}}} 
  e^{-\frac{\pi i}{6k} (6 j+1)}
  + \mathop{\sum ~~(-)^j}_{\substack{j \mod
      2k\\ (6j-1)^2 \equiv \nu \pmod{ 24 k}}}  e^{\frac{\pi
      i}{6k}(6 j-1)}  \right\}
$$

In both terms the range of summation is $j \mod 2k$ and there is the
further condition which restricts $j$. So
$$
  A_k (n) = \frac{1}{2} \sqrt{\frac{k}{3}} \sum
    _{\substack{j \mod 2k\\ (6j\pm 1)^2 \equiv \nu \pmod{ 24 k}}}(-)^j 
  e^{-\frac{\pi i}{6k} (6 j\pm 1)}
$$ 

Write $6j \pm 1= \ell$. $6 j \pm 1$ thus modulo $24 k$. $j= \frac{\ell
  +1}{6}$, so it is the integer nearest to $\frac{\ell}{6}$ since
$(\ell, 6)=1$. So write $j= \left\{ \dfrac{\ell}{6} \right\}$ where
$\{ x\}$ denotes the integer nearest to $x$. Then
$$
  A_k (n) = \frac{1}{2} \sqrt{\frac{k}{3}} \sum 
    _{\substack{\ell  \mod 2k\\ (\ell, 6)=1, \ell^2 \equiv \nu \pmod{
        24 k}}}(-)^{\left\{ \frac{\ell}{6}\right\}} e^{\frac{\pi i \ell}{6k}} 
$$ 

And\pageoriginale one final touch. The ranges for $\ell$ in the two
conditions are modulo $12k$ and modulo $24k$. Make these ranges the
same. Then
$$
  A_k (n) = \frac{1}{4} \sqrt{\frac{k}{3}} \sum 
    _{\substack{\ell  \mod 24k\\ \ell^2 \equiv \nu \pmod{
        24 k}}}(-)^{\left\{ \frac{\ell}{6}\right\}} e^{\frac{\pi i \ell}{6k}} 
$$ 

We prefer the formula in this form which is much handler. We shall
utilise this to get the multiplicative property of  $A_k(n)$.
