\chapter{Lecture}\label{part3:lec21}
\markboth{\thechapter. Lecture}{\thechapter. Lecture}

We\pageoriginale have rewritten the generating function $f(x)$ as a
sum consisting of certain functions which we called $\Phi_k (x)$:
$$
\displaylines{\hfill f(x) = 2 \pi \left(
  \frac{\pi}{12}\right)^{\frac{3}{2}} \sum^\infty_{k=1} k^{-5/2}
  \sideset{}{'}\sum_{h \mod k} \omega_{hk} \Phi_k \left(xe^{-2 \pi i
    \frac{h}{k}}\right)\hfill \cr
  \text{where} \hfill \Phi_k (\mathfrak{z}) = \sum^\infty_{n=0}
  L_{3/2} \left( \frac{\pi^2}{6k^2} (n- \alpha)\right)
  \mathfrak{z}^n,\hfill }
$$
with $\alpha= \frac{1}{24}$. $\Phi_k (\mathfrak{z})$ could also be
written as 
$$
\Phi_k (\mathfrak{z}) = \sum^\infty_{r=0} \frac{\left( \frac{\pi}{k
    \sqrt{6}}\right)^{2r}}{r! \Gamma \left( \frac{5}{2} + r\right)}
\varphi_r (\mathfrak{z}'')
$$
where $\varphi_r (\mathfrak{z})$ is a rational function as we found
out. We got $\varphi$ explicitly by means of a certain $\psi$: 
$$
\varphi_r (\mathfrak{z}) = (-)^{r+1} \sum^r_{j=0} \Delta^j_\alpha
(\alpha+1)^r \frac{1}{(\mathfrak{z}-1)^{j+1}}
$$

What we need for questions of convergence is an estimate of
$\varphi_r$; this is not difficult.
$$
\Delta f(x) = f(x+1) - f(x) = f'(\xi_1), x < \xi_1 < x+1,
$$
by the mean-value theorem; and because $\Delta$ is a finite linear
process we can interchange it with the operation of applying the mean
value\pageoriginale theorem and obtain
\begin{align*}
  \Delta^2 f(x) & = \Delta (\Delta f(x)) = \Delta f'(\xi_1) =
  f'(\xi_1+1)- f' (\xi_1)\\
  & = f''(\xi_2), x < \xi_1 < \xi_2< \xi_1 +1 < x+2;\\
  \Delta^3 f(x) & = \Delta (\Delta^2 f(x)) = \Delta^2 f'(\xi), x< \xi
  < x+1, \\
  & = f'''(\xi_3), x< \xi <\xi_3 < \xi +2 < x+3;
\end{align*}
and in general
$$
\Delta^k f(x) = f^k (\xi), x< \xi < x+k. 
$$

This was to be expected. Take $|1-\mathfrak{z}|\geq \delta$, $0<
\delta<1$ so that $z$ is not too close to 1. $\dfrac{1}{\delta} > 1$
and $0 < \alpha <1$ 
\begin{align*}
  |\varphi_r (\mathfrak{z}| & \leq \sum^{r} _{j=0} r(r-1) \cdots
  (r-j+1)(1+ \alpha +j)^{r-j}\cdot \frac{1}{\delta^{j+1}}\\
  & < \sum^r_{j=0} \frac{(\alpha +1 + r)^r}{\delta^{j+1}}\\
  & < (r+1) \frac{(\alpha + 1 + r)^r}{\delta^{r+1}}\\
  & < \frac{(\alpha+1 + r)^{r+1}}{\delta^{r+1}}
\end{align*}

Originally we know that the formula for $f(x)$ was good for $|x|<
1$. From this point on we give a new meaning to
$\varphi_r(\mathfrak{z})$ for all $\mathfrak{z} \neq 1$.

This\pageoriginale is a new step. We prove that the series for
$\Phi_k (\mathfrak{z})$ is convergent not merely for
$|\mathfrak{z}|<1$ but also elsewhere. The sum in $\Phi_k
(\mathfrak{z})$ is majorised by
$$
\frac{1}{\delta} \sum^\infty_{r=0} \frac{\left(
  \frac{\pi^2}{6}\right)^r}{r! \Gamma \left( \frac{5}{2}+ r\right)}
\cdot \frac{(\alpha+1 + r)^{r+1}}{\delta^r} 
$$

This is convergent, for thought the numerator increases with $r$, we
have by Stirling's formula
$$
\frac{r^r}{r!} \sim \frac{r^r}{\sqrt{2 \pi} r^{r+ \frac{1}{2}} e^{-r}}
= \frac{e^r}{\sqrt{2 \pi r}}
$$

So as far as convergence is concerned it is no worse than
$$
\frac{1}{\delta} \sum^\infty_{r=1} \frac{\left(\frac{e \pi^2}{6
    \delta} \right)^r}{\Gamma \left(\frac{5}{2} + r \right)} (\alpha +
1 + r)\left(1+ \frac{\alpha+1}{r}\right)^r 
$$
which is $\leq C_\delta$, the power series still being rapidly
converging because of the factorial in the denominator and $\frac{e
  \pi^2}{6 \delta}$ is fixed and $\left( 1+
\frac{\alpha+1}{r}\right)^r$ is bounded. So we have absolute
convergence and indeed uniformly so for $|1-\mathfrak{z}| \geq
\delta$.

We have now a uniformly convergent series outside the point $z=1$, and
$\Phi_k (\mathfrak{z})$ is explained at every point except $z=1$ which
is an essential singularity. $\Phi_k (\mathfrak{z})$ is entire in
$\dfrac{1}{1- \mathfrak{z}}$. From this moment if we put it back into
our argument we have $f(x)$ in the whole plane if $x e^{-2 \pi i h/k}$
keep\pageoriginale away from 1. And we are sure of that; either
$|x|\leq 1- \delta$ or $|x|\geq 1+ \delta$. Originally $x$ was only
inside the unit circle; now it can be outside also. In both cases
$f(x)$ is majorised by
$$
\sum^\infty_{k=1} k^{-\frac{5}{2}}, k \cdot C_\delta = C_\delta
\sum^\infty_{k=1} k^{-3/2},
$$
which is absolutely convergent.

Therefore we have now a very peculiar situation. In this notation of
$\Phi_k$ we have obtained a function which represents \textit{two}
analytic functions separated by a natural boundary which is full of
singularities and cannot be crossed. They are not analytic
continuations. The outer function is something new; it is analytic
because the series is uniformly convergent in each compact subset. 

Consider the circle. We state something more explicit which explains
the behaviour at each point near the boundary. Since every convergence
is absolute there are no difficulties and convergence prevails even if
we take each piece separately. 
\begin{multline*}
  f(x) =- 2 \pi \left( \frac{\pi}{12}\right)^{3/2} \sum^\infty_{k=1}
  k^{-\frac{5}{2}} \sideset{}{'}\sum_{h \mod k} \omega_{hk} \\
  \sum^\infty_{r=0} \frac{\left(- \frac{\pi^2}{6k} \right)^r}{r!
    \Gamma \left( \frac{5}{2}+ r\right)} \sum^\infty_{j=0}
  \Delta_\alpha^j (\alpha+1)^r \cdot \frac{1}{(x e^{-2 \pi i h/k}-1)^{j+1}}
\end{multline*}

We\pageoriginale can now rearrange at leisure.
\begin{multline*}
  f(x) =- 2 \pi \left( \frac{\pi}{12}\right)^{3/2} \sum^\infty_{k=1}
  k^{-\frac{5}{2}} \sideset{}{'}\sum_{h \mod k} \omega_{hk} \\
  \sum^\infty_{j=0} \frac{e^{-2 \pi i \, \frac{h}{k}(j+1)}}{(x-e^{2 \pi i
      h/k})^{j+1}} \sum^\infty_{r=j} \Delta^j_\alpha (\alpha+ 1)^r
  \frac{\left( - \frac{\pi^2}{6k^2}\right)^r}{r! \Gamma \left(
    \frac{5}{2} + r\right)} 
\end{multline*}

However, if we replaced $\sum^\infty_{r=j}$ by $\sum^\infty_{r=0}$
it would not to any harm because the summation is applied to a
polynomial of degree $r$ and the order of the difference is one more
than the power. We can therefore write, taking $\Delta$ outside, 
\begin{multline*}
  f (x)  = - 2 \pi \left(\frac{\pi}{12}\right)^{3/2} \sum^\infty_{k=1}
  k^{-\frac{5}{2}} \sideset{}{'}\sum_{h \mod k} \omega_{hk} \\
   \qquad\sum^\infty_{j=0} \frac{e^{2 \pi i} \frac{h}{k}(j+1)}{(x-e^{2 \pi i
      h/k})^{j+1}}  \Delta^j_\alpha \sum^\infty_{r=0}(\alpha+ 1)^r
  \frac{\left( - \frac{\pi^2}{6k^2}\right)^r}{r! \Gamma \left(
    \frac{5}{2} + r\right)}\\
   = -2 \pi \left( \frac{\pi}{12}\right)^{-5/2} \sum^\infty_{k=1}
  k^{\frac{5}{2}} \sideset{}{'}\sum_{h \mod k} \omega_{hk} \sum^\infty_{\ell=1}
  \frac{e^{2 \pi i \frac{h}{k} \ell}}{(x- e^{2 \pi i h/k})^\ell}
  \Delta_\alpha^{\ell-1} L_{3/2} \left(- \frac{\pi^2}{6k^2} (\alpha+1) \right)
\end{multline*}
 
It is quite clear what has happened. $x$ appears only in the
denominator, a root of unity is subtracted and the difference raised\pageoriginale
to a power 1. Choose specific $h$, $x$, 1; then we have a term
$\dfrac{B}{(x- e^{2 \pi i h/k})^\ell}$. We have a conglomerate of
terms which look like this, a conglomerate of singularities at each
root of unity. So we have a partial fraction decomposition not
exactly of the Mittag-Leffler type. Here of course the singularities
are not poles, and they are everywhere dense on the unit circle. Each
series $\sum\limits^\infty_{\ell =1}$ represents one specific point
$e^{2 \pi i h/k}$. 

Let us return to our previous statement. $f(x)$ is regular and
analytic outside the unit circle. What form has it there? Inside it is
$\prod\limits^{\infty}_{m=1}(1-x^m)$. We shall expand $f(x)$
about the point at infinity. We want the $\varphi's$ explicitly. 
\begin{align*}
  \varphi_0 (\mathfrak{z}) & = \frac{1}{1- \mathfrak{z}}\\
  \varphi_{r+1} (\mathfrak{z}) & = \mathfrak{z} \varphi'_r
  (\mathfrak{z}) - \alpha \varphi_r (\mathfrak{z})\\
  \varphi_0 (\mathfrak{z}) & = \frac{\mathfrak{z}^{-1}}{\mathfrak{z}^{-1}
    -1} =- \frac{\mathfrak{z}^{-1}}{1- \mathfrak{z}^{-1}} =-
  \sum^\infty_{m=1} \mathfrak{z}^{-m}\\
  \varphi_1 (\mathfrak{z}) & = \sum^\infty_{m=1} m\mathfrak{z}^{-m} +
  \alpha \sum^\infty_{m=1} \mathfrak{z}^{-m} = \sum^\infty_{m=1}
  (m+\alpha) \mathfrak{z}^{-m}
\end{align*}

The following thing will clearly prevail
$$
\varphi_r (\mathfrak{z}) = (-)^{r+1} \sum^\infty_{m=1} (m+\alpha)^r
\mathfrak{z}^{-m} 
$$

This speaks for itself.
$$
\varphi_{r+1} (\mathfrak{z}) = (-)^r \sum^\infty_{m=1} m(m + \alpha)
\mathfrak{z}^{-m} + (-)^r \alpha \sum^\infty_{m=1} (m + \alpha)^r
\mathfrak{z}^{-m} 
$$
So\pageoriginale the general formula is justified by induction.
$$
\Phi_k (\mathfrak{z}) =- \sum^\infty_{r=0} \frac{\left(-
  \frac{\pi^2}{6k^2} \right)^r}{r! \Gamma\left( \frac{5}{2}+ r\right)}
\sum^\infty_{m=1}  (m+\alpha)^r \mathfrak{z}^{-m}
$$
for all $|\mathfrak{z}|>1$. Exchanging summations,
\begin{align*}
  \Phi_k (\mathfrak{z}) & = - \sum^\infty_{m=1} \mathfrak{z}^{-m}
  \sum^\infty_{r=0} \frac{\left(\frac{\pi^2}{6k^2} (- m - \alpha)
    \right)^r}{r! \Gamma \left(\frac{5}{2} + r \right)}\\
  & = - \sum^\infty_{m=1}\mathfrak{z}^{-m} L_{3/2}
  \left(\frac{\pi^2}{6k^2} (-m -\alpha) \right)
\end{align*}

Put this back into $f(x)$; we get for $|x|>1$,
$$
\displaylines{\hfill 
  f(x) =- 2 \pi \left( \frac{\pi}{12}\right)^{3/2} \sum^\infty_{k=1}
  k^{- \frac{5}{2}} \sideset{}{'}\sum_{h \mod k} \omega_{hk} \sum^\infty_{m=1}
  \left( x^{-1} e^{2 \pi i \frac{h}{k}}\right)^m L_{3/2} \left(
  \frac{\pi^2}{6k^2} (- m - \alpha)\right),\hfill \cr
  \text{and since} \hfill A_k (n)= \sideset{}{'}\sum_{h \mod k} \omega_{hk} e^{-2
    \pi i h/k},\hfill \cr
  \hfill f(x) = - 2 \pi \left( \frac{\pi}{12}\right)^{3/2} \sum^\infty_{k=1}
  k^{-5/2}  \sum^\infty_{m=1} A_k (-m) x^{-m} L_{3/2} \left(
  \frac{\pi^2}{6k^2} (-m - \alpha)\right) \hfill }
$$

Again\pageoriginale interchanging summations,
$$
f(x) = - 2 \pi \left( \frac{\pi}{12}\right)^{3/2} \sum^\infty_{m=1}
  x^{-m}  \sum^\infty_{k=1} A_k (-m) k^{-5/2} L_{3/2} \left(
  \frac{\pi^2}{6k^2} (-m - \alpha)\right)
$$

The inner sum we recognize immediately; it is exactly what we had for
$p(n)$; so 
$$
f(x) =- 2 \pi \left( \frac{\pi}{12}\right)^{3/2} \sum^\infty_{m=1}
p(-m) x^{-m}
$$

And here is a surprise which could not be foreseen! By its very
meaning $p(-m)=0$. So
$$
f(x)\equiv 0
$$
outside the unit circle. This was first conjectured by myself and
proved by H.Petersson by a completely different method. Such
expressions occur in the theory of modular forms. Petersson got the
outside function first and then the inner one, contrary to what we
did. 

The function is represented by a series inside the circle, and it is
zero outside, with the circle being a natural boundary. There exist
simpler examples of this type of behaviour. Consider the partial sums:
\begin{multline*}
  1+ \frac{x}{1-x} = \frac{1}{1-x}\\ 
  1+ \frac{x}{1-x} +
  \frac{x^2}{(1-x)(1-x^2)}= \frac{1}{1-x} + \frac{x^2}{(1-x)(1-x^2)}
    = \frac{1}{(1-x)(1-x^2)}\\
    1+ \frac{x}{1-x} +
    \frac{x^2}{(1-x)(1-x^2)}+ \frac{x^3}{(1-x)(1-x^2)(1-x^3)} + \cdots
    \text{to} ~ n+1 ~\text{terms}\\ 
    = \frac{1}{(1-x)(1-x^2)\cdots (1-x^n)}
\end{multline*}

For\pageoriginale $|x|<1$, the partial sum converges to
$\dfrac{1}{\prod\limits^\infty_{m=1} (1-x^m)}$. For $|x|>1$ also it
has a limit; the powers of $x$ far outpace 1 and so the denominator
tends to infinity and the limit is zero. The Euler series here is
something just like our complicated function. Actually the two are the
same. For suppose we take the partial sum
$\dfrac{1}{(1-x)(1-x^2)\cdots (1-x^n)}$ and break it into partial
fractions. We get the roots of unity in the denominator, so that we
have a decomposition
$$
\sum \frac{B_{h, k, l, n}}{\left( x- e^{2 \pi i \frac{h}{k}}\right)^\ell}
$$
$k \leq n$ and $\ell$ not too high. For a higher $n$ we get a finer
expression into partial fractions. Let us face one of these, keeping
$h$, $k$, $\ell$ fixed:
$$
\frac{B_{h, k, l, n}}{\left( x- e^{2 \pi i \frac{h}{k}}\right)^\ell}
$$

Let $n \to \infty$. Then I have the opinion that
$$
B_{h, k, l, n} \to - 2 \pi \left( \frac{\pi}{12}\right)^{3/2}
\omega_{hk} k^{- \frac{5}{2}} e^{2 \pi i \frac{h}{k} \ell}
\Delta_\alpha^{\ell-1} L_{3_k} \left( - \frac{\pi^2}{6k^2} (\alpha+1)\right)
$$

The $B'$s all appear from algebraic relations and so are algebraic
numbers - in sufficiently high cyclotonic fields. And this is equal to
something which looked highly transcendental! though we cannot vouch
for this. The verification is difficult even in simple cases - and no
finite number of experiments would prove the result.

$\dfrac{B_{0, 1, 1, n}}{x-1}$\pageoriginale is itself very complicated. Let us
evaluate the principal formula for $f(x)$ and pick out the terms
corresponding to $h=0$, $k=l$, $\ell =1$.

$L_{3/2}$ is just the sine function and terns out to be
$-\dfrac{6}{25}- \dfrac{12\sqrt{3}}{75 \pi}$. Since $\frac{1}{1-x} =-
\frac{1}{x-1}$, $-1$ is the first approximation to $B_{0, 1, 1,
  n}$. If we take the partial fraction decomposition for 
$$
\frac{1}{(1-x)(1-x^2)}, \frac{1}{(1-x)(1-x^2)}= \frac{\cdot
   \cdot}{(x-1)^2}+\frac{\cdot \cdot}{(x-1)}+\frac{\cdot\cdot}{(1+x)},
$$
the numerator of the second term would give the second
approximation. If indeed these successive approximations converge to
$B_{0, 1,1,n}$ we could get a whole new approach to the theory of
partitions. We could start with the Euler series and go to the
partition function. 

We are now more prepared to go into the structure of $\omega_{hk}$. We
shall study next time the arithmetical sum $A_k(n)$ and the discovery
of A.Selberg. We shall then go back again to the $\eta$-function. 
