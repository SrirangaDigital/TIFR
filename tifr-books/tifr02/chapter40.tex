\chapter{Lecture}\label{part4:lec40} %%% 40
\markboth{\thechapter. Lecture}{\thechapter. Lecture}

Let\pageoriginale us look at $S_r(n)$ a little more explicitly.
$$
S_r (n) = \gamma_2 (n) \gamma_3 (n)
$$

\heading{$r\equiv 0 \pmod{4}$.} 

In this case we need not bother about
the sign of the Gaussian sums; the fourth power of the coefficient
becomes 1.
$$
\gamma_2 (n) = 1+ V_2 (n) + V_{2^2} (n)+ \cdots
$$ 
which is a finite sum. If $2 \nmid n$, then $\gamma_2 (n) =1$. If
$2\mid n$, $n=2^\alpha n'$, $2 \nmid n'$, then 
\begin{align*}
  V_2 (n) & = 0\\
  V_{2^\lambda} (n) & = 
  \begin{cases}
    \frac{(-)^{r/4}}{2^{(\lambda-1)\left(\frac{r}{2}-1\right)}}. &
    \text{if}~ \lambda < \alpha +1;\\
    - \frac{(-)^{r/4}}{2^{(\lambda -1)\left(\frac{r}{2}-1 \right)}}, &
    \text{if}~ \lambda = \alpha +1;\\
    0, & \text{if}~ \lambda > \alpha + 1 
  \end{cases}
\end{align*}
So
\begin{align*}
  \gamma_2 (n) & = 1+ (-)^{r/4} \left\{ \frac{1}{2^{\frac{r}{2}-1}} +
  \frac{1}{2^{2\left(\frac{r}{2}-1 \right)}} + \cdots +
  \frac{1}{2^{(\alpha-1)\left(\frac{r}{2}-1 \right)}} -
  \frac{1}{2^\alpha \left(\frac{r}{2}-1 \right)}\right\}\\
  & = 1+ (-)^{\frac{r}{4}} \sum^{\alpha}_{\mu=1}
  \frac{(-)^{\frac{n}{2 \mu}}}{2^\mu \left( \frac{r}{2} -1\right)},
\end{align*}
if\pageoriginale $2^\alpha||n$ ($2^\alpha$ is the highest power of 2 dividing
$n$). If $2 \nmid n$, $\gamma_2 (n)=1$.
\begin{align*}
  \gamma_p (n) & = \left(1- \frac{1}{p^{r/2}} \right) \left(1-
  \frac{1}{p^{r/2-1}} + \cdots \frac{1}{p^{\alpha \left(\frac{r}{2}-1
      \right)}}\right), p^\alpha ||n\\
  S_r(n) & = \gamma_2 (n) \prod_{p \geq 3} \left(1- \frac{1}{p^{r/2}}
  \right) \prod_{p\mid n, p ~\text{odd}} \left(1- \frac{1}{p^{r/2-1}}+
  \cdots + \frac{1}{p^{\alpha\left( \frac{r}{2}-1\right)}} \right)\\
  & = \gamma_2 (n) P_1\cdot P_2(n),
\end{align*}
where $P_1$ is a fixed factor and 
\begin{align*}
  P_2 (n) & = \prod_{p\mid n, p ~\text{odd}} \left(1-
  \frac{1}{p^{\frac{r}{2}-1}}+ \cdots + \frac{1}{p^{\alpha
      \left(\frac{r}{2}-1 \right)}} \right)\\
  & = \sum_{d\mid n, d ~\text{odd}} \frac{1}{d^{\frac{r}{2}-1}}.\\
  P_1 & = \left(1- \frac{1}{2^{r/2}} \right)^{-1} \prod_{p \geq 2}
  \left(1- \frac{1}{p^{r/2}} \right) \\
  & = \frac{2^{r/2}}{2^{r/2-1}} \times \frac{1}{\zeta \left(\frac{r}{2}
    \right)}. 
\end{align*}

It is known (vide: Whittaker \& Watson) that
$$
\zeta (2k) = (-)^{k-1} \frac{(2 \pi)^{2k} B_{2k}}{2(2k)!}, k \geq 1,
$$
where\pageoriginale $B_{2k}$ are the Bernoulli numbers.
\begin{gather*}
  \left( B_1 =- \frac{1}{2}, B_3 = B_5= B_7= \cdots =0; B_{2k} \neq 0;
  sgn B_{2k} = (-)^{k-1}\right)\\
  P_1 = \frac{2^{r/2}}{2^{r/2}-1} \times \frac{2 \left(
    \frac{r}{2}\right)!}{(2 \pi)^{r/2} |B_{r/2}|}
\end{gather*}

So for $r> 4$, the principal term
\begin{align*}
  A_r (n) & \thicksim \frac{\pi^{r/2}}{\Gamma
    \left(\frac{r}{2}\right)} n^{\frac{r}{2-1}} S_n (n)\\
  & = C_r (n),
\end{align*}
say, where
\begin{align*}
  C_r (n) & = \frac{\pi^{r/2}}{\Gamma (r/2)} \frac{2^{r/2}}{2^{r/2}-1}
  \frac{2 \left( \frac{r}{2}\right)!}{(2 \pi)^{r/2} |B_{r/2}|}
    n^{\frac{r}{2}-1} \gamma_2 (n) \sum_{d\mid n, d\text{odd}}
    \frac{1}{d^{\frac{r}{2}-1}} 
\end{align*}
(a\pageoriginale divisor sum! which is interesting, but not
surprising, because the Jacobi formula contains it).
$$
C_r(n) = \frac{r}{2^{\frac{r}{2}-1}|B_{r/2}|} n^{\frac{r}{2}-1}
\sum_{d\mid n , d ~\text{odd}} \frac{1}{d^{\frac{r}{2}-1}}
$$

\heading{$r \equiv 0 \pmod{8}$}

\begin{align*}
  n^{\frac{r}{2}-1}\gamma_2 (n) \cdot & \sum_{d\mid n, d
    ~\text{odd}}\frac{1}{d^{r/2-1}}\\ 
  & = n^{\frac{r}{2}-1} \left(1+
  \frac{1}{2^{\frac{r}{2}-1}} + \cdots + \frac{1}{2^{(\alpha
        -1)(\frac{r}{2}-1)}} - \frac{1}{2^{\alpha \left(
      \frac{r}{2}-1\right)}}  \sum_{d\mid n, d ~\text{odd}}
    \frac{1}{d^{\frac{r}{2}-1}}\right).\\
    & = n^{\frac{r}{2}-1} \sum_{\delta\mid n}
    \frac{(-)^{\frac{n}{\delta}}}{\delta^{\frac{r}{2}-1}}, ~\text{if} ~n
    ~\text{is even};\\
    & \quad n^{\frac{r}{2}-1} \sum_{\delta\mid n}
    \frac{1}{\delta^{\frac{r}{2}-1}}, \text{if $n$ is odd}.
\end{align*}

So for any $n$,
\begin{align*}
  n^{\frac{r}{2}-1} \gamma_2(n) \sum_{d\mid n, d \text{odd}}
  \frac{1}{d^{\frac{r}{2}-1}} & = n^{\frac{r}{2}-1} (-)^n
  \sum_{\delta\mid n} \frac{(-)^\frac{n}{\delta}}{\delta^{\frac{r}{2}-1}}\\
  & = (-)^n \sum_{\delta\mid n} (-)^{n/\delta} \left(
  \frac{n}{\delta}\right)^{\frac{r}{2}-1}\\
  & = (-)^n \sum_{t\mid n} (-)^t t^{\frac{r}{2}-1}
\end{align*}

So\pageoriginale 
$$
\displaylines{
C_r(n) = Q_r (-)^n \sum_{t\mid n} (-)^t t^{\frac{r}{2}-1},\cr
\text{here}\hfill Q_r =
\frac{r}{2^{\frac{r}{2}-1}|B_{\frac{r}{2}}|}\hfill }
$$

This is exactly what appears for $r=4$ in the Jacobi formula.

\heading{$r \equiv 4 \pmod{8}$}

\begin{align*}
  n^{\frac{r}{2} -1} \gamma_2 (n) & \sum_{d\mid n, d \text{odd}}
  \frac{1}{d^{\frac{r}{2}-1}}\\ & = n^{\frac{r}{2}-1} \left(1-
  \frac{1}{2^{\frac{r}{2}-1}} - \cdots -
  \frac{1}{2^{(\alpha-1)(\frac{r}{2}-1)}} + \frac{1}{2^{\alpha
      \left(\frac{r}{2}-1 \right)}} \right)  \sum_{d\mid n}, d ~\text{odd}~
  \frac{1}{d^{\frac{r}{2}-1}}\\
  & = n^{\frac{r}{2}-1} \sum_{\delta\mid n} \frac{(-)^{\delta+
      \frac{n}{\delta}+ 1}}{\delta^{\frac{r}{2}-1}}\\
  & = \sum_{\delta\mid n} (-)^{\delta+ \frac{n}{\delta}+1} \left(
  \frac{n}{\delta}\right)^{\frac{r}{2}-1}\\
  & = \sum_{t\mid n} (-)^{\frac{n}{t}+ t+ 1} t^{\frac{r}{2}-1}, ~\text{if
    $n$ is even};\\
  & \quad \sum_{t\mid n} t^{\frac{r}{2}-1}, ~\text{if $n$ is odd};
\end{align*}\pageoriginale
or in either case
$$
\displaylines{(-)^n \sum_{t\mid n} (-)^{t+ \frac{n}{t} +1}
  t^{\frac{r}{2}-1} \cr
  \text{So} \hfill C_r (n) = (-)^n Q_r \sum_{t\mid n} (-)^{\frac{n}{t} +
    t-1} t^{\frac{r}{2}-1}\hfill \cr
  A_r (n) \thicksim Q_n (-)^n \sum_{t\mid n} (-)^{t+ \frac{r}{4}
    \left(\frac{n}{t}+1 \right)} t^{\frac{r}{2}-1};\cr 
\text{where} \hfill A_r = \frac{r}{2^{\frac{r}{2}-1} |B_{r/2}|}\hfill }
$$

The Bernoulli numbers are all rational numbers and we can show that
$2(2^{r/2}-1)B_{r/2}$ is an odd integral i.e., $2(2^{2k}-1)B_{2k} (k
~\text{integral})$\pageoriginale is an all integer. Suppose $q$ is an
odd prime; then, by Fermat's theorem, 
\begin{align*}
  2^{q-1} & \equiv 1(\mod q)
\end{align*}

Let $(q-1)\mid 2k$. Then
\begin{align*}
  2^{2k} & \equiv 1 \pmod{q}\\
  2^{2k}-1 & \equiv 0 \pmod{q}
\end{align*}

We now appeal to the non-Steadt-Clausen theorem, which is a beautiful
theorem describing fully the denominators of the Bernoulli numbers:
$$
B_{2k} = G_{2k} - \sum_{(p-1)\mid 2k} \frac{1}{p}
$$
where $G_{2k}$ is an integer.
\begin{align*}
  \therefore \quad (2^{2k}-1) B_{2k} & = (2^{2k}-1) G_{2k}- (2^{2k}-1)
  \sum_{(p-1)\mid 2k} \frac{1}{p}\\
  & = \text{integer $+ \frac{1}{2}$ integer}
\end{align*}

So $2(2^{2k}-1) B_{2k}$ is an odd integer.

Let us obtain some specimens of 
\begin{gather*}
  Q_r = \frac{2r}{(2(2^{r/2}-1) |B_{r/2}|)}\\
  A_4 =8, \quad Q_8=16, \quad Q_{12}= 8, \quad Q_{16}= \frac{32}{17},\\
  Q_{20}= \frac{8}{31}, \quad Q_{24} = \frac{16}{691}, \quad Q_{28}=
  \frac{8}{5461}, \quad q_{32}= \frac{64}{929569} 
\end{gather*}

The\pageoriginale conspicuous prime 691 appears in connection with the
representation as the sum 24 squares; it has to do with $\eta^{24}$.

Can $A_r(n)$ be exactly equal to the asymptotic expression? (as for
$r=4$). $A_4(n)=C_4(n), A_8(n)=C_8(n)$. From $Q_{16}$ on wards,
$A_{16} (n) \neq C_{16} (n)$. This is because $Q_{16}$ has an odd
prime factor in the denominator. Suppose $p$ divides the
denominator. Then the fraction produced by $Q_{16}$ cannot be
destroyed by the other factor and $C_r (n)$ is not always an
integer. If $p \mid n$. the numerator of $C_{r}(p^\alpha)$ is congruent to
$\pm 1 \mod p$.

