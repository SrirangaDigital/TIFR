\chapter{Lecture}\label{part3:lec20} %% 20
\markboth{\thechapter. Lecture}{\thechapter. Lecture}

We\pageoriginale found a closed expression for $p(n)$; we shall now
look back at the generating function and get some interesting results.
$$
f(x) = \frac{1}{\prod\limits^\infty_{n=1} (1- x^n)} = \sum^\infty_{n=0}
p(n) x^n,
$$
and we know $p(n)$. $p(n)$ in its simplest form before reduction to
the traditional Bessel functions is given by
$$
\displaylines{\hfill p(n) = 2 \pi \left(\frac{\pi}{12} \right)^{3/2}
  \sum^\infty_{k=1} A_k (n) k^{-5/2} L_{3/2} \left(\left(
  \frac{\pi}{12k}\right)^2 (24n-1) \right), \hfill \cr
  \text{where} \hfill L_{3/2} \left( \frac{\pi^2}{6k^2} \left(n-
  \frac{1}{24}\right)\right) = \sum^\infty_{r=0} \frac{\frac{\pi}{6k^2} (n-
    \frac{1}{24})^r}{r! \Gamma \left( \frac{5}{2} + r\right)}\hfill }
$$

We wish first to give an appraisal of $L$ and show that the series for
$p(n)$ converges absolutely. The series is
$$
f(x) = 2 \pi \left( \frac{\pi}{12}\right)^{3/2} \sum^\infty_{n=0} x^n
\sum^\infty_{k=1} A_k (n) k^{-5/2} L_{3/2} \left( \frac{\pi^2}{6k^2}
(n- \alpha)\right),
$$
where we write $\frac{1}{24}= \alpha$ for abbreviation - it will be
useful for some other purposes also to have a symbol there instead of
a number. 

We make only a crude estimate.
\begin{align*}
  \left|L_{3/2} \left( \frac{\pi^2}{6k^2} (n- \alpha)\right) \right|
  & \leq \sum^\infty_{r=0} \frac{\left( \frac{\pi^2}{6}n\right)^r}{r!
    \Gamma \left( \frac{5}{2}+r \right)}\\
  & = \sum^\infty_{r=0}\frac{\left(\frac{\pi^2}{6} n \right)^r}{r!
    \Gamma \left(\frac{1}{2} \right) \cdot \frac{1}{2} \cdot
    \frac{3}{2} \cdots \left( \frac{3}{2} + r\right)}\\
  & \frac{2^2}{\sqrt{\pi}} \sum^\infty_{r=0} \frac{\left( \frac{2
      \pi}{3} \pi n\right)^r}{(2 r +1)! (3+2r)}\\
  & \leq 4 \sum^{\infty}_{r=0} \frac{(C\sqrt{n})^{2r}}{(2r)!}, C = \pi
  \sqrt{\frac{2}{3}},\\
  & \leq 4 \sum^{\infty}_{\rho=0} \frac{(C\sqrt{n})^\rho}{\rho !}\\
  & = 4 e^{C \sqrt{n}}
\end{align*}

So\pageoriginale $f(x)$ is majorised by
$$
\text{constant} ~x \sum^\infty_{n=1} |x|^n e^{C\sqrt{n}}
\sum^\infty_{k=1} \frac{1}{k^{3/2}}
$$
and this is absolutely convergent for $|x|< 1$, indeed uniformly so
for $|x|\leq 1- \delta$, $\delta > 0$, because $e^{C \sqrt{n}}=
0(e^{\delta n})$, $\delta > 0$, so that we need take $|x e^\delta|
<1$. We can therefore interchange the order of summation:
\begin{align*}
  f(x) & = 2\pi \left( \frac{\pi}{12} \right)^{3/2} \sum^\infty_{k=1}
  k^{- 5/2} \sum^\infty_{n=0} A_k (n) x^n L_{3/2} \left(
  \frac{\pi^2}{6k^2} (n- \alpha)\right)\\
  & = 2 \pi \left(\frac{\pi}{12} \right)^{3/2} \sum^\infty_{k=1} k^{-
    5/2} \sideset{}{'}\sum_{h \mod k} \omega_{hk} \sum^\infty_{n=0} \left(xe^{- 2
    \pi i \frac{h}{k}} \right)^n L_{3/2} \left( \frac{\pi^2}{6k^2} (n-
  \alpha)\right)
\end{align*}
where\pageoriginale the middle sum is a finite sum. This is all good
for $|x|<1$. Now call
$$
\Phi_k (\mathfrak{z})= \sum^\infty_{n=0} L_{3/2} \left(
\frac{\pi^2}{6k^2} (n- \alpha)\right) \mathfrak{z}^n
$$

So in a condensed form $f(x)$ appears as
$$
f(x) = 2 \pi \left( \frac{\pi}{12}\right)^{3/2} \sum^\infty_{k=1}
k^{-5/2} \sideset{}{'}\sum_{h \mod k} \omega_{hk} \Phi_k \left(x e^{- 2\pi i
    \frac{h}{k}}\right) 
$$

We have now a completely new form for our function. It is of great
interest to consider $\Phi_k (\mathfrak{z})$ for its own sake; it is a
power series $(|\mathfrak{z}|<1)$ and the coefficients of
$\mathfrak{z}^n$ are functions of $n-\alpha$. 
$$
L_{3/2} (\nu) = \sum^\infty_{r=0} \frac{\nu^r}{r! \Gamma \left(
  \frac{5}{2} +r\right)}
$$

This is an entire function of $\nu$, for the convergence is rapid
enough in the whole plane. Looking into the Hadamard theory of entire
functions, we could see that the order of this function is
$\dfrac{1}{2}$. This is indeed plausible, for the denominator is
roughly $(2r)!$ and $\sum \frac{\nu^r}{(2 r)!}= \sum
\frac{(\sqrt{\nu})^{2r}}{(2r)!} \sim e^{\sqrt{\nu}}$; or the function
grows like $e^{\sqrt{\nu}}$, and this is characteristic of the growth
of an entire function or order $\frac{1}{2}$. The coefficients of
$\mathfrak{z}^n$ are themselves entire functions in the subscript
$n$. 

We now quote a theorem of Wigert to the following effect. Suppose that
we have a power series $\Phi (\mathfrak{z}) = \sum\limits^\infty_{n=0}
g(n) \mathfrak{z}^n$ where $g(\nu)$\pageoriginale is in entire
function of order less than 1; then we can say something about $\Phi
(\mathfrak{z})$ which has been defined so far for $|\mathfrak{z}| <
1$. This function can be continued analytically beyond the circle of
convergence, and $\Phi (\mathfrak{z})$ has only $z=1$ as a
singularity; it will be an essential singularity in general, but a
pole if $g(\nu)$ is a rational function. We can extract the proof of
Wigert's theorem from our subsequent arguments; so we do not give a
separate proof here. 
\begin{gather*}
  \Phi_k (\mathfrak{z}) ~\text{is a double series}:\\
  \Phi_k (\mathfrak{z}) = \sum^\infty_{n=0} \mathfrak{z}^n
  \sum^\infty_{r=0} \frac{\left( \frac{\pi^2}{6k^2} (n-d)\right)^r}{r!
    \Gamma \left( \frac{5}{4} + r\right)}, |\mathfrak{z}|<1
\end{gather*}

This is absolutely convergent; so we can interchange summations and
write
\begin{align*}
  \Phi_k (\mathfrak{z}) & = \sum^\infty_{r=0} \frac{\left(
    \frac{\pi}{k\sqrt{6}}\right)^{2r}}{r! \Gamma \left( \frac{5}{2}+
    r\right)} \sum^\infty_{n=0} (n- \alpha)^r \mathfrak{z}^n\\
  & = \sum^\infty_{r=0} \frac{\left(
    \frac{\pi}{k\sqrt{6}}\right)^{2r}}{r! \Gamma \left(\frac{5}{2} + r
    \right)} \varphi_r (\mathfrak{z})
\end{align*}
where $\varphi_r (\mathfrak{z})$ is the power series
$\sum\limits^\infty_{n=0} (n- \alpha)^r \mathfrak{z}^n$. Actually it
turns out to be a rational function. $\Phi_k (\mathfrak{z})$ can be
extended over the whole plane.
$$
\varphi_r (\mathfrak{z}) = \sum^\infty_{n=0} \mathfrak{z}^n = 
\frac{1}{1- \mathfrak{z}}. 
$$

Differentiating\pageoriginale $\varphi_r (\mathfrak{z})$,
\begin{align*}
  \varphi'_r (\mathfrak{z}) & = \sum^\infty_{n=0} n (n- \alpha)^r
  \mathfrak{z}^{n-1},\\
  \mathfrak{z} \varphi'_r (\mathfrak{z}) & = \sum^{\infty}_{n=0} n(n-
  \alpha)^r \mathfrak{z}^n,\\
  \alpha \varphi_r (\mathfrak{z}) & = \sum^\infty_{n=0} \alpha (n-
  \alpha)^r \mathfrak{z}^n;
\end{align*}
so, 
$$
\mathfrak{z} \varphi'_r (\mathfrak{z}) - \alpha \varphi_r
(\mathfrak{z}) = \sum^\infty_{n=0} (n- \alpha)^{r+1} \mathfrak{z}^n =
\varphi_{r+1} (\mathfrak{z}) 
$$

This says that we con derive $\varphi_{r+1} (\mathfrak{z})$ from
$\varphi_r (\mathfrak{z})$ by rational processes and
differentiation. This will introduce no new pole; the old pole $z=1$
(pole for $\varphi_\circ (\mathfrak{z})$) will be enhanced. So
$\varphi_r (\mathfrak{z})$ is rational. Let us express the function a
little more explicitly in terms of the new variable $u=
\frac{1}{\mathfrak{z}-1}$ or $\frac{1}{u} +1=\mathfrak{z}$. Introduce
$(-)^{r+1}\varphi_r (\mathfrak{z}) = (-)^{r+1} \varphi_r (1+u)= \psi_r
(u)$, say he last equation which was a recursion formula now becomes
$$
\displaylines{\hfill
(-)^{r+2} \psi_{r+1} (u) = \left(\frac{1}{u} +1 \right) (-)^r u^2
\psi'_r (u) - \alpha (-)^{r+1} \psi_r (u) \hfill \cr
\text{because } \hfill \psi'_r (u)= (-)^{r+1} \varphi'_r \left(1+
\frac{1}{u} \right)\left( - \frac{1}{u^2}\right) = (-)^r \varphi'_r
\left( 1+ \frac{1}{u}\right) \frac{1}{u^2} \hfill \cr
\therefore \hfill \psi_{r+1} (u) = u(u+1) \psi'_r (u) + \alpha \psi_r
(u) \hfill }
$$

This is a simplified version of our recursion formula. We have a mind
to expand about the singularity $z=1$. Let us calculate the $\psi's$. 
\begin{align*}
  \psi_0 (u) & =u\\
  \psi_1(u) & = u(u+1)+ \alpha u = (1+\alpha) u+u^2\\
  \psi_2 (u) & = u(u+1)(2u+1+\alpha)+ \alpha(1+\alpha)u + \alpha u^2\\
  & = (1+ \alpha)^2u + (2 \alpha+ 3)u^3 + 2 u^3
\end{align*}
$\psi_r (u)$\pageoriginale is a polynomial of degree $r+1$ without
the constant term. The coefficients are a little complicated. If we
make a few more trials we get by induction the following:

\begin{theorem*}
  $$
  \psi_r (u) = \sum^r_{j=0} \Delta^j (\alpha + 1)^r u^{j+1},
  $$
  where $\Delta^j$ is the $j^{th}$ difference.
\end{theorem*}

By definition, 
\begin{align*}
  \Delta f(x) & = f(x+1) - f(x),\\
  \Delta^2 f(x) & = \Delta \Delta f(x) = \Delta f(x+1) - \Delta f(x)\\
  & = f(x+2) - 2f(x+1) + f(x)
\end{align*}

The binomial coefficients appear, and 
$$
\Delta^k f(x) = \sum^{k}_{\ell =0} (-)^{k-\ell} \binom{k}{\ell}
f(x+\ell) 
$$

How\pageoriginale does the formula for $\psi_r$ fit? For induction one has to make
sure that the start is good.
\begin{align*}
  \psi_0 (u) & = (\alpha + 1)^\circ u=u\\
  \psi_1 (u) & = (\alpha +1)' u' + \Delta (\alpha+1)' u^2 =
  (\alpha+1)u+u^2\\
  \psi_2 (u) & = (\alpha +1)^2 u' + \Delta (\alpha+1)^2 u^2 + \Delta^2
  (\alpha+1)^2 u^3\\
  & = (\alpha+1)^2u + \left( (\alpha+2)^2- (\alpha+1)^2 \right)u^2 +
  2u^3\\
  & = (\alpha+1)^2u + (2 \alpha+3)u^2 + 2 u^3
\end{align*}
So the start is good. We assume the formula up to $r$.
\begin{align*}
  \psi_{r+1} (u) & = \sum^r_{j=0} \left\{ (u^2 +u) (j+1) \Delta^j
  (\alpha+1)^r u^j+ \alpha \Delta^j (\alpha+1)^r u^{j+1}\right\}\\
  & = \sum^{r+1}_{j=0} \left\{ j \Delta^{j-1} (\alpha+1)^r u^{j+1} +
  (j+1+\alpha) \Delta^j (\alpha+1)^r u^{j+1}\right\}
\end{align*}

(A Seemingly negative difference need not bother us because it is
accompanied by the term $j=0$).
$$
= \sum^{r+1}_{j=0} u^{j+1} \left( j \Delta^{j-1} (\alpha+1)^r + (j+ 1+
\alpha) \Delta^j (\alpha+1)^r\right)
$$

To\pageoriginale show that the last factor is $\Delta^j (\alpha+1)^{r+1}$, we need a
side remark. Introduce a theorem corresponding to Leibnitz's theorem
on the differentiation of a product. We have
\begin{align*}
  \Delta f(x) g(x) & = f(x+1) g(x+1) - f(x) g(x)\\
  & =f(x+1) \Delta g(x) + f(x+1) g(x) - f(x) g(x)\\
  & = f(x+1) \Delta g(x) + \Delta f(x) \cdot g(x)
\end{align*}

The general rule is 
$$
\Delta^k f(x) g(x) = \sum^k_{\ell =0} \binom k \ell \Delta^{k-\ell}
f(x+\ell) \Delta^\ell g(x)
$$

This is true for $k=1$. We prove it by induction,
$$
\Delta^{k+1} f(x) g(x) = \Delta (\Delta^k f(x) g(x),
$$
and since $\Delta$ is a linear process, this is equal to 
$$
\sum^k_{\ell =0} \binom k \ell \left\{ \Delta^{k- \ell} f(x+ \ell +1)
\Delta ^{\ell +1} g(x) + \Delta ^{k+ 1- \ell} f(x+\ell) \Delta^\ell
g(x) \right\},  
$$
which becomes, on rearranging summands,
$$
\sum^{k+1}_{\ell =0} \Delta^{k+1-\ell} f(x+\ell) \Delta^{\ell} g(x)
  \left\{ \binom k \ell + \binom{k}{\ell -1}
 \right\},
$$
and the last factor is $\binom{k+1}{\ell}, \left( \binom{k}{-1} =
\binom{k}{k+1}=0 \right)$ This proves the rule.

Applying\pageoriginale this to $(\alpha+1)^r$,
$$
(\alpha+1)^{r+1}= (\alpha+1)(\alpha+1)^r; ~\text{write}~ f= \alpha+1,
g= (\alpha+1)^r,
$$
and observe that $f$ being linear permits only $0^{th}$ and $1^{st}$
differences; 
\begin{align*}
  \Delta^k (\alpha+1)^r & = \binom{k}{k-1} \Delta^{k-1} (\alpha+1)^r+
  \binom{k}{k} (\alpha+k+1) \Delta^k (\alpha+1)^r\\
  & = k \Delta^{k-1} (\alpha+1)^r + (\alpha+k+1) \Delta^k (\alpha+1)^r
  \\
  \therefore \qquad \psi_r (u) & = \sum^r_{j=0} \Delta^j (\alpha+1)^r
  u^{j+1} 
\end{align*}

We can now rewrite the $\varphi's$:
\begin{align*}
  \varphi_r (\mathfrak{z}) & = (-)^{r+1} \varphi_r (n)\\
  & = (-)^{r+1} \sum^{r}_{j=0} \Delta^j (\alpha+1)^r
  \frac{1}{(\mathfrak{z}-1)^{j+1}}
\end{align*}
$\varphi_r$ has now been defined in the whole plane.
