\chapter{Lecture}\label{part3:lec22} %% 22
\markboth{\thechapter. Lecture}{\thechapter. Lecture}

We\pageoriginale shall speak about the important sum $A_k(n)$ which
appeared in the formula for $p(n)$, defined as
$$
A_k(n) = \sideset{}{'}\sum_{h \mod k} \omega_{hk} e^{- 2 \pi i n h/k}.
$$ 

we need the explanation of the $\omega_{hk}$; they appeared as factors
in a transformation formula in the following way:
\begin{gather*}
  f\left(e^{2 \pi i \frac{h+ i \mathfrak{z}}{k}} \right) = \omega_{hk}
  \sqrt{\mathfrak{z}} e^{\frac{\pi}{12k}
    \left(\frac{1}{\mathfrak{z}}-\mathfrak{z} \right)} f\left(e^{2 \pi i
    \frac{h'+i/\mathfrak{z}}{k}} \right),\\ 
  hh'+1 \equiv 0 \pmod{k} 
\end{gather*}

Here, as we know,
$$
\displaylines{\hfill  f(x) = \frac{1}{\prod^\infty_{m=1} (1-x^m)}\hfill
  \cr
  \text{and as} \hfill \eta(\tau) = e^{\pi i \tau/12}
  \prod^\infty_{m=1} (1- e^{2 \pi i m \tau}), \hfill \cr
  \hfill f(e^{2\pi i \tau}) = e^{\pi i \tau/12} (\eta (\tau))^{-1} \hfill }
$$

We know how $\eta (\tau)$ because. $\omega_{hk}$ is something
belonging to the behaviour of the modular form $\eta(\tau)$. What is
$\omega_{hk}$ explicitly? We had a formula
$$
\eta\left(\frac{a \tau +b}{c \tau +d} \right)= \epsilon \sqrt{\frac{c
    \tau+d}{i}} \eta(\tau), c > 0,
$$
and\pageoriginale\ $\epsilon$ is just the question. Our procedure will be
to study $\epsilon $ and $\eta$ and then go back to $f$ where $\omega_{hk}$
appeared. The trick in the discussion will be that we shall not use
the product formula for $\eta(\tau)$, but the infinite series from the
pentagonal numbers theorem. This was carried out at my suggestion by
W.Fischer (Pacific Journal of Mathematics; vol. 1). However we shall
not copy him. We shall make it shorter and dismiss for our purpose all
the long and complicated discussions of Gaussian sums
$$
G(h, k) = \sum^k_{v=1} e^{2 \pi i \nu^2 h/k}
$$
which are of great interest arithmetically, having to do with law of
reciprocity to which we shall return later.

We are able to infer that a formula of the sort quoted for $\eta$
should exist from the discussion of $\mathscr{V}_1' (0/ \tau)$. We had
the formula (see hechire 14)
$$
\mathscr{V}_1 \left( 0\Big/ \frac{a \tau+b}{c \tau +d}\right) = \cdots
$$
where the right side contains a doubtful root of unity, which we could
discuss in some special cases, and by iteration in all cases. We shall
use as further basis of our argument that such a formula has been
established with the proviso $|\epsilon|=1$. We then make a statement about
$\epsilon$ and use it directly.

After all this long talk let us go to work. We had $\tau' = (h' +
i/\mathfrak{z}/k)$, $\tau = (h+ i\mathfrak{z})/k$. The question is how
is $\tau'$ produced\pageoriginale from $\tau$? It was obtained by
means of the substitution
$$
\begin{pmatrix}
  a & b\\ c & d
\end{pmatrix}= 
\begin{pmatrix}
  h' & -\frac{hh'+1}{k}\\ k & -h
\end{pmatrix}
$$

We can therefore get what we are after if we specify the formula by
these particular values.
$$
\eta \left( \frac{h' + i \mathfrak{z}}{k}\right)= \epsilon
\sqrt{\mathfrak{z}} \eta \left(\frac{h+ i \mathfrak{z}}{k} \right)
$$
with the principal value for $\sqrt{z}$. We wish to determine $\epsilon $
defined by this. We shall expand both sides and compare the
results. For expansion we do not use the infinite product but the
pentagonal numbers formula.
\begin{align*}
  \eta(\tau) & = e^{\pi i \tau/12} \sum^\infty_{\lambda =- \infty}
  (-)^\lambda e^{2 \pi i \tau \lambda (3 \lambda -1)/2}\\
  & = \sum^\infty_{\lambda =- \infty} (-)^\lambda e^{\frac{\pi i
      \tau}{12} (1+ 36 \lambda^2 - 12 \lambda)}\\
  & = \sum^\infty_{\lambda =- \infty} (-)^\lambda e^{3 \pi i \tau
    (\lambda - 1/6)^2}
\end{align*}

Most determinations of $\eta(\tau)$ make use of the infinite product formula;
the infinite series is simpler here
$$
  \eta \left( \frac{h+ i \mathfrak{z}}{k}\right) = \sum^\infty_{\lambda =- \infty}
  (-)^\lambda e^{3 \pi i \frac{h+i\mathfrak{z}}{k} (\lambda -1/6)^2}
$$

In\pageoriginale order to get the root of unity a little more clearly
exhibited, we replace $\lambda \mod 2k$.

$\lambda = 2k q + j$, $j=0,1,\ldots, 2k-1$ and $q$ runs from $-
\infty$ to $\infty$. So
$$
\eta \left( \frac{h+i \mathfrak{z}}{k}\right) =
\sum^\infty_{q=-\infty} \sum^{2k-1}_{j=0} (-)^j e^{3 \pi i \frac{h}{k}
(2 k q + j- \frac{1}{6})^2} e^{-3 \pi \frac{\mathfrak{z}}{k}(2 kq + j-
  \frac{1}{6})^2} 
$$ 

\begin{tabbing} 
  The product term in the exponent \= $= 4 kq (j- \frac{1}{6}). 3 \pi
  i \frac{h}{k}$\\
  \> $= 2 \pi i h q (6j-1)$\\
  \> = an integral multiple of $2 \pi i$
\end{tabbing}
(This is the reason why we used $\mod 2k$).

$$
\eta \left( \frac{h+ i \mathfrak{z}}{k}\right)= \sum^{2k-1}_{j=0}
(-)^j e^{3 \pi i \frac{h}{k} (j- \frac{1}{6})^2}
\sum^\infty_{q=-\infty} e^{- 12 \pi \mathfrak{z} k(q+ \frac{j- 1/6}{2k})^2}.
$$ 

We did this purposely in order to make it comparable to what we did in
the theory of $\mathscr{V}$-functions. For $\mathscr{R} t >0$, we have 
$$
\sum^\infty_{q=- \infty} e^{-\pi t (q+ \alpha)^2}= \frac{1}{\sqrt{t}}
\sum^\infty_{m=- \infty} e^{-\frac{\pi}{t} m^2} e^{2 \pi i m \alpha}
$$

This is a consequence of a $\mathscr{V}$-formula we had:
$$
e^{\pi i \tau \nu^2} \mathscr{V}_3 (\nu \tau/ \tau) =
  \sqrt{\frac{1}{\tau}} \mathscr{V}_3 \left(\nu/ - \frac{1}{\tau}\right)
$$

If\pageoriginale we write this explicitly,
$$
\mathscr{V}_3 (\nu/ \tau) = \sum^\infty_{n=- \infty} e^{\pi i \tau
  n^2} e^{2 \pi i n \nu}, 
$$
and put $i \tau =- t$,
$$
\displaylines{\hfill e^{- \pi t \nu ^2} \sum^\infty_{n=- \infty} e^{-
    \pi t n^2} e^{-2 \pi n t \nu}= \frac{1}{\sqrt{t}} \sum^\infty_{n=-
    \infty} e^{- \pi n^2/t} e^{2 \pi in \nu},\hfill \cr
  \text{or} \hfill \sum^\infty_{n=- \infty} e^{-\pi t (\nu +n)^2} =
  \frac{1}{\sqrt{t}} \sum^\infty_{n=- \infty} e^{- \frac{\pi}{t} n^2}
  e^{2 \pi i n \nu},\hfill }
$$
which is the formula quoted. We now apply this deep theorem and get
something completely new. Putting $t= 12zk$ and $\alpha=
\dfrac{j-1/6}{k}$,
$$
\eta \left(\frac{h+ i \mathfrak{z}}{k} \right)= \sum^{2k-1}_{j=0}
(-)^j e^{ 2\pi i \frac{h}{k} \left(j- \frac{1}{6} \right)^2}
\frac{1}{\sqrt{12 k \mathfrak{z}}} \sum^\infty_{m=- \infty}
e^{-\frac{\pi m^2}{12 \mathfrak{z} k}} e^{\frac{\pi i m}{k} \left(j- \frac{1}{6} \right)}
$$

We rewrite this, emphasizing the variable and exchanging the orders of
summation. Then
$$
\eta \left(\frac{h+ i \mathfrak{z}}{k} \right)= \frac{1}{2
  \sqrt{3k\mathfrak{z}}} \sum^{\infty}_{m=- \infty}
 e^{- \frac{\pi m^2}{12 k\mathfrak{z}}}
 \sum^{2k-1}_{j=0} e^{\pi i \left( j 
+ \frac{2h}{k} \left( j - \frac{1}{6}\right)^2+ \frac{m}{12k} (6j -1)\right)}
$$\pageoriginale

Let us use an abbreviation. 
$$
\displaylines{\hfill 
\eta\left(\frac{h+ i\mathfrak{z}}{k} \right)= \frac{1}{\sqrt{2 k
    \mathfrak{z}}} \sum^\infty_{m=- \infty} e^{-\frac{\pi m^2}{12
    k\mathfrak{z}}} T_{(m)},\hfill \cr
\text{where} \hfill T_{(m)} = \frac{1}{2} \sum^{2K-1}_{j=0} e^{\pi i
  \left( j+ \frac{2h}{k} (j- \frac{1}{6})^2 + \frac{m}{12k}
  (6j-1)\right)}. \hfill \cr
\hfill \eta \left( \frac{h+ i \mathfrak{z}}{k}\right) = \frac{1}{\sqrt{3
    k\mathfrak{z}}} \left\{ T_{(0)} + \sum^\infty_{m=1} e^{- \frac{\pi
    m^2}{12k \mathfrak{z}}}(T(m) + T(-m)) \right\} \hfill }
$$

This is a function in $\frac{1}{ \mathfrak{z}}$. Also
$$
\eta \left( \frac{h+ i \mathfrak{z}}{k}\right) = \frac{\epsilon^{-1}}{\sqrt{
    \mathfrak{z}}} \sum^\infty_{\lambda =- \infty} (-)^\lambda  e^{- \frac{\pi
    }{12k \mathfrak{z}} (6 \lambda -1)^2} e^{\frac{\pi ih'}{12k}(6
  \lambda -1)^2}
$$

Now $\eta\left( \frac{h+ i \mathfrak{z}}{k}\right)$ has been obtained
in two different ways. We have in both cases a power series in $e^{-
  \pi /(12\mathfrak{z}k)}=x$, both for $|x|<1$. But an analytic
function has only one power series; so they are identical. This
teaches us something. The second teaches us that by no means do all
sequences appear in the \pageoriginale exponent. Only $m^2 = (6
\lambda-1)^2$ can occur. There is no constant term in the second
expression. So $m$ has the form $|6 \lambda-1|= 6 \lambda \pm 1$, 
$\lambda >0$. Make the comparison; the coefficients are
identical. They are almost always zero. In particular
$T(0)=0$. $T(m)$ for $m$ other than $\pm 1 \pmod{6}$ also
vanish. So we have the following identification. 
$$
\frac{1}{\sqrt{3 k}}  \left( T (6 \lambda-1)+ T(-6\lambda+1)\right)=
\epsilon^{-1} (-)^\lambda e^{\frac{\pi i h'}{12k}} (6 \lambda-1)^2
$$

Realise that we have acknowledged here that a transformation formula
exists. The root of unity $\epsilon $ is independent of $\lambda$. This we
can assume but $W$. Ruscher does not. Take in particular
$\lambda=0$. Then we have for $m= \pm 1$, 
$$
\frac{1}{\sqrt{3k}} (T(-)+ T(1))= \epsilon^{-1} e^{\frac{\pi i h'}{12k}}
$$

This is proved by Fischer by using Gaussian sums. Therefore
$$
\epsilon^{-1} = \frac{e^{-\frac{\pi i h'}{12k}}}{\sqrt{3k}} \left\{
\sum^{2k-1}_{j=0} e^{\pi i(j+ \frac{3h}{k} (j- \frac{i}{6})^2)-
  \frac{6j-1}{6k}}+ \sum^{2k-1}_{j=0} e^{\pi i \left(j+ \frac{3h}{k} (j-
  \frac{1}{6})^2 + \frac{6j-1}{6k}\right)}\right\}
$$

Now $j$ matters only $\mod 2k$. We can beautify things slightly:
$$
\epsilon^{-1} = \frac{e^{-\frac{\pi i h'}{12k}+ \frac{\pi i
    h}{12k}}}{2\sqrt{3k}} \left\{ e^{\frac{\pi i}{6k}}
\sum_{j \mod 2k}  e^{\frac{\pi i}{k} (3 hj^2 + j(k-h-1))}
  + e^{-\frac{\pi i}{6k}} \sum_{j\mod 2k} e^{\frac{\pi i}{k} \left(3hj^2+ j(
    k-h-1)\right)}\right\}
$$

The sum appears complicated but will collapse nicely; however
complicated it should be a root of unity. In $A_k(n)$ the sums are
summed over $h$ and for that purpose we shall not need to compute the
sums explicitly. 

