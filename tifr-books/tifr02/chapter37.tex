\chapter{Lecture}\label{part4:lec37} %%% 37
\markboth{\thechapter. Lecture}{\thechapter. Lecture}

We\pageoriginale have finished to some extent Gaussian sums; we
treated then only in view of their occurrence in the singular series
defined as
$$
\displaylines{S^{(r)}_{(n)} = \sum^\infty_{k=1} V_k^{(r)}(n) \cr
\text{with} \hfill V_k^{(r)}(n) = V_k (n) = \sum_{\substack{h \mod
    k\\ (h, k)=1}} \left(\frac{G(h, k)}{k}\right)^r e^{- 2 \pi i
  \frac{h}{k} n},\hfill }
$$
which appeared as the principal term in the expression for the number
of representation of $n$ as the sum of $r$ squares:
$$
A_r (n) = \frac{\pi^{r/2}}{\Gamma \left(\frac{r}{2} \right)}
n^{\frac{r}{2}-1} S^{(r)}_{(n)} + O \left(n^{r/4} \right),
$$
$r \geq 5$. We did not bother to do this for lower $r$, although we
could for $r=4$, in which case we know an exact formula; but this is
another question. We consider first a fundamental property of the
singular series, viz. its expression as an infinite product. 

\heading{Fundamental Lemma.}

$$
S^{(r)}_{(n)} = \prod_p \left\{ 1+ V_p(n) + V_{p^2}(n) + V_{p^3} (n) +
\cdots \right\},
$$
$p$ prime.

We first prove the multiplicative property of $V_k(n)$: for $(k_1,
k_2)=1$, 
$$
V_{k_1} (n) V_{k_2} (n) = V_{k_1 k_2} (n)
$$

We\pageoriginale had a similar situation in connection with $A_k(n)$ for the
partition function; but there the multiplication was more
complicated. Here we have
$$
V_{k_1 k_2} (n) = \frac{1}{(k_1, k_2)^r} \sum_{\substack{h \mod k_1
    k_2\\(h, k_1 k_2)=1}} G(h, k_1 k_2)^r e^{- 2 \pi i \frac{hn}{k_1 k_2}}.
$$ 

Writing $h= k_2 h_1+ k_1 h_2$ with the conditions $(h_1, k_1)=1 =
(h_2, k_2)$, $h$, running modulo $h_1$ and $h_2$ modulo $k_2$, this
becomes
\begin{multline*}
  \frac{1}{(k_1 k_2)^r} \sum_{h_1} \sum_{h_2} G(k_2 h_1 + k_1 h_2 ,
  k_1 k_2) e^{- 2 \pi i \frac{h}{k_1 k_2}n}\\
  = \frac{1}{(k_1 k_2)^r} \sum_{h_1} \sum_{h_2} G\left((k_2 h_1 + k_1
  h_2)k_1, k_2 \right)^r \\
  G((k_2 h_1 + k_1 h_2) k_2 k_1)^r e^{- 2 \pi i (k_2 h_1 + k_1 h_2)
    \frac{n}{k_1 k_2}}
\end{multline*}
on using the multiplicativity of the Gaussian sums; and suppressing
multiples of $k_1$, $k_2$, as we may, this gives
$$
\frac{1}{k_1^r k_2^r} \sum_{h_1 \mod k_1} ~\sum_{h_2 \mod k_2} G(h_2
k_1^2, k_2)^r G(k_2^2 h_1, k_1)^r e^{-2 \pi i \frac{h_1}{k_1} n-2\pi i
  \frac{h_2}{k_2} n} 
$$

Now\pageoriginale 
$$
G (ha^2, h) = \sum_{\ell \mod k} e^{2 \pi i \frac{h}{k} a^2 \ell^2}
$$

If $(a, k)=1$, $al$ also runs modulo $k$ when $\ell$ does, so that the
right side is
$$
\sum_{n \mod k} e^{2 \pi i \frac{h}{k} m^2}= G(h, k)
$$

In our case $(k_1, k_2)=1$. So we have
\begin{gather*}
  \frac{1}{k_1^r} \sum_{h_1 \mod k_1} G(h_1, k_1)^r e^{- 2 \pi i
    \frac{h_1}{k_1} n}\frac{1}{k_2^r} \sum_{h_2 \mod k_2} G(h_2,
  k_2)^r e^{- 2 \pi i \frac{h_2}{k_2}n}\\
  = V_{k_1} (n) V_{k_2} (n)
\end{gather*}

We can then break each summand in $S^{(r)}_n$ into factors corresponding
to prime powers and multiply them again together, and the
rearrangement does not count because of absolute convergence; so
\begin{align*}
  S^{(r)}_{(n)} & = \prod_p \left\{ 1+ V_p (n) + V_{p^2}(n) +
  V_{p^3}(n) + \cdots \right\}\\
  & = \prod_p \gamma_p (n),
\end{align*}
say;\pageoriginale this is an absolutely convergent product. This
simplifies matters considerably. We have to investigate $V$ only for
those $G's$ is which prime powers appear.

We first take $p=2$. then
\begin{align*}
  \gamma_2 (n) & = 1+ V_2 (n) + V_{2^2}(n) + \cdots \\
  V_{2^\lambda} (n) & = \frac{1}{2^{\lambda r}} \sum_{\substack{h \mod
        2^\lambda\\ 2 \nmid h}} G(h, 2^\lambda)^r e^{-2\pi i h \frac{n}{2^\lambda}} 
\end{align*}

(i) $\lambda=1$ Since $G(h, 2)=0$ for odd $h$,
  $$
  V_2 (n) =0
  $$

(ii) $\lambda$ even. For $\lambda \geq 4$, 
  
\begin{align*}
    G(h, 2^\lambda) & = 2^{\frac{\lambda}{2}-1} 2(1+ i^h) =
    2^{\frac{\lambda}{2}} (1+ i^h)\\
    V_{2^\lambda} (n) & = \frac{1}{2^{\lambda r}} 2^{\lambda r/2}
      \sum_{\substack{h \mod 2 \lambda\\2\nmid h}} (1+ i^h)^r e^{- 2 \pi i
        \frac{h}{2^\lambda}n}\\
      & = \frac{1}{2^{\lambda r/2}} \left\{ \sum_{\substack{h \equiv 1
      \pmod{4}\\ h \mod 2^\lambda}} (1+ i)^r e^{-2 \pi i
        \frac{h}{2^\lambda} n}+ \sum_{\substack{h \equiv - \pmod{4}\\h
      \mod 2^\lambda}} (1- i)^r e^{-2 \pi i \frac{h}{2^\lambda} n}
      \right\}\\
      & =\frac{2^{r/2}}{2^{\lambda r/2}} \left\{ \sum_{h \equiv 1
        \pmod{4}} e^{\pi i \frac{r}{4}} e^{- 2 \pi i
        \frac{h}{2^\lambda}n} + \sum_{h \equiv -1 \pmod{4}} e^{- \pi i
      \frac{r}{4}} e^{- 2 \pi i \frac{h}{2^\lambda}n} \right\}\\
    & = \frac{1}{2^{\frac{\lambda-1}{2}r}} \left\{ e^{\pi i
        \frac{r}{4}- 2 \pi i \frac{r}{2^\lambda}} \sum_{s \mod
        2^{\lambda-2}} e^{- 2 \pi i \frac{s}{2^{\lambda-2}}n} +
      \right.\\
      & \hspace{3cm}\left. + e^{- \pi i \frac{r}{4}+ 2 \pi i \frac{r}{2^\lambda}}
      \sum_{s \mod 2^{\lambda -2}} e^{ -2 \pi i \frac{s}{2^{\lambda
            -2}}n} \right\}\\
     &  = 0, ~\text{if}~ 2^{\lambda-2} +n;\\
     & \quad  \frac{2^{\lambda-2}}{2^{\frac{\lambda-1}{2}r}} \cos \left(\pi
      \frac{r}{4} - 2 \pi \frac{\nu}{4} \right), ~\text{if}~
      2^{\lambda-2}/n, n= 2^{\lambda-2}.\nu\\
     \text{i.e.,} &  \frac{1}{2^{(\lambda-1) (\frac{r}{4}-1)}} \cos
     \frac{\pi}{4} (2 \nu -r)
\end{align*}\pageoriginale

Hence, for $\lambda$ even, $\lambda\geq 4$,
\begin{equation*}
V_{2^\lambda} (n) =
\begin{cases}
  \quad 0, & \text{if}~ 2^{\lambda -2} +n;\\
  \frac{\cos \frac{\pi}{4} (2 \nu - r)}{2^{(\lambda-1)(\frac{r}{2}
      -1)}}, & \text{if}~ 2^{\lambda -2}. \nu =n.
\end{cases}\tag{*}\label{part4:lec37:eq*}
\end{equation*}

(iii) \textit{$\lambda$ odd, $\lambda \geq 3$.}
\begin{align*}
  G(h, 2^\lambda) &= 2 G(h, 2^{\lambda-2})= 2^{\frac{\lambda -3}{2}}
  G(h, 2^3)\\
  & = 2^{\frac{\lambda-3}{2}} 4 e^{\pi i h/4} =
  2^{\frac{\lambda+1}{2}} e^{\pi i h/4}\\
  V_{2^\lambda} (n) & = \frac{1}{2^{\lambda r}} 2^{\frac{\lambda+1}{2}}
  \sum_{\substack{h \mod 2^\lambda\\2\nmid h}} e^{\pi i h \frac{r}{4}} e^{-
  2 \pi i \frac{h}{2^\lambda} n},
\end{align*}
or,\pageoriginale writing \quad $h= 8 s+t$, $t= 1, 3, 5, 7$,
\begin{align*}
  & = \frac{1}{2^{\frac{\lambda -1}{2}r}} \sum_t
  \sum^{2^{\lambda-3}}_{s=1} e^{\pi i t r/4} e^{-2 \pi i (8 s+t)
    \frac{n}{2^\lambda}}\\
  & = \frac{1}{2^{\frac{\lambda-1}{2} r}} \sum_t e^{\pi i t r/4 - 2
    \pi i t n/2 \lambda } \sum^{2^{\lambda-3}}_{s=1} e^{- 2 \pi i s n
    /2^{\lambda-3}}\\
  & = 0, ~\text{if}~ 2^{\lambda-3}\nmid n.
\end{align*}

If, however, $2^{\lambda-3}|n$, $n= 2^{\lambda -3}. \nu$, this is 
\begin{align*}
  & \frac{2^{\lambda-3}}{2^{\frac{\lambda-1}{2}r}} \sum_t e^{\pi i
    t\nmid 4
    (r-\nu)} = o, ~\text{if}~ 4 /(r-\nu);\\
  & \frac{2^{\lambda-1}}{2^{\frac{\lambda-1}{2}r}} e^{\pi i
    (r-\nu)\mid 4},
  ~\text{if}~ 4/(r - \nu)\\
  \text{i.e.,} \quad & \frac{1}{2^{(\lambda-1)(\frac{r}{2}-1)}} \cdot
  (-)^{\frac{\nu -r}{4}}.
\end{align*}

Hence\pageoriginale for $\lambda$ odd, $\lambda \geq 3$,
\begin{equation*}
  V_{2^\lambda} (n) = 
  \begin{cases}
    \quad 0, & \text{if}~ 2^{\lambda -3} \nmid n;\\
    \quad 0, & \text{if}~ 2^{\lambda -3}. \mid n, n=2^{\lambda -3}\nu, 4
    \nmid (\nu - r);\\
    \frac{(-)^{\frac{\nu -r}{4}}}{2^{(\lambda -1)(\frac{r}{2}-1)}}, &
    \text{if}~ 2^{\lambda -3}\mid n, 4\mid (\nu-r)
  \end{cases} \tag{**}\label{part4:lec37:eq**}
\end{equation*}

Now, given $n$, only a finite number of powers of 2 can divide it. So
the situation $2^{\lambda -3}/n$ will occur sometime or the other, so
that $\gamma_2(n)$ is always a finite sum.
\begin{align*}
  |\gamma_2 (n) -1| & \leq \sum^\infty_{\lambda =2}
  \frac{1}{2^{(\lambda -1)(\frac{r}{2}-1)}}\\
  & = \frac{1}{2^{\frac{r}{2}-1}} \cdot \frac{1}{1-1/2
    \frac{r}{2}-1}\\
  & = \frac{1}{2^{r/2-1}-1};
\end{align*}
and this is valid for $r \geq 3$. so the singular series behaves much
better than we expected.

