\chapter{Lecture}\label{part1:lec5}
\markboth{\thechapter. Lecture}{\thechapter. Lecture}

Let\pageoriginale\  us consider some applications of formal differentiation of power
series. Once again we start from the pentagonal numbers theorem:
\begin{align*}
  \prod^\infty_{m=1} (1-x^m) & = \sum^\infty_{\lambda =- \infty}
  (-)^\lambda x^{\lambda (3 \lambda -1)/2}\\
  & = \sum^\infty_{\lambda=-\infty} (-)^\lambda x^{\omega_\lambda},
  \tag{1}\label{part1:lec5:eq1} 
\end{align*}
with $\omega_\lambda= \dfrac{\lambda(3 \lambda-1)}{2}$. Taking the
logarithmic derivative - and this can be done piecemeal-
$$
\sum^\infty_{m=1} \frac{-mx^{m-1}}{1-x^m} =
\frac{\sum\limits^\infty_{\lambda=-\infty} (-)^\lambda \omega_\lambda
  x^{\omega_{\lambda-1}}}{\sum\limits^\infty_{\lambda=-\infty}
  (-)^\lambda x^{\omega_\lambda}} 
$$

Multiplying both sides by $x$, 
\begin{equation*}
\sum^\infty_{m=1} \frac{-mx^{m}}{1-x^m} =
\frac{\sum\limits^\infty_{\lambda=-\infty} (-)^\lambda \omega_\lambda
  x^{\omega_\lambda}}{\sum\limits^\infty_{\lambda=-\infty}
  (-)^\lambda x^{\omega_\lambda}} \tag{2} \label{part1:lec5:eq2}
\end{equation*}

The left side here is an interesting object called a Lambert series, with a
structure not quite well defined; but it plays some role in number
theory. Let us transform the Lambert series into a power series; it
becomes
$$
- \sum^\infty_{m=1} m \sum^\infty_{k=1} x^{km} = -
\mathop{\sum\sum}^\infty_{k_1 m=1} m x^{km},
$$
and these are all permissible power series, because though there are
infinitely many of them, the inner ones begin with later and later
terms. 

Rearranging,\pageoriginale\  this gives
\begin{align*}
  - \sum_{n=km} \sum^\infty_{m=1} m x^n & = - \sum^\infty_{n=1} x^n
  \sum_{m/n} m\\
  & = - \sum^\infty_{n=1} \sigma (n) x^n,
\end{align*}
where $\sigma(n)$ denotes the sum of the divisors of $n$, $\sigma(n) =
\sum\limits_{d|n} d$.

(Let us study $\sigma (n)$ for a moment.
$$
\sigma(1) =1, \sigma(2)=3, \sigma_3=4, \sigma(5) =6; ~\text{indeed}~
\sigma(p) = p+1
$$  
for a prime $p$. And $\sigma (n) = n+1$ implies that $n$ is
prime. $\sigma(n)$ is not too big; there can be at most $n$ divisions 
of $n$ and so roughly $\sigma(n) =O (n^2)$. In fact it is known that
$\sigma(n) = O (n^{1+epsilon})$, $\in > 0$, that is, a little larger than the
first power. We shall however not be studying $\sigma(n)$ in detail).

Equation (\ref{part1:lec5:eq2}) can now be rewritten as 
$$
\sum^\infty_{n=1} \sigma(n) x^n
\sum^\infty_{\lambda=-\infty}(-)^\lambda x^{\omega_\lambda} =
\sum^\infty_{\lambda=-\infty}(-)^{\lambda-1} \omega_\lambda x^{\omega_\lambda}
$$

Let us look for the coefficient of $x^m$ on both sides. Remembering
that the first few $\omega'_\lambda s$ are 0, 1, 2, 5, 7, 12, 15
$\cdots$, the coefficient of $x^m$ on the left side is 
$$
\sigma(m) - \sigma(m-1) - \sigma(m-2) + \sigma(m-6)+ \sigma(m-7) -- ++
    \cdots 
$$ 

On the right side the coefficient is $0$ most frequently, because the
pentagonal numbers are rather rare, and equal to $(-)^{\lambda-1}
\omega_\lambda$ exceptionally, when $m= \omega_\lambda$.
$$
\sigma(m) - \sigma(m-1) - \sigma(m-2) ++ -- \cdots = 
\begin{cases}
  \quad 0 & \text{usually},\\
  (-)^{\lambda-1}\omega_\lambda & \text{for}~ m= \omega_\lambda.
\end{cases}
$$

We\pageoriginale\  now single out $\sigma(m)$.

We may write
$$
\sigma(m) = \sum_{0 < \omega_\lambda < m} (-)^{\lambda-1} \sigma(m-
\omega_\lambda) + 
\begin{cases}
  0 & \text{usually},\\
  (-)^{\lambda-1} \omega_\lambda & \text{for}~ m= \omega_\lambda 
\end{cases}
$$

This is an additive recursion formula for $\sigma(n)$. We can make it
even more striking. The inhomogeneous piece on the right side is a
little annoying. $\sigma(m-m)$ can occur on the right side only for
$m=\omega_\lambda$; $\sigma(0)$ does not make sense; however,
\textit{for our purpose} let us define
$$
\sigma(m-m)=m.
$$

Then $\sigma(\omega_\mu- \omega_\mu)= \omega_\mu$, and the previous
formula can now be written uninterruptedly as 
\begin{equation*}
  \sigma(m) = \sum_{0 < \omega_\lambda \leq m} (-)^{\lambda-1} \sigma
  (m- \omega_\lambda)\tag{3}\label{part1:lec5:eq3}
\end{equation*}

We have proved earlier that
\begin{equation*}
  p(m) = \sum_{0 < \omega_\lambda \leq m} (-)^{\lambda-1} p
  (m- \omega_\lambda)\tag{4}\label{part1:lec5:eq4}
\end{equation*}
which is a formula completely identical with (\ref{part1:lec5:eq3}). Here $p(m-m)=
p(0)=1$. It is extraordinary that $\sigma(m)$ and $p(m)$ should have
the same recursion formula, differing only in the definition of the
term with $n=0$. This fact was noted by Euler. In fact $p(m)$ is
increasing monotonically, while the growth of $\sigma(m)$ is more
erratic. 

There are more relations between $p(m)$ and $\sigma(m)$. Let us start
again with the identity
\begin{equation*}
  \prod^\infty_{m=1} (1-x^m) \sum^\infty_{m=0} p(m) x^m
  =1\tag{5}\label{part1:lec5:eq5} 
\end{equation*}
 
We\pageoriginale\  know that for a pair of power series $A$, $B$ such that $AB=1$, on
taking logarithmic derivatives, we have $\dfrac{A'}{A} + \dfrac{B'}{B}
=0$ or $\dfrac{A'}{A} =- \dfrac{B'}{B}$. So from (\ref{part1:lec5:eq5}),
$$
\displaylines{\hfill \sum^\infty_{m=1} \sigma(m) x^m =
  \frac{\sum\limits^\infty_{n=0} n p(n) x^n}{\sum\limits^\infty_{n=0}
    p(n) x^n}, \hfill \cr
\text{or} \hfill \sum^\infty_{m=1} \sigma(m) x^m \sum^\infty_{k=0}
p(k) x^k = \sum^\infty_{n=0} np(n) x^n.\hfill }
$$

Comparing coefficients of $x^n$,
$$
np(n) = \sum_{m+ k=n} \sigma (m) p(k),
$$
or more explicitly,
\begin{equation*}
  np(n) = \sum^\infty_{m=1} \sigma(m) p(n-m) \tag{6}\label{part1:lec5:eq6}
\end{equation*}

This is a bilinear relation between $\sigma(n)$ and $p(n)$. This can
be proved directly also in the following way. Let us consider all the
partitions of $n$; there are $p(n)$ such:
\begin{align*}
  n & = h_1 + h_2 + \cdots \\
  n & = k_1 + k_2 + \cdots \\
  n & = \ell_1 + \ell_2 + \cdots \\
   & \dots \dots \dots 
\end{align*}

Adding up, the left side gives $np(n)$. Let us now evaluate the sum of
the right sides. Consider a particular summand $h$ and let us look for
those partitions in which $h$ figures. These are $p(n-h)$ partitions
in which $h$ occurs at least once, $p(n- 2h)$ in which $h$ occurs at
least twice; in general, $p(n- rh)$ in which $h$ occurs at least $r$
times. Hence the number of those partitions which contain $h$
\textit{exactly} $r$\pageoriginale\  times is $p (n-nh)- p(n-\overline{n+1} h)$. Thus the
number of times $h$ occurs in all partitions put together is 
$$
\sum_{nh\leq n} n\left\{ p(n-nh) - (n-\overline{n+1} h)\right\}
$$

Hence the contribution from these to the right side will be
$$
h\sum_{nh\leq n} n\left\{ p(n-nh) - (n-\overline{n+1} h)\right\}= h
\sum_{nh\leq n} p(n-nh)
$$
on applying partial summation. Now summing over all summands $h$, the
right side becomes
$$
\sum_{h} h \sum_{nh \leq n} p(n-nh) =\ \sum_{n/m} \frac{m}{n} \sum_{m
  \leq n} p(n-m),
$$
on putting $rh=m$; and this is 
$$
\sum_{m \leq n} p(n-m) \sum_{n.m} \frac{m}{n} = \sum^n_{m=1} p(n-m)
  \sigma(m). 
$$

Let us make one final remark.

Again from the Euler formula,
\begin{align*}
  \sum^\infty_{m=1} \sigma(m) x^m & = \frac{\sum\limits^\infty_{\lambda
    =-\infty} (-)^{\lambda-1} \omega_\lambda x
    ^{\omega_\lambda}}{\sum\limits^\infty_{\lambda =-\infty}
    (-)^\lambda x^{\omega_\lambda}}\\
  & = \frac{\sum\limits^\infty_{\lambda
    =-\infty} (-)^{\lambda-1} \omega_\lambda x
    ^{\omega_\lambda}}{\prod\limits^\infty_{m=1} (1- x^m)}\\
  &= \sum\limits^\infty_{\lambda
    =-\infty} (-)^{\lambda-1} \omega_\lambda x^{\omega_\lambda}
  \sum^\infty_{m=0} p(m) x^m
\end{align*}

Comparing\pageoriginale\  the coefficients of $x^m$ on both sides,

\begin{align*}
  \sigma(m) & = p(m) - 1\cdot p(m-1) - 2\cdot p(m-2) + 5 \cdot p(m-5)\\
&\qquad + 7\cdot p(m-7) - + \cdots\\
  & = \sum_{0 \leq \omega_\lambda \leq m} (-)^{\lambda-1}
  \omega_\lambda p(m-\omega_\lambda)
\end{align*}

This last formula enables us to find out the sum of the divisors
provided that we know the partitions. This is not just a curiosity; it
provides a useful check on tables of partitions computed by other
means. 

We go back to power series leading up to some of Ramanujan's
theorems. Jacobi introduced the products
$$
\prod^\infty_{n=1} (1-x^{2n})(1+ \mathfrak{z}x^{2n-1}) (1+
\mathfrak{z}^{-1} x^{2n-1}).
$$

This is a power series in $x$; though these are infinitely many
factors they start with progressively higher powers. The coefficients
this time are not polynomials in $z$ but from the field $R(z)$, the
field of rational functions of $z$, which is a perfectly good
field. Let us multiply out and we shall have a very nice surprise. The
successive coefficients are:

\begin{tabbing}
  1 \quad \= \\
  $x$ \>:  $\mathfrak{z}+  \mathfrak{z}^{-1}$ \qquad (note that this
  is unchanged when $ \mathfrak{z}\to  \mathfrak{z}^{-1}$)\\
  $x^2$ \> : $(1+1)=0$\\
  $x^3$ \> : $(\mathfrak{z} + \mathfrak{z}^{-1}-  \mathfrak{z}-
  \mathfrak{z}^{-1})=0$\\
  $x^4$ \> : $(-1-1+ \mathfrak{z}^2+1+1+ \mathfrak{z}^{-2})=
  \mathfrak{z}^2+ \mathfrak{z}^{-2}$ (again unchanged when
  $\mathfrak{z} \to  \mathfrak{z}^{-1}$)\\
  \> $\dots \dots \dots \dots $
\end{tabbing}

We\pageoriginale\  observe that non-zero coefficients are associated only with square
exponents. We may therefore provisionally write 
\begin{align*}
  \prod^\infty_{n=1} (1-x^{2n}) (1+ \mathfrak{z}x^{2n-1})(1+
  \mathfrak{z}^{-1} x^{2n-1}) & = 1+ \sum^\infty_{k=1}(\mathfrak{z}^k+
  \mathfrak{z}^{-k}) x^{k^2}\\
  & = \sum^\infty_{k=-\infty} \mathfrak{z}^k x^{k^2} \tag{7}\label{part1:lec5:eq7}
\end{align*}
(with the terms corresponding to $\pm k$ folder together). This is a
$\mathcal{V}$- series; only quadratic exponents occur.

We shall now prove the identity (\ref{part1:lec5:eq7}). But we have got to be
careful. Consider the polynomial
$$
\Phi_N (x, \mathfrak{z}) = \prod^N_{n=1}
(1-x^{2n})(1+\mathfrak{z}x^{2n-1}) (1+\mathfrak{z}^{-1} x^{2n-1})
$$

This consists of terms $\mathfrak{z}^j x^k$ with $-N \leq j \leq N$,
$0 \leq k \leq N (N+1) + 2N^2= 3N^2+N$. We can rearrange with respect
to powers of $z$. The coefficients are now polynomials in $x$. $z$ and
$z^{-1}$ occur symmetrically.
$$
  \Phi_N (x, \mathfrak{z}) = C_\circ (x) +
  (\mathfrak{z}+\mathfrak{z}^{-1}) C_1 (x) + (\mathfrak{z}^2+
  \mathfrak{z}^{-2}) C_2(x)+ \cdots + (\mathfrak{z}^N +
  \mathfrak{z}^{-N}) C_N (x).
$$

Let us calculate the $C's$. It is cumbersome to look for $C_\circ$,
for so many cancellations may occur. It is easier to calculate
$C_N$. Since the highest power of $z$ can occur only from the terms
with the highest power of $x$, we have
\begin{align*}
  C_N (x) & = \prod^N_{n=1} (1-x^{2n}) \times x^{1+3+\cdots + (2N-1)}\\
  & = x^{N^2} \prod^N_{n=1} (1-x^{2n})
\end{align*}

Now\pageoriginale\  try to get a recursion among the $C's$. Replacing $z$ by $zx^2$,
we get
$$
\Phi_N (x, \mathfrak{z}x^2) = \prod^N_{n=1} (1-x^{2n})
(1+\mathfrak{z}x^{2n+1}) (1+\mathfrak{z}^{-1} x^{2n-3}).
$$

Compare $\Phi_N (x, \mathfrak{z}x^2)$ and $\Phi_N (x, \mathfrak{z})$; these are
related by the equation
$$
\Phi_N (x, \mathfrak{z}x^2) (1+\mathfrak{z}x)(1+\mathfrak{z}^{-1}
x^{2N-1}) = \Phi_N (x,
\mathfrak{z})(1+\mathfrak{z}x^{2n+1})(1+\mathfrak{z}^{-1} x^{-1})
$$

The negative power in the last factor on the right is particularly
disgusting; to get rid of it we multiply both sides by $xz$, leading
to
\begin{multline*}
\qquad \Phi_N (x, \mathfrak{z}x^2)(x \mathfrak{z}+ x^{2N}) = \Phi_N (x,
\mathfrak{z}) (1+\mathfrak{z}x^{2N+1}),\\
\text{or} \qquad (1+ \mathfrak{z}x^{2N+1})(C_\circ (x) +
  (\mathfrak{z}+\mathfrak{z}^{-1}) C_1 (x) + \cdots + (\mathfrak{z}^N+
  \mathfrak{z}^{-N}) C_N (x))\qquad \\
  = (x \mathfrak{z}+ x^{2n}) (C_\circ (x) + (\mathfrak{z}x^2+
  \mathfrak{z}^{-1} x^{-2}) C_1 (x)+\\
  + (\mathfrak{z}^2 x^4+ \mathfrak{z}^{-2} x^{-4})C_2 (x)+\cdots +
  (\mathfrak{z}^N x^{2N} + \mathfrak{z}^{-N} x^{-2N}) C_N (x) )
\end{multline*}

These are perfectly harmless polynomials in $x$; we may compare
coefficients of $\mathfrak{z}^k$. Then
$$
\displaylines{\hfill C_k (x) + C_{k-1} (x) x^{2N+1}= C_k (x) x^{2k+2N}
  + x^{2k-1} C_{k-1}(x),\hfill \cr
  \text{or} \hfill C_k(x) (1-x^{2N+2k}) = C_{k-1} (x) x^{2k-1}
  (1-x^{2N-2k+2}) \hfill }
$$
(We proceed from $C_k$ to $C_{k-1}$ since $C_N$ is already known).
$$
C_{k-1} (x) = \frac{x^{-2k+1}(1- x^{2N+2k})}{1-x^{2N-2k+2}} C_k(x)
$$

Since\pageoriginale\  $C_N (x) = x^{N^2} \prod\limits^N_{n=1} (1- x^{2n})$, we have in
succession 
\begin{align*}
  C_{N-1} (x) & = x^{N^2-2N+1} \frac{1-x^{4N}}{1-x^2} \prod^N_{n=1}
  (1-x^{2n})\\
  & = x^{(N-1)^2} \prod^N_{n=2} (1-x^{2n}) \cdot (1-x^{4N});\\
  C_{N-2}(x) & = x^{(N-2)^2} \prod^N_{n=3} (1-x^{2n})\cdot
  (1-x^{4N})(1-x^{4N-2})\\
  & \qquad \dots \dots \dots \dots
\end{align*}

In general,
$$
C_{N-j} (x) = x^{(N-j)^2} \prod^N_{n=j+1} (1-x^{2n}) \prod^{j-1}_{m=0} (1-x^{4N-2m})
$$
or, with $j= N-n$,
\begin{equation*}
  C_n (x) = x^{n^2} \prod^N_{n=N-n+1} (1-x^{2n}) \prod^{N-n-1}_{m=0}
  (1-x^{4N-2m}) \tag{8}\label{part1:lec5:eq8}
\end{equation*}

Equation (\ref{part1:lec5:eq8}) leads to some congruence relations. The lowest terms of
$C_n (x)$ have exponent
$$
n^2 +2 (N-n+1) = 2N + (n^2-2n+1) +1 \geq 2N+1
$$

Hence 
\begin{equation*}
  C_n (x) \equiv x^{k^2} \pmod{x^{2N+1}} \tag{9}\label{part1:lec5:eq9}
\end{equation*}

From the original formula,
\begin{align*}
  \Phi_N (x, \mathfrak{z})& = \prod^N_{n=1} (1-x^{2n})
  (1+\mathfrak{z}x^{2n+1})(1+\mathfrak{z}^{-1} x^{2n-1})\\
  & \equiv 1+ (\mathfrak{z}+\mathfrak{z}^{-1}) x+ (\mathfrak{z}^2 +
  \mathfrak{z}^{-2}) x^4 + \cdots \pmod{x^{2N+1}}\\
  & \equiv \sum^\infty_{k=-\infty} \mathfrak{z}^k x^{k^2} \pmod{x^{2N+1}},
\end{align*}
since\pageoriginale\  the infinite series does not matter, the higher powers being
absorbed in the congruence. Hence
$$
\Phi_N (x, \mathfrak{z}) \equiv \prod^\infty_{n=1} (1-x^{2n})
(1+\mathfrak{z}x^{2n-1})(1+\mathfrak{z}^{-1} x^{2n-1}) \pmod{x^{2N+1}}
$$

The new terms $x^{2N+2}, \ldots$, are absorbed by $\mod x^{2N+1}$. We
have
$$
\prod^\infty_{n=1}
(1-x^{2n})(1+\mathfrak{z}x^{2n-1})(1+\mathfrak{z}^{-1} x^{2n-1})
\equiv \sum^\infty_{k=-\infty} \mathfrak{z}^k x^{k^2} \pmod{x^{2N+1}}
$$

Thus both expansions agree as far as we wish, and this is what we mean
by equality of formal power series. Hence we can replace the
congruence by equality, and Jacobi's identity (\ref{part1:lec5:eq7}) is proved. 

As an application of this identity, we shall now give a new proof of
the pentagonal numbers theorem. We replace $x$ by $y^3$, as we could
consistently in the whole story; only read modulo $y^{6N+3}$. Then we
have
$$
\prod^\infty_{n=1}
(1-y^{6n})(1+\mathfrak{z}y^{6n-3})(1+\mathfrak{z}^{-1}y^{6n-3})
=\sum^\infty_{k=-\infty} \mathfrak{z}^k y^{3k^2}
$$

We now do something which needs some justification. Replace $z$ by
$-y$. This is something completely strange, and would interfere
seriously with our reasoning. For $\Phi_N (y^3, \mathfrak{z})$ we had
congruences modulo $y^{6N+3}$. If we replaced $z$ by $y^3$ nobody
could forbid that. Since $z$ occurs in negative powers, the powers of
$y$ might be lowered too by as much as $N$. We obtain polynomials in
$y$ alone on both sides, but true modulo $y^{5N+3}$, because we\pageoriginale\  may
have lowered powers of $y$. With this proviso it is justified to
replace $z$ by $-y$; so that ultimately we have
$$
\prod^\infty_{n=1} (1-y^6)(1-y^{6n-2}) (1-y^{6n-4})= 
\sum^\infty_{k=-\infty} (-)^k y^{3k^2+k} \pmod{y^{5N+3}}
$$

We can carry over the old proof step by step. Since we now have only
even powers of $y$, this leads to
$$
\prod^\infty_{m=1} (1-y^{2m})= \sum^\infty_{k=-\infty} (-)^k y^{k(3k+1)}
$$

These are actually power series in $y^2$. Set $y^2=x$, then
$$
\prod^\infty_{m=1} (1-x^m) = \sum^\infty_{k=-\infty} (-)^K x^{k(3k+1)/2}
$$
which is the pentagonal numbers theorem.

