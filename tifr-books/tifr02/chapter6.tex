\chapter{Lecture}\label{part1:lec6}
\markboth{\thechapter. Lecture}{\thechapter. Lecture}

In\pageoriginale  the last lecture we used the Jacobi formula:

\begin{equation*}
  \prod^\infty_{n=1} (1-x^{2n})
  (1+\mathfrak{z}x^{2n-1})(1+\mathfrak{z}^{-1} x^{2n-1}) =
  \sum^\infty_{k=-\infty} \mathfrak{z}^k x^{k^2} \tag{1}\label{part1:lec6:eq1}
\end{equation*}
to give a new proof of Euler's pentagonal numbers theorem. We proceed
to give another application. We observe again that the right side of
(\ref{part1:lec6:eq1}) is a power series in $x$; we cannot do anything about the $z'$s
and no formal differentiation can be carried out with respect to
$z$. Let us make the substitution $\mathfrak{z} \to -
\mathfrak{z}x$. This again interferes greatly with our variable
$x$. Are we entitled to do this? Let us look back into our proof of
(\ref{part1:lec6:eq1}). We started with a curtailed affair 
$$
\Phi_N (x, \mathfrak{z}) = \prod^\infty_{n=1} (1-x^{2n}) (1+
\mathfrak{z}x^{2n-1}) (1+\mathfrak{z}^{-1} x^{2n-1}) 
$$  
and this was a polynomial of the proper size and everything went
through. When we replace $z$ by $-zx$ and multiply out, the negative
powers might accumulate and we might be destroying $x^N$ possibly;
nevertheless the congruence relations would be true this time modulo
$x^{N+1}$ instead of $x^{2N+1}$ as it was previously; but this is all
we went. So the old proof can be reproduced step by step and every
thing matches modulo $x^{N+1}$. (Let us add a side remark. In the
proof of (\ref{part1:lec6:eq1}) we had to replace $z$ by $zx^2$ - and this was the
essential step in the proof. We cannot do the same here as this would
lead to congruences $\mod x$ only. Before we had the congruences we
had identities and there we could carry out any substitution. Then we
adopted\pageoriginale  a new point of view and introduced congruences; and that step
bars later the substitution $\mathfrak{z} \to \mathfrak{z}x^2$.

So let us make the substitution $\mathfrak{z} \to -\mathfrak{z}x$
without further compunction. This gives us 
$$
\prod^\infty_{n=1} (1-x^{2n})(1-\mathfrak{z}x^{2n})
(1-\mathfrak{z}^{-1}x^{2n-2})= \sum^\infty_{k=-\infty} (-)^{k}
\mathfrak{z}^k x^{k^2+k}
$$

This is not nicely arranged. There appears an extraordinary term
without $x$- corresponding to $n=1$ in the last factor on the left
side; let us keep this apart. Also on the right side the exponent of
$x$ is $k(k+1)$, so that every number occurs twice; let us keep these
two pieces together. We then have
\begin{align*}
  (1-\mathfrak{z})^{-1} & \prod^\infty_{n=1}
  (1-x^{2n})(1-\mathfrak{z}x^{2n}) (1-\mathfrak{z}^{-1} x^{2n})\\
  & = \sum^\infty_{k=0} (-)^k \mathfrak{z}^k x^{k(k+1)} +
  \sum^\infty_{k=0} (-)^{-k-1} \mathfrak{z}^{-k-1} x^{k(k+1)}\\
  & \qquad \text{(where in the second half we have replaced $k$ by
    $-k-1$),}\\
  & = \sum^\infty_{k=0} (-)^k x^{k(k+1)}
  (\mathfrak{z}^k-\mathfrak{z}^{-k-1})\\
  & = \sum^\infty_{k=0} (-)^k x^{k(k+1)} \mathfrak{z}^k
  (1-\mathfrak{z}^{-2k-1})\\
  & = \sum^\infty_{k=0} (-)^k x^{k(k+1)}
  \mathfrak{z}^k (1-\mathfrak{z}^{-1})  (1+ \mathfrak{z}^{-1} +
  \mathfrak{z}^{-2}+ \cdots + \mathfrak{z}^{-2k})
\end{align*}

We now have an infinite series in $x$ equal to another. Now recollect
that our coefficients are from the field $R(z)$ which has no zero
divisors. So we may cancel $1-z^{-1}$ on both sides; this is a
non-zero factor\pageoriginale  in $R(z)$ and has nothing to do with
differentiation. This leads to
$$
\prod^\infty_{n=1} (1-x^{2n})
(1-\mathfrak{z}x^{2n})(1-\mathfrak{z}^{-1} x^{2n}) = \sum^\infty_{k=0}
(-)^k x^{k(k+1)} (\mathfrak{z}^k + \mathfrak{z}^{k-1} + \cdots +
\mathfrak{z}^{-k}). 
$$

In the field $R(z)$ we can replace $z$ by 1. We can do what we like in
the field and that is the essence of the power series method. So
putting $z=1$,
$$
\prod^\infty_{n=1} (1-x^{2n})^3 = \sum^\infty_{k=0} (-)^k x^{k(k+1)} (2k+1).
$$

This is a power series in $x^2$; give it a new name, $x^2=y$. Then
\begin{equation*}
  \prod^\infty_{n=1} (1-y^n)^3 = \sum^\infty_{k=0} (-)^k (2k+1)
  y^{k(k+1)/2} \tag{2}\label{part1:lec6:eq2}
\end{equation*}

This is a very famous identity of Jacobi, originally proved by him by
an altogether different method using the theory of functions. Let us
juxtapose it with the Euler pentagonal formula:
\begin{equation*}
  \prod^\infty_{n=1} (1-y^n) = \sum^\infty_{\lambda=-\infty}
  (-)^\lambda x^{\lambda(3\lambda -1)/2} \tag{2a}\label{part1:lec6:eq2a}
\end{equation*}

Let us proceed to yet another application of the triple product
formula; we shall obtain some of Ramanujan's formulas. Taking away the
first part of the triple product formula we have
\begin{equation*}
  \prod^\infty_{n=1} (1+\mathfrak{z}x^{2n-1})(1+\mathfrak{z}^{-1}
  x^{2n-1})  = \sum^\infty_{k=-\infty} \mathfrak{z}^k x^{k^2}
  \frac{1}{\prod\limits^\infty_{n=1}(1-x^{2n})}  \tag{3}\label{part1:lec6:eq3}
\end{equation*}

The\pageoriginale  second part on the right side here is of interest, because it is
the generating function of the partition. We had earlier the formula
\begin{equation*}
  \begin{aligned}
    \prod^\infty_{n=1} (1+ \mathfrak{z}x^{2n-1}) & = \sum^\infty_{m=0}
    \mathfrak{z}^m C_m (n),\\
    C_m(x) & = \frac{x^{m^2}}{(1-x^2)\cdots (1-x^{2m})} 
  \end{aligned}\tag{4}\label{part1:lec6:eq4}
\end{equation*}
and these are permissible power series, beginning with later and later
powers of $x$, and so the right side of (\ref{part1:lec6:eq4}) makes sense, as a formal
power series in $x$.

Substituting (\ref{part1:lec6:eq4}) in (\ref{part1:lec6:eq3}), we have
\begin{equation*}
  \sum^\infty_{r=0} \mathfrak{z}^n C_r (x) \sum^\infty_{s=0}
  \mathfrak{z}^{-s} C_s (x) = \sum^\infty_{k=-\infty} \mathfrak{z}^k
  x^{k^2} \frac{1}{\prod\limits^\infty_{n=1}(1-x^{2n})}
  \tag{5}\label{part1:lec6:eq5}  
\end{equation*}

We can compare $z^O$ on both sides for, for very high $x^N$ the left
side will contain only finitely many terms and all others will
disappear below the horizon; we can also add as many terms as we
wish. So equating coefficients of $z^O$, we have
$$
\displaylines{\hfill 
\sum^\infty_{r=0} C_r(x) C_r (x) = \frac{1}{\prod\limits^\infty_{n=1}
  (1- x^{2n})}, \hfill \cr
\text{or} \hfill \sum^\infty_{r=0} \frac{x^{2r^2}}{(1-x^2)^2\cdots
  (1-x^{2n})^2} = \frac{1}{\prod\limits^\infty_{n=1} (1-x^{2n})}
\hfill }
$$

We have even powers of $x$ consistently on both sides; so replace
$x^2$ by $y$, and write down the first few terms
explicitly:
\begin{align*}
  1+ \frac{y}{(1-y)^2} + \frac{y^4}{(1-y)^2(1-y^2)^2}
  & +\frac{y^9}{(1-y)^2(1-y^2)^2(1-y^3)^2} + \cdots\\ 
  & = \frac{1}{\prod\limits^\infty_{n=1}(1-y^n)} \tag{6}\label{part1:lec6:eq6}  
\end{align*}

This\pageoriginale  formula is found in the famous paper of Hardy and Ramanujan
(1917) and ascribed by them to Euler. It is very useful for rough
appraisal of asymptotic formulas. Hardy and Ramanujan make the cryptic
remark that it is ``$a$ formula which lends itself to wide
generalisations''. This remark was at first not very obvious to me;
but it can now be interpreted in the following way. Let us look for
$\mathfrak{z}^k$ in (\ref{part1:lec6:eq5}). Then
$$
\sum_{\substack{r,s\\r-s=k}} C_r (x) C_s (x)
=\frac{x^{k^2}}{\prod\limits^\infty_{n=1} (1-x^{2n})}
$$ 
or, replacing $r$ by $s+k$, and writing $C_s$ for $C_s(x)$, the left
side becomes
\begin{multline*}
  \sum^\infty_{s=0} C_s C_{s+k}  = 1\cdot \frac{x^{k^2}}{(1-x^2)
    \cdots (1-x^{2k})}
    +\frac{x^{1+(k+1)^2}}{(1-x^2)^2(1-x^4)\cdots(1-x^{2k+2})}+\\
       + \frac{x^{4+(k+2)^2}}{(1-x^2)^2(1-x^4)^2(1-x^6)\cdots
        (1-x^{2k+4})}+ \cdots
\end{multline*}

Let\pageoriginale  us divide by $x^{k^2}$. The general exponent on the right side is
$\ell^2 + (k+ \ell)^2$, so on division it becomes $2\ell^2 +
2k\ell$. Every exponent is even, which is a very nice
situation. Replace $x^2$ by $y$, and we get the `wide generalisation'
of which Hardy and Ramanujan spoke:
\begin{multline*}
  \frac{1}{(1-y)(1-y^2)\cdots (1-y^k)} +
  \frac{y^{k+1}}{(1-y)^2(1-y^2)\cdots (1-y^{k+1})}\\
  + \frac{y^{2(k+2)}}{(1-y)^2(1-y^2)^2(1-y^3)\cdots (1-y^{k+2})}+ \cdots
  \\
  \frac{y^{l (k+l)}}{(1-u)^2\cdots (1-y^l)^2 (1-y^{l+1})\cdots
    (1-y^{k+l})} + \cdots = \frac{1}{\prod\limits^\infty_{n=1}
    (1-y^n)} \tag{7}\label{part1:lec6:eq7} 
\end{multline*}
$k$ is an assigned number and it can be taken arbitrarily.

So such expansions are not unique.

Thus (\ref{part1:lec6:eq6}) and (\ref{part1:lec6:eq7}) give two
different expansions for 
$$
\frac{1}{\prod\limits^\infty_{n=1} (1-y^n)}.
$$

We are now slowly coming to the close of our preoccupation with power
series; we shall give one more application due to Ramanujan (1917). In
their paper Hardy and Ramanujan gave a surprising asymptotic formula for
$p(n)$. It contained an error term which was something unheard of
before, $\mathcal{O}(n^{-1/4})$, error term \textit{decreasing} as $n$
increases. Since $p(n)$ is an integer it is enough to take a few terms
to get a suitable value. The values calculated on the basis of\pageoriginale  the
asymptotic formula were checked up with those given by Macmahon's
tables and were found to be astonishingly close. Ramanujan looked at
the tables and with his peculiar insight discovered something which
nobody else could have noticed. He found that the numbers $p(4)$,
$p(9)$, $p(14)$, in general $p(5k+4)$ are all divisible by 5; $p(5),
p(12), \cdots p (7k+5)$ are all divisible by 7; $p(11k +6)$ by 11. So
he thought this was a general property. $A$ divisibility property of
$p(n)$ is itself surprising, because $p(n)$ is a function defined with
reference to addition. The first and second of these results are
simpler than the third. Ramanujan in fact suggested more. If we chose
a special progression modulo $5^\lambda$, then all the terms are
divisible by $5^\lambda$. There are also special progressions modulo
$7^{2 \lambda-1}$; so for 11. Ramanujan made the general conjecture
that if $\delta = 5^a 7^b 11^c$ and $24n\equiv 1 \pmod{\delta}$, then
$p(n)\equiv 0 \pmod{\delta}$. In this form the conjecture is
wrong. These things are deeply connected with the theory of modular
forms; the cases 5 and 7 relate to modular forms with subgroups of
genus 1, the case 11 with genus 2.

Let us take the case of 5. Take $p(5k+4)$. Consider $\Sigma p(n) x^n$;
it is nicer to multiply by $x$ and look for $x^{5k}$. We have to show
that the coefficients of $x^{5k}$ in $x \Sigma p(n) x^n$ are congruent
to zero modulo 5. We wish to juggle around with series a bit. Take
$\Sigma a_n x^n$; we want to study $x^{5k}$. Multiply by the series
$1+b_1x^5 +b_2 x^{10}+ \cdots $ where the $b'$s are integers. We get a
new power series
$$
\sum a_n x^n \cdot(1+ b_1 x^5 + b_2 x^{10} + \cdots ) = \sum c_n x^n, 
$$  
which\pageoriginale   is just as good. It is enough if we prove that
for this series every fifth coefficient $\equiv 0 \pmod{5}$.

For, 
\begin{align*}
  \sum a_n x^n & = \frac{\sum c_nx^n}{1+b_1 x^5 + b_2 x^{10} +
    \cdots}\\
  & = \sum c_n x^n, (1+ d_1 x^5 + d_2 x^{10}+ \cdots ), ~\text{say}.
\end{align*}

Then if every fifth coefficient of $\Sigma c_n x^n$ is divisible by 5,
multiplication by $\Sigma d_n x^{5n}$ will not disturb this. For a
prime $p$ look at
$$
(1+x)^p= 1+ \binom{p}{1} x + \binom{p}{2} x^2 + \binom{p}{3} x^3 +
\cdots + \binom{p}{p} x^p.
$$

All except the first and last coefficients on the right side are
divisible by $p$, for in a typical term $\binom{p}{q} =
\frac{p!}{(p-q)! q!}$, the $p$ in the numerator can be cancelled only
by a $p$ in the denominator. So
$$
(1+x)^p \equiv 1+x^p \pmod{p}.
$$

This means that the difference of the two sides contains only
coefficients divisible by $p$. This
$$
(1-x)^5 \equiv 1+x^5 \pmod{5}
$$

We\pageoriginale  now go to Ramanujan's proof that $p(5k+4) \equiv 0 \pmod{5}$ We
have
$$
x \sum p(n) x^n = \frac{x}{\prod (1-x^n)}
$$

It is irrelevant here if we multiply both sides by a series containing
only $x^5, x^{10}, x^{15}, \cdots$. This will not ruin our plans as we
have declared in advance. So
\begin{align*}
  x \sum p(n) x^n \prod^\infty_{m=1} (1-x^{5m}) &
  =\frac{x}{\prod(1-x^n)} \prod^\infty_{m=1} (1-x^{5m})\\
  & \equiv \frac{x}{\prod (1-x^n)} \prod^\infty_{m=1} (1-x^m)^5
  ~\text{modulo}\, 5\\  
  \Big(\prod (1-x^{5m}) -\prod (1-x^m)^5 
  & \text{has only coefficients divisible by 5}\Big)\\
  & \equiv x \prod^\infty_{m=1} (1-x^m)^4 ~\text{modulo}~ 5\\
  & = x \prod^\infty_{m=1} (1-x^m) \prod^\infty_{m=1} (1-x^m)^3.
\end{align*}

For both products on the right side we have available wonderful
expressions. By (\ref{part1:lec6:eq2}) and (\ref{part1:lec6:eq2a}),
$$
x \prod (1-x^m) \prod (1-x^m)^3 = x \sum^\infty_{\lambda=-\infty}
(-)^{\lambda (3 \lambda -1)/2} \sum^\infty_{k=0} (-)^k (2k+1) x^{k(k+1)/2}
$$

The typical term on the right side is 
$$
\sum^\infty_{k=0} (-)^{\lambda+k} x^{1+\lambda(3 \lambda-1)/2} + k(k+1)/2
$$

The exponent $= 1+ \lambda (3 \lambda-1)/2+ k(k+1)/2$, and we want
this to be of the form $5m$. Each such combination
contributes\pageoriginale  to $x^{5m}$. We want
$$
1+ \frac{\lambda(3 \lambda-1)}{2} + \frac{k(k+1)}{2} \equiv 0 \pmod{5}
$$

Multiply by 8; that will not disturb it. So we want
\begin{align*}
  8 + 12 \lambda^2 - 4 \lambda + 4k^2 + 4k & \equiv 0(5),\\
  3+ 2 \lambda^2 - 4 \lambda + 4 k^2 + 4k & \equiv 0 (5),\\
  2 (\lambda-1)^2+ (2k+1)^2 & \equiv 0 (5).
\end{align*}

This is of the form:
\begin{center}
  2. a square $+$ another square $\equiv 0 (5)$
\end{center}

Now
\begin{align*}
  A^2 & \equiv 0, 1, 4(5),\\
  2B^2 & \equiv 0, 2, 3 (5);
\end{align*}
and so $A^2+ 2B^2\equiv 0(5)$ means only the combination $A^2 \equiv
0(5)$ and $2B^2\equiv 0 (5)$; each square must therefore separately be
divisible by 5, or
$$
2k+1 \equiv 0 (5)
$$

So to $x^{5m}$ has contributed only those combinations in which $2k+1$
appeared; and every one of these pieces carried with it a factor of
5. This proves the result.

The case $7k+5$ is even simpler. We multiply by a series in $x^7$
leading to $(1-x^m)^6$ which is to be broken up into two Jacobi
factors $(1-x^m)^3$. These are examples of very beautiful theorems
proved in a purely formal way.

We shall deal in the next lecture with one more starting instance, the\break
Rogers-Ramanujan identities which one cannot refrain from talking
about.


