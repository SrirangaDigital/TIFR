\chapter{Lecture 23}\label{chap23:lec23}%lec 23

\begin{defi*}%def 0
The\pageoriginale $\Phi$- dimension of a space $X$, $\Phi- \dim X$, is
$\sup\limits_{F \in \Phi} \dim F$.   
\end{defi*}

$\Phi - \dim X \leqq n$ \textit{if and only if every $\Phi$- covering
  has a $\Phi$- refinement of order $\leqq n$.} 

\begin{proof}
{\em Necessity}. Let $\Phi - \dim X \leqq n$ and let $\mathscr{U}$ be
a $\Phi$-covering of $X$. Let $G$ be a neighbourhood of $\bigcup
\limits_{U \in \mathscr{U} -(U_*) } \bar{U}$ with $\bar{G} \in
\Phi$. Then $\mathscr{U}'= \big\{ (\mathscr{U} -(U^*)) \cap  \bar{G}, U_*
\cap \bar{G} \big\}$ forms a locally finite covering of $\bar{G}$. Since
$\bar{G}$ is normal and $\dim \bar{G} \leq n$, the covering
$\mathscr{U}'$ (see Lecture \ref{chap11:lec11}) has a (locally finite) proper
refinement $\mathscr{W}'$ of order $\leq n$. Let $V_*$ be the union of
$X- \bar{G}$ and those elements $\mathscr{W}'$ which are contained in
$U_* \cap \bar{G}$ together with $V_*$ form a $\Phi$- covering
$\mathscr{W}$ of order $\leq n$ which is a $\Phi$- refinement of
$\mathscr{U}$. 
\end{proof}

\medskip
\noindent\textbf{Sufficiency.} Let $F \in \Phi$ and let $\mathscr{U}$ be a
finite proper covering of $F$. Let $G$ be an open set with $F \subset
G$ and $\bar{G} \in \Phi$. Extend each $U \in \mathscr{U}$ to an open
set $V$ of $G$ with $V \cap F  = U$. These sets together with $V_* = X
- F $ form a $\Phi$- covering $\mathscr{W}$ of $X$. Then $\mathscr{W}$
has a $\Phi-$ refinement $\mathscr{W}$ of order $\leqq n $  and $\{ W
\cap F\}_{ W \in \mathscr{W}}$ is a refinement of $\mathscr{U}$ of order
$\leqq n $. Thus $\dim F \leqq n$, and hence $\Phi - \dim X \leqq n$.  

\begin{note*}
 The paracompactness of the sets of $\Phi$ was not
used in this proof.  
\end{note*}

\begin{example*}%exam 0
In\pageoriginale Example $M$ (see C. H. Dowker, Quart. Jour. Math 6 (1955),
p. 115) let $\Phi$ be the family of all paracompactsets of $M$ (the
space is also denoted by $M$). Then $\Phi - \dim M = 0 $ and $\dim M =
1 $.  
\end{example*}

\begin{remark*}%rema 0
It is always true that $\Phi - \dim X \leq \dim x$. 
\end{remark*}

\begin{proposition}\label{chap23:prop17}%prop 17
Let 
$$
\cdots \to \mathscr{S}^{q-1 } \xrightarrow{d^q} \mathscr{S}^{q}
\xrightarrow{d^{q+1}} \mathscr{S}^{q+1 } \to \cdots  
$$
be a sequence of homomorphisms of sheaves of $A-$ modules
  with $\im d^q = \ker d^{q + 1}$ for $q \neq 0 $  and $\im d^0
\subset \ker d^1 $ and let $\mathscr{G}=  \ker d^1 / \im
d^0$. If $\Phi - \dim X$ is finite or if the degrees of 
$\{\mathscr{S}^q\}$ are bounded below there is an isomorphism 
$\eta : H^p C_{\Phi } (\mathscr{S}) \to H^p_\Phi (X, \mathscr{G})$.  

If 
$$
\cdots \to \mathscr{S}^{'q-1 } \xrightarrow{d^q} \mathscr{S}^{'q}
\xrightarrow{d^{q+1}} \mathscr{S}^{'q+1 } \to \cdots  
$$
is another such sequence (with $\mathscr{G}'$
isomorphic to $\mathscr{G}$ and identified with 
$\mathscr{G}$) and if $h : \mathscr{S}'^{q} \to
\mathscr{S}^q$ are homomorphisms commuting with $d^q$
such that the induced homomorphism $h : \mathscr{G}' \to
\mathscr{G}$ is the identity, then there are induced
  homomorphisms $h^* : H^P C _{\Phi } (\mathscr{S}') \to H^P C_{\Phi}
(\mathscr{S})$ with commutativity in  
\[
\xymatrix{
H^p C_{\Phi} (\mathscr{S}')\ar[r]^{\eta}\ar[d]_{h^{\ast}} &
H^p_{\Phi} (X, \mathscr{G}) \ar[d]^{h^{\ast} = \text{ identity}}\\
H^p C_{\phi} (\mathscr{S}) \ar[r]^{\eta} & H^p_{\phi} (X,\mathscr{G}).
}
\]
\end{proposition}


\begin{proof}
If the degrees of $\{ \mathscr{S}^q\}$ are bounded below, then
$C^p_{\Phi } (\mathscr{U}, \mathscr{S}^q) = 0 $ for $q < n $ for some
$n$, and so the system $\bigg\{C_\Phi (\mathscr{U}, \mathscr{S}),
\Phi_{\mathscr{W} \mathscr{U}}\bigg\}_{ \mathscr{U}, \mathscr{W} \in
  \Omega_*}$ is bounded on the left. If $\Phi - \dim X \leq m$,
then\pageoriginale there is a cofinal directed set $\Omega'_*$,
consisting of $\Phi-$coverings of order $\leqq m$. If $\mathscr{U}
\in \Omega'_*$, 
$C^P_{\Phi } (\mathscr{U}, \mathscr{S}^q) = 0 $ for $p > m$; thus the
system $\bigg\{C_\Phi (\mathscr{U}, \mathscr{S}), \Phi_{\mathscr{W}
  \mathscr{U}}\bigg\}_{ \mathscr{U}, \mathscr{W} \in \Omega'_*}$ is
bounded below.  
\end{proof}

If $q \neq 0$, we have $\mathscr{H}^q = 0 $, and by a result in
Lecture \ref{chap22:lec22}, we have $H^{p,q}_{21} C_\Phi (\mathscr{S})
\approx H^p_\Phi 
(\chi, \mathscr{H^q}) = 0$, hence $H^{p,q}_{21} C_\Phi (\mathscr{S}) =
0$. Therefore by Proposition 16-a, there is an isomorphism  
$$
\theta : H^p C_\Phi (\mathscr{S}) \to H^{p,o}_{21} C_\Phi
(\mathscr{S}). 
$$

Also for $q = 0$, there is an isomorphism (see Lecture \ref{chap22:lec22}) 
$$
\psi^* : H^{p,o}_{21} C_\Phi (\mathscr{S}) \to H^p_{\Phi } (x, \mathscr{G}).
$$

Let $\eta$ be the composite isomorphism 
$$
\eta = \psi^{\ast \theta}: H^P C_{\Psi } (\mathscr{S}) \to H^P_\Psi (X,
\mathscr{G}). 
$$

Next, the homomorphisms $ h : \mathscr{S}'^q \to \mathscr{S}^q$
induce homomorphisms $h : C^P_\Phi (\mathscr{U}, \mathscr{S'}^q) \to
C^P_\Phi (\mathscr{U}, \mathscr{S}^q)$ which commute with $d$, $\delta$
and $\phi_{\mathscr{W} \mathscr{U}}$, and hence give rise to maps $h :
C_\Phi (\mathscr{U}, \mathscr{S}') \to C_\Phi (\mathscr{U},
\mathscr{S})$ which commute with $\phi_{\mathscr{W}
  \mathscr{U}}$. Therefore, there are induced homomorphisms $h^*$
which commute with $\theta$,  
\[
\xymatrix{
H^p C_{\Phi} (\mathscr{S}') \ar[r]^{\theta}\ar[d]_{h^{\ast}} &
H^{p,0}_{21} C_{\Phi} (\mathscr{S}')\ar[d]^{h^{\ast}} \\
H^pC_{\Phi} (\mathscr{S}) \ar[r]^{\theta} & H^{p,0}_{21} C_{\Phi}
(\mathscr{S}). 
}
\]

Since\pageoriginale $h$ commutes with $d^q$, $h$ maps $ \mathfrak{z}'^q $ into
$\mathfrak{z}^q$ and $\mathbb{B}'^q$ into $\mathbb{B}^q$, hence induces
a homomorphism $h : \mathscr{H}'^q  \to  \mathscr{H}^q$, and there is
commutativity in  
\[
\xymatrix{
C^p_{\Phi} (\mathscr{U}, \mathfrak{z}'^q)\ar[r]^{\psi}  \ar[d]_h &
C^p_{\Phi} (\mathscr{U}, \mathscr{H}'^q)\ar[d]^h\\
C^p_{\Phi} (\mathscr{U}, \mathfrak{z}^q) \ar[r]^{\psi} & C^p_{\Phi}
(\mathscr{U}, \mathscr{H}^q),
}
\]
where $\psi$ is the homomorphism induced by the natural homomorphisms
$\mathfrak{z}'^q \to \mathscr{H}'^q$ and $\mathfrak{z}^q \to
\mathscr{H}^q$. Therefore there is commutativity in  
\[
\xymatrix{
H^{p,q}_{21} C_{\Phi} (\mathscr{S}') \ar[r]^{\psi^{\ast}}
\ar[d]_{h^{\ast}} & H^p_{\Phi} (X, \mathscr{H}'^q)\ar[d]^{h^{\ast}}\\
H^{p,q}_{21} C_{\Phi} (\mathscr{S}) \ar[r]^{\psi^{\ast}} & H^p_{\Phi}
(X,\mathscr{H}^q) .
}
\]

Therefore, taking $q=0$, we see that $h^*$ commutes with $\eta =
\psi^* \theta$. 
\[
\xymatrix{
H^pC_{\Phi} (\mathscr{S}')  \ar[r]^{\theta} \ar[d]_{h^{\ast}} &
H^{p,0}_{21} C_{\Phi} (\mathscr{S}') \ar[r]^{\psi^{\ast}}\ar[d] &
H^p_{\Phi} (X,\mathscr{G})\ar[d]^{h^{\ast} = \text{ identity}}\\
H^pC_{\Phi} (\mathscr{S}) \ar[r]^{\theta} & H^{p,0}_{21} C_{\Phi}
(\mathscr{S}) \ar[r]^{\psi^{\ast}} & H^p_{\Phi} (X,\mathscr{G}).
}
\]

