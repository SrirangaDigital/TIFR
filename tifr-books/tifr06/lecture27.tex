\chapter{Lecture 27}

\textit{Given\pageoriginale any sheaf $\mathscr{G}$} of $A$-modules,
there exists an exact sequences of sheaves  
$$
O \to \mathscr{G}\xrightarrow{e} \mathscr{S}^o \xrightarrow{d^1}
\mathscr{S}^1 \to \cdots \to \mathscr{S}^{q-1} \xrightarrow{d^q}
\mathscr{S}^q \to \cdots 
$$
\textit{where each $\mathscr{S}^q (q \geq O)$ is finite.}

\begin{proof}
For each open set $U$ of $X$, let $S^q_U(q=O,1, \ldots)$ be the
abelian group of integer valued functions $f(x_0, \ldots ,x_q)$ of
$q+1$ variables $x_0, \ldots\break x_q \in U$. If $V$ is an open set with $V
\subset U$, the restriction of the functions $f$ gives a homomorphism
$\rho_{_{VU}} : S^q_U \to S^q_V$. There is a homomorphism
$d^{q+1}_{U}:S^q_U \to S^{q+1}_U$ defined by 
$$
d^{q+1}f(x_o, \ldots ,x_{q+1})= \sum ^{q+1}_{j=o}(-1)^j f(x_o, \ldots,
\hat{x}_j,\ldots,x_{q+1}). 
$$

If $e:Z \to S^o_U$ is the inclusion homomorphism of the constant
functions on $U$, the sequence 
$$
O \to Z \xrightarrow{e}  S^o_U \xrightarrow{d^1} \cdots \to S^{q-1}_U
\xrightarrow{d^q}  S^q_U \to \cdots 
$$
is exact (Cartan Seminar 1948-49, Expose 7, \S 8). Clearly
$\rho_{_{VU}}d^{q+1} f=d^{q+1} \rho_{_{VU}}f$ and $\rho_{_{VU}}$ commutes
with $e$. The presheaves $\{ S^q_U, \rho_{_{VU}} \} (q = 0, 1, \ldots)$
determine sheaves $\mathscr{S}^q$ and there is an induced exact
sequence 
$$
O \to Z \xrightarrow{e}  S^o_U \xrightarrow{d^1} \cdots \to S^{q-1}_U 
\xrightarrow{d^q} S^q \to \cdots 
$$

It is easily verified that each abelian group $S^p_U$ is without
torsion, and\pageoriginale this property is preserved in the direct
limit. Hence 
each stalk $S^q_x=dir \lim \{ S^q_U, \rho_{_{VU}} \}_{x \in U}$ is
without torsion, i.e., the sheaves $\mathscr{S}^q$ are without
torsion. Therefore the sequences of sheaves of $A$-modules 
$$
O \to Z \otimes_Z \mathscr{G}\xrightarrow{e} \mathscr{S}^o \otimes_Z
\mathscr{G} \xrightarrow{d^1} \cdots \to \mathscr{S}^{q-1} \otimes_Z \mathscr{G}
\xrightarrow{d^q} \mathscr{S}^q  \otimes_Z \mathscr{G} \to \cdots 
$$
is exact, (this follows from the fact that if $O \to F' \to F \to F''
\to O$ is an exact sequence of abelian group, $F''$ is without
torsion, and $G$ is an abelian group, then the sequence 
$$
O \to F' \otimes G \to F \otimes G \to F'' \otimes G \to O  
$$
is exact), that is, the sequence
\begin{equation*}
0 \to \mathscr{G}\xrightarrow{e} \mathscr{S}^o
  \otimes_Z \mathscr{G} \xrightarrow{d^1} \cdots \to \mathscr{S}^{q-1}
  \otimes_Z \mathscr{G} \xrightarrow{d^q}  \mathscr{S}^q \otimes_Z
  \mathscr{G} \to \cdots\tag{1}\label{chap27:eq1}
\end{equation*}
is exact.

We now show that \textit{each of the sheaves} $\mathscr{S}^q
\otimes_Z \mathscr{G}$ \textit{is a fine sheaf}. To do this, let $E
\subset G$ with $E$ closed and $G$ open and let $h:S^q_U \to S^p_U$ be
the homomorphism defined by  
\begin{align*}
h f(x_o, \ldots ,x_q) & = f(x_o, \ldots ,x_q) \text{ if } x_o, \ldots
,x_q \in U \cap G,\\ 
& = 0 \text{ otherwise.}
\end{align*}

Then $h$ commutes with $\rho_{_{VU}}$ and induces a homomorphism
$h:\mathscr{S}^q \to \mathscr{S}^q $ for which $h_x:S^q_x \to S^q_x$
is the identity if $x \in  \bar{G}$, and $h(S^q_x) = 0_x$ if $x \in X
- \bar{G}$. There is then an induced homomorphism $h:\mathscr{S}^q
\otimes_Z \mathscr{G}\to \mathscr{S}^q \otimes_Z$ which is the
identity on the stalks $S^q_x 
\otimes_Z G_x $ for $x \in G$ and zero on\pageoriginale $S^q_x
\otimes_Z G_x$ for 
$x \in X - \bar{G}$. Thus $\mathscr{S}^q \otimes_Z \mathscr{G}$ is
fine, and the sequence (\ref{chap27:eq1}) is a fine resolution of
$\mathscr{G}$.  

We now give a definition of the cochains of a covering of a space $X$
coefficients in a sheaf $\mathscr{G}$ and support in a $\Phi$-family
and also give an alternative definition of the cohomology groups of
$X$ with coefficients in $\mathscr{G}$ and supports in the family
$\Phi$. We then prove Leray's theorem on acyclic coverings.  
\end{proof}

\begin{defi*}
Let $C^p_\Phi(\mathscr{U},\mathscr{G})$, for an arbitrary covering
$\mathscr{U}=\{ U_i \}_{i \in I}$ be the
$C^p(\mathscr{U},\mathscr{G})$ consisting  of those cochains $f$ such
that the closure of the set $supp f=\{ x:f(i_o, \ldots ,i_p)(x)$ is
defined for some $\sigma=(i_o, \ldots ,i_p)$ and $\neq 0_x \}$,
belongs to the family $\Phi$. (If $p=0$, this definition does not
agree with the previous one even when $\mathscr{U}$ is a
$\bar{\Phi}$-covering.) 
\end{defi*}

Then, since the homomorphisms
\begin{align*}
\delta^{p+1}&:C^p(\mathscr{U},\mathscr{G})\to
C^{p+1}(\mathscr{U},\mathscr{G})\\ 
\tau^+ &: C^p (\mathscr{U},\mathscr{G}) \to C^p
(\mathscr{W},\mathscr{G}) 
\end{align*}
decrease supports, they induce homomorphisms
$$
\delta ^{p+1}:C^p_\phi(\mathscr{U},\mathscr{G})\to
C^{p+1}_\Phi(\mathscr{U},\mathscr{G})\\ 
$$
and
$$
\tau ^+ : C^p_\Phi (\mathscr{U},\mathscr{G}) \to C^p_\phi
(\mathscr{W},\mathscr{G}) 
$$
respectively. Hence there are cohomology module $H^p_ \Phi
(\mathscr{U},\mathscr{G})$ and homomorphisms\pageoriginale
$\tau_{\mathscr{W}\mathscr{U}}:H^p_\Phi (\mathscr{U},\mathscr{G})\to
H^p_\Phi (\mathscr{W},\mathscr{G})$. For $p=0$, $H^o_ \Phi
(\mathscr{U},\break\mathscr{G})=\Gamma_ \Phi(X, \mathscr{G})$ for every
covering $\mathscr{G}$, and  
$$
\tau_{\mathscr{W}\mathscr{U}}:\Gamma_ \Phi(X, \mathscr{G}) \to \Gamma_
\Phi(X, \mathscr{G}) 
$$
is the identity. Using the directed set $\Omega$ of all proper
covering of \break $X, \{ H^p_ \Phi(\mathscr{U},\mathscr{G}),
\tau_{\mathscr{W}\mathscr{U}} \}_{\mathscr{U},\mathscr{W} \in \Omega}$
is a direct system. The direct limit of this system will be denoted by
$ H^p_\Phi(X, \mathscr{G})$. This module also is called the $p$-\textit{th cohomology module of the space $X$ with coefficients in
  the sheaf $\mathscr{G}$ and supports in the family $\Phi$}. (This
cohomology module is isomorphic with that previously defined by means
of $\Phi$-coverings, Lecture \ref{chap21:lec21}.) There are
homomorphisms into the 
direct limits, $\tau_ \mathscr{U}:H^p_ \Phi
(\mathscr{U},\mathscr{G})\to H^p_ \Phi (X,\mathscr{G})$. 

Let $X$ be a \textit{paracompact normal} space, $\mathscr{G}$ a sheaf
of $A$-modules over $X$, and let $\mathscr{U}=\{ U\}$ be locally
finite proper covering of $X$ where each $U \in \mathscr{U}$ is an $F_
\sigma$ set. ($\mathscr{U}$ is indexed by itself.) Let $\Phi_\sigma$
be the set of all intersection $E \cap U_ \sigma$ with $E \in
\Phi$. Since $U_\sigma$ is an open $F_\sigma$ set in $X,U_\sigma$
is paracompact and normal. Hence each $E \cap U_\sigma \in
\Phi_\sigma$ is paracompact and normal. One easily verifies that
$\Phi_\sigma$ is a $\Phi$-family in $U_\sigma$. We now assume the 
following conditions on the family $\Phi$ and the covering
$\mathscr{U}$. 
\begin{enumerate}[(i)]
\item For some infinite cardinal number $m$, the union of fewer than
  $m$ elements of $\Phi$ is contained in a set belonging to
  $\Phi$. (If $X \in \Phi$, choose $m$ greater than the number of
  closed sets of $X; $ if $\Phi$ is family of compact sets, let
  $m=\mathscr{N}_o$.) 

\item Each\pageoriginale set in $\Phi$ meets fewer than $m$ sets of
  the covering $\mathscr{U}$.  

\item Each $U \in \mathscr{U}$ is an $F_\sigma$ sets, i.e., is a
  countable union of closed subsets of $X$. 

\item For each $U_ \sigma=U_o\cap \cdots U_p$,
  $H^p_{\Phi_\sigma}(U_\sigma,\mathscr{G})=0$ for $q>0$; here
  $\mathscr{G}$ denotes the restriction of $\mathscr{G}$ to $U_\sigma$  

(A covering $\mathscr{U}$ is called \textit{acyclic} is conditions
  (iv) is satisfied.) 
\end{enumerate}

\textit{Under the conditions {\em (i), (ii), (iii), (iv)} stated
  above, the homomorphism $\tau _\mathscr{U}: H^p_ \Phi
  (\mathscr{U},\mathscr{G}) \to H^p _ \Phi (X,\mathscr{G})$ is an
  isomorphism.}  

\begin{proof}
Let
$$
O \to \mathscr{G} \xrightarrow{e} \mathscr{S}^o \xrightarrow{d^1}
\cdots \to \mathscr{S}^{q-1} \rightarrow{d^q} \mathscr{S}^q \to \cdots 
$$
be any fine resolution of $\mathscr{G}$. Then there is a system $\{
(C^p _ \Phi (\mathscr{U},\mathscr{S}^q)) \}_{\mathscr{U} \in \Omega'}$
of double complexes, where $\Omega'$ is the cofinal directed set of all
locally finite proper coverings of $X$. This system is bounded above
by $p=0$ and on the left by $q=0$. 
\end{proof}

Since $X$ is normal, $\mathscr{S}^q$ is fine and $\mathscr{U}$ is
locally finite, there is a homotopy (see Lecture \ref{chap16:lec16}) 
$$
k^{p-1}:C^p(\mathscr{U},\mathscr{S}^q) \to 
C^{p-1}(\mathscr{U},\mathscr{S}^q) \qquad (p>0). 
$$

Since $k^{p-1}$ decreases supports, it induces a homomorphism
$$
k^{p-1} : C^p_{\Phi} (\mathscr{U}, \mathscr{S}^q) \to C^{p-1}
(\mathscr{U}, \mathscr{S}^q) \quad (p > 0),
$$
with\pageoriginale $\delta^p k^{p-1} + k^p \delta^{p+1} = 1$. Hence $H^p_\Phi
(\mathscr{U}, \mathscr{S}^q)=0$ for $p>0$, and for $p=0$, $H^o_ \Phi
(\mathscr{U},\mathscr{S}^q)= \Gamma _\Phi (X, \mathscr{S}^q)$. Hence 
$$
H^{O,q}_{12}C_ \Phi (\mathscr{U},\mathscr{S})=H^q \Gamma _ \Phi
(X,\mathscr{S}). 
$$

Thus we have the isomorphisms indicated below (see Lecture
\ref{chap20:lec20}): 
\[
\xymatrix{
H^q\Gamma_{\Phi} (X,\mathscr{S}) \ar[d]_{\approx} & H^{0,q}_{12}
C_{\Phi} (\mathscr{U},\mathscr{S}) \ar[d] \ar[l]_{\approx} &
H^qC_{\Phi} (\mathscr{U},\mathscr{S})\ar[d]\ar[l]_{\approx}\\
H^q\Gamma_{\Phi} (X,\mathscr{S}) & H^{0,q}_{12} C_{\Phi} (\mathscr{S})
\ar[l]_{\approx} & H^qC_{\Phi} (\mathscr{S}) \ar[l]_{\approx}.
 }
\]

Since $\mathscr{S}^q$ is fine and $X$ is normal, $\mathscr{S}^q$ is
locally fine, hence its restriction to $U_ \sigma$ is locally
fine. But $U_\sigma$ is paracompact and normal, so that the
restriction of $\mathscr{S}^q$ to $U_ \sigma$ is fine. Hence there is
an isomorphism (see Proposition \ref{chap23:prop17} and
\ref{chap25:prop18}, lectures \ref{chap23:lec23} and \ref{chap25:lec25} 
respectively), 
$$
\eta \rho^{-1}: H^q\Gamma_{\Phi_ \sigma}(U_\sigma, \mathscr{S}) \to
H^q_{\Phi_ \sigma}(U_\sigma,\mathscr{G}). 
$$

\noindent
Hence by condition (iv), $H^q \Gamma_{\Phi_\sigma}(U_\sigma, \mathscr{S})=0$
for $q>0$, $H^o\Gamma_\Phi(U_\sigma, \mathscr{S}) \approx H^o_{\Phi_
  \sigma}(U_\sigma, \mathscr{G})= \Gamma_{\Phi_\sigma}(U_ \sigma,
\mathscr{G})$. 

If $f \in C^p_\Phi (\mathscr{U},\mathscr{S}^q) \; (q>O)$ and
$d^{q+1}f=O$, then $(d^{q+1}f)(U_o , \ldots,\break U_p)=O$ in each $U_\sigma
=U_o \cap \cdots \cap U_p$. Since $H^q \Gamma _{\Phi_\sigma}(U_
\sigma, \mathscr{S})=0(q>0)$, there is a section $g(U_o 
,\ldots ,U_p) \in \Gamma_{\Phi _\sigma}(U_ \sigma,
\mathscr{S}^{q-1})$ with $dg(U_o ,\ldots,\break U_p)=f(U_o ,\ldots ,U_p)$
(choose $g(U_o ,\ldots ,U_p)=0$ if $f(U_o ,\ldots ,U_p)=0$. There is
then a cochain $g \in C^p(\mathscr{U},\mathscr{S}^{q-1})$) with
$dg=f$, (see p.57). Since $f \in C^o_\Phi
(\mathscr{U},\mathscr{S}^q)$, $\supp f$ is contained in a set belonging
to $\Phi$ and\pageoriginale hence $f(\sigma)$ is different from zero
on fewer than 
$m$ sets $U_\sigma$. Then $g(\sigma)$ is different from zero on fewer
than $m$ sets $U_\sigma$ and hence supp $g$ is the union of fewer than
$m$ set $\{ x \in U_\sigma :g(\sigma)(x)\neq 0$, each of which is in
$\Phi_\sigma$ and hence has its closure in $\Phi$. Hence $\supp g$ is
contained in a set belonging to $\Phi$ and $g \in C^p_\Phi
(\mathscr{U},\mathscr{S}^{q-1})$. Hence $H^{p,q}_2 C_\phi
(\mathscr{U},\mathscr{S})=0 (q>0)$. 

Since the sequences
$$
0 \to C^p(\mathscr{U},\mathscr{G}) \overset{e}\to
C^p(\mathscr{U},\mathscr{S}^o) \overset{d^1}\to C^p
(\mathscr{U},\mathscr{S}^1)
$$ 
is exact, if $f \in C^p_\Phi
(\mathscr{U},\mathscr{S}^O)$ and $d^1 f=0$, then $f=e(g)$ for some $g
\in C^p (\mathscr{U},\mathscr{G})$ and clearly $g \in C^p_\Phi
(\mathscr{U},\mathscr{G})$. Thus  
$$
H^{p,o}_2 C_\Phi (\mathscr{U},\mathscr{S})\approx C^p_\Phi
(\mathscr{U},\mathscr{G}) \text{ and  } H^{p,0}_{21}C_\Phi
(\mathscr{U},\mathscr{S}) \approx H^p_\Phi (\mathscr{U},\mathscr{G}).
$$  

Thus we have the isomorphism indicated below: 
\[
\xymatrix{
H^pC_{\Phi} (\mathscr{U},\mathscr{S}) \ar[r]^{\approx}\ar[d] &
H^{p,0}_{21} C_{\Phi}(\mathscr{U},\mathscr{S}) \ar[d] \ar[r]^{\approx}
& H^p_{\Phi} (\mathscr{U},\mathscr{G}) \ar[d]^{\tau_{\mathscr{U}}}\\
H^pC_{\Phi} (\mathscr{S}) \ar[r]^{\approx} & H^{p,0}_{21} C_{\Phi}
(\mathscr{S}) \ar[r]^{\approx} & H^p_{\Phi} (X,\mathscr{G}).
}
\]

Combining this diagram with the previous one, we see that the
homomorphism $\tau:{H^p_\Phi}(\mathscr{U},\mathscr{G}) \to H^p_ \Phi
(X, \mathscr{G})$ is an isomorphism. 
\hfill {Q.e.d}

In particular, we have proved the following proposition (Cartan
Seminar, 1953-54, Expose 17, p.7). 

\begin{proposition}%Prop 20
If $\mathscr{U}$\pageoriginale is a locally finite proper covering of a
  paracompact normal space $X$ by open $F_\sigma$ sets, and if
  $\mathscr{G}$ is a sheaf of $A$-modules such that $H^q(U_
  \sigma,\mathscr{G})=O (q>O)$ for every $U_\sigma =U_o \cap \cdots
  \cap U_k (k=O,1, \ldots)$, then 
$$
\tau _\mathscr{U}:H^p(\mathscr{U},\mathscr{G}) \to H^p (X,\mathscr{G})
$$
is an isomorphism.
\end{proposition}

In the case that $\Phi$ is the family of all compact sets of $X$, we
write $H^p_*$ instead of $H^p_{\bar{\Phi}}$. 

\setcounter{proposition}{19}
\begin{proposition}[-a]%Prop 20 a
If $\mathscr{U}$ is a locally finite proper covering of a locally
  compact and paracompact Hausdorff space by open $F_\sigma$ sets
  with compact closures, and if $\mathscr{G}$ is a sheaf of
  $A$-modules such that $H^p(U_\sigma, \mathscr{G})=O(q>O)$ for
 every 
$$
U_ \sigma=U_o \cap \cdots \cap U_k (k=0,1,\ldots) 
$$
then
$$
\tau _ \sigma:H^p_* (\mathscr{U},\mathscr{G})\to H^p_*(X,\mathscr{G}) 
$$
is an isomorphism.
\end{proposition}

\begin{note*}
It is no restriction to assume that $\mathscr{U}$ is a proper
covering. If $\mathscr{W}$ is any covering, there is an equivalent
proper covering $\mathscr{U}$ with the open sets. Then
$\tau_{\mathscr{W}\mathscr{U}}$ and $\tau _{\mathscr{U},\mathscr{W}}$
are isomorphisms. 
\end{note*}

