\chapter{Lecture 4} % Lecture 4

If\pageoriginale $(S', \pi', X)$ and $(S, \pi, X)$ are two sheaves of
$a$-modules with 
$S' \subset S$ and if the inclusion map $i : S' \rightarrow S$ is a
homomorphism, then $S'$ is an open subset of $S$ since i is an open
map; further the topology on $S'$ is the one induced from $S$. Then
$\pi' = \pi \cdot  i = \pi | S'$ and since $i | S'_x : S'_x
\rightarrow S_x$ is a homomorphism, $S'_x$ is a sub - $A_x$ - module
of $S_x$. 

This suggests the following definition of a subsheaf.

\begin{defi*}% defintion 
$(S', \pi | S', X)$ is called a subsheaf of $(S, \pi,X)$
  if $S'$ is open in $S$ and, for each $x$, $S'_x = S' \cap S_x$ is a
  sub - $A_x$- module of $S_x$. 
\end{defi*}

\textit{A subsheaf is a sheaf and the inclusion map $i$ is a
  homomorphism}. 

\begin{proof}% proof
For each $p \in S'$ there exists an open set $G$, $p \in G \subset S$,
with $\pi |  G$ a homeomorphism. Then $G \cap S'$ is open in $S'$
and $(\pi |S') | G \cap S' = \pi  |  G \cap S'$ is a
homeomorphism. $S'_x$ is an $A_x$-module and the operations which are
continuous in $S$ are continuous in the subspace $S'$. Therefore $(S',
\pi  | S', X)$ is a sheaf. 
\end{proof}

Since $i : S' \rightarrow S$ is a map, and $\pi \cdot  i = \pi | S'$ and
$i  | S'_x : S'_x \rightarrow S_x$ is the inclusion homomorphism of
the submodule $S'_x$, it follows that $i$ is a homomorphism. 

\textit{The set of all zeros in $S$ is a subsheaf of} $S$.

\begin{proof}
The set of zeros is open in $S$, and $0_x$ is a sub-$A_x$-module of
$S_x$. 
\end{proof}

This\pageoriginale sheaf is called the 0-sheaf (zero sheaf) and is usually
identified with the constant sheaf $0 = (X \times 0, \pi, X)$. 

\textit{If $h : \mathscr{S} \rightarrow \mathscr{R}$ is a homomorphism
  of sheaves, the set $S'$ of $p \in S$ such that $h(p) = 0_{\pi(p)}$
  forms a subsheaf $\mathscr{S}'$ of $\mathscr{S}$ called the kernel
  of $h(\mathscr{S}' = \ker h)$, and the image set $S'' = h(S)$ forms a
  subsheaf of $\mathscr{R}$ called the image of $h(\mathscr{S}'' = im 
  h)$ }. 

\begin{proof}% proof
\begin{enumerate}[(1)]
\item  Since $h$ is continuous and 0 (0 denotes the set of zeros
  in $R$) is open, therefore $S' = h^{-1}(0)$ is open. Each $S'_x = S'
  \cap S_x$ is the kernel of $h | S_x : S_x \rightarrow R_x$,
  hence $S'_x$ is a sub-$A_x$-module of $S_x$. 

\item Since $h$ is an open map, $S''$ is open. Each $S''_x = S''_x \cap
  R$ is the image of the homomorphism $h | S_x : S_x \rightarrow
  R_x$, hence $S''_x$ is a sub - $A_x$ - module of $R_x$. 
\end{enumerate}
\end{proof}

\begin{defi*}% definition
A homomorphism $h : \mathscr{S} \rightarrow \mathscr{R}$ is
    called a monomorphism if $\ker h = 0$, an epimorphism if
$\im h = R$, and and isomorphism if $\ker h = 0$ and $\im h
= R$. 
\end{defi*}

\begin{defi*}% definition 
A sequence
$$
\cdots \rightarrow \mathscr{S}_{j-1} \xrightarrow{h_j} \mathscr{S}_j
\xrightarrow{h_{j+1}} \mathscr{S}_{j+1} \rightarrow \cdots 
$$
of homomorphisms of sheaves is called {\em{exact at}} $\mathscr{S}_j$
if $\ker h_{j+1} = \im  h_j$; it is called {\em{exact}} if it is exact
at each $\mathscr{S}_j$. 
\end{defi*}
 
\textit{If $h : \mathscr{S} \rightarrow \mathscr{R}$  is a
  homomorphism}, the sequence 
$$
0 \rightarrow \ker h \xrightarrow{i} \mathscr{S} \xrightarrow{h}  \im
 h \rightarrow 0 
$$
\textit{is exact}.\pageoriginale

Here $i : \ker h \rightarrow \mathscr{S}$ is the inclusion
homomorphism, and $h' : \mathscr{S} \rightarrow \im h$ is defined
by $h'(p) = h(p)$. It is a homomorphism. The other two homomorphisms
are, of course, uniquely determined. 

\begin{proof}% proof
The statement is the composite of the three trivial statements: 
\begin{enumerate}[(i)]
\item $i : \ker h \rightarrow \mathscr{S}$ is a monomorphism,

\item $\ker h' = \ker h$,

\item $h' : \mathscr{S} \rightarrow  im h$ is an epimorphism.
\end{enumerate}
\end{proof}

\begin{defi*}
A {\em{directed set}} $(\Omega, <)$ is a set $\Omega$ and a relation
$<$, such that  
\begin{enumerate}[1)]
\item $\lambda < \lambda \quad (\lambda \in \Omega)$,

\item if\pageoriginale $\lambda < \mu$ and $\mu < \nu$ then $\lambda <
  \nu ( \lambda, \mu, \nu \in \Omega)$, 

\item if $\lambda$, $\mu \in \Omega$, there exists a $\nu \in \Omega$
  such that $\lambda < \nu$ and $\mu < \nu$. 
\end{enumerate}
\end{defi*}

That is, $<$ is reflexive and transitive and each finite subset of
$\Omega$ has an upper bound. (We also write $\mu > \lambda$ for
$\lambda < \mu$). 

\begin{example*}% example 
Let $\Omega$ be the family of all compact subsets $C$ of the plane let
$C < D$ mean $C \subset D$. $\Omega$ is then a directed set. 
\end{example*}

\begin{defi*}% definition
A {\em{direct system}} $\{ G_\lambda, \phi_{\mu\lambda} \}$ of abelian
groups, indexed by a directed set $\Omega$, is a system $\{G_\lambda
\}_{\lambda \in \Omega}$ of abelian groups and a system
$\{\phi_{\mu\lambda} : G_\lambda \rightarrow G_{\mu} \}_{\lambda <
  \mu}$ of homomorphisms such that  
\begin{enumerate}[(i)]
\item $\phi_{\lambda \lambda} : G_\lambda \rightarrow G_\lambda$ is
  identity, 

\item if $\lambda < \mu < \nu$, $\phi_{\nu\mu} \phi_{\mu\lambda} =
  \phi_{\nu\lambda }: G_\lambda \rightarrow G_{\nu}$. 
\end{enumerate}
\end{defi*}

\noindent
Thus if $ \lambda < \mu < \nu $ and $\lambda < k < \nu$ , then
$\phi_{\nu\mu} \phi_{\mu\lambda} = \phi_{\nu k} \phi_{k \lambda}$. 

The definition of a direct system will be the same even when the
$G_\lambda'$ s are any algebraic systems. 

\begin{defi*}% definition
Two elements $a \in G_\lambda$ and $b \in G_\mu$ of $\cup_{\lambda \in
  \Omega} G_\lambda $ are said to be {\em{equivalent}} $(a \sigma b)$
if for some $\nu$, $\phi_{\nu\lambda} a = \phi_{\nu\mu} b $. 
\end{defi*}

This relation is easily verified to be an equivalence relation. The
equivalence class determined by $a$ will be denoted by $(a)$. 

We will now define addition of equivalence classes. If $(a)$ and $(b)$ 
are equivalence classes, $a \in G_\lambda$, $b \in G_\mu$, choose a $\nu
>  \lambda$ and $> \mu$ and define $(a) + (b) = (\phi_{\nu\lambda} a +
\phi_{\nu\mu}b)$. 
\[
\xymatrix@R=0.5cm{
G_{\lambda} \ar[d]\ar[drr] & & G_{\mu}\ar[d] \ar[dll]\\
G_{\nu} \ar[dr] & & G_{\nu_1}\ar[dl]\\
& G_{\nu_2} & 
}
\]

To show that this does not depend on the choice of
$\nu$ choose $\nu_1 > \lambda$ and $> \mu$, let $\nu_2 > \nu$ and $> 
\nu_1$. Then 
\begin{align*}
\phi_{\nu_2 \nu} (\phi_{\nu\lambda}  a + \phi_{\nu\mu}  b) & =
\phi_{\nu_2 \lambda}  a + \phi_{\nu_2 \mu }  b\\ 
& = \phi_{\nu_2\nu_1} (\phi_{\nu_1\lambda}  a + \phi_{\nu_1 \mu}  b),
\end{align*}
hence $\phi_{\nu\lambda}  a + \phi_{\nu\mu} b \sim \phi_{\nu_1\mu}
 b$. Clearly the class $(\phi_{\nu\lambda}  a + \phi_{\nu\lambda} b)$
 is independent of the choice of $a$ and $b$.  

\textit{If $\{ G_{\lambda}, \phi_{\mu\lambda}\}$ is a direct system of
  abelian groups, the equivalence classes form an abelian group $G$
  called the direct limit of the system}. 

\begin{proof}
That $G$ is an abelian group follows easily from the fact that each
$G_{\lambda}$ is an abelian group. 
\end{proof}

The\pageoriginale zero element of $G_{\lambda}$ is the class
containing all the zeros of all the groups $G_{\lambda}$.   

Clearly, if each $G_{\lambda}$ is a ring, then $G$ is a ring, and
similarly for any other algebraic system. 

\textit{The function $\phi_{\lambda} : G_{\lambda} \rightarrow G$, where
  $\phi_{\lambda} a = (a)$ is a homomorphism and if $\lambda < \mu$,
  $\phi_{\mu} \phi_{\mu\lambda} = \phi_\lambda$}. 

\begin{proof} 
\begin{align*}
\phi_\lambda (a + b) & =  (a + b) = (a) + (b) = \phi_\lambda a +
\phi_\lambda b,\\ 
\phi_{\mu}(\phi_{\mu\lambda} a) & = (\phi_{\mu\lambda}  a) = (a) =
\phi_{\lambda}  a. 
\end{align*}
\end{proof}

\begin{example*}% example
Let $(N, \le)$ be the directed set of natural numbers. For each
natural number $n$ let $G_n = Z$ and if $n \le m$, let $\phi_{mn} :
G_n \rightarrow G_m$ be defined by $\phi_{mn}  a = \dfrac{m !  a}{n
  !}$. The direct limit is isomorphic to the group of rational
numbers. 
\end{example*}

\begin{example*}% example
Let $(N,\le)$ be as before. For each natural number $n$, let $G_n$ be
the group of rational numbers modulo 1 and if $n \le m$ let
$\phi_{mn} : G_n \rightarrow G_m$ be defined by $\phi_{mn}  a =
\dfrac{m !  a}{n !}$. The direct limit is zero. 
\end{example*}

\textit{If $\{ G_\lambda, \phi_{\mu\lambda} \}$ is a direct system of
  abelian groups and if $\{f_\lambda : G_\lambda \rightarrow H \}$ are
  homomorphisms into an abelian group $H$ with $f_\mu \phi_{\mu\lambda
    = f_\lambda}$, there is a unique homomorphism $f : G \rightarrow
  H$ of the limit group $G$ such that $f \phi_\lambda = f_\lambda$.} 

\begin{proof}
Since $f_{\mu} \phi_{\mu\lambda} = f_\lambda$, all elements of an
equivalence class have the same image in $H$. Then $f$ is uniquely
determined by $f(\phi_\lambda  a) = f_\lambda  a$. 
\end{proof}

For\pageoriginale any two equivalence classes, choose representatives
$a_1$, $b_1$ in some $G_\nu$. Then   
\begin{align*}
f(\phi_{\nu} a_1 + \phi_\nu b_1) & = f \phi_\nu (a_1 + b_1)\\ 
& =  f \phi_\nu a_1 + f \phi_\nu b_1 
\end{align*}
since $f \phi_\nu = f_\nu$ is a homomorphism. Thus $f$ is a
homomorphism. 
