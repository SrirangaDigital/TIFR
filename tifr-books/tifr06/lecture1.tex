\chapter{Lecture 1}

\noindent
\textbf{Sheaves.}\pageoriginale

\begin{defi*}
A sheaf $\mathscr{S} = (S, \tau, X)$ of abelian groups is a  map $\pi:
S \xrightarrow{\rm onto}X$, where  $S$ and $X$ are topological spaces,
such that  
\begin{enumerate}
\item $\pi$ is a local homeomorphism,

\item for each $x \in X$, $\pi^{-1} (x)$ is an abelian group,

\item addition is continuous.
\end{enumerate}
\end{defi*}

That $\pi$ is a local homeomorphism means that for each point $p \in
S$, there is an open set $G$ with $p \in G$ such that $\pi |G$ maps
$G$ homeomorphically onto some open set $\pi(G)$. 

Sheaves were originally introduced by Leray in Comptes Rendus
222(1946)p. 1366 and the modified definition of sheaves now used was
given by Lazard, and appeared first in the Cartan Sem. 1950-51
Expose 14. 

In the definition of a sheaf, $X$ is not assumed to satisfy any
separation axioms. 

$S$ is called the sheaf space, $\pi$ the projection map, and $X$ the
base space. 

\textit{The open sets of $S$ which project homeomorphically onto open
  sets of $X$ form a base for the open sets of $S$.}
 
\begin{proof}
If $p$ is in an open set $H$, there exists an open $G$, $p \in G$ such
that $\pi|G$ maps $G$ homeomorphically onto an open set $\pi
(G)$. Then $H \cap G$ is open, $p \in H \cap  G \subset H$, and
$\eta | H \cap G$ maps $H \cap G$ homeomorphically onto $\pi(H \cap
G)$ open in $\pi(G)$, hence open in $X$.  
\end{proof}

\textit{$\pi$\pageoriginale is a continuous open mapping}

\begin{proof}
Continuity of $\pi$ follows from the fact that it is a local
homeomorphism, and openness follows from the result proved above. 
\end{proof}

The set $S_x = \pi^{-1}(x)$ is called the stalk of $S$ at $x$. $S_x$ is
an abelian group. If $\pi(p) \neq \pi(q)$, $p+q$ is not defined. 

\textit{$S_x$ has the discrete topology}.

\begin{proof}
This is a consequence of the fact that $\pi$ is a local
homeomorphism. 
\end{proof}

Let $S \times S$ be the cartesian product of the space $S$ with itself
and let $S + S$ be the subspace consisting of those pairs $(p,q)$ for
which $\pi(p) = \pi(q)$. \textit{Addition is continuous} means that
$f:S+S \to S$ defined by $f(p,q)=p+q$ is continuous. In other words,
if $p$, $q \in S$ and $\pi (p)= \pi(q)$, then given an open $G$
containing $p+q$, there exist open sets $H$, $K$ with $p \in H$, $q \in K$
such that if $r \in H$, $s \in K$ and $\pi(r) = \pi(s)$, then $r+s \in
G$. We may write this as $H + K \subset G$. 

\begin{proposition}%%% 1
 Zero and inverse are continuous.
\end{proposition}
\begin{enumerate}[(i)]
\item Writing $O_x$ for the zero element of the group $S_x$,
  \textit{zero is continuous} means that $f:X \to S$, where $f(x)
  =O_x$, is continuous. 

\begin{proof}
Let $x \in X$ and let $G$ be an open set containing $f(x)=O_x$. Then
there is an open set $G_1$ such that $O_x \in G_1 \subset G$ and
$\pi|G_1$ is a homeomorphism of $G_1$ onto open $\pi(G_1)$. Since
$O_x+ O_x = O_x$,\pageoriginale and addition is continuous, there
exist open sets 
$H$, $K$ with $O_x \in H$, $O_x \in K$ such that $H+K \subset G_1$. Let $L=
G_1 \cap H \cap K$, then $L$ is open, $O_x \in L$ and $\pi|L$ is a
homeomorphism of $L$ onto open $\pi(L)$. Then $x= \pi(O_x) \in \pi(L)$
and if $y \in \pi(L)$ there exists $q \in L$ with $\pi(q) =y$. Then $q
\in H$, $q \in K$ and hence $q+q \in G_1$. But $q \in G_1$, and
$\pi|G_1$ is $1-1$; hence $q= S_y \cap G_1$. Therefore, since $q+q \in
S_y \cap G_1$, $q+q=q$, hence $q=O_y$. Thus if $y \in \pi (L)$, $f(y) =
O_y \in L$. Thus $x$ is in open $\pi (L)$, with $f(\pi(L))=L \subset
G$. Hence $f$ is continuous. (Incidentally we have proved that each
$O_x$ is contained in an open set which consists  of zeros only and
which projects homeomorphically onto an open set of  $X$.) 
\end{proof}

\item Writing-$p$ for the inverse of $p$ in the group
  $S_{\pi(p)}$ \textit{inverse is continuous} means that $g: S \to S$ 
  where $g(p)=-p$ is continuous. 

\begin{proof}
Let $p \in S$, and let $G$ be an open set containing-$p$. Then there
exists open $L$ containing $O_{\pi (p)}$ and consisting of zeros
only. Since $p+(-p)= O_{\pi (p)}$ and addition is continuous, there
exist open $H$, $K$, with $p \in H$, $-p \in K$ and $H+K \subset L$. Hence
if $q \in H$, $r \in K$, $\pi(r) = \pi(q)$, then $q+r = O_{\pi(q)}$,
i.e. $r=-q$. We may assume that $\pi|H$ is a homeomorphism. Let  
$$
H_1= (\pi|H)^1 (\pi(H)\cap \pi (K \cap G)), 
$$
then $p \in H_1$, and since $\pi$ is open, continuous, $H_1$ is
open. Then if $q \in H_1$, there exists $r \in K \cap G$ with $\pi(r)
= \pi(q)$; then $r=-q=g(q)$.\pageoriginale Thus $H_1$ is open and
$g(H_1) \subset 
G$. Hence $g$ is continuous. 
\end{proof}
\end{enumerate}

\begin{corollary}%coro 1
Subtraction is continuous.
\end{corollary}

\noindent
i.i. $f:S + S \to S$, where $f(p,q)=p-q$ is continuous.

\begin{corollary}\label{chap1:cor2}
 The set of all zeros is an open set.
\end{corollary}

\begin{exam}%exe1
If $X$ is a topological space, and $G$ is an abelian group furnished
with the discrete topology, let $\mathscr{S}= (X \times G, \pi, X)$
where $\pi(x,g)=x$ and $(x,g_1)+(x,g_2)= (x,g_1 + g_2)$. Each stalk
$S_x$ is isomorphic to $G$. Axioms a) and c) are easily
verified. This sheaf is called the \textit{constant sheaf} associated
with $G$. 
\end{exam}

\begin{exam}\label{chap1:exam2}%exe 2
Form the constant sheaf $(A \times Z, \pi , A)$ where A is the square 
$\{ (x,y): 0 \leq x \leq 1, 0 \leq 0 \leq y \leq 1 \}$, and $Z$ is the
group of integers. Then identify $(x,0)$ with $(1-x,1)$ in $A$ to get
a M\"obius band $X$, and identify $(x,0,n)$ with $(1-x,1, -n)$ in $A
\times Z$ to get $S$. The resulting sheaf $\mathscr{S}=(S, \pi , X)$
is the sheaf of ``twisted integers'' over the M\"obius band. Each $S_x$ is
isomorphic to the group of integers. 
\end{exam}

\medskip
\begin{exam}\label{chap1:exam3}%exe 3
Let $X$ be the sphere of complex numbers. Let $S_x$ be the additive
group of function elements at $x$, each function element being a power
series converging in some neighbourhood of $x$. Let $S = \cup_x S_x$
and define $\pi : S \to X$ by $\pi(S_x) =x$. If $p$ is\pageoriginale a function
element, a neighbourhood of $p$ in $S$ is defined by analytic
continuation. Then $\mathscr{S}= (S, \pi, X)$ is the sheaf of \textit{
  analytic function elements}. Each component (maximal connected
subset) of $S$ is a Riemann surface without branch points. The sheaf
space $S$ is Hansdorff.  
\end{exam}
