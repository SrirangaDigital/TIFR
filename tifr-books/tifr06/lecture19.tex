\chapter{Lecture 19}%lect 19

\begin{defi*}%defi 0
A \textit{double complex}\pageoriginale $K$  is a system of  $ A-$
modules  ($A$ is 
a commutative ring with unit element $ \bigg\{ K^{p,q} \bigg\} $,
indexed by pairs $(p,q)$ of integers, together with homomorphisms
$d_1$ and $d_2$ with  
\begin{gather*}
d^{p,q}_1 : K^{p-1,q} \rightarrow K^{p,q},\quad   d^{p,q}_2 : K^{p,q -1}
\rightarrow K^{p,q}, \\ 
d^{p+1,q}_1 \cdot d^{p,q}_1 = 0, \quad d^{p,q+1}_2 d^{p,q}_2 = 0, \\ 
d^{p+1,q}_2 d^{p+1,q-1}_1 +  d^{p+1,q}_1 d^{p,q}_2 = 0, 
\end{gather*}
 (i.e., $d_1$ and $d_2$ are differential operators of bi degree (1, 0)
and (0, 1) respectively, which anticommute. Usually we omit 
the  superscripts attached to $d_1$ and $d_2$.) We have then the
\textit{anticommutative} diagram:  
\[
\xymatrix{
& \ar[d] & \ar[d] & \ar[d] & \\
\ldots \ar[r] & K^{p-1,\, q-1} \ar[r]^{d_2} \ar[d]_{d_1} & K^{p-1, \,
  q} \ar[d]_{d_1}\ar[r]^{d_2} & K^{p-1, \, q+1} \ar[r]\ar[d]_{d_1}&
\ldots\\
\ldots \ar[r] & K^{p,q-1} \ar[r]^{d_2} \ar[d] & K^{p,q} \ar[r]^{d_2}
\ar[d] & K^{p,p+1} \ar[r] \ar[d] & \ldots\\
& & & &
}
\]

\noindent
(Each row and column of a double complex forms a (single) complex
with the homomorphisms  $d_2$ and $d_1$  respectively.)   
\end{defi*}

\begin{defi*}%defi 0 
A \textit{subcomplex} $L$ of $K$  is a system of submodules $ L^{p,q} 
\subset K^{p,q} $ stable under $d_1$ and $d_2$; thus $ d_1 ( L^{p-1,q}
) \subset L^{p,q} $ and\pageoriginale $d_2 ( L^{p,q-1} ) \subset L^{p,q}$. 
\end{defi*}

If $L$ is a subcomplex of a double complex $K$, then clearly $K/L = 
\bigg\{K^{p,q} / L^{p,q} \bigg\} $ with the homomorphisms induced by $
d_1 $ and $ d_2 $ is again a double complex. 

Let $ Z^{p,q}_1 (K) $ be the kernel of $ d_1 : K^{p,q} \rightarrow
K^{p+1,q}$ and let $ B^{p,q}_1 (K) $  be the image of $d_1 : K^{p-1,q}K^{p,q}
$. Since $ d^2_1 = 0 $, $ B^{p,q}_1 \subset Z^{p,q}_1  \subset K^{p,q}$. 

Now $d_1 (Z^{P-1,Q})_1 = 0 \in Z^{p,q}_1 $ and, since $d_1 d_2   (
Z^{p,q-1}_1)\break  = -d_2 d_1 (Z^{p,q-1}_1 ) = 0$,  $d_2 ( Z^{p,q-1}_1 )
\subset  Z^{p,q}_1 $. Thus  $Z_1 (K) = \bigg\{Z^{p,q}_1 \bigg\} $ is  a
subcomplex of $K$. 

Also $ d_1 (B^{P-1,q}_1 ) = 0 \in B^{p,q}_1 $ and  $ d_2 (B^{p,q-1}_1 ) 
= d_2d_1 ( K^{p-1,q-1} ) = -d_1d_2 ( K^{p-1,q-1} ) \subset B^{p,q}_1
$. Thus $ B_1 (K)  = \bigg\{ B^{p,q}_1 \bigg\} $ is a subcomplex of  $
Z_1 (K) $. Let $ H_1 (K) = Z_1 (K) / B_1 (K) $ with $ H_1 (K)
=\bigg\{ H^{p,q}_1 (K) = Z^{p,q}_1 /B^{p,q}_1 \bigg\} $.  In the
double complex $H_1 (K) $, the homomorphism induced by $d_1$ is the
trivial (zero) homomorphism. 
\[
\xymatrix@R=0.5cm{
\ldots \ar[r] & H^{p-1,q-1}_1 \ar[r]^{d_2} & H^{p-1,q}_1 \ar[r]^{d_2}
& H^{p-1, q+1}_1 \ar[r] & \ldots\\
\ldots \ar[r] & H^{p,q-1}_1 \ar[r]^{d_2} & H^{p,q}_1 \ar[r]^{d_2} &
H^{p,q+1}_1 \ar[r] & \ldots
}
\]

Similarly if $ Z^{p,q}_2 = \ker d_2 $ and $B^{p,q}_2 = \im d_2$ there is a
double complex $ H_2 (K) $ with $ H_2 (K) = \bigg\{H^{p,q}_2 (K) =
Z^{p,q}_2| B^{p,q}_2 \bigg\} $. In $ H_2 (K) $, the homomorphism
induced by $d_2$ is the trivial homomorphism. 
\[
\xymatrix{
\ar[d] & \ar[d] & \ar[d]\\
H^{p-1,q-1}_2 \ar[d]_{d_1}& H^{p-1,q}_2 \ar[d]_{d_1}&
H^{p-1,q+1}\ar[d]_{d_1} \\
H^{p,q-1}_2 \ar[d] & H^{p,q}_2 \ar[d] & H^{p,q+1}_2\ar[d]\\
& & & 
}
\]\pageoriginale

In particular, there is a double complex $ H_2 (H_1 (K)) $, which we
write as $ H_{12} (K) = \bigg\{H^{p,q}_{12} (K) \bigg\} $, where $
H^{p,q}_{1 2} = \bigg\{Z^{p,q}_{1 2} B^{p,q}_{1 2} \bigg\} $ and  $
Z^{p,q}_{12} = \ker d_2 : H^{p,q}_1 \rightarrow  H^{p,q+1}_1$; $ B_{1
  2} = \im d_2 : H^{p,q-1}_1 \rightarrow  H^{p,q}_1 $. In the double
complex $H_{1 2} (K) $, the induced homomorphisms $d_1$ and $d_2$ are
the trivial homomorphisms.  

Similarly there is a double complex 
$$H_{21} (K) = \bigg\{ H^{p,q}_{2
  1} (K) \bigg\} = H_1 (H_2 (K)).$$  

\noindent{\textbf{Notations.}}
In terms of the more usual notation, $ H^{p,q}_{1 2} (K)   = H^q_{II}
( H^p_I  (K) ) $ and $ H^{p,q}_{2 1} (K) ) =  H^{p}_{I} (K) (H^q_{II}
(K) )$. 
 
 To the double complex $ K = \bigg\{ K^{p,q}, d_1,d_2 \bigg\} $  we
 can now associate the (single) complex $ \bigg\{ K^n,d \bigg\} K^n$
 being the direct sum $ K^n = \sum \limits_{p+q =n} K^{p,q} $ (each
 $K^n $ is an $A-$ module) with the differential operator $ d = d_1 +
 d_2 : K^{n-1} \rightarrow K^n$. ($d$ is a homomorphism and  $ d^2
 = d^2_1 + d_1 d_2 + d_2 d_1 + d^2_2 = 0$).   
 $$ 
 \cdots \rightarrow K^{n-1} \xrightarrow{d^n} K^n
 \xrightarrow{d^{n+1}} K^{n+1} \rightarrow \cdots  
 $$
 
 Thus $ \im d^n \subset \ker d^{n+1} $ and there are  cohomology
 modules $H^n (K) = \ker d^{n+1} / im d^n $. 

\begin{defi*}%defi 
A {\em homomorphism}\pageoriginale $f: K \rightarrow L$ (of bidegree
$(r,s)$) of double complexes is a system of homomorphisms $f : K^{p,q}
\rightarrow L^{p+r , q+s}$. 
\end{defi*}

\begin{defi*}%defi 0
{\em A map} $f : K \rightarrow L$ of double complexes is a
homomorphism of bidegree (0, 0), which commutes with $d_1$ and
$d_2$. 
\end{defi*}

Clearly a map $f : K \rightarrow L$ induces homomorphisms
\begin{align*}
& f^+ : H^{p,q}_1 (K) \rightarrow H^{p,q}_1 (L),  f^* : H^{p,q}_{12} (K) 
\rightarrow H^{p,q}_{12} (L)\\
\text{and } \qquad & f^* : H^{p,q}_{21} (K) \rightarrow
H^{p,q}_{21} (L) 
\end{align*}

Also, $f$ determines obvious homomorphisms $f : K^n \rightarrow L^n$
which commute with $d = d_1 + d_2$ and there are induced homomorphisms
$f^* : H^n (K) \rightarrow H^n(L)$ 

\begin{defi*} %defi 0
A sequence
$$
\cdots \rightarrow K_{r-1} \xrightarrow{h_r} K_r \xrightarrow{h_{r+1}}
K_{r+1}\rightarrow \cdots 
$$
of homomorphisms of bidegree (0, 0) of double complexes is called {\em
  exact} if, each pair $(p, q)$, the sequence 
$$
\cdots \rightarrow K^{p,q}_{r-1} \rightarrow K^{p,q}_r \rightarrow
K^{p,q}_{r-1} \rightarrow \cdots 
$$
is exact. 
\end{defi*}

\textit{Given an exact sequence of maps of double complexes}  
$$
0 \rightarrow K' \xrightarrow{i} K \xrightarrow{j} K'' \rightarrow 0,  
$$
\textit{there is an exact cohomology sequence}  
$$
\cdots \rightarrow H^n(K') \xrightarrow{i^\ast} H^n(K) \xrightarrow{j^\ast} 
H^n(K'') \xrightarrow{d^\ast} H^{n+1}(K') \rightarrow \cdots 
$$  

\begin{proof}
The sequences\pageoriginale
$$
0 \rightarrow K'^n \xrightarrow{i} K^n \xrightarrow{j} K''^n
\rightarrow 0 
$$
are clearly exact for each each $n$, and $d$ commutes with $i$ and 
$j$. Then, using the standard arguments of Lecture 10, we obtain the
result. 
\end{proof}

\begin{defi*}% defi 0
Two maps of double complexes, $f: K \rightarrow L$ and $g: K 
\rightarrow L$ are called {\em homotopic} ($f \circ g$) if there exist
homomorphisms $h_1 : K^{p+1,q} \rightarrow L^{p,q}$ and  $h_2 :
K^{p,q+1} \rightarrow L^{p,q}$ (i.e., $h_1$ and $h_2$ are
homomorphisms $K \rightarrow L$ of bidegree ($-1, 0$) and ($0, -1$)
respectively) such that 
\begin{gather*}
d_1 h_1 + h_1 d_1 + d_2 h_2 + h_2 d_2 = g-f, \\
d_1 h_2 = - h_2 d_1, \qquad d_2 h_1 = -h_1 d_2. 
\end{gather*}
\end{defi*}
(\textit{Homotopy of maps is obviously an equivalence relation.})

\textit{Homotopic maps $f : \rightarrow L$ and $g : K \rightarrow L$
induce the same homomorphisms} 
\begin{enumerate}[(i)]
 \item $f^*  = g^* : H^{p,q}_{12}(K) \rightarrow H^{p,q}_{12}(L)$,
 \item $f^*  = g^* : H^{p,q}_{21}(K) \rightarrow H^{p,q}_{21}(L)$,
 \item $f^*  = g^* : H^n (K) \rightarrow H^n (L)$.
\end{enumerate}

\begin{proof}
Let ($h_1 , h_2$)  be the pair of homomorphisms $K \rightarrow L$
which express the homotopy between $f$ and $g$. Since $h_2$, like
$d_2$, anticommutes with $d_1$, there are induced homomorphisms 
$$
h^+_2 :  H^{p, q+1}_1 (K) \rightarrow  H^{p, q}_1 (L).
$$ 

Further,\pageoriginale since	
$$
d_1 h_1 + h_1 d_1 = g - f - d_2 h_2 - h_2 d_2,
$$
$h_1$ expresses the homotopy of $g$ and $f + d_2 h_2 + h_2 d_2$ from a
column complex of $K$ to the corresponding column complex of
$L$.
\end{proof}

 Hence 
\begin{align*}
f^+ + d_2 h^+_2 + h^+_2 d_2 & = g^+ :  H^{p, q}_1 (K) \rightarrow  H^{p,
  q}_1 (L), \\
\text{i.e., } \qquad \qquad 
d_2 h^+_2 + h^+_2 d_2 & = g^+ - f^+. \qquad 
\end{align*}

Thus $h^+_2$ expresses the homotopy of $g^+$ and $f^+$ from a row
complex of $\bigg\{  H^{p, q}_1 (K) \bigg\}$ to the corresponding row
complex of $\bigg\{  H^{p, q}_1 (L) \bigg\}$.  

Hence
$$
f^\ast = g^\ast :  H^{p, q}_{12} (K) \rightarrow  H^{p, q}_{12} (L). \text {
  This proves } (i). 
$$

The proof of (ii) is carried out in a similar manner, using the
other anti-commutativity $d_2 h_1 = - h_1 d_2$. 

To prove (iii), let $h = h_1 + h_2 : K^{n+1} \rightarrow L^n$. Then 
\begin{align*}
dh + hd & =(d_1 + d_2) (h_1 + h_2) + (h_1 + h_2) (d_1 + d_2)\\ 
& = (d_1 h_1 + h_1 d_1 + d_2 h_2 + h_2 d_2) + (d_1 h_2 + h_2 d_1) +
(d_2 h_1 + h_1 d_2)\\ 
& = g-f.
\end{align*}

Thus $h$ is a homotopy of the complexes $\big\{ K^n \big\}$ and
$\big\{ L^n \big\}$, and we obtain $f^\ast = g^\ast : H^n(K) \rightarrow H^n
(L)$. 
  
  \begin{note*}%note 0
In the double complexes which occur in the usual applications, one has
commutativity $d_1 d_2 = d_2 d_1$, $d_1 h_2 = h_2 d_1$
and\pageoriginale $d_2 h_1 = 
h_1 d_2$ rather than anti-commutativity. The commutative case can be
transformed into the anti-commutative case and vice versa by replacing
$d_2 $ by $(-1)^P d_2 : K^{p,q-1} \rightarrow K^{p,q}$ and $h_2$ by
$(-1)^P h_2 : K^{p,q-1} \rightarrow K^{p,q}$. These substitutions do
not change ker $d_2$, $\im d_2$, etc., and so the cohomology modules
$H^{p,q}_{1}$, $H^{p,q}_{12}$, etc., remain unchanged. But if $K$ is a
commutative double complex, $K^n$, $d$, $H^n(K)$ are understood to refer
to the associated anticommutative double complex. 
\end{note*}  
  
