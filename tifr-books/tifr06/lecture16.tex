 \chapter{Lecture 16}\label{chap16:lec16}%chap 16 
 
 Let\pageoriginale $\{l_i\}_{i \in I}$ be a system of endomorphisms of a
 fine sheaf $\mathscr{S}$ corresponding to a locally finite covering
 $\{U_i\}_{i \in I}$ of a normal space $X$. Each $l_i$ gives a
 homomorphism $l_i (U) :\Gamma (U, \mathscr{S}) \to \Gamma (U,
 \mathscr{S})$ for each open $U$ and $\sum\limits_{i \in I}
 l_i (U)$ has a meaning and is the identity endomorphism of $\Gamma (U
 \mathscr{S})$. Also $l_i$ determines a homomorphism  
 $$
 \bar{l_i} (U) : \Gamma (U_i \cap U, \mathscr{S}) \to \Gamma (U,
 \mathscr{S}) 
 $$
 defined by 
 \begin{align*}
(\bar{l}_i (U) g) (x) & = l_i (g(x)) \qquad \text{ if } x \in
   U_i \cap U,\\ 
& = 0 \qquad  \qquad \text{ if } x \in (X -U_i) \cap U. 
 \end{align*} 
 
 One verifies that the following diagrams are commutative.
{\fontsize{9}{11}\selectfont
\[
\xymatrix{
\Gamma(U,\mathscr{S}) \ar[r]^{l_i(U)} \ar[d]_{\rho (U_i \cap U,
  \mathscr{S})} & \Gamma(U,\mathscr{S}) \ar[d]^{\rho(U_i\cap U,
  \mathscr{S})} \\
\Gamma(U_i \cap U, \mathscr{S}) \ar[r]_{l_i (U_i \cap U)}
\ar[ur]^{\bar{l}_i (U)}& \Gamma
(U_i \cap U, \mathscr{S})
}\quad
\xymatrix{
\Gamma(U_i\cap U, \mathscr{S}) \ar[r]^{\bar{l}_i(U)} \ar[d]_{\rho (U_i \cap V,
 U_i\cap U)} & \Gamma(U,\mathscr{S}) \ar[d]^{\rho_{_{VU}}} \\
\Gamma(U_i \cap V, \mathscr{S}) \ar[r]^{\bar{l}_i(V)}
&  \Gamma(V, \mathscr{S})
}
\]}\relax

\textit{If $X$ is normal, $\mathscr{S}$ is fine and $\mathscr{U} = \{
  U_i\}_{i \in I}$ is a locally finite covering of $X$, then
  $H^q (\mathscr{U}, \mathscr{S}) \to 0$ for $q \geqq 1$.}
 
\begin{proof}
Let $k^{q - 1} : C^q (\mathscr{U}, \mathscr{S}) \to C^{q-1}
(\mathscr{U}, \mathscr{S})$ for $q \geqq 1$ be the homomorphism
defined by 
 $$
 (k^{q-1} f) (\sigma) = \sum_{i \in I} \bar{l}_i (U_\sigma) (f 
(i \sigma)),  
$$ 
 where $i \sigma=i$, $i_o ,\ldots,i_{q-1}$ if $\sigma = i_o
 ,\ldots,i_{q-1}$. (This infinite sum of sections is finite neglecting
 zeros, in some neighbourhood of each point.)\pageoriginale Using the
 fact that 
 $\partial_o (i \sigma) = \sigma$ and $\partial_j (i \sigma) =i
 \partial_{j-1} (\sigma)$ for $j > 0$, one verifies that  
 $$
 \delta^q k^{q-1} f + k^q \delta^{q+1} f =f. 
 $$
 
 (The computation is given at the end of the lecture.)
 
 Hence each cocycle $f$ is a coboundary $q.e.d$.
\end{proof}

 \begin{proposition}\label{chap16:prop12}%pro 12
For a fine sheaf $\mathscr{S}$ over a paracompact, normal space
$X$, $H^q (X, \mathscr{S}) = 0$ for $q \geqq 1$ 
\end{proposition}

\begin{proof}
$H^q (X, \mathscr{S}) = 0 \;  q \geqq 1$ for each locally finite covering
  $\mathscr{U}$ of the space $X$. Since the space $X$ is paracompact,
  this means that $H^q (X, \mathscr{S}) = 0, q \geqq 1$.   
\end{proof}  

\begin{coro*}%coro 
If $X$ is paracompact and normal, any exact sequence of sheaves  
$$
0 \to \mathscr{G} \xrightarrow{e} \mathscr{S}^o \xrightarrow{d^1}
\cdots \to \mathscr{S}^{q-1} \xrightarrow{d^q} \mathscr{S}^q \to
\ldots, 
$$
where each $\mathscr{S}^q (q \geqq 0)$ is fine, is a 
  resolution of $\mathscr{G}$. 
\end{coro*}

\begin{defi*}%defi 0
$A$ sheaf $ \mathscr{S}$ is called locally fine, if for each
  open $U$ and each $x \in U$, there is an open $V$ with $x
  \in V \subset U$ such that the restriction of $\mathscr{S}$
  to $\bar{V}$ is fine. 
\end{defi*}

\textit{If $X$ normal, a fine sheaf $\mathscr{S}$ is locally fine.} 

\begin{proof}
The restriction of $\mathscr{S}$ to an arbitrary closed set is fine. 
\end{proof}

\begin{proposition}\label{chap16:prop13}%prop   13
If\pageoriginale $X$ is paracompact normal, a locally fine sheaf
$\mathscr{S}$ is fine. 
\end{proposition}

\begin{proof}
Let $E \subset G_1$, with $E$ closed and $G_1$ open. If $G_2 = X-E$
then $\{ G_i\}_{i=1,2}$ is a covering of $X$. Since $\mathscr{S}$ is
locally fine, for each $x \in G_i$, there is an open $V_x$
with $x \in V_x \subset G_i$ such that the restriction of
$\mathscr{S}$ to $\bar{V}_x$ is fine. Since $X$ is paracompact, there
is a locally finite refinement $ \mathscr{U}\{U_j\}_{j \in J}$
of $\{V_x\}_{x \in X}$, hence $\mathscr{U}$ is also a
refinement of $\{G_i\}$. If $U_j \subset V_x$, then $\bar{U}_j \subset
\bar{V}_x$, and $\bar{V}_x$, being closed in $X$, is normal. Since the
restriction of $\mathscr{S}$ to $\bar{V}_x$ is fine, the restriction
of $\mathscr{S}$ to $\bar{U}_j$ is also fine. Now by proposition 11,
there exist endomorphisms $l_j$ such that $\sum\limits_{j \in
  J} l_j =1$ and $l_j$ is zero outside a closed set $E_j \subset
U_j$. Choose the function $\tau : J \to (1,2)$ so that $U_j \subset
G_{\tau (j)}$ and let   
$$
l_i = \sum_{ \tau (j) = i} l_j, \quad i= 1, 2. 
$$

Then $I_1 + I_2 = l$ and
$$
l_i (s)  = 0 \text{ if } \pi (s) \in X - \bigcup _{\tau (j) =
  i} U_j \supset X -G_i. 
$$
Hence
$$
l_1 (s)  = 0  \quad \text{if} \quad  \pi (s) \in X - G_1
$$  
and 
$$
l_1 (s)  = s \quad \text{if} \quad \pi (s) \varepsilon X -G_2 = E.
$$

$l_1$ thus gives the required function, and this completes the proof. 
\end{proof}

\begin{coro*}
If $X$ is paracompact and normal, any exact sequence of sheaves  
$$
0 \to \mathscr{G} \xrightarrow {e} \mathscr{S}^o \xrightarrow{d^1}
\dots \mathscr{S}^{q-1} \xrightarrow{d^q} \mathscr{S}^q \cdots, 
$$
where\pageoriginale each $\mathscr{S}^q (q \geqq 0)$ is locally fine,
is a resolution of $\mathscr{G}$.  
\end{coro*}

The following examples shows that, in more general spaces, fineness
need not coincide with local fineness.  

\begin{exam}%exam 26
Let $X$ have points $a, b ,\ldots, h$ with base for open sets
consisting of $(f)$, $(g)$, $(h)$, $(d,f,h)$, $(e,g,h)$, $(c,f,g)$,
$(b,e,g,h)$, $(a,d,e,f,\break g,h)$. Let $\mathscr{S}$ be the subsheaf of the
constant sheaf $Z_2$ formed by omitting $(c,1)$, $(f,1)$. Then
$\mathscr{S}$ is fine but not locally fine. In fact, $V = (c,f,g)$ is
the least open set containing $c$ and the restriction of $\mathscr{S}$
to $\bar{V} = X -(h)$ is not fine. ($X$ is not normal.)  
\end{exam}

\begin{exam}%exam 27
Let $T$ be the space of ordinal numbers $\leqq \omega_1$ with the
usual topology induced by the order. Let $A$ be the space or ordinal
numbers $\leqq \omega_o$ and let $X$ be the subspace of $T X A$
formed by omitting the point $(\omega_1, \omega_o)$. Let $\mathscr{S}$
be the constant sheaf $Z_2$ over $X$. Then $\mathscr{S}$ is locally
fine, for every point has a closed neighbourhood which is normal 
 and zero dimensional. But $\mathscr{S}$ is not fine. If $B$ is
the set of even numbers, then $B \subset A$. Let $E = \omega_1 \times
B$ and $G = T \times B$. Then $E \subset G \subset X$ with $E$ closed
and $G$ open. There is no endomorphism of $\mathscr{S}$ which is the
identity on $E$ and is zero outside $\bar{G}$. ($X$ in neither
paracompact nor normal.)     
\end{exam}

\begin{exam}%exam 28
The\pageoriginale space $M$ of Quart. Jour. Math. 6 (1955), p. 101 is
normal and locally zero dimensional but not zero
dimensional. Therefore the constant sheaf $Z_2$ is locally fine but
not fine. ($M$ is not paracompact).\pageoriginale    
\begin{align*}
& \delta^q k^{q-1} f+ k^q \delta^{q+1} f =f.\\
& \delta^q k^{q-1} f(\sigma)  = \sum^q_{j=0} (-1)^j \rho (U_\sigma,
  U_{\partial_j}\sigma) (k^{q-1} f) (\partial_j \sigma)\\ 
= & \sum^q_{j=o} (-1)^j \rho (U_{\sigma} ,U_{\partial_j} \sigma)
\sum_i \bar{I}_i (U_{\partial_j} \sigma) f (i \partial_j \sigma)\\ 
= & \sum^q_{j=0} (-1)^j \sum_i \bar{l}_1 (U_\sigma) \rho (U_{i
  \sigma}, U_{j \partial_j} \sigma) f (i \partial_j \sigma).\\ 
& k^q \delta^{q+1} = \sum_i \bar{l}_1 (U_\sigma) (\delta^{q+1} f) (i
\sigma)\\ 
= & \sum_i \bar{l}_i (U_\sigma) \sum^{q+1}_{j=0} (-1)^j \rho (U_i
\sigma, U_{\partial_j} i \sigma) f (\partial_j i \sigma)\\ 
= &\sum_i \bar{l}_i (U\sigma) \rho (U_i \sigma, U_\sigma) f (\sigma)\\
&{} +\sum_i \bar{l}_i (U\sigma) \sum^{q+1}_{j=1} (-1)^j \rho (U_j \sigma,
U_i \partial_{j-1} \partial) f (i\partial_{j-1} \sigma)\\ 
= & \sum_i l_i (U_\sigma) f (\sigma) \sum_i \bar{l}_i (U_\sigma)
\sum^q_{j=0} (-1)^{j+1} \rho (U_{i_\sigma}, U_{i \partial_j \sigma}) f (i
\partial_j \sigma).\\ 
& \delta^q k^{q-1} f(\sigma) + k^q \delta^{q+1} = \sigma_i l_i
(U_\sigma) f (\sigma) = f (\sigma)
\end{align*}
\end{exam}


