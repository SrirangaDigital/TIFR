\chapter{Lecture 10}

Let\pageoriginale $h = \{ h_U \} : \Sigma' \to \sum$ be a homomorphism of
presheaves, i.e., each $h_U : S'_U \to S_U$ is a homomorphism and, if
$V \subset U$, $h_V \rho'_{VU} = \rho_{VU} h_U$. We define, for each
$q\ge0$, the mapping  
$$
h^+ : C^q (\mathscr{U}, \Sigma') \to C^q (\mathscr{U}, \sum) 
$$
by $(h^+ f) (\sigma) = h_{U_\sigma} f(\sigma)$. Then, $h^+$ is a
homomorphism since each $h_{U_\sigma}$ is a homomorphism.  

\textit{$h^+$ commutes with $\delta$} 

\begin{proof}%proof
\begin{align*}
(h^+ \delta^{q+1} f) (\sigma) & = h_{U_\sigma} (\delta^{q+1} f) (\sigma)\\
& = h_{U_\sigma} \sum^{q+1}_{j=0} (-1)^j \rho' (U_\sigma ,
  U_{\partial_j \sigma}) f(\partial_j \sigma), 
\end{align*}
\begin{align*}
\text{and }\qquad  (\delta^{q+1} h^+ f) (\sigma) & = \sum^{q+1}_{j=0}
(-1)^j \rho(U_\sigma , U_{\partial_j \sigma}) (h^+ f) (\partial_j
\sigma)\\ 
& = \sum^{q+1}_{j=0} (-1)^j \rho(U_\sigma , U_{\partial_j \sigma})
h_{U_\partial j \sigma } f(\partial_j \sigma)\\ 
& = h_{U_\sigma} \sum^{q+1}_{j=0} (-1)^j \rho' (U_\sigma ,
U_{\partial_j \sigma}) f(\partial_j \sigma) 
\end{align*}
\hfill Q. e. d.

\noindent
\textit{Hence $h^+$ induces a homomorphism $h_{\mathscr{U}} : H^q
  (\mathscr{U}, \Sigma') \to H^q (\mathscr{U}, \sum)$. $h^+$ commutes
  with $\tau^+$} 
\end{proof}

\begin{proof}%proof
\begin{align*}
(h^+ \tau^+ f) (\sigma ) & = h_{V_\sigma} (\tau^+ f) (\sigma )\\
& = h_{V_\sigma} \rho' (V_\sigma , U_{\tau \sigma}) f(\tau \sigma )\\
& = \rho (V_{\sigma}, U_{\tau \sigma}) h_{U_{\tau \sigma}} f(\tau \sigma )\\
& = (\tau^+ h^+ f) (\sigma).
\end{align*}

\textit{Hence}\pageoriginale $h_{\mathscr{W}} \tau_{\mathscr{W} \mathscr{U}} =
\tau_{\mathscr{W} \mathscr{U}} h_{\mathscr{U}} : H^q (\mathscr{U},
\Sigma') \to H^q (\mathscr{W}, \sum)$, i.e., the following diagram is
commutative. 
\end{proof}
\[
\xymatrix{
H^q(\mathscr{U}, \sum') \ar[r]^{h_{\mathscr{U}}}
\ar[d]_{\tau_{\mathscr{W}\mathscr{U}}} &
H^q(\mathscr{U},\sum)\ar[d]^{\tau_{\mathscr{W}\mathscr{U}}}\\
H^q (\mathscr{W}, \sum') \ar[r]_{h_{\mathscr{W}}} & H^q(\mathscr{W},\sum).
}
\]

\textit{If $\sum' \xrightarrow{h} \sum \xrightarrow{g} \sum''$ is
  a  sequence of homomorphisms of presheaves, then $g h$ induces a
  homomorphism 
$$
(gh)^+ : C^q (\mathscr{U}, \sum') \to C^q (\mathscr{U}, \sum'') 
$$
such that $(gh)^+ = g^+ h^+$ and, if $h$ is the identity, $h^+$ is the
identity.} 

\begin{proof}%proof
\begin{align*}
((gh)^+ f) (\sigma) & = (gh)_{U_\sigma} f(\sigma)\\
& = g_{U_\sigma} h_{U_\sigma} f(\sigma)\\
& = (g^+ h^+ f) (\sigma).
\end{align*}

If $h$ is the identity, i.e., if each $h_U$ is the identity, then
$(h^+ f) (\sigma) = h_{U_\sigma} f(\sigma) = f(\sigma)$; so $h^+$ is
the identity, q.e.d. 
\end{proof}

If one has the commutative diagram of homomorphisms of preshea\-ves,
i.e., $g h = h_1 g_1$, then 
\[
\xymatrix{
\sum \ar[r]^{h} \ar[d]_{g_1} & \sum'\ar[d]^g\\
\sum''_1 \ar[r]_{h_1} & \sum''
}
\]
$$
 g^+ h^+ = (gh)^+ = (h_1 g_1)^+ = h^+_1 g^+_1, 
$$ 
and hence $g_{\mathscr{U}} h_{\mathscr{U}} = h_{1 \mathscr{U}}
g_{1 \mathscr{U}}$. 

\textit{If\pageoriginale $\mathscr{U} =\{U_i\}_{i \in I}$ is a covering and the 
  sequence $\sum' \xrightarrow{h} \sum \xrightarrow{g} \sum''$ of
  homomorphisms of presheaves is exact, then the sequence}  
$$
C^q (\mathscr{U}, \sum') \xrightarrow{h^+} C^q (\mathscr{U}, \sum)
\xrightarrow{g^+} C^q (\mathscr{U}, \sum'') 
$$
is also exact.

\begin{proof}%proof
\begin{enumerate}[(i)]
\item If $f \in C^q (\mathscr{U}, \sum)$ is an element of $\im h^+$,
  clearly $f \in \ker g^+$, hence $\im h^+ \subset \ker g^+$. 

\item Linearly order the index set $I$, and let $f \in C^q
  (\mathscr{U}, \sum)$ be an element of $\ker g^+$. Then $f(\sigma)
  \in \ker g_{U_\sigma} = im h_{U_\sigma}$ for each $q$-simplex
  $\sigma$, hence there is an element $r$ in the module corresponding
  to the open set $U_\sigma$, of the presheaf $\Sigma'$, such that
  $h_{U_\sigma} (r) = f(\sigma)$. If $\sigma = (i_0, \ldots , i_q)$
  with $i_0 < \cdots < i_q$, define the function $t$ on $\sigma$ by
  $t(\sigma) = r$. If $\sigma' = (j_0, \ldots , j_q)$ is a permutation
  of $\sigma = (i_0 , \ldots,  i_q)$ define $t$ on $\sigma '$ by
  $t(\sigma ') = \pm t(\sigma)$ 
  according as $\sigma '$ is an even or odd permutation of
  $\sigma$. If $\sigma$ is a $q$-simplex in which two indices are
  repeated, define $t(\sigma)$ to be zero. It then follows that $t \in
  C^q (\mathscr{U}, \Sigma')$ and it is easily verified that $h^+(t) =
  f$, hence $\ker g^+ \subset im h^+$. 
\end{enumerate}
\end{proof}

\textit{If the sequence $ 0 \to \sum' \xrightarrow{i} \sum
  \xrightarrow{j} \sum'' \to 0$ of homomorphisms of presheaves is
  exact, there is an induced homomorphism} 
$$
\delta_{\mathscr{U}} : H^q (\mathscr{U}, \sum'') \to H^{q+1}
(\mathscr{U}, \sum'). 
$$

\begin{proof}
Since the homomorphisms $i^+$, $j^+$ commute with the homomorphism
$\delta$, there is commutativity in the following diagram: 
\[
\xymatrix{
& \ar[d]_{\delta} & \ar[d]_{\delta} & \ar[d]_{\delta} & \\
0 \ar[r] & C^q(\mathscr{U},\sum') \ar[r]^{i^+} \ar[d]_{\delta} &
C^q(\mathscr{U}, \sum) \ar[d]_{\delta} \ar[r]^{j^+} &
C^q(\mathscr{U},\sum'') \ar[r]\ar[d]_{\delta} & 0 \\
0  \ar[r] & C^{q+1}
(\mathscr{U},\sum')\ar[r]^{i^+}\ar[d]_{\delta} &
C^{q+1}(\mathscr{U},\sum) \ar[r]^{j^+}\ar[d]_{\delta} & C^{q+1}
(\mathscr{U},\sum'') \ar[r]\ar[d]_{\delta} & 0\\
& & & & 
}\tag{$\ast$}
\]
\end{proof}\pageoriginale

Since the sequence $0 \to \sum'\to \sum \to \sum'' \to 0$ is
exact, each row of the diagram is an exact sequence of
homomorphisms. We will construct a homomorphism $\theta : Z^q
(\mathscr{U}, \sum'') \to H^{q+1} (\mathscr{U}, \sum')$ which is
zero on $B^q(\mathscr{U}, \sum'')$, and hence $\theta$ will induce
a homomorphism from $H^q(\mathscr{U}, \sum'') \to
H^{q+1}(\mathscr{U}, \sum')$. 

To do this, let $r \in Z^q (\mathscr{U}, \Sigma'')$, and choose $s \in
C^q (\mathscr{U}, \sum)$ with $j^+ s = r$. Since $\delta j^+ s = j^+
\delta s = \delta r = 0, \delta s \in \ker j^+$ and by exactness,
there is a unique $t \in C^{q+1} (\mathscr{U}, \Sigma')$ with $i^+ t =
\delta s$. Then $i^+ \delta t = \delta i^+ t = \delta \delta s = 0$,
hence $\delta t =0$. Let $\tau \in H^{q+1}(\mathscr{U}, \Sigma')$ be
the element represented by $t$. To show that $\tau$ is unique. let
$s_1, t_1$ be the result of a second such choice, then $j^+ (s - s_1)
= r-r = 0$ and $s - s_1 = i^+ u$ for a unique $u \in C^q (\mathscr{U},
\Sigma')$. Then since $i^+$ is a monomorphism and 
$$
i^+ (t - t_1) = \delta(s - s_1) = \delta i^+ u = i^+ \delta u,
$$
hence $t-t_1 = \delta u$. Thus $t$ and $t_1$ represent the same
element $\tau \in H^{q+1}(\mathscr{U}, \Sigma')$. 

Let $\tau = \theta (r)$. If $r = a r_1 + b r_2 \in Z^q (\mathscr{U},
\Sigma'')$, suppose that $r_1 = j^+ s_1$, $\delta s_1 = i^+ t_1$ and
that $r_2 = j^+ s_2, \delta s_2 = i^+ t_2$, and let $\tau_1 = \theta
(r_1)$, $\tau_2 = \theta (r_2)$ be the elements represented by
$t_1$\pageoriginale and $t_2$. Then since $j^+$, $\delta$ and $i^+$
are homomorphisms,  
$$
r = j^+(a s_1 + b s_2), \delta (a s_1 + b s_2) = i^+ (a t_1 + b t_2)
$$
and, since $a t_1 + bt_2$ represents $a\tau_1 + b \tau_2$, we have
$$
\theta(r) = a \theta (r_1) + b \theta (r_2) .
$$

\noindent
Thus $\theta : Z^q (\mathscr{U}, \Sigma'') \to H^{q+1} (\mathscr{U},
\Sigma')$ is a homomorphism. 

If $r \in B^q(\mathscr{U}, \Sigma'')$, let $r = \delta_v$. For some $w
\in C^{q-1} (\mathscr{U}, \sum) v = j^+ w$. Then $j^+ (\delta w)
=\delta j^+ w = r $ and there exists a unique $t \in C^{q+1}
(\mathscr{U}, \Sigma')$ with $i^+ t = \delta (\delta w) = 0$, hence $t
= 0$; i.e., $\theta (r) = 0$. Thus $\theta$ induces a homomorphism 
$$
\delta_{\mathscr{U}} : H^q (\mathscr{U}, \Sigma'') \to H^{q+1}
(\mathscr{U}, \Sigma') . 
$$
\hfill{Q.e.d.}

\textit{$\tau_{\mathscr{W}, \mathscr{U}}$ commutes with
  $\delta_{\mathscr{U}}$, i.e., the following diagram is
  commutative.} 
\[
\xymatrix{
H^q(\mathscr{U}, \sum'')
\ar[r]^{\delta_{\mathscr{U}}}\ar[d]_{\tau_{\mathscr{W}\mathscr{U}}} &
H^{q+1}(\mathscr{U}, \sum')\ar[d]^{\tau_{\mathscr{W}\mathscr{U}}} \\
H^q(\mathscr{W},\sum'') \ar[r]_{\delta_{\mathscr{W}}} &
H^{q+1}(\mathscr{W},\sum') 
}
\]

\begin{proof}
$\tau^+$ commutes with $j^+$, $\delta$, $i^+$.
\end{proof}

\textit{If $0 \to \sum' \xrightarrow{i} \sum \xrightarrow{j}
  \sum'' \to 0$ is exact, then the sequence} 
\begin{multline*}
0 \to H^0 \left(\mathscr{U}, \sum' \right) \to \cdots \to H^q \left(\mathscr{U},
\sum'\right) \xrightarrow{i_{\mathscr{U}}} \\
 H^q\left(\mathscr{U}, \sum \right)
\xrightarrow{j_{\mathscr{U}}} H^q \left(\mathscr{U},
\sum''\right)\xrightarrow{\delta_{\mathscr{U}}} H^{q+1} \left(\mathscr{U},
\sum' \right) 
\end{multline*}
\textit{is exact.}

\begin{proof} 
The\pageoriginale exactness of this sequence is the result of six
properties of the 
form $\ker \subset \im$ and $\im \subset \ker$. Each can be easily 
verified in $(*)$. (See Eilenberg-Steenrod, Foundations of Algebraic
Topology, p. 128). 
\end{proof}

\textit{If $0 \to \sum' \xrightarrow{i} \sum \xrightarrow{j}
  \sum'' \to 0$ and $0 \to \sum'_1 \xrightarrow{i_1} \sum_1
  \xrightarrow{j_1} \sum''_1 \to 0$ are exact sequence, and if $h :
  (\sum' , \sum, \sum'') \to (\sum'_1 , \sum_1 ,
  \sum''_1) $ is a homomorphism commuting with $i$, $j$, $i_1$, and
   $j_1$, then $h_{\mathscr{U}}$ commutes with
  $\delta_{\mathscr{U}}$.} 

\begin{proof}
The homomorphism $h^+$ commutes with the homomorphisms $j^+$, $\delta$
and $i^+$, $q. e. d$. 
\end{proof}

With the same assumptions as in the above statement, we then have the
following commutative diagram, in which each row is exact. 
{\fontsize{7}{10}\selectfont
\[
\xymatrix@C=0.3cm{
0\ar[r]& H^0(\mathscr{U}, \sum') \ar[r]\ar[d]_{h_{\mathscr{U}}} &
\ldots \ar[r]& H^q(\mathscr{U}, \sum)\ar[r]^{i_{\mathscr{U}}}
\ar[d]_{h'_{\mathscr{U}}} & H^q(\mathscr{U},
\sum)\ar[r]^{j\mathscr{U}}\ar[d]_{h\mathscr{U}} & H^q (\mathscr{U},
\sum'') \ar[r]^{\delta_{\mathscr{U}}} \ar[d]_{h''_{\mathscr{U}}} &
H^{q+1}(\mathscr{U}, \sum') \ar[d]_{h'_{\mathscr{U}}} \ar[r] & \ldots\\
0 \ar[r]& H^0(\mathscr{U},\sum'_1) \ar[r] & \ldots \ar[r] &
H^q(\mathscr{U},\sum'_1)  \ar[r]^{i_{1\mathscr{U}}} & H^q(\mathscr{U},
\sum_1) \ar[r]^{j_{1\mathscr{U}}} & H^q (\mathscr{U}, \sum''_1)
\ar[r]^{\delta_{1\mathscr{U}}} & H^{q+1}(\mathscr{U},\sum'_1) \ar[r]& \ldots
}
\]}\relax

\begin{defi*}%def
A proper covering of $X$ is a set of open sets whose union is $X$. 
\end{defi*}

A proper covering $\mathscr{U} =\{ U\}$ of $X$ may be regarded as an
indexed covering $\big\{U_U \big\}_{U \in \mathscr{U}}$ if each open
set of the covering is indexed by itself. Every covering $ \big\{U_i
\big\}_{i \in I}$ has a refinement which is a proper covering, e.g.,
the set of all open sets $U$ such that $U = U_i$ for some $i \in I$.  

Let\pageoriginale $\Omega$ be the set of  all proper coverings of $X$ and let
$\mathscr{U} < \mathscr{W}$ mean that $\mathscr{W}$ is a refinement of
$\mathscr{U}$. Then \textit{$\Omega$ is a directed set}, for 1)
$\mathscr{U}< \mathscr{U}$ 
2)  if $\mathscr{U} < \mathscr{W}$ and $\mathscr{W}<
\mathscr{W}$ then trivially $\mathscr{U} < \mathscr{W}$ and 3) given
$\mathscr{U}$, $\mathscr{W}$ there exists $\mathscr{W}$ with 
$\mathscr{U} < \mathscr{W}$, $\mathscr{W}<\mathscr{W}$, e.g,
$\mathscr{W}$ may be chosen to consists of all open sets $W$ with
$W=\cup \cap V$ for some $U \in \mathscr{U}$, $V \in \mathscr{W}$. 

(There is no set of \textit{all} indexed coverings). 

The system $\big\{ H^q (\mathscr{U},\sum), \tau _{\mathscr{W}\mathscr{U}}
\big\}_{\mathscr{U},\mathscr{W} \in \Omega}$ is then a direct system. Its
direct limit $H^q(X,\sum)$ is called the $q$-\textit{th cohomology
  module (over $A$) of $X$ with coefficients in $\sum$.} Let
$\tau_\mathscr{U}:H^q(\mathscr{U},\sum) \to H^q(X ,\sum)$ denote the
usual homomorphism in to the direct limit. 

\textit{If $h: \sum' \to \sum$ is a homomorphism of presheaves, there
  is a induced homomorphism} 
$$
h^*:H^q (X,\sum') \to H^q(X, \sum) \text{with } h^* \tau_\mathscr{U}=
\tau_\mathscr{U} h_\mathscr{U}. 
$$
\begin{proof}
This follows from the fact that $h_\mathscr{U} \tau_{\mathscr{W}
  \mathscr{U}}=\tau _{\mathscr{W}\mathscr{U}} h_\mathscr{U}$. 

\hfill {$Q.e.d$.}
\end{proof}

\textit{If $O\to \sum ' \xrightarrow{i} \sum \xrightarrow{j} \sum'' 
\to O$ is an exact sequence of presheaves, there is an
  induced exact sequence} 
{\fontsize{8}{10}\selectfont
\[
\xymatrix@C=0.3cm{
0\ar[r]&  H^0 (X, \sum')  \ar[r] & \cdots \ar[r]& H^q(X,\sum') \ar[r]
&  H^q(X,\sum) \ar[r] & H^q (X,\sum'') \ar[r]&  H^{q+1}(X, \sum ')
\ar[r]&  \cdots 
}
\]}\relax

\begin{proof}
This is a consequence of the fact that the direct limits of exact
sequences is again an exact sequence. 
\end{proof}

\textit{If\pageoriginale $h: (\sum ', \sum ,\sum'') \to (\sum '_1 ,\sum _1,
\sum''_1)$  is a homomorphism of exact sequence of presheaves} 
\[
\xymatrix{
0 \ar[r] & \sum'\ar[r]^i\ar[d]_{h'} & \sum \ar[r]^j\ar[d]_h &
\sum''\ar[r]\ar[d]_{h''} & 0\\
0 \ar[r] & \sum'_1 \ar[r]^{i_1} & \sum_1 \ar[r]^{j_1} & \sum''_1
\ar[r] & 0
 }
\]

\noindent
\textit{and $h$ commutes with $i$, $j$, $i_1$, $j_1$ then the
  following diagram, where $h^*$ is the homomorphism induced from $h$,
  is a commutative diagram.} 
{\fontsize{9}{11}\selectfont
\[
\xymatrix@C=0.5cm{
\ldots \ar[r] & H^q (X,\sum') \ar[r]^{i^{\ast}} \ar[d]_{h^{\ast}} &
H^q(x,\sum) \ar[r]^{j^{\ast}} \ar[d]_{h^{\ast}} & H^q(x,\sum'')
\ar[r]^{\delta^{\ast}} \ar[d]_{h^{\ast}} & H^q(x,\sum')
\ar[r]\ar[d]_{h^{\ast}} & \ldots\\
\ldots \ar[r] & H^q(X,\sum'_1) \ar[r]^{i^{\ast}_1} & H^q(X,\sum_1)
\ar[r]^{j^{\ast}_1} & H^q(X,\sum''_1) \ar[r]^{\delta^{\ast}_1}&
H^{q+1}(X,\sum'_1) \ar[r] & \ldots
}
\]}\relax


\begin{proof}
The result is a consequences of the fact that $h_{\mathscr{U}}$
commutes with $i_\mathscr{U}$, $j_\mathscr{U}$ and $\delta
_\mathscr{U}$. 
\end{proof}

