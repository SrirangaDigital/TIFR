\chapter{Lecture 13}

\begin{exam}
Let\pageoriginale $X$ consist of the natural numbers together with two special
points $p$ and $q$. Each natural number forms an open set. A
neighbourhood of $p$ (resp. $q$) consists of $p$ (resp; $q$) together
with all but a finite number of the natural numbers. Let $S_U = Z$ if
$U$ consists of all but a finite number of the natural numbers and if
$V \subset U$ is another such set, let $\rho_{_{VU}}:Z \to Z$ be the
identity. If $U$ is an open set not containing all but a finite number
of the natural numbers or if it contains either $p$ or $q$, let
$S_U=O$ and let $\rho_{_{VU}}$, $\rho_{_{UW}}$ be the zero homomorphisms. Then
$\sum =\{ S_U,\rho_{_{VU}} \}$ is a presheaf determining the 0-sheaf,
but $H^1(X,Z)=Z$. The space $X$ is $T_1$ and paracompact but not
normal. 
\end{exam}

\begin{exam}\label{chap13:exam18}%%% 18
Let $R$ be a set with cardinal number $\mathscr{N}_1$ let $S=2^R$ be
the set of all subsets of $R$ and let $T=2^S$ be the set of all
subsets of $S$. If $r \in R$, let $r' \in T$ be the largest subset
of $S$, which is such that, each of its elements considered as a
subset of $R$ contains the elements $r$. Let $R'\subset T$ consists of
all $r$, for all $r \in R$ and let $T_1=T-R'$. 
\end{exam}

Let $X$ be a space consisting of (1) all elements $r \in R$ and
(2) all triples $(t,r_1,r_2)$ with $t \in T_1$, $r_1$, $r_2 \in R$ and
$r_1 \neq r_2$. Each point $(t,r_1,r_2)$ is to form an open
set. Neighborhoods of points $r$ of the first kind are sets $N(r;s_1,
\ldots ,s_k)$, where $o \leq k< \infty$ and $s_1, \ldots ,s_k \in S$,
consisting of $r$ together with all points $(t,r_1,r_2)$ with $r \in
(r_1, r_2)$ and, for each\pageoriginale $i=1, \ldots, k$, either $r
\in s_i \in t$ 
or $r \notin s_i \notin t$. [cf. Bing's Example G, Canadian Jour,
  of Math. 3 (1951) p.184]. 

For  sets $U \subset X$ of cardinal number $\ge 2$ and consisting of
points of the second kind, let $S_U=Z$ and, if $V \subset U$ is
another such set, let $\rho_{_{VU}}: Z \to Z$  be the identity. If $U$ is
an open set containing any point of the first kind or consisting  of
at most one point, let $S_U=0$. Then $\sum = \bigg\{ S_U,
\rho_{_{VU}}\bigg\}$ is a presheaf determining the 0-sheaf, but $H'(X,
\sum) \neq 0$ (although dim $X=0$). X is a completely normal,
Hausdorff space but is not paracompact. (A space $X$ is said to be
completely  normal if each subspace of $X$ is normal). 

\textit{If $0 \to \mathscr{S'} \xrightarrow{i} \mathscr{S}
  \xrightarrow{j} \mathscr{S}'' \to 0$ is an exact sequence of
    sheaves, let $\bar{\mathscr{S}''}_o$, $\bar{Q}$, be the
    image and quotient presheaves in $0 \to \bar{\mathscr{S}}'
  \xrightarrow{i} \mathscr{S} \xrightarrow{j} \mathscr{S}''$
  for which the sequences} 
\begin{align*}
&0 \to \bar{\mathscr{S}} \xrightarrow{i} \bar{\mathscr{S}}
  \xrightarrow{j_o}        \bar{\mathscr{S}_o''} \to 0,\\ 
&0 \to \bar{\mathscr{S}}''_0 \xrightarrow{\bar{i}} \bar{\mathscr{S}}''
  \xrightarrow{\bar{j}} \bar{Q} \to 0,
\end{align*}
\textit{are exact. Then  $\bar{Q}$ determines the zero sheaf}.

\begin{proof}
An arbitrary element of a stalk $Q_k$ of the induced sheaf has the
form $\bar{\rho}_{_{xU}} \bar{j_U}f$ where $x \in  U$ and $f \in \Gamma
(U, \mathscr{S}'')$. Since $j$  maps $\mathscr{S}$ onto $\mathscr{S}$,
there is an open set $V, x \in  V \subset U$, for which $f| V \im
j_V$. Then  $\rho''_{_{VU}} f =f|V \in \im  j_V= \im \bar{i}_V$ and by
exactness $\bar{j}_V \rho''_{_{VU}}  f = 0$. Hence 
$$
\bar{\rho}_{xU} \bar{j}_U f= \bar{\rho}_{xV}\bar{\rho}_{VU} \bar{j}_U 
f= \bar{\rho}_{xV} \bar{j}_V \rho''_{VU} f=0. 
$$

Therefore\pageoriginale the sheaf determined by $\bar{Q}$ is the 0-sheaf.
\end{proof}

\begin{note*}
In example \ref{chap12:exam16}, if $\Gamma_o(U, \mathscr{S}'')= im j_U$, we have
$\Gamma_o(U, \mathscr{S}'')=\Gamma(U, \mathscr{S}'')$ for all $U$
expect $X$, but $\Gamma_o(X, \mathscr{S}'')=0$, $\Gamma_o(X,
\mathscr{S}'')= Z_2$. Thus  $\bar{Q}_X= Z_2$, $\bar{Q}_U=0$ for all
smaller $U$, and thus $\bar{Q}$ determine the 0-sheaf. 
\end{note*}

\begin{proposition}\label{chap13:prop9}%pro 9
If $X$ is paracompact and normal  and if $0 \to \mathscr{S'}
 \xrightarrow{i} \mathscr{S} \xrightarrow{j} \mathscr{S}'' \to 0$
 is an exact sequence of sheaves, there is an exact
cohomology sequence 
\begin{multline*}
0 \to  H^o(X, \mathscr{S'}) \to \cdots \to  H^q(X,\mathscr{S'})
\xrightarrow{i^*} \\
H^q(X, \mathscr{S}) \xrightarrow{j^*} H^q(X,
\mathscr{S''}) \xrightarrow{\delta^*} H^{q+1}(X,\mathscr{S'}) \to. 
\end{multline*}

If $h:(\mathscr{S'},\mathscr{S},\mathscr{S''}) \to
(\mathscr{S'}_1, \mathscr{S}_1,\mathscr{S''}_1)$ is a
  homomorphism of exact sequences, commuting with $i$ and $j$, 
\[
\xymatrix{
0\ar[r] & \mathscr{S}'\ar[r]^i\ar[d]_h & \mathscr{S} \ar[r]^j \ar[d]_h
& \mathscr{S}''\ar[r]\ar[d]_h & 0\\
0 \ar[r] & \mathscr{S}'_1 \ar[r]^i & \mathscr{S}_1 \ar[r]^j &
\mathscr{S}''_1 \ar[r] & 0
}
\]
the induced homomorphisms $S^*$ of the cohomology sequences commute
with $i^*$, $j^*$ and $\delta^*$, i.e. the following diagram is
commutative: 
{\fontsize{9}{11}\selectfont
\[
\xymatrix@C=0.45cm{
\ldots \ar[r] & H^q(x,\mathscr{S}')
\ar[r]^{i^{\ast}}\ar[d]_{h^{\ast}} & H^q 
(X,\mathscr{S})  \ar[r]^{j^{\ast}} \ar[d]_{h^{\ast}} &
H^q(x,\mathscr{S}'')\ar[r]^{\delta^{\ast}} \ar[d]_{h^{\ast}} &
H^{q+1}(X,\mathscr{S}') \ar[r]\ar[d]_{h^{\ast}} & \ldots\\
\ldots \ar[r] & H^q(X,\mathscr{S}'_1) \ar[r]^{i^{\ast}} & H^q (X,S_1)
\ar[r]^{j^{\ast}}& 
H^q(X,\mathscr{S}''_1) \ar[r]^{\delta^{\ast}} & H^{q+1}
(X,\mathscr{S}'_1) \ar[r] & \ldots
}
\]}\relax
\end{proposition}

\begin{proof}
As before, if $\bar{\mathscr{S}}''_o$, $\bar{Q}$ denote the image and
quotient presheaves in the exact sequence of presheaves of sections 
$$
0 \to \bar{\mathscr{S'}} \xrightarrow{i} \bar{\mathscr{S}}
\xrightarrow{j} \bar{\mathscr{S}}'', 
$$
we\pageoriginale obtain the exact sequence of presheaves
$$
0 \to \bar{\mathscr{S'}} \xrightarrow{i} \bar{\mathscr{S}}
\xrightarrow{j_0} \bar{\mathscr{S}}''_0 \to 0, 
$$
and 
$$
0 \to \bar{\mathscr{S}''} \xrightarrow{\bar{i}} \bar{\mathscr{S}''}
\xrightarrow{\bar{j}} \bar{Q} \to 0, 
$$
\end{proof}

From  these exact sequences of presheaves we obtain the following
exact cohomology sequences: 
{\fontsize{9}{11}\selectfont
\[
\xymatrix@C=0.45cm{
& & & \ar[d] & & \\
& & & H^{q-1} (X,\bar{Q}) = 0 \ar[d]_{\bar{\delta}^{\ast}} & & \\
\ldots \ar[r] & H^q(X,\mathscr{S}') \ar[r]^{i^{\ast}} &
H^q(X,\mathscr{S}) \ar[r]^{j^{\ast}_o} \ar[dr]_{j^{\ast}}&
H^{q}(X,\mathscr{\bar{S}}''_o) 
\ar[r]^{\delta^{\ast}_o} \ar[d]_{\bar{i}^{\ast}}&
H^{q+1}(X,\mathscr{S}') \ar[r] & \ldots\\
& & & H^q(X,\mathscr{S}'') \ar[ur]_{\delta^{\ast}}
\ar[d]_{\bar{j}^{\ast}} & & \\ 
& & & H^q(X,\bar{Q})=0 \ar[d]  & & \\
& & & & & 
 }
\]}\relax

Since $\bar{Q}$ determine the 0-sheaf, by Proposition 8, $H^q(X,
\bar{Q})=0$ for all $q \ge 0$ and hence, by exactness, $\bar{i}^*$ is
an isomorphism. Hence if $\delta^*= \delta^*_o(\bar{i}^*)^{-1}: H^q(X,
\mathscr{S}'') \to H^{q+1}(X, \mathscr{S}')$, the cohomology sequence 
\begin{gather*}
0 \to  H^o(X,\mathscr{S}')  \to  H^o (X,\mathscr{S}) \to  H^o(X,
\mathscr{S}'') \to \cdots\\ 
\cdots \to  H^q(X,\mathscr{S}') \to H^q(X,\mathscr{S}) \to
H^q(X,\mathscr{S}'') \to H^{q+1}(X,\mathscr{S}') \to \cdots  
\end{gather*}
is exact.

Next,\pageoriginale since the homomorphism $h$ commutes with $i$ and
$j$, the induced homomorphism $h$ of presheaves also commutes with
$i$ and $j$:  
\[
\xymatrix{
0\ar[r] & \bar{\mathscr{S}}'\ar[r]^i\ar[d]_h &
\bar{\mathscr{S}}\ar[r]^j\ar[d]_h & \bar{S}''\ar[d]_h\\
0 \ar[r] & \bar{\mathscr{S}}'_1 \ar[r]^i & \bar{\mathscr{S}}_1 \ar[r]
& \bar{\mathscr{S}}''_1 \quad .
}
\]

Hence, the induced homomorphism $h^*$ of the cohomology modules
commutes with $i^*$, $j^*_o$, $\bar{i}^*$ and $\delta^*_o$. Thus in the
exact cohomology sequences, $h^*$ commutes with $i^*$, $j^*$ and
$\delta^*$. 

\begin{note*}
If $X$  is not paracompact and normal, in general,
$\bar{i}^*$ is not an isomorphism, to be precise, the cohomology
sequence is not defined. One does, however, have the exact sequence  
\begin{multline*}
0 \to H^o(X, \mathscr{S}') \to H^o(X,\mathscr{S})\to
H^o(X,\mathscr{S}'')\to \\
H^1(X,\mathscr{S}')\to H^1(X,\mathscr{S})\to
H^1(X,\mathscr{S}'') 
\end{multline*}
as one sees from the exact sequences
\newpage

\begin{landscape}
\[
\xymatrix@C=0.5cm{
  & & & 0 \ar[d] & & & 0 = H^o(X,\bar{Q})\ar[d]\\
0 \ar[r] & H^o(X,\mathscr{S}') \ar[r]^{i^{\ast}} & H^o(X,\mathscr{S})
\ar[r]^{j^{\ast}_o}\ar[dr]_{j^{\ast}} & H^o(X,\mathscr{S}''_o)
\ar[r]^{\delta^{\ast}_o} \ar[d]_{\bar{i}^{\ast}} & H^1(X,\mathscr{S}')
\ar[r]^{i^{\ast}} & H^1(X,\mathscr{S}) \ar[r]^{j^{\ast}_o}
\ar[dr]_{j^{\ast}} & H^1(X,\mathscr{S}''_o)\ar[d]_{\bar{i}^{\ast}}\\
& & & H^o(X,\mathscr{S}'') \ar[ur]\ar[d] & & & H^1(X,\mathscr{S}'') \\
& & & H^o(X,\bar{Q}) = 0.
}
\]
\end{landscape}
 
\newpage

The following examples show that Proposition \ref{chap13:prop9} is not
true,\pageoriginale in general,unless the space is both paracompact
and normal.  
\end{note*}

\begin{exam}%exe 19
Let $X$ consist of the unit segment $I$ with the usual topology and of
two points $p$ and $q$. A neighbourhood of $p$ (resp. $q$) consists of
$p$ (resp. $q$) together with the whole of $I$. Let
$\mathscr{S}'$, $\mathscr{S}$, $\mathscr{S}''$ be the sheaves
$\mathscr{S}'$, $\mathscr{S}$, $\mathscr{S}''$ of Example
\ref{chap12:exam16} over $I$ 
together with zeros at $p$ and $q$. A neighbourhood of $0_p$
(resp. $0_q$) consists of the zeros over a neighbourhood of $p$
(resp. $q$). Then there is an exact  sequence  
$$
0 \to  \mathscr{S}' \xrightarrow{i} \mathscr{S} \xrightarrow{j}
\mathscr{S}'' \to 0 
$$ 
where $i$, $j$ correspond to those in Example
\ref{chap12:exam16}. Since $H^1(X, 
\mathscr{S})= H^2(X, \mathscr{S}')=0$ and $H^1(X, \mathscr{S}'')=Z_2$,
there is no exact cohomology sequence. The space $X$ is paracompact
but not normal.  
\end{exam}

\begin{exam}%exe 20
Let $X$ consists of a sequence of copies $I_n$ of the unit segment
together with two special points $p$ and $q$. A neighbourhood of
$p$ (resp $q$) consists of $p$ (resp. $q$) together with all but a
finite number of the segments $I_n$. Let $G$ be the 4-group and let
$\mathscr{S}$ be the subsheaf of $(X \times G, \pi, X)$ consisting  of
zero at $p$ and $q$ and on each $I_n$ a copy of the sheaf
$\mathscr{S}$ of Example \ref{chap12:exam16}. Let $\mathscr{S}''$ be
the subsheaf of 
$(X \times Z_2, \pi, X)$ formed by omitting the points $(p,1)$, $(q,1)$ and
let the homomorphism $j :  \mathscr{S} \to \mathscr{S}''$ be induces
by $j: G \to Z_2$ as defined in Example \ref{chap12:exam16}. Then
there is exact sequence 
$$
0 \to  \mathscr{S}' \xrightarrow{i} \mathscr{S} \xrightarrow{j} 
\mathscr{S}'' \to 0 
$$
but\pageoriginale $H^1(X, \mathscr{S}) = H^2(X,  \mathscr{S}')=0$ while $H^1(X,
\mathscr{S}'') \neq 0$. Thus there is no exact cohomology sequence. The 
space $X$ is paracompact and $T_1$ but not normal. 
\end{exam} 

\begin{exam}%exe 21
Let $R$, $S$, $T_1$ be as in Example \ref{chap13:exam18}. Let $X$ be
the space consisting 
of (1) the elements $r \in R$ and (2) segments $I_{n tr_1 r_2}$
where $n$ is a natural number, $t \in T_1$, $r_1$ and $r_2$ are in $R$,
and $r_1 \neq r_2$. Neighbourhoods of the points $r$ are
sets $N(r;n,s_1,., s_k)$ where $n$ is a natural number and $s_1, \ldots
, s_k \in S$, consisting of $r$ together  with all segments $I_{mtr_1
  r_2}$ with $m > n$, $r \in  (r_1, r_2)$ and, for each $i=1, \ldots
, k$, either $r \in  s_i \in t$ or $r \notin s_i \notin t$. 
\end{exam}

Let $G$ be the 4-group and let $\mathscr{S}$ be the subsheaf of $(X
\times G, \pi, X)$  consisting of zero at each $r$ and a copy  of the
sheaf $\mathscr{S}$ of Example \ref{chap12:exam16} on each $I_{ntr_1 r_2}$. Let
$\mathscr{S}''$ be the subsheaf of $(X \times Z_2,  \pi , X)$ formed
by omitting $(r,1)$ for all $r$, and let the homomorphism $j :
\mathscr{S} \to \mathscr{S}''$ be induced by $j :  G \to  Z_2$ as
mentioned before. Then there is an exact sequence 
$$
0 \to  \mathscr{S}' \to \mathscr{S} \to \mathscr{S}'' \to 0 
$$
but $H^1(X, \mathscr{S})= H^2(X, \mathscr{S}')=0$ while $H^1(X,
\mathscr{S}'') \neq 0$. Thus again, there is no exact cohomology
sequence. $X$  is a  perfectly normal Hausdorff space but is not
paracompact. ($A$ space $X$  is said to be perfectly normal if, for
each closed set $C$ of $X$ there is a continuous real valued function
defined on $X$ and vanishing on $C$ but not at point $x \in
X-C$. Perfectly normal spaces are completely normal.) 

