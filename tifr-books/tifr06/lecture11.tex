\chapter{Lecture 11}\label{chap11:lec11}

If\pageoriginale the coefficient presheaf is the presheaf of sections
of some sheaf 
$\mathscr{S}$ of $A$-modules, we write $C^q(\mathscr{U},\mathscr{S})$
instead of $C^q(\mathscr{U}, \bar{\mathscr{S}})$ etc. Then, if
$U_\sigma =U _{i_o} \cap \cdots \cap U_{i_q}$ is called the \textit{
  support} of the simplex $\sigma= i_o, \ldots ,i_q$, a $q$-cochain $f
\in C^q (\mathscr{U},\mathscr{S})$ is an alternating function which
assigns to each $q$-simplex $\sigma$ a section over the support of
$\sigma$. 

\textit{If $\mathscr{U}=\{ U_i \} i\in I$  is any covering
  of $X$, $H^o(\mathscr{U},\mathscr{S})$ is isomorphism to
$\Gamma (X,\mathscr{S})$.}
 
\begin{proof}
A $0$-cochain belonging to $C^0(\mathscr{U},\mathscr{S})$ is a system
$(f_i)_{i \in I}$, each $f_i$ being a section of $\mathscr{S}$ over
$U_i$. In order that this cochain be a cocycle, it is necessary and
sufficient that $f_i-f_j=O$ over $U_i \cap U_{j}$; in other words,
that there exist a section $f \in \Gamma (X, \mathscr{S})$ which
coincides with $f_i$ on $U_i$ for each $i \in I$. Thus there is an
isomorphism $\phi _\mathscr{U}:\Gamma (X, \mathscr{S})\to
Z^0(\mathscr{U},\mathscr{S}) \to H^0 (\mathscr{U},\mathscr{S})$. 
\end{proof}

\begin{proposition}\label{chap11:prop7} %%%% proposition 7
$H^0 (X,\mathscr{S})$ can be identified with $\Gamma (X,
    \mathscr{S})$. 
\end{proposition}

\begin{proof}
Since $\tau_{\mathscr{W}\mathscr{U}} \phi _\mathscr{U} =\phi
_\mathscr{W}$, there is an induced isomorphism $\phi: \Gamma (X,
\mathscr{S}) \to H^0 (X, \mathscr{S})$ with $\tau_{\mathscr{U}}
\phi_{\mathscr{U}} = \phi$. A homomorphism  
$h: \mathscr{S} \to \mathscr{S}_1$ of sheaves induces a homomorphism
$\{ h_U \}$ of the presheaves of section and hence induced homomorphism
$h_{\mathscr{U}'}$, $h^*$ with commutativity in  
\end{proof}
\[
\xymatrix{
\Gamma(X,\mathscr{S}) \ar[r]^{\phi_{\mathscr{U}}} \ar[d]_{h_X} &
H^0(\mathscr{U},\mathscr{S}) \ar[r]^{\tau_{\mathscr{U}}}
\ar[d]_{h_{\mathscr{U}}} & H^0(X,\mathscr{S})\ar[d]^{h^{\ast}}\\
\Gamma(X,\mathscr{S}_1) \ar[r]^{\phi_{\mathscr{U}}} &
H^0(\mathscr{U},\mathscr{S}_1) \ar[r]^{\tau_{\mathscr{U}}}  &
H^0(X,\mathscr{S}_1) 
}
\]

Thus\pageoriginale we can identify $\Gamma (X, \mathscr{S})$ with
$H^o(X,\mathscr{S})$ under $\phi$ if we also identify $h_x:\Gamma
(X,\mathscr{S}) \to \Gamma (X,\mathscr{S}_1)$ with $h^* : H^o (X,
\mathscr{S}) \to H^o (X, \mathscr{S}_1)$. 

\begin{defi*}%%%% 7 
A system $\big\{ A_i \big\}_{i \in I}$ of subset of a space $X$ is
called \textit{finite}
if $I$ is finite, \textit{countable} if $I$ is countable. The system
is said to be \textit{locally finite} if each point $x \in X$ has a
neighbourhood $V$ such that $V \cap A_i= \phi$ expect for a finite
number of $i$. (This finite number may also be zero). 
\end{defi*}

We notice that a locally finite system $\big\{ A_i \big\}_{i \in I}$ is always
point finite. (A system $\big\{ A_i \big\}_{i \in I}$ of subsets of $X$ is said
to be point finite if each point $x \in X$ belongs to $A_i$ for only a
finite number of $i$). 

If $\big\{ A_i \big\}_{i \in I}$ is locally finite, so is $\big\{ B_j
\big\}_{j \in J}$ if 
$J \subset I$ and each $B_j \subset A_j$. If $\big\{ A_i \big\}_{i \in I}$ is
locally finite, so is $\big\{ \bar{A}_i \big\}_{i \in I}$, where $\bar{A}_i$
denotes the closure of $A_i$, and  $ \underset{i \in I}{\overline
  {\bigcup A_i}}= \bigcup\limits_{i \in I} \bar{A}_i$. In particular, if
each $A_i$ is closed, so is $\bigcup\limits_{i \in I} A_i$ 

\begin{defi*}
The  {\em order} of a system $\{ A_i \}_{i \in I}$ of subsets of $X$
is $-1$ if $A_i$ is the empty set for each $i \in I$. Otherwise the
order is the largest integer $n$ such that for $n+1$ values of $i \in
I$, the $A_i's$ have a non-empty intersection, and it is infinity if
there exists no such largest integer. 
\end{defi*}

\begin{defi*}
The \em{dimension} of $X$, denoted as $\dim X$, is the least integer
$n$ such that every finite covering of $X$ has a
refinement\pageoriginale of order $\le n$, and the dimension is
infinity if there is no such integer. 
\end{defi*}

\begin{defi*}
A space $X$ is called \em{normal}, if for each pair $E$, $F$ of closed
sets of $X$ with $E \cap F= \phi$, there are open sets $G$, $H$ with $E
\subset G$, $F \subset H$ and $G \cap H = \phi$. 
\end{defi*}

\begin{defi*}
A covering $\mathscr{U}=\{ U_i \} _ {i \in I}$ of the space $X$ is
called \em{shrinkable} if there is a refinement $\mathscr{W}=\{ V_i
\} _{i \in I}$ of $\mathscr{U}$ with $\bar{V}_i \subset U_i$ for each
$i \in I$. 
\end{defi*}

$X$ \textit{is normal if and only if every locally finite covering
  of $X$ is shrinkable.} 

\begin{proof}
See S. Lefschetz, Algebraic Topology, p.26.
\end{proof}

\textit{If $X$ is normal, $\dim X \le n$ if and only if every locally
  finite covering of $X$ has a refinement of order $\le n$.} 

\begin{proof}
See C.H.Dowker, Amer.Jour of Math. (1947), p.211.
\end{proof}

\begin{defi*}
A space $X$ is called \em{paracompact} if every covering of $X$ has a
refinement which is locally finite. 
\end{defi*}

\textit{If $X$ is paracompact and normal, $\dim X \leq n$ if and only
  if every covering of $X$ has a refinement of order $\leq n$.} 

\begin{proof}
\begin{enumerate}[(i)]
\item Since $X$ is paracompact, every covering of $X$ has a locally
  finite refinement and since $X$ is normal and $\dim X \leq n$, using
  the above result, every locally finite covering has a refinement of
  order\pageoriginale $\leq n$, thus every covering has a refinement of
  order $\leq  n$. 

\item Since every covering of $X$ has a refinement of order $\leq n$,
  in particular, every locally finite of $X$ has a refinement of order
  $\leq n$, hence, since $X$ is normal, using the above result, we
  obtained $\dim \leq n$. 
\end{enumerate}
\end{proof}

\begin{remark*}
Since a paracompact Hausdorff space is normal, (see
J. Dieu\-donne, Jour. de Math. 23, (1944), p.(66), this result holds,
in particular, when $X$ is a paracompact Hausdorff space. 
\end{remark*}

\textit{If a covering $\mathscr{U}$ of  $X$ has a
  refinement $\mathscr{W}$ of order $\leq n$, 
  then it has a proper refinement $\mathscr{W}$ of order
$\leq n$.} 

\begin{proof}
If $\mathscr{W}= \{ V_j \}_{j \in J}$ has order $\leq n$, let
$\mathscr{W}$ be the proper covering formed by all open sets $W$ such
that, for some $j \in j$, $W=V_j$. Then $\mathscr{W}$ has order $\leq n$
and is a refinement of $\mathscr{U}$. 
\end{proof}

\textit{If $X$ is paracompact and normal and $\dim X \leq n$, then
  $H^q (X, \sum)=0$ for $q > n$ and an arbitrary presheaf $\sum$.} 

\begin{proof}
Replace the directed set $\Omega$ of all proper coverings of $X$ by
the cofinal sub-directed set $\Omega'$ of all proper covering of order
$\leq n$. If $\mathscr{U} \in \Omega'$, $q > n$ and $f \in C^q
(\mathscr{U}, \sum)$, then $f(U_0, U_1, \ldots ,U_q)\in S_{U_o \cap
\cdots \cap_{U_q}} = S_\phi = 0$ for any $q+1$ distinct open sets of
$\mathscr{U}$. If the open sets $U_0,U_1, \ldots ,U_q$ are not all
distinct, then $f(U_0,U_1, \ldots ,U_q)=0$ since $f$ is
alternating. Hence $C^q(\mathscr{U}, \sum)=0$ and hence also $H^q
(\mathscr{U}, \sum)=0$. Therefore $H^q(X, \sum)=0,q >n$. 
\end{proof}

\textit{If\pageoriginale $\sum$ is a presheaf which determines the
  zero sheaf, then $H^o(X, \sum)=0$.}  

\begin{proof}
For any element $\eta \in H^0 (X, \sum)$ choose a representative
\break $f
\in z^0 (\mathscr{U},\sum) \mathcal{S} H^0 (\mathscr{U},\sum)$, where
$\mathscr{U}$ is some proper covering of $X$, so that $\tau
_\mathscr{U} f =\eta$. For each $x \in X$ choose an open set $U=\tau
(x)$ such that $x \in \tau (x)\in \mathscr{U}$. Since $\sum$
determines the 0-sheaf, one can choose an open set $V_x$ such that
$x \in V_x \subset \tau (x)$ and $\rho (V_x, \tau (x))f (\tau
(x))=0$. Then $\mathscr{W} = \big\{ V_x \big\}_{x \in X}$ is a refinement of
$\mathscr{U}$, and, for each $x$, $(\tau ^+ f)(x)=\rho (V_x, \tau
(x))f(\tau (x))=0$, hence $\tau^+ f=0$. If $\mathscr{W}$ is a proper
refinement of $\mathscr{W}$, choose $\tau_1: \mathscr{W} \to X$ so
that each $W \subset V_{\tau _1(W)}$ and $(\tau \tau_1)^+f=\tau _1^+ \tau^+
f=0$. Thus $\tau _{\mathscr{W}\mathscr{U}}f=0$ and hence $\eta =\tau
_\mathscr{U} f=\tau_\mathscr{W}
\tau_{\mathscr{W}\mathscr{U}}f=0$. Hence $H^o (X, \sum)=0$.  
\end{proof}

This result is not true in general for the higher dimensional
cohomology groups. However, if the space $X$ is assumed to be \textit{
  paracompact and normal}, we will prove the result to be true for the
higher dimensional cohomology groups. 

