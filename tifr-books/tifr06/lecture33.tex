\chapter{Lecture 33}

\begin{defi*}
A sheaf\pageoriginale $\mathfrak{a}$ of rings with unit is called a
{\em coherent 
  sheaf of rings} if it is coherent as a sheaf of
$\mathscr{a}$-modules, i.e., it has property $(b)$ (Property (a) is
trivially satisfied.)  
\end{defi*}

\textit{If $\mathfrak{a}$ is coherent and $\mathscr{S}$ is a sheaf of
  $\mathfrak{a}$- modules, then $\mathscr{S}$ is coherent if and
  only if for each point $x$ there is an open set $U$ with $x \in U$,
  and an exact sequence} 
$$
\sum^{l}_{i=1} \mathfrak{a}_U \xrightarrow{\psi} \sum^{k}_{i=1} 
\mathfrak{a}_U \xrightarrow{\phi} \mathscr{S}_U \rightarrow 0.  
$$

\begin{proof}
{\em Necessity}. If $\mathscr{S}$ is coherent, property $(a)$ implies 
the existence of $\phi$ and property $(b)$ implies the existence of
$\psi$. 
\end{proof}

\textit{Sufficiency}. Since $\mathfrak{a}$ is coherent,
$\mathfrak{a}_U$ is coherent in $U$ for each open set $U$ and so are
$\sum^{l}_{i=1} \mathfrak{a}_U$ and 
$\sum^{k}_{j=1} \mathfrak{a}_U$. As the image of $\sum^{l}_{i=1}
\mathfrak{a}_U$, 
im $\psi$ has property $(a)$, and as a subsheaf of $\sum^{k}_{j=1}
\mathfrak{a}_U$, $\im \psi$ has property $(b)$. Thus im $\psi$ is
coherent, and there is an exact sequence 
$$
0 \rightarrow \im \psi \rightarrow \sum^{k}_{i=1} \mathfrak{a}_U \rightarrow
\mathscr{S}_U \rightarrow 0. 
$$

Hence, since two of the sheaves are coherent, the third,
$\mathscr{S})U$, is coherent for a neighbourhood of each point $x$ and
hence $\mathscr{S}$ is coherent. 

\begin{exam}%exe 36
In the ring $B = Z [y, x_1 , x_2 , \ldots]$ of polynomials in
infinitely many variables with integer coefficients, let $I$
be\pageoriginale the 
ideal generated by $y x_1 , y x_2, \ldots$ and let $A = B/I$. Then
multiplication by $y$ gives a homomorphism $f : A \rightarrow A$ whose
kernel $C$ consists of polynomials in $Z [x_1 , x_2, \ldots]$ without
constant terms. Then the $A$- module $C$ is not finitely generated. Hence
if $X$ is a consisting of one point, the constant sheaf $A$ is not a
coherent sheaf of rings. 
\end{exam}

\begin{exam}%ex e37
Let $X = \{ x : 0 \le x \le 1\}$ and let $F$ be the ring of functions
$f : X \rightarrow Z_4 $ for which $f(x) = f(1)$ for $x > 0$ and $f(0)
= f(1)$ mod 2. There are eight functions in $F$. Let $a$ be the
sheaf of germs of function in $F$. Let $g$ be the (constant) global
section defined by the function $g : X \rightarrow Z_4$ where $g(0) =
2$ and $g(x) = 0$ for $x > 0$. The sheaf $\mathscr{R}$ of relations
for this section is obtained by omitting from $\mathfrak{a}$ the germs at 0 of
functions $f$ with $f(0) = 1$ or 3. Then the sections of $\mathscr{R}$
over any connected neighbourhood $U$ of 0 contain only the germs of
even valued functions, hence do not generate $\mathscr{R}_U$. Thus
$\mathscr{R}$ does not have $(a_1)$ and hence $a$ has neither $(b_1)$
nor $(b)$. 
\end{exam}

\textit{The sheaf of germs of analytic functions in the complex plane
  is a coherent sheaf of rings}. 

\begin{proof}
Let $\mathfrak{a}$ be the sheaf of germs of analytic functions in the complex
place, and let $f_1, \ldots , f_k$ be sections of $\mathfrak{a}$ over a
neighbourhood $U$ of a point $z_o$, i.e., $f_i$ is an analytic function
in $U$. We can write $f_i(z) = (z - z_o)^n g_i (z)$ where $g_i$ does not
vanish at $z_o$ and\pageoriginale hence does not vanish in a
neighbourhood $V_i$ of 
$z_o$, $z_o \in V_i \subset U$. Let $V = \bigcap^K_{i = 1} V_i$. Let
$\mathscr{R}$ be the sheaf of relations between the sections $f_i | V$,
$i = 1, \ldots , k$. We will show that $\mathscr{R}$ is finitely
generated in $V$. 
 \end{proof}
 
 Let $P = C[z]$ be the ring of polynomials in $z$ with complex
 coefficient and let $M$ be the submodule (over $P$) of the direct sum
 $\sum^k_{i=1} P$ consisting of elements $(p_1 , \ldots , p_k)$ for
 which $\sum^k_{i=1} p_i (z) (z - z_o)^{n_i} = 0$. Since $P$ is a
 euclidean ring and $\sum^k_{i=1} P$ is finitely generated over $P$,
 the submodule $M$ is finitely generated over $p$. (See van der
 Waerden , Modern Algebra, Vol, I, p. 106). Let $(p^j_1, \ldots ,
 p^j_k)$, $j = 1, \ldots , l$, be a system of generators for $M$. Let
 $r^j_i(z) = p^j_i(z) / g_i(z)$; then each $r^j_i(z)$, $i = 1, \ldots,
 k$, is an analytic function in $V$, and  
 $$
 \sum^k_{i=1} r^j_i(z) f_i (z) = \sum^k_{i=1} p^j_i(z)  (z -
 z_o)^{n_i} = 0 \quad (j = 1, \ldots , l). 
 $$
 
 Thus for each $j$, the section determined by $(r^j_i , \ldots ,
 r^j_k)$ is in $\mathscr{R}$. We will show that these sections
 generate $\mathscr{R}$. 
 
 Let $\sum^k_{i=1} t_{i z_1}$, $f_{i z_1} = 0$ be a relation between
 germs at $z_1 \in V$, i.e., let $t_i (z)$, $i = 1, \ldots, k$, be
 analytic functions at $z_1$ such that $\sum^k_{i=1} t_i(z) f_i (z) =
 0$. Then $\sum^k_{i=1} t_i(z) g_i(z) (z-z_o)^{n_i} \equiv  0$. Let $n =
 \max(n_1, \ldots , n_k)$ and suppose that $n_k = n$. We can write 
  $$
 t_i(z) g_i(z) = (z - z_o)^n q_i (z) + s_i(z), \quad (i =1, \ldots , k-1) 
 $$
 where\pageoriginale $q_i(z)$ is analytic at $z_1$ and $s_i(z) \equiv
 0$ of $z_1 \neq z_o$ and is a polynomial of degree less than $n$ if
 $z_1 =  z_o$. Then 
\begin{align*}
& (t_1(z)g_1(z), \ldots, t_k(z) g_k (z))=\\
& \qquad q_1(z)((z-z_o)^n, 0, \ldots,0,- (z-z_o)^{n_1}+ \cdots \\
& + \quad   q_{k-1}(z)(0, \ldots, 
 (z-z_o)^n,- (z-z_o)^{n_{k-1}}+ (s_1 (z), \ldots, s_k(z))
\end{align*}
where $s_k(z)$ is the analytic function defined by  
$$
s_k(z) =t_k(z) g_k(z) + \sum^{k-1}_{i=1} q_i (z)(z-z_0)^{n_i}. 
$$

Now, $((z-z_0)^n, 0, \ldots, 0, -(z-z_0)^{n_1}$, etc. are in $M$ and by
direct verification we have $\sum^{k-1}_{i=1} s_i (z) (z-z_0)^{n_i}
\equiv 0$. Since $s_1(z), \ldots , s_{k-1}(z)$, are polynomials, it
follows that $s_k(s) (z-z_o)^n$, is a polynomial. 
\begin{enumerate}[(i)]
\item If $z_1 \neq z_0$, then $s_1(z), \ldots, s_{k-1}(z)$ are all
  zero and  
$$
s_k(z)= t_k(z) g_k(z) + \sum^{k-1}_{i=1}q_i(z) (z-z_0)^{n_i},  
$$
\begin{align*}
(z-z_0)^n s_k(z) & = t_k(z) g_k(z) (z-z_0)^n +
  \sum^{k-1}_{i=1}q_i(z) (z-z_0)^{n} (z-z_0)^{n_i}\\ 
&= t_k(z) g_k(z) (z-z_0)^n + \sum^{k-1}_{i=1}t_i(z) g_i(z)
  (z-z_0)^{n_i}\\ 
& \equiv 0,\\
\text{hence }  s_k(z) & \equiv 0,
\end{align*}
\item If $z_1 =z_0$, then $s_k(z)$ has a series expansion
  $\sum^{\infty}_{r=0} a_r(z-z_0)^r$ and on multiplication by
  $(z-z_o)^n$ this series has a finite number of terms. Hence $s_k(z)$
  is already a polynomial. 
\end{enumerate}

In\pageoriginale either case, $(t_1(z)g_1(z), \ldots, t_k(z) g_k (z))$ is a linear
combination of elements of $M$ with coefficients analytic at
$z_1$. Hence  
$$
(t_1(z)g_1(z), \ldots, t_k(z) g_k (z))= \sum^l_{j=1}h_j(z) 
(p^j_1(z), \ldots, p^j_k(z)), 
$$
where $h_j(z)$ is analytic at $z_1$. Then 
$$
(t_1(z), \ldots , t_k(z))= \sum^l_{j=1}h_j(z) (r^j_1(z), \ldots, 
r^j_k(z)). 
$$

Thus the sheaf $\mathscr{R}$ of relations is generated by a finite
number of sections, hence the sheaf $\mathfrak{a}$ is coherent. 

This result is a special case of Oka's theorem, Cartan Seminar,
1951-52, Expose 15, \S 5. The following proposition on the
extension of coherent sheaves is based on Expose 19, \S 1, of the
same seminar. 

\begin{proposition}%proposi 22.
 Let $Y \subset X$ and either, let $X$ be paracompact and normal
  with $Y$ closed, or, let$X$ be hereditarily paracompact and normal.
  Let $\mathfrak{a}$ be a coherent sheaf of rings with unit over $X$
  and let $\mathscr{S}$ be a coherent sheaf of $\mathfrak{a}_Y$-
  modules over $Y$. Then there is an open set $U$ with $Y \subset U$,
  and a coherent sheaf $\mathcal{J}$ of $\mathfrak{a}_U$- modules over
  $U$ whose restriction to $Y$ is isomorphic to  $\mathscr{S}$ .   
\end{proposition}

\begin{proof}
Since $\mathscr{S}$ is coherent, there is a covering $\{ V_j \cap Y
\}_{ j \in J}$ of $Y$, where $V_j$ is open in $X$, and for each $j \in
J$ an exact sequence  
$$ 
\sum^{l_j}_{i=1} \mathfrak{a}_{V_{j \cap Y}} \xrightarrow{\psi_j}
\sum^{k_j}_{i=1} \mathfrak{a}_{V_j \cap Y} \xrightarrow{\phi_j} \mathscr{S}_{V_j
  \cap Y} \to 0. 
$$
\end{proof}\pageoriginale

From the properties of $X$ there exist system $s\{ G_i\}_{i \in I}$ and
$\{ H_i \}_{i \in I}$ of open sets of $X$ such that, if $G=\cup_{i}G_i$,
 \begin{enumerate}[(i)]
\item $\bar{H}_i \cap G \subset G_i$,

\item the system $\{ G_i\}_{i \in I}$ is locally finite in $G$,  

\item $Y \subset \cup_i H_i$,

\item each $G_i$ is contained in some $V_j$, 
\end{enumerate}

For the first case, we can assume that each $G_i$ is an $F_{\sigma}$
set in $X$, hence $G$ and all intersections $\cap^{k}_{r=1} G_{i_r}$
are $F_{\sigma}$-sets and hence are paracompact and normal. In the
second case, all subsets of $X$ are paracompact and normal. 

Since each $G_i$ is contained in some $V_j$, there are exact sequences  
$$
\sum^{l_j}_{i=1} \mathfrak{a}_{G_i \cap Y} \xrightarrow{\psi_i}
\sum^{k_i}_{r=1} \mathfrak{a}_{G_i \cap Y} \xrightarrow{\phi_j}
\mathscr{S}_{G_j   \cap Y} \to 0. 
$$

Either $G_i$ is paracompact and normal with $G_i \cap Y$ closed in
$G_i$ or $G_i$ is hereditarily paracompact and normal, and the sheaves
$\sum^{l_i}_{r=1} \mathfrak{a}_{G_i}$ and  $\sum^{k_i}_{r=1}
\mathfrak{a}_{G_i}$ are coherent. Hence (see Lecture
\ref{chap32:lec32}) there is an 
open set $G'_i$ with $G_i \cap Y \subset G'_i \subset G_i$ and an
extension   
$$
\psi' : \sum^{l_i}_{r=1} \mathfrak{a}_{G'_i} \to \sum^{k_i}_{r=1}
\mathfrak{a}_{G'_i}   
$$
of $\psi_i$. For the first case we may assume that $G'_i$ is also an 
$F_{\sigma}$- set. Let $\mathscr{S}^i = (\sum^{k_i}_{r=1}
\mathfrak{a}_{G'_i})/ \im \psi'$. Then, if $\phi'$ is the natural
homomorphism, the sequence    
$$
\sum^{l_i}_{r=1} \mathfrak{a}_{G'_i} \xrightarrow{\psi'} \sum^{k_i}_{r=1}
\mathfrak{a}_{G'_i} \xrightarrow{\phi'} \mathscr{S}^i \to 0 
$$\pageoriginale
is exact, i.e., 
$$
0 \to \im \psi' \to \sum^{k_i}_{r=1} \mathfrak{a}_{G'_i} \to 
\mathscr{S}^i \to 0 
$$
is exact and hence $\mathscr{S}^i$is coherent. There is an isomorphism  
$$
g_i : \mathscr{S}^i_{G'_i \cap Y} \to \mathscr{S}_{G'_i \cap Y}. 
$$

There are open sets $H'_i$ with $\bar{H}_i \cap Y \subset H'_i$,
$\bar{H}'_i \cap G' \subset G'_i$ where $G' = \bigcup \limits_{i \in
  I}G'_i$. For the first case, $G'$ and all intersections
$\cap^q_{r=1}G'_i$ are $F_{\sigma}$ sets, hence are paracompact and
normal. 

For $i$, $j \in I$, there is an open set $G_{ij}$, $G'_i \cap G'_j \cap Y
\subset G_{ij} \subset G'_i \cap G'_j$, and a homomorphism $f_{ij}:
\mathscr{S}^j_{G_{ij}} \to \mathscr{S}^i_{G_{ij}}$ such that $f_{ij} \Big|
\mathscr{S}^j_{G_{ij} \cap Y} = g^{-1}_i g_j \Big| \mathscr{S}^j_{G_{ij}
  \cap Y}$. For the first case, we may assume that $G_{ij}$ is an
$F_{\sigma}$ set. Then there is an open set $G'_{ij}$ with $G'_i \cap
G'_j \cap Y \subset G'_{ij} \subset G_{ij} \cap G_{ji}$ such that $f_{ij}
f_{ji} \Big|\mathscr{S}^j_{G'_{ij} \cap Y}$ is the identity and $f_{ji}
f_{ij} \Big|\mathscr{S}^j_{G_{ij} \cap Y}$. is the identity. Let $E_{ij} =
\bar{H'}_i \cap \bar{H'}_j \cap (G' - G'_{ij})$, then $E_{ij}$ is
closed in $G'$. 

For $i$, $j$, $k \in I$ there is an open set $G_{ijk}$, $G'_i \cap
G'_j \cap G'_k \cap Y \subset G_{ijk} \subset G'_i \cap G'_j \cap
 G'_k$, such that $f_{ij} f_{jk}| G_{ijk} = f_{ik}| \big|
 G_{ijk}$. Let $E_{ijk} = \bar{H}'_i \cap \bar{H}'_i \cap \bar{H}'_k
 \cap (G' - G_{ijk})$, then $E_{ijk}$ is closed in $G'$.  

Let $E = (\bigcup_{i,j} E_{ij}) \bigcup (\bigcup_{i,j,k} E_{i j k})$
and\pageoriginale let $U = (G' - E) \cap(\bigcup_i H'_i)$. Then $E$ is
closed in 
$G'$ and $U$ is open with $Y \subset U$. Over each $H'_i \cap U$ there
is a sheaf $\mathscr{S}^{i}_{H'_i \cap U}$; over each $H'_i \cap H'_j
\cap U$ there is an isomorphism  
$$
f'_{ij} = f_{ij} \Bigg| \mathscr{S}^{i}_{H'_i \cap H'_j \cap U}:
\mathscr{S}^{j}_{H'_i \cap H'_j \cap U} \to \mathscr{S}^{i}_{H'_i \cap
  H'_j \cap U}  
$$
and $f'_{ij} = (f'_{ij})^{-1}$. Further, over each $H'_i \cap H'_j \cap
H'_k \cap U$ these isomorphisms are consistent, i.e., $f'_{ij} f'_{jk}
= f'_{ik}$. Then by identification there is determined a sheaf
$\mathcal{J}$ over $U$ such that $\mathcal{J}_{H'_i \cap U}$ is
identified with $\mathscr{S}^i_{H'_j \cap U}$. Then the isomorphisms
$g_i: \mathscr{S}^i_{H'_i \cap Y} \to \mathscr{S}_{H'_i \cap Y}$
induce an isomorphism $ g: \mathcal{J}_Y \to \mathscr{S}$. Since  each
$\mathscr{S}^i_{H'_i \cap U}$ is coherent. $\mathcal{J}$ is
coherent.
 

\begin{thebibliography}{99}\pageoriginale
\bibitem{key1} {H. Cartan}: Seminaire de L' Ecole Normale Superiure,
  1948-49, 1950-51, 1951-52, 1953-54. 

\bibitem{key2} {H. Cartan}: Varietes analytiques complexes et cohomologie,
  Colloque de Bruxelles, (1953), pp. 41-55. 

\bibitem{key3} {H. Cartan and S. Eilenberg}: Homological Algebra, Princeton
  University Press, 1956. 

\bibitem{key4} {H. Cartan and J.P. Serre}: C. R. Acad. Sci. (Paris),
  237(1953), pp. 128-130. 

\bibitem{key5} {S.S. Chern}: Complex Manifolds, Scientific Report on the
  second summer institute. Bull. Amer. Soc. 62(1956). pp. 101-117. 

\bibitem{key6} {R. Deheuvels}: C. R. Acad. Sci. (Paris), 238 (1954)
  pp. 1089; 240 (1955), p. 1183. 

\bibitem{key7} {F. Dolbeault}: Formes differentielles et cohomologie sur
  une varieta anlytique complex I, Annalso of Math., 64(1956),
  pp. 83-130. 

\bibitem{key8} {S. Eilenberg and N. Steenrod}: Foundations of algebraic
  topology, Princetion University Press, 1952. 

\bibitem{key9} {I. Fary}:Notion axiomatique de l'algebre de cochaines dans
  la theorie de J. Leray, Bull. Soc. Math. France, 82 (1954),
  pp. 97-135. 

\bibitem{key10} {J. Frenkel}:\pageoriginale C. R. Acad. Sci. (Paris),
  241(1955),  p.17. 

\bibitem{key11} {A. Grothendieck}: A general theory of fibre spaces with
  structue sheaf, National Science Foundation Research Project,
  University of Kansas, 1955. 

\bibitem{key12} {A. Grothendieck}: Theoremes de finitude pour la
  cohomologie des faisceaux, Bull. Soc. Math., France
  84(1956). pp 1-7. 

\bibitem{key13} {F. Hirzebruch}: Neue Topologische methoden in der
  algebraischen geometrie, Springer, 1956. 

\bibitem{key14} {W.V.D. Hodge and M.F. Atiyah}: Integrals of the second kind
  on an algebraic variety, Annals of Math., 62(1955), pp. 56-91. 

\bibitem{key15} {K. Kodaira}: On cohomology groups of compact analytic
  varieties with coefficients in some analytic faisceaux,
  Proc. Nat. Acad. Sci. U.S.A., 39(1953). pp. 865-868. 

\bibitem{key16} {K. Kodaira}: On a differential geometric method in the
  theory of analytic stacks, Ibid, pp. 1268-73.  

\bibitem{key17} {K. Kodaira and D.C. Spencer}: On Arithmetic Genera of
  algebraic varieties, Proc. Nat. Acad. Sci. U.S.A.,
  39(1953), PP. 641-649. 

\bibitem{key18} {K. Kodaira and D.C. Spencer}: Groups of complex line
  bundles over complex Kahler varieties, Ibid, pp. 868-872. 

\bibitem{key19} {K. Kodaira and D.C. Spencer}: Divisor calss groups on
  algebraic varieties, Ibid,pp. 872-877. 

\bibitem{key20} {K. Kodaira and D.C. Spencer}:\pageoriginale On a
  theorem of Lefschetz 
  and the lemma of Enriques-Severi-Zariski,
  Proc. Nat. Acad. Sci. U.S.A, 39(1953), pp. 1273-1278. 

\bibitem{key21} {J. Leray}: C.R. Acad. Sci. (Paris), 232(1946), p. 1366. 

\bibitem{key22} {J. Leray}: L'anneau spectral etlanneau filtre d'homologie
  d'un espace localement compat etd'une application continue, Jour. De
  Math.Pures et Appliquees, 29(1950), pp. 1-139. 

\bibitem{key23} {G. de Rham}: Varietes differentiables (Formes, Courants,
  formes harmoniques), Paris, Hermann, 1955. 

\bibitem{key24} {L. Schwartz}: Lectures on Complex Analytic Manifolds,
  Tata Institute of Fundamental Research, Bombay, 1955.  

\bibitem{key25} {J. P. Serre}: Queleques problems globaux relatifs aux
  varietes de Stein, Colloque de Bruxells, (1953), pp 57-68. 

\bibitem{key26} {J. P. Serre}: Un theorem de daulite,
  Com. Math. Helv. 29(1955), pp. 9-26. 

\bibitem{key27} {J. p. Serre}: Faisceaux algebriques coherents, Annals of
  Math., 61 (1955), pp. 197-278. 

\bibitem{key28} {N. E. Steenrod}: Homology with local coefficients, Annals
  of Math., 44(1943), pp. 610-627. 

\bibitem{key29} {A. Weil}:\pageoriginale Sur les theorems de de Rham,
  Comm. Math. Helv., 26(1952), pp. 119-145.  

\bibitem{key30} {O. Zariski}: Algebraic Sheaf theory, Scientific Report on
  the second summer institute, Bull. Amer. Math. Soc.,
  62(1956). pp. 177-141. 
\end{thebibliography} 

