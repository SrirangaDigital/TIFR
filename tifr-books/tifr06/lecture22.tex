\chapter{Lecture 22}\label{chap22:lec22}%lectu 22 

\noindent{\textbf{Proposition 8-a.}}
\textit{If\pageoriginale $\sum$ is a presheaf which determines the
  zero sheaf, then $H^p_{\Phi}(X, \sum) = 0$ for all $p \geqq 0$.} 

\begin{proof}
Let $f \in C^p_{\Phi} \left(\mathscr{U}, \sum \right)$. Then $\bigcup
\limits_{ U \in \mathscr{U} -(U_\ast)} \bar{U} \in \Phi$, and hence
has neighbourhoods 
$G$, $H$ with $\bar{G} \subset H$; $\bar{G}, \bar{H} \in \Phi$. Shrink
the covering 
$\mathscr{U}'=\bigg\{ \mathscr{U}-(U_*),U_* \cap \bar{H} \bigg\}$ of
$\bar{H}$ to a covering $\mathscr{W}' = \big\{ W_U \big\}_{U
  \in\mathscr{U}'}$ with $\bar{W}_U \subset U$. For each $x \in H$
choose a neighbourhood $V_x$ of $x$ such that  
\begin{enumerate}[a)]
\item if $x \in U$, $V_x \subset U \cap H$,

\item if $x \in W_U$, $V_x \subset W_U \cap H$,

\item if $x \notin \bar{W}_U, V_x \cap W_U = \phi$,

\item if $x \in U_o \cap \ldots \cap U_p = U_{\sigma}$, $\rho_{V_x
  U_{\sigma}} f(\sigma) = 0$, 
\end{enumerate}
and let $V_\ast = X- \bigcup\limits_{U \in \mathscr{U}-(U_*)} \bar{U}$. 
\end{proof}

Then $\bigg\{V_*,V_x \bigg\}_{x \in H}$ is a refinement of
$\mathscr{U}$. Choose $\tau : H \cup (\ast) \to \mathscr{U}$ such that $
x \in W_{\tau(x)}$ and $\tau(*)=U_*$. Then it can be verified that
$\tau^+ f =0$. 

The covering $\bigg\{ V_x \cap \bar{G}\bigg\}_{ x \in H}$ of $\bar{G}$ 
has a locally finite refinement $\{Y_i \}_{i \in I}$. Let
$\mathscr{U}_1$ be the proper covering consisting of $V_*$ together
with all $V$ such that $V =Y_i \cap G$ for some $i \in I$. Then
$\mathscr{W}_1$ is a $\Phi$- covering which is a refinement of
$\bigg\{ V_* ,V_x \bigg\}_{x \in H}$. Hence there is a function $\tau_1
: \mathscr{W}_1 \to \mathscr{U}$ with $V \subset \tau_1 (V)$ and such
that $\tau^+_1 f = 0$. Therefore $H^p_{\Phi}(X,\sum)=0$, $q.e.d$.  

Let\pageoriginale 
$$
\cdots \to \mathscr{S}^{q-1} \xrightarrow{d^q} \mathscr{S}^q
\xrightarrow{d^{q+1}} \mathscr{S}^{q+1} \to \cdots 
$$
be a sequence of homomorphisms of sheaves of $A$-modules with im $d^q
\subset \ker d^{q+1}$. Let $\mathbb{B}^q = \im d^q$, $\mathfrak{z}^q =
\ker d^{q+1}$ and $\mathscr{H}^q = \mathfrak{z}^q/ \mathbb{B}^q$. 

There is an induced sequence of homomorphisms of presheaves   
$$ 
\cdots \to \bar{\mathscr{S}}^{q-1} \xrightarrow{d^q}
\bar{\mathscr{S}}^q\xrightarrow{d^{q+1}} \bar{\mathscr{S}}^{q+1} \to
\cdots 
$$
with im $d^q = \bar{\mathbb{B}^q_o} \subset \bar{\mathbb{B}^q}$ and
$\ker d^{q+1}= \bar{\mathfrak{z}^q}$. Also there is an induced
sequence of homomorphisms  
$$ 
\cdots C^p_{\Phi} \left(\mathscr{U},\mathscr{S}^{q-1} \right) \xrightarrow{d^q}
C^p_{\Phi} \left(\mathscr{U},\mathscr{S}^{q} \right) \xrightarrow
{d^{d+1}} C^p_{\Phi} \left(\mathscr{U},\mathscr{S}^{q+1} \right) \to
\cdots  
$$
where $C^p_{\Phi}\left(\mathscr{U},\mathscr{S}^{q} \right)=
C^p_{\Phi} \left(\mathscr{U},\bar{\mathscr{S}}^{q} \right)$, with im
$d^q =C^p_{\Phi}\left(\mathscr{U},\bar{\mathbb{B}}^{q}\right)$ and
$\ker d^{q+1} = C^p_{\Phi}\left(\mathscr{U},\mathfrak{z}^{q}
\right)$. Then $H^{p,q}_2 C_{\Phi} \left(\mathscr{U},\mathscr{S}
\right)= C^p_{\Phi} \left(\mathscr{U},\mathfrak{z}^{q} \right)/  
C^p_{\Phi}\break \left(\mathscr{U},\bar{\mathbb{B}}^{q}_o \right)$. 

Let $\psi : \mathfrak{z}^q \to \mathscr{H}^q$ be the natural
homomorphism. There is an induced homomorphism $\psi :
\bar{\mathfrak{z}^q} \to \mathscr{H}^q$ with $\psi (\bar{\mathbb{B}}^q_o
)=0$. 

Hence there is an induced homomorphism
$$
\psi :  C^p_{\Phi}(\mathscr{U},\mathfrak{z}^{q}) \to
C^p_{\Phi}(\mathscr{U},\mathscr{H}^{q}), 
$$
which commutes with $\delta$ and $\phi_{\mathscr{W}\mathscr{U}}$ such
that $\psi (C^p_{\Phi}(\mathscr{U},\bar{\mathbb{B}}^{q}))=0$. 

Hence there are induced homomorphisms :  
\begin{align*}
& \psi : H^{p,q}_2 C_{\Phi}(\mathscr{U},\mathscr{S}) \to
  C^p_{\Phi}(\mathscr{U},\mathscr{H}^{q}),\\ 
& \psi* : H^{p,q}_{21} C_{\Phi}(\mathscr{U},\mathscr{S}) \to
  H^p_{\Phi}(\mathscr{U},\mathscr{H}^{q}),\\ 
&\psi* : H^{p,q}_{21} C_{\Phi}(\mathscr{S}) \to
  H^p_{\Phi}(X,\mathscr{H}^{q}). 
\end{align*}

\textit{The homomorphism\pageoriginale $\psi* : H^{p,q}_{21} C_{\Phi}(\mathscr{S})
  \to C^p_{\Phi}(X,\mathscr{H}^{q})$ is an isomorphism.} 

\begin{proof} 
The exact sequence 
$$
0 \to \bar{\beta}^q_o \to \bar{\mathfrak{z}^q} \to 
\bar{\mathfrak{z}}^q/ \bar{\mathbb{B}}^q_o \to 0 
$$
gives rise to an exact sequence  
$$
0 \to C^p_{\Phi}(\mathscr{U} , \bar{\mathbb{B}}^q_o) \to 
C^p_{\Phi}(\mathscr{U}, \mathfrak{z}^q) \to
C^p_{\Phi}(\mathscr{U} , \bar{\mathfrak{z}}^q/ \bar{\mathbb{B}}^q_o) \to
0. 
$$

Hence the induced homomorphism 
$$
H^{p,q}_{21} C_{\Phi}(\mathscr{U},\mathscr{S})=
C^p_{\Phi}(\mathscr{U},\mathfrak{z}^q)/
C^p_{\Phi}(\mathscr{U},\bar{\mathbb{B}}^q_o) \to
C^{p_{\Phi}}(\mathscr{U},\bar{\mathfrak{z}}^q / \bar{\mathbb{B}}^q_o) 
$$
is an isomorphism. Therefore 
$$
H^{p,q}_{21} C_{\Phi}(\mathscr{U},\mathscr{S}) \to
H^p_{\Phi}(\mathscr{U},\bar{\mathfrak{z}}^q / \bar{\mathbb{B}}^q_o) 
$$ 
and hence the homomorphisms 
\begin{equation*}
H^{p,q}_{21} C_{\Phi}(\mathscr{S}) \to
H^p_{\Phi}(X,\bar{\mathfrak{z}}^q / \bar{\mathbb{B}}^q_o) \cdots
\tag{1}\label{chap22:eq1}  
\end{equation*}
are isomorphisms.
\end{proof}

The exact sequence 
$$
0 \to \bar{\mathbb{B}}^q_o \to \bar{\mathbb{B}}^q \to
\bar{\mathbb{B}}^q/ \bar{\mathbb{B}}^q_o \to 0 
$$
gives rise to an exact sequence 
$$
0 \to C^p_{\Phi}(\mathscr{U},\bar{\mathbb{B}}^q_o) \overset{i}{\to}
C^p_{\Phi}(\mathscr{U},\bar{\mathbb{B}}^q) \overset{j}{\to}
C^p_{\Phi}(\mathscr{U},\bar{\mathbb{B}}^q / \bar{\mathbb{B}}^q_o) \to
0 
$$
and $i$, $j$ commute with $\delta$. Hence there is an exact cohomology 
sequence\pageoriginale 
\begin{multline*}
\cdots \to H^{p-1}_{\Phi}(\mathscr{U},\bar{\mathbb{B}}^q /
\bar{\mathbb{B}}^q_o) \xrightarrow{\delta^*}
H^{p}_{\Phi}(\mathscr{U},\bar{\mathbb{B}}^q_o)\xrightarrow{i^*}\\
H^{p}_{\Phi}(\mathscr{U}, \mathbb{B}^q) \xrightarrow{i^*}
H^{p}_{\Phi}(\mathscr{U},\bar{\mathbb{B}}^q / \bar{\mathbb{B}}^q_o)
\to. 
\end{multline*}

Since $i$, $j$ and $\delta$ commute with $\phi_{\mathscr{W}
  \mathscr{U}}$, there is an exact cohomology sequence of the direct
limits 
\begin{multline*}
\cdots \to H^{p-1}_{\Phi}(X,\bar{\mathbb{B}}^q / \bar{\mathbb{B}}^q_o)
\to H^{p-1}_{\Phi}(X,\bar{\mathbb{B}}^q_o) \to\\
H^{p}_{\Phi}(X,\bar{\mathbb{B}}^q) \to
H^{p}_{\Phi}(X,\bar{\mathbb{B}}^q / \bar{\mathbb{B}}^q_o) \to \cdots 
\end{multline*}

The presheaf $\bar{\mathbb{B}}^q / \bar{\mathbb{B}}^q_o$ determines
the 0-sheaf and hence $H^{p}_{\Phi}(X,\break\bar{\mathbb{B}}^q /
\bar{\mathbb{B}}^q_o) =o $ for all $p$. Hence, by exactness, 
$$
i^* : H^{p}_{\Phi}(X,\bar{\mathbb{B}}^q_o) \to
H^{p}_{\Phi}(X,\bar{\mathbb{B}}^q_o) 
$$
is an isomorphism. 

From the exact sequences of homomorphisms  
\[
\xymatrix{
0 \ar[r] & \bar{\mathbb{B}}^q_0 \ar[r] \ar[d] & \bar{\mathfrak{z}}^q
\ar[r] \ar[d] & \bar{\mathfrak{z}}^q/\bar{\mathbb{B}}^q_0 \ar[r]\ar[d]
& 0\\
0 \ar[r] & \bar{\mathbb{B}}^q \ar[r] & \bar{\mathfrak{z}}^q \ar[r] &
\bar{\mathfrak{z}}^q / \bar{\mathfrak{B}}^q \ar[r] & 0,
}
\]
one obtains exact sequences of homomorphisms
{\fontsize{8}{10}\selectfont
\[
\xymatrix{
H^p_{\Phi} (X,\bar{\mathbb{B}}^q_0) \ar[r] \ar[d] &
  H^p_{\Phi}(X,\mathfrak{z}^q) \ar[r]\ar[d] & H^p_{\Phi}
  (X,\bar{\mathfrak{z}}^q/\bar{\mathbb{B}}^q_0) \ar[r] \ar@{-->}[d]
  & H^{p+1}_{\Phi} (X,\bar{\mathbb{B}}^q_0) \ar[r]\ar[d] &
  H^{p+1}_{\phi} (X,\mathfrak{z}^q)\ar[d]\\
H^p_{\Phi}(X,\bar{\mathbb{B}}^q) \ar[r] & H^p_{\Phi}(X,\mathfrak{z}^q)
\ar[r] & H^p_{\Phi} (X,\bar{\mathfrak{z}}^q/\bar{\mathbb{B}}^q) \ar[r] &
H^{p+1}_{\Phi} (X,\bar{\mathbb{B}}^q) \ar[r] &
H^{p+1}_{\Phi}(X,\mathfrak{z}^q)  
}
\]}\relax
where four of the vertical homomorphisms are isomorphisms. Hence, by
the ``five'' lemma (see Eilenberg-Steenrod, Foundations of Algebraic
Topology, p. 16), the homomorphism 
\begin{equation*}
H^{p}_{\Phi}(X, \mathfrak{z} \bar{\mathbb{B}}^q_o) \to H^{p}_{\Phi}(X,
\mathfrak{z}^q \bar{\mathbb{B}}^q) \cdots \cdots \tag{2} \label{chap22:eq2}
\end{equation*}\pageoriginale
is an isomorphism.

Next, the exact sequence 
$$
0 \to \mathbb{B}^q \to \mathfrak{z}^q \to \mathscr{H}^q \to 0 
$$
gives rise to an exact sequence  
$$
0 \to \bar{\mathbb{B}}^q \to \bar{\mathfrak{z}}^q \to
\bar{\mathscr{H}}^q . 
$$

Let $\bar{\mathscr{H}}^q_o$ be the image of $\bar{\mathfrak{z}}^q $ in
$\bar{\mathscr{H}}^q $. Then there is an exact sequence  
$$
0 \to \bar{\mathfrak{z}}^q / \bar{\mathbb{B}}^q \to
\bar{\mathscr{H}}^q \to \bar{\mathscr{H}}^q /\bar{\mathscr{H}}^q_o \to
0 
$$
and the presheaf $\bar{\mathscr{H}}^q  / \bar{\mathscr{H}}^q_o$
determines the 0-sheaf. Hence $H^p_{\Phi}(X,\break\bar{\mathscr{H}}^q  /
\bar{\mathscr{H}}^q_o) =0$, and in the exact cohomology sequence, the
homomorphism  
\begin{equation*}
H^p_{\Phi}(X,\bar{\mathscr{H}}^q / \bar{\mathbb{B}}^q) \to 
H^p_{\Phi}(X,\bar{\mathscr{H}}^q) \cdots \cdots \tag{3} \label{chap22:eq3}
\end{equation*}
is an isomorphism.

Then $\psi^*$ is the composite isomorphism 
$$
 H^{p,q}_{21} C_{\Phi}(\mathscr{S}) \to H^p_{\Phi}(X,\bar{\mathfrak{z}}^q /
\bar{\mathbb{B}}^q_o) \to H^p_{\Phi}(X,\bar{\mathfrak{z}}^q /
\bar{\mathbb{B}}^q) \to H^p_{\Phi}(X,\bar{\mathscr{H}}^q). 
$$ 

\begin{defi*}%defi 0
We say that the degrees of $\{ \mathscr{S}^q \}$ are bounded below if
there is an integer $m$ such that $\mathscr{S}^q = 0$ for $q < m$. 
\end{defi*}




