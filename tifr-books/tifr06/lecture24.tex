\chapter{Lecture 24}\label{chap24:lec24}%chap 24

\textit{Every\pageoriginale $\Phi$ covering is shrinkable} as is shown
by the following result. 

\textit{Let $\{U_i \}_{i \in I}$  be a locally finite covering of space
  $X$ with some $i_* \in I$ such that $\bar{U}_i$ is normal for $i \in
  I-(i_*)$. Then there is a refinement $\{ V_i \}_{i \in I}$ with
  $\bar{V}_i \subset U_i$}. 

\begin{proof}
The union of the locally finite system $\{\bar{U}_i\}_{i \in I
  -(i_*)}$ of normal closed sets is normal and closed and $X-U_{i_\ast}
\subset \bigcup\limits_{i\neq i_\ast} U_i \subset \bigcup\limits_{i
  \neq i_\ast}\bar{U}_i$
with $X -U_i*$ closed and $\bigcup\limits_{i \neq i_\ast}{U}_i$
open. Hence there are open sets $G$, $H$ with $X-U_{i_\ast} \subset G$, $\bar{G}
\subset H$, $\bar{H} \subset \bigcup\limits_{i \neq i_\ast} U_i$. Let
$V_{i_*}=X - \bar{G}$, then $\bar{V}_{i_*} \subset X- G \subset
U_{i_*}$. 
\end{proof}

Since $\{\bar{H} \cap U_i \}_{i \in I -(i_*)}$ is a covering of
$\bar{H}$ and the closed subset $\bar{H}$ of $\bigcup\limits_{i \neq
  i_{\ast}} \bar{U}_i$ is normal, there is a covering $\{ P_i \}_{i
  \in I-(i_{\ast})}$ 
of $\bar{H}$ with $\bar{P}_i \subset \bar{H} \cap U_i$. Let $V_i = H
\cap P_i$ for $i \in I - (i_*)$. Then $V_i$ is open, $\bar{V}_i
\subset U_i$ and $\bigcup\limits_{i \neq i_*} V_i =H$. Then $\{ V_i \}_{i
\in I}$ is a covering of $X$ and $\bar{V}_i \subset U_i$ for all $i
\in I$ q.e.d.  

\textit{If $\mathscr{S}$ is a fine sheaf and if $C \subset U \subset
  X$ with $C$ closed, $U$ open and $\bar{U}$ normal, then the
  restriction of $ \mathscr{S}$ to $C$ is fine}. 

\begin{proof}
This result is proved in the same way as in the case (see Lecture
\ref{chap15:lec15}) 
that $X$ is normal except that the open set $H$ is to be 
replaced by its intersection with $U$ if necessary, so that $H \subset
U$. 
\end{proof}

\medskip
\noindent{\textbf{Proposition 11-a}}
\textit{If\pageoriginale $\{U_i\}_{i \in I}$ is a locally finite
  covering of a space $X$ 
  with some $i_* \in I$ such that $\bar{U}_i$ is normal for $i \in I
- (i_*)$, and if $\mathscr{S}$ is a sheaf whose restriction to each
closed subset $C$ of each $U_i$, $i \in I - (i_*)$ is fine (in
particular this is true if $\mathscr{S}$ is fine), then there is a
system $\{l_i\}_{i \in I}$ of homomorphisms $l_i : \mathscr{S}
\rightarrow \mathscr{S}$ such that}  
\begin{enumerate}[(i)]
\item \textit{for each $i \in I$ there is a closed set $E_i \subset
  U_i$ such that $1_i(S_x) = 0_x$ if $x \notin E_i$,}  

\item $\sum\limits_{i \in I} l_i = 1$.
\end{enumerate}

\begin{proof}% proof
Shrink to a covering $\{W_i\}_{i \in I}$ with $\bar{W}_i \subset V_i$,
$\bar{V}_i \subset G_i$, $\bar{G}_i \subset U_i$, where $W_i$, $V_i$ and
$G_i$, are open. Using the fineness of the restriction of
$\mathscr{S}$ to $\bar{G}_i$. one constructs homomorphisms $h_i :
\mathscr{S} \rightarrow \mathscr{S} \; i \neq i_*$, (actually the
homomorphisms are $h_i : \mathscr{S}_{\bar{G}_i} \rightarrow
\mathscr{S}_{\bar{G}_i}$, and we extend these by zero outside
$\bar{G}_i; \mathscr{S}_{\bar{G}_i}$ denotes the restriction of
$\mathscr{S}$ to $\bar{G}_i$) with  
\begin{align*}
h_i (s) & = s \quad \text{ if } \pi(s) \in \bar{W}_i,\\
& = 0_{\pi(s)} \text{ if } \pi(s) \in X - \bar{V}_i.
\end{align*}

Let the set $I - (i_*)$ be well-ordered and define
\begin{align*}
l_i & = \Big( \prod_{j < i} (1-h_j)\Big) h_i \quad (i \neq i_*),\\ 
l_{i_*} & = \prod_{\{ j ~ \in ~ I - (i_*) \}} (1-h_j).
\end{align*} 

(In\pageoriginale a neighbourhood of each point of $X$, $l_i$, $i \in
I$, is only a finite product.) Then $l_i : \mathscr{S} \rightarrow
\mathscr{S}$ is a 
homomorphism. Let $E_i = \bar{V}_i$ for $i \neq i_*$; then if $\pi(s)
\in X - \bar{V}_i$, we have $l_i(s) = 0_{\pi(s)}$ since $h_i(s) =
0_{\pi(s)}$. Let $E_{i_*} = X- \bigcup\limits_{i \in I - (i_*)} W_i$;
then $E_{i_*} \subset W_{i_*} \subset U_{i_*}$. If $\pi(s) \in X -
E_{i_*}$, then, for some $i \in I - (i_*)$, $\pi(s) \in W_i$ and hence
$h_i(s) = s$; so $l_{i_*}(s) = 0_{\pi(s)}$. 
\end{proof}

If $\pi(s) = x$, choose a neighbourhood $N_x$ of $x$ meeting at most a
finite number of the sets $U_i$, $i \in I - (i_*)$, say, for $i = i_1,
\ldots, i_q$ with $i_1 < \cdots < i_q$. Then 
\begin{align*}
\sum_{i \in I} l_i (s) & = h_{i_1} (s) + (1-h_{i_1}) h_{i_2}(s)\\ 
& + (1 - h_{i_1}) \dots (1 - h_{i_{q-1}}) h_{i_1} (s)\\
& + (1 - h_{i_1}) \dots (1-h_{i_q}) ~ (s)\\
& = s,
\end{align*}
and this completes the proof.

Let 
$$\cdots \rightarrow \mathscr{S}^{q-1} \xrightarrow{dq}
\mathscr{S}^q \xrightarrow{d^{q+1}} \mathscr{S}^{q+1} \rightarrow
\cdots
$$
 be a sequence of homomorphisms of sheaves with $d^{q+1} d^q =
0$. Such a sequence of sheaves is called a \textit{complex} of
sheaves. 

\begin{defi*} % definition
A complex of sheaves $\{\mathscr{S}^q \}$ is called {\em{homotopically
    fine}}, if, for each locally finite covering $\{U_i \}_{i \in I}$
with some $i_* \in I$ such that $\bar{U}_i$ is normal for $i \in I -
(i_*)$, there exist\pageoriginale homomorphisms $h^{q-1} :
\mathscr{S}^q \rightarrow \mathscr{S}^{q-1}$ and a family
$\{l^q_i\}_{i \in I}$ of endomorphisms of $\mathscr{S}^q$ such that   
\begin{enumerate}[(i)]
\item for each $i \in I$ there is a closed set $E^q_i \subset U_i$
  such that $l^q_i (S^q_x) = 0_x$ if $x \notin E^q_i$,  

\item $\sum\limits_{i \in I} l^q_i = 1 + d^q h^{q-1} + h^q d^{q+1}$. 
\end{enumerate}
\end{defi*}

\textit{If each $\mathscr{S}^q$ is fine, then the sequence $\{
  \mathscr{S}^q\}$ is homotopically fine}. 

\begin{proof}%Prf
Taking $h^q = 0$, this result follows immediately from Proposition 11-a. 
\end{proof}

\textit{If the sequence $\{ \mathscr{S}^q\}$ is homotopically fine,
  and $\mathscr{U}$ is a locally finite covering satisfying the
  conditions of the previous definition, then $H^{p,q}_{12} C_\Phi 
  (\mathscr{U}, \mathscr{S}) = 0$ for all $p > 0$. (This result is
  trivially true for $p < 0$)} 

\begin{proof}%Prf
As in the proof of Proposition \ref{chap16:prop12}, there are induced
homomorphisms 
$l^q_i (U) : \Gamma (U,\mathscr{S}^q) \rightarrow \Gamma
(U,\mathscr{S}^q)$ induced by $l^q_i$, and homomorphisms 
\begin{equation*}
k^{p-1} : C^p_\Phi (\mathscr{U}, \mathscr{S}^q) \rightarrow
C^{p-1}_\Phi (\mathscr{U},\mathscr{S}^q) \quad (p > 0) 
\end{equation*}
such that 
\begin{align*}
\delta^p  k^{p-1}  f(\sigma) + k^p \delta^{p+1}  f(\sigma) & =
\sum_{i \in I} l^{q}_i (U_\sigma) f(\sigma)\\ 
& = f(\sigma) + d^q h^{q-1} f(\sigma) + h^q d^{q+1} f(\sigma)
\end{align*}

Thus\pageoriginale $\delta k  f + k \delta f = f + d  h  f + h  d  f$. and 
hence 
$$
\delta  k + k \delta = 1 + d h + h  d : C^p_\Phi
(\mathscr{U},\mathscr{S}^q) \rightarrow C^p_\Phi  (\mathscr{U},
\mathscr{S}^q). 
$$

Since $d$ and $h$ commute with $\delta$, there are induced
homomorphisms 
\begin{align*}
& d^q : H^p_{\Phi} (\mathscr{U}, \mathscr{S}^{q-1}) \rightarrow
  H^p_{\Phi} (\mathscr{U}, \mathscr{S}^q),\\ 
& h^{+^q} : H^p_\Phi (\mathscr{U}, \mathscr{S}^{q+1}) \rightarrow
  H^p_\Phi (\mathscr{U}, \mathscr{S}^q). 
\end{align*}


Now, $H^{p,q}_1 C_\Phi (\mathscr{U}, \mathscr{S}) = H^p_\Phi 
(\mathscr{U}, \mathscr{S}^q)$ and, from the homotopy $k$ we have  
$$
dh^+ + h^+  d = 0 - 1 : H^{p,q}_1  C_\Phi (\mathscr{U},
\mathscr{S}) \rightarrow H^{p, q}_1  C_\Phi (\mathscr{U},
\mathscr{S}) 
$$
is homotopic to zero and hence $H^{p, q}_{12} C_\Phi  (\mathscr{U},
\mathscr{S}) = 0$. 
\end{proof}


