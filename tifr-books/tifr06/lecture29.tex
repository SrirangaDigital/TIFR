\chapter{Lecture 29}\label{chap29:lec29}

\textit{Given\pageoriginale homomorphisms $g_i : \mathscr{S}^i \to
  \mathscr{T}(i \in I)$, there is an induced homomorphism $g : \sum
  \limits_{i \in  I} \mathscr{S}^i \to \mathscr{T}$ with $g
  h^i=g_i$.}  

\begin{proof}
If $s= \sum_i s^i \in S_x$, let $g(s)= \sum_{ig_i}(s^i) \in T_x$.Then
$g|S_x : S_x \to T_x$ is clearly a homomorphism. Choose an open set
$U$ and sections $f_j : U \to S^{i_j} $ so that the section defined by
$f(y) = \sum h^{i_j}f(y)$ goes through $s$. Then $gf(y)= \sum gi_j f_j
(y)$ and $gf$ being the sum of a finite number of sections, is a
section. Thus $g$ is continuous and is a sheaf homomorphism.  
\end{proof}

\begin{note*}
 Since $\mathfrak{a}$ itself is a sheaf of $\mathfrak{a}$-modules,
 there is, for any $I$, a direct sum $\sum \limits_{i \in I}
 \mathfrak{a}$ where each direct  summand is $\mathfrak{a}$. This is again a
 sheaf of $\mathfrak{a}$ -modules.    
 \end{note*}
 
If $\mathscr{S}$ is a sheaf of $\mathfrak{a}$-modules over a space
$X$, and $\mathfrak{a}_Y$ is the restriction of $\mathfrak{a}$ to a
subset $Y$, then clearly the restriction $\mathscr{S}_Y$ of
$\mathscr{S}$ to $Y$ is a sheaf of $\mathfrak{a}_Y$-modules.  

\begin{defi*}
The following properties of sheaves $\mathscr{S}$ of $\mathfrak{a}$-
modules over a space $X$ are called property $(a_1)$ and property $(a):$  
\end{defi*}

\medskip
\noindent{\textbf{Property ({\boldmath$a_1$}).}}%Property (a1) 
 \; There is a covering $\{U_j \}_{j \in J}$ of $X$, and for each $j \in
J$, there is an index set $I_j$ and an epimorphism $\phi_j :
\sum\limits_{i \in I_j} \mathfrak{a}_{U_j} \to \mathscr{S}_{U_j}$.  

\medskip
\noindent{\textbf{Property (a).}}%Property (a)
\; There is a covering $\{U_j \}_{j \in J}$ of $X$, and for each $j \in
J$, there is a natural number $k_j$ and an epimorphism $\phi_j :
\sum^{k_j}_{i=1} \mathfrak{a}_{U_j} \to \mathscr{S}_{U_j}$.\pageoriginale

$\mathscr{S}$ \textit{has property $(a_1)$ if and only if each point 
  $x \in X$ has a neighbourhood $U$ such that the sections in $\Gamma
  (U, \mathscr{S})$ generate} $\mathscr{S}_U$. (That is, for each $y
\in U$ and $s \in S_y$, $s= \sum^k_{j=1} a^j_y f_j(y)$ for some
finite number $k$ of elements $a^j_j \in A_y$ and sections $f_j \in
\Gamma (U,\mathscr{S})$.) 

\begin{proof}
{\em Necessity.} Let $\mathscr{S}$ have property $(a_1)$. Since
$\{U_j\}$ is a covering, $x \in U_j$ for some $j$; let $U =U_j$. Then
there is an index set $I$ and an epimorphism $\phi : \sum \limits_{i
  \in I} \mathfrak{a}_U \to \mathscr{S}_U$.  
\end{proof}

Let $f^i = \phi h^i 1 : U \to S$ where 1 is the unit section in
$\mathfrak{a}_U$, 
$$
U \xrightarrow {1} \mathfrak{a}_U\xrightarrow {h^i}\sum \limits_{i \in
  I} \mathfrak{a}_U \xrightarrow{\phi} \mathscr{S}_U.  
$$

Then if $s \in \mathscr{S}_U$, $s=\phi(\sum_i a^i_y)$ for some $a^i_y
\in A_y$ with $a^i_y = O_y$ except for a finite number of $i$, say
$i=i_1, i_2, \ldots, i_k$. Then  
\begin{align*}
\sum^k_{j=1} a^{i_j}_y f_{i_j}(y) & = \sum^k_{j=1} a^{i_j}_y \phi
h^{i_j}1_y = \sum^k_{j=1} \phi h^{i_j} a^{i_j}_y 1_y\\ 
& = \phi (\sum^k_{j=1} h^{i_j} a^{i_j}_y ) = \phi(\sum_{i \in I}a^i_y)
=s. 
\end{align*}

\medskip
\noindent{\textbf{Sufficiency.}}
 For each $x \in X$, there is a neighbourhood $U_x$ of $x$ such that
 the sections over $U_x$ generate $\mathscr{S}_{U_x}$. Then
 $\{U_x\}_{x \in X}$ is a covering of $X$. Let $I_x$ be the set of
 sections $\Gamma (U_x, \mathscr{S})$. For each section $i \in I_x$,
 there is a sheaf homomorphism $\phi^i_x : a_{U_x} \to
 \mathscr{S}_{U_x}$ given, for $a \in A_y$, by $ \phi^i_x (a) = a\cdot
 i(\pi a)$.    

Then\pageoriginale there is an induced homomorphism
$$
\phi_x : \sum_{i \varepsilon I_x} a_{U_x} \to \mathscr{S}_{U_x}.
$$

Then for $s \in S_y$ and $y \in U_x$,
$$
s=\sum^k_{j=1} a^j_y i_j (y)= \sum^k_{j=1} \phi^{i_j}_x (a^j_y) \in
\phi_x ( \sum_{i \in I_x} a_{U_x}). 
$$

Thus $\phi_x$ is an epimorphism.

\medskip
$\mathscr{S}$ \textit{has property $(a)$ if and only if each point $x$
  has a neighbourhood $U$ such that some finite number of sections 
 $f_i \in \Gamma (U, \mathscr{S})~(i=1,\ldots,k)$  generate
$\mathscr{S}_U$.} 

\begin{proof}
Similar to the proof given above.
\end{proof}

It is clear that $(a)$ implies $(a_1)$, i.e., each sheaf with property
$(a)$ has property $(a_1)$. The sheaf $\sum \limits_{i \in I}
\mathfrak{a}$ has property $(a_1)$ and $\sum^k_{i=1} \mathfrak{a}$ has property
$(a)$. In particular, the sheaf $\mathfrak{a}$ of
$\mathfrak{a}$-modules has property (a). 


\textit{If $\mathscr{S}_i, i=1, \ldots,k$ are sheaves of $\mathfrak{a}$- modules
  with property $(a_1)$ (\resp $(a)$), then the direct sum
  $\sum^k_{i=1} \mathfrak{a}_i$ has property $(a_1)$ (resp $(a)$)}. 

\begin{proof}
Clear.

Statements (1), (2), (3), (4), (5), (6), (Lectures \ref{chap29:lec29},
\ref{chap30:lec30}, \ref{chap31:lec31},) are required to prove Serre's
theorem on coherent 
sheaves, (see the\pageoriginale next lecture for the definition of coherent
sheaves), i.e., if $0 \to \mathscr{S}' \to \mathscr{S} \to
\mathscr{S}'' \to 0$ is an exact sequence of sheaves, and if two of
them are coherent, then the third is also coherent.  

\noindent
(1) \qquad \textit{If $f : \mathscr{S}' \to \mathscr{S}$ is an
  epimorphism and $\mathscr{S}'$ has property $(a_1)$ (resp (a)),
  then $\mathscr{S}$ has property} $(a_1)$ (resp (a)). 
\end{proof}

\begin{proof}
Clear.
\end{proof}

\begin{example*}
If $M$ is a finitely generated $A$- module, the constant sheaf $M$ has
property (a) with respect to the constant sheaf $A$. If a constant
sheaf is a sheaf of $\mathfrak{a}$- modules, then it has property
$(a_1)$. If $X$ 
is the unit segment $0 \leq x \leq 1$, the subsheaf $\mathscr{S}$ of
the constant sheaf $Z_2$ obtained by omitting (1, 1) does not have
property $(a_1)$ either as a sheaf of $Z$- modules or as a sheaf of
$Z_2$ -modules. With the same $X$, the sheaf $\mathscr{S}$ of germs of
functions $f : X \to Z_2$, considered as a sheaf of $Z_2$ -modules,
has property $(a_1)$ but not property $(a)$. But, considering
$\mathscr{S}$ as a sheaf of rings with unit, it has property $(a)$
with respect to itself, The sheaf $\mathfrak{a}$ of germs of analytic
functions in the complex plane has property $(a)$ as a sheaf of
$\mathfrak{a}$- modules, but as a sheaf of $C$- modules (where $C$ is
the field of complex numbers) it does not even have property
$(a_1)$. For, there are natural boundaries for analytic functions,
e.g., let $f$ be an analytic function in $|z|<1$, with $|z| = 1$ as
natural boundary. If $\mathfrak{a}$ has property $(a_1)$ as a sheaf of
$C$-modules, then by considering a point on the boundary, we see that
$f$ can be continued to a neighbourhood of this boundary point and
this is a contradiction.  
\end{example*}

