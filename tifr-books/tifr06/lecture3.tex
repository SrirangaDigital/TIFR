\chapter{Lecture 3}

\noindent
\textbf{Sections}.

\begin{defi*}
A\pageoriginale \textit{section} of a sheaf $(S, \pi , X)$ over an open set $U
\subset X$ is a map $f: U \to S$ such that $\pi \cdot f=1|U$  where 
$1|U$ denotes the identity function on $U$. (A map is a continuous
function). 
\end{defi*}

By abuse of language, the image $f(U)$ is also called a section.

\textit{For each open set $U \subset X$ the function $f: U \to S$,
  where $f(x)=0_x$ is a section.} 

\begin{proof}
Zero is continuous, and $\pi(0_x)=x$.
\end{proof}

\noindent
This section will be called the 0-section (zero section).

\begin{proposition}%%%% 2
 If $\mathscr{S}=(S, \pi , X)$ is a sheaf of abelian groups, the
  set of all sections of $\mathscr{S}$ over a non-empty open set $U$
  forms an abelian group $\Gamma(U,\mathscr{S})$. If $\mathscr{S}$ is a
  sheaf of sings with unit, $\Gamma(U,\mathscr{S})$ is ring with unit
  element. If $\mathscr{S}$ is a sheaf of $\mathfrak{a}$-modules,
  $\Gamma(U,\mathscr{S})$ is a unitary left $\Gamma(U,
  \mathfrak{a})$-module. If $\mathscr{S}$ is a sheaf of $B$-modules,
  $\Gamma(U,\mathscr{S})$ is a unitary left $B$-module.  
\end{proposition}

\begin{note*}
The operations are the usual ones for functions. If
$f,g \in \Gamma(U,\mathscr{S})$, $a \in \Gamma(U, \mathfrak{a})$, $b
\in B$, define   
\begin{align*}
(f+g) (x) & =f(x)+g(x), (fg) (x)  = f(x) \cdot g(x),\\
(af)(x) & = a(x) \cdot f(x), (bf) (x)  = b \cdot f(x) .
\end{align*}
\end{note*}

\noindent
The zero element of $\Gamma(U,\mathscr{S})$  is the 0-section over
$U$. If $\mathscr{S}$ is a sheaf of rings with unit, the unit element
of $\Gamma(U,\mathscr{S})$ is the unit section $1: U \to S$ where
$1(x) = 1_x$. 

\begin{proof}
The\pageoriginale proposition follows from the fact that addition
zero, inverse, 
unit and multiplication (ring multiplication as well as scalar
multiplication) are continuous. 
\end{proof}

\begin{remark*}
If $\mathfrak{a} = (X \times B, \tau ,X)$ is a constant sheaf of rings with
units, for each open set $U \subset X$ we can identify $B$ with the
ring of constant sections $f_b$, $b \in B$ where $f_b(x)=(x,b)$ over
$U$. Then $B \subset \Gamma (U, \mathfrak{a})$ is a subring, and by
restricting the ring of scalars, each $\Gamma (U,\mathfrak{a})$-module
becomes a  $B$-module. ($B$ need not be the whole of $\Gamma (U,
\mathfrak{a})$).   
\end{remark*}

\textit{Abuse of} $\phi$. If $\mathscr{S}$ is a sheaf of
$\mathfrak{a}$-modules, we agree that the unique section over the
empty set $\phi$ is the 0-section, and the set
$\Gamma(\phi\mathscr{S})=0 $.  


\setcounter{exam}{5}
\begin{exam}%%%% 6
If $\mathscr{S}=(S, \pi , X)$ is the sheaf of function elements over
the complex sphere $X$, $\Gamma(X, \mathscr{S})$ can be identified with
the ring of functions, analytic in $U$. Then $\Gamma (X, \mathscr{S})$
is the ring of functions, analytic everywhere, hence is isomorphic to
the ring of complex numbers $C$.  
\end{exam}

\begin{note*}
Usually a sheaf $\mathscr{S}=(S, \pi , X)$ may be
interpreted as describing some local property of the space $X$; then
$\Gamma(X, \mathscr{S})$ gives the corresponding global property. 
\end{note*}

\textit{A section $f: U \to S$ is an open mapping}

\begin{proof}
This is proved using the fact that $f$ is continuous and that $\pi$ is
a local homeomorphism. 
\end{proof}

We\pageoriginale can now characterise the sections of $S$.

\textit{The necessary and sufficient condition that a set $G \subset
  S$ is a section $f(U)$ over some open set $U \subset X$ is that $G$
  is open and $\pi |G$ is a homeomorphism.} 

\begin{proof}
The sufficiency is easy to prove. To prove the necessity let $f$ be a
section over $U$, then since $f$ is open $f(U)$ is open. Since $f: U
\to f(U)$ is 1-1, open, continuous, $\pi|f(U):f(U) \to U$ \textit{is
  a homeomorphism} 
\end{proof}

\textit{We have shown that if $f$ is a section over $U$ then $f$ is a 
  homeomorphism of $U$ onto $f(U)$.} 

\textit{The sections $f(U)$ form a base for the open sets of $S$.} 

\begin{proof}
We have already proved that the open sets of $S$ which project
homeomorphically onto open sets of $X$ form a base for the open sets
of $S$ 
\end{proof}

\textit{The intersection $f(U) \cap g(V)$ of two sections is a
  section.}

\begin{proof}
$f(U) \cap g(V)$ is open and projects homeomorphically onto an open
  set of $X$ since each of $f(U)$ and $g(V)$ has this property. 
\end{proof}

\textit{If $f:U \to S$ is a section, the set $\{ x : f(x) = 0_x \}$  
  is open in $U$.}  

\begin{proof}
$\{ x : f(x) = 0_x \}= \pi (f(U) \cap 0(U))$ (0 denotes the 0-section
  over $U$), hence is open in $U$. 
\end{proof}

\begin{defi*}
If $f:U \to S$ is a section, the support of $f$, denoted as supp $f$,
is the set $\{ x : f(x) \neq  0_x \}$. This set is
closed\pageoriginale in $U$. If 
$f$ is a section over $X$, supp $f$ is a closed subset of $X$ 
\end{defi*}

\begin{note*}
Since the sections of $S$ form a base for the open sets
of $S$, the topology of $S= \cup S_x$ can be described by specifying
the sections. (See appendix at the end of the lecture). 
\end{note*}

Let $\mathfrak{a}=(A, \tau,X)$ be a sheaf of rings with unit, and let
$\mathscr{S}=(S, \pi ,X)$ and $\mathscr{R}= (R, \rho ,X)$ be sheaves
of $\mathfrak{a}$-modules (all over the same base space $X$). 

\begin{defi*}
A \textit{homomorphism} $h: \mathscr{S} \to \mathscr{R}$ is a
\textit{map} $h:S \to R$ such that $\rho \cdot h = \pi$ and its restriction
$h|S_x=h_x: S_x \to R_x$ is an $A_x$ homomorphism for each $x \in X$ 
\end{defi*}

This definition includes as a special case the definition of
homomorphisms of sheaves of $B$-modules and sheaves of abelian
groups. 

\textit{If $h: \mathscr{S} \to \mathscr{R}$ is a homomorphism, the
  image of each section is a section}. 

\begin{proof}
If $f: U \to S$ is a section, then $hf:U \to R$, the image of the
section $f$ defined by $(hf)(x)=h(f(x))$, is continuous and $\rho (hf)
= \Pi \cdot f=1|U$. 
\begin{enumerate}[(1)]
\item $h$ \textit{is an open mapping}.

\item $h$ \textit{is a local homeomorphism}.
\end{enumerate}
\end{proof}

\begin{proof}
\textbf{(1)} If $G \subset S$ is open, then $G$ is a union of sections, hence
  $h(G)$ is a union of sections of $R$, hence is open  

\textbf{(2)} Each\pageoriginale $p\in S$ is contained in some section
$f(U)$. So $hf(U)$ is a section in $R$ and $ h | f(U) = hf \pi
| f(U)$, and each of $\pi | f(U) : f(U) \rightarrow U$ and $hf :
U \rightarrow hf(U)$ is a homeomorphism. 
\end{proof}

\noindent
\textbf{Appendix.}

A sheaf may be described by specifying its sections as follows:
Suppose that we are given a space $X$ and mutually disjoint abelian
groups $S_x$, one for each point $x \in X$. Also suppose that we
are given a family $\sum = \{ s\}$ of functions with domain open in
$X$ and values in $\cup_x S_x$, $s: \dom (s) \rightarrow \cup_x 
S_x$, such that, if $x \in \dom  (s)$, $s(x) \in S_x$. Suppose
further that 
\begin{enumerate}[(i)]
\item the images for all $s \in \sum$ cover $\cup S_x$,

\item if $s_1  (x)$, $s_2 (x) $ are defined then, for some open $U$ with
  $x  \in \subset\break \dom (s_1) \cap\dom  (s_2)$, $(s_1 + s_2) |  U
   \in  \sum$, 

\item if $s(x) = 0_x$ then, for some open $U$ with $x \in U 
  \subset \dom (s) s(U)$ consists entirely of zeros. 
\end{enumerate}

\textit{If $S = \cup S_x$ with $\{s(U) \}$, for all $s \in \sum$ and
  open $U \subset \dom (s)$, as base for open sets and if $\pi(p) = x$
  for $p \in S_x$ then $(S,\pi,X)$ is a sheaf}. 

\begin{proof}
Let $p  \in  s(U) \cap s_1 (U_1)$ and let $x = \pi (p)$. By (i)
there exists $s_2$ with $s_2(x) = -p$, by (ii) there exists a
neighbourhood $V$ of $x$ with $(s + s_2)  | V \in \sum$ and by
(iii), since $(s + s_2)x = p -p = 0_x$ there is a smaller
neighbourhood $V'$ of $x$ with $(s + s_2) (V')$ consisting of
zeros. Similarly there is a neighbourhood $V'_1$ of $x$ with $(s_1 +
s_2) (V'_1)$ consisting of zeros. 
\end{proof}

\noindent
Let\pageoriginale $W = V' \cap V'_1$, then $(s + s_1 + s_2)  | W = s  |  W  =
s_1  | W$. Then $s(W)$ is in the proposed base and $p \in s(W)
\subset s(U) \cap s_1(U_1)$. Therefore the axioms for a base are
satisfied. 

Then $s : U \rightarrow S$ is continuous. For if $x \in \dom (s)$ and
$p = s(x)$, any neighbourhood $G$ of $p$ contains a neighbourhood $s_1
(U_1)$ and again there is an open $W$ with $p \in s(W) = s_1(W)
\subset s(U_1) \subset G$. Hence $s : U \rightarrow s(U)$ is a
homeomorphism, since $s$ is clearly 1-1 and open. Then $\pi  |
s(U)$ is the inverse homeomorphism and its image $U$ is open. Thus
$\pi$ is a local homeomorphism. 

Addition is continuous. For if $p$, $q \in S_x$, with $p + q \in
s_2(U_2)$ suppose $p \in s(U)$ and $q \in s_1(U_1)$. By (ii) there
is a neighbourhood $V$ of $x$ with $(s_1 + s_2)  | V \in
\sum$. Then $p + q \in s_2 (U_2) \cap (s + s_1) (V)$ and hence for
some $W$, with $x \in W \subset U_2 \cap V$, $s_2  |  W = (s + s_1) 
|  W$. Thus, if $ r \in s (W)$, $t \in s_1 (W)$ and $\pi (r) = \pi
(t)$, then $s(r) + s_1(t) = s_2(t) \in s_2(U_2)$. 

