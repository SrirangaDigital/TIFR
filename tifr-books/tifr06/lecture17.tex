\chapter{Lecture 17}%lecut 17

In\pageoriginale this and the next lecture, we shall give a proof of
de Rham's theorem.  

Let $X$ be an indefinitely differentiable $(C^\infty)$ manifold of
dimension $n$, which is countable at infinity (i.e) a countable union
of compact sets); we assume that $X$ is a Hausdorff space. Then $X$ is
paracompact and normal. (Dieudonne, Jour. de Math. 23 (1944)). The set
$\mathscr{E}^p (U)$ of all $\underline{C^\infty}$ (alternating)
differential $p$-forms on an open set $U$ forms a vector space over the
field of real numbers. Exterior differentiation gives a homomorphism
$d^p$,    
$$
d^p : \mathscr{E}^{p-1} (U) \to \mathscr{E}^p (U)  
$$
with $d^{p+1} d^p = 0$. In particular, there is a sequence 
$$
0 \xrightarrow{d^o} \mathscr{E}^o (X)  \xrightarrow{d^1}
\mathscr{E}^1(X) \to \cdots \xrightarrow{d^p}  \mathscr{E}^p (X) \to
\cdots \xrightarrow{d^n}  \mathscr{E}^n (X)  \xrightarrow{d^{n+1}} 0 
$$
with  im $ d^p \subset \ker d^{p+1}$. Let 
$$
H^p ( \mathscr{E}(X)) = \ker d^{p+1} / \text{ im } d^p. 
$$

This vector space of the closed $p$-forms modulo the derived $p$-forms is
called the $p$-th \textit{de Rham cohomology vector space} of the
manifold $X$. 

If $V\subset U$, the inclusion map $i : V \to U$ induces a
homomorphism 
$$
\rho_{_{VU}} = i^{-1} :  \mathscr{E}^p (U) \to  \mathscr{E}^p (V)
$$
which\pageoriginale commutes with $d$. Thus the system $ \mathscr{E}^p = \{
\mathscr{E}^p (U), \rho_{_{VU}} \}$ is a presheaf which determines a
sheaf $\Omega p$, called the sheaf of germs of $p$-forms, and  
$$
d^p :  \mathscr{E}^{p-1} \to  \mathscr{E}^p
$$
is a homomorphism of presheaves which induces a homomorphism 
$$
d^p : \Omega^{p-1} \longrightarrow \Omega^p.
$$

There is a constant presheaf $\{ R, \rho_{_{VU}}\}$ where $R$ is the 
filed of real numbers and $\rho_{_{VU}} : R \to R$ is the identity. This
presheaf determines the constant sheaf $R$. There is a homomorphism $e
: R \to  \mathscr{E}^o (U)$, ($\mathscr{E}^o(U)$ is the space of
$C^\infty$ functions on $U$) where $e(r)$ is the function on $U$ with
the constant value $r$, and further $e$ commutes with
$\rho_{_{VU}}$. Thus, there is an induced sheaf homomorphism $e : R \to
\Omega^o$. Hence, we have a sequence of homomorphisms of sheaves  
$$
0 \to R \xrightarrow{e} \Omega^o \xrightarrow{d^1} \cdots \to 
\Omega^{p-1} \xrightarrow{d^p} \Omega^p \to \ldots, 
$$
with $d^{p+1} d^p=0$.

There is a homomorphism
$$
\{f_U\} :\{\mathscr{E}^p (U), \rho_{_{VU}}\} \to \bar{\Omega^p}, 
$$
$(\bar{\Omega}^p$ denotes the presheaf of sections of $\Omega^p$),
where the image of an element of $\mathscr{E}^p (U)$ is the section
over $U$ which it determines in $\Omega^p$.\pageoriginale Then $d$ commutes with
$f_U$ and, in particular, with $f_x $. Thus we have the commutative
diagram: 
\[
\xymatrix{
\mathscr{E}^{p-1}(X)\ar[r]^{d^p}\ar[d]_{f_X} &
\mathscr{E}^p(X)\ar[d]^{f_X}\\
\Gamma (X,\Omega^{p-1}) \ar[r]^{d^p} & \Gamma (X,\Omega^p).
}
\]

If a $p$-form $\omega \in \mathscr{E}^p (X)$ is not zero, there
is some point $x \in X$ at which it does not vanish, and hence
$\rho_{x X} \omega \neq 0$. Thus $f_X \omega \neq 0$, i.e., $f_X$ is a
monomorphism. (In the same way, $f_U$ is a monomorphism for each open
set $U$.) $f_X$ is also onto, hence an isomorphism. For, if $g
\in \Gamma (X, \Omega^p)$, then since $\Omega^p$ is the sheaf
of germs of $p$-forms, for each $x \in X$ there is a
neighbourhood $U_x$ of $x$ and a $p$-form $\omega_x$ defined on $U_x$
such that section $g$ and the section determined by $\omega_x$
coincide on $U_x$. Then $\{ U_x\}$ forms a covering for $X$, and since
the section determined by $\omega_x$ and $\omega_y$ coincide on $U_x
\cap U_y$, using the fact that $f_U$ is a monomorphism for each open
set $U$, we see that the forms $\omega_x$ and $\omega_y$ themselves
coincide on $U_x \cap U_y$; hence they define a global from $\omega$,
such that $f_X(\omega) = g$. Thus $f_X$ gives an isomorphism of the
sequences:  
\[
\xymatrix{
0\ar[r] & \mathscr{E}^{o}(X) \ar[r]^{d^1} \ar[d]_{f_X} & \ldots
\ar[r]& \mathscr{E}^{p-1}(X)\ar[r]^{d^p} \ar[d]_{f_X} &
\mathscr{E}^p(X) \ar[r] \ar[d]_{f_X} & \ldots\\
0 \ar[r] & \Gamma(X,\Omega^o) \ar[r]^{d^1} & \ldots \ar[r] &
\Gamma(X,\Omega^{p-1}) \ar[r]^{d^p} & \Gamma (X,\Omega^p)\ar[r] & \ldots
 }
\]

 Hence\pageoriginale there is an induced isomorphism of the cohomology
 vector  spaces:  
 $$
 f^*_X : H^q (\mathscr{E} (X)) \to H^q \Gamma (X, \Omega) \qquad (q
 \geqq 0). 
 $$

 \textit{Poincare's lemma. The sequence} 
 $$
 0 \to R \to \Omega^o \xrightarrow {d^1} \cdots \to \Omega{^{p-1}}
 \xrightarrow{d^p} \Omega^p \to \cdots 
 $$
 \textit{is exact}.

 \begin{proof}
We have to prove that for each point $a \in X$, the sequence 
$$ 
0 \to R_a \xrightarrow{e} \Omega{^o_a} \xrightarrow{^{d^1}} \cdots \to
\Omega^{p-1}_a \xrightarrow{d^p} \Omega^p_a \cdots 
$$
is exact, where the subring $R_a$ of $\Omega^o_a$, consisting of the
germs of constant functions at $\underbar{a}$, is identified with the
filed $R$ of real number. Choose a coordinate neighbourhood $W$ of
$\underbar{a}$ with coordinates $(x_1 ,\ldots,x_n)$ and suppose that $a
=(0,\ldots,0)$. Then $\Omega^p_a$ is the direct limit of the system
$\left\{\mathscr{E}^p (U), \rho_{_{VU}} \right\}_{a \varepsilon U'}$
where $U$ belongs to the cofinal set of those spherical neighbourhoods
$x^2_1 + \cdots + x^2_n < r^2$ which are contained in $W$.   
 \end{proof} 
 
For each such $U$, let
 \begin{align*}
h  & : \mathscr{E}^o (U) \to R  \\
\text{and } \qquad  k^{p-1}  & : \mathscr{E}^p (U) \to
\mathscr{E}^{p-1} (U) \qquad (p \geqq 1 )
 \end{align*}
   be the homomorphisms defined by
 $$
 h (f) = f (0, \ldots, 0), 
  $$
and
\begin{align*}
&k^{p-1} ( f (x_1, \ldots , x_n) dx_{_{i}} \cdots dx_{i_{p}} ) \\ 
= &( \int^1_0 f ( tx_1, \ldots, tx_n) t^{p-1} dt ) \cdot\sum^P_{j=1} ( -1
)^{j-1} x_{i_{j}} dx_{i_{1}} \dots \hat{dx}_{i_{j}} \dots dx_{i_{p}} 
\end{align*}\pageoriginale
respectively. (The formula on the right is an alternating function of
$ i_1,\ldots, i_p $; $h$ and  $k^{p-1} $ are then extended by
linearity to $\mathscr{E}^o (U) $ and  $ \mathscr{E}^p (U)$
respectively.) One now verifies that 
\begin{align*}
e h + k^0 d^1 & =  1,\\
d^p k^{p-1} + k^p d^{p+1} & = 1 \; ( p  \geqq 1 ),
\end{align*}
where 1 denotes the identity map.

(The computation is carried out at the end of the lecture.)

Thus $ f \in \ker d'$ implies that $ f \in \im e$
and $ \omega \in  \ker d^{p+1}$ implies that $ \omega \in  \im
d^p $. Hence $ \ker d^1 = \im e $ and  $ \ker d^{p+1}  = im d^p$,
since already $ \im e \subset \ker d^1 $ and $ \im d^p \subset \ker
d^{p+1}$. Hence the  sequence  
$$
0 \rightarrow R_U \xrightarrow{e} \mathscr{E}^o (U) \xrightarrow{d^1}
\dots \mathscr{E}^{p-1} (U) \xrightarrow{d^p} \mathscr{E}^p (U)
\rightarrow \dots  
$$
is exact, and since exactness is preserved under direct limits,
therefore the limit sequence 
$$
0 \rightarrow R_a  \xrightarrow{e} \Omega^o_a \xrightarrow{d^1} \dots
\rightarrow \Omega^{p-1}_a \xrightarrow{d^p} \Omega^p_a \rightarrow
\dots 
$$
is exact, $q.e.d$.

\textit{The sheaf $\Omega^p$ is fine.}

\begin{proof}%proof 0
Since the space $X$ is paracompact and normal, by Proposition
\ref{chap16:prop13} 
(Lecture \ref{chap16:lec16}), it is enough to prove that the sheaf $
\Omega^p $ is  
locally fine. Let $U$ be an open set of $X$, and let  $ a \in U $. 
\end{proof} 

We\pageoriginale may assume that $ \bar{U} $ is  compact and that it
is contained in 
some coordinate neighbourhood $N$ of $a$. Let $V$ be an open subset with
$ a \in V $ and $ \bar{V} \subset U $. We will now prove that the
restriction $ \Omega^p_{\bar{V}} $ of $ \Omega^p $ to  $ \bar{V} $ is
fine. 
 
Let $ E \subset G $ with $E$ closed and $G$ open in $\bar{V}
$. Extend $G$ to an open set $ H \subset U $, so that $ G = \bar{V}
\cap H $. Then $ \bar{U} $ is covered by a finite number of spherical
neighbourhoods $S_i$ contained in $N$, such that $ \bar{S}_i $ either
does not meet $E$ or is contained in $H$. 

For each $i$, choose an indefinitely differentiable function $f_i$
which is positive inside $S_i$ and vanishes  out side $S_i $. We
construct one such function as follows: For the spherical
neighbourhood $ \sum\limits^n_{j=1} ( x_j - b_{ij} )^2 < r^2_i$, let  
\begin{align*}
g_i (r) &= 0  \hspace{3cm } (r \geqq r_i),\\ 
&= \int^{r_i}_r \exp \left\{ \frac{1}{(t - \frac{r_i}{2})
  (t-r_i)}\right\} \,dt\; ( \frac{r_i}{2} \leqq r \leqq r_i), \\ 
&= \int^{r_i}_{r_{i/2}} \exp \left\{ \frac{1}{(t - \frac{r_i}{2})
  (t-r_i)}\right\} \, dt \; (0 \leqq r \leqq \frac{r_i}{2}) 
\end{align*}
and define $f_i$ by 
$$
f_i ( x_1, \ldots, x_n ) = g_i ( \sqrt{\sum^n_{j=1}  ( x_j -
  b_{ij})^2. }  
$$

Let $\varphi_1 (x) = \sum f_i (x)$, summed  for all $i$ for which $
\bar{S}_i $ meets $E$ and let $ \varphi_2 (x) = \sum f_i (x) $, summed
for all the remaining $i$. Then $ \varphi_1 + \varphi_2 $ is
positive in $ U $ and, if 
$$
\theta (x) = \varphi_1 (x) / ( \varphi_1 (x) + \varphi_2 (x) ), 
$$
$\theta$\pageoriginale is indefinitely differentiable in $U$, is zero
outside $H$ and is constant, equal to 1, in a neighbourhood of $E$.  

Let $ h :  \mathscr{E}^p (W) \rightarrow \mathscr{E}^p (W) $, for open
$ W \subset U $ be defined by $ h (\omega ) = \theta \cdot \Omega $. Then
$h$ is a homomorphism commuting with $ \rho_{_{YW}}$, $Y $ open in $ U$,  $Y
\subset W $. Hence $h$ induces a homomorphism $ h : \Omega^p (U)
\rightarrow \Omega^p (U) $ for which 
\begin{align*}
 h ( \omega_b ) &=  \omega_b \text{ if } b \in E, \\
  h ( \omega_b ) &=  0_b \text{ if } b \in \bar{V} - \bar{G} \subset U
  -H . 
\end{align*}
 ($\omega_b $ denotes the germ determined by $ \omega $ at $ b \in U$.) 

\begin{proposition}%proposition 14
There is an isomorphism
$$
\eta f^*_X : H^p ( \mathscr{E} (X) ) \rightarrow H^P  ( X, R ). 
$$
\end{proposition}

\begin{proof}
By the corollary to Proposition \ref{chap16:prop12}, the exact sequence
$$
0 \rightarrow R \xrightarrow{e} \Omega^o \xrightarrow{d^1} \dots
\rightarrow \Omega^{p-1} \xrightarrow{d^p} \Omega^p \rightarrow \dots 
$$
is a resolution of the constant sheaf $R$ and hence, by Proposition
\ref{chap14:prop10}, there is an isomorphism 
$$
\eta : H^p \Gamma ( x, \Omega )  \rightarrow H^p ( X,R ); 
$$
 but we already have (as proved in the earlier part of this lecture)
 an isomorphism 
 $$
 f^*_X : H^p ( \mathscr{E} (X) ) \rightarrow H^p \Gamma ( X, \Omega
 ). 
 $$
\end{proof}

\noindent
$ \underline{(1) eh + k^0 d^1 = 1 } $.  \quad $ \underline{(2) d^p
  k^{p-1} + k^p d^{p+1} = 1 ~ ( p \geqq 1)} $.\pageoriginale 

\begin{enumerate}[(1)]
\item  If $ f (x) = f ( x_1, \ldots , x_n )  \in \mathscr{E}^0 (U ) $, 
$$  
eh f (x) = f( 0, \ldots , 0 ) \text{ and }  d^1 f(x) = \sum^n_{i =1}
D_i f (x)  dx_i,  \hfill{(*)},  
$$ 
 hence $ k^o d^1 f(x) = \sum^n_{i=1}  \int^1_0 D_i f (tx) dt \cdot  x_i =
 \int^1_0 \dfrac{d}{dt} f (tx) dt  = f (x) - f(0)$, thus $eh f (x) +
 k^o d^1 f(x) = f (x) $. 

\item If $ \omega = f ( x_1, \ldots , x_n ) dx_{i_1} \cdots dx_{i_{p}}$,
\end{enumerate}
\begin{align*}
 & d^p k^{p-1} = d^p ( \bigg\{ \int^1_0  f (tx) t^{p-1} dt \bigg\}
  \cdot \sum^p_{j=1} (-1)^{j-1} x_{i_{j}} dx_{i_{1}} \cdots d
  \hat{x}_{i_{j}} \cdots dx_{i_{p}}\\ 
 &  = \bigg\{ ( \sum^n_{i=1} ( \int^1_0 D_i f (tx) t^p dt ) ) \cdot
  \sum^p_{j=1} (-1)^{j-1} x_{i_{j}} dx_{i_{1}} \cdots d
  \hat{x}_{i_{j}} \dots dx_{i_{p}} \bigg\}\\ 
 & \qquad + ( \int^1_0 f (tx) t^{p-1} dt ) \cdot  p dx_{i_{1}} \cdots
  dx_{i_{p}}.\\ 
 & k^p d^{p+1} \omega = k^p ( \sum^n_{i=1} Di f (x)  dx_i
  dx_{i_{1}} \cdots dx_{i_{p}})\\  
& \sum^n_{i=1}  ( \int^i_0 D_i f (tx) t^p \cdot dt ) \bigg\{ x_i dx_{i_{1}}
  \cdots dx_{i_{p}}\\
&\qquad\quad{} - \sum^p_{j=1} (-1)^{j-1} x_{i_{j}} dx_i
  dx_{i_{1}} \cdots d \hat{x}_{i_{j}} \cdots dx_{i_{p}} \bigg\}. 
\end{align*}
$ (*) \; D_i $ denotes partial derivation with respect to the $i$ - th
variable concerned.  

Thus\pageoriginale $ d^p k^{p-1} \omega + k^p d^{p+1} \omega $ 
\begin{align*}
& = ( \int^1_0 f(tx) p \cdot t^{p-1} dt ) dx_{i_{1}} \dots dx_{i_{p}}\\
& {} +  \sum^n_{i=1}  \int^1_0 D_i f (tx) t^p \cdot dt \cdot x_i dx_{i_{1}} \dots
  dx_{i_{p}}\\ 
& = \bigg\{ f (tx) t^p  \bigg]^1_0 - \int^1_0 \sum^{n}_{i = 1} D_i f
    (tx)\cdot x_i t^p dt \bigg\} dx_{i_{1}} \dots dx_{i_{p}}\\ 
& \qquad + \sum^n_{i=1} \int^1_0  D_i f (tx)\cdot t^p dt \cdot  x_i
    dx_{i_{1}} \dots     dx_{i_{p}} \\ 
&=  f(x) dx_{i_{1}} \dots dx_{i_{p}} \\
&= \omega .
 \end{align*} 
 

