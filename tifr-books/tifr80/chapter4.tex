
\chapter{Applications}\label{chap4}

THE\pageoriginale THEOREMS OF the preceding chapter show that the
short exponential sums in quesion depend on the rational approximation
of $f'(n)$ in the interval of summation. But in long sums the value of
$f'(n)$ may vary too much to be approximated accurately by a single
rational number, and therefore it is necessary to split up the sum
into shorter segments such that in each segment $f'(n)$ lies near to a
certain fraction $r$. By suitable averaging arguments, it is possible
to add these short sums - in a transformed shape - in a non-trivial
way. Variations on this theme are given in \S\S~ \ref{chap4:sec4.2} -
\ref{chap4:sec4.4}. But as a preliminary for \S\S ~ \ref{chap4:sec4.2}
and \ref{chap4:sec4.4}, we first work out in \S~ \ref{chap4:sec4.1}
the transformation formulae of Chapter \ref{chap3} in the special case
of Dirichlet polynomials related to $\zeta^2(s)$ and $\varphi(s)$.

\section{Transformation Formulae for Dirichlet
  Polynomials}\label{chap4:sec4.1} 

The general theorems of the preceding chapter are now applied to\break
Dirichlet polynomials
\begin{align}
S\left(M_1,M_2\right) &= \sum\limits_{M_1\leq m\leq M_2}d(m)
m^{-1/2-it},\label{chap4:eq4.1.1}\\
S_{\varphi}\left(M_1,M_2\right) &= \sum\limits_{M_1\leq m\leq m_2}
a(m)m^{-k /2-it},\label{chap4:eq4.1.2}
\end{align}
as well as to their smoothed variants
\begin{align}
\tilde{S}\left(M_1,M_2\right) &= \sum\limits_{M_1\leq m\leq M_2}
\eta_J(m)d(m)m^{-1/2-it},\label{chap4:eq4.1.3}\\
\tilde{S}_\varphi\left(M_1,M_2\right) &= \sum\limits_{M_1\leq m\leq
M_2} \eta_J(m)a(m)m^{-k /2-it},\label{chap4:eq4.1.4}
\end{align}
where\pageoriginale $\eta_J(x)$ is a weight function defined in
\eqref{chap2:eq2.1.2}.

We shall suppose for simplicity that $t$ is a sufficiently large
positive number, and put $L=\log t$. The function $\chi(s)$ is as in
the functional equation $\zeta(s)=\chi(s)\zeta(1-s)$, thus 
$$
\chi(s)=s^s\pi^{s-1}\sin\left(\frac{1}{2}s\pi\right)\Gamma(1-s).
$$

If $\sigma$ is bounded and $t$ tends to infinity, then (see
\cite{key27}, p.~68)
\begin{equation}\label{chap4:eq4.1.5}
\chi(s) = \left(2\pi/t\right)^{s-1/2}e^{i(t+\pi/4)}\left(1+o\left(
t^{-1}\right)\right).
\end{equation}

Define also
\begin{equation}\label{chap4:eq4.1.6}
\phi(x)=ar\sin h \left(x^{1/2}\right)+\left(x+x^2\right)^{1/2}.
\end{equation}

As before, $\delta_1,\delta_2,\ldots$ will denote positive constants
which may be supposed to be arbitrarily small.

\begin{thm}\label{chap4:thm4.1}
Let $r=h/k$ be a rational number such that 
\begin{gather}
M_1<\frac{t}{2\pi r}<M_2\label{chap4:eq4.1.7}\\
\intertext{and}
1\leq k\ll M_1^{1/2-\delta_1}.\label{chap4:eq4.1.8}
\end{gather}

Write
\begin{gather}
M_j=\frac{t}{2\pi r}+(-1)^jm_j,\label{chap4:eq4.1.9}\\
\intertext{and suppose also that $m_1\asymp m_2$ and}
t^{\delta_2}\max\left(t^{1/2}r^{-1},hk\right)\ll m_1\ll
M_1^{1-\delta_3}.\label{chap4:eq4.1.10} 
\end{gather}

Let
\begin{equation}\label{chap4:eq4.1.11}
n_j=h^2m_j^2M_j^{-1}.
\end{equation}

Then\pageoriginale
\begin{align}
S(M_1,M_2) &= \left\{(hk)^{-1/2}\left(\log\left(t/2\pi\right)+2\gamma
-\log(hk)\right)\right.\label{chap4:eq4.1.12}\\
&\quad +\pi^{1/4}(2hkt)^{-1/4}\sum\limits_{j=1}^2\sum\limits_{n<n_j}
d(n)e\left(n\left(\frac{\bar{h}}{k}-\frac{1}{2hk}\right)\right)
n^{-1/4}\times\notag\\
&\quad\times\left.\left(1+\frac{\pi n}{2hkt}\right)^{-1/4}\exp\left(i
(-1)^{j-1}\left(2t\phi\left(\frac{\pi n}{2hkt}\right)+\frac{\pi}{4}
\right)\right)\right\}r^{it}\notag\\
& ~\chi\left(\frac{1}{2}+it\right)
+o\left(h^{-3/2}k^{1/2}m_1^{-1}t^{1/2}L\right)+o\left(
hm_1^{1/2}t^{-1/2}L^2\right)\notag\\
& \hspace{3cm} +o\left(h^{-1/4}k^{3/4}m_1^{-1/4}L \right).\notag 
\end{align}
\end{thm}

\begin{proof}
We apply Theorem \ref{chap3:thm3.1} with $-r$ in place of $r$, and
with 
\begin{gather}
f(z)=-(t/2\pi)\log z,\label{chap4:eq4.1.13}\\
\intertext{and}
g(z)=z^{-1/2}.\label{chap4:eq4.1.14}
\end{gather}

Then the assumptions of the theorem are obviously satisfied with 
\begin{align}
F&=t,\label{chap4:eq4.1.15}\\
G&=M_1^{-1/2}\asymp r^{1/2}t^{-1/2},\label{chap4:eq4.1.16}
\end{align}
and 
\begin{equation}\label{chap4:eq4.1.17}
M(-r)=\frac{t}{2\pi r}.
\end{equation}

Then the number $n_j$ in \eqref{chap3:eq3.1.10} equals the one in
\eqref{chap4:eq4.1.11}.

The leading term on the right of \eqref{chap3:eq3.1.11} is
$$
(hk)^{-1/2}(\log(t/2\pi)+2\gamma-\log(hk))r^{it}(2\pi/t)^{it}
e^{i(t+\pi/4)}
$$
which can also be written, by \eqref{chap4:eq4.1.5}, as 
\begin{equation}\label{chap4:eq4.1.18}
(hk)^{-1/2}(\log(t/2\pi)+2\gamma-\log(hk))r^{it}\chi\left(\frac{1}{2}
+it\right)\; \left(1+o\left(t^{-1}\right)\right).
\end{equation}

The\pageoriginale function $p_{j,n}(x)$ reads in the present case
\begin{equation}\label{chap4:eq4.1.19}
-(t/2\pi)\log x+rx+(-1)^{j-1}\left(2\sqrt{nx}/k-1/8\right)
\end{equation}
and the numbers $x_{j,n}$ are roots of the equation
\begin{equation}\label{chap4:eq4.1.20}
p'_{j,n}=-\frac{t}{2\pi x}+r+(-1)^{j-1}n^{1/2}x^{-1/2}k^{-1}=0,
\end{equation}
and thus roots of the quadratic equation
\begin{equation}\label{chap4:eq4.1.21}
x^2-\left(\frac{t}{\pi r}+\frac{n}{h^2}\right)x+\left(\frac{t} {2\pi
r}\right)^2=0.
\end{equation}

Moreover, since $x_{1,n}<x_{2,n}$, we have 
\begin{gather}
x_{j,n}=\frac{t}{2\pi r}+\frac{n}{2h^2}+\frac{(-1)^j}{h^2}\left(
\frac{n^2}{4}+\frac{hknt}{2\pi}\right)^{1/2}\label{chap4:eq4.1.22}\\
\intertext{and}
(t/2\pi r)^2x_{j,n}^{-1}=\frac{t}{2\pi r}+\frac{n}{2h^2}-\frac{(-1)^j}
{h^2}\left(\frac{n^2}{4}+\frac{hknt}{2\pi}
\right)^{1/2}.\label{chap4:eq4.1.23}
\end{gather}

Next we show that 
\begin{equation}\label{chap4:eq4.1.24}
2^{-1/2}k^{-1/2}x_{j,n}^{-3/4}p''_{j,n}\left(x_{j,n}\right)^{-1/2}=
\pi^{1/4}(2hkt)^{-1/4}\left(1+\frac{\pi n}{2hkt}\right)^{-1/4}.
\end{equation}

Indeed, by \eqref{chap4:eq4.1.19} we have 
$$
2kx_{j,n}^{3/2}p''_{j,n}\left(x_{j,n}\right)=\pi^{-1}ktx_{j,n}^{-1/2}
+(-1)^jn^{1/2},
$$
which by \eqref{chap4:eq4.1.20} and \eqref{chap4:eq4.1.23} is further
equal to 
\begin{gather*}
(-1)^{j-1}h^2n^{-1/2}\left(2\left(\frac{t}{2\pi r}\right)^2
x_{j,n}^{-1}-\frac{t}{\pi r}\right)+(-1)^jn^{1/2}\\
=\pi^{-1/2}(2hkt)^{1/2}\left(1+\frac{\pi n}{2hkt}\right)^{1/2}.
\end{gather*}

This proves \eqref{chap4:eq4.1.24}.

To\pageoriginale complete the calculation of the explicit terms in
\eqref{chap3:eq3.1.11}, we still have to work out
$p_{j,n}(x_{j,n})$. Note that by \eqref{chap4:eq4.1.22} and
\eqref{chap4:eq4.1.23} 
\begin{align*}
\left(2\pi rt^{-1}x_{j,n}\right)(-1)^j &= 1+\frac{\pi n}{hkt}+
\left(\left(\frac{\pi n}{hkt}\right)^2+\frac{2\pi n}{hkt}
\right)^{1/2}\\
&= \left(\left(\frac{\pi n}{2hkt}\right)^{1/2}+\left(1+\frac{\pi n}
{2hkt}\right)^{1/2}\right)^2,
\end{align*}
whence
\begin{equation}\label{chap4:eq4.1.25}
\log\left(2\pi rt^{-1}x_{j,n}\right)=(-1)^j 2 ar\sin h \left(\left(
\frac{\pi n}{2hkt}\right)^{1/2}\right).
\end{equation}

Also, by \eqref{chap4:eq4.1.20} and \eqref{chap4:eq4.1.22},
\begin{align*}
&2\pi rx_{j,n}+4\pi(-1)^{j-1}n^{1/2}x_{j,n}^{1/2}k^{-1}\\
&= 2t-2\pi rx_{j,n}\\
&= t-\frac{\pi n}{hk}+(-1)^{j-1}2t\left(\frac{\pi n}{2hkt}+\left(
\frac{\pi n}{2hkt}\right)^2\right)^{1/2}
\end{align*}
Together with \eqref{chap4:eq4.1.19}, \eqref{chap4:eq4.1.25}, and
\eqref{chap4:eq4.1.6}, this gives
$$
2\pi p_{j,n}\left(x_{j,n}\right)=(-1)^{j-1}\left(2t\phi\left(
\frac{\pi n}{2hkt}\right)-\frac{\pi}{4}\right)-t\log (t(2\pi)+t\log
r+t-\frac{\pi n}{hk}.
$$

Hence, using \eqref{chap4:eq4.1.5} again, we have
\begin{align}
&i(-1)^{j-1}e\left(p_{j,n}\left(x_{j,n}\right)+1/8\right)
=\label{chap4:eq4.1.26}\\
&\qquad =e\left(-\frac{n}{2hk}\right)\exp\left(i(-1)^{j-1}\left(2t\phi \left(
\frac{\pi n}{2hkt}\right)+\frac{\pi}{4}\right)\right)r^{it}\notag\\
& \qquad \chi\left(\frac{1}{2}+it\right)\;\left(1+0\left(t^{-1}\right)
\right).\notag 
\end{align}

By \eqref{chap4:eq4.1.18}, \eqref{chap4:eq4.1.24}, and
\eqref{chap4:eq4.1.26}, we find that the explicit terms on the right
of \eqref{chap3:eq3.1.11} coincide with those in
\eqref{chap4:eq4.1.12}, up to the factor $1+o(t^{-1})$. The correction
$o(t^{-1})$ can be omitted with an error 
$$
\ll t^{-1}\left((hk)^{-1/2}L+(hkt)^{-1/4}n_1^{3/4}L\right),
$$ 
which\pageoriginale is 
\begin{gather}
\ll t^{-1}\left((hk)^{-1/2}L+t^{-1}h^2k^{-1}m_1^{3/2}L\right)\notag\\
\ll hm_1^{1/2}t^{-1}L\label{chap4:eq4.1.27}
\end{gather}
by \eqref{chap4:eq4.1.11}, \eqref{chap4:eq4.1.7}, and
\eqref{chap4:eq4.1.10}. This is clearly negligible in
\eqref{chap4:eq4.1.12}. 

Finally, the error terms in \eqref{chap3:eq3.1.11} give those in
\eqref{chap4:eq4.1.12} by \eqref{chap4:eq4.1.15} and
\eqref{chap4:eq4.1.16}.

An application of Theorem \ref{chap3:thm3.2} yields an analogous
result for \break $S_\varphi(M_1,M_2)$. 
\end{proof}

\begin{thm}\label{chap4:thm4.2}
Suppose that the conditions of Theorem \ref{chap4:thm4.1} are
satisfied. Then 
\begin{align}
&S_\varphi(M_1,M_2) = \pi^{1/4}(2hkt)^{-1/4}\left\{\sum\limits_{j=1}^2
\sum\limits_{n<n_j}a(n)e\left(n\left(\frac{\bar{h}}{k}-\frac{1}{2hk}
\right)\right)\times\right.\label{chap4:eq4.1.28}\\
&\quad \times \left.n^{1/4-k /2}\left(1+\frac{\pi n}{2hkt}
\right)^{-1/4}\exp\left(i(-1)^{j-1}\left(2t\phi\left(\frac{\pi n}
{2hkt}\right)+\frac{\pi}{4}\right)\right)\right\}r^{it}\notag\\
& \hspace{2cm} \chi (1/2+it)
+0\left(hm_1^{1/2}t^{-1/2}L^2\right)+0\left(h^{-1/4}k^{3/4} 
m_1^{-1/4}L\right).\notag 
\end{align}
\end{thm}

Turnign to smoothed Dirichlet polynomials, we first state a
transformation formula for $\tilde{S}(M_1,M_2)$.

\begin{thm}\label{chap4:thm4.3}
Suppose that the conditions of Theorem \ref{chap4:thm4.1} are
satisfied. Let 
\begin{equation}\label{chap4:eq4.1.29}
U\gg r^{-1}t^{1/2+\delta_4}
\end{equation}
and let $J$ be a fixed positive integer exceeding a certain bound
(which depends on $\delta_4$). Write for $j=1,2$
$$
M'_j=M_j+(-1)^{j-1}JU=\frac{t}{2\pi r}+(-1)^jm'_j,
$$
and\pageoriginale suppose that $m_j\asymp m'_j$. Define
\begin{equation}\label{chap4:eq4.1.30}
n'_j=h^2(m'_j)^2\left(M'_j\right)^{-1}.
\end{equation}

Then, defining the weight function $\eta_J(x)$ in the interval $[M_1,M_2]$ with the aid of the parameters $U$ and  $J$, we have 
\begin{gather}
\tilde{S}\left(M_1,M_2\right)=\left\{(hk)^{-1/2}(\log(t/2\pi)+2\gamma -\log (hk))\right.\label{chap4:eq4.1.31}\\
+\pi^{1/4}(2hkt)^{-1/4}\sum\limits_{j=1}^2 \sum\limits_{n<n_j}w_j(n)d(n)e \left(n\left(\frac{\bar{h}}{k}-\frac{1}{2hk}\right)\right)n^{-1/4} \times\notag\\
\times \left.\left(1+\frac{\pi n}{2hkt}\right)^{-1/4}\exp\left(i(-1)^{j-1} \left(2t\phi\left(\frac{\pi n}{2hkt}\right)+\frac{\pi}{4}\right)\right) \right\}r^{it}\chi(1/2+it)\notag\\
+o\left(h^2k^{-1}m_1^{1/2}t^{-3/2}UL\right),\notag\\
\intertext{where}
w_j(n)=1\quad\text{for}\quad n<n'_j,\label{chap4:eq4.1.32}\\
w_j(n)\ll 1\quad\text{for}\quad n<n_j,\label{chap4:eq4.1.33}
\end{gather}
$w_j(y)$ and $w'_j(y)$ are piecewise continuous in the interval $(n'_j,n_j)$ with at most $J-1$ discontinuities, and 
\begin{equation}\label{chap4:eq4.1.34}
w'_j(y)\ll \left(n_j-n'_j\right)^{-1}\quad\text{for}\quad n'_j<y<n_j
\end{equation}
whenever $w'_j(y)$ exists
\end{thm}

\begin{proof}
We apply Theorem \ref{chap3:thm3.3} to the sum $\tilde{S}(M_1,M_2)$ with $f,g,F,G$ and $r$ as in the proof of Theorem \ref{chap4:thm4.1}; in particular, $F=t$. Hence the condition \eqref{chap3:eq3.2.1} on $U$ holds by \eqref{chap4:eq4.1.29}. The other assumptions of the theorem are readily verified, and the explicit terms in \eqref{chap4:eq4.1.31} were already calculated in the proof of Theorem \ref{chap4:thm4.1},up to the properties of the weight functions $w_j(y)$, which follow from \eqref{chap3:eq3.2.4} - \eqref{chap3:eq3.2.6}.

The error term in \eqref{chap3:eq3.2.3} gives that in
\eqref{chap4:eq4.1.31}. It should also be noted that as in the proof
of Theorem \ref{chap4:thm4.1} there is an extra error
term\pageoriginale caused by $o(t^{-1})$ in the formula
\eqref{chap4:eq4.1.5} for $\chi(1/2+it)$. This error term is $\ll
hm_1^{1/2}t^{-1}L$, as was seen in \eqref{chap4:eq4.1.27}. By
\eqref{chap4:eq4.1.29} this can be absorbed into the error term in
\eqref{chap4:eq4.1.31}, and the proof of the theorem is complete. 

The analogue of the preceding theorem for $\tilde{S}_\varphi(M_1,M_2)$ reads as follows.
\end{proof}

\begin{thm}\label{chap4:thm4.4}
With the assumptions of Theorem \ref{chap4:thm4.3}, we have 
\begin{align}
\tilde{S}_\varphi\left(M_1,M_2\right) &= \pi^{1/4}(2hkt)^{-1/4}\left\{ \sum\limits_{j=1}^2\sum\limits_{n<n_j}w_j(n)a(n) \times\right.\label{chap4:eq4.1.35}\\
&\quad \times e\left(n\left(\frac{\bar{h}}{k}-\frac{1}{2hk} \right)\right)n^{1/4-k /2}\left(1+\frac{\pi n}{2hkt}\right)^{-1/4} \times\notag\\
&\quad \times\left.\exp\left(i(-1)^{j-1}\left(2t\phi\left(\frac{\pi n} {2hkt}\right)+\frac{\pi}{4}\right)\right)\right\}r^{it}\chi (1/2+it)\notag\\
&\qquad +o\left(h^2k^{-1}m_1^{1/2}t^{-3/2}UL\right).\notag
\end{align}
\end{thm}

\begin{Remark}\label{chap4:rem1}
It is an easy corollary of Theorem \ref{chap4:thm4.1} that
\begin{equation}\label{chap4:eq4.1.36}
\left|\sum\limits_{x_1\leq n\leq x_2}d(n)n^{-1/2-it}\right|\ll\log
t\quad\text{for}\quad t\geq 2\quad\text{and}\quad\left|x_i-t/2\pi
\right|\ll t^{2/3}. 
\end{equation}
\end{Remark}

\begin{Remark}\label{chap4:rem2}
In Theorems \ref{chap4:thm4.3} and \ref{chap4:thm4.4} the error term
is minimal when $U$ is as small as possible, \ie 
\begin{equation}\label{chap4:eq4.1.37}
  U\asymp r^{-1}t^{1/2+\epsilon}.
\end{equation}

The error term then becomes
\begin{equation}\label{chap4:eq4.1.38}
  0\left(hm_1^{1/2}t^{-1+\epsilon}\right).
\end{equation}

This is significantly smaller than the error terms in Theorems
\ref{chap4:thm4.1} and \ref{chap4:thm4.2}. for example, if $r=1$ and
$m_1=t^{3/4}$, then the error in Theorem \ref{chap4:thm4.1} is $\ll
t^{-1/8}L^2$, while \eqref{chap4:eq4.1.38} is just $\ll
t^{-5/8+\epsilon}$. The lengths $n_j$ of the transformed sums are
about $t^{1/2}$, which is smaller than the length $\asymp t^{3/4}$
of\pageoriginale the original sum. A trivial estimate of the right
hand side of \eqref{chap4:eq4.1.12} is $\ll t^{1/8}L$, which is a
trivial estimate of the original sum is $\ll t^{1/4}L$.  

Thanks to good error terms, Theorems \ref{chap4:thm4.3} and \ref{chap4:thm4.4} are useful when a number of sums are dealt with and there is a danger of the accumulation of error terms.
\end{Remark}

\begin{Remark}\label{chap4:rem3}
With suitable modifications, the theorems of this section hold for
negative values of $t$ as well. In this case $r$ (and thus also $h$)
will be negative. Because $p''_{j,n}$ is now negative, our saddle
point theorems take a slightly different form (see the remark in the
end of \S~ \ref{chap2:sec2.1}. When the calculations in the proof of
Theorem \ref{chap4:thm4.1} are carried out, then instead of
\eqref{chap4:eq4.1.12} we obtain, for $t<0$, 
\begin{align*}
S\left(M_1,M_2\right) &=
\left\{(|h|k)^{-1/2}(\log|t|/2\pi)+2\gamma-\log (|h|k)\right.\\ 
& +\pi^{1/4}(2hkt)^{-1/4}\sum\limits_{j=1}^2\sum\limits_{n<n_j}d(n)
e\left(n\left(\frac{\bar{h}}{k}-\frac{1}{2hk}\right)\right)n^{-1/4}\times\\ 
& \times \left.\left(1+\frac{\pi
  n}{2hkt}\right)^{-1/4}\exp\left(i(-1)^{j-1}
\left(2t\phi\left(\frac{\pi
  n}{2hkt}\right)-\frac{\pi}{4}\right)\right)\right\}|r|^{it}\\
&\chi\left(
\frac{1}{2}+it\right)+o\left(|h|^{-3/2}k^{1/2}m_1^{-1}|t|^{1/2}L\right)
+ o\left(|h|m_1^{1/2}|t|^{-1/2}L^2\right)\\
& \hspace{3cm}+o\left(|h|^{-1/4}k^{3/4}m_1^{-1/4}L\right),
\end{align*}
and similar modifications have to be made in the other theorems. Of
course, this formula can also be deduced from \eqref{chap4:eq4.1.12}
simply by complex conjugation.  
\end{Remark}

\section{On the Order of $\varphi(k /2+it)$}\label{chap4:sec4.2}

Dirichlet series are usually estimated by making use of their
approximate functional equations. For $\zeta^2(s)$, this result is
classical-due to Hardy-Littlewood and Titchmarsh - and states that for
$0\leq\sigma\leq 1$ and $t\geq 10$ 
\begin{equation}\label{chap4:eq4.2.1}
\zeta^2(s)=\sum\limits_{n\leq x}d(n)n^{-s}+\chi^2(s)\sum\limits_{n\leq
  y}d(n) n^{s-1}+o\left(x^{1/2-\sigma}\log t\right), 
\end{equation}\pageoriginale
where $x\geq 1, y\geq 1$, and $xy=(t/2\pi)^2$. Analogously, for
$\varphi(s)$ we have  
\begin{gather}
\varphi(s)=\sum\limits_{n\leq x}a(n)n^{-s}+\psi(s)\sum\limits_{n\leq
  y}a(n) n^{s-k }+o\left(x^{k /2-\sigma}\log
t\right),\label{chap4:eq4.2.2}\\ 
\intertext{where}
\psi(s)=(-1)^{k /2}(2\pi)^{2s-k }\Gamma(k -s)/\Gamma(s).\notag
\end{gather}

For proofs of \eqref{chap4:eq4.2.1} and \eqref{chap4:eq4.2.2}, see \eg
\cite{key19}. 

The problem of the order of $\varphi(k /2+it)$ can thus be reduced to
estimating sums 
\begin{equation}\label{chap4:eq4.2.3}
\sum\limits_{n\leq x}a(n)n^{-k /2-it}
\end{equation}
for $x\ll t$; we take here $t$ positive, for the case when $t$ is negative is much the same, as was seen in Remark \ref{chap4:rem3} in the preceding section. 

Estimating the sum \eqref{chap4:eq4.2.3} by absolute values, we obtain
$$
\left|\varphi(k /2+it)\right|\ll t^{1/2}L,
$$
which might be called a ``trivial'' estimate. If there is a certain amount of cancellation in this sum, then one has 
\begin{equation}\label{chap4:eq4.2.4}
\left|\varphi(k /2+it)\right|\ll t^\alpha,
\end{equation}
where $\alpha < 1/2$. An analogous problem is estimating the order of $\zeta(1/2+it)$, and in virtue of the analogy between $\zeta^2(1/2+it)$ and $\varphi(k /2+it)$, one would expect that if 
\begin{equation}\label{chap4:eq4.2.5}
\left|\zeta(1/2+it)\right|\ll t^c,
\end{equation}
then \eqref{chap4:eq4.2.4} holds with $\alpha =2c$. (Recently it has
been shown by E. Bombieri and H. Iwaniec that $c=9/56+epsilon$ is
admissible). In particular, as an analogue\pageoriginale of the
Lindel\"of hypothesis for the zeta-function, one may conjecture that
\eqref{chap4:eq4.2.4} holds for all positive $\alpha$. A counterpart of
the classical exponent $c=1/6$ would be $\alpha=1/3$, for which
\eqref{chap4:eq4.2.4} is indeed known to hold, up to an unimportant
logarithmic factor. More precisely, Good \cite{key9} proved that 
\begin{equation}\label{chap4:eq4.2.6}
\left|\varphi(k /2+it)\right|\ll t^{1/3}(\log t)^{5/6}
\end{equation}
as a corollary of his mean value theorem \eqref{int:eq0.11}. The proof
of the latter, being based on the spectral theory of the hyperbolic
Laplacian, is sophisticated and highly non-elementary. 

A more elementary approach to $\varphi(k /2+it)$ via the transformation formulae of the preceding section leads rather easily to an estimate which is essentially the same as \eqref{chap4:eq4.2.6}.

\begin{thm}\label{chap4:thm4.5}
We have 
\begin{equation}\label{chap4:eq4.2.7}
\left|\varphi(k /2+it)\right|\ll (|t|+1)^{1/3+\epsilon}.
\end{equation}
\end{thm}

\begin{proof}
We shall show that for all large positive values of $t$ and for all numbers $M,M'$ with $1\leq M<M'\leq t/2\pi$ and $M'\leq 2M$ we have 
\begin{equation}\label{chap4:eq4.2.8}
\left|S_\varphi(M,M')\right|=\left|\sum\limits_{M\leq m\leq M'}a(m) m^{-k /2-it}\right|\ll t^{1/3+\epsilon}.
\end{equation}

A similar estimate could be proved likewise for negative values of $t$, and the assertion \eqref{chap4:eq4.2.7} then follows from the approximate functional equation \eqref{chap4:eq4.2.2}.

Let $\delta$ be a fixed positive number, which may be chosen arbitrarily small. for $M\leq t^{2/3+\delta}$ the inequality \eqref{chap4:eq4.2.8} is easily verified on estimating the sum by absolute values.

Let\pageoriginale now
\begin{gather}
M_\circ = t^{2/3+\delta},\notag\\
M_\circ < M < M' \leq t/2\pi,\label{chap4:eq4.2.9}\\
K= (M/M_\circ)^{1/2},\label{chap4:eq4.2.10}
\end{gather}
and consider the increasing sequence of reduced fractions $r=h/k$ with $1\leq k\leq K$, in other words the Farey sequence of order $K$.

The \emph{mediant} of two consecutive fractions $r=h/k$ and $r'=h'/k'$ is 
$$
\rho=\frac{h+h'}{k+k'}
$$
The basic well-known properties of the mediant are :$r<\rho <r'$, and 
\begin{equation}\label{chap4:eq4.2.11}
\rho-r=\frac{1}{k(k+k')}\asymp\frac{1}{kK},r'-\rho= \frac{1}{k'(k+k')} \asymp\frac{1}{k'K}.
\end{equation}

Subdivide now the interval $[M,M']$ by all the points 
\begin{equation}\label{chap4:eq4.2.12}
M(\rho)=\frac{t}{2\pi\rho}
\end{equation}
lying in this interval; here $\rho$ runs over the mediants. Then the sum $S_\varphi(M,M')$ is accordingly split up into segments, the first and last one of which may be incomplete. Thus, the sum $S_\varphi(M,M')$ now becomes a sum of subsums of the type
\begin{equation}\label{chap4:eq4.2.13}
S_\varphi(M(\rho'),M(\rho)),
\end{equation}
up to perhaps one or two incomplete sums. This sum is related to that fraction $r=h/k$ of our system which lies between $\rho$ and $\rho'$. We are going to apply Theorem \ref{chap4:thm4.2} to the sum \eqref{chap4:eq4.2.13}. The numbers $m_1$ and $m_2$ in the theorem are now $M(r)-M(\rho')$ and $M(\rho)-M(r)$. Hence $m_1\asymp m_2$ by\pageoriginale \eqref{chap4:eq4.2.12} and \eqref{chap4:eq4.2.11}, which imply moreover that 
\begin{equation}\label{chap4:eq4.2.14}
m_j\asymp tr^{-2}(r-\rho)\asymp k^{-1}K^{-1}M^2 t^{-1}\asymp k^{-1}M^{3/2} t^{-2/3+\delta/2}.
\end{equation}

This gives further
\begin{equation}\label{chap4:eq4.2.15}
Mt^{-1/3+\delta}\ll m_j\ll Mt^{-1/6+\delta/2}.
\end{equation}

It follows that the incomplete sums contribute 
$$
\ll M^{1/2}t^{-1/6+\delta}\ll t^{1/3+\delta},
$$
which can be omitted.

Next we check the conditions of Theorem \ref{chap4:thm4.2}, \ie the conditions \eqref{chap4:eq4.1.8} and \eqref{chap4:eq4.1.10} of Theorem \ref{chap4:thm4.1}. The validity of \eqref{chap4:eq4.1.8} is clear by \eqref{chap4:eq4.2.10} and \eqref{chap4:eq4.2.9}. In \eqref{chap4:eq4.1.10}, the upper bound for $m_1$ follows from \eqref{chap4:eq4.2.15}. As to the lower bound, note that $m_1\gg Mt^{-1/2+\delta}\asymp t^{1/2+\delta}r^{-1}$ by \eqref{chap4:eq4.2.15}, and that 
$$
hk=rk^2\ll \left(M^{-1}t\right)\;\left(MM_\circ^{-1}\right)\asymp t^{1/3-\delta} \ll m_j t^{-3\delta}.
$$

The error terms in \eqref{chap4:eq4.1.28} can be estimated by \eqref{chap4:eq4.2.10}, \eqref{chap4:eq4.2.12}, and \eqref{chap4:eq4.2.14}. The first of them is $\ll L^2$, and the second is smaller. The number of subsums is 
$$
\asymp\left(tM^{-1}\right)K^2\asymp t^{1/3-\delta}.
$$

Hence the contribution of the error terms is $\ll t^{1/3}$.

Next we turn to the main terms in \eqref{chap4:eq4.1.28}. A useful observation will be that the numbers
$$
n_j\asymp m_j^2h^2M^{-1}
$$
are of the same order for all relevant $r$, namely 
\begin{equation}\label{chap4:eq4.2.16}
n_j\asymp t^{2/3+\delta}.
\end{equation}\pageoriginale

This is easily seen by \eqref{chap4:eq4.2.14}.

To simplify the expression in \eqref{chap4:eq4.1.28}, we omit the factors
$$
\left(1+\frac{\pi n}{2hkt}\right)^{-1/4}=1+o\left(k^{-2}Mnt^{-2}\right),
$$
which can be done with a negligible error $\ll 1$.

We now add up the expression in \eqref{chap4:eq4.1.28} for different fractions $r$. Putting
$$
\tilde{a}(n)=a(n)n^{-(k -1)/2},
$$
we end up with the problem of estimating the multiple sum 
\begin{gather}
t^{-1/4}\Bigg|\sum\limits_{h,k}(hk)^{-1/4}(h/k)^{it}\sum\limits_{n<n_j}
\tilde{a}(n)n^{-1/4}e\left(n\left(\frac{\bar{h}}{k}-\frac{1}{2hk}\right)
\right)\times\label{chap4:eq4.2.17}\\ 
\times \exp\left(i(-1)^{j-1}\left(2t\phi\left(\frac{\pi n}{2hkt}
\right)+\pi/4\right)\right)\Bigg|.\notag 
\end{gather}

As a matter of fact, the numbers $n_j$ depend also on $r=h/k$, but
this does not matter, for only the order of $n_j$ will be of
relevance. 

For convenience we restrict in \eqref{chap4:eq4.2.17} the summations
to the intervals $K_\circ\leq k\leq K'_{\circ}, N_\circ\leq n\leq
N'_\circ$, where $K'_\circ\leq\min(2K_\circ,K)$, and $N'_\circ\leq
2N_\circ,N_\circ\ll t^{2/3+\delta}$. Also we take for $j$ one of its
two values, say $j=1$. The whole sum is then a sum of $o(t^\delta)$
such sums. 

It may happen that some of the n-sums are incomplete. In order to have formally complete sums, we replace $\tilde{a}(n)$ by $\tilde{a}(n)\delta (h,k;n)$, where 
\begin{equation*}
\delta(h,k;n)=
\begin{cases}
1\quad\text{for}\quad n<n_1(h,k),\\
0\quad\text{otherwise};
\end{cases}
\end{equation*}
the dependence of $n_1$ on $h/k$ is here indicated by the notation $n_1(h,k)$. Then,\pageoriginale changing in \eqref{chap4:eq4.2.17} the order of the summations with respect to $n$ and the pairs $h,k$, followed by applications of Cauchy's inequality and Rankin's estimate \eqref{chap1:eq1.2.4}, we obtain
\begin{multline*}
\ll t^{-1/4}N_\circ^{1/4}\left\{\sum\limits_n\left|\sum\limits_{h,k}\delta (h,k;n)\;(hk)^{-1/4}(h/k)^{it}\times\right.\right.\\
\times \left.\left. e\left(n\left(\frac{\bar{h}}{k}-\frac{1}{2hk}\right)+ (1/\pi)t\phi\left(\frac{\pi n}{2hkt}\right)\right)\right|^2\right\}^{1/2}.
\end{multline*}

Here the square is written out as a double sum with respect to $h_1,k_1$, and $h_2,k_2$, and the order of the summations is inverted. Then, since $N_\circ\ll t^{2/3+\delta}$ and 
\begin{equation}\label{chap4:eq4.2.18}
hk\asymp K_\circ^2M^{-1}t,
\end{equation}
the preceding expression is 
{\fontsize{10}{12}\selectfont
\begin{gather}
 \ll K_\circ^{-1/2}M^{1/4}t^{-1/3+\delta}\left(\sum\limits_{h_1,k_1} \sum\limits_{h_2,k_2}\left|s\left(h_1,k_1;h_2,k_2\right)\right| \right)^{1/2},\label{chap4:eq4.2.19}\\
\intertext{where}
 s\left(h_1,k_1;h_2,k_2\right)=\sum\limits_{N_\circ\leq n\leq N'_\circ}\delta\left( h_1,k_1;n\right)\delta\left(h_2,k_2;n\right) e(f(n))\label{chap4:eq4.2.20}\\
\intertext{with}
 f(x)=x\left(\frac{\bar{h}_1}{k_1}-\frac{1}{2h_1k_1}-\frac{\bar{h}_2}{k_2}+ \frac{1}{2h_2k_2}\right)+(t/\pi)\;\left(\phi\left(\frac{\pi x}{2h_1k_1t} \right)-\phi\left(\frac{\pi x}{2h_2k_2t}\right) \right).\label{chap4:eq4.2.21}
\end{gather}}

Thus $s(h_1,k_1;h_2,k_2)$ is a sum over a subinterval of $[N_\circ,N'_\circ]$. It will be estimated trivially for quadruples $(h_1,k_1,h_2,k_2)$ such that $h_1k_1=h_2k_2$, and otherwise by van der Corput's method, applying the following well-known lemma (see \cite{key27}, Theorem 5.9).
\end{proof}

\begin{lem}\label{chap4:lem4.1}
Let $f$ be a twice differentiable function such that 
$$
0<\lambda_2\leq f''(x)\leq h\lambda_2\quad\text{or}\quad \lambda_2\leq -f''(x)\leq h\lambda_2
$$
throughout\pageoriginale the interval $(a,b)$, and $b\geq a+1$. Then
$$
\sum\limits_{a<n\leq b+1}e(f(n))\ll h(b-a)\lambda_2^{1/2}+\lambda_2^{-1/2}.
$$
\end{lem}

Now, by \eqref{chap4:eq4.1.6},
$$
\phi''(x)=-\frac{1}{2}x^{-3/2}(1+x)^{-1/2},
$$
so that for our function $f$ in \eqref{chap4:eq4.2.21}
\begin{gather*}
f''(x)=-2^{-3/2}\pi^{-1/2}t^{1/2}x^{-3/2}\left(\left(h_1k_1\right)^{-1/2}\left( 1+\frac{\pi x}{2h_1k_1t}\right)^{-1/2}\right.\\
-\left.\left(h_2k_2\right)^{-1/2}\left(1+\frac{\pi x}{2h_2k_2t}\right)^{-1/2}\right),\\
\intertext{and accordingly}
\lambda_2\asymp N_\circ^{-3/2}t^{1/2}\left|\left(h_1k_1\right)^{-1/2}-\left( h_2k_2\right)^{-1/2}\right|\\
\asymp K_\circ^{-3}M^{3/2}N_\circ^{-3/2}t^{-1}\left|h_1k_1-h_2k_2\right|,
\end{gather*}
where we used \eqref{chap4:eq4.2.18}. Hence, by Lemma \ref{chap4:lem4.1},
\begin{align*}
s\left(h_1,k_1;h_2,k_2\right) &\ll K_\circ^{-3/2}M^{3/4}N_\circ^{1/4}t^{-1/2} \left|h_1k_1-h_2k_2\right|^{1/2}\\
&+ K_\circ^{3/2}M^{-3/4}N_\circ^{3/4}t^{1/2}\left|h_1k_1-h_2k_2\right|^{-1/2}
\end{align*}
if $h_1k_1\neq h_2k_2$. By \eqref{chap4:eq4.2.18}
\begin{align*}
\sum\limits_{h_1k_1\neq h_2k_2}\left|h_1k_1-h_2k_2\right|^{1/2} &\ll \left( K_\circ^2M^{-1}t\right)^{5/2}\\
\intertext{and}
\sum\limits_{h_1k_1\neq h_2k_2}\left|h_1k_1-h_2k_2\right|^{-1/2} &\ll \left(K_\circ^2M^{-1}t\right)^{3/2}t^\delta.
\end{align*}

Thus
\begin{align*}
\sum\limits_{h_1,k_1}\sum\limits_{h_2,k_2}\left|s\left(h_1,k_1;h_2,k_2\right) \right| &\ll K_\circ^{7/2}M^{-7/4}N_\circ^{1/4}t^2\\
&\quad +K_\circ^{9/2}M^{-9/4}N_\circ^{3/4}t^{2+\delta}+K_\circ^2M^{-1}N_\circ t\\
&\ll K_\circ^{7/2}M^{-7/4}t^{13/6+\delta}\\
&\quad +K_\circ^{9/2}M^{-9/4}t^{5/2+2\delta}+K_\circ^2M^{-1}t^{5/3+\delta}.
\end{align*}

Hence\pageoriginale the expression \eqref{chap4:eq4.2.19} is
\begin{align*}
& \ll K_\circ^{5/4}M^{-5/8}t^{3/4+2\delta}+K_\circ^{7/4}M^{-7/8}t^{11/12+2\delta}+ K_\circ^{1/2}M^{-1/4}t^{1/2+2\delta}\\
& \ll t^{1/3+2\delta},
\end{align*}
and the proof of the theorem is complete.
\begin{remark*}
The preceding proof works for $\zeta^2(s)$ as well, and gives
$$
\left|\zeta^2(1/2+it)\right|\ll t^{1/3+\epsilon}.
$$
\end{remark*}

This is, of course, a known result, but the argument of the proof is
new, though there is a van der Corput type estimate (Lemma
\ref{chap4:lem4.1}) as an element in common with the classical
method. 

\section{Estimation of ``Long'' Exponential Sums}\label{chap4:sec4.3} 

The method of the preceding section is now carried over to more
general exponential sums 
\begin{equation}\label{chap4:eq4.3.1}
\sum\limits_{M\leq m\leq M'}b(m)g(m)e(f(m)), b(m)=d(m)\quad\text{or}\quad a(m),
\end{equation}
which are ``long'' in the sense that the length of a sum may be of the order of $M$. ``Short'' sums of this type were transformed in Chapter \ref{chap3} under rather general conditions. Thus the first steps of the proof of Theorem \ref{chap4:thm4.5} --dissection of the sum and transformation of the subsums-can be repeated in the more general context of sums \eqref{chap4:eq4.3.1} without any new assumptions, as compared with those in Chapter \ref{chap3}. But it turned out to be difficult to gain a similar saving in the summation of the transformed sums without more specific assumptions on the function $f$. However, if we suppose that $f'$ is approximately a power, the analogy with the previous case of Dirichlet polynomials will be perfect. The result\pageoriginale is as follows.

\begin{thm}\label{chap4:thm4.6}
Let $2\leq M<M'\leq 2M$, and let $f$ be a holomorphic function in the domain
\begin{equation}\label{chap4:eq4.3.2}
D=\left\{z\left| |z-|<cM\quad\text{for some}\quad x\in [M,M']\right.\right\},
\end{equation}
where $c$ is a positive constant. Suppose that $f(x)$ is real for $M\leq x\leq M'$, and that either
\begin{equation}\label{chap4:eq4.3.3}
f(z)=Bz^\alpha\left(1+0\left(F^{-1/3}\right)\right)\quad\text{for}\quad z\in D 
\end{equation}
where $\alpha\neq 0,1$ is a fixed real number, and
\begin{gather}
F=|B|M^\alpha,\label{chap4:eq4.3.4}\\
\intertext{or}
f(z)=B\log z\left(1+o\left(F^{-1/3}\right)\right)\quad\text{for}\quad
z\in D,\label{chap4:eq4.3.5} \\
\intertext{where}
F=|B|.\label{chap4:eq4.3.6}
\end{gather}

Let $g\in C^1[M,M']$, and suppose that for $M\leq x\leq M'$
\begin{equation}\label{chap4:eq4.3.7}
|g(x)|\ll G,|g'(x)|\ll G'.
\end{equation}

Suppose also that 
\begin{equation}\label{chap4:eq4.3.8}
M^{3/4}\ll F \ll M^{3/2}.
\end{equation}

Then
\begin{equation}\label{chap4:eq4.3.9}
\left|\sum\limits_{M\leq m\leq M'}b(m)g(m)e(f(m))\right|\ll
(G+MG')M^{1/2} F^{1/3+\epsilon}, 
\end{equation}
where $b(m)=d(m)$ or $\tilde{a}(m)$.
\end{thm}

\begin{proof}
We give the details of the proof for $b(m)=d(m)$ only; the other case is similar, even slightly simpler. 

It\pageoriginale suffices to prove the assertion for $g(x)=1$, because the general case can be reduced to this by partial summation.

The proof follows that of Theorem \ref{chap4:thm4.5}, which corresponds to the case $f(z)= -t\log z$. Then $F=t$, so that the condition \eqref{chap4:eq4.3.8} states $t^{2/3}\ll M\ll t^{4/3}$. We restricted ourselves to the case $t^{2/3}\ll M\ll t$, which sufficed for the proof of Theorem \ref{chap4:thm4.5}, but the method gives, in fact,
$$
\left|\sum\limits_{M\leq m\leq M'}\tilde{a}(m)m^{it}\right|\ll M^{1/2} t^{1/3+\epsilon}\quad\text{for}\quad t^{2/3}\ll M\ll t^{4/3}.
$$

This follows, by the way, also from the previous case by a ``reflection'', using the approximate functional equation \eqref{chap4:eq4.2.2}.

The analogy between the number $t$ in Theorem \ref{chap4:thm4.5} and the number $F$ in the present theorem will prevail throughout the proof. Accordingly, we put
$$
M_\circ=F^{2/3+\delta}
$$
and define, as in \eqref{chap4:eq4.2.10},
\begin{equation}\label{chap4:eq4.3.10}
K=\left(M/M_\circ\right)^{1/2}.
\end{equation}

We may suppoe that $M\geq M_\circ$, for otherwise the assertion to be proved, viz.
\begin{equation}\label{chap4:eq4.3.11}
\left|\sum\limits_{M\leq m\leq M'}d(m)e(f(m))\right|\ll M^{1/2}F^{1/3+\epsilon},
\end{equation}
is trivial.

Consider the case when $f$ is of the form \eqref{chap4:eq4.3.3}; the
case \eqref{chap4:eq4.3.5} is analogous and can be dealt with by
obvious modifications. We suppose\pageoriginale that $B$ is of a
suitable sign, so that $f''(x)$ is positive. 

The equation \eqref{chap4:eq4.3.3} can be formally differentiated once or twice to give correct results, for by Cauchy's integral formula we have 
\begin{gather}
f'(z)=\alpha Bz^{\alpha-1}\left(1+o\left(F^{-1/3}\right) \right)\label{chap4:eq4.3.12}\\
\intertext{and}
f''(z)=\alpha(\alpha-1)Bz^{\alpha-2}\left(1+o\left(F^{-1/3}\right) \right)\label{chap4:eq4.3.13}
\end{gather}
for $z$ lying in a region $D'$ of the type \eqref{chap4:eq4.3.2} with $c$ replaced by a smaller positive number. Hence, for $z\in D'$, 
\begin{gather}
\left|f'(z)\right|\asymp FM^{-1},\label{chap4:eq4.3.14}\\
\intertext{and}
\left|f''(z)\right|\asymp FM^{-2},\label{chap4:eq4.3.15}
\end{gather}
so that the parameter $F$ plays here the same role as in Chapters \ref{chap2} and \ref{chap3}.

The proof now proceeds as in the previous section. The set of the mediants $\rho$ is constructed for the sequence of fractions $r=h/k$ with $1\leq k\leq K$, and the interval $[M,M']$ is dissected by the points $M(\rho)$ such that 
\begin{equation}\label{chap4:eq4.3.16}
f'(M(\rho))=\rho.
\end{equation}

The numbers $m_1$ and $m_2$ then have formally the same expressions as before by \eqref{chap4:eq4.3.15} and \eqref{chap4:eq4.3.16} 
\begin{equation}\label{chap4:eq4.3.17}
m_j\asymp k^{-1}K^{-1}M^2F^{-1}\asymp k^{-1}M^{3/2}F^{-2/3+\delta/2}
\end{equation}
in\pageoriginale analogy with \eqref{chap4:eq4.2.14}. This implies that
\begin{equation}\label{chap4:eq4.3.18}
MF^{-1/3+\delta}\ll m_j\ll\min\left(MF^{-1/6+\delta/2}, M^{1/2}F^{1/3+\delta/2}\right); 
\end{equation}
note that $k\gg F^{-1}M$ for $M\gg F$ by \eqref{chap4:eq4.3.16} and \eqref{chap4:eq4.3.14}. The upper estimate in \eqref{chap4:eq4.3.18} shows that the possible incomplete sums in the dissection can be omitted.

The subsums are transformed by Theorem \ref{chap3:thm3.1}, the assumptions of which are readily verified as in the proof of Theorem \ref{chap4:thm4.5}.

Of the three error terms in \eqref{chap3:eq3.1.11}, the second one is $\ll M^{1/2}\break\log^2M$, and the others are smaller. Since the number of subsums is $\asymp F^{1/3-\delta}$, the contribution of these is $\ll M^{1/2} F^{1/3}$. 

The leading term in \eqref{chap3:eq3.1.11} is $\ll F^{-1/2}k^{-1}M\log F$. For a given $k,h$ takes $\ll FM^{-1}k$ values. Hence the contribution of the leading terms is 
$$
\ll F^{1/2}K\log F\ll M^{1/2}F^{1/6}.
$$

Consider now the sums of length $n_j$ in \eqref{chap3:eq3.1.11}. By
\eqref{chap3:eq3.1.16} and \eqref{chap4:eq4.3.17} we have
(cf. \eqref{chap4:eq4.2.16}) 
\begin{equation}\label{chap4:eq4.3.19}
n_j\asymp F^{2/3+\delta}.
\end{equation}

For convenience, we restrict the triple sum with respect to $j,n$, and $h/k$ by the conditions $j=1, N_\circ\leq n\leq N'_\circ$, and $K_\circ\leq k\leq K'_\circ$, where $K_\circ\asymp K'_\circ$ and $N_\circ\asymp N'_\circ$. Denote the saddle point $x_{1,n}$ by $x(r,n)$ in order to indicate its dependence on $r$. Then, for given $r$, the sum with respect to $n$ can be written as 
\begin{equation}\label{chap4:eq4.3.20}
\sum\limits_nQ(r,n)d(n)e\left(f_r(n)\right),
\end{equation}
where\pageoriginale
\begin{align}
Q(r,n) &= i2^{-1/2}k^{-1/2}n^{-1/4}x(r,n)^{-1/4}\left(f''(x (r,n))-\right.\label{chap4:eq4.3.21}\\
&\qquad \left.-\frac{1}{2}k^{-1}n^{1/2}x(r,n)^{-3/2} \right)^{-1/2}\notag\\
\intertext{and}
f_r(n) &= -n\bar{h}/k+f(x(r,n))-rx(r,n)+2k^{-1}n^{1/2}x (r,n)^{1/2}.\label{chap4:eq4.3.22}
\end{align}

The range of summation in \eqref{chap4:eq4.3.20} is either the whole interval $[N_\circ,N'_\circ]$, or a subinterval of it if $n_1\leq N'_\circ$. Since
$$
|Q(r,n)|\asymp F^{-1/2}K_\circ^{-1/2}M^{3/4}N_\circ^{-1/4},
$$
we may write
$$
Q(r,n)=F^{-1/2}K_\circ^{-1/2}M^{3/4}N_\circ^{-1/4}q(r,n),
$$
where $|q(r,n)|\asymp 1$. Then, using Cauchy's inequality as in the proof of Theorem \ref{chap4:thm4.5}, we obtain
\begin{align}
& \sum\limits_r\sum\limits_n Q(r,n)d(n)e\left(f_r(n) \right)\label{chap4:eq4.3.23}\\
& \ll F^{-1/2+\delta}K_\circ^{-1/2}M^{3/4}N_\circ^{1/4}\left(\sum\limits_{r_1,r_2}\left| s\left(r_1,r_2\right)\right|\right)^{1/2},\notag
\end{align}
where
$$
s\left(r_1,r_2\right)=\sum\limits_nq\left(r_1,n\right)q\left(r_2,n\right)e \left( f_{r_1}(n)-f_{r_2}(n)\right).
$$

The saddle point $x(r,n)$ is defined implicitly by the equation 
\begin{equation}\label{chap4:eq4.3.24}
f'(x(r,n))-r+k^{-1}n^{1/2}x(r,n)^{-1/2}=0.
\end{equation}

Therefore, by the implicit function theorem,
\begin{equation}\label{chap4:eq4.3.25}
\left|\frac{dx(r,n)}{dn}\right|\asymp \left(K_\circ^{-1}M^{-1/2}N_\circ^{-1/2} \right)\;\left(FM^{-2}\right)^{-1}\asymp F^{-1}K_\circ^{-1}M^{3/2}N_\circ^{-1/2}
\end{equation}

Then\pageoriginale it is easy to verify that 
$$
\left|\frac{dq(r,n)}{dn}\right|\ll N_\circ^{-1}F^{\delta/2};
$$
the assumption $M\ll F^{4/3}$ is needed here. Consequently, if 
\begin{equation}\label{chap4:eq4.3.26}
\left|\sum\limits_ne\left(f_{r_1}(n)-f_{r_2}(n)\right)\right|\leq\sigma \left(r_1, r_2\right)
\end{equation}
whenever $n$ runs over a subinterval of $[N_\circ, N'_\circ]$, then by partial summation 
$$
\left|s\left(r_1,r_2\right)\right|\ll\sigma\left(r_1,r_2\right)F^{\delta/2}.
$$

Thus, in order to prove that the left hand side of \eqref{chap4:eq4.3.23} is $\ll M^{1/2}F^{1/3+\circ(\delta)}$, it suffices to show that
\begin{equation}\label{chap4:eq4.3.27}
\sum\limits_{r_1,r_2}\sigma\left(r_1,r_2\right)\ll F^{5/3+o(\delta)}K_\circ M^{-1/2} N_\circ^{-1/2}.
\end{equation}

With an application of Lemma \ref{chap4:lem4.1} in mind, we derive bounds for $\frac{d^2}{dn^2}(f_{r_1}(n)-f_{r_2}(n))$, where $n$ is again understood for a moment as a continuous variable. First, by \eqref{chap4:eq4.3.22} and \eqref{chap4:eq4.3.24},
\begin{align*}
\frac{df_r(n)}{dn} &= -\bar{h}/k+\left(f'(x(r,n))-r+k^{-1}n^{1/2}x(r,n)^{-1/2} \right)\frac{dx(r,n)}{dn}\\
&\qquad +k^{-1}n^{-1/2}x(r,n)^{1/2}\\
&= -\bar{h}/k+k^{-1}n^{-1/2}x(r,n)^{1/2},
\end{align*}
and further
\begin{equation}\label{chap4:eq4.3.28}
\frac{d^2f_r(n)}{dn^2}=\frac{1}{2}k^{-1}n^{-1/2}x(r,n)^{-1/2}\frac{dx(r,n)}{dn}- \frac{1}{2}k^{-1}n^{-3/2}x(r,n)^{1/2}.
\end{equation}

Here the first term, which is 
\begin{equation}\label{chap4:eq4.3.29}
\ll F^{-1}K_\circ^{-2}MN_\circ^{-1}
\end{equation}
by \eqref{chap4:eq4.3.25}, will be less significant.

The saddle point $x(r,n)$ is now approximated by the point $M(r)$, which is easier to determine. By definition
$$
f'(M(r))=r;
$$
hence\pageoriginale by \eqref{chap4:eq4.3.12}
$$
\alpha BM(r)^{\alpha-1}=r\left(1+o\left(F^{-1/3}\right)\right),
$$
which gives further, by \eqref{chap4:eq4.3.4},
\begin{equation}\label{chap4:eq4.3.30}
M(r)=\left(\frac{M^\alpha}{|\alpha|F}\right)^{2\beta}|r|^{2\beta}\left(1+o\left(
F^{-1/3}\right)\right) 
\end{equation}
with
$$
\beta=\frac{1}{2(\alpha-1)}. 
$$

But the difference of $M(r)$ and $x(r,n)$ is at most the maximum  of $m_1$ and $m_2$, so that by \eqref{chap4:eq4.3.17}
\begin{align*}
x(r,n) &= M(r)+o\left(F^{-1}K_\circ^{-1}K^{-1}M^2\right)\\
&= M(r)\;\left(1+o\left(F^{-1}K_\circ^{-1}K^{-1}M\right)\right).
\end{align*}

Hence by \eqref{chap4:eq4.3.30}
$$
x(r,n)=\left(\frac{M^\alpha}{|\alpha|F}\right)^{2\beta}|r|^{2\beta}\left(1+o\left( F^{-1}K_\circ^{-1}K^{-1}M\right)\right);
$$
note that by \eqref{chap4:eq4.3.10}
$$
F^{-1}K_\circ^{-1}K^{-1}M\geq F^{-1}K^{-2}M=F^{-1/3+\delta}.
$$

So the second term in \eqref{chap4:eq4.3.28} is 
$$
-\frac{1}{2}\left(|\alpha|F\right)^{-\beta}k^{-1}M^{\alpha\beta}n^{-3/2}|r|^\beta+o
\left(F^{-1}K_\circ^{-2}K^{-1}M^{3/2}N_\circ^{-3/2}\right). 
$$

The expression \eqref{chap4:eq4.3.29} can be absorbed into the error term here, for $N_\circ\ll K^{-2}M$ by \eqref{chap4:eq4.3.10} and \eqref{chap4:eq4.3.19}. Hence \eqref{chap4:eq4.3.28} gives
\begin{align}
  &  \frac{d^2}{dn^2} \left(f_{r_1}(n)-f_{r_2} (n)
  \right) \label{chap4:eq4.3.31}\\  
  &= \frac{1}{2}\left(|\alpha|F\right)^{-\beta} M^{\alpha\beta}
  n^{-3/2} \left(|h_2|^\beta k_2^{-1-\beta}-|h_1 |^\beta
  k_1^{-1-\beta} \right)\notag\\
  & \hspace{3cm}+o \left(F^{-1}K_\circ^{-2} K^{-1}M^{3/2}
  N_\circ^{-3/2} \right).\notag
\end{align}

By\pageoriginale the following lemma, the differences $|h_2|^\beta k_2^{-1-\beta}- |h_1|^\beta k_1^{-1-\beta}$ are distributed as one would expect on statistical grounds.
\end{proof}

\begin{lem}\label{chap4:lem4.2}
Let $H\geq 1, K\geq 1$, and $0<\Delta\ll 1$. Let $\alpha$ and $\beta$ be non-zero real numbers. Then the number of quadruples $(h_1,k_1,h_2,k_2)$ such that 
\begin{gather}
H\leq h_i\leq 2H, \; K\leq k_i \leq 2K\label{chap4:eq4.3.32}\\
\intertext{and}
\left|h_1^\alpha k_1^\beta - h_2^\alpha k_2^\beta \right|\leq \Delta H^\alpha K^\beta\label{chap4:eq4.3.33}\\
\intertext{is at most}
\ll HK\log^2(2HK)+\Delta H^2K^2,\label{chap4:eq4.3.34}
\end{gather}
where the implied constants depend on $\alpha$ and $\beta$.
\end{lem}

We complete first the proof of Theorem \ref{chap4:thm4.6}, and that of Lemma \ref{chap4:lem4.2} will be given afterwards.

In our case, the number of pairs $(r_1,r_2)$ such that 
\begin{gather}
\left||h_2|^\beta k_2^{-1-\beta}-|h_1|^\beta k_1^{-1-\beta}\right|\leq\Delta K_\circ^{-1} \left(FM^{-1}\right)^\beta \label{chap4:eq4.3.35}\\
\intertext{is at most}
\ll FK_\circ^2M^{-1}\log^2F+\Delta F^2K_\circ^4M^{-2}\label{chap4:eq4.3.36}
\end{gather}
by Lemma \ref{chap4:lem4.2}. Let 
$$
\Delta_\circ=c_\circ F^{-1}K_\circ^{-1}K^{-1}M,
$$
where $c_\circ$ is a certain positive constant. For those pairs $(r_1,r_2)$ satisfying \eqref{chap4:eq4.3.35} with $\Delta=\Delta_\circ$ we estimate trivially $\sigma(r_1,r_2)\ll N_\circ$. Then, by \eqref{chap4:eq4.3.36}, their contribution to the sum in \eqref{chap4:eq4.3.27} is 
$$
\ll FK_\circ^2M^{-1}N_\circ\log^2F\ll F^{5/3+2\delta}K_\circ M^{-1/2}N_\circ^{-1/2}.
$$\pageoriginale

Let now $\Delta_\circ\leq\Delta\ll 1$, and consider those pairs
$(r_1,r_2)$ for which the expression on the left of
\eqref{chap4:eq4.3.35} lies in the interval $(\Delta
K_\circ^{-1}(FM^{-1})^\beta$, $2\Delta K_\circ^{-1}(FM^{-1})^\beta]$. If
  $c_\circ$ is chosen sufficiently large, then the main term (of order
  $\asymp \Delta K_\circ^{-1}M^{1/2}N_\circ^{-3/2}$) on the right of
  \eqref{chap4:eq4.3.31} dominates the error term. Then by Lemma
  \ref{chap4:lem4.1} 
$$
\sigma\left(r_1,r_2\right)\ll\Delta^{1/2}K_\circ^{-1/2}M^{1/4}N_\circ^{1/4}+ \Delta^{-1/2}K_\circ^{1/2}M^{-1/4}N_\circ^{3/4}.
$$

The number of the pairs $(r_1,r_2)$ in question is $\ll\Delta F^2K_\circ^3KM^{-2}\break \log^2F$ by \eqref{chap4:eq4.3.36} and our choice of $\Delta_\circ$, so that they contribute
\begin{align*}
&\ll \Delta^{3/2}F^{2+\delta}K_\circ^{5/2}KM^{-7/4}N_\circ^{1/4}+\Delta^{1/2}F^{2+\delta} K_\circ^{7/2}KM^{-9/4}N_\circ^{3/4}\\
&\ll F^{2+\delta}K_\circ M^{-1/2}N_\circ^{-1/2}\left(K^{5/2}M^{-5/4}N_\circ^{3/4}+K^{7/2} M^{-7/4}N_\circ^{5/4}\right)\\
&\ll F^{5/3+\delta}K_\circ M^{-1/2}N_\circ^{-1/2}.
\end{align*}

The assertion \eqref{chap4:eq4.3.27} is now verified, and the proof of Theorem \ref{chap4:thm4.6} is complete.

{\bf Proof of Lemma \ref{chap4:lem4.2}}. To begin with, we estimate the number of quadruples satisfying, besides \eqref{chap4:eq4.3.32} and \eqref{chap4:eq4.3.33}, also the conditions 
\begin{equation}\label{chap4:eq4.3.37}
\left(h_1,h_2\right)=\left(k_1,k_2\right)=1.
\end{equation}

By symmetry, we may suppose that $H\geq K$. The condition \eqref{chap4:eq4.3.33} can be written as
$$
h_1^\alpha k_1^\beta =h_2^\alpha k_2^\beta(1+o(\Delta)).
$$

Raising both sides to the power $\alpha^{-1}$ and dividing by $h_2k_1^{\beta/\alpha}$, we obtain
$$
\left|\frac{h_1}{h_2}-\left(\frac{k_2}{k_1}\right)^{\beta/\alpha}\right|\ll\Delta.
$$\pageoriginale
for given $k_1$ and $k_2$, the number of fractions $h_1/h_2$ satisfying this,\break\eqref{chap4:eq4.3.32}, and \eqref{chap4:eq4.3.37}, is $\ll 1+\Delta H^2$ by the theory of Farey fractions. Summation over the pairs $k_1,k_2$ in question gives
$$
\ll K^2+\Delta H^2K^2\ll HK+\Delta H^2K^2.
$$

Consider next quadruples satisfying \eqref{chap4:eq4.3.32} and \eqref{chap4:eq4.3.33} but instead of \eqref{chap4:eq4.3.37} the conditions
$$
\left(h_1,h_2\right)=h,\left(k_1,k_2\right)=k
$$
for certain fixed integers $h$ and $k$. Then, writing $h_1=hh'_i,
k_i=kk'_i$, we find that the quadruples $(h'_1,k'_1,h'_2,k'_2)$
satisfy the conditions \eqref{chap4:eq4.3.32}, \eqref{chap4:eq4.3.33},
and \eqref{chap4:eq4.3.37} with $H$ and $K$ replaced by $H/h$ and
$K/k$. Hence, as was just proved, the number of these quadruples is  
$$
\ll HK/hk+\Delta H^2K^2(hk)^{-2}.
$$

Finally, summation with respect to $h$ and $k$ gives \eqref{chap4:eq4.3.34}.

\begin{example*}
To illustrate the scope of Theorem \ref{chap4:thm4.6}, let us consider
the exponential sum 
\begin{equation}\label{chap4:eq4.3.38}
S=\sum\limits_{M\leq m\leq M'}b(m)e\left(\frac{X}{m}\right)
\end{equation}
where $b(m)$ is $d(m)$ or $\tilde{a}(m)$. By the theorem,
$$
S\ll M^{1/6}X^{1/3+\epsilon}\quad\text{for}\quad M^{7/4}\ll X \ll M^{5/2}.
$$

Thus, for $M\asymp\chi^{1/2}$, one has $S\ll M^{5/6+\epsilon}$. In the case $b(m)=d(m)$ it is also possible to interpret $S$ as the double sum
$$
\sum_{\substack{m,n\geq 1\\ M\leq mn\leq M'}}e\left(\frac{X}{mn}\right).
$$\pageoriginale
\end{example*}

This can be reduced to ordinary exponential sums, fixing first $m$ or
$n$, but it can be also estimated by more sophisticated methods in the
theory of multiple exponential sums. For instance, B.R. Srinivasan's
theory of n-dimensional exponent pairs gives, for $M\asymp X^{1/2}$
and $b(m)=d(m)$, 
\begin{equation}\label{chap4:eq4.3.39}
S\ll M^{1-\ell_1+\ell_\circ},
\end{equation}
where $(\ell_\circ,\ell_1)$ is a two-dimensional exponent pair (see
\cite{key13}, \S~2.4). Of the pairs mentioned in \cite{key13}, the
sharpest result is given by $(\frac{23}{250},\frac{56}{250})$, namely
\eqref{chap4:eq4.3.39} with the exponent $217/250=0.868$. The optimal
exponent given by this method is $0.86695\ldots$ (see
\cite{key10}). If a conjecture concerning one- and two-dimensional
exponent pairs (Conjecture P in \cite{key10}) is true, then the
exponent could be improved to $0.8290\ldots,$ which is smaller than
$5/6$. But in any case, for $b(m)=\tilde{a}(m)$ the sum $S$ seems to
be beyond the scope of ad hoc methods because of the complicated
structure of the coefficients $a(m)$.  

\section[The Twelth Moment of]{The Twelfth Moment of $\zeta(1/2+it)$
  and Sixth Moment of $\varphi(k /2+it)$}\label{chap4:sec4.4} 

In this last section, a unified approach to the mean value theorems
\ref{int:eq0.7} and \ref{int:eq0.9} will be given. 
\begin{thm}\label{chap4:thm4.7}
For $T\geq 2$ we have
\begin{align}
& \int\limits_\circ^T \left|\zeta(1/2+it)\right|^{12}\,dt\ll
  T^{2+\epsilon}\label{chap4:eq4.4.1}\\ 
\intertext{and}
& \int\limits_\circ^T \left|\varphi(k /2+it)\right|^6\,dt\ll
T^{2+\epsilon}.\label{chap4:eq4.4.2} 
\end{align}
\end{thm}

\begin{proof}
The\pageoriginale proofs of these estimates are much similar, so it
suffices to consider \eqref{chap4:eq4.4.2} as an example, with some
comments on \eqref{chap4:eq4.4.1}. 

It is enough to prove that 
\begin{equation}\label{chap4:eq4.4.3}
\int\limits_T^{2T}\left|\varphi(k /2+it)\right|^6\,dt\ll T^{2+\epsilon}.
\end{equation}

Actually we are going to prove a discrete variant of this, namely that 
\begin{equation}\label{chap4:eq4.4.4}
\sum\limits_\nu\left|\varphi(k /2+it_\nu)\right|^6 \ll T^{2+\epsilon}
\end{equation}
whenever $\{t_\nu\}$ is a ``well-spaced'' system of numbers such that 
\begin{equation}\label{chap4:eq4.4.5}
T\leq t_\nu \leq 2T, \left|t_\mu-t_\nu\right|\geq 1\quad\text{for}\quad\mu\neq\nu.
\end{equation}

Obviously this implies \eqref{chap4:eq4.4.3}. Again, \eqref{chap4:eq4.4.4} follows if it is proved that for any $V>0$ and for any system $\{t_\nu\},\nu=1, \ldots,R$, satisfying besides \eqref{chap4:eq4.4.5} also the condition
\begin{align}
& \left|\varphi(k /2+it_\nu)\right|\geq V,\label{chap4:eq4.4.6}\\
\intertext{one has}
& R\ll T^{2+\epsilon}V^{-6}.\label{chap4:eq4.4.7}
\end{align}

The last mentioned assertion is easily verified if 
\begin{equation}\label{chap4:eq4.4.8}
V\ll T^{1/4+\delta}
\end{equation}
where $\delta$ again stands for a positive constant, which may be chosen as small as we please, and which will be kept fixed during the proof. Indeed, one may apply the discrete mean square estimate
\begin{equation}\label{chap4:eq4.4.9}
\sum\limits_\nu\left|(k /2+it_\nu)\right|^2\ll T^{1+\delta}
\end{equation}
which is an analogue of the well-known discrete mean fourth power estimate for $|\zeta(1/2+it)|$ (see \cite{key13}, equation (8.26)), and can be proved in the same\pageoriginale way. Now \eqref{chap4:eq4.4.9} and \eqref{chap4:eq4.4.6} together give 
\begin{equation}\label{chap4:eq4.4.10}
R\ll T^{1+\delta}V^{-2},
\end{equation}
and thus $R\ll T^{2+5\delta}V^{-6}$ if also \eqref{chap4:eq4.4.8} holds.

Henceforth we may assume that
\begin{equation}\label{chap4:eq4.4.11}
V\gg T^{1/4+\delta}.
\end{equation}

Then by \eqref{chap4:eq4.4.10}
\begin{equation}\label{chap4:eq4.4.12}
R\ll T^{1/2-\delta}.
\end{equation}

Large values of $\varphi(s)$ on the critical line can be investigated in terms of large values of partial sums of its Dirichlet series, by the approximate functional equation \eqref{chap4:eq4.2.2}. The partial sums will be decomposed as in the proof of Theorem \ref{chap4:thm4.5}. However, in order to have compatible decompositions for different values $t\in[T,2T]$, we define the system of fractions $r=h/k$ in terms of $T$ rather than in terms of $t$. As a matter of fact, the ``order'' $K$ of the system will not be a constant, but it varies as a certain function $K(r)$ of $r$. More exactly, write
\begin{equation}\label{chap4:eq4.4.13}
M(r,t)=\frac{t}{2\pi r},
\end{equation}
and letting $R$ be the cardinality of the system $\{t_\nu\}$ satisfying \eqref{chap4:eq4.4.5}, \eqref{chap4:eq4.4.6}, and \eqref{chap4:eq4.4.11}, define
\begin{equation}\label{chap4:eq4.4.14}
K(r)=M(r,T)^{1/2}T^{-1/3}R^{-1/3}.
\end{equation}

We now construct the (finite) set of all fractions $r=h/k\geq 1$ satisfying the conditions
\begin{align}
k &\leq K(r),\label{chap4:eq4.4.15}\\
K(r) &\geq T^\delta,\label{chap4:eq4.4.16}
\end{align}\pageoriginale
and arrange these into an increasing sequence.

This sequence determines the sequence $\rho_1<\rho_2<\cdots <\rho_P$ of the mediants, and we define moreover $\rho_\circ=\rho_1^{-1}$. We apply \eqref{chap4:eq4.2.2} for $\sigma =k /2$, choosing
\begin{equation}\label{chap4:eq4.4.17}
x=x(t)=M(\rho_\circ,t),y=y(t)=(t/2\pi)^2x^{-1}.
\end{equation}

Then, if \eqref{chap4:eq4.4.6} and \eqref{chap4:eq4.4.11} hold, at least one of the sums of length $x(t_\nu)$ and $y(t_\nu)$ in \eqref{chap4:eq4.2.2} exceeds $V/3$ in absolute value. Let  us suppose that for at least $R/2$ points $t_\nu$ we have
\begin{equation}\label{chap4:eq4.4.18}
\left|\sum\limits_{n\leq x(t_\nu)}\tilde{a}(n)n^{-1/2-it_\nu}\right|\geq V/3;
\end{equation}
the subsequent arguments would be analogous if the other sum were as large as often.

The sum in \eqref{chap4:eq4.4.18} is split up by the points $M(\rho_i,t_\nu)$ as in \S~ \ref{chap4:sec4.2}. As to the set of points $t_\nu$ satisfying \eqref{chap4:eq4.4.18}, there are now two alternatives: either
\begin{equation}\label{chap4:eq4.4.19}
\left|\sum\limits_{n\leq M(\rho_P,t_\nu)}\tilde{a}(n)n^{-1/2-it_\nu}\right|\geq V/6
\end{equation}
for $\gg R$ points, or there are functions $M_1(t),M_2(t)$ of the type $M(\rho_i,t)$ such that $M_1(t)\asymp M_2(t)$ and 
\begin{equation}\label{chap4:eq4.4.20}
\left|S_\varphi\left(M_1(t_\nu),M_2(t_\nu)\right)\right|\gg VL^{-1},
\end{equation}
with $L=\log T$, for at least $\gg RL^{-1}$ points $t_\nu$. We are going to derive an upper bound for $R$ in each case.

Consider\pageoriginale first the former alternative. We apply the following large values theorem of M.N. Huxley for Dirichlet polynomials (for a proof, see \cite{key12} or \cite{key15}).
\end{proof}

\begin{lem}\label{chap4:lem4.3}
Let $N$ be a positive integer,
\begin{equation}\label{chap4:eq4.4.21}
f(s)=\sum\limits_{n=N+1}^{2N}a_n n^{-s},
\end{equation}
and let $s_r=\sigma_r+it_r, r=1,\ldots,R$, be a set of complex numbers such that $\sigma_r\geq 0, 1\leq |t_r-t_{r'} |\leq T$ for $r\neq r'$, and
$$
\left|f(s_r)\right|\geq V>0.
$$

Put
$$
G=\sum\limits_{n=N+1}^{2N}\left|a_n\right|^2.
$$

Then
\begin{equation}\label{chap4:eq4.4.22}
R\ll \left(GNV^{-2}+TG^3NV^{-6}\right)(NT)^\epsilon.
\end{equation}
\end{lem}

This lemma cannot immediately be applied to the Dirichlet polynomial
in \eqref{chap4:eq4.4.19}, for it is not of the type
\eqref{chap4:eq4.4.21}, and the length of the sum depends moreover on
$t_\nu$. To avoid the latter difficulty, we express the Dirichlet
polynomials in question by Perron's formula using the function 
$$
f(w)=\sum\limits_{n\leq N}\tilde{a}(n)n^{-w}
$$
with $N=M(\rho_P,2T)$. Letting $y=N^{1/2}$ and $\alpha=1/\log N$, we have 
\begin{multline*}
\sum\limits_{n\leq
  M(\rho_P,t_\nu)}\tilde{a}(n)n^{-1/2-it_\nu}=\frac{1}{2\pi i}
\int\limits_{\alpha-iY}^{\alpha+iY}f\left(1/2+it_\nu+w\right)\\ 
M \left(\rho_P,t_\nu\right)^w w^{-1}\,dw + o\left(T^\delta\right). 
\end{multline*}

Now,\pageoriginale in view of \eqref{chap4:eq4.4.19}, there is a
number $X\in [1,Y]$ and numbers $N_1,N_2$ with $N_1<N_2\leq
\max(2N_1,N)$ such that writing 
$$
f_\circ(w)=\sum\limits_{N_1\leq n\leq N_2}\tilde{a}(n)n^{-w}
$$
we have 
\begin{equation}\label{chap4:eq4.4.23}
\int\limits_{-X}^X\left|f_\circ\left(1/2+\alpha+i\left(t_\nu+u\right) \right)\right|\,du\gg V\chi L^{-2}
\end{equation}
for at least $\gg RL^{-2}$ points $t_\nu$. Next we select a sparse set of $R_\circ$ numbers $t_\nu$ with 
\begin{equation}\label{chap4:eq4.4.24}
R_\circ\ll 1+R X^{-1}L^{-2}
\end{equation}
such that \eqref{chap4:eq4.4.23} holds for these, and moreover
$|t_\mu-t_\nu|\geq 3 X$ for $\mu\neq\nu$. Further, by
\eqref{chap4:eq4.4.23} and similar quantitative arguments as above, we
conclude that there exist a number $w\gg VL^{-2}$, a subset of
cardinality $\gg R_\circ L^{-1}$ of the set of the $R_\circ$ indices
just selected, and for each $\nu$ in this subset a set $\gg
VW^{-1}X L^{-3}$ points $u_{\nu,\mu}\in[-X, X]$ such that  
\begin{equation}\label{chap4:eq4.4.25}
W\leq\left|f_\circ
\left(1/2+\alpha+i\left(t_\nu+u_{\nu,\mu}\right)\right) \right|\leq 2W 
\end{equation}
and
$$
\left|u_{\nu,\lambda}-u_{\nu,\mu}\right|\geq
1\quad\text{for}\quad\lambda\neq\mu. 
$$

The system $t_\nu+u_{\nu,\mu}$ for all relevant pairs $\nu,\mu$ is
well-spaced in the sense that the mutual distance of these numbers is
at least $1$, and its cardinality is  
$$
\gg R_\circ VW^{-1}X L^{-4}.
$$

On the other hand, its cardinality is by Lemma \ref{chap4:lem4.3} 
$$
\ll \left(N_1W^{-2}+TN_1W^{-6}\right)T^\delta\ll W^{-1}\left(NV^{-1}+TNV^{-5} \right)T^\delta L^{10}.
$$\pageoriginale

These two estimates give together
$$
R_\circ X\ll\left(NV^{-2}+TNV^{-6}\right)T^\delta L^{14}.
$$

But $R_\circ X\gg RL^{-2}$ by \eqref{chap4:eq4.4.24}, so finally
\begin{equation}\label{chap4:eq4.4.26}
R\ll\left(NV^{-2}+TNV^{-6}\right)T^{2\delta}\ll NV^{-2}T^{2\delta}
\end{equation}
by \eqref{chap4:eq4.4.11}. This means that a direct application of Lemma \ref{chap4:lem4.3} gives a correct result in the present case though the conditions of the lemma are not formally satisfied.

Since $\rho_P$ was the last mediant, we have by \eqref{chap4:eq4.4.14}, \eqref{chap4:eq4.4.16}, and the definition of $N$
$$
N\ll T^{2/3+2\delta}R^{2/3}.
$$

Together with \eqref{chap4:eq4.4.26}, this implies
$$
R\ll T^{2/3+4\delta}R^{2/3}V^{-2},
$$
whence
\begin{equation}\label{chap4:eq4.4.27}
R\ll T^{2+12\delta}V^{-6}
\end{equation}

We have now proved the desired estimate for $R$ in the case that \eqref{chap4:eq4.4.19} holds for $\gg R$ indices $\nu$.

Turning to the alternative \eqref{chap4:eq4.4.20}, we write
\begin{equation}\label{chap4:eq4.4.28}
S_\varphi\left(M_1(t),M_2(t)\right)=\sum\limits_{i=i_1}^{i_2}S_\varphi\left(M\left( \rho_{i+1},t\right),M\left(\rho_i,t\right)\right)
\end{equation}
for $T\leq t\leq 2T$. The sums $S_\varphi$ here are transformed by Theorem \ref{chap4:thm4.2}. That unique fraction $r=hk$ which lies between $\rho_i$ and $\rho_{i+1}$ will be used as\pageoriginale the fraction $r$ in the theorem. Write $M=M_1(T)$ and 
\begin{equation}\label{chap4:eq4.4.29}
K=M^{1/2}T^{-1/3}R^{-1/3}.
\end{equation}

Then by \eqref{chap4:eq4.4.14} we have $K(r)\asymp K$ for those $r$ related to the sums in \eqref{chap4:eq4.4.28}. Since for two consecutive fractions $r$ and $r'$ of our system we have $r'-r\leq[K(r')]^{-1}$, it is easily seen that $K(r)-K(r')<1$. Thus either $r$ and $r'$ are consecutive fractions in the Farey system of order $[K(r)]$, or exactly one fraction $r''=h''/k''$ with $K(r')<k''\leq K(r)$ of this system lies between them. Then, in any case, $|r-\rho_j|\asymp (kK)^{-1}$ for $j=i$ and $i+1$, whence as in \eqref{chap4:eq4.2.14} we have 
\begin{equation}\label{chap4:eq4.4.30}
m_j\asymp k^{-1}K^{-1}M^2T^{-1}\asymp k^{-1}M^{3/2}T^{-2/3}R^{1/3}.
\end{equation}

Hence $m_j\ll M^{1-\delta/3}$ by \eqref{chap4:eq4.4.12}, so that the upper bound part of the condition \eqref{chap4:eq4.1.10} is satisfied. The other conditions of Theorem \ref{chap4:thm4.2} are easily checked as in the proof of Theorem \ref{chap4:thm4.5}.

The error terms in Theorem \ref{chap4:thm4.2} are now by
\eqref{chap4:eq4.4.30} and \eqref{chap4:eq4.4.13}
$o(k^{1/2}k^{-1/2}L^2)$ and $o(K^{3/4}K^{1/4}M^{-1/4}L)$, and the sum
of these for different $r$ is  
\begin{align*}
& \ll k^2M^{-1}TL^2+K^3TM^{-5/4}L\\
& \ll T^{1/3}R^{-2/3}L^2+M^{1/4}R^{-1}L\ll T^{1/3}R^{-2/3}L^2.
\end{align*}

If
$$
T^{1/3}R^{-2/3}\ll VT^{-\delta},
$$
then these error terms can be omitted in \eqref{chap4:eq4.4.20}. Otherwise
\begin{align*}
R &\ll T^{1/2+3\delta/2} V^{-3/2}\\
&\ll \left(T^{1/2+3\delta/2} V^{-3/2}\right)^4=T^{2+6\delta}V^{-6}
\end{align*}
and\pageoriginale we have nothing to prove. Hence, in any case, we may omit the error terms in \eqref{chap4:eq4.1.28}.

Consider now the explicit terms in Theorem \ref{chap4:thm4.2}. For the numbers $n_j$ we have by \eqref{chap4:eq4.1.11}, \eqref{chap4:eq4.4.30}, and \eqref{chap4:eq4.4.29}
\begin{equation}\label{chap4:eq4.4.31}
n_j\asymp K^{-2}M\asymp T^{2/3}R^{2/3}.
\end{equation}

Denote by $S_r(t)$ the explicit part of the right hand side of \eqref{chap4:eq4.1.28} for the sum related to the fraction $r$. Then by \eqref{chap4:eq4.4.20}, \eqref{chap4:eq4.4.28}, and the error estimate just made we have 
\begin{equation}\label{chap4:eq4.4.32}
\left|\sum\limits_r S_r(t_\nu)\right|\gg VL^{-1}
\end{equation}
for at least $\gg RL^{-1}$ numbers $t_\nu$.

At this stage we make a brief digression to the proof of the estimate \eqref{chap4:eq4.4.1}. So far everything we have done for $\varphi(s)$ goes through for $\zeta^2(s)$ as well, except that in Theorem \ref{chap4:thm4.1} there is the leading explicit term and the first error term which have no counterpart in Theorem \ref{chap4:thm4.2}. The additional explicit term is $\asymp(hk)^{-1/2}L$, and the sum of these over the relevant fractions $r$ is $\ll T^{1/6}L$, which can be omitted by \eqref{chap4:eq4.4.11}. The additional error term in \eqref{chap4:eq4.1.12} is also negligible, for it is dominated in our case by the second one. So the analogy between the proofs of \eqref{chap4:eq4.4.1} and \eqref{chap4:eq4.4.2} prevails here, like also henceforth.

It will be convenient to restrict the fractions $r=h/k$ in \eqref{chap4:eq4.4.32} suitably. Suppose that $K_\circ\leq k\leq K'_\circ$, where $K_\circ\asymp K'_\circ$ and $K_\circ\ll K$, and suppose also that for two different fractions $r=h/k, r'=h'/k'$ in our\pageoriginale system we have 
\begin{gather}
\left|r-r'\right|\gg K_\circ^{-2}T^\delta \label{chap4:eq4.4.33}\\
\intertext{and}
0<\left|\frac{1}{hk}-\frac{1}{h'k'}\right|<\left(K'_\circ \right)^{-2}\label{chap4:eq4.4.34}
\end{gather}

An interval $[K_\circ,K'_\circ]$ and a set of fractions of this kind can be found such that 
\begin{equation}\label{chap4:eq4.4.35}
\left|\sum\limits_r S_r\left(t_\nu\right)\right|\gg VT^{-2\delta}
\end{equation}
for at least $R_1\gg RT^{-2\delta}$ numbers $t_\nu$. The sum over $r$ here is restricted as indicated above.

Let
\begin{equation}\label{chap4:eq4.4.36}
Z=K_\circ^2 M^{-1}T.
\end{equation}

There exists a number $R_2$ such that those intervals $[T+pZ,T+(P+1)Z]$ containing at least $R_2/2$ and at most $2R_2$ of the $R_1$ numbers $t_\nu$ contain together $\gg R_1L^{-1}$ of these. Omit the other numbers $t_\nu$, and suppose henceforth that the $t_\nu$ under consideration lie in these $\ll R_1R_2^{-1}$ intervals. Summing \eqref{chap4:eq4.4.35} with respect to those $t_\nu$ lying in the interval $[T+pZ,T+(p+1)Z]$, we obtain by Cauchy's inequality
\begin{equation}\label{chap4:eq4.4.37}
R_2VT^{-2\delta}\ll \left(R_2\sum\limits_\nu\left|\sum\limits_r S_r\left(t_\nu \right)\right|^2\right)^{1/2}.
\end{equation}

The following inequality of P.X. Gallagher (see \cite{key23}, Lemma 1.4) is now applied to the sum over $t_\nu$.

\begin{lem}\label{chap4:lem4.4}
Let $T_\circ, T\geq\delta >0$ be real numbers, and let $A$ be a finite set in the interval $[T_\circ +\delta/2, T_\circ +T-\delta/2]$ such that $|a'-a|\geq\delta$ for any two distinct numbers $a, a'\in A$. Let $S$ be a continuous complex valued\pageoriginale function in $[T_\circ,T_\circ+T]$ with continuous derivative in $(T_\circ,T_\circ +T)$. Then
$$
\sum\limits_{a\epsilon A}|S(a)|^2\leq\delta^{-1}\int\limits_{T_\circ}^{T_\circ+T}|S(t)|^2 \,dt+\left(\int\limits_{T_\circ}^{T_\circ +T}|S(t)|^2\,dt\right)^{1/2}\left( \int\limits_{T_\circ}^{T_\circ +T}|S'(t)|^2\,dt\right)^{1/2}.
$$
\end{lem}

The lengths $n_j$ of the sums in $S_r(t)$ depend linearly on
$t$. However, the variation of $n_j$ in the interval $T+pZ\leq
t<T+(p+1)Z$ is only $o(1)$, so that \eqref{chap4:eq4.4.35} and
\eqref{chap4:eq4.4.37} remain valid if we redefine $S_r(t)$ taking
$n_j$ constant in this interval. Lemma \ref{chap4:lem4.4} then gives 
\begin{gather}
\sum\limits_\nu\left|\sum\limits_r S_r\left(_\nu\right)\right|^2\ll \int\limits_\circ^Z\left|\sum\limits_rS_r(T+pZ+u)\right|^2 \,du\label{chap4:eq4.4.38}\\
+\left(\int\limits_\circ^Z\left|\sum\limits_rS_r(T+pZ+u)\right|^2\,du\right)^{1/2}\left(\int\limits_\circ^Z\left|\sum\limits_rS'_r(T+pZ+u)\right|^2\,du \right)^{1/2}.\notag
\end{gather}

Let $\eta(u)$ be a weight function of the type $\eta_J(u)$ such that
$JU=Z,\eta(u)=1$ for $0\leq u\leq Z,\eta(u)=0$ for $u\notin (-Z,2Z)$,
and $J$ is a large positive integer. Then 
\begin{gather}
\int\limits_\circ^Z\left|\sum\limits_r S_r(T+pZ+u)\right|^2\,du\leq \int\limits_{-Z}^{2Z}\eta(u)\left|\sum\limits_rS_r\right|^2 \,du\label{chap4:eq4.4.39}\\
=\sum\limits_{r,r'}\int\limits_{-Z}^{2Z}\eta(u)S_r\overline{S_{r'}}\,du.\notag
\end{gather}

We now dispose of the nondiagonal terms. Put $t(u)=T+pZ+u$. When the integral on the right of \eqref{chap4:eq4.4.39} is written as a sum of integrals, recalling the definition \eqref{chap4:eq4.1.28} of $S_r(t)$, a typical term is 
\begin{equation}\label{chap4:eq4.4.40}
\int\limits_{-Z}^{2Z}\eta(u)g(u)e(f(u))\,du,
\end{equation}
where\pageoriginale
\begin{align*}
g(u) &=
\pi^{1/2}2^{-1/2}(hkh'k')^{-1/4}\tilde{a}(n)\overline{\tilde{a}(n')}\;
(nn')^{-1/4}\times\\ 
&\quad \times
e\left(n\left(\frac{\bar{h}}{k}-\frac{1}{2hk}\right)-n'\left(
\frac{\overline{h'}}{k'}-\frac{1}{2h'k'}\right)\right)t(u)^{-1/2}\\
& \hspace{3cm}\left(1+
\frac{\pi n}{2hkt(u)}\right)^{-1/4}\left(1+\frac{\pi n}{2h'k't(u)}
\right)^{-1/4},\\ 
f(u) &= (-1)^{j-1}\left\{(t(u)/\pi)\phi\left(\frac{\pi n}{2hkt(u)}\right)+1/8 \right\}\\
&\quad -(-1)^{j'-1}\left\{(t(u)/\pi)\phi\left(\frac{\pi n'}{2h'k't(u)}\right)+ 1/8\right\}+(t(u)/\pi)\log(r/r'),
\end{align*}
$r=h/k,r'=h'/k',j'$ and $j'$ are $1$ or $2$, and $n<n_j,n'<n_{j'}$. Now 
$$
\left|\frac{d}{du}\left(t(u)\log\left(r/r'\right)\right)\right|=\left|\log \left(r/r'\right)\right|\gg K_\circ^{-2}MT^{-1+\delta}
$$
by \eqref{chap4:eq4.4.33}, while by \eqref{chap4:eq4.1.6}
$$
\left|\frac{d}{du}\left(t(u)\phi\left(\frac{\pi n}{2hkt(u)}\right)\right)\right|\asymp (hkT)^{-1/2}n^{1/2}\asymp K_\circ^{-1} M^{1/2} n^{1/2} T^{-1},
$$
which is by \eqref{chap4:eq4.4.31}
$$
\ll K_\circ^{-1}K^{-1}MT^{-1}\ll K_\circ^{-2}MT^{-1}.
$$

Accordingly,
$$
|f'(u)|\asymp |\log(r,/r')|\gg T^\delta Z^{-1}.
$$

We may now apply Theorem \ref{chap2:thm2.3} to the integral \eqref{chap4:eq4.4.40} with $\mu\asymp Z, M\asymp |\log(r,r')|\gg T^\delta Z^{-1}$, and $U\asymp Z$. If $J\asymp\delta^{-2}$ and $\delta$ is small, then this integral is negligible. A similar argument applies to the integral involving $S'_r$ in \eqref{chap4:eq4.4.38}. Consequently, it follows from \eqref{chap4:eq4.4.37} - \eqref{chap4:eq4.4.39} that 
\begin{align*}
& R_2VT^{-2\delta}\ll R_2^{1/2}\left\{\sum\limits_r\int\limits_{-Z}^{2Z}\left| S_r(t(u))\right|^2\,du\right.\\
& +\left.\left(\sum\limits_r\int\limits_{-Z}^{2Z}\left|S_r(t(u))\right|^2\,du \right)^{1/2}\left(\sum\limits_r\int\limits_{-Z}^{2Z}\left|S'_r(t(u))\right|^2\, du\right)^{1/2}\right\}^{1/2}.
\end{align*}

Summing\pageoriginale these inequalities with respect to the $\ll R_1R_2^{-1}$ values of $p$, we obtain by Cauchy's inequality
{\fontsize{10}{12}\selectfont
\begin{align*}
& R_1L^{-1}VT^{-2\delta}\ll R_1^{1/2}\left\{\sum\limits_{p,r}\int\limits_{-Z}^{2Z} \left|S_r(T+pZ+u)\right|^2\,du\right.\\
& +\left.\left(\sum\limits_{p,r}\int\limits_{-Z}^{2Z}\left|S_r(T+pZ+u)\right|^2 \, du\right)^{1/2}\left(\sum\limits_{p,r}\int\limits_{-Z}^{2Z}\left|S'_r(T+pZ+u) \right|^2\,du\right)^{1/2}\right\}^{1/2}.
\end{align*}}

For each $p$, the integrals here are expressed by the mean value theorem. Then by \eqref{chap4:eq4.4.36} this implies (recall that $RT^{-2\delta}\ll R_1\leq R$)
\begin{align}
RV & \ll R^{1/2}K_\circ M^{-1/2}T^{1/2+4\delta}L\left\{\sum\limits_{p,r}\left|S_r (t_p)\right|^2\right.\label{chap4:eq4.4.41}\\
&\quad \left.+\left(\sum\limits_{p,r}\left|S_r(t_p)\right|^2\right)^{1/2}\left( \sum\limits_{p,r}\left|S'_r(t'_p)\right|^2\right)^{1/2}\right\}^{1/2},\notag
\end{align}
where $\{t_p\}$ is a set of numbers in the interval $(T-Z,2T+2Z)$ such that 
\begin{equation}\label{chap4:eq4.4.42}
\left|t_p-t_{p'}\right|\geq Z\quad\text{for}\quad p\neq p',
\end{equation}
and similarly for $\{t'_p\}$. 

The rest of the proof will be devoted to the estimation of the double
sums on the right of \eqref{chap4:eq4.4.41}. For convenience we
restrict in $S_r$ and $S'_r$ the summation to an interval $N\leq n\leq
N'$, where $N\asymp N'$, and take $j=1$. The notation $S_r$ is still
retained for these sums. The original sum can be written as a sum of
$o(L)$ new sums. We are going to show that  
\begin{equation}\label{chap4:eq4.4.43}
\sum\limits_{p,r}\left|S_r(t_p)\right|^2\ll \left(K_\circ^{-2}K^{-2}M^2T^{-1}+ K_\circ^{-1}M^{1/2}R\right)T^{2\delta}.
\end{equation}

It will be obvious that the argument of the proof of this gives the same estimate for the similar sum involving $S'_r$ as well. Then the inequality \eqref{chap4:eq4.4.41}\pageoriginale becomes 
$$
RV\ll\left(K^{-1}M^{1/2}R^{1/2}+K_\circ^{1/2}M^{-1/4}T^{1/2}R\right)T^{6\delta}\ll R^{5/6}T^{1/3+6\delta};
$$
recall the definition \eqref{chap4:eq4.4.29} of $K$. This gives
$$
R\ll T^{2+36\delta}V^{-6},
$$
as desired.

To prove the crucial inequality \eqref{chap4:eq4.4.43}, we apply methods of Hal\'asz and van der Corput. The following abstract version of Hal\'asz's inequality is due to Bombieri (see \cite{key23}, Lemma 1.5, or \cite{key13}, p.~ 494).

\begin{lem}\label{chap4:lem4.5}
If $\xi,\varphi_1,\ldots,\varphi_R$ are elements of an inner product space over the complex numbers, then 
$$
\sum\limits_{r=1}^R\left|\left(\xi,\varphi_r\right)\right|^2\leq\parallel\xi \parallel^2\underset{1\leq r\leq R}{\max}\sum\limits_{s=1}^R\left|\left( \varphi_r,\varphi_s\right)\right|.
$$
\end{lem}

Suppose that the numbers $N$ and $N'$ above are integers, and define the usual inner product for complex vectors $a=(a_N,\ldots,a_{N'}), b=(b_N,\ldots,b_{N'})$ as 
$$
(a,b)=\sum\limits_{n=N}^{N'}a_n\bar{b}_n.
$$

Define vectors
\begin{gather*}
\xi=\left\{\overline{\tilde{a}(n)}n^{-1/4}\right\}_{n=N}^{N'},\\
\varphi_{p,r}=\left\{\left(1+\frac{\pi n}{2hkt_p}\right)^{-1/4}e\left(n\left( \frac{\bar{h}}{k}-\frac{1}{2hk}\right)+\left(t_p/\pi\right)\phi\left(\frac{\pi n}{2hkt_p}\right)\right)\right\}_{n=N}^{N'}
\end{gather*}
with the convention that if $n_1<N'$, then in $\varphi_{p,r}$ the components for $n_1\leq n\leq N'$ are understood as zeros. Then by \eqref{chap4:eq4.1.28} we have 
$$
\left|S_r(t_p)\right|\ll K_\circ^{-1/2}M^{1/4}T^{-1/2}\left|\left(\xi,\varphi_{p,r} \right)\right|.
$$

Hence,\pageoriginale by Lemma \ref{chap4:lem4.5}, there is a pair $p',r'$ such that 
\begin{equation}\label{chap4:eq4.4.44}
\sum\limits_{p,r}\left| S_r(t_p)\right|^2 \ll K_\circ^{-1}M^{1/2}N^{1/2}T^{-1+\delta} \sum\limits_{p,r}\left| \left( \varphi_{p,r},\varphi_{p',r'}\right)\right|.
\end{equation}

If now
\begin{equation}\label{chap4:eq4.4.45}
\sum\limits_{p,r}\left|\left(\varphi_{p,r},\varphi_{p',r'}\right)\right| \ll \left( K_\circ^{-1}K^{-1}M+KM^{-1/2}RT\right)T^\delta,
\end{equation}
then \eqref{chap4:eq4.4.43} follows from \eqref{chap4:eq4.4.44};
recall that $N\ll K^{-2}M$ by \eqref{chap4:eq4.4.31}. Hence it remains
to prove \eqref{chap4:eq4.4.45}. 

Let 
$$
f_{p,r}(x)=x\left(\frac{\bar{h}}{k}-\frac{1}{2hk}\right)+\left(t_p/\pi\right)\phi\left(\frac{\pi x}{2hkt_p}\right).
$$

The estimation of $|(\varphi_{p,r},\varphi_{p',r'})|$ can be reduced, by partial summation, to that of exponential sums
$$
\sum\limits_n e\left(f_{p,r}(n)-f_{p',r'}(n)\right).
$$

Namely, if this sum is at most $\Delta(p,r)$ in absolute value whenever $n$ runs over a subinterval of $[N,N']$, then 
$$
\left|\left(\varphi_{p,r},\varphi_{p',r'}\right)\right|\ll \Delta(p,r).
$$

So in place  of \eqref{chap4:eq4.4.45} it suffices to show that
\begin{equation}\label{chap4:eq4.4.46}
\sum\limits_{p,r}\Delta(p,r)\ll \left(K_\circ^{-1}K^{-1}M+KM^{-1/2}RT\right)T^\delta.
\end{equation}

The quantity $\Delta(p,r)$ will be estimated by van der Corput's method. To this end we need the first two derivatives of the function $f_{p,r}(x)-f_{p',r'}(x)$ in the interval $[N,N']$. By the definition \eqref{chap4:eq4.1.6} of the function $\phi(x)$ we have 
\begin{align*}
\phi'(x) &= \left(1+x^{-1}\right)^{1/2},\\
\phi''(x) &= -\frac{1}{2}x^{-3/2}(1+x)^{-1/2}.
\end{align*}\pageoriginale

Then by a little calculation it is seen that 
\begin{gather}
f'_{p,r}(x)-f'_{p',r'}(x)=\frac{\bar{h}}{k}-\frac{1}{2hk}-\frac{\bar{h'}}{k'}+ \frac{1}{2h'k'}\label{chap4:eq4.4.47}\\
+BK_\circ^{-3}M^{3/2}N^{-1/2}\left|hkt_p^{-1}-h'k't_p^{-1}+\frac{1}{2}\pi x\left(t_p t_{p'}\right)^{-1}\left(\frac{hk}{h'k'}-\frac{h'k'}{hk}\right) \right|,\notag
\end{gather}
where $|B|\asymp 1$, and 
\begin{align}
\left|f''_{p,r}(x)-f''_{p',r'}(x)\right| & \asymp K_\circ^{-3}M^{3/2}N^{-3/2}\left| hkt_p^{-1}-h'k't_{p'}^{-1}\right.\label{chap4:eq4.4.48}\\
&\qquad +\left.\frac{1}{2}\pi x \left(t_p^{-2}-t_{p'}^{-2}\right)\right|.\notag
\end{align}

We shall estimate $\Delta(p,r)$ either by Lemma \ref{chap4:lem4.1}, or by the following simple lemma (see \cite{key27}, Lemmas 4.8 and 4.2). We denote by $\parallel\alpha\parallel$ the distance of $\alpha$ from the nearest integer.

\begin{lem}\label{chap4:lem4.6}
Let $f\in C^1[a,b]$ be a real function with $f'(x)$ monotonic and $\parallel f'(x)\parallel\geq m>0$. Then
$$
\sum\limits_{a<n\leq b}e(f(n))\ll m^{-1}.
$$
\end{lem}
 Turning to the proof of \eqref{chap4:eq4.4.46}, let us first consider the sum over the pairs $p,r'$. Trivially,
$$
\Delta(p',r')\ll N\ll K^{-2}M\ll K_\circ^{-1}K^{-1}M.
$$

If $p\neq p'$, then by \eqref{chap4:eq4.4.47}
$$
\left|f'_{p,r'}(x)-f'_{p',r'}(x)\right|\asymp K_\circ^{-1}M^{1/2}N^{-1/2}T^{-1} \left| t_p-t_{p'}\right|.
$$

We\pageoriginale may apply Lemma \ref{chap4:lem4.6} if
$$
\left|t_p-t_{p'}\right|\ll K_\circ M^{-1/2}N^{1/2} T,
$$
and the corresponding part of the sum \eqref{chap4:eq4.4.46} is 
$$
\ll K_\circ M^{-1/2} N^{1/2} T\sum\limits_p\left|t_p-t_{p'}\right|^{-1}\ll K_\circ^{-1} K^{-1}MT^\delta;
$$
recall \eqref{chap4:eq4.4.42}, \eqref{chap4:eq4.4.36}, and \eqref{chap4:eq4.4.31}.

Ohterwise $\Delta(p,r)$ is estimated by Lemma \ref{chap4:lem4.1}. Now by \eqref{chap4:eq4.4.48}
$$
\left|f''_{p,r'}(x)-f''_{p',r'}(x)\right|\asymp K_\circ^{-1}M^{1/2}N^{-3/2}T^{-1} \left|t_p-t_p\right|\gg N^{-1},
$$
so that these values of $p$ contribute 
\begin{align*}
\ll\sum\limits_p\left(N\left(K_\circ^{-1}M^{1/2}N^{-3/2}\right)^{1/2}+N^{1/2}\right) &\ll \left(K_\circ^{-1/2}M^{1/4}N^{1/4}+N^{1/2}\right)R\\
&\ll M^{1/2}R,
\end{align*}
which is clearly $\ll KM^{-1/2}RT$.

For the remaining pairs $p,r$ in \eqref{chap4:eq4.4.46} we have $r\neq r'$. 
Let $p$ be fixed for a moment. Then 
\begin{equation}\label{chap4:eq4.4.49}
\left|hkt_p^{-1}-h'k't_{p'}^{-1}\right|\gg T^{-1},
\end{equation}
save perhaps for one ``exceptional'' fraction $r=h/k$; note that by the assumption \eqref{chap4:eq4.4.34} no two different fractions in our system have the same value for $hk$. If \eqref{chap4:eq4.4.49} holds, then by \eqref{chap4:eq4.4.48}
$$
\left|f''_{p,r}(x)-f''_{p',r'}(x)\right|\asymp K_\circ^{-3}M^{3/2}N^{-3/2}\left| hkt_p^{-1}-h'k't_{p'}^{-1}\right|.
$$

Then, if $r$ runs over the non-exceptional fractions,
\begin{align*}
\sum\limits_r\left|f''_{p,r}(x)-f''_{p',r'}(x)\right|^{-1/2} &\ll K_\circ^{3/2} M^{-3/4}N^{3/4}T^{1/2}\sum\limits_{m\ll K_\circ^2M^{-1}T}m^{-1/2}\\
&\ll K_\circ^{5/2}M^{-5/4}N^{3/4}T\\
&\ll KM^{-1/2}T,
\end{align*}
and\pageoriginale
$$
\sum\limits_rN\left|f''_{p,r}(x)-f''_{p',r'}(x)\right|^{1/2}\ll K_\circ^{3/2}M^{-3/4}N^{1/4}T\ll KM^{-1/2}T.
$$

Hence by Lemma \ref{chap4:lem4.1}
\begin{equation}\label{chap4:eq4.4.50}
\sum\limits_r\Delta(p,r)\ll KM^{-1/2}T.
\end{equation}

Consider finally $\Delta(p,r)$ for the exceptional fraction. We shall need the auxiliary result that for any two different fractions $h/k$ and $h'/k'$ of our system we have 
\begin{equation}\label{chap4:eq4.4.51}
\parallel\frac{\bar{h}}{k}-\frac{1}{2hk}-\frac{\overline{h'}}{k'}+\frac{1}{2h'k'} \parallel\gg K_\circ^{-2}M^2T^{-2}.
\end{equation}

For if $k\neq k'$, then the left hand side is $\gg K_\circ^{-2}$ by the condition \eqref{chap4:eq4.4.34}, like also in the case $k=k'$ if $h\nequiv h'\pmod k$. On the other hand, if $k=k'$ and $h\equiv h'\pmod k$, then $|h-h'|\gg K_\circ$, and the left hand side is
$$
\left|\frac{1}{2hk}-\frac{1}{2h'k}\right|\gg(hh')^{-1}\gg K_\circ^{-2}M^2 T^{-2}.
$$
 
Let $r$ be the exceptional fraction (for given $p$), and suppose first that for a certain small constant $c$ 
\begin{equation}\label{chap4:eq4.4.52}
\left|hkt_p^{-1}-h'k't_{p'}^{-1}\right|\leq cK_\circ M^{1/2}N^{1/2}T^{-2}
\end{equation}
in addition to the inequality
\begin{equation}\label{chap4:eq4.4.53}
\left|hkt_p^{-1}-h'k't_p^{-1}\right|\ll T^{-1}
\end{equation}
which defines the exceptionality. Then, by \eqref{chap4:eq4.4.51}, the first four terms in \eqref{chap4:eq4.4.47} dominate, and we have 
$$
\parallel f''_{p,r}(x)-f'_{p'r'}(x)\parallel\gg K_\circ^{-2}M^2T^{-2}.
$$

Hence\pageoriginale by Lemma \ref{chap4:lem4.6}
$$
\Delta(p,r)\ll K^2M^{-2}T^2\ll KM^{-3/2}T^{5/3}\ll KM^{-1/2}T,
$$
since $K\ll M^{1/2}T^{-1/3}$ and $M\gg T^{2/3}$.

On the other hand, if \eqref{chap4:eq4.4.52} does not hold, then by \eqref{chap4:eq4.4.48} and \eqref{chap4:eq4.4.53}
$$
K_\circ^{-3}M^{3/2}N^{-3/2}T^{-1}\gg\left|f''_{p,r}(x)-f''_{p',r'}(x)\right|\gg K_\circ^{-2}M^2N^{-1}T^{-2}.
$$

Hence by Lemma \ref{chap4:lem4.1}
\begin{align*}
\Delta(p,r) &\ll K_\circ^{-3/2}M^{3/4}N^{1/4}T^{-1/2}+K_\circ M^{-1}N^{1/2}T\\
&\ll MT^{-1/2}+M^{-1/2}T\ll M^{-1/2}T.
\end{align*}

Now we sum the last estimations and those in \eqref{chap4:eq4.4.50} with respect to $p$ to obtain
$$
\sum_{\substack{p,r\\r\neq r'}}\Delta(p,r)\ll KM^{-1/2}RT.
$$

Taking also into account the previous estimations in the case $r=r'$, we complete the proof of \eqref{chap4:eq4.4.46}, and also that of Theorem \ref{chap4:thm4.7}.


\bigskip
 
\addcontentsline{toc}{subsection}{Notes}
\section*{Notes}

Theorems \ref{chap4:thm4.1} and \ref{chap4:thm4.2} were proved in \cite{key16} for integral values of $r$. The results of \S~ \ref{chap4:sec4.1} as they stand were first worked out in \cite{key17}.

In \S\S~ \ref{chap4:sec4.2} - \ref{chap4:sec4.4} we managed (just!) to dispense with weighted versions of transformation formulae. The reason is that in all the problems touched upon relatively large values of Dirichlet polynomials and exponential sums occurred, and therefore even the comparatively weak error\pageoriginale terms of the ordinary transformation formulae were not too large. But in a context involving also small or ``expected'' values of sums it becomes necessary to switch to smoothed sums in order to reduce error terms. A challenging application of this kind would be proving the mean value theorems 
\begin{align*}
\int\limits_T^{T+T^{2/3}}\left|\zeta(1/2+it)\right|^4\,dt &\ll T^{2/3+\epsilon}\\
\int\limits_T^{T+T^{2/3}}\left|\varphi(k /2+it)\right|^2\,dt &\ll T^{2/3+\epsilon},
\end{align*}
respectively due to $H$. Iwaniec \cite{key14} and $A$. Good \cite{key9} (a corollary of \eqref{int:eq0.11} in a unified way using methods of this chapter.

The estimate \eqref{chap4:eq4.4.2} for the sixth moment of $\varphi(k
/2+it)$ actually gives the estimate for $\varphi(k /2+it)$ in Theorem
\ref{chap4:thm4.5} as a corollary, so that strictly speaking the
latter theorem is superfluous. However, we found it expedient to work
out the estimate of $\varphi(k /2+it)$ in a simple way in order to
illustrate the basic ideas of the method, and also with the purpose of
providing a model or starting point for the more elaborate proofs of
Theorem \ref{chap4:thm4.6} and \ref{chap4:thm4.7}, and perhaps for
other applications to come. 

The method of \S~ \ref{chap4:sec4.4} can probably be applied to give
results to the effect that an exponential sum involving $d(n)$ or
$a(n)$ which depends on a parameter $X$ is ``seldom'' large as a
function of $X$. A typical example is the sum
\eqref{chap4:eq4.3.38}. An analogue of Theorem \ref{chap4:thm4.7}
would be  
$$
\int\limits_{X_1}^{X_2}\left|\sum\limits_{M_1\leq m\leq
  M_2}b(m)e\left( \frac{X}{m}\right)\right|^6\,dx \ll
X_1^{5/2+\epsilon} 
$$ 
for $M_1\asymp M_2, X_1\asymp X_2, X_1\gg M_1^2$, and $b(m)=d(m)$ or
$\tilde{a}(m)$.  


\bigskip



\begin{thebibliography}{99}
\addcontentsline{toc}{subsection}{Bibliography}
\bibitem{key1} T.M.\pageoriginale APOSTOL: \emph{Modular Functions and
  Dirichlet Series in Number Theory}. Craduate Texts in Math. 41,
  Springer, 1976. 
\bibitem{key2} F.V. ATKINSON: The mean-value of the Riemann
  zeta-function. \emph{Acta Math}. 81 (1949), 353--376. 
\bibitem{key3} B.C. BERNDT: Identities involving the coefficients of a
  class of Dirichlet series. \emph{I. Trans. Amer. Math. Soc}. 137
  (1969), 354--359; II. ibid, 361--374; III. ibid, 146 (1969),
  323--348; IV. ibid, 149 (1970), 179--185; V. ibid, 160 (1971),
  139--156; VI. ibid, 157--167; VII. ibid, 201 (1975), 247--261. 
\bibitem{key4} B.C. BERNDT: The Voronoi summation formula. \emph{The
  Theory of Arithmetic Functions}, 21--36. Lecture Notes in Math. 251,
  Springer 1972. 
\bibitem{key5} P. DELIGNE: La Conjecture de Weil. \emph{Inst. Hautes
  \'Etudes Sci publ. Math}. 53 (1974), 273--307. 
\bibitem{key6} A.L. DIXON and W.L. FERRAR: Lattice-point summation
  formulae. \emph{Quart. J. Math}. Oxford 2 (1931), 31--54. 
\bibitem{key7} C. EPSTEIN, J.L. HAFNER and P. SARNAK: Zeros of
  L-functions attached to Maass forms. \emph{Math. Z.} 190 (1985),
  113--128. 
\bibitem{key8} T. ESTERMANN: On the representation of a number as the
  sum of two products. \emph{Proc. London Math. Soc}. (2) 31 (1930),
  123--133. 
\bibitem{key9} A. GOOD: The square mean of Dirichlet series associated
  with cusp forms. \emph{Mathematika} 29 (1982), 278--295. 
\bibitem{key10} S.W.\pageoriginale GRAHAM: An a algorithm for
  computing optimal exponent pairs. \emph{J. London Math. Soc}. (2) 33
  (1986), 203--218. 
\bibitem{key11} D.R. HEATH-BROWN: The twelth power moment of the
  Riemann zeta-function. \emph{Quart. J. Math}. oxford 29 (1978),
  443--462. 
\bibitem{key12} M.N. HUXLEY: Large values of Dirichlet polynomials
  III. \emph{Acta Arith}. 26 (1975), 435--444. 
\bibitem{key13} A. Ivic: \emph{The Riemann Zeta-function}. John Wiley
  \& Sons, New York, 1985. 
\bibitem{key14} H. IWANIEC: Fourier Coefficients of Cusp Forms and the
  Riemann Zeta-function. \emph{S\'eminare de Th\'eorie des Nombres},
  Univ. Bordeaux 1979/80, expos\'e no 18, 36 pp. 
\bibitem{key15} M. JUTILA: Zero-density estimates for
  L-functions. \emph{Acta Arith}. 32 (1977), 55--62. 
\bibitem{key16} M. JUTILA: Transformation formulae for Dirichlet
  polynomials. \emph{J. Number Theory} 18 (1984), 135--156. 
\bibitem{key17} M. JUTILA: Transformation formulae for Dirichlet
  polynomials II. (unpublished) 
\bibitem{key18} M. JUTILA: On exponential sums involving the divisor
  function. \emph{J. Reine Angew. Math}. 355 (1985), 173--190. 
\bibitem{key19} M. JUTILA: On the approximate functional equation for
  $\zeta^2(s)$ and other Dirichlet
  series. \emph{Quart. J. math}. Oxford (2) 37 (1986), 193--209. 
\bibitem{key20} N.V. KUZNETSOV: Petersson's conjecture for cusp forms
  of weight zero and Linnik's conjecture. sums of Kloosterman
  sums. \emph{Mat. Sb}. 111 (1980), 334--383. (In Russian). 
\bibitem{key21} H.\pageoriginale MAASS: \"Uber eine neue Art von
  nichtanalytischen automorphen Funktionen und die Bestimmung
  Dirichletcher Reihen durch
  Funktionalgleichungen. \emph{math. Ann}. 121 (1949), 141--183. 
\bibitem{key22} T. MEURMAN: On the mean square of the Riemann
  zeta-function. \emph{Quart J. Math}. Oxford. (to appear). 
\bibitem{key23} H.L. MONTGOMERY: \emph{Topics in Multiplicative Number
  Theory}. Lecture Notes in Math. 227, Springer, 1971. 
\bibitem{key24} R.A. RANKIN: Contributions to the theory of
  Ramanujan's function $\tau(n)$ and similar arithmetical functions
  II. The order of Fourier coefficients of integral modular
  forms. \emph{Math. Proc. Cambridge Phil. Soc}. 35 (1939), 357--372. 
\bibitem{key25} P. SHIU: A Brun-Titchmarsh theorem for multiplicative
  functions. \emph{J. Reine Angew. math}. 31 (1980), 161--170. 
\bibitem{key26} E.C. TITCHMARSH: \emph{Introduction to the Theorey of
  Fourier Integrals}. oxford univ. Press, 1948. 
\bibitem{key27} E.C. TITCHMARSH:{The Theory of the Riemann
  Zeta-function}. Oxford Univ. Press, 1951.
\bibitem{key28} K.-C. TONG: On divisor problems III. \emph{Acta
  Math. Sinica} 6 (1956), 515--541. (in Chinese) 
\bibitem{key29} G.N. WATSON': \emph{A Treatise on the Theory of Bessel
  Functions} (2nd ed.). Cambridge Univ. Press, 1944. 
\bibitem{key30} J.R. WILTON: A Note on Ramanujan's arithmetical
  function $\tau(n)$. \emph{Proc. Cambridge Phil. soc}. 25 (1929),
  121--129. 
\end{thebibliography}



 
