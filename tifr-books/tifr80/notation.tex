\chapter{Preface}

\markboth{Preface}{Preface}


\indent These lectures were given at the Tata Institute  of Fundamental
Research in October - November 1985. It was my first object to present
a self-contained introduction to summation and transformation formulae
for exponential sums involving either the divisor function $d(n)$ or
the Fourier coefficients of a cusp form; these two cases are in fact
closely analogous. Secondly, I wished to show how these formulae - in
combination with some standard methods of analytic number theory - can
be applied to the estimation of the exponential sums in question. 

I would like to thank Professor K. Ramachandra, Professor
R. Balasubramanian, Professor S. Raghavan, Professor T.N. Shorey, and
Dr. S. Srinivasan for their kind hospitality, and my whole audience
for interest and stimulating discussions. In addition, I am grateful
to my colleagues D.R. Heath-Brown, M.N. Huxley, A. Ivic, T. Meurman,
Y. Motohashi, and many others for valuable remarks concerning the
present notes and my earlier work on these topics.


\chapter{Notation}

\markboth{Notation}{Notation}

The following notation, mostly standard, will occur repeatedly in these notes.
{\fontsize{9}{11}\selectfont
\tabcolsep=3pt
\begin{longtable}{ll}
$\gamma$ & Euler's constant.\\
$s$ & $=\sigma +it$, a complex variable.\\
$\zeta(s)$ & Riemann's zeta-function.\\
$\Gamma(s)$ & The gamma-function.\\
$\chi(s)$ & $=2^s\pi^{s-1}\Gamma(1-s)\sin(\pi s/2)$.\\
$J_n(z), Y_n(z), K_n(z)$ & Bessel functions.\\
$e(\alpha)$ & $=e^{2\pi i\alpha}$.\\
$e_k(\alpha)$ & $=e^{2\pi i\alpha/k}$.\\
$\Res(f,a)$ & The residue of the function $f$ at the point $a$.\\
$\int\limits_{(c)}f(s)\,ds$ & The integral of the function $f$ over the line Re $s=c$.\\
$d(n)$ & The number of positive divisors of the integer $n$.\\
$a(n)$ & Fourier coefficient of a cusp form.\\
$\varphi(s)$ & $=\sum\limits_{n=1}^\infty a(n)n^{-s}$.\\
$\kappa$ & The weight of a cusp form.\\
$\tilde{a}(n)$ & $=a(n)n^{-(\kappa-1)/2}$.\\
$r$ & $=h/k$, a rational number with $(h,k)=1$ and $k\geq 1$.\\
$\bar{h}$ & The residue $\pmod k$ defined by $h\bar{h}\equiv 1\pmod k$.\\
$E(s,r)$ & $=\sum\limits_{n=1}^\infty d(n)e(nr)n^{-s}$.\\
$\varphi(s,r)$ & $=\sum\limits_{n=1}^\infty a(n)e(nr)n^{-s}$.\\
$\parallel\alpha\parallel$ & The distance of $\alpha$ from the nearest integer.\\
$\sideset{}{'}\sum\limits_{n\leqq x}f(n)$ & $=\sum\limits_{1\leqq n\leqq x}f(n)$, except that if $x$ is an integer,\\
&  then the term $f(x)$ is to be replaced by $\frac{1}{2}f(x)$.\\
$\sideset{}{'}\sum\limits_{a\leqq n\leqq b}f(n)$ & A sum with similar conventions as above\\
& if $a$ or $b$ is an integer.\\
$D(x)$ & $=\sideset{}{'}\sum\limits_{n\leqq x}d(n)$.\\
$A(x)$ & $=\sideset{}{'}\sum\limits_{n\leqq x}a(n)$.\\
$D(x,\alpha)$ & $=\sideset{}{'}\sum\limits_{n\leqq x}d(n)e(n\alpha)$.\\
$A(,\alpha)$ & $=\sideset{}{'}\sum\limits_{n\leqq x}a(n)e(n\alpha)$.\\
$D_a(x)$ & $=\frac{1}{a!}\sideset{}{'}\sum\limits_{n\leqq x}d(n)\;(x-n)^a$.\\
$A_a(x), D_a(x,\alpha), A_a(x,\alpha)$ & are analogously defined.\\
$\epsilon$ & An arbitrarily small positive constant.\\
$A$ & A constant, not necessarily the same at\\
& each occurrence.\\
$C^n[a,b]$ & The class of functions having a continuous\\
& nth derivative in the interval $[a,b]$. 
\end{longtable}}

The symbols $0(), \ll$, and $\gg$ are used in their standard
meaning. Also, $f\asymp g$ means that $f$ and $g$ are of equal order
of magnitude, \ie that $1\ll f/g\ll 1$. The constants implied by these
notations depend at most on $\in$.  
