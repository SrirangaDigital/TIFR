\chapter*{Introduction}

\addcontentsline{toc}{chapter}{Introduction}

\markboth{Introduction}{Introduction}


ONE\pageoriginale OF THE basic devices (usually called ``process B'';
see \cite{key13}, $\S$ 2.3) in van der Corput's method is to transform
an exponential sum into a new shape by an application of van der
Corput's lemma and the saddle-point method. An exponential sum 
\begin{equation}
\sum\limits_{a<n\leq b} e(f(n)),\tag{0.1}\label{int:eq0.1}
\end{equation}
where $f\epsilon C^2[a,b], f''(x)<0$ in $[a,b],f'(b)=\alpha$, and
$f'(a)=\beta$, is first written, by use of van der Corput's lemma, as 
\begin{equation}
\sum\limits_{\alpha -\eta<n<\beta +\eta} \int\limits_a^b e(f(x)-nx)\,dx
+0 (\log(\beta -\alpha +2)),\tag{0.2}\label{int:eq0.2}
\end{equation}
where $\eta\epsilon$ \eqref{int:eq0.1} is a fixed number. The exponential
integrals here are then evaluated approximately by the saddle-point
method in terms of the saddle points $x_n\epsilon(a,b)$ satisfying
$f'(x_n)=n$. 

If the sum \eqref{int:eq0.1} is represented as a series by Poisson's
summation formula, then the sum in \eqref{int:eq0.2} can be
interpreted as the ``interesting'' part of this series, consisting of
those integrals which have a saddle point in $(a,b)$, or at least in a
slightly wider interval. 

The same argument applies to exponential sums of the type 
\begin{equation}
\sum\limits_{a\leq n\leq b} d(n)g(n)e(f(n))\tag{0.3}\label{int:eq0.3}
\end{equation}
as well. The role of van der Corput's lemma or Poisson's summation
formula is now played by Voronoi's summation formula 
\begin{align*}
\sideset{}{'}\sum\limits_{a\leq n\leq b}d(n)f(n) &= \int\limits_a^b(\log
x+2\gamma)f(x)\,dx+\sum\limits_{n=1}^\infty d(n)\int\limits_a^b
f(x)\alpha(nx)\,dx,\\
\alpha (x) & =4K_\circ(4\pi x^{1/2})-2\pi Y_\circ(4\pi x^{1/2}).\notag
\tag{0.4}\label{int:eq0.4} 
\end{align*}\pageoriginale

The well-known asymptotic formulae for the Bessel functions $K_\circ$
and $\gamma_\circ$ imply an approximation for $\alpha(nx)$ in terms of
trigonometric functions, and, when the corresponding exponential
integrals in  \eqref{int:eq0.4} with $g(x)e(f(x))$ in place of
$f(x)$-are treated by the saddle-point method, a certain exponential
sum involving $d(n)$ can be singled out, the contribution of the other
terms of the series \eqref{int:eq0.4} being estimated as an error
term. The leading integral normally represents the expected value of
the sum in question.

As a technical device, it may be helpful to provide the sum
\eqref{int:eq0.3} with suitable smooth weights $\eta(n)$ which do not
affect the sum too much but which make the series in Voronoi formula
for the sum 
$$
\sideset{}{'}\sum\limits_{a\leq n\leq b}\eta(n)d(n)g(n)e(f(n))
$$
absolutely convergent.

Another device, at first sight nothing but a triviality, consists of
replacing $f(n)$ in \eqref{int:eq0.3} by $f(n)+rn$, where $r$ is an
integer to be chosen suitably, namely so as to make the function
$f'(x)+r$ small in $[a,b]$. This formal modification does not, of
course, affect the sum itself in any way, but the outcome of applying
Vornoi's summation formula and the saddle-point method takes quite a
new shape.

The last-mentioned argument appeared for the first time \cite{key16},
where a transformation formula for the Dirichlet polynomial 
\begin{equation}
S(M_1,M_2)=\sum\limits_{M_1\leq m\leq M_2}d(m)
m^{-1/2-it}\tag{0.5}\label{int:eq0.5} 
\end{equation}
was\pageoriginale derived. An interesting resemblance between the
resulting expression for $S(M_1,M_2)$ and the well-known formula of
F.V. Atkinson \cite{key2} for the error term $E(T)$ in the asymptotic
formula 
\begin{equation}
\int\limits_0^T|\zeta\left(\frac{1}{2}+it\right)|^2\,dt
=(\log(T/2\pi)+2\gamma-1)T+E(T) \tag{0.6} \label{int:eq0.6}
\end{equation}
was clearly visible, especially in the case $r=1$. This phenomenon
has, in fact, a natural explanation. For differentiation of
\eqref{int:eq0.6} with respect to $T$, ignoring the error term
$o(\log^2T)$ in Atkinson's formula for $E(T)$, yields heuristically an
expression for $|\zeta(\frac{1}{2}+it)|^2$, which can be indeed
verified, up to a certain error, if 
$$
|\zeta\left(\frac{1}{2}+it\right)|^2=\zeta^2\left(\frac{1}{2}+it\right)\chi^{-1}
(\frac{1}{2}+it)
$$
is suitably rewritten invoking the approximate functional equation for
$\zeta^2(s)$ and the transformation formula for $S(M_1,M_2)$ (for
details, see Theorem 2 in \cite{key16}).

The method of \cite{key16} also works, with minor modifications, if
the coefficients $d(m)$ in \eqref{int:eq0.5} are replaced by the
Fourier coefficients $a(m)$ of a cusp form of weight $\kappa$ for the
full modular group; the Dirichlet polynomial is now considered on the
critical line $\sigma =\kappa/2$ of the Dirichlet series
$$
\varphi(s)=\sum\limits_{n=1}^\infty a(n)n^{-s}.
$$

This analogy between $d(m)$ and $a(m)$ will prevail
throughout these notes, and in order to avoid repetitions, we are not
going to give details of the proofs in both cases. As we shall see,
the method could be generalized to other cases, related to Dirichlet
series satisfying a functional\pageoriginale equation of a suitable
type. But we are leaving these topics aside here, for those two cases
mentioned above seem to be already representative enough. 

The transformation formula of \cite{key16} has found an application
in the proof of the mean twelfth power estimate 
\begin{equation}
\int\limits_0^T|\zeta\left(\frac{1}{2}+it\right)|^{12}\,dt\ll
T^2\log^{17}T\tag{0.7}\label{int:eq0.7} 
\end{equation}
of D.R. Heath-Brown \cite{key11}. The original proof by Heath-Brown
was based on Atkinson's formula. The details of the alternative
approach can be found in \cite{key13}, \S ~ 8.3.

The formula of \cite{key16} is useful only if the Dirichlet polynomial
to be transformed is fairly short and the numbers $t(2\pi M_i)^{-1}$
lie near to an integer $r$. But in applications to Dirichlet series it
is desirable to be able to deal with ``long'' sums as well. It is not
advisable to transform such a sum by a single formula; but a more
practical representation will be obtained if the sum is first split up
into segments which are individually transformed using an optimally
chosen value of $r$ for each of them. The set of possible values of
$r$ can be extended from the integers to the rational numbers if a
summation formula of the Voronoi type, to be given in \S~
\ref{chap1:sec1.9}, for for sums
$$
\sideset{}{'}\sum\limits_{a\leq n\leq b} b(n)e_k(hn)f(n),b(n)=d(n)
\quad\text{or}\quad a(n),
$$
is applied. The transformation formula for Dirichlet polynomiala are
deduced in \S~ \ref{chap4:sec4.1} as consequences of the theorems of
Chapter \ref{chap3} concerning the\pageoriginale transformation of
more general exponential sums 
\begin{equation}
\sum\limits_{M_1\leq m\leq M_2}b(m)g(m)e(f(m))\tag{0.8}\label{int:eq0.8}
\end{equation}
or their smoothed versions.

An interesting problem is estimating long exponential sums of the type
\eqref{int:eq0.8}. A result of this kind will be given in \S~
\ref{chap4:sec4.3}, but only under rather restrictive assumptions on
the function $f$, for we have to suppose that $f'(x)\approx
Bx^\alpha$. It is of course possible that comparable, or at least
nontrivial, estimates can be obtained in concrete cases without this
assumption, making use of the special properties of the function $f$.

In view of the analogy between $\zeta^2(s)$ and $\varphi(s)$, a mean
value result corresponding to Heath-Brown's estimate \eqref{int:eq0.7}
should be an estimate for the sixth moment of
$\varphi(\kappa/2+it)$. However, the proofs of \eqref{int:eq0.7} given
in \cite{key11} and \cite{key13} utilize special properties of the
function $\zeta^2(s)$ and cannot be immediately carried over to
$\varphi(s)$. An alternative approach will be presented in \S~
\ref{chap4:sec4.4}, giving not only \eqref{int:eq0.7} (up to the
logarithmic factor), but also its analogue 
\begin{equation}
\int\limits_0^T|\varphi(\kappa/2+it)|^6\,dt \ll
T^{2+\epsilon}.\tag{0.9}\label{int:eq0.9} 
\end{equation}

This implies the estimate
\begin{equation}
|\varphi(\kappa/2+it)|\ll t^{1/3+\epsilon}\quad\text{for}\quad
t\geq 1,\tag{0.10}\label{int:eq0.10} 
\end{equation}
which is not new but neverthless essentially the best known
presently. In fact, \eqref{int:eq0.10} is a corollary of the mean
value theorem 
\begin{equation}
\int\limits_0^T|\varphi(\kappa/2+it)|^2\,dt = (C_\circ\log T+C_1)T +o
((T\log T)^{2/3})\tag{0.11}\label{int:eq0.11}
\end{equation}
of\pageoriginale A. Good \cite{key9}. The estimate \eqref{int:eq0.10}
can also proved directly in a relatively simple way, as will be shown
in \S ~ \ref{chap4:sec4.2}.

The plan of these notes is as follows. Chapters \ref{chap1} and
\ref{chap2} contain the necessary tools - summation formulae of the
Voronoi type and theorems on exponential integrals - which are
combined in Chapter \ref{chap3} to yield general transformation
formulae for exponential sums involving $d(n)$ or $a(n)$. Chapter
\ref{chap4}, the contents of which were briefly outlined above, is
devoted to specializations and applications of the results of the
preceding chapter. Most of the material in Chapters \ref{chap3} and
\ref{chap4} is new and appears here the first time in print.

An attempt is made to keep the presentation selfcontained, with an
adequate amount of detail. The necessary prerequisites include, beside
standard complex function theory, hardly anything but familiarity with
some well-known properties of the following functions: the Riemann and
Hurwitz zeta functions, the gamma function, Bessel functions, and cusp
forms together with their associated Dirichlet series. The method of
van der Corput is occasionally used, but only in its simplest form. 

As we pointed out, the theory of transformations of exponential sums
to be presented in these notes can be viewed as a continuation or
extension of some fundamental ideas underlying van der Coroput's
method. A similarity though admittedly of a more formal nature can
also be found with the circle method and the large sieve method,
namely a judicious choice of a system of rational
numbers\pageoriginale at the outset. In short, our principal goal is
to analyse what can be said about Dirichlet series and related
Dirichlet polynomials or exponential sums by appealing only to the
functional equation of the allied Dirichlet series involving the
exponential factors $e(nr)$ and making only minimal use of the actual
structure or properties of the individual coefficients of the
Dirichlet series in question. 
