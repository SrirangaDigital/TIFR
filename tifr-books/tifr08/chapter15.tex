\chapter{Lecture 15}\label{chap15}% LECTURE 15

\section{Proof of the Friedrichs - Lax - Nirenberg theorem}\label{chap15:sec1}

To\pageoriginale prove the Friedrichs - Lax - Nirenberg theorem, we need three lemmas:
\begin{Lemma}\label{chap15:sec1:lem1}% lemma 1
 If $u_o$ is of order $i$ in $R_1$ and if $\tilde{D}(s)_{u_o}$ is of
 order $j$ in $R$, for all $s$ with $|s| \le i$, then $u_o$ is of
 order $i + j$ in $R_1$. If $u_o$ is of order $i + j$ in $R$, then
 $\tilde{D}(s)_{u_o}$ is of order $j$ for $|s|\le i$. 
\end{Lemma}

\begin{Lemma}\label{chap15:sec1:lem2}% lemma 2
 Let $R_1$ be a relatively compact subdomain of $R$ and let $u_o \in
 L_2 (R_1)$. Then for any positive integer $s$ 
 $$
 (I + (-\triangle)^s)h = u_o \quad (\triangle~~ \text{is the Laplacian})
 $$
 has weak solution of order $2s$ in $R_1$.
\end{Lemma}

\begin{Lemma}\label{chap15:sec1:lem3}% lemma 3
 Let $u_o \in L_2(R_1)$ be of order $n$ in $R_1$ and 
 \begin{gather*}
  | (L^* \varphi, u_o) | \le \const || \varphi ||_{n-1},
  ~\text{ for all}~ \varphi \in \mathscr{D}^\infty (R_1) \\ 
  \left[(\varphi,\psi)_o = \int\limits_{R_1} ~ \varphi \bar{\psi} ~ d ~ x
   ; || ~ \varphi ~ ||^2_k = \sum_{|s|\le k} ~ \int\limits_{R_1} |
   D^{(s)}\varphi |^2 ~dx \right] 
 \end{gather*}
\end{Lemma}

Then $u_o$ is of order $n + 1$ in $R_1$.

Assuming these lemmas for a moment, we shall give a 
\textit{Proof of the Friedrichs - Lax - Nirenberg theorem}

\begin{first step}% first step
 If $u_o \in L_2 (R_1)$ is of order $n$ in $R_1$ and satisfies $|
 (L^* \varphi, u_o)| \le \const || \varphi ||_{n-j}$ for all
 $\varphi \in \mathscr{D}^\infty(R_1)$, then $u_o$ is of order $n+j$
 in $R_1$. This is proved by induction on $j$. The result is true for
 $j = 1$ (Lemma \ref{chap15:sec1:lem3}). Let us assume that $j > 1$
 and that the result is true for $j - 1$ Suppose 
 $$
 | (L^* \varphi, u_o) | \le \const || \varphi ||_{n-j};
 $$
 since\pageoriginale $|| \varphi ||_{n-j} \le || \varphi ||_{n-(j-1)}, u_o$ is of
 order $(n+j-1)$ in $R_1$ by the inductive assumption. For any first
 order derivation $D$, we have $| (L^*D\varphi,u_o)| \const\break || D
 \varphi ||_{n-j} \le \const || \varphi ||_{n-j+1}$. Since $u_o$ is
 of order $n+1$, we have 
 \begin{align*}
  (L^* D\varphi, u_o) & = \sum_{|\varrho |, |\sigma | \le n} 
  \left((-1)^{| \rho | + | \sigma |} D^{(\sigma)} ~ a^{\rho,\sigma}
  ~ D^{(\varrho)} ~ D ~ \phi, u_o \right)\\ 
  & = \sum \left((-1)^{|\varrho |} D ~ D^{\varrho}\varphi,
  a^{\varrho,\sigma} ~ \tilde{D}^{\sigma} ~ u_o\right)\\ 
  & = \sum \left((-1)^{|\varrho | + 1} ~ D^{\varrho} \varphi,
  D(a^{\varrho,\sigma}, \tilde{D}^\sigma u_o)\right)\\ 
  & = \sum \left((-1)^{|\varrho | + 1} ~ D^{\varrho} \varphi, (D ~
  a^{\varrho, \sigma}) \tilde{D}^{\sigma} ~ u_o\right)\\ 
  & \quad + \sum \left((-1)^{|\varrho| + 1} ~ D^{\varrho} \varphi,
  a^{\varrho,\sigma} ~ \tilde{D}^{\sigma}\tilde{D} ~ u_o\right)\\ 
  & = \sum \left((-1)^{|\varrho | + 1} ~ D^{\varrho} \varphi, (D ~
  a^{\varrho,\sigma}) ~ \tilde{D}^\sigma ~ u_o\right) - (L^* ~
  \varphi, \tilde{D} ~ u_o). 
 \end{align*}
\end{first step}

Since $u_o$ is of order $\underline{(n + j -1)}$ we see by partial
integration that 
\begin{align*}
 | (L^* \varphi, \tilde{D} ~ u_o) | & \le | (L^* D \varphi, u_o) | +
 \const|| \varphi ||_{2n} - (n+j-1)\\ 
 & \le \const || \varphi ||_{n-(j-1)} 
\end{align*}

By Lemma \ref{chap15:sec1:lem1}, $\tilde{D} ~u_o$ is of order $\ge ~ n+ j -1 \ge n+ j -2
\ge n $ (as $j\ge 2$). Hence by the induction assumption
$\tilde{D}u_o$ is of order $n + j - 1$. So, by lemma
\ref{chap15:sec1:lem1} $u_o$ is of order $n+j$. 

\begin{second step}[Friedrich's theorem ]% first step 
 Let $u_o \in L_2(R_1)$ be a weak solution of $L u = f$ and $f$ be
 order $p$ in $R_1$. If $u_o$ is of order $n$ in $R_1$, then $u_o$ is
 of order $2 n+p$ in $R_1$. 
\end{second step}

\begin{proof}
 This holds for $p = 0$. For, from $(L^* \varphi, u_o)_o =
 (\varphi,f)_o$, we have 
 $$
 | (L^* \varphi, u_o)_o | \le \const || \varphi ||_o = \const ||
 \varphi ||_{n-n}. 
 $$
 So,\pageoriginale by the first step $u_o$ is of order $n + n =
 2n$. Suppose $p =  1$. We have, as above, 
 \begin{align*}
  (L^*\varphi,\tilde{D} ~u_o)_o & = - (D ~ L^* \varphi, u_o) =
  (-1)^{|\varrho|+|\sigma | + 1} (D^\sigma ~ D ~ a^{\varrho,\sigma}
  ~ D^{\varrho}\varphi, u_o)_o\\ 
  & = (-1)^{|\rho|+ | \sigma | +1 } ~ (D^{(\sigma)} ~
  a^{\varrho,\sigma} ~ D^\varrho ~ D ~ \varphi, u_o)\\ 
  & + (-1)^{|\varrho| + | \sigma | + 1} ~ (D^\sigma (D ~
  a^{\varrho,\sigma}) D^\varrho \varphi, u_o)_o\\ 
  & = (L^* ~ D \varphi, u_o)_o + (\varphi, \tilde{L}' u_o),
 \end{align*}
 where $L'$ is a differential operator of degree $2n$. 
 \begin{align*}
  (L^* \varphi, \tilde{D} u_o) & = (D \varphi, f)_o + (\varphi,
  \tilde{L}' u_o)\\ 
  & = -(\varphi, \tilde{D} f) + (\varphi, \tilde{L}' u_o)
 \end{align*}
 (since $f$ is of order $1$ at least; the case $p = 0$ is already
 proved). Thus 
 $$
 | (L^* \varphi, D ~ u_o)_o | \le \const || \varphi ||_o = \const ||
 \varphi ||_{n-n} 
 $$
 and $\tilde{D} ~ u_o$ is of order $2n - 1 \ge n$. So by the first
 step, $\tilde{D} ~ u_o$ is of order $n + n = 2n$. By Lemma
 \ref{chap15:sec1:lem1}, $u_o$
 is of order $2n + 1$. For $ p > 1$, we may repeat the argument. 
\end{proof}

\begin{third step}% third step
 Let $u_o \in L_2(R_1)$ be a weak solution of $L ~ u = f$ and $f$ be
 of order $p$ in $R_1$. Then $u_o$ is of order $2n+p$ in $R_1$. 
\end{third step}

\begin{proof}% proof
 Let $h_o$ of order $2n$ be a weak solution of 
 $$
 (I + (-A)^n)h = u_o.
 $$
 $h_o$ exists by Lemma \ref{chap15:sec1:lem2}. Then $h_o$ of order $2n$ is a weak
 solution of 
 $$
 L(I + (-\triangle)^n)h = f;
 $$
 $L(I + (-\triangle)^n)$\pageoriginale is an elliptic operator of order $4n$. $f$
 being of order $p,h_o$ is of order $4n + p$, by the second
 step. Hence, by Lemma \ref{chap15:sec1:lem1}, 
 $$
 u_o = (I + (-\triangle)^n)h_o
 $$
 is of order $4n + p - 2n = 2n + p$.
\end{proof}
