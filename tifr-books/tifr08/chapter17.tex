\chapter{Lecture 17}\label{chap17} % LECTURE 17

\section{Proof of Lemma 2}\label{chap17:sec1}

 We\pageoriginale define
 \begin{equation*}
  \tilde{u}_o ~ (x) =
  \begin{cases}
   u_o(x) &\text{ if } x \in R_1\\
   \quad 0 & \text{ if } x \in E^m - R_1.
  \end{cases}
 \end{equation*}

Let $U_o(y) = (\mathscr{F}\tilde{u}_o)(y)$. Then
$$
h_o(x) =\mathscr{F}^{-1} \frac{U_o ~ (Y)}{1+(\sum_{j=1}^m ~ (2 \pi y_j)^2)^s}
\quad (x) 
$$
satisfies the conditions of the lemma. In the first place,  
$$
h_o(x) = \int\limits_{|y|\le n} ~ \frac{U_o ~ (y)}{1+(\sum_{j=1}^m ~
 (2 \pi y_j)^2)^s} \exp ~ (2\pi\sqrt{-1} ~ yx)dy 
$$
is $C^\infty(E^m)$. For, since $U_o(y) \in L_2 (E^m)$, 
$$
\frac{U_o ~ (y)}{1+(\sum_{j=1}^m ~ (2 \pi y_j)^2)^s} \prod_{j=1}^{m}
(2\pi\sqrt{-1}~ y_j)^{k_j} \exp ~ (2\pi\sqrt{-1} ~ yx) 
$$
is, for any set of integers $k_j \ge 0$, integrable over $| y | \le n$
and majorised uniformly in $x$ by a summable function (in $y$). So 
{\fontsize{10pt}{12pt}\selectfont
$$
\frac{\partial^{k_1 + \cdots + k_m}}{{ \partial x_1}^{k_1} \cdots
 {\partial x_n}^{k_n}} ~ h_n(x) = \int\limits_{| y | \le n} \left\{
\frac{U_o(y)\prod\limits_{j=1}^{m} ~ (2\pi\sqrt{-1}~ y_j)^{k_j}}{1+
 (\sum_{j=1} ~ (2\pi ~y_i)^2)^s}\right\} ~ \exp (2\pi\sqrt{-1} ~
yx)dy 
$$}\relax

Moreover, for $| k | \le 2s$, the function under the curly brackets
$\{ \cdots \}$ is in $L_2 ~ (E^m)$, so that for $| k | \le 2s, D^{(k)}
~ h_n ~ (E^x)$. converges in $L_2(E^m)$ Therefore $h_o(x)$ is of order
$2s$ in $E^m$. 

Next for any $\varphi \in \mathscr{D}^\infty (E^m)$, we have, by
partial integration, 
\begin{align*}
\int\limits_{E^m} ~ (I + (-\triangle)^s) ~ \varphi(x) ~ h_o (x) ~ dx &
=\lim_{n \rightarrow \infty} \int\limits_{E^m} ~ (I + (-\triangle)^s)
\varphi(x) ~ h_n(x) ~ dx\\ 
 & = \lim_{n \rightarrow \infty} ~ \int\limits_{E^m} ~ \varphi(x) ~
 (I + (-\triangle)^s) ~ h_n (x) ~ dx\\ 
 & = \int\limits_{E^m} ~ \varphi (x) ~ (\mathscr{F}^{-1} ~ U_o) ~ (x) ~ dx
\end{align*}

This\pageoriginale proves that $h_o$ is a weak solution in $E^m$ of $(I +
(-\triangle)^s) ~ h = \tilde{u}_o = \mathscr{F}^{-1} U_o$. Thus $h_o$
is a weak solution in $R_1$ of $(I + (-\triangle)^s)h = u_o$. 

\section{Proof of Lemma 1}\label{chap17:sec2}

 In the proof of Lemma \ref{chap15:sec1:lem1} we have to make use of
 the notion of 
 ``regularisation'' or ``mollifiers''. Let $j(x) \in C^\infty(E^m)$ such
 that 
 \begin{enumerate}[\rm i)]
 \item $j(x) \ge 0$,
 \item $j(x) = 0$, for $| x | \ge 1$
 \item $\int\limits_{E^m} ~ j(x) ~ dx = 1$.
 \end{enumerate}

Let for $\varepsilon > 0$ 
$$
j_\varepsilon (x) = \varepsilon^{-n} ~ j(x/\varepsilon)
$$

We have then
\begin{enumerate}[i)]
\item $j_\varepsilon (x) \ge 0$,
\item $j_\varepsilon(x) = 0$, for $| x | \ge \varepsilon$
\item $\int\limits_{E^m} ~ j_\varepsilon(x) ~ dx = 1$.
\end{enumerate}

Let $R_1$ be a relatively compact subdomain of $R \subset E^m$ and
$u(x) \in L_2 (R_1)$. Let $R_2$ be a subdomain relatively compact in
$R_1$. Let $d > 0$ be the distance between $R_2$ and the boundary of
$R_1$. Let $\varepsilon > 0$ be such that $\varepsilon < d$. For $x
\in R_2$, define 
$$
(J_\varepsilon ~ u) (x) = \int\limits_{R_1} ~ j_\varepsilon (x-y) u' (y) ~ dy.
$$
$((J_\varepsilon ~u) (x)$\pageoriginale is called regularisation of $u(x)$ and the
operators $J_\varepsilon$ are called mollifiers). Let 
$$
 ||v||^2_{o,R_i} = \int\limits_{R_i} |v |^2 dx.
$$

We then have
\begin{enumerate}[i)]
\item $|| ~J_\varepsilon ~ u ~||_{o,R_2} \le ~ || ~ u ~ ||_{o,R_1}$
\item $\lim_{\varepsilon \downarrow o} || ~ J_\varepsilon ~ u - u ~ ||_{o,R_2} = 0$
\item $(J_\varepsilon ~u) (x)$ is $C^\infty$ in $R_2$ and if $h$ is of order $i$ in $R_1$, 
\end{enumerate}
then
$$
D^{(s)} ~ (J_\varepsilon ~ u)(x) = (J_\varepsilon ~ \tilde{D}^s ~
u)(x) \text{ for } | s | \le i 
$$
in $R_2$.

\noindent \textbf{Proof of (iii):} 
 We have, for each derivation $D^{(s)}$,
 $$
 (D^{(s)}_x ~ J_\varepsilon ~ u) (x) = \int\limits_{R_1} ~ D^{(s)}_x ~
 j_\varepsilon (x - y) ~ u(y) ~ dy. 
 $$

Suppose $u$ is of order $i$ in $R$. We have then, for $| s | \le i$,
by partial integration, 
\begin{align*}
 \int\limits_{R_1} ~ D^{(s)}_x ~ j_\varepsilon ~(x-y) ~ u(y) & =
 \int\limits_{R_1} ~ (-1)^{|s|} \{D^{(s)}_y ~ j_\varepsilon ~(x-y)\}
 ~ u(y) ~ dy\\ 
 & \int\limits_{R_1} ~ j_\varepsilon ~ (x-y) ~ \tilde{D}^{(s)} ~ u(y) ~ dy
\end{align*}
since, for each $x \in R_2, j_\varepsilon (x-y)$ considered as a
function of $y$, has compact support in $R_1$. 

\noindent \textbf{Proof of (ii):} 
 We have, for $x \in R_2, \int\limits_{R_1} ~ j_\varepsilon ~ (x-y) ~
 dy = 1$. Hence 
 $$
 (J_\varepsilon ~ u)(x) -u(x) = \int\limits_{R_1} ~ j_\varepsilon ~
 (x-y) ~ (u(y) - u(x)) ~ dy. 
 $$


By Schwarz's inequality
\begin{align*}
 & \int\limits_{R_2} ~ | ~ (J_\varepsilon ~ u) (x) - u(x) ~ |^2 ~ dx\\
 & \le \int\limits_{R_2} ~ dx ~ \left[ \int\limits_{R_1} ~ j_\varepsilon ~
  (x-y) ~ dy ~ \int\limits_{R_1} ~ j_\varepsilon (x-y) | ~ u(y) -
  u(x) |^2 ~ dy\right]\\ 
 & = \int\limits_{R_2} ~ dx ~ \int\limits_{R_1} ~j_\varepsilon ~(x-y)
 ~ | ~ u(y) - u(x) |^2 ~ dy\\ 
 & \le \int\limits_{R_2} ~ dx ~ \int\limits_{| z | < \varepsilon} ~
 j_{\varepsilon} ~ (z) ~ | u(x-z) - u(x) |^2 ~ dz\\ 
 & = \int\limits_{| z | < \varepsilon} ~ j_\varepsilon ~ (z)
 \left\{\int\limits_{R_2} ~ |~ u(x-z) - u(x) |^2 ~ dx \right\} ~ dz 
\end{align*}

Since\pageoriginale $\int\limits_{R_2} ~ | ~ u (x-z) - u(x) |^2 ~ dx $ tends to zero
as $| z | \rightarrow 0, (ii)$ is proved. 

\noindent \textbf{Proof of (i):} 
 We have, by calculations similar to the above calculations,
 \begin{align*}
  || ~ J_\varepsilon ~ u ~ ||^2_{o,R_2} & = \int\limits_{R_2} ~ dx ~
  \int\limits_{R_1} ~ j_\varepsilon ~ (x-y) ~ | ~ u(y) ~ |^2 ~ dy\\ 
  & \le \int\limits_{|z|< \varepsilon} ~ j_\varepsilon (z) \left\{
  \int\limits_{R_2} ~ | u(x-z) ~ |^2 ~ dx \right\} ~ dz\\ 
  & \le \int\limits_{|z|< \varepsilon} ~ j_\varepsilon (z) \left\{
  \int\limits_{R_2} ~ | ~ u(x) ~ |^2 ~ dx \right\} ~ dz\\ 
  & = || ~ u ~ ||^2_{o,R_1}.
 \end{align*}

\setcounter{Prooff}{0}
\begin{Prooff}%Prf of lem 1
 Let $u$ be of order in $R_1$ and let $\tilde{D}^{(s)} ~ u$ be of
 order $j$ in $R_1$ for each $s$ with $| s | \le i$. Then for $| t |
 \le j$, 
 $$
 D^{(t)} ~ D^{(s)} ~ J_\varepsilon ~ u = D^{(t)} ~ J_\varepsilon ~
 \tilde{D}^{(s)} ~ u = J_\varepsilon ~ \tilde{D}^t ~ \tilde{D}^{(s)}
 ~ u\, ( | s | \le i) 
 $$
 by (iii). Hence by (ii), $u$ is of order $i + j$, in $R_1$.
\end{Prooff}

Next let $u$ be of order $i + j$ in $R_1$. Since
$$
D^{(t)} ~ J_\varepsilon ~ \tilde{D}^{(s)} ~ u = J_\varepsilon ~
\tilde{D}^{(t)} ~ \tilde{D}^{(s)} ~ u ~ (|t|\le j, |s| \le i) 
$$
we see by $(ii)$ that $\tilde{D}^{(s)} u $ is of order $j$ in $R_1$.
