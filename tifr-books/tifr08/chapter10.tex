\chapter{Lecture 10}\label{chap10} % lect 10

\section{Supplementary results}\label{chap10:sec1}

We\pageoriginale shall now prove some results which supplement our earlier results;
these will be useful in applications. 
\begin{theorem*}%Thm
 \begin{enumerate}[\rm 1.]
 \item For a semi-group $\{T_t\}$ the infinitesimal generator $A$ may
  be defined by 
  $$
  w-\lim_{h \downarrow o} \frac{T_h-I}{h}x.
  $$ 
  i.e., if $\tilde{A}$ is the operator with $\mathscr{D}(\tilde{A})=
  \left\{x | w-\lim\limits_{h \downarrow o}\dfrac{T_h-I}{h}x
  \text{ exists } \right\}$ and $\tilde{A}x= w-\lim\limits_{h
   \downarrow o} \dfrac{T_h-I}{h}x$, then $\tilde{A}=A$. 
 \item If $\{ T_t\}_{t \ge o}$ is a family of linear operators on a
  Banach space $X$ such that $T_{t+s}= T_t T_s, T_o=I$ and $|| T_t
  || \le e^{\beta t}$, $\beta \ge 0$ then the following two
  conditions are equivalent: 
  \begin{enumerate}[\rm (i)]
  \item strong continuity of $T_t$, i.e., $w-\lim\limits_{t \to t_o}$ $T_t x=
   T_{t_o}x$ for each $t_o \ge 0$ and $x \in X$. 
  \item weak right continuity at $t=0$, i.e., $w-\lim\limits_{h \downarrow
   o}$ $T_t x=x$, for $x \in X$. 
  \end{enumerate}
 \item The infinitesimal generator is a semi-group is a closed operator.
 \end{enumerate}
\end{theorem*}

\begin{Proof}
 It is evident that $\tilde{A}$ is an extension of $A$. We shall show
 that $A$ is an extension of $\tilde{A}$, i.e., if $x \in
 \mathscr{D}(\tilde{A})$, then $x \in \mathscr{D}(A)$ and $Ax=
 \tilde{A}x$. If $x \in \mathscr{D}(A)$, 
 $$ 
 w-\lim_{h \downarrow o} \frac{ T_{t+h}- T_t}{h} x= T_t \left[w-\lim_{h
   \downarrow o} \frac{T_h -I}{h}x\right] = T_t A x. 
 $$
\end{Proof}

(For, if $w-\lim_{h \downarrow o} x_h =y$, and $T$ is a linear
operator, then $w-\lim\limits_{h \downarrow o} T {x_h}= T_y$; in fact, if $f
\in X^*$, $\hat{f}(y)=f(TY)$ is a linear functional\pageoriginale on $X$, as
$|\hat{f}(y) | \le ||f || \, ||T_y || \le || f ||\, ||T ||\, ||y ||$, and
$f(Ty)-f(Tx_h)= \hat{f}y- \hat{f}x_h \to 0$ as $h \downarrow 0$). So,
if $x \in \mathscr{D} (\tilde{A})$, $f(T_t x)$ has right derivative
$\dfrac{d^+}{dt} f(T_t x) =f (T_t \tilde{A}x)~(t \ge 0)$, which is
continuous for $t \ge 0$, by the strong continuity of $T_t$. Therefore
the derivative $\dfrac{d}{dt} f(T_t x)$ exists for each $t \ge 0$ and
is continuous. 

So 
\begin{align*}
 f(T_t x-x) &= f(T_t x) -f(x)= \int^t_o f(T_s \tilde{A}x ) ds \\
 &= f \left( \int^t_o T_s \tilde{A} \,x ~ds\right), \text{ for each } f
 \in X^*. 
\end{align*}

Continuously, by the Hahn-Banach theorem,
$$
T_t x-x= \int^t_o T_s \tilde{A}x ~ds.
$$

Since $T_t$ is strongly continuous in $t$ it follows that 
$$
s-\lim_{t \downarrow o} \frac{T_t-I}{t} x= T_o \tilde{A} x= \tilde{A} x.
$$

Thus if $x \in \mathscr{D}(\tilde{A})$, then $x \in \mathscr{D}(A)$
and $\tilde{A} x=A x$. 

\begin{Proof} % prof 2 
 Evidently (i) implies (ii). $T_o$ prove that (ii) implies (i),
 let $x_o$ be a fixed element of $X$. We shall show that
 $w-\lim\limits_{t \downarrow t_o} T_t x_o = T_{t_o} x_o$ for each $t
 \ge 0$. Consider the function $x(t)= T_t x_o$. For $t_o \ge 0$,
 $x(t)$ is right continuous at $t_o$, as $w-\lim\limits_{t \downarrow
  t_o} T_t x_o= w-\lim\limits_{h \downarrow o} T_h T_{t_o}
 x_o$. $x(t)$ has the following three properties: 
 \begin{enumerate}[\rm (a)]
 \item $x(t)$ is weakly measurable, i.e., for any $f \in X^*$,
  $f(x(t))$ is measurable (since a right continuous numerical
  function is measurable). 
 \item $||x(t) ||$ is bounded in any bounded interval of $t$.
 \item there\pageoriginale exists a countable set $M= \{ x_n\}$ such that $x(t)\,(t
  \ge 0)$ is contained in the closure of $M$. 
 \end{enumerate}
\end{Proof}

To prove $(c)$, let $\{t_k\}$ be the totality of positive rational
numbers. Consider finite linear combinations $\sum \alpha_kx(t_k)$
where $\alpha_k$ are rational numbers if $X$ is real and if $X$ is
complex $\alpha_k =a_k+ ib_k$ with $a_k$ and $b_k$ rational. These
elements form a countable set $M=\{x_n\}$. The closure of $M,
\bar{M}$, contains $x(t)$, for each $t \ge 0$. 

For, if not, let $t_0 \ge 0$ be a number such that $x(t_o)$ does not
belong to $\bar{M}. \bar{M}$ is a closed linear subspace of $X$. By
the Hahn-Banach theorem, there exists a linear functional $f_o$ on
$X$ such that $f_o(x(t_o))\neq 0$ and $f_o(x)=0$ for $x' \in
\bar{M}$. Take a sequence $t'_k \downarrow t_o$ ( $t'_k$ positive
rational). By the weak right continuity of $x(t)$ at $t_o$, 
$$
f_o(x(t'_k)) \to f_o (x(t_o)).
$$

But $f_o (x(t'_k))=0$ and $f_o(x(t_o)) \neq 0$. We have thus arrived
at a contradiction. 

We next prove a result, due to $N$. Dunford (On one parameter group of
linear transformations, Ann, of Math., $39(1938), 569-573)$, according
of which the properties $(a)$, (b) and $(c)$ listed above imply the
strong continuity of $x(t)$. First we show that $|| x(t) ||$ is
measurable in $t$. Let $f_n \in X^*$ be such that $f_n (x_n)= || x_n
||$ and $|| f_n ||=1$. Let $f(t)= \sup\limits_{n \ge 1} f_n
(x(t))$; since each $f_n(x(t))$ is measurable, $f(t)$ is measurable in
$t$. But $|| x(t) ||= f(t)$; for 
\begin{align*}
 f(t) \ge |f_n(x(t)) | & \ge | f_n (x_n) | -| f_n(x(t) -x_n) |\\
 & \ge || x_n || - || x(t) -x_n || 
\end{align*} 
and\pageoriginale $x(t)$ is in the closure of the set $M$ so that $f(t) \ge || x(t)
||$; since $|f_n (x(t) |\le ||x(t) ||$, $f(t) \le || (t) ||$. Thus
$f(t)= || x(t) ||$ and $|| x(t) ||$ is measurable. 
 
By a similar argument, $|| x(t) -x_n ||$ is measurable in $t$ for each
$n$. it follows, using $(c)$, that the half-line $[0 \le t < \infty)$
 can be represented, for each integer $m$, as a countable union of
 measurable sets $S_{m,n}$, 
 $$
 [0, \infty) = \bigcup_{n=1}^\infty S_{m,n}, S_{m, n} = \left\{t | || x(t)
  -x_n || \le m^{-1} \right\} 
 $$
 
 If we define 
 $$
 S'_{m,1}= S_{m,1}, \ldots, S'_{m,n}= S_{m,n} - \bigcup_{k=1}^{n-1} S'_{m,k}, 
 $$
 we have a decomposition of $[0, \infty)$ into disjoint measurable sets
 $S'_{m,n}(n=1,2, \ldots)$ such that $|| x(t) -x_n || \le m^{-1}$ in
 $S'_{m,n}$. 
 
 Therefore the strongly measurable step-function (i.e., a countably
 valued function taking each of its values exactly on a measurable
 set) 
 $$
 x^m (t) =x_n \text{ for } t \in S'_{m,n}
 $$
converges to $x(t)$ as $m \to \infty$ uniformly in $[0,t)$, Thus
 $x(t)$ is a strongly measurable function, a strongly measurable
 function being a functional which is the uniform limit of a sequence
 of strongly measurable step functions. We may then define the
 Bochner integral of $x(t)$ by: 
 $$
 \int\limits^{\beta}_{\alpha } x(t) dt= s-\lim_{, \to \infty}
 \int\limits^{\beta}_{\alpha } x^{(m)} (t) dt, 0 \le \alpha <\beta <\infty 
 $$
 ($\int\limits^{\beta}_{\alpha } x^m(t) dt$ may be defined, as in the case of
 the ordinary Lebesgue integral, as the strong limit of finitely
 valued functions, each taking each of its values exactly on a
 measurable set). We have 
 $$
 || \int\limits_{\alpha}^{\beta} x(t) dt || \le
 \int\limits_{\alpha}^{\beta} || x(t) || dt. 
 $$

Let\pageoriginale $0 \le \alpha < \eta < \beta < \xi - \varepsilon < \xi\,
(\varepsilon > 0)$.  

Since
$$
x(\xi) = T_\xi x_o = T_\eta T_{\xi - \eta} x_o = T_\eta x(\xi - \eta),
$$
we have
$$
(\beta - \alpha) x (\xi) = \int\limits_{\alpha}^{\beta} x (\xi) d \eta
= \int\limits_\alpha^\beta T_\eta (\xi - \eta ) d \eta, 
$$
the integrals being Bochner integrals. So
$$
(\beta - \alpha) \{x( \xi \pm \varepsilon ) - x (\xi ) \} =
\int^\beta_\alpha T_\eta \{ x (\xi \pm \varepsilon - \eta) - x(\xi -
\eta) \} d \eta. 
$$

Thus
$$
| \beta - \alpha |~ || x (\xi \pm \varepsilon) - x (\xi) || \le \sup_{
 \alpha \le \eta \le \beta} || T_\eta|| \int\limits_{\xi -
 \beta}^{\xi - \alpha} || x (\tau \pm \varepsilon ) - x (\tau) || d
\tau 
$$

But the right side tends to zero as $\varepsilon \downarrow 0$. (This
we see by approximating $x(\xi)$, in bounded interval, uniformly with
bounded. finitely valued strongly measurable functions. For, then the
result is reduced to the case of numerical measurable step
functions.) Thus $x (\xi)$ is strongly continuous for $\xi > 0$. 

To prove the strong continuity at $\xi = 0$ we proceed as follows: For
positive rational $t_k$, since 
$$
T_\xi x(t_k) = T_\xi t_{t_k} x_o = T_{\xi + t_k} x_o = x(\xi + t_k),
$$
we have, using the continuity for $\xi > 0$ proved above,
$$
s-\lim_{\xi \downarrow 0} T_\xi x(t_k) = x(t_k).
$$

It follows that $s-\lim\limits_{\xi \downarrow 0} T_\xi x_n = x_n$ for
each $x_n$; also $x(t), t \ge 0$, in particular $x(0) = x_o$, belongs
to $\bar{M} ~ (M = \{x_n\})$. It follows therefore, from the
inequalities, 
\begin{align*}
 || x (\xi ) - x_o || & \le || T_\xi x_n - x_n || + || x_n - x_o || +
 || T_\xi (x_o - x_n) ||\\ 
 & \le || T_\xi x_n - x_n || + || x_n - x_o || + \sup_{o \le \xi \le
 1} || T_\xi ||. || x_o - x_n ||, 
\end{align*}
that\pageoriginale $\lim\limits_{\xi \downarrow o} x (\xi) = x_o ~ $ i.e., $T_\xi$
is strongly continuous at $\xi = 0$. 

\begin{Proof}%Prf of 3
 An additive operator $A$ (with domain $\mathscr{D} (A)$) is said to
 be closed if it possesses the following property: if $\{ x_n\}$ is a
 sequence of elements of $\mathscr{D}(A)$ such that $s-\lim\limits_{n
  \to \infty} x_n = x$ and $s-\lim\limits_{n \to \infty} A x_n =
 y$, then $x$ belongs to $\mathscr{D} (A)$ and $Ax = y$. Evidently a
 linear operator is closed. 
\end{Proof}

To prove (3) let $k > \beta$. Then $J_k = \left(I -
\dfrac{A}{k}\right)^{-1}$ 
is a linear operator. Let $\{x_n\}$ be a sequence, $x_n \in
\mathscr{D} (A)$ such that $s-\lim\limits_{n \to \infty} x_n = x,
s-\lim\limits_{n \to \infty} A x_n = y$. Then $s-\lim\limits_{n \to
 \infty} \left(x_n - \dfrac{A}{k} x_n\right) = x - \dfrac{y}{k}$. By the
continuity of $J_k, s-\lim\limits_{n \to \infty} J_k \left(x_n -
\dfrac{A}{k} x_n \right) = J_k \left(x - \dfrac{y}{k}\right)$, i.e.,
$x= J_k \left(x - \dfrac{y}{k}\right)$. So $x (\in \mathscr{D} (A)$. Since 
$$
\left(I - \frac{A}{k}\right) x = \left(I - \frac{A}{k}\right) J_k \left(x
- \frac{y}{k}\right) = x -\frac{y}{k}, 
$$
we have $A x = y$.

\begin{remark*} %remk 
 It is to be noted that the theory has been extended for $\{ T_t\}_{o
  < t}$ satisfying 
  $$
 T_t T_s = T_{t + s}
 $$
 and the strong continuity in $t$ for $t > 0$.
\end{remark*}
