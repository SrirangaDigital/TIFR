\chapter{Lecture 1}\label{chap1}

\section{Introduction}\label{chap1:sec1}

The\pageoriginale analytical theory of one-parameter semi-groups deals with the
exponential function in infinite dimensional function spaces. It is a
natural generalization of the theorem of Stone on one-parameter groups
of unitary operators in a Hilbert space. 

In these lectures, we shall be concerned with the differentiability
and the representation of one-parameter semi-groups of bounded linear
operators on a Banach space and with some of their applications to the
initial value problem (Cauchy's problem) for differential equations,
especially for the diffusion equation (heat equation) and the wave
equation. 

The ordinary exponential function solves the initial value problem:
$$ 
\frac{dy}{dt} = \alpha y, \quad y (0) = C.
$$

We consider the diffusion equation
$$
\frac{\partial u}{\partial t} = \Delta u,
$$
where $\Delta = \sum\limits_{i = 1}^m \dfrac{\partial^2}{\partial
 x^2_i}$ is the Laplacian in the Euclidean m-space $E^m$; we wish to
find a solution $u = u(x, t), t \ge 0$, of this equation satisfying
the initial condition $u(x, 0) = f(x)$, where $f(x) = f(x_1, \ldots,
x_n)$ is a given function of $x$. We shall also study the wave
equation 
$$
\frac{\partial^2 u}{\partial t^2} = \Delta u, - \infty \le t \le \infty
$$
with the initial data 
$$
u (x, 0) = f(x) \text { and } \big( \frac{\partial u}{\partial t}
\big)_{t = o} = g (x), 
$$
$f$\pageoriginale and $g$ being given functions. This may be written in the vector
form as follows: 
$$
\frac{\partial}{\partial t} \binom{u}{v} = 
\binom{0 \quad I}{\Delta \quad 0} 
\binom{u}{v}, v = \frac{\partial u}{\partial t} 
$$
with the initial condition 
$$
\binom {u(0)}{v(0)} \quad = \quad \binom{f(x)}{g(x)}. 
$$

So in a suitable function space the wave equation is of the same form
as the heat equation - differentiation with respect to the time
parameter on the left and another operator on the right - or again
similar to the equation $\dfrac{d y}{d t} = \alpha y$. Since the
solution in the last case is the exponential function, it is suggested
that the heat equation and the wave equation may be solved by properly
defining the exponential functions of the operators $\Delta$ and
$\begin{pmatrix} 0 & I \\ \Delta & 0 \end{pmatrix}$ in suitable
function spaces. This is the motivation for the application of the
semi-group theory to Cauchy's problem. 

Our method will give an explanation why in the case of the heat
equation the time parameter is restricted to non-negative values,
while in the case of the wave equation it may extend between $-
\infty$ and $\infty$. 

Before going into the details, we give a survey of some of the basic
concepts and results from the theory of Banach spaces and Hilbert
spaces. 

\part[Survey of some basic concepts and ...]{Survey of some basic concepts and results from the
 theory of Banach spaces}\label{chap1:p1} 

\begin{defi*}%def
 A\pageoriginale set $X$ is called a {\em linear space} over a field $K$ if the
 following conditions are satisfied: 
 \begin{enumerate}[1)]
 \item $X$ is an abelian group (written additively).
 \item There is defined a scalar multiplication: to every element $x$
  of $X$ and each $\alpha \in K$ there is associated an element of
  $X$, denoted by $\alpha x$, such that 
 \begin{align*}
  (\alpha + \beta)x & = \alpha x + \beta x, \quad \alpha, \beta \in
  K, \quad x \in X\\ 
  \alpha (x + y) & = \alpha x + \alpha y, \qquad \alpha \in K, x, y
  \in X\\ 
  (\alpha \beta)x & = \alpha (\beta x)\\
  1x & = x, \qquad 1 \in K \text{is the unit element of K.}
 \end{align*}
 \end{enumerate}
\end{defi*}

We shall denote by Greek letters the elements of $K$ and by Roman
letters the elements of $X$. The zero of $X$ and the zero of $K$ will
both be denoted by $0$. We have $0.x = 0$. 

In the sequel we consider linear spaces only over the real number
field or the complex number field. A linear space will be said to be
real or complex according as the field is the real number field or the
complex number field. In what follows, by a linear space we always mean
a real or a complex linear space. 

\begin{defi*}%def
 A subset $M$ of a linear space $X$ is called a {\em linear subspace}
 (or a {\em subspace}) if whenever $x, y \in M$ and $\alpha, \beta
 \in K_2$ then $\alpha x+ \beta y \in M$. 
\end{defi*}

\section{Normed linear spaces:}\label{chap1:sec2}
\pageoriginale

\begin{defi*}%def
 A linear space $X$ (real or complex) is called a {\em normed linear
  space} if, for every $x \in X$ there is associated a real number,
 denoted by $|| x ||$, such that 
 \begin{enumerate}[\rm i)]
 \item $|| x|| \ge 0$ and $|| x || = 0$ if and only if $x = 0$.
 \item $|| \alpha x|| = |\alpha| || x||$, ($\alpha$ is a scalar and
  $|\alpha|$ is the modulus of $\alpha$). 
 \item $|| x + y|| \le || x|| + || y||, x, y \in X$ (triangle
  inequality). $|| x||$ is called the norm of $x$. 
 \end{enumerate}
\end{defi*}

A normed linear space becomes a metric space if the distance $d(x, y)$
between two elements $x$ and $y$ is defined by $d(x, y) = || x -
y||$. We say that a sequence of elements $\{x_n\}$ of $X$
\textit{converges strongly} to $x \in X$, and write $s-\lim\limits_{n
 \to \infty} x_n = x$ (or simply $\lim\limits_{n \to \infty} x_n =
x$), if $ \lim\limits_{n \to \infty} ||x_n - x || = 0$. (This limit,
if it exists, is unique by the triangle inequality). 

\begin{prop*}%proposition
 If $\lim\limits_{n \to \infty} \alpha_n = \alpha (\alpha_n, \alpha
 \in K), s- \lim\limits_{n \to \infty} x_n = x$ and $s-\lim\limits_{n
  \to \infty} y_n = y$, then $s-\lim\limits_{n \to \infty} \alpha_n
 x_n = \alpha x$ and $s- \lim\limits_{n \to \infty} (x_n + y_n) = x +
 y$. 
\end{prop*}

\noindent \textit{Proof.}
 \begin{align*}
  ||(x_n + y_n) - (x+y)|| & = || (x_n - x) + (y_n - y)||\\ 
  & \le || (x_n -
  x) || + || (y_n - y)|| \text{ (Triangle inequality) }\\ 
  & \to 0.\\
  || \alpha_n x_n - \alpha x|| & \le || \alpha x - \alpha_n x|| + ||
  \alpha_n x - \alpha_n x_n||\\ 
  & = |\alpha -\alpha_n| || x || + |
  \alpha_n | ||x -x_n||\\ 
  & \to 0. \tag*{$\Box$}
 \end{align*}

\begin{prop*}%prop
 If\pageoriginale $s-\lim\limits_{n \to \infty} x_n = x$ then $\lim\limits_{n \to
  \infty} || x_n || = || x ||$, i.e., norm is a continuous
 function. 
\end{prop*}

\begin{proof}
 We have, from the triangle inequality,
 $$
 \big| || x || - || y || \le || x-y || ;
 $$
 now take $y = x_n$ and let $n \to \infty$.
\end{proof}

\section{Pre-Hilbert spaces}\label{chap1:sec3}

A special class of normed linear spaces - pre-Hilbert spaces-will be
of fundamental importance in our later discussion of differential
equations. These normed linear spaces in which the norm is defined by
scalar product. 

\begin{defi*}%def
 A linear space $X$ is called a {\em pre-Hilbert space} if for every
 ordered pair of elements $(x, y) (x, y \in X)$ there is associated a
 number (real number if $X$ is a real linear space and complex number
 if $X$ is a complex linear space) such that 
 \begin{enumerate}[\rm i)]
 \item $(x, x) \ge 0$ and $(x,x) = 0$ if and only if $x = 0$.
 \item $(\alpha x, y) = \alpha(x, y)$, for every number $\alpha$.
 \item $(x, y) = (\overline{y, x}) [(\overline{y, x})$ denotes the
  complex conjugate of $(y, x).]$ 
 \item $(x + y, z) = (x, z) + (y, z) \quad x, y, z \in X$.
 \end{enumerate}
 
 $(x, y)$ is called the \textit{scalar product} between $x$ and $y$.
 
 If we define $|| x || = \sqrt{(x, x)}$, a pre-Hilbert space becomes
 a normed linear space, as is verified easily using Schwarz's
 inequality proved below 
\end{defi*}

\begin{prop*}%prop
 \begin{enumerate}[\rm i)]
 \item $ | (x, y) | \le || x || || y ||$ \hfill(Schwarz's inequality)
 \item $|| x+ y ||^2 + || x-y ||^2 = 2(||x||^2 + ||y ||^2)$ \hfill 
    {(Euclidean property)}
 \end{enumerate}
\end{prop*}

\begin{proof}
 (ii)\pageoriginale is easily verified. To prove (i), we observe that, for every
 real number $\alpha$, 
 \begin{align*}
  0 & \le (x + \alpha (x, y)y, x + \alpha (x, y)y)\\
  & = (x, x) + 2 \alpha | (x, y) |^2 + \alpha^2 |(x, y)|^2 (y, y).
 \end{align*}
 
 This quadratic form in $\alpha$, being always non-negative should
 have non-positive discriminant so that 
 $$
 | (x, y)|^4 - || x||^2 ||y||^2 | (x, y) |^2 \le 0.
 $$

 If $(x, y) = 0, (i)$ is obviously satisfied; if $(x, y) \ne 0$,
 Schwarz's inequality follows from the above inequality. 
\end{proof}

\section{Example of a pre-Hilbert space}\label{chap1:sec4}

Let $R$ be a domain in Euclidean $m$-space $E^m$. Let $\mathscr{D}^k
(R)$ denote the set of all complex valued functions $f(x) = f(x_1,
\ldots, x_n)$ which are of class $C^k$ in $R$(i.e., $k$ times
continuously differentiable) and which have compact support. These
functions form a linear space with the ordinary function sum and
scalar multiplication. Define the scalar product between two functions
$f$ and $g$ by 
$$
(f, g)_k = \sum_{|n|\le k} \int_R D^{(n)} f(x) \overline{D^{(n)} g(x)}
dx \quad, \quad 0 \le k < \infty, 
$$
where $n = (n_1, \ldots, n_m)$ is a system of non-negative integers,
$|n| = n_1 + \qquad + n_m$ and 
$$
D^{(n)} = \frac{\partial^{|n|}} {\partial x_1^{n_1} \partial x_2^{n_2}
 \cdots \partial x_m^{n_m}} 
$$

\section{Banach spaces}\label{chap1:sec5}

\begin{defi*}%def
 A normed linear space is called a {\em Banach spaces} if it is
 complete in the sense of the metric given by the norm. 

 (Completeness\pageoriginale means that every Cauchy sequence is convergent: if
 $\{x_n\} \subset X$ is any Cauchy sequence, i.e., a sequence
 $\{x_n\}$ for which $|| x_m - x_n|| \to 0$ as $m, n \to \infty$
 independently, then there exists an element $x \in X$ such that
 $\lim \limits_{n \to \infty} || x_n - x|| = 0. x$ is unique). 
\end{defi*}

\section{Hilbert space}\label{chap1:sec6}

\begin{defi*}%def
 A pre-Hilbert space which is complete (considered as a normed linear
 space) is called a \textit{Hilbert space}. 
\end{defi*}

The pre-Hilbert space $\mathscr{D}^K (R)$ defined in the last example
is not complete 

\section{Example of Banach spaces}\label{chap1:sec7}

\begin{enumerate}[1)]
\item $C \underline{[\alpha, \beta]}$: Let $[\alpha, \beta]$ be
 a closed interval $- \infty \le \alpha < \beta \le \infty$. Let
 $C[\alpha, \beta]$ denote the set of all bounded continuous
 complex-valued functions $x(t)$ on $[\alpha, \beta]$. (If the
 interval is not bounded, we assume further that $x(t)$ is uniformly
 continuous). Define $x + y$ and $\alpha x$ by 
 \begin{align*}
  (x + y) (t) & = x(t) + y(t)\\
  (\alpha x) (t) & = \alpha. x(t).
 \end{align*}
 $C[\alpha, \beta]$ is a Banach space with the norm given by
 $$
 || x || = \sup_{t \in [ \alpha, \beta]} |x(t) |
 $$

 Converges in this metric is nothing but uniform convergence on the
 whole space. 
\item $L_p (\alpha, \beta). \quad (1 \le p < \infty)$. This is the
 space of all real or complex valued Lebesgue functions $f$ on the
 open interval $(\alpha, \beta)$ for which $|f(t)|^p$ is Lebesgue
 summable over $(\alpha, \beta)$; two functions $f$\pageoriginale and $g$ which
 are equal almost everywhere are considered to define the same vector
 of $L_p (\alpha, \beta)$. $L_p (\alpha, \beta)$ is a Banach space
 with the norm: 
 $$
 || f || = \left( \int\limits_\alpha^\beta |f(t)|^p dt \right)^{1/p}
 $$
 The fact that $|| ~ ||$ thus defined is a norm follows from
 Minkowski's in-equality; the Riesz-Fischer theorem asserts the
 completeness of $L_p$. 
\item $L_\infty (\alpha, \beta)$: This is the space of all measurable
 (complex valued) functions $f$ on $(\alpha, \beta)$ which are
 essentially bounded, i.e., for every $f \in L_\infty (\alpha,
 \beta)$ there exists $a \wp > 0$ such that $| f(t) | \le \wp$ almost
 everywhere. Define $|| f ||$ to be the infimum of such $\wp$. 
 
 (Here also we identify two functions which are equal almost everywhere).
\end{enumerate}

\section{Example of a Hilbert space}\label{chap1:sec8}

$L_2 (\alpha, \beta): L_2 (\alpha, \beta)$ (see example $(2)$
above), is a Hilbert space with the scalar product 
$$
(f, g) = \int\limits_\alpha^\beta f(t) \overline{g(t)} dt. 
$$

\section{Completion of a normed linear space}\label{chap1:sec9}

Just as the completeness of the real number field plays a fundamental
role in analysis, the completeness of a Banach space will play an
essential role in some of our subsequent discussions. If we have an
incomplete normed linear space we can always complete it; we can imbed
this space in a Banach spaces as an everywhere dense subspace and this
Banach spaces is essentially unique. We have, in fact, the 

\begin{theorem*}%thm
 Let $X_0$ be a normed linear space. Then there exists a {\em
  complete} normed linear space (Banach spaces ) $X$ and a norm
 preserving isomorphism $T$ of $X_0$ onto a subspace $X'_0$ of $X$
 which is dense in\pageoriginale $X$ in the sense of the norm topology. (That $T$
 is a norm preserving isomorphism means that $T$ is one-to-one,
 $T(\alpha x_0 + \beta y_0) = \alpha T(x_0) + \beta T(y_0)$ and $|| x
 || = || T(x) ||$). Such an $X$ is determined uniquely upto a norm
 preserving isomorphism 
\end{theorem*}

\noindent
{\bf Sketch of the proof:} The proof follows the same idea as that
utilized for defining the real numbers from the rational numbers. Let
$X$ be the totality of all Cauchy sequences $\{x_n\} \subset X_0$
classified according to the equivalence: $\{ x_n\} \sim \{y_n\}$ if
and only if $\lim \limits_{n \to \infty} || x_n - y_n || = 0$. Denote
by $\{\overline{x_n}\}$ the class containing $\{x_n\}$. 

If $\tilde{x}, \tilde{y} \in X$ and $\tilde{x} = \{\overline{ x_n}\},
\tilde{y} = \{\overline{y_n}\}$, define $\tilde{x} + \tilde{y} =
\{\overline{x_n + y_n}\}, \alpha \tilde{x} = \{\overline{\alpha
 x_n}\}, || \tilde{x} || = \lim \limits_{n \to \infty} || x_n
||$. These definitions do not depend on the particular representatives
for $\tilde{x}, \tilde{y}$ respectively. Finally if $x_0 \in X_0$
defines $T(x_0) = \{\overline{x_n}\}$ where each $x_n = x_0$. 


\section{Additive operators}\label{chap1:sec10}

\begin{defi*}%def
 Let $X$ and $Y$ be linear spaces over $K$. An {\em additive
  operator} from $X$ to $Y$ is a single-valued function $T$ from a
 subspace $M$ of $X$ into $Y$ such that 
 $$
 T(\alpha x + \beta y) = \alpha Tx + \beta Ty, \quad x, y \in M,
 \alpha, \beta \in K. 
 $$
 
 $M$ is called the domain of $T$ and is denoted by $\mathscr{D} (T)$;
 the set $\{ z | z \in Y $ such that $ z =Tx$ for some $ x \in
 \mathscr{D} (T) \}$ is called the range of $T$ and is denoted by
 $\mathfrak{W} (T)$. 
\end{defi*}

If $Y$ is the space of real or complex numbers (according as $X$ is a
real or a complex linear space) and $T$ is an additive operator from
$X$ to $Y$ we say that $T$ is an \textit{additive functional}. 

\begin{defi*}%def
 Let\pageoriginale $X$ and $Y$ be two normed linear spaces. An additive operator
 $T$ is said to be {\em continuous} at $x_0 \in \mathscr{D} (T)$ if
 for every sequence $\{ x_n\} \subset \mathscr{D} (T)$ with $x_n \to
 x_0$ we have $Tx_n \to Tx_0$. An additive operator is said to be
 {\em continuous} (on $\mathscr{D}(T)$) if it is continuous at every
 point of $\mathscr{D}(T)$. It is easy to see that an additive
 operator $T$ is continuous on $\mathscr{D} (T)$ if it is continuous
 at one point $x_0 \in \mathscr{D} (T)$. 
\end{defi*}

\begin{prop*}%prop
 An additive operator $T: X \to Y$ between two normed linear spaces
 is continuous if and only if there exists a real number $\wp > 0$
 such that 
 $$
 || Tx || \le \wp || x|| \quad \text{ for every } \quad x \in
 \mathscr{D} (T) 
 $$
\end{prop*}

\begin{proof}
 The sufficiency of the condition is evident, for if $x_n \to x_0 ~ ||
 Tx_0 - Tx_n || = || T(x_0 - x_n) || \le \wp || x_0 - x_n || \to 0$. 

 Now assume that $T$ is continuous. If there exists no $\wp$ as in
 the proposition, then there exists a sequence $\{x_n\} \subset
 \mathscr{D}(T)$ such that $|| Tx_n || > n || x_n ||$. Since $T(0) =
 0, x_n \ne 0$. Define $y_n = x_n/ \sqrt{n} || x_n ||$. Then $|| y_n
 || = \dfrac{1}{\sqrt{n}} \to 0$ as $n \to \infty$; as $T$ is
 continuous $T_{y_n}$ must tend to zero as $n \to \infty$. But $Ty_n
 = \dfrac{1}{\sqrt{n}||x_n||} Tx_n$ and $|| Ty_n || =
 \dfrac{1}{\sqrt{n}||x_n||} ||Tx_n || > \sqrt{n}$ and so $Ty_n$ does
 not tend to zero. This is a contradiction. 
\end{proof}

Let $T$ be an additive operator from a linear space $X$ into a linear
space $Y$. $T$ is one-one if and only if $Tx = 0$ implies $x =0 $. If
$T$ is one-one it has an inverse $T^{-1}$, which is an additive
operator from $Y$ into $X$ with domain $\mathfrak{w}(T)$, defined by 
$$
T^{-1} y = x \quad if \quad y =Tx. 
$$

$T^{-1}$\pageoriginale satisfies the relations $T^{-1}Tx = x$ for $x \in \mathscr{D}
(T)$ and $T T^{-1} y = y$ for $y \in \mathscr{D}(T^{-1}) =
\mathfrak{W}(T)$. If $X$ and $Y$ are normed linear spaces, $T$ has a
continuous inverse if and only if there exists a $\delta > 0$ such
that $|| T x || \ge \delta || x || $ for $x \in \mathscr{D}(T)$. 

The \textit{sum} of two operators $T$ and $S$, with $\mathscr{D}(T),
\mathscr{D}(S) \subset x$ and $\mathfrak{W}(T), \mathfrak{W}(S)
\subset Y$ is the operator $(T + S)$, with domain $\mathscr{D}(T) \cap
\mathscr{D}(S)$, defined by: 
$$
(T + S)x = Tx + Sx. 
$$
