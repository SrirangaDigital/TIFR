\chapter{Lecture 22}\label{chap22} %Lect 22

\section[Integration of the Fokker-Planck...]{Integration of the Fokker-Planck equation\hfill\break (Continued)}\label{chap22:sec1}

Before\pageoriginale going into the proof of the differentiability of the
operator-theo\-retical solution $u(t, x)=(T_t u)(x)$ we shall discuss
the question of the denseness of the range of the set 
$$
\left\{ (I- \alpha^{-1}A) f,f \in D(A) \right\}.
$$

If the range of $(I-\alpha^{-1}A)$ were not dense in $L_1 (R)$, there
will exists $h \in M(R)=L_1(R)^*,h\neq 0$ such that 
$$
\int \limits_R (I- \alpha^{-1}A)f. hdx=0, f\in D(A).
$$
$h$ is a weak solution of the equation $(I- \alpha^{-1}
A^*)h=0$. Since $h \in L_2 (R)$ and $A^*$ is elliptic, $h$ is almost
everywhere equal to a bounded $C^\infty$ solution of $(I-
\alpha^{-1}A^*)h=0$. Let $\{ R_k \}$ be a monotone increasing sequence
of domains $\subseteq R$ with smooth boundary such that $\partial R_k$ tends
to $\partial R$ very smoothly. Then we have 
\begin{align*}
 0 &= \int \limits_{R_k}h (I- \alpha^{-1}A) fdx- \int \limits_R f(I-
 \alpha^{-1}A^*)hdx\\ 
 &= \alpha^{-1}\int \limits_{R_k} (h f-f A^* h)dx\\
 &= \alpha^{-1} \left\{ \int \limits_{\partial R_k} \sqrt{g} a^{ij} \left(h
 \frac{\partial f}{\partial x_j}-f \frac{\partial h}{\partial x_j}\right)
 \cos (n,x^i)dS\right.\\ 
 & \left.+ \int \limits_{R_k} \left(\frac{\partial \sqrt{g}a^{ij}}{\partial
  x^j}-\sqrt{g}b^i\right) \cos (n,x^i) f(x)h(x)dS \right\}. 
\end{align*}

By the boundedness of $h$ and the boundary condition on $f$, we have
$$
\lim_{k \to \infty} \int \limits_{\partial R_k} \left\{ \sqrt {g} a^{ij}
\frac{\partial f}{\partial x^j}+ \left(\frac{\partial
 \sqrt{g}a^{ij}}{\partial x^j}- \sqrt{g}b^i\right)f\right\} \cos
(n,x^i)hd S=0. 
$$

Therefore\pageoriginale $h$ must satisfy the boundary condition

$$
\lim_{k \to \infty} \int \limits_{\partial R_m} \sqrt{g} a^{ij} f
\frac{\partial h}{\partial x^j} \cos (n,x^i) d S=0 \text{ for every }
f \in D(A). 
$$

Such a bounded solution $h$ of $(I- \alpha^{-1} A^*)h=0$ is
identically zero and hence $\bar {A}$ is the infinitesimal generator
of a semi-group $T_t$ in $L_1(R)$ in either of the following cases: 
\begin{enumerate}[(i)]
\item $R$ is compact (without boundary).
\item $R$ is a half-line is a finite closed interval on the real line
 and $A=d^2/dx^2$. 
\end{enumerate}

\begin{proof}
 \begin{enumerate}[(i)]
 \item At a maximum (or minimum) point $x_ \circ$ of $h(x)$ we must
  have $A^*h(x_ \circ) \leq 0$ (\resp $\geq 0$) so that the
  continuous solution $h(x)$ of $A^*h= \alpha h$ cannot have either
  a positive maximum or a negative minimum. 
 \item The boundary condition for $h$ is $\dfrac{\partial h}{\partial
  n}=0$ and the general solution of $A^*h =\alpha h$ is 
 \end{enumerate}
\end{proof}

\begin{align*}
 h &=C_1 e^{\sqrt{\alpha} x}- C_2 e^{-\sqrt{\alpha} x}\\
 \frac{dh}{dx} &= C_1 \sqrt{\alpha} x + C_2
 \sqrt{\alpha}-e^{\sqrt{\alpha} x}
\end{align*}
so that the vanishing of $\dfrac{dh}{dx}$ at two points implies that
$C_1=C_2=0$. And the vanishing of $\dfrac{dh}{dx}$ at one point
implies either $C_1=C_2=0$ or $C_1$ and $C_2 \neq 0$. The latter
contingency contradicts the boundedness of $h$. 

\noindent \textbf{A parametrix for the operator $-\bigg (
 \dfrac{\partial}{\partial \tau}+ A^* \bigg )$} 

Let $\Gamma (P,Q)=r (P,Q)^2$ be the square of the shortest distance
between the points $P$ and $Q$ according to the metric $dr^2=a_{ij}dx^i
dx^j$, where $(a_{ij})=(a^{ij})^{-1}$. We have the 

\begin{theorem*}%Thm
 For\pageoriginale any positive $k$ we may construct a parametrix $H_k$ ($P$, $Q$, $t-\tau$)
 for $-\bigg ( \dfrac{\partial}{\partial \tau}+ A^* \bigg )$ of the
 form 
 $$
 H_k (P,Q,t-\tau)= (t- \tau)^{-m/2} \exp \left(-\frac{\Gamma
  (P,Q)}{4(t,\tau)} \sum^k_{i=0} u_i (P,Q)(t - \tau)^i\right), 
 $$
 where $u_i(P,Q)$ are $C^ \infty$ functions in a vicinity of $P$ and
 $u_i(P,P)=1$, we have 
 $$
 \left( -\frac{\partial}{\partial \tau}- A^*_Q \right )H_k
 (P,Q,t-\tau)=(t-\tau)^{k-m/2} \exp \left(-\frac{\Gamma
  (P,Q)}{4(t-\tau)}\right)C_k (P,Q) 
 $$
 $C_k(P,Q)$ being $C^ \infty$ functions in a vicinity of $P$.
\end{theorem*}

\begin{proof}
 We introduce the normal coordinates* $y^\sigma$ of the point
 $Q=(x^1,\break \ldots$, $x^m)$ in suitable neighbourhood of $P$. 
 $$
 y^\sigma= \{ \Gamma (P,Q)\}^{\frac{1}{2}} \left(\frac
 {dx^\sigma}{dr}\right)_{P=Q} 
 $$

 Let $dr^2=\alpha_{ij}(y)dy^i dy^j$. We first show that when we apply
 the operator 
 $$
 A^*=Ay^*= \alpha^{ij} \frac{\partial^2}{\partial y_i \partial}+
 \beta^i \frac{\partial}{\partial y^i}+e 
 $$
 on a function $f(\Gamma,y)$ ($\Gamma$ is function on $y$) we have,

 \begin{align*}
  A^*_y&= 4 \Gamma \frac{\partial^2 f}{\partial \Gamma^2} +4
  y^\sigma \frac{\partial^2 f}{\partial \Gamma \partial^\sigma}+M
  \frac {\partial f}{\partial \Gamma} +N(f)\\ 
  N(f)&= \alpha^{ij} \frac{\partial^2 f}{\partial y^i \partial y^j}+
  \beta^i \frac{\partial f}{\partial y^i}+ef 
 \end{align*}
\end{proof}

 (The differentiations have to be performed as though $\Gamma$ and
 $y$ were independent variables). To prove this, we need the
 well-known formulae: 
\begin{equation}
 \Gamma (P,Q)= \alpha_{ij}(0)y^i y^j\tag{1}
\end{equation}
$$
\alpha_{ij} (y) y^i =\alpha_{ij} (0) y^j.
$$

Define\pageoriginale
$$
\frac{d}{dy^i}f(y,\Gamma)= \frac{\partial f}{\partial y^j}+
\frac{\partial f}{\partial \Gamma}. \frac{\partial \Gamma}{\partial
 y^j}. 
$$

Then
\begin{align*}
 \frac{d^2}{dy^i dy^j}\{f (y, \Gamma) \} &= \frac{\partial}{\partial
  y^i} \left(\frac{\partial f}{\partial y^j}+ \frac{\partial f}{\partial
  \Gamma}\cdot \frac{\partial \Gamma}{\partial y^j}\right) +
 \frac{\partial f}{\partial \Gamma} \left(\frac{\partial
  f}{\partial y^j}+ \frac{\partial f}{\partial
  \Gamma}. \frac{\partial \Gamma}{\partial y_j}\right)\frac{\partial
  \Gamma}{\partial y^i}\\ 
 &=\frac{\partial^2 f}{\partial y^i\partial y^j}+\frac{\partial^2
  f}{\partial y^i \partial \Gamma}. \frac{\partial \Gamma}{\partial
  y^j}+\frac{\partial f}{\partial \Gamma} \frac{\partial^2
  \Gamma}{\partial y^i \partial y^j}\\ 
 &+ \frac{\partial^2 f}{\partial \Gamma \partial y^j} \frac{\partial
  \Gamma}{\partial y^i}+ \frac{\partial^2f}{\partial \Gamma
 ^2}.\frac{\partial \Gamma}{\partial y^j} \frac{\partial
  \Gamma}{\partial y^i} 
\end{align*}

So, by (1)
\begin{align*}
\alpha ij \frac{d^2 f}{dy^idy^j}& + \beta^i \frac{d}{dy^i}f+ef\\
 &= \bigg( \alpha^{ij} \frac{\partial \Gamma}{\partial y^i}
 \frac{\partial \Gamma}{\partial y^j} \bigg) \frac{\partial^2
  f}{\partial \Gamma^2}+ 2 \alpha^{ij} \frac{\partial
  \Gamma}{\partial y^j} \frac{\partial^2 f}{\partial y^i \partial
  \Gamma}+ M \frac {\partial f}{\partial \Gamma}+N(f)\\ 
 &=4 \Gamma \frac{\partial^2 f}{\partial \Gamma^2}+4 y^\sigma
 \frac{\partial^2 f}{\partial \Gamma \partial y^\sigma}+M
 \frac{\partial f}{\partial \Gamma}+N (f). 
\end{align*}

Now applying the above formula to $H_k$, we have 
\begin{align*}
 -A^*_yH_k &= \sum^k_{i=0} \frac{\Gamma}{4} (t- \tau)^{i-2-m/2}\exp
 \left(-\frac {\Gamma}{4(t- \tau)} \right )u_i\\ 
 &+\sum^k_{i=0}(t- \tau)^{i-1-m/2}\exp \left(-\frac {\Gamma}{4(t-
  \tau)} \right) \left\{ y^\sigma \frac{\partial u_i}{\partial
  y^\sigma}+\frac{M}{4}u_i -N(u_{i-1})\right\}\\ 
 &-(t- \tau)^{k-m/2}\exp \left(-\frac {\Gamma}{4(t- \tau)}\right)N(u_k) 
\end{align*}
where $u_{-1}=0,N(u_{-1})=0$. Since 
$$
-\frac{\partial}{\partial \tau}H_k= \sum^k_{i=0}(t-\tau)^{i-1-m/2}
\exp \bigg (-\frac {\Gamma}{4(t- \tau)} \bigg ) u_i \bigg
(-\frac{m}{2}+i+\frac {\Gamma}{4(t- \tau)} \bigg ) 
$$
the theorem will be prove d if we can choose $u_i$ satisfying the relations 
$$
y^\sigma \frac{\partial u_i}{\partial y^\sigma}+
\left(-\frac{m}{2}+i+\frac{M}{4}\right)u_i=N (u_{i-1}), 
$$
$u_i(P,Q)$\pageoriginale being $C^\infty$ in a vicinity of $P$ with $u_{-1}=0$ and
$u_\circ (P,P)=1$. To see that we may choose such $u_i$, put
$y^\sigma= \eta^\sigma s$ and transform the equation as an ordinary
differential equation in $s$ containing the parameters $\eta$: 
$$
\frac{d u_i}{ds} +\left(-\frac{m}{2}+i+ \frac{N}{4}\right)u_i=N (u_{i-1}).
$$

Choose
$$
u_ \circ =\exp \left(-\int \limits^s_\circ
s^{-1}\left(-\frac{m}{2}+\frac{M}{4}\right)ds\right).
$$
$u_i$ is $C^\infty$ near $s= \circ$, because of the order relation
$M=2m+o(s)$. Define $u_i$ successively by the formula 
$$
u_i (P,Q)=u_ \circ s^{-1} \int \limits^s_ \circ s^{i-1}u^{-1}_ \circ
N(u_{i-1})ds\, (i=1,2,\ldots, k). 
$$
