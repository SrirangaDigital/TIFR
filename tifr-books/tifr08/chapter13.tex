\part[Regularity properties of solutions of linear elliptic...]{Regularity properties of solutions of linear elliptic
 differential equations}\label{chap13:p3} 

\chapter{Lecture 13}\label{chap13}

The\pageoriginale results proved in this part will be needed in the application of
the semi-group theory to Cauchy's problem. 

\section{Strong differentiability}\label{chap13:sec1}

Let $R$ be a subdomain of $E^m$. We denote by $C^\infty (R)$ the space
of indefinitely differentiable functions in $R$ and by
$\mathscr{D}^{\infty}(R)$ the space of $C^\infty$ functions in $R$
with compact support. We denote by $L_2 (R)_{loc}$ the space of
locally square summable functions in $R, $ (i.e., functions in $R$
which are square summable on every compact subset of $R$). $A$
function $u(x) \in L_2 (R)_{loc}$ is said to be $k$-times strongly
differentiable in $R$ (or of order $k$ in $R$) if for every subdomain
$R_1$ of $R$ relatively compact in $R$ there exists a sequence $u_n
(x) ( = u_{n, R_1}(x) ~ )$ of $C^\infty$ functions in $R_1$, such that 
$$
\displaylines{\hfill 
\lim_{n \to \infty} \int\limits_{R_1} | u - u_n |^2 dx = 0\hfill\cr
\text{and}\hfill 
\lim_{n, 1 \to \infty} \int\limits_{R_1} | D^{(s)} u_n - D^{(s)} u_1
|^2 dx = 0 \quad \text { for } |s | \le k. \hfill }
$$

Then there exists, for $| s | \le k$, functions
\begin{gather*}
 u^{(s)} (x) = u_{R_1}^{(s)} \in L_2 (R_1) \qquad \text { such that}\\
 \lim_{n \to \infty} \int\limits_{R_1} \big | u^{(s)} (x) - D^{(s)}
 u_n (x) \big |^2 dx = 0. 
\end{gather*}

$u^{(s)}_{R_1}(x)$ is determined independently of the approximating
sequence $u_n$; for\pageoriginale we have, for each $C^\infty$ function $\varphi$
with compact support in $R_1$ 
\begin{align*}
 \int\limits_{R_1} \varphi(x) u^{(s)} (x) dx & = \lim_{n \to \infty}
 \int\limits_{R_1} \varphi (x) D^{(s)} u_n (x) dx\\ 
 & = \lim_{n \to \infty} (-1)^{| s |} \int\limits_{R_1} u_n (x)
 D^{(s)} \varphi (x) dx\\ 
 & = (-1)^{|s|} \int\limits_{R_1} u(x) D^{(s)} \varphi (x) dx
\end{align*}
and $C^\infty$ functions with compact support in $R_1$ are dense in
$L_2(R_1)$. 
It also follows that, for $|s| \le k$, there exists a function in
$L_2(R)_{loc}$, denoted by $\tilde{D}^{(s)} u(x)$, such that for each
subdomain $R_1$ relatively compact in $R$, $\tilde{D}^{(s)} u(x)$ coincides
with $u^{(s)}_{R_1} (x)$ almost everywhere in $R_1$. $\tilde{D}^{(s)}
u(x)$ is called the strong derivative of $u$  corresponding to the
derivation $D^{(s)}$. 

\section{Weak solutions of linear differential operators}\label{chap13:sec2}

Let
$$
L = \sum^n_{| \rho | = | \sigma | = o} D^{(\rho)} a^{\rho \sigma}
D^{(\sigma)}, a^{\rho, \sigma} (x) \in C^\infty (R), a^{\rho, \sigma}
= a^{\sigma, \rho} ~\text{ for }~ | \sigma | = | \rho | = n, 
$$
 be a linear differential operator in $R$ with $C^\infty$
 coefficients. Let $f \in L_2 (R)_{loc}$. A function $u \in L_2
 (R)_{loc}$ will be said to be a weak solution of the equation $L u =
 f $ if for every $\varphi \in \mathscr{D}^\infty (R)$ we have 
 $$
 \int\limits_{R} ~ L^* \varphi u dx = \int\limits_{R} \varphi f dx
 $$
 where $L^*$ is the adjoint of $L$:
 $$
 L^* = \sum^n_{| \rho | = | \sigma | = o} (-1)^{|\rho | + |\sigma|}
 D^{(\sigma)} a^{\rho, \sigma} D^{(\rho)}. 
 $$

\section{Elliptic operators}\label{chap13:sec3} 

\noindent \textbf{Friedrichs - Lax - Nirenberg, theorem:}  
Let\pageoriginale $L$ be elliptic in $R$ in the sense that there
exists a constant $C_o > 0$ such that 
$$
\sum_{| \rho | = | \sigma | = n} \xi_1^{\rho_1} \cdots \xi_m^{\rho_m}
a^{\rho_1 \cdots \rho_m ; \sigma_1 \cdots \sigma_m} (x)
\xi_1^{\sigma_1} \cdots \xi_m^{\sigma_m} \ge C_o \left( \sum^m_{i = 1}
\xi^2_i \right)^ n 
$$
for every $x \in R$ and every real vector $(\xi_1, \ldots,
\xi_m)$. Then if $u_o$ is a weak solution of $Lu= f$ and if $f$ is of
order $p$ in $R$, then $u_o$ is of order $2n + p$ in $R$. 

\noindent \textbf{Sobolev's lemma:} 
If $u_o (x)$ is of order $k$ in $R$, then, for $k > m/2 + \sigma, h_o
(x)$ is equal almost everywhere (in $R$) to a function which is
$\sigma$ times continuously differentiable. 

\noindent \textbf{Weyl-Schwartz theorem:}
Let $L$ be an elliptic operator in $R$, and $u_o$ a weak solution of
$Lu = f$. If $f$ is indefinitely differentiable in $R$, then $u_o$ is
almost everywhere equal to an indefinitely differentiable function in
$R$. 

This theorem is an immediate consequence of the Friedrichs
Lax-Nirenberg theorem and Sobolev's lemma. 

\section{Fourier Transforms:}\label{chap13:sec4}

For the proofs we need the following facts about Fourier transforms:

\noindent \textbf{Plancherel's theorem:} 
Let $f(x) \in L_2 (E^m)$, $x = (x_1, \ldots, x_n)$. Then the functions 
$$
\phi_n (y) = \int\limits_{| x | \le n} f(x) \exp (-2 \pi i x. y) ~ dx~
(x.y = \sum x_i y_i) 
$$
converge\pageoriginale in the $L_2$-norm to a function $\varphi (y_1, \ldots, y_n)
\in L_2$ and the transformation $\mathscr{F}$ defined by $\mathscr{F}
f = \varphi (y) = \lim\limits_{n \to \infty} \int\limits_{| x | \le n}
f(x) \exp (-2 \pi ix . y) dx$ is a unitary transformation of $L_2$
onto itself. (i.e., $ (\mathscr{F} f, \mathscr{F} g) = (f, g)$, for
$f, g \in L_2$ onto $L_2)$. The inverse $\mathscr{F}^{-1} $ of
$\mathscr{F}$ is given by 
$$
\mathscr{F}^{-1} \varphi (x) = \lim_{n \to \infty} \int\limits_{| y |
 \le n} \varphi (y) \exp (2 \pi i y. x ) dy 
$$
$\mathscr{F}(f)$ is called the Fourier transform of $f$.

As regards the Fourier transform of the derivatives, we have: if $f$
in $L_2(E^m)$ is also in $C^k (E^m)$ and $D^{(s)} f(x) \in L_2 (E^m)$
for $| s | \le k$, $(D^{(s)} = \partial^{s_1 + \cdots + s_n}/
\partial x_1^{s_1} \cdots \partial x_m^{s_m}, | s | = \sum^n_{i = 1}
s_j$), then 
$$
(\mathscr{F} D^{(s)} f) (y) = \prod^m_{j = 1} (2 \pi i y_j)^s
j. \mathscr{F} (f) (y). 
$$

\noindent \textbf{Proof of Sobolev's lemma:}
Let $R_1$ be any relatively compact subdomain of $R$ and $\alpha (x)\, a\,
C^\infty$ function with compact support in $R$ such that $\alpha (x)
\equiv 1$ on $R_1$. Since $u_o$ is assumed to be of order $k$ there
exists a sequence $\{u_n\}$ of $C^\infty$ functions in $R_1$ such that 
$$
\lim_{n \to \infty} \sum_{| s | \le k} \int\limits_{R_1} \big |
\tilde{D}^{(s)} u_o - D^{(s) } u_n \big |^2 dx = 0. 
$$

We have, using Leibnitz's formula,
$$
\lim_{n \to \infty} \sum_{| s | \le k} \int \big | \tilde{D}^{(s)}
\alpha u_o - D^{(s)} \alpha u_n \big |^2 dx = 0. 
$$

Let $\tilde{u}_n$ (\resp $\tilde{u}_o)$ denote the function in $E^m$
defined by: 
$$
\tilde{u}_n^{(x)} = 
\begin{cases}
 \alpha u_n (x), &x \in \text { Support of } \alpha\\
 \quad 0 & x \in E^m - \text { supp } \alpha ;
\end{cases}
$$
similar\pageoriginale definition for $\tilde{u}_o ( = \alpha u_o$ in
supp. $\alpha$). Since the Fourier transform is a unitary
transformation, we have 
$$
\lim_{n \to \infty} || \mathscr{F} D^{(s)} \tilde{u}_n ~ - ~
\mathscr{F} \tilde{D}^{(s)} \tilde{u}_o ||_{o, E^m} = 0. 
$$

But, as remarked earlier,
$$
(\mathscr{F} D^{(s)} \tilde{u}_n ) (y) = (2 \pi i)^s y_1^{s_1} \cdots
y_m^{s_m} \tilde{U}_n (y) 
$$
where $\tilde{U}_n = \mathscr{F} \tilde{u}_n$; also since $\mathscr{F}
$ is unitary, 
$$
\lim_{n \to \infty} || \tilde{U}_n - \tilde{U}_o ||_{o, E_m} = 0,
\text { where } \tilde{U}_o = \mathscr{F} (\tilde{u}_o). 
$$

Therefore there exists a subsequence $\{ n' \} $ of $\{n\}$ such that
for almost all $y \in E^m$ 
$$
\displaylines{\hfill 
 \lim_{n' \to \infty} \tilde{U}_{n'} (y) = \tilde{U}_o (y) \quad \text
   {(pointwise limit)} \hfill \cr
   \hfill \lim_{n' \to \infty} \tilde{U}_{n'} (y) y_1^{s_1} \cdots y_m^{s_m}
   (2 \pi i)^{| s |} = \tilde{U}_o^{|s|} = \tilde{U}_o^{(s)} (y)
   \hfill\cr 
   \text{where} \hfill 
   \tilde{U}_o^{(s)} = \mathscr{F} \tilde{D}^{(s)} \tilde{u}_o.\hfill} 
$$
Thus for almost all \quad $y \in E^m, 
\tilde{U}_o (y) y_1^{s_1} \cdots y_m^{s_m} (2 \pi i)^{|s|} =
\tilde{U}_o^{(s)} (y), | s| \le k$. 

We shall now show that $\tilde{U}_o (y)\cdot y_1^{q_1} \cdots y_m^{q_m}$
is integrable on $E^m$ provided $k > \dfrac{m}{2} + \sigma$, where
$\sigma = | q| \sum\limits^m_{j=1} q_j$. We have 
$$
\tilde{U}_o (y) y_1^{q_1} \cdots y_m^{q_m} = \frac{y_1^{q_1} \cdots y
_m^{q_m}}{1 + | \sum^m_{i = 1} y^2_i |^{k/2}} \tilde{U}_o (y) \left(1+ |
\sum^m_{i = 1} y^2_i |^{k/2}\right). 
$$

Now, in polar coordinates 
$$
dy = dy_1 \cdots dy_m = r^{m - 1} dr d \Omega_{m - 1}
$$
$(\Omega_{m- 1} $ is the surface of unit sphere in $E^m)$. So
$\dfrac{y_1^{q_1} \cdots y_m^{q_m}}{1 + | \sum\limits^m_{i = 1} y^2_i
 |^{k / 2}}$ is square integrable in $| z | > \alpha (Z \in E^m)$ if
$2 | q | - 2k + m - 1 < - 1$, i.e., if $k > \dfrac{m}{2}+
\sigma$. Already\pageoriginale we know that $U_o (y) (1 + \sum\limits_{i = 1}^m
y^2_i)^{k/2}$ is square integrable in $| z | > \alpha$. So $U_o (y)
y_1^{q_m}\cdots y_m^{qm}$, begin the product of two square integrable
functions, is 
integrable in $| z | > \alpha$. We see also that $U_o (y) y_1^{q_1}
\cdots y_m^{q_m}$ is integrable in $|z|\le \alpha$. 

Thus if $k > \dfrac{m}{2} + | q|, U_o (y) y_1^{q_1} \cdots y_m^{q_m}$
is integrable over $E^m$. 

Suppose $k > \dfrac{m}{2} +\sigma$, ($\sigma > 0$ integer). Then
$\tilde{U}_o(y) \in L_2 \cap L_1$ so that $(\mathscr{F}^{-1}
\tilde{U}_o) (y) = \int\limits_{E^m} \tilde{U}_o (y) \exp (2 \pi i
y. x) dy, a.e $ on $E^m$; i.e.,$ \tilde{u}_o (x) = \int\limits_{E^m}
\tilde{U}_o (y) \exp (2 \pi i y. x) dy a. e.$ on $E^m$. 

Let $| q | \le \sigma (k > \dfrac{m}{2} + \sigma) ; $ then
$$
\displaylines{\hfill 
 D^{(q)}_x \left\{ \tilde{U}_o (y) \exp (2 \pi i. y. x)\right\} =
 \tilde{U}_o (y) 
 \prod^m_{j = 1} (2 \pi i y_j)^{q_j} \exp 2 \pi i y. x \hfill\cr
 \text{and}\hfill 
 \big | \tilde{U}_o (y) \prod^m_{j = 1} (2 \pi i y_j)^{q_j} \exp 2 \pi
 i y x \big | \le \big | \tilde{U}_o (y) \prod^m_{j = 1} (2 \pi i
 y_j)^{q_j} \big| \hfill }
$$
and $\big | \tilde{U}_o (y) \prod\limits^{m}_{j = 1} (2 \pi i
y_j)^{q_j} \big |$ is a function independent of $x$ and summable (as a
function of $y$) over $E^m$. Therefore $D^q(x) \tilde{u}_o (x)$ exists
and $D^{(q)} \tilde{u}_o (x) = \int\limits_{E^m} \tilde{U}_o (y)
\prod\limits_{j = 1}^{m} (2 \pi i y_j)^{q_j} (\exp 2 \pi i y . x)
dy$. 

This representation also shows that $D^{(q)} \tilde{u}_o (x)$ is
continuous. Thus $\tilde{u}_o (x)$ is $\sigma$-times continuously
differentiable; so $u_o (x) $ is $\sigma$-times continuously
differentiable in $R_1$. 
