\chapter{Lecture 4}\label{chap4}

\section{Local weak compactness of a Hilbert space}\label{chap4:sec1}\pageoriginale

\begin{theorem*}
 Let $\{x_n\}$ be a bounded sequence of elements of a Hilbert space
 (i.e., $|| x_n || \leq C < \infty, n = 1, 2, \ldots )$; then we
 can choose a subsequence of $\{x_n\}$ which converges weakly to an
 element of $X$. 
\end{theorem*}

\begin{proof}
 Let $M$ be the closed linear space spanned by $\{x_n \}$. ($M$ is
 the closure in the sense of the norm of the set of all finite
 linear combinations $\sum \alpha_i x_i$ of the
 elements$\{x_i\}$). $M$ is separable, there exists a countable set
 of elements $\{y_n\}$ which is dense in $M$. We may take for example,
 the rational linear combinations of $\{ x_i \}$ if $X$ is real 
 and if $X$ is complex, linear combinations of $\{x_i\}$ with
 coefficients of the form $p + iq, p,q$ rational. 
\end{proof}

For each $y_k$ from $\{y_n\}$ the sequence $\big\{(x_n, y_k)\big\}$
is bounded ; $|(x_n. y_k)| \leq ||x_n|| || y_k || \leq C || y_k ||
$. By the Bolzano - Weierstrass theorem and a diagonal process we can
find a subsequence $\{x'_n\}$ of $\{x_n\}$ such that $\big\{(x'_n,
y_k)\big\}$ converges for every $k$. Actually $\{(x'_n, z)\}$
converges for each $x \in X$. To prove this, let $z = y + \omega$
where $y = P_M z, \omega \in M^\perp$. Then $(x_n, z ) = (x_n, y)$
and we have to prove that $\{(x_n, y )\} (y \in M)$ is convergent. We
have 
\begin{align*}
 |(x_{n'} - x_{m'}, y )| & = |(x_{n'} - x_{m'}, y - y_k + y_k)|\\
 & \leq |(x_{n'} - x_{m'}, y_k)| + |(x_n - x_m., y - y_k )|\\
 & \leq |(x_{n'} - x_{m'}, y_k)| + || x_n - x_m ., y - y_k ||\\
 & \leq |(x_{n'} - x_{m'}, y_k)| + 2C||y-y_k||.
\end{align*}

Since\pageoriginale $\{(x_n, y_k)\}$ is convergent and $\{y_k\}$ is dense in $M$,
it follows that $(x_n, y)$ is a Cauchy sequence; so $\{(x_n, y)\}$
is convergent. Define $g(z) = \lim\limits_{ n \to \infty}$ and $|f(z)|
= |g(z)| = \lim\limits_{ n' \to \infty} | (x_n ., z)| \leq C || z||,
f(z)$ is continuous. By the Riesz theorem there exists and element
$x_\infty \in X$ such that $f(z) = (z, x_\infty)$ for each $z \in
X$. Since $\lim\limits{n '\to \infty} (x_n, z) = (x_\infty, z )$ for
each $z \in X, w - \lim\limits_{ n \to \infty} x_n' = x_\infty $ (by
Riesz's Theorem)  

We mention without proof that $L_p (\alpha, \beta)$, $1 < p < \infty$
is locally weakly compact. But $L(\alpha, \beta)$, $L_\infty (\alpha,
\beta)$ and $C[\alpha,
 \beta]$ are not locally weak\-ly compact. 

We next prove a theorem which will be needed in the study of Cauchy's
problem. 

\section{Lax-Milgram theorem}\label{chap4:sec2} 

Let $B(u, v )$ be a bilinear functional on a real Hilbert space $X$ such that 
\begin{enumerate}[(i)]
\item there exists a $\wp > 0$ such that $|B (u, v) |\leq \wp || u
 || || v||$ for all $u, v \in X$, 
\item there exists a $\delta > 0$ such that $\delta || u||^2 \leq B
 (u, u)$ for each $u \in X$. 

 Then there exists a linear operator $S$ from $X$ to $X$ such that 
 $$
 (u, v ) = B(u, S v)
 $$
 and $|| S || \leq \delta^{-1}$. 
\end{enumerate}

\begin{proof}
 Let $V$ be the set of elements $v$ for which there exists an element
 $v^*$ such that $(u, v )= B (u, v^*)$ for all $u \in X$. ($V$ is
 non-empty;\pageoriginale $0 \in V$). $v^*$ is uniquely determined by $v$. For, if
 $w \in X$ be such that $B(u, w) = 0$ for all $u$, then $w = 0 $ as
 $\delta || w^2 || \leq B (w, w ) = 0$ or $|| w|| = 0$. $V$ is a
 linear subspace. We have an additive operator $S$ with domain $V$,
 defined by $S v = v^*$. $S$ is continuous; 
 $$
 \delta || S v ||^2 \leq B (Sv, Sv) = (Sv, v)\leq ||Sv|| ||v||
 $$
 so that $|| Sv|| \leq \delta^{-1} || v ||$ (if $|| Sv|| = 0$ this is
 trivially true). Moreover $V$ is closed subspace of $X$. For, of
 $v_n \in V$ and $v_n \to v \in X$, then $S v_n$ is a Cauchy
 sequences and so has a limit $v^*$; but $(u, v_n) \to (u, v )$ and
 by $(i)$ $B(u, Sv_n) \to B(u, v^*)$ so that $(u, v ) = B(u, v^*)$
 for each $u$; so $v \in V$. The proof will be complete if we show
 that $V = X$. Suppose $V \neq X$. Then there exists $w \in X$ such
 that $w \neq 0$ and $(w, v) = 0$ for each $v \in V$. Consider the
 functional, as 
 $$
 |F(z)| = | B(z, w)|\leq \wp || z || || w||.
 $$
 
 So by Riesz's theorem, there exists, $w' \in X$ such that $B(z, w) =
 (z, w')$ for each $z \in X$. So $w' \in V$ and $Sw' = w$. So 
 \begin{align*}
  \delta || w ||^2 \leq B(w, w) & = (w,  w')\\
  & = 0,\\
  \text{i.e.,} \hspace{2cm} w & = 0
 \end{align*}
 which is a contradiction. 
\end{proof}

\part{Semi-group Theory}\label{chap4:p2} 

\begin{defi*}
 Let\pageoriginale $\{T_t\}_{ t \geq 0}$ be a one-parameter family of linear
 operators on a Banach space $X$ into itself satisfying the following
 conditions: 
 \begin{enumerate}[\rm (1)]
 \item $T_t T_s = T_{ t +s}, T_o = I, I$ denoting the identity
  operator on $X$ (Semi -group property). 
 \item $s - \lim\limits_{ t \to t_0 } T_ t x = T_{t_\circ}x\leq 0$ and
  each $x \in X$(strong continuity). 
 \item there exists a real number $\beta \geq 0$ such that $|| T_t ||
  \leq e^{ \beta t}$ for $t \geq 0$. 
 \end{enumerate}
 
 We call such a family $\{T_t\}$ a {\em semi group} of linear
 operators of {\em normal type} on the Banach space $X$, or simply a
 {\em semi - group}. 
\end{defi*}

\begin{remark*}
 The third condition may look a bit curious but it is nothing but a
 restriction of the order of $|| T_t||$ near $ t = 0$, because we can
 prove the following. 
\end{remark*}

\begin{prop*}
 The two conditions $(1)$ and $(2)$ imply the following: 

 $(3')$ $\lim\limits_{ t \to \infty} t^{-1} \log || T_t || = \wp <
 \infty (\wp$ may be$-\infty )$. 

 $(4)$ $||T_t||$ is bounded in any bounded interval $[0, t_o], o <
 t_o < \infty$. 
\end{prop*}

\begin{proof}
 We first prove $(4)$. Suppose $|| T_t||$ is unbounded in some
 interval $[ 0, t_o], 0, < t_o < \infty $. Then there would exist a
 sequence $\{t_n\} ~(n = 1 1, 2, \ldots )$ such that $\|T_{ t_n}\| \geq
 n$ and $o \leq \lim\limits_{ n \to \infty} t_n = t_\infty \leq
 t_\circ < \infty$. Since $\big\{ || T_{
  t_n}||\big\}$ is unbounded, by the resonance theorem,
 $\big\{|| T_{t_n}x|| \big\}$ is unbounded at least for
 one $x \in X$; but by strong continuity, $ s - \lim\limits_{ n \to
  \infty} T_{ t_n} x= T_{t_\infty}x$ for each $x \in X$. This is a
 contradiction.  

 To\pageoriginale prove $(3')$, let $p(t) = \log|| T_t||, p(t) < \infty$ (may be
 $-\infty$). Since $|| T_{t + s} ||= ||T_t T_s|| \leq || T_t || ||
 T_s ||$, we have $p(t+s)\leq p(t) + p(s)$. Let $\wp
 \inf\limits_{t > 0} t^{-1} p(t)$. is either finite or $- \infty$. We
 shall show that $\lim\limits_{t \to \infty t^{-1} p(t)}$ exists and is equal to
 $\wp$. Assume, first, 
 $\wp$ is finite. Choose for any $\mathcal{E} > o$, a number $a>o$ in
 such a way that $p(a) \leq (\wp + \mathcal{E}) a$. Let $n$ be an
 integer such that $n a \leq t < (n +1)a$. 

 Then 
 \begin{align*}
  \wp & \leq \frac{p(t)}{t} \leq \frac{p(na)}{t} + \frac{p(t - na)}{t}\\
  & \leq \frac{na }{t} \frac{p(a)}{a} + \frac{p(t -na)}{t}\\
  & \leq \frac{na}{t} (\wp + \mathcal{E})+\frac{p(t - na)}{t}.
 \end{align*}

 Letting $t \to \infty, \dfrac{p(t -na )}{t}$ tends to zero since
 $p(t -na )$ is bounded from above (since, as we have proved above,
 $|| T_s ||$ is bounded in any finite interval of $s$). Thus
 $\lim\limits_{t \to \infty} t^{-1} p(t) = \wp $. The case $\wp = -
 \infty$ can be treated similarly.
\end{proof}
