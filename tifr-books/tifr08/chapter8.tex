\chapter{Lecture 8}\label{chap8} % Lecture 8

\section{An application of the representation theorem}\label{chap8:sec1}

In\pageoriginale $C[o, \infty]$ consider $(T_t x)(s)=x(t+s)$. By the representation
theorem 
\begin{align*}
 (T_t x) (s)= x(t+s) &= s-\lim_{n \to \infty} \exp \,(t A J_nx) (s)\\
 & =s-\lim_ {n \to \infty} \sum_{m =o}^\infty \frac{t^m}{m!} \,(A J_n)^m x(s)
\end{align*}
uniformly in any bounded interval. From this we get an operation
theoretical proof of the Weirstrass approximation theorem. Let $z(s)$
be a continuous function on the closed interval $[0, \alpha ],0 <
\alpha < \infty$. Let $x(s) \in C[o, \infty]$ be such that
$x(s)=z(s)$ for $s \in [0, \alpha]$ (such functions trivially
exist). Put $s=0$ in the above formula 
$$
(T_t x) (0)=x(t) = s-\lim_{n \to \infty} \sum_{m=o}^\infty \frac{t^m
 \left[(AJ_n)^{m_x}\right]\,(0)}{m!} 
$$
uniformly in $[0, \alpha]$. Thus shown that $z(s)$ is the uniform
limit of polynomials on $[0, \alpha]$. 

\section[Characterization of the...]{Characterization of the infinitesimal general of a
  semi-group}\label{chap8:sec2} 

We next wish to characterize the infinitesimal generator of a
semi-group by some of the properties we have established. First we
prove the 

\begin{prop*}%prop
 Let $A$ be an additive operator on a Banach space $X$ into itself
 with the following properties: 
 \begin{enumerate}[\rm (a)]
 \item $\mathscr{D}(A)$ is dense in $X$;
 \item there exists a $\beta \ge 0$ such that for $n > \beta$ the
  inverse $J_n=(I-n^{-1} A)^{-1}$ exists as a linear operator
  satisfying 
  $$
  ||J_n|| \le (1-n^{-1}\beta)^{-1}\,(n > \beta).
  $$
 \end{enumerate}
\end{prop*}

Then\pageoriginale we have 
\begin{enumerate}[i)]
\item $A J_n x=n(J_n -I)x, x \in X$
\item $A J_n x= J_n \,Ax=n(J_n -I)x,x \in \mathscr{D}(A)$
\item $s-\lim\limits_{n \to \infty} J_n x=x$, for $x \in X$.
\end{enumerate}

\begin{proof}
 i) and ii) are evident. To prove iii) let $y \in \mathscr{D}(A)$.
\end{proof}

Then $y=J_n y- n^{-1}J_n A y$ and hence
\begin{align*}
 || y-J_n y || &\le n^{-1} ||J_n || || Ay ||\\
 & \le n^{-1} (1-n^{-1}\beta)^{-1} || Ay || \to 0 \text{ as } n \to \infty.
\end{align*}

Let $x \in X$. Since $\mathscr{D}(A)$ is dense in $X$, given
$\varepsilon >0$, there exists $y \in \mathscr{D}(A)$ such that $|| y-
x || \le \varepsilon $. We then have 
\begin{align*}
 || x- J_n x || & \le || x-y || + || y- J_n y || + ||J_n y- J_n x || \\
 & \le \varepsilon + || y- J_n y || + (1-n^{-1} \beta)^{-1} \varepsilon. 
\end{align*}
 
As $|| y- J_n y || \to$ as $n \to \infty$,
$$
\overline{\lim_{n \to \infty}} || x- J_n x || \le \varepsilon,
$$
and $\varepsilon$ being arbitrary positive number, $iii)$ is proved.

\begin{theorem*}%Thm
 An additive operator $A$ with domain $\mathscr{D}(A)$ dense in a
 Banach space $X$ and with values in $X$ is the infinitesimal
 generator of a uniquely determined semi-group $\{ T_t\}$ with $||
 T_t || \le e^{\beta t}$ if (and only if ), for $n > \beta$, the
 inverse $J_n=(I- n^{-1} A)^{-1}$ exists as a linear operator
 satisfying $|| J_n || \le (1-n^{-1}\beta)^{-1}$. 
\end{theorem*} 
 
\noindent \textit{Proof.}
 We put $T^{(n)}_t=(\exp t A J_n)$. We have 
 \begin{align*}
  \| T_t^{(n)}\| & \le \exp (-nt) \exp (nt || J_n ||) \\
  & \le \exp \frac{\beta_t}{1-n^{-1} \beta},
 \end{align*} 
 $$
 \displaylines{\hfill 
 D_t T_t^{(n)}x = A J_n T_t^{(n)} x = T_t^{(n)} A J_n x, x \in
 X,\hfill \cr
 \text{and}\hfill 
 T^{(n)}_t x-x = \int^{t}_o T^{(n)}_s A J_n x ~ds.\hfill \Box}
 $$
 
It\pageoriginale is easy to see $J_n J_m = J_m J_n$; so $A J_n =n(J_n - I)$
commutes with $T^{(m)}_t = \exp (t A J_m)$. Thus, as in the proof of
the representation theorem, we have, for any $x \in \mathscr{D}(A)$, 
{\fontsize{10pt}{12pt}\selectfont
\begin{align*}
 || T^{(m)}_t x- T^{(n)}_t x || &= || \int^t_o D_s \bigg\{
 T^{(n)}_{t-s} T^{(m)}_s x \bigg\} ds ||\\ 
 &= \| \int_o^t T^{(n)}_{t-s} T^{(m)}_s (A J_m -A J_n )x~ ds ||\,
 (\text{as} D_s T^{(m)}_s x= T^{(m)}_s AJ_m x)\\ 
 &\le || (J_m A- J_n A)x || \int^t_o \exp
 \frac{\beta(t-s)}{1-n^{-1}\beta}. \exp \frac{\beta s}{1-m^{-1}\beta}
 ds 
\end{align*}}\relax

So $\lim\limits_{m,n \to \infty} ||T^{(m)}_t x- T^{(n)}_t x || =0$
uniformly in any finite interval of $t$. Let $y \in X$. Given
$\varepsilon > 0$, there exists $x \in \mathscr{D} (A)$ such that
$||y-x || \le \varepsilon$. Then 
\begin{align*}
 \| T^{(m)}_t y- T^{(n)}_t y \| &= || T^{(m)}_t y-T^{(n)}_t x || +
 \|T^{(m)}_t x- T^{(n)}_t x \| \\ 
 &+ || T^{(n)}_t x - T^{(n)}_t y \| \\
 &\le \exp \left(\frac{\beta t}{1-m^{-1} \beta}\right) \varepsilon +
 \|T^{(m)}_t x- T^{(n)}_t x \| +\exp \frac{\beta t}{1-n^{-1}\beta} \varepsilon. 
\end{align*} 

So ${\varlimsup\limits_{m,n \to \infty}} || T^{(m)}_t y- T^{(n)}_t y
|| \le \varepsilon$. $2 \exp (\beta t)$ uniformly in any finite
interval of $t$. Therefore, by the completeness of $X$,
$s-\lim\limits_{n \to \infty} T^{(n)}_t y = T_t y$ exist and the
convergence is uniform in any bounded interval of $t$. 
 
By the resonance theorem $T_t$ is a linear operator; since $T^{(n)}_t$
are strongly continuous in $t$ and the convergence is uniform in any
bounded interval of $t, T_t$ is strongly continuous in $t$. Also, 
\begin{align*}
 || T_t || &\le \lim_{n \to \infty} || T^{(n)}_t || \text{ (Cor. to
  response theorem ) }\\ 
 &\le \exp (\beta t)
 \end{align*} 
 
We\pageoriginale now prove that $T_t T_s = T_{t+s} (T =I$, evidently).
 
Since $T^{(n)}_t T^{(n)}_t =T^{(m)}_{t+s}$,
\begin{align*}
 || T_{t+s}x - T_t T_s x || &\le || T_{t+s}x- T^{(n)}_{t+s} x || + ||
 T^{(n)}_{t+s}x -T^{(n)}_t T^{(n)}_s x ||\\ 
 &\hspace{1cm} + \|T^{(n)}_t T^{(n)}_s x-
 T^{(n)}_t T_s x || + || T^{(n)}_t T_s x- T_t T_s x || \\ 
 & \le || T_{t+s} x- T^{(n)}_{t+s} || + \exp \frac{\beta t}{1-n^{-1}
  \beta}\| T^{(n)}_s- T_s x \|\\ 
 & \hspace{1cm} + || T^{(n)}_t (T_s x)- T_t (T_s x)||\\ 
 & \to 0 ~\text{ as }~ n \to \infty. 
\end{align*} 
 
Finally let $A'$ be the infinitesimal generator of the semi-group
$T_t$ We shall show that $A' =A$. For this it is enough to prove that
$A'$ is an extension of $A$ (i.e., $x \in \mathscr{D}(A)$ implies $x
\in \mathscr{D}(A')$ and $A' x = Ax$). For, $(I-n^{-1}A')(n > \beta)$
maps $\mathscr{D}(A')$ onto $X$ in a one-one manner; by assumption
$(I-n^{-1}A)$ maps $\mathscr{D}(A')$ onto $X$ in a one-one manner; but
on $\mathscr{D}(A), (I-n^{-1}A)= (I-n^{-1}A')$ and hence
$\mathscr{D}(A)= \mathscr{D}(A')$. To prove that $A'$ is an extension
of $A$, we start with the formula  
 $$
 T^{(n)}_t x-x = \int^t_o T^{(n)}_s A J_n x ds, x \in X.
 $$
 
If $x \in \mathscr{D}(A)$
\begin{align*}
 || T_s Ax- T^{(n)}_s A J_n x || & \le || T_s A x- T^{(n)}_s A x || +
 || T^{(n)}_s A x- T^{(n)}_s A J_n x || \\ 
 & \le || (T_s- T^{(n)}_s) Ax || + \exp \frac{\beta s}{1 \beta
 n^{-1}} || Ax- J_n A x || 
 \end{align*} 
 $$
 (A J_n x= J_n A x, \text{ if } x \in \mathscr{D}(A) ).
 $$
 
As $n \to \infty$ the first on the right tends to zero, uniformly in
any bounded interval of $s$; the second term also tends to zero,
uniformly in any bounded interval of $s$, as $\exp \dfrac{\beta
 s}{1-\beta n^{-1}}$ stays in such an interval\pageoriginale and we know that 
 $$
 s-\lim_{n\to \infty} J_n y=y, y \in X.
 $$
 
Hence
\begin{align*}
 T_t x-x &= s-\lim_{n \to \infty} (T^{(n)}_t x-x) = s-\lim_{n\to
  \infty} \int^t_o T^{(n)}_s A J_n x ~ ds \\ 
 &= \int^t_o s-\lim_{n\to \infty} (T^{(n)}_s A J_n x )ds \\
 &= \int^t_o T_s Ax ds 
\end{align*} 
(using the uniformly of convergence in $[o,t]$). Therefore 
$$
s-\lim\limits_{n\to \infty}\frac{T_t x-x}{t} = T_o A x=A x.
$$
i.e., if $x \in \mathscr{D}(A)$ then $x \in \mathscr{D}(A')$ and $A' x=Ax$.

The uniqueness of the semi-group $\{T_t\}$ with $A$ as the
infinitesimal generator follows from the representation theorem for
semi-groups pro\-ved earlier. 
