\chapter{Lecture 19}\label{chap19} % lecture 19

\section{The Cauchy problem for the wave equation}\label{chap19:sec1}

We\pageoriginale consider the Cauchy problem for the `wave equation' in the
$m$-dimensional Euclidean space $E^m$: 
\begin{equation*}
 \begin{cases}
  \frac{\partial^2 u(t, x)}{\partial t^2}=A u (t,x), & x \in E^m \\
  u(0, x)= f(x), & u_t (0, x) ~g(x),~f, g, \text{ given },
 \end{cases}
\end{equation*}
$$
\displaylines{\text{where}\hfill 
A = a^{ij}(x) \frac{\partial^2}{\partial x_i ~ \partial x_j}+ b^i (x)
\frac {\partial}{\partial x_i}+ c(x)\hfill } 
$$
is a second-order elliptic operator. This problem is equivalent to the
matricial equation 
\begin{equation*}
 \begin{cases} 
  \frac{\partial}{\partial t} \begin{pmatrix} u (t, x)\\ v (t,
   x) \end{pmatrix}= \begin{pmatrix} 0 & I \\ A &
   0 \end{pmatrix} \begin{pmatrix} u (t, x)\\ v (t,
   x) \end{pmatrix} (I = \text { identity }).\\ 
  \begin{pmatrix} u (o, x)\\ v (o, x) \end{pmatrix}
  = \begin{pmatrix} f (x)\\ g (x) \end{pmatrix} \tag{1} 
 \end{cases} 
\end{equation*}

We may apply the semi-group theory to integrate $(1)$, by considering,
in a suitable Banach-space the ``resolvent equation" 
$$
\begin{bmatrix}
 \begin{pmatrix}
  I & o \\ o & I
 \end{pmatrix}
 -n^{-1}
 \begin{pmatrix}
  o & I \\ A & o
 \end{pmatrix}
\end{bmatrix}
\begin{pmatrix}
 u \\ v
\end{pmatrix}
=
\begin{pmatrix}
 f \\ g
\end{pmatrix}
$$
 for large $|n|$ ($n$, integral) and obtaining the estimate
 $$ 
 || 
 \begin{pmatrix} 
  u \\ 
  v 
 \end{pmatrix}
 || \leq (1+ |n^{-1}|
 \beta) || 
 \begin{pmatrix} 
  f \\ 
  g 
 \end{pmatrix} || 
 $$
 with a positive $\beta$ independent of $u, f$ and $g$. As a matter
 of fact, the estimate implies (see Lecture $9$) that
 $\begin{pmatrix} 0 & I \\ A & 0 \end{pmatrix}$ is the infinitesimal
 generator of a \textit {group} $\{T_t \}_{-\infty < t < \infty}$ and 
 $$
 T_t 
 \begin{pmatrix} 
  f (x) \\ 
  g(x) 
 \end{pmatrix} 
 = 
 \begin{pmatrix} 
  u (t, x) \\ 
  v (t, x)
 \end{pmatrix} 
 $$
 will\pageoriginale give a solution of $(1)$ if the initial functions $f(x)$ and
 $g(x)$ are prescribed properly. 
 
 We have the
\begin{theorem*}
  Suppose that the coefficients $a^{ij}(x), b^i(x)$ and $c(x)$ are
  $C^\infty$ and that there exists a positive constant $\lambda$
  such that 
  $$
  a^{ij}(x) \xi_i \xi_j \geq \lambda \sum_i \xi^2_i
  $$
  $(x \in E^m, (\xi_1, \ldots, \xi_m) \in E^m)$. Assume further that
  \begin{multline*}
   \eta = \max \left\{\sup_{x,i,j}| a^{ij}(x)|, \sup_{x,i,j,k} \left
   |\frac{\partial a^{ij}}{\partial x_k} \right|\right.\\
   \left. \sup_{x;i,j,k,s} \left|\frac{\partial^2 a^{ij}}{\partial x_k ~
    \partial x_s} \right|, \sup_{x;i} | b^i (x)|, \sup_{x;i,k}
   \frac{\partial b^i}{\partial x_k}, \sup_x | c(x)| \right\} 
  \end{multline*}
  is finite. Then there exists a positive constant $\beta$ such that for
  sufficiently small $\alpha_\circ$, the equation $(1)$ is solvable in
  the following sense: for any pair of $C^\infty$ functions $\big\{f(x),
  g(x)\big\}$ on $E^m$ for which $A^k f, A^k g$ and their partial
  derivatives are square integrable (for each integer $k \geq 0$) over
  $E^m$, the equation $(1)$ admits of a $C^\infty$ solution $u(t, x)$
  satisfying the ``energy inequality" 
  \begin{gather*}
   ((u - \alpha_\circ A u, u)_\circ + \alpha_\circ (u_t, u_t)_\circ
   )^{\frac{1}{2}}
   \leq \exp ( \beta |t| ((f-\alpha_\circ A f, f)_\circ + \alpha_\circ
   (g, g)_\circ )^{\frac{1}{2}} 
  \end{gather*}
\end{theorem*}
 
\begin{proof}
The proof will be carried out in several steps.

 \noindent
 {\bf First step:} Let $H$ be the space of real-valued $C^\infty$
 functions in $E^m$ for which 
 $$
 || f ||_1 = \left\{\int \limits_{E^m} | f |^2 dx + \sum_i \int
 \limits_{E^m} | f_{x_i}|^2 dx \right\}^{\frac{1}{2}} < \infty, 
 $$
 and\pageoriginale let $\tilde{H}_1 (E^m)$ be the completion of $H$ with respect to
 the norm $|| ~||_1$. The completion of $H$ with respect to $|| f
 ||_\circ = \left\{ \int \limits_{E^m} | f |^2 dx \right
 \}^{\frac{1}{2}}$ will be denoted by $\tilde{H}_\circ
 (E^m). \tilde{H}_\circ (E^m)$ and $\tilde{H}_1(E^m)$ are Hilbert
 spaces; actually $\tilde{H}_\circ (E^m)= H_\circ (E^m) = L_2
 (E^m)$. 
 
 One can prove that there exists $\chi > 0$ and $\alpha_\circ > 0$
 such that for $0 < \alpha < \alpha_\circ$ there correspond $\gamma >
 0$ and $\delta_\alpha > 0$ satisfying 
\begin{align*}
 \delta_\alpha || f ||^2_1 & \leq 
 \begin{cases}
  (f -\alpha Af, f)_\circ & \text{ for } f \in H, A f \in H_\circ\\
  (f -\alpha A^*f, f)_\circ &\text{ for } f \in H \quad A^* f \in H_\circ.
 \end{cases}\\ 
  | (f -\alpha Af, g)_\circ | & \leq (1 + \alpha \gamma) || f ||_1 ||
  | g ||_1 ~ \text{ for } f, g \in H, A f \in H_\circ.\\ 
  | (f -\alpha A^*f, g)_\circ | & \leq (1 + \alpha \gamma ) || f ||_1
  | || g ||_1 ~ \text{ for } f, g \in H; A^* f \in H_\circ.\\ 
  | (af, g)_\circ - (f, Ag)_\circ | & \leq \chi || f ||_1 || g
  | ||_\circ, \text{ for } f, g \in H; Af, Ag \in H_\circ. 
\end{align*}
 (The proofs of these inequalities will be given in the next lecture).
\end{proof}
 Thus the bilinear form
 $$
 B^\wedge_\alpha(u,v) = (u - \alpha A^* u, v)_\circ \text{ for } u, v
 \in H, A^* u \in H_\circ 
 $$
 can be extended to a bilinear functional $B_\alpha (u, v)$ on $H_1$
 satisfying 
 \begin{equation*}
  \begin{cases}
   \delta_\alpha || u ||^2_1 \leq B_\alpha (u, u) \\
   | B_\alpha (u, v) | \leq (1+ \alpha \gamma ) || u ||_1 || v ||_1.
  \end{cases}
 \end{equation*} 
 
 \medskip
 \noindent
 {\bf Second step:} Let $0 < \alpha \leq \alpha_\circ$. For any $f
 \in H$, the equation $u -\alpha A u =f$ admits of a uniquely
 determined solution $u(x) =u_f (x) \in H$. 
 
\begin{proof}
 The additive functional $F(u) = (u, f)_\circ$ is bounded on $H_1$, because
 $$
 | F (u)| = |(u, f)_\circ | \leq || u ||_\circ || f ||_\circ \leq ||
 u ||_1 || f ||_\circ. 
 $$

 So,\pageoriginale by Riesz's representation theorem, there exists a uniquely
 determined $v (f) \in \tilde{H}_1$ such that 
 $$
 (u, f)_\circ =(u, v(f))_1.
 $$
 
 By the Lax-Milgram theorem (see lecture $4$) as applied to the
 bilinear form $B_\alpha (u, v)$ in $\tilde{H}_1$, there corresponds a
 uniquely determined element $Sv(f)$ in $\tilde{H}_1$ such that 
 $$
 (u, f)_\circ = (u, v(f))_1 = B_\alpha (u, Sv(f)), \text{ for } u \in H_1.
 $$
 $u_\circ =Sv(f)$ is a weak solution of the equation $u - \alpha A u = f,
 i.e$., for each $u \in \mathscr{D}^\infty (R)$ we have $(u, f)_\circ =
 (u -\alpha A^* u, Sv(f))_\circ$. In fact, let $\{v_k\} \subset H$ be a
 sequence such that $v_k \to Sv(f)$ in $\tilde{H}_1$; then, for 
 \begin{align*}
  u \in \mathscr{D}^\infty (R), B_\alpha (u, S v(f)) & = \lim_{n \to
   \infty} B_\alpha (u, v_n)\\ 
  & = \lim_{n \to \infty} (u -\alpha A^* u, v_n)\\
  & = (u- \alpha A^* u, S v (f)).
 \end{align*}
 Since $f$ is $C^\infty$ in $E^m$ and $A$ is elliptic, $u_\circ = S
 v(f)$ is almost everywhere equal to a $C^\infty$ function 
(Weyl-Schwartz theorem). We thus have a solution $u_\circ \in H$ of the
 equation $u -\alpha A u=f$. The uniqueness of the solution follows
 from the inequalities given in the first step. 
\end{proof}

 \medskip
 \noindent
{\bf Third step:} If the integer $n$ is such that $| n |^{-1}$ is
sufficiently small, then for any pair of functions $\{f,g\}$ with $f,
g \in H$ and $A f \in H_\circ$, the equation 
 \begin{equation*}
  \begin{bmatrix}
   \begin{pmatrix}I & 0 \\ 0 & I\end{pmatrix}
    -n^{-1}
    \begin{pmatrix}0 & I \\ A & 0\end{pmatrix}
  \end{bmatrix}
  \begin{pmatrix}u \\ v\end{pmatrix}
   =
   \begin{pmatrix}f \\ g \end{pmatrix}
   \tag {2}
 \end{equation*} 
 or
 \begin{align*}
  u  - n^{-1} v & = f \\
  v  - n^{-1} A u & =g
 \end{align*}
 admits\pageoriginale of a uniquely determined solution $\{u,v\}~ u, v \in
 H$. Moreover, we have 
 $$
 \left[ B_{\alpha_\circ}(u, u) + \alpha_\circ (v, v) \right
 ]^{\frac{1}{2}} \leq (1+ | n|^{-1} \beta) (B_{\alpha_\circ}(f,f) +
 \alpha_\circ (g,g)_\circ )^{\frac{1}{2}} 
 $$
 with a positive constant $\beta$.

\begin{proof}
 Let $u_1, v_1 \in H$ be such that
 $$
 u_1 - n^{-2} A u_1 = f \qquad v_2 - n^{-2} A v_2 = g.
 $$
 (See the second step). Then
 $$
 u = u_1 + n^{-1}v_2 \qquad v=n^{-1} A u_1 + v_2
 $$
 satisfies (2).
 
 We have
 $$
 A u = n(v-g) \in H \subset H_\circ, \quad A v = n(Au - Af) \in H_\circ.
 $$
 
 We may therefore apply the inequalities of the first step.
 
 Thus by (2),
 \begin{align*}
  (f -\alpha_\circ Af, f)_\circ & = (u - n^{-1} v - \alpha_\circ A(u -n^{-1}
  v), u -n^{-1}v)_\circ \\ 
  & = (u - \alpha_\circ Au, u)_\circ -2n^{-1} (u, v)_\circ +
  \alpha_\circ n^{-1} (Au, v)_\circ \\ 
  & \qquad + \alpha_\circ n^{-1} (Av, u)_\circ + n^{-2} (v - \alpha_\circ A
  v, v)_\circ 
 \end{align*}
 and
 \begin{align*}
  \alpha_\circ (g, g)_\circ & = \alpha_\circ (v -n^{-1} A u, v -
  n^{-1} A u)_\circ \\ 
  & = \alpha_\circ (v,v)_\circ - \alpha_\circ n^{-1} (v, A u)_\circ -
 \alpha_\circ n^{-1} (Au, v)_\circ + \alpha_\circ n^{-2}(A~A)_\circ 
\end{align*}
Hence\pageoriginale 
 \begin{align*}
  B \alpha_\circ & (f, f)_\circ + \alpha_\circ (g, g)\\
  \geq & B \alpha_\circ (u,u)+ \alpha_\circ (v, v)_\circ -2|n|^{-1}
  (u,v)_\circ - \alpha_\circ |n|^{-1} | (Av, u)_\circ - (Au, v)_\circ
  |\\ 
  \geq & B \alpha_\circ (u,u)+ \alpha_\circ (u, v)_\circ -2|n^{-1} |
  ~|| u ||_1 || v ||_\circ - \alpha_\circ |n|^{-1} \chi || u ||_1 || v
  ||_\circ \\ 
  \geq & B \alpha_\circ (u,u)+ \alpha_\circ (v,v)_\circ\\ 
  & \qquad -|n^{-1}|
  \left\{ || u ||^2_1 \tau + \tau^{-1} || v ||^2_\circ 
  + \frac{\alpha_\circ}{2} \chi ( || u ||^2_1 \tau + \tau^{-1} || v
  ||^2_\circ )\right \} ( \tau > 0) 
 \end{align*}
 
 Thus, by taking $\tau > 0$ sufficiently large and then taking $| n |$
 sufficiently large, we have the desired inequality. 
\end{proof}

\medskip
 \noindent
 {\bf Fourth step:} The product space $\tilde{H}_1 \times
 \tilde{H}_\circ$ is a Banach space with the norm 
 $$
 || (^u_v) || = [B_{\alpha_\circ} (u, u) + \alpha_\circ (v, v)]^{\frac{1}{2}}
 $$
 We define now an operator $\mathscr{O}$ in $\tilde{H}_1 \times
 H_\circ:$ the domain of $\mathscr{O}$ consists of the vectors
 $(^u_v) \in H_1 \times H_\circ $ such that $u, v \in H$ and
 $A(u-n^{-1}v) \in H_\circ$ and $v - n^{-1} A u \in H$ and on such
 elements $\mathscr{O} (^u_v)$ is defined to be 
 $$
 \begin{pmatrix} 0 & I \\ A & 0 \end{pmatrix}\begin{pmatrix}
   u\\  v \end{pmatrix}. 
 $$
 

The third step shows that for sufficiently large $|n|$, the range of
the operator $\begin{pmatrix} I & 0 \\ 0 & I \end{pmatrix} - n^{-1}
\mathscr{O}$ coincides with the set pairs $(^f_g)$ such that $f, g \in
H, A f \in H_\circ$; such vectors $(^f_g)$ are dense in the Banach
space $\tilde {H}_1 \times \tilde{H}_\circ$. It follows that the
smallest closed extension $\bar{\mathscr{O}}$ of is such that 
$$
(\mathscr{I} - n^{-1}\bar{\mathscr{O}}), \qquad \mathscr{I}
= \begin{pmatrix} I & 0 \\ 0 & I \end{pmatrix} 
$$
admits,\pageoriginale for sufficiently large $| n |$, of an inverse $\mathscr{I}_n =
(\mathscr{I} -n^{-1} \bar{\mathscr{O}})^{-1}$ which is linear operator
on $\tilde{H}_1 \times H_\circ$ satisfying 
$$
|| \mathscr{I}_n || \leq (1 + \beta | n^{-1}| ).
$$

So, there exists a uniquely determined \textit{group} $\{T_t\}_{-
 \infty < t< \infty}$ with $\bar{\mathscr{O}}$ as the infinitesimal
generator and such that 
$$
|| T_t || \leq \exp (\beta t),
$$
strong $\lim \limits_{h \to o} \dfrac{T_{t+h}-T_t}{h}(^f_g) =
\bar{\mathscr{O}}T_t (^f_g) = T_t \bar{\mathscr{O}}(^f_g)$ if $(^f_g)
\in $ domain of $\bar{\mathscr{O}}$ (See Lecture $9$). 
