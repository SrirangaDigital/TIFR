\chapter{Lecture}\label{chap16} % LECTURE 16

\section{Proof of Lemma 3}\label{chap16:sec1}

 Let\pageoriginale $R$ be a bounded domain of $E^m$. Let $u_o$ of order $n$ satisfy
 \begin{gather*}
 | (L^* \varphi, u_0)_0 ~ | \quad |\sum_{| \rho | = |
   \sigma | = 0}^{n} ~
 (D^{(\sigma)} ~ a^{\rho,\sigma} ~ D^{(\varrho)} ~ \varphi, u_0)_0\\
 \le \const || \varphi ||_{n-1}\quad \text{ for all } \varphi \in D^\infty (R) 
 \end{gather*}

Let $R_2 \subset R_1 \subset R, R_2, R_1$ being subdomains, such that
the closure of $R_1$ in $R$ is compact. Let $\zeta \in
\mathscr{D}^\infty$ with $\zeta (x) = 1$ on $R_2$. Let 
$$
v^h(x) = \frac{v(x^h) - v(x)}{h}, x^h = (x_1 + h, x_2, \ldots, x_m),
$$
$h$ sufficiently small. Then, as will be proved below,
$$
|| ~ v^h ~ ||_n \le \const \quad (\text{ for all sufficiently small } h).
$$

Since the Hilbert space $H_n (R)$ (completion of $\mathscr{D}^\infty
(R)$ by the norm $|| ~~ ||_n$) is locally weakly compact, there exists
a sequence $\{h_i \}$ with $\lim\limits_{i \rightarrow \infty }\break h_i
= 0$ such that for $ | k | \le n$ 
\begin{align*}
 & \underset{i \rightarrow \infty}{\text{weak}\lim\limits} \quad v^{h_i}
 = \hat{v}\\ 
 & \underset{i \rightarrow \infty}{\text{weak}\lim\limits} \quad
 \tilde{D}^k v^{h_i} = v^{(k)} 
\end{align*}
exist in $L_2(R_1)$. We shall show that 
\begin{align*}
 & \hat{v} = \tilde{D}_1 ~ v ~ (D_1 = \partial/_{\partial_{x_1}})\\
 & v^{(k)} = \tilde{D}_1 \tilde{D}^{(k)}_v = \tilde{D}^{(k)} ~
 \tilde{D}_1 ~ v 
\end{align*}
proving that $\tilde{D}_1 ~ v$ is of order $n$ in $R_1$. Similarly
$\tilde{D}_i ~ v (i = 2, \ldots, m)$ will be of order $n$ in
$R_1$. Thus by lemma \ref{chap15:sec1:lem1}, $v$ is of order $n + 1$ in $R_1$ and hence
$u$ is of order $n + 1$ in $R$. That $\hat{v} = \tilde{D}_1 ~ v$ may
be proved as follows: For any $\varphi \in \mathscr{D}^\infty(R_1)$ we
have, $\theta$ being a real number such that $0 < \theta < 1$, 
\begin{align*}
 (\varphi, \hat{v})_o & = \lim_{i \rightarrow \infty} (\varphi, v^{h_i})_o\\
 & = \lim_{i \rightarrow \infty} (\varphi^{-h_i}, v)\\
 & = \lim_{i \rightarrow \infty} (\varphi_{x_1} (x^{-\theta_{h_i}}), v(x))_o\\
 & = \lim_{i \rightarrow \infty} (\varphi(x^{(-\theta h_i)}), \tilde{D}_1 v(x))_o\\
 & = (\varphi, D_1 v)_o.
\end{align*}

We\pageoriginale have also 
$$
(\tilde{D}^k ~ v)^h = \tilde{D}^k ~ v^h
$$
and thus, in $L_2$,
\begin{align*}
 v^k = \underset{i \rightarrow \infty}{\text{weak}\lim} ~\tilde{D}^k
 ~ v^{h_i} & = 
 w-\lim_{i \rightarrow \infty} ~ (\tilde{D}^k ~ v)^{h_i}\\ 
 & = \tilde{D}_1 ~D^{(k)} ~ v.
\end{align*}

We prove that 
$$
|| v^h ||_m \le \const ~(\text{for all small} ~h).
$$

We shall make use of Garding's inequality for the $2n$ order elliptic
differential operator $L^*$: there exist constants $C_1$, $C_2$ and $C_3$
such that 
\begin{gather*}
 C_1 || \varphi ||^2_n \le (L^* \varphi, \varphi)_o + C_2 || \varphi
 ||^2_o\\ 
 | (L^* \varphi, \psi) | \le C_3 || \varphi ||_n ~ || \psi ||_n,
 \quad \varphi, \psi \in \mathscr{D}^\infty(R). 
\end{gather*}

Now,
\begin{align*}
 (L^* \varphi, v^h)_o & = (-1)^{|\varrho|} ~ (D^\varrho \varphi,
 a^{\varrho,\sigma} ~ \tilde{D}^{(\sigma )} ~ (\zeta ~ u_o)^h)_o\\ 
 & = (-1)^{|\varrho |} ~ (D^\varrho \varphi, a^{\varrho,\sigma} ~
 (\tilde{D}^{(\sigma)} ~ \zeta ~ u_o)^h)_o\\ 
 & = (-1)^{|\varrho|} ~ (D^{(\varrho)} \varphi, a^{\varrho,\sigma} ~
 (\zeta.\tilde{D}^{\sigma} u_o)_o\tag{*}\\ 
 & \quad + (-1)^{|\varrho|} ~ C^{\sigma,\sigma'} ~ (D^\varrho \varphi,
 a^{\varrho,\sigma} ~ \left[D^{\sigma'} \zeta ~ D^{(\sigma - \sigma')}
  ~ u_o\right]^h_o ~ (|\sigma' |\ge ~1) 
\end{align*}
by applying the Leibnitz formula.

On\pageoriginale the other hand, we have, for any function $w$ of order $j$ in $R$
with support completely interior to $R$ 

$|| w^h ||_{J-1,R_1} \le || w ||_j$, ~\text{ for sufficiently small }~ $| h |$,
because, for any approximating functions $\{ u_i \} \le C^\infty (R)$
\begin{align*}
 || ~ w^h ~ ||_{j- 1,R_1} & = \lim_{i\rightarrow \infty} || ~ u^h_i ~
 ||_{j - 1, R_1}\\ 
 & = \lim_{i \rightarrow \infty} ~ || ~ u_{x_1} ~ (x^{(\theta h)}) ~
 ||_{j-1,R_1}\\ 
 & \le ~|| ~ w ~ ||_{j,R} = || ~ w ~ ||_j.
\end{align*}

Thus the absolute value of the second term on the right of $(*)$ is by
Schwarz's inequality $\le \const ~ || ~ \varphi ~ ||_n ~ || ~ u ~ ||_n
= \const || ~ \varphi ~ ||_n$. Since 
$$
(ef)^h (x) = e^h(x) ~ f(x^h) - e(x) ~ f^h(x),
$$
we have 
\begin{align*}
  (-1)^{|\varrho|} & (D^{\varrho} \varphi, a^{\varrho,\sigma}
 ~(\zeta.\tilde{D}^{(\sigma)} ~u_o)^h)_o\\ 
 = & (-1)^{|\varrho|} ~ (D^\varrho \varphi,[(a^{\varrho,\sigma}
  ~\zeta.\tilde{D}^{\sigma} ~ u_o)^h - (a^{\varrho,\sigma})^h ~
  \zeta(x^h). \tilde{D}^\sigma ~ u_o (x^h)])_o\\ 
 = & (-1)^{|\varrho|} ~ ((D^{\varrho}\varphi)^{-h},
 a^{\varrho,\sigma} \zeta ~ \tilde{D}^\sigma ~ u_o)_o\\ 
 + & (-1)^{|\varrho|+1} ~ (D^\varrho \varphi, (a^{\varrho,\sigma})^h
 ~ \zeta (x^h) ~ \tilde{D}^\sigma ~ u_o ~ (x^h))_o 
\end{align*}

The\pageoriginale absolute value of the second term on the right is
$$
\le \const || ~ \varphi ~ ||_m. 
$$

We have also
\begin{align*}
 & (-1)^{| \varrho |} ~ ((D^\varrho\varphi)^{-h}, a^{\varrho,\sigma}
 ~ \zeta ~\tilde{D}^\sigma ~ u_o)_o\\ 
 = & (-1)^{| \varrho |} ~ (D^\varrho\varphi^{-h}, a^{\varrho,\sigma}
 ~ \zeta ~\tilde{D}^\sigma ~ u_o)_o\\ 
 = & (-1)^{| \varrho |} ~(a^{\varrho,\sigma} \zeta ~ D^\varrho ~
 \varphi^{-h}, \tilde{D}^\sigma ~ u_o)_o\\ 
 = & (-1)^{| \varrho |} ~(a^{\varrho,\sigma} ~ D^\varrho ~\zeta ~
 \varphi^{-h}, \tilde{D}^\sigma ~ u_o)_o\\ 
 = & (-1)^{|\varrho|} ~ C^{\varrho,\varrho'} ~ (a^{\varrho,\sigma}
 ~(D^{\varrho'} \zeta ~D^{(\varrho-\varrho')}
 \varphi^{-h}),\tilde{D}^{\sigma} ~ u_o)_o ~ (| \varrho' | \ge 1). 
\end{align*}

The absolute value of the second term on the right is
$$
\le \const ||~ \varphi^{-h} ~||_{n-1} \le \const || ~ \varphi ||_{n-1}.
$$

Therefore, by applying the original hypothesis,
\begin{align*}
 | (L^*\varphi,(\zeta ~ u_o)^h)_o| & \le | (L^* \zeta ~ \varphi^{-h},
 u_o) | + \const || \varphi ||_n\\ 
 & \le \const || \zeta \varphi^{-h} ~||_{n-1} + \const ~ || \varphi ||_n\\
 & \le K ~ || ~ \varphi ~ ||_n, K \text{ a positive constant }.
\end{align*}

Thus letting $\varphi$ tend, in $|| ~~ ||_n$, to $(\zeta ~ u_o)^h$, we
have 
$$
C_1 || (\zeta ~ u_o)^h ||^2_n \le K || (\zeta ~ u_o)^h ||_n + C_2 ||
(\zeta ~ u_o)^h ||_o 
$$

Since $|| (\zeta ~ u_o)^h ||_o \le \const || \zeta ~ u_o ||_1$, the
right hand side being independent of $h$, we must have 
$$
|| (\zeta ~ u_o)^h ||_n \le \const ~ (\text{ independent of } h).
$$
