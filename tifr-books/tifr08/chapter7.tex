\chapter{Lecture 7}\label{chap7}

\section{The exponential of a linear operator}\label{chap7:sec1}

\begin{example*}[III ]
 In\pageoriginale $C [ -\infty, \infty ] $ consider the semi-group
 associated with Poison process, viz., 
 $$
 (T_t x) (s) = e^{- \lambda t} \sum \frac{( \lambda t)^k}{k!} x ( s
 - k \mu ) ~ \lambda, \mu > 0 
 $$
\end{example*}

Since $ e^{- \lambda t} \sum \limits^{\infty}_{k=0} ~ \dfrac{(
 \lambda t)^k}{k!} = 1 $, we have 
\begin{align*}
 \frac{(T_t \,x) \,(s)-x(s)}{t} &= \frac{e^{-\lambda t}}{t}
 \sum^{\infty}_{k=0} \frac{( \lambda t)^k}{k!} ( x ( s - k \mu ) -x
 (s) ) \\ 
 &= \frac{e^{-\lambda t}}{t} ( x ( s - k \mu ) -x (s) ) \\
 &+ \frac{e^{-\lambda t}}{t} \sum^{\infty}_{k=2} \frac{( \lambda t
  )}{k!} ( x ( s - k \mu ) -x (s) ).
\end{align*}

As $t \downarrow 0 $ the first term on the right tends uniformly
with respect to $ s $ to $ \lambda ( x ( s - \mu ) - x (s) ) $; the
absolute value of the second term is majorized by $ 2 \mid\mid x
\mid\mid ~\dfrac{e^{-\lambda t}}{t} ~ \sum \limits^{\infty}_{k=2} ~
\dfrac{(\lambda t)^k}{k!}$ which tends to zero as $ t \downarrow 0
$. Thus for any $ x \in C [ - \infty, \infty ] $, we have $
Ax = \lambda ( x ( s - \mu ) - x (s) ) $. So in this case the
infinitesimal generator is the \textit{ linear operator } defined by: 
$$
( Ax ) (s) = \lambda [ x ( s-\mu ) ~ -x (s) ],
$$
for $ x \in C [ - \infty, \infty ] $.

This is the difference generator.

We now intend to represent the original semi-group $ \{ T_t \} $ by
its infinitesimal generator. We expect, by analogy with the case of
the ordinary exponential function, the result to be given by 
$$
T_t x = ~ \exp(t A) x. 
$$

But in general $A$ is not defined over the whole space. So if we
attempt to define $(\exp t ~ A)x$ by a power series $ \sum
\limits^{\infty}_{k=0} ~ \dfrac{(tA)^k}{k!} x$, we encounter
some\pageoriginale difficulties. First, we have to choose $x$ form $ \bigcap
\limits^{\infty}_{k=0} \mathscr{D} ( A^k ) $ and we do not know how
big this space is. Even if we do this, it will be difficult to prove
the convergence of the series, let alone its convergence to $T_t x$.
 So we proceed to define the exponential in another way. As a
preparation to the definition of the exponential function of an
additive operator - not necessarily linear - we consider the
exponential of a linear operator. 

\begin{prop*}
 Let $B$ be a linear operator from the Banach space $X$ into
 $X$. Then for each $ x \in X$, $s- \lim \limits_{n \rightarrow \infty}
 \sum \limits^{\infty}_{k=0} ~ \dfrac {B^k}{k!} x $ exists ; denote
 this by $\exp Bx $. Then $\exp B$ is a linear operator and $
 \mid\mid \exp B \mid\mid \leq \exp ( \mid\mid B \mid\mid ) $. 
\end{prop*}

\begin{proof}
 We have $ \mid\mid B^k \mid\mid \leq ~ ( \mid\mid B \mid\mid^k ) $
 $(k \ge 0)$. $ \sum \limits_{k=0}$ $\dfrac{B^k}{k!} x $ is a
 Cauchy sequence; for $l> j $ we have
 $$
 \Big\| \sum^{l}_{k=0} \frac{B^k}{k!} - \sum^{j}_{k=0} ~
 \frac{B^k}{k!} \Big\| = ~ \Big\| \sum^{1}_{k= j+1} ~
 \frac{B^k}{k!} \Big\| \sum^{1}_{j+1} ~ \frac{\|B\|}{k!}^k
 $$
 and $< \sum^{(\infty)}_{k=0} \dfrac{\mid\mid B \mid\mid^k}{k!} ~
 \mid \mid x \mid\mid $ is convergent. So, by the completeness of the
 space, $ s- \lim \limits_{n \rightarrow \infty} ~ \sum
 \limits^{\infty}_{k=0} ~ \dfrac{B^k}{k!} x $ exists; and the
 convergence is uniform in every sphere $ \mid\mid x \mid\mid \leq M
 $; the above inequality shows that 
 $$
 \mid\mid \exp B \mid \mid \leq \exp ( \mid\mid B \mid\mid ) ~
 \mid\mid x \mid\mid. 
 $$ 
\end{proof}

So $ \exp B $ is a linear operator and 
$$
\mid\mid \exp B \mid\mid \leq \exp ( \mid\mid B \mid\mid ). 
$$

\begin{remark*}
 In a similar manner one can prove the following: Let a sequence of
 linear operators $ \{ S_n \} $, on a linear normed space 
\end{remark*}

$X$\pageoriginale with values in a Banach space $Y$ be a Cauchy sequence, i.e.,
$\lim_{ n, m} || S_n - S_m || = 0$. Then there exists a linear
operator $S$  forms $X$ to $Y$ such that $\lim\limits_{ n \to \infty} || S_n
- S|| = 0 $ and $\| S \| \leq \varliminf\limits_{ n \to \infty} \| S_n\|$. 
\begin{theorem*}
 Let $B$ and $C$ be two linear operators from a Banach space $X$ into
 $X$. Assume that $B$ and $C$ commute, i.e., $BC = CB$. Then 
 \begin{enumerate} [1)]
 \item $\exp B. \exp C =E\exp\, ( B + C)$
 \item $D_t \exp (tB)x = s -\lim\limits_{ h \to \infty} \dfrac{\exp (t + h)
  B - \exp t B}{h}x$ exists and has the value $B(\exp t Bx) = (\exp
  tB).Bx$. 
 \end{enumerate}
\end{theorem*}

\begin{proof}
 \begin{enumerate}[i)]
 \item If $\beta$ and $\wp $ are complex numbers, we have 
  $$
  \sum_{ j = 0}^{\infty} \frac{(t\beta)^j}{j!} \sum^\infty_{ l = 0}
  \frac{(t \wp)^l}{l!} = \frac{t( \beta + \wp)^m}{m!} \hspace{1cm}(t
  > 0); 
  $$
  for, by the absolute convergence of each of the series on the left
  and the commutativity of $\beta$ and $\wp$ we may arrange the
  product on the left to be equal to the power series on the
  right. A similar proof holds when $\beta$ and $\wp$ are replaced
  by commuting linear operators $B$ and $C$ on a Banach space. 
 \item Since $tB$ and $hB$ commute, we have by $1)$
  $$
  \exp (t + h )B = \exp (tB). \exp (hB) = exp (hB). \exp tB.
  $$  
  So,
  \begin{align*}
   \frac{\exp (t + h)B - \exp tB}{h} & = \frac{\exp tB (\exp
    (hB) - I)}{h}\\ 
   & = \frac{\exp (hB) - I}{h}\exp tB.
  \end{align*}
 \item follows since 
  \begin{align*}
   \Big\|\frac{\exp (hB) - I}{h} - B \Big\| = \Big\| \sum_{k = 2}^{ \infty}
   \frac{(hB)^k}{k!}\Big\| 
   \leq \sum_{ k = 2}^{\infty} \frac{B^k}{k!} h^{ k - 1} \to 0,
   \text{ as } h \to 0. 
  \end{align*}
 \end{enumerate}
\end{proof}

\section{Representation of semi-groups}\label{chap7:sec2} \pageoriginale

\begin{theorem*}
 Let $A$ be the infinitesimal generator of a semi-group $\{T_t\}$.
\end{theorem*}

Then for each $y \in X$
$$
T_t y = s- \lim_{n \to \infty} \exp (t A J_n) y
$$ 

\textit{uniformly in any bounded interval of $t$}. ($J_n$ is the
resolvent $(I-n^{-1}A)^{-1}$, $n > \beta )$. 

\begin{proof}
 $(t A J_n)= nt (J_n -I)$ is a linear operator and so $\exp(t A
 J_n)$ can be defined. Since $n t I$ and $nt J_n$ commute we have 
 \begin{align*}
  (\exp t A J_n) &= \exp (-n t I). \exp (n t J_n) \\
  &= \exp (-nt). \exp (n t J_n).
 \end{align*}
\end{proof}

Since $|| J_n || \le 1 / (1- \beta n^{-1})$ $(n > \beta)$, we have 
\begin{align*}
 || \exp (t A J_n) || & \le \exp (-nt) ||\exp (nt J_n )|| \\
 & \le \exp (-nt) \exp (nt J_n ||) \\
 &\le \exp (-nt) \exp (nt /1 - \beta n^{-1}) \\
 &= \exp (tB/(1- \beta n^{-1}))
\end{align*}

If $x \in \mathscr{D}(A)$, $D_t T_t x= A T_t x= T_t A x$ and hence 
{\fontsize{10pt}{12pt}\selectfont
\begin{equation*}
 D_s \left\{ \exp [(t-s) A J_n)] T_s x \right\} = \exp ((t-s)A J_n)
 T_s A x - \exp((t-s)A J_n).AJ_n. T_s x. 
\end{equation*}}\relax

Since $T_t T_s= T_s T_t (= T_{t+s})$,
$$
J_n = n \int^{\infty}_o e^{-nt} T_t dt
$$
is the limit of Riemannian sums each of which commutes with each
$T_s$; so $J_n$ commutes with each $T_s$ so that $A J_n =n(J_n-I)$
commutes with each $T_s$. Now 
$$
T_t x- \exp (t A J_n) x= [ \exp ((t-s) A J_n) T_s x]^t_{s=o}
$$

Since\pageoriginale $\exp ((t-s) A J_n)T_s(A-A J_n)x$ is strongly continuous in $s$,
we have, for $x \in \mathscr{D} (A)$, 

\begin{align*}
 T_t x- \exp (t A J_n)x &= \int^t_o D_s \bigg\{ \exp ((t-s)A J_n)T_s
 x \bigg\} ds\\ 
 & = \int^t_o \exp(( t-s)A J_n) T_s (A- J_n A x)\, ds\\ 
 & \hspace{2cm}(\text{as}~ A J_n x= J_n A
 x, as x \in \mathscr{D}(A)) 
\end{align*}
So
\begin{align*}
 || T_t x - \exp(t A s_n x) || & \le \int^t_o || \cdots || ds\\
 & \le \int^t_o || \exp (t-s) A J_n || || T_s || || Ax- J_n A x || ds\\
 & \le || Ax - J_n A x || \int^t_o \exp \frac{\beta(t-s)}{1- \beta
  n^{-1}} \exp \beta s ds 
\end{align*}

For each fixed $t_o >0$ and $n > \beta$, the integral is uniformly
bounded for $0 \le t \le t_o$ as $n(> \beta)\to \infty$ ; also we know
that for each $x \in X$, $s-\lim\limits_{n \to \infty} J_n
x=x$. Thus 
$$
T_t s=s- \lim_{n \to \infty} \exp (t A J_n x) \uniformly_{ \text{ if }
 x \in \mathscr{D}(A)}\text{ in}~ \,0 \le t \le t_o, 
$$

We now prove the formula for arbitrary $y \in X$. Since
$\mathscr{D}(A)$ is dense in $X$, given $\varepsilon >0$ we can find
$x \in \mathscr{D}(A)$ such that $|| y-x|| \le \varepsilon$. Then 

\begin{align*}
 || T_t y-\exp (t A J_n y) || & \le || T_ty - T_tx || + || T_t x-
 \exp (tA J_n x) ||\\ 
 & \quad + || \exp (t A J_n)x - \exp(t A J_n)y ||\\
 & \le \exp(\beta t) \varepsilon + || T_t x- \exp ( t A J_n)x ||\\
 & \quad + \exp \left(\frac{t}{1-n^{-1} \beta }\right) \varepsilon.
\end{align*}

Since\pageoriginale $x \in \mathscr{D}(A)$, the middle term on the right tends to zero
as $n \to \infty$ uniformly in any bounded interval of $t$. So 
$$
\overline{\lim\limits_{n \to \infty}} || T_t y- \exp (t A J_n)y || \le
2 \exp (\beta t) \varepsilon, 
$$
and $\varepsilon$ being arbitrary,
$$
T_t y = s- \lim_{n \to \infty} (\exp t A J_n)y, y \in X,
$$
\textit{uniformly in any bounded interval of $t$}

\begin{remark*}%Remk 
 The above representation of the semi-group was obtained
 independently of $E$. Hille who gave many representations in his
 book. One of them reads as follows: 
 $$
 T_t x= s- \lim_{n \to \infty} \left(I - \frac{tA}{n}\right)^{-1} x
 $$ 
 uniformly in any bounded interval of $t$. It also shows the
 exponential character of the representation. 
\end{remark*}
