
\chapter{Generalities on rational maps-Zariski's main
  theorem}\label{chap3}%chap 3   

\markright{\thechapter. Generalities on rational maps-Zariski's main
  theorem}

Let\pageoriginale $X$ be a \textit{reduced} $B-$prescheme, and $Y$ a 
$B$-\textit{scheme}. Let $\varphi$ be a rational map of $X/B$ into
$Y/B$. Then there is a unique pair $(U,f)$ consisting of an open dense
subprescheme $U$ of $X$ and a $B$-morphism $f$ of $U$ into $Y$ such
that $(i) (U, f)$ represents $\varphi$, and $(ii)$ if $g$ is a
$B$-morphism of an open dense subscheme $V$ of $X$ into $Y$ which
represents $\varphi$, then $V \subset U$ and $f/U = g$ ($\EGA$   I,
7.2.1). When these assumptions are fulfilled, we call $U$ the
\textit{(maximal open)  set of definition} of $\varphi$, and we say
that $\varphi$ is \textit{defined} or \textit{regular} at the points
of $U$. Also, we shall use the same symbol $\varphi$ to mean
the $B$-morphism $f$ defined on $U$.  

Let $X$ be a reduced $B$-scheme which is either locally noetherian or
irreducible. Let $Y$ be a $B$-proper scheme, and $\varphi$ a rational
map of $X/B$ into $Y/B$. If $x$ is any of $X$ such that
$\mathscr{O}_x$ is a discrete valuation ring, $\varphi$ is defined at
$x$. 

\begin{proof}
  When $X$ is locally noetherian, since $\mathscr{O}_x$ is an integral
  domain, we can find a neighbourhood $X'$ of $x$ which is
  irreducible, and we may clearly restrict ourselves to this
  neighbourhood. Thus, it is sufficient to prove the theorem in the
  case when $X$ is irreducible. 
\end{proof}

If $U$ is the set of definition of $\varphi$, and if $\pi_1$ and
$\pi_2$ are the first and second projections of $U \times_B Y$, the graph
morphism $U \xrightarrow{\Gamma_\varphi} U \times_B Y$ defined by the
conditions that $\pi_1\circ\Gamma_\varphi = I_U$, $\pi_2\circ\Gamma_\varphi
= \varphi$, is a closed immersion,\pageoriginale  since $Y$ is separated
over $B$. Let $\Gamma$ be the unique reduced subprescheme of $X \times_B Y$
whose support is the closure in $X \times_B y$ of $\iim (\Gamma_\varphi)$,
and let $g$ be the restriction to $\Gamma$ of the projection $X \times_B Y
\rightarrow X$. Then $\Gamma$ is irreducible, $\Gamma \cap(U \times_B Y) = \iim
(\Gamma_\varphi)$ is an open subprescheme of $\Gamma$ and the
restriction of $g$ to this open subscheme is an isomorphism onto
$U$. Hence $g$ is birational. Further $g$ is proper, being the
composite of the closed immersion $\Gamma \rightarrow X \times_B Y$ and the
proper morphism $X \times_B Y \rightarrow X$. It follows that $g$ is
surjective. 

Let $z$ be a point of $\Gamma$ such that $g(z) = x$. Choose an affine
open neighbourhood $B'$ of the image of $x$ in $B$, an affine
neighbourhood $\Gamma'$ of $x$ in $X$ which is mapped into $B^1$, and
an affine neighbourhood $\Gamma'$ of $z$ in in $\Gamma$ such that $g(\Gamma')
\subset X'$. If $\Gamma(B', \mathscr{O}_B), \Gamma (X',
\mathscr{O}_X)$ and $\Gamma (\Gamma, \mathscr{O}_\Gamma)$ are
respectively denoted by $P, Q$ and $R, Q$ and $R$ are $P$-algebras,
and $R$ is finitely generated over $Q$. We have the comutative diagram
of $P$-algebras  

\[
\xymatrix{R \ar@{^{(}->}[r]  & \mathscr{O}_z  \ar@{^{(}->}[r] & R (\Gamma) \\
Q \ar[u]^{g*} \ar@{^{(}->}[r] & \ar[u] \mathscr{O}_x \ar@{^{(}->}[r] &
R(X) \ar[u]_\wr  
}
\]

Since a discrete valuation ring is a maximal proper subring of its
quotient field, $\mathscr{O}_x \rightarrow \mathscr{O}_z$ is an
isomorphism. Let $h_1,\ldots , h_n$ be generators of $R$ over $Q$. Then
we can find an $h \in Q$ which is invertible in $\mathscr{O}_x$ such
that $h_i \in g*\left(Q \left[ \dfrac{1}{h} \right]\right)$. It
follows that $g^*$ 
induces an isomorphism of $X^1_h$ onto $\Gamma^1_{g^*(h)}$. The
composite morphism\pageoriginale $X^1_h \xrightarrow{g^{-1}} \Gamma \rightarrow Y$
clearly represents the rational map $\varphi$ so that by the
maximality of $U, X^1_h \subset U $ and $x \in U$.  

This proves the assertion.

In particular, it follows from this proposition that the complete
regular model of a function field of one variable is unique.  

We now state that celebrated ``Main theorem'' of Zariski (Z.M.T for
short) without proof. This has several variants, but we shall base all
our conclusions on the following formulation: 
\begin{theorem*}[Zariski ]
  Let $Y$ be a a locally noetherian prescheme and $f: X \rightarrow Y$
  a birational morphism of finite type. Let $Y$ be any (not
  necessarily closed) point $Y$, such that  
  \begin{enumerate}[(i)]
  \item the local ring $\mathscr{O}_y$ of $Y$ at $y$ is normal(that
    is, an integrally closed integral domain), and  
  \item there is a point $x \in f^{-1} (y)$ which is isolated (that
    is, open) in $f^{-1}(y)$. 
  \end{enumerate}
\end{theorem*}

Then there is an open neighbourhood $U$ of $x$ in $X$ and an open
neighbourhood $V$ of $y$ in $Y$ such that $f/U$ is an isomorphism of
$U$ onto $V$. If further $f$ is separated, we have $U = f^{-1}(V)$. 

The geometric meaning of this theorem is: if no subvariety of $X$ of
positive dimension going through $x$ is blown down by $f, ~ f$ is
regular at $x$.  

For a proof, we refer the reader to ($\EGA$, III, 4.4).


\subsubsection*{Some geometric examples of dilatations.}

We shall for simplicity consider only (irreducible and reduced)
varieties over an algebraically closed field $K$, and we shall only
consider points\pageoriginale which are rational over $K$. 

Let $\varphi$ be a rational map of a variety $X$ into a projective
space $\mathbb{P}^n$ (with homogeneous co-ordinates $(y_\circ, \ldots ,
y_n)$), and let $x$ be a point of $X$ where $\varphi$ is defined. Then
$\varphi(x)$ is contained in one of the `standard' affine open sets of
$\mathbb{P}^n$, say in $y_\circ \neq 0$. Hence there exist rational
functions $f_1,\ldots, f_n$ on $X$, all of them regular in a
neighbourhood of $x$, such that $\varphi(x^1) = (1, f_1(x^1), \ldots ,
f_n(x^1))$ for any $x^1$ in this neighbourhood. Conversely, any system
$(f_\circ, f_1, \ldots , f_n)$ of rational functions on $X$ with not all
$f_i \equiv 0$, defines a rational map of $X$ into $\mathbb{P}^n$ which
is regular at atleast all points where the $f_i$ are all regular and
atleast one $f_i$ does not vanish. Two such systems ($f_\circ, \ldots
,f_n$) and $(g_\circ, \ldots , g_n)$ define the same rational map if and
only if there is a $h \in R(X)$ such that $g_i = h f_i$. The rational
map defined by the system $(f_\circ, \ldots , f_n)$ is regular at $x$ if
and only if there is an $h \in R(X)$ such that $hf_i$ are regular at
$x$ and $(h f_i)(x) \neq 0$ for atleast one $i$. 

For any $\lambda = (\lambda_\circ,\ldots , \lambda_n) \in \mathbb{P}^n$, we
shall denote by $H_ \lambda (H = Hyper\-plane)$ the divisor in
$\mathbb{P}^n$ defined by $\sum \limits^n_{i=0} \lambda_i y_i = 0$. If
$\varphi$ is a rational map of $X$ into $\mathbb{P}^n$ determined by a
system $(f_\circ, \ldots , f_n)$ of functions on $X$ all regular and not
all $0$ at a point of $x$, the inverse image divisor $\varphi^*(H_
\lambda)$ is defined in a neighbourhood of $x$ by the equation $f_
\lambda = \sum \limits^n_0 \lambda_i f_i = 0$. 

We shall say $\varphi$ is \textit {biregular} ar $x$ if it is regular
at $x$, and induces an isomorphism of a neighborhood of $x$ onto a We
shall say that $\varphi$ is \textit{biregular} at $x$ if it is regular
at $x$, and induces an isomorphism of a neighbourhood of $x$ onto a
locally\pageoriginale closed subvariety of $\mathbb{P}^n$
(i.e.$\varphi$ is a local 
immersion at $x$, in the sense of $\EGA$. (Ch.I, 4.5.1). We shall now
find conditions for this. Let $\varphi$ be described by $(f_\circ, \ldots,
f_n)$, where the $f_i$ are all regular and at least one $f_i$ does not
vanish in a neighbourhood $U$ of $x$ on $X$. Suppose $\varphi$ is an
immersion on $U$. Then firstly, $\varphi$ must be injective in $U$, so
that given $x_1,x_2 \in U$ with $x_1 \neq x_2$, there is a hyperplane
$H_{\lambda}$ containing $x_1$, but not $ x_2$. This may be expressed
by the condition that  
\begin{enumerate}[(A)]
\item \textit{for any} $x_1, x_2 \in U$ \textit{with} $x_1 \neq x_2$,
  \textit{there is a } $\lambda = ( \lambda_\circ  , \ldots, \lambda_n)$
  \textit{such that} $f_{\lambda}(x_1)= 0, f_{\lambda}(x_2) \neq 0$. 

  Secondly, under the homomorphism $\mathscr{O}_{\varphi
    (x),\mathbb{P}^n} \to \mathscr{O}_{x,X}$, the image of the maximal
  ideal $\mathcal{M}_{\varphi (x)}$ must be the maximal ideal
  $\mathcal{M}_x$ at $x$. Since $\mathcal{M}_{\varphi (x)}$ is itself
  generated over $\mathscr{O}_{\varphi (x),\mathbb{P}^n}$ by the linear
  functions' $\dfrac {L_1(y)}{L_2(y)}$ where $L_1$ and $L_2$ are
  linear forms with $L_2(\varphi (x)) \neq 0,\mathcal{M}_x$ must be
  generated by the functions $L_1(f(x'))$ of $x' \in U$ for
  $L_2(\varphi (x'))$ are invertible at $x$. Thus we have the second
  condition. 

\item  \textit{As $\lambda = (\lambda_\circ, \ldots , \lambda_n)$
  ranges through values such that $f_{\lambda}(x)=0$, the
    functions $f_{\lambda}$ generate the maximal ideal
  $\mathcal{M}_x$, or what is the same, their images
  $df_{\lambda}$ in $\dfrac{\mathcal{M}_{x}}{\mathcal{M}^2_x}$
    span this $K$-vector space}. 
\end{enumerate}

Speaking geometrically, $(A)$ says that for distinct points $x_1, x_2$
of $U$, there is a member of `linear system' of divisors
div $(f_{\lambda})$\pageoriginale which passes through $x_1$ but not
$x_2$. When $X$ is nonsingular, $(B)$ says that the tangent spaces at
$x$ to those members of the linear system which pass through $x$ and
are non-singular at $x$ do not contain any one-dimensional subspace of
the tangent space to $X$ at $x$ in common.  

Conversely, if $U$ is a neighbourhood of $x$ and $\varphi$ is given by
functions $(f_\circ, \ldots , f_n)$ which are all regular on $U$ and at
least one vanishes nowhere in $U$, and if the conditions $(A)$ and
$(B)$ above are fulfilled, \textit{and} if further the closed subvariety
$\overline{\varphi (U)}$ of $\mathbb{P}^n$ is \textit{normal} at
$\varphi(x),\varphi$ is biregular at $x$. In fact, because of $(A)$,
the morphism $\varphi: U \to  \overline{\varphi (U)}$ must be radical
(that is, $R(U)$ is a purely inseparable finite algebraic extension of
$R(\overline{\varphi (U)})$ ($\EGA$ I, 3.5)). But (B) is evidently
equivalent to saying that the mapping of the Zariski tangent space to
$U$ at $x$ int that of $\overline{\varphi (U)}$ at $\varphi(x)$ is
injective, and this implies that $R(U)$ is separable over
$R(\overline{\varphi (U)})$. Hence $R(U) = R(\overline{\varphi (U)})$
and $\varphi$ is birational. But now, by $(A),x$ is the unique point
of the fibre $\varphi^{-1}(x)$ (where $\varphi$ is considered as
restricted to $U$),and since $\varphi(x)$ is a normal point of
$\overline{\varphi (U)}$, it follows by Z.M.T. that $\varphi$ is an
isomorphism of a nighbourhood of $x$ onto a neighbourhood of
$\varphi(x)$ in $\overline{\varphi (U)}$. This proves the converse
assertion. 

Suppose now that $X$ is itself imbedded as a closed subvariety of a
projective space $\mathbb{P}^m$, and let $k [x_\circ, \ldots , x_m ]$ be
its homogeneous co-ordinate ring. If $x \in X$, a rational function on
$X$ regular\pageoriginale at $x$ can be written in the form
$\dfrac{f}{g}$, where 
$f,g$ are homogeneous elements of $K[x_\circ, \ldots , x_m], g(x) \neq 0$
and $\deg f = \deg g$. Thus, in this case, a rational map regular at $x$
can be described by a system of elements $(f_\circ ,\ldots ,f_n)$
homogeneous of the same degree in $K[x_\circ, \ldots , x_m]$ with at least
one $f_i$ such that $f_i(x) \neq 0$. The statement of condition $(A)$
for biregularity remains unaltered, whereas in the condition $(B)$,
one replaces the $f_{\lambda}$ (which are not functions on $X$) by
$\dfrac{f_{\lambda}}{g}$ where $g$ is \textit{any} homogeneous element
of $K[x_\circ, \ldots , x_m]$ with $g(x) \neq 0$ and deg. $g = \deg
f_i$. The geometric statement of these conditions are again the same.  

Let $X$ be a non-singular surface, and $( f_\circ , \ldots , f_n)$ a
system of regular functions on $X$, which have a point $x \in X$ for
common zero (but no other common zero). The rational map $X
\xrightarrow{\varphi}\mathbb{P}^n$ is then not defined at $x$. But
suppose further that the condition $(B)$ above is fulfilled at
$x$. Let $X' \xrightarrow{\sigma} X$ be the dilatation of $X$ at
$x$. We shall show that $\varphi\circ\sigma$ is regular on $X'$. It is
clearly sufficient to prove this at any point $z$ of the fibre
$\sigma^{-1}(x)$. Let $u = 0$ be the defining equation of
$\sigma^{-1}(x)$ in a neighbourhood of $z$ in $X'$. We have seen that
there is a one-one correspondence between the points of
$\sigma^{-1}(x)$ and the set of one-dimensional subspaces of the
tangent space to $X$ at $x$, such that any curve which is non-
singular\pageoriginale at $x$ has for proper transform a curve on
$X'$, which passes 
through the unique point of $\sigma^{-1}(x)$ corresponding to the
tangent line of the base curve at $x$. Because of condition $(B)$, we
can find an $f_i$ such that $f_i = 0$ is non- singular at $x$ and its
proper transform does not pass through $z$. Now, the rational map
$\varphi\circ\sigma: X' \to \mathbb{P}^n$ is defined in a
neighbourhood of $z$ in $X'$ by the system of rational functions
$\left(\dfrac{f_\circ\circ\sigma }{u},\dfrac{f_1\circ\sigma }{u}, \ldots
,\dfrac{f_n\circ\sigma }{u}\right)$. All of these are regular at $z$, and
since div $\left(\dfrac{f_n\circ\sigma }{u}\right)= \sigma^* (div f_i)$- div $u$ =
the proper transform $\sigma '(div(f_i))$, we see that $f_i/u$ does
not vanish at $z$. Hence $\varphi \circ \sigma $ is defined at $z$. 

One can similarly deduce conditions for $\varphi \circ \sigma $  to
be biregular at points of $\sigma^{-1}(x)$. Suppose in fact that the
condition $(A)$ above is fulfilled for $x_2 \neq x$, and assume that  

\noindent
$(B)'$ \textit{for any curve $C$ through $x$ which
  is non-singular at $x$, there is a $\lambda = (\lambda_\circ , \ldots ,
\lambda_n)$ such that the divisor $f_{\lambda}=0$ is
  non-singular at $x$ and touches $C$ at $x$,
but does not have contact of order 2 with $C$
at $x$}. (This condition is fulfilled if, the images of the
$f_i$ in $\dfrac{\mathcal{M}_x}{\mathcal{M}^3_x}$ generate this as a
$K$-vector space, but is not equivalent to this). 

Then $\varphi \circ \sigma $ is biregular on $X'$ if
$\overline{\varphi(X)}$ is normal. In fact, in view of what we have
seen, the proper transforms of two curves which are non-singular and
touch at $x$ touch at a point of $\sigma^{-1}(x)$ if and only if the
base curves have order of contact atleast two. By assumption $(B')$,
given any point of $\sigma^{-1}(x)$, we can find $\lambda= (\lambda_\circ
, \ldots , \lambda_n) $ and $\mu=(\mu_\circ, \ldots , \mu_n)$ such that
the curves $f_{\lambda}=0 $ and  $f_{\mu}=0$ are non- singular and
touch at $x$, but do not have second\pageoriginale order contact at $x$, and such
that their proper transforms pass through $z$. Hence $\varphi \circ
\sigma$ satisfies the conditions for biregularity. 

One has similar results when $X$ is a projective non-singular surface
and $f_i$ are homogeneous elements of the same degree in the
homogeneous coordinate ring of $X$. 

The above considerations suggest a method of imbedding the dilatation
of a non-singular (and hence, by a theorem of Zariski,
quasi-projective) surface in some projective space. One could take
$(f_\circ , \ldots , f_n)$ to be a basis of the homogeneous elements of
degree $m$ in the homogeneous ring of $X$ which vanish at a given
point $x$, and hope to prove that $m$ is large, the $f_i$ satisfy the
above conditions. This can indeed be done and it is even sufficient to
take $m=2$. We shall content ourselves with examining some
particular cases. 
\begin{enumerate}[i)]
\item Let $X=\mathbb{P}^n$, and take  $(f_\circ, \ldots , f_N)$ to be the
  set of monomials of degree $m \geq 1$ in the homogeneous
  co-ordinates $(x_\circ , \ldots , x_n)$. The conditions for an imbedding
  are trivially fulfilled, and we get an imbedding of $\mathbb{P}^n$ in
  $\mathbb{P}^N(N= (\overset{m+n}m)-1)$, called the \textit{Veronese}
  imbedding. The hyperplane sections of the image are precisely the
  hypersurfaces in $\mathbb{P}^n$ of degree $m$, and this imbedding
  is useful for precisely this reason, that problems of intersection
  with hypersurfaces are reduced to problems of hyperplane sections in
  the image. 

  Since the intersection of $n$ general hyperplanes in $\mathbb{P}^N$
  and the Veron\-esean variety is the intersection of $n$ hypersurfaces
  of degree $m$ in $\mathbb{P}^n$, by the theorem of Bezout, it
  consists of $m^n$ points,\pageoriginale  and this is the degree of
  the imbedded  variety.  

\item Let $ X =\mathbb{P}^2$, and $x$ any point of $\mathbb{P}^2$. By
  applying a projective transformation, we can assume that
  $x=(0,0,1)$. Let  $(f_\circ , \ldots , f_4)$ be the monomials of degree
  two which vanish at this point. One checks easily that these satisfy
  the conditions for projective imbedding of the blown up variety $X'$
  in $\mathbb{P}^4$ (A point $x' \neq x$ may be brought to (1,0,0) by a
  projective transformation which fixes (0,0,1), and this leaves the
  vector space $\sum \limits^4_0 K f_i$ stable. Hence it is sufficient
  to check the conditions at (1, 0, 0) and (0, 0, 1)). A general
  hyperplane section of $X'$ in $\mathbb{P}^4$ is the proper transform
  in $X'$ of a general conic in $\mathbb{P}^2$ through $x$. Hence, the
  intersection of $X'$ and a general linear variety of dimension
  $2$must be the intersection of proper transforms of two general
  conics in $\mathbb{P}^2$ through $x$. Since two general conics
  through $x$ are non-tangential at $x$ and meet at three other points
  of $\mathbb{P}^2$, it follows that degree of $X'$ in $\mathbb{P}^4$
  is 3. 

  Suppose we try to get an imbedding of $\mathbb{P}^2$ blown up at two
  points in $\mathbb{P}^3$, by using the quadratic forms vanishing at
  these points. We do get a morphism of the blown up variety onto a
  quadratic surface in $\mathbb{P}^3$. However, since any conic must
  intersect a line at almost two points unless it contains it as a
  component, the whole of the line joining the two chosen points is
  mapped into a single point in $\mathbb{P}^3$. It can be checked
  directly that the morphism $X' \to \mathbb{P}^3$ so obtained is
  biregular in the complement of the proper transform of this line, but
  contracts\pageoriginale the proper transform of this line to a point on the
  quadric surface. 

\item Next we consider imbeddings by cubic forms. The dimension of the
  vector space of cubic forms is 10. If we impose the condition that
  they vanish at $k(\leq 10)$ points in `general position' we get $k$
  homogeneous linear equations for the coefficients, and the dimension
  of the space of solutions is $10-k$. Thus, to get an imbedding of
  $\mathbb{P}^2$ blown up at $k$ points in $\mathbb{P}^{9-k}$, we must
  have $9-k \geq 2, k \leq 7$. Further for $k=7$, we cannot get an
  isomorphism of $X$ blown up at 7 points with $\mathbb{P}^2$
  (blowing up a point increases 2nd betti-number by one, for instance
  when $K = \mathbb{C}$). Thus to get an imbedding of
  $\mathbb{P}^2$ blown up at $k$ points by cubic forms, we must have
  $k \leq 6$. 
\end{enumerate}

Further, when $k \geq 3$, on three of these points must lie on a
line. For, a cubic curve through 3 points on a line, if it also
contains a fourth point of the line, must contain the entire line as a
component, so that the corresponding rational map will contract the
entire line through these three points to a single point. For, similar
reasons, when $k=6$, all the six points should not lie on a conic. 

Suppose then that $k \leq 3$ and a set $P_1 , \ldots , P_k$ of $k$
points in $\mathbb{P}^2$ are given satisfying the above conditions,
Let $(f_\circ , \ldots , f_n)$ be a basis of the vector space of cubic
forms vanishing at these points.We shall verify that condition (A)
above is fulfilled by this set of functions and $x_1 \neq P_i$ for any
$i$. ($x_1x_2$ are as in statement of condition $(A)$). By choosing
further points suitably if necessary, it is sufficient\pageoriginale
to verify this  when $k=6$, the `worst possible' case. We may also
clearly assume that $x_2 \neq P_i$  for any $i$. The dimension of the
space of forms vanishing at $P_i(1 \leq i \leq 6)$ and $x_i$ is at
least $10-7 =3$. Suppose the line $x_1 x_2$ does not contain any
$P_i$. Choose 
distinct points $y,z$ of this line distinct form $x_1, x_2$. Then
there is a non-zero cubic form $F$ vanishing at the $P_i,x_1,x_2,y$
and $z$. Since a cubic form not vanishing identically on a line can
have at most 3 zeros on this line, $F$ must vanish identically on
the line $x_1 x_2$. Hence, $F=0$ must have as its other component a
conic through all the $P_i$. But this is impossible. Next suppose the
line $x_1 x_2$ contains just one $P_i$, say $P_1$. Choose a $y$ on
this line distinct from $ x_1, x_2$ and $P_1$. The space of cubic
forms vanishing at the $P_i,x_1$ and $y$ has dimension 2,and all
these forms vanish on the line $x_1 x_2$. By dividing out by a
defining linear form of this line, we get a vector space of quadratic
forms of dimension $\geq 2$, all of which vanish at the points $P_2,
\ldots ,P_6$. Hence,we can find a quadratic form of this space which is
non-zero and vanishes also at $P_i$. This is a contradiction. The case
when the line $x_1,x_2$ contains two points among the $P_i$ is
similarly dealt with.  

We shall next verify the condition $(B)$ for points $x$ distinct from
the $P_i$. Suppose all the forms $F$ vanishing at the $P_i$ and $x$
vanish to the second order at $x$, or suppose all the curves $F=0$
which are non-singular at $x$ have fixed tangent line at $x$. In
either case we can find a line $L$ through $x$ such that the
restrictions to $L$ of all these forms vanish to the second order at
$x$. If $L$ does not pass through\pageoriginale any of the $P_i$,
choose two points 
$y,z$ on $L$ distinct from $x$. The dimension of the space of cubic
forms vanishing at the $P_i,x,y$ and $z$ is $\geq 10-6-3=1$, so that
we can find an $F \neq 0$ and vanishing at these points. Since $F$
vanishes to the second order at $x$ and also at $y$ and $z$ when
restricted to $L,F$ vanishes identically on $L$. Thus, $F=0$ has a
component which is a conic passing through all the $P_i$. This is a
contradiction. One argues similarly when $L$ passes through one or two
of the $P_i$  

Finally, we verify condition $(B)'$, at the point $P_1$, say.We need a
simple fact, namely that given four points of $\mathbb{P}^2$, on three
of which are collinear, and given a non-zero vector at one of them,
there is a conic through these points (which may be degenerate) which
has the given tangent vector at the specified point. In fact, we may
assume after a projective transformation that the points are
(1, 0, 0), (0, 1, 0), (0, 0, 1) and (1, 1, 1), and conics through these
points are given by $fx_1x_2 + gx_2x_\circ + hx_\circ x_1 = 0$, and this has
the tangent line $hx_1 + gx_2 =0$ at (1, 0, 0). But this represents an
arbitrary line through (1, 0, 0). This proves the assumption. Now let
$C$ be an arbitrary non- singular curve through $P_1$ and $D$ the
conic through $P_1, P_2, P_3$ and $P_4$ which is non-singular at $P_1$
and touches $C$ at $P_1$. Suppose every cubic through $P_1, \ldots
,P_6$ which is non-singular at $P_1$ and touches $C$ at $P_1$ has
second order contact with $C$ at $P_1$. In particular, the degenerate
cubic consisting of $D$ and the line $P_5 P_6$ has second order
contact with $C$ at $P_1$, so that we may assume that
$C=D$.\pageoriginale Now, the space of all cubic forms which vanish at
$P_2, P_3, P_4, P_5$ and 
$P_6$, and whose restriction to $D$ vanishes to the second order at
$P_1$ is of dimension 3. We know that $D$ does not containing $P_5$ or
$P_6$. Choose a point $x$ on $D$ distinct from the $P_i$, and a point
$y$ not on $D$ or the line $P_5P_6$. Then there is a non-zero cubic
form $F$ vanishing at $P_2 , \ldots,P_6$, $x$ and $y$, and whose
restriction to $D$ vanishes to the second order at $P_1$. But this
last condition means that the curve $F=0$ is non-singular at $P_1$ and
touches $D$ at $P_1$. By assumption, it must therefore have contact
of order $2$ with $D$ at $P_1$, or equivalently, its restriction to
$D$ vanishes to the $3$rd order at $P_1$. Since $F$ also vanishes at
$4$ other distinct points of $D,F$ must vanish on $D$. The other
component of $F=0$ must therefore be the line $P_5 P_6$. But $F$
vanishes also at $y$, and $y$ does not lie on either $D$ or $P_5 P_6$,
which is a contradiction. 

Hence, we have proved that for a set of $k \leq 6$ points of
$\mathbb{P}^2$ such that no three lie on a line (if $k \geq 3)$, and
not all six lie on a conic (if $k= 6$), the variety $X'$ got by
blowing up $\mathbb{P}^2$ at these points can be imbedded biregularly
in $\mathbb{P}^{9-k}$ by means of the cubic forms vanishing at the
given points. Since hyperplane sections on $X'$ are proper transforms
of cubic curves on $\mathbb{P}^2$   through the $k$ points, we see
that the degree of $X'$ for this imbedding is $9-k$. 

The resulting surfaces are called the \textit{Del Pezzo} surfaces. It
has been shown (\cite{key18} \S 11) that these are the only non-singular
surfaces in $\mathbb{P}^n$ of degree $n$ which are not contained in a
linear subvariety. 

In\pageoriginale particular, all cubic surfaces in $\mathbb{P}^3$ which are
non-singular are obtained by six blowings up from the plane
$\mathbb{P}^2$. We shall now derive the notorious $27$ lines on such a
cubic surface. 

A line in $\mathbb{P}^3$ is characterised by the fact that it is a
curve which has exactly one point of intersection, counting
multiplicity, with a general hyperplane. Since the fibre in $X'$ over
a blown up point $P_i$ has a unique point of intersection with the
proper transform of a general cubic curve through the $P_j(1 \leq j
\leq 6)$, each such fibre is a line on $X'$. Any other line on $X'$ is
the proper transform of an irreducible curve $C$ on
$\mathbb{P}^2$. Further, since a line on $X'$ can be realised as a
(proper) component of a hyperplane section of $X'$ it follows that $C$
must be a (proper) component of a cubic curve through the $P_j$ in
$\mathbb{P}^2$. Hence, $C$ must be either a line or a non-degenerate
conic. Since any point of intersection of $C$ and a cubic curve
through the $P_j$ which is distinct from all the $P_j$ corresponds to
a point of intersection of the proper transform of $C$ in $X'$ and a
hyperplane, it follows that $C$ can intersect a general cubic through
the $P_{j}$ at most one point of $\mathbb{P}^2$  besides
possibly the $P_j(1 \leq j \leq 6)$, counting multiplicity. Suppose
$C$ is a line on $\mathbb{P}^2$. Then $C$ meets a general cubic
through the $P_j$ at 3 points, counting multiplicity. Since the cubic
curves through the $P_j$ `separate directions' at the $P_j$ as we have
shown earlier, a general cubic through the $P_j$ is not tangential to
$C$ at any point $P_j$ contained in $C$; so that such a point
$P_j$occurs with multiplicity $1$ the  intersection\pageoriginale of
$C$ and a general cubic through the $P_j$. It follows that $C$ must
contain at 
least two points among the $P_j$, since other wise it would have at least
two points of intersection, counting multiplicity, outside the $P'_j
s$, which is impossible.Similarly, if $C$ is a conic, one sees that $C$
must contain $5$ of the 6 points $P_j$. Conversely, we see that any
line through two of the points $P_j$ or any conic through 5 of the
points $P_j$ has for proper transform in $X'$ a curve of degree 1 on
$X'$, that is, a line on $X'$. Thus we get exactly $ 6 + 6 + (
^6_2)=27$ lines on the cubic surface $X'$, and the relations of
incidence in this configuration of straight lines are also very
clearly seen by means of this representation. 

By means of cubic forms vanishing at $7$ points of $\mathbb{P}^2$ in
general position, in the above sense, we get a morphism of the
surface obtained by blowing up $\mathbb{P}^2$ at these points onto
$\mathbb{P}^2$. This morphism is in fact a (ramified) covering of
$\mathbb{P}^2$ of order 2 described on the affine open set $x_\circ \neq
0$ of $\mathbb{P}^2$ by an equation $Z^2 = f_4(x_1, x_2)$, where $f_4$
is a polynomial of degree 4. 

The surface obtained by blowing up $ k \leq 7$ points in general
position on $\mathbb{P}^2$ (together with the limiting cases where
blowing up certain points of the fibres of earlier dilatations are also
allowed), have been characterised as the only non-singular surfaces
ion $\mathbb{P}^n$ all of whose hyperplane sections are of genus
1. (\cite{key4}). 

\noindent
\textit{The\pageoriginale resolution of singularities of a curve on a
  surface by dilatations.}   

\noindent
\begin{theorem*}[M.Noether]
Let $X$ be a noetherian equidimensional regular prescheme of
    dimension 2, and $C$ an irreducible, reduced, closed subscheme
    of dimension one in $X$. Assume that for any affine open subset
    $U$ of $C$, the integral closure of $\Gamma (U, \mathscr{O}_C)$ in
    its quotient field is a module of finite type over $\Gamma (U,
    \mathscr{O}_C)$. 

Then the set of singular (i.e., non-regular) points of $C$ is
  a finite set of closed points of $C$. Further, there is a prescheme
  $Y$and a morphism $\tau : Y \to X$ satisfying the following
  conditions: 
\begin{enumerate}[(i)]
\item $\tau$ can be factorised as $Y = X_n
  \xrightarrow{\sigma_{ n-1}} X_{n-1} \xrightarrow{\sigma_{ n-2}}
  \cdots \xrightarrow{\sigma_\circ} X_\circ = X, \tau =
  \sigma_\circ\circ\sigma_1 o 
  \ldots\circ\sigma_{n-1}$, where each $ \sigma _i : X_{i+1}
  \to X_i$ is $X_i$ isomorphic to the dilatation of $X_i$
  at a closed point $x_i$ of $X_i$, whose image in
    $X$ by  $\sigma_\circ~\circ\ldots\circ\sigma_{i-1}$ is a singular
    point of $C$; 

\item The unique reduced, irreducible closed subscheme $C'$ of
  $Y$ such that $\tau(C') = C$ (which shall be referred to as the
  proper transform of $C$ on $Y$) is non -singular. 
\end{enumerate}
\end{theorem*}

\begin{proof}
  Let $U$ be any affine open set on $C, B = \Gamma (U,\mathscr{O}_C)$
  and $\bar B$ the integral closure of $B$ in its quotient field. Then
  for any closed point $x$ of $U$ defined by the maximal ideal
  $\mathcal{M}_x$  of $B$, the integral closure of
  $\mathscr{O}_{x,C}=B_{\mathcal{M}x}$ is precisely the localisation
  $\bar{(B)}_{\mathcal{M}_x}$ of $\bar B$ with respect to the
  multiplicatively closed subset $B -\mathcal{M}_x$. Now, a
  one-dimensional noetherian local domain is regular (i.e., is a
  discrete\pageoriginale valuation ring) if and only if it is
  integrally closed.  
\end{proof}

\noindent
Thus, the set of singular points of $C$ in $U$ is precisely the set of
points $x$ in $U$ such that $B_{\mathcal{M}_x} \neq
\bar{(B)}_{\mathcal{M}_x}$, or equivalently, the set of points $x$ such
that $\bar{(B}/B)_{\mathcal{M}_x} \neq 0$, that is, the support of the
$B$-module $\bar{(B}/B)$. Since $\bar B$ is a finite $B$-module by
assumption, this set is closed. This proves the first part of the
theorem, since $X$ is quasi-compact, and the singular set is closed
and discrete. 

Let us now prove the second part.For any $x \in C$, let $
\mathscr{O}_{x,C}$ denote the local ring of $x$ on $C$, and
$\bar{\mathscr{O}}_{x,C}$ its integral closure in the quotient
field. Since by assumption $\bar{\mathscr{O}}_{x,C}$ is an
$\mathscr{O}_{x,C}$-module of finite type, and since there is a
non-zero element a $\in \mathscr{O}_{x,C}$ such that a $
\bar{\mathscr{O}}_{x,C} \subset \mathscr{O}_{x,C}$it follows that the
$\mathscr{O}_{x,C}$-module $ \bar{\mathscr{O}}_{x,C /
 \mathscr{O}_{x,C}}$ is of finite length. We denote the length of
this module by ${}^1\mathscr{O}_{x,C}
(\bar{\mathscr{O}}_{x,C/\mathscr{O}_{x,C}})$, and we put 
$$
N(C,X) = \sum_{x=C}
l_{\mathscr{O}_{x,C}}(\bar{\mathscr{O}}_{x,C/\mathscr{O}_{x,C}})  
$$

Then $N(C,X)$ is a non-negative integer associated to $C$ and $X$, and
$N(C,X) =0$ if and only if $C$ is regular. 

Suppose then that $C$ is not regular, and let $x$ be a singular point
of $C$. Let $X'$ be the dilatation of $X$ at $x$, and $C'$ the proper
transform of $C$ on $X'$. If we can show that $N(C',X')< N(C,X)$ the
theorem clearly follows by induction on the integer $N( C,X)$.  

We thus\pageoriginale have to show that for the dilatation $X'$ of $X$
at a singular point $x$ of $C,N (C',X')< N (C,X)$. Let $\sigma : X' \to X $
be the dilatation morphism. Since $\sigma$ is an isomorphism of $X' -
\sigma^{-1}(x)$ onto $X-\{x\}$, we have only to show that 
$$
\sum_{ \substack {\sigma(y)=x \\ {y \in
      C'}}} l_{\mathscr{O}_{y,C'}}(\bar{\mathscr{O}}_{y,C'/\mathscr{O}_{y,C'}})
< l_{\mathscr{O}_{x,C}}(\bar{\mathscr{O}}_{x,C/\mathscr{O}_{x,C}}) 
$$

Suppose we can prove that there is a ring $ A \neq \mathscr{O}_{x,C}$
integral over $\mathscr{O}_{x,c}$ and contained in its quotient field,
such that the $\mathscr{O}_{y,C'}$ for $ y \in \sigma^{-1}(x) \cap C'$
are the distinct localisations of $A$ at its maximal ideals. We would
then have 
\begin{gather*}
  \sum_{\substack {\sigma (y) = x \\ {y \in
        C'}}} l_{\mathscr{O}_{y,C'}}(\bar{\mathscr{O}}_{y,C'/\mathscr{O}_{y,C'}})
  \leq \sum_{ \substack {\sigma (y) = x \\{y \in C'}}}
       [k(y):k(x)]l_{\mathscr{O}_{y,C'}}(\bar{\mathscr{O}}_{y,C'/\mathscr{O}_{y.C'}})\\ 
       =\sum_{ \substack {\sigma (y)= x \\{y \in
             C'}}}l_{\mathscr{O}_{x,C}}(\bar{\mathscr{O}}_{y,C'/\mathscr{O}_{y,C'}})
       = l_{\mathscr{O}_{x,C}}(\bar{\mathscr{O}}_{x,C}/A)\\ 
       < l_{\mathscr{O}_{x,C}}(\bar{\mathscr{O}}_{x,C/\mathscr{O}_{x,C}}),
\end{gather*}
as required.

Let $\tau$ be the restriction of $\sigma$ to $C'$, considered as a
morphism into $C$. Since $\tau$ is proper with finite fibres, one
could appeal to ($\EGA$, III, 4.4.2) to conclude that $\tau$ is a finite
morphism, from which the existence of an $A$ integral over
$\mathscr{O}_{x,C}$ such that the localisations of $A$ at its maximal
ideals are the local rings $\mathscr{O}_{y,C'}$ of\pageoriginale points $y$ of
$\tau^{-1}(x)$, follows. However, we shall prove this directly,
without appeal to the above mentioned theorem. Let us put $A = \bigcap
\limits_{y \in \tau^{-1}(x)}\mathscr{O}_{y,C'}$, the intersection
being taken in the quotient field $R(C)$. Let $\bar{C}$ be the
normalisation of $C$ in $R(C)$ and $\pi :\bar{C} \to C$ the
projection. Since $C'$ is $C$-proper, by the statement proved at the
beginning of the lecture, there is a morphism $\pi':\bar{C} \to C'$
such that $\tau \circ\pi' = \pi$. It follows that $\bar{C}$ is the
normalisation of $C'$ also. Now, $A$ is a subring of
$\bigcap\limits_{\substack{\pi (z) = x \\ {z \in \tau}}}
\mathscr{O}_{z,\bar{C}} = \bar{\mathscr{O}}_{x,C}$, so that $A$ is
integral over $\mathscr{O}_{x,C}$. It remains to be shown that each
$\mathscr{O}_{y,C'}(y \in \tau^{-1}(x))$ is a localisation of
$A$. Since $\bar{C}$is the normalisation of $C'$, we can find an
integer $n > o$ such that $\bigcap\limits_{\pi' (z)=y}
\mathcal{M}^n_z, \bar{C} \subset \mathcal{M}_{y,C'}$ for all $y \in
\tau^{-1}(x)$. Let $y_1, \ldots , y_n$ be the distinct points of
$\tau^{-1}(x)$. By the theorem of independence of valuations (or by
the Chinese remainder theorem, applied to the co-ordinate ring of an
affine open set $\tau^{-1}(U)$ of $\bar{C}$, where $U$ is an affine
open neighbourhood of $x$ on $C$), we can find a rational function
$\varphi$ on $\bar{C}$ such that $\varphi \equiv 1 (\mod
\mathcal{M}^n_{z,\bar C}$ for all $z$ with $\pi ' (z)=y_1$ and
$\varphi \equiv 0 (\mod \mathcal{M}^n_{z,\bar C}$ for all $z$ with
$\pi'(z) = \gamma_{\xi}(i > 1)$. It then follows that $\varphi \in A$
and $\varphi$ vanishes at $y_2, \ldots , y_n$ but not at $y_1$. Hence,
for any $f \in \mathscr{O}_{y_1,C'}$, we have $\varphi^m f \in
\bigcap\limits^{n}_{i=1} \mathscr{O}_{y_i,C'}=A$ for $m$ large. This
proves that $\mathscr{O}_{y_1,C'}$ (and similarly also
$\mathscr{O}_{y_2,C'}$,etc). is a localisation of $A$. 

It only\pageoriginale remains to show that (when $x$ is a singular
point) $A \neq 
\mathscr{O}_{x,C}$, or equivalently that $\tau : C' \to C$ is not an
isomorphism. Suppose $C$ is defined at $x$ by an element $f$, so that
we have $ f \in \mathcal{M}^l_{x,X}$, $f \notin \mathcal{M}^{l+1}_{x,X}$
for some $l>1$. If $(u,v)$ generate  $\mathcal{M}_{x,X}$, we can then
write  
$$
f=a_\circ u^l + a_1 u^{l-1} v + - + a_l v^l, a_i \in \mathscr{O}_{x,X},  
$$
and not all $a_i$ belonging to $\mathcal{M}_{x,X}$. Let $\bar{a}_i$
denote the canonical image of $a^i$ in $k(x)=k$. Identifying $
\sigma^{-1}(x)$ with $\mathbb{P}^1(k)$ = proj $(k[W', W''])$ as in
Lecture \ref{chap2}, we have shown in that  lecture that the points of
intersection of $C'$ and $ \sigma^{-1}(x)$ are the ``zeros'' (i.e.,
correspond to the irreducible factors ) of the binary from
$\bar{a}_\circ     
W''^l + \bar{a}_1 W''^{l-1} W' + \cdots + \bar a_l W'^l$. If $\tau $
were an isomorphism, there can in particular be only one irreducible
factor, and this factor should further be linear (since otherwise, the
residue field $k(y)$ at $y \in \tau^{-1}(x)$ would be a non-trivial
extension of $k(x))$. Hence, by changing $(u,v)$ if necessary, we may
assume that  
$f = v^l + g$, $g \in \mathcal{M}^{l+1}_{x,X}$. In this case, there is a
unique point $y$ in $\sigma^{-1}(x)\cap C'$, and $w' = \dfrac{v}{u}$
is regular and vanishes at this point. Let $\omega'$ be the image of
$w'$ in $\mathscr{O}_{y,C'}$ so that $\omega' \in
\mathcal{M}_{y,C'}$. If $\tau$ were an isomorphism, we must have
$\mathscr{O}_{x,C} = \mathscr{O}_{y,C'}$, so that we would have $v_C =
\omega u_C$, where $v_C$ and $u_C$ are the images of $v$ and $u$
respectively, in $\mathscr{O}_{x,C}$. But then, it would follow that 
$$
\mathcal{M}_{x,C} = (u_C, v_C) = (u_C),
$$
so that $x$ would be a regular point of $C$. Contradiction.

The\pageoriginale theorem is completely proved.

\begin{remarks*}
  \begin{enumerate}[1)]
  \item The assumption  made on $C$, that for any affine open subset
    $U$ of $C$, the integral closure of $\Gamma (U, \mathscr{O}_C)$ is
    of finite type over $\Gamma (U,\mathscr{O}_C)$, is necessary not
    only for the closeness of the set of singular points, but is also
    essential even for the resolution of the singularity at a single
    point of $C$ by dilatations. In fact, it is easily shown (from what
    has been established during the course of the above proof) that if
    $x$ is a singular point of $C$ which can be resolved by a finite
    sequence of dilatations, $\bar{\mathscr{O}}_{x,C}$ must be an
    $\mathscr{O}_{x,C}$ module of finite type. 

    Incidentally, we may note that for a one-dimensional local domain
    $A$, its integral closure $\bar{A}$ is a finite $A$-module if and
    only if the completion $\hat{A}$ of $A$ has no nilpotent
    elements.(L.R. Chap. V, \S 33, Ex. 1). 

    This assumption is further fulfilled in all `good' cases, for
    instance when $X$ is of finite type over a field or over
    $\mathbb{Z}$. More generally,the suitable assumption to make on
    $X$ is that it be a {\small ``Japanese prescheme''} ($\EGA$, O, \S 23), in order
    to ensure the above condition. 

  \item It is not necessary to assume that $C$ is irreducible, but
    only that it is reduced. Indeed, we may first resolve all the
    singularities of all the components of $C$ by a finite sequence of
    dilatations. After this is done, one has only to separate two
    components of $C$ which may intersect at a point $x$ of $X$, by
    further dilatations. Suppose $C_1, C_2$\pageoriginale are two
    components of $C$ 
    simple at $x$, and having order of contact $l$ at $x$. We know that
    after a dilatation at $x$, the proper transforms $C'_1,C'_2$ of
    $C_1,C_2$ respectively have contact of order $l-1$ at a point of
    the fibre over $x$. Thus, after $l + 1$ dilatations, we see that
    the proper transform  of $C_1$ and $C_2$ do not intersect at any
    point lying over $x$. 
  \end{enumerate}
\end{remarks*}
