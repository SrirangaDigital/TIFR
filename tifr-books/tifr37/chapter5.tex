

\chapter[The behaviour of various groups associated to...]{The behaviour 
of various groups associated to a scheme under 
birational transformations}\label{chap5}%chap 5 

\markright{\thechapter. The behaviour of various groups associated to...}

Let\pageoriginale $X$ be a locally noetherian, everywhere two
dimensional, regular prescheme, $x$ a closed point of $X$ and $\sigma
:X' \to X$ the dilatation of $X$ at $x$. Let $\mathscr{F}$ be a
coherent sheaf of $\mathscr{O}_{X'}$-modules on $X'$. For any point $y
\neq x$ on $X$, let $V$ be an affine open neighbourhood of $y$ on $X$
not  containing 
$x$. Then $\sigma^{-1}(V)$ is isomorphic to $V$, and hence is also
affine, so that $H^{p}(\sigma^{-1}(V),\mathscr{F})=0$ for $p \geq 1
(\EGA ~III)$. This proves that the higher direct images
$R^{p}\sigma(\mathscr{F})$ for $p \geq 1$ are concentrated at the
point $x$ of $X$. Further, let $X_o= \Spec A_o$ be an affine open
neighbourhood of $x$ in $X$, such that the maximal ideal
$\mathcal{M}_o$ of $x$ in $A_o$ is generated by two elements $u, v$ of
$A_o$. Then the open set $X'_o =  \sigma^{-1}(X_o)$ of $X'$ is covered
by two affine open sets $Y'_o = \Spec A_o[w']$ and $Y''_o = \Spec
A_o[w'']$, where $w'= \dfrac{v}{u}$ and $w''
=\dfrac{u}{v}$. Denoting this affine covering $\{Y'_o, Y''_o\}$ by
$\mathscr{G}$, we have canonical isomorphisms 
$$
H^{p}(X'_o, \mathscr{F}) \tilde{\leftarrow} \c{H}^p(\mathscr{G},
\mathscr{F}) 
$$
where $\c{H}^{*}$ denotes the \v{C}ech cohomology groups of this
covering ($\EGA$ ~III, 1,2). But now, the \v{C}ech groups
$\c{H}^{p} (\mathscr{G},\mathscr{F})$ can be computed by using the
complex of alternating cochains of $\mathscr{G}$ with values in
$\mathscr{F}$ ($\EGA$ III, 1.4.1). Since $\mathscr{G}$ contains just two
elements, any alternating $p$-cochain on $\mathscr{F}$ is $0$ if $p \geq
2$. Hence, $H^{p}(X'_o, \mathscr{F})\simeq \c{H}^{p} (\mathscr{G},
\mathscr{F})=0$ for $p \geq 2$, and $R^{p} \sigma(\mathscr{F}) = 0$
for $p \geq 2$. 

Hence,\pageoriginale in this case, the (convergent) spectral sequence
of Leray 
$$
E^{p,q}_2 = H^{p}(X, R^q \sigma (\mathscr{F})) \Longrightarrow H^n
(X', \mathscr{F}) 
$$
degenerates into an exact sequence
\begin{multline*}
  0 \to H' (X, R^o \sigma (\mathscr{F})) \xrightarrow{\sigma^*_1} H^1(X',
  \mathscr{F})^{\alpha} \to H^o(X, R' \sigma (\mathscr{F}))\\ 
  \to H^2(X,R^o \sigma (\mathscr{F})) \xrightarrow{\sigma^*_2} H'(X',
  \mathscr{F}) \to 0. 
\end{multline*}

Note that here, $\sigma^*_1$ and $\sigma^*_2$ are simply the canonical
homomorphisms induced in cohomology by the morphism $\sigma$, and
$\alpha : H' (X', \mathscr{F}) \to H^o (X, R' \sigma (\mathscr{F}))$ is
the homomorphism of a presheaf into the sections of the associated
sheaf. 
 
Let us take for $\mathscr{F}$ the structure sheaf $\mathscr{O}_{X'}$,
of $X'$. The homomorphism $\mathscr{O}_X \to R^o \sigma
(\mathscr{O}_{X'})$  is clearly an isomorphism outside the point
$x$. Let $s$ be a section of $R^o \sigma (\mathscr{O}_{X'})$ in a
neighbourhood $U$ of $x$, so that $s$ is a regular function in a
neighbourhood of $\sigma^{-1}(x)$ on $X'$. But now, $s$ defines a
section $s'$ of $\mathscr{O}_X$ in $U-\{x\}$. But we have seen earlier
that any rational function on a regular (or even normal) prescheme
$Y$, if it is regular at all points $y \in Y$ such that
$\mathscr{O}_y$ is a discrete valuation ring, is regular on
$Y$. Hence, $s'$ extends to a section $s_1$ of $\mathscr{O}_x$ on
$U$. Denoting the image of $s_1$ in $R^o \sigma (\mathscr{O}_{X'})$ by
$s_2, s- s_2$ is a regular function in a neighbourhood of
$\sigma^{-1}(x)$ which to $0$ outside $\sigma^{-1}(x)$, so that
$s=s_2$. Thus, $\mathscr{O}_X \to R^o \sigma (\mathscr{O}_{X'})$ is an
isomorphism. We next show that $R' \sigma(\mathscr{O}_{X'})=0$. With
notations as in the first paragraph, we have to show that
$\c{H}'(\mathscr{G},\mathscr{O}_{X'})=0$. An alternating
1-cocycle of the covering $\mathscr{G}$ is simply\pageoriginale an
element $f$ of $\Gamma (Y'_o Y''_o, \mathscr{O}_{X'})= A_o [w', w'']$,
so that can write  
 $$
 f=\sum\limits_{O}^{n} a_n w'^n + \sum\limits_{O}^{n} b_n w''^n = f' -
 f'',  
 $$
 where $f' \in \Gamma (Y'_o, \mathscr{O}_{X'})$ and $f'' \in \Gamma
 (Y''_o, \mathscr{O}_{X'})$. This means that any such cocycle is a
 coboundary, and our assertion is proved. 
 
We have thus the results

$R^o \sigma (\mathscr{O}_{X'})\backsimeq \mathscr{O}_X$

$R^p \sigma (\mathscr{O}_X')=0 , p\geq 1, H^p(X',
\mathscr{O}_X')\backsimeq H^p (X, \mathscr{O}_x), p \geq 0$. 

In view of the theorems of domination and composition of the earlier
lecture, we deduce that if $X_1$ and $X_2$ are two regular,
noetherian, proper $B$-schemes ($B$ being arbitrary) of dimension two,
and $\varphi$ a birational map of $X_1$ into $X_2$, $\varphi$ induces
isomorphisms 
$$
H^p (X_1, \mathscr{O}_{X_1})\simeq H^p (X_2, \mathscr{O}_{X_2}), \; p\geq
0. 
$$

When in particular $X$ is a proper scheme over $B = \Spec k$ where $k$
is a field, the $k$-vector spaces $H^i (X, \mathscr{O}_X)$ are finite
dimensional ($\EGA$, III~ 3.2.4), and the integer $p_a (X)= \dim_k
H'(X, \mathscr{O}_X)$ is called the \textit{arithmetic genus} of
$X$. We have thus proved the invariance of arithmetic genus under
birational transformations. ($H^p( X, \mathscr{O})$ is invariant under
birational isomorphisms.) 

It is\pageoriginale not known whether this is true for complete non-singular
varieties of arbitrary dimension. However, when the base field is the
field $\mathbb{C}$ of complex numbers and $X$ and $Y$ are assumed
projective, we have from the theory of Kahler manifolds the equalities
$\dim_\mathbb{C} H^{p}(X, \mathscr{O}_X)= \dim_\mathbb{C} H^o (X,
\Omega^{p}_{X}), \dim_\mathbb{C} H^{p}(Y, \mathscr{O}_Y)=
\dim_\mathbb{C} H^{o}(Y, \Omega^{p}_{Y})$, where $\Omega^{p}_{X}$ and
$\Omega^{p}_{Y}$ denote the sheaves of germs of holomorphic $p$-forms
on $X$ and $Y$ respectively ;  and by an argument similar to the one
used for proving that $R^o \sigma (\mathscr{O}_{X'}) =
\mathscr{O}_{X}$, we see that $R^{o} f (\Omega^{p}_{X}) =
\Omega^{p}_{Y}$, so that $H^{o}(X, \Omega^{p}_{X}) \backsimeq H^{o}(Y,
\Omega^{p}_{Y})$. Hence we have in this case the equalities
$$
\dim_\mathbb{C} H^p (X, \mathscr{O}_X) = \dim_\mathbb{C} H^p (Y,
\mathscr{O}_Y), p \geq 0. 
$$

This naturally holds when $X$ and $Y$ are projective non-singular over
any algebraically closed field of characteristic zero. 

Again, let $X$ be a regular, proper, two dimensional schemes over a
field $K$, and $\sigma : X' \to X$ the dilatation of $X$ at a closed
point $x$ of $X$. Let $\Omega'_{X} =  \Omega'_{X / K}$
(resp. $\Omega'_{X' / K})$ be the sheaf of $K$-differentials of
$X$(resp, $X'$). We have a canonical map $\Omega'_X \to R^o \sigma
(\Omega'_{X'})$ which is clearly an isomorphism when restricted to
$X-\{x\}$, and we shall show that it is surjective. Let $X_o = \Spec
A_o$ be an affine open neighbourhood of $x$ satisfying the conditions
mentioned in the first paragraph, $X'_o = \sigma^{-1}(X_o)$, and let
$Y'_o = spec A_o [w']$, $Y''_o = spec A_o[w'']$ be the usual
affine open sets of $X'_o$ covering $X'_o$. We put\pageoriginale  $B' = A _o [w']$,
$B''= A_o [w'']$ and $C= B' . B''= A_o [w', w'']$. For any
commutative $K$-algebra $\wedge$, let us denote by $D_K(\wedge)$ the
$\wedge$- module of differentials of $\wedge$ over $K$. An element
$\omega \in \Gamma (X_o, R^o \sigma (\Omega'_X)) = \Gamma(X'_o,
\Omega'_{X'})$ is represented by a pair of elements $\omega ' \in D_K
(B')$ and $\omega'' \in D_K(B'')$ such that the images of
$\omega'$ and $\omega''$  in $D_K(C)$ coincide. Now, $\omega'$ can
be written as $\omega' = \sum b'_{i} da'_i + f'(w') dw'$ with
$f' \in A_o [X], a'_i \in A_o, b'_i \in B'$ and similarly
$\omega'' = \sum b''_j da''_j + f''(w'')dw''$ with
$a''_j \in A_o, b''_j \in B''$ and $f''\in A_o [X]$. Let
us denote by $\bar{f}'$ and $\bar{f}''$ the polynomials over
$k[X]=\dfrac{A}{\mathcal{M}}[X]$ got by reduction modulo $\mathcal{M}$
from $f'$ and $f''$. Let $l$ be the algebraic closure of $k$, and
$\theta : \mathbb{P}' (l) \to \mathbb{P}'(k) \backsimeq \sigma^{-1}(x)
\hookrightarrow X'$ the composite morphism. Then $\theta (\omega)$ is
a regular differential form on $\mathbb{P}'(l)$, and hence is $0$, as
is well-known (and trivial to check). But $\theta^* (\omega)$ is
represented on a standard affine open set of $\mathbb{P}'(\ell)$ by
$\bar{f}'(X) dX$, so that $\bar{f}' \equiv 0$, and similarly
$\bar{f}'' \equiv 0$. Thus the coefficients of $f'$ and $f''$ belong
to $\mathcal{M}$. Since $\mathcal{M} B' =  u B'$, we can write $f'(w')
dw' = g'(w'). u dw' = g'(w') dv - g'(w') w' du$ with $g' \in B'
[X]$. Thus, we may assume without loss of generality that $\omega' =
\sum b'_i da'_i, \omega'' = \sum b''_j da''_j, a'_i,
a''_j \in A_o, b'_i \in B''b''_j \in B''$. Now, since
$C= A_o [w', w'^{-1}]$ and $A_o[w']$ is the quotient of the polynomial
ring $A_o[w]$ by the principal ideal $(uW-v)$, it follows from well -
known facts concerning the behaviour of modules of differentials under
passage to quotient\pageoriginale  and formation of rings of fractions
($\EGA$, O,~ \S~ 20) that   
$$
D_K(C) \simeq \frac{C \otimes_{A_o} D_K (A_o) \oplus C. dw'}{C.(w' du + u
  dw' - dv)}. 
$$
where $C.dw'$ denotes a free module of rank 1. It follows that the
canonical homomorphism $C \otimes_{A_o} D_K(A_o) \to D_K(C)$ is an
injection. Since $\sigma _* (\mathscr{O}_{X'}) = \mathscr{O}_X$, we
have an exact sequence of $A_o$-modules  
$$ 
0 \to A_o \xrightarrow{\xi} B' \oplus B'' \xrightarrow{\eta} C \to 0
$$
where $\xi (a) = (a, a)$ and $\eta (b', b'') = b' - b''$, and
tensoring with $D_K(A_o)$, we obtain the exact sequence 
$$
D_K(A_o) \xrightarrow{\xi \otimes 1} B' \otimes_{A_o} D_K(A_o) \oplus
B'' \otimes_{A_o} D_K(A_o) \xrightarrow {\eta \otimes 1} C
\otimes_{A_o} D_K(A_o) \to 0 
$$

Since by assumption, $(\omega', \omega'')$ belong to the kernal of
$\eta \otimes 1$, it follows that $\omega = \sigma^* (\omega_1)$ where
$\omega_1 \in D_K(A_o)$. In other words, the canonical homomorphism
$\Omega'_X \to \sigma_* (\Omega'_{X'})$ is surjective. The kernel of
this homomorphism has support at the point $x$, or is empty. It
follows from the cohomology exact sequence that we have an isomorphism
$H'(X, \Omega'_X) \simeq H'(X, R^o \sigma (\Omega'_{X'}))$. 

Let us look back at the exact sequence obtained from the Leray
spectral sequence with $\mathscr{F}$ replaced by the sheaf
$\Omega'_{X'}$. We shall show that in this case, the homomorphism
$\alpha $ is non-trivial. Let $\mathcal{U} = \{U_o, U_1, U_2\}$ be the
covering of $X$ given by $U_o = X' - \sigma^{-1}(x), U_1 = Y'_o,
U_2 = Y''_o$,  and let $\xi$ be the cohomology class of $H'(X',
\Omega'_{X'})$ given by\pageoriginale  the following alternating cocycle on the
covering $\mathcal{U} : $ $\xi_{o1} = \dfrac{du}{u} \in \Gamma (U_o
\cap U_1, \Omega'_{X'}), \xi_{o2} = \dfrac{dv}{v} \in \Gamma(U_o
\cap U_2, \Omega'_{X'})$, 
$$
\xi_{12} = \frac{dw}{w} \in \Gamma (U_1, \cap U_2, \Omega'_{X'})
$$

(This is the element of $H'(X', \Omega'_{X'})$ given by the divisor
$\sigma^{-1} (x)$ of $X')$. We shall show that $\alpha (\xi) \neq
0$. As we have already remarked $\alpha$ is obtained by passage to the
limit from the restriction maps in cohomology to neighbourhoods of the
fibre $\sigma^{-1}(x)$. Identifying $\sigma^{-1}(x)$ with
$\mathbb{P}'(k)$ as usual, we have a homomorphism $\Omega'_{X'}\to
\Omega'_{\mathbb{P'}(k)}$, and hence a linear map 
$$ 
H^{o}(X, R' \sigma (\Omega'_{X'})) \xrightarrow{\beta} H'
(\mathbb{P'}(k), \Omega'_{\mathbb{P}'(k)}) 
$$
and it is sufficient to show that $\beta o \alpha (\xi ) \neq 0$. If
$V_1 = \Spec k[W]$ and $V_2 = \Spec k[W^{-1}]$ from the standard
affine covering of $\mathbb{P}' (k), \eta = \beta o \alpha (\xi)$ is
represented by the alternating 1-cocycle 
$$
\eta_{12} = \frac{dW}{W},
$$
which is easily seen not to be a coboundary (represents the divisor
class of degree -$1$ on $\mathbb{P}'(k)$). 

Hence, when $\mathscr{F}= \Omega'_{X'}, \alpha$ is not
trivial. Thus we obtain that the map $\sigma^*_1 : H'(X, R^o) \sigma
(\Omega'_{X'})) \to H' (X' , \Omega'_{X'})$ is not surjective,
and hence the inequality 
$$
\dim_K H'(X, \Omega'_{X}) = \dim_K H'(X, R^o \sigma
(\Omega'_{X'})) < \dim_K H'(X' , \Omega'_{X'}). 
$$
This will be useful to us later on.

We remark\pageoriginale that more exact information can easily be
obtained by an 
explicit computation of $R' \sigma (\Omega'_{X'})$ by using the
affine covering $\mathscr{G} = \{ Y'_{o'} , Y''_o \}$ of
$X'_o$. We summarise the results, \textit{ when } $K$ \textit{is
  assumed algebraically closed}, both for $\Omega'$ and $\Omega^2$,
the sheaf of forms of degree $2$. 
\begin{align*}
  R^o \sigma (\Omega'_{X'}) &= \Omega'_{X}, \dim_K H^o (X,
  \Omega'_{X}) = \dim_K H^o(X', \Omega'_{X'})\\
  R^o \sigma (\Omega'_{X'}) \simeq K & =
  \dfrac{\mathscr{O}_x}{\mathcal{M}_x} ~\text{at}~ x, \dim_K H'(X,
  \Omega'_{X}) = \dim_K H'(X', \Omega'_{X'}) -1 \\
  R^o \sigma (\Omega'_{X'}) & = 0 ~\text{for}~ p \geq 2, \dim_K
  H^{2}(X, \Omega'_{X}) = \dim_K H^{2}(X', \Omega'_{X'})\\  
  R^o \sigma (\Omega^{2}_{X'}) & = \Omega^{2}_{X},\\
  R^o \sigma (\Omega^{2}_{X'}) &= 0, p \geq 1, \dim_K H^{p}(X',
  \Omega^{2}_{X}) = \dim_K H^{p}(X, \Omega^{2}_{X'}), p \geq 0. 
\end{align*}

The results for $\Omega^{2}$ also follow from Serre duality, of course
(\cite{key27}).

We shall next investigate the behaviour of the group $\vartheta(X)$
of divisors and the Picard group Pic $(X)$ of classes of inversible
sheaves on a noetherian, two dimensional, regular prescheme under a
dilatation $\sigma : X' \to X$. Since $\sigma$ induces an isomorphism
of $X' - \sigma^{-1}(x)$ onto $X- \{x\}, \sigma^* : \vartheta (X) \to
\vartheta (X')$ is clearly injective. Further, any divisor on $X'$ is
a sum of an element of $\sigma^* (\vartheta(X))$ and a multiple of the
fibre $\sigma^{-1}(x)$, and no multiple of $\sigma^{-1}(x)$ belongs to
$\sigma^* (\vartheta(X))$. Thus we have the isomorphism 
$$
\vartheta(X') \simeq \sigma^{*} (\vartheta(X)) \oplus
\mathbb{Z}. \sigma^{-1}(x) \simeq \vartheta(X) \oplus \mathbb{Z}. 
$$

Further let $\vartheta_l(X)$ denote the subgroup of $\vartheta(X)$
consisting of principal\pageoriginale divisors, so that we have a canonical
isomorphism Pic $(X) \simeq
\dfrac{\vartheta(X)}{\vartheta_l(X)}$. Since $\vartheta_l (X') =
\sigma^* (\vartheta_l (X))$, it follows that we have 
$$
Pic ~ (X') \simeq Pic (X) \oplus \mathbb{Z}.
$$

Suppose now that $X$ is a non-singular surface over an algebraically
closed field. In this case, algebraic equivalence of divisors is
defined (\cite{key15}, p. 55) and if $\vartheta_a(X)$ is the subgroup of
divisors of $X$ which are algebraically equivalent to $0$, the
quotient group $S(X) = \dfrac{\vartheta(X)}{\vartheta_a(X)}$ is called
the \textit{ Neron-Severi group } of $X$. It is known (\cite{key16}) that
$S(X)$ is a finitely generated abelian group. Now, we shall soon
define for a pair of elements $\alpha, \beta \in
\dfrac{\vartheta(X)}{\vartheta_l(X)}$ an intersection number $(
\alpha. \beta ) \in \mathbb{Z}$, when $X$ is a complete variety. We
shall also show that if $L = \sigma^{-1}(x)$ and $(L)$  its class in
$\vartheta (X')/\vartheta_l (X'), ((L). \sigma^* (\vartheta(X))) = 0$
and $(L).(L)=-1$. But it can be shown that $((L). \vartheta_a(X'))=0$
(the intersection number is preserved under algebraic equivalence on
an algebraic variety). It follows that $\vartheta_a(X') \subset
\sigma^*(\vartheta(X))$. Since it can also be shown that if $\alpha
\in \vartheta_a(X')$, the direct image $\sigma_*(\alpha) \in
\vartheta_a(X)$, and since we have for $\alpha \in \sigma^*
(\vartheta(X)), \alpha = \sigma^* \sigma_* (\alpha)$, we deduce that
$\vartheta_a(X') = \sigma(\vartheta_a(X))$. We therefore have the
following relationship between the Neron-Severi groups of $X$ and
$X'$: 
$$
S(X') \simeq S(X) \oplus \mathbb{Z}.
$$


\noindent
\textbf{ A diversion }

Several\pageoriginale questions, especially of number theory, are connected with
birational isomorphisms of surfaces $X$ defined over a field $K$ which
is not necessarily algebraically closed. The principal problem is to
classify under $K$-birational isomorphism those surfaces that become
birationally isomorphic when one extends the base field to the
algebraic closure $\bar{K}$ of $K$. We have constructed above some
numbers that are attached to the surface and are birational invariants
e.g. the arithmetic genus. But such invariants are unaltered by
extension of the base field and so are not useful for our purpose. We
have to construct finer invariants for this problem. The situation is
simple in the case of curves, where birational isomorphism essentially
coincides with isomorphism of schemes. 

Let us consider the simplest possible example, namely that of a scheme
$X$ over a field $K$, such that the scheme $\bar{X} = X X_K \bar{X}$
over the algebraic closure $\bar{K}$ of $K$ is isomorphic to the
projective line $\mathbb{P}'(\bar{K})$ over $\bar{K}$. It is easily
shown that such an $X$ is isomorphic as a $K$-scheme to a conic
$Q(X_o, X_1, X_2)=0$ in $\mathbb{P}^2 (K)$, where $Q$ is a
non-degenerate quadratic form with coefficients in $K$. In fact, from
the isomorphism $X X_K \bar{K} \backsimeq \mathbb{P}' \bar{K}$, it
follows that $X$ is an integral scheme defined over $K$, such that its
function field $R(K)$ is a regular extension of $K, R(\bar{X}) =  R(X)
\otimes_K \bar{K}$, and $X$ is absolutely simple. In particular, the
sheaf $\Omega_{\bar{X}, \bar{K}}$ of differentials of $\bar{X}$ over
$\bar{K}$ is got by base extension from the sheaf of $K$-differentials
of $X$, that is, $\Omega_{\bar{X}, \bar{K}} = \Omega_{X, K} \otimes_K
\bar{K}$. Hence we have a canonical isomorphism $H^o (\bar{X},
\Omega^{*}_{\bar{X}}) \simeq H^o (X, \Omega^{*}_{X}) \otimes_K
\bar{K}$.\pageoriginale Further, $\Omega^{*}_{\bar{X}}$ is the unique invertible
sheaf of degree 2 on $\bar {X} \simeq \mathbb{P}'(\bar{X})$, so that
we can choose a basis $\sigma_o, \sigma_1, \sigma_2$ of $H^o (\bar{X},
\Omega^{*}_{\bar{X}})$ such that if $t$ denotes the identity function
of $\mathbb{P}'(\bar{K}), \dfrac{\sigma_1}{\sigma_o} = t,
\dfrac{\sigma_2}{\sigma_o}=t^2$. The rational map of $\bar{X} \to
\mathbb{P}^2(\bar{K})$ given by $x \mapsto (1,
\dfrac{\sigma_1}{\sigma_o}(x), \dfrac{\sigma_2}{\sigma_o}(x))$ is
therefore a closed immersion of $\bar{X}$ onto a non-degenerate conic
in $\mathbb{P}^2(\bar{K})$ (defined by $X_o X_2 = X^2_1)$. Hence, a
similar assertion also holds for any other choice
$\sigma^{1}_{o},\sigma^{1}_{1},\sigma^{1}_{2}$ of basis for $H^o
(\bar{X}, \Omega^*_{\bar{X}})$. Now, we can choose such a basis
$\sigma'_{i}$ of $H^o (X, \Omega^*_{X})$ over $K$, and its remains
a basis of $H^o (\bar{X}, \Omega^*_{\bar{X}})$. But for such a choice
of $'_{i}$, we have a \textit{$K$-morphism } of $X/K$ onto a conic
in $\mathbb{P}^2(K)$ defined over $K$, which must of necessity be
non-degenerate. This morphism is again a closed immersion, since it is
so after a base extension. This proves that any $X/K$ as above is
isomorphic to a conic in $\mathbb{P}^2(K)$ defined over $K$. We
naturally ask, when such conics $C_1$ and $C_2$ defined over $K$ in
$\mathbb{P}^2(K)$ are $K$-isomorphic. A $K$-isomorphism of $C_1$ and
$C_2$ induces an isomorphism of the vector spaces $H^o (C_1,
\Omega'^*_{C_1})$ and $H^o (C_2, \Omega'^*_{C_2})$. But by what we
have said above, $\Omega^{*}_{C_i}$ is simply the restriction of the
canonical sheaf $\mathscr{O}(1)$ of $\mathbb{P}^2(K)$ to $C_i$. This
means that the isomorphism in question is actually induced by a
projective transformation over $K$ of $\mathbb{P}^2(K)$.\pageoriginale  Thus, the
problem of classifying all $K$-schemes $X$ such that $\bar{X} = X X_K
\bar{X} \backsimeq \mathbb{P}'(\bar{K})$ is equivalent ot the
problem of finding all the classes of non-degenerate quadratic forms
over $K$ for the equivalence upto a constant factor defined by the
full linear group over $K$. This problem has been solved, for example,
when $K$ is a $p$-adic field (i.e., a complete inequicharacteristic
discrete valuation ring with finite residue field) or when $K$ is an
algebraic number field. In either case (and more generally, when the
characteristic of $K$ is different from $2$), any non-degenerate
ternary quadratic form is equivalent to a form $Z^2- a X^2 - b Y^2, a,
b \in K^*$. In the $p$-adic case, the forms $Z^2- a X^2 - b Y^2$ and
$Z^2- a' X^2 - b' Y^2$ are equivalent if and only if the Hilbert
symbols $(a, b)_{\mathscr{G}}$ and $(a', b')_{\mathscr{G}}$ are equal
(\cite{key10}). In the case of an algebraic number field, the forms are
equivalent if and only if the forms have the same signature at all the real
infinite primes (i.e., are equivalent at the infinite prime spots)
and the forms are equivalent in all the $\mathscr{G}$-adic completions
of $K$ with respect to all the prime divisors $\mathscr{G}$ of $K$
(\cite{key10}).  

Finally, we mention that over any field $K$, such an $X$ is isomorphic
to $\mathbb{P}'(K)$  over $K$ if and only if $X$ contains a rational
point over $K$. The necessity is clear, since $\mathbb{P}'(K)$
contains at least three rational points $(0,1),(1,0)$ and $(1,1)$ over
$K$. Suppose then that $x \in X$ is a rational point, and $\vartheta$
the inversible sheaf of ideals of $\mathscr{O}_X$ which defines the
point $x$. The inversible sheaf $\vartheta^{-1} \otimes_K \bar{K}$ on
$\bar{X} \backsimeq \mathbb{P}'(\bar{K})$\pageoriginale is then of
degree one, and 
hence admits two independent regular sections. Since $H^o (\bar{X},
\vartheta^{-1} \otimes_K \bar{K}) \backsimeq H^o (X, \vartheta^{-1})
\otimes_K \bar{K}$, it follows that there is an element $f$ of
$R_K(X)$ which has a simple pole at the point $x$ of $\bar{X}$, and
hence defines a $K$-isomorphism of $X$ and $\mathbb{P}'(K)$. This
proves our statement. 

We proceed to discuss the analogous question for surfaces. We shall
attach to a surface $X$ over $K$ (which is defined and absolutely
simple over $K$) a group which is `manageable' as we shall show by an
example, and which is not necessarily the same for surfaces which
become isomorphic over the algebraic closure $\bar{K}$ of $K$. We need
some preliminary definitions. 

Let $X$ be scheme of finite type over a field $K$, and let $K_s$ and
$\bar{K}$ denote respectively the separable and algebraic closures of
$K$. Let $\mathscr{G}=\mathscr{G}(K_S/K)$ be the Galois group of $K_s$
over $K$ with the Krull topology. Then $\mathscr{G}$ also acts as the
groups of automorphisms of $\bar{K}$ over $K$. Let $\bar{X} = X X_K
\bar{K}$ and $\pi : \bar{X} \to X$ the first projection. If $x$ is any
point of $X$, $k(x)$ the field of residues at $x$ and $l_s(x)$ the
largest separable algebraic extension of $K$ in $k(x)$, it follows
from $\EGA$, I,(3.4.9), that the points of $\pi^{-1}(x)$ are
canonically in one-one correspondence with the $K$-monomorphisms of
$l_s(x)$ into $\bar{K}$(or $K_s$), and in particular, there are
precisely $[l_s(x) :K]$ points in the fibre $\pi^{-1}(x)$. Points of
the same fibre are said to be conjugate over $K$. Further, if $Y$  is a
an irreducible closed set of $X$  with generic point $y$, the
components of $\pi^{-1}(Y)$ are precisely the closures in $\bar{X}$ of
the points of the fibre $\pi^{-1}(y)$, all these components have the
same dimension and are mapped onto $Y$ by $\pi$.\pageoriginale These
components are 
again said to be conjugate over $K$, so that two irreducible closed
sets of $\bar{X}$ are conjugate over $K$ if and only if their generic
points are conjugate over $K$. 

If $\sigma$ is any element of $\mathscr{G}, \sigma $ induces a
$K$-isomorphism $\tilde{\sigma}: \Spec \bar{K}~\to \Spec \bar{K}$,
and hence also a $K$-isomorphism $\c{\sigma} = I_X \times
\tilde{\sigma} : \bar{X} \to \bar{X}$. We have clearly $(\sigma
\tau)^{\c{}} = \c{\tau} \c{\sigma}$, so that we may consider
$\mathscr{G}$ as acting (as an abstract group) on the right on
$\bar{X}$. From what we have said above, it follows that $\mathscr{G}$
acts transitively on the fibres $\pi^{-1}(x)$, and hence also on any
complete set of conjugate irreducible closed subsets. If $\bar{x}$ is
any point of $\pi^{-1}(x)$, to which there corresponds a
$K$-monomorphism $\theta : l_s(x) \to K_s$, the subgroup of
$\mathscr{G}$ which fixes $\bar{x}$ is precisely the subgroup which
fixes $\theta (l_s(x))$. 

Let us now assume further that $X$ is two dimensional and $K$-proper
such that $\bar{X} = X X_K \bar{K}$ is irreducible and regular (that
is, $R(K)$ is a regular extension of $K$, and $X$ is absolutely
regular, in particular regular). We can make $\mathscr{G}$ act on the
left on the group $\vartheta(\bar{X})$ of divisors of $\bar{X}$ by
defining for $\sigma \in \mathscr{G}$ and $D \in \vartheta (\bar{X})$,
$\sigma D =\c{\sigma}^* (D)$. If we provide $\vartheta(\bar{X})$
with the 
discrete topology, it follows from our earlier remarks that
$\mathscr{G}$ acts continuously on $\vartheta(\bar{X})$. Further, the
subgroups $\vartheta_l (\bar{X})$ and $\vartheta_a(\bar{X})$ are seen
to be stable for this action. Thus, $\mathscr{G}$ acts continuously on
the factors $\vartheta(\bar{X}) /  \vartheta_l (\bar{X})$ and
$S(\bar{X})=\vartheta(\bar{X})/\vartheta_a (\bar{X})$ (both with
discrete topology). In future, we shall denote $S(\bar{X})$ simply by
$S(X)$. If $Y$ is another $K$-scheme satisfying the above assumptions
and $f : X \to Y$ a $K$-morphism, the induced\pageoriginale
homomorphisms $\vartheta(\bar{Y}) \to \vartheta(\bar{X}), 
\dfrac{\vartheta(\bar{Y})}{\vartheta_l(\bar{Y})} \to
\dfrac{\vartheta(\bar{X})}{\vartheta_l(\bar{X})}$ and $S(\bar{Y})
\xrightarrow{f*} S(\bar{X})$ are $\mathscr{G}$-homomorphisms. 

Now, let $K$ be a perfect field, and $X$ a regular two dimensional
$K$-proper scheme such that $K$ is algebraically closed in $R(K)$. It
follows automatically that $X X_K \bar{K}$ is regular and irreducible
[$\EGA$~ III, 4.3.5]. It is not difficult to show that there is a
family $\{x_\alpha, \varphi_\alpha \}_{\alpha \in I}$ of couples
consisting of regular $K$-proper schemes $X_\alpha$ and $K$-morphisms
$\varphi_\alpha : X_\alpha \to X$ which are birational, indexed by a
set  $I$, such that if $Y$ is any regular $K$-proper scheme
and $\psi : Y \to X$ a birational $K$-morphism, there is a
\textit{unique} $\alpha \in I$ and an isomorphism $\psi : Y \to
X_\alpha$ such that $\varphi_\alpha o \psi_\alpha = \psi$ (one can in
fact take the $X_\alpha$ to be suitable collections of regular local rings of
$R(X)$ containing $K$, that is, schemes in the sense of
Chevalley). For $\alpha, \beta \in I$, let us define $\alpha \geq
\beta$ if there is a $K$-morphism $\varphi^{\alpha}_\beta : X_\alpha
\to X_\beta$ such that $\varphi_\beta o \varphi^{\alpha}_\beta =
\varphi_\alpha$. From what we have seen in lecture \ref{chap4}, $I$ is filtered
for this partial ordering. Each $\bar{X}_\alpha = X_\alpha X_K
\bar{X}$ is irreducible and $K$-regular (and it is for this that we
assumed $K$ perfect). Since for $\alpha \geq \beta \geq \gamma $ we
have clearly $\varphi^{\beta}_\gamma o \varphi^{\alpha}_\beta =
\varphi^{\alpha}_\gamma$, if we put 
$$
\bar{\varphi}^{\alpha}_\beta = \varphi^{\alpha}_\beta X_K \bar{K} :
\bar{X}_\alpha \to \bar{X}_\beta, \text { and } \psi^{\beta}_\alpha =
(\bar{\varphi}^{\alpha}_\beta)^* : S(X_\beta) \to S(X_\alpha)), 
$$
$(S(X_\alpha), \psi^{\beta}_\alpha)$ form an inductive system of
discrete $\mathscr{G}$ modules. We define 
$$
\gamma(X) = \varinjlim_\alpha S(X_\alpha),
$$
so that\pageoriginale $\gamma (X)$ is a discrete $\mathscr{G}$-module
associated to 
$X$. This group can be interpreted as the Severi-group of the infinite
dimensional object that was mentioned in lecture \ref{chap1} (projective limit
of all models of the filed $R(X)$) It is clear that $\gamma (X)$ is
`independent' of the choice of the family  $\{X_\alpha ,
\varphi_{\alpha}\}_{\alpha \in I}$. Further, $\gamma (X)$ is a
contravariant functor on the category of proper, $K$-regular
absolutely irreducible schemes over $K$. We shall show that $\gamma
(X)$ is a $K$-birational invariant, in the sense that if $Y$ is again
regular and $K$-proper and $\psi : Y \to X$ a $K$-birational map,
$\psi$ induces a canonical $\mathscr{G}$-isomorphism of $\gamma (X)$
and $\gamma (Y)$. Indeed, by the theorem of domination, we may assume
that $\psi$ is actually a morphism, so that we may take $Y=X_{\alpha},
\psi = \varphi_{\alpha}$ for some $\alpha \in I$. Since the set
$I_{\alpha}= \{\beta \in I | \beta \geq \alpha\}$ is cofinal in $I$,
we have canonical isomorphisms $\gamma (Y) \backsimeq
\varinjlim_{\beta \in I_{\alpha}} S(X_\beta) \backsimeq
\varinjlim_{\beta \in I} S(X_{\beta}) \backsimeq \gamma (X)$. 

Now, $\gamma (X)$ itself is `much too big' to be of any use by
itself. However, it was pointed out by Manin that the cohomology group
$H'\break(\mathscr{G,\gamma (X)})$ (in the sense of cohomology of profinite
groups); (see \cite{key21}) is of manageable proportions. In fact, we shall
show that if $\psi : S(X) \to \gamma (X)$ is the canonical
homomorphism, $\psi$ induces isomorphisms 
$$
H'(\mathscr{G}, S(X)) \to H'(\mathscr{G} , \gamma (X)).
$$

By an elementary lemma on the cohomology of profinite groups we have
that $\varinjlim_{\alpha} H' (\mathscr{G}, S(X_{\alpha})) \backsimeq
H' (\mathscr{G}, \gamma (X))$. 

Thus,\pageoriginale it sufficient to show that $\varphi_{\alpha} :
X_{\alpha} \to X$ 
induces an isomorphism $H' (\mathscr{G}, S(X)) \tilde{\to} H'
(\mathscr{G}, S(X_\alpha))$. Moreover, by the theorem of decomposition of
birational morphisms, $\varphi_{\alpha}$ is a composite of
dilatations, so that it is sufficient to show that if $\sigma : X' \to
X$ is the dilatation of $X$ at a closed point $x$ of $X, \sigma$
induces an isomorphism $H' (\mathscr{G}, S(X)) \tilde{\to} H'
(\mathscr{G}, S(X'))$. Let $\bar{x}^{(1)}, \ldots \bar{x}^{(t)}$ be
the complete set of conjugate points of $\bar{X}$ lying over $x$, so
that $t = [k(x) : K]_s = [k (x):K]$. Since $\bar{X}$ is regular,
$\bar{\sigma}: \bar{X}' \to \bar{X}$ induces an isomorphism of
$\bar{X}'- \bigcup\limits_{i} \bar{\sigma}^{-1} (\bar{x}^{(i)})$ onto
$\bar{X}- \{\bar{x}^{(-1)}, \ldots, \bar{x}^{(t)}\}$ and
$\bar{\sigma}^{-1}(\bar{x}^{(i)})\backsimeq \sigma ^{-1} (x) X_{k(x)}
\bar{K} \backsimeq \mathbb{P}' (\bar{K})$, we see that $\bar{X}'$ is
the dilatation of $\bar{X}$ at the points $\bar{x}^{(1)}, \ldots ,
\bar{x}^{(t)}$. Let $l$ be the image  of the monomorphism of residue
fields $k(x) \to k(x^{(1))}= \bar{K}$, and $\mathscr{G}$ the subgroup
of $\mathscr{J}$ which fixes $l$. By our remarks of the previous
paragraph, we have a bijection $\mathscr{J}/ \mathscr{G}\to
\{\bar{x}^{(1)} , \ldots, \bar{x}^{(t)}\}$, given by $g \mathscr{G} \to g
\bar{x}^{(1)}$, and this is compatible with the action of $\mathscr{J}$
on both the sets. Since $\bar{\sigma}$ again commutes with the action of
$\mathscr{J}$ on $\bar{X}'$ and $\bar{X}$, we see that the set of
divisors $\{\bar{\sigma}^{-1} (\bar{x}^{(1)})\}$ of $\bar{X}'$  is again
stable for $\mathscr{J}$, and identifies itself as a
$\mathscr{J}$-set with $\mathscr{J}/ \mathscr{G}$. Now, we have proved
earlier that $\bar{\sigma}^*$ is an injection of $S(\bar{X})$ into
$S(\bar{X'})$, and $S(\bar{X'})$ decomposes as a direct sum 
$S(\bar{X}') \simeq \bar{\sigma}^* (S(\bar{X})) \oplus
\sum\limits_{i=1}^{t} \mathbb{Z} \bar{\sigma}^{-1} (\bar{x}^{(i)})$ and
it follows that this is a decomposition of $\mathscr{J}$-modules. As
a $\mathscr{J}$-module, $\sum\limits_{i=1}^{t} \mathbb{Z}
\bar{\sigma}^-1 (\bar{x}^{(i)})=T$ is isomorphic\pageoriginale to the
$\mathbb{Z}$-free module $\mathbb{F}_{\mathbb{Z}}(\mathscr{J}/
\mathscr{G})$ over 
$\mathscr{J}/ \mathscr{G}$ considered as a left $\mathscr{J}$-module
in the natural way. In order words this module is `induced' from the
trivial $\mathscr{G}$- module $\mathbb{Z}$. Hence, we have the
isomorphisms $H^p (\mathscr{J}, T)\tilde{\to} H^p (\mathscr{G},
\mathbb{Z})$ (\cite{key21}), and since $H'(\mathscr{G},\mathbb{Z})= \varinjlim
H' (H_\alpha , \mathbb{Z})$ where $H_\alpha$ are  finite quotients of
the profinite group $\mathscr{G}$, and $H' (H_\alpha, \mathbb{Z})= \Hom
(H_\alpha , \mathbb{Z})=0$, it follows that $H' (\mathscr{J},
T)=0$. This proves our assertion that $H' (\mathscr{J},
S(X))\backsimeq H' (\mathscr{J}, \gamma (X))$. Now, as we have stated
already, $S(X)$ is a finitely generated abelian group. Suppose
further that $S(X)$ is torsion free. There is a normal subgroup
$\mathscr{J}_\circ$ which is open and of finite index in $\mathscr{J}$
which fixes the generators of $S(X)$, and hence the whole of $S(X)$
$(\mathscr{J}_\circ$ can be taken as the subgroup which fixes a Galois
extension $L$ of $K$ such that all the generators of $S(X)$ are
defined over $L'$). Since $S(X)$ has no torsion by assumption $H'
(\mathscr{J}_\circ, S(X))=0$, and it follows that 
$$
H' (\mathscr{J}, \gamma (X))\tilde{\leftarrow} H'
(\mathscr{J}, S(X))\tilde{\leftarrow} H' (\mathscr{J}/
\mathscr{J}_\circ,S(X)). 
$$

In particular, $H' (\mathscr{J}, \gamma (X))$ is a finite group which
is `Computable'. 

We shall illustrate this by an example.

Let $K$ be a perfect filed of characteristic $\neq 3$, and containing
a primitive cube root $\varrho$ of unity. Let $X$ be the (absolutely
irreducible and absolutely simple) projective surface $X$ defined by
the equation 
\begin{equation*}
  x^3_1 + x^3_2 + x^3_3 = a z^3_0, \tag{$*$}
\end{equation*}
where\pageoriginale $a \neq 0$. We shall compute $H' (\mathscr{G},
\gamma (X))= H' (\mathscr{G}, S(X))$. 

Consider the set of six points $(1,0,0)$, $(0,1,0)$, $(0, 0,1)$,
$(1,1,1)$, $(1,\varrho, \varrho^2),(1, \varrho^2, \varrho)$ of
$\mathbb{P}^2$. Keeping in mind the fact that this set of 
points is invariant under permutations of co-ordinates, one verifies
trivially that no three of these points lie on a line and all six do
not lie on a conic. The cubic forms $U_1 = X_O(X_O X_1 - X^2_2), U_2 =
X_O (X_O X_2 - X^2_1), U_3 = X_1 (X_1 X_2 - X^2_O), U_4 = X_1 (X_1 X_O
- X^2_2)$ are linearly independent and vanish at all these points, and
hence form a basis for all the cubic forms vanishing at these
points. Further, these satisfy the equation 
$$
U_1 U_3 (U_1+ U_3)= U_2 U_4 (U_2 + U_4),
$$
so that $\mathbb{P}^2$ blown up at the above six points is isomorphic
to the cubic surface in $\mathbb{P}^3$ (with $U_1, U_2 , U_3$ and
$U_4$ as homogeneous co-ordinates) defined by the above equation. If
we put $U_1 =x_1 + x_2 ,U_3 = \varrho (x_1 + \varrho x_2), U_2 =
\sqrt[3]{a} \,  x_o - x_3, U_4 = \varrho (\sqrt[3]{a}  \, x_o- \varrho x_3)$, the
above equation gets transformed into $(*)$. Thus, over
$K(\sqrt[3]{a}), X$ becomes birationally isomorphic to $\mathbb{P}^2$. In
particular, if a belongs to $K^{*3}$, that is, if a is a cube in $K$,
$X$ is birationally equivalent to $\mathbb{P}^2$ over $K$.  

Now, choose a line $L$ in $\mathbb{P}^2(\bar{K})$ defined over
$K$. Any divisor of $\mathbb{P}^2 (\bar{K})$ is linearly equivalent to
a multiple of $L$. On the other hand, since the self intersection
number of $L$ is $+1$, no multiple of $L$ is algebraically equivalent
to $0$. 

Hence,\pageoriginale $S(\mathbb{P}^2 (\bar{K})) \backsimeq \mathbb{Z}$, and the
action of $\mathscr{G}$ on $S(\mathbb{P}^2(\bar{K}))$ is trivial. It
follows that \textit{if} $a \in K^{*3}$, 
$$
H' (\mathscr{G}, \gamma (X))= H' (\mathscr{G}, \gamma (\mathbb{P}^2))=
H' (\mathscr{G}, \mathbb{Z})=0. 
$$

Next, suppose that $a \not\in K^{*3}$. The group $S(\bar{X})$ is a
free group on seven generators. Six of these generators are the fibers
of the blown up points in $\mathbb{P}^2$, and the seventh is the
inverse image of a line in $\mathbb{P}^2$. But since this line may be
chosen to pass through two of the blown up points, we see that the
classes of the lines lying on $\bar{X}$ generate $S(\bar{X})$. 

The 27 lines on $\bar{X}$ are easily written down by inspection. They
are given by the equations 
\begin{equation*}
  \left.
  \begin{aligned}
    &x_i + \varrho^m x_j =0\\
    &x_k - \varrho^n \sqrt[3]{a} x_\circ =0
  \end{aligned}\right\}
(i, j, k) ~\text{a permutation of}~ (1, 2, 3), 0 \leq m, n < 3,
\end{equation*}
where $\sqrt[3]{a}$ denotes any fixed cube root of $a$. All these lines
are defined over the field $K ( \sqrt[3]{a})$ which is Galois over $K$,
and which has for Galois group $G$ a cyclic group of order $3$
generated by an element $g$ with $g(\sqrt[3]{a})= \varrho
\sqrt[3]{a}$. It follows that if $l$ denotes the line defined by the
equations 

$x_i + \varrho^m x_j =0,x_k - \varrho^n \sqrt[3]{a} \, x_o =0 $, then $l +
gl + g^2l$ is precisely the divisor cut out on $\bar{X}$ by the
hyperplane $x_i + \varrho^m x_j =0$. 

We shall first establish that the group $S(\bar{X})^G$ of $G$-invariant
elements of $S(\bar{X})$ is isomorphic to $\mathbb{Z}$. Since
$S(\bar{X})$ is torsion free, we have\pageoriginale only to show that
$S(\bar{X})^G$ is of rank 1. Let us put $S_Q (\bar{X}) = S(\bar{X})
\oplus_{\mathbb{Z}} \mathbb{Q}$, where $\mathbb{Q}$ denotes the
rational filed, so that we have $S^G(\bar{X}) \oplus_\mathbb{Z}
\mathbb{Q}\backsimeq (S_\mathbb{Q}(\bar{X}))^G$. If we put $N = 1
+g+g^2\in \mathbb{Z}[G]$, we have clearly $S_{\mathbb{Q}}(\bar{X})^G =
N. S_{\mathbb{Q}}(\bar{X})$, since for $\alpha \in S_{\mathbb{Q}}
(\bar{X})^G$, we have $\alpha =\dfrac{1}{3} N \alpha$. Now,
$S_{\mathbb{Q}}(\bar{X})$ is generated over $\mathbb{Q}$ by the
canonical images of the above lines. If $l$ is a line, we have shown
above that $Nl =l +gl + g^2 l$ is a hyperplane section. Finally, the
difference of two hyperplane of two hyperplane section is linearly
equivalent to $0$ on $\bar{X}$ (since any two hyperplanes are
linearly equivalent in $\mathbb{P}^3$). In consequence of these
remarks, it follows that the rank of $S(\bar{X})^G$= dimension over
$\mathbb{Q}$ of $S_{\mathbb{Q}}(\bar{X})^G \leq 1$. Now, $G$ admits
precisely two irreducible representations over the rational filed,
viz., the trivial one-dimensional representation and the two
dimensional representation in the cyclotomic filed
$\mathbb{Q}(\varrho)$. Since $S_{\mathbb{Q}}(\bar{X})$ is seven
dimensional over $\mathbb{Q}$, it follows that we must necessarily
have $S_{\mathbb{Q}}(\bar{X})\backsimeq \mathbb{Q} \oplus \mathbb{Q}
(\varrho)^3$ as $\mathbb{Q}[G]$-modules, because of the complete
reducibility of $\mathbb{Q}$-representations of $G$. In particular, we
must have $\dim_\mathbb{Q} S_\mathbb{Q}(\bar{X})^G = 1$, $S(\bar{X})^G
= \mathbb{Z}$. 

Now, for any finitely generated $G$-module $M$, the cohomology\break groups
$H^p (G,M) (p \geq 1)$ are finite dimensional vector spaces over
$\mathbb{Z}_3 = \mathbb{Z}/_{3 \mathbb{Z}}$. Let us put $\chi (M) =
\dim _{\mathbb{Z}_3}H^2(G,M) - \dim_{\mathbb{Z}_3} H' (G,M)$. (The
rational number $3^{\chi (M)}$ is 
usually called the Herband quotient of $M$).\pageoriginale It is $\lambda$ then
well-known (\cite{key21}) that (i) if $0 \to M' \to M \to M'' \to 0$ is an
exact sequence of finitely generated $G$-modules, $\chi(M) = \chi (M')
+ \chi (M'')$, and (ii) $\chi (M) =0$ if $M$ is finite. It easily
follows from (i) and (ii), that if $M_1$ and $M_2$ are two
finitely generated $G$-modules such that $M_1 \oplus_\mathbb{Z}
\mathbb{Q}$ and $M_2 \oplus_\mathbb{Z} \mathbb{Q}$ are isomorphic
$\mathbb{Q}[G]$- modules, $\chi (M_1)= \chi (M_2)$. 

Applying this remark to $S(\bar{X})$, we deduce that
$$
\chi (S(\bar{X}))= \chi (\mathbb{Z} \oplus \mathbb{Z}[\varrho]^3)=
\chi (\mathbb{Z})+3 \chi (\mathbb{Z}[\varrho]). 
$$

Now, the exact sequence of $G$-modules $0 \to \mathbb{Z} . N \to
\mathbb{Z} [G] \to \mathbb{Z} [\varrho] \to 0$ yields that $\chi
(\mathbb{Z}[\varrho]) + \chi (\mathbb{Z}) = \chi (\mathbb{Z}
       [G])=0$. Further, since $H'(G,\mathbb{Z}) =0$ and $H^2 (G
       \mathbb{Z}) \backsimeq \dfrac{\mathbb{Z}^G}{N \mathbb{Z}} 
\simeq \dfrac{\mathbb{Z}^G}{N \mathbb{Z}} \simeq \mathbb{Z}_3$, $\chi
(\mathbb{Z})=1$. It follows that
\begin{gather*}
  \chi (S(\bar{X}))= \dim_{\mathbb{Z}_3} H^2(G, S(\bar{X}))-
  \dim_{\mathbb{X}_3} H'(G,S(\bar{X}))=-2,\\ 
  H' (\mathscr{G}, \gamma (X)) = H' (G,S(\bar{X}))\neq 0.
\end{gather*}

In view of our earlier remarks, it follows that if $a \not\in K^{*3}$
the $K$-surface $X$ is not birationally equivalent to $\mathbb{P}^2
(K)$ \textit{over} $K$. It follows that if $a, b \in K^*$ with $K (
\sqrt[3]{a}) \neq K( \sqrt[3]{b})$, and if $X_a$ an $X_b$ denoted the
corresponding surfaces in $\mathbb{P}^3$, $X_a$ and $X_b$ are not
birationally equivalent even over $K(\sqrt[3]{a})$, and a fortiori not
equivalent over $K$. Now, by Kummer theory, $K(\sqrt[3]{a}) =
K(\sqrt[3]{b})$ if and only if $a$ and $b$ generate the same subgroup in
$K^* / K^{*3}$, that is, if and only if either $a= bc^3$ or $a= c^3/b$
for same $c \in K^*$. In the first case, $X_a$ and $X_b$ are clearly
even projectively isomorphic\pageoriginale over $K$. However, we are unable to
decide whether $X_a$ and $X_b$ are birationally equivalent over $K$ or
not when $a= c^3/b$ with $c \in K^*$. 

Let us mention here another problem of geometric interest, where the
difficulties are similar. A variety $X$ over a filed $K$ is said to be
\textit{rational} if it is birationally equivalent to a projective space
over $K$, and \textit{unirational} if there is a dominant $k$-rational
map of a projective space onto $X$. Thus, $X$ is rational if and only
if its function field $R(X)$ is a purely transcendental extension of
$K$, and $X$ is unirational if and only if $R(X)$ is $K$-isomorphic to
a subfield of a pure transcendental extension of $K$. It has been
proved by Luroth that any unirational curve is rational. Castelnuovo
(\cite{key3}) (in the classical case $K = \mathbb{C}$) and Zariski
(\cite{key24}, \cite{key25}) have proved that when $K$ is algebraically closed and
$L$ an extension of $K$ contained in pure transcendental extension
$K(X,Y)$ of $K$ such that $K(X,Y)$ is separable over $L$, then $L$ is
purely transcendental over $K$. It has been generally believed that the
analogous statement  in dimension three is false. Many examples of
unirational three dimensional varieties which are thought not to be
rational have been suggested, the most notable being the general cubic
hypersurface in $\mathbb{P}^4$. But in no case has it been established
that the suggested variety is not rational. The difficulty in this
case is again the lack of suitable invariants\pageoriginale which distinguish
between varieties which are rational and those which are merely
unirational, the dimensions of sheaf cohomologies being insufficient
for this purpose. Incidentally, there is a closer connection between
the problem we discussed earlier and the present problem than is
apparent. Indeed, the function filed $R$ of a cubic hypersurface in
$\mathbb{P}^4$ over a field $K$ is of the form $R =K(X,Y,Z,T)$ with
$X,Y,Z$ and $T$ connected by a single cubic polynomial relation $F_3
(X,Y,Z,T)=0$. Now $R$ may also be considered as the field of
functions of a cubic surface over $K(T)$. But now, since a general
cubic in $\mathbb{P}^3$ over an algebraically closed filed is
rational, as we have stated earlier, if $L$ is the algebraic closure of
$K(T)$, the composite field $L$ $R$ is a field of rational functions
over $L$ in two variables. In other words, the field extension
$R/K(T)$, becomes a rational function filed on extension of base from
$K(T)$ to $L$. But the nature of the extension $R/K(T)$ itself is
unknown. Of course the real difficulty lies in the fact that the
subfield $K(T)$ is not uniquely determined in $K(X,Y,Z,T)$. 
