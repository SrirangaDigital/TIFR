
\chapter{The existence of relatively minimal models}\label{chap7}%\chap 7

\markright{\thechapter. The existence of relatively minimal models}

We\pageoriginale need the following

\begin{theorem*}%\tho 
  Let $Y$ be locally noetherian and $f : X \rightarrow Y$ a morphism
  of finite type; for any $x \in X$, put $\delta (x)= \dim_x f^{-1}
  f(x)$. Then for any integer $n \ge 0$, the set $\{x \in X / \delta
  (x) \ge n\} = E_n$ is closed in $X$. If further $X$ is irreducible,
  $x$ is a generic point of $X$ and $e=\dim f^{-1}(f(x))=
  tr. deg_{k(f(x))}{^{k(x)}}$, then  $E_e = X$ and $f(E_{e+1})^c$
  contains a non-void open set of $Y$.  
\end{theorem*}

For the proof, see ($\EGA$, IV, 13.1.1).

We note however that under the assumptions of the theorem, it is
\textit{ not } true that if $X, Y$ are irreducible and $f$ dominant,
$\dim X= \dim Y+e$. This holds however when $X,Y$ are irreducible
algebraic schemes over a field $K$ and $f$ a $K$-morphism. 

The following lemma is well known. but we add a proof since we could
not find a ready reference. (For varieties over an algebraically
closed filed, see \cite{key12}). 

\begin{lemma*}%lemma
  Let $X,Y$ be irreducible noetherian schemes, and $f:X\rightarrow Y$
  a morphism of finite type such that if $y$ is a generic point of
  $Y$, the generic fibre $f^{-1}(y)$ is geometrically
  irreducible. Then there is a non-void open subset $U$ of $Y$ such
  that for any $z \in U$, $f^{-1} (z)$ is geometrically irreducible.  
\end{lemma*}

\begin{proof}
  One may clearly assume $X,Y$ integral, $f$ dominant and $Y=\Spec A$,
  where $A$ is a noetherian domain. We first remark if $U$ is a\pageoriginale
  non-void open subset of $X$ and $g$ the restriction of $f$ to $U$,
  the lemma holds for $(X, Y,f)$ if and only if it holds for $(U, Y,
  g)$. Indeed, if we put $F = X -U$ and $F_1 ,\ldots,F_r$  are the
  irreducible components of $F$, since each $F_i$ intersects  the
  generic fibre of $f$ in a proper closed subset, it follows from the
  above theorem that there is a  non-void open subset $V$ of $Y$ such
  that for any $y \in V$, we have  
  \begin{align*}
    &\dim_x (f^{-1}(y))=e, \text{ for every } x \in f^{-1}(y)\\
    &\dim ~ (f^{-1}(y) \cap F_i) < e
  \end{align*}
  It follows therefore that for any $ y \in V, f^{-1}(y)$ is
  geometrically irreducible if and only if $g^{-1} (y) = f^{-1} (y) \cap
  U$ is geometrically irreducible. This proves our assertion.
\end{proof}

Let $T_1,\ldots, T_n$ be a transcendence basis of $R(X)$ over
$R(Y)$, and $T_{n+1}$ an element of $R(X)$ satisfying an irreducible
separable monic equation over $R(Y)$ $(T_1,\ldots,T_n)$  
$$
T^r_{n+1} + f_1 T^{r-1}_{n+1}+\cdots + f_r =0, f_i \in A [T_1,\ldots,T_n ],
$$
such that $R(X)$ is purely inseparable over
$R(y)(T_1,\ldots,T_{n+1})$. It follows easily from ($\EGA$, I (6.5.1))
that there is an open subset $U$ of $X$, an open subset $U'$ of $\Spec
A [T_1,\ldots T_{n+1},]$ and a $Y$-morphism of $U$ onto $U'$ which is
radical and finite. The fibres in $U$ and $U'$ of any
point of $Y$ are therefore both geometrically irreducible or both
geometrically reducible. In view of our earlier remarks, it is
sufficient\pageoriginale to prove the theorem in the case when $Y=
\Spec A$, where 
$A$ is a noetherian domain and $X= \Spec C$, where $C$ is the
$A$-algebra  
$$
C=\frac{A[T_1,\ldots,T_{n+1}]}{\varphi A[T_1,\ldots,T_{n+1}]},
$$
where $\varphi = \varphi (T_1,\ldots,T_{n+1})$ is a polynomial in
$T_1,\ldots,T_{n+1}$ which is irreducible over the quotient field
$R(Y)$ of $A$, and  considered as a polynomial in $T_{n+1}$ over $R(Y)
(T_1,\ldots,T_n)$, is separable. The assumption that the generic fibre
of $f$ is geometrically irreducible simply means that $\varphi
(T_1,\ldots,T_{n+1})$ remains irreducible even over the algebraic
closure of $R(Y)$. If for, any y $\in \Spec A$ we denote by  
$\varphi_y(T_1,\ldots,T_{n+1})$  the image of $\varphi$ by the
canonical homomorphism. 

$A [ T_1 ,\ldots T_{n+1},] \rightarrow k(y) [ T_1 ,\ldots T_{n+1},]$
we have to show that the set of $y \in Y$ such that $\varphi_y$ is
irreducible over the algebraic closure $\overline{k(y)}$ of $k(y)$
contains an open subset of $Y$. Let $d$ be the degree of $\varphi$,
and  for any integer $p >0$, Let $N(p)$ denote the number of
monomials of degree $\le p$ in $(n+1)$ variables. For $p, q \ge 0$ with
$p+q=d$, we have a morphism  
$$
\mathbb{A}^{N(p)} (A) \times_A \mathbb{A}^{N(q)}_{A}
\xrightarrow{M_{p,q}}\mathbb{A}^{N(d)}   \, \,(A) 
$$
which corresponds to multiplication of polynomials of degrees $p$ and
$q$. (Here $,\mathbb{A}^N(A)$) denotes the `affine space' Spec $A [
  X_1,\ldots X_n, ]$ over $A$. Now, for $y \in Y, \varphi_y$ over the
algebraic closure  $\overline {k(y)} $ of  $k(y)$ into factors of
degrees $p$ and $q$ if and only if the point of 
$\mathbb{A}^{N(d)}(k(y)) = \mathbb{A}^{N(d)} (A) \times_A
k(y)$\pageoriginale 
corresponding to $\varphi_y$ is in the image of the morphism 
$$
\mathbb{A}^{N(p)}(k(y))
\times_{k(y)} \mathbb{A}^{N(q)} (k(y))
\xrightarrow{M_{p,q} \times_A k(y)} \mathbb{A}^{N(d)} (k(y))
$$

Now, $\varphi$ itself defines a morphism $Y=\Spec A
\xrightarrow{\Phi}\mathbb{A}^{N(d)}(A)$. Let $Z_{p,q}$ be the
image of $M_{p,q}$, and put $Z=\bigcup\limits_{p+q=d}Z_{p,q}$. Then
$Z$ is a constructable subset of $\mathbb{A}^{N(d)} (A)$, by a theorem
of Chevalley (EGA, IV, (18.5)). Hence, $\Phi^{-1}(Z)$ is also
constructable. Since the generic point of $Y$ does not belong to
$\Phi^{-1}(Z)$, there is an open subset of $Y$ disjoint with
$\Phi^{-1}(Z)$. This implies the Lemma.

We now come to the main theorem.

\subsubsection*{Theorem of existence of relatively minimal models}

Let $B$  be a noetherian prescheme, and $X_i (1 \le i < \infty)$
$B$-schemes of finite type which are equi-two dimensional and
regular. Suppose given $B$-morphisms $\varphi_i : X_i \rightarrow
X_{i+1} (1 \le i < \infty)$ which are proper and birational. Then
there is an integer $n$ such that for $i \ge n, \varphi_i$  is an
isomorphism. 

\begin{proof}
  Since each $X_i$ has only a finite number of components (these are
  disjoint and open in $X_i$), and each component of $X_{i+1}$ is the
  image\pageoriginale of one and only one component of $X_i$, we may
  assume without 
  loss of generality that $X_i$ are irreducible. Hence, we may also
  assume that $B$ is irreducible and reduced. Since $B$ is covered by
  a finite number of affine open set, we 
  may further restrict ourselves to the case when $B=\Spec A$, where
  $A$ is a noetherian domain. Since one may also assume that all the
  morphisms $X_i \rightarrow B$ are dominant, we have injections $A
  \rightarrow R (X_i)$ such that the diagrams  
\end{proof}

\[
\xymatrix{& A \ar[dl] \ar[dr] & \\
R(X_{i+1}) \ar[rr]^\sim &  & R(X_i)
}
\]
are commutative. Let $A'$ be a finite type $A$-algebra contained in
$R(X_1)$ such that the quotient field of $A'$ is algebraically closed
in $R(X_1)$. Since the $X_i$ are regular and hence normal, $\Gamma
(X_i, \mathscr{O}_{X_i})$ are integrally closed rings, so that  the
inverse image of $A'$ by the composite isomorphism $R(X_i) \rightarrow
R(X_{i-1}) \cdots \rightarrow R(X_1)$ is actually contained in $\Gamma
(X_i, \mathscr{O}_x{_{_i}})$. Hence, we may consider the $X_i$ as
schemes over Spec $A'$ and $\varphi_i$ are morphisms over Spec
$A'$. To sum up, we may assume that $B$-Spec $A$, such that the image
of the quotient field of $A$ in $R(X_i)$ is algebraically closed in
$R(X_i)$ for every $i$. Finally, by the theorem of decomposition of
lecture \ref{chap4}, we may assume that $\varphi_i$ is either an isomorphism
or $(X_i, \varphi_i)$ is $X_{i+1}-$ isomorphic to a dilatation of
$X_{i+1}$ at a closed point of $X_{i+1}$. 

 Let $r$\pageoriginale be the dimension of the generic fibre of $X_1$ over
 $B$. Since the generic fibre is homeomorphic to a subspace of $X_1$,
 we see that $r \le 2$. 

\medskip
\noindent\textbf{Case (i) {\boldmath$r \le 1$}.}

Let $\psi_i : X_i \rightarrow B$ denote the structural morphism. By
the theorem started at the beginning of this lecture and the next
lemma, we can find a non-void open subset $U$ of $B$ such that for
any $b \in U,\psi_1^{-1} (b)$ is geometrically irreducible and of
dimension $r$ (Note that the generic fibre of $\psi_1$ is
geometrically irreducible since the quotient field of $A$ is
algebraically closed in $R(X_1)$ by assumption). Since
$\psi_i^{-1}(b) \xrightarrow{\varphi_i X_B k(b)} \psi _{i+1}^{-1} (b)$
is surjective, ($\EGA$, I, (3.5.2)) it follows from theorem above that
for every  $b \in U$, $\psi^{-1}_{i}(b)$ is irreducible and of dimension
$r$. We assert that the restriction of $\varphi_i$ to $\psi_i^{-1}(U)$
is an isomorphism of $\psi^{-1}_i(U) $ onto  $\psi^{-1}_{i+1}
(U)$. For otherwise, since $(X_i,\varphi_i)$ is the dilatation of
$X_{i+1}$ at a point $x$, we must have 
$$
\displaylines{\hfill  
  \varphi_i^{-1} (x) \simeq \mathbb{P'} (k(x)),\hfill \cr  
  \text{and}\hfill \varphi^{-1}_i (x) \cap \psi^{-1}_i (U) \neq
  \Phi,\hfill \cr 
  \text{so that} \hfill \varphi_i^{-1} (x) = \psi ^{-1}_i (\psi
  _{i+1}(x))\hfill \cr 
  \text{since}\hfill  r\leq 1  ~\text{and} \quad  \psi_i^{-1}
  (\psi_{i+1}(x)) \hfill }
$$ 
is irreducible of dimension $r$. It follows that $r=1$ necessarily,
and $\psi^{-1}_{i+1}$ $(\psi_{i+1} (x)) =
\varphi_i(\psi^{-1}_{i}(\psi_{i+1}(x))) = x$. This is impossible since
we must have dim $\psi_{i+1}^{-1}(\psi_{i+1}(x)) =1$. Thus,
$\varphi_i$ induces an isomorphism of $\psi_i^{-1}(U)$ onto
$\psi_{i+1}^{-1}(U)$. Now, $F_i = \psi^{-1}_i (B-U)$ is a proper
closed subset of $X_i$, and hence has only a finite number $s_i$ of
components (of dimension = 1). Since $\varphi_i (F_i) = F_{i+1}$, it
follows\pageoriginale that $ s_i \ge s_{i+1}$. If further $\varphi_i$ is not an
isomorphism, $\varphi_i$ must contract an irreducible component of
$F_i$ to a point of $F_i$ to a point of $F_{i+1}$, so that $s_i > s_{i
  + 1}$. Since $s_i \ge 0 $, the sequence $s_1, s_2,\ldots$ must
become stationary, which proves that $\varphi_i$, must be an
isomorphism for $i$ large. 

\medskip
\noindent\textbf{Case (ii)  {\boldmath$r=2$}.}
 
In this case, we assert that $\psi_1 (X_1)$ is a single (generic)
point of $B$. If not, we can fine a $b \in B$ distinct from the
generic point $B$ such that $\psi_1^{-1} (b) \neq \Phi $ and  $\neq
X_1$ and $\dim \psi_1^{-1}(b) = 2$. Let $F_2 \supset F_1\supset F_0$
be a sequence of distinct irreducible closed sets of
$\psi_1^{-1}(b)$. Then $\overline{F_2} \supset \overline{F_1} \supset
\overline{F_0}$, and $\bar{F}_{i+1} \neq \overline{F}_i$. Since
$\overline{F}_{i+1} \cap \psi^{-1}_1 (b) = F_{i+1} \neq F_i =
\overline{F}_i \cap \psi_1^{-1}(b) $, and  $\overline{F}_2 \neq X$ 
since $\overline{F}_2$ is contained in $\psi^{-1}_1 (\overline{b})$,
and $\overline\{b\} \neq B$ since $b$ is not generic. Thus we have a
strict chain $ X \supset \overline{F}_2 \supset \overline{F}_1 \supset
\overline{F}_0 $ of irreducible closed sets, which contradicts the
assumption that $\dim X=2$. 
 
 Hence we must have $\psi_1(X_1) = b \in B $, and consequently
 $\psi_i(X_i) = b$. Since the $X_i$ are reduced schemes, it follows
 that each $\psi_i$ can be factored as $X_i \rightarrow \Spec k(b)
 \rightarrow B$, so that the $X_i$  can be considered as algebraic
 schemes over $k(b)$. Further, since Spec $k(b) \rightarrow B$ is a
 monomorphism in the category of preschemes, the $\varphi_i$ are
 actually $k(b)$- morphisms. Thus, we are reduced to the case when
 $B=\Spec K$, $K$ being a field. By Corollary $2$ to the theorem of
 Castelnuovo, we can find an irreducible reduced projective $k(b)$
 -scheme which we denote by $\overline{X}_1$, such that $X_1$ is
 isomorphic to an open subscheme\pageoriginale of $\overline{X}_1$
 (since the proof 
 of this Corollary uses the reduction of singularities of a surface,
 we remark that it is sufficient to have any $\overline{X}$ which is
 $k(b)$ proper and contains an isomorphic copy of $X_1$; and the
 existence of such an $\overline{X}_1$ is ensured by the theorem of
 \textit{ Nagata } on the embedding of abstract varieties in a
 complete variety \cite{key19}). Now, there is a finite set $F_i$ of closed
 points of $X_i$ such that if we put $\chi_i = \varphi_{i-1} \circ
 \varphi_{i-2} \circ \cdots \circ \varphi_1 : X_1 \to X_i$, $\chi_i$ is an
 isomorphism of $X_1 - \chi_i^{-1}(F_i) $ onto  $X_i - F_i$. Further,
 $\chi_i^{-1} (F_i)$ is closed in  $\overline{X}_1$ since
 $\chi_i^{-1}(F_i)$ is $k$-proper. Let $\overline{X}_i$  be the
 variety obtained by `patching up' the varieties $X_i$ and
 $\overline{X}_1- \chi_i^{-1}(F_i)$ by means of the isomorphism
 $X_i- F_i \xrightarrow{\chi_i^{-1}} X_1 - \chi_i^{-1} 
 (F_i)$. Then each $\overline{X}_i$ is a $k(b)$-proper scheme, and
 $\varphi_i$ extends to a morphism $\overline{\varphi}_i :
 \overline{X}_i \rightarrow \overline{X}_{i+1}$.  Further, each
 $\overline{\varphi}_i$ is either an  isomorphism, or is isomorphic to
 the dilatation of $\bar{X}_{i+1}$  at a regular point of
 $\bar{X}_{i+1}$. 
 
 Let $\Omega^1 \dfrac{1}{X_i}$ be the sheaf of $1$-differentials on
 $\bar{X}_i$. Since $\Omega^1 \dfrac{1}{X_i}$ are coherent
 $\mathscr{O}_{X_i}$-modules, $\dim_K H^1 (\overline{X}_i,
 \Omega^1_{\overline{X}_i}) < \infty $. Further, we have seen in lecture
 \ref{chap5} that if $\overline{\varphi}_i : \overline{\chi}_i \rightarrow
 \overline{X}_{i+1}$  is a dilatation at a closed point of
 $\overline{X}_{i+1}$, 
 $$
 \dim_K H^1(\overline{X}_{i+1},\Omega^1 \frac{1}{\overline{X}_{i+1}}) <
 \dim_K H^1 (\overline{X}_{i}, \Omega^1 \frac{1}{\overline{X}_i})  
 $$
 (The assumption that the varieties are everywhere regular plays no
 role in the of this inequality). It follows that since the
 sequence\pageoriginale  of non-negative integers  
 $$
 \dim H^1(\overline{X}_{1},\Omega^1 \frac{1}{\overline{X}_1}) \ge \dim H^1
 (\overline{X}_{2},\Omega^1 \frac{1}{\overline{X}_2} \ge  
 $$
must become stationary $\overline{\varphi}_i$ must be an isomorphism
for $i$ large. Consequently, $\varphi_i$ is also an isomorphism for
$i$ large. 

This completes the proof of the theorem.

We shall now state an equivalent formulation of the theorem. For this,
we need the following  

\begin{defi*}%definition
  A noetherian, two dimensional, regular scheme $X$ over a noetherian
  base scheme $B$ is said to be  a \textit{relatively minimal model}
  if for any $B$-scheme $Y$ which is regular, any proper birational
  $B$-morphism $\varphi : X \to Y$ is an isomorphism. 
\end{defi*}

\begin{coro*}%corollary
  Let $B$ be a noetherian scheme and $X$ a $B$-scheme of finite type
  which is regular and of dimension $2$. Then there is a relatively
  minimal model $Y$ over $B$ and a proper birational $B$-morphism of $X$
  onto $Y$. 
\end{coro*}

Indeed, if this  were not true, we can clearly find a sequence $X
\xrightarrow{\varphi_o} X_1 \xrightarrow{\varphi_1} X_2
\xrightarrow{\varphi_2} ---$ of regular $B$-schemes  $X_i$ and proper
$B$-morphisms $\varphi_i$ such that no $\varphi_i$ is an
isomorphism. But this is impossible by the theorem. 

In terms of the `birationality class' $\mathscr{B} (X/B)$ of $X$
introduced in lecture \ref{chap4}, we can restate the Corollary by saying that
there is a $Y$ in $\mathscr{B} (X/B)$ such that $X \ge Y$, but
such that $Y \ge Z$ for any $Z$ implies that $Y = Z$. This is indeed
the justification for the terminology\pageoriginale `relatively
minimal model'.  

\begin{remark*}%remark
  In the proof of the theorem when $B =  \Spec  K$ where $K$ is an
  algebraically closed field, after imbedding the $X_i$ as open
  subsets of projective normal surfaces $\overline{X}_i$, one could
  also use the fact that the Neron-Severi groups of $\overline{X}_i$
  are of finite rank and that this rank is decreased by one when we
  blow down a line to a regular point, to show that the $\varphi_i$
  are isomorphisms for $i$ large. When $K$ is the complex field, the
  second Betti-numbers of the $\overline{X}_i$ would also serve the
  same purpose. 
\end{remark*}

\noindent
\textit{ Minimal Models. Classification of complete, relatively
  minimal models over an algebraically closed field when minimal
  models do not exist}. 

Let the base scheme be arbitrary noetherian.

Let $X$ be an irreducible, regular, proper $B$-scheme of  dimension
two, such that the structural morphism $X \to B$ is dominant and hence
surjective. We say that $X$ is a \textit{ relatively minimal model
  over $B$ } if any $B$-birational morphism $\varphi : X \to X'$ of
$X$ onto another regular, proper $B$-scheme $X'$ is an isomorphism. It
is said to be a \textit {minimal model} if for any regular, $B$-proper
scheme $Y$, any $B$-birational map $\psi : Y \to X$ is a
$B$-morphism. In view of the  theorem of existence of a minimal model
dominated by any given model we see that a relatively minimal model
$X/B$ is a minimal model if and only if a birational map $\varphi : X
\to X'$ of $ X $ onto any other relatively\pageoriginale minimal model
$X'/B$ is a $B$-isomorphism.  

Our intention in this paragraph is to show that when $B =\Spec  K$,
where $K$ is an algebraically closed field, except in comparatively
few cases (rational and ruled surfaces), a minimal model exists, and
to classify all the relatively minimal models in the exceptional
cases. 

Before we set out on this, we shall establish two results which ought
to have found their places earlier. We now suppose that $B =  \Spec
K$, where $K$ is an algebraically closed field. 

The first concerns itself with the behaviour of the canonical class
when the surface is blow up. Let $X$ be a non-singular surface, and
$\Omega^2$ the line bundle on $X$ whose sections are exterior $2$
forms on $X$. The divisor class determined by this line bundle (i.e.,
divisor class of the divisor determined by a non-zero rational section
of $\Omega^2)$ is called \textit{the canonical class} $K$ of $X$. Let
$x$ be any point of $X$, and $\sigma : X^1 \to X$ the dilatation of
$X$ at $x$. As usual, we denote the fibre $\sigma^{-1} (x)$ by $L$. We
want the relationship between the canonical classes $K_X$ and $K_{X'}$
of $X$ and $X'$ respectively. Let $\omega$ be a rational $2$-form on
$X$ which is regular and non-vanishing at $x$, and $\sigma^* (\omega)$
its inverse image on $X'$. Since $\sigma$ induces an isomorphism of
$X'-L$ onto $X-\{x \}$, the zeros and poles of $\omega$, and the
multiplicities of the components of the zero and polar divisors are
preserved. It only remains to compute the coefficient of $L$ in the
divisor of $\sigma^* (\omega)$. Let $u$, $v$\pageoriginale be
uniformising parameters 
at $x$, regular in an affine neighbourhood of $x$ with co-ordinate
ring $A$. Then $\omega$ has an expression of the form $f$ $du \wedge
dv$ in this neighbourhood of $x$, where $f$ is regular and
non-vanishing at $x$. In the open subset of $X'$ with co-ordinate ring
$A [\dfrac{v}{u}], \sigma^*(\omega)$ takes the form $(f \circ \sigma) d(u \circ
\sigma) \wedge d(v \circ \sigma)$. Now, we know that at points of $L$
which lie in this neighbourhood, $u \circ \sigma$ and $w=\dfrac{v \circ
  \sigma}{u \circ \sigma}$ are uniformising parameters. In terms of these,
we have 
\begin{align*}
  \sigma^* ~ (\omega) &= (f \circ \sigma) ~ d(u \circ \sigma) \wedge ~
  d((u \circ
  \sigma) w )\\ 
  & = (f \circ \sigma). ~ (u \circ \sigma) ~ d(u \circ \sigma) \wedge d w
\end{align*}
Since $f \circ \sigma$ is regular and does not vanish along $L$ and since
$u \circ \sigma$ vanishes to the first order on $L$, we obtain that 
$$
\text{ div } (\sigma^*(\omega))  ~ = \sigma^* (\text { div } (\omega))
+ L, 
$$
and consequently we have between the canonical classes of $X$ and $X'$
the relation 
\begin{equation*}
  K_{X'} ~ = \sigma^* ~ (K_X) + L\tag{1}\label{chap7:eq1}
\end{equation*}

The second result that we need is concerned with the arithmetic (or
virtual) genus of a (possibly reducible) curve $D$ on a non-singular
complete surface. The arithmetic genus is defined by the formula. 
$$
\pi (D) ~ = ~ 1 - \chi (\mathscr{O}_D) ~ = ~ 1 - \dim_K H^{\circ} 
(D, \mathscr{O}_D) + \dim_K H^1 (D, \mathscr{O}_D) 
$$
where\pageoriginale $\mathscr{O}_D = \mathscr{O}_{X/\mathscr{I}_D}$
and $\mathscr{I}_D$ 
is the invertible sheaf of ideals in $\mathscr{O}_X$ defining the
effective divisor $D$. Denoting as always by $\chi(F)$ the Euler
characteristic $\sum\limits_{i=0}^{2}(-1)^i \dim_K H^i(X, F)$ of a
coherent sheaf $F$ on $X$, the exact sequence 
$$
O \to \mathscr{I}_D \to \mathscr{O}_X \to \mathscr{O}_D \to O
$$
gives the relation
$\chi(\mathscr{O}_D)=\chi(\mathscr{O}_X)-\chi(\mathscr{I}_D)$. By
Serre duality, $\chi(\mathscr{O}_X)=\chi(\Omega^2)$ and
$\chi(\mathscr{I}_D)=\chi(\mathscr{I}^{-1}_D \otimes
\Omega^2)$. Finally, the exact sequence $0 \to \Omega^2 \to \Omega^2
\otimes \mathscr{I}_D^{-1} \to \Omega^2 \otimes \mathscr{I}^{-1}_{D/
  \mathscr{O}_{\chi}} \to 0$ gives the relation 
\begin{align*}
  \chi(\mathscr{O}_D)& =\chi(\mathscr{O}_X)-\chi(\mathscr{I}_D)\\
  &=\chi(\Omega^2)-\chi(\Omega^2 \otimes \mathscr{I}^{-1}_{D})\\
  &=-\chi(\Omega^2 \otimes \frac{\mathscr{I}^{-1}_D}{\mathscr{O}_X})
\end{align*}

Now an elementary argument (used in the proof of the Riemann-Roch
theorem for curves) shows that $\chi(\Omega^2 \otimes
\mathscr{I}^{-1}_{D/\mathscr{O}_X})=((K+D).D)+\chi(\mathscr{O}_D)$,
and we obtain by substitution that 
\begin{align*}
2\chi(\mathscr{O}_D)& =2-2 \pi(D)=-((K+D).D)\\
\pi(D)& =1+\frac{1}{2}((K+D).D) \tag{2}\label{chap7:eq2}
\end{align*}

As an application of formula (\ref{chap7:eq1}) and (\ref{chap7:eq2})
one can find how $\pi(D)$ 
changes under a single dilatation. Namely, applying formula (\ref{chap6:eq10}) of
lecture \ref{chap6} one easily finds that  
$$
\pi(D')=\pi(D)-\frac{l(l-1)}{2},
$$\pageoriginale
$D'=\sigma'(D)$ and $l$ is the multiplicity of $D$ at the center of
$\sigma$. In particular, if $C$ is irreducible we can resolve the
singularities of $C$ by means of dilatations. We obtain a nonsingular
curve $\overline{C}$ with genus $g$ and $\pi(\overline{C})=g$. The
above formula gives 
$$
\pi(C)=g+\sum \frac{l_i(l_i-1)}{2}
$$
where $l_i$ is the multiplicity of the image of $C$ under the $i^{\rm th}$
consecutive dilatation. 

We see that $\pi(C)\geq 0$ and even $\pi(C) \geq g;$ if $\pi(C)=g$,
then $C$ is nonsingular. 

Let us return to the general case of $X \to B$. Suppose that $X$ does
not possess a minimal models. Then there are two relatively minimal
models $X$ and $Y$ with function fields isomorphic to the given
function field and a birational map $\varphi:X \to Y$ which is not
regular either way. 

By the theorem of elimination of indeterminacies and the theorem of
decomposition, there exist $B$-proper regular  schemes\pageoriginale
$X_i(0 \leq i 
\leq m), Y_j(0 \leq j \leq n)$ and morphisms 
$$
\sigma_i:X_{i+1} \to X_i(0
\leq i < m), \tau _j: Y_{j+1} \to Y_j(0 \leq j < n)
$$
 such that
$X_0=X,Y_0=Y, X_{m+1}=Z=Y_{n+1}$, and each $(X_{i+1}, \sigma_i)$ is
$X_i$ isomorphic to the dilatations of $X_i$ at a point and each
$(Y_{j+1},\tau_j)$ is $Y_j$-isomorphic to the dilatation of $Y_j$ at a
point. We choose the $X_i,Y_j, \sigma_i, \tau_j$ in such a way that
the integer $m+n$ is minimal among all such possible choices. We can
suppose $m>0$, $n>0$, because otherwise either $X$ or $Y$ would not be
relatively minimal. Let us denote by $L_{m+1}$ the line on $Z$ which
is contracted to a point on $Y_n$ by $\tau_n$. We assert that the
image of $L_{m+1}$ by $\sigma_\circ  \circ \sigma_1 \circ \cdots \circ
\sigma_m$ in $X=X_\circ$
is not a point of $X$ but a curve. For if it were, we obtain a
morphism $f:Y_n \to X$ such that $f \circ \tau_n =\sigma_\circ
\circ \cdots \circ \sigma_m$. If we write $f \circ \tau_n$ as
the composite of a 
sequence of dilatations, say $f \circ \tau_n=\lambda_o \circ \lambda_1 \circ
\cdots \circ \lambda_p$, then $p$ equals the number of irreducible
components of the closed subset of $Y_n$ where $f$ fails to be an
isomorphism. From similar interpretation of the integer $m$, it follows
that\pageoriginale $p \leq m$. Thus, we have a situation depicted by
the following diagram:  
\[
\xymatrix@C=.5cm{&&&&Y_\circ=Y& \\
  &&& \ar[ur]^{\tau_\circ}&&\\
  && \ar@{.}[ur]& &&\\
  & Y_{n-1} \ar[ur]&  & && \\
  \ar[ur]^{\tau_{n-1}}
  Y_n \ar[dr]_{\lambda_p} &&&&&\\
  & Z_p \ar[r]_{\lambda_{p-1}} & Z_{p-1} \ar[r] & \ar@{.}[r] &
  \ar[r]_{\lambda_\circ} & Z_\circ=X
}
\]

But now, we have $p+(n-1) \leq m+n-1$, which contradicts the
minimality of $(m+n)$ for all choices of $X_i,Y_j, \sigma_i,
\tau_j$. Thus we have shown that $L_{m+1}$ is not contradicted to a
point in $X$. Let us denote by $L_i(0 \leq i \leq m+1)$ the curve in
$X_i$ which is the image of $L_{m+1}$ by $\sigma_1 \circ \sigma_{i+1} \circ
\cdots \circ \sigma_m$. It is then clear that for $i>0$, $L_i$ is the proper
transform $\sigma'_{i-1}(L_{i-1})$ of the curve $L_{i-1}$ on
$X_{i-1}$. 

Now $L_{m+1}$ is isomorphic to $\mathbb{P}'(K)$ and
$(L_{m+1}^2)=-[K:k(b)]$, where $K$ is a finite algebraic extension of
$k(b)$, since $L_{m+1}$ can be blown down to a point in $Y_n$ (Lecture
\ref{chap6} (\ref{chap6:eq9})), By formula (\ref{chap6:eq11}) of
Lecture \ref{chap6}, if 
we denote by $s_i$ the 
multiplicity of $L_i$ at the point $x_i$ of $X_i$ which is blown up in
$X_{i+1}$, we have 
$$
  (L_{i+1})^2=(L_i)^2-s_i^2[k(x_i):k(b)] \; (i=0,1.\ldots,m),
$$\pageoriginale
so that
$$
-[K:k(b)]=(L_{m+1}^2)=(L_0^2)-\sum_{i=0}^{m}s_i^2[k(x_i):k(b)]
$$

(We put $s_i=0$ if $X_i$ does not lie on $L_i$). Now, suppose all the
$s_i$ are $0$, so that $x_i$ does not lie on $L_i$ for any $i$; there
is a neighbourhood of $L_{m+1}$ which is mapped isomorphically a
neighbourhood of $L_o$ by $\sigma_\circ \circ \cdots \circ
\sigma_m$. Hence $L_\circ$ 
can be contracted to a point of a regular $B$-scheme, which
contradicts over assumption that $X_o=X$ is a relatively minimal
model. Hence, at least one $s_i>0$. Let $i$ be the largest integer for
which $s_i>0$. Then, by (\ref{chap6:eq10}) of Lecture \ref{chap6}, 
$$
(L_{i+1},\sigma_i^{-1}(x_i))=s_i[k(x_i):k(b)],
$$
and since $L_{i+1}\simeq L_{m+1} \simeq \mathbb{P}'(K)$ the residue field at
every point of $L_{i+1}$ contains $K$, so that the left side is $\geq
[K: k(b)]$ 

It follows from the above equation for $(L_0^2)$ that 
\begin{equation*}
  \left.
  \begin{aligned}
    &(L_0^2)\geq 0,\\
    &(L_0^2)>0, if L_0 \text{ possess singularities}
  \end{aligned}\quad 
  \right\}\tag{3}\label{chap7:eq3}
\end{equation*}
Thus, we have proved the following fundamental lemma

\begin{lemma*}
  If $X$ is a proper, regular, irreducible, two dimensional
    scheme over $B$, which is a relatively minimal model but not a
    minimal model, then there is a closed, irreducible,
    one-dimensional subscheme\pageoriginale $L$ of $X$ such that $L$
    is contained in 
    the fibre over a closed point $b$ of $B$. $(L^2)\geq 0$ and
    $(L^2)>0$ if $L$ is not regular and $L$ is birationally isomorphic
    to $\mathbb{P}^1(K)$, where $K$ is a finite algebraic extension of
    $k(b)$. 
\end{lemma*}

Now, suppose that $B= \Spec K$, $K$ an algebraically closed field. Apply
formula (\ref{chap7:eq2}) to the curve $L_i$. Since $\pi(L_i)=\dim
H^1(\mathbb{P}^1,\mathscr{O})=0$ we have $(K_i, L_i)=-1$, where $K_i$
are the canonical classes of $X_i$. By (\ref{chap7:eq1}) we have 
$$
(K_i,L_i)=(\sigma^*(K_{i-1})+L_i,L_i)=(K_{i-1},L_{i-1})+l
$$ 
by formulae (\ref{chap6:eq8}) and (\ref{chap6:eq10}) of lecture
\ref{chap6}. So, 
$$
(K_{i-1},L_{i-1})=-1-l<0.
$$

From (\ref{chap7:eq1}) it again follows that the same is true for $L_0$. We have
thus the following supplement to the principal lemma: 

\textit{In the case where $X$ is a surface over an algebraically
  closed field, we have $(K.L)<0$ where $K$ is the canonical class of
  $X$} 

Upto the end of this lecture, we will consider the case $B=\Spec K, K$
an algebraically closed field. The case of a one-dimensional $B$ will
be discussed in the next lecture.	 

Now, for any complete non-singular surface $X$, the integers $P_n=\dim$
 $H^0(X,(\Omega ^2)^{\otimes n})(n 1)$ are called the
\textit{plurigenera} of $X$. From the theorem of elimination of
indeterminacies and the theorem of decomposition, one deduces by a
standard argument\pageoriginale (using the fact that a section of a
vector bundle in 
the complement of a closed subset of codimension $\geq 2$ extends to a
section on the entire surface (see Lecture \ref{chap5} for $P_1$) that the
integers $P_n(X)$ are birational invariants of $X$. Further, since any
non-zero section $s$ of $(\Omega^2)^{\otimes n}$ leads to a non-zero
section $s^{\otimes m}$ of $(\Omega^2)^{\otimes mn}$ for $m \geq 1$,
we see that $P_{mn}=0$ implies that $P_n=0$. 
 
Now let $X$ be as in the fundamental lemma above. We shall then show
that $P_{n}(X)=0$ for all $n \geq 1$ (or in mere classical
terminology, the pluri-canonical systems do not exist). For, suppose
$P_n(X)\geq 1$ for some $n \geq 1$, so that there is an effective
divisor $D$ belonging to the class $n K$. If we write $D=D_1+r.L$
where $r \in \mathbb{Z}$ and $D_1$ does not have $L$ as a component,
we have 
$$
n(K.L)=(D_1.L)+r(L^2)\geq 0
$$
 since $(D_1.L)\geq 0$ and $(L^2)\geq 0$ by the fundamental lemma. But
 the above inequality contradicts the second inequality of the same
 Lemma. Thus, $P_n=0$ for all $n \geq 1$. 
 
 We now make the following

\begin{defi*}%def
  A {\em ruled surface} is the total space of a locally trivial fibre
  bundle with base a complete non-singular curve, typical fibre the
  projective line $\mathbb{P}^1$ and structure group the projective
  group $PGL(1)$ 
\end{defi*}
 
It is an easy matter to show that for a ruled surface, all the
plurigenera vanish. Indeed, the bundle of which it is the total space\pageoriginale
being locally trivial, the surface is birationally equivalent to a
product $C \times \mathbb{P}^1$. If $\pi_1$ and $\pi_2$ are the projections
of $C \times  \mathbb{P}^1$ onto the first and second factors, we have
$\Omega^2(C  \times \mathbb{P}^1)=\pi^*_1(\Omega^1_C)\otimes \pi^*_2
(\Omega^1_{\mathbb{P}^1})$, so that 
$$
(\Omega^2(C \times \mathbb{P'}))^{\otimes n} \simeq
\pi_1^*((\Omega'_C)^{\otimes n})\otimes
\pi_2^*((\Omega'_{\mathbb{P'}})^{\otimes n}) 
$$

It follows that $H^0(C \times \mathbb{P}',(\Omega^2)^{\otimes
  n})=H^0(C,(\Omega'_C)^{\otimes n})^{\otimes k} \otimes_K
H^0(\mathbb{P}',\break(\Omega'_{\mathbb{P}'})^{\otimes n})=0$ since
$(\Omega'_{\mathbb{P}})^{\otimes n}$ is a locally free sheaf whose
rational section have degree $-2n$. This shows that $P_n(C \times
\mathbb{P}')=0$ for $n \geq 1$, and consequently, because of the
birational invariance of the $P_n, P_n=0$ for $n \geq 1$ for any ruled
surface. 

It is a theorem of Enriques (\cite{key2} Sh. IV) that conversely if $X$ is a
complete non-singular surface with $P_4=P_6=0$ (or $P_{12}$ alone
$=0$), then $X$ is birationally equivalent to $C \times
\mathbb{P}'$. The proof of this theorem is long and we shall not give
it. We note however that this yields the result that any $X$ as in the
fundamental Lemma  is birationally a product $C \times
\mathbb{P}'$. We shall prove this (and more, namely that $X$ is
actually a ruled surface) directly below. 

From now on till the end of this lecture, $X$ and $L$ will have the
same connotations as in the fundamental Lemma. We consider two cases. 

\medskip
\noindent
{\bf Case (i). The Albanese variety $A$ of $X$ is non-trivial.}\pageoriginale

In this case, choose a point $x_0$ of $L$, and denote by $\psi:X \to
A$ the canonical morphism of $X$ into the Albanese such that
$\psi(x_\circ)=0$. The image $\psi(X)$ of $X$ in $A$ must be an
irreducible variety of dimension 1 or 2, since this image must
generate $A$. Further since $L$ is a rational curve, by a well known
result on abelian, since $L$ is a rational curve, by a well known
result on abelian varieties (\cite{key15} p. 25), $\psi(L)=0$. Suppose $\psi(X)$
is of dimension 2. In this case, the restriction of the bilinear
form $(.)$ (intersection number) to the irreducible components of
dimension one of any fibre is negative definite, as proved in lecture
\ref{chap6}. But then, since $L \subset \psi^{-1}(0)$, we must have $(L^2)<0$,
which is a contradiction. 

Thus, $\psi(X)$ must be a curve in $A$. Identifying the function field
$R(\psi(X))$ of $\psi(X)$ with a subfield of $R(X)$ by means of
$\psi$, let $F$ be the algebraic closure of $R(\psi(X))$ in $R(X)$,
and let $C$ be the normalisation of the curve $\psi(X)$ in $F$, with
canonical morphism $\pi:C \to \psi(X)$. Since $X$ is nonsingular (and
hence normal), the morphism $\psi:X \to \psi(X)$ factorises $\psi=\pi
\circ \psi'$ where $\psi :X\to C$ is a morphism. Since the fibres of $\pi$
are finite, and since $L$ is contained in a fibre of $\psi, L$ is also
contained in a fibre of $\psi'$,i.e., a fibre of $\psi, L$ is also
contained in a fibre of $\psi'$,i.e., $\psi'(L)=y \in C$. Further
since $R(C)$ is algebraically closed in $R(X)$, the generic fibre of
$\psi'$ is geometrically irreducible (see Lemma at the beginning of
lecture) and any fibre of $\psi'$ is geometrically\pageoriginale
connected by the 
connectedness theorem of Zariski. Further, we have seen in lecture \ref{chap6}
that if the fibres are connected, the restriction of the intersection
number to the free abelian group generated by components of any fibre
is negative semi-definite, and the null-space of this bilinear form is
precisely the maximal subgroup of rank one containing the whole
fibre(considered as a divisor). Since $(L^2)\leq 0$ and $(L^2)\geq 0$
by assumption $(L^2)=0$, so that $L$ is non-singular and is the only
irreducible component of its fibre. Hence the divisor, $\psi'^{-1}(y)$
is of the form $nL$, where $n$ is a positive integer. Our aim is to
show that $n=1$. Since $L$ is isomorphic to $\mathbb{P}',\pi(L)=0$ and
hence $(KL)=-2$. For any $z \in C$, we have by what we have seen in
Lecture \ref{chap6} that 
$$
(K. \psi'^{-1}(z))=(K, \psi'^{-1}(y))=(K.nL)=-2n,
$$
so that for the arithmetic genus of $\psi'^{-1}(z)$ we have
$$
\pi(\psi'^{-1}(z))=1+\frac{1}{2}(\psi'^{-1}(z).(\psi'^{-1}(z)+K))=1-n 
$$
since $(\psi'^{-1}(z)^2)=0$.

Now, we have the

\begin{lemma*}%lem
  Let $R$ be a function field in one variable over a perfect field
  $K$, and suppose $R$ is contained and is algebraically closed in a
  field $S$ is separable over $R$. 
\end{lemma*}

\begin{proof}
  Since $R$ is separably generated over $K$, if $\{x\}$ is a
  separating transcendence basis of $R/K$, we have that
  $R^{1/p}=R(x^{1/p})$ We have to show that $S$ and $R^{1/p}$ are
  linearly disjoint\pageoriginale over $R$ in order to establish that $S/R$ is
  separable. But now, $x^{1/p}\notin S$ since $R$ is algebraically
  closed in $S$, and since $X \in R \subset S$, we have
  $[S(x^{1/p}):S]=p=[R(x^{1/p}):R]$, which proves the assertion. 
\end{proof}

Now, $R(C)$ is algebraically closed in $R(X)$, so that $R(X)$ is also
separable over $R(C)$, by the above lemma. But in this case, it can be
shown as in the first lemma of this lecture that there is an open
subset $U$ of $C$ such that for $z \in U, \psi'^{-1}(z)$ is an
integral scheme, or equivalently that as a divisor, $\psi'^{-1}(z)$
has a single component and this component occurs with multiplicity
one. Hence for $z \in U$, we have
$H^0(\psi'^{-1}(z),\mathscr{O}_{\psi'^{ -1} (z)})\simeq K$, so that    
\begin{align*}
  \pi(\psi'^{-1}(z))& =1-\dim H^{0}(\psi'^{-1}(z),
  \mathscr{O}_{\psi'^{-1}(z)})+\dim
  H^{1}(\psi'^{-1}(z),\mathscr{O}_{\psi'^{-1}(z)})\\
  & =\dim  H^{1}(\psi'^{(-1)}(z),\mathscr{O}_{\psi'^{-1}(z)})\geq 0 
\end{align*}

It follows from our earlier result that $1-n \ge 0$, so that
$n=1$. Thus the \textit{divisor} $\psi'^{-1} (y)$ is equal to $L$, and
$(K. \psi'^{-1}(z)) = - 2$ for all $z \in C$. We now assert that
\textit{every} fibre $\psi'^{-1}(z)$ in the schematic sense is
isomorphic to $\mathbb{P}'$. To prove this, first suppose that
$\psi'^{-1}(z)=n.D$ where $n$ is an integer $> 1$ and $D$
irreducible. Then we have $(D^2)=0$ and $-0=(n D.K) = n (D.K)$, so
that $n=2$ and $(D.K)=-1$. But this is impossible, since $(D^2) +
(D.K)$ must be an even integer. Thus if $\psi'^{-1} (z) = nD$ with $D$
irreducible, then $n=1$ and $(D.K)=-2$, so that $\pi (D)=0$. Now
suppose that $\psi'^{-1} (z) = \sum\limits^r_1 n_i D_i$,\pageoriginale
with $n_i > 0$ and $D_i$ running through the irreducible components of
$\psi'^{-1}(z)$, with $r \ge 2$. In this case, we must have $(D_i^2)
\le -1$, and $(D_i^2) + (D_i.K) \ge -2$ for $i=1,\ldots , r$. We
assert that there exists an i such that $(D_i^2) = -1 $ and $(D_i^2) +
(D_i. K) = -2$. If not, for every $i$, either $(D_i^2) \le - 2 $ or
$(D_i^2) + (D_i.K) \ge - 1$, with $(D_i^2) = - 1$, so that $(D_i . K) \ge
0$ for all $i$. But this is impossible since $(\psi'^{-1} (z). K) =
\sum n_i (D_i K) = - 2$. Thus, there is an $i$ for which $(D_i^2) = -
1$ and $D_i^2 + (D_i.K) = -2$, that is, $\pi (D_i) = 0$. But as we
have seen in connection with formula (\ref{chap7:eq2}) $\pi \ge g$, and $\pi = g$
only if the curve is nonsingular. So $D_i$ is nonsingular and of
genus $0$ and consequently isomorphic to $\mathbb{P}'$. Since
$(D_i^2) = -1$, it follows by the Theorem of Castelnuovo that $D_i$
can be blown down to a point of a non-singular surface, thus
contradictions the assumption that $X$ is a relatively minimal
model. Thus we finally deduce that for any $z \in C, \psi'^{-1} (z)$
(in the schematic sense) is isomorphic to $\mathbb{P}'$. 

We shall now deduce that $\psi' : X \to C $ is a locally trivial
fibration with $PGL (1)$ as structure group and $\mathbb{P}'$ as
fibre. 

Let $x$ be a generic point of $C$ and $\psi'^{-1} (x)$ the generic
fibre. By the lemma proved earlier, the function field $R (\psi'^{-1}
(x)) (\simeq R(X))$ is a regular extension of $k(x) (\simeq
R(C))$. Thus, the curve $\psi'^{-1} (x)$ defined over $k(x)$ has genus
$0$. ($\EGA$ III, 79). It is then well-known that $\psi'^{-1} (x)$ is
birationally isomorphic \textit{over} $k(x)$ to a conic $Q(X_oX_1X_2)
= 0$ in $\mathbb{P}^2$ defined over\pageoriginale $k(x)$ (see lecture
\ref{chap5}). But by a 
theorem of Tsen, any form in $n$ variable of degree in a function field
of one variable over an algebraically closed field has a non-trivial
(i.e. different from $(0)$) zero in that field provided that $n>d$. In
particular $\psi'^{-1} (x)$ carries a $k(x)$-rational point is a
rational section $\sigma:C \to X$ such that
$\psi' \circ \sigma=Id_C$. However, since $C$ is a normal curve and $X$ is
complete, $\sigma$  extends to a $C$-morphism of $C$ into $X$. If we
denote by $D$ the curve $\sigma (C)$ in $X,\sigma(C)$ intersects every
fibre $\psi'^{-1}(z)$ transversally at a unique point $\sigma(z)$ for
all $z \in C$, so that $(D. \psi'^{-1}(z))=1$ for all $z \in C$. Let
$\mathcal{L}(u)$ denote the invertible sheaf defined by the divisor
$u$ and put $F=\psi'^{-1}(z)$, for some fixed $z \neq y$. For any $m$,
we have the standard exact sequence: 
$$
0 \to \mathscr{L}(D+(m-1)F)(m-1)F) \to \mathscr{L}(D+mF)\to
\mathscr{L}_F (D+mF). F\to 0 
$$ 
where the last homomorphism is the restriction to $F$. Obviously
$\mathscr{L}((D+mF).F)=\mathscr{L}_F(D.F)$. 

We shall prove that, for $m$ large enough the corresponding homomorphism
$$
H^0(X,\mathscr{L}(D+mF))\to H^0(F,\mathscr{L}_F(D.mF))
$$  
is surjective. Because of the cohomology exact sequence it is sufficient
to prove that 
$$
H^1(X,\mathscr{L}(D+(m-1)F))\to H^1(X,\mathscr{L}(D+(m-1)F))
$$ 
is an isomorphic for large $m$. But as $H^1(F,\mathscr{L}(D.F))=0$
(note that\pageoriginale $F \simeq \mathbb{P}'$ and $D.F$ is a point
on $F$). This 
homomorphism is a certain surjective. So the numbers
$\dim.H^1(X,\mathscr{L}(D+mF))$ form a decreasing sequences of
positive integers and thus remain constant for large $m$. This is just
what we need. As $D.F$ is a point on $\mathbb{P}^1,\dim
H^0(F,\mathscr{L}(D.F))=2$. Choose two section $s_0$ and $s_1$ of
$\mathscr{L}(D+mF)$ over $X$ whose images form a basis for
$H^0(F,\mathscr{L}(D.F))$. 

The linear system
$$
\lambda_o s_o+\lambda_1s_1 = 0 (\lambda_\circ,\lambda_1 \in K,
(\lambda_\circ, \lambda_1) \neq (0,0))
$$
has no base points on $\psi'^{-1}(z)$, by choice of the $s_i$, so that
because of the properness of $\psi '$, we may, by choosing $U$ smaller
if necessary, assume that this linear system has no base points on
$\psi'^{-1}(U)$. Thus, we get a morphism $\lambda:\psi'^{-1}(U)\to
\mathbb{P}'$, and restricted to $\psi'^{-1}(z)$ is an isomorphism onto
$\mathbb{P}'$. Let  $\psi :\psi'^{-1}(U)\to U \times \mathbb{P}'$
denote the morphism $\chi=(\psi'|\psi'^{-1}(U), \lambda)$. For any $x
\in \psi'^{-1}(z),\chi^{-1}(\chi(x)=\{x\}$, and the tangent mapping
$d\chi$ is injective at $x$. Since $\chi$ is also proper, we deduce
that $\chi$ is birational, and it follows by $z.M.T$ that $\chi$ is an
isomorphism of a neighborhood of $\psi'^{-1}(z)$ onto a neighbourhood of
$z \times \mathbb{P}'$ over $U$. Again because of the properness of
$\psi'$ and the projection $U \times \mathbb{P}'\to U$, it follows that
there is a neighbourhood $V$ of $z$ such that there is a
$V$-isomorphism of $\psi'^{-1} (V)$ onto $V \times \mathbb{P}'$. This
establishes the local
triviality of $X \overset{\psi}\to C$. It can be deduced easily that
this is indeed an associated fibre space on\pageoriginale $C$ to a principle fibre
space with $PGL(1)$ as structure group. (We have only to show that if
there are two `trivializations' on an open subset, they are `connected'
by projective transformations `varying algebraically' in the open
subset). 

Thus, our investigation in case $(i)$ lead to the  

\begin{theorem*}%\the
  Let $X$ be a complete non-singular surface which is a
    relatively minimal model but which is not a minimal model. Assume
    that the Albanese variety of the surface is non-trivial (or in other
    words, the surface is irregular). Then $X$ is a ruled surface over
    a curve of positive genus.   
\end{theorem*}

Thus, the problem of classification of relatively minimal (but not
minimal) models reduced to the classification of (locally trivial)
principal $PGL(1)$ bundled on curves. Now, it is not difficult to show
that any $PGL(1)$ bundle arises from a $GL(2)$-bundle by `extension of
structure group, and two $GL(2)$-bundles give rise to the same
$PGL(1)$ bundle if and only if, of the corresponding vector line
bundles of rank 2, each is got from the other by tensoring with a
line bundle. Let us fix once and for all a line bundle $L$ of degree
bundles of rank two one $C$ whose determinantal bundle is the trivial
line bundle (\resp is the line bundle $L$). Let $\pi$  be the (finite)
group of elements of order two on the Jacobian of the curve. Then it is
clear that $\pi$ acts on $S_o$ and $S_1$ (by tensorisation). The
quotient set of $S_o \cup S_1$ for the action of $\pi$ can be
canonically\pageoriginale identified with the set of relatively minimal models or
ruled surfaces on the give curve $C$. 

We now turn to the case of regular surfaces which are relatively
minimal but not minimal. 

\noindent
\textbf{Case (ii). $X$ is (as in the fundamental lemma, and) such that
  the Albanese variety of $X$ is trivial.} 

In this case, by using a criterion due to \textit{Castelnuovo} for
the rationality of a projective surface, we shall show first that $X$
is rational, and we shall then state without proof the theorem of
classification of rational surfaces which are relatively minimal. 

It is known that \textit{when the first integer $P_1$ (usually called
  the geometric genus and denoted by $p_g$)} among the plurigenera
$P_i$ \textit{of a non-singular surface is $0$, the irregularity $q$
  (equal by definition to the dimension of the Albanese of the
  surface) is given by} (\cite{key8}, 236, 2.10) 
$$
q=-p_a
$$  

It follows that for our relatively minimal model $X,p_a=o$. We now
apply the following criterion of rationality (\cite{key2},
\cite{key24}, \cite{key25}). 

\begin{theorem*}[(Castelnuovo)]%the
  If for a complete non-singular surface we have $p_a=p_2=0$, the
  surface is rational. 
\end{theorem*}

Thus we deduce that $X$ is rational. We want to describe here all
relatively minimal models of a rational surface. First, we recall the
definition of the line bundle $G(D)$ over a curve $X$ associated with
a divisor $D$. 

Let\pageoriginale $X= \bigcup\limits_I U_i$ be a covering of $X$ such that $D$ is
defined on each $U_i$ by a function $u_i$. Then $G(D)$ is covered by
open subset $U_i \times \mathbb{P}^1$ which are patched up together by means
of the transition rule 
\begin{align*}
  x \times \xi \sim y \times \eta, &  \quad x \in U_i, y \in U_j,\\
  \text {if} x &= y \in U_i \cap U_j\\
  \text { and } ~\eta & = \xi \frac{u_i}{u_j}(x) \tag{3}
\end{align*}

Here we understand that the point at $\infty$ on $\mathbb{P}^1$
remains unchanged on multiplication by $\dfrac{u_i}{u_j}(x)$. Thus,
$G(D)$ has two obvious sections: $S_o=\{x \times o \}$ and
$S_\infty=\{x \times \infty \}$. As is well- known $G(D)\simeq G(D')$
if $D \sim D'$. We shall apply this construction to
$X=\mathbb{P}^1$. Since a divisor class on $\mathbb{P}^1$  is
determined by its degree, we can take $D=nx$ and we shall denote the
corresponding $G(D)$ by $\mathbb{F}_n$. One can check readily that 
$$
x \times \xi \longrightarrow \qquad x \times \xi^{-1}
$$  
defines an isomorphism between $G(D)$ and $G(-D)$. This explains why we
can restrict ourselves to considering $\mathbb{F}_\mu,n \ge o$. 

We shall first compute the group of divisor classes on
$\mathbb{F}_n$. Note first that all the fibres of $\mathbb{F}_n$ are
linearly equivalent. Indeed, let $F_x$ and $F_y$  be fibres over $x,y
\in \mathbb{P}^1, \pi :\mathbb{F}_n \to \mathbb{P}^1$ be the
projection and let $f$ be a function on $\mathbb{P}^1$ such that
$(f)=(x)-(y)$; then $F_x-F_y=(\pi ^* f)$. We shall prove that the
group\pageoriginale of divisor classes on $\mathbb{F}_n$ has two
generators -any fibre 
$F_o$ and the section $S_o$. Let $D$ be any divisor on
$\mathbb{F}_n$. Choose a proper open set $U$ of the base $\mathbb{P'}$
such that if $\pi :\mathbb{F}_n \to \mathbb{P'}$ is the projection,
$\pi^{-1} (U)\simeq U\times \mathbb{P'}$. Then $\pi ^{-1}(U)-(S_o\cap
U)$ is isomorphic to $U \times K$, hence to an open subset of
$K^2$. Thus, the restriction of $D$ to $\pi ^{-1}(U)$ is a principal
divisor $(f)$ an $\pi ^{-1}(U)$. The divisor $D-(f)$ has therefore
components contained in $S_o \cup \pi^{-1}(\mathbb{P'}-U)$, that is,
$D-(f)$ is a linear combination of $S_o$ and a finite number of
fibres. Since all fibre are linearly equivalent, our assertion
follows. 

We now derive an important equivalence between $S_o,S_ \infty$ and
$F_o$. Let us look at $\mathbb{F}_n$ as being of the form $G(D), D$ a
divisor of degree $n$ on $\mathbb{P}^1$. Consider the function  
$$
f(x \times \xi)=\xi u_i (x)
$$   
on the open subset $U_i \times \mathbb{P}^1$. In view of the
transition rule, it is obvious that we get a function $f$ on the
whole of $G(D)$. The divisor $(f)$ can be easily computed, the answer
is  
$$
 (f)=\pi^{-1}(D)+S_o-S_\infty
$$   
In particular, we have on $\mathbb{F}_n$
\begin{equation*}
  n F_o + S_o-S_ \infty \sim 0 \tag{4}\label{chap7:eq4}
\end{equation*} 
 
 If we\pageoriginale consider the intersections of both sides of
 (\ref{chap7:eq4}) with $S_o$  and $S_ \infty$ and use the equalities
 $(S_o.F_o)=(S_ \infty. F_o)=1$  we obtain  
 $$
 (S^2_o)=-n~ \text { and }~(S^2_ \infty)=n
 $$
 
 From this we can deduce that $S_o$ is (if $n>o$) the only
 irreducible curve on $\mathbb{F}_n$ with negative self-intersection
 (while on $\mathbb{F}_o$ there are none). 
 
 Indeed, if $C$ is another irreducible curve, $C \sim  lF_\circ +mS_\circ$ and
 thus either $C \sim lF_\circ$ and $(C^2)=0$  
 \begin{alignat*}{4}
   \text { or } & \qquad & m &=(C.F_o)>0\\
   &&1-mn &=(C.S_o)\le 0\\
   \text { and }&& \qquad (C^2) & =
   m(2l-n) = m(2(l-mn)) + (2m-1)n) > 0\qquad 
 \end{alignat*} 
 
 We see from this that the $\mathbb{F}_n$ are non-isomorphic not only
 as fibrations over $\mathbb{P}^1$ but also as surfaces. On the other
 hand all of them except $\mathbb{F}_1$ are relatively minimal models-
 they have no curves $C$ with $(C^2)=-1$. Indeed, $\mathbb{F}_1$ has
 an exceptional curve of the first kind, namely $S_o$. From the
 Castelnuovo theorem proved in lecture \ref{chap6} it follows that $S_o$ on
 $\mathbb{F}_1$ can be contracted to a point. It is easy to verify
 directly that the resulting surfaces is the projective plane
 $\mathbb{P}^2$. Thus, we have obtained a series of surfaces.   
  $$
 \mathbb{P}^2 ,\mathbb{P}^1 \times \mathbb{P}^1 (\simeq
 \mathbb{F}_o),\mathbb{F}_n (n >1) 
 $$   
 and have also' shown that all of them are relatively minimal models
 of $\mathbb{P}^2$\pageoriginale and are non-isomorphic to one
 another. It can be proved that these are the only relatively minimal
 models of $\mathbb{P}^2$. The proof however is much complicated. The
 reader is referred to Nagata (\cite{key18}).  
