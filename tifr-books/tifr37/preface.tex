\thispagestyle{empty}
\begin{center}
{\Large\bf Lectures on Minimal Models And}\\[5pt]
{\Large\bf Birational Transformations of}\\[5pt]
{\Large\bf Two Dimensional Schemes}
\vskip 1cm


{\bf By}
\medskip


{\large\bf I. R. Shafarevich}
\vfill

{\bf Tata Institute Of Fundamental Research, Bombay}


{\bf 1966}
\end{center}
\eject


\thispagestyle{empty}
\begin{center}
{\Large\bf Lectures on Minimal Models and}\\[5pt] 
{\Large\bf Birational Transformations}\\[5pt]
{\Large\bf of Two Dimensional Schemes}
\vskip 1cm


{\bf By}
\medskip

{\large\bf I.R. Shafarevich}
\vfill

{\bf Notes by}\\
\medskip

{\large\bf C.P. Ramanujam}
\vfill

\parbox{0.7\textwidth}{No part of this book may be reproduced
in any form by print, microfilm or any
other means without written permission
from the Tata Institute of Fundamental
Research, Colaba, Bombay 5}
\vfill


{\bf Tata Institute of Fundamental Research, Bombay}
\medskip

{\bf 1966}
\end{center}
\eject


\chapter{Preface}


These lectures contain an exposition of fundamental concepts and
results of the theory of birational transformations and minimal models
for schemes of dimension two. In the case of surfaces over an
algebraically closed field of characteristic zero, these results were
obtained by old Italian geometers. In the case of fields of arbitrary
characteristic, they are due to Zariski (cf. his book on minimal
models). Later Neron observed the importance of obtaining such results
in the case of schemes of dimension $2$; for certain questions of
number theory. In particular, he proved the existence of absolute
minimal models for two dimensional schemes over rings of integers of
global fields in the case where the genus of the generic fibre is~1. 

The first aim of these lectures was to give a proof of the
corresponding result in the case of arbitrary genus $g \geq 1$. This
proof is quite short and is contained in Lecture 7. (Curiously the
proof is much simpler for  schemes over a one dimensional base than
for surfaces over a field). However, in the proof, one has to use
certain results on  the geometry of two dimensional schemes that were
proved only for surfaces. The correctness of these results in the
general situation is more or less obvious but as the proofs were never
written down, I preferred to give an exposition of the subject in the
required generality starting from the beginning. This is why the
lectures grew so Long.   

For general properties of schemes used in the lectures, the reader is
referred to ``Elements'' of Grothendieck. Of course, it does not mean
that the knowledge of the whole treatise is assumed. For example, a
general survey given at the beginning of Mumford's  ``Lectures on
curves on an algebraic surface'' would be sufficient (with isolated
exceptions).  

The notes of these lectures were taken and the manuscript was prepared
by $C.P$. Ramanujam. I want to thank him here for the splendid job he
has done. He has not only corrected several mistakes but also
complemented proofs of many results that were only stated in  the
oral exposition. To mention some of them, he has written the proofs of
the Castelnuovo theorem in lecture 6, of the ``chain condition'' in
lecture 7, the example of Nagata of a non-projective surface in
lecture 6, the proof of Zariski's theorem in lecture~6. 

I also thank A. Bialynicki-Birual and J. Manin who have read the
manuscript and made many useful remarks.  
\bigskip

\hfill{\large\bf I.R. Shafarevich}
