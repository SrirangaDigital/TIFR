\chapter{Kolmogorov's Theorem}\label{chap2}

\begin{defi*}
LET\pageoriginale $(\Omega,\mathscr{B},P)$ BE A probability space. A
stochastic process in $\mathbb{R}^{d}$ is a collection $\{X_{t}:t\in
I\}$ of $\mathbb{R}^{d}$-valued random variables defined on $(\Omega,\mathscr{B})$.
\end{defi*}

\begin{note}\label{chap2-note1}
I will always denote a subset of $\mathbb{R}^{+}=[0,\infty)$. 

\smallskip
2.~ $X_{t}$ is also denoted by $X(t)$.

Let $\{X_{t}:t\in I\}$ be a stochastic process. For any collection
$t_{1}$, $t_{2},\ldots,t_{k}$ such that $t_{i}\in I$ and $0\leq
t_{1}<t_{2}<\ldots <t_{k}$ and any Borel set $\Lambda$, in
$\mathbb{R}^{d}\times \mathbb{R}^{d}\times \cdots \times \mathbb{R}^{d}$
($k$ times),.\@ define
$$
F_{t_{1}}\ldots t_{k}(\Lambda)=P(w\in
\Omega:(X_{t_{1}}(w),\ldots,X_{t_{k}}(w))\in \Lambda).
$$

If
$$
\{t_{1},\ldots,t_{k}\}\subset \{s_{1},\ldots,s_{\ell}\}\subset
I,\quad\text{with}\quad l\geq k
$$
such that
$$
s^{(0)}_{1}<\ldots<s^{(0)}_{n_{0}}<t_{1}<s^{(1)}_{1}\ldots
<s^{(1)}_{n}<t_{2}\ldots <t_{k}<s^{(k)}_{1}\ldots <s^{(k)}_{n_{k}},
$$
let then
$$
\pi:\mathbb{R}^{d}\times\cdots\times \mathbb{R}^{d}(1\text{~
  times})\to \mathbb{R}^{d}\times\cdots\times \mathbb{R}^{d}(k\text{~
  times})
$$
be the canonical projection. If $E_{t_{i}}\subset \mathbb{R}^{d}$ is
any Borel set in $\mathbb{R}^{d}$, $i=1,2,\ldots,k$, then
$$
\pi^{-1}(E_{t_{1}}\times\cdots\times
E_{t_{k}})=\mathbb{R}^{d}\times\cdots\times E_{t_{1}}\times
\mathbb{R}^{d}\times\cdots\times E_{t_{2}}\times\cdots\times
\mathbb{R}^{d}
$$
($l$ times). The following condition always holds.
\begin{equation*}
E_{t_{1}}\ldots t_{k}(E_{t_{1}}\times\cdots\times
E_{t_{k}})=F_{s_{1}}\ldots s_{1}(\Pi^{-1}(E_{t_{1}}\times\cdots\times E_{t_{k}})).\tag{*}
\end{equation*}

If\pageoriginale $(*)$ holds for an arbitrary collection
$\{F_{t_{1}}\ldots t_{k}: 0\leq t_{1}<t_{2}\ldots <t_{k}\}$
$(k=1,2,3\ldots)$ of distributions then it is said to satisfy
the {\em consistency condition.}
\end{note}

\setcounter{exercise}{0}
\begin{exercise}\label{chap2-exer1}
\begin{enumerate}
\renewcommand{\theenumi}{\alph{enumi}}
\renewcommand{\labelenumi}{(\theenumi)}
\item Verify that $F_{t_{1}}\ldots t_{k}$ is a probability measure on
  $\mathbb{R}^{d}\times\cdots\times \mathbb{R}^{d}$ ($k$ times).

\item Verify $(*)$. (If $B_{m}$ denotes the Borel $\sigma$ field of
  $\mathbb{R}^{m}$, $B_{m+n}=B_{m}\times B_{n}$).
\end{enumerate}
\end{exercise}

The following theorem is a converse of Exercise \ref{chap2-exer1} and
is often used to identify a stochastic process with a family of
distributions satisfying the consistency condition.

\medskip
\noindent
{\bf Kolmogorov's Theorem.}
\smallskip

{\em Let $\{F_{t_{1},t_{2},\ldots t_{k}}0\leq t_{1}<t_{2}<\ldots
  <t_{k}<\infty\}$ be a family of probability distributions (on
  $\mathbb{R}^{d}\times\cdots \times \mathbb{R}^{d}$, $k$ times,
  $k=1,2,\ldots$) satisfying the consistency condition. Then there
  exists a measurable space $(\Omega_{k},\mathbb{B})$, a unique
  probability measure $P$ an $(\Omega_{k},\mathscr{B})$ and a
  stochastic process $\{X_{t}:0\leq t<\infty\}$ such that the family
  of probability measures associated with it is precisely}
$$
\{F_{t_{1},t_{2},\ldots t_{k}}:0\leq t_{1}<t_{2}<\ldots
<t_{k}<\infty\},\quad k=1,2,\ldots.
$$

A proof can be found in the APPENDIX. We mention a few points about
the proof which prove to be very useful and should be observed
carefully.

\begin{enumerate}
\item The\pageoriginale space $\Omega_{k}$ is the set of all
  $\mathbb{R}^{d}$-valued functions defined on $[0,\infty)$:
$$
\Omega_{K}=\prod_{t\in [0,\infty)}\mathbb{R}^{d}
$$

\item The random variable $X_{t}$ is the $t^{\text{th}}$-projection of
  $\Omega_{K}$ onto $\mathbb{R}^{d}_{t}$.

\item $\mathscr{B}$ is the smallest $\sigma$-algebra with respect to
  which all the projections are measurable.

\item $P$ given by
$$
P(w:X_{t_{1}}(w)\in A_{1},\ldots X_{t_{k}}(w)\in A_{k})=F_{t_{1}\ldots
  t_{k}}(A_{1}\times\cdots\times A_{k})
$$
where $A_{i}$ is a Borel set in $\mathbb{R}^{d}$, is a measure on the
algebra generated by $\{X_{t_{1}},\ldots X_{t_{k}}\}(k=1,2,3\ldots)$
and extends uniquely to $\mathscr{B}$.
\end{enumerate}

\begin{remark*}
Although the proof of Kolmogorov's theorem is very constructive the
space $\Omega_{K}$ is too ``large'' and the $\sigma$-algebra
$\mathscr{B}$ too ``small'' for practical purposes. In applications
one needs a ``nice'' collection of $\mathbb{R}^{d}$-valued functions
(for example continuous, or differentiable functions), a ``large''
$\sigma$-algebra on this collection and a probability measure
concentrated on this family.
\end{remark*}



