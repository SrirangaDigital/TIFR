\chapter{Dirichlet Problem and Brownian Motion}\label{chap10}

LET\pageoriginale $G$ BE ANY bounded open set in
$\mathbb{R}^{d}$. Define the exit time $\tau_{G}(w)$ as follows:
$$
\tau_{G}(w)=\{\inf t:w(t)\not\in G\}.
$$

If $w(0)\in G$, $\tau_{G}(w)=0$; if $w(0)\in G$, $\tau_{G}(w)$ is the
first time $w$ escapes $G$ or, equivalently, it is the first time that
$w$ hits the boundary $\p G$ of $G$. Clearly $\tau_{G}(w)$ is a
stopping time. By definition $X_{\tau_{G}}(w)\in \p G$, $\forall w$
and $X_{\tau_{G}}$ is a random variable. We can define a Borel
probability measure on $\p G$ by
\begin{align*}
\pi_{G}(x,\Gamma) &= P_{x}(X_{\tau_{G}}\in\Gamma)\\
&= \text{probability that~ $w$ hits $\Gamma$.}
\end{align*}

If $f$ is a bounded, real-valued measurable funciton defined on $\p
G$, we define
$$
u(x)=E_{x}(f(X_{\tau_{G}}))=\int\limits_{\p G}f(y)\pi_{G}(x,dy)
$$
where 
$$
E_{x}=E^{P_{x}}.
$$

In case $G$ is a sphere centred around $x$, the exact form of
$\pi_{G}(x,\Gamma)$ is computable.

\begin{theorem*}
Let $S=S(0;r)=\{y\in \mathbb{R}^{d}:|y|<r\}$. Then
$$
\pi_{S}(0,r)=\frac{\text{surface area of~} \Gamma}{\text{surface area
    of~} S}.
$$
\end{theorem*}

\begin{proof}
The\pageoriginale distributions $\{F_{t_{1},\ldots,t_{k}}\}$ defining Brownian
motion are invariant under rotations. Thus $\pi_{S}(0,\cdot)$ is a
rotationally invariant probability measure. The result follows from
the fact that the only probability measure (on the surface of a
sphere) that is invariant under rotations is the normalised surface area.
\end{proof}

\begin{theorem*}
Let $G$ be any bounded region, $f$ a bounded measurable real valued
function defined on $\p G$. Define $u(x)=E_{x}(f(X_{\tau_{G}}))$. Then
\begin{itemize}
\item[\rm(i)] $u$ is measurable and bounded;

\item[\rm(ii)] $u$ has the mean value property; consequently,

\item[\rm(iii)] $u$ is harmonic in $G$.
\end{itemize}
\end{theorem*}

\begin{proof}
\begin{itemize}
\item[(i)] To prove this, it is enough to show that the mapping $x\to
  P_{x}(A)$ is measurable for every Borel set $A$.

Let $\mathscr{C}=\{A\in\mathscr{B}:x\to P_{x}(A)$ is measurable$\}$

It is clear that $\pi^{-1}_{t_{1},\ldots,t_{k}}(B)\in\mathscr{C}$,
$\forall$ Borel set $B$ in $\mathbb{R}^{d}\times\cdots\times
\mathbb{R}^{d}$. As $\mathscr{C}$ is a monotone class
$\mathscr{C}=\mathscr{B}$.

\item[(ii)] Let $S$ be any sphere with centre at $x$, and $S\subset
  G$. Let $\tau=\tau_{S}$ denote the exit time through $S$. Clearly
  $\tau\leq \tau_{G}$. By the strong Markov property,
$$
u(X_{\tau})=E(f(X_{\tau_{G}})|\mathscr{F}_{\tau}).
$$

Now 
\begin{align*}
u(x) &=
E_{x}(f(X_{\tau_{G}}))=E_{x}(E(f(X_{\tau_{G}}))|\mathscr{F}_{\tau})\\
&= E_{x}(u(X_{\tau}))=\int\limits_{\p S}u(y)\pi_{S}(x,dy)\\
&= \frac{1}{|S|}\int\limits_{\p S}u(y)dS;\ |S|=\text{~ surface area
  of~ } S.
\end{align*}\pageoriginale

\item[(iii)] is a consequence of (i) and (ii). (See exercise below).
\end{itemize}
\end{proof}

\begin{dashexer*}
Let $u$ be a bounded measurable function in a region $G$ satisfying
the mean value property, i.e.
$$
u(x)=\frac{1}{|S|}\int\limits_{\p S}u(y)dS
$$
for every sphere $S$ $G$. Then
\begin{itemize}
\item[(i)] $u(x)=\dfrac{1}{v\in lS}\int\limits_{S}u(y)dy$.

\item[(ii)] Using (i) show that $u$ is continuous.

\item[(iii)] Using (i) and (ii) show that $u$ is harmonic.
\end{itemize}

We shall now solve the boundary value problem under suitable
conditions on the region $G$.
\end{dashexer*}

\begin{theorem*}
Let $G$, $f$, $u$ be as in the previous theorem. Further suppose that
\begin{itemize}
\item[\rm(i)] $f$ is continuous;

\item[\rm(ii)] $G$ satisfies the exterior cone condition at every
  point of $\p G$, i.e.\@ for each $y\in \p G$ there exists $a$ cone
  $C_{h}$ with vertex at the point $y$ of height $h$ and such that
  $C_{h}-\{y\}\subset$ exterior of $G$. Then
$$
\lim\limits_{x\to y,x\in G}u(x)=f(y),\ \forall y\in \p G.
$$\pageoriginale
\end{itemize}
\end{theorem*}

\begin{proof}
\setcounter{step}{0}
\begin{step}%1
$P_{y}\{w:w(0)=y,w$ remains in $\overline{G}$ for some positive
  time$\}=0$.

Let $A_{n}=\{w:w(0)=y,w(s)\in \overline{G}$ for $0\leq s\leq 1/n\}$,

$B_{n}=\Omega-A_{n}$, $A=\bigcup\limits^{\infty}_{n=1}A_{n}$,
$B=\bigcap\limits^{\infty}_{n=1}B_{n}$. 
\end{step}

As $A_{n}$'s are increasing, $B_{n}$'s are decreasing and $B_{n}\in
\mathscr{F}_{1/n}$; so that $B\in \mathscr{F}_{0+}$. We show that
$P(B)>0$, so that by Bluementhal's zero-one law, $P(B)=1$, i.e.\@
$P(A)=0$.
$$
P_{y}(B)=\lim\limits_{n\to \infty}P_{y}(B_{n})\geq
\varlimsup\limits_{n\to \infty}P_{y}\{w:w(0)=y,w(\frac{1}{2n})\in C_{h}-\{y\}\}
$$

Thus
\begin{gather*}
P_{y}(b)\geq \varlimsup
\int\limits_{C_{h}-\{y\}}1/\surd(2\pi/2n)^{d}\exp
(-|z-y|^{2}/2/2n)dz\\
=\int\limits_{C_{\infty}}1/\surd(2\pi)e^{-|y|^{2}/2}dy,
\end{gather*}
where $C_{\infty}$ is the cone of infinite height obtained from
$C_{h}$. Thus $P_{y}(B)>0$.

\begin{step}%2
If $C$ is closed then the mapping $x\to P_{x}(C)$ is upper semi-continuous.
\end{step}

For, denote by $X_{C}$ the indicator function of $C$. As $C$ is closed
(in a metric space) $\exists$ a sequence of continuous functions
$f_{n}$ decreasing to $X_{C}$ such that $0\leq f_{n}\leq 1$. Thus
$E_{x}(f_{n})$ decreases to $E_{x}(X_{C})=P_{x}(C)$. Clearly $x\to
E_{x}(F_{n})$ is continuous. The result follows from the fact that the
infimum of any collection of continuous functions is upper
semi-continuous.

\begin{step}%3
Let\pageoriginale $\delta>0$,
\begin{gather*}
N(y;\delta)=\{z\in \p G:|z-y|<\delta\},\\
B_{\delta}=\{w:w(0)\in G, X_{\tau_{G}}(w)\in \p G-N(y;\delta)\},
\end{gather*}
i.e.\@ $B_{\delta}$ consists of trajectories which start at a point of
$G$ and escape for the first time through $\p G$ at a point not in
$N(y;\delta)$. If $C_{\delta}=\overline{B}_{\delta}$, then
$$
C_{\delta}\cap\{w:w(0)=y\}\subset A\cap \{w:w(0)=y\}
$$
where $A$ is as in Step 1.
\end{step}

For, suppose $w\in C_{\delta}\cap \{w:w(0)=y\}$. Then there exists
$w_{n}\in B_{\delta}$ such that $w_{n}\to w$ uniformaly on compact
sets. If $w\not\in A\cap \{w:w(0)=y\}$ there exists $\epsilon>0$ such
that $w(t)\in \overline{G}\,\forall t$ in $(0,\epsilon]$. Let
  $\delta^{*}=\inf\limits_{0\leq t\leq
    \epsilon}d(w(t),G-N(y,\delta))$. Then $\delta^{*}>0$. If
  $t_{n}=\tau_{G}(w_{n})$ and $t_{n}$ does not converge to $0$, then
  there exists a subsequence, again denoted by $t_{n}$, such that
  $t_{n}\geq k\epsilon>0$ for some $k\in (0,1)$. Since
  $w_{n}(k\epsilon)\in \overline{G}$ and $w_{n}(k\epsilon)$,
  $w(k\epsilon)\in \overline{G}$, a contradiction. Thus we can assume
  that $t_{n}$ converges to $0$ and also that $\epsilon\geq
  t_{n}\forall n$, But then
\begin{equation*}
|w_{n}(t_{n})-w(t_{n})|\geq \delta^{*}.\tag{*}
\end{equation*}

However, as $w_{n}$ converges to $w$ uniformly on $[0,\epsilon]$, 
$$
w_{n}(t_{n})-w(t_{n})\to w(0)-w(0)=0
$$
contradicting (*). Thus $w\in A\{w:w(0)=y\}$.

\begin{step}%4
$\lim\limits_{x\to y,x\in G}P_{x}(B_{\delta})=0$.

For,\pageoriginale
\begin{align*}
\varlimsup\limits_{x\to y}P_{x}(B_{\delta}) &\leq
\varlimsup\limits_{x\to y}P_{x}(C_{\delta})\leq P_{y}(C_{\delta})\q
\text{(by Step 2)}\\
&= P_{y}(C_{\delta}\cap \{w:w(0)=y\})\\
&\leq P_{y}(A)\q\text{(by Step 3)}\\
&= 0.
\end{align*}
\end{step}

\begin{step}%5
\begin{align*}
& |u(x)-f(y)|=|\int\limits_{\Omega}f(X_{\tau_{G}}(w))dP_{x}(w)-\int\limits_{\Omega}f(y)dP_{x}(w)|\\
&\qq \leq
  \int\limits_{\Omega-B_{\delta}}|f(X_{\tau_{G}}(w))-f(y)|dP_{x}(w)+|\int\limits_{B_{\delta}}(f(X_{\tau_{G}}(w))-f(y))dP_{x}(w)|\\
&\qq \leq \int\limits_{\Omega-B_{\delta}}|f(X_{\tau_{G}}(w))-f(y)|dP_{x}(w)+2||f||_{\infty}P_{x}(B_{\delta})
\end{align*}
and the right hand side converges to $0$ as $x\to y$ (by Step $4$ and
the fact that $f$ is continuous). This proves the theorem.
\end{step}
\end{proof}

\begin{remark*}
The theorem is local.
\end{remark*}

\begin{theorem*}
Let $G=\{y\in \mathbb{R}^{d}:\delta<|y|<R\}$, $f$ any continuous
function on $\p G=\{|y|=\delta\}\cap \{|y|=R\}$. If $u$ is any harmonic
function in $G$ with boundary values $f$, then
$u(x)=E_{x}(f(X_{\tau_{G}}))$. 
\end{theorem*}

\begin{proof}
Clearly $G$ has the exterior cone property. Thus, if
$$
v(x)=E_{x}(f(X_{\tau_{G}})), 
$$
then $v$ is harmonic in $G$ and has
boundary values $f$ (by the previous theorem). The result follows from
the uniqueness of the solution of the Dirichlet problem for the
Laplacian operator.

The function $f=0$ on $|y|=R$ and $f=1$ on $|y|=\delta$ is of special
interest. Denote by $\cup^{R,0}_{\delta,1}$ the corresponding solution
of the Dirichlet problem.
\end{proof}

\begin{exer*}
\begin{itemize}
\item[\rm(i)] If\pageoriginale $d=2$ then
$$
U^{R,0}_{\delta,1}(x)=\frac{\log R-\log |x|}{\log R-\log
  \delta},\ \forall x\in G.
$$

\item[(ii)] If $d\geq 3$ then
$$
U^{R,0}_{\delta,1}(x)=\frac{|x|^{-n+2}-R^{-n+2}}{\delta^{-n+2}-R^{-n+2}}. 
$$
\end{itemize}

\noindent
{\bf Case (i):}~ $d=2$. Then
$$
\frac{\log R-\log |x|}{\log R-\log \delta}=U^{R,0}_{\delta,1}(x).
$$

Now,
$$
E_{x}(f(X_{\tau_{G}}))=\int\limits_{|y|=\delta}\pi_{G}(x,dy)=P_{x}(|X_{\tau_{G}}|=\delta), 
$$
i.e.
\begin{align*}
& \frac{\log R-\log|x|}{\log R-\log
    \delta}=P_{x}(|X_{\tau_{G}}|=\delta)\\
& P_{x}\q (\text{the particle hits~ }|y|=\delta\text{~ before it hits~ }|y|=R).
\end{align*}

Fix $R$ and let $\delta\to 0$; then
$0=P_{x}$ (the particle hits $0$ before hitting $|y|=R$).

Let $R$ take values $1,2,3,\ldots$, then $0=P_{x}$ (the particle hits
$0$ before hitting any of the circles $|y|=N$). Recalling that
$$
P_{x}(\varlimsup |X_{t}|=\infty)=1,
$$
we get
\end{exer*}

\begin{prop*}
A two-dimensional Brownian motion does not visit a\break point.

Next, keep $\delta$ fixed and let $R\to \infty$, then,
$$
1=P_{x}(|w(t)|=\delta\q\text{for some time}\q t>0).
$$

Since\pageoriginale any time $t$ can be taken as the starting time for
the Brownian motion, we have
\end{prop*}

\begin{prop*}
Two-dimensional Brownian motion has the recurrence\break pro\-perty.
\end{prop*}

\noindent
{\bf Case (ii):}~ $d\geq 3$. In this case
\begin{align*}
& P_{x}(w:w\text{~ hits~ }|y|=\delta\text{~ before it hits~ }|y|=R)\\
&=(1/|x|^{n-2}-1/R^{n-2})/(1/\delta^{n-2}-1/R^{n-2}). 
\end{align*}

Letting $R\to \infty$ we get
$$
P_{x}(w:w\text{~ hits~ }|y|=\delta)=(\delta/|x|)^{n-2}
$$
which lies strictly between $0$ and $1$. Fixing $\delta$ and letting
$|x|\to\infty$, we have

\begin{prop*}
If the particle start at a point for away from $0$ then it has very
little chance of hitting the circle $|y|=\delta$.

If $|x|\leq \delta$, then
$$
P(w\text{~ hits~ } S_{\delta})=1\text{~ where~ }S_{\delta}=\{y\in
R^{d}:|y|=\delta\}.
$$

Let
$$
V_{\delta}(x)=(\delta/|x|)^{n-2}\text{~ for~ } |x|\geq \delta.
$$

In view of the above result it is natural to extend $V_{\delta}$ to
all space by putting $V_{\delta}(x)=1$ for $|x|\leq \delta$. As
Brownian motion has the Markov property
\begin{gather*}
P_{x}\{w:w\text{~ hits~ }S_{\delta}\text{~ after time~ }t\}\\
=\int V_{\delta}(y)1/\surd(2\pi t)^{d}\exp -|y|^{2}/2t\ dy\to 0\text{~
  as~ }t\to +\infty.
\end{gather*}
\end{prop*}

Thus\pageoriginale $P(w$ hits $S_{\delta}$ for arbitrarily large
$t)=0$. In other words, $P(w:\varliminf\limits_{t\to \infty}|w(t)|\geq
\delta)=1$. As this is true $\forall \delta>0$, we get the following
important result.

\begin{prop*}
$P(\varliminf\limits_{t\to \infty}|w(t)|=\infty)=1$,

i.e.\@ for $d\geq 3$, the Brownian particle wander away to $+\infty$.
\end{prop*}


