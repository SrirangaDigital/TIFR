\chapter{Solution of Poisson's Equations}\label{chap17}

LET\pageoriginale $X(t,\cdot)$ BE A $d$-dimensional Brownian motion
with $(\Omega,\mathscr{F}_{t},P)$ as usual. Let
$u(x):\mathbb{R}^{d}\to \mathbb{R}$ be such that $\dfrac{1}{2}\Delta
u=f$. Assume $u\in C^{2}_{b}(\mathbb{R}^{d})$. Then $L_{s,w}u\equiv
\dfrac{1}{2}\Delta u=f$ and we know that
$u(X(t,\cdot))-\int\limits^{t}_{0}f(X(s,\cdot))ds$ is a
$(\Omega,\mathscr{F}_{t},P)$-martingale. Suppose now that $u(x)$ is
defined only on an open subset $G\subset \mathbb{R}^{d}$ and
$\dfrac{1}{2}\Delta u=f$ on $G$. We would like to consider
$$
Z(t,\cdot)=u(X(t,\cdot))-\int\limits^{t}_{0}f(X(s,\cdot))ds
$$
and ask whether $Z(t,\cdot)$ is still a martingale relative to
$(\Omega,\mathscr{F}_{t},P)$. Let $\tau(w)=\inf\{t,X(t,w)\in \p
G\}$. Put this way, the question is not well-posed because
$Z(t,\cdot)$ is defined only upto time $\tau(w)$ for $u$ is not
defined outside $G$. Even if at a time $t>\tau(w)X(t,w)\in G$, one
needs to know the values of $f$ for $t>\tau(w)$ to compute the
integral.

To answer the question we therefore proceed as follows. Let
$A_{t}=[w:\tau(w)>t]$. As $t$ increases, $A_{t}$ have decreasing
measures. We shall give a meaning to the statement `$Z(t,\cdot)$ is a
martingale on $A_{t}$'. Define
$$
\overline{Z}(t,\cdot)=u(X(\tau\wedge t,\cdot))-\int\limits^{\tau\wedge
  t}_{0}f(X(s,\cdot))ds.
$$

Therefore
$$
\overline{Z}(t,\cdot)
=
\begin{cases}
Z(t), & \text{on~ }A_{t}\\
Z(\tau,\cdot), &\text{on~ } (A_{t})^{c}.
\end{cases}
$$

Since\pageoriginale $Z(t,\cdot)$ is progressively measurable upto time
$\tau$, $\overline{Z}(t,\cdot)$ is $\mathscr{F}_{t}$-measurable.

\begin{theorem*}
$\overline{Z}(t,\cdot)$ is a martingale.
\end{theorem*}

\begin{proof}
Let $G_{n}$ be a sequence of compact sets increasing to $G$ such that
$G_{n}\subset G^{0}_{n+1}$. Choose a $C^{\infty}$ function $\phi_{n}$
such that $\phi_{n}=1$ on $G_{n}$ and support $\phi_{n}\subset G$. Put
$u_{n}=\phi_{n}u$ and $f_{n}=\dfrac{1}{2}\Delta u_{n}$. Then
$$
Z_{n}(t,\cdot)=u_{n}(X(t,\cdot))-\int\limits^{t}_{0}f_{n}(X(s,\cdot))ds
$$
is a martingale for each $n$. Put
$$
\tau_{n}=\inf \{t:X(t,\cdot)\not\in G_{n}\}
$$

Then $Z_{n}(\tau_{n}\wedge t,\cdot)$ is also a martingale (See
exercise below). But 
$$
Z_{n}(\tau_{n}\wedge t)=Z(\tau_{n}\wedge t).
$$

Therefore $M_{n}(t,\cdot)=Z(\tau_{n}\wedge t,\cdot)$ is a
martingale. Observe that $\tau_{n}\leq \tau_{n+1}$ and since
$G_{n}\shortuparrow G$ we have $\tau_{n}\shortuparrow \tau$. Therefore
$Z(\tau_{n}\wedge t)\to Z(\tau\wedge t)$ (by continuity); also
$|M_{n}(t,\cdot)|\leq ||u||_{\infty}+||f||_{\infty}t$. Therefore
$Z(\tau\wedge t)=\overline{Z}(t,\cdot)$ is a martingale.
\end{proof}

\begin{exer*}
If $M(t,\cdot)$ is a $(\Omega,\mathscr{F}_{t},P)$-martingale, show
that for my stopping time $\tau$, $M(\tau\wedge t,\cdot)$ is also a
martingale relative to $(\mathscr{F}_{t})$. 

\noindent
[Hint: One has to show that if $t_{2}>t_{1}$,
$$
\int\limits_{A}M(\tau\wedge t_{2},w)dP(w)=\int\limits_{A}M(\tau\wedge
t_{1},w)dP(w),\forall A\in \mathscr{F}_{t_{1}}.
$$

The right side =
$$
\int\limits_{A\cap (\tau>t_{1})}M(t_{1},w)dP(w)+\int\limits_{A\cap
  (\tau<t_{1})}M(\tau,w)dP(w). 
$$

The\pageoriginale left side
$$
=\int\limits_{A\cap (\tau>t_{1})}M(t_{2},w)dP(w)+\int\limits_{A\cap
  (\tau<t_{1})}M(\tau,w)dP(w). 
$$

Now use optional stopping theorem].
\end{exer*}

\begin{lemma*}
Let $G$ be a bounded region and $\tau$ be as above. Then
$E_{x}(\tau)<\infty$, $\forall x\in G$, where $E_{x}=E^{P_{x}}$.
\end{lemma*}

\begin{proof}
Without loss of generality we assume that $G$ is a sphere of radius
$R$. The function $u(x)=\dfrac{R^{2}-|x|^{2}}{d}\geq 0$ and satisfies
$\dfrac{1}{2}\Delta u=-1$ in $G$. By the previous theorem
$$
u(X(\tau\wedge t,\cdot)+\int\limits^{\tau\wedge t}_{0}ds
$$
is a martingale. Therefore
$$
E_{x}(u(X(\tau\wedge t,\cdot)))+E_{x}(\tau\wedge t)=u(X(0))=u(x).
$$

Therefore $E_{x}(\tau\wedge t)\leq u(x)$ (since $u\geq 0$). By Fatou's
lemma, on letting $t\to \infty$, we obtain $E_{x}(\tau)\leq
u(x)<\infty$. Thus the mere existence of a $u$ satisfying
$\dfrac{1}{2}\Delta u=1$ helps in concluding that
$E_{x}(\tau)<\infty$. 
\end{proof}

\begin{theorem*}
Let $u\in C^{2}_{b}(G)$ and suppose that $u$ satisfies
\begin{align*}
\frac{1}{2}\Delta u &= f\text{~ in~ }G,\tag{*}\\
u &= g\text{~ on~ } \p G.
\end{align*}

Then $u(x)=E_{x}[g]-E_{x}[\int\limits^{\tau}_{0}f(X(s,\cdot))ds]$
solves (*).
\end{theorem*}

\begin{remark*}
The first part of the solutin $u(x)$ is the solution of the
homogeneous equation, and the second part accounts for the
inhomogeneous term.
\end{remark*}

\begin{proof}
Define\pageoriginale $\overline{Z}(t,\cdot)=u(X(\tau\wedge
t))-\int\limits^{\tau\wedge t}_{0}f(X(s,\cdot))ds$. Then
$\overline{Z}$ is a martingale. Also $|\overline{Z}|\leq
||u||_{\infty}+\tau||f||_{\infty}$. Therefore, by the previous Lemma,
$\overline{Z}(t,\cdot)$ is a uniformly integrable
martingale. Therefore we can equate the expectations at time $t=0$ and
at time $t=\infty$ to get
$$
u(x)=E_{x}(g)-E_{x}[\int\limits^{\tau}_{0}f(X(s,\cdot))ds].
$$
\end{proof}


