\chapter{Brownian Motion with Drift}\label{chap20}

LET\pageoriginale $\Omega=C[0,\infty;\mathbb{R}^{d}]$,
$\mathscr{F}=\text{BOREL}$ $\sigma$-FIELD of $\Omega$,
$\{X(t,\cdot)\}\equiv$ Brownian motion,
$\mathscr{F}_{t}=\sigma[X(s,\cdot):0\leq s\leq t]$, $P_{x}\equiv$
probability measure on $\Omega$ corresponding to the Brownian motion
starting at time $0$ at
$x$. $\mathscr{F}=\sigma({\displaystyle{\mathop{U}_{t\geq
      0}}}\mathscr{F}_{t})$. Let $b:\mathbb{R}^{d}\to \mathbb{R}^{d}$
be any bounded measurable funciton. Then the map $(s,w)|\to b(w(s))$
is progressively measurable and 
$$
Z(t,\cdot)=\exp[\int\limits^{t}_{0}\langle
  b(X(s,\cdot)),dX(s,\cdot)\rangle-\frac{1}{2}\int\limits^{t}_{0}|b(X(s,\cdot))|^{2}ds]
$$
is a martingale relative to $(\Omega,\mathscr{F}_{t},P_{x})$. Define
$Q^{t}_{x}$ on $\mathscr{F}_{t}$ by
$$
Q^{t}_{x}(A)=\int\limits_{A}Z(t,\cdot) dP_{x},
$$
i.e.\@ $Z(t,\cdot)$ is the Radon-Nikodym derivative of $Q^{t}_{x}$
with respect to $P_{x}$ on $\mathscr{F}_{t}$.

\begin{prop*}
\begin{itemize}
\item[\rm(i)] $Q^{t}_{x}$ is a probability measure.

\item[\rm(ii)] $\{Q^{t}_{x}:t\geq 0\}$ is a consistent family on
  ${\displaystyle{\mathop{\cup}_{t\geq 0}}}\mathscr{F}_{t}$, i.e.\@ if
  $A\in \mathscr{F}_{t_{1}}$ and $t_{2}\geq t_{1}$ then
  $Q^{t}_{x}1(A)=Q^{t}_{x}2(A)$. 
\end{itemize}
\end{prop*}

\begin{proof}
$Q^{t}_{x}$ being an indefinite integral, is a measure. Since
  $Z(t,\cdot)\geq 0$, $Q^{t}_{x}$ is a positive
  measure. $Q^{t}_{x}(\Omega)=E_{x}(Z(t,\cdot))=E_{x}(Z(0,\cdot))=1$. This
  proves (i). (ii) follows from the fact that $Z(t,\cdot)$ is a
  martingale.

If $A\in \mathscr{F}_{t}$, we define
$$
Q_{x}(A)=Q^{t(A)}_{x}.
$$

The\pageoriginale above proposition shows that $Q_{x}$ is well defined
and since $(\mathscr{F}_{t})$ is an increasing family, $Q_{x}$ is
finitely additive on $\bigcup\limits_{t\geq
      0}\mathscr{F}_{t}$. 
\end{proof}

\begin{exer*}
Show that $Q_{x}$ is countably additive on the algebra
$\bigcup\limits_{t\geq 0}\mathscr{F}_{t}$.

Then $Q_{x}$ extends as a measure to
$\mathscr{F}=\sigma(\bigcup\limits_{t\geq
      0}\mathscr{F}_{t})$. Thus we get a family of measures
$\{Q_{x}:x\in \mathbb{R}^{d}\}$ defined on $(\Omega,\mathscr{F})$.
\end{exer*}

\begin{prop*}
If $s<t$ then
$$
Q_{x}(X_{t}\in A|\mathscr{F}_{s})=Q_{X(s)}(X(t-s)\in
A)\quad\text{a.e.}
$$
\end{prop*}

\begin{defi*}
If a family of measures $\{Q_{x}\}$ satisfies the above property it is
called a homogeneous Markov family.
\end{defi*}

\begin{proof}
Let $B\in \mathscr{F}_{s}$. Therefore $B\cap X^{-1}_{t}(A)\in
\mathscr{F}_{t}$ and by definition,
\begin{align*}
& Q_{x}((X(t)\in A)\cap B))=\int\limits_{B\cap
    X_{t}^{-1}(A)}Z(t,\cdot)dP_{x}\\
& E^{P_{x}}(Z(t,\cdot)\chi^{(w)}_{B}\chi_{A}(X(t,\cdot)))\\
&=
  E^{P_{x}}(E^{P_{x}}(\frac{Z(t,\cdot)}{Z(s,\cdot)}Z(s,\cdot)\chi^{(w)}_{B}\chi_{A}(X(t,\cdot))|\mathscr{F}_{s})\\
&= E^{P_{x}}([\chi_{B}Z(s,\cdot))E^{P_{x}}(\frac{Z(t,\cdot)}{Z(s,\cdot)}\chi_{A}(\chi(t,\cdot))]|\mathscr{F}_{s})\\
\end{align*}
(since $B\in\mathscr{F}_{s}$ and $Z(s,\cdot)$ is
$\mathscr{F}_{s}$-measurable)
\begin{align*}
&
 = E^{Q_{x}}[\chi_{B}E^{P_{x}}(\frac{Z(t,\cdot)}{Z(s,\cdot)}\chi_{A}(\chi(t,\cdot))|\mathscr{F}_{s})]\ldots\tag{1}\\
& = E^{Q_{x}}[\chi_{B}E^{P_{x}}(\exp[\int\limits^{t}_{s}\langle
      b,dX\rangle
      -\frac{1}{2}\int\limits^{t}_{0}|b|^{2}]\chi_{A}(X(t,\cdot))|\mathscr{F}_{s})]\\
&= E^{Q_{x}}[\chi_{B}E^{P_{X(s)}}(\exp [\int\limits^{t-s}_{0}\langle
     b,dX\rangle
     \frac{1}{2}\int\limits^{t-s}_{0}|b|^{2}]X_{A}(X(t-s))]\\
&\q \text{(by Markov property of Brownian motion)}\\
&=
 E^{Q_{x}}(\chi_{B}E^{Q^{t-s}_{X(s)}}(\chi_{A}(\chi(t-s)))\q(\text{since~
 }\frac{dQ^{t-s}_{X(s)}}{dP_{X(s)}}=Z(t-s,\cdot)\\
&= E^{Q_{x}}(X_{B}E^{Q_{X(s)}}\chi_{A}(X(t-s,\cdot))
\end{align*}\pageoriginale
\end{proof}

The result follows from definition.

Let $b:[0,\infty]\times\mathbb{R}^{d}\to \mathbb{R}^{d}$ be a bounded
measurable function, $P_{s,x}$ the probability measure corresponding
to the Brownian motion starting at time $s$ at the point $x$. Define,
for $t\geq s$,
\begin{align*}
Z_{s,t}(w) &= \exp[\int\limits^{t}_{s}\langle
  b(\sigma,X(\sigma,w)),dX(\sigma,w)\rangle\\
&\qq -\frac{1}{2}\int\limits^{t}_{s}|b(\sigma,X(\sigma,w))|^{2}d\sigma]
\end{align*}

\begin{exer*}
\begin{enumerate}
\renewcommand{\theenumi}{\roman{enumi}}
\renewcommand{\labelenumi}{(\theenumi)}
\item $Z_{s,t}$ is a martingale relative to $(\mathscr{F}^{s}_{t},
  P_{s,x})$. 

\item Define $Q^{t}_{s,x}$ by
  $Q^{t}_{s,x}(A)=\int\limits_{A}Z_{s,t}dP_{s,x}$, $\forall A\in
  \mathscr{F}^{s}_{t}$. 

Show that $Q^{t}_{s,x}$ is a probability measure on
$\mathscr{F}^{s}_{t}$.

\item $Q^{t}_{s,x}$ is a consistent family.

\item $Q_{s,x}$ defined on ${\displaystyle{\mathop{U}_{t\geq
      s}}}\mathscr{F}^{s}_{t}$ by
  $Q_{s,x}|\mathscr{F}^{s}_{t}=Q^{t}_{s,x}$ is a finitely additive set
  function which is countably additive.

\item The family $\{Q_{s,x}:0\leq s<\infty, x\in \mathbb{R}^{d}\}$ is
  an inhomogeneous Markov family, i.e.
$$
Q_{s,x}(X(t,\cdot)\in
A|\mathscr{F}^{s}_{\sigma})=Q_{\sigma,X(\sigma,\cdot)}(X(t,\cdot)\in
A), \forall s<\sigma<t,A\in \mathscr{F}^{s}_{t}.
$$
[Hint:\pageoriginale Repeat the arguments of the previous section with obvious
  modifications]. 
\end{enumerate}
\end{exer*}

\begin{prop*}
Let $\tau$ be a stopping time, $\tau\geq s$. Then
\begin{align*}
Q_{s,x}[X(t,\cdot)\in A|\mathscr{F}^{s}_{\tau}] &=
Q_{\tau,X_{\tau}(\cdot)}(X(t,\cdot)\in A)\text{~ on~ }\tau(w)<t,\\
&= \chi_{A}(X(t,\cdot))\text{~ on~ } \tau(w)\geq t.
\end{align*}
\end{prop*}

\begin{proof}
Let $B\in \mathscr{F}^{s}_{\tau}$ and $B\subset \{\tau<t\}$ so that
$B\in \mathscr{F}^{s}_{t}$.
\begin{align*}
E^{Q_{s,x}}(\chi_{B}\chi_{A}(X_{t})) &=
E^{P_{s,x}}(Z_{s,t}\chi_{B}\chi_{A}(X_{t}))\\
&=
E^{P_{s,x}}[E^{P_{s,x}}(Z_{\tau,t}Z_{s,\tau}\chi_{B}\chi_{A}(X_{t})|\mathscr{F}^{s}_{\tau})] 
\end{align*}
(since $Z$ satisfies the multiplicative property)
$$
=E^{P_{s,x}}(Z_{s,\tau}\chi_{B}E^{P_{s,x}}(Z_{\tau,t}\chi_{A}(X_{t})|\mathscr{F}^{s}_{\tau})]\\ 
$$
(since $Z_{s,\tau}$ is $\mathscr{F}^{s}_{\tau}$-measurable)
\begin{equation*}
=E^{P_{s,x}}[Z_{s,\tau}X_{B}E^{P_{\tau},X_{\tau}}(Z_{\tau,t}\chi_{A}(X_{t}))]\tag{*} 
\end{equation*}
(by strong Markov property).

Now 
$$
\frac{dQ_{s,x}}{dP_{s,x}}\Big|_{\mathscr{F}^{s}_{t}}=Z_{s,t}.
$$
so that the optional stopping theorem,
$$
\frac{dQ_{s,x}}{dP_{s,x}}\Big|_{\mathscr{F}^{s}_{\tau}}=Z_{s,}\q\text{on}\q
\{\tau<t\},\ \forall x.
$$

Putting\pageoriginale this in (*) we get
$$
E^{Q_{s,x}}[X_{B}\chi_{A}(X_{t})]=E^{P_{s,x}}[Z_{s,\tau}\chi_{B}E^{Q_{\tau,X_{\tau}}(\chi_{A}X_{t})}].
$$

Observe that
$$
\chi_{B}E^{Q_{\tau,X_{\tau}}}(\chi_{A}(X_{t}))
$$
is $\mathscr{F}^{s}_{\tau}$-measurable to conclude the first part of
the proof. For part (ii) observe that
$$
X^{-1}_{t}(A)\cap \{\tau\geq t\}\cap \{\tau\leq k\}
$$
is in $\mathscr{F}^{s}_{k}$ if $k>s$, so that
$$
X^{-1}_{t}(A)\cap \{\tau\geq t\}\in \mathscr{F}^{s}_{\tau}.
$$

Therefore
$$
E^{Q_{s,x}}(X_{t}\in A\cap(\tau\geq
t))|\mathscr{F}^{s}_{\tau})=\chi_{A}(X_{t})\chi_{\{\tau\geq t\}},
$$
or 
$$
E^{Q_{s,x}}[(X_{t}\in
  A)|\mathscr{F}^{s}_{\tau}]=\chi_{A}(X_{t})\q\text{if}\q \tau\geq t.
$$
\end{proof}

\begin{prop*}
Let $b:[0,\infty)\times \mathbb{R}^{d}\to \mathbb{R}^{d}$ be a
  bounded measurable function, $f:\mathbb{R}^{d}\to \mathbb{R}$ any
  continuous bounded function. If
\begin{gather*}
\frac{\p u}{\p s}+\frac{1}{2}\Delta u+\langle b(s,x),\Delta
u\rangle=0,\q 0\leq s\leq t,\\
u(t,x)=f(x)
\end{gather*}
has a solution $u$, then
$$
u(s,x)=\int\limits_{\Omega}f(X_{t})dQ_{s,x}.
$$
\end{prop*}

\begin{remark*}
$b$\pageoriginale is called the {\em drift}. If $b=0$ and $s=0$ then
  we recover the result obtained earlier. With the presence of the
  drift term, the result is the same except that instead of $P_{s,x}$
  one has to use $Q_{s,x}$ to evaluate the expectation.
\end{remark*}

\begin{proof}
Let
$$
Y(\sigma,\cdot)=\int\limits^{\sigma}_{s}\langle
b(\theta,X(\theta,\cdot)), dX(\theta,\cdot)\rangle
-\frac{1}{2}\int\limits^{\sigma}_{s}|b(\theta,X(\theta,\cdot)|^{2}d\theta.
$$

\setcounter{step}{0}%1
\begin{step}
$(X(\sigma,\cdot)-X(s,\cdot),Y(\sigma,\cdot))$ is a
  $(d+1)$-dimensional It\^o process with parameters
$$
(0,0,\ldots,0,-\frac{1}{2}|b(\sigma,X(\sigma,\cdot))|^{2})\q\text{and}
$$
$d$ terms
$$
a=
\begin{bmatrix}
I_{d\times d} & b_{d\times 1}\\
b^{\ast}_{1\times d} & b
\end{bmatrix}
$$

Let $\lambda =(\lambda_{1},\ldots,\lambda_{d})$. We have to show that
\begin{align*}
& \exp [\mu
  Y+\frac{\mu}{2}\int\limits^{\sigma}_{s}|b(\sigma,X(\sigma,\cdot)|^{2}d\sigma+\langle
  \lambda,X(\sigma,\cdot)-X(s,\cdot)\rangle -\\
&\qq -\frac{1}{2}\int\limits^{\sigma}_{s}\langle
  \lambda,\lambda\rangle +2\mu \langle\lambda,
  b+\mu^{2}|b(\sigma,\cdot)|^{2}d\sigma] 
\end{align*}
is a martingale, i.e.\@ that
$$
\exp[\langle \lambda,X(\sigma,\cdot)-X(s,\cdot)\rangle
  +\mu\int\limits^{\sigma}_{s}\langle b,dX\rangle
  -\frac{1}{2}\int\limits^{\sigma}_{s}|\lambda+b\mu|^{2}d\rho]. 
$$
is a martingale; in other words that
$$
\exp[\int\limits^{\sigma}_{s}\langle \lambda+b\mu, dX\rangle
  -\frac{1}{2}\int\limits^{\sigma}_{s}|\lambda+b\mu|^{2}d\rho]
$$\pageoriginale
is a martingale. But this is obvious because
$$
Z(\sigma,\cdot)=\int\limits^{\sigma}_{s}\langle \lambda +\mu
b,dX\rangle
$$
is a stochastic integral and hence an It\^o process with parameters
$(0,|\lambda +\mu b|^{2})$. (Refer to the section on vector-valued
It\^o process).
\end{step}

\begin{step}%2
Put
$\phi(\sigma,X(\sigma,\cdot),Y(\sigma,\cdot))=u(\sigma,X(\sigma,\cdot))e^{Y(\sigma,\cdot)}$. By
It\^o formula,
$$
d\phi=e^{Y}\frac{\p u}{\p t}dt+e^{Y}\langle \nabla u, dX\rangle
+ue^{Y}dY+\frac{1}{2}\sum a_{ij}\frac{\pi^{2}\phi}{\p z_{i}\p z_{j}},
$$
where $z=(x,y)$, or
\begin{gather*}
d\phi=e^{Y}[\frac{\p u}{\p t}dt+\langle \nabla u, dX\rangle + u\langle
  b,dX\rangle -\frac{\mu}{2}|b|^{2}dt+\frac{1}{2}\nabla udt+\langle
  b,\nabla u\rangle dt+\\
+\frac{1}{2}u|b|^{2}dt]\\
=e^{Y}[\langle \nabla u, dX\rangle +u\langle b,dX\rangle].
\end{gather*}

Therefore $\phi$ is an It\^o process and hence a martingale. Therefore
\begin{gather*}
E(\phi(t,\cdot))=E(\phi(s,\cdot))\\
u(s,x)=E^{P_{s,x}}[(f(X(t))e^{\int^{t}_{s}\langle
    b,dX\rangle-\frac{1}{2}\int\limits^{t}_{0}|b|^{2}d\theta}]\\
=E^{Q_{s,x}}[f(X(t))],
\end{gather*}
which proves the theorem.
\end{step}
\end{proof}

\noindent
{\bf\em Alternate Proof.}\pageoriginale

\begin{exer*}
Let $Y(\sigma,\cdot)$ be progressively measurable for $\sigma\geq
s$. Then $Y(\sigma,\cdot)$ is a martingale relative to
$(Q_{s,x},\mathscr{F}^{s}_{t})$ if and only if $Y(\sigma)Z_{s,\sigma}$
is a martingale relative to $(P_{s,x},\mathscr{F}^{s}_{t})$.

Now for any function $\theta$ which is progressively measurable and
boun\-ded,
$$
\exp[\int\limits^{t}_{s}\langle \theta,dX\rangle
  -\frac{1}{2}\int\limits^{t}_{s}|\theta|^{2}d\sigma]
$$
is a martingale relative to $(\Omega,\mathscr{F}^{s}_{t},P_{s,x})$. In
particular let $\theta$ be replaced by
$\theta+b(\sigma,w(\sigma))$. After some rearrangement one finds that
$X_{t}$ is an It\^o process with parameters $b$, $I$ relative to
$Q_{s,x}$. Therefore 
$$
u(t,X_{t})-\int\limits^{t}_{s}\left(\frac{\p u}{\p \sigma}+\langle b,\nabla
u\rangle +\frac{1}{2}\nabla u\right)d\sigma
$$
is a martingale relative to $Q_{s,x}$. But
$$
\frac{\p u}{\p \sigma}+\langle b,\nabla u\rangle+\frac{1}{2}\nabla
u=0.
$$

Therefore
$$
E^{Q_{s,x}}(u(t,X(t))=u(s,X).
$$

We have defined $Q_{s,x}$ by using the notion of the Radon-Nikodym
derivative. We give one more relation between $P$ and $Q$.
\end{exer*}

\begin{theorem*}
Let $T:C([s,\infty),\mathbb{R}^{d})\to C([s,\infty), \mathbb{R}^{d})$
    be given by 
$$
TX=Y\q\text{where}\q
Y(t)=X(t)-\int\limits^{t}_{s}b(\sigma,X(\sigma))d\sigma.
$$
($b$ is as before). Then
$$
Q_{s,x}T^{-1}=P_{s,x}.
$$
\end{theorem*}

\begin{proof}
Define\pageoriginale $Y(t,w)=X(t,Tw)$ where $X$ is a Brownian
motion. We prove that $Y$ is a Brownian motion with respect to
$Q_{s,x}$. Clearly $Y$ is progressively measurable because $T$ is
$(\mathscr{F}_{t}-\mathscr{F}_{t})$-measurable for every $t$, i.e.\@
$T^{-1}(\mathscr{F}_{t})\subset \mathscr{F}_{t}$ and $X$ is
progressively measurable. Clearly $Y(t,w)$ is continuous
$\forall\ w$. We have only to show that $Y(t_{2})-Y(t_{1})$ is
$Q_{s,x}$-independent of $\mathscr{F}^{s}_{t_{1}}$ and has
distribution $N(0;(t_{2}-t_{1})I)$ for each $t_{2}>t_{1}\geq s$. But
we have checked that
$$
\exp [\langle \theta,X_{t}-x\rangle
  -\frac{1}{2}|\theta|^{2}(t-s)-\int\limits^{t}_{s}\langle
  \theta,b\rangle d\sigma]
$$
is a martingale relative to $Q_{s,x}$. Therefore
$$
E^{Q_{s,x}}(\exp\langle \theta,Y_{t_{2}}-Y_{t_{1}}\rangle
|\mathscr{F}^{s}_{t_{1}})=\exp(\frac{1}{2}|\theta|^{2}(t_{2}-t_{1})), 
$$
showing that $Y_{t_{2}}-Y_{t_{1}}$ is independent of
$\mathscr{F}^{s}_{t_{1}}$ and has normal distribution
$N(0;(t_{2}-t_{1})I)$. Thus $Y$ is a Brownian motion relative to
$Q_{s,x}$. Therefore
$$
Q_{s,x}T^{-1}=P_{s,x}.
$$
\end{proof}

