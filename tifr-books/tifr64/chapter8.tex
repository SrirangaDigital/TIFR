\chapter{Blumenthal's Zero-One Law}\label{chap8}

LET\pageoriginale $X_{t}$ BE $A$ $d$-dimensional Brownian motion. If
$A\in \mathscr{F}_{0+}=\bigcap\limits_{t>0}\mathscr{F}_{t}$, then
$P(A)=0$ or $P(A)=1$.

\medskip
\noindent
{\bf Interpretation.}~ If an event is observable in every interval
$[0,t]$ of time then either it always happens or it never happens.

We shall need the following two lemmas.

\setcounter{lemma}{0}
\begin{lemma}\label{chap8-lem1}
Let $(\Omega,\mathscr{B},P)$ be any probability space,
$\mathscr{C}_{a}$ sub-algebra of $\mathscr{B}$. Then
\begin{itemize}
\item[\rm(a)] $L^{2}(\Omega,\mathscr{C},P)$ is a closed subspace of
  $L^{2}(\Omega,\mathscr{B},P)$.

\item[\rm(b)] If $\pi:L^{2}(\Omega,\mathscr{B},P)\to
  L^{2}(\Omega,\mathscr{C},P)$ is the projection map then $\pi f
  =E(f|\mathscr{C})$. 
\end{itemize}
\end{lemma}

\begin{proof}
Refer appendix.
\end{proof}

\begin{lemma}\label{chap8-lem2}
Let $\Omega=C([0,\infty);\mathbb{R}^{d})$, $P_{0}$ the probability
  corresponding to the Brownian motion. Then the set
  $\{\phi(\pi_{t_{1}},\ldots,t_{k})\in \phi$ is continuous, bounded on
  $\mathbb{R}^{d}\times\cdots\times \mathbb{R}^{d}$ ($k$ times),
  $\pi_{t_{1}},\ldots,t_{k}$ the canonical projection) is dense in
  $L^{2}(\Omega,\mathscr{B},P)$. 
\end{lemma}

\begin{proof}
Functions of the form $\phi(x(t_{1}),\ldots,x(t_{k})$ where $\phi$
runs over continuous functions is clearly dense in
$L_{2}(\Omega,\mathscr{F}_{t_{1},t_{2},\ldots,t_{k}},P)$ and
$$
\bigcup\limits_{k}\bigcup\limits_{t_{1},\ldots,t_{k}}L_{2}(\Omega,\mathscr{F}_{t_{1},\ldots,t_{k}},P) 
$$
is clearly dense in $L_{2}(\Omega,\mathscr{B},P)$. 
\end{proof}

\medskip
\noindent
{\bf Proof of zero-one law.}~ Let\pageoriginale 
$$
H_{t}=L^{2}(\Omega,\mathscr{F}_{t},P),H=L^{2}(\Omega,\mathscr{B},P),H_{0+}=\bigcap\limits_{t>0}H_{t}. 
$$

Clearly $H_{0+}=L^{2}(\Omega,\mathscr{F}_{0+},P)$.

Let $\pi_{t}:H\to H_{t}$ be the projection. Then $\pi_{t}f\to
\pi_{0+}f\,\forall_{f}$ in $H$. To prove the law it is enough to show
that $H_{0+}$ contains only constants, which is equivalent to
$\pi_{0+}f=$ constant $\forall f$ in $H$. As $\pi_{0+}$ is continuous
and linear it is enough to show that $\pi_{0+}\phi=$ const $\forall
\phi$ of the Lemma \ref{chap8-lem2}:
\begin{align*}
\pi_{0+}\phi=\Lt\limits_{t\to 0}\pi_{t}\phi &= \Lt\limits_{t\to
  0}E(\phi|{}_{t})\q\text{by Lemma \ref{chap8-lem1},}\\
&= \Lt\limits_{t\to 0}E(\phi(t_{1},\ldots,t_{k})|\mathscr{F}_{t}).
\end{align*}

We can assume without loss of generality that $t<t_{1}<t_{2}<\ldots<
t_{k}$. 
{\fontsize{10pt}{12pt}\selectfont
\begin{align*}
&
  E(\phi(t_{1},\ldots,t_{k})|\mathscr{F}_{t})=\int\phi(y_{1},\ldots,y_{k})1/\surd
  (2\pi(t_{1}-t))e^{-|y_{1}-X_{t}(w)|^{2}/2(t_{1}-t)}\ldots\\
&\qq \ldots
  1/\sqrt(2\pi(t_{k}-t_{k-1}))e^{\frac{-|y_{k}-y_{k-1}|^{2}}{2(t_{k}-t_{k-1})}}dy_{1}\ldots
    dy_{k}. 
\end{align*}}\relax

Since $X_{0}(w)=0$ we get, as $t\to 0$,
$$
\pi_{0+}\phi=\text{~ constant.}
$$

This completes the proof.

\medskip
\noindent
{\bf APPLICATION.}~ Let $\alpha\geq
1\,A=\{w:\int\limits^{1}_{0}|w(t)|/t^{\alpha}<\infty\}$. Then $A\in
\mathscr{F}_{0+}$. For, if $0<s<1$, then
$\int\limits^{1}_{s}|w(t)|/t^{\alpha}<\infty$. Therefore $w\in A$ or
not according as $\int\limits^{s}_{0}|w(t)|/t^{\alpha}dt$ converges or
not. But this convergence can\pageoriginale be asserted by knowing the
history of $w$ upto time $s$. Hence $A\in
\mathscr{F}_{s}$. Blumenthal's law implies that
$$
\int\limits^{1}_{0}|w(t)|/t^{\alpha}dt<\infty\text{~ a.e.w., or,~ }
\int\limits^{1}_{0}|w(t)|/t^{\alpha}dt=\infty\text{~ a.e.w.}
$$

A precise argument can be given along the following lines. If $0<s<1$,
\begin{align*}
A &= \{w:\int\limits^{s}_{0}|w(t)|/t^{\alpha}<\infty\}\\
&= \{w:\sup I_{n,s}(w)<\infty\}
\end{align*}
where $I_{n,s}(w)$ is the lower Riemannian sum of
$|w(t)^{n}|/t^{\alpha}$ corresponding to the partition
$\{0,s/n,\ldots,s\}$ and each $I_{n,s}\in \mathscr{F}_{s}$.
