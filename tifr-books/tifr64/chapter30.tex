\chapter{Behaviour of Diffusions for Large Times}\label{chap30}

LET\pageoriginale $L_{2}=\Delta/2+b\cdot\nabla$ WITH
$b:\mathbb{R}^{d}\to \mathbb{R}^{d}$ measurable and bounded on each
compact set. We assume that there is no explosion. If $P_{x}$ is the
$d$-dimensional Brownian measure on
$\Omega=C([0,\infty);\mathbb{R}^{d})$ we know that there exists a
  probability measure $Q_{x}$ on $\Omega$ such that 
$$
\dfrac{dQ_{x}}{dP_{x}}\Big|_{t}=\exp \left[\int\limits^{t}_{0}\langle
  b,dX\rangle -\frac{1}{2}\int\limits^{t}_{0}|b|^{2}ds\right]
$$

Let $K$ be any compact set in $\mathbb{R}^{d}$ with non-empty
interior. We are interested in finding out how often the trajectories
visit $K$ and whether this `frequency' depends on the starting point
of the trajectory and the compact set $K$.

\begin{theorem*}
Let $K$ be any compact set in $\mathbb{R}^{d}$ having a non-empty
interior. Let
\begin{align*}
E^{K}_{\infty} &= \{w:w\text{~ revisits $K$ for arbitrarily large
  times}\}\\
&= \{w:\text{~ there exists a sequence~ } t_{1}<t_{2}<\cdot <\infty\\
&\qq \text{with~ } t_{n}\to \infty\text{~ such that~ } w(t_{n})\in K\} 
\end{align*}

Then,
\begin{align*}
& \text{either~ }Q_{x}(E^{K}_{\infty})=0,\ \forall x,\text{~ and~ }
  \forall K,\\
&\text{or}\q Q_{x}(E^{K}_{\infty})=1,\ \forall x,\text{~ and~ }
  \forall K.
\end{align*}
\end{theorem*}

\begin{remark*}
\begin{enumerate}
\item In the first case $\lim\limits_{t\to +\infty}|X(t)|=+\infty$
  a.e.\@ $Q_{x}$, $\forall x$, i.e.\@ almost all trajectories stay
  within $K$ only for a short period.

These\pageoriginale trajectories are called {\em transient}. In the
second case almost all trajectories visit $K$ for arbitrary large
times. Such trajectories are called {\em recurrent}.

\item  If $b=0$ then $Q_{x}=P_{x}$. For the case $d=1$ or $d=2$ we
  know that the trajectories are recurrent. If $d\geq 3$ the
  trajectories are transient.
\end{enumerate}
\end{remark*}


\begin{proof}
\setcounter{step}{0}
\begin{step}%1
We introduce the following sets.
\begin{align*}
& E^{K}_{0} =\{w:X(t,w)\in K\text{~ for some~ } t\geq 0\},\\
& E^{K}_{t_{0}}=\{w:X(t,w)\in K\text{~ for some~ }t\geq
  t_{0}\},\ 0\leq t_{0}<\infty.
\end{align*}

Then clearly
$$
E^{K}_{\infty}=\bigcap\limits^{\infty}_{n=1}E^{K}_{n}=\bigcap\limits_{t_{0}\geq 0}E^{K}_{t_{0}}.
$$

Let
\begin{align*}
\psi(x) &= Q_{x}(E^{K}_{\infty}),\ F=\chi_{E^{K}_{\infty}}.\\
E^{Q_{x}}(F|\mathscr{F}_{t}) &=
E^{Q_{x}}(\chi_{E^{K}_{\infty}}|\mathscr{F}_{t}) =Q_{X(t)}(E^{K}_{\infty})\\
&\qq\q \text{by the Markov property,}\\
&=\psi(X(t))\text{~ a.e.~ } Q_{x}.
\end{align*}
\end{step}

Next we show that $\psi(X(t))$ is a martingale relative to
$Q_{x}$. For, if $s<t$,
\begin{align*}
& E^{Q_{x}}(\psi(X(t))|\mathscr{F}_{s})\\
& =E^{Q_{x}}(E^{Q_{x}}(F|\mathscr{F}_{t})|\mathscr{F}_{s})\\
& =E^{Q_{x}}(F|\mathscr{F}_{s})\\
& =\psi(X(s)).
\end{align*}

Equating\pageoriginale the expectations at time $t=0$ and time $t$ one
gets 
\begin{align*}
\psi(x) &= \int\limits_{\Omega}\psi(X(t))dQ_{x}\\
&= \int \psi(y)q(t,x,y)dy,
\end{align*}
where $q(t,x,A)=Q_{x}(X_{t}\in A)$, $\forall A$ Borel in
$\mathbb{R}^{d}$. 

We assume for the present that $\psi(x)$ is continuous (This will be
shown in Lemma \ref{chap31-lem4} in the next section). By definition
$0\leq \psi\leq 1$.

\begin{step}%2
$\psi(x)=1$, $\forall x$ or $\psi(x)=0$, $\forall x$.

Suppose that $\psi(x_{0})=0$ for some $x_{0}$. Then
$$
0=\psi(x_{0})=\int \psi(y)q(t,x_{0},y)dy.
$$ 
As $q>0$ a.e.\@ and
$\psi\geq 0$ we conclude that $\psi(y)=0$ a.e.\@ (with respect to
Lebesgue measure). Since $\psi$ is continuous $\psi$ must vanish
identically.
\end{step}

If $\psi(x_{0})=1$ for some $x_{0}$, we apply the above argument to
$1-\psi$ to conclude that $\psi=1$, $\forall x$. We now show that the
third possibility $0<\psi(x)<1$ can never occur.

Since $K$ is compact and $\psi$ is continuous,
$$
0<a=\inf\limits_{y\in K}\psi(y)\leq \sup\limits_{y\in K}\psi(y)=b<1.
$$

From an Exercise in the section on martingales it follows that
$$
\psi(X(t))\to \chi_{E^{K}_{\infty}}\q\text{a.e.}\q Q_{x}\q\text{as}\q
t\to +\infty.
$$

Therefore $\lim\limits_{t\to \infty}\psi(X(t))(1-\psi(X(t)))=0$ a.e.\@
$Q_{x}$. Now
{\fontsize{10pt}{12pt}\selectfont
\begin{align*}
\psi(x_{0})=Q_{x_{0}}(E^{K}_{\infty}) &= Q_{x_{0}}\{w:w(t)\in K\text{~
  for arbitrary large time}\}\\
&\leq Q_{x_{0}}\{w:a\leq \psi(X(t,w))\leq b\text{~ for arbitrarily
  large times}\}\\
&\leq Q_{x_{0}}\{w:(1-b)a\leq \psi(X(t))[1-\psi(X(t)]\leq b(1-a)\\
&\qq \text{for arbitrarily large times}\}\\
&= 0.
\end{align*}}\relax\pageoriginale

Thus $\psi(x)=0$ identically, which is a contradiction. Thus for the
given compact set $K$,
\begin{align*}
&\text{either}\q Q_{x}(E^{K}_{\infty}) =0,\ \forall x,\\
&\text{or}\qq\;\, Q_{x}(E^{K}_{\infty})=1,\ \forall x.
\end{align*}

\begin{step}%3
If $Q_{x}(E^{K_{0}}_{\infty})=1$ for some compact set
$K_{0}(\overset{\circ}{K}_{0}\neq \emptyset)$ and $\forall x$, then
$Q_{x}(E^{K}_{\infty})=1$, $\forall$ compact set $K$ with non-empty
interior.
\end{step}

We first given an intuitive argument. Suppose
$Q_{x}(E^{K_{0}}_{\infty})=1$, i.e.\@ almost all trajectories visit
$K_{0}$ for arbitrarily large times. Each time a trajectory hits
$K_{0}$, it has some chance of hitting $K$. Since the trajectory
visits $K_{0}$ for arbitrarily large times it will visit $K$ for
arbitrarily large times. We now give a precise arguent. Let
\begin{align*}
& \tau_{0}=\inf \{t:X(t)\in K_{0}\}\\
& \tau_{1}=\inf \{t\geq t_{0}+1\q X(t)\in K_{0}\}\\
& \ldots\q \ldots\q \ldots\q \ldots\\
& \tau_{n}=\inf\{t\geq t_{n-1}+1:X(t)\in K_{0}\}\\
& \ldots\q \ldots\q \ldots\q \ldots\\
\end{align*}

Clearly $\tau_{0}<\tau_{1}<\ldots<$ and $\tau_{n}\geq n$.
\begin{align*}
Q_{x}(E^{K}_{n}) &\geq Q_{x}\{X(t)\in K\text{~ for~ } t\geq
\tau_{n}\}\\
&\geq Q_{x}\{X(t)\in K\text{~ for~ } t\in
\bigcup\limits^{\infty}_{j=n}[\tau_{j},\tau_{j}+1]\}\\ 
&= 1-Q_{x}\left\{\bigcap\limits_{j\geq n}X(t)\not\in K\text{~ for~ } t\in
              [\tau_{j},\tau_{j}+1]\right\}\\
&\geq 1-Q_{x}\left\{\bigcap\limits_{j\geq n}X(\tau_{j}+1)\in K\right\}
\end{align*}\pageoriginale

We claim that
$$
Q_{x}(\bigcap\limits_{j\geq n}X(\tau_{j}+1)\not\in K)=0,
$$
so that $Q_{x}(E^{K}_{n})=1$ for every $n$ and hence
$Q_{x}(E^{K}_{\infty})=1$, completing the proof of the theorem.

Now 
$$
q(1,x,K)\geq q(1,x,\overset{\circ}{K})>0,\q \forall
x,\ \overset{\circ}{K}\q\text{interior of}\q K.
$$

It is clear that if $x_{n}\to x$, then
$$
\varliminf q(1,x_{n},\overset{\circ}{K})\geq
q(1,x,\overset{\circ}{K}).
$$

Let $d=\inf\limits_{x\in K_{0}}q(1,x,\overset{\circ}{K})$. Then there
exists a sequence $x_{n}$ in $K_{0}$ such that $d=\Lt\limits_{n\to
  \infty}q(1,x_{n},\overset{\circ}{K})$. $K_{0}$ being compact, there
exists a subsequence $y_{n}$ of $x_{n}$ with $y_{n}\to x$ in $K_{0}$,
so that
$$
d=\lim\limits_{n\to \infty}q(1,x,\overset{\circ}{K})=\varliminf
q(1,y_{n},\overset{\circ}{K})\geq q(1,x,\overset{\circ}{K})>0.
$$

Thus
$$
\inf\limits_{x\in K_{0}}q(1,x,K)\geq d>0.
$$

Now 
\begin{align*}
& Q_{x}\left(\prod^{N}_{j=n}X(\tau_{j}+1)\not\in
K|\mathscr{F}_{\tau_{N}}\right)\\
&= \prod^{N-1}_{j=n}\chi(X(\tau_{j}+1)\not\in K)Q_{x}(X(\tau_{N}+1)\in
K|\mathscr{F}_{\tau_{N}})\q\text{because}\\
&\qq \tau_{j}+1\leq \tau_{N}\q\text{for}\q j<N,\\
&= \prod^{N-1}_{j=n}(\chi_{X(\tau_{j}+1)}\not\in
K)Q_{X(\tau_{N})}(X(1)\not\in K)\q\text{by the strong}\\
&\qq\qq\text{Markov property,}\\
&= \prod^{N-1}_{j=1}q(1,X(\tau_{N}),K^{c})\,\chi_{(X(\tau_{j}+1)\not\in
  K)}.  
\end{align*}\pageoriginale

Therefore
\begin{align*}
& Q_{x}\left(\bigcap^{N}_{j=n}X(\tau_{j}+1)\not\in K\right)\\
&= E^{Q_{x}}(Q_{x}(\bigcap^{N}_{j=n}X(\tau_{j}+1)\not\in
  K|_{\tau_{N}})))\\
&= E^{Q_{x}}\left(\prod^{N-1}_{j=n}(\chi_{[X(\tau_{j}+1)\not\in
      K]})q(1,X(\tau_{N}), K^{c})\right)
\end{align*}

Since $K_{0}$ is compact and $X(\tau_{N})\in K_{0}$,
$$
q(1,X(\tau_{N}),K^{c})=1-q(1,X(\tau_{N}),K)\leq 1-d
$$

Hence
$$
Q_{x}\left(\bigcap\limits^{N}_{j=n}X(\tau_{j}+1)\not\in K\right)\leq
(1-d)Q_{x}\left(\bigcap\limits^{N-1}_{j=n}X(\tau_{j}+1)\not\in
K\right).
$$

Iterating, we get
$$
Q_{x}\left(\bigcap\limits^{N}_{j=n}X(\tau_{j}+1)\not\in K\right)\leq
(1-d)^{N-n+1},\ \forall N.
$$

Let\pageoriginale $N\to \infty$ to get
$$
Q_{x}\left(\bigcap_{j=n}X(\tau_{j}+1)\not\in K\right)=0,
$$
since $0\leq 1-d<1$. Thus the claim is proved and so is the theorem.
\end{proof}

\begin{coro*}
Let $K$ be compact, $\overset{\circ}{K}\neq \emptyset$. Then
$Q_{x}(E^{K}_{\infty})=1$ if and only if $Q_{x}(E^{K}_{0})=1$,
$\forall x$.
\end{coro*}

\begin{proof}
Suppose $Q_{x}(E^{K}_{\infty})=1$; then $Q_{x}(E^{K}_{0})=1$ because
$E^{K}_{\infty}E^{K}_{0}$. Suppose $Q_{x}(E^{K}_{0})=1$, then
\begin{align*}
Q_{x}(E^{K}_{n}) &=
E^{Q_{x}}(E^{Q_{x}}(\chi_{E^{K}_{n}}|\mathscr{F}_{n}))\\
&= E^{Q_{x}}(Q_{X(n)}(E^{K}_{0}))\\
&= E^{Q_{x}}(1)\\
&= 1,\ \forall n.
\end{align*}

Therefore $Q_{x}(E^{K}_{\infty})=1$.
\end{proof}

\begin{remark*}
If $Q_{x}(E^{K}_{\infty})=0$ then it need not imply that
$$
Q_{x}(E^{K}_{0})=0.
$$
\end{remark*}

\begin{example*}
Take $b=0$ and $d=3$. LEt $K=S_{1}=\{x\in \mathbb{R}^{3}$ such that
$|x|\leq 1\}$. Define
$$
\psi(n)=
\begin{cases}
1, & \text{for}\q |x|\leq 1,\\
\frac{1}{|x|}, & \text{for}\q |x|\geq 1.
\end{cases}
$$
$P_{x}(E^{K}_{0})\neq$ constant but $P_{x}(E^{K}_{\infty})=0$. In
fact, $P_{x}(E^{K}_{0})=\psi(x)$ (Refer Dirichlet Problem).
\end{example*}

