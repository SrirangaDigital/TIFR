\chapter{Application of Stochastic Integral}\label{chap33}

LET\pageoriginale $b$ BE A bounded function. For every Brownian measure
$P_{x}$ on $\Omega=C([0,\infty);\mathbb{R}^{d})$ we have a probability
  measure $Q_{x}$ on $(\Omega,\mathscr{F})$.


\begin{problem*}
Let $q(t,x,A)=Q_{x}(X_{t}\in A)\cdot q(t,x,\cdot)$ is a probability
measure on $\mathbb{R}^{d}$. We would like to know if $q(t,x,\cdot)$
is given by a density function on $\mathbb{R}^{d}$ and study its
properties. 
\end{problem*}

\noindent
{\bf Step (i).}~ $q(t,x,\cdot)$ is absolutely continuous with respect
to the Lebesgue measure.

For, $p(t,x,A)=P_{x}(X(t)\in A)$ is given by a density
function. Therefore $p(t,x,\cdot)\gg m_{d}$ (Lebesgue measure). Since 
\begin{align*}
& Q_{x}\ll P_{x}\text{~ on~ } \mathscr{F}_{t},\\
& q(t,z,\cdot)\leq M_{d}\text{~ on~ } \mathscr{F}_{t}.
\end{align*}

\noindent
{\bf Step (ii).}~ Let $q(t,x,y)\geq 0$ be the density function of
$q(t,x,\cdot)$ and write $p(t,x,y)$ for the density of
$p(t,x,\cdot)$. Let $1<\alpha<\infty$. Put
\begin{align*}
r(t,x,y) &= \frac{q(t,x,y)}{p(t,x,y)}.\\
\int\limits_{\mathbb{R}^{d}}q^{\alpha}dy &= \int
r^{\alpha}p^{\alpha}dy\\
&= \int r^{\alpha}p^{1/\alpha}P^{\frac{\alpha-1}{\alpha}}dy \leq
\left(\int r^{\alpha^{2}}pdy\right)^{1/\alpha}\times\\
&\qq \times \left(\int p^{\alpha+1}dy\right)^{\alpha-1/\alpha}
\end{align*}


\noindent
{\bf Step (iii).}~ \pageoriginale
\begin{align*}
Q_{x}(X(t)\in A) &= \int q(t,x,y)dy\\
&= \int r(t,x,y)p(t,x,y)dy\\
&= \int r(t,x,y)P_{x}(X_{t}\in dy).
\end{align*}

Therefore
$$
\frac{dQ_{x}}{dP_{x}}\Big|^{t}_{t}=r(t,x,y)
$$

Therefore
\begin{align*}
& \left(\int
  r^{\alpha^{2}}pdy\right)^{1/\alpha}=\left|\left|\frac{dQ_{x}}{dP_{x}}\Big|^{t}_{t}\right|\right|^{\alpha}_{\alpha^{2},P_{x}}\\
&\leq
  \left|\left|\frac{dQ_{x}}{dP_{x}}\Big|_{t}\right|\right|^{\alpha}_{\alpha^{2},P_{x}},\q\text{since}\q
  \mathscr{F}^{t}_{t}\subset \mathscr{F}_{t},\\
&\qq \{E^{P_{x}}[Z(t)^{\alpha^{2}}]\}^{1/\alpha}\\
&= \{E^{P_{x}}[\exp (\alpha^{2}\int\limits^{t}_{0}\langle b,dX\rangle
    -\frac{\alpha^{2}}{2}\int\limits^{t}_{0}|b|^{2}ds)]\}^{1/\alpha}\\
&= \{E^{P_{x}}[\exp (\alpha^{2}\int\limits^{t}_{0}\langle b, dX\rangle
    -\frac{\alpha^{4}}{2}\int\limits^{t}_{0}|b|^{2}ds+\frac{\alpha^{4}-\alpha^{2}}{2}\int\limits^{t}_{0}|b|^{2}ds)]\}^{1/\alpha}, 
\end{align*}
i.e.,
{\fontsize{10pt}{12pt}\selectfont
$$
\left(\int r^{\alpha^{2}}pdy\right)^{1/\alpha}\leq
\left\{E^{P_{x}}\left[\exp\left(\frac{\alpha^{4}-a^{2}}{2}ct+\alpha^{2}\int\limits^{t}_{0}\langle
  b,dX\rangle
  -\frac{\alpha^{4}}{2}\int\limits^{t}_{0}|b|^{2}ds\right)\right]\right\}^{1/\alpha} 
$$}\relax
where $c$ is such that $|b|^{2}\leq c$. Using Schwarz inequality we
then get
$$
\left(\int r^{\alpha^{2}}pdy\right)^{1/\alpha}\leq
\left[\exp\left(\frac{\alpha^{4}-\alpha^{2}}{2}ct\right)\right]^{1/\alpha}. 
$$

Hence\pageoriginale
$$
\int q^{\alpha}dy\leq \left(\exp
\left[\frac{\alpha^{4}-\alpha^{2}}{2}ct\right]\right)^{1/\alpha}\left(\int P^{\alpha+1}dy\right)^{\alpha-1/\alpha}
$$ 

\noindent
{\bf Significance.}~ Pure analytical objects like $q(t,x,y)$ can be
studied using stochastic integrals.

