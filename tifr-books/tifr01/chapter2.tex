
\chapter{The Phragmen-Lindel\"of principle}\label{chap2}

We shall\pageoriginale first prove a crude version of the
Phragmen-Lindel\"of theorem and then obtain a refined variant of it. The
results may be viewed as extensions of the maximum-modulus principle
to infinite strips.

\setcounter{thm}{0}
\begin{thm}\label{chap2:thm1}
\cite[p.168]{key6} We suppose that
\begin{itemize}
\item[(i)] $f$ is regular in the strip $\alpha < x < \beta$; $f$ is
  continuous in $\alpha \leq x \leq \beta$

\item[(ii)] $|f| \leq M$ on $x = \alpha$ and $x = \beta$

\item[(iii)] $f$ is bounded in $\alpha \leq x \leq \beta$

Then, $|f| \leq M$ in $\alpha < x < \beta$; and $|f| = M$ in $\alpha <
x < \beta$ only if $f$ is a constant.
 \end{itemize}
\end{thm}

\begin{proof}
\begin{enumerate}
\renewcommand{\theenumi}{\roman{enumi}}
\renewcommand{\labelenumi}{(\theenumi)}
\item If $f(x+iy) \to 0$ as $y \to \pm \infty$ uniformly in $x$,
  $\alpha \leq x \leq \beta$, then the proof is simple. Choose a
  rectangle $\alpha \leq x \leq \beta$, $|y| \leq \eta$ with $\eta$
  sufficiently large to imply $|f(x \pm i \eta)| \leq M$ for $\alpha
  \leq x \leq \beta$. Then, by the maximum-modulus principle, for any
  $z_0$ in the interior of the rectangle,
$$
|f(z_0)| \leq M.
$$

\item If $|f(x+iy)| \not\to 0$, as $y \to \pm \infty$, consider the
  modified function $f_n (z) = f(z) e^{z^2 / n}$. Now
\begin{align*}
& |f_n(z)| \to 0 \text{ as } y \to \pm \infty \text{ uniformly in 
  $x$, and} \\
& |f_n (z)|  \leq M e^{c^2/n}
\end{align*}
on the boundary of the strip, for a suitable constant $c$.

Hence\pageoriginale
$$
|f_n (z_0)| \leq M e^{c^2/n}
$$
for an interior point $z_0$ of the rectangle, and letting $n \to
\infty$, we get
$$
|f(z_0)| \leq M.
$$

\item If $|f(z_0)| = M$, then $f$ is a constant. For, if $f$ were not
  a constant, in the neighbourhood of $z_0$ we would have points $z$,
  by the maximum-modulus principle, such that $|f(z)| > M$, which is
  impossible. 
\end{enumerate}
\end{proof}

Theorem \ref{chap2:thm1} can be restated as:

\medskip
\noindent{\textbf{Theorem $1'$:}}  Suppose that 
\begin{enumerate}
\renewcommand{\theenumi}{\roman{enumi}}
\renewcommand{\labelenumi}{(\theenumi)}
\item $f$ is continuous and bounded in $\alpha \leq x \leq \beta$

\item $|f(\alpha + iy)| \leq M_1$, $|f(\beta + iy)| \leq M_2$ for all
  $y$ 
\end{enumerate}

Then
$$
|f(x_0 + iy_0)| \leq M^{L(x_0)}_1 M^{1-L(x_0)}_2
$$
for $\alpha < x_0 < \beta$, $|y_0| < \infty$, where $L(t)$ is a linear
function of $t$ which takes the value 1 at $\alpha$ and 0 at
$\beta$. If equality occurs, then 
$$
f(z) = c M^{L(z)}_1 M^{1-L(z)}_2; \quad |c|=1.
$$

\begin{proof}
Consider
$$
f_1 (z) = \frac{f(z)}{M^{L(z)}_1 \; M^{1-L(z)}_2}
$$
and\pageoriginale apply Theorem \ref{chap2:thm1} to $f_1(z)$.

More generally we have \cite[p.107]{key12}
\end{proof}

\begin{thm}\label{chap2:thm2}
Suppose that
\begin{itemize}
\item[(i)] $f$ is regular in an open half-strip $D$ defined by:
$$
z = x + iy, \quad \alpha < x < \beta, \quad y > \eta
$$

\item[(ii)] $\overline{\lim\limits_{\substack{z \to \xi\\ \text{in
      }D}}} |f| \leq M$, \text{ for } $\xi \in Bd \; D$

\item[(iii)] $f = \mathcal{O} \left[ \exp \left\{e^{\frac{\theta \pi |z|}{\beta
    - \alpha}} \right\}\right]$, $\theta < D$
\end{itemize}
uniformly in $D$.

Then, $|f| \leq M$ in $D$; and $|f|<M$ in $D$ unless $f$ is a constant.
\end{thm}

\begin{proof}
Without loss of generality we can choose
$$
\alpha = -\frac{\pi}{2}, \quad \beta = + \pi/2, \quad \eta =0
$$

Set 
\begin{align*}
g(z) & = f(z). \exp (-\sigma e^{-ikz})\\
& \equiv f(z). \{\omega(z)\}^{\sigma} , \text{ say},
\end{align*}
where $\sigma >0$, $\theta < k < 1$. Then, as $y \to + \infty$,
$$
g = \mathcal{O} \left[\exp \left\{e^{\theta|z|} - \sigma e^{ky} \cos
  \left(\frac{k\pi}{2}\right) \right\}\right] 
$$
uniformly\pageoriginale in $x$, and 
\begin{align*}
e^{\theta|z|} - \sigma e^{ky} \cos\left(\frac{k\pi}{2}\right) & \leq e^{\theta(y +
  \pi/2)} - \sigma e^{ky} \cos\left(\frac{k\pi}{2}\right)\\
& \to - \infty, \text{ as } y \to \infty,
\end{align*}
uniformly in $x$, since $\theta < k$. Hence
$$
|g| \to 0 \text{ uniformly as } y \to \infty,
$$
so that
$$
|g| \leq M, \text{ for } y > y', \alpha < x < \beta.
$$

Let $z_0 \in D$. We can then choose $H$ so that
$$
|g(z)| \leq M
$$
for $y = H$, $\alpha < x < \beta$, and we may also suppose that $H>
y_0 = \im (z_0)$. 

Consider now the rectangle defined by $\alpha < x < \beta$, $y = 0$,
$y = H$. Since $|\omega^\sigma| \leq 1$ we have
$$
\overline{\lim\limits_{\substack{z \to \xi\\ \text{in $D$}}}} \;\;
|g| \; \leq \; M
$$
for every point $\xi$ on the \textit{boundary of this
  rectangle}. Hence, by the maximum-modulus principle,
\begin{align*}
& |g(z_0)| < M, \text{ unless $g$ is a constant};\\
\text{or } \qquad & |f(z_0)| < M | e^{\sigma e^{-ikz_0}} |\qquad 
\end{align*}

Letting\pageoriginale $\sigma \to 0$, we get
$$
|f(z_0)| < M.
$$
\end{proof}

\begin{remarks*}
The choice of $\omega$ in the above proof is suggested by the critical
case $\theta =1$ of the theorem when the result is {\em no longer
  true}. For take 
$$
\theta = 1, \quad \alpha = - \pi/2, \quad \beta = \pi/2, \quad \eta =
0, \quad f = e^{e^{-iz}}
$$
Then
$$
|e^{e^{-iz}}| = e^{\re(e^{-iz})} = e^{\re\{e^{y - ix}\}} = e^{e^y \cos
x}
$$
So
$$
|f| = 1 \text{ on } x = \pm \pi/2, \quad \text{ and } f = e^{e^y}
\text{ on } x =0.
$$
\end{remarks*}

\begin{corollary}\label{chap2:coro1}
If $f$ is regular in $D$, continuous on $Bd$ $D$, $|f|\leq M$ on $Bd$
$D$, and 
$$
f = O \left(e^{e \frac{|z| \pi \theta}{\beta - \alpha}} \right), \quad
\theta < 1,
$$
then $|f|  \leq M$ in $D$.
\end{corollary}

\begin{corollary}\label{chap2:coro2}
Suppose that hypotheses (i) and (iii) of Theorem \ref{chap2:thm2} hold; suppose
further that $f$ is continuous on $Bd$ $D$,
\begin{gather*}
f = O(y^a) \text{ on } x = \alpha, \quad f = O (y^b) \text{ on } x =
\beta. \text{ Then }\\
f = O (y^c) \text{ on } x = \gamma,
\end{gather*}
uniformly in $\gamma$, where $c = p \gamma + q$, and $px + q$ is the
linear function which equals a at $x = \alpha$ and $b$ at $x =
\beta$.
\end{corollary}

\begin{proof}
Let\pageoriginale $\eta >0$. Define $\varphi (z) = f (z) \psi (z)$,
where $\psi$ is the single-valued branch of $(-zi)^{-(pz+q)}$ defined
in $D$, and apply Theorem \ref{chap2:thm2} to $\varphi$. We first observe that
$\varphi$ is regular in $D$ and continuous in $\overline{D}$. Next
\begin{align*}
|\psi(z)| & = |(y - ix)^{-(px + q) - ipy}|\\
& = |y - ix|^{-(px + q)} \exp \left(py \im \frac{\log y - ix}{y}\right)\\
& = y^{-(px + q)} \left|1 + O \left( \frac{1}{y}\right)\right|^{O(1)} \exp
\left[py \left\{ -\frac{x}{y} + O \left(\frac{1}{y^2} \right) \right\}
\right]\\
& = y^{-(px + q)} e^{-px} \left\{1+O \left(\frac{1}{y} \right)\right\},
\end{align*}
the $O$'s being uniform in the $x$'s.

Thus $\varphi = O(1)$ on $x = \alpha$ and $x = \beta$. Since $\psi = O
(y^k) = O (|z|^k)$, we see that condition (iii) of Theorem \ref{chap2:thm2} holds for
$\varphi$ if $\theta$ is suitably chosen. Hence $\varphi = O(1)$
uniformly in the strip, which proves the corollary.
\end{proof}

\begin{thm}\cite[p.108]{key12}\label{chap2:thm3}
Suppose that conditions (i) and (iii) of Theorem \ref{chap2:thm2} hold. Suppose
further that $f$ is continuous in $Bd$ $D$, and 
$$
\overline{\lim\limits_{y \to \infty}} |f| \leq M \text{ on } x =
\alpha, \quad x = \beta 
$$
Then
$$
\overline{\lim\limits_{y \to \infty}} |f| \leq M \; \text{ uniformly in } \; 
\alpha \leq x \leq \beta.
$$ 
\end{thm}

\begin{proof}
$f$ is\pageoriginale bounded in $\overline{D}$, \textit{by the previous
    theorem}. Let $\eta \geq 0$, $\epsilon > 0$. Let $H= H
  (\epsilon)$ be the ordinate beyond which $|f|<M + \epsilon$ on
  $x =\alpha$, $x = \beta$.

Let $h >0$ be a constant. Then
$$
\left|\frac{z}{z+h i} \right| <1
$$
in the strip. Choose $h = h (H)$ so large that
$$
\left|f(z) \cdot \frac{z}{z+hi} \right| < M + \epsilon \text{ on } y =
H 
$$
(possible because $\left| f \cdot \dfrac{z}{z+hi} \right| \leq |f| <
M$). Then the function
$$
g(z) = f \cdot \frac{z}{z+hi}
$$
satisfies the conditions of Theorem \ref{chap2:thm2} (with $M + \epsilon$ in place
of $M$) in the strip \textit{above} $y = H$. Thus
$$
|g| \leq M + \epsilon \text{ in {\em this} strip}
$$
and so $\overline{\lim\limits_{y \to \infty}} |g| \leq M +
\epsilon$ uniformly.

That is, 
$$
\overline{\lim} |f|\leq M + \epsilon,
$$
since $\dfrac{z}{z + hi} \to 1$ uniformly in $x$. Hence
$$
\overline{\lim} |f| \leq M.
$$
\end{proof}

\setcounter{corollary}{0}
\begin{corollary}\label{chap2:addcoro1}
If\pageoriginale conditions (i) and (iii) of Theorem \ref{chap2:thm2} hold, if $f$ is
continuous on $Bd$ $D$, and if 
$$
\overline{\lim} |f| \leq 
\begin{cases}
a & \text{ on } x = \alpha,\\
b & \text{ on } x = \beta,
\end{cases}
$$
where $a \neq 0$, $b \neq 0$, then
$$
\overline{\lim_{y \to \infty}} |f| \leq e^{px + q},
$$ 
uniformly, $p$ and $q$ being so chosen that $e^{px+q} = a$ for $x =
\alpha$ and $ = b$ for $x = \beta$.
\end{corollary}

\begin{proof}
Apply Theorem \ref{chap2:thm3} to $g = f e^{-(pz+q)}$ 
\end{proof}

\begin{corollary}\label{chap2:addcoro2}
If $f = O(1)$ on $x = \alpha$, $f = o(1)$ on $x = \beta$, then $f =
o(1)$ on $x = \gamma$, $\alpha < \gamma \leq \beta$.

[if conditions (i) and (iii) of Theorem \ref{chap2:thm2} are satisfied].
\end{corollary}

\begin{proof}
Take $b = \epsilon$ in Corollary \ref{chap2:addcoro1}, and note that
$e^{p\gamma + q} \to 0$ as $\epsilon \to 0$ for fixed $\gamma$,
provided $p$ and $q$ are chosen as specified in Corollary \ref{chap2:addcoro1}. 
\end{proof}
