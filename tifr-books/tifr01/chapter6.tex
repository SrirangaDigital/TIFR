\chapter{The Gamma Function}\label{chap6}

\section[Elementary Properties]{Elementary Properties \protect \cite[p.148]{key15}}\label{chap6:sec1}\pageoriginale 

Define the function $\Gamma$ by the relation.
$$
\Gamma (x) = \int\limits^\infty_0 t^{x-1} e^{-t} dt, \; x \text{ real. }
$$
\begin{enumerate}
\renewcommand{\theenumi}{\roman{enumi}}
\renewcommand{\labelenumi}{(\theenumi)}
\item This integral converges at the `upper limit' $\infty$, for all
  $x$, since $t^{x-1} e^{-t} = t^{-2} t^{x+1} e^{-t} = O(t^{-2})$ as
  $t \to \infty$; it converges at the lower limit $0$ for $x>0$.

\item The integral converges uniformly for $0<a \leq x \leq b$. For 
\begin{align*}
\int\limits^\infty_0 t^{x-1} e^{-t} dt = \int\limits^1_0 +
\int\limits^\infty_1 & = O \left( \int\limits^1_0 t^{a-1} dt\right) +
O \left( \int\limits^{\infty}_1 t^{b-1} e^{-t} dt\right)\\
& = O (1), \text{ independently of $x$.}
\end{align*}

Hence the integral represents a \textit{continuous function for $x>0$.}

\item If $z$ is complex, $\int\limits^\infty_0 t^{z-1} e^{-t} dt$ is
  again uniformly convergent over any finite region in which $\re z
  \geq a > 0$. For if $z = x + iy$, then 
$$
|t^{z-1}| = t^{x-1},
$$
and we use (ii). Hence $\Gamma (z)$ is an \textit{analytic function
  for $\re z >0$}

\item If $x>1$, we integrate by parts and get
\begin{align*}
\Gamma (x) & = \left[ -t^{x-1} e^{-t}\right]^{\infty}_0 + (x-1)
\int\limits^{\infty}_0 t^{x-2} e^{-t} dt\\
& = (x-1) \Gamma (x-1).
\end{align*}
However,\pageoriginale  
\begin{align*}
\Gamma (1) & = \int\limits^\infty_0 e^{-t} dt = 1; \text{ hence}\\
\Gamma (n) & = (n-1) !,
\end{align*}
if $n$ is a $+$ve integer. Thus $\Gamma (x)$ is a generalization of $n!$

\item We have
$$
\log \Gamma (n) = \left(n - \frac{1}{2}\right) \log n - n + c + O (1),
$$
where $c$ is a constant.
$$
\log \Gamma (n) = \log (n -1) ! = \sum\limits^{n-1}_{r=1} \log r.
$$

We estimate log $r$ by an integral: we have
\begin{align*}
\int\limits^{r + \frac{1}{2}}_{r-\frac{1}{2}}  \log t \; dt  & =
\int\limits^{\frac{1}{2}}_{-\frac{1}{2}} \log (s+r) ds =
\int\limits^{\frac{1}{2}}_0 + \int\limits^0_{-\frac{1}{2}}\\
& = \int\limits^{1/2}_0 \left[\log (t+r) + \log (r-t) \right] \; dt\\
& = \int\limits^{1/2}_0 \left[ \log r^2 + \log \left(1-\frac{t^2}{r^2}
  \right)\right] dt\\
& = \log r + c_r, \text{ where } c_r = O \left(\frac{1}{r^2} \right).
\end{align*}

Hence
\begin{align*}
\log \Gamma (n) & = \sum\limits^{n-1}_{r=1}
\left(\int\limits^{r+\frac{1}{2}}_{r-\frac{1}{2}} \log t \; dt - c_r \right)\\
& = \int\limits^{n-\frac{1}{2}}_{\frac{1}{2}}  \log t \; dt -
\sum^{n-1}_{r=1} c_r\\
& = \left\{\left(n-\frac{1}{2} \right) \log \left(n-\frac{1}{2}
\right) - \frac{1}{2} \log \frac{1}{2}\right\} - \left(n-\frac{1}{2}\right) +
\frac{1}{2}\\
& \hspace{2cm} - \sum\limits^{\infty}_{r=1} c_r + \sum\limits^\infty_n
C_r\\
& = \left(n-\frac{1}{2}\right) \log n - n + c + o(1), \text{ where $c$ is a constant.}
\end{align*}\pageoriginale 

Hence, also 
$$
(n!) = a \cdot e^{-n} n^{n+\frac{1}{2}} e^{o(1)},
$$
where $a$ is a constant such that $c = \log a $

\item We have
$$
\frac{\Gamma (x) \Gamma (y)}{\Gamma (x+y)} = \int\limits^\infty_0
\frac{t^{y-1}}{(1+t)^{x+y}}  dt, \; x>0, \; y >0.
$$
For 
$$
\Gamma (x) \Gamma (y) = \int\limits^\infty_0 t^{x-1} e^{-t} dt \cdot
\int\limits^\infty_0 s^{y-1} e^{-s} ds, \; x >0,\; y >0
$$
Put $s = tv$; then
\begin{align*}
\Gamma (x) \Gamma (y) & = \int\limits^\infty_0 t^{x-1} e^{-t} \; dt
\int\limits^\infty_0 t^y v^{y-1} e^{-tv} \; dv\\
& = \int\limits^\infty_0 v^{y-1} dv \int\limits^\infty_0 t^{x+y-1}
e^{-t(1+v)} dt\\
& = \int\limits^\infty_0 v^{y-1} dv \int\limits^\infty_0
\frac{u^{x+y-1}e^{-u}}{(1+v)^{x+y}}  du\\
& = \Gamma (x+y) \int\limits^\infty_0 \frac{v^{y-1}}{(1+v)^{x+y}}dv.
\end{align*}

The inversion\pageoriginale of the repeated integral is justified by
the use of the fact that the individual integrals converge uniformly
for $x \geq \epsilon >0$, $y \geq \epsilon > 0$.

\item Putting $x = y = \dfrac{1}{2}$ in (vii), we get [using the
  substitution $t = \tan^2 \theta$]
$$
\left\{\Gamma\left(\frac{1}{2}\right)\right\}^2 = 2 \Gamma (1) \int\limits^{\pi/2}_0 d\theta
= \pi.
$$
Since $\Gamma \left(\frac{1}{2}\right) > 0$, we get $\Gamma \left(\frac{1}{2}\right) = \sqrt{\pi}$

Putting $y = 1 -x$, on the other hand, we get the important relation 
$$
\Gamma (x) \Gamma (1-x) = \int\limits^\infty_0 \frac{u^{-x}}{1+u} du =
\frac{\pi}{\sin (1-x)\pi}, \; 0 < x < 1, 
$$
since 
$$
\int\limits^\infty_0 \frac{x^{a-1}}{1+x} dx = \frac{\pi}{\sin a\pi} 
\;\; \text{ for } \;\; 0 < a < 1, \; \text{  by contour}
$$
integration. Hence
$$
\Gamma (x) \Gamma (1-x) = \frac{\pi}{\sin x \pi}, \quad 0 < x< 1.
$$
\end{enumerate}


\section[Analytic continuation of $\Gamma (z)$]{Analytic continuation of {\boldmath$\Gamma (z)$} \protect \cite[p. 148]{key15}}\label{chap6:sec2} 

We have seen that the function
$$
\Gamma (z) = \int\limits^\infty_0 e^{-t} t^{z-1} \; dt
$$
is a regular function for $\re z>0$. We now seek to extend it
analytically into the rest of the complex $z$-plane. Consider
$$
I(z) \equiv \int\limits_C e^{-\zeta} (-\zeta)^{z-1} d\zeta,
$$
where $C$ consists of the real axis from $\infty$ to $\delta >0$, the
circle\pageoriginale $|\zeta|=\delta$ in the `positive' direction
$\curvearrowleft$ and again the real axis from $\delta$ to $\infty$.

We define 
$$
(-\zeta)^{z-1} = e^{(z-1) \log (-\zeta)}
$$
by choosing $\log (-\zeta)$ to be real when $\zeta = - \delta$

The integral $I(z)$ is uniformly convergent in any finite region of
the $Z$-plane, for the only possible difficulty is at $\zeta =
\infty$, but this was covered by case (iii) in \S\ \ref{chap6:sec1}.  \textit{Thus
$I(z)$ is regular for all finite values of $z$.}

We shall evaluate $I(z)$. For this, set
$$
\zeta = \rho e^{i\varphi}
$$
so that
$$
\log (-\zeta) = \log \rho + i (\varphi - \pi) \text{ on the contour }
$$
[so as to conform with the requirement that $\log (-\zeta)$ is real
  when $\zeta = - \delta$]

Now the integrals on the portion of $C$ corresponding to $(\infty,
\delta)$ and $(\delta, \infty)$ give [since $\varphi = 0$ in the first
case, and $\varphi = 2 \pi$ in the second]:
\begin{align*}
&\quad \int\limits^\delta_\infty e^{-\rho + (z-1) \cdot (\log \rho - i \pi)}
d \rho + \int\limits^\infty_\delta e^{-\rho + (z-1) (\log \rho + i
  \pi)} d\rho\\
& =  \int\limits^\infty_\delta \left[-e^{-\rho + (z-1) (\log \rho
    -i\pi)}  + e^{-\rho + (z-1) (\log \rho + i \pi)}\right] d\rho\\
& = \int\limits^\infty_\delta e^{-\rho + (z-1) \log \rho}
\left[e^{(z-1)i\pi} -e^{-(z-1)i\pi} \right] d\rho\\
& = -2 i \sin z\pi \int\limits^\infty_\delta e^{-\rho} \cdot
\rho^{z-1} d\rho
\end{align*}

On the circle $|\zeta| =\delta$, we have
\begin{align*}
|(-\zeta)^{z-1}| & = |e^{(z-1) \log (-\zeta)}| = |e^{(z-1)[\log \delta
+ i (\varphi - \pi)]}|\\
& = e^{(x-1) \log \delta - y (\varphi-\pi)}\\
& = O(\delta^{x-1}).
\end{align*}\pageoriginale 

The integral round the circle $|\zeta| =\delta$ therefore gives 
$$
O(\delta^x) = O(1), \text{ as } \delta \to 0, \text{ if } x > 0.
$$
Hence, letting $\delta \to 0$, we get
\begin{align*}
I(z) & = - 2 i \sin \pi z \int\limits^{\infty}_0 e^{-\rho} \rho^{z-1}
d\rho, \re z > 0\\
& = - 2 i \sin \pi z \Gamma (z), \; \re z > 0.
\end{align*}

We have already noted that $I(z)$ is regular for {\bf all} finite $z$; so 
$$
\frac{1}{2} i I (z) \cosec \pi z
$$
is regular for all finite $z$, {\em except} (possibly) for the poles of
$\cosec \pi z$ namely, $z=0$, $\pm 1$, $\pm 2, \ldots $; and it equals
$\Gamma (z)$ for $\re z > 0$.

\textit{Hence  - $\frac{1}{2} i I (z) \cosec z \pi$ is the analytic
  continuation of $\Gamma (z)$ all over the $z$-plane.}

We know, however, by (iii) of \S\ \ref{chap6:sec1}, that $\Gamma (z)$ is regular for
$z=1,2,3,\ldots$. Hence the only {\em possible} poles of $\dfrac{1}{2}i I
(z) \cosec \pi x$ are $z =0,-1,-2, \ldots$. {\em These are actually} poles
of $\Gamma (z)$, for if $z$ is one of these numbers, say $-n$, then
$(-\zeta)^{z-1}$ is single-valued in $0$ and $I(z)$ can be evaluated
directly by Cauchy's theorem. Actually
\begin{align*}
& (-1)^{n+1} \int\limits_C \frac{e^{-\zeta}}{\zeta^{n+1}} d\zeta =
  \frac{2\pi i}{n!} (-1)^{n+n+1} =  \frac{-2\pi i}{n!}\\
\text{i.e. } & I(-n) = - \frac{-2\pi i}{n!}.
\end{align*}

So the\pageoriginale poles of $\cosec \pi z$, at $z=0$, $-n$, are
actually poles of $\Gamma (z)$. The residue at $z=-n$ is therefore
$$
\lim\limits_{z\to -n} \left(\frac{-2\pi i}{n!} \right)
\left(\frac{z+n}{-2i \sin z\pi} \right) = \frac{(-1)^n}{n!}
$$ 
The formula in (viii) of \S\ \ref{chap6:sec1} can therefore be upheld for complex
values.

Thus
$$
\Gamma (z) \cdot \Gamma (1-z) = \pi \cosec \pi z 
$$
for all \textit{non-integral values of $z$}. Hence, also,
$$
\frac{1}{\Gamma(z)} \text{ is an entire function.}
$$
(Since the poles of $\Gamma (1-z)$ are cancelled by the zeros of $\sin
\pi z$)


\section{The Product Formula}\label{chap6:sec3}

We shall prove that $\dfrac{1}{\Gamma (z)}$ is an entire function of
order 1.

Since $I(z) = - 2 i \sin z \pi$ $\Gamma (z)$, and 
$$
\Gamma (z) \cdot \Gamma (1-z) = \frac{\pi}{\sin \pi z},
$$
we get
\begin{align*}
I(1-z) & = - 2 i \sin (\pi - z \pi) \Gamma (1-z)\\
& = -2 i \sin z \pi \frac{\pi}{\sin \pi z} \cdot \frac{1}{\Gamma(z)}\\
& = - \frac{2 i \pi}{\Gamma (z)}
\end{align*}
or
$$
\frac{1}{\Gamma(z)} = \frac{I(1-z)}{-2i\pi}.
$$
Now\pageoriginale 
\begin{align*}
I(z) & = O \left[e^{\pi|z|} \left\{\int\limits^1_0 e^{-t} t^{x-1} d t
  + \int\limits^\infty_1 e^{-t} t^{x-1} dt \right\} \right]\\
& = O \left[e^{\pi|z|} \left\{ 1+\int\limits^\infty_1 e^{-t} t^{|z|}
  dt\right\} \right] \\
& = O \left[e^{\pi|z|} \left\{1 + \int\limits^{1+|z|}_1 +
  \int\limits^{\infty}_{1+|z|} \right\} \right] \\
& = O \left[e^{\pi|z|} (|z| +1)^{|z| +1} \right]\\
& = O \left[e^{|z|^{1+\epsilon}} \right], \; \epsilon > 0.
\end{align*}

Hence $\dfrac{1}{\Gamma(z)}$ is of order $\leq 1$. However, the
exponent of convergence of its zeros equals 1. Hence the order is
$=1$, and the genus of the canonical product is also equal to 1. 

Thus 
$$
\frac{1}{\Gamma(z)} = z\cdot e^{az+b} \prod\limits^{\infty}_{n=1}
\left(1+\frac{z}{n}\right) e^{\frac{-z}{n}}
$$
by Hadamard's theorem. However, we know that $\lim\limits_{z\to 0+}
\dfrac{1}{z\Gamma (z)} =1$ and $\Gamma(1)=1$. These substitutions give
$b=0$, and 
$$
1 = e^a \Pi \left(1+\frac{1}{\pi} \right) e^{-1/n}
$$
or,
\begin{align*}
0 &= a + \sum\limits^\infty_1 \left[\log \left(1+\frac{1}{n} \right)
  -\frac{1}{n} \right] \\
& = a - \lim\limits_{m \to \infty} \left\{\sum\limits^m_{1} \left[\log
\left(1+\frac{1}{n} \right) - \frac{1}{n} \right] \right\} \\
& = a + \lim\limits_{m \to \infty} \left\{ \sum\limits^m_1 \left[\log
  (n+1) - \log n \right] - \sum\limits^m_1 \frac{1}{n}\right\}\\
& = a + \lim\limits_{m\to\infty} \left[\log(m+1) - \sum\limits^m_1
  \frac{1}{n} \right] \\
& = a - \gamma, \text{ where $\gamma$ is Euler's constant.}
\end{align*}\pageoriginale 

Hence
$$
\frac{1}{\Gamma (z)} = e^{\gamma z} z \prod\limits^\infty_{n=1}
\left\{\left(1+\frac{z}{n} \right)e^{-z/n} \right\}
$$

As a consequence of this we can derive Gauss's expression for
$\Gamma(z)$. For 
\begin{align*}
\Gamma(z) & = \lim\limits_{n\to\infty} \left[ e^{z(\log n -1 -
    \frac{1}{2} - \ldots -\frac{1}{n})} \frac{1}{z} \cdot
  \frac{e^{z/1}}{1+\frac{z}{1}} \ldots \frac{e^{z/n}}{1+\frac{z}{n}}
\right]\\
& = \lim\limits_{n \to \infty} \left[n^z \cdot \frac{1}{z} \cdot
  \frac{1}{z+1} \cdot \frac{2}{z+2} \ldots \frac{n}{z+n} \right]\\
& = \lim\limits_{n\to\infty} \left[\frac{n^z \cdot n!}{z(z+1) \ldots
    (z+n)} \right] 
\end{align*}

We shall denote
$$
\frac{n^z \cdot n!}{z(z+1) \ldots (z+n)} \equiv \Gamma_n(z),
$$
so that\pageoriginale
$$
\Gamma (z) = \lim\limits_{n \to \infty} \Gamma_n (z).
$$

Further, we get by logarithmic-differentiation,
$$ 
\frac{\Gamma' (z)}{\Gamma(z)} = -\gamma - \frac{1}{z} -
\sum\limits^\infty_{n=1} \left(\frac{1}{z+n} -\frac{1}{n} \right).
$$

Thus
$$
\frac{d^2}{dz^2} [\log \Gamma (z)] = \sum\limits^\infty_{n=0}
\frac{1}{(n+z)^2}; 
$$

Hence
$$
\frac{d^2}{dz^2} [\log \Gamma (z)] > 0 \text{ for real, positive $z$.}
$$
