\chapter{Schwarz's Lemma}\label{chap3}

\setcounter{thm}{0}
\begin{thm}\label{chap3:thm1}
Let\pageoriginale $f$ be regular in $|z| < 1$, $f(0) =0$.
If, for $|z|<1$, we have $|f(z)| \leq 1$, then
$$
|f(z)| \leq |z|, \quad \text{for} \quad |z|<1.
$$

Here, equality holds only if $f(z) \equiv c \; z$ and $|c| =1$. We
further have $|f'(0)| \leq 1$.
\end{thm}

\begin{proof}
$\dfrac{f(z)}{z}$ is regular for $|z|<1$. Given $o< r<1$ choose $\rho$
  such that $r < \rho <1$; then since $|f(z)| \leq 1$ for $|z| =
  \rho$, it follows by the maximum modulus principle that
$$
\left|\frac{f(z)}{z} \right| \leq \frac{1}{\rho},
$$
also for $|z| =r$. Since the L.H.S is independent of $\rho \to 1$, we let
$\rho$ and obtain $|f(z)| \leq |z|$, for $|z|<1$.

If for $z_0$,  $(|z_0| <1)$, we have $|f(z_0)| = |z_0|$, then $\left|
\dfrac{f(z_0)}{z_0}\right| =1$, \Big(by the maximum principle applied to
$\dfrac{f(z)}{z}$\Big) hence $f=cz$, $|c|=1$.

Since $f(z) = f'(0)z+f''(0)z^2 + \cdots$ in a neighbourhood of the
origin, and since $\left| \dfrac{f(z)}{z}\right| \leq 1$ in $z$ 1, we get
$|f' (0)| \leq 1$. More generally, we have
 \end{proof}

\medskip
\noindent{\textbf{Theorem {\boldmath$1'$}:}}
Let $f$ be regular and $|f| \leq M$ in $|z| < R$, and $f(0) =0$. Then 
$$
|f(z)| \leq \frac{M|z|}{R}, \quad |z|<R.
$$
In particular,\pageoriginale this holds if $f$ is regular in $|z|<R$,
and continuous in $|z|\leq R$, and $|f| \leq M$ on $|z|=R$.

\begin{thm}[Caratheodory's Inequality]\label{chap3:thm2}
 \cite[p.112]{key12} Suppose that 
\begin{itemize}
\item[(i)] $f$ is regular in $|z|<R$ $f$ is non-constant

\item[(ii)] $f(0) = 0$

\item[(iii)] $\re f \leq U$ in $|z| < R$. (Thus $U > 0$, since $f$ is
  not a constant).
\end{itemize}

Then,
$$
|f| \leq \frac{2U\rho}{R-\rho}, \text{ for } |z| = \rho < R.
$$
\end{thm}

\begin{proof}
Consider the function
$$
\omega (z) = \frac{f(z)}{f(z) - 2U}.
$$

We have $\re \{f- 2 U\} \leq - U <0$, so that $\omega$ is regular in
$|z|<R$.

If $f=u + iv$, we get
$$
|\omega| = \sqrt{\frac{u^2 + v^2}{(2U - u)^2 + v^2}}
$$
so that
$$
|\omega| \leq 1, \text{ since } 2U - u \geq |u|.
$$

But $\omega(0) = 0$, since $f(0) = 0$. Hence by Theorem \ref{chap3:thm1}, 
$$
|\omega(z)| \leq \frac{|z|}{R}, \quad |z| < R.
$$

\eject

But $f = - \dfrac{2U\omega}{1-\omega}$. Now take $|z| = \rho <R$.



Then, we have,\pageoriginale 
$$
|f| \leq \frac{2U |\omega|}{1-|\omega|} \leq \frac{2U
  \rho}{R-\rho} 
$$
\indent
N.B.~  If  \quad $|f(z_0)| = \dfrac{2U
  \rho}{R-\rho}$, for $|z_0| < R$, then
\begin{align*}
f(z) & = \frac{2U \cdot cz}{1-cz}, \quad |c| = 1.
\end{align*}

More generally, we have
\end{proof}

\medskip
\noindent{\textbf{Theorem {\boldmath$2'$:}}} Let $f$ be regular in
$|z|<R$; $f$ non-constant
$$
f(0) = a_0 = \beta + i \gamma
$$
$\re f \leq U (|z| < R)$, so that $U \geq \beta$

Then
$$
|f(z) - a_0| \leq \frac{2(U - \beta)\rho}{R-\rho}
$$
for $|z| = \rho < R$


\begin{remark*}
If $f$ is a constant, then Theorem \ref{chap3:thm2} is trivial.
We shall now prove Borel's inequality which is sharper than Theorem \ref{chap3:thm2}.
\end{remark*}

\begin{thm}[Borel's Inequality]\label{chap3:thm3}
\cite[p.114]{key12} Let $f(z) = \sum\limits^\infty_{n=0}
a_n z^n$, be regular in $|z| <1$. Let $\re f \leq U$. $f(0) = a_0 =
\beta + i \gamma$.
Then
$$
|a_n| \leq 2 (U - \beta) \equiv 2 \quad U_1, \text{ say, } n >0.
$$
\end{thm}

\begin{proof}
We shall first prove that $|a_1| <U_1$, and then for general $a_n$.
\begin{itemize}
\item[(i)] Let\pageoriginale $f_1 \equiv \sum\limits^\infty_{n=1} a_n
  z^n$. Then $\re f_1 \leq U_1$.

\eject

For $|z| = \rho < 1$, we have, by Theorem \ref{chap3:thm2}, 
\begin{align*}
|f_1| &\leq \frac{2U_1 \rho}{1-\rho}\\
\text{i.e.}\quad \left|\frac{f_1(z)}{z} \right| &\leq \frac{2U_1}{1-\rho}
\end{align*}

Now letting $z \to 0$, we get
$$
|a_1| \leq 2 U_1
$$

\item[(ii)] Define $\omega =e^{2\pi i/k}$, $k$ a $+$ve integer.

Then
\begin{align*}
\sum\limits^k_{v=1} \omega^{\nu m} & = 0\;
\begin{cases}
& \text{if } m \neq 0\\
\& & \text{if } m \neq \text{a multiple of $k$}
\end{cases}\\
& = k  \;
\begin{cases}
& \text{if } m =0\\
\text{or} & \text{if } m = \text{ a multiple of $k$}.
\end{cases}
\end{align*}
\end{itemize}

We have
$$
\frac{1}{k} \sum\limits^{k-1}_{r=0} f_1 (\omega^r z) =
\sum\limits^\infty_{n=1} a_{nk} z^{nk} = g_1 (z^k) \equiv g_1 (\zeta),
\text{ say}.
$$

The series for $g_1$ is convergent for $|z| <1$, and so for $|\zeta|
<1$. Hence $g_1$ is regular for $|\zeta|<1$. Since $\re g_1 \leq
U_1$, we have

$|$ coefficient of $\zeta| \leq 2 U_1$ by the first part of the proof 
i.e.\pageoriginale $|a_k| \leq 2 U_1$.
\end{proof}

\setcounter{corollary}{0}
\begin{corollary}\label{chap3:coro1}
Let $f$ be regular in $|z| < R$, and $\re \{f(z) -f(0)\} \leq
U_1$. Then
$$
|f'(z)| \leq \frac{2U_1 R}{(R -\rho)^2} , \quad |z| =\rho < R
$$
\end{corollary}

\begin{proof}
Suppose $R = 1$. Then
$$
|f'(z)| \leq \sum n |a_n| \rho^{n-1} \leq 2 U_1 \sum n \rho^{n-1} 
$$

This argument can be extended to the $n^{\rm th}$ derivative.
\end{proof}

\begin{corollary}\label{chap3:coro2}
If $f$ is regular for every finite $z$, and 
$$
e^{f(z)} = \mathcal{O} (e^{|z|^k}), \text{ as } |z| \to \infty,
$$
then $f(z)$ is a polynomial of degree $\leq k$.
\end{corollary}

\begin{proof}
Let $f(z) =\sum a_n z^n$. For large $R$, and a fixed $\zeta$,
$|\zeta|<1$, we have
$$
\re \{f(R \zeta )\} < c R^k.
$$

By Theorem \ref{chap3:thm3}, we have
$$
|a_n R^n| \leq 2 (2 R^k - \re a_0), \quad n >0.
$$

Letting $R \to \infty$, we get
$$
a_n = 0 \text{ if } n > k.
$$

Apropos Schwarz's Lemma we give here a formula and an inequality which
are useful.
\end{proof}

\begin{thm}\label{chap3:thm4}
Let\pageoriginale $f$ be regular in $|z-z_0| \leq r$
$$
f = P(r,\theta) + Q (r,\theta).
$$

Then
$$
f'(z_0) = \frac{1}{\pi r} \int\limits^{2\pi}_0 P(r,\theta) e^{ - i\theta}
d \theta.
$$
\end{thm}

\begin{proof}
By Cauchy's formula,
 $$
{\rm (i)} \qquad  f(z_0) = \frac{1}{2\pi i } \int\limits_{|z-z_0|=r}
\frac{f(z)dz}{(z-z_0)^2} =\frac{1}{2\pi r} \int\limits^{2\pi}_0 (P+iQ)
e^{-i\theta}d \theta 
$$
 
By Cauchy's theorem,
$$
{\rm (ii)} \qquad 0 = \frac{1}{r^2} \cdot \frac{1}{2\pi i}
\int\limits_{|z-z_0|=r} f(z) dz = \frac{1}{2\pi r}
\int\limits^{2\pi}_{0} (P + iQ) e^{i\theta} d\theta
$$

We may change $i$ to $-i$ in this relation, and add (i) and (ii).

Then
$$
f'(z_0) = \frac{1}{\pi r} \int\limits^{2\pi}_0 P(r, \theta)
e^{-i\theta} d\theta
$$
\end{proof}

\begin{coro*}
If $f$ is regular, and $|\re f| \leq M$ in $|z-z_0| \leq r$, then
$$
|f'(z_0)| \leq \frac{2M}{r}.
$$
\end{coro*}

\medskip
\noindent{\textbf{Aliter:}}
We have obtained a series of results each of which depended on the
preceding. We can reverse this procedure, and state one general result
from which the rest follow as consequences.

\begin{thm}\label{chap3:thm5}
\cite[p.50]{key11} Let\pageoriginale $f(z) \equiv \sum\limits^\infty_{n=0}$ $c_n (z-z_0)^n$
be regular in $|z-z_0| <R$, and $\re f < U$. Then
\begin{equation*}
|e_n| \leq \frac{2(U - \re c_0)}{R^n}, \quad n = 1,2,3, \ldots \tag{5.1}\label{c3:eq5.1}
\end{equation*}
and in $|z-z_0| \leq \rho <R$, we have
\begin{align*}
& |f(z) - f(z_0)| \leq \frac{2\rho}{R - \rho} \left\{ U - \re
  f(z_0)\right\} \tag{5.2}\label{c3:eq5.2}\\
& \left|\frac{f^{(n)} (z)}{n!} \right| \leq \frac{2R}{(R-\rho)^{n+1}}
  \left\{ U - \re f(z_0) \right\} n = 1,2,3, \ldots \tag{5.3}\label{c3:eq5.3}
\end{align*}
\end{thm}

\begin{proof}
We may suppose $z_0 = 0$.
\begin{align*}
\text{Set } \quad \phi (z) = U - f(z) & = U - c_0 -
\sum\limits^{\infty}_1 c_n z^n \\
& \equiv \sum\limits^\infty_0 b_n z^n, \quad |z| <R.
\end{align*}
Let $\gamma$ denote the circle $|z| = \rho < R$. Then
\begin{align*}
b_n = \frac{1}{2\pi i} \int\limits_{\gamma} \frac{\phi(z)dz}{z^{n+1}}
= \frac{1}{2\pi \rho^n} \int\limits^\pi_{-\pi} (P +
iQ)e^{-\text{in}\theta} d \theta, n \geq 0, \tag{5.4}\label{c3:eq5.4}
\end{align*}
where $\phi (r, \theta) = P(r, \theta) + i Q (r, \theta)$. Now, if $n
\geq 1$, then $\phi (z) z^{n-1}$ is regular is $\gamma$, so that
\begin{align*}
0 = \frac{\rho^n}{2r} \int\limits^\pi_{-\pi} (P+iQ) e^{in\theta} d
\theta, n \geq 1 \tag{5.5}\label{c3:eq5.5}
\end{align*}

Changing $i$ to $-i$ in (\ref{c3:eq5.5}) and adding this to (\ref{c3:eq5.4}) we get
$$
b_n \rho^n = \frac{1}{\pi} \int\limits^\pi_{-\pi} P e^{-n \theta i}
d\theta, \quad n \geq 1.
$$
But\pageoriginale $P= U - \re f \geq 0$ in $|z| < R$ and so in
$\gamma$. Hence, if  $n \geq 1$,
\begin{equation*}
|b_n| \rho^n \leq \frac{1}{\pi} \int\limits^\pi_{-\pi} |P e^{-n\theta
  i}| d \theta = \frac{1}{\pi} \int\limits^{\pi}_{-\pi} P d \theta = 2
\re b_0, \tag{5.6}\label{c3:eq5.6}
\end{equation*}
on using (\ref{c3:eq5.4}) with $n=0$. Now letting $\rho \to R$, we get
$$
|b_n| R^n \leq 2 \beta_0 \equiv 2 \re b_0.
$$

Since $b_0 = U - c_0$, and $b_n = - c_n$, $n \geq 1$, we at once
obtain (\ref{c3:eq5.1}). We then deduce, for $|z| \leq \rho < R$
\begin{equation*}
|\phi (z) - \phi (0)| = \left|\sum\limits^\infty_1 b_n z^n\right| \leq
\sum\limits^\infty_1 2 \beta_0 \left( \frac{\rho}{R}\right)^n =
\frac{2\beta_0\rho}{R-\rho}  \tag{5.7} \label{c3:eq5.7}
\end{equation*}
and if $n \geq 1$,
\begin{align*}
|\phi^{(n)} (z)| & \leq \sum\limits^\infty_{r=n} r(r-1) \ldots (r-n+1)
\frac{2\beta_0 \rho^{r-n}}{R^r}\\
& = \left(\frac{d}{d\rho} \right)^n \sum\limits^\infty_{n=0} 2 \beta_0
\left(\frac{\rho}{R} \right)^r\\
& = \left(\frac{d}{d\rho} \right)^n \frac{2\beta_e R}{R-\rho}\\
& = 2 \frac{\beta_0 R\cdot n!}{(R-\rho)^{n+1}} \tag{5.8}\label{c3:eq5.8}
\end{align*}
(\ref{c3:eq5.7}) and (\ref{c3:eq5.8}) yield the required results on substituting $\phi =
U - f$ and $\beta_0 = U - \re f(0)$.
\end{proof}

