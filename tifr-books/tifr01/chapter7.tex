\chapter{The Gamma Function: Contd.}\label{chap7}

\setcounter{section}{3}
\section[The Bohr-Mollerup-Artin Theorem]{The Bohr-Mollerup-Artin Theorem\hfil\break 
\protect \cite[Bd.I, p.276]{key7}}\label{chap7:sec4} 

We\pageoriginale shall prove that the functional equation of
$\Gamma(x)$, namely $\Gamma(x+1) =x \Gamma(x)$, together with the fact
that $\dfrac{d^2}{dx^2} [\log \Gamma (x)] > 0$ for $x>0$, determines
$\Gamma(x)$ `essentially' uniquely-essentially, that is, except for a
constant of proportionality. If we add the normalization condition
$\Gamma(1) = 1$, then these three properties determine $\Gamma$
uniquely. N.B. The functional equation \textit{alone} does not define
$\Gamma$ uniquely since $g(x) = p(x) \Gamma(x)$, where $p$ is any
analytic function of period 1 also satisfies the same functional
equation.

We shall briefly recall the definition of `Convex functions'. A
real-valued function $f(x)$, defined for $x>0$, is \textit{convex}, if
the corresponding function
$$
\phi (y) = \frac{f(x+y)-f(x)}{y},
$$
defined for all $y> -x$, $y \neq 0$, is monotone increasing throughout
the range of its definition.

If $0 < x_1 < x < x_2$ are given, then by choosing $y_1 = x_1 - x$ and
$y_2 = x_2 -x$, we can express the condition of convexity as:
$$
f(x) \leq \frac{x_2 -x}{x_2 -x_1} f(x_1) + \frac{x-x_1}{x_2 -x_1}
f(x_2) 
$$
 
Letting\pageoriginale $x\to x_1$, we get $f(x_1 + 0) \leq f(x_1)$; and
letting $x_2 \to x$, we get $f(x) \leq f(x+0)$ and hence $f(x_1 +0) =
f(x_1)$ for all $x_1$. Similarly $f(x_1 -0) = f(x_1)$ for all $x_1$.


Thus \textit{a convex function is continuous}. 

Next, if $f(x)$ is twice (continuously) differentiable, we have
$$
\varphi'(y) = \frac{yf'(x+y)-f(x+y) + f(x)}{y^2}
$$
and writing $x+y = u$, we get
$$
f(x) = f(u-y) = f(u) - y f' (u) + \frac{y^2}{2} f''[u-(1-\theta)y], 0
\leq \theta \leq 1
$$
which we substitute in the formula for $\varphi'(y)$ so as to obtain
$$
\varphi'(y) = \frac{1}{2} f''[x + \theta y].
$$
Thus if $f'' (x )\geq 0$ for all $x>0$, then $\varphi'(y)$ is monotone
increasing and $f$ is convex. [The converse is also true].

Thus $\log \Gamma(x)$ is a convex function; by the last formula of the
previous section we may say that $\Gamma(x)$ is
\textit{`logarithmically convex'} for $x>0$. Further $\Gamma(x)>0$ for
$x>0$.

\begin{theorem*}
$\Gamma(x)$ is {\em the} positive function uniquely defined for $x>0$ by the
  conditions: 
\begin{itemize}
\item[1.] $\Gamma(x+1) = x\Gamma(x)$

\item[2.] $\Gamma(x)$ is logarithmically convex

\item[3.] $\Gamma (1) =1$.
\end{itemize}
\end{theorem*}

\begin{proof}
Let $f(x)>0$ be any function satisfying the above three conditions
satisfied by $\Gamma(x)$.

Choose\pageoriginale an integer $n>2$, and $0<x\leq 1$.

Let 
$$
n-1 < n < n + x \leq n +1.
$$
By logarithmic convexity, we get
\begin{align*}
\frac{\log f(n-1) - \log f(n)}{(n-1)-n} & < \frac{\log f(n+x) - \log
  f(n)}{(n+x)-n} \\
& \leq \frac{\log f(n+1) - \log f(n)}{(n+1)-n}
\end{align*}
Because of conditions 1 and 3, we get
$$
f(n-1) = (n-2)!, \; f(n) = (n-1)!, f(n+1) = n!
$$
and
$$
f(n+x) = x(x+1) \cdots (x+n-1) f(x).
$$
Substituting these in the above inequalities we get
$$
\log (n-1)^x < \log \frac{f(n+x)}{(n-1)!} \leq \log n^x,
$$
which implies, since log is monotone,
\begin{gather*}
\text{or } \qquad (n-1)^x < \frac{f(n+x)}{(n-1)!} \leq n^x\\
\frac{(n-1)! (n-1)^x}{x(x+1) \cdots (x+n-1)} < f(x) \leq
\frac{(n-1)!n^x}{x(x+1) \cdots (x+n-1)}
\end{gather*}
Replacing $(n-1)$ by $n$ in the first inequality, we get
$$
\Gamma_n(x) < f(x) \leq \Gamma_n(x) \cdot \frac{x+n}{n}, n = 2, 3, \ldots
$$ 
where $\Gamma_n$ is defined as in Gauss's expression for $\Gamma$.

Letting $n\to \infty$, we get
$$
f(x) = \Gamma(x), \; 0 < x \leq 1.
$$

For values of $x>1$, we uphold this relation by the functional equation.
\end{proof}

\section[Gauss's Multiplication Formula]{Gauss's Multiplication Formula. \cite[Bd.I, p.281]{key7}}\label{chap7:sec5}

Let\pageoriginale  $p$ be a positive integer, and 
$$
f(x) = p^x \Gamma \left(\frac{x}{p} \right) \Gamma \left(\frac{x+1}{p}
\right) \cdots \Gamma \left( \frac{x+p-1}{p}\right)
$$
[If $p=1$, then $f(x) = \Gamma(x)$]. For $x>0$, we then have 
$$
\frac{d^2}{dx^2} [\log f(x)] > 0.
$$
Further 
$$
f(x+1) = xf(x).
$$
Hence by the Bohr-Artin theorem, we get
$$
f(x) = a_p \Gamma(x).
$$
Put $x=1$. Then we get
\begin{equation*}
a_p = p \Gamma \left(\frac{1}{p} \right) \cdot \Gamma
\left(\frac{2}{p} \right) \cdots \Gamma \left(\frac{p}{p} \right)
\tag{5.1}\label{c7:eq5.1}
\end{equation*}
[$a_1 = 1$, by definition]. Put $k = 1, 2, \ldots, p$, in the relation
$$
\Gamma_n \left( \frac{k}{p}\right) = \frac{n^{k/p} n! p^{n+1}}{k(k+p)
  \cdots (k+np)},
$$
and multiply out and take the limit as $n \to \infty$. We then get
from (\ref{c7:eq5.1}),
$$
a_p = p. \lim\limits_{n\to\infty} \frac{n^{\frac{p+1}{2}} (n!)^p p^{np+p}}{(np+p)!}
$$
However
$$
(np+p)! = (np)! (np)^p \left[\left(1+\frac{1}{np} \right) \;
  \left(1+\frac{2}{np}  \right)  \cdots \left(
  1+\frac{p}{np}\right)\right]  
$$
Thus, for fixed $p$, as $n \to \infty$, we get
$$
a_p = p \lim\limits_{n \to \infty} \frac{(n!)^p p^{np}}{(np)!
  n^{(p-1)/2}} 
$$

Now\pageoriginale by the asymptotic formula obtained in (v) of \S\ \ref{chap6:sec1},
[see p.49]
\begin{align*}
(n!)^p & = a^p n^{np+p/2} e^{-np} e^{o(1)}\\
(np)! & = a n^{np+\frac{1}{2}} p^{np+\frac{1}{2}} e^{-np}e^{o(1)}
\end{align*}

Hence
$$
a_p = \sqrt{p} \cdot a^{p-1}.
$$
Putting $p=2$ and using (\ref{c7:eq5.1}) we get
\begin{equation*}
a = \frac{1}{\sqrt{2}} , \; a_2 = \frac{1}{\sqrt{2}} 2 \Gamma
\left(\frac{1}{2} \right) \Gamma (1) = \sqrt{2\pi} \tag{5.2}\label{c7:eq5.2}
\end{equation*}

Hence
$$
a_p = \sqrt{p} (2\pi)^{(p-1)/2}. 
$$

Thus
$$
p^x \Gamma \left( \frac{x}{p}\right) \cdots \Gamma \left(
\frac{x+p-1}{p}\right) = \sqrt{p} (2\pi)^{(p-1)/2} \Gamma(x)
$$
For $p=2$ and $p=3$ we get:
$$  
\Gamma \left(\frac{x}{2} \right) \Gamma \left(\frac{x+1}{2} \right) =
\frac{\sqrt{\pi}}{2^{x-1}}  \Gamma(x) ;\;\; \Gamma \left(\frac{x}{3}
\right) \Gamma \left(\frac{x+1}{3} \right) \Gamma \left(\frac{x+2}{3}
\right)  = \frac{2\pi}{3^{x-\frac{1}{2}}} \Gamma (x).
$$

\section[Stirling's Formula]{Stirling's Formula \cite[p.150]{key15}}\label{chap7:sec6}

From (v) of \S\ \ref{chap6:sec1}, p.49 and (\ref{c7:eq5.2}) we get
$$
(n!) = \sqrt{2\pi} \cdot e^{-n} \cdot n^{n+\frac{1}{2}} e^{o(1)},
$$
which is the same thing as 
\begin{equation*}
\log (n-1)! = \left(n - \frac{1}{2} \right) \log n - n + \log \sqrt{2\pi} + o(1)
\tag{6.1}\label{c7:eq6.1} 
\end{equation*}\pageoriginale 
Further
\begin{equation*}
1 + \frac{1}{2} + \cdots + \frac{1}{N-1} - \log N = \gamma + o (1),
\text{ as } N \to \infty \tag{6.2}\label{c7:eq6.2}
\end{equation*}
and 
\begin{align*}
\log (N+z) & = \log N + \log \left( 1+\frac{z}{N}\right)\\
& = \log N + \frac{z}{N} + O \left(\frac{1}{N^2} \right), \text{ as }
N \to \infty. \tag{6.3}\label{c7:eq6.3}
\end{align*}

We need to use (\ref{c7:eq6.1})-(\ref{c7:eq6.3}) for obtaining the asymptotic formula for
$\Gamma (z)$ where $z$ is complex.

From the product-formula we get
\begin{equation*}
\log \Gamma (z) = \sum\limits^\infty_{n=1} \left\{\frac{z}{n} - \log
\left(1+\frac{z}{n} \right) \right\}  - \gamma z - \log z, \tag{6.4}\label{c7:eq6.4}
\end{equation*}
each logarithm having its principal value.

Now
\begin{align*}
& \int\limits^N_0 \frac{[u] - u + \frac{1}{2}}{u+z} du =
\sum\limits^{N-1}_{n=0} \int\limits^{n+1}_n \left(\frac{n+\frac{1}{2}
  +z}{u+z} -1 \right) du\\
& = \sum\limits^{N-1}_{n=0} \left(n + \frac{1}{2}+z\right) [\log (n+1+z) -\log
  (n-z)] - N\\
& = \sum\limits^{N-1}_{n=0} n [\log (n+1+z) - \log (n+z)]\\
&\qquad + \left(z +
\frac{1}{2}\right) \sum\limits^{N-1}_{n=0}
\left[\log (n+1+z) - \log (n+z) \right] -N\\
& = (N-1) \log (N+z) - \sum\limits^{N-1}_{n=1} \log (n+z)\\
&\qquad +
\left(z+\frac{1}{2} \right) \log (N+z)
 - \left(z+\frac{1}{2} \right) \log z - N\\
& = \left(N - \frac{1}{2} + z \right) \log (N+z) - \left(z+\frac{1}{2}
\right) \log z\\
&\qquad - N - \sum\limits^{N-1}_{n=1} \left[\log n + \log
  \left(1+\frac{z}{n} \right) \right]\\
& = \left(N - \frac{1}{2} +z \right) \log(N+z) - \left(
z+\frac{1}{2}\right) \log  z - N - \log (N-1)!\\
& \qquad + \sum\limits^{N-1}_{n=1} \left\{\frac{z}{n} - \log
\left(1+\frac{z}{n} \right) \right\}  - z \sum\limits^{N-1}_{1}
\frac{1}{n}
\end{align*}\pageoriginale 

Now substituting for $\log (N+z)$ etc., from (\ref{c7:eq6.1})-(\ref{c7:eq6.3}), we get
\begin{align*}
\int\limits^{N}_0 \frac{[u] - u + \frac{1}{2}}{u + z} du & = \left(N - 
\frac{1}{2} + z\right) \left\{\log N + \frac{z}{N} + O \left(\frac{1}{N^2}
  \right) \right\}\\
&\qquad  - \left(z+\frac{1}{2} \right)\log z - N- \left(N - \frac{1}{2} \right) \log N + N - c\\
&\qquad + O(1) +\sum\limits^{N-1}_{n=1} \left\{\frac{z}{n} - \log
  \left(1+\frac{z}{n} \right) \right\} -z \sum\limits^{N-1}_1
  \frac{1}{n} \\
& = z \left(\log N - \sum\limits^{N-1}_1 \frac{1}{n} \right)  + z  +
  O(1) - \log \sqrt{2\pi}\\
& \qquad - \left(z-\frac{1}{2} \right) \log z - \log z + \sum\limits^{N-1}_{n=1}
  \left\{\frac{z}{n} - \log \left(1+\frac{z}{n} \right) \right\}
\end{align*}

Now\pageoriginale letting $N \to \infty$, and using (\ref{c7:eq6.4}), and (\ref{c7:eq6.2}), we get  
\begin{equation*}
\int\limits^\infty_0 \frac{[u] - u + \frac{1}{2}}{u+z} du +
\frac{1}{2} \log 2 \pi - z + \left(z - \frac{1}{2} \right) \log z = \log \Gamma
(z)\tag{6.5}\label{c7:eq6.5}
\end{equation*}
If $\varphi (u) = \int\limits^u_0 \left([v] - v + \frac{1}{2}\right)dv$, then
$$
\varphi (u) = O(1), 
$$
since $\varphi(n+1) = \varphi (n)$, if $n$ is an integer.

\eject

Hence
\begin{align*}
\int\limits^\infty_0 \frac{[u] - u + \frac{1}{2}}{u+z} du =
\int\limits^\infty_0 \frac{d\varphi(u)}{u+z} & = \int\limits^\infty_0
\frac{\varphi(u)}{(u+z)^2}
 = O \left[ \int\limits^\infty_0 \frac{dv}{|u+z|^2}\right]
\end{align*}

If we write $z = re^{i\theta}$, and $u = ru_1$, then
$$
\int\limits^\infty_0 \frac{[u]-u + \frac{1}{2}}{u+z} du = O
\left[\frac{1}{r} \int\limits^\infty_0
  \frac{du_1}{|u_1+\theta^{i\theta}|^2}\right]  
$$

The last integral is finite if $\theta \neq \pi$. Hence
$$
\int\limits^\infty_0 \frac{[u] - u + \frac{1}{2}}{u+z} du = O
\left(\frac{1}{r} \right), \text{ uniformly for } |\arg z | \leq \pi -
\epsilon < \pi 
$$

Thus we get from (\ref{c7:eq6.5}): 
$$
\log \Gamma (z) = \left(z -\frac{1}{2} \right) \log z -z  + \frac{1}{2} \log 2 \pi
+ O \left( \frac{1}{|z|}\right)
$$
for $|\arg z| \leq \pi - \epsilon < \pi$

\begin{corollaries*}
\begin{enumerate}
\renewcommand{\labelenumi}{(\theenumi)}
\item $\log \Gamma (z+\alpha) = \left(z+\alpha -\frac{1}{2}\right) \log z - z +
\frac{1}{2}  \log 2 \pi + O \left(\frac{1}{|z|} \right)$ as $|z| \to
  \infty$\pageoriginale uniformly for $|\arg z| \leq \pi - \epsilon
  < \pi$, $\alpha$ bounded.

\item For any fixed $x$, as $y \to \pm \infty$
$$
|\Gamma (x+iy)| \sim e^{-\frac{1}{2} \pi |y|} |y|^{x -\frac{1}{2}}
\sqrt{2\pi} 
$$

\item $\dfrac{\Gamma'(z)}{\Gamma(z)} = \log z -\dfrac{1}{2z} + O
  \left(\dfrac{1}{|z|^2} \right)$, $|\arg z| \leq \pi - \epsilon <
  \pi$ \cite[p.57]{key11}

This may be deduced from (1) by using 
$$
f'(z) = \frac{1}{2\pi  i } \int\limits_C \frac{f(\zeta)}{(\zeta -z)^2}
d \zeta, 
$$
where $f(z) = \log \Gamma (z) - \left(z-\frac{1}{2} \right) \log z + z
- \frac{1}{2} \log 2 \pi$ and $C$ is a circle with centre at $\zeta =
z$ and radius $|z|\sin\epsilon/2$.
\end{enumerate}
\end{corollaries*}
