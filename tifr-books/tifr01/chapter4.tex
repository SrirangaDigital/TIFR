
\chapter{Entire Functions}\label{chap4}

An entire\pageoriginale function is an analytic function (in the
complex plane) which has no singularities except at $\infty$. A
polynomial is a simple example. A polynomial $f(z)$ which has zeros at
$z_1, \ldots , z_n$ can be factorized as 
$$
f(z) = f(0) \left(1 -\frac{z}{z_1} \right) \cdots
\left(1-\frac{z}{z_n} \right) 
$$
The analogy holds for entire functions in general. Before we prove
this, we wish to observe that if $G(z)$ is an entire function with no
zeros it can be written in the form $e^{g(z)}$ where $g$ is
entire. For consider $\dfrac{G'(z)}{G(z)}$; every (finite) point is an
`ordinary point' for this function, and so it is entire and equals
$g_1(z)$, say. Then we get
$$
\log \left\{\frac{G(z)}{G(z_0)} \right\} = \int\limits^{z_1}_{z_0} g_1
(\zeta) d \zeta = g(z) - g(z_0), \text{ say,}
$$
so that
$$
G(z) = G (z_0) e^{g(z)-g(z_0)} = e^{g(z) - g(z_0) + \log G(z_0) }
$$
As a corollary we see that if $G(z)$ is an entire function with $n$
zeros, distinct or not, then 
$$
G(z) = (z-z_1) \cdots (z-z_n) e^{g(z)}
$$
where $g$ is entire. We wish to uphold this in the case when $G$ has
an infinity of zeros.


\setcounter{thm}{0}
\begin{thm}[Weierstrass]\label{chap4:thm1}
\cite[p.\,246]{key15} Given\pageoriginale a sequence of complex numbers $a_n$
such that
$$
0 < |a_1| \leq |a_2| \leq \cdots \leq |a_n| \leq \cdots
$$
whose sole limit point is $\infty$, there exists an entire function
with zeros at these points and these points only.
\end{thm}

\begin{proof}
Consider the function
$$
\left(1-\frac{z}{a_n} \right) e^{\mathcal{Q}_\nu (z)},
$$
where $\mathcal{Q}_\nu(z)$ is a polynomial of degree $q$. This is
an entire function which does not vanish except for $z = a_n$. Rewrite
it as 
\begin{align*}
\left(1- \frac{z}{a_n} \right) e^{\mathcal{Q}_\nu (z)} & =
e^{\mathcal{Q}_\nu (z) + \log \left(1-\frac{z}{a_n} \right)}\\
& = e^{ - \frac{z}{a_n} - \frac{z^2}{2a^2} \cdots + \mathcal{Q}_\nu(z)},
\end{align*}
and choose
$$
\mathcal{Q}_\nu(z) = \frac{z}{a_n} +\cdots + \frac{z^\nu}{\nu a^{\nu_n}}
$$
so that
\begin{align*}
\left(1-\frac{z}{a_n} \right) e^{\mathcal{Q}_\nu (z)} & =
e^{-\left(\frac{z}{a_n} \right)^{\nu +1}_{\frac{1}{\nu+1}} -\cdots} \\
& \equiv 1 + U_n (z), \text{ say.}
\end{align*}

We wish to determine $\nu$ in such a way that
$$
\prod\limits^\infty_1 \left(1-\frac{z}{a_n} \right) e^{\mathcal{Q}_\nu(z)}
$$\pageoriginale
is absolutely and uniformly convergent for $|z| < R$, however large
$R$ may~be.

\vfill\eject

Choose an $R >1$, and an $\alpha$ such that $0 < \alpha < 1$; then
there exists a $q$ ($+$ve integer) such that
$$
|a_q| \leq \frac{R}{\alpha}, \quad |a_{q+1}| > \frac{R}{\alpha}.
$$

Then the partial product
$$
\prod\limits^q_1 \left(1-\frac{z}{a_n} \right) e^{\mathcal{Q}_\nu (Z)}
$$
is trivially an entire function of $z$. Consider the remainder
$$
\prod\limits^\infty_{q+1} \left(1-\frac{z}{a_n} \right)
e^{\mathcal{Q}_\nu (z)}
$$
for $|z| \leq R$.

We have, for $n > q$, $|a_n| > \dfrac{R}{\alpha}$ or
$\left|\dfrac{z}{a_n} \right|< \alpha < 1$. Using this fact we shall
estimate each of the factors $\{1 + U_n (z)\}$ in the product, for
$n > q$.
\begin{align*}
|U_n(z)| & = \left|e^{-\frac{1}{\nu+1} \left(\frac{z}{a_n}
  \right)^{\nu+1} - \cdots } -1\right|  \\
& \leq e^{\left| \frac{1}{\nu+1} \left(\frac{z}{a_n} \right)^{\nu+1}
 + \cdots\right|} -1
\end{align*}
\end{proof}

Since\pageoriginale $|e^m -1| \leq e^{|m|} -1$, for $e^m -1 = m +
\dfrac{m^2}{2!}+\cdots$ or
$|e^m-1| \leq |m| + \dfrac{|m^2|}{2} \cdots = e^{|m|}-1$.
\begin{align*}
|U_n(z)| & \leq e^{\left|\frac{z}{a_n} \right|^{\nu+1}
  \left(1+\left|\frac{z}{a_n} \right| + \cdots \right)}  -1,\\
 & \leq e^{\left|\frac{z}{a_n} \right|^{\nu+1} (1+\alpha+\alpha^2 +
  \cdots)}  -1\\ 
& = e^{\frac{1}{1-\alpha} \left|\frac{z}{a_n} \right|^{\nu+1}}-1\\
& \leq \frac{1}{1-\alpha} \left|\frac{z}{a_n} \right|^{\nu+1}  \cdot
e^{\frac{1}{1-\alpha} \;  \left|\frac{z}{a_n} \right|^{\nu+1}}, \text{
  since if $x$ is} 
\end{align*}
real, $e^x -1 \leq x e^x$, for 
$$
e^x -1 = x \left(1+\dfrac{x}{2!} +
\dfrac{x^2}{3!}+\cdots \right) \leq x \left(1+x + \dfrac{x^2}{2!} +\cdots\right)
= x e^x
$$ 

Hence
$$
|U_n (z)| \leq \frac{e^{\frac{1}{1 - \alpha}}}{1-\alpha} \left|
\frac{z}{a_n}\right|^{\nu+1} 
$$

Now\pageoriginale arise two cases: (i) either there
exists an integer $p>0$ such that $\sum\limits^\infty_{n=1}
\dfrac{1}{|a_n|^p} < \infty$ or (ii) there does not. In case (i) take
$\nu = p -1$, so that 
$$ 
|U_n (z)| \leq \frac{R^p}{|a_n|^p} \cdot \frac{e^{\frac{1}{1-a}}}{1-\alpha},
\text{ since } |z| \leq R; 
$$
and hence $\sum |U_n(z)|< \infty$ for $|z| \leq R$,
i.e. $\prod\limits^{\infty}_{q+1}(1+U_n(z))$ is absolutely and
uniformly convergent for $|z| \leq R$.

In case (ii) take $\nu = n - 1$, so that
\begin{align*} 
|U_n(z)| < \frac{e^{\frac{1}{1 - \alpha}}}{1-\alpha} \left|\frac{z}{a_n}
\right|^n, \;  & n > q\\
& \left|\frac{z}{a_n} \right| < 1\\
& |z| \leq R\\
& |a_n| \to \infty
\end{align*}

Then by the `root-test' $\sum |U_n (z)| < \infty$, and the same result
follows as before. Hence the product
$$
\prod\limits^\infty_1 \left(1-\frac{z}{a_n} \right) e^{\mathcal{Q}_\nu(z)}
$$
is analytic in $|z| \leq R$; since $R$ is arbitrary and $\nu$ does not
depend on $R$\pageoriginale we see that the above product represents
an entire function.

\begin{remark*}
If in addition $z = 0$ is also a zero of $G(z)$ then
$\dfrac{G(z)}{z^mG(z)}$ has no zeros and equals $e^{g(z)}$, say, so
that 
$$
G(z) = e^{g(z)} \cdot z^m \prod\limits^\infty_{n=1} 
\left(1-\frac{z}{a_n} \right) e^{Q_\nu (z)}
 $$
\end{remark*}

In the above expression for $G(z)$, the function $g(z)$ is an
\textit{arbitrary} entire function. If $G$ is subject to further
restrictions it should be possible to say more about $g(z)$; this we
shall now proceed to do. The class of entire functions which we shall
consider will be called ``entire functions of finite order''.

\begin{defi*}
An entire function $f(z)$ is of finite order if there exists a
constant $\lambda$ such that $|f(z)| < e^{r^\lambda}$ for $|z| = r >
r_0$. For a non-constant $f$, of finite order, we have $\lambda
>0$. If the above inequality is true for a certain $\lambda$, it is
also true for $\lambda' > \lambda$. Thus there are an infinity of
$\lambda$'s $o$ satisfying this. The {\em lower bound} of such
$\lambda$'s is called the ``order of $f$''. Let us denote it by
$\rho$. Then, given $\epsilon >0$, there exists an $r_0$ such that 
$$
|f(z)| < e^{r^{\rho + \epsilon}}
$$
for $|z| = r> r_0$.
\end{defi*}

This would imply that
$$
M(r) = \max\limits_{|z|=r} |f(z)| < e^{r^{\rho + \epsilon}} , r> r_0,
$$
while\pageoriginale 
$$
M(r) > e^{r^{\rho-\epsilon}} \text{ for an infinity
of values of $r$ tending}
$$
to $+ \infty$

Taking logs. twice, we get
$$
\frac{\log \log M(r)}{\log r} < \rho + \epsilon
$$
and
$$
\frac{\log \log M(r)}{\log r} > \rho - \epsilon \text{ for an
  infinity of values of $r \to \infty$}
$$

Hence
$$
\rho =\overline{\lim\limits_{r\to \infty}} \frac{\log \log M(r)}{\log R}
$$

\begin{thm}\label{chap4:thm2}
If $f(z)$ is an entire function of order $\rho < \infty$, and has an
infinity of zeros, $f(0) \neq 0$, then given $\epsilon >0$, there
exists an $R_\circ$ such that for $R \geq R_0$, we have 
$$
n \left(\frac{R}{3} \right) \leq \frac{1}{\log 2} \cdot \log
\frac{e^{R^{\rho+\epsilon}}}{|f(0)|} 
$$

Here $n(R)$ denotes the number of zeros of $f(z)$ in $|z| \leq R$.
\end{thm}

\begin{proof}
We first observe that if $f(z)$ is analytic in $|z| \leq R$, and $a_1,
\ldots, a_n$ are the zeros of $f$ inside $|z| < R/3$, then for 
$$
g(z)  = \frac{f(z)}{\left(1-\frac{z}{a_1}\right) \cdots
  \left(1-\frac{z}{a_n} \right)} 
$$
we get\pageoriginale the inequality
$$
|g(z)| \leq \frac{M}{{}_2 n}
$$
where $M$ is defined by: $|f(z)| \leq M$ for $|z| = R$.
For, if $|z| = R$, then since $|a_p| \leq \dfrac{R}{3}$, $p = 1, \ldots,
n$, we get $\left|\dfrac{z}{a_p} \right| \geq 3$ or
$\left|1-\dfrac{z}{a_p} \right| \geq 2$ 

By the maximum modules theorem,
$$
|g(0)| \leq \frac{M}{{}_2 n} , \text{ i.e. } |f(0)| \leq \frac{M}{{}_2
n}
$$
or  
$$
n \equiv n \left(\frac{R}{3} \right) \leq \frac{1}{\log2 } \cdot \log
\left( \frac{M}{f(0)}\right) 
$$

If, further, $f$ is of order $\rho$, then for $r > r_\circ$, we have $M <
e^{r^{\rho+\epsilon}}$ which gives the required results.

N.B. The result is trivial if the number of zeros is finite.
\end{proof}

\begin{coro*}
$
n(R ) = O (R^{\rho + \epsilon})$. 

For 
\begin{align*}
& n \left(\frac{R}{3} \right) \leq \frac{1}{\log 2} \cdot \left[R^{\rho
    + \epsilon} - \log |f(0)| \right],\\
\text{and } \qquad & -\log |f(0)| < R^{\rho + \epsilon}, \text{ say}.
\end{align*}

Then for $R \geq R'$
$$
n \left(\frac{R}{3} \right) < \frac{2}{\log 2} R^{\rho +
  \epsilon} 
$$
and hence the result.
\end{coro*}

\begin{thm}\label{chap4:thm3} 
If $f(z)$\pageoriginale is of order $\rho < \infty$, has an infinity
of zeros 
$$ 
a_1, a_2, \ldots , f(0) \neq 0, \sigma > \rho, \text{ then } \sum
\frac{1}{|a_n|^\sigma} < \infty  
$$
\end{thm}

\begin{proof}
Arrange the zeros in a sequence:
$$
|a_1| \leq |a_2| \leq \cdots
$$
Let
$$
\alpha_p = |a_p|
$$
Then in the circle $|z| \leq r = \alpha_n$, there are exactly $n$
zeros. Hence
$$
n < c \; \alpha^{\rho + \epsilon}_n, \text{ for } n  \geq N
$$
or
$$
\frac{1}{n} > \frac{1}{c} \cdot \frac{1}{\alpha^{\rho + 
    \epsilon}_n}  
$$
Let $\sigma > \rho + \epsilon$ (i.e. choose $0< \epsilon <
\sigma - \rho$). Then
$$
\frac{1}{n^{\sigma/\rho+ \epsilon}} >
\frac{1}{c^{\sigma/\rho+\epsilon}} \cdot
\frac{1}{\alpha^{\sigma}_n}, 
$$ 
or 
$$
\frac{1}{\alpha^{\sigma}_n} < c^{\sigma/\rho+\epsilon} \cdot
\frac{1}{n^{\sigma/\rho+\epsilon}} \text{ for } n \geq N
$$
Hence $ \sum\dfrac{1}{\alpha^\sigma_n} < \infty$.
\end{proof}

\begin{remark*}
\begin{itemize}
\item[(i)] There cannot be too dense a distribution of zeros, since $n
  < c < \alpha^{\rho + \epsilon}_n$. Nor can their moduli increase
  too slowly, since, for instance, $\sum\dfrac{1}{(\log n)^p}$ does
  not converge.

\item[(ii)] The result\pageoriginale is of course trivial if there are
  only a finite number of zeros.
\end{itemize}
\end{remark*}

\begin{defi*}
The {\em lower bound} of the numbers `$\sigma$' for which
$$
\sum\dfrac{1}{|a_n|^{\sigma}} < \infty
$$ 
is called the {\em exponent of
  convergence} of $\{a_n\}$. We shall denote it by $\rho_1$.
Then 
$$ 
\sum \frac{1}{|a_n|^{\rho_1 + \epsilon}} < \infty, \quad \sum
\frac{1}{|a_n|^{\rho_1 - \epsilon}} = \infty, \;\; \epsilon >0.
$$
By Theorem \ref{chap4:thm1}, we have $\rho_1 \leq \rho$.
\end{defi*}

\medskip
\noindent{\textbf{N.B.}} If the $a_n$'s are finite in number or nil,
then $\rho_1 =0$. Thus $\rho_1 >0$ implies that $f$ has an infinity of
zeros. Let $f(z)$ be entire, of order $\rho < \infty$; $f(0) \neq 0$,
and $f(z_n) =0$, $n = 1,2,3, \ldots$. Then there exists an integer
$(p+1)$ such that $\sum \dfrac{1}{|z_n|^{\rho+1}} < \infty$.

(By Theorem \ref{chap4:thm1}, any integer $> \rho$ will serve for $\rho+1$). Thus, by
Weierstrass's theorem, we have
$$
f(z) = e^{\mathcal{Q}(z)} \prod\limits^{\infty}_1
\left(1-\frac{z}{z_n} \right) e^{\frac{z}{z_n} + \frac{z^2}{2z_n2} +
  \cdots + \frac{z^\nu}{\nu z^\nu_n}}, 
$$
where $\nu = p$ (cf. proof of Weierstrass's theorem).

\begin{defi*}
The smallest {\em integer} $p$ for which $\sum \frac{1}{|z_n|^{p+1}}
< \infty$ is the 
`genus' of the `canonical product' $\prod \left(1-\dfrac{z}{z_n}
\right)e^{\frac{z}{z_n} + \cdots + \frac{z^p}{pz^p_n}}$, 
with\pageoriginale zeros $z_n$. If $z_n$'s are finite in
number, we (including nil) define $p=0$, and we define the product as
$\prod\left(1-\dfrac{z}{z_n} \right)$.
\end{defi*}

\begin{examples*}
\begin{itemize}
\item[{\rm (i)}] $z_n = n$, $p=1$ \quad {\rm (ii)} $z_n = e^n$, $p =0$

\smallskip

\item[{\rm (iii)}] $z_1 = \frac{1}{2} \log 2$, $z_n = \log n$, $n \geq 2$, no
finite $p$.
\end{itemize}
\end{examples*}

\begin{remark*}
If $\sigma > \rho_1$, then $\sum \dfrac{1}{|z_n|^\sigma} < \infty$.

Also, 
$$
\sum \frac{1}{|z_n|^{p+1}} < \infty.
$$

But 
$$
\sum \frac{1}{|z_n|^p} = \infty.
$$
Thus, if $\rho_1$ is {\em not} an integer, $p=[\rho_1]$, and if $\rho_1$ is
an integer, then either $\sum \dfrac{1}{|z_n|^{\rho_1}} < \infty$ or
$\sum \dfrac{1}{|z_n|^{\rho_1}} = \infty$; in the first case,
$p+1=\rho_1$, which in the second case $p=\rho_1$. Hence,
$$
\phi \leq \rho_1
$$
But $\rho_1 \leq \rho$. Thus
$$
p \leq \rho_1 \leq \rho
$$

Also
$$
p \leq \rho_1 \leq p +1, \text{ since } \sum \frac{1}{|z_n|^{p+1}} <
\infty.  
$$
\end{remark*}

