\chapter{Entire Functions (Contd.)}\label{chap5}

\setcounter{thm}{0}
\begin{thm}[Hadamard]\label{chap5:thm1}
\cite{key8}\pageoriginale Let $f$ be an entire function with
zeros $\{z_{n}\}$ such that
$$
|z_1| \leq |z_2| \leq \cdots |z_{n}|\leq\cdots \to \infty 
$$
Let $f(0) \neq 0$. Let $f$ be of order $\rho < \infty$. Let $G(z)$ be
the canonical product with the given zeros.

Then
$$
f(z) = e^{\mathcal{Q}(z)} G(z),
$$
where $\mathcal{Q}$ is a polynomial of degree $\leq \rho$.

We shall give a proof without using Weierstrass's theorem. The proof
will depend on integrating the function
$\dfrac{f'(z)}{f(z)}\cdot \dfrac{1}{z^{\mu} (z-x)}$ over a sequence of
expending circles.

Given $R>0$, let $N$ be defined by the property:
$$
|z_N| \leq R, \quad |z_{N+1}| > R.
$$
We need the following 
\end{thm}

\begin{lemma*}
Let $0 < \epsilon < 1$. Then 
$$
\left|\frac{f'(z)}{f(z)} - \sum\limits^N_{n+1} \frac{1}{z-z_n}\right|
< c R^{\rho -1 + \epsilon}
$$
for $|z| = \dfrac{R}{2} > R_0$, and $|z| = \dfrac{R}{2}$ is free from
any $z_n$.
\end{lemma*}

For we can choose an $R_0$ such that for $|z| = R> R_0$,
\begin{itemize}
\item[(i)] $|f(z)| < e^{R^{\rho+\epsilon}}$,\pageoriginale

\item[(ii)] no $z_n$ lies on $|z| = \dfrac{R}{2}$, and 

\item[(iii)] $\log \dfrac{1}{|f(0)|} < (2R)^{\rho + \epsilon}$
\end{itemize}

For such an $R > R_0$, define 
$$
g_R (z) = \frac{f(z)}{f(0)} \prod\limits^N_{n=1} \left(1-\frac{z}{z_n}
\right)^{-1} 
$$

Then for $|z| = 2R$, we have
$$
|g_R(z)| < \frac{1}{|f(0)|} e^{(2R)^{\rho+\epsilon}}
$$
since $|f| < e^{(2R)^{\rho+\epsilon}}$, and $\left|1-\dfrac{z}{z_n}\right|
\geq 1$ for $1 \leq n \leq N$.

Hence, by the maximum modulus principle,
$$
|g_R(z)| < \frac{1}{f(0)} e^{(2R)^{\rho + \epsilon}} \text{ for }
|z| = R.
$$
That is
$$
\log |g_R (z)| < 2^{\rho +2} R^{\rho + \epsilon}.  \; \; (0 < \epsilon
< 1)
$$
If $h_R(z) \equiv \log g_R (z)$, and the log vanishes for $z=0$, then,
since $g_R(z)$ is analytic and $g_R (z) \neq 0$ for $|z| \leq R$, we
see that $h_R (z)$ is analytic for $|z| \leq R$, and $h_R
(0)=0$. Further, for $|z| =R$,
$$
\re h_R (z) = \log |g_R (z)| < 2^{\rho +2} R^{\rho + \epsilon}
$$

Hence,\pageoriginale by the Borel-Caratheodory inequality, we have for
$|z|=R/2$, 
$$
|h'_R (z)| < c R^{\rho -1 + \epsilon},
$$
which gives the required lemma.

\medskip
\noindent{\textbf{Proof of theorem.}}
Let $\mu = [\rho]$ Then consider
$$
I \equiv \int \frac{f'(z)}{f(z)} \cdot \frac{1}{z^{\mu} (z-x)} dz
$$
taken along $|z| = R/2$, where $|x| < \dfrac{R}{4}$. The only poles of
the integrand are: $0$, $x$ and those $z_n$'s which lie inside $|z| =
R/2$. Calculating the residues, we get
\begin{align*}
\frac{1}{2\pi i} \int\limits_{|z| = R/2} \frac{f'(z)}{f(z)}
\frac{1}{z^{\mu} (z-x)} dz & = \sum\limits_{|z_n|< R/2} \frac{1}{z_n^\mu
\cdot (z_n -x)}\\
& \quad + \frac{f'(x)}{f(x)} \cdot \frac{1}{x^{\mu}}\\
& \quad + \frac{1}{(\mu -1)!} \left( D^{\mu-1}\right)_{z=0}
\left(\frac{f'(z)}{f(z)} - \frac{1}{z-x} \right)
\end{align*}


We shall prove that the l.h.s tends to $0$ as $R \to \infty$ in such a
way that $|z| = R /2$ is free from any $z_n$. Rewriting $I$ as 
\begin{align*}
I & = \int \left(\frac{f'}{f} - \sum^N_1 \frac{1}{z-z_n} \right)
\frac{dz}{z^\mu(z-x)} + \sum\limits^N_1 \int \frac{1}{z-z_n} \cdot
\frac{dz}{z^{\mu} (z-x)}\\
&  \equiv I_1 + I_2,
\end{align*}
we get\pageoriginale for $|z| = R/2$ (by the Lemma),
$$
|I_1| = O (R^{\rho -\mu -1 + \epsilon}) = O (1).
$$
Let $k$ denote the number of $z_n$'s lying inside $|z| = R/2$

Then 
\begin{align*}
I_2 & = \sum\limits^k_{n=1} \int\limits_{|z|=R/2} \cdots +
\sum\limits^N_{n=k+1} \int \ldots \\
& \equiv I_{2,1} + I_{2,2}, \text{ say}.
\end{align*}

The value of $I_{2,1}$ remains unaltered when we integrate along $|z|=
3 R/4$, and along the new path, $|z-z_n| \geq R/4$ since $n \leq k$. Since $N = O (R^{\rho+\epsilon})$ we get 
$$
I_{2,1} = O(R^{\rho - \mu-1+\epsilon}) =o (1)
$$
Similarly integrating $I_{2,2}$ over $|z| = R/4$, we get
$$
I_{2,2} = o(1)
$$
Thus $I = o(1)$. Hence
\begin{align*}
& \frac{f'(x)}{f(x)} \frac{1}{x^\mu} + \sum\limits^\infty_{n=1} \cdot
  \frac{1}{z^{\mu}_n (z-x)} \frac{1}{(\mu-1)!} \Big[D \Big]^{\mu
    -1}_{z=0} \left(\frac{f'}{f} \; \frac{1}{z-x} \right) =0\\
\text{i.e. } & \frac{f'(x)}{f(x)} + \sum\limits^{\mu-1}_0 b_n x^n
\sum\limits^\infty_{n=1} \left(\frac{x^{-1}}{z^{\mu}_n (z-x)} \right)
= 0, \quad b_n =a_{\mu -n}, a_n = -
\frac{\left[\frac{f'}{f} \right]^{(\mu-n)}_{z=0}}{(\mu-n)!} \\
\text{i.e. } & \frac{f'(x)}{f(x)} = \sum\limits^{\mu-1}_0 c_n x^n +
\sum\limits^{\infty}_{n=1} \left(\frac{x^{\mu-1}}{z^{\mu}_n} +
\frac{x^{\mu-2}}{z^{\mu-1}_n} + \cdots + \frac{1}{x-z_n} 
\right) 
\end{align*}\pageoriginale

Integrating and raising to the power of $e$ we get the result:
$$
f(x) = e^{\sum\limits^\mu_0 d_n x^n} \prod\limits^\infty_1
\left(1-\frac{x}{z_n} \right) e^{x/z_n + \cdots + x^{\mu}/\mu z^\mu_n},
d_n = \frac{c_{n-1}}{n}.
$$

If $f$ is an entire functions of order $\rho$, which is not a
constant, and such that $f(0) \neq 0$, and $f(z_n) =0$, $n = 1, 2, 3,
\ldots $ $|z_n| \leq |z_{n+1}|$, then
$$
f(z) = e^{\mathcal{Q}_1(z)} G_1 (z),
$$
where $\mathcal{Q}_1(z)$ is a polynomial of degree $q\leq \rho$ and 
$$
G_1 (z) = \prod\limits^\infty_{n=1} \left(1 -\frac{z}{z_n} \right)
e^{\frac{z}{z_n} + \cdots + \frac{1}{p} \left(\frac{z}{z_n} \right)^p}, 
$$
where $p+1$ is the smallest integer for which $\sum
\dfrac{1}{|z_n|^{p+1}} < \infty$,  this integer existing, since $f$ is
of finite order.

\begin{remark*}
This follows at once from Weierstrass's theorem but can also be
deduced from the infinite-product representation derived in the proof
of Hadamard's theorem (without the use of Weierstrass's theorem). One
has only to notice that
$$
\prod e^{-\frac{1}{p+1} \left(\frac{z}{z_n} \right)^{p+1} - \cdots -
  \frac{1}{\mu}  \left( \frac{z}{z_n}\right)^{\mu} }  
$$
is convergent, where $p$ is the genus of the entire function in
question,\pageoriginale so that $f(z) = e^{\mathcal{Q}(z)} G(z)$ may
be multiplied by this product yielding $f(z) =e^{\mathcal{Q}_1(z)}
G_1(z)$, where $\mathcal{Q}_1$ is again of degree $\leq \mu$.

We recall the convention that if the $z_n$'s  are finite in number
(or if there are no $z_n$'s), then $p=0$ and $\rho_1=0$.

We have also seen that
$$
q \leq \rho, \rho_1 \leq \rho \text{ i.e. } \max (q, \rho_1) \leq \rho
$$ 

We shall now prove the following 
\end{remark*}

\begin{thm}\label{chap5:thm2}
$\rho \leq \max (q , \rho_1)$ 

Let
$$
E(u) \equiv (1-u) e^{u + \dfrac{u^2}{2} +  \cdots + \dfrac{u^p}{p}}
$$
If $|u| \leq \frac{1}{2}$, then
\begin{align*}
|E(u)| & = \left| e^{\log (1-u) + u + \cdots +\frac{u^p}{p}} \right| =
\left|e^{\frac{-u^{p+1}}{p+1} - \frac{u^{p+2}}{p+2} - \cdots}
\right|\\
& \leq e^{|u|^{p+1} (1+ \frac{1}{2} + \frac{1}{4} + \cdots)}\\
& = e^{2|u|^{p+1}} \leq e^{|2u|^{p+1}}\\
& \leq e^{|2u|^\tau}, \text{ if } \tau \leq p +1
\end{align*}

If $|u| > \frac{1}{2}$, then
\begin{align*}
|E(u)| & \leq (1+|u|) e^{|u| + \cdots + \frac{|u|^p}{p}}\\
& \leq (1+|u|) e^{|u|^p \cdot (|u|^{1-p} + \cdots + 1 )}\\
& \leq (1 + |u|) e^{|u|^p (2^{p-1} + \cdots +1)}  \\
& < e^{|2u|^p + \log (1+|u|)}\\
& \leq e^{|2u|^\tau + \log (1+|u|)}, \text{ if } p \leq \tau, \text{
  since } |2u| >1.
\end{align*}\pageoriginale

Hence, for all $u$, we have
$$
|E(u)| \leq e^{|2u|^\tau + \log (1+|u|)},
$$
if $p \leq \tau \leq p +1$.

It is possible to choose $\tau$ in such a way that 
\begin{itemize}
\item[(i)] $p \leq \tau \leq p +1$,

\item[(ii)] $\tau > 0$,

\item[(iii)] $\sum \dfrac{1}{|z_n|^\tau} < \infty$, (if there are an
  infinity of $z_n$'s) and

\item[(iv)] $\rho_1 \leq \tau \leq \rho_1 + \epsilon$
\end{itemize}

For, if $\sum\dfrac{1}{|z_n|^{ \rho_1}} < \infty$ (if the series is
infinite this would imply that $\rho_1 > 0$), we have only to take
$\tau = \rho_\tau > 0$. And in all other cases we must have $p \leq
\rho_1 < p +1$, [for, if $\rho_1 = p +1$, then
  $\sum\dfrac{1}{|z_n|^{\rho_1}} < \infty$] and so we choose $\tau$ such
that $\rho_1 < \tau < p + 1$; this implies again that $\sum
\dfrac{1}{|z_n|^\tau }< \infty$ and $\tau > 0$. For such a $\tau$ we
can write the above inequality as
$$
|E(u)| \leq e^{c_1|u|^\tau}, c_1 \text{ is a constant.}
$$

Hence\pageoriginale
\begin{align*}
& |f(z)| \leq e^{|\mathcal{Q}(z)|+c_1} \cdot \sum\limits^\infty_{n=1}
  \left|\frac{z}{z_n} \right|^\tau\\
\text{i.e. } & M(r) < e^{c_2 r^q + c_3 r^\tau}, |z| = r > 1,
\text{ and } c_1 \sum |1/z_n|^\tau = c_3.
\end{align*}

\eject

Hence
\begin{align*}
\log M(r) & = O(r^q) + O(r^\tau), \text{ as } r \to \infty,\\
& = O(r^{\max (q,\tau)}), \text{ as } r \to \infty \tag{2.1}\label{c5:eq2.1}
\end{align*}
so that
$$
\rho \leq \max (q, \tau)
$$
Since $\tau$ is arbitrarily near $\rho_1$, we get
$$
\rho \leq \max (q, \rho_1),
$$
and this proves the theorem.
\end{thm}

\begin{thm}\label{chap5:thm3}
If the order of the canonical product $G(z)$ is $\rho'$, then
$\rho'=\rho_1$
\end{thm}

\begin{proof}
As in the proof of Theorem \ref{chap5:thm2}, we have
$$
|E(u)| \leq e^{c_1|u|^\tau}
\begin{cases}
& p \leq \tau \leq  p+1;\\
& \tau > 0\\
& \sum \dfrac{1}{|z_n|^\tau} < \infty\\
& \rho_1 \leq \tau \leq \rho + \epsilon, \quad \epsilon > 0
\end{cases}
$$

Hence
\begin{align*}
|G(z)| & \leq e^{c_1 \sum\limits^\infty_{n=1} \left| \frac{z}{z_n}\right|^\tau} \\
& \leq e^{c_3 r^\tau}, |z| = r.
\end{align*}
If $M(r) = \max\limits_{|z|=r} |G(z)|$\pageoriginale, then
$$
M(r) < e^{c_3 r^\tau}
$$
Hence
$$
\rho' \leq \tau
$$
which implies $\rho'\leq \rho_1$; but $\rho_1 \leq \rho'$. Hence
$\rho' = \rho_1$
\end{proof}

\begin{thm}\label{chap5:thm4}
If, either 
\begin{itemize}
\item[{\rm (i)}] $\rho_1 < q$  or 

\item[{\rm (ii)}] $\sum \dfrac{1}{|z_n|^{\rho_1}} < \infty$ (the
  series being infinite),
\end{itemize}
then
$$
\log M(r) = O (r^\beta), \text{ for }  \beta = \max (\rho_1, q). 
$$
\end{thm}

\begin{proof}
\begin{itemize}
\item[{\rm (i)}] If $\rho_1 < q$, then we take $\tau < q$ and from
  (\ref{c5:eq2.1}) we get
$$
\log M(r) = O (r^q) = O (r^{\max (\rho_1, q)})
$$

\item[{\rm (ii)}] If $\sum\limits^\infty_{n=1} \dfrac{1}{|z_n|\rho_1}
  < \infty$, then we may take $\tau = \rho_1 >0$ and from (\ref{c5:eq2.1}) we get
\begin{align*}
\log M(r) & = O(r^q) + O (r^{\rho_1})\\
& = O(r^\beta)
\end{align*}
\end{itemize}
\end{proof}

\begin{thm}\label{chap5:thm5}
\begin{itemize}
\item[{\rm (i)}] We have
$$
\rho = \max (\rho_1, q)
$$
(The existence of either side implies the other)

\item[{\rm (ii)}] If\pageoriginale $\rho > 0$, and if $\log M(r) = O
  (r^\rho)$ does not hold, then $\rho_1 = \rho$, $f(z)$ has an
  infinity of zeros $z_n$, and $\sum |z_n|^{-\tau}$ is divergent.
\end{itemize}
\end{thm}

\begin{proof}
The first part is merely an affirmation of Hadamard's theorem, $\rho_1
\leq \rho$ and Theorem \ref{chap5:thm2}. 

For the second part, we must have $\rho_1 = \rho$; for if
$\rho_1<\rho$, then $\rho = q$ by the first part, so that $\rho_1 <
q$, which implies, by Theorem \ref{chap5:thm4}, that $\log M(r) = O(r^q) = O(r^\rho)$
in contradiction with our hypothesis. Since $\rho>0$, and $\rho_1 =
\rho$ we have therefore \textit{$\rho_1 >0$, which implies that $f(z)$
has an infinity of zeros.} Finally $\sum\limits^\infty_1
|z_n|^{-\rho_1}$ is divergent, for if $\sum |z_n|^{\rho_1} < \infty$,
then by Theorem \ref{chap5:thm4}, we would have $\log M(r) = O (r^\rho)$ which
contradicts the hypothesis.
\end{proof}

\begin{remarks*}
\begin{enumerate}
\renewcommand{\labelenumi}{(\theenumi)}
\item $\rho_1$ is called the `real order' of $f(z)$, and $\rho$ the
  `apparent order', Max $(q,p)$ is called the `genus' of $f(z)$.

\item If $\rho$ is not an integer, then $\rho = \rho_r$, for $q$ is an
  integer and $\rho = \max (\rho_1, q)$. In particular, if $\rho$ is
  non-integral, then $f$ must have an infinity of zeros.

\end{enumerate}
\end{remarks*}
