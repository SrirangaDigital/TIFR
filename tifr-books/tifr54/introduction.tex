\chapter{Introduction}
\markboth{Introduction}{}


These notes are based on a series of lectures given at the Tata
Institute in January-February, 1973. The lectures are centered about
the work of M. Scahlessinger and R. Elkik on infinitesimal
deformations. In general, let $X$ be a flat scheme over a local Artin
ring $R$ with residue field $k$. Then one may regard $X$ as an
infinitesimal deformation of the closed fiber
$x_{0}=\fprod{X}{\Spec(k)}{\Spec (R)}$. 
Schlessinger's main result
proven in part (for more information see his Harvard Ph.D. thesis) is
the construction, under certian hypotheses, of a ``versal deforamtion
space'' for $X_{0}$. He shows that $\exists$ a {\em complete}
local k-algebra $A=\lim A/m_{A}^{n}$ and a sequence of deformations
$X_{n}$ over $\Spec(A/m_{A}^{n})$ such that the formal $A$-prescheme
$\mathscr{X}=\varinjlim X_{n}$ is versal in this sense: For all Artin
local rings $R$, every deformation $S/Spec(R)$ of $X_{0}$ may be
obtained from some homomorphism $A\to R$ by setting
$X=\fprod{\mathscr{X}}{\Spec (R)}{\Spec (A)}$. 

Note that by ``versal deformation'' we do not mean that there is in
fact a deformation of $X_{0}$ over $A$. The versal deforamtion is
given only as a formal scheme. However, Elkik has proven
(cf. ``Algebrisation du module formel d'une singularite isol\'{e}e''
S\'{e}minaire, E.N.S., 1971-72) that such a deformation of $X_{0}$
over $A$ does exist when $X_{0}$ is a affine and has isolated
singularities. We give a proof of this result in these lectures.

Finally, some of the work of M. Schaps, A. Iarrobino, and H.~Pin\-kham
is considered here. We prove schaps's result that every Cohen-Macaulay
affine scheme of pure codimension $2$ is determinantal. Moreover, we
outline the proof of her result that every unmixed Cohen-Macaulay
scheme of codimension $2$ in an affine space of dimension $<6$ has
nonsingular deformations. For more details see  her
Harvard Ph. D. thesis, ``Deforamtions of Cohen-Macaulay schemes of
codimension $2$ and non-singular deformation of space curves''. We
also reproduce Iarrobino's counterexample (cf. ``Reducibility of the
families of $0$-dimensional schemes on a variety'', University of
Texas, 1970) that not every $0$-dimensional projective scheme (in
$\mathbb{P}$ for $n>2$) has a smooth deformation. Finally, we give
some of Pinkham's results on deformations of cones over rational
curves (cf. his Harvard Ph.D. thesis, ``Deformations of algebraic
varieties with $\mathbb{G}_{m}$ action).

There is of course much more literature in this subject. Two papers
relevant to these notes are Mumford's ``Pathologies-IV''
(Amer. J. Math., Vol. XCVII, p. 847-849) in which expanding on
Iarrobino's methods he proves that not every 1-dimensional scheme has
nonsingular deformations, and M. Artin's ``Versal deformations and
algebraic stacks'' (Inventions mathematicae, vol. 27, 1974), in which
is shown that formal versality is an open condition.

Thanks are due to the Institute for Advanced study for providing
excellent facilities in getting this manuscript typed.


