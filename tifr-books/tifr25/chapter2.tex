
\chapter{Exterior differential systems}\label{chap2} % chapter II

\section{}\label{chap2:sec2.1} % section 2.1

In\pageoriginale this chapter,  as we mentioned in the beginning of chapter I. We
study the construction and properties of the submanifolds of a given
manifold which are integrals of a certain differential system. We
make this more explicit in the following. We begin with certain
notations which we employ throughout this and the following chapter. 

All manifolds,  submanifolds,  differential forms which we consider in
the following will be real analytic,  so we omit the adjective real
analytic. Let $M$ be a manifold of dimension $n$. For any point $z \in
M,  (M)_z$ denotes the tangent vector space to $M$ at $z$. 

Let $\varphi$ be a homogeneous differential form of degree $h$ on
$M$. $\varphi$ associates to every point $z$ an anti-symmetric $h$-tuple
multilinear mapping $\varphi_z$ on $(M)_z$. If $L^1,  \ldots,  L^h$
are tangent vectors in $(M)_z$ the value of the function $\varphi_z$
on $(L^1,  \ldots,  L^h)$ is denoted by $\langle \varphi,  L^1 \wedge
\cdots \wedge L^h \rangle$. If $\varphi$ has an expression $\varphi =
\sum a^{i_1 \cdots i_h} du_{i_1} ~ \wedge \cdots \wedge du_{i_h}$ in
terms of a system of local coordinates then we have $\langle \varphi,
L^1 \cdots L^h \rangle = \sum a^{i_1 \cdots i_h} ~ \det \langle
du_{i_\nu},  L_\mu \rangle$. For any subspace $E$ of the tangent
vector space $(M)_z$ of $M$ at $z$,  we denote by $\varphi | E$ the
restriction of the function $\varphi_z$ to $E$. 

A $q$-dimensional subspace $E$ of $(M)_z$ is called a $q$-dimensional
\textit{contact elements} at $z$. Let $\mathscr{G}^q_z (M)$ be the set
of all contact elements\pageoriginale of $M$ at $z$. Then $\mathscr{G}^q(M) = \cup \{
\mathscr{G}^q_z (M) : z \in M\}$ is the set of all $q$-dimensional
contact elements of $M$. $\mathscr{G}^q(M)$ can be provided with a
structure of real analytic manifold such that each $\mathscr{G}^q_z
(M) (z \in M)$,  is a real analytic submanifold of $\mathscr{G}^q (M)$
and such that there is a real analytic projection mapping $\rho$ of
$\mathscr{G}^q(M)$ onto $M$,  mapping every contact element onto its
origin. Now we shall explicitly give the coordinate system in
$\mathscr{G}^q(M)$. 

Let $x_1,  \ldots,  x_q$ be (real analytic) functions defined on an
open set $U$ of $M$ such that $dx_1,  \ldots,  dx_q$ are linearly
independent	at each point of $U$. Let $U (x_1,  \ldots,  x_q)$ be
the set of all $q$-dimensional contact elements $E$ in
$\mathscr{G}^q(M)$ such that $\rho(E)$,  the origin of $E$,  is in $U$
and such that the restrictions $(dx_1 | E,  \ldots,  dx_q | E)$ are
linearly independent. IF $E$ is in $U (x_1,  \ldots,  x_q)$ let $L^1
(E),  \ldots,  L^q (E)$ be a basis in $E$ dual to $dx_1 | E,  \ldots,
dx_q|$ $E$. Clearly this choice of the dual basis depends on the choice
of $(x_1,  \ldots$,  $x_q)$. Suppose that $(x_1,  \ldots,  x_q; w_1,
\ldots,  w_{n-q})$ be a coordinate system of  $M$ defined on
$U$. Then,  for every $E$ in $U (x_1,  \ldots,  x_q),  L^i (E)$ can be
expressed by  
$$
\frac{\partial }{\partial x_i} + \sum_{\lambda =1}^{n-q} y^i_\lambda
(E) \frac{\partial}{\partial w_\lambda} 
$$
where $y^i_\lambda$ are functions defined on $U(x_1,  \ldots,
x_q)$. The mapping which associates to every $E$ in $U(x_1,  \ldots,
x_q)$ the system (origin of $E,  \ldots,  y^i_\lambda (E)$,  $\ldots)$
defines a coordinate system $(y_1 ~ o ~ \rho,  \ldots,  y_n ~ o ~
\rho,  \ldots,  y^i_\lambda (E),  \ldots )$ where $(y_1, \ldots,
y_n)$ is a coordinate system on\pageoriginale $M$ defined in $U$. Thus $(y_1  ~ o ~
\rho,  \ldots$,  $y_n ~ o ~ \rho,  \ldots,  y^i_\lambda (E),  \ldots)$
is a coordinate system in $\mathscr{G}^q (M)$. $\mathscr{G}^q(M)$ is
the so called Grassman manifold. 

\section{}\label{chap2:sec2.2} % section 2.2

Hereafter we consider only the domains in a Euclidean space. Let $D$
be a domain in $R^n$; ler $\Lambda^k(D)$ denote the module of
homogeneous exterior differential forms of degree $k$ on $D$ and the
direct sum $\Lambda (D) $ of $\lambda^o (D),  \ldots,  \Lambda^n (D)$,
where $n$ denote the dimension of $D$,  is the algebra over $\Lambda^o
(D)$ of homogeneous exterior differential forms on $D$. Here
$\Lambda^0 (D)$ denotes the ring of (real analytic) functions on
$D$. The operators $\wedge$ and $d$ always denote the exterior
multiplication and the exterior derivation respectively in the algebra
$\Lambda (D)$. If $f$ is a real analytic mapping of a manifold $M$
into another manifold $M'$ then the inverse image on $M$ of an
exterior differential form $\phi$ on $M'$ by $f$ is denoted by $f^*
\varphi$ and we have the equality $d(f^* \varphi) = f^* (d
\varphi)$. 

Let $(\sum)$ be a subset of $\wedge (D)$.

\begin{defi*} % definition
  A submanifold $N$ of $D$ is said to be in {\em{integral
      submanifold}} or an {\em{integral}} of $(\sum)$ if the
  restrictions of $\varphi$ to $N$ vanish for all $\varphi \in
  (\sum)$. 
\end{defi*}

\begin{defi*}% definition
  A $q$-dimensional contact element $E$ of $D$ is called an
  {\em{integral element}} of dimension $q$ of the system $(\sum)$ if
  the restrictions of $\varphi $ (in $\sum$) to $E$ vanish. A
  0-dimensional integral element is also sometimes called an
  {\em{integral point}}.  
\end{defi*}

The following proposition is an immediate consequence of this definition.
\setcounter{proposition}{0}
\begin{proposition}\label{chap2:sec2.2:prop1}% proposition 1
  A\pageoriginale submanifold $N$ of $D$ is an integral of the system $(\sum)$ of
  exterior differential forms if and only if for any point $z$ in $N,
  (N)_z$ is an integral element of $(\sum)$. 
\end{proposition}

Now let $N$ be a submanifold of $D$ and let $i$ denote the natural
inclusion mapping of $N$ into $D$. If $\varphi = \varphi^0 + \cdots +
\varphi^n$,  where $\varphi^j \in \Lambda^j (D)$,  is an exterior
differential form on $D,  i^* \varphi$ is nothing but the restriction
of $\varphi$ to $N$. Therefore we remark that,  if the restriction of
$\varphi$ to $N$ also  vanishes,  because $i^* \varphi = 0$ implies
that $i^* (d \varphi) = d(i^* \varphi) = 0$. If $\varphi$ and $\psi$
are two differential forms on $D$ with $i^* \varphi = 0$,  then $i^*
(\psi \wedge \varphi) = (i^* \psi ) \wedge (i^* \varphi) = 0$.  If
moreover $i^* \psi = 0$ then clearly $i^* (\alpha \varphi + \beta
\psi) = 0$. Hence we conclude that the homogeneous ideal (closed under
the operation $d$) generated by homogeneous parts of elements in
$(\sum)$ in the exterior algebra $\Lambda (D)$ also possesses the same
integrals as $(\sum)$. 

\begin{defi*}% definition
  A homogeneous ideal $(\sum)$ in $\Lambda(D)$ is said to be closed if
  $(d \sum) \subset (\sum)$. 
\end{defi*}

\begin{proposition}\label{chap2:sec2.2:prop2}% proposition 2
  If $(\sum)$ is a homogeneous ideal in $\Lambda (D)$ then the
  homogeneous ideal generated by $(\sum)$ and $(d \sum) $ is
  closed. This follows easily if we use the fact that $d ~ o ~ d =
  0$. 
\end{proposition}

\begin{defi*}% definition
  A homogeneous ideal $(\sum)$ in the exterior algebra $\Lambda(D)$ is
  called an exterior differential system if (i) $(\sum)$ is closed
  and (ii)  $(\sum)$ is finitely generated as an ideal. 
\end{defi*}

As remarked above,  as far as the set of integrals are concerned,  the
situation will not change when we replace a finite set  $(\sum) = \{
\varphi_1, \ldots,  \varphi_h\}$\pageoriginale by the homogeneous ideal generated by
$\{ \varphi_1,  \ldots,  \varphi_h,  d \varphi_1,  \ldots,  d
\varphi_h\}$ in the algebra $\wedge (D)$. Then the ideal is closed by
Proposition \ref{chap2:sec2.2:prop2} and so hereafter we will consider
only exterior 
differential systems instead of finite subsets in $\wedge (D)$. 

Let $(\sum)$ be an exterior differential system. $(\sum)$ being a
homogeneous ideal,  it can be decomposed as $\sum = \sum^{(0)} +
\sum^{(1)} + \cdots + \sum^{(n)}$ where $\sum^{(k)} = \sum \cap
\Lambda^k (D)$. Let $E$ be a fixed element in $\mathscr{G} ^{q}_z
(D)$. Any set of vectors $L^1,  \ldots L^r$ in $E$ and a differential
form $\varphi$ in $\sum^{(r +1)}$ define a linear functional
$\alpha_\varphi$ on the tangent space $(D)_z$ at $z$,  as follows: 
$$
\alpha_\varphi (L) = \langle \varphi,  L^1 \wedge \cdots \wedge L^r
\wedge L \rangle \quad (L \in (D)_z). 
$$

The subspace of the dual of $(D)_z$ generated by all the
$\alpha_\varphi$ \break $\left(\varphi \in \sum^{(r+1)}; L^1,  \ldots,
L^r  \in E; 
r = 0,  1,  \ldots,  q\right)$ is denoted by $J(E,  \sum)$\break (or simply by
$J(E)$ when there is no possible confusion regarding $\sum$) and the
dimension of $J(E,  \sum)$ is denoted by $t(E,  \sum)$ (or simply by
$t(E)$). $J(E,  \sum)$ is called the \textit{space of polar forms} of
$\sum$ at $E$. $t$ can be regarded as an integral valued (not
necessary real analytic) function on $(\mathscr{G}^q (D)$. The
following are immediate consequences of this definition. 

\begin{proposition}\label{chap2:sec2.2:prop3}% proposition 3
  The subspace of the tangent space $(D)_z$ of $D$ at $z$,  spann\-ed by
  an integral element $E$ (subspace of $(D)_z$) and a tangent vector
  $L$ is an integral element of $(\sum)$ if and only if $L$ is a
  solution of the equation $J(E) = 0$. 
\end{proposition}

\begin{proposition}\label{chap2:sec2.2:prop4}% proposition 4
  If\pageoriginale $E$ and $E'$ are two integral elements of $(\sum)$ with $E'
  \subset E$ then $J(E') \subseteq J(E)$ and hence $t(E') \le t(E)$. 
\end{proposition}

\begin{defi*}% definition
  $J(E) = 0$ is called the polar equation of $\sum$ at $E$.
\end{defi*}

\begin{defi*}%  definition
  Let $F$ be a set of (real analytic) functions defined in a
  neighbourhood of a point $z$ in $D; F = 0$ is said to be a regular
  local equation of a subset $N$ of $D (z\in N)$ around $z$ if  
  \begin{enumerate}[\rm (i)]
  \item there exists a neighbourhood $U$ of $z$ in $D$ such that $U
    \cap N$ is a submanifold; 
  \item $f = 0$ on $N$ for every $f$ in  $F$;  and 
  \item there exists functions $f_1,  \ldots,  f_{n-h}$ in $F$ ( $h$
    being the dimension of the submanifold $U \cap N$) such that
    $df_1,  \ldots,  df_{n-h}$ are linearly independent at $z$. 
  \end{enumerate}
\end{defi*}

The set  of all $q$-dimensional integral elements of $(\sum)$ will be
denoted by $\ell^q \sum$. we shall now define the notions of ordinary and
regular integral elements  of $(\sum)$ by induction on the dimension
$q$. $\ell^q \sum$ is provided with the topology induced by
$\mathscr{F}^q D$. 

\begin{defi*}% definition
  An integral point $x$ of $\sum$ is said to be an {\em{ordinary
      integral point}} if $\sum^{(0)} = 0$ is a regular local equation
  of $\ell^o \sum$ around $x$. An integral point $x$ is said to be a
  {\em{regular integral point}} if $x$ is an ordinary integral point
  and the function $t$ is a constant in a neighbourhood of $x$ in
  $\ell^o \sum$. Suppose that the ordinary and regular integral
  elements of dimensions $q'$,  for $q' < q$,  are defined. 
\end{defi*}

\begin{defi*} % definition
  A $q$-dimensional integral element is said to be an  \textit{
    ordinary integral element}\pageoriginale of $(\sum)$ if it contains atleast one
  $(q-1)$-dimensional regular integral element. $A$ $q$-dimensional
  integral element is said to be a \textit{regular integral element}
  of $(\sum)$ if it is an ordinary integral element and the function
  $t$ is a constant in a  neighbourhood of it in $\ell^q \sum$. 
\end{defi*}

\begin{example*}
  Let $D$ be the plane $R^2$ represented by $(x,  y)$. Let $(\sum)$ be
  the differential system generated (as a closed ideal) in $\Lambda
  (D)$ by $\big \{x,  dx,  xdy$,  $dx \wedge dy \big \}$. If $z \in D$
  then clearly $J(z)$ is generated by  $\{(dx)_{z}\}$ if $z \in
  \ell^o \sum$ since $x(z) = 0$ and by $\{(dx)_z,  (dy)_z \}$ if $z
  \notin \ell^o \sum$. Hence any integral point $z$ is a regular
  integral element of $(\sum)$. 
\end{example*}

\section{}\label{chap2:sec2.3} 

Let $(M,  M',  \varpi)$ be a real analytic fibered manifold. 

\begin{defi*}
  A homogeneous differential form $\theta$ on $M$ of degree $r$ is
  said to be a fibred differential form if,  for every $z \in M$ and
  every pair of sets of tangent vectors $(L',  \ldots ,  L^r)$ and
  $(L^{' 1},  \ldots ,  L^{' r})$ in $(M)_z$ such that $\varpi (L^i) =
  \varpi(L^{' i}) ~ (i = 1,  2,  \ldots ,  r)$,  we have 
  $$
  \langle \theta ,  L^1 \wedge \cdots \wedge L^r \rangle = \langle
  \theta ,  L^{' 1} \wedge \cdots \wedge L^{' r} \rangle. 
  $$
\end{defi*}

If $z$ is any point in $M$,  a fibred differential form on $M$ defines
an antisymmetric multilinear form $\theta [z]$,  on the tangent space
of $M'$ at $\varpi(z)$ such that for any $L^1 ,  \ldots ,  L^r \in
(M)_z$ 
$$
\langle \theta,  L^1 \wedge \cdots \wedge L^r \rangle = \langle \theta
        [z],  \varpi (L^1) \wedge \cdots \wedge \widetilde{\omega } (L^r )
        \rangle. 
$$

If\pageoriginale $(x'_1 ,  \ldots ,  x'_n)$ is a coordinate system in an open subset
$U$ of $M'$ and if we set $x_i = x'_i o \varpi $ on $\varpi^{-1} (U)$
the fibred differential form $\theta$ has an expression of the form
$\theta = \sum^{i_1 \cdots i_r} dx_{i_1} \wedge \cdots \wedge
dx_{i_r}$. Then $\theta [z],  z \in \varpi^{-1} (U)$ is given by
$\theta$ [$z$] = $\sum a^{i_1 \cdots i_r} (z) (dx'_{i_1} )_{\varpi_(z)}
\wedge \cdots \wedge (dx'_{i _r})_{ \varpi (z)}$. This expression
shows clearly that $\theta $[$z$] depends real analytically on $z$. 

Let $(x_1 ,  \ldots ,  x_q)$ be a set of functions defined on a domain
$D$ of $R^n$ such that $dx_1,  \ldots ,  dx_q$ are linearly
independent at each point of $D$. Let $\mathscr{G}^q (D; x_1 ,  \ldots
,  x_q)$ denote the subset of $\mathscr{G}^q (D)$ consisting of all
elements $E$ in $\mathscr{G}^q (D)$ for which the restrictions of
$dx_1 ,  \ldots ,  dx_q$ to $E$ are linearly independent. Let $\rho$
be the canonical projection of $\mathscr{G}^q (D; x_1,  \ldots ,
x_q)$ onto $D$ which associates  to every element $E$ in
$\mathscr{G}^q (D; x_1,  \ldots ,  x_q)$ its origin. When
$\mathscr{G}^q (D; x_1,  \ldots ,  x_q)$ is provided with the manifold
structure induced  from $\mathscr{G}^q (D)$,  it is easy to see that
$(\mathscr{G}^q (D; x_1,  \ldots ,  x_q),  D,  \rho)$ is a fibred
manifold. Given a homogeneous differential form $\varphi \in \sum^{(r
  +1)}(D)$ and a set of integers $(i_1 ,  \ldots ,  i_r)$ with $1 \le
i_1 < \cdots < i_r \le q,  ~ a $ \textit{Pfaffian form},  denoted by
$\varphi^{\{i_1 \cdots i_r\}}$ on $\mathscr{G}^q (D; x_1,  \ldots ,
x_q)$ is defined by the identity 
$$
\langle \varphi^{\{i_1 \cdots i_r\}},  L \rangle = \langle \varphi ,
L^{i_1} (E) \wedge \cdots \wedge L^{i_r } (E) \wedge d \rho L \rangle 
$$
where $L \in (\mathscr{G}^q (D))_E$ and $L^1 (E),  \ldots ,  L^q (E)$
is a basis of $E$ dual to the restrictions of $dx_1 ,  \ldots ,  dx_q$
to $E$. It is immediate to see that $\varphi^{\{i_1 \cdots i_r\}}$ is
a fibred differential form on $(\mathscr{G}^q (D; x_1,  \ldots ,
x_q))$ with\pageoriginale respect to the fibred manifold $(\mathscr{G}^q (D; x_1,
\ldots ,  x_q) ; D,  \rho)$. 

Let $(\sum)$ be an exterior differential system on $D$ and let
$\varphi_1 ,  \ldots ,  \varphi_\ell$ be a set of homogeneous
differential forms (of degree $d(1),  \ldots ,  d(\ell))$ in $(\sum)$
which generate $(\sum)$ (as a closed ideal in $\Lambda (D))$. Let
$\big \{\varphi_1 ,  \ldots,  \varphi_h \big \}$ be the subset of
$\varphi_1 ,  \ldots,  \varphi_\ell$ for which $1 \le d(i) \le q$. As
before choose a set $(x_1 ,  \ldots ,  x_q)$ of functions on $D$ such
that $dx_1 ,  \ldots ,  dx_q$ are linearly independent at each point
of $D$ and let $E \in \mathscr{G}^q (D; x_1,  \ldots ,  x_q)$. 

\begin{proposition}\label{chap2:sec2.3:prop5} % pro 5
  If $E$ is in $\ell^q \sum \cap \mathscr{G}^q (D; x_1, \ldots, x_q)$
  then $J(E)$ is generated by $\big \{\varphi^{i_1 \cdots i_{d
    (\sigma) - 1}} $ [$E$] $; \rho = 1,  \ldots,  k,  1 \le i_j \le q
  \big \}$ $(1 \le i_1 < \cdots < i_{d(\rho) - 1} \le q)$ 
\end{proposition}

\begin{proof}
  By definition $J(E)$ is generated by all $\alpha_\varphi$ defined by
  $\alpha_ \varphi (L) = \langle \varphi,  L^1 \wedge \cdots \wedge
  L^r \wedge L \rangle$ where $ \varphi \in \sum^{(r+1)},  L^1,
  \ldots ,  L^r \in E $ and $L' \in (D)_z,  z$ being the origin of
  $E$. If $L^1 (E),  \ldots,  L^q (E)$ is  a basis of $E$ dual to
  $dx_1 \big | E,  \ldots,  dx_q \big | E$ we can write $L^j = \sum
  b^j_i L^i (E)$. Therefore $\alpha_\varphi$ is in the space generated
  by $\varphi^{\{i_1 \cdots i_r\}} $[$E$]. 
\end{proof}

On the otherhand,  since $\deg \varphi \le q$ we see that $\alpha_
\varphi = 0$ for $r \ge q$. Hence one can assume that $r < q$ and one
can write $\varphi = \sum\limits^{h}_{\sigma = 1} \psi_\sigma \wedge
\varphi_\sigma + \sum_{f_j} \wedge \xi_j (f_j \in
\sum^{(0)})$. Therefore,  for $L \in (D)_z$ 
\begin{align*}
  \langle \varphi^{\{i _1  \cdots i_r \}} [E],  L \rangle & = \langle
  \varphi,  L^{i_1} (E) \wedge \cdots \wedge L^{i_r} (E) \wedge L
  \rangle\\ 
  & = \sum^{h}_{\sigma = 1} \langle \psi_\sigma \wedge \varphi_\sigma,
  L^{i_1} (E) \wedge \cdots \wedge L^{i_r}(E) \wedge  L \rangle\\ 
  & \hspace{2cm}+  \sum \langle f_j \wedge \xi_j,   L^{i_1} (E) \wedge \cdots
  \wedge L^{i_r}(E) \wedge  L \rangle. 
\end{align*}\pageoriginale

But since $E$ is an integral element and $f_j \in \sum^{(0)}$ the
second term in the right member vanishes. Hence 
\begin{multline*}
  \left\langle \varphi^{\{i_1 \cdots i_r\}},  L \right\rangle = \sum_{\sigma = 1}^h
  \langle \psi_\sigma \wedge \varphi_\rho,  L^{i_1} (E) \wedge \cdots
  \wedge L^{i_r} (E) \wedge L \rangle\\ 
  = \sum_{\sigma = 1}^h \pm \langle \psi_\sigma,  L^{h_1} (E) \wedge \cdots
  \wedge L^{h_{r_1}^\sigma}  (E) \rangle \langle \varphi_\sigma L^{k_1}(E)
  \wedge \cdots \wedge L{}^{k_{r_2}^\sigma-1} (E)\\ 
  + \sum_{\sigma = 1}^h \pm
  \langle \psi_\sigma, L {}^{h_1} (E) \wedge \cdots \wedge
  L{}^{h_{r_1}^\sigma-1} (E) \wedge L \rangle \langle \varphi_\sigma,
  L^{k_1}(E) \wedge . \wedge L{}^{k_{r_2}^\sigma} (E) 
\end{multline*}

Here $r^\sigma_1,  r^\sigma_2$ are the degrees of $\psi_\sigma$ and
$\varphi_\sigma$ respectively. Again $E$ being an integral element,  the
latter term in the right member vanishes. Therefore we obtain 
$$
\langle \varphi^{ \{i_1 \cdots i_r \}} ,  L \rangle = \sum^h_{\sigma =
  1} \pm * \langle \varphi_\sigma^{k_1 \cdots k_{r_2}\sigma_{-1}} ,  L
\rangle. 
$$

This completes the proof of the proposition.

Let $(\sum)$ be a differential system on a domain $D$ of $R^n$. If $E$
is an integral element of $(\sum)$,  let $G^r_E$ (when there is no
possible confusion,  simply $G$) be the set of all $r$-dimensional
contact elements of $D$ contained in $E$. $G$ can be provided with the
structure of a real analytic manifold as follows. Let $E'$ be an
element of $G$ and let $\varphi_1 ,  \ldots ,  \varphi_h$ be a base of
the dual $E^*$ of $E$ such that the restrictions of $\varphi_1 ,
\ldots ,  \varphi_r$ to $E'$ are linearly independent.\pageoriginale Then a
coordinate neighbourhood of $E'$ in $G$ can be defined to be $U
(\varphi_1 ,  \ldots ,  \varphi_r) = \{E'' \in G : \varphi_1 \big |
E'' ,  \ldots ,  \varphi_r \big | E'' $ are linearly independent $\}$.
The coordinate systems can be explicitly constructed without much
difficulty. But we do not go into the details since we do not need the
explicit coordinate systems. Clearly $G^r_E$ is a submanifold of
$\mathscr{G}^r (D)$. 

\begin{defi*} % defn
  $A$ subset $A$ of manifold $M$ is said to be a \em{ (real analytic)
    subvariety} ( or a \em {real analytic subset}) if for every $a$ in
  $M$ there exists a neighbourhood $U$ of a in $M$ and a finite number
  of real analytic functions $f_1 ,  \ldots ,  f_h$ defined on $U$
  such that $A \cap U$ is the set of common zeros of $f_1 ,  \ldots ,
  f_h$. 
\end{defi*}

We say that a subvariety $A$ is proper when $A \neq M$. Clearly this
definition is local. Every real analytic subset is closed in $M$. If
$M$ is connected and $A$ is proper then $M - A$ is everywhere dense. 

\begin{proposition}\label{chap2:sec2.3:prop6} % prop 6
  Let $E$ be an integral element of $\sum$. Then there exists a proper
  real analytic subset $A$ of $G^r_E = G$ and an integer $k > 0$ such
  that $t (E') = k$ for every $E'$ in $G - A$ and $t (E')
  \underset{\neq}{<} k$ for every $E'$ in $A$.  (We will denote this
  $k$ by $t_r(E))$. 
\end{proposition}

\begin{proof}
  Let $E' \in G$ and let $(x_1 \ldots ,  x_q,  y_1 ,  \ldots ,  y_m)$
  be a coordinate system in $D$ such that $dx_1 \big | E' ,  \ldots ,
  dx_r \big | E'$ are linearly independent. \break $\mathscr{G}^r (D; x_1,
  \ldots ,  x_r)$ being an open subset of $\mathscr{G}^r(D)$ and the
  property of a real analytic subset being local,  it is sufficient to
  prove that there exists a real analytic subset $A'$ of $G \cap
  \mathscr{G}^r (D; x_1,  \ldots ,  x_r)$ satisfying\pageoriginale the assertion of
  the proposition. 
\end{proof}

We know that there exists a finite number of Pfaffian forms $\theta_1,
\cdots ,\break  \theta _h$ of the fibred manifold $ \mathscr{G}^r (D; x_1,
\ldots ,  x_r),  D,  \rho) $ such that $\theta_1 [E'],  \ldots ,\break
\theta_h  [E']$ generate $J(E')$ for any integral element $E'$ in
$\mathscr{G}^r (D; x_1, \ldots , x_r)$ in particular. for any $E'$ in $G
\cap  \mathscr{G}^r (D; x_1,  \ldots ,  x_r)$. We can write
$\theta_\sigma = \sum\limits_{i = 1}^r a^i_\sigma dx_i +
\sum\limits_{\lambda = 1}^m a^{r + \lambda}_\sigma dy_\sigma$. Hence for an
element $E'$ in $ \mathscr{G}^r (D; x_1,  \ldots ,  x_r) \cap G$,  the
dimension $t(E')$ of $J(E')$ is nothing but the rank of the matrix 
$$
\begin{pmatrix}
\cdots a^i_1 (E') \cdots a^{r + \lambda }_1 (E') \cdots\\
\cdots \cdots \cdots \cdots \cdots \cdots \cdots \cdots \cdots\\
\cdots a^i _h (E') \cdots a^{r + \lambda}_h (E') \cdots
\end{pmatrix}
$$

Since $a^i_\sigma (E')$ and $a^{r+ \lambda}_\sigma (E')$ are real
analytic functions on a neighbourhood of $E'$ in $G \cap \mathscr{G}^r
(D; x_1,  \ldots ,  x_r)$,  it can be easily seen that $t$ is a
constant $k > 0$ except on a real analytic subset $A'$ of $G \cap
\mathscr{G}^r (D; x_1,  \ldots ,  x_r)$ and this proves the required
assertion. 

It is easy to prove the following

\noindent
\textbf{Corollary to Proposition 6.} Let $E$ be an ordinary
integral element. If $E'$ is regular and is contained in $E$,  then
$t(E') = t_r (E)$ where $r = \dim (E')$. 

We shall denote the set of all regular integral elements of dimension
$q$ of $(\sum)$ by $\mathscr{R}^q \sum$ and the set of all ordinary
integral elements of dimension $q$ of $(\sum)$ by $\mathfrak{G}^q
\sum$. 

\section{}\label{chap2:sec2.4} %sec 2.4

Let $(x_1,  \ldots ,  x_q,  y_1 ,  \ldots ,  y_m)$ be a coordinate
system of $D$. Any\pageoriginale differential form $\varphi \in \Lambda^r (D)$ and a
finite set of integers $(i_1,  \ldots ,  i_r)$ such that $1 \leq i_1 <
\cdots < i_r \leq  q$ define a function  $\varphi [ i_1 ,  \ldots ,
  i_r ] ~ (r= 0,  1,  \ldots ,  q)$ on $\mathscr{G}^q (D; x_1,  \ldots
,  x_q)$ as follows: If $E \in \mathscr{G}^q (D; x_1,  \ldots ,  x_q)$
and $\{L^1 (E)$,  $\ldots ,  L^q (E)\}$ is a basis of $E$ dual to $dx_1
\big | E,  \ldots ,  dx_q \big | E$ then  
$$
\varphi [i_1 ,  \ldots ,  i_r ](E) = \langle \varphi ,  L^{i_1} (E)
\cdots L^{i_r} (E) \rangle. 
$$

Let $F = F (\sum ; x_1 ,  \ldots ,  x_q)$ be the set of all $\varphi
[i_1 ,  \ldots,  i_r]$ with $\varphi \in \sum^{(r)} ~ (1 \le i_1 <
\cdots < i_r \leq q ; r = 0,  1 . \ldots ,  q)$. 

\begin{proposition}\label{chap2:sec2.4:prop7} % prop 7
  $\theta^q \sum$ and $\mathscr{R}^q \sum$ are open subsets of $\ell^q
  \sum$. If $E^0 \in \theta^q \sum \cap$ $\mathscr{G}^q (D; x_1,  \ldots
  ,  x_q)$ then there exists a neighbourhood of $E^0$ in
  $\mathscr{G}^q$ $(D)$ such that $\theta^q \sum \cap \mathscr{U}$ is a
  submanifold of $\mathscr{U}$ and  $F = 0$ is a regular local
  equation of $\theta^q \sum$. Moreover,  if $E'$ is a regular
  $(p-1)$-dimensional integral element in $E^0$,  then $\dim
  (\theta^q \sum \cap \mathscr{U}) = \dim (\theta^{q-1} \sum \cap
  \mathscr{U}' ) - (q-1) + (n-q-t (E'))$,  when $n = \dim D $ and
  $\mathscr{U}'$ is a sufficiently small neighbourhood of $E'$. 
\end{proposition}

\begin{proof}
  The proposition is obviously true in the case $q = 0$. We proceed
  now by induction on $q$. Let us assume the proposition to have been
  proved for all dimensions $q' < q$. If $E$ is an $\theta^q \sum $
  there exists a subspace $E'$ of $E$ in $\mathscr{R}^{q-1}
  \sum$. Choose the coordinate system $(x_1,  \ldots ,  x_q,  y_1 ,
  \ldots ,  y_m)$ in $D$ such that $dx_1 \big | E^0 ,  \ldots ,  dx_q
  \big | E^0$ are linearly independent and such that 
  $$
  E' = \{ L \in E : \langle dx_q ,  L \rangle= 0 \}
  $$
  We\pageoriginale shall define a map $\pi$ of $\mathscr{G}^q (D; x_1,  \ldots ,
  x_q)$ into $\mathscr{G}^{q-1} (D; x_1 ,  \ldots x_{q-1})$ as follows :
  if $E \in \mathscr{G}^q (D; x_1,  \ldots ,  x_q)$ then $\pi(E)$ is the
  space generated by $L^1 (E),  \ldots ,  L^{q-1}(E)$. For $E''$ in
  $\mathscr{G}^{q-1} (D; x_1 ,  \ldots ,  x_{q-1})$ denote by $L^1 (E'')
  ,  \ldots ,  L^{q-1}(E'')$ the base of $E''$ dual to $dx_1 \big | E''
  ,  \ldots ,  dx_{q-1} \big | E''$. We can write 
  $$
  L^r (E'') = \frac{\partial}{\partial x_r} + w^r (E'')
  \frac{\partial}{\partial x_q} + y^r_\lambda (E'')
  \frac{\partial}{\partial y_\lambda} ~ (r=1,  \ldots,  q-1) 
  $$
  where $w^r (E'') ,  y^r_\lambda (E'')$ form part of a coordinate
  system in $\mathscr{G}^{q-1} (D; x_1$,  $\ldots ,  x_{q-1})$. Therefore the
  equations $w_1 = \cdots = w_{q-1} = 0$ define a submanifold $W$ of
  $\mathscr{G}^{q-1} (D; x_1,  \ldots ,  x_{q-1})$ and $\dim W + (q-1) =
  \dim _{E'}\break (\ell ^{q-1} \sum )$. Then it is clear that $\pi$ maps
  $\mathscr{G}^q (D; x_1,  \ldots ,  x_q)$ onto $W$ and $\mathscr{G}^q
  (D; x_1$,  $\ldots ,  x_q),  W,  \pi)$ is a fibred manifold. 
\end{proof}

By the induction assumption,  there exists a neighbourhood
$\mathscr{U}'$ of $E'$ in $\mathscr{G}^{q-1}(D)$ such that
$\mathscr{U}' \cap \mathscr{R}^{q-1} \sum$ is a submanifold of
$\mathscr{U}'$ and $F_{E'} = 0$ is a regular local equation. We now
assert that,  for a suitable $\mathscr{U}' ,  \mathscr{U}' \cap W \cap
\mathscr{R}^{q-1} \sum$ is a submanifold of $\mathscr{R}^{q-1}
\sum$. Let $z_0$ be the origin of $E^0$. Given real numbers $a^1,
\ldots,  a^{q-1}$,  and $u$,  let $E_u \in \mathscr{G}^{q-1}(D)$,  be
generated by $L^j (E^0)+ u a^j L^q (E^0) ~ (j = 1,  \ldots,  q-1)$
. $E_u$ being a subspace of $E^0$,  it is an integral
element. Therefore $E_u$ is a real analytic curve in
$\mathscr{R}^{q-1} \sum$ for sufficiently small $u$. Because of the
choice of $E' ,  (E_u)_{u = 0}= E'$ and $w^j (E_u) = u ~ a^j$. The
last equality implies that $\langle d w^j,  X^a \rangle = a^j$ where
$X^a$ is the tangent vector at $u = 0 $ of the curve $E_u$ in
$\mathscr{R}^{q-1} \sum$. Because $a_1,  \ldots , \ldots , a_{q-1}$
are arbitrary, it follows that $(dw^1)_E , \ldots , (dw^{q-1)} E'$,
restricted to the  tangent\pageoriginale vector space to $\mathscr{R}^{q-1} \sum$ at
$E'$ are linearly independent. Hence $\mathscr{U}' \cap W \cap
\mathscr{R}^{q-1} \sum$ is a submanifold $N$ of $\mathscr{U}'$. Let
$\mathscr{U}$ be a neighbourhood of $E^0$ in $\mathscr{G}^q (D)$ such
that $\pi (\mathscr{U}) \subset \mathscr{U}'$. It is clear that any
element $E$ in $\mathscr{U}$ is in $\ell^q \sum$ if and only if its
image $\pi (E)$ is in $N$ and $L^q (E)$ is a zero of $J (\pi
(E))$.Take real analytic functions $f_1,  \ldots ,  f_a$ on
$\mathscr{U}'$ where $a = \dim \mathscr{G}^{q-1} (D) - \dim
(\ell^{q-1} \sum \cap \mathscr{U}'$) such that $df_1,  \ldots ,  df_a$
are linearly independent at $E'$ and $f_1 = \cdots = f_a = 0$ define
$\ell ^{q-1} \sum \cap \mathscr{U}'$.  

By the induction assumption we can choose fibred differential forms
$\theta_1,  \ldots ,  \theta_{t )(E')}$ of $(\mathscr{G}^{q-1} (D; x_1
,  \ldots ,  x_{q-1}) ,  D,  \rho)$ such that $\theta_1 [E''],  \ldots
,  \theta_{t (E') }$ $[E'']$ are linearly independent and generate
$J(E'')$ for integral elements $E''$ near $E'$. We can write  
$$
\theta _\sigma = \sum_{j = 1}^q a^j_\sigma dx_j + \sum_{\lambda = 1}^m
a^{q+ \lambda}_\sigma dy_\lambda. 
$$

We recall that the functions $y^q_\lambda$ on $\mathscr{G}^q (D; x_1,
\ldots ,  x_q)$ defined by $L^q (E)$ $= \dfrac{\partial}{\partial x_q} +
\sum\limits_{\lambda} y^q_\lambda (E) \dfrac{\partial}{\partial
  y_\lambda}$,  form a part of the coordinate system in $\mathscr{G}^q
(D; x_1$,  $\ldots,  x_q)$. Then the conditions $\pi (E) \in N$ and
$\langle J (\pi (E)),  ~ L^q (E) \rangle = 0$ are analy\-tically
expressed by the conditions: 
\begin{equation*}
  (I)\quad 
  \begin{cases}
    f_1 o \pi = \cdots = f_a o \pi = 0&\\
    a^q_\sigma o \pi + (a^{q+ \lambda}_\sigma o \pi ) y^q_\lambda = 0&
    ~ (\sigma = 1,  \ldots,  t (E')) 
  \end{cases}
\end{equation*}

Because $dx_1,  \ldots ,  dx_q,  \theta_1 [E'],  \ldots ,  \theta_t
[E']$ are linearly independent,  it is clear now that the
differentials of the above functions at $E$ are linearly independent
at each $E$ in $\mathscr{U}$,  for $\mathscr{U}$ sufficiently\pageoriginale
small. Therefore $\ell^q \sum \cap \mathscr{U}$ form a submanifold of
$\mathscr{G}^q (D)$. Since $\dim \pi^{-1} (E') = n - q $,  we see
easily that $\dim (\ell^q \sum \cap \mathscr{U})$ is equal to $\dim
(\mathscr{U}' \cap \ell^{q-1} \sum ) - (q-1)+ (n-t(E') -q)$. By
induction assumption we can choose $f_1,  \ldots,  f_a$ in
$F_{E'}$. Then it is clear by the definition of $F_E$ that  $f_1 o
\pi,  \ldots ,  f_a o \pi$ are in $F_E$. If we choose $\theta_\sigma$ as
constructed in in Proposition \ref{chap2:sec2.3:prop5},  we see easily that the function
$\langle \theta_\sigma \big [\pi_0 E \big ]$,  $L^q (E) \rangle$ are in
$F_E$. Thus we can choose $f_h$ and $\theta_\sigma$ in such a way that
the equations $(I)$ is a part of the equation $F_E = 0$. Therefore
$F_E = 0$ is a regular local equation of $\theta^q \sum$ at $E$.  

Because $pi$ is continuous and $\mathscr{R}^{q-1} \sum$ is open in
$\ell^{q-1} \sum$ it follows that $\theta^q \sum$ is open. Then,
because of the definition of regular integral elements,
$\mathscr{R}^q \sum$ is also open. 

\begin{proposition}\label{chap2:sec2.4:prop8} % prop 8
  Let $E$ be an ordinary integral element of dimension $q$. Let $E'
  \subset E$. Assume that there is a sequence of subspaces $E' =
  E^{(1)} \subset E^{(2)} \subset \cdots E^{(h)} = E$ of $E$  such
  that $\dim (E^{i +1)}) = \dim (E^{(i)}) + 1$ and such that each
  $E^{(i)} ~ (i = 1,  \ldots ,  h-1)$ is regular. Then for each
  neighbourhood $\mathscr{U}$ of $E$ in $\ell^q \sum$ there is a
  neighbourhood $\mathscr{U}'$ of $E'$ with the following property:
  for any $E'' \in \mathscr{U}' \cap \ell ^r \sum (r = \dim E')$ there
  is an element $E_1 \in \mathscr{U} \cap \Theta^q \sum$ such
  that $E_1 \supset E''$. 
\end{proposition}

\begin{proof}
  By an induction argument the problem can be reduced to the case when
  $r = q-1$. Take a neighbourhood $\mathscr{U}'_1$ of $E'$ and a
  system $\theta_1,  \ldots ,  \theta_t$,  of fibred differential
  forms on $(\mathscr{U}'_1 ,  D,  \rho)$ such that for any $E'' \in
  \mathscr{U}'_1 \cap \ell^{q-1} \sum,  \theta_1 [E''],  \ldots,
  \theta_t,  [E'']$ are linearly\pageoriginale independent and generate $J(E'')
  $. Then the proposition follows from the fact that a non-zero
  solution of $J(E'')$ together with   $E''$ generate a $q$-dimensional
  integral element for $E''$ in $\ell^{q-1} \sum$. 
\end{proof}

When $\dim E' = 0$,  that is when $E'$ is the origin of $E$,  we have
the following corollary. 

\noindent
\textbf{Corollary to Proposition 8.} Let $E$ be an ordinary integral
element of origin $z$. Then for neighbourhood $\mathscr{U}$ of $E$
there is a neighbourhood $U $ of $z$ such that for any $z'$ in $U \cap
\ell^0 \sum$ there is an integral element $E'' \in \mathscr{U}$ with
origin $z'$. 

\section{}\label{chap2:sec2.5} 

Let $(\sum)$ be a differential system on a domain $D$ in $R^n$ Then we
pose the following definitions. 

\begin{defi*} % def
  Let $E$ be a $q$-dimensional contact element of $D$. A flag on $E$ is
  defined to be a finite sequence of subspaces of $E$: 
  $$
  \{ 0 \} = E_0 \subset E_1 \subset \cdots \subset E_q = E, 
  $$
  such that the dimension of $E_r$ is $r (r = 0,  1,  \ldots ,  q)$.
\end{defi*}

$q$ is called the dimension of the flag and $E_r$ is called the
$r^{th}$ component of the flag on $E ~ (r = 0,  1,  \ldots ,  q)$. 

\begin{defi*} % def
  Let $E$ be an integral element of $(\sum)$. A flag in $E$ with
  components $E_r$,  is said to be normal if $t(E_r) = t_r (E)$ holds
  for $r= 0,  1,  \ldots ,  q-1$ (cf. Proposition \ref{chap2:sec2.3:prop6}). 
\end{defi*}

\begin{defi*} % def
  For an ordinary integral element $E$ of $(\sum)$ a flag on $E$ is
  called regular if each component $E_r$ is a regular integral element
  $(r = 0,  1,  \ldots ,  q-1)$. 
\end{defi*}

It\pageoriginale is clear by definition and by Corollary to Proposition
\ref{chap2:sec2.3:prop6} that if
$E$ is an ordinary integral element of $(\sum)$,  a regular flag on
$E$ is a normal flag. 

\begin{proposition}\label{chap2:sec2.5:prop9}  % prop 9
  Given a $q$-dimensional ordinary integral element of $(\sum)$ there
  exists atleast one regular flag on $E$. 
\end{proposition}

\begin{proof}
  Take a regular $E'_1$ in $E$. There exists a neighbourhood $U_1$ of
  $E'_1$ in $G^1_E$ such that any $E''_1 \in U_1$ is regular. Let
  $U'_2 = \{E''_2 \in G^2_E $ : there exists $E''_1 \in U_1$ such that
  $E''_1 \subset E''_2 \}$. Clearly this is a non-empty open subset of
  $G^2_E$. There exists a regular $E''_2 \in U'_2$ such that $t
  (E''_2) = t_2 (E)$. Thus we find $E''_1 \subset  E''_2 \subset E$
  such that $E''_r$ is regular and $r$-dimensional for $r = 1,  2$. Now
  we proceed similarly by induction on $r$ upto $q-1$ and thus the
  assertion is proved.  
\end{proof}

Given an element $E$ in $\mathscr{G}^q (D)$ the set $\widetilde{G(E)}$ of all
flags on $E$,  can be made into a real analytic manifold by
considering it as a homogeneous space as follows: Let $G \ell (q)$ be
the general linear group of $E$. Fix a flag $E_0 \subset E_1 \subset
\cdots \subset E_q = E$ on $E$. We define a map $f$ of $G \ell (q)$
into $\tilde{G} (E)$ as follows: For $A \in G \ell (q), f(A)$ is the
flag $AE_0 \subset AE_1 \subset \cdots \subset AE_q = E$. It is clear
that $f$ is surjective. Denoting by $H$ the subgroup of all the
elements which  leaves each $E_q$ invariant,  we see easily that $f$
induces a bijective mapping of $G \ell (q)  / H$ onto $\tilde{G}
(E)$. Hence we can identify $\tilde{G} (E)$ with $G \ell (q) / H$ and
take its real analytic structure. It is easy to see  the following: 

\begin{proposition}\label{chap2:sec2.5:prop10}  % prop 10
  The set of all normal flags on an integral element  $E$\pageoriginale is a
  non-empty open subset of $\tilde{G} (E)$. 
\end{proposition}

Proof is similar to that of Proposition \ref{chap2:sec2.5:prop9}.

Let $E$ be an element of $\ell^q \sum$ of origin $z$. We assume that
$\ell^0 \sum$ is a submanifold on a neighbourhood of $z$ and
$\sum^{(0)}= 0$ is a regular local equation of $\ell^0 \sum$ around
$z$. We define a sequence of integers $s_r (E) ~ (r = - 1,  0,  \ldots
,  q)$ by setting 
\begin{align*}
  s_{-1} (E) & = t_{-1} (E)  = ~\text {co-dimension of }~ \ell^0 \sum
  \text{around}  ~ z\\ 
  s_r (E) & = t_r (E) - t_{r-1} (E) ~\text{ for }~ r = 0,  1,  \ldots ,  q-1\\
  s_q (E) & = n - q-t_{q-1} (E).
\end{align*}

Hence $s_{-1} (E) + \cdots + s_q (E) = n-q$.

\begin{proposition}\label{chap2:sec2.5:prop11}  % prop 11
  If $E \in \ell^q \sum$ satisfies the above conditions,  then $s_r
  (E) \geq 0 ~ (r = - 1,  \ldots ,  q)$. In particular,  if $E$ is
  ordinary then $s_r (E) \ge 0$. 
\end{proposition}

\begin{proof}
  Since $\sum^{(0)} = 0$ is a regular local equation $\ell^0 \sum$ is
  a submanifold around the origin $z$ of $E$. By
  Proposition \ref{chap2:sec2.5:prop9} there
  exists a normal flag on $E$,  say $E_0 \subset E_1 \subset \cdots
  \subset E_q = E$ with $t_r (E) = t(E_r)$. By Proposition
  \ref{chap2:sec2.2:prop4}, $t(E_r)
  \ge t (E_{r-1})$. Hence $s_r (E) \geq 0$ for $r=1,  \ldots ,  q-1$.  
\end{proof}

\noindent
\textbf{Case when {\boldmath $r = 0$}} $(\sum)$ being closed $f \in \sum^{(0)}$
implies $df \in \sum^{(1)}$. The set of all $(df)_z$,  where $f \in
\sum^{(0)}$,  will generate a subspace $A$ of the dual of $E$,  whose
dimension is $s_{-1} (E)$. Since $(\sum)$ is  closed $A \subset J(z)$
and hence $s_0 (E) = t_0 (E) - t _{-1} (E) \ge 0$. 
 
\noindent
\textbf{Case when {\boldmath $r = q$}.}\pageoriginale For every $L \in E$ and for any $\alpha
\in J(E_{q-1}),  \alpha (L) = 0$ implies $\dim J(E_{q-1}) \le n - \dim
E = n-q$ and hence $s_q = n - q - t_{q-1} \ge 0$. 
 
 From the second part of Proposition \ref{chap2:sec2.4:prop7},  we
 easily deduce the following 

\noindent 
\textbf{Corollary to Proposition 7.}
Let $E$ be a $q$-dimensional ordinary integral element of $\sum$. Then
the dimension of $\ell^q \sum$ is equal to $(n - s_{-1} (E)) +
\sum^q_{r = 1} r s_r (E)$,  where $n = \dim D$. 

\section{}\label{chap2:sec2.6}

Let $(\sum)$ be a differential system on $D$. Let $D_1$ be a
submanifold of $D$. Denote by $(\sum_1)$ the differential system on
$D_1$ generated by the restriction of $(\sum)$ to $D_1$. Then
$\mathscr{G}^q (D_1)$ is a submanifold of $\mathscr{G}^q (D)$ and so
$\ell^q \sum_1$ is a subset of $\mathscr{G}^q D$. We easily see the
following: 
 $$
 \ell^q \sum_1 = \ell^q \sum \cap \mathscr{G}^q (D_1).
 $$
 
Let $z$ be a point of $D_1$. Then the injection mapping $i : D_1 \to
D$ induces a homomorphism $i^*_z$ of $(D)^*_z$ onto $(D_1)^*_z$. Let
$E$ be an integral element of $(\sum_1)$ with $z$ as its origin. Then
by the remark above $E$ is also an integral element of
$(\sum)$. Therefore we  have $J(E, \sum)$ and $J (E,  \sum_1)$ which
are subspaces of $(D)^*_z$ and $(D_1)^*_z$ respectively. Let $A_z$
denote the kernel of $i^*_z$. 

\begin{proposition}\label{chap2:sec2.6:prop12} % prop 12
  Notations being as above,  $i^*_z$ induces a surjective mapping of
  $J(E,  \sum)$ onto $J (E,  \sum_1)$. If,  moreover,  for any $\beta
  (\neq 0)$ in $A_z$ there is an integral element $E'$ of $(\sum)$
  containing $E$ and there\pageoriginale is an $L$ in $E'$ with $\beta (L) \neq 0$,
  then $i^*_z$ induces a bijective mapping of $J (E,  \sum)$ onto
  $J(E,  \sum_1)$. 
\end{proposition} 

\begin{proof}
  The first assertion is an immediate consequence of the definitions
  of $J(E,  \sum)$ and $J(E,  \sum_1)$. As for the second,  take
  $\beta$ in $J(E,  \sum) \cap A_z$. Then,  because $E'$ is an
  integral element,  $\beta (L) = 0$ for any $L$ in $E'$. Therefore
  $\beta = 0$ because of the assumption that there exists an $L$ in
  $E'$ with $\beta (L) \neq 0$ for non-zero $\beta$. 
\end{proof} 
 
Let $f$ be a function on $D$ such that $df \big | E_q \neq 0$ where
$E_q$ is a $q$-dimensional ordinary integral element of $(\sum)$ on $D$.
Let us assume further that $D_1$ is the submanifold of $D$ defined by
the equation $f = 0$. Take a regular flag of $(\sum)$ on $E$ : $E_0
\subset E_1 \subset \cdots \subset E_q$. Assume that $E_{q-1} = \{L
\in E : \langle df,  L \rangle = 0 \}$. Then we have the following
proposition: 
\begin{proposition}\label{chap2:sec2.6:prop13} % prop 13
  $E_r (r \le q - 1)$,  regarded as a contact element of $D_1$,  is a
  regular integral element of $(\sum_1)$. Moreover,  we have  
  \begin{multline*}
    s_r \left(E_{q-1},  \sum_1\right) = s_r \left(E_q,  \sum\right)
    \quad \text { for } r =  - 1,  \ldots ,  q - 2;\\ 
    s_{q-1} \left(E_{q-1} ,  \sum_1\right) = s_{q-1} \left(E_q,
    \sum\right) + s_q \left(E_q, \sum\right). 
  \end{multline*}
\end{proposition}
 
\begin{proof}
  First,  we show that there is a neighbourhood $U^r$ of $E_r$ in
  $\mathscr{G}^r (D)$ such that for any $E''$ in $U^r \cap \ell^r
  \sum_1 t(E'',  \sum_1) = t(E_r. \sum)$. Take a neighbourhood $U$ of
  $E_q$ in $\mathscr{G}^q D$ such that,  for any $E'$ in $U$,  $df
  \big | E' \neq 0$. By Proposition \ref{chap2:sec2.4:prop8} there is
  a neighbourhood $U^r$ 
  of $E_r$ such that for any $E''$ in $U^r \cap \ell^r \sum$,  there
  is\pageoriginale $E'$ in $U \cap \ell^q \sum$ containing $E''$ as a
  subspace. Then 
  if $E''$ is in $U^r \cap \ell^r \sum_1$,  the conditions in the
  second part of Proposition \ref{chap2:sec2.6:prop12} are satisfied
  for $E''$ and hence it 
  follows that $t (E'',  \sum_1) = t(E'',  \sum)$ from Proposition
  \ref{chap2:sec2.6:prop12}. If we take $U^r$ sufficiently small,  then $t(E'',  \sum) =
  t(E_r,  \sum)$. Now the proof can be completed by induction on
  $r$. When $r = 0$,  the condition $df \big | E_q \neq 0$ implies
  that $\ell^0 \sum_1= \ell^0 \sum \cap D_1$ is a submanifold of $D_1$
  on a neighbourhood of $z$ having regular local equation
  $\sum^{(0)}_1 = 0$. Thus $E_0$ is ordinary and $\dim \ell^0 \sum_1 +
  1 = \dim \ell^0 \sum$. In particular,  $s_{-1} (z,  \sum_1) = s_{-1}
  (z,  \sum)$. On the other hand we have already shown that $t (w,
  \sum_1) = t (z,  \sum)$ for any integral point $w$ of $(\sum_1)$ in
  a sufficiently small neighbourhood of $E_0$ in $\ell^0
  \sum_1$. Hence $E_0$ is a regular  integral point of $(\sum_1) $ and
  $s_0 (z,  \sum_1) = s_0 (z,  \sum)$. Assuming the case of all $r' <
  r$ to show that $E_r$ is in $\mathscr{R}^r \sum_1$ ; it is only
  necessary to show that $t(E',  \sum_1)$ remains constant when $E'$
  is an integral element of $\sum_1$ sufficiently near $E_r$. We have
  already shown this and moreover,  $t(E',  \sum_1) = t (E_r,
  \sum)$. So we have $s_r (E_{q-1},  \sum_1) = s_r (E_q,  \sum)$ for
  $r \le q - 2$. By definition,   
  \begin{multline*}
    s_{q-1} \left(E_{q-1},  \sum_1\right)  = \dim D_1 - (q-1) - t
    \left(E_{q-2},  \sum_1\right)\\ 
     = \dim D- q -t \left(E_{q-1},  \sum\right) + \left(t \left(E_{q-1},
    \sum \right) -t \left(E_{q-2},  \sum\right)\right)\\
      = s_q \left(E_q,  \sum\right) + s_{q-1} \left(E_q, \sum\right).
  \end{multline*}
This completes the proof of the proposition.
\end{proof}

\section{Differential systems with independent
  variables}\label{chap2:sec2.7} %sec 2.7 
 
Let\pageoriginale $(D,  D',  \varpi)$ be a fibred manifold where $D$ and $D'$ are
domains in Euclidean spaces $R^{q+m} $ and $R^q$ respectively. Let
$(\sum)$ be a differential system defined on $D$. A pair consisting of
a fibred manifold $(D,  D',  \varpi)$ and a differential system
$(\sum)$ on $D$ is called a differential system with
  independent variables and is denoted by $\big [ \sum ; (D,  D' ,
  \varpi) \big]$.
 
\begin{defi*}
  A cross-section $f$ of $(D,  D',  \varpi)$ over an open set $U$ of
  $D'$ is said to be an \textit{integral } of the differential system
  $\big [\sum,  (D,  D' ,  \varpi \big]$ if the map $f$ of $U$ into
  $D$ defines a submanifold of $D$ which is an integral submanifold of
  $(\sum)$. 
\end{defi*} 
  
Let $\mathscr{G}^r (D. D',  \varpi)$ denote the set of all
$r$-dimensional contact elements $E$ in $\mathscr{G}^r (D)$ which are
such that if $z$ is the origin of $E,  (d \varpi)_z$ is injective on
$E$. We set  
\begin{align*}
  \ell^r \ \left [\sum,  (D,  D',  \varpi \right ] & = \ell^r \sum
  \cap \mathscr{G}^r (D,  D',  \varpi);\\ 
  \theta^r \left [\sum,  (D,  D',  \varpi \right] & = \theta^r \sum \cap
  \mathscr{G}^r (D,  D',  \varpi);\\ 
  \text{and} \hspace{2cm}\mathscr{R}^r \left[\sum,  (D,  D',  \varpi \right]
  & = \mathscr{R}^r \sum \cap \mathscr{G}^r (D,  D',  \varpi).
\end{align*} 
  
Let $E$ be a $q$-dimensional ordinary integral element of this system
(an element of $\theta^q \left [ \sum,  (D,  D',  \varpi \right
])$ with origin $z$.  Let $(x_1,  \ldots ,  x_q ,  y_1,  \ldots$,
$y_m)$ be a coordinate system of $(D,  D',  \varpi)$ around $z$. 

\begin{defi*} % def 
  A coordinate system $(x_1,  \ldots ,  x_q ,  y_1,  \ldots ,  y_m)$
  of $(D,  D',  \varpi)$\break around\pageoriginale $z$ is said to be regular with respect
  to an ordinary integral element $E$ of the system $\left[ \sum,  (D,
    D',  \varpi) \right]$ if $x_i (z) = y_\lambda (z) = 0$ and if the
  following conditions are satisfied: 
  \begin{enumerate} [\rm (i)]
  \item $E_r = \{L \in E : \langle dx_{r+1} ,  L \rangle = \cdots =
    \langle dx_q ,  L \rangle = 0$ is a regular integral element of
    $\left(\sum\right)$ for $r = 0,  1,  \ldots ,  q-1$;  
  \item $y_1,  \ldots,  y_{s_{-1}}$,  where $s_{-1}=s_{-1}(E)$,  are
    in $\sum^{(0)}$;  

  \item $(dx_1)_z ,  \ldots ,  (dx_q)_z,  (dy_{t_r+1})_z,  \ldots ,
    (dy_m)_z$,  where $t_r = t_r (E) = s_{-1}\break (E) + \cdots + s_r (E)$,
    are linearly independent modulo $J (E_r,  \sum)$ for $r \le q-1$;
    and  
  \item $E$ is equal to the tangent vector space at $z$ to the
    submanifold $y_1 = \cdots = y_m = 0$. 
  \end{enumerate}
\end{defi*} 


 Put $S = (s_0,  \ldots ,  s_q)$,  where $s_r = s_r (E) ~ (s_q$ may be
 zero; $S$ is a system of characters (cf. $1.2)$. $A \xi $ in $H(S)$
 has $s_0 + \cdots + s_q = m-s_{-1}$ components. Namely,  if $s_{r+1}
 \neq 0$,  $\xi_{s_0 +s_1 + \cdots + s_r + j}$ is in $H_{r+1}$ for $j =
 1,  \ldots ,  s_{r+1}$. Let $(x,  y)$ be a fixed regular coordinate
 system and $\xi$ in $H(S)$ be given. 

\begin{defi*} % def
   A cross-section $f$ of $(D,  D',  \varpi)$ over an open
   neighbourhood $U$ of $\varpi(z)$ is said to have initial condition
   $\xi$ with respect to the regular coordinate system $(x,  y)$ with
   center $z$ if 
   \begin{enumerate} [\rm (i)]
   \item $f(\varpi(z))$ is in the domain of $(x,  y) $ and  
   \item when $f$ is expressed by $y_\lambda = \eta _ \lambda (x_1 ,
     \ldots ,  x_q )$ we have $\eta _1 (x) = \cdots = \eta_{s_{-1}}(x)
     = 0$ and  
     $$
     y_{s_{-1}+s_0+s_{r-1}+ j} (x_1,  \ldots ,  x_r,  0,  \ldots ,  0)
     = \xi_{s_0 + \cdots + s_{r-1}+ j }(x_1 ,  \ldots ,  x_r). 
     $$
   \end{enumerate} 
\end{defi*} 
 
We\pageoriginale note that the last condition has a meaning since $s_{-1}+ s_0 +
\cdots + s_q = \dim D - q = m$. 
\begin{defi*}
  A mapping $F$ of $H(S; u,  v)$ into $H_q^m (u',  v')$ is said to be a
  \em {solution mapping } of the system $\left[\sum,  (D,  D',
    \tilde{\omega)} \right ]$ with respect to a regular coordinate
  system $(x,  y)$ if $y_ \lambda = F_ \lambda (\xi)$ is an integral
  of $(\sum)$ with the initial condition $\xi$. 
\end{defi*}  

\section{}\label{chap2:sec2.8}

\begin{proposition}\label{chap2:sec2.8:prop14} % prop 14
  For any ordinary integral element $E$ of $\left [\sum,  (D,  D',
    \varpi) \right ]$ of dimension $q = \dim D'$,  there exists a
  regular coordinate system in $(D,  D',  \varpi)$ with respect to $E$
  and $\sum$. 
\end{proposition}
 
\begin{proof}
  Take a regular flag $E_0 \subset E_1 \subset \cdots \subset E_q = E$
  on $E$. Take a coordinate system $(x'_1,  \ldots ,  x'_q)$ around
  $\varpi (z)$ in $D'$ $(z$ being the origin of $E$) such that  
  $$
  E_r = \{L \in E : \langle dx_{r+1},  L \rangle = \cdots = \langle
  dx_q,  L \rangle = 0 \} 
  $$
  where $x_j = x'_j o \varpi$. Since $z$ is an ordinary integral point
  there are functions $y_1,  \ldots ,  y_{t_{-1}}$ in $\sum^{(0)}$
  such that $(dy_1)_z,  \ldots ,  (dy_{t_{-1}})_z$ are linearly
  independent. Then,  since $dy_\sigma \big | E = 0,  dy_1 ,  \ldots ,
  dy_{t_{-1}}$,  $dx_1,  \ldots ,  dx_q$ are linearly independent at
  $z$. Therefore $(x_1,  \ldots ,  x_q,  y_1,  \ldots ,  y_{t_{-1}}) $
  can be completed into a coordinate system $(x_1,  \ldots ,  x_q,
  y_1,  \ldots ,  y_m)$ around $z$ in $(D,  D',  \varpi)$. This
  coordinate system satisfies (i) and (ii) of the definition of a
  regular coordinate system.  Since $J (E_r,  \sum)$ are ascending
  with $t_r = \dim (E_r,  \sum)$ for $r= 1,  2,  \ldots ,  q-1$ and
  since they contain $(dy_1)_z,  \ldots ,  (dy_{t_{-1}})_z$ (because
  $(\sum)$\pageoriginale is closed),  it is now clear that,  by applying a linear
  transformation of $x_1,  \ldots ,  x_q ~ y_1,  \ldots ,  y_m$ which
  fixes each $x_j$ (if necessary),  we can construct a coordinate
  system satisfying (i), (ii),  (iii) and (iv) of the
  definition of a regular coordinate system. 
\end{proof} 

\begin{proposition}\label{chap2:sec2.8:prop15} % prop 15
  Let $(x_1,  \ldots ,  x_q ,  y_1 ,  \ldots ,  y_m)$ be a regular
  coordinate system in $(D,  D',  \varpi)$ with respect to an ordinary
  integral element $E$ of $\left [\sum,  (D,  D',  \varpi) \right
  ]$. Denote by $D_1 $ (\resp  $D'_1$) the submanifold of $D$
  (\resp\break  $D'$) defined by $x_q = 0$ (\resp  $x'_q = 0$,  where $x_q =
  x'_q o \varpi)$. Let $(\sum_1)$ be the restriction of $(\sum)$ to
  $D_1$. Let $E_{q-1} = \big \{L \in E : \langle dx_q ,  L \rangle = 0
  \big \}$. Then $E_{q-1}$ is an ordinary integral element of $\left
       [\sum_1,  (D_1,  D'_1,  \varpi) \right ]$ and $(x_1,  \ldots ,
       x_{q-1},  0$,  $y_1 ,  \ldots ,  y_m)$ is a regular coordinate
       system of $(D_1,  D'_1,  \varpi)$ with respect to $E_{q-1}$ and
       $(\sum_1)$  
\end{proposition} 

\begin{proof}
  $E_{q-1}$ is a regular integral element of $(\sum_1)$ by Proposition
  \ref{chap2:sec2.6:prop13} and the coordinate system $(x_1,  \ldots ,  x_{q-1} ,  0 ,  y_1
  ,  \ldots ,  y_m)$ satisfies the condition (i) of a regular
  coordinate system with respect to $E^{q-1}$ and $(\sum_1)$. The
  condition (ii) is clear,  because $s_{-1} (E,  \sum) = s_{-1}
  (E^{q-1},  \sum_1)$ by the same proposition. The restriction mapping
  $i^*_z$ of $(D)^*_z$ onto $(D_1)^*_z$ induces an isomorphism of $J
  (E_r,  \sum)$ onto $J (E_r,  \sum_1)$ and the kernel of $i^*_z$ is
  $(dx_q)_z$. Then the verification of the condition (iii) for
  $(x_1,  \ldots ,  x_{q-1} , 0 ,  y_1 ,  \ldots ,  y_m)$ is
  immediate. 
\end{proof} 

\section{}\label{chap2:sec2.9} %sec 2.9

We fix in this $n^0$ once for all a regular coordinate system $(x,
y)$ in \break $(D. D',  \varpi)$ with respect to an ordinary integral
element  $E^0$\pageoriginale and $(\sum)$. So,  $\dim E^0 = \dim D$. We use the
notations used in the definition of a regular coordinate system and
Proposition \ref{chap2:sec2.8:prop15}. In particular $E_r = \{L \in E^0 ; \langle dx_{r+1},
L \rangle = \cdots = \langle dx_q,  L \rangle = 0 \}$. Set $S = (s_0,
\ldots ,  s_q)$ where $s_r = s_r (E^0)$ and $S_1 = (s_0,  \ldots ,
s_{q-2}$,  $s_{q-1} + s_q)$. We define an infinite analytic mapping
$P$ (everywhere defined) of $H(S)$ onto $H(S_1)$ by setting 
$$
p (\xi ) = (\xi_1,  \ldots ,  \xi_{t_{q-1}},  \xi_{t_{q-1}+1} (x_1 ,
\ldots ,  x_{q-1},  0),  \ldots ,  \xi_m (x_1,  \ldots ,  x_{q-1},
0)) 
$$
 
We remark that the elements of $H(S)$ (\resp  $H(S_1))$ can be regar\-ded
as initial conditions for integrals of $\left [\sum,  (D,  D',  \varpi
  ) \right]$ (\resp  $\big[ \sum_1,  (D_1,\break  D'_1 ,  \varpi) \big]$) as explained in $n^0$ 2.7. 

Take a cross-section $f$ of $(D,  D',  \varpi)$ over an open
neighbourhood of $\varpi (z) ~ (z$  being the origin of $E$),  which
is represented by $y_ \lambda = \eta_\lambda (x_1,  \ldots ,\break  x_q)$
$(\lambda = 1,  \ldots ,  m)$,  with the initial condition $\xi$ in
$H(S)$. Now we state necessary conditions on $\eta_ \lambda$ so that
$f$ is an integral of $\left [ \sum,  (D,  D',  \varpi ) \right ]$
with initial condition $\xi$ in $H(S)$. 

\begin{enumerate} [I.]
\item $(x_1,  \ldots ,  x_{q-1} ,  0 ,  y_1 ,  (x_1,  \ldots ,
  x_{q-1} ,  0),  \ldots ,  y_m (x_1,  \ldots ,  x_{q-1} ,  0))$\hfill\break
  should be an integral of $\left [\sum_1 (D_1,  D'_1, \varpi_1 ) \right
  ]$ with the initial condition $P(\xi)$. This follows easily from the
  definitions involved. 

\item Consider the fibred manifold $(\mathscr{G}^{q-1} (D),  D$,  the
  canonical projection). Let $\alpha_1,  \ldots ,  \alpha_{t_{q-1}}$
  be fibred differential forms on $(\mathscr{G}^{q-1} (D),  D$,
  canonical projection) defined in a neighbourhood of $E_{q-1}$ such
  that $\alpha_1 [E'] ,  \ldots ,  \alpha_{t_{q-1}}[E']$ generate $J
  (E' ,  \sum)$ for  any\pageoriginale integral element $E'$ near $E_{ q -1}$ in
  $\mathscr{G}^{q-1}(D)$ (cf. $n^o 2.5$, and Proposition
  \ref{chap2:sec2.3:prop5}). Now
  suppose $f$, expressed by $y_ \lambda = \eta_\lambda (x_1, \ldots ,
  x_q)$ is an integral of $( \sum )$ over an open neighbourhood $U$ of
  $\tilde{\omega}(z)$ in $D, M = f(U)$ is an integral submanifold of
  $D$. If $(x, \eta (x))\in M$ denote by $E_x$ the tangent vector
  space $(M)_{x, \eta (x)}$ to $M$ at $(x, \eta (x))$. $E_x$ is an
  integral element of $( \Sigma)$. Let $L' (E_x) , \ldots , L^q(E_x)$
  be a basis of $E_x$ such that $\tilde{\omega} (L' (E_x)) , \ldots ,
  \tilde{\omega}(L^q(E_x))$ are dual to $dx_1 | \tilde{\omega} (E_x),
  \ldots , dx_q | \tilde{\omega}( E_x)$.\break Let $E'_x$ be the subspace of
  $E_x$ generated by $L' (E_x), \ldots , L^{q-1}(E_x)$. Clearly $E'_x$ is
  an integral element of $( \Sigma )$. $L^q(E_x)$ must be a solution
  of $J(E'_x, \Sigma) =0$ and hence a solution of $\alpha_\sigma[E'_x]
  (L^q(E_x))=0$ $(\sigma =1, \ldots , t_{q-1}) (*)$. Let us express this
  condition $(*)$ by using the coordinate system. We have, for
  $\sigma=1, \ldots , t_{q-1}$, 
  $$
  \alpha_ \sigma = ^{'} b^i_{\sigma} {dx_i} +  ^{'} a^\lambda_\sigma dy_\lambda
  $$   
  where $^{'}b^i_\sigma$ and $^{'}a^\lambda_\sigma$ are real analytic
  functions on a neighbourhood of $E_{q-1}$ in
  $\mathscr{G}^{q-1}(D)$. Since $(x, y)$ is a regular coordinate
  system $(dx_1)_z , \ldots , (dx_q) _z, \alpha_1 [E_{q-1}] , \dots ,
  \alpha_{t_{q-1}} [E_{q-1}], (dy_{t_{q-1}}+1)_z \ldots , (dy_m)_z$
  are linearly independent. Hence we can find real analytic functions
  $c^{\sigma'}_\sigma ( \sigma, \sigma' = 1, \ldots , t_{q-1})$ on a
  neighbourhood of $E_{q-1}$ such that  
  \begin{multline*}
    c^{\sigma'}_\sigma \alpha_{\sigma'} = dy_\sigma - d^i_\sigma dx_i
    - a^{t+\mu}_\sigma \text{dy}_{t + \mu}\\ 
    (\sigma, \sigma' =1 , \ldots , t_{q-1}, t=t_{q-1}, \mu= 1 , \ldots , m-t)
  \end{multline*}
  
  Since\pageoriginale $dx_1| E_{q-1}, \ldots, dx_{q-1} |  E_{q-1}$ are linearly
  independent, we have the coordinate system $(x, y, w^1, \ldots ,
  w^{q-1}, \ldots , y^r_\lambda , \ldots) ( \lambda =1, \ldots , m$;
  $r=1, \ldots , q-1)$ on a neighbourhood of $E_{q-1}$ in
  $\mathscr{G}^{q-1}(D)$ such that  
  $$
  L^r (E') = \frac{\partial}{\partial x^r} + w^r(E')
  \frac{\partial}{\partial x_q} +  y^r_ \lambda (E')
  \frac{\partial}{\partial y_\lambda} 
  $$
  where $L^1 (E'), \ldots, L^{q-1}(E')$ is a basis of $E'$ dual to
  $dx_1| E' , \ldots$,\break  $dx_{q-1}| E'$. Then  
  $$
  L^r(E'_x) = L^r(E_x) = \frac{\partial}{\partial x_r} +
  \frac{\partial \eta_\lambda}{\partial x_r} \frac{\partial}{\partial
    y_\lambda} 
  $$
  and therefore $E'_x$ has the coordinates $w^r(E'_x) = 0$,
  $y^r_\lambda(E'_x) = \dfrac{\partial \eta_\lambda}{\partial x_r}$,
  Let $A^{t  +  \mu}(x, y, w, \ldots , y^r_\lambda , \ldots)$,
  $B_\sigma (x,y,w, \ldots , y^r_\lambda , \ldots )$ be the expression
  of the functions $a^{t+\mu}_\sigma, b^q_\sigma$ in terms of the
  coordinate system. Then the condition $(*)$ can be expressed as  
  \begin{multline*}
  \frac{\partial \eta_\sigma}{\partial x_q}= B_\sigma \left(x, \eta,\ldots
  ,\frac{\partial \eta_\lambda}{\partial x_r},  \ldots \right)\\  
  +  A^{t + \mu}_\sigma \left(x, \eta, \ldots , \frac{\partial
    \eta_\lambda}{\partial x_r}, \ldots \right) \frac{\partial \eta_{t+
      \mu}}{\partial x_q} \quad ( r=1, \ldots , q-1) 
  \end{multline*}
  
  On the otherhand, by the initial condition of $\eta$, we have 
  $$
  \eta_{t + \mu}(x_1,  \ldots , x_q) = \xi_{s_o +  \cdots + s_{q-1}+
    \mu} (x_1, \ldots , x_q). 
  $$

  Hence, by  setting $C^{\xi}_\sigma (x_1, \ldots , x_q)\, y_1 ,
  \ldots , y_t , y^r_k , \ldots)~ (k=1, \ldots
  , t ; r =1, \ldots , q-1)$, the function obtained by substituting
  $y_{t + \mu}= \xi_{s_o +\cdots + s_{q-1}}\break + \mu (x_1, \ldots , x_q),
  y^i_{t + \mu}= \dfrac{\partial \xi_{s_o+ \cdots + s_{q-1}}+ \mu}{
    \partial x_i} (i=1, \ldots , q)$  in $B_\sigma (x, y, \ldots$,
  $y^r_\lambda , \ldots) + A^{t+ \mu} (x,y, \ldots , y^r_\lambda,
  \ldots ) y^q_{t +  \mu} $ in the above differential\pageoriginale equation, the
  condition $(*)$ can be expressed by  
  \begin{multline*}
    \frac{\partial \eta_\sigma}{\partial x_q}= C^\xi_\sigma \left(x_1,
    \ldots , x_q, \eta_1 , \ldots \eta_t , \ldots, \frac{\partial
      \eta_k}{\partial x^r}, \ldots \right) \\
    ( \sigma, k=1 , \ldots , t ;  r=1, \ldots , q-1).
  \end{multline*}
\end{enumerate}
  Now we claim that the above two conditions are also sufficient. More
  precisely we have the following: 
\begin{proposition}\label{chap2:sec2.9:prop16}% prp 16
  Let $f$ be a cross-section of $(D, D' \tilde{\omega})$ over an open
  neighbourhood $U$ of $\tilde{\omega}(z)$, expressed by $y_\lambda =
  \eta_\lambda (x_1, \ldots , x_q)$, with the initial condition
  $\xi$. Assume that the tangent space of $M=f(U)$ at the point over
  $\tilde{\omega}(z)$ is sufficiently near the ordinary integral
  element $E ~( \dim E = q)$. Then $f$ is an integral of $( \Sigma)$ if
  and only if the  following two conditions are satisfied: 
  \begin{enumerate}[\rm(i)]
  \item $y_\lambda = \eta_\lambda (x_1, \ldots  , x_{q-1}, 0)$
    represent an integral  of $(\Sigma_1)$ with the initial condition
    $P (\xi)$; 
  \item $(y_1= \eta_1 (x_1, \ldots , x_q) , \ldots , y_t= \eta_t( x_1
    , \ldots, x_q))$ is a solution of the system of equations 
  \end{enumerate}
  \begin{equation}
    \frac{\partial y_\sigma}{\partial x_q}= C^{\xi}_\sigma  \left(x_1,
    \ldots , x_q, y_1 , \ldots , y_t , \ldots , \frac{\partial
      y_k}{\partial x_r}, \dots \right)  \tag{**} 
  \end{equation}
\end{proposition}

In order to prove the proposition we make the following preliminaries:

Let $\varphi$ be a homogeneous differential form of degree  $h$ on $D$
and let $f$ be a cross-section of $(D, D' , \tilde{\omega})$ over an
open neighbourhood of $\tilde{\omega}(z)$ in $D$. Denote, as before,
by $M$ the image $f(U)$. Let $i_M: M \to D$ be the injection
mapping. Then  
$$
i^*_M \varphi = \sum_{i_1 < \cdots < i_h} \varphi _M^{i_1 \cdots i_h}
dx_{i_1} \wedge \cdots \wedge dx_{i_h}, 
$$ 
where\pageoriginale $x_i o i_M$ is also denoted by $x_i$. Then by definition
(cf. $n^o 2.4$) of $\varphi [i_1 , \ldots , i_h]$ we have  
$$
\varphi^{i_1, \ldots i_h}_M (x) = ( \varphi [i_1 , \ldots , i_h])
(E_x), 
$$
where again $E_x$ is the tangent vector space to $M$ at $x$. $E_x$ is
regarded as an element of $\mathscr{G}^q(D)$. Since $i_M^* (d \varphi
) =  d(i_M^* \varphi)$, the above equality implies 
$$
(\chi)  \sum_{s=1}^{h+1} (-1)^s \frac{\partial \varphi^{i_1 , \ldots
    \hat{i}_s \cdots i_{h+1}}}{\partial x_{i_s}} = (( d \varphi ) [i_1
  , \ldots , i_{h+1}]) (E_x)  
$$

For a fibred differential form $\alpha$ of the fibred manifold
$\mathscr{G}^{q-1} D,D,$ the canonical projection), we define a
function $\tilde{\alpha}$ of $\mathscr{G}^q (d ; x_1 , \ldots ,  x_q)$
as follows: For $E$ in $\mathscr{G}^q (D ; x_1 , ,\ldots , x_q)$ let
$\pi (E)$ be the subspace defined by $dx_q = 0$, or equivalently
generated by $L^1(E), \ldots , L^{q-1}(E) $. Set $\tilde{\alpha}(E) =
( \alpha [\pi (E)] ) (L^q(E))$. In this notation the condition $(*)$
can be expressed as  
\begin{equation}
  \tilde{\alpha}_\sigma (E_x) = 0 ~( \sigma= 1, \ldots , t_{q-1}= t), \tag{* 1}
\end{equation} 
where we adopt same notations as before. We choose real analytic
functions $f_1, \ldots,f_k$ (defined on a neighbourhood of $E_{q-1}=
\pi (E))$ from $F(\sum , x_1, \ldots , x_q)$ (cf. Proposition
\ref{chap2:sec2.4:prop7}), 
such that $f_1= \cdots = f_k =0$ is a regular local equation of
$\ell^{q-1} \sum$ on a neighbourhood of $E_{q-1}$ in
$\mathscr{G}^{q-1}D$. We recall that $f_1 o \pi =  \cdots = f_{k^o}
\pi =  \tilde{\alpha}_1 = \cdots = \tilde{\propto}_t = 0$ is a regular
local equation of $\ell^q \sum$ on a  neighbourhood $\mathcal{U}$ of
$E$ (cf. Proof of Proposition \ref{chap2:sec2.4:prop7}). Take  $\mathcal{U}$ so small\pageoriginale that,
besides the above property, we have real analytic functions
$C^{\sigma'}_\sigma$, as in  $(\#)$, defined on $\mathcal{U}$ and such
that every element $E$ of $\mathcal{U} \cap \ell^q \sum$ is ordinary
and $\pi(E_1)$ is regular. 

\noindent 
\textbf{proof of the proposition 16.}  
We have only to show that the conditions are sufficient. We assume
that the tangent vector space of $M$ at the point over
$\tilde{\omega}(z)$ is in the above neighbourhood $\mathcal{U}$ of
$E$. 
To prove that $M$ is an integral submanifold is the same as proving
that $E_x$ are in $\ell^q \sum$.  By the condition (2) which is
equivalent to $(*)$ (and hence to $(*')$), $\tilde{\alpha}_\sigma
(E_x)=0$. Hence it remains to show that $f_i o \pi (E_x) = \dots  =
f_{k} o \pi (E_x) =0$. Set $g_\theta (x_1 , \ldots , x_q) =f_\theta o  \pi
(E_x)$ for $\theta = 1 , \ldots , k$. Since $f_\theta$ are in $F(\sum,
x_1 , \ldots , x_{q-1})$ each $f_\theta$ is expressed as  
$$
\displaylines{\hfill 
f_\theta = \varphi  [i_1, \ldots , i_h]  , \varphi \in \sum^{(h)},\hfill
\cr 
\text{where}\hfill 1 \le i_1 \cdots < i_h \le q-1.\hfill } 
$$

Therefore, by $( \chi )$, we obtain 
$$
 (\chi \chi) (-1)^h \frac{\partial g_\theta}{\partial x_q} =
\sum^h_{s=1} (-1)^s \frac{\partial \varphi^{i_1 \cdots \hat{i}_s
    \cdots i_hq}_M}{\partial x_{i_s}} - (( d \varphi) [i_1, \ldots ,
  i_h ,q] ) (E_x) 
$$
 
Since $\varphi$ and $d \varphi$ are in $( \sum)$ and since
$$
f_1 o \pi = \cdots = f_k o \pi = \tilde{\alpha}_1 = \cdots =
\tilde{\alpha}_t =0 
$$ 
is a regular local  equation of $\ell^q \sum$ around $E$, we  have 
\begin{multline*}
  \varphi [i_1 , \ldots , \hat{i}_s , \ldots , i_h ,q] =
  X^\theta_s. (f_\theta o \pi ) +  Y_s^\lambda
  . \tilde{\alpha}_\lambda, \\ 
  d \varphi[ i_1 , \ldots , i_h,q]= 'X^\theta . (f_\theta o \pi )
  + 'Y^\lambda . \tilde{\alpha}_\lambda ,  
\end{multline*}
where\pageoriginale $X^\theta_s, 'X^\theta, Y^\lambda_s, 'Y^\lambda$ are real
analytic functions on the neighbourhood of  $E$. (This is here,
where we use the closeness of $( \sum )$ most
essentially). Therefore, by the definition of $\varphi^{i_1 , \ldots
  i_h}_{M}$, the equation $(\chi \chi)$ can be written in the form 
$$
\frac{\partial g_\theta}{\partial x_q} = W^{\theta
  '}_{\theta^{g_{\theta'}}}+ Z_\theta^{\theta ',r}  \frac{\partial
  g_{\theta'}}{\partial x_r}.   
$$ 
 
In the above $\tilde{\alpha} (E'_x)$ and their derivatives do not
appear because they are already known to be zero. This equation is of
Cauchy Kowalewski type. Hence the solution $g _{\theta}(x_1 , \ldots ,
x_q)$ is uniquely determined by functions $g_\theta (x_1 , \ldots ,
x_{q-1}, 0)$. But $g_\theta (x_1 , \ldots , x_{q-1}, 0)$ are zero,
because of the condition (1). Then $g_\theta = 0$ is clearly the
solution, so we proved that $g_\theta (x_1 , \ldots , x_q)=0$
. Therefore $M$ is an integral submanifold and $f$ is an integral of
$( \sum)$. 
 
\section{}\label{chap2:sec2.10} % 2.10 

Now we are in a position to formulate and prove the main theorems of
this Chapter. Let $[ \sum , (D, D', \tilde{\omega})]$ be a
defferential system with independent variables. Set $q =  \dim
D'$. Let $ E^o$ be a $q$-dimensional \textit{ ordinary } integral
element of the system. Take a regular coordinate system $(x_1, \ldots
, x_q, y_1, \ldots , y_m)= (x, y)$ in $(D,D' , \tilde{\omega})$ with
respect to $E^o$ and $\sum$, which is known to exist by Proposition  
\ref{chap2:sec2.8:prop14}. To $E^o$ we have already associated a system if characters $S$,
and $H(S)$ was considered to be initial conditions for cross sections
over open neighbourhoods of $\tilde{\omega} (z)$ where $z$ is the
origin of $E^o$ (cf, 2.7). We remark that $x_i (z) = y_\lambda (z)
=0$. 

Under the above notations and assumptions, we have the following theorem:
\setcounter{theorem}{0}
\begin{theorem}\label{chap2:sec2.10:thm1}%Thm 1
  \begin{enumerate} [\rm (i)]
  \item For\pageoriginale any given $\xi$ in $H(S)$, the germ of integrals of $[
    \sum, (D, D', \tilde{\omega} )]$ over $\tilde{\omega} (z)$ with
    the initial condition $\xi$ (with respect to $(x,y))$ is unique if
    it exists; 
  \item there exists a solution mapping $F$ with respect to $(x, y)$
    such that the mapping  $\xi \to F(\xi)- F(0)$ is finite analytic; 
  \item if $y_\lambda = 0$ is an integral, then there exists a regular
    solution mapping and any two such regular solution mappings are
    equivalent and determine a germ of infinite analytic mapping. 
  \end{enumerate}
  
  Let $p$ be the largest integral such that $s_p \neq 0$, $s_{p +1} =
  \cdots = s_q = 0$. 
\end{theorem} 

\begin{theorem}\label{chap2:sec2.10:thm2}%Thm 2
  Let $M_0$ be a germ of integral submanifolds of $[\sum, (D,D'$,
    $\tilde{\omega})]$ at $z$. Assume that the tangent vector space
  $E^0$ to $M_0$ at $z$ is an ordinary integral element of
  $(\sum)$. Then the set of germs over $\tilde{\omega} (z)$ of
  integrals of $[ \sum , (D, D' , \tilde{\omega})]$ depend on $s_p$
  functions in $p$ variables around $M_0$. 
\end{theorem} 

The proofs of theorems \ref{chap2:sec2.10:thm1} and
\ref{chap2:sec2.10:thm2} are given in the following section. 

\section{}\label{chap2:sec2.11} % 2.11
 
We reduce the problem to the case of $(q-1)$-variables. For this
purpose we introduce the mapping $P$ of $H(S)$ into $H(S')$, where $S=
(s_0, \ldots , s_q)$ and $S' (s_0, \ldots, s_{q-2}, s_{q-1} +  s_q)$,
defined by  
\begin{multline*}
  P_\sigma ( \xi) (x_1, \ldots , x_{q-1})  = \xi_\sigma (x_1 , \ldots,
  x_{q-1}) , \sigma  \le s_o + \cdots + s_{q-1} \\ 
  P_{s_o+ \cdots +  s_{q-1}+ \sigma} ( \xi) (x_1 , \ldots , x_{q-1})\\
   = \xi_{s_o+  \cdots + s_{q-1} +  \sigma} (x_1 , \ldots , x_{q-1}, 0)
  \text{ for } 1 \le \sigma \le s_q  
\end{multline*}
 
 Let\pageoriginale $(D_1, D'_1 , \tilde{\omega})$ be the subfibred manifold of $(D,
 D' , \tilde{\omega})$ where $D_1$ (\resp  $D'_1$) is defined by the
 coordinate system $(x_1, \ldots , x_{q-1}, 0, y_1 , \ldots ,\break
 y_m)$ (\resp  $(x_1, \ldots , x_{q-1}, 0)$) and $\tilde{\omega}_1$ is
 the restriction of $\tilde{\omega}$ to $D_1$. Let $(\sum_1)$ be the
 differential system on $D_1$ generated by the restriction of $( \sum
 )$ to $D_1$ and let $E$ be an ordinary integral element of $[ \sum,
   (D, D', \tilde{\omega})]$. 
 
Now according to Proposition \ref{chap2:sec2.9:prop16} a section $f$ of $(D, D',
\tilde{\omega})$ represented by $y_ \lambda = \eta_\lambda (x_1,
\ldots , x_q)$ with the initial condition $\xi$ at $\tilde{\omega}
(z)$ is an integral of $[\sum, (D, D', \tilde{\omega})]$ if and only
if the following two conditions are satisfied: 
\begin{enumerate}[(1)]
\item $y_ \lambda = \eta_\lambda (x_1, \ldots , x_{q-1},0)$ represents
  an integral of $[\sum_1, (D_1, D'_1, \tilde{\omega}_1)]$ with the
  initial condition $P(\xi)$ at $\tilde{\omega}(z)$  
\item $y_k = \eta_k (x_1, \ldots , x_q)$ for $k=1, \ldots t =
  t_{q-1}$, is the system of solutions of the Cauchy -Kowalewski
  system  of equations 
\end{enumerate} 
$$
(\mathscr{G}^\xi) \frac{\partial y_\sigma}{\partial x_q} =
A^\xi_\sigma (x,y_1 , \ldots , y_t , \frac{\partial y_\sigma'}{\partial
  x_r} (\sigma, \sigma' = 1, \ldots , t; r=1 \cdots q-1) 
$$ 

Here $A^\xi_\sigma (x_1, \ldots , x_q, y_1 , \ldots , y_t , \ldots ,
y^r_{\sigma'}, \ldots )$ is equal to  
\begin{multline*}
A_\sigma \Big(x_1 , \ldots ,x_q, y_1
, \ldots , y_t , \ldots , y^r_\sigma , \ldots , \xi_{t+1}(x_1 , \ldots
, x_q) ,\\ \ldots , \xi_{m}(x_1 , \ldots , x_q), \ldots ,
\dfrac{\partial \xi_{t + \mu}}{\partial x_j} , \ldots \Big)
\end{multline*}
with
$A_\sigma(x_1 , \ldots , x_q , y_1 , \ldots , y_t , \ldots ,
y^r_{\sigma'} , \ldots , y_{t+1}, \ldots  y_m , \ldots , y^j_{t +  \mu},
\ldots)$ are\break analytic functions of all their arguments for $t=s_{-1}
+ \cdots + s_{q-1} ; r=1, \ldots , q-1; j=1 , \ldots , q' \sigma,
\sigma' =1, \ldots ,t$. $(\mathscr{G}^\xi)$ is a system of equations
of Cauchy-Kowalewski type with parameter in which the derivatives of
the parameters also applear . There exists a unique solutions of the
system $(\mathscr{G}^\xi)$ by Theorem \ref{chap1:sec1.9:thm5} of\pageoriginale Chapter I. Hence, by
Theorems \ref{chap1:sec1.9:thm5} and \ref{chap1:sec1.9:thm6} of
Chapter $I$, there exists a solution mapping 
$F''$ of $(\mathscr{G}^\xi)$ and $\exists ''$ represents a germ of
infinite analytic maps of $H^t_{q-1} + H^{m-t}_q$ into $H^t_q$. 
 
We complete the proof of Theorem \ref{chap2:sec2.10:thm1} by induction on $q$.
\begin{enumerate}[(i)]
\item Let $f$ be a germ of integrals of  $[\sum, (D, D',
  \tilde{\omega} )]$ over $\tilde{\omega} (z)$ with the initial
  condition $\xi$. Then the restriction $f_1$ of the cross-section $f$
  to $D_1$ is a germ of integrals of $[\sum_1 , (D, D'_1,
    \tilde{\omega}_1)]$ with the initial condition $P (\xi)$ at
  $\tilde{\omega} (z)$ . So $f_1$ is uniquely determined by the
  induction assumption. Then $f$ is unique since $\eta_1 (x), \ldots,
  \eta_t(x)$ is the solution of $(\mathscr{G}^\xi)$ with the
  prescribed initial functions $\eta_\sigma(x_1, \ldots , x_{q-1}, 0)$
  and $\eta_{t + \tau} (x) = \xi_\tau (x)$. 
\item Let $E'$ be the subspace of $E$ defined  by $dx_q=0$. By
  Proposition \ref{chap2:sec2.9:prop16}, $E'$ is an ordinary integral element of $[ \sum_1
    (D_1, D'_1, \tilde{\omega}_1)]$ and $(x_1, \ldots , x_{q-1}, 0,y)$
  is a regular coordinate system of $(D_1, D'_1, \tilde{\omega}_1)$
  with respect to $E'$ and $(\sum_1)$. Let $F'$ be a solution mapping
  of $ [ \sum_1 (D_1, D'_1, \tilde{\omega}_1)] $ with respect to $(x_1
  , \ldots , x_{q-1}, 0 ,y)$. $F'$ is a mapping of $H(S'; u_1, v_1)$
  into $H^m_{q-1}(u'_1, v'_1)$ with suitable $u_1, v_1, u'_1, v'_1$,
  clearly we can  choose $u'_1$ arbitrarily small. Since the mapping
  $\zeta \to F' (\zeta) - F' (0)$ is finite analytic, choosing $v_1$
  small, we can make sup $\left\{\bigg| F' (\zeta ) - F' (0)  \bigg| ;
  \zeta \in H(S';  u_1, v_1)\right\}$ arbitrarily small. On the
  otherhand $[F(0)] (0)=0$ because the cross-section $y_\lambda =
  F_\lambda (0)$ must pass through $z$. Therefore we can choose $v'_1$
  arbitrarily small (cf. Proposition \ref{chap1:sec1.2:prop1}, Chapter I). Hence we can
  assume without loss of generality that $\mathscr{F} ''$ has a
  representative $F''$ which maps\pageoriginale $H^t_{q-1} (u'_1, v'_1) +
  H^{m-1}_qu'_1, v'_1)$ into $H^t_q(u', v')$. Set $u= \max (u_1,
  u'_1), v = \min (v_1, v'_1)$. Let $F$ be the mapping of $H(S; u, v)$
  into $H^m_q(u' ,v')$ defined by  
  $$
  F_\sigma (\xi)  = F'' ((F'_1 (P(\xi)),  \ldots , F'_t(P( \xi))), (
  \xi_{t+1}, \ldots , \xi_m)) (\sigma =1 \cdots t), 
  $$
  $F_{t + \lambda}(\xi) = \xi_{t +  \lambda} (\lambda =1 \cdots
  m-t)$. Then by Proposition \ref{chap2:sec2.9:prop16} it is easy to verify that $F$
  satisfies the conditions of the solution mapping except for the fact
  that the mapping $\xi \to F (\xi) - F(0)$ is infinite analytic. But
  this follows from the remark in section \ref{chap1:sec1.9} (see
  page~\pageref{chap1:sec1.9:rem34}),  
\item is clear from (ii).
\end{enumerate} 
 
\setcounter{proof of theorem}{1}
\begin{proof of theorem}% 2
  Choose $(x, y)$ such that $M_0$ is represented by $y_\lambda =
  0$. We define a germ $\mathscr{Y}$ of finite analytic mappings of
  $H^m_q$ into $H(S)$ as follows: Take $v$ sufficiently small and for
  each $\eta \in H^m_q$ let $f_\eta$ be the germ of cross-sections of
  $(D, D' , \tilde{\omega})$ represented by $y_\lambda = \eta_\lambda
  (x_1 , \ldots , x_q)$ around $\tilde{\omega} (z)$. Denote by $G(
  \eta)$ the initial condition of $f_\eta$ with respect to $(x,
  y)$. Now it is clear that $\mathscr{Y}_0 \mathscr{F}$ is identity
  since $\mathscr{Y}$ associates the initial condition. The definition
  of $\mathscr{F}$ implies the second condition of the definition of
  the parametrization. On the otherhand the choice of $S$ and $p$
  implies that $H(S) \cong H^{s_p}_p$. This proves the Theorem
  \ref{chap2:sec2.10:thm2}.  
\end{proof of theorem} 

\begin{proposition}\label{chap2:sec2.11:prop17} %\prop 17
  Under the notation of Theorem \ref{chap2:sec2.10:thm1}, let $M$ be the integral of $[
    \sum, (D, D', \tilde{\omega})]$ with initial condition $0$. Then
  the tangent vector space of $M$ at  $z$ is equal to $E^o$.  
\end{proposition}

\begin{proof}
  (by\pageoriginale induction on $q$) Denote by $E$ the tangent vector space. By
  induction assumption applied to $[\sum_1, (D_1 , D'_1,
    \tilde{\omega_1})]$,, we can assume that $L_r(E)= L_r (E^o)$ for
  $r=1, \ldots, q-1$. Since $M$ has initial condition $0$, the
  condition $(iv)$ of the regular coordinate system implies that
  $\langle dy_{t_+ \rho}, L_q(E) \rangle$ $=  \langle dy_{t_+ \rho},
  L_q(E^o) \rangle$ for $\rho =1  , \ldots , m-t$. Since $L_q(E)$
  and $L_q(E^o)$ are solutions  of $J(E_{q-1}, \sum ) =0$, the
  condition (iii) of the regular coordinate system implies that
  $L_q(E)= L_q(E^o)$. Hence $E = E^0$. 
\end{proof}

\noindent 
\textbf{corollary or proposition 17.} 
Let $E$ be an ordinary integral element (with origin $z$) of a
differential system $( \sum )$. Then there is an integral submanifold
$M$ of $(\sum )$ such that the tangent vector space to $M$ at $z$ is
equal to $E$. 
