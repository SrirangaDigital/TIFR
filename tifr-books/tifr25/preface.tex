\thispagestyle{empty}

\begin{center}
  \textbf{\Large Lectures on}\\[5pt]
  \textbf{\Large Exterior Differential Systems}
  \vskip 1cm

  {\bf By}\\
  {\large\bf M. Kuranishi}
  \vfill

  {\bf Tata Institute of Fundamental Research}

  {\bf Bombay}

  {\bf 1962}
\end{center}
\eject

\thispagestyle{empty}


\begin{center}
  \textbf{\Large Lectures on}\\[5pt]
  \textbf{\Large Exterior Differential Systems}
\vskip 1cm

  {\bf By}\\
  {\large\bf M. Kuranishi}
\vfill

{\bf Notes by}\\
{\large\bf  M.K. Venkatesha Murthy}\\
  \vfill

  \begin{tabular}{p{8.5cm}}
    No part of this book may be reproduced in any form by print,
    microfilm or any other means without written permission from the
    Tata Institute of Fundamental Research, Colaba, Bombay 5-BR 
  \end{tabular}

  \vfill
  {\bf Tata Institute of Fundamental Research}

  {\bf Bombay}

  {\bf 1962}
\end{center}




\chapter{Introduction}

To\pageoriginale begin with,  we shall roughly state the main problem that we shall
be considering in the following. Let $D$ denote a domain in the
$n$-dimen\-sional Euclidean space $R^n$ and let $\theta_1,  \ldots,
\theta_m$ be a system of homogeneous differential forms on $D$,  which
we shall denote by $\sum$. We adopt the convention that a function is
a homogeneous differential form of degree zero. A submanifold $M$ of
$D$ is called an integral submanifold or simply an integral of $\sum$
if the restrictions of $\theta_1, \ldots,  \theta_m$ to $M$ vanish. We
will be concerned mainly with the following problem: given a system
$\sum$ of homogeneous differential forms on $D$,  to determine
sufficient conditions for constructing all the integrals of $\sum$,
and to obtain some information regarding the structure of the set of
integrals of $\sum$. We shall discuss such conditions given by
E.Cartan. He called systems satisfying his conditions ``systems in
involution''. We shall also discuss the prolongations of differential
systems,  the main idea of which is also due to him. 

The above mentioned problem is essentially a problem in the theory of
partial differential equations. This fact is made  clear by the
following simple example. 

Let $u(x, y)$ be a function of two independent real variables defined
in a certain domain $D$ in $R^2$ and satisfy the system of\pageoriginale partial
differential equations 
$$
F_\alpha \left(x,  y,  u,  \frac{\partial u}{\partial x},  \frac{\partial
  u}{ \partial y}\right) = 0 \quad (\alpha = 1,  2,  \ldots,  m) 
$$
$u$ may be assumed to be once continuously differentiable. We will
construct,  introducing new variables $p$ and $q$,  a system $\sum$ of
homogeneous differential forms in a suitable domain $D_1$ in $R^5$ of
coordinate system $(x,  y,  u,  p,  q)$ 
$$
\left(\sum\right)
\begin{cases}
  F_\alpha (x,  y,  u,  p,  q), \\
  du - pdx - qdy
\end{cases}
$$

Let $M^2$ be a two dimensional submanifold of $D_1$ expressed
parametrically by $(x,  y,  u(x,  y),  p(x,  y),  q(x,  y))$. It can
be easily seen that $M^2$ is an integral of the system $\sum$ if and
only if $u(x,  y)$ is a solution of the system of differential
equations 
$$
F_\alpha \left(x,  y,  u,  \frac{\partial u}{\partial x},  \frac{\partial
  u}{\partial y}\right) = 0 \quad (\alpha = 1,  \ldots,  m) 
$$
together with $p(x, y) = \dfrac{\partial u(x, y)}{\partial x},  q(x,
y) = \frac{\partial u(x, y)}{\partial y}$ . 

However,  it seems that,  in our approach,  it is convenient to handle
the system of homogeneous differential forms rather than solving the
system of partial differential equations. Moreover,  sometimes our
approach is quite useful for certain geometric problems also. 

We shall restrict our attention only to the case of systems of real
analytic differential forms. The extension of our results\pageoriginale to the case
of $C^\infty$ forms (differentiable case) appears to be very much more
complicated and remains unsolved. we shall also confine ourselves to
the so called local problem. 
