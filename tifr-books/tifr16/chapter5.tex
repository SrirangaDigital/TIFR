\chapter{Inequalities, etc.}\label{chap5}

\setcounter{section}{11}
\section{Bernstein's and Markoff's Inequalities}\label{chap5:sec12}%sec12

\begin{theorem}[Bernstein]\label{chap5:sec12:thm21}%them 21
  If\pageoriginale $t(x) = \dfrac{1}{2} a_o + \sum^n_1 (a_k \cos kx +
  b_k \sin kx)$ then $\qquad |t'(x) | \leq n \sup |t(x)|$. 
\end{theorem}

\begin{proof}
  Suppose, on the contrary, that 
  $$
  \sup|t'(x)| =nl,
  $$
  where \hspace{3cm} $l > \sup|t(x)|$.
\end{proof}

$t'(x)$, being continuous, attains its bounds and so, for some
$c,t'(a) = \pm nl$ and we will suppose that  
$$
t'(c) =nl.
$$

Since $nl$ is a maximum value of $t' (x)$,
$$
\displaylines{\hfill   t''(c) =0.\hfill \cr
  ~\text{Define}\hfill  S(x) = l \sin n(x-c) -t(x).\hfill \cr
  ~\text{Then}~ r(x) =S' (x) =nl \cos n(x-c) -t'(x).\hfill} 
$$
$S(x)$ ~ and ~$r(x)$ ~ both have order ~ $n$.

Consider the points 
$$
u_o = c+ \pi / 2n,  u_k = u_o + k \pi/n (1 \leq k \leq 2n).
$$
Then \begin{align*}
  S(u_o) &= 1-t(u_o) > 0\\
  S(u_1) &= 1-t(u_1) < 0\\
  & \cdots \cdots\\
  S(u_{2n}) &= 1 - t(u_{2n}) > 0
\end{align*}

\noindent
Each of the $2n$ intervals $(u_o,u_1),(u_1,u_2), \ldots, (u_{2n-1},
u_{2n})$ then contains a zero of $S(x)$.say 
$$
S(y_i) =0,
$$
where\pageoriginale $u_i < y_i < u_{i+1}, (0 \leq i \leq 2n-1)$. Clearly
$$
\displaylines{\hfill y_{2n-1} < y_o + 2 \pi.\hfill \cr
  ~\text{Write} \hfill  y_{2n}  = y_o + 2 \pi.\hfill \cr
  ~\text{Then}\hfill S(y_{2n})  = S(y_o) = 0.\hfill}
$$

By Rolle's Theorem, there is a zero $x_i$ of $r(x)$ inside each
interval $(y_i, y_{i+1})$ where $0 \leq i \leq 2n-1$. Clearly  
$$
x_{2n-1} < x_o + 2 \pi.
$$
Now $\qquad r(c) = nl- t'(c) =0$.

Since the polynomial $r(x)$ of order $n$ has at most $2n$ zeros, it
follows that, for some $k$, 
$$
c \equiv x_k \pmod {2 \pi}.
$$

But$\qquad r'(c) = -t''(c) = 0$.

Therefore $c$(and so $x_k$) is a double zero (at least) of $r(x)$.

Therefore the $x_i(0 \leq i \leq 2n-1)$ provide at least $2n+1$ zeros
of $r(x)$. This is only possible if $r(x) \equiv 0$, and so $S(x)$ is a
constant. But $S(u_o)>0$ and $S(u_1) < 0$ and we have a contradiction 

\setcounter{corollary}{0}
\begin{corollary}\label{chap5:sec12:coro1} %corollary 1
  $t(x) = \sin nx$ shows that the result is the best possible.
\end{corollary}

\begin{corollary}\label{chap5:sec12:coro2} %corollary 2
  The algebraic equivalent is- If $p(x)$ has degree $n$ and $|p(x) |
  \leq M$ in $(-1,1)$, then  
  $$
  |p'(x)| \leq n M \sqrt{(1-x^2)}. 
  $$
\end{corollary}

\begin{proof}Put
  \begin{align*}
    t(\theta)& = p (\cos \theta)\\
    t'(\theta) & =-p'(\cos \theta) \sin (\theta). 
  \end{align*}
  The\pageoriginale bound for $|p'(x)|$ given in
  Corollary \ref{chap5:sec12:coro2}  fails at the
  end-points $\pm 1$. A better result, due to Markoff, is  
  $$
  |p'(x)| \leq M n^2,
  $$
  as will be proved in Theorem \ref{chap5:sec12:thm22}.
\end{proof}

\setcounter{lem}{0}
\begin{lem}\label{chap5:sec12:lem1} %lemma 1
  Let
  $$
  x_k = \cos \frac{(2k-1) \pi}{2n} \qquad (k=1,2, \ldots, n)
  $$
  be the zeros of the Chebyschev polynomial $T_n(x)$. If $q(x)$ is in
  $P_{n-1}$, then 
  $$
  q(x)= \frac{1}{n} \sum^n_{k=1}(-1)^{k-1} \sqrt{(1-x^2_k)}
  q(x_k)\cdot \frac{T_n(x)}{x-x_k}. 
  $$
\end{lem}

\begin{proof}
  Both sides are in $P_{n-1}$ and so it is sufficient to show that
  they agree for the $n$ values $x_k$. As $x \to x_k$, 
  \begin{align*}
    \frac{T_n(x)}{x-x_k} \to T'_n(x_k) & = \frac{n}{\sqrt{(1-x^2_k)}}
    \sin(n ~\text{arc}~ \cos x_k)\\ 
    & = \frac{n(-1)^{k-1}}{\sqrt{(1-x^2_k)}}
  \end{align*}
  
  \noindent
  Also, for $x = x_k$, every term on the R.H.S. except the k-th vanishes.
\end{proof}

\begin{lem}\label{chap5:sec12:lem2} %lemma 2
  Suppose that $q(x)$ is in $P_{n-1}$ and $|q(x)| \leq
  \dfrac{1}{\sqrt{(1-x^2)}}(-1<x<1)$. 
\end{lem}
Then $|q(x)| \leq n \,(-1 \leq x \leq 1)$.

\begin{proof}
  With the notation of Lemma \ref{chap5:sec12:lem1}, if $-x_1=x_n \leq x \leq x_1$,
  $$
  \sqrt{(1 - x^2)} \geq \sqrt{(1-x^2_1)} = \sin \frac{\pi}{2n} \geq \frac{1}{n}.
  $$

  Therefore Lemma \ref{chap5:sec12:lem2} is true for $x_n \leq  x \leq
  x_1$. If $x_1 < x \leq 1$ (or $-1 \leq x < x_n)$
  Lemma \ref{chap5:sec12:lem1}  gives  
  $$
  |q(x) | \leq \frac{1}{n} | \sum \frac{T_n(x)}{x-x_k}|,
  $$
  since\pageoriginale all the $x-x_k$ are positive (or all negative). Now 
  $$
  \displaylines{\hfill 
    T_n(x) = 2^{n-1} \prod (x-x_k),\hfill \cr
    ~\text{and so}\hfill  
    \frac{T'_n(x)}{T_n(x)} = \sum \frac{1}{x-x_k}.\hspace{2cm}\hfill}
  $$
\end{proof}

Therefore $\qquad |q(x)| \leq \dfrac{1}{n} |T'_n (x)|$.

But, if $x = \cos \theta, T'_n(x) = \dfrac{n \sin n \theta}{\sin
  \theta}$, which gives  
$$
|T'_n(x) | \leq n^2. 
$$

\begin{theorem}[Markoff]\label{chap5:sec12:thm22} %theorem 22
  If $p(x)$ is in $P_n$, then
  $$
  |p'(x)| \leq n^2 \sup |p(x)| \qquad -1 \leq x \leq 1.
  $$
\end{theorem}

\begin{proof}
  If $\sup |p(x)|=M$, take in Lemma \ref{chap5:sec12:lem2},
  $$
  q(x) = \frac{p'(x)}{M_n}.
  $$
\end{proof}

\begin{coro*}
  $p(x) = T_n (x)$ shows that the result is the best possible.
\end{coro*}

\section[Structural Properties Depend on the...]{Structural Properties Depend on the closeness of the
  approximation}\label{chap5:sec13} 

Theorem \ref{chap5:sec12:thm21} can be used to prove theorems of a
type converse to Theorems 18-20. Theorem \ref{chap5:sec13:thm23},
which is complementary to Theorem \ref{chap4:sec11:thm18}, Corollary
\ref{chap4:sec10:coro2}, will suffice to show the method.  

\begin{theorem}\label{chap5:sec13:thm23}  %theorem 23
  Let $f(x)$ be $C(2 \pi)$. Suppose that, for all $n$, the best
  approximation in $T_n$ to $f(x)$ is less than $A/n^{\alpha}$, where
  $0< \alpha <1$. Then $f(x)$ is Lip.$\alpha$. 
\end{theorem}

\begin{proof}
  Let $t_n(x)$ satisfy
  $$
  \displaylines{\hfill 
    |f(x) -t_n (x)| \leq \frac{A}{n^{\alpha}}.\hfill \cr
    \text{Define}\hfill u_0(x) = t_1(x)\qquad \hfill \cr
    \hfill a_n(x) t_{2^n}(x) - t_{2^{n-1}} (x) \qquad(n \geq 1).\hfill}
  $$
\end{proof}

  Then\pageoriginale $f(x)$ is the sum of the uniformly convergent series $\sum
\limits^{\infty}_0 u_n (x)$. Choose $\delta$ with $0 < \delta \leq
\dfrac{1}{2}$, and $m$ such that   
$$ 
2^{m-1} \leq \frac{1}{\delta} < 2^m.
$$

Suppose $|x-y| \leq \delta$. We have 
$$
|f(x) - f(y)| \leq \sum^{m-1}_0 | u_n(x) - u_n(y)| +
\sum^{\infty}_m|u_n(x)| + \sum^{\infty}_m|u_n(y)|. 
$$

We shall find upper bounds for the terms on the R.H.S.
\begin{align*}
  |u_n(x)| & \leq |t_{2^n}(x) - f(x) | + | f(x) - t_{2^{n -1}}(x)|\\
  & \leq \frac{A}{2^{n \alpha}} + \frac{A}{2^{(n-1)}\alpha} =
  \frac{A(1+2^{\alpha})}{2^{n \alpha}}. 
\end{align*}

Therefore 
$$
\sum^{\infty}_m |u_n(x)| \leq A(1+2^{\alpha}) \sum^{\infty}_m
\dfrac{1}{2^{n \alpha}}= \frac{A(1+2^{\alpha})}{1-2^{- \alpha}}
\frac{1}{2^{m \alpha}}. 
$$

This gives  
$$
|f(x) -f(y)| \leq \sum^{m-1}_o|u_n(x) - u_n (y)| + \frac{B}{2^{m
    \alpha}}. 
$$
Theorem \ref{chap5:sec12:thm21} applied to $u_n(x)$ gives 
$$
|u'_n(x) | \leq 2^n \sup |u_n(x)| \leq A(1+2^{\alpha}) 2^{n(1-\alpha)}.
$$

By the mean-value theorem, 
$$
|u_n(x) - u_n (y) | \le |u'_n (\xi) || x-y| \le
A(1+2^\alpha)2^{n(1-\alpha)} \delta 
$$

Therefore 
$$
|f(x) -f(y) | \leq A(1+2^{\alpha}) \delta \sum^{m-1}_o 2^{n(1-
  \alpha)} + \frac{B}{2^{m \alpha}}. 
$$

Putting $C=A(1 + 2^{\alpha})$ and using $\dfrac{1}{2^m} < \delta$, we have 
$$
\omega(\delta) \leq C \delta \sum^{m-1}_o 2^{n (1- \alpha)} + B
\delta^{\alpha}. 
$$
If\pageoriginale now $\alpha < 1$,
$$
\sum^{m-1}_o 2^{n (1- \alpha)} =
\frac{2^{m(1-\alpha)}-1}{2^{1-\alpha}-1} <
\frac{2^{m(1-\alpha)}}{2^{1-\alpha}-1}. 
$$

Use now $2^m \leq \dfrac{2}{\delta}$ and we find
$$
\omega (\delta) < \left(\dfrac{2^{1-\alpha}}{2^{1-\alpha}-1} C+B
\right) \delta^{\alpha} 
$$
See Notes 1-4 at the end of the Chapter.

\section{Divergence of the Lagrange Sequence}\label{chap5:sec14}

There is a sense in which the Lagrange polynomial of degree $n$ (\S
\ref{chap2:sec4}) 
fitted to a function $f(x)$ at $n+1$ points equally spaced through an
interval follows the function closely. It is natural to expect that,
by increasing $n$, the approximation would improve and we might, for
instance, find another proof of Theorem \ref{chap1:sec1:pofthm1} on these lines. Such
expectations are falsified. Unless heavy restrictions are laid on
$f(x)$, the sequence of Lagrange polynomials diverges except for
certain special values of $x$.  

We shall construct an example of this phenomenon.

\begin{lemma*}
  Let $p(x)$ be the Lagrange polynomial which takes the value $0$ at
  the $2m$ values of $x$ 
  $$
  k/m \qquad (-m \leq k \leq m; k \neq 2)
  $$
  and takes the value $1/m$ when $x = 2/m$. Then, if $m$ is odd,
  $|p \left(\dfrac{1}{2}\right)| \to \infty$ as $m \to \infty$. 
\end{lemma*} 

\begin{proof}
  The polynomial $p(x)$, of degree $2m$, is 
  $$
  \frac{1}{m} \frac{(x+1) (x+ \frac{m-1}{m}) \ldots x (x-1/m) (x-3/m)
    \ldots (x-1)}{(2/m+1) (2/m + \frac{m-1}{m}) \ldots 2/m(2/m-1/m)
    (2/m-3/m) \ldots (2/m-1)} 
  $$
  This\pageoriginale gives for $|p\left(\dfrac{1}{2}\right)|$
  $$
  \frac{1}{m} \frac{3m(3m-2)\ldots m(m-2) (m-6) \ldots 1.1.3.\ldots
    m}{2^{2m} (m+1) (m+1) \ldots2. 1.1.2 \ldots \qquad (m-2)} 
  $$
\end{proof}

This can be estimated by forming it into factorials and using
Stirbing's theorem. More simply, we can prove that it tends to
$\infty$ by grouping the factors as follows: 
\begin{align*}
  |p\left(\frac{1}{2}\right)| & = \frac{m-1}{2(m+1) (m+2) (m-4)} A^2 B
  C,\hspace{2cm}\\ 
  ~\text{where} \hspace{2.5cm} A& = \frac{3.54 \ldots m}{2.4 \ldots (m-1)}\\
  B & = \frac{(m+2)(m+4) \ldots (2m +1)}{(m+1) (m+3) \ldots 2m}\\
  C& = \left(- \frac{2m+3}{m+1}\right) \left(\frac{3m+5}{m+3}\right)
  \ldots \left(\frac{3m}{2m-2}\right). 
\end{align*}

Here $A>1,B>1$, and the factors of $C$ decrease from left to right,
the last being greater than $3/2$. So $C> (3/2)^{m-1}$.  

\noindent
\textbf{Note}.$x = \dfrac{1}{2}$ has been taken for ease of
calculation. The conclusion holds for other values of $x$. 

\begin{theorem}[Borel]\label{chap5:sec14:thm24} %theorem 24
  There is $f(x)$ in $C(-1,1)$ whose nth Lagrange polynomial
  does not converge to $f(x) $ as $n \to \infty$. 
\end{theorem}

\begin{proof}
  Define a continuous curve $C_k$ which coincides with Ox outside the
  interval $(3^{-k-1}, 3^{-k})$ and has maximum $3^{-k-1}$ at the
  midpoint of that interval. For example we can define $C_k$ by  
  $$
  y = 3^{-k-1} \sin \left\{(3^{k+1} x-1) \pi/2 \right\}.
  $$
\end{proof}

We shall use the $C_k$ to construct a curve $S. P_{k,S}(x)$ will
denote the Lagrange polynomial which takes the same values as $S$ for
the values $x = 1/3^k$, where $-3^k \leq 1$ (integer) $\leq{3^k}$. 

We\pageoriginale shall construct $S$ so that $P_{k,S} \left(\dfrac{1}{2}\right)$ does not
converge to the point on $S$ where $x = \dfrac{1}{2}$. Observe first
that $P_{k,C_{k-1}}$ is the Lagrange polynomial in the Lemma with $m =
3^k$. From the Lemma, given A, there is $h_1$ such that  
$$
|P_{k,C_{k-1}}\left(\frac{1}{2}\right)| > 2 A ~\text{if}~ k-1 > h_1.
$$

There are two possibilities:
\begin{enumerate}[(a)]
\item With $h_1$ fixed, $P_{k,C_{h1}}\left(\dfrac{1}{2}\right)$ does not tend to
  $0$ as $k \to \infty$. Then $S$ can be taken to be $C_{h1}$; or  
\item there exists $r$ such that 
  $$
  |P_{k,C_{h1}}\left(\frac{1}{2}\right)| < \frac{1}{2} A ~\text{for all}~ k> r_1.
  $$
\end{enumerate}
Choose $h_2 > \max(h_1,r_1)$.

Define $D_{2,k}$ to be the sine-curves in $C_{h1}$ and $C_{k-1} (k-1
\geq h_2)$, and, for the rest, the x-axis in $(-1,1)$. 

$D_{2,k}$ is a continuous curve; its ordinate for $x = \dfrac{1}{2}$ is
$0$, and  
$$
P_{k,D_{2,k}} = P_{k,C_{h_1}} + P_{k,C_{k-1}}.
$$
From above, since $k-1 \geq h_2$,
$$
|P_{k,D_{2,k}}\left(\frac{1}{2}\right)| > 2 A - \frac{1}{2}A.
$$

Again, there are two possibilities:
\begin{enumerate}[(a)]
\item With $h_2$ fixed,$P_{k,D_{2,k}}\left(\dfrac{1}{2}\right)$ does not tend to
  0 as $k \to \infty$. Then $S$ can be taken to be $D_{2, h_2}$; or 
\item there exists $r_2$ such that 
  $$
  |P_{k,D_{2,k_{h_2}}}\left(\frac{1}{2}\right)| < \frac{1}{4}\cdot A ~\text{for all}~ k>r_2.
  $$
\end{enumerate}

Choose\pageoriginale $h_3 > \max (h_2,r_2)$.

Define $D_{3,k}$ to be the sine-curves in $C_{h_{1}},C_{h_{2}}$ and
$C_{k-1}(k-1 \geq h_3)$ and, for the rest, the -axis in
$(-1,1)$. After $n$ repetitions, there are two possibilities: 
\begin{enumerate}[(a)]
\item There is a $D_{n, h_n}$ for which the $k$ th Lagrange polynomial
  does not tend to $0$ at $x = \dfrac{1}{2}$; and this serves for $S$;
  or  
\item there is an infinite sequence $D_{n, h_n}$  for which 
  $$
  |P_{h_n+1,D_{n, h_n}} \left(\frac{1}{2}\right)| > 2 A - \frac{1}{2} A -
  \frac{1}{4}A- \ldots - \frac{A}{2^{n-1}}>A. 
  $$
\end{enumerate}

As $n \to \infty,  D_{n, h_n}$ defines a continuous curve $S$ whose
ordinate for $x = \dfrac{1}{2}$ is $0$. Its Lagrange polynomial takes
values greater than $A$ for $x = \dfrac{1}{2}$ when its degree is
$h_1, h_2, \ldots,  h_n, \ldots$.  

\begin{center}
  \textbf{Notes}\pageoriginale
\end{center}

\begin{enumerate}
\item Weierstrass's function $\sum a^r$ cos $b^r x$ illustrates
  Theorem \ref{chap4:sec11:thm18} (Corollary \ref{chap4:sec11:coro2})
  and Theorem \ref{chap5:sec13:thm23}. See Chapter IV,  note 4. 
\item If $\alpha =1$, the best that can be proved in
  Theorem \ref{chap5:sec13:thm23} is
  that $\omega (\delta) < A \delta \log (1/ \delta)$. The latter part
  of the argument can be adapted for this purpose (Natanson, 91). 
  
  The function $\sum_1^\infty \frac{\sin nx}{n^2}$ satisfies $d_n <
  \dfrac{1}{n}$, but is not in Lip.$1$ (Natanson, $93$). 
\item A condition which is necessary and sufficient of $d_n <  A/n$ is that 
  $$
  |f(x+h)- 2f (x) + f(x-h) | <   Bh.
  $$
  (Zymund, Duke Mathematical Journal, $12(1945) 47$ or Natanson, $96$).
\item (Extension of Theorem \ref{chap5:sec13:thm23}). If, for $f(x)$,
  $d_n <A/n^{p+ 
  \alpha}$  ($p=\break \text{integer},  0 <  \alpha < 1$),  then $f(x)$ has a
  pth derivative $f^{(p)}(x)$ in Lip $\alpha$. 
\item For further `negative results' like Theorem \ref{chap5:sec14:thm24}, see Natanson,
  369 -- 388. For example, the Lagrange polynomial taking the values
  of $| x|$ at $n$ equally spaced points in $(-1,1)$ 
  converges to $|x|$ as $n \to \infty$ for no value of $x$ expect $0, \pm 1$.
\end{enumerate}
