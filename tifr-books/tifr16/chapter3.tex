\chapter{Approximations to \texorpdfstring{$|x|$}{|x|}}\label{chap3}

\setcounter{section}{6}
\section{}\label{chap3:sec7}

We\pageoriginale now take up the central problem of polynomial
  approximation,\break namely 

Given a function $f(x)$, how high is the degree of the polynomial
necessary to approach it with an assigned accuracy? 

The answer may well depend on structural properties of $f(x)$. For
instance, we may guess (rightly) that we can predict a lower degree if
$f(x)$ is assumed to be differentiable instead of only continuous. The
best theorems on these matter lie fairly deep. We shall go through
some heuristic motion of finding from particular cases what truth
appears to be and then deciding how to try to establish it. 

A useful function to study with care is $|x|$ in $(-1,1)$. This
function was the basis of Lebesgue's proof of Theorem \ref{chap1:sec1:pofthm1}. 

From Exercises 1, 2 at the end of Chapter $I$, the deviation from
$|x|$ of a polynomial of degree $n$ of any of the three types used in
proving Theorem \ref{chap1:sec1:thm1} is of order $1/ \sqrt{n}$. 

Let us clarify our mode of speech. If, for some $p(x)$ in $P_n$,
$$
|f(x)-p(x)|=0 \{\varphi (n)\}
$$
we will say that the approximation is $0\{\varphi (n) \}$. If,
moreover, there is no $p(x)$ in $P_n$ for which 
$$
|f(x)-p(x)|=0 \{\varphi (n)\}
$$
we will say that the approximation is \textit{actually} $0 \{\varphi
(n)\}$. 

Study of the proofs of Theorem \ref{chap1:sec1:pofthm1} might lead us to conjecture that
the best approximation in $p_n$ to $|x|$ is actually $0(1/ \surd n)$. 

We\pageoriginale proceed to show that, in fact, it is actually $0(1/n)$. This will
be proved, following Bernstein, Lecons Ch.I, by elementary (though
rather lengthy) algebra. 

To approximation to $|x|$ in $(-1,1)$ is the same thing as to
approximate to $x$ in $(0,1)$ by polynomials whose exponents are all
even, and this is what we shall do. 

If $d_{2n}$ is the best approximation to $x$ in $(0,1)$ by 
$$
a_ 0 +a_1 x^2+ \cdots + a_n x^{2n}
$$
we shall prove that
$$
\frac{1}{2n+1} > d_{2n}> \frac{1}{4(1+ \sqrt{2})}\cdot \frac{1}{2n-1}.
$$

Bernstein went further and proved that $d_{2n} \sim C/n$, where $C$ is
a constant which he evaluated as $0.282 \pm 0.004$. 

The theorems of this Chapter will not be used later in the course, and
any one who wishes may note above inequalities for $d_{2n}$ and pass
on to Chapter IV. 

\section{Oscillating polynomials}\label{chap3:sec8}

\begin{defi*}
  If $0 \leq \alpha_ 0 < \alpha_1 \cdots < \alpha_n$ and $A_i \neq 0$
  (all $i$), we say that 
  $$
  p(x)=A_ 0 x^{\alpha _0}+A_1 x^{\alpha_ 1}+ \cdots + A_n x^{\alpha _ n}
  $$
  is an \textit{oscillating polynomial} in $(0,1)$ if $\sup |p(x)|$ is
  attained for $n+1$ values of $x$ in $0 \leq x \leq 1$. We shall
  suppose the $\alpha's$ integers. 

\noindent\textbf{Illustrations.} 

  \begin{enumerate}[\quad \rm (1)]
    \item $\alpha _i =2i+1$  \quad $T_{2n+1}(x)$ \quad satisfies
    \item $\alpha_i = i$  \qquad $T_{2n}(\sqrt{x})$.
  \end{enumerate}
\end{defi*}

\setcounter{lem}{0}
\begin{lem}\label{chap3:sec8:lem1}%lemma 1
  The\pageoriginale polynomial $p(x)$ in the definition has at most $n$ positive
  zeros. If it has $n$ the coefficients alternate in sign. 
\end{lem}

From Descrates' rule fo signs.
\begin{lem}\label{chap3:sec8:lem2}%lemma 2
  The coefficients of an oscillating polynomial alternate in sign.
\end{lem}

\begin{proof}
  \begin{enumerate}[(1)]
  \item Let $\alpha_ 0=0$ (and $A_ 0\neq 0$). There are at most $n-1$
    changes of sign in the coefficients of $p'(x)$. Therefore $p'(x)$
    has at most $n-1$ positive zeros. The $n+1$ values of $x$ at which
    $\sup |p(x)|$ is attained must be $n-1$ zeros of $p'(x)$ and
    $x=0,x=1$. So $p'(x)$ has $n-1$ zeros, say $x_1,x_2, \ldots
   , x_{n-1}$ lying inside $(0,1)$ and its coefficients $A_1, \ldots
   , A_{n-1}$ alternate in sign. 
    
    $p(x)$ has no maxima or minima other than these $n-1$. Therefore
    $$
    p(0),p(x_1), \ldots, p(x_{n-1}),p(1)
    $$
    alternate in sign. Therefore $p(x)$ has $n$ zeros. Therefore $A_
    0, A_1, \ldots,\break A_n$ alternate in sign 
  \item Let $\alpha _ 0 >0$. Then $p(0)=0$. So $\sup |p(x)|$ is
    attained at $n$ points inside $(0,1)$ which are roots of
    $p'(x)=0$. Therefore the coefficients alternate in sign. 
  \end{enumerate}
\end{proof}

\begin{coro*}
  $p(x)$ takes the values $\pm \sup |p(x)|$ with $+$ and $-$ sign
  alternately. 
\end{coro*}

\begin{theorem}\label{chap3:sec8:thm10}%theorem 10
  $p(x)=\sum \limits^n_{i= 0}A_i x ^{\alpha _i}$ is an
    oscillating polynomial in $(0,1).q(x)$ is another polynomial
    $\sum B_i x^{\alpha_i}$ with the same exponents. One coefficient
    of $p$ is the same as the corresponding one of $q$ (say
    $A_j=B_j$), where $\alpha_j >0$. Then 
    $$
    \sup |q|> \sup |p|.
    $$
\end{theorem}

\begin{proof}
  If\pageoriginale not, $p-q$ takes alternate signs (may be $0$) for the $n+1$
  values of $x$ for which $p$ takes its numerically greatest
  value. Therefore $p-q$ has at least $n$ zeros in $0 \leq x \leq
  1$. But, since $A_j=B_j$ it has only $n$ terms, and so at most $n-1$
  changes of sign in its coefficients and so (by Lemma \ref{chap3:sec8:lem1}) at most
  $n-1$ positive zeros. This is a contradiction. 
\end{proof}

\noindent
\textbf{Converse of Theorem 10}. $p(x)$ and $q(x)$ are two
polynomials with the same exponents and one coefficient the same
$(A_j=B_j$, where $\alpha _j >0$). If  
$$
\sup |p|< \sup |q|
$$
for every such $q$, then $p$ is an oscillating polynomial.

\begin{proof}
  We gives the gist of the proof, without setting out all the detail
  in full. It uses a 'deformation' argument like that of theorem
  \ref{chap2:sec5:thm5}.   
\end{proof}

Suppose that $p(x)$ is not an oscillating polynomial. Then $p(x)$
takes the values $\pm M$, where $M=\sup |p(x)|$, at $h$ points, say
$x_k(k=1,\ldots, h)$, where $h< n+1$. We can construct a polynomial
$r(x)= \sum C_i x^{\alpha_i}$ with $C_j=0$ and $r(x_k)=p(x_k)$. (The
$n+1$ coefficients $C_i$ have to satisfy at most $n+1$ equations; the
determinant can be proved $\neq 0$). 

We can take $\varepsilon$ and intervals round the $x_k$, outside which
$|p(x)|<M-\varepsilon$ and inside each of which $p(x)$ and $r(x)$ have
the same sign. 
\begin{gather*}
  ~\text{Choose}~ \lambda ~\text{to make}~ \lambda |r(x)|<
  \varepsilon ~\text{for}~ 0 \leq x \leq 1.\\ 
  ~\text{Then} \qquad \sup |p-\lambda r|< \sup |p|.
\end{gather*}

But $p-\lambda r$ satisfies the conditions for a $q$, giving a contradiction.

Apply\pageoriginale theorem \ref{chap3:sec8:thm10}, taking $p(x)$ to be a constant multiple of one of
the oscillating polynomials $T_{2n}(\sqrt{x})$ and $T_{2n+1}(x)$. We
obtain  

\begin{corollary}\label{chap3:sec8:coro1} %corollary 1
  If $q(x)=a_ 0 +a_1x + \cdots a_n x^n$ and $M=\sup |q(x)|$ in $(0,1)$, then
  $$
  |a_i|\leq M|t_i ~|(i=0,1, \ldots, n),
  $$
  where\pageoriginale $t_i$ is the coefficient of $x^i$ in $T_{2n}(\surd x)$.
\end{corollary}

\begin{corollary}\label{chap3:sec8:coro2} %corollary 2
  If $q(x)=a_ 0 x +a_1 x^3+ \cdots a_n x^{2n+1}$ and $M= \sup |q(x)|$
  in $(0,1)$, then 
  $$
  |a_i|\leq M|t_i|(i=0, \ldots,n)
  $$
  where $t_i$ is the coefficient of $x^{2i+1}$ in $T_{2n+1}(x)$.
\end{corollary}

\begin{theorem}\label{chap3:sec8:thm11} %theorem 11
  To a given set of exponents there corresponds an oscillating
  polynomial in $(0,1)$, which is unique except for a constant factor.
\end{theorem}

\begin{proof}
  Let $\alpha _ 0, \alpha_1, \ldots, \alpha _n$ be the given exponents
  in ascending order. Suppose that the coefficient of $x^{\alpha_k}$
  is given to be $K$. 
\end{proof}

We need to prove that among all the polynomials with the given exponents
$$
q(x)=B_ 0 x^{\alpha _ 0}+ \cdots + B_{k-1}x^{\alpha_{k-1}}+k
x^{\alpha_k}+ \cdots + B_n x^{\alpha_n}, 
$$
there is a unique $q(x)$ for which $\sup \limits _{0 \leq x \leq 1}
|q(x)|$ attains its lower bound. 

Clearly $\sup |q(x)|$ is a continuous function of the $n$ variables
($B_ 0 \cdots,$ $B_{k-1},B_{k+1},\ldots, B_n$). Its lower bound is
greater than $0$ by Corollary \ref{chap3:sec8:coro1} of
Theorem \ref{chap3:sec8:thm10}. It is less than or
equal to $K$, as is seen by taking the $B's$ to be small. Again by
Corollary \ref{chap3:sec8:coro1}, we need only consider values of $B_i$ for which 
$$
|B_i|\leq K|t_i|\quad (i=0,1, \ldots, k-1,k+1,\ldots, n),
$$
where $t_i$ is the coefficient of $x^{\alpha_i}$ in $T_{2 \alpha_n}(\sqrt{x})$. 

The $B_i$ lie in a bounded closed region of $n$ space, and so they
have at least one set of values for which $\sup |q(x)|$ attains its
lower bound. This proves the existence of an oscillating
polynomial. Uniqueness follows from Theorem \ref{chap3:sec8:thm10}. 

\begin{theorem}\label{chap3:sec8:thm12}%theorem 12
   If
   \begin{align*}
     p(x) & = x^{\alpha _ 0}+A_1x^{\alpha_1}_{\beta_1}+ \cdots +A_n
     x^{\alpha _n}_{\beta _n}\\ 
     \text{and}\qquad q(x)&= x^{\alpha_0}+B_1
     x^{\beta_1}+\cdots +B_n x^{\beta_n}
   \end{align*}
   are both oscillating polynomials in $(0,1)$ where
   $$
   0<\alpha_0<\beta_1< \alpha_1 < \beta_2 < \cdots <\beta_n <\alpha_n,
   $$
   $$
   ~\text{then} \qquad \sup|p(x)|>\sup|q(x)|.
   $$
\end{theorem}

\begin{proof}
  By Lemma \ref{chap3:sec8:lem2}, the coefficients fo  of $p(x)$ alternate in sign and
  so do those of $q(x)$.  
  $$
  q(x)-p(x)=B_1x^{\beta_1}-A_1x^{\alpha_1}+B_2 x^{\beta_2}-\cdots -A x_n^{\alpha_n}
  $$
  has $n$ variation of sign, and so the equation
  $$
  q(x)-p(x)=0
  $$
  has at most $n$ positive roots.
\end{proof}

Suppose the theorem false and
$$
\sup |p(x)|\leq \sup|q(x)|.
$$

Then $q(x)-p(x)$ has the sign of $q(x)$ (it may be $0$) for the values
$x_n(k=1,\ldots, n+1)$ at which $|q(x)|$ takes its maximum
value. Therefore $q(x)-p(x)$ vanishes for $n$ values $\xi_1, \ldots, 
\xi_n$ such that  
$$
x_i\leq \xi_1 \leq x_2 \leq \xi_2 \leq \cdots \leq \xi_n \leq x_{n+1}.
$$

Moreover,\pageoriginale there are $n+1x's$ and only $n \xi 's$ so at least one $x$,
say $x_i$ must satisfy $\xi_{i-1}<x_i<\xi_i$ (giving meaning to $\xi_
0,\xi_{n+1}$). 

We shall now compute the sign of $q(x_i)$ by two different methods and
obtain contradictory results. 

Firstly, in $(0, \xi_1), q(x) -p(x)$ has the sign of its dominant term
$B_1x^{\beta_1}$, which is negative. By following the changes of sign
along the sequence, $q(x)-p(x)$ has sign $(-1)^i$ in
$(\xi_{i-1},\xi_i)$. At $x_i, q(x)-p(x)$ and also $q(x)$ have the sign
$(-1)^i$. 

Secondly, for small values of $x, q(x)$ had the sign of its first
term, which is positive. Therefore $q(x_1)>0$. So $q(x_2)<0$, and
generally, $q(x_i)$ has the sign $(-1)^{i+1}$. 

This is a contradiction.

The same arguments can be used to prove
\begin{theorem*}[12 (Extension)]%theorem 12
  If
  \begin{align*}
    p(x)&=A_ 0 x^{\alpha_ 0}+\cdots +
    A_{i-1}x^{\alpha_{i-1}}+x^m+A_{i+1}x^{\alpha_{i+1}}+ \cdots
    +A_nx^{\alpha_n}\\ 
    q(x)&= B_ 0 x^{\beta_ 0}+\cdots +
    B_{i-1}x^{\beta_{i-1}}+x^m+B_{i+1}x^{\beta_{i+1}}+ \cdots
    +B_nx^{\beta_n} 
  \end{align*}
  are both oscillating polynomials in $(0,1)$, where
  $$
  0 \leq \alpha_ 0 < \beta_ 0< \cdots < \alpha_{i-1}< \beta_{i-1}<m<
  \beta_{i+1}< \alpha_{i+1}\cdots <\beta_n <\alpha_n, 
  $$
  then
  $$
  \sup |p(x)|>\sup |q(x)|.
  $$
\end{theorem*}

\section{Approximation to \texorpdfstring{$|x|$}{|x|}}\label{chap3:sec9}

\begin{theorem}\label{chap3:sec9:thm13}%theorem 13
  If
  $$
  p(x)=x+a_1 x^2+a_2 x^4+\cdots + a_n x^{2n}
  $$
  is an oscillating polynomial in $(0,1)$, then
  $$
  \frac{1}{2n+1} >\sup|p(x)|> \frac{1}{2(1+\sqrt{2})(2n-1)} (n>1).
  $$
\end{theorem}

\begin{note*} 
  If\pageoriginale $n=1$, the second inequality
  is to be replaced by equality. The oscillating polynomial is
  $x-\left(\dfrac{1}{2}+\dfrac{1}{\sqrt{2}}\right)x^2$. 
\end{note*}

\begin{proof}
  Take $n>1$. By Theorem \ref{chap3:sec8:thm12}, $\sup |p(x)|$ is less than the supremum
  of the oscillating polynomial 
  $$
  x+b_1 x^3 +\cdots
  $$
  with exponents $1,3,5,\ldots 2n+1$. But that polynomial is $(-1)^n
  T_{2n+1}(x)/$ $(2n+1)$. This gives the first inequality of the theorem.  
\end{proof}

By Theorems \ref{chap3:sec8:thm12} and \ref{chap3:sec8:thm10}, the oscillating polynomial $x+b_1x^3 +\cdots
+ b_{n-1}x^{2n-1}$ has smaller maximum modulus than the polynomial $x+
c_2 x^4 +\cdots +c_rx^{2n}$ with exponents $1,4,6,\ldots, 2n$. But the
former polynomial is $(-1)^{n-1}T_{2n-1}(x)/(2n-1)$, with maximum
modules $1/(2n-1)$. We shall now construct a polynomial of the latter
form (with no term in $x^2$). 

With the notation of the hypothesis for $p(x)$, write $\sup|p(x)|=m$.

Then, if $\mu >0$,
$$
\bigg| \frac{x}{1+ \mu}+a_1 \left(\frac{x}{1+ \mu} \right)^2 \cdots
+a_n \left(\frac{x}{1+ \mu} \right)^{2n} \bigg| \leq m. 
$$

Therefore
$$
\displaylines{\hfill 
  | x(1+ \mu)+a_1 x^2+a'_2 x^4 + \cdots + a'_n x^{2n}|\leq m (1+
  \mu)^2 \hfill \cr
  \text{i.e.,}\hfill | p(x)+\mu (x+c_2 x^4 +\cdots +c_n x^{2n})|\leq
  m(1+ \mu)^2.\hfill } 
$$

Therefore 
$$
\displaylines{\hfill 
  |\mu(x +c_2 x^4  +\cdots +c_n x^{2n})|\leq m \left\{(1+\mu)^2+1
  \right\}\hfill\cr 
  \text{and so}\hfill |x + c_2 x^4 +\cdots +c_n x^{2n}|\leq m \left\{
  (1+\mu)^2+1 \right\} / \mu \hfill } 
$$

This\pageoriginale is true for all positive values of $\mu$, and so we can replace
the right-hand side by its minimum, which is $2m(1+\sqrt{2})$. 

As we said, the maximum modulus of a polynomial with exponents
$1,4,6,\ldots, 2n$ is greater than $1/(2n-1)$ and therefore 
$$
m>\frac{1}{2(1+ \sqrt{2})}.\frac{1}{2n-1} 
$$

We have now all the material for the final result announced at the end
of \S \ref{chap3:sec7}.
 
\begin{theorem}\label{chap3:sec9:thm14}%theorem 14
  If $d_{2n}$ is the best approximation to $x$ in $(0,1)$ by
  $$
  \displaylines{\hfill 
  a_ 0 +a_1 x^2 +\cdots + a_n x^{2n},\hfill \cr
  \mbox{then}\hfill 
  \frac{1}{2n+1}>d_{2n}> \frac{1}{4(1+\sqrt{2})}\cdot
  \frac{1}{2n-1}.\hfill} 
  $$
\end{theorem}

\begin{proof}
  $d_{2n}$ is the maximum modulus of the oscillating polynomial
  $$
  A_ 0 +x+A_1 x^2 +A_2 x^4+ \cdots +A_n x^{2n}.
  $$

  Let $p(x)$ and $m$ have the same meanings as in
  Theorem \ref{chap3:sec9:thm13}. 
\end{proof}

By Theorem \ref{chap3:sec8:thm10}, \qquad $d_{2n}<m$.

Write $q(x)-A_0 =x+A_1 x^2 +\cdots + A_n x^{2n}$.

So, by Theorem \ref{chap3:sec8:thm10},
$$
\sup |q(x)-A_ 0|
$$
is greater than the maximum modulus of $p(x)$, the oscillating
polynomial with exponents $1,2,4, \ldots, 2n$ and coefficient of $x$
equal to 1; that is to say, is greater than $m$. 

But $\sup|q(x)-A_ 0| \leq d_{2n} +|A_ 0|\leq 2 d_{2n}$.

Therefore \qquad $2 d_{2n}>m$.

The inequalities of Theorem \ref{chap3:sec9:thm13} for $m$ give the
started inequalities for $d_{2n}$. 

\begin{center}
{\bf Notes}
\end{center}

\begin{enumerate}[1.]
\item Example.\pageoriginale Find the polynomial in $P_n$ for which the coefficient
  of $x^k$ is $1$ and which deviates least from $0$. 
\item The definition on page $22$ of an oscillating polynomial can be
  extended to a system 
  $$
  A_0 \varphi _0 + \cdots + A_n \varphi_n (x),
  $$
  if the $\varphi 's$ satisfy certain conditions. See Bernstein,
  Lecons, $1$ or Aschieser, $67$ 
\end{enumerate}

\begin{center}
{\bf Hint}
\end{center}

\begin{enumerate}[1.]
\item If $k,n$ are both even or both odd, consider $T_n(x)$, otherwise
  $T_{2n}(\sqrt{x})$. 
\end{enumerate}
