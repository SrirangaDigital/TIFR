\chapter{Trigonometric Polynomials}\label{chap4}

\setcounter{section}{9}
\section{Trigonometric polynomials. Modulus of Continuity}\label{chap4:sec10}

The\pageoriginale central problem of approximation, namely the degree of the
polynomial required an assigned closeness to a given function, yields
more easily to trigonometric than to algebraic
treatment. Trigonometric series and in particular Fourier series have
been in the fore-front of Analysis for something like a century, and
knowledge about them has been available for any problem of
approximation. 

\textit{A trigonometric polynomials is}

$t(x)=\dfrac{1}{2}a_0 + (a_1 \cos x+ b_1 \sin x)+ \cdots + (a_n \cos
nx+ b_n \sin nx)$. This can be written $t_n(x)$ if $a_n \neq 0$ or $b_n
\neq 0$ and we wish to display the \textit{order} of the
polynomial. We can denote by $T_n$ the set of all polynomials which
are sums of multiples of $\cos kx$ and $\sin kx$ for $< \leq k \leq
n$. (There will be no confusion with the Chebyshev polynomials $T_n
(x)$ of \S \ref{chap2:sec6}). 

The function $t(x)$ is periodic with period $2 \pi$ (and, in general,
with no smaller period). We say that $f(x)$ is $C(2\pi)$ if it is
continuous with period $2 \pi$. 

The problem of approximating to a $C(2\pi)$ function by a
trigonometric polynomial is essentially the same as that of
approximating to a $C(a,b)$ function by an algebraic polynomials. In
the first place, the analogue of Theorem 1 holds. 

\begin{theorem}[Weierstrass]\label{chap4:sec10:thm15} %theorem 15
  If $f(x)$ is $C(2\pi)$ then, given $\varepsilon$,
  there is $t(x)$ such that 
  $$
  |f(x)-t(x)|< \varepsilon \quad \text(all ~x)
  $$
  
  This will emerge as a by-product of theorem \ref{chap4:sec11:thm18}, and we shall give an
  independent proof here. You should, however, read Notes $1-3$ at the
  end of this chapter. 
\end{theorem}

In\pageoriginale statements about periodic functions, values of $x$ differing by
multiple of $2 \pi$ will be regarded as the same. 

\setcounter{lem}{0}
\begin{lem}\label{chap4:sec10:lem1}%lemma 1
  The equation $t_n(x)=0$ has at most $2n$ roots.
\end{lem}
(Prove by expressing in term of $\tan \dfrac{1}{2}x$ or of $\exp ix$).

\setcounter{corollary}{0}
\begin{corollary}\label{chap4:sec10:coro1} %corollary 1
  Two $t_n's $ which take the same values at $2n+1$ points are identical.
\end{corollary}

\begin{corollary}\label{chap4:sec10:coro2}%corollary 2
  If two $t_n's$ have $2n$ common zeros one is a consult multiple of the order
\end{corollary}

The reader should verify that there is an analogue of the Lagrange
polynomial of \S \ref{chap2:sec4}, namely 

The polynomial in $T_n$ which takes the values $c_i$ at $x_i
(i=0,1,\ldots, 2n)$ is  
\begin{align*}
  & P(x) \sum \frac{1}{2\sin \frac{x-xi}{2}} \frac{c_i}{P'(x_i)}\\
  ~\text{where}\quad & P(x) =\prod \sin \frac{x-xi}{2}.
\end{align*}

We shall take for granted the trigonometric analogues of Theorem
$4-7$ (pages $14-17$) about best approximation. Briefly, for a given
$f(x)$ in $C(2 \pi)$, there is unique $t(x)$ of best approximation in
$T_n$ which is characterized by $f(x)-t(x)$ taking its greatest
numerical value, with alternating sign, for alt least $2n+2$ values of
$x$. Proofs can be found in the book of de la Vallee Poussin or
Natanson. 
\newpage

\noindent \textbf{Illustrations:}

\begin{enumerate}[1)]
\item If \qquad $f(x)=t_{n-1}(x)+ (a_n \cos nx +b_n \sin nx)$.
  then $t_{n-1}(x)$ gives the best approximation in $T_{n-1}$ to $f(x)$.
  \begin{proof}
    $f-t_{n-1}$ takes the values $\pm \sqrt{(a^2_n +b^2_n)}$ alternately
    at $2n$ points. 
  \end{proof}
\item An\pageoriginale interesting example is Weiertrass's non-differentiable function 
  $$
  f(x)= \sum^ \infty _{r=0} a^r \cos b^r x
  $$
  where $0< a<1,b$ is an odd integer and $ab> 1$. We shall prove that
  the best approximation in  $T_n$ to $f(x)$ is 
  $$
  t(x)=\sum ^k_{r=0}a^r \cos b^rx, ~\text{where}~ b^k\leq n< b^{k+1}.
  $$
\end{enumerate}

\begin{proof}
  $f(x)-t(x)=\sum^\infty_{k+1}a^r \cos b^r x$.
\end{proof}

This takes its greatest value $\sum \limits^ \infty _{k+1}a^r$ at
$x=0$. Cos $b^{k+1}x$ takes the values $\pm 1$ alternately at integral
multiples of $\pi/b^{k+1}$, of which there are $2b^{k+1}$ in a
period. 

Since $b$ is an odd integer, $\cos b^r x$ for $r>k+1$ takes the same
values at those points as $\cos b^{k+1}x$. 

Now $2b^{k+1}\geq 2n+2$ and so $f(x)-t(x)$ takes its numerically
greatest value for at least $2n+2$ values of$x$. 

\begin{coro*}\label{page33}
  The approximation given by this $t(x)$ is $A/n^ \alpha$, where
  $\alpha =\log (1/a)/ \log	 b$. 
\end{coro*}

\begin{proof}
  The approximation is 
  $$
  \frac{a^{k+1}}{1-a}=\frac{b^{-\alpha(k+1)}}{1-a}\sim
  \frac{1}{1-a}\cdot\frac{1}{n^ \alpha}  
  $$
\end{proof}

\noindent
\textit{Modulus of continuity.} Let $f(x)$ be $C(a,b)$ and define
$$
\omega (\delta)=\sup |f(x_2)-f(x_1)| ~\text{for}~ |x_2-x_1|\leq \delta.
$$

Then $\omega(\delta)$ is continuous, increases as $\delta$ increases,
and tends to $0$ as $\delta$ tends to $0$. We shall find that the
rapidity with which $\omega(1/n)$ tends to $0$ as $n \to \infty$ gives
the clue to the approximation to $f(x)$ attainable in $P_n$ or $T_n$. 

If\pageoriginale $f(x)$ is $C(2 \pi)$, the same definition of $\omega (\delta)$
holds. Observe that now the greatest value of$\omega (\delta)$ is
$\omega (\pi)$  

Properties of $\omega (\delta)$ are collected in the following
theorem.

\begin{theorem}\label{chap4:sec10:thm16}%theorem 16
  \begin{enumerate}[(1)]
  \item If $n$ is an integer,
    $$
    \omega(n \delta) \leq n \omega (\delta).
    $$
  \item If $k>0, \omega (k \delta) \leq (k+1) \omega(\delta)$.
  \item If $\omega(\delta)=$ for some $\delta > 0$, then $f(x)$ is a constant.
  \end{enumerate}
\end{theorem}

\begin{proof}
  \begin{enumerate}[(1)]
  \item $f(x+nh) - f(x) = \sum \limits^{n-1}_{k=o} \bigg\{
    f(x+kh+h)-f(x+kh) \bigg \}$. 
    
    For $h \leq \delta$, the R.H.S. is numerically at most $n \omega(\delta)$.
  \item If $k$ is not an integer, let $n$ be the integer next greater. Then 
    $$
    \omega(k \delta) \leq \omega(n \delta) \leq n \omega(\delta) \leq
    (k+1) \omega(\delta). 
    $$
  \item $f(x)$ is constant in any interval less that $\delta$, and so
    everywhere.
  \end{enumerate}
\end{proof}

\textit{Lipschitz condition. Def. $f(x)$} satisfies the Lipschitz
condition of order $\alpha$(briefly, is Lip. $\alpha$) in a given
interval, if for every $x_1,x_2$ in it, 
$$
|f(x_2)-f(x_1)| \leq A|x_2 -x_1|^{\alpha}.
$$
It follows that $\omega (\delta) \leq A \delta^{\alpha}$.

In this, $0 < \alpha \leq 1$. If $\alpha >1, f(x)$ can only be a
constant,be a constant, because then  
$$
\omega (\delta)\leq n \omega (\delta/{n}) \leq A \delta^{\alpha}/n^{\alpha-1}.
$$
Making $n \to \infty$, we have $\omega (\delta) =0$.

\section{Fourier and Fejer Sums}\label{chap4:sec11}

We\pageoriginale collect for reference in Theorem \ref{chap4:sec11:thm17} some well-known facts. Proofs
can be found in any text-book of analysis which includes a chapter on
Fourier Series. 
\begin{theorem}\label{chap4:sec11:thm17} %theorem 17
  \begin{enumerate}[\rm (1)]
  \item \textit{The sum}
    $$
    S_n = \frac{1}{2} a_0 + \sum^n_{r=1} (a_r \cos  rx + b_r \sin rx)
    $$
    for the Fourier Series of $f(x)$ is equal to 
    $$
    \frac{1}{\pi} \int^{\pi}_0 \left\{f(x+t) + f(x-t) \right\}
    \frac{\sin \left(n+ \frac{1}{2}\right)t}{2 \sin \frac{1}{2}t}dt. 
    $$
  \item $|f(x) - S_n (x) | < M (A \log n+ B)$, 
    where $M = \sup |f(x)|$ and $A,B$ are constants.
  \item If $\sigma_n = (S_0 + S_1 \cdots + S_{n-1})/n$
    
    is the Fejer $(C1)$ sum of the Fourier series of $f(x)$, then 
    $$
    \sigma_n = \frac{1}{n \pi} \int^{\pi}_0 f(x+ 2t) \left(\frac{\sin nt
   }{\sin t}\right)^2 dt. 
    $$
  \item $\dfrac{1}{\sin^2 t} = \sum^{\infty}_{-\infty} \dfrac{1}{(t +
    k \pi)^2} \qquad (t \neq k \pi).$ 
  \end{enumerate}
\end{theorem}

The result (2), which cannot be improved, shows that, in general,
the Fourier series of a function gives a poor approximation in the
sense measured by $\sup|f(x) -S_n (x) |$. As the R.H.S. of (2) tends
to infinity with $n,(2)$ does not include Weierstrass's Theorem
\ref{chap4:sec10:thm15}. The sense in which the Fourier series does
give the best 
approximation is the mean-square sense (omitted here). It is known
that the Fejer sums $\sigma_n$ of (3) behave more regularly than the
Fourier sums $S_n$; this is due to the kernel $(\sin nt / \sin t)^2$
in the integral for ${}^\sigma_ n$  being positive, whereas the kernel in
$S_n$ takes both signs. The next theorem gives the approximation to
$f(x)$ afforded by $({}^\sigma_n (x)$.  

\begin{theorem}\label{chap4:sec11:thm18}%theorem 18
  If\pageoriginale $f(x)$ has modulus of continuity $\omega (\delta)$, then 
  $$
  |f(x) - \sigma_n (x) | \leq A \omega (1/n) | log \omega (1/ n) |.
  $$
\end{theorem}

\begin{proof}
  We first put $\sigma_n(x)$ into a form more convenient than that of
  Theorem \ref{chap4:sec11:thm17}(3). Since $f$ is periodic 
  $$
  \int^{\pi}_0 f(x + 2t) \frac{\sin ^2 nt}{(t+k\pi)^2}dt =
  \int^{(k+1)\pi}_{k \pi}f(x + 2t) \frac{\sin ^2 nt}{(t^2)} dt. 
  $$
\end{proof} 

Then, from (3) and (4) of Theorem \ref{chap4:sec11:thm17},
$$
\sigma_n = \frac{1}{n \pi} \int^{\pi}_0 f(x + 2t) \frac{\sin ^2
  nt}{\sin^2t}dt = \frac{1}{n \pi} \int^{\infty}_{-\infty} f(x + 2t)
\frac{\sin ^2 nt}{t^2}dt, 
$$
and so, by changing the variable from $t$ to $t/n$,
\begin{align*}
  \sigma_n & = \frac{1}{\pi} \int^{\infty}_{-\infty} f(x+
  \frac{2t}{n}) \frac{\sin ^2 t}{t^2}dt \\ 
  \sigma_n -f & = \frac{1}{\pi} \int^{\infty}_{-\infty} \left\{f(x+
  \frac{2t}{n}) -f(x) \right\}\frac{\sin ^2 t}{t^2}dt. 
\end{align*}

Therefore 
$$
|f-\sigma_n| \leq \frac{2}{\pi} \int^{\infty}_{0} \omega (2t/n)
\frac{\sin ^2 t}{t^2}dt. 
$$

The integral on the R.H.S. is the sum of the integrals over
$(0,1)$, $(1,X)$ and $(X,\infty)$. This gives  
\begin{align*}
  |f - \sigma_n| & \leq \frac{2}{\pi}\left\{\omega(2/n) + \int^X_1 \omega
  (2t/n) \frac{dt}{t^2} + \omega (\pi) \int^{\infty}_X \frac{dt}{t^2}
  \right\}\\ 
  & \leq \frac{2}{\pi}\left\{\omega(2/n) + \omega (2/n) \int^X_1
  \frac{t+1}{t^2} dt+ \frac{\omega (\pi)}{X}\right\}\\ 
  & \leq \frac{2}{\pi} \left\{\omega(2/n) (2 + \log X) + \frac{\omega
    (\pi)}{X} \right\}. 
\end{align*}

Choose $X = 1/ \omega (2/n)$ and we have a result equivalent to that stated.

\setcounter{corollary}{0}
\begin{corollary}\label{chap4:sec11:coro1} %corollary 1
  Theorem \ref{chap4:sec10:thm15}\pageoriginale
\end{corollary}

\begin{corollary}\label{chap4:sec11:coro2} %corollary 2
  If $\omega (\delta) < A \delta^{\alpha} (0 < \alpha <1)$, then $| f
  - \sigma_n| < \frac{AB}{n^{\alpha}}$, where $B = B(\alpha)$ is
  independent of $f$. 
\end{corollary}

\begin{proof}
  \begin{align*}
    | f - \sigma_n| & \leq \frac{2}{\pi} \int^{\infty}_0 \omega (2t/n)
    \frac{\sin^2 t}{t^2} dt\\ 
    & \leq \frac{2^{\alpha + 1}A}{\pi n^{\alpha}} \int^{\infty}_0
    t^{\alpha} \frac{\sin^2 t}{t^2} dt  
  \end{align*}
  The estimates in Chapter III would lead us to suspect that, if we
  can find a $t(x)$ which approximates to $f(x)$ more closely than
  $\sigma_n (x)$ does, we may get rid of the $\log \omega (1/n)$ on the
  R.H.S. of Theorem \ref{chap4:sec11:thm18}. It is easy to see how to try to do this. The
  logarithm arises from integrating a term in $1/t$. The Fejer sum is  
  $$
  F_r(x,n) = \frac{1}{J_r} \int^{\infty}_{-\infty} f \left(x+ \frac{2t}{n}\right)
  \left(\frac{\sin t}{t}\right)2r\, dt 
  $$
  for $r=1$ and $J_r = \pi$. If $r \geq 2$, there will be no term in $1/t$.  
  We shall achieve our purpose by taking $r=2$.
\end{proof}

\setcounter{lem}{0}
\begin{lemma}\label{chap4:sec11:lem1} %lemma 1
  \begin{enumerate}[(1)]
  \item $J_2 = \int \limits^{\infty}{-\infty} \left(\frac{\sin t}{t}\right)^4  dt
    = \frac{2 \pi}{3}$.\\ 
  \item $F_2(x,n) $ is in $T_{2n -1}$.
  \end{enumerate}
\end{lemma}

\begin{proof}
  \begin{enumerate}[(1)]
  \item \begin{align*}
    J_2 & = \int\limits^{\pi}_0 \sin^4 t \sum^{\infty}_{-\infty}
    \frac{1}{(t+k\pi)^4}dt\\ 
    & = \frac{1}{6} \int\limits^{\pi}_0 \sin^4 t \frac{d^2}{dt^2}
    \left(\frac{1}{\sin ^2 t}\right) dt\\ 
    & = \frac{1}{6}\int\limits^{\pi}_0 \sin^4 t \left(\frac{6}{\sin^4 t}-
    \frac{4}{\sin^2 t}\right) dt = \frac{2 \pi}{3}. 
  \end{align*}
  \item Reversing the steps by which $F_1(x,n)$ was obtained in the
    first part of the proof of Theorem 18. we have  
    $$
    F_2(x,n) = \frac{3}{2 \pi} \frac{3}{6n} \int\limits^{\pi}_0 f(x +
    2t) \sin^4nt \frac{d^2}{dt^2} \left(\frac{1}{\sin^2t}\right) dt. 
    $$
  \end{enumerate}

  Then\pageoriginale $\sin^4nt \dfrac{d^2}{dt^2} \left(\dfrac{1}{\sin^2t}\right) =
  \sin^4nt \left(\dfrac{6}{\sin^4t}- \dfrac{4}{\sin^2 t}\right)$. 
\end{proof}

Now $\dfrac{\sin nt}{\sin t}$ is the sum of multiples of cos $kt$
where $k \leq n-1$. Hence $\sin^4nt \dfrac{d^2}{dt^2}
\left(\dfrac{1}{\sin^2t}\right)$ is the sum of multiples of cos $kt$ where $k
\leq 4n-2$. Moreover, the expression is even and has period $\pi$, so
$k$ takes only even values, 21 say, where $1 \leq 2n-1$. Finally, 
$$
\int^{\pi}_0 f(x+2n) \cos 21t dt = \dfrac{1}{2} \int^{2 \pi}_0 f(u) 
\cos l (u-x) du  
$$
and $F_2(x,n) $ is in $T_{2n-1}$.

\begin{theorem}\label{chap4:sec11:thm19}%theorem 19
  $|f(x) - F_2(x,n) | \leq 3 \omega (1/n)$.
\end{theorem}

\begin{proof}
  $F_2(x,n) -f(x) = \dfrac{3}{2 \pi} \int \limits^{\infty}_{-\infty}
  \bigg\{f(x+ \dfrac{2t}{n}) - f(x) \bigg\} \left(\dfrac{\sin t}{t}\right)^4
  dt$. 
\end{proof}

Now $|f(x+ \dfrac{2t}{n}) - f(x)| \leq \omega \left(\dfrac{2|t|}{n}\right) \leq
(2 |t| +1) \omega (1/n)$ from Theorem \ref{chap4:sec10:thm16} (2). Therefore  
$$
|f(x) -F_2 (x,n)| \leq \omega(1/n) \frac{3}{\pi} \int^{\infty}_0
(2t+1) \left(\frac{\sin t}{t}\right)^4 dt = A \omega (1/n), 
$$
where $A = 1+ \dfrac{6}{\pi} \int^{\infty}_0  \dfrac{\sin^4
  t}{t^3}dt$. But  
$$
\int^{\infty}_0  \frac{\sin^4 t}{t^3}dt \int^{\infty}_0 |\dfrac{\sin^3
  t}{t^2}| dt = \frac{1}{2} \int^{\pi}_0 \frac{\sin^3t}{\sin^2t} dt =1 
$$
(again by use of Theorem \ref{chap4:sec11:thm17}(4)).

So $\qquad A< 1+ \dfrac{6}{\pi}< 3$.

\begin{theorem}\label{chap4:sec11:thm20} %theorem 20
  If $f(x)$ is $C(2 \pi)$ and $f'(x)$ is continuous with modulus of
  continuity $\omega_1 (\delta)$, then  
  $$
  |f(x) -F_2(x, n)| \leq \dfrac{A}{n} \omega_1 (1/n)~\text{where}~ A < 5/2.
  $$
\end{theorem}

\begin{proof}
  {\fontsize{10}{12}\selectfont$F_2(x,n) -f(x) = \dfrac{3}{2 \pi} \int \limits^{\infty}_0 \left\{
  f \left(x+ \dfrac{2t}{n}\right) + f \left(x-\dfrac{2t}{n}\right) -
  2f(x) \right\}  \left(\dfrac{\sin t}{t}\right)^4 dt$.}\relax 

  The\pageoriginale modulus of the term within $\{~\}$ is 
  \begin{align*}
    & |\frac{2}{n} \int^{t}_0 \left\{f'\left(x+ \frac{2u}{n}\right) -
    f'\left(x-\frac{2u}{n}\right) \right\}du| \\
    & \leq \frac{2}{n}\int \limits^{t}_0 \omega_1 \left(\frac{4u}{n}\right)du\\
    & \leq \frac{2}{n} \omega_1 \left(\frac{1}{n}\right)  \int
    \limits^{t}_0 (4u +1) du\\ 
    & = \frac{2}{n} \omega_1 \left(\frac{1}{n}\right) (2t^2 +t)
  \end{align*}
\end{proof}
Therefore 
{\fontsize{10}{12}\selectfont
$$
\displaylines{\hfill 
  |f(x) -F_2(x,n) | \leq \frac{A}{n} \omega_1
  \left(\frac{1}{n}\right),\hfill \cr 
  \text{where}\hfill \cr  
  A= \frac{3}{\pi} \int \limits^{\infty}_0 (2t^2 +t)
  \left(\frac{\sin t}{t}\right)^4 dt = \frac{3}{\pi} \sin^2t dt + \frac{3}{\pi}
  \int \limits^{\infty}_0 \frac{\sin^4 t}{t^3}dt < \frac{3}{\pi}
  \left(\frac{\pi}{2} + 1\right) < \frac{5}{2}\hfill }   
$$}\relax 

Theorem \ref{chap4:sec11:thm20} can be extended to higher derivatives. If $f(x)$ has an
r-th derivative with modulus of continuity $\omega_r (\delta)$, the
approximation attainable in $T_n$ is a constant multiple of $n^{-r}
\omega_r (1/n)$. 
\begin{center}
\textbf{Notes\pageoriginale on Chapter IV}
\end{center}

\begin{enumerate}[1.]
\item Use the singular integral (de la Vallee Poussin)
  $$
  \frac{1}{J_n} \int\limits^{\pi}_{-\pi} \cos^{2n} \frac{1}{2} (t-x) f(t) dt,
  $$
  where $J_n$ is the value of the integral when $f(t) =1,$ to give a
  direct proof of Theorem \ref{chap4:sec10:thm15}. 
\item Assuming Theorem \ref{chap4:sec10:thm15} proved, deduce
  Theorem \ref{chap1:sec1:thm1} from it.
\item Deduce Theorem \ref{chap4:sec10:thm15} from Theorem
  \ref{chap1:sec1:thm1} as follows: 
  \begin{enumerate}[(a)]
  \item Prove that, if $f(x)$ is $C(0, \pi)$, it can be approximated
    uniformly by a $t(x)$ containing cosines only. 
  \item By applying $ \qquad (a)$ to the even functions
    \begin{align*}
      2g(x)& = f(x) + f(-x)\\
      2h(x)& = \bigg\{f(x) - f(-x) \bigg\} \sin x,
    \end{align*}
    deduce that $f(x)$ is uniformly approximately by a $t(x)$.
  \end{enumerate}
\item With the notation of Illustration (2), Corollary, page \pageref{page33},
  prove that Weierstrass's function $\sum a^r \cos b^r x$ satisfies a
  Lipschitz condition of order $\alpha$. 
\end{enumerate}

\begin{center}
\textbf{Hints}\pageoriginale
\end{center}

\begin{enumerate}[1]
\item Follow Theorem \ref{chap1:sec2:thm2}. Detail is in Natanson, 10.
\item Approximate to $\cos kx$ and sin $kx$ by a finite number of
  terms of their expansions in powers of $x$. 
\item
  \begin{enumerate}[(a)]
  \item Put $y = \cos x$.
  \item $g(x), h(x)$ are uniformly approximately in $(-\pi, \pi)$. So is
    $g(x)\break \sin^2 x + h(x) \sin^2 x$. So is $f(x) \cos^2 x$, and hence
    $f(x) (\sin^2x + \cos^2 x)$. 
  \end{enumerate}
\item Given $h$, choose $n$ so that $b^n h \leq 1 < b^{n+1}h$.
  \begin{align*}
    f(x+h)-f(x-h) & = -2 \sum^{\infty}_1 a^r \sin b^r h \sin b^r x\\
    & =  \sum^n_1 + \sum^{\infty}_{n+1}\\
    \bigg| \sum^{\infty}_{n+1} \bigg| \leq 2  \sum^{\infty}_{n+1}a^r &
    = \dfrac{2a^{n+1}}{1-a}\\ 
    \bigg|\sum^n_1\bigg| \leq 2h \sum^n_1 a^r b^r & = 2abh \dfrac{a^n
      b^n -1}{ab-1}< \dfrac{2b a^{n+1}}{ab-1}
  \end{align*}
\end{enumerate}

But $\qquad a^{n+1} = b^{-\alpha (n+1)}< h^{\alpha}$.

Hence  $\qquad | f(x+h) - f(x-h)| < A h^{\alpha}$.

With more trouble (Aschieser and Krein, 167) this can be proved best possible.
