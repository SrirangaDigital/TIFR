\chapter[Mixed Complexes, the Connes Operator B, and ...]{Mixed
  Complexes, the Connes Operator B, and Cyclic 
  Homology}\label{chap5}\index{Connes' operator B}

THE\pageoriginale DOUBLE COMPLEX $CC_{\ast,\ast}(A)$ has acyclic
columns in odd degrees, and this property leads to the concept of a
mixed complex. Thus we effectively suppress part of the cyclic
homology complex $CC_{\ast}(A)$. In the second section this new
definition is shown to be equivalent to the old one. Yet another way
of simplifying the Connes-Tsygan double complex is to normalize the
Hochschild complexes, and this is considered in \S\ \ref{chap5-sec3}.

\section{The operator $B$ and the notion of a mixed
  complex}\label{chap5-sec1}

Let $A$ be an algebra over $k$. The last simplicial operator defines a
homotopy operator $s:C_{q}(A)\to C_{q+1}(A)$ by the relation
$s=(-1)^{q}s_{q}$. It has the basic property that $sb'+b's=1$, and
this is a general property of simplicial objects over an abelian
category.

\begin{definition}\label{chap5-defi1.1}
Let $A$ be an algebra over $k$. The Connes operator is
$B=(1-T)sN:C_{q}(A)\to C_{q+1}(A)$ on the standard complex.
\end{definition}

For any cyclic object $X_{\bigdot}$ the Connes operator is
$$
B=(1-T)sN:X_{q}\to X_{q+1}
$$
a morphism of degree $+1$. The corresponding diagram is
\[
\xymatrix{
X_{q+1} & \ar[l]_{1-T} X_{q+1}\ar@<.25em>[d]^{s} & \\
 & X_{q}\ar@<.25em>[u]^{b'} & X_{q}\ar[l]_{N}
}
\]

\begin{proposition}\label{chap5-prop1.2}
Let $X_{\bigdot}$ be a cyclic object over an abelian category
$\A$. The Connes operator $B$ of degree $+1$ and the usual boundary
operator $b$ satisfy the following relations
$$
b^{2}=0,B^{2}=0,\q\text{and}\q Bb+bB=0.
$$
\end{proposition}

\begin{proof}
The\pageoriginale first relation was already contained in 3(2.4), and
the second $BB=(1-T)sN(1-T)sB=0$ since $N(1-T)=0$ by 3(3.3). For the
last relation we calculate using 3(3.4)
\begin{align*}
Bb+bB &= (1-T)(sNb+b(1-T)sN=(1-T)sb'N+(1-T)b'sN\\
      &= (1-T)(sb'+b's)N=(1-T)N=0.
\end{align*}

This proves the proposition.
\end{proof}

\begin{remark}\label{chap5-rem1.3}
For the standard cyclic object $C_{\bigdot}(A)$ associated with an
algebra $A$, the following formula defines $B$ on an element,
{\fontsize{10}{12}\selectfont
\begin{align*}
B(a_{0}\otimes\cdots\otimes a_{q}) &=
\sum(-1)^{iq}(a_{q-i}\otimes\cdots\otimes a_{q}\otimes
a_{0}\otimes\cdots\otimes a_{q-1-i}\otimes 1)- \\
&\q-\sum(-1)^{(i-1)q}(1\otimes a_{q-i}\otimes\cdots\otimes
a_{q}\otimes a_{0}\otimes\cdots\otimes a_{q-1-i}).
\end{align*}}
\end{remark}

This leads to a new structure called a mixed complex\index{mixed complex} which is a
complex with two operators one of degree $-1$ and one of degree $+1$
which commute in the graded sense, that is, anticommute in the
ungraded sense. This is the relation $Bb+bB=0$. Each mixed complex has
homology in the usual sense with its operator of $-1$. Using the two
operators, we can associate a second complex, which can be thought of
as the total complex of a double complex associated with the mixed
complex. The homology of this complex is called the cyclic homology of
the mixed complex. This terminology is justified because the cyclic
homology of a mixed complex associated with a cyclic object will shown
to be isomorphic to the cyclic homology of the cyclic object as
defined in the previous chapter. A second point justifying the
terminology is that there is a Connes exact couple\index{Connes' exact couple} relating the
ordinary and cyclic homology of a mixed complex.

There are two advantages in considering mixed complexes. The complex
defining cyclic homology of the mixed complex is smaller than
$CC_{\bigdot\bigdot}(X)$ for a cyclic object. Then there are mixed
complexes which do not come from cyclic objects which are useful,
namely the one corresponding to the normalized\pageoriginale standard
complex $\overline{C}_{\ast}(A)$ for Hochschild homology.

\section{Generalities on mixed
  complexes}\label{chap5-sec2}

\begin{definition}\label{chap5-defi2.1}
Let $\A$ be an abelian category. A mixed complex $X$ is a triple
$(X_{\ast},b,B)$ where $X_{\ast}$ is a $\bfZ$-graded object in $\A$,
$b:X_{\ast}\to X_{\ast}$ is a morphism of degree $-1$, and
$B:X_{\ast}\to X_{\ast}$ is a morphism of degree $+1$ satisfying the
relations
$$
b^{2}=0, B^{2}=0, Bb+bB=0.
$$

A morphism $f:X\to Y$ of mixed complxes is a morphism of graded
objects such that $bf=fb$ and $Bf=fB$. A mixed complex is positive if
$X_{q}=0$ if $q<0$.
\end{definition}

Let $\Mix(\A)$ denote the category of mixed complexes and
$\Mix^{+}(\A)$ the full subcategory of positive mixed complexes.

\begin{remark}\label{chap5-rem2.2}
We have the following functors associated with mixed complexes. Let
$\A$ denote an abelian category.
\begin{enumerate}
\renewcommand{\labelenumi}{(\theenumi)}
\item The functor which assigns to a mixed complex $(X,b,B)$ the
  complex $(X,b)$ is defined $\Mix(\A)\to \C(\A)$ and it restricts to
  $\Mix^{+}\break(\A)\to \C^{+}(\A)$ to the full subcategories of positive
  objects. When it is composed with the homology functor $H:\C(\A)\to
  Gr_{Z}(\A)$, it defines the homology $H(X)$ of the mixed object $X$.

\item The functor which assigns to a cyclic object $X_{\bigdot}$ the
  mixed object $(X,b,B)$ as in (\ref{chap5-prop1.2}) is defined
  $\Lambda(\A)\to \Mix^{+}(\A)$, and when composed with
  $\Mix^{+}(\A)\to \C^{+}(\A)$ gives the usual simplicial differential
  object whose homology is the ordinary homology of the cyclic object.

\item Finally the standard cyclic object $C_{\bigdot}(A)$ associated
  with an algebra $A$ over a ring $k$ is a functor defined
  $\Alg_{k}\to \Lambda(k)$ which can be composed with the above
  functors to give a positive mixed complex of $k$-modules\index{module} whose
  homology is in turn its Hochschild homology.
\end{enumerate}

Now\pageoriginale we wish to define a functor $\Mix^{+}(\A)\to
\C^{+}(\A)$ whose homology is to be the cyclic homology. There is a
similar construction for $\Mix(\A)\to \C(\A)$ which is not given since
it is not needed for our purposes.
\end{remark}

\begin{definition}\label{chap5-defi2.3}
Let $(X,b,B)$ be a positive mixed complex over an abelian category
$\A$. The cyclic complex $(X[B],b_{B})$ associated with the mixed
complex\index{cyclic complex associated to a mixed complex} $(X,b,B)$ is defined as a graded object by
$X[B]_{n}=X_{n}\oplus X_{n-2}\oplus\ldots$ which is a finite sum since
$X$ is positive and $b_{B}:X[B]_{n}\to X[B]_{n-1}$ is defined using
the projections $p_{i}:X[B]_{n}\to X_{i}$ by the relation
$p_{i}b_{B}=bp_{i+1}+Bp_{i-1}$. The cyclic homology $HC_{\ast}(X)$ of
the mixed complex $X_{\ast}$ is $HC_{\ast}(X)=H_{\ast}(X[B])$, the
homology of cyclic complex associated with $X_{\ast}$.
\end{definition}

If the abelian category $\A=(k)$, the category of $k$-modules, then
the boundary in the cyclic complex can be described by its image on
elements $(x_{n},x_{n-2},x_{n-4},\ldots)\in X[B]_{n}$, and the above
definition gives
$$
b_{B}(x_{n},x_{n-2},x_{n-4},\ldots)=(b(x_{n})+B(x_{n-2}),b(x_{n-2})+B(x_{n-4}),\ldots). 
$$

\begin{remark}\label{chap5-rem2.4}
To $(X,b,B)$, a positive mixed complex over an abelian complex $\A$,
we associate an exact sequence
$$
0\to (X,b)\xrightarrow{i}(X[B],b_{B})\to (s^{-2}X[B],b_{B})\to 0
$$
where $i:X_{n}\to X[B]_{n}$ is defined by $p_{n}i=X_{n}$ and
$p_{i}i=0$ for $i<n$. Observe that $i$ is a monomorphism of complexes
with quotient of $X[B]$ equal to $s^{-2}X[B]$ which is $X[B]$ shifted
down by 2 degrees. The exact triangle of this short exact sequence is
the Connes exact couple for mixed complexes\index{Connes' double complex}
\[
\xymatrix{
HC_{\ast}(X)\ar[rr]^{S} & & HC_{\ast}(X)\ar[dl]\\
 & H_{\ast}(X)\ar[ul] & 
}
\]
and as usual $\deg(S)=-2$, $\deg(B)=+1$, and $\deg(I)=0$.
\end{remark}

If\pageoriginale we can show that the Connes exact sequence of the
previous Remark (\ref{chap5-rem2.4}) is the same as the Connes exact
sequence for a cyclic object in terms of $CC_{\bigdot}(X)$, then we
have a new way of calculating cyclic homology for a cyclic object and
hence also for an algebra. This we do in the next section.

First we remark that the above construction of the complex
$(X[B],\break b_{B})$ from a mixed complex $(X,b,B)$ can be viewed as the
total complex of a double complex $\B(X)$. 

\begin{definition}\label{chap5-defi2.5}
Let $(X,b,B)$ be a positive mixed complex over an abelian category
$\A$. The Connes double complex $\B(X)$ associated with $X$ is defined
by the requirement that $\B(X)_{p,q}=X_{q-p}$ for $p$, $q\geq 0$ and
$0$ otherwise, the differential $d'=B$ and $d''=b$.
\end{definition}

The double complex $\B(X)$ is concentrated in the $2^{nd}$ octant of the
first quadrant, that is, above the line $p=q$ in the first quadrant.
\[
\xymatrix@C=.2cm@R=.5cm{
\ldots & \ldots & \ldots & \ldots & \ldots & \ldots & \ldots & \ldots
& \ldots & \ldots & \ldots\\
X_{q}\ar[d]^{b} & \xleftarrow{B} & X_{q-1}\ar[d]^{b} & \xleftarrow{B}
& X_{q-2}\ar[d]^{b} & \xleftarrow{B} & \ldots & \xleftarrow{B} & 
X_{1}\ar[d]^{b}  & \xleftarrow{B} & X_{0}\\
\ldots\ar[d] & \ldots &  \ldots\ar[d] &  \ldots &  \ldots\ar[d] &
\ldots &  \ldots & \ldots & & & \\
X_{2}\ar[d]^{b} & \xleftarrow{B} & X_{1}\ar[d]^{b} & \xleftarrow{B} & X_{0} & & & & & \\
X_{1}\ar[d]^{b} & \xleftarrow{B} & X_{0} & & & & & & & &\\
X_{0} & 
}
\]

The associated single complex of the double complex $\B(X)$ is just
$X[B]$, $b_{B}$. Once again one can see the double periodicity which
arises by deleting the first column.

\section[Comparison of two definition of cyclic
  homology...]{Comparison of two definition of cyclic homology for a
  cyclic 
  object}\label{chap5-sec3}\pageoriginale 

We have two functors defined on category $\Lambda(\A)$ of cyclic
objects over an abelian category $\A$ with values in the category of
positive complexes $\C^{+}(\A)$ over $\A$. The first is
$CC_{\ast}(X)$, the associated complex of the cyclic homology double
complex $CC_{\bigdot\bigdot}(X)$, and the second is $X_{\bigdot}[B]$
where $X_{\bigdot}$, $b$, $B$ is the mixed complex associated with
$X$, see (\ref{chap5-defi1.1}) and (\ref{chap5-prop1.2})

\begin{notation}\label{chap5-not3.1}
For a cyclic object $X_{\bigdot}$ over an abelian category $\A$ we
define a comparison morphism $f:X_{\bigdot}[B]\to CC_{\ast}(X)$ by the
following relations in degree $n$. For $f_{n}:X_{\bigdot}[B]_{n}\to
CC_{n}(X)$ we require that
$$
pr_{i}f=
\begin{cases}
pr_{i} & \text{for~ } i\text{~ even}\\
s'N\,pr_{i-1} & \text{for~ } i\text{~ odd}
\end{cases}
$$
where is degree $n$ the diagram takes the form
\[
\xymatrix{
X_{\bigdot}[B]_{n}=\coprod_{i}X_{n+2i}\ar[r] &
\coprod_{i}X_{n+i}\ar[d]^{pr_{i}}\\ 
& X_{n+i}. 
}
\]

If $X_{\bigdot}$ is a cyclic $k$-module, then this definition can be
given in terms of elements,
$$
f(x_{n},x_{n-2},x_{n-4},\ldots)=(x_{n},s'Nx_{n-2},x_{n-2},s'Nx_{n-4},x_{n-r},\ldots).
$$
\end{notation}

\begin{lemma}\label{chap5-lem3.2}
The graded morphism $f:X_{\bigdot}[B]\to CC_{\ast}(X)$ is a morphism
of differential objects.
\end{lemma}

\begin{proof}
There is a general argument that says that abelian categories can be
embedded in a category of modules. The result is that we can check the
commutativity of $f$ with boundary morphisms using elements. Now the
differential of 
$$
f(x_{n},x_{n-2},x_{n-4},\ldots)=(x_{n},s'Nx_{n-2},x_{n-2},s'Nx_{n-4},x_{n-4}x\ldots) 
$$
is\pageoriginale the element
$$
(bx_{n}+(1-t)s'Nx_{n-2'}-b's'Nx_{n-2}+Nx_{n-2},\ldots).
$$
If we apply $f$ to the element
$$
b_{B}(x_{n},x_{n-2},x_{n-4},\ldots)=(bx_{n}+Bx_{n-2},bx_{n-2}+Bx_{n-4},\ldots), 
$$
then we obtain
$$
(bx_{n}+Bx_{n-2},s'Nbx_{n-2}+s'NBx_{n-4},\ldots).
$$

A direct inspection shows that the differential of
$f(x_{n},x_{n-2},\ldots)$ and $f(b_{B}(x_{n},x_{n-2},\ldots))$ have
the same even coordinates. For the odd indexed coordinates, we
calculate
\begin{align*}
s'Nbx_{n-2}+s'NBx_{n-4} &= s'Nbx_{n-2}+s'N(1-t)s'Nx_{n-4}\\
&= s'b'Nx_{n-2}\\
&= Nx_{n-2}-b's'Nx_{n-2}.
\end{align*}
This shows that $f$ is a morphism of complexes and proves the lemma.
\end{proof}

The following result shows that the two definitions of cyclic homology
are the same.

\begin{theorem}\label{chap5-thm3.3}
Let $X_{\bigdot}$ be a cyclic object in an abelian category $\A$. The
above comparison morphism $f:X_{\bigdot}[B]\to CC_{\ast}(X)$ induces
an isomorphism $H(f):H_{\ast}(X_{\bigdot}[B])\to HC_{\ast}(X)$.
\end{theorem}

\begin{proof}
The first index of the double complex $X_{\bigdot\bigdot}$ determines
a filtration $F_{p}CC_{\ast}(X)$ on $CC_{\ast}(X)$ where
$$
F_{p}CC_{n}(X)=\coprod\limits_{i+j=n,i\leq p}X_{i,j}
$$
and there is a related filtration $F_{p}X_{\bigdot}[B]$ on
$X_{\bigdot}[B]$
$$
F_{p}X_{\bigdot}[B]_{n}=\coprod_{2i\leq p}X_{n-2i}. 
$$

From\pageoriginale the definition of $f$, we check that $f$ is
filtration preserving. The morphism $E^{0}(f)$ is a monomorphism and
$d^{0}=b$ with $E^{0}_{2k,\ast}(f)$ and isomorphism,
$E^{0}_{2k+1,\ast}X_{\bigdot}[B]=0$, and $E^{0}_{2k+1,\ast}CC(X)$
acyclic. Thus $E^{1}(f)$ is an isomorphism. By 1(5.6) the induced
morphism $H_{\ast}(f)$ is an isomorphism. This proves the theorem.
\end{proof}

\begin{remark}\label{chap5-rem3.4}
The morphism $f$ considered above can be viewed as
$f:\B(X)_{\ast}=X_{\bigdot}[B]\to CC_{\ast}(X)$. These complexes come
from double complexes with a periodic structure. The first vertical
column of $\B(X)$ maps to the total subcomplex of $CC_{\ast}(X)$
determined by the first two vertical columns of
$CC_{\bigdot\bigdot}(X)$. The resulting subcomplexes have homology
equal to Hochschild homology while the quotient complexes have the
form of $\B(X)_{\ast}$ and $CC_{\ast}(X)$ respectively. We arrive at a
sharper form of the isomorphism in (\ref{chap5-thm3.3}), namely that
$f$ induces an isomorphism of the Connes' exact couple defined by
mixed complexes onto the Connes' exact couple defined by the cyclic
homology double complex.
\end{remark}

\section{Cyclic structure on reduced Hochschild
  complex}\label{chap5-sec4}\index{reduced Hochschild complex}

In 3(2.6), we remarked that for a simplicial $k$-module $X$, the
subcomplex $D(X)$ generated by degenerate elements was contractible,
and thus the quotient morphism induces an isomorphism on homology
$$
H_{\ast}(X)\to H_{\ast}(X/D(X)).
$$

For the standard complex $C_{\ast}(A)$ of an algebra $A$ the quotient
complex $C_{\ast}(A)/DC_{\ast}(A)$ is the reduced standard complex
$\overline{C}_{\ast}(A)$ where
$$
\overline{C}_{q}(A)=A\otimes \overline{A}^{q\otimes}
$$
as noted in 3(2.7). To study the cyclic homology $HC_{\ast}(A)$ with
the reduced standard complex, we use the mixed complex construction
and the following formula for the Connes' operator $B$.

\begin{proposition}\label{chap5-prop4.1}
The operators $b$ and $B$ on the standard complex
$C_{\ast}(A)$\pageoriginale define operators $b$ and $B$ on the
quotient reduced standard complex $\overline{C}_{\ast}(A)$ given by
the formulas
\begin{multline*}
b(a_{0}\otimes \cdots\otimes a_{q})=a_{0}a_{1}\otimes
a_{2}\otimes\cdots\otimes a_{q}\\
+\sum_{0<i<q}(-1)^{i}a_{0}\otimes\cdots\otimes
a_{i}a_{i+1}\otimes \cdots\otimes a_{q}\\
(-1)^{q}a_{q}a_{0}\otimes a_{1}\otimes\cdots\otimes a_{q-1}
\end{multline*}
where the ambiguity in $a_{0}a_{1}$ and in $a_{q}a_{0}$ is cancelled
with the terms $i=1$ and $i=q-1$ respectively in the sum and 
$$
B(a_{0}\otimes\cdots\otimes a_{q})=\sum_{1\leq i\leq
  q}(-1)^{iq}1\otimes a_{i}\otimes\cdots\otimes a_{q}\otimes
a_{0}\otimes\cdots\otimes a_{i-1}. 
$$ 
\end{proposition}

\begin{proof}
The first formula is just a quotient of the usual formula, and for the
second we calculate immediately that
$$
sN(a_{0}\otimes\cdots\otimes a_{q})=\sum_{1\leq i\leq
  q}(-1)^{iq}1\otimes a_{i}\otimes\cdots\otimes a_{q}\otimes
a_{0}\otimes \cdots\otimes a_{i-1}. 
$$

The statement follows from the fact that $tsN(a_{0}\otimes\cdots
a_{q})=0$ in the reduced complex with $1$ in the nonzero place giving
a degenaracy and the formula $B=(1-t(sN_{\bigdot}$ This proves the
proposition. 
\end{proof}

Now we rewrite the $b$, $B$ double complex for the reduced standard
complex $\overline{C}_{\ast}(A)$ where $\overline{C}_{q}(A)=A\otimes
\overline{A}^{q\otimes}$. It is in this form that we will compare it
with complexes of differential forms in the next two chapters.
\[
\xymatrix@C=0cm@R=.5cm{
\ldots & \ldots & \ldots & \ldots & \ldots & \ldots & \ldots & \ldots
& \ldots & \ldots & \ldots\\
A\otimes \overline{A}^{(q+1)\otimes}\ar[d]^{b} & \xleftarrow{B} &
A\otimes \overline{A}^{q\otimes}\ar[d]^{b} & \xleftarrow{B} & A\otimes
\overline{A}^{(q-1)\otimes}\ar[d]^{b} 
& \xleftarrow{B} & & \xleftarrow{B} & A\otimes \overline{A}\ar[d]^{b}
& \xleftarrow{B} & A\\
\ldots\ar[d]^{b} & \ldots & \ldots\ar[d]^{b} & \ldots & \ldots
\ar[d]^{b} & \ldots & \ldots & \ldots & A\\
A\otimes \overline{A}^{2\otimes}\ar[d]^{b} & \xleftarrow{B} & A\otimes
\overline{A}\ar[d]^{b} & \xleftarrow{B} & A \\
A & & A 
}
\]




