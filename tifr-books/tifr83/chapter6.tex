\chapter[Cyclic Homology and de Rham Cohomology for...]{Cyclic
  Homology and de Rham Cohomology for Commutative 
  Algebras}\label{chap6} 

THIS\pageoriginale CHAPTER DEALS with the relations between Hochschild
homology and de Rham cohomology for commutative algebras. In the case
of algebras over a field of characteristic zero, we can go further to
prove that the de Rham cohomology groups occur as components in a
direct sum expression for cyclic homology. We begin with a discussion
of differential forms and show how closely related they are to
Hochschild homology. Then we introduce a product structure on
$HH_{\ast}(A)$ in the special case where $A$ is commutative. This
gives us a comparison morphism between graded algebras, and then we
sketch the Hochschild-Kostant-Rosenberg theorem which says that this
morphism is an isomorphism for smooth algebras. We then calculate the
cyclic homology of smooth algebras over a filed of characteristic
zero. This is a case where the first derived couple of the Connes'
exact couple splits and the first differential is the exterior
differential of forms.

Finally, we continue with a discussion of the algebra
$A=C^{\infty}(M)$ of smooth functions on a manifold and prove Connes'
theorem, which says roughly that this smooth case is parallel to the
algebra case.

\section{Derivations and differentials over a commutative
  algebra}\label{chap6-sec1}\index{derivation of a commutative algebra}

In this section, let $A$ denote a commutative algebra over $k$.

\begin{definition}\label{chap6-defi1.1}
Let $M$ be an $A$-module. A derivation $D$ of $A$ with values in $M$
is a $k$-linear map $D:A\to M$ such that
$$
D(ab)=aD(b)+bD(a)\q\text{for}\q a,b\in A.
$$

Let $\Der_{k}(A,M)$ or just $\Der(A,M)$ denote the $k$-module of all
derivations of $A$ with values in $M$.
\end{definition}

The\pageoriginale module $\Der(A,M)$ has a left $A$-module structure
where $cD$ is defined by $(cD)(a)=cD(a)$ for $c$, $a\in A$. For $M=A$
the $k$-module $\Der(A,A)$ has the structure of a Lie algebra over
$k$, where the Lie bracket is given by $[D,D']=DD'-D'D$ for $D$,
$D'\in \Der(A,A)$. A simple check shows that $[D,D']$ satisfies the
derivation rule on products.

\begin{definition}\label{chap6-defi1.2}
The $A$-module of K\"ahler differentials\index{K\"ahler differentials} is a pair,
$(\Omega^{1}_{A/k},d)$ where $\Omega^{1}_{A/k}$, or $\Omega^{1}_{A}$
or simply $\Omega^{1}$, is an $A$-module and $d:A\to \Omega^{1}_{A/k}$
is a derivation such that for any derivation $D:A\to M$, there exists
a unique $A$-linear morphism $f:\Omega^{1}_{A/k}\to M$ with $D=fd$.
\end{definition}

The derivation $d$ defines an $A$-linear morphism
$$
\Hom_{A}(\Omega^{1}_{A/k},M)\to \Der_{k}(A,M)
$$
by assigning to $f\in \Hom_{A}(\Omega^{1}_{A/k},M)$ the derivation
$fd\in \Der_{k}(A,M)$. The universal property is just the assertion
that this morphism is an isomorphism of $A$-modules. The universal
property shows that two possible $A$-modules of differentials are
isomorphic with a unique isomorphism preserving the derivation $d$.

There are two constructions of the module of derivations
$\Omega^{1}_{A/k}$. The first one as the first Hochschild homology
$k$-module of $A$ and the second by a direct use of the derivation
property.

\medskip
\noindent
{\bf Construction of {\boldmath$\Omega^{1}_{A/k}1$}. 1.3.} Let $I$
denote the kernel of the multiplication morphism $\phi(A):A\otimes
A\to A$. To show that
$$
\Omega^{1}_{A/k}=I/I^{2}=HH_{1}(A),
$$
we give $I/I^{2}$ an $A$-module structure by $ax=(1\otimes
a)x=(a\otimes 1)x$, observing that $1\otimes a-a\otimes 1\in I$ and
$(1\otimes a-a\otimes 1)x\in I^{2}$ for $a\in A$, $x\in I$. We define 
$$
d:A\to I/I^{2}\q\text{by}\q d(a)=(1\otimes a-a\otimes
1)\text{mod}\,I^{2}\q\text{for}\q a\in A,
$$
and\pageoriginale check that it is a derivation by
\begin{align*}
d(ab) &= 1\otimes ab-ab\otimes 1\\
&=(1\otimes a)(1\otimes b-b\otimes
1)+(b\otimes 1)(1\otimes a-a\otimes 1)\\
&= ad(b)+bd(a).
\end{align*}

To verify the universal property, we consider a derivation $D:A\to M$,
and note that $f(a\otimes b)=aD(b)$ defined on $A\otimes A$ restricts
to $I$. Since $D(1)=0$, we see that $f(d(a))=f(1\otimes a-a\otimes
1)=D(a)$ or $fd=D$. The uniqueness of $f$ follows from the fact that
$I$, and hence also $I/I^{2}$, is generated by the image of $d$. This
is seen from the following relation,
$$
\sum_{i}a_{i}\otimes b_{i}=\sum_{i}(a_{i}\otimes 1)(1\otimes
b_{i}-b_{i}\otimes 1)=\sum_{i}a_{i}db_{i}
$$
which holds for $\displaystyle{\sum_{i}} a_{i}\otimes b_{i}\in I$ or
equivalently if 
$\displaystyle{\sum_{i}} a_{i}b_{i}=0$ in $A$. Finally, we note that
$f(I^{2})=0$ by 
applying $f$ to $(\displaystyle{\sum_{i}}a_{i}\otimes b_{i})(1\otimes
  c-c\otimes 1)$ 
to obtain 
\begin{align*}
  f\left(\sum_{i}a_{i}\otimes b_{i}c-\sum_{i}a_{i}c\otimes
  b_{i}\right)& =\sum_{i}a_{i}D(b_{i}c)-\sum_{i}a_{i}cD(b_{i})\\
  & =\left(\sum_{i}a_{i}b_{i}\right)D(c)=0.  
\end{align*}

Thus $d:A\to I/I^{2}$ is a module of differentials.

\medskip
\noindent
{\bf Construction of {\boldmath$\Omega^{1}_{A/k}$} II. 1.4.}
Let $L$ be the $A$-submodule of $A\otimes A$ generated by all
$1\otimes ab-a\otimes b-b\otimes a$ for $a$, $b\in A$ where the
$A$-module structure on $A\otimes A$ is given by $c(a\otimes
b)=(ca)\otimes b$ for $c\in A$, $a\otimes b\in A\otimes A$. Next, we
define $d:A\to (A\otimes A)/L$ by $d(b)=(1\otimes b)\text{mod\,}L$ and
from the nature of the generators of $L$, it is clearly a
derivation. Further, if $D\in \Der_{k}(A,M)$, then $f:(A\otimes
A)/L\to M$ defined by $f(a\otimes b\text{mod\,}L)=aD(b)$ is a
well-defined morphism of $A$-modules, and it is the unique one with
the property that $fd=D$.

\setcounter{theorem}{4}
\begin{remark}\label{chap6-rem1.5}
In the first construction, we saw that $\Omega^{1}_{A/k}=HH_{1}(A)$
and in the second construction we see that
$$
\Omega^{1}_{A/k}=\coker(b:C_{2}(A)=A^{\otimes 3}\to A^{\otimes
  2}=C_{1}(A))
$$
in\pageoriginale the standard complex for calculating Hochschild
homology. Now we introduce the algebra of all differential forms in
order to study the higher Hochschild homology modules in terms of
differential forms.
\end{remark}

\begin{definition}\label{chap6-defi1.6}
The algebra of differential forms over an algebra $A$ is the graded
exterior algebra $\Lambda^{*}_{A}\Omega^{1}_{A}$ over $A$, denoted
$\Omega^{*}_{A}$ or $\Omega^{*}_{A/k}$. The elements of
$\Omega^{q}_{A}=\Lambda^{q}_{A}\Omega^{1}_{A}$ are called differential
forms of degree $q$, or simply $q$-forms over $A$.
\end{definition}

A $q$-form is a sum of expressions of the form $a_{0}da_{1}\ldots
da_{q}$ where $a_{0},\ldots,a_{q}\in A$. If $\Omega^{1}_{A}$ is a free
$A$-module with basis $da_{1},\ldots,da_{n}$, then $\Omega^{q}_{A/k}$
has a basis consisting of
$$
da_{i(1)}\ldots da_{i(q)}\q\text{for all}\q i(1)<\ldots<i(q)
$$
as an $A$-module.

\begin{remark}\label{chap6-rem1.7}
The algebra $\Omega^{*}_{A/k}$ is strictly commutative in the graded
sense. This means that
$$
(1)~ \omega_{1}\omega_{2}=(-1)^{pq}\omega_{2}\omega_{1}\q\text{for}\q
\omega_{1}\in \Omega^{p}_{A/k},\omega_{2}\in \Omega^{q}_{A/k} 
$$
(this is commutativity in the graded sense), and
$$
(2)~ \omega^{2}=0\q\text{for}\q \omega\q\text{of odd degree}
$$
(this is strict commutativity).
\end{remark}

Moreover, the exterior algebra is universal for strictly commutative
algebras, in the sense that if $f:M\to H_{1}$ is a $k$-linear morphism
of a $k$-module into the elements of degree $1$ in a strictly
commutative algebra $H$, then there exists a morphism of graded
algebras $h:\Lambda^{*}M\to H$ with the property that
$f=h_{|_{M}}=\Lambda^{1}M\to H^{1}$.

Since $\Omega^{1}_{A/k}\to HH_{1}(A)$ is a natural isomorphism by 
(\ref{chap6-defi1.2}), we wish to define a strictly commutative
algebra structure on $HH_{\ast}(A)$ for any commutative algebra
$A$. We do this in the next section, and before that, we
describe\pageoriginale the exterior derivative which also arises from
the universal property of the exterior algebra.

\begin{proposition}\label{chap6-prop1.8}
There exists a unique morphism $d$ of degree $+1$ defined
$\Omega^{*}_{A/k}\to \Omega^{*}_{A/k}$ satisfying
\begin{enumerate}
\renewcommand{\theenumi}{\alph{enumi}}
\renewcommand{\labelenumi}{\rm(\theenumi)}
\item $d^{2}=0$

\item $d$ is a derivation of degree $+1$, that is,
{\fontsize{10}{12}\selectfont
$$
d(\omega_{1}\omega_{2})=(d\omega_{1})\omega_{2}+(-1)^{p}\omega_{1}(d\omega_{2})\q\text{for}\q
\omega_{1}\in \Omega^{p}_{A/k},\omega_{2}\in \Omega^{q}_{A/k}. 
$$}

\item $d$ restricted to $A=\Omega^{0}$ is the canonical derivation
  $d:A\to \Omega^{1}$.
\end{enumerate}
\end{proposition}

\begin{proof}
The uniqueness follows from the relation
$$
d(a_{0}d_{a_{1}}\ldots da_{q})=da_{0}da_{1}\ldots da_{q}
$$
since the elements $a_{0}d_{a_{1}}\ldots da_{q}$ generate
$\Omega^{q}_{A}=\Lambda^{q}\Omega^{1}_{A}$, and the existence is
established with this formula.
\end{proof}

\begin{definition}\label{chap6-defi1.9}
For an algebra $A$ over $k$, the complex $(\Omega^{*}_{A/k},d)$ is
called the de-Rham complex of $A$, and the cohomology algebra
$H^{*}(\Omega^{*}_{A/k},d)$, denoted $H^{*}_{DR}(A)$, is called the de
Rham cohomology of $A$ over $k$.
\end{definition}

\section{Product structure on $HH_{\ast}(A)$}\label{chap6-sec2}

The basis for a product structure is usually a K\"unneth morphism and
a K\"unneth theorem which says when the morphism is an
isomorphism\index{K\"unneth morphism and isomorphism}. The K\"unneth morphism usually comes from the morphism
$\alpha$ for the homology of a tensor product $X_{\bigdot}\otimes
Y_{\bigdot}$ of two complexes.

\begin{definition}\label{chap6-defi2.1}
Let $X_{\bigdot}$ and $Y_{\bigdot}$ be two complexes of
$k$-modules. The tensor K\"unneth morphism is
$$
\alpha:H_{\bigdot}(X_{\bigdot})\otimes H_{\bigdot}(Y_{\bigdot})\to
H_{\bigdot}(X_{\bigdot}\otimes Y_{\bigdot})
$$
defined by the relation $\alpha(u\otimes v)=w$ where $u\in H_{p}(X)$
is represented by $x\in X_{p}$, $v\in H_{q}$ is represented by $y\in
Y_{q}$ and $w$ is represented by $x\otimes y\in (X\otimes Y)_{p+q}$.
\end{definition}


If\pageoriginale $k$ is a field, then $\alpha$ is always an
isomorphism. Under the assumption that $X_{\bigdot}$ and $Y_{\bigdot}$
are flat over $k$, it follows that $\alpha$ is an isomorphism if
either $H_{\bigdot}(X_{\bigdot})$ or $H_{\bigdot}(Y_{\bigdot})$ is
flat over $k$.

\begin{remark}\label{chap6-rem2.2}
Let $B$ and $B'$ be two algebras over $k$. If $L$ is a right
$B$-module and $L'$ a right $B'$-module, then $L\otimes L'$ is a right
$B\otimes B'$ module, and if $M$ is a left $B$-module and $M'$ a left
$B'$-module, then $M\otimes M'$ is a left $B\otimes B'$-module. Using
the natural associativity and commutativity isomorphisms for the
tensor product over $k$, we have a natural isomorphism 
$$
\theta:(L\otimes_{B}M)\otimes (L'\otimes_{B'}M')\to (L\otimes
L')_{B\otimes B'}(M\otimes M').
$$

If $P_{\bigdot}\to L$ is a projective resolution of $L$ over $B$, and
if $P'_{\bigdot}\to L'$ is a projective resolution of $L'$ over $B'$,
then $P_{\bigdot}\otimes P'_{\bigdot}\to L\otimes L'$ is a projective
resolution of $L\otimes L'$ over $B\otimes B'$. This assertion holds
in either the absolute projective or $k$-split projective
cases. Combining the isomorphism of complexes
$$
(P_{\bigdot}\otimes_{B}M)\otimes (P'_{\bigdot}\otimes_{B'}M')\to
(P_{\bigdot}\otimes P')\otimes_{B\otimes B'}(M\otimes M')
$$
with the K\"unneth morphism of (\ref{chap6-defi2.1}), we obtain the
following: 
\end{remark}

\noindent
{\bf K\"unneth morphism for Tor 2.3.}\index{K\"unneth morphism for Tor} Let $B$ and $B'$ be two algebras
with modules $L$ and $M$ over $B$ and $L'$ and $M'$ over $B'$. The
isomorphism $\theta$ extends to a morphism of functors 
$$
\alpha:\Tor^{B}_{\ast}(L,M)\otimes \Tor^{B'}_{\ast}(L',M')\to
\Tor^{B\otimes B'}_{\ast}(L\otimes L',M\otimes M')
$$
which we call the K\"unneth morphism for the $\Tor$ functor. This
morphism is defined for both the absolute and $k$-split $\Tor$
functors.

Let $A$ and $A'$ be two algebras, and form the algebras
$A^{e}=A\otimes A^{op}$ and ${A'}^{e}=A'\otimes {A'}^{op}$. There is a
natural commuting isomorphism $(A\otimes A')^{e}\to A^{e}\otimes
{A'}^{e}$ which we combine with the K\"unneth morphism for the $\Tor$
to obtain: 

\medskip
\noindent
{\bf K\"unneth\pageoriginale morphism for Hochschild homology
  2.4.}\index{K\"unneth morphism for Hochschild homology}  
Let $M$ be an $A$-bimodule, and let $M'$ be an $A'$-bimodule. A
special case of the K\"unneth morphism for $\Tor$ is
$$
\alpha:H_{\ast}(A,M)\otimes H_{\ast}(A',M')\to H_{\ast}(A\otimes
A',M\otimes M')
$$
called the K\"unneth morphism for Hochschild homology. In particular,
we have $\alpha:HH_{\ast}(A)\otimes HH_{\ast}(A')\to
HH_{\ast}(A\otimes A')$. 

\setcounter{theorem}{4}
\begin{definition}\label{chap6-defi2.5}
The K\"unneth\_morphisms for $\Tor$ and for Hochschild homology
satisfy associativity, commutativity, and unit properties which we
leave to the reader to formulate. If $k$ is a field, then the
K\"unneth morphism is an isomorphism.
\end{definition}

We are now ready to define the product structure $\phi(HH_{\ast}(A))$
on $HH_{\ast}(A)$ when $A$ is commutative. Recall that an algebra $A$
is commutative if and only if the structure morphism is a morphism of
algebras $A\otimes A\to A$.

\begin{definition}\label{chap6-defi2.6}
For a commutative $k$-algebra $A$ the multiplication\break 
$\phi(HH_{\ast}(A))$ on $HH_{\ast}(A)$ is the composite
$HH_{\ast}(\phi(A))\alpha$ defined by
$$
HH_{\ast}(A)\otimes HH_{\ast}(A)\to HH_{\ast}(A\otimes A)\to
HH_{\ast}(A).
$$
\end{definition}

From the above considerations $HH_{\ast}(A)$ is an algebra which is
commutative over $A=HH_{0}(A)$ in the graded sense.

\begin{remark}\label{chap6-rem2.7}
Let $B\to A$ be an augmentation of the commutative algebra $B$. If
$K_{\ast}\to A$ is a $B$-projective resolution of $A$ such that
$K_{\ast}$ is a differential algebra and $K_{\ast}\to A$ is a morphism
of differential algebras, then we have the following morphisms
$$
(A\otimes_{B}K_{\ast})\otimes (A\otimes_{B}K_{\ast})\to (A\otimes
A)\otimes_{B\otimes B}(K_{\ast}\otimes K_{\ast})\to
A\otimes_{B}K_{\ast}
$$
where the first is a general commutativity isomorphism for the tensor
product and the second is induced by the algebra structures on $A$,
$B$ and $K_{\ast}$. If\pageoriginale the composite is denoted by
$\psi$, then the algebra structure on $\Tor^{B}(A,A)$ is the K\"unneth
morphism composed with $H(\psi)$ in
$$
H(A\otimes_BK_{\ast})\otimes H(A\otimes_{B}K_{\ast})\to
H((A\otimes_{B}K_{\ast})\otimes (A\otimes_{B}K_{\ast}))\to
H(A\otimes_{B}K_{\ast}). 
$$
\end{remark}

\begin{remark}\label{chap6-rem2.8}
There is a natural $A$-morphism of the abelianization of the tensor
algebra $T(HH_{1}(A))$ on $HH_{1}(A)$, viewed as a graded algebra over
$A=HH_{0}(A)$ with $HH_{1}(A)$ in degree 1 defined
$T_{A}(HH_{1}(A))^{ab}\to HH_{\ast}(A)$. This is a morphism of
commutative algebras. Since the square of every element in $HH_{1}(A)$
is zero, we have in fact a morphism of the exterior algebra on
$HH_{1}(A)$ into $HH_{\ast}(A)$, 
$$
\psi(A):\Lambda_{A}(HH_{1}(A))\to HH_{\ast}(A).
$$

Note that if $k$ is a field of characterisitic different from 2, then
the natural algebra morphism $T_{A}(X)^{ab}\to \Lambda_{A}(X)$ is an
isomorphism when $X$ is graded, with nonzero terms in odd degrees.
\end{remark}

In this chapter we will show that $\psi(A)$ is an isomorphism, for a
large class of algebras $A$ which arise in smooth geometry.

We conclude by mentioning another way of defining the product on
$HH_{\ast}(A)$ by starting with a product, called the shuffle
product\index{shuffle product},
on the simplicial object $C_{\ast}(A)$. In the commutative case
$C_{\ast}(A)$ is a simplicial $k$-algebra, i.e.\@ each $C_{q}(A)$ is a
$k$-algebra and the morphisms $d_{i}$ and $s_{j}$ are morphisms of
algebras. 

\begin{definition}\label{chap6-defi2.9}
Let $R_{\bigdot}$ be a simplicial $k$-algebra. The shuffle product
$R_{p}\otimes R_{q}\to R_{p+q}$ is defined by the following sum for
$\alpha\in R_{p}$, and $\beta\in R_{q}$,
$$
\alpha_{\bigdot}\beta=\sum_{\mu,\nu}\epsilon(\mu,\nu)(s_{\mu}(\alpha)(s_{\nu}(\beta)\q\text{in}\q R_{p+q}
$$
where $\mu$, $\nu$ is summed over all $(q,p)$ shuffle permutations of
$[0,\ldots,p+q-1]$ of the form
$(\mu_{1},\ldots,\mu_{q},\nu_{1},\ldots,\nu_{p})$ where
$\mu_{1}<\ldots<\mu_{q}$ and
$\nu_{1}<\ldots<\nu_{p}$. Also\pageoriginale $\epsilon(\mu,\nu)$
denotes the sign of the permutation $\mu$, $\nu$, and the iterated
operators are
$$
s_{\mu}(\alpha)=s_{\mu_{q}}(\ldots(s_{\mu_{1}}(\alpha))\ldots)\q\text{and}\q
s_{\nu}(\beta)=s_{\nu_{p}}(\ldots(s_{\nu_{1}}(\beta))\ldots). 
$$
\end{definition}

\begin{remark}\label{chap6-rem2.10}
With the shuffle product on a simplicial $k$-algebra $R_{\bigdot}$,
the differential module $(R_{\bigdot},d)$ becomes a differential
algebra over $k$. If $R_{\bigdot}$ is a commutative simplicial
algebra, then $(R_{\bigdot},d)$ is a commutative differential
algebra. This applies to $HH_{\ast}(A)$ for a commutative algebra $A$,
and again we obtain a natural morphism
$$
\Lambda^{*}HH_{1}(A)\to HH_{\ast}(A).
$$
\end{remark}

\begin{example}\label{chap6-exam2.11}
For $\alpha=(a,x)$, $\beta=(a',y)$ the shuffle product is
\begin{gather*}
\alpha_{\bigdot}\beta=(s_{0}\alpha)\cdot(s_{1}\beta)-(s_{1}\alpha)(s_{0}\beta)=(a,x,1)(a',1,y)-(a,1,x)(a',y,1)\\
=(aa',x,y)-(aa',y,x).
\end{gather*}

For $\alpha_{j}=(a_{j},x_{j})$ where $j=1,\ldots,p$ this formula
generalizes to 
$$
\alpha_{1}\ldots\alpha_{p}=\sum_{\alpha\in
  \Sym_{p}}sgn(\alpha)(a_{1}\ldots
a_{p},x_{\alpha_{(1)}},\ldots,x_{\alpha_{(p)}}). 
$$
\end{example}

\section{Hochschild homology of regular algebras}\label{chap6-sec3}

In this section we outline the proof that Hochschild homology is just
the K\"ahler differential forms for a regular $k$-algebra $A$, i.e.\@
that $HH_{q}(A)$ is isomorphic to $\Omega^{q}_{A/k}$. We start with
some background from commutative algebra.

\begin{definition}\label{chap6-defi3.1}
A sequence of elements\index{regular sequences of elements} $y_{1},\ldots,y_{d}$ in a commutative
$k$-algebra $B$ is called regular provided the image of $y_{i}$ in the
quotient algebra $B/B(y_{1},\ldots,y_{i-1})$ is not a zero divisor.
\end{definition}

Let\pageoriginale $K(b,B)$ denote the exterior differential algebra on
one generator $x$ in degree $1$ with boundary $dx=b\in
B=K(b,B)_{0}$. If $y_{1},\ldots,y_{d}$ is a regular sequence of
elements, then
$$
K(y_{i},B/B(y_{1},\ldots,y_{i-1})\to B/B(y_{1},\ldots,y_{i})
$$
is a free resolution of $B/B(y_{1},\ldots,y_{i})$ by
$B(y_{1},\ldots,y_{i-1})$-modules. 

\begin{notation}\label{chap6-not3.2}
Let $B$ be a commutative algebra, and let $b_{1},\ldots,b_{m}$ be
elements of $B$. We denote by $K(b_{1},\ldots,b_{m})$ the differential
algebra which is the tensor product
$$
K(b_{1},\ldots,b_{m};B)=K(b_{1},B)\otimes_{B}\ldots
\otimes_{B}K(b_{m},B).
$$

This algebra is zero in degrees $q>m$ and $q<0$ and free of rank
$\binom{n}{q}$ in degree $q$, further the differential on a basis
element is given by
$$
d(x_{k(1)}\wedge\ldots\wedge x_{k(q)})=\sum_{1\leq i\leq
  1}(-1)^{i-1}b_{i}(x_{k(1)}\wedge\ldots\wedge
x_{k(i)}\wedge\ldots\wedge x_{k(q)}),
$$
and the augmentation is defined by $K(b_{1},\ldots,b_{m};B)\to
B/B(b_{1},\ldots,b_{m})$. Filtering $K(B_{1},\ldots,b_{m};B)$ in two
steps with respect to degrees of $K(b_{m},\break B)$, and looking at the
associated spectral sequence, we obtain immediately the following
proposition. 
\end{notation}

\begin{proposition}\label{chap6-prop3.3}
For $b_{1},\ldots,b_{m}$ a sequence of elements in a commutative
algebra $B$ the augmentation morphism induces an isomorphism
$$
H_{0}(K(b_{1},\ldots,b_{m};B)\to B/B(b_{1},\ldots,b_{m}).
$$

If $b_{1},\ldots,b_{m}$ is a regular sequence, then the augmentation
morphism induces isomorphisms $H_{0}(K(b_{1},\ldots,b_{m};B))\to
B/B(b_{1},\ldots,b_{m})$ and 
$$
K(b_{1},\ldots,b_{m};B)\to B/B(b_{1},\ldots,b_{m}).
$$
\end{proposition}

This resolution is called the {\bf Koszul resolution}\index{Koszul
  resolution} of the quotient of $B$ by free $B$-modules. 

\begin{definition}\label{chap6-defi3.4}
An\pageoriginale ideal $J$ in a commutative $k$-algebra $B$ is said to
be regular if it is generated by a regular sequence. An algebra $A$ is
$\phi$-regular provided the kernel $I$ of $\phi(A):A\otimes A\to A$ is
regular in the algebra $B=A\otimes A$.
\end{definition}

The next theorem is the first case where we identify the Hochschild
homology of a commutative algebra as the exterior algebra on the first
Hochschild homology module.

\begin{theorem}\label{chap6-thm3.5}
If $A$ is a commutative $\phi$-regular algebra, then the natural
morphisms of algebras
$$
\Lambda_{A}(HH_{1}(A))\to HH_{\ast}(A)
$$
or equivalently
$$
\Lambda^{*}_{A}(I/I^{2})=\Omega^{*}_{A/k}\to HH_{\ast}(A)
$$
is an isomorphism of graded commutative algebras.
\end{theorem}

\begin{proof}
By (\ref{chap6-rem1.5}) we have the natural isomorphisms between
$HH_{1}(A)$, $I/I^{2}$ and $\Omega^{1}_{A/k}$. By hypothesis for
$B=A\otimes A$ the previous proposition (\ref{chap6-defi3.4}) applies
and we have a resolution of $A=B/I$ by a differential algebra of free
$B$-modules $K_{\ast}=K(b_{1},\ldots,b_{m};B)\to A$, such that the
augmentation morphism is a morphism of algebras. Hence
$HH_{\ast}(A)=H_{\ast}(K(b_{1},\ldots,b_{m};B)\otimes_{B}A$ since the
coefficients in the formula of (\ref{chap6-not3.2}) are in $I$ and the
resulting algebra over $A$ is the exterior algebra on $I/I^{2}$. This
proves the theorem. 
\end{proof}

\begin{remark}\label{chap6-rem3.6}
The hypothesis of being a $\phi$-regular algebra\index{regular
  algebras and ideals} is rather restricted,
except in the local case where it is equivalent to the maximal ideal
being generated by a regular sequence. This means that the above
construction applies to a regular local algebra, i.e.\@ a local
algebra whose maximal ideal is generated by a regular sequence.
\end{remark}

\begin{definition}\label{chap6-defi3.7}
An algebra $A$ over a field $k$ is regular provided each localisation
$A_{\mathfrak{P}}$ at a prime ideal $\mathfrak{P}$ is regular.
\end{definition}

These\pageoriginale are the algebras with the property that their
Hochschild homology is the algebra of differential forms. This leads
to the theorem of Hochschild, Kostant and Rosenberg\index{Hochschild,
  Kostant, and Rosenberg theorem}.

\begin{theorem}\label{chap6-thm3.8}
The natural morphism of graded commutative algebras
$\Omega^{*}_{A/k}\to HH_{\ast}(A)$ is an isomorphism for a regular
algebra $A$ over a field $k$. 
\end{theorem}

\begin{proof}
For each prime ideal $\mathfrak{P}$ in $A$, the localisation of this
morphism in the statement of the theorem
$$
\Omega^{*}_{A_{\mathfrak{P}}/k}=(\Omega^{*}_{A/k})_{\mathfrak{P}}\to
HH_{\ast}(A)_{\mathfrak{P}}=HH_{\ast}(A_{\mathfrak{P}})
$$
is an isomorphism by (\ref{chap6-thm3.5}). Hence the morphism is an
isomorphism by a generality about localisation at each prime
ideal. This proves the main theorem of this section.
\end{proof}

\section{Hochschild homology of algebras of smooth
  functions}\label{chap6-sec4}\index{smooth manifolds, functions, and forms}

In this section we outline the proof that Hochschild homology is just
the algebra of differential forms for an algebra $A$ of smooth complex
valued functions on a smooth manifold $X$.

\begin{remark}\label{chap6-rem4.1}
Let $X$ be a smooth $n$-dimensional manifold, and $A=\break C^{\infty}(X)$
denote the algebra of smooth complex valued functions on $X$. Then the
Lie algebra of derivations $\Der_{C}(C^{\infty}(X))$ is just the space
of smooth vector fields on $X$ with complex coefficients, and
$\Omega^{1}_{A/C}=A^{1}(X)$ is the $A$-module of $1$-forms and
$A^{q}(X)$ is the $A$-module of $q$-forms on $X$. This means that
$HH_{1}(A)=A^{1}(X)$, by the characterization of $HH_{1}(A)$ in terms
of K\"ahler $1$-forms of a commutative algebra. We will outline the
proof that $HH_{q}(A)=A^{q}(X)$, the module of $q$-forms over
$A=A^{0}(X)$, the algebra of smooth functions on $X$. Thus we have the
same calculation in degree $1$, and following the lead from the
previous section, we see that there must be a resolution of the ideal
$\ker(A^{0}(X)\otimes A^{0}(X)\to A^{0}(X))$. This we do by relating
this multiplication with $A^{0}(X\times X)\to A^{0}(X)$ coming from
restriction\pageoriginale to the diagonal. Observe that there is an
embedding
$$
A^{0}(X)\otimes A^{0}(X)\to A^{0}(X\times X)
$$
given by assigning to a tensor product of functions, a function of two
variables and then using the normal bundle to the diagonal in $X\times
X$. The result corresponding to the $\phi$-regular algebra
construction is the following proposition.
\end{remark}

\begin{remark}\label{chap6-rem4.2}
Let $E\to Y$ be a complex vector bundle with dual bundle $E\sphat$. If
$s\in \Gamma(Y,E)$ is a cross section of $E$, then its inner product
with an element of a fibre of $E\sphat$ defines a scalar varying from
fibre to fibre. We define a morphism $s^{\vdash}:E\sphat\to
\Lambda^{0}E\sphat$, the trivial bundle. This $s^{\vdash}$ extends to
a complex
$$
\ldots\xrightarrow{s^{\vdash}}\Lambda^{2}E\sphat\xrightarrow{s^{\vdash}}\Lambda^{1}E\sphat
\xrightarrow{s^{\vdash}}\Lambda^{0}E\sphat\to 0
$$
which is exact at all points where $s\neq 0$.
\end{remark}

Now assume that $Y$ is a smooth manifold, $E$ is a smooth vector
bundle, and $X$, the set of zeros of $s$ is transverse to the zero
section, and that the tangent morphism $ds_{y}:T_{y}Y\to E_{y}$ is
surjective. Then $X$ is a submanifold of $Y$ of codimension $q$ where
$q=\dim E$ and the normal bundle to the zero set $X$ in $Y$ is
isomorphic to $E|_{X}$.

\begin{proposition}\label{chap6-prop4.3}
With the above notations the complex of Fr\'echet spa\-ces
\begin{gather*}
R(Y,E):\ldots\to \Gamma(Y,\Lambda^{q}E\sphat)\qquad
\xrightarrow{s^{\vdash}}\Gamma(Y,\Gamma^{q-1}E\sphat)\to\ldots\\
\ldots\xrightarrow{s^{\vdash}}\Gamma(Y,\Lambda^{1}E\sphat)\xrightarrow{s^{\vdash}}
\Gamma(Y)\xrightarrow{res}\Gamma(X)\to 0
\end{gather*}
is contractible.
\end{proposition}

\begin{proof}
The first step is to show that if the result holds locally, then it
holds globally. Let $\displaystyle{Y=\bigcup_{i\in I}U_{i}}$ be an open covering with a
smooth partition of unity $\displaystyle{\sum_{i\in I}\eta_{i}=1}$ where
$U_{i}\supset$ closure of $\eta^{-1}_{1}((0,1])$ and
  $R(U_{i},E_{|_{U_{i}}})$ is contractible\pageoriginale with
  contracting homotopy $h_{i}$ for each $i\in I$. For $\pi:E\to Y$ the
  complex $R(Y,E)$ has a retracting homotopy
$$
h(x)=\sum_{i\in I}\eta_{i}(\pi(x))h_{i}(x|_{U_{i}}).
$$

If $N$ is the normal bundle of $X$ in $Y$, then the induced tangent
mapping $ds_{x}:N_{x}\to E_{x}$ is an isomorphism by the
transversality hypothesis. Thus locally the bundle is of the form 
$$
\bfR^{\bfq}\times \bfR^{\bfq}\times \bfR^{\bfp}=T(\bfR^{\bfq})\times
\bfR^{\bfp}\to \bfR^{\bfq}\times \bfR^{\bfp}
$$
with projection from the middle $\bfR^{\bfq}$ coordinate or
$T(\bfR^{\bfq})\to \bfR^{\bfq}$ with parameters from $\bfR^{\bfp}$.
\end{proof}

\begin{remark}\label{chap6-rem4.4}
For a submanifold $X$ of $Y$ and a smooth bundle\index{smooth bundles} $E$ over $Y$, the
restriction from the space of cross sections induces an isomorphism
$\Gamma(X)\Omega_{\Gamma(Y)}\Gamma(Y,E)\to \Gamma(X,E_{|_{X}})$.
\end{remark}

\begin{theorem}\label{chap6-thm4.5}
For a smooth manifold we have a natural isomorphism
$HH_{q}(A^{0}(X))\to \Gamma(X,\Lambda^{q}T^{\ast}(X))=A^{q}(X)$. 

We do not give a proof of this theorem here see Connes [1985].
\end{theorem}

\section[Cyclic homology of regular algebras and...]{Cyclic homology
  of regular algebras and smooth 
  manifolds}\label{chap6-sec5} 

We calculate the cyclic homology by comparing the basic standard
complex with the complex of differential forms. For this, we consider
a basic morphism from the standard complex to the complex of
differential forms and study to what extent it is a morphism of mixed
complexes. 

\begin{notation}\label{chap6-not5.1}
The morphism $\mu$ is defined in two situations:
\begin{enumerate}
\renewcommand{\labelenumi}{(\theenumi)}
\item Let $A$ be a commutative algebra over a field $k$ of
  characteristic zero. Denote by $\mu:A^{(q+1)\otimes}\to
  \Omega^{q}_{A/k}$ defined by
$$
\mu(a_{0}\otimes\ldots\otimes a_{q})=(1/q!)a_{0}da_{1}\ldots da_{q}. 
$$

\item Let\pageoriginale $X$ be a smooth manifold. Denote by
  $\mu:A^{0}(X^{q+1})\to A^{q}(X)$ defined by
$$
\mu(f(x_{0},\ldots,x_{q}))=(1/q!)\Delta^{*}(fd_{1}f\ldots d_{q}f)
$$
where $d_{i}f(x_{0},\ldots,x_{q})$ is the differential of $f$ along
the $x_{i}$ variable in $X^{q+1}$ and $\Delta:X\to X^{q+1}$ is the
diagonal map.
\end{enumerate}
\end{notation}

\begin{remark}\label{chap6-rem5.2}
Both $A^{q+1}\otimes$ and $A^{0}(X^{q+1})$ are the terms of degree $q$
of cyclic vector spaces and hence the operators $b$ and $B$ are
defined. Under the morphism $\mu$ we have the following result.
\end{remark}

\begin{proposition}\label{chap6-prop5.3}
We have, with the above notations
$$
\mu b=0\q\text{and}\q \mu B=d\mu
$$
where $d$ is the exterior differential on differential forms.
\end{proposition}

\begin{proof}
Given $a_{0}\otimes\cdots\otimes a_{q}\in A^{(q+1)\otimes}$, we must
show that the following sum of differentials is zero,
{\selectfont{\fontsize{10}{8}{
$$
a_{0}a_{1}da_{2}\ldots da_{q}+\sum_{0<i<q}(-1)^{i}a_{0}da_{1}\ldots
d(a_{i}a_{i+1})\ldots da_{q}+(-1)^{q}a_{q}a_{0}da_{1}\ldots da_{q+1}.
$$}}}

A direct check shows that terms with coefficients $a_{0}a_{i}$ come in
pairs with opposite signs. Hence $\mu b=0$. Since $\mu$ factors
through $A\otimes \overline{A}^{q\otimes}$, we can calculate by
5(4.1),
\begin{align*}
(\mu B)(a_{0}\otimes\ldots\otimes a_{q}) &= \mu(\sum_{0\leq i\leq
    q}(-1)^{iq}(1\otimes a_{i}\otimes\ldots\otimes a_{q}\otimes
  a_{0}\otimes \ldots \otimes a_{i+1})\\
&= (1/(q+1)!)(q+1)da_{0}\ldots da_{q}\\
&= (1/q!)d(a_{0}da_{1}\ldots da_{q})\\
&= d\mu(a_{0}da_{1}\ldots da_{q}).
\end{align*}

This shows that $\mu B=d\mu$. The above calculation works also for
$\mu$ in the smooth manifold case. This proves the proposition.
\end{proof}

\begin{remark}\label{chap6-rem5.4}
The above morphism $\mu$ induces a morphism
$$
\mu:HH_{q}(A)\to \Omega^{q}_{A/k}
$$
which\pageoriginale when composed with the natural
$\Omega^{q}_{A/k}\to HH_{q}(A)$ on the right is multiplication by
$q+1$ on $\Omega^{q}_{A/k}$. Thus $\mu$ is a morphism of mixed
complexes
$$
\mu:(C_{\ast}(A),b,B)\to \Omega^{\ast}(A/k,0,d)
$$
which induces an isomorphism $HH_{\ast}(A)\to \Omega^{*}_{A/k}$. Thus
the mixed complex of differential forms $(\omega^{*}_{A/k},0,d)$ can
be used to calculate the cyclic homology of $A$ or $A^{0}(X)$.
\end{remark}

\begin{theorem}\label{chap6-thm5.5}
Let $A$ be a regular $k$-algebra over a field $k$ of characteristic
zero. Then the cyclic homology is given by
$$
HC_{p}(A)=\Omega^{p}_{A/k}/d\Omega^{p-1}_{A/k}\oplus
H^{p-2}_{DR}(A)\oplus H^{p-4}_{DR}(A)\oplus\ldots
$$

Let $A$ be the $\bfC$-algebra of smooth functions on a smooth
manifold. Then the cyclic homology is given by
$$
HC_{p}(A)=A^{p}(X)/dA^{p-1}(X)\oplus H^{p-2}_{DR}(X)\oplus
H^{p-4}_{DR}(X)\oplus\ldots 
$$

In both cases, the projection of $HC_{p}(A)$ onto the first term is
induced by $\mu$ and in the Connes' exact sequence, we have:
\begin{enumerate}
\renewcommand{\labelenumi}{\rm\theenumi.}
\item $I:HH_{p}(A)\to HC_{p}(A)$ is the projection of $HH_{p}(A)$, the
  $p$-forms, onto the first factor of $HC_{p}(A)$,

\item $S:HC_{p}(A)\to HC_{p-2}(A)$ is injection of the first factor of
  $HC_{p}(A)$ into the second factor $H^{p-2}_{DR}$ and the other
  factors map isomorphically on the corresponding factor of
  $HC_{p-2}$.

\item $B:HC_{p-2}(A)\to HH_{p-1}(A)$ is zero on all factors except the
  first one where it is $d:\Omega^{p-2}/d\Omega^{p-3}\to \Omega^{p-1}$.
\end{enumerate}

Finally in the first derived couple of the Connes' exact couple we
have $B=0$ and the exact couple is the split exact sequence
$$
0\to HH_{p}\to HH_{p}\oplus HH_{p-2}\oplus HH_{p-4}\oplus\ldots\to
HH_{p-2}\oplus HH_{p-4}\oplus\ldots\to 0.
$$
\end{theorem}

\begin{proof}
Everything\pageoriginale in this theorem follows from the fact that we
can calculate cyclic homology, Hochschild homology, and the Connes'
exact couple with the mixed complex $(\Omega,0,d)$ and it is an easy
generality on mixed complexes with the first differential zero.
\end{proof}

\section{The Chern character in cyclic
  homology}\label{chap6-sec6}\index{Chern character}

Recall that for topological $K$-theory, we have a ring homomorphism
$$
ch:K(X)\to H^{ev}(X,\bfQ)
$$
such that $ch\otimes\bfQ$ is an isomorphism. Here the superscript $ev$
denotes the homology groups of even degree. We wish to define a
sequence of morphisms $ch_{m}:K_{0}(A)\to HC_{2m}(A)$ for all $m$ such
that $S(ch_{m})=ch_{m-1}$, in terms of $S:HC_{2m}(A)\to
HC_{2m-2}(A)$. In this section $k$ is always a field of characteristic
zero. 

\begin{remark}\label{chap6-rem6.1}
$K$-theory is constructed from either vector bundles over a space or
  from finitely generated projective modules over a ring. The vector
  bundles under consideration are always direct summands of a trivial
  bundle. In either case, it is a direct summand which is represented
  by an element $e=e^{2}$ in a matrix ring $M_{r}(A)$ over $A$. Here
  $A$ is an arbitrary ring or the algebra of either the continuous
  functions on the space or of smooth functions on a smooth base
  manifold. Our approach to the Chern character is motivated by
  differential geometry where a differential form construction of the
  Chern character is made from $e$. The choice of $e=e^{2}$ is not
  uniquely defined by the element of $K$-theory but it amounts to the
  choice of a connection on a vector bundle.
\end{remark}

\begin{proposition}\label{chap6-prop6.2}
If $e=e^{2}\in M_{r}(A)$ for a commutative ring, then in
$M_{r}\Omega^{1}_{A/k}$ we have the relations
$$
e(de)=de(1-e)\q\text{and}\q (de)e=(1-e)de.
$$

In\pageoriginale particular, $e(de)e=0$ and $e(de)^{2}=(de)^{2}e$
where $M_{r}(A)$ acts on $M_{r}\Omega^{1}_{A/k}$ by matrix
multiplication of a matrix valued form with a matrix valued function
on either side.
\end{proposition}

\begin{proof}
We calculate $de=d(e^{2})=e(de)+(de)e$ and use this to derive the
relations immediately.
\end{proof}

\begin{remark}\label{chap6-rem6.3}
For $e=e^{2}\in M_{r}(A)$ we denote by $\Gamma(E)=\Iim(e)\subset
A^{r}$ where we think of $\Gamma(E)$ as the cross sections of the
vector bundle $E$ corresponding to $e$. The related connection is
$D(s)=eds$ for $s\in \Gamma(E)$ where $eds\in \Gamma(E\otimes
\Omega^{1}_{A/k})$, and the curvature is
$$
D^{2}s=ed(eds)=e(de)^{2}_{s}.
$$

In order to see how the second formula follows from the first, we
calculate
$$
ed(eds)=ededs=eded(es)=ede(de)s+e(de)eds=e(de)^{2}s.
$$

Thus the curvature is given by $D^{2}=e(de)^{2}$ and this means that
$$
(D^{2})^{q}=e(de)^{2}\ldots e(de)^{2}=e(de)^{2q}
$$
which leads to the following definition by analogy with classical
differential geometry.
\end{remark}

\begin{definition}\label{chap6-defi6.4}
The Chern character form of $e=e^{2}\in M_{r}(A)$ with curvature
$D^{2}=e(de)^{2}$ is given by the sum
$$
ch(e)=tr(e^{D^{2}})=\sum_{q\geq 0}(1/q!)tr(e(de)^{2q}).
$$

Now we will see how this Chern character form defines a class in
cyclic homology. The guiding observation is the fact that up to a
scalar, $tr(e(de)^{2q})$ is $\mu(tr(e^{(2q+1)\otimes}))$ where $\mu$
was introduced in (\ref{chap6-not5.1}) of the previous section. We
have two preliminary results in the cyclic homology complex.
\end{definition}

\begin{proposition}\label{chap6-prop6.5}
Let\pageoriginale $A$ be an algebra over a field $k$. For an element $a\in A$ and
$a^{(q+1)\otimes}\in C_{q}(A)$ in the standard complex, we have
\begin{align*}
(t-1)(a^{(q+1)\otimes}) &= -2a^{(q+1)\otimes}\text{~ for~ } q\text{~
    odd}\\
&= 0\text{~ for~ } q\text{~ even.}
\end{align*}

For $e=e^{2}\in A$ and $e^{(q+1)\otimes}\in C_{q}(A)$ we have the
relation
\begin{align*}
b(e^{(q+1)\otimes}) &= e^{q\otimes}\text{~ for~ }q \text{~ even}\\
&=0\text{~ for~ }q\text{~ odd.}
\end{align*}
\end{proposition}

\begin{proof}
The first formula follows from the relation
$t(a^{(q+1)\otimes})=(-1)^{q}\break a^{(q+1)\otimes}$. Since $e=ee$ the sum
$b(e^{(q+1)\otimes})$ is an alternating sum of $q+1$ terms
$e^{q\otimes}$, and they either cancel to yield zero or reduce to
$e^{q\otimes}$. This proves the proposition.
\end{proof}

\begin{corollary}\label{chap6-coro6.6}
If $e=e^{2}\in M_{r}(A)$, then the boundary
$$
b(tr(e^{(2q+1)\otimes}))=0\text{~ in~ } C_{2q-1}(A)/\Iim(1-t).
$$

Thus $tr(e^{(2q+1)\otimes})$ defines a class $ch_{q}(e)\in
HC_{2q}(A)$, for $e=e^{2}\in M_{r}(A)$ and this is the Chern character
form upto a scalar factor. This was the aim of this section, and we
finish with the following summary assertion.
\end{corollary}

\begin{theorem}\label{chap6-thm6.7}
Let $e=e^{2}\in M_{r}(A)$ with Chern character form
$ch_{q}(e)=(1/q!)tr(e(de)^{2q})$ in degree $2q$. Then in degree $2q$
we have
$$
\mu(ch_{q}(e))=ch_{q}(e)\text{~ in ~ } HC_{2q}(A).
$$

Moreover, under $S:HC_{2q}(A)\to HC_{2q-2}(A)$, we have for this Chern
character class, $S(ch_{q}(E))=ch_{q-1}(E)$.
\end{theorem}




