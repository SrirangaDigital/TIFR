\chapter{Abelianization and Hochschild Homology}\label{chap2}

IN\pageoriginale THIS CHAPTER we first consider abelianization in the
contexts of associative algebras and Lie algebras together with the
adjunction properties of the related functors. In degree zero,
Hochschild and cyclic homology of an algebra $A$ are isomorphic and
equal to a certain abelianization of $A$ which involves the related
Lie algebra structure on $A$. We will give the axiomatic definition of
Hochschild homology $H_{\ast}(A,M)$ of $A$ with values in an
$A$-bimodule $M$, discuss existence and uniqueness, and relate in
degree zero $H_{0}(A,A)$, the Hochschild homology of $A$ with values
in the $A$-bimodule $A$, to the abelianization of $A$. The $k$-modules
$H_{\ast}(A,A)$ are the absolute Hochschild homology $k$-modules
$HH_{\ast}(A)$ which were considered formally in the previous chapter
in conjunction with cyclic homology.

\section{Generalities on adjoint
  functors}\label{chap2-sec1}\index{adjoint functors}

Abelianization is defined by a universal property relative to the
subcategory of abelian objects. The theory of adjoint functors, which
we sketch now, is the formal development of this idea of a universal
property, and this theory also gives a means for constructing
equivalences between categories. We approach the subject by
considering morphisms between the identity functor and a composite of
two functors.

For an object $X$ in a category, we frequently use the symbol $X$ also
for the identity morphism $X\to X$ along with $1_{X}$, and similarly,
for a category $\X$ the identity functor on $\X$ is frequently
denoted $\X$. Let (sets) denote the category of sets.

\begin{remark}\label{chap2-rem1.1}
Let $\X$ and $\Y$ be two categories and $S:\X\to \Y$ and $T:\Y\to \X$
be two functors. Morphisms of functors $\beta:\X\to TS$ are in
bijective correspondence with morphisms
$$
b:\Hom_{\Y}(S(X),Y)\to \Hom_{\X}(X,T(Y))
$$
as\pageoriginale functors of $X$ in $\X$ and $Y$ in $\Y$ with values
in (sets). A morphism $\beta$ defines $b$ by the relation
$b(g)=T(g)\beta(X)$ and $b$ defines $\beta$ by the relation
$b(1_{S(X)})=\beta(X):X\to TS(X)$. In the same way, morphisms of
functors $\alpha:ST\to \Y$ are in bijective correspondence with
morphisms
$$
a:\Hom_{\X}(X,T(Y))\to \Hom_{\Y}(S(X),Y)
$$
as functors of $X$ and $Y$ with values in (sets). A morphism $\alpha$
defines $a$ by the relation $a(f)=\alpha(Y)S(f)$, and $a$ defines
$\alpha$ by the relation
$$
a(1_{T(Y)})=\alpha(Y):ST(Y)\to Y.
$$
\end{remark}

\begin{definition}\label{chap2-defi1.2}
An adjoint pair of functors is a pair of functors $S:\X\to \Y$ and
$T:\Y\to \X$ together with an isomorphism of functors of $X$ in $\X$
and $Y$ in $\Y$
$$
b:\Hom(S(X),Y)\to \Hom(X,T(Y)),
$$
or equivalently, the inverse isomorphism
$$
a:\Hom(X,T(Y))\to \Hom(S(X),Y).
$$

The functor $S$ is called the left adjoint of $T$ and $T$ is the right
adjoint of $S$.

This situation is denoted $(a,b):S\dashv T(\X,\Y)$ or just $S\dashv
T$.

In terms of the morphisms $\beta(X)=b(1_{S(X)}):X\to TS(X)$ and
$\alpha(Y)=a(1_{T(Y)}):ST(Y)\to Y$ we calculate, for $f:X\to T(Y)$
$$
b(s(f))=T(a(f))\beta(X)=T(\alpha(Y))TS(f)\beta(X)=T(\alpha(Y))\beta(T(Y))f 
$$
and for $g:S(X)\to Y$
$$
a(b(g))=\alpha(Y)S(b(g))=\alpha(Y)ST(g)S(\beta(X))=g\alpha(S(X))S(\beta(X))
$$
\end{definition}

\begin{remark}\label{chap2-rem1.3}
With\pageoriginale the above notations we have
\begin{gather*}
b(a(f))=f\q\text{if and only if}\q
T(\alpha(Y))\beta(T(Y))=1_{T(Y)}\q\text{and}\\
a(b(g))=g\q\text{if and only if}\q \alpha(S(X)S(\beta(X))=1_{S(X)}.
\end{gather*}

An adjoint pair of functors can be defined as a pair of functors
$S:\X\to \Y$ and $T:\Y\to \X$ together with two morphisms of functors
$$
\beta:\X\to TS\q\text{and}\q \alpha:ST\to \Y
$$
satisfying $\alpha(S(Y))S(\beta(X))=1_{S(X)}$ and
$T(\alpha(Y))\beta(T(Y))=1_{T(Y)}$. This situation is denoted
$(\alpha,\beta):S\dashv T:(\X,\Y)$ or just $(\alpha,\beta):S\dashv T$.
\end{remark}

\begin{remark}\label{chap2-rem1.4}
If $S:\X\to \Y$ is the left adjoint of $T:\Y\to \X$, then for the dual
categories $S:\X^{op}\to \Y^{op}$ is the right adjoint of
$T:\Y^{op}\to \X^{op}$. 
\end{remark}

The relation between adjoint functors and universal constructions is
contained in the next proposition.

\begin{proposition}\label{chap2-prop1.5}
Let $S:\X\to \Y$ be a functor, and for each object $Y$ in $\Y$, assume
that there exists an object $t(Y)$ in $\X$ and a morphism
$\alpha(Y):S(t(Y))\to Y$ such that for all $g:S(X)\to Y$, there exists
a unique morphism $f:X\to t(Y)$ such that $\alpha(Y)S(f)=g$. Then
there exists a right adjoint functor $T:\Y\to \X$ of $S$ such that for
each object $Y$ in $\Y$ the object $T(Y)$ is $t(Y)$ and
$$
a:\Hom(X,T(Y))\to \Hom(S(X),Y)
$$
is given by $a(f)=\alpha(Y)S(f)$.
\end{proposition}

\begin{proof}
To define $T$ on moprhisms, we use the universal property. If $v:Y\to
Y'$ is a morphism in $\Y$, then there exists a unique morphism
$T(v):T(Y)\to T(Y')$ such that $\alpha F(Y')(S(T(v))=v\alpha(Y)$ as
morphisms $ST(Y)\to Y'$. The reader can check that this defines a
functor $T$, and the rest follows from the fact that the universal
property asserts that $a$ is a bijection.\pageoriginale This proves
the proposition.
\end{proof}

\setcounter{starprop}{4}
\begin{starprop}\label{chap2-starprop1.5}
Let $T:\Y\to \X$ be a functor, and for each object $X$ in $\X$ assume
there exists an object $s(X)$ in $\Y$ and a morphism $\beta(X):X\to
T(s(X))$ such that for all $f:X\to T(Y)$ there exists a unique
$g:s(X)\to Y$ such that $T(g)\beta(X)=f$. Then there exists a left
adjoint functor $S:\X\to \Y$ such that for each object $X$ in $\X$ the
object $S(X)$ is $s(X)$ and 
$$
b:\Hom(S(X),Y)\to \Hom(X,T(Y))
$$
is given by $b(g)=T(g)\beta(X)$. 
\end{starprop}

\begin{proof}
We deduce (\ref{chap2-starprop1.5})$^{*}$ immediately by applying
(\ref{chap2-prop1.5}) to the dual category.
\end{proof}

\section{Graded commutativity of the tensor product and
  algebras}\label{chap2-sec2}\index{commutative algebra} 

Let $\Theta$ denote an abelian group with a morphism
$\epsilon:\Theta\to \{\pm 1\}$, and define a corresponding
bimultiplicative $\epsilon:\Theta\times\Theta\to \{\pm 1\}$, by the
requirement that
$$
\epsilon(\theta,\theta')=
\begin{cases}
+1 & \text{if~ } \epsilon(\theta)=1\text{~ or~ }\epsilon(\theta')=1\\
-1 & \text{if~ } \epsilon(\theta)=-1\text{~ and~ } \epsilon(\theta')=-1.
\end{cases}
$$

\begin{definition}\label{chap2-defi2.1}
The commuting morphism\index{commutative morphism} $\sigma$ or $\sigma_{\epsilon}$ of the tensor
product $\times:\Theta(k)\times \Theta(k)\to \Theta(k)$ relative to
$\epsilon$ is the morphism
$$
\sigma(L,M)=\sigma:L\otimes M\to M\otimes L
$$
defined for $x\otimes y\in L_{\theta}\otimes M_{\theta'}$ by the
relation
$$
\sigma_{\epsilon}(x\otimes y)=\epsilon(\theta,\theta')(y\otimes x).
$$

Observe that $\sigma(M,L)\sigma(L,M)=L\otimes M$, the identity on the
object $L\otimes M$.
\end{definition}

\begin{definition}\label{chap2-defi2.2}
A $\Theta$-graded $k$-algebra $A$ is commutative (relative to
$\epsilon$) provided $\phi(A)\sigma_{\epsilon}(A,A)=\phi(A):A\otimes
A\to A$. The full subcategory of $\Alg_{\Theta,k}$\pageoriginale
determined by the commutative algebras is denoted by
$C\Alg_{\Theta,k}$. 
\end{definition}

\begin{remark}\label{chap2-rem2.3}
Let $A$ be a $\Theta$-graded $k$-algebra. For $a\in A_{\theta}$ and
$b\in A_{\theta'}$ the Lie bracket of $a$ and $b$ is 
$$
[a,b]=ab-\epsilon(\theta,\theta')ba
$$
which is an element of $A_{\theta+\theta'}$. Let $[A,A]$ denote the
$\Theta$-graded $k$ - submodule of $A$ generated by all Lie brackets
$[a,b]$ for $a$, $b\in A$. Observe that $A$ is commutative if and only
if $[A,A^{c}]=0$. Let $(A,A)$ denote the two-sided ideal generated by
$[A,A]$. 
\end{remark}

\begin{definition}\label{chap2-defi2.4}
A $\Theta$-graded Lie algebra over $k$ is a pair $\underline{g}$
together with a graded $k$-linear map $[,]:\underline{g}\otimes
\underline{g}\to \underline{g}$, called the Lie bracket, satisfying
the following axioms:
\begin{enumerate}
\renewcommand{\labelenumi}{(\theenumi)}
\item For $a\in\underline{g}_{\theta}$ and $b\in
  \underline{g}_{\theta'}$ we have
$$
[a,b]=-\epsilon(\theta,\theta')[b,a].
$$

\item (Jacobi identity) for $a\in \underline{g}_{\theta'}$, $b\in
  \underline{g}_{\theta'}$, and $c\in \underline{g}_{\theta''}$ we
  have
$$
\epsilon(\theta,\theta'')[a,[b,c]]+\epsilon(\theta',\theta)[b,[c,a]]+\epsilon(\theta'',\theta')[c,[a,b]]=0. 
$$

A morphism $f:\underline{g}\to \underline{g}'$ of $\Theta$-graded Lie
algebras over $k$ is a graded $k$-module morphism such that
$f([a,b])=[f(a),f(b)]$ for all $a$, $b\in \underline{g}$. Since the
composition of morphisms of Lie algebras is again a morphism of Lie
algebras, we can speak of the category $\Lie_{\Theta,k}$ of
$\Theta$-graded Lie algebras over $k$ and their morphisms. Following
the lead from algebras, we define a Lie algebra $\underline{g}$ to be
commutative if $[,]=0$ on $\underline{g}\otimes \underline{g}$, or
equivalently, $[\underline{g},\underline{g}]$ is the zero
$k$-submodule where $[\underline{g},\underline{g}]$ denotes the
$k$-submodule generated by all Lie brackets $[a,b]$. The full
subcategory of commutative Lie algebras is denoted $C\Lie_{\Theta,k'}$
and it is essentially the category $\Theta(k)$ of $\Theta$-graded
modules. 
\end{enumerate}
\end{definition}

\begin{example}\label{chap2-exam2.5}
If $A$ is a $\Theta$-graded $k$-algebra, then $A$ with the Lie bracket
$[a,b]=ab-\epsilon(\theta,\theta')ba$\pageoriginale for $a\in
\underline{g}_{\theta'}$, $b\in \underline{g}_{\theta'}$ is a
$\Theta$-graded Lie algebra which we denote by Lie (A). This defines a
functor
$$
\Lie:\Alg_{\Theta,k}\to \Lie_{\Theta,k}.
$$
\end{example}

\section{Abelianization of algebras and Lie
  algebras}\label{chap2-sec3}\index{abelianization}\index{Lie algebra}

In this section we relate several categories by pairs of adjoint
functors. For completeness, we include $(gr)$, the category of groups
and group morphisms together with the full subcategory $(ab)$ of
abelian groups. Also $(ab)$ and $(\bfZ)$ are the same categories. We
continue to use the notation of the previous section for the group
$\Theta$ which indexes the grading.

\begin{definition}\label{chap2-defi3.1}
Abelianization is the left adjoint functor to any of the following
inclusion functors
$$
C\Alg_{\Theta,k}\to \Alg_{\Theta,k},\, C\Lie_{\Theta,k}\to
\Lie_{\Theta,k},\q\text{and}\q (ab)\to (gr).
$$
\end{definition}

\begin{proposition}\label{chap2-prop3.2} 
Each of the inclusion functors
$$
C\Alg_{\Theta,k}\to \Alg_{\Theta,k},\, C\Lie_{\Theta,k}\to
\Lie_{\Theta,k}\q \text{and}\q (ab)\to (gr)
$$
have left adjoint functors
$$
\Alg_{\Theta,k}\to C\Alg_{\Theta,k},\, \Lie_{\Theta,k}\to
C\Lie_{\Theta,k'}\q\text{and}\q (gr)\to (ab).
$$
each of them denoted commonly by $(\;)^{ab}$.
\end{proposition}

\begin{proof}
If the inclusion functor is denoted by $J$, then we will apply
(\ref{chap2-starprop1.5})$^{*}$ to $T=J$ and form the commutative
$s(A)=A/(A,A)$, $s(L)=L/[L,L]$ and $s(G)/(G,G)$ algebra, Lie algebra,
and group respectively by dividing out by commutators. In the case of
an algebra $A$, the commutator ideal $(A,A)$ is the bilateral ideal
generated by $[A,A]$ and $\beta(A):A\to J(s(A))=A/(A,A)$
is\pageoriginale the quotient morphism. For each morphism $f:A\to
J(B)$ where $B$ is a commutative algebra $f((A,A))=0$ and hence it
defines a unique $g:s(A)\to B$ in $C\Alg_{k}$ such that
$J(g)\beta(A)=f$. Hence there exists a left adjoint functor $S$ of $J$
which we denote by $S(A)=A^{ab}$. The same line of argument applies to
Lie algebras where
$\underline{g}^{ab}=\underline{g}/[\underline{g},\underline{g}]$ and
$[\underline{g},\underline{g}]$ is the Lie ideal of all brackets
$[a,b]$ and groups where $G^{ab}=G/(G,G)$ and $(G,G)$ is the normal
subgroup of $G$ generated by all commutators $(s,t)=sts^{-1}t^{-1}$ of
$s$, $t\in G$. This proves the proposition.  
\end{proof}

Now we consider functors from the category of algebras to the category
of Lie algebras and the category of groups.

\begin{notation}\label{chap2-not3.3}
We denote the composite of the functor $\Lie:\Alg_{\Theta,k}\to
\Lie_{\Theta,k}$, which assigns to an algebra $A$ the same underlying
$k$-module together with the Lie bracket $[a,b]$, with the
abelianization functor of this Lie algebra $\Lie(A)^{ab}$, and denote
it by $A^{\alpha\beta}$. This is just the graded $k$-module $[A,A]$.
\end{notation}

We remark that there does not seem to be standard notation for $A$
divided by the $k$-module generated by the commutators, and we have
hence introduced the notation $A^{\alpha\beta}$. Note that the
quotient $A^{\alpha\beta}$ is not an algebra but an abelian Lie
algebra, that is, a graded $k$-module.

\begin{remark}\label{chap2-rem3.4}
The importance of $A^{\alpha\beta}$ lies in the fact that it is
isomorphic to the zero dimensional Hochschild homology, as we shall
see in (6.3)(2), and thus to the zero dimensional cyclic homology, see
1(3.6). 
\end{remark}

\begin{remark}\label{chap2-rem3.5}
The multiplicative group\index{multiplicative group} functor $(\;)^{*}:\Alg_{k}\to (gr)$ is
defined as the subset consisting of $u\in A$ with an inverse
$u^{-1}\in A$ and the group law being given by multiplication in
$A$. It is the right adjoint of the group algebra functor
$k[\;]:(gr)\to \Alg_{k}$ where $k[G]$ is the free module with the set
$G$ as basis and multiplication given by the following formula
on\pageoriginale linear combinations in $k[G]$,
$$
\left(\sum_{t\in G}a_{t}t\right) \left(\sum_{r\in
  G}b_{r}r\right)=\sum_{s\in G} \left(\sum_{tr=s}a_{t}b_{r}\right)s.
$$

The adjunction condition is an isomorphism
$$
\Hom(k[G],A)\to \Hom(G,A^{*}).
$$
\end{remark}

\section[Tensor algebras and universal enveloping
  algebras]{Tensor algebras and universal enveloping\hfil\break
  algebras}\label{chap2-sec4}\index{tensor algebra}\index{universal
  enveloping algebra}

Adjoint functors are also useful in describing free objects or
universal objects with respect to a functor which reduces
structure. These are called structure reduction functors, stripping
functors, or forgetful functors.

\begin{proposition}\label{chap2-prop4.1}
The functor $J:\Alg_{\Theta,k}\to \Theta(k)$ which assigns to the
graded algebra $(A,\phi,\eta)$ the graded $k$-module $A$ has a left
adjoint $T:\Theta(k)\to \Alg_{\Theta,k}$ where $T(M)$ is the tensor
algebra on the graded module $M$.
\end{proposition}

\begin{proof}
From the $n^{th}$ tensor power $M^{n\otimes}$ of a graded module
$M$. For each morphism $f:M\to J(A)$ of graded modules where $A$ is an
algebra we have defined $f_{n}:M^{n\otimes}\to J(A)$ as
$f_{n}=\phi_{n}(A)f^{n\otimes}$, where $\phi_{n}(A):A^{n\otimes}\to A$
is the $n$-fold multiplication.

We give $T(M)=\displaystyle{\coprod_{n}}M^{n\otimes}$ the structure of algebra
$(T(M),\phi,\eta)$ where $\eta:k=M^{0\otimes}\to T(M)$ is the natural
injection into the coproduct and $\phi|M^{p\otimes}\otimes
M^{q\otimes}$ is the natural injection of $M^{(p+q)\otimes}$ into
$T(M)$ defining $\phi:T(M)\otimes T(M)\to T(M)$. For a morphism
$f:M\to J(A)$ the sum of the $f_{n}:M^{n\otimes}\to J(A)$ define a
morphism $g:T(M)\to A$ of algebras. The adjunction morphism is
$\beta(M):M\to J(T(M))$ the natural injection of $M^{1\otimes}=M$ into
$J(T(M))$. Clearly $J(g)\beta(M)=f$ and this defines the bijection
giving the adjunction from the universal property. This proves the
proposition. 
\end{proof}

Now\pageoriginale we consider the question of abelianization of the
tensor algebra. Everything begins with the commutativity symmetry
$\sigma:L\otimes M\to M\otimes L$ of the tensor product.

\medskip
\noindent
{\bf Algebra abelianization of {\boldmath$T(M)$} 4.2.}
The abelianization $T(M)^{ab}$ of the algebra $T(M)$, like $T(M)$, is
of the form $\prod\limits_{0\leq n}S_{n}(M)$ where
$S_{n}(M)=(M^{n\otimes})_{\Sym_{n}}$ is the quotient of the $n^{th}$
tensor power of $M$ by the action of the symmetric group $\Sym_{n}$
permuting the factors with the sign $\epsilon(\theta,\theta')$ coming
from the grading. This follows from the fact that the symmetric group
$\Sym_{n}$ is generated by transpositions of adjacent indices, and
thus $(T(M),T(M))$ is generated by
$$
x\otimes y-\epsilon(\theta,\theta')y\otimes x\q\text{for}\q x\in
M_{\theta},\q y\in M_{\theta'}
$$
as a two sided ideal.

\medskip
\noindent
{\bf Lie algebra abelianization {\boldmath$T(M)^{\alpha\beta}$} of
  {\boldmath$T(M)$}. 4.3.}\index{Lie algebra abelianization} We form $\Lie(T(M))$ and divide by the
$\Theta$-graded $k$-submodule $[T(M),T(M)]$ to obtain\break
$T(M)^{\alpha\beta}$, which like $T(M)$ and $T(M)^{ab}=S(M)$, is of
the form $\coprod\limits_{0\leq n}L_{n}(M)$ where
$L_{n}(M)=(M^{n\otimes})_{\Cyl_{n}}$ is the quotient of the $n^{th}$
tensor power of $M$ by the action of the cyclic group $\Cyl_{n}$
permuting the factors cyclically with the sign
$\epsilon(\theta,\theta')$ coming from the grading. In $M^{n\otimes}$,
we must divide by elements of the form
{\fontsize{10}{12}\selectfont
$$
[x_{1}\otimes\cdots\otimes x_{p},x_{p+1}\otimes\cdots\otimes
  x_{n}]=x_{1}\otimes\cdots\otimes
x_{n}-\epsilon(\theta,\theta')x_{p+1}\otimes\cdots\otimes x_{n}\otimes
x_{1}\otimes\cdots\otimes x_{p}
$$} 
where $x_{1}\otimes\cdots\otimes x_{p}\in (M^{p\otimes})_{\theta}$ and
$x_{p+1}\otimes\cdots\otimes x_{n}\in
(M^{(n-p)\otimes})_{\theta'}$. These elements generate $[T(M),T(M)]$
and they are exactly the elements mapping to zero in the quotient,
under the action of the cyclic group $\Cyl_{n}$ on $M^{n\otimes}$. 

\setcounter{theorem}{3}
\begin{proposition}\label{chap2-prop4.4}
The functor $\Lie:\Alg_{\Theta,k}\to \Lie_{\Theta,k}$ has a left
adjoint functor $U:\Lie_{\Theta,k}\to \Alg_{\Theta,k}$.
\end{proposition}

\begin{proof}
The functor Lie starts with the functor $J$ of (\ref{chap2-prop4.1})
which has $T(\underline{g})$ as\pageoriginale its left adjoint
functor. This is not enough because $\underline{g}\to
T(\underline{g})$ is not a morphism of Lie algebras, so we form the
quotient $u(\underline{g})$ of $T(\underline{g})$ by what is needed to
make it a Lie algebra morphism, namely the two sided ideal generated
by all
$$
x\otimes y-\epsilon(\theta,\theta')y\otimes x=[x,y]\q\text{for}\q x\in
\underline{g}_{\theta}, y\in \underline{g}_{\theta'}. 
$$

The resulting algebra $U(\underline{g})$ has the universal property
which follows from the universal property for $T(M)$ in
(\ref{chap2-prop4.1}). This proves the proposition.
\end{proof}

\begin{definition}\label{chap2-defi4.5}
The algebra $U(\underline{g})$ is called the universal enveloping
algebra of the Lie algebra $\underline{g}$.
\end{definition}

\begin{example}\label{chap2-exam4.6}
The abelianization $U(\underline{g})^{ab}=U(\underline{g}^{ab})$ while
$U(\underline{g})^{\alpha\beta}$ is\break $U(\G)_{\}'}$ the universal
quotient where the action of $\underline{g}$ on $U(\underline{g})$ is
trivial. 
\end{example}

\begin{example}\label{chap2-examp4.7}
The abelianization $k[G]^{ab}=k[G^{ab}]$ while $k[G]^{\alpha\beta}$ is\break
$k[G]_{G}$, the universal quotient where the action of $G$ on $k[G]$
is trivial. This is just a free module on the conjugacy classes of $G$.
\end{example}

\section{The category of $A$-bimodules}\label{chap2-sec5}

Let $A$ be a $\Theta$-graded algebra over $k$ with multiplication
$\phi(A):A\otimes A\to A$ and unit $\eta(A):k\to A$.

\begin{definition}\label{chap2-defi5.1}
A left $A$-module $M$ is a $\Theta$-graded $k$-module $M$, together
with a morphism $\phi(M):A\otimes M\to M$ such that
\begin{enumerate}
\renewcommand{\labelenumi}{(\theenumi)}
\item (associativity) as morphisms $A\otimes A\otimes M\to M$ we have
  $\phi(M)(A\otimes \phi(M))=\phi(M)(\phi(A)\otimes M)$, and

\item (unit property) the composite $(\phi(M)(\eta(A)\otimes M)$ is
  the natural morphism $k\otimes M\to M$.
\end{enumerate}

A morphism $f:M\to M'$ of left $A$-modules is a graded $k$-linear
morphism satisfying $f\phi(M)=\phi(M')(A\otimes f)$. The composition
of two morphisms of left $A$-modules as $k$-modules is a morphism of
left $A$-modules. Thus we can speak of the category ${}_{A}\Mod$ of
left $A$-modules and their morphisms. 
\end{definition}

\begin{definition}\label{chap2-defi5.2}
A\pageoriginale right $A$-module $L$ is a $\Theta$-graded $k$-module
$L$ together with a morphism $\phi(L):L\otimes A\to M$ satisfying an
associativity and unit property which can be formulated to say that
$L$ together with $\phi(L)\sigma(A,L)$ is a left $A^{op}$-module where
$A^{op}=(A,\phi(A)\sigma(A,A),\eta(A))$. A morphism of right
$A$-modules is just a morphism of the corresponding left
$A^{op}$-modules, and composition of $k$-linear morphisms induces
composition of right $A$-modules. Thus we can speak of the category
$\Mod_{A}$ of right $A$-modules and their morphisms. 
\end{definition}

We have the natural identification of categories
${}_{A}\Mod=\Mod_{(A^{op})}$ and ${}_{(A^{op})}\Mod=\Mod_{A}$. 

\begin{definition}\label{chap2-defi5.3}
An $A$-bimodule $M$ is a $\Theta$-graded $k$-module together with two
morphisms $\phi(M):A\otimes M\to M$ making $M$ into a left $A$-module,
and $\phi'(M):M\otimes A\to M$ making $M$ into a right $A$-module,
such that, as morphisms $A\otimes M\otimes A\to M$ we have
$$
\phi(M)(A\otimes \phi'(M))=\phi'(M)(\phi(M)\otimes A).
$$

A morphism of $A$-bimodules $f:M\to M'$ is a $k$-linear morphism which
is both a left $A$-module morphism and a right $A$-module
morphism. The composition as $k$-linear morphisms is the composition
of $A$-bimodules. Thus we can speak of the category ${}_{A}\Mod_{A}$
of $A$-bimodules\index{bimodules}.
\end{definition}

We have the natural identification of categories
${}_{A}\Mod_{A}={}_{A\otimes (A^{op})}\break\Mod=\Mod_{(A^{op})\otimes A}$
in terms of left and right modules over $A$ tensored with its opposite
algebra $A^{op}$.

\begin{definition}\label{chap2-defi5.4}
Let $M$ be an $A$-bimodule. Let $[A,M]$ denote the graded
$k$-submodule generated by all elements of the form
$$
[a,x]=ax-\epsilon(\theta,\theta')xa
$$
where $a\in A_{\theta,x}\in M_{\theta'}$. As a graded $k$-module we
denote by $M^{\alpha\beta}=M/[A,M]$. 
\end{definition}

If\pageoriginale $f:M\to M'$ is a morphism of $A$-bimodules, the
$f([A,M])\subset [A,M']$ and $f$ induces on the quotient
$f^{\alpha\beta}:M^{\alpha\beta}\to {M'}^{\alpha\beta}$, and this
defines a functor ${}_{A}\Mod_{A}\to \Theta(k)$ which is the largest
quotient of an $A$-bimodule $M$ such that the left and right actions
are equal. It is a kind of abelianization\index{bimodules abelianization}, in the sense that for the
$A$-bimodule $A$ the result $A/[A,A]$ is just the abelianization of
the Lie algebra $\Lie(A)$.

\begin{remark}\label{chap2-rem5.5}
In fact the abelization functor is just a tensor product. Any
$A$-bimodule is a left $A\otimes A^{op}$-module and $A^{op}$ is a
right $A\otimes A^{op}$-module. Then $M^{\alpha\beta}$ is just
$A^{op}\otimes_{(A\otimes A^{op})}M$, because the tensor product over
$A\otimes A^{op}$ is the quotient of $A\otimes M$ divided by the
submodule generated by $ab\otimes x-a\otimes bx$ for $a\in A^{op}$,
$x\in M$, and $b\in A\otimes A^{op}$, that is, by relations of the
form $a\otimes x-1\otimes ax$ and $a\otimes x-1\otimes xa$.
\end{remark}

In fact $M\to M^{\alpha\beta}$ is a functor $\Theta\Bimod\to
\Theta(k)$. Here $\Theta\Bimod$ is the category of pairs $(A,M)$ where
$A$ is $\Theta$-graded algebra over $k$ and $M$ is an $A$-bimodule,
and the morphisms are $(u,f):(A,M)\to (A',M')$ where $u:A\to A'$ is a
morphism of algebras and $f:M\to M'$ is $k$-linear such that
$f\phi(M)=\phi(M')(A\otimes f)$ and $f\phi'(M)=\phi'(M')(f\otimes
A)$. Observe that when $u$ is the identity on $A$, then $f:M\to M'$ is
a morphism ${}_{A}\Mod_{A}$.

\begin{remark}\label{chap2-rem5.6}
The abelianization functor $M^{\alpha\beta}$, being a tensor product,
has the following exactness property. If $L\to M\to N\to 0$ is an
exact sequence in ${}_{A}\Mod_{A}$, then $L^{\alpha\beta}\to
M^{\alpha\beta}\to N^{\alpha\beta}\to 0$ is exact in $\Theta(k)$. Even
if $L\to M$ is a monomorphism, it is not necessarily the case that
$L^{\alpha\beta}\to M^{\alpha\beta}$ is a monomorphism.
\end{remark}

Since $M^{\alpha\beta}$ is only right exact, the functor generates a
sequence of functors of $(A,M)$ in $\Theta\Bimod$, denoted
$H_{n}(A,M)$ and called Hochschild homology of $A$ with values in the
module $M$, such that $H_{0}(A,M)$ is isomorphic to
$M^{\alpha\beta}$. More precisely, in the following section we have a
theorem which gives an axiomatic characterisation of Hochschild
homology.

\section{Hochschild homology}\label{chap2-sec6}

\begin{definition}\label{chap2-defi6.1}
An\pageoriginale $A$-bimodule $M$ is called extended provided it is of the form
$A\otimes X\otimes A$ where $X$ is a graded $k$-module.
\end{definition}

\begin{remark}\label{chap2-rem6.2}
There is a natural morphism to $A$-bimodule $k$-module\break 
$\Hom_{(A)}(A\otimes X\otimes A,M')$, denoted
$$
a:\Hom_{\Theta(k)}(X,M')\to \Hom_{(A)}(A\otimes X\otimes A,M'),
$$
defined by the relation
$$
a(f)=\phi'(M')(\phi(M')\otimes A)(A\otimes f\otimes
A)=\phi(M')(A\otimes \phi'(M')).
$$

Moreover, $a$ is an isomorphism defining $S(X)=A\otimes X\otimes A$ as
a left adjoint functor to the stripping functor ${}_{A}\Mod_{A}\to
\Theta(k)$ which deletes the $A$-bimodule structure leaving a $\Theta$
graded $k$-module. The extended modules\index{extended bimodules} have an additional property,
namely that for an exact sequence
$$
0\to M'\to M\to A\otimes X\otimes A\to 0
$$
which is $k$-split exact, we have the short exact sequence
$$
0\to {M'}^{\alpha\beta}\to M^{\alpha\beta}\to (A\otimes X\otimes
A)^{\alpha\beta}\to 0.
$$

This follows from the fact that under the hypothesis, we have a
splitting of the $A$-bimodule sequence given by a morphism $A\otimes
X\otimes X\otimes A\to M$.
\end{remark}

The reader can easily check that the projectives in the category
${}_{A}\break\Mod_{A}$ are direct summands of extended modules $A\otimes
X\otimes A$ where $X$ is a free $\Theta$-graded $k$-module.

\begin{theorem}\label{chap2-thm6.3}
There exists a functor $H:\Theta\Bimod\to \bfZ(\Theta(k))$ together
with a sequence of morphisms $\p:H_{q}(A,M'')\to H_{q-1}(A,M')$ in
$\Theta(k)$ associated to each exact sequence split in $\Theta(k)$ of
$A$-modules $0\to M'\to M\to M''\to 0$ such that 
\begin{enumerate}
\renewcommand{\labelenumi}{\rm(\theenumi)}
\item the\pageoriginale following exact triangle is exact
\[
\xymatrix{
H_{\ast}(A,M')\ar[rr] & & H_{\ast}(A,M)\ar[dl]\\
 & H_{\ast}(A,M'')\ar[ul]^{\p} & 
}
\]
and $\p$ is natural in $A$ and the exact sequence,

\item in degree zero $H_{0}(A,M)$ is naturally isomorphic to
  $M^{\alpha\beta}= M/[A,\break M]$ 

\item if $M$ is an extended $A$-bimodule, then $H_{q}(A,M)=0$ for
  $q>0$. 
\end{enumerate}

Finally two such functors are naturally isomorphic in a way that the
morphisms $\p$ are preserved.
\end{theorem}

\begin{proof}
Since $M^{\alpha\beta}$ is isomorphic to the tensor product
$A^{op}\otimes_{(A\otimes A^{op})}M$, the functor $H_{\ast}(A,M)$ can
be defined as $\Tor^{A\otimes A^{op}}_{\ast}(A^{op},M)$, not as the
absolute $Tor$, but as a $k$-split relative $Tor$ functor. Since
this concept is not so widely understood, we give an explicit version
by starting with a functorial resolution of $M$ by extended
$A$-bimodules. The first term is the resolution is $A\otimes M\otimes
A\to M$ given by scalar multiplication and $M$ in $A\otimes M\otimes
A$ viewed as a $\Theta$-graded $k$-module. The next term is $A\otimes
W(M)\otimes A\to A\otimes M\otimes A$, where $W(M)=\{\ker(A\otimes
M\otimes A\to M)\}$, and the process continues to yield a complex
$Y_{\ast}(M)\to M$ depending functorially on $M$. We can define
$H_{\ast}(A,M)=H_{\ast}(Y_{\ast}(M)^{\alpha\beta})$, and to check the
properties, we observe that for an exact sequence of $A$-bimodules
which is $k$-split
$$
0\to M'\to M\to M''\to 0
$$
the corresponding sequence of complexes
$$
0\to Y_{\ast}(M')^{\alpha\beta}\to Y_{\ast}(M)^{\alpha\beta}\to
Y_{\ast}(M'')^{\alpha\beta}\to 0
$$
is exact, and the homology exact triangle results give property (1)
for the homology $H_{\ast}(A,M)$. The relation (2) that
$H_{0}(A,M)=M^{\alpha\beta}$ follows from the\pageoriginale 
right exactness of the
functor. Finally (3) results from the last statement in
(\ref{chap2-rem6.2}). 
\end{proof}

The uniqueness of the functor $H_{q}$ is proved by induction on $q$
using the technique call dimension shifting. We return to the
canonical short exact sequence associated with any $A$-bimodule $M$
$$
0\to W(M)\to A\otimes M\otimes A\to M\to 0.
$$

This gives an isomorphism $H_{q}(A,M)\to H_{q-1}(A,W(M))$ for $q>1$,
and an isomorphism $H_{1}(A,M)\to \ker(H_{0}(A,W(M))\to
H_{0}(A,A\otimes M\otimes A))$. In this way the two theories are seen
to be isomorphic by induction on the degree. This proves the theorem. 



