\chapter{Exact Couples and the Connes Exact Couple}\label{chap1}

In\pageoriginale this chapter we review background material on graded
objects, differential objects or complexes, spectral sequences, and on
exact couples. Since the Connes' exact couple relating Hochschild and
cyclic homology plays a basic role in the theory of cyclic homology,
this material will serve as background material and as a means of
introducing other technical topics needed in the subsequent
chapters. We discuss the basic structure of the Connes' exact couple
and the elementary conclusions that can be drawn from this kind of
exact couple.

\section{Graded objects over a
  category}\label{chap1-sec1}\index{graded objects}

Given a category we formulate the notion of graded objects over the
category and define the category of graded objects. There are many
examples of gradings indexed by groups $\bfZ$, $\bfZ/2\bfZ$,
$\bfZ/8\bfZ$, or $\bfZ^{r}$ which arise naturally. Then, a bigraded
object is a $\bfZ^{2}$-graded object, that is, an object graded by the
group $\bfZ^{2}$. 

\begin{definition}\label{chap1-defi1.1}
Let $\mathcal{C}$ be a category and $\Theta$ an abelian group. The
category $G\tau_{\Theta}(\mathcal{C})$, also denoted $\Theta\mathcal{C}$,
of $\Theta$-graded objects over $\mathcal{C}$ has for objects
$X=(X_{\theta})_{\theta\in \Theta}$ where $X$ is a family of objects
$X_{\theta}$ in $\mathcal{C}$ indexed by $\Theta$, for morphisms
$f:X\to Y$ families $f=(f_{\theta})_{\theta\in \Theta}$ of morphisms
$f_{\theta}:X_{\theta}\to Y_{\theta}$ in $\mathcal{C}$, and
composition $gf$ of $f:X\to Y$ and $g:Y\to Z$ given by
$(gf)_{\theta}=g_{\theta}f_{\theta}$ in $\mathcal{C}$. 
\end{definition}

The identity on $X$ is the family $(1_{\theta})_{\theta\in \Theta}$ of
identities $1_{\theta}$ on $X_{\theta}$. Thus it is easily checked
that we have a category, and the morphism sets define a functor of two
variables 
$$
\Hom_{\Theta\mathcal{C}}=\Hom:(\Theta\mathcal{C})^{op}\times
\Theta\mathcal{C}\to \text{(sets)}
$$
extending $\Hom:\mathcal{C}^{op}\times \mathcal{C}\to$ (sets) in the
sense that for two $\Theta$-graded objects $X$ and $Y$ we have
$\Hom_{\Theta\mathcal{C}}(X,Y)=\displaystyle{\prod_{\theta'\in\Theta}}\Hom_{\mathcal{C}}(X_{\theta'},Y_{\theta'})$. Note 
that\pageoriginale we do not define graded objects as either products
or coproducts, but the morphism set is naturally a product. This
product description leads directly to the notion of a morphism of
degree\index{morphism of given degree} $\alpha\in \Theta$ such that a morphism in the category is of
degree $0\in\Theta$.

\begin{definition}\label{chap1-defi1.2}
With the previous notations for two objects $X$ and $Y$ in $\Theta\C$,
the set of morphisms of degree $\alpha\in\Theta$ from $X$ to $Y$ is
$\Hom_{\alpha}(X,Y)=\displaystyle{\prod_{\theta'\in\Theta}}\Hom(X_{\theta},Y_{\theta+\alpha})$. If
$f:X\to Y$ has degree $\alpha$ and $g:Y\to Z$ has degree $\beta$, then
$(gf)_{\theta}=g_{\theta+\alpha}f_{\theta}$ is defined $gf:X\to Z$ of
degree $\alpha+\beta$, i.e.\@ it is a function $(f,g)\mapsto gf$
defined
$$
\Hom_{\alpha}(X,Y)\times \Hom_{\beta}(Y,Z)\to
\Hom_{\alpha+\beta}(X,Z).
$$
\end{definition}

Thus this $\Theta$-graded $\Hom$, denoted $\Hom_{*}$, is defined
$$
\Hom_{*}:(\Theta\C)^{op}\times \Theta\C\to \Theta~\text{(Sets)}
$$
as a functor of two variables with values in the category of
$\Theta$-graded sets.

\begin{remark}\label{chap1-rem1.3}
Recall that a zero object\index{zero object} in a category $\C$ is an object denoted $0$
or $*$, such that $\Hom(X,0)$ and $\Hom(0,X)$ are sets with one
element. A category with a zero object is called a pointed
category. The zero morphism $0:X\to Y$ is the composite $X\to 0\to Y$.
\end{remark}

\begin{remark}\label{chap1-rem1.4}
If $\A$ is an additive (resp.\@ abelian) category\index{abelian category}, then $\Theta\A$ is
an additive (resp.\@ abelian) category\index{additive category} where the graded homomorphism
functor is defined
$$
\Hom_{*}:(\Theta\A)^{op}\times \Theta\A\to \Theta(ab)
$$
with values in the category of $\Theta$-graded abelian
groups. A
sequence $X'\to X\to X''$ is exact in $\Theta\A$ if and only if
$X'_{\theta}\to X_{\theta}\to X''_{\theta}$ is exact in $\A$ for each
$\theta\in\Theta$. 
\end{remark}

\begin{remark}\label{chap1-rem1.5}
Of special interest is the category $(k)$ of $k$-modules over a
commutative\pageoriginale ring $k$ with unit. This category has an
internal $\Hom$ functor and tensor functor defined
$$
\otimes :(k)\times (k)\to (k)\q\text{and}\q \Hom:(k)^{op}\times (k)\to (k)
$$
satisfying the adjunction formula with an isomorphism
$$
\Hom(L\otimes M, N)\simeq \Hom(L,\Hom(M,N))
$$
as functors of $L$, $M$, and $N$. These functors extend to
$$
\otimes :\Theta(k)\times \Theta(k)\to \Theta(k)\q\text{and}\q
\Hom:\Theta(k)^{op}\times\Theta(k)\to \Theta(k)
$$
satisfying the same adjunction formula by the definitions
$$
(L\otimes
M)_{\theta}=\coprod\limits_{\alpha+\beta=\theta}L_{\alpha}\otimes
M_{\beta}\q\text{and}\q
\Hom(M,N)_{\theta}=\prod\limits_{\alpha\in\Theta}\Hom(M_{\alpha},N_{\alpha+\theta}). 
$$

We leave it to the reader to check the adjunction formula, and we come
back to the question of the tensor product of two morphisms of
arbitrary degrees in the next section, for it uses an additional
structure on the group $\Theta$.
\end{remark}

\begin{notation}\label{chap1-not1.6}
For certain questions, for example those related to duality, it can be
useful to have the upper index convention for an element $X$ of
$\Theta\C$. This is $X^{\theta}=X_{-\theta}$ and
$\Hom(X,Y)^{\theta}=\Hom(X,Y)_{-\theta}$. In the classical case of
$\Theta=\bfZ$ the effect is to turn negative degrees into positive
degrees. For example in the category $(k)$ the graded dual in degree
$n$ is $\Hom(M,k)^{n}=\Hom(M_{n},k)$. The most clear use of this
convention is with cohomology which is defined in terms of the dual of
the homology chain complex for spaces.
\end{notation}

\section{Complexes}\label{chap1-sec2}

To define complexes, we need additional structure on the grading
abelian group $\Theta$, and this leads us to the next definition.

\begin{definition}\label{chap1-defi2.1}
An\pageoriginale oriented abelian group $\Theta$ is an abelian group $\Theta$
together with a homomorphism $e:\Theta\to \{\pm 1\}$ and an element
$\iota\in\Theta$ such that $e(\iota)=-1$.
\end{definition}

\begin{definition}\label{chap1-defi2.2}
A complex\index{complex} $X$ in a pointed category $\chi$ graded by an oriented
abelian group $\Theta$ is a pair $(X,d(X))$ where $X$ is in
$\Theta\chi$ and $d(X)=d:X\to X$ is a morphism of degree $-\iota$ such
that $d(X)d(X)=0$. A morphism $f:X\to Y$ of complexes is a morphism in
$\Theta \chi$ such that $fd(X)=d(Y)f$.
\end{definition}

The composition of morphisms of complexes is the composition of the
corresponding graded objects. We denote the category of complexes in
$\chi$ graded by the oriented abelian group by $C_{\Theta}(\chi)$ or
simply $C(\chi)$.

In order to deal with complexes, we first need some additive structure
on $\Hom(X,Y)$ for two $\Theta$-graded objects $X$ and $Y$, which are
the underlying graded objects of complexes and second, kernels and
cokernels, which are used to define the homology functor. To define
the homology, the base category must be an abelian category $\A$, for
example, the category $(k)$ of $k$-modules. Then $\Theta\A$ and
$C_{\Theta}(\A)$ are abelian categories, and homology will be defined
as a functor $H:C_{\Theta}(\A)\to \Theta\A$. Tha basic tool is the
snake lemma which we state now.

\begin{snakelemma}\label{chap1-snakelem2.3}\index{snake lemma}
Let $\A$ be an abelian category, and consider a morphism of exact
sequences $(u',u,u'')$ all of degree $\nu\in \Theta$
\[
\xymatrix{
 & L'\ar[d]_{u'}\ar[r]^{f} & L\ar[d]_{u}\ar[r]^{f'} &
  L''\ar[d]_{u''}\ar[r] & 0\\
0\ar[r] & M'\ar[r]^{g} & M\ar[r]^{g'} & M'' &
}
\]

Then $f$ and $g$ induce morphisms $k(f):\ker (u')\to \ker(u)$ and
$c(g):\coker (u')\to \coker (u)$ and the commutative diagram induces a
morphism $\delta:\ker(u'')\to \coker (u')$ of degree $\nu$ such that
the following sequence, called the sequence of the snake, is exact
$$
\ker(u')\to \ker(u)\to \ker(u'')\xrightarrow{\delta}\coker (u')\to
\coker (u)\to \coker (u'').
$$

Further,\pageoriginale if $f$ is a monomorphism, then $\ker(u')\to
\ker(u)$ is a mono\-mor\-phism, and $g'$ is an epimorphism, then $\coker
(u)\to \coker (u'')$ is an epimorphism. Finally the snake sequence is
natural with respect to morphisms of the above diagrams which give
arise to the snake sequence. Here a morphism of the diagram is a
family of morphisms of each respective object yielding a commutative
three dimensional diagram.
\end{snakelemma}

For a proof, see Bourbaki, {\em Alg\'ebre homologique.}

\begin{notation}\label{chap1-not2.4}
Let $X$ be a complex in $C_{\Theta}(\A)$, and consider the
kernel-cokernel sequence in $\Theta\A$ of $d(x)=d$, which has degree
$-\iota$,
$$
0\to Z(X)\to X\xrightarrow{d}X\to Z'(X)\to 0.
$$

This defines two functors $Z$, $Z':C_{\Theta}(\A)\to \Theta\A$, and
this sequence is a sequence of functors $C_{\Theta}(A)\to \Theta
A$. Since $d(X)d(X)=0$, we derive three factorizations of $d(X)$
namely
\begin{gather*}
d'=d'(X):Z'(X)\to X, d''=d''(X):X\to Z(X),\q\text{and}\\
\hat{d}=\hat{d}(X):Z'(X)\to Z(X)
\end{gather*}
from which we have the following diagram, to which the snake sequence
applies, 
\[
\xymatrix{
 & X\ar[d]_{d''}\ar[r]^{d} & X\ar[d]_{1}\ar[r] &
  Z'(X)\ar[d]_{d'}\ar[r] & 0\\
0\ar[r] & Z(X)\ar[r] & X\ar[r] & X &
}
\]
and the boundary morphism $\delta:\ker(\delta')=H'(X)\to H(X)=\coker
(\delta'')$ has zero kernel and cokernel. Thus it is invertible of
degree $-\iota$, and it can be viewed as an isomorphism of the functor
$H'$ with $H$ up to the question of degree.
\end{notation}

The next application of the snake lemma \ref{chap1-snakelem2.3} is to
a short exact sequence $0\to X'\to X\to X''\to 0$ of complexes in
$C_{\Theta}(\A)$ and this is possible because\pageoriginale the
following diagram is commutative with exact rows arising from the
snake lemma applied to the morphism $(d(X'),d(X)$, $d(X''))$
\[
\xymatrix{
 & Z'(X')\ar[d]^{\hat{d}(X')}\ar[r] & Z'(X)\ar[d]^{\hat{d}(X)}\ar[r] &
  Z'(X'')\ar[d]^{\hat{d}(X'')}\ar[r] & 0\\
0\ar[r] & Z(X')\ar[r] & Z(X)\ar[r] & Z(X'') &
}
\]

Since $H'$ is the kernel of $\hat{d}$ and $H$ is the cokernel of
$\hat{d}$, we obtain the exact sequence
$$
H'(X'')\to H'(X)\to H'(X'')\xrightarrow{\delta}H(X')\to H(X)\to
H(X''), 
$$
and using the isomorphism $H'\to H$, we obtain an exact triangle which
we formulate in the next basic theorem about homology.

\begin{theorem}\label{chap1-thm2.5}
Let $0\to X'\to X\to X''\to 0$ be a short exact sequence of complexes
in $C_{\Theta}(\A)$. Then there is a natural morphism
$\delta:H(X'')\to H(X')$ such that the following triangle is exact
\[
\xymatrix{
H(X')\ar[rr] & & H(X)\ar[dl]\\
 & H(X'')\ar[ul] & 
}
\]

Here the degree of $\delta$ is $-\iota$, the degree of $d$.
\end{theorem}

\section{Formal structure of cyclic and Hochschild
  homology}\label{chap1-sec3}\index{homology}

\begin{definition}\label{chap1-defi3.1}
An algebra\index{algebra} $A$ over $k$ is a triple $(A,\phi(A),\eta(A))$ where $A$ is
a $k$-module, $\phi(A):A\otimes A\to A$ is a morphism called
multiplication, and $\eta(A):k\to A$ is a morphism called the unit
which satisfies the following axioms:
\begin{enumerate}
\renewcommand{\labelenumi}{(\theenumi)}
\item (associativity) As morphisms $A\otimes A\otimes A\to A$ we have
$$
\phi(A)(\phi(A)\otimes A)=\phi(A)(A\otimes \phi(A))
$$
where\pageoriginale as usual $A$ denotes both the object and the
identity morphism on $A$.

\item (unit) As morphisms $A\otimes k\to A$ and $k\otimes A\to A$, the
  morphisms
$$
\phi(A)(A\otimes \eta(A))\q\text{and}\q \phi(A)(\eta(A)\otimes A)
$$
are the natural isomorphisms for the unit $k$ of the tensor
product. Let $\Theta$ be an abelian group. A $\Theta$-graded algebra
$A$ over $k$ is a triple $(A,\phi(A),\eta(A))$ where $A$ is a
$\Theta$-graded $k$-module, $\phi(A):A\otimes A\to A$ is a morphism of
$\Theta$-graded $k$-modules, and $\eta(A):k\to A$ is a morphism of
$\Theta$-graded $k$-modules satisfying the above axioms (1) and (2).
\end{enumerate}

A morphism $f:A\to A'$ of $\Theta$-graded algebras is a
morphism\index{algebra morphism} of
$\Theta$-graded modules such that $\phi(A')(f\otimes f)=f\phi(A)$ as
morphisms $A\otimes A\to A'$ and $f\eta(A)=\eta(A')$ as morphisms
$k\to A'$. If $f:A\to A'$ and $f':A'\to A''$ are two morphisms of
$\Theta$-graded algebras\index{algebra, graded}, then $f'f:A\to A''$ is a morphism of
$\Theta$-graded algebras. Let $\Alg_{\Theta,k}$ denote the category of
$\Theta$-graded algebras over $k$, and when $\Theta=0$, the zero
grading, then we denote $\Alg_{0,k}$ by simply $\Alg_{k}$. 
\end{definition}

\begin{notation}\label{chap1-not3.2}
For an abelian group $\Theta$ and a pointed category $\C$ we denote by
$(Z\times \Theta)^{+}(\C)$ the full subcategory of $(\bfZ\times
\Theta)(\C)$ determined by all $X_{\bigdot}=(X_{n,\theta})$ with
$X_{n,\theta}=\ast$ for $n<0$ and $(\bfZ\times \Theta)^{-}(\C)$ the
full subcategory determined by all $X_{\bigdot}=(X_{n,\theta})$ with
$X_{n,\theta}=\ast$ for $n>0$. The intersection $(\bfZ\times
\Theta)^{+}(\C)\cap (\bfZ\times \Theta)^{-}(\C)$ can be identified
with $\Theta(\C)$.
\end{notation}

\begin{remark}\label{chap1-rem3.3}
As functors, cyclic homology and Hochschild homology, denoted by
$HC_{\ast}$ and $HH_{\ast}$ respectively, are defined
$$
HC_{\ast}:\Alg_{\Theta,k}\to (\bfZ\times\Theta)^{+}(k)\q\text{and}\q
HH_{\ast}:\Alg_{\Theta,k}\to (\bfZ\times\Theta)^{+}(k). 
$$

This is the first indication of what kinds of functors these are.
\end{remark}

When he first introduced cyclic homology $HC_{\ast}$, Connes'
emphasised that cyclic homology and Hochschild homology were linked
with exact sequences which can be assembled into what is called an
exact couple. We introduce\pageoriginale exact couples with very
general gradings to describe this linkage.

\begin{definition}\label{chap1-defi3.4}
Let $\Theta$ be an abelian group with $\theta$, $\theta'$,
$\theta''\in\Theta$ and let $\A$ be a abelian category. An exact
couple over $\A$ with degrees $\theta$, $\theta'$, $\theta''$ is a
pair of objects $A$ and $E$ and three morphisms $\alpha:A\to A$ of
degree $\theta$, $\beta:A\to E$ of degree $\theta'$, and $\gamma:E\to
A$ of degree $\theta''$ such that the following triangle is
exact\index{homology exact triangle}.
\[
\xymatrix{
A\ar[rr]^-{\alpha} & & A\ar[dl]^-{\beta}\\
 & E\ar[ul]^-{\gamma} & 
}
\]

In particular, we have $\Iim(\alpha)=\ker(\beta)$,
$\Iim(\beta)=\ker(\gamma)$, and $\Iim(\gamma)=\ker(\alpha)$. 
\end{definition}

Let $(A,E,\alpha,\beta,\gamma)$ and $(A',E',\alpha',\beta',\gamma')$
be two exact couples of degree $\theta$, $\theta'$, $\theta''$. A
morphism from the first to the second is pair of morphisms $(h,f)$,
where $h:A\to A'$ and $f:E\to E'$ are morphisms of degree $0$ in
$\Theta(\A)$ such that $h\alpha=\alpha'h$, $f\beta=\beta'h$,
$h\gamma=\gamma'f$. The composition of two morphisms $(h,f)$ and
$(h',f')$ is $(h',f')(h,f)=(h'h,f'f)$ when defined. Thus we can speak
of the category $ExC(\A;\Theta;\theta,\theta',\theta'')$ of exact
couples $(A,E,\alpha,\beta,\gamma)$ in $\Theta(\A)$ of degrees
$\theta$, $\theta'$, $\theta''$. 

We can now describe the Cyclic-Hochschild homology\index{cyclic
  homology}\index{Hochschild homology} linkage in terms of a single functor.

\begin{remark}\label{chap1-rem3.5}
The Connes' exact sequence (or exact couple)\index{Connes' exact
  couple} is a functor 
$$
(HC_{\ast},HH_{\ast},S,B,I):\Alg_{\Theta,k}\to
ExC((k),\bfZ\times\Theta,(-2,0),(1,0),(0,0)) 
$$
which, incorporating the remark \eqref{chap1-rem3.3} satisfies
$HC_{n}(A)=0=HH_{n}(A)$ for $n<0$. The special feature of the degrees
formally gives two elementary results.
\end{remark}

\begin{proposition}\label{chap1-prop3.6}
The natural morphism $I:HH_{0}(A)\to HC_{0}(A)$ is an\pageoriginale
isomorphism of functors $\Alg_{\Theta,k}\to \Theta(k)$. 
\end{proposition}

\begin{proof}
We have an isomorphism since $\ker(I)$ is zero for reasons of degree
and
$$
\Iim(I)=\ker(S:HC_{0}(A)\to HC_{-2}(A))=HC_{0}(A)
$$
again, due to degree considerations. This proves the proposition.
\end{proof}

\begin{proposition}\label{chap1-prop3.7}
Let $f:A\to A'$ be a morphism in $\Alg_{\Theta,k}$. Then
$HC_{\ast}(f)$ is an isomorphism if and only if $HH_{\ast}(f)$ is an
isomorphism. 
\end{proposition}

\begin{proof}
The direct implication is a generality about morphisms $(h,f)$ of
exact couples in any abelian category, namely, if $h$ is an
isomorphism, then by the five-lemma $f$ is an isomorphism. Conversely,
if we assume that $HC_{i}(f)$ is an isomorphism for $i<n$ and
$HH_{\ast}(f)$ is an isomorphism, then $HC_{n}(f)$ is an isomorphism
by the five-lemma applied to the exact sequence
$$
HC_{n-1}\xrightarrow{B}HH_{n}\xrightarrow{I}HC_{n}\xrightarrow{s}HC_{n-2}\xrightarrow{B}HH_{n-1}. 
$$

The induction begins with the result in the previous proposition. This
proves the proposition.
In the next section we study the category of exact couples as a
preparation for defining Hochschild and cyclic homology and
investigating its properties. We also survey some of the classical
examples of exact couples. 
\end{proof}

\section{Derivation of exact couples and their spectral
  sequence}\label{chap1-sec4}\index{exact couple} 

The snake lemma (\ref{chap1-snakelem2.3}) is a kernel-cokernel exact
sequence coming from a morphism of exact sequences. There is another
basic kernel-cokernel exact sequence coming from a composition of two
morphisms. We announce the result and refer to Bourbaki, {\em
  Alg\'ebre homologique} for the proof.

\begin{lemma}\label{chap1-lem4.1}
Let $f:X\to Z$ and $g:Z\to Y$ be two morphisms in an abelian category
$\A$. Then there is an exact sequence-
{\selectfont{\fontsize{10}{8}{
$$
0\to \ker (f)\to \ker(gf)\xrightarrow{f'}\ker (g)\to \coker
(f)\xrightarrow{g'}\coker (gf)\to \coker (g)\to 0
$$}}}
where\pageoriginale $f':\ker(gf)\to \ker(g)$ is induced by $f$,
$g':\coker(f)\to \coker (gf)$ is induced by $g$, and the other three
arrows are induced respectively by the identities on $X$, $Z$, and
$Y$.
\end{lemma}

We wish to apply this to an exact couple $(A,E,\alpha,\beta,\gamma)$ in
the category $ExC(\A,\Theta;\theta,\theta',\theta'')$ to obtain a new
exact couple, called the derived couple. In fact there will be two
derived couples one called the left and the other the right derived
couple differing by an isomorphism of nonzero degree.

First, observe that $\alpha:A\to A$ factorizes naturally as the
composite of the natural epimorphism $A\to \coker(\gamma)$, an
invertible morphism $\alpha^{\#}:\coker (\gamma)\to \ker(\beta)$, and
the natural monomorphism $\ker(\beta)\to A$. Secondly, since
$(\beta\gamma)(\beta\gamma)=0$, we have an induced morphism
$\beta\gamma:\coker (\beta\gamma)\to \ker(\beta\gamma)$ whose kernel
and cokernel are naturally isomorphic to $H(E,\beta\gamma)$ by the
snake exact sequence as is used in \ref{chap1-not2.4}. Finally, there
is a natural factorization of $\beta\gamma:\coker(\beta\gamma)\to
\ker(\beta\gamma)$ as a quotient $\gamma^{\#}:\coker (\beta\gamma)\to
A$ of $\gamma$ composed with a restriction $\beta^{\#}:A\to
\ker(\beta\gamma)$ of $\beta$. Then we have
$\ker(\beta)=\ker(\beta^{\#})$ and
$\coker(\gamma)=\coker(\gamma^{\#})$. Now we apply
(\ref{chap1-lem4.1}) to the factorization of
$\beta\gamma=\beta^{\#}\gamma^{\#}$ and consider the middle four terms
of the exact sequence
$$
H(E,\beta\gamma)\xrightarrow{\gamma^{0}}\ker(\beta)\xrightarrow{\delta}\coker(\gamma)\xrightarrow{\beta^{0}}H(E,\beta\gamma).
$$

\begin{definition}\label{chap1-defi4.2}
We denote $ExC(\A,\Theta;\theta,\theta',\theta'')$ by simply
$ExC(\theta,\theta',\theta'')$. The left derived couple functor
defined\index{derived exact couple}
$$
ExC(\theta,\theta',\theta'')\to ExC(\theta,\theta',\theta''-\theta)
$$
assigns to an exact couple $(A,E,\alpha,\beta,\gamma)$, the exact
couple
$$
(\coker(\gamma),H(E,\beta\gamma),\alpha_{\lambda},\beta_{\lambda},\gamma_{\lambda})
$$
where $\alpha_{\lambda}=\delta\alpha^{\#}$,
$\beta_{\lambda}=\beta^{0}$, and
$\gamma_{\lambda}=(\alpha^{\#})^{-1}\gamma^{0}$, using the above
notations. The right derived couple functor defined
$$
ExC(\theta,\theta',\theta'')\to ExC(\theta,\theta'-\theta,\theta'')
$$
assigns\pageoriginale to an exact couple $(A,E,\alpha,\beta,\gamma)$,
the exact couple
$$
(\ker(\beta),H(E,\beta\gamma),\alpha_{\rho},\beta_{\rho},\gamma_{\rho})
$$
where $\alpha_{\rho}=\alpha^{\#}\delta$,
$\beta_{\rho}(\alpha^{\#})^{-1}$, and $\gamma_{\rho}=\gamma^{0}$ using
the above notations.

Observe that $(\alpha^{\#},H(E,\beta\gamma))$ is an invertible
morphism
$$
(\coker(\gamma),H(E,\beta\gamma),\alpha_{\lambda},\beta_{\lambda},\gamma_{\lambda})\to (\ker(\beta),H(E,\beta\gamma),\alpha_{\rho},\beta_{\rho},\gamma_{\rho})
$$
which shows that the two derived couple functors differ only by the
degree of the morphism. The only point that remains, is to check
exactness of the derived couple at $H(E,\beta\gamma)$, and for this we
use  (\ref{chap1-lem4.1}) as follows. The composite of
$$
\gamma^{\#}:\coker(\beta\gamma)\to A\q\text{and}\q \beta^{\#}:A\to
\ker(\beta\gamma) 
$$
is $\beta\gamma:\coker(\beta\gamma)\to \ker(\beta\gamma)$, and by
(\ref{chap1-lem4.1}) we have a six term exact sequence
{\fontsize{10}{12}\selectfont
$$
0\to\ker(\gamma^{\#})\to
H(E)\xrightarrow{\gamma^{0}}\ker(\beta)\xrightarrow{\delta}\coker(\gamma)\xrightarrow{\beta^{0}}H(E)\to
\coker(\beta^{\#})\to 0.
$$}
\end{definition}

Hence the following two sequences
$$
H(E)\xrightarrow{\gamma_{\lambda}}\coker(\gamma)\xrightarrow{\alpha_{\lambda}}\coker(\gamma)\xrightarrow{\beta_{\lambda}}H(E)
$$
and
$$
H(E)\xrightarrow{\gamma_{\rho}}\ker(\beta)\xrightarrow{\alpha_{\rho}}\ker(\beta)\xrightarrow{\beta_{\rho}}H(E).
$$
are exact. It remains to show that the derived couple is exact at
$H(E)$. For this, we start with the exact sequence
$A\xrightarrow{\beta}E\xrightarrow{\gamma}A$ of the given exact couple
and observe that $\Iim(\beta\gamma)\subset
\Iim(\beta)=\ker(\gamma)\subset \ker(\beta\gamma)$. Hence the sequence
$\coker(\gamma)\to \ker(\beta\gamma)/\Iim(\beta\gamma)=H(E)\to
\ker(\beta)$ is exact where the first arrow is $\beta^{0}$ and the
second is $\gamma^{0}$. Using the invertible morphism $\alpha^{\#}$,
we deduce that the left and right derived couples are exact
couples. This\pageoriginale completes the discussion of definition
(\ref{chap1-defi4.2}). 

\begin{remark}\label{chap1-rem4.3}
Let $C_{\Theta,-\iota}(\A)$ denote the category of complexes over
$\A$, graded by $\Theta$, and with differential of degree $-\iota$. We
have used the functor $ExC(\A,\Theta:\theta,\theta',\theta'')\to
C_{\Theta,\theta'+\theta''}(\A)$ which assigns to an exact couple
$(A,E,\alpha,\beta,\gamma)$ the complex $(E,\beta\gamma)$. Further,
composing with the homology functor, we obtain $H(E)$ which is the
second term in the derived couple of $(A,E,\alpha,\beta,\gamma)$. 
\end{remark}

\begin{remark}\label{chap1-rem4.4}
Now we iterate the process of obtaining the derived couple. For an
exact couple $(A,E)=(A,E,\alpha,\beta,\gamma)$ in
$ExC(\theta,\theta',\theta'')$, we have a sequence of exact couples
$(A^{r},E^{r})$ where $(A,E)=(A^{1},E^{1})$, $(A^{r},E^{r})$ is the
derived couple of $(A^{r-1},E^{r-1})$, and $E^{r+1}=H(E^{r},d^{r})$
with $d^{r}=\beta\gamma^{r}$. As for degrees $(A^{r},E^{r})$ is in
$ExC(\theta,\theta',\theta''-(r-1)\theta)$ for a sequence of left
derived couples and in $ExC(\theta,\theta'-(r-1)\theta,\theta'')$ for
a sequence of right derived couples. In either case the complex
$(E^{r},d^{r})$ is in $C_{\Theta,\theta'+\theta''-(r-1)\theta}(\A)$,
and the sequence of complexes $(E^{r},d^{r})$ is an example of a
spectral sequence because of the property that
$E^{r+1}=H(E^{r},d^{r})$. We can give a direct formula for the terms
$E^{r}$ as subquotients of $E=E^{1}$. Firstly, we know that
\begin{align*}
E^{2} &=
H(E^{1},\beta\gamma)=\ker(\beta\gamma)/\Iim(\beta\gamma)=\gamma^{-1}(\ker(\beta)/\beta(\Iim(\gamma))\\ 
&= \gamma^{-1}(\Iim(\alpha))/\beta(\ker(\alpha)), 
\end{align*}
and by analogy, the general formula is the following:
$$
E^{r}=\gamma^{-1}(\Iim(\alpha^{r-1}))/\beta(\ker(\alpha^{r-1})).
$$

We leave the proof of this assertion to the reader.
\end{remark}

\section[The spectral sequence and exact couple of...]{The spectral
  sequence and exact couple of a filtered 
  complex}\label{chap1-sec5} 

The most important example of an exact couple and its associated
spectral sequence is the one coming from a filtered complex.

\begin{definition}\label{chap1-defi5.1}
A filtered object $X$ in a category $\C$ is an object $X$
together\pageoriginale with a sequence of subobjects, $F_{p}X$ or
$F_{p}(X)$, indexed by the integers $\cdots\to F_{p-1}X\to F_{p}X\to
\cdots\to X$. A morphism $f:X\to Y$ of filtered objects in $\C$ is a
morphism $f:X\to Y$ in $\C$ which factors for each $p$ by
$F_{p}(f):F_{p}X\to F_{p}Y$.
\end{definition}

The factorization $F_{p}(f)$ is unique since $F_{p}Y\to Y$ is a
monomorphism. The composition $gf$ in $\C$ of two morphisms $f:X\to Y$
and $g:Y\to Z$ of filtered objects\index{filtered object} is again a morphism of filtered
objects. Thus we can speak of the category $F\cdot\C$ of filtered
objects over $\C$. 

\begin{remark}\label{chap1-rem5.2}
We are interested in the category $\mathcal{F}\cdot C_{\Theta}(A)$ of filtered
complexes. In particular we construct a functor
$$
E^{0}:\F\cdot C_{\Theta,-\iota}(\A)\to C_{\bfZ\times
  \Theta,(0,-\iota)}(\A) 
$$
by assigning to the filtered complex $X$ the complex $E^{0}(X)$,
called the associated graded complex, with graded term
$$
E^{0}_{p,\theta}=F_{p}X_{\theta}/F_{p-1}X_{\theta}
$$
and quotient differential in the following short exact sequence
$$
0\to F_{p-1}X\to F_{p}X\to E^{0}_{p}\to 0
$$
in the category $C_{\Theta}(\A)$. The homology exact triangle is a
sequence of $\Theta$-graded exact triangles which can be viewed as a
single $(\bfZ\times\Theta)$-graded exact triangle and this exact
triangle is an exact couple 
\[
\xymatrix{
H_{\ast}(F_{\ast}X)\ar[rr]^-{\alpha} & &
H_{\ast}(F_{\ast}X)=A^{1}_{\ast,\ast}\ar[dl]^{\beta}\\
 & H_{\ast}(E^{0}_{\ast})=E^{1}_{\ast,\ast}\ar[ul]^{\gamma}
}
\]
where the $\bfZ\times \Theta$-degree of $\alpha$ is $(1,0)$, of
$\beta$ is $(0,0)$, and of $\gamma$ is $(-1,-\iota)$. The theory of
the previous section says that we have a spectral sequence
$(E^{r},d^{r})$ and the degree of $d^{r}$ is $(-r,-\iota)$. Moreover,
we have defined a functor $(A^{r},E^{r})$\pageoriginale on the
category $\mathcal{F} \cdot  
C_{\Theta,-\iota}(\A)\to ExC(\A,\Theta;(1,0),0,\break (-r,-\iota))$ such that
$(A^{r+1},E^{r+1})$ is the left derived couple of $(A^{r},E^{r})$.
\end{remark}


In the case where $\Theta=\bfZ$, the group of integers, and
$\iota=+1$, there is a strong motivation to index the spectral
sequence with the filtration index $p$, as above, and the
complementary index $q=\theta-p$ where $\theta$ denotes the total
degree of the object. In particular, we have
$H_{p+q}(E^{0}_{p,\ast})=E^{1}_{p,q}$ in terms of the complementary
index. The complementary index notation is motivated by the
Leray-Serre spectral sequence of a map $p:E\to B$ where the main
theorem asserts that there is a spectral sequence with
$E^{2}{p,q}=H_{p}(B,H_{q}(F))$ coming from a filtration on the chains
of the total space $E_{\bigdot}$, $F$ being the fibre of the morphism
$p$.

\begin{remark}\label{chap1-rem5.3}
The filtration on a filtered complex $X$ defines a filtration on the
homology $H(X)$ of $X$ by the relation that
$$
F_{p}H_{\ast}(X)=\Iim(H_{\ast}(F_{p}X)\to H_{\ast}(X)).
$$

Now this filtration has something to do with the terms
$E^{r}_{p,\ast}$ of the spectral sequence. We carry this out for the
following special case which is described by the following
definition. 
\end{remark}

\begin{definition}\label{chap1-defi5.4} 
A filtered object\index{filtered object} $X$ in a pointed category is
positive provided 
$F_{p}X=0$ (cf. (\ref{chap1-rem1.3})) for $p<0$. A filtered
$\Theta$-graded object $X$ has a locally finite
filtration\index{filtered objects with locally finite filtration} provided
for each $\theta\in\Theta$ there exists integers $m(\theta)$ and
$n(\theta)$ such that
$$
F_{p}X_{\theta}=0\q\text{for}\q p<m(\theta)\q\text{and}\q
  F_{p}X_{\theta}=X_{\theta}\q\text{for}\q n(\theta)<p.
$$
\end{definition}

\begin{proposition}\label{chap1-prop5.5}
Let $X$ be a locally finite filtered $\Theta$-graded complex $X$ over
an abelian category $\A$. Then for a given $\theta\in\Theta$ and
filtration index $p$, if $r>\max(n(\theta)+1-p$, $p-m(\theta-\iota))$,
then we have
$E^{r}_{p,\theta}=E^{r+1}_{p,\theta}=\ldots=F_{p}H_{\theta}(X)/F_{p-1}H_{\theta}(X)=E^{0}H_{\theta}(X)$. 
\end{proposition}

\begin{proof}
We\pageoriginale use the characterization of the terms $E^{r}$ given at the end of
(\ref{chap1-rem4.4}). For
$A^{1}_{p,\theta}\xrightarrow{\beta}E^{1}_{p,\theta}\xrightarrow{\gamma}A^{1}_{p-1,\theta-\iota}$
we form a subquotient using $\alpha^{r-1}:A^{1}_{p,\theta}\to
A^{1}_{p+r-1,\theta}$ and $\alpha^{r-1}:A^{1}_{p-r,\theta-\iota}\to
A^{1}_{p-1,\theta-\iota}$ where $A^{1}_{p+r-1,\theta}=H_{\theta}(X)$
and $A^{1}_{p-r,\theta-\iota}=0$ under the above conditions on
$r$. Thus the term
$E^{r}=\gamma^{-1}(\Iim(\alpha^{r-1}))/\beta(\ker(\alpha^{r-1}))$ has
the form
$$
E^{r}_{p,\theta}=\gamma^{-1}(\Iim(0))/\beta(\ker(\alpha^{r-1}))=\Iim(\beta)/\beta(\ker(H_{\theta}(F_{p}X)\to
H_{\theta}(X)), 
$$
and this is isomorphic under $\beta$ to the quotient 
$$
A^{1}_{p,\theta}/((\ker(H_{\theta}(F_{p}X)\to
H_{\theta}(X))+\Iim(H_{\theta}(F_{p-1}X)\to H_{\theta}(F_{p}X))). 
$$

This quotient is mapped isomorphically by $\alpha^{r-1}$ to the
following subquotient of $H_{\theta}(X)$, which is just the associated
graded object\index{associated graded object} for the filtration on $H(X)$ defined in
(\ref{chap1-rem5.3}), 
$$
\Iim(H_{\theta}(F_{p}X)\to H_{\theta}(X))/\Iim(H_{\theta}(F_{p-1}X)\to
H_{\theta}(X))=E^{0}H_{\theta}(X). 
$$

This proves the proposition.
\end{proof}

This proposition and the next are preliminaries to the spectral
mapping theorem.

\begin{proposition}\label{chap1-prop5.6}
Let $f:L\to M$ be a morphism of locally finite filtered
$\Theta$-graded objects over an abelian category $\A$. If the morphism
of associated $\bfZ\times \Theta$-graded objects $E^{0}(f):E^{0}(L)\to
E^{0}(M)$ is an isomorphism, then $f:L\to M$ is an isomorphism.
\end{proposition}

\begin{proof}
For $F_{p}L_{\theta}=0$, $F_{p}M_{\theta}=0$ if $p<m(\theta)$ and
$F_{p}L_{\theta}=L_{\theta}$, $F_{p}M_{\theta}=M_{\theta}$ if
$p>n(\theta)$ we show inductively on $p$ from $m(\theta)$ to
$n(\theta)$ that $F_{p}f:F_{p}L_{\theta}\to F_{p}M_{\theta}$ is an
isomorphism. To begin with, we note that by hypothesis
$F_{m(\theta)}=E^{0}_{m(\theta),\theta}$ is an isomorphism. If the
inductive statement is true for $p-1$, then it is true for $p$ by
applying the ``5-lemma'' to the short exact sequence
$$
0\to F_{p-1,\theta}\rightharpoondown F_{p,\theta}\to
E^{0}_{p,\theta}\to 0.
$$

Since\pageoriginale the induction is finished at $n(\theta)$, this
proves the proposition.
\end{proof}

This proposition is true under more general circumstances which we
come back to after the next theorem.

\begin{theorem}\label{chap1-thm5.7}
Let $f:X\to Y$ be a morphism of locally finite filtered
$\Theta$-graded complexes over an abelian category $\A$. If for some
$r\geq 0$ the term $E^{r}(f):E^{r}(X)\to E^{r}(Y)$ is an isomorphism,
then $H(f):H(X)\to H(Y)$ is an isomorphism.
\end{theorem}

\begin{proof}
Since $E^{r+1}=H(E^{r})$ as functors, we see that all $E^{r'}(f)$ are
isomorphisms for $r'\geq r$. For given $\theta\in \Theta$ and
filtration index $p$ we know by (\ref{chap1-prop5.5}) that
$E^{r}_{p,\theta}=E^{0}_{p}H_{\theta}$ for $r$ large enough. Hence
$E^{0}H(f)$ is an isomorphism, and by (\ref{chap1-prop5.6}) we deduce
that $H(f)$ is an isomorphism. This proves the theorem.
\end{proof}

This theorem illustrates the use of spectral sequences to prove that a
morphism of complexes $f:X\to Y$ induces an isomoprhism $H(f):H(X)\to
H(Y)$. The hypothesis of locally finite filtration is somewhat
restrictive for general cyclic homology, but the general theorem,
which is contained in \cite{Eilenberg1962}, is clearly given in
their article. The modifications involve limits, injective limits as
$p$ goes to plus infinity and projective limits as $p$ goes to minus
infinity. We explain these things in the next section on the filtered
complex related to a double complex. 

\section{The filtered complex associated to a double
  complex}\label{chap1-sec6}

For the theory of double complexes we use the simple $\bfZ\times \bfZ$
grading which is all we need in cyclic homology. Firstly, we consider
an extension of (\ref{chap1-prop5.6}) for filtered objects which are
constructed from a graded object.

\begin{remark}\label{chap1-rem6.1}
Let $\A$ denote an abelian category with countable products and
countable coproducts. For a $\bfZ$-graded object $SX_{p}$ we form the
object $X_{\bigdot}=\prod\limits_{i\leq a}X_{i}\times
\coprod\limits_{a<i}X_{i}$ with filtration
$F_{p}X_{\bigdot}=\prod\limits_{i\leq p}X_{i}$. The definition of
$X_{\bigdot}$ is independent\pageoriginale of $a$. With these
definitions the natural morphisms
$$
X_{\bigdot}\to \lim\limits_{\leftarrow
  p}X_{\bigdot}/F_{p}X_{\bigdot}\q\text{and}\q \lim\limits_{\to
  p}F_{p}X_{\bigdot}\to X.
$$
are isomorphisms. In general for any filtered object $X$ these natural
morphisms are defined. If the first morphism is an isomorphism, then
$X$ is called complete, if the second morphism is an isomorphism, then
$X$ is called cocomplete, and if the two morphisms are isomorphisms,
then $X$ is called bicomplete. With these definitions we have the
following extension of (\ref{chap1-prop5.6}) not proved here.
\end{remark}

\begin{remark}\label{chap1-rem6.2}
Let $f:L\to M$ be a morphism of bicomplete filtered objects over an
abelian category with countable products and coproducts. If
$E^{0}(f):E^{0}(L)\to E^{0}(M)$ is an isomorphism, then $f:L\to M$ is
an isomorphism of filtered objects.
\end{remark}

Now we consider double complexes and their associated filtered
complexes which will always be constructed so as to be bicomplete.

\begin{definition}\label{chap1-defi6.3}
Let $\A$ be an abelian category. A double complex\index{double complex} $X_{\bigdot\bigdot}$
over $\A$ is a $\bfZ\times \bfZ$-graded object with two morphisms
$d'=d'(X)$, $d''=d''(X):X_{\bigdot\bigdot}\to X_{\bigdot\bigdot}$ of
degree $(-1,0)$ and $(0,-1)$ respectively satisfying $d'd'=0$,
$d''d''=0$, and $d'd''+d''d'=0$. A morphism of double complexes
$f:X_{\bigdot\bigdot}\to Y_{\bigdot\bigdot}$ is a morphism of graded
objects such that $d'(Y)f=fd'(X)$ and $d''(Y)f=fd''(X)$. With the
composition of graded morphisms we define the composition of morphisms
of double complexes. We denote the category of double complexes over
$\A$ by $DC(\A)$. 
\end{definition}

There are two functors $DC(\A)\to \F\cdot C(\A)$ from double complexes
to bicomplete filtered complexes corresponding to a filtration on the
first variable or on the second variable.

\begin{definition}\label{chap1-defi6.4}
Let $X_{\bigdot\bigdot}$ be a double complex over the abelian category
$\A$. We form:
\begin{enumerate}
\renewcommand{\labelenumi}{(\theenumi)}
\item the\pageoriginale filtered graded object ${}^{I}X_{\bigdot}$
  with
$$
{}^{I}X_{n}=\prod\limits_{i+j=n,i\leq
  a}X_{i,j}\times\coprod\limits_{i+j=n,i>a}X_{i,j}\q\text{and}\q
{}^{I}F_{p}X_{n}=\prod\limits_{i+j=n,i\leq p}X_{i,j},
$$

\item the filtered graded object ${}^{II}X_{\bigdot}$ with
$$
{}^{II}X_{n}=\prod\limits_{i+j=n,j\leq
  a}\times\coprod\limits_{i+j=n,j>a}X_{i,j}\q\text{and}\q
{}^{II}F_{p}X_{n}=\coprod\limits_{i+j=n,j\leq p}X_{i,j}
$$
and in both cases the differential is $d=d'+d''$, making the filtered
graded objects into bicomplete filtered complexes.
\end{enumerate}
\end{definition}

\begin{remark}\label{chap1-rem6.5}
Using the complementary degree indexing notation considered in
(\ref{chap1-rem5.2}), we see that
$$
E^{0}_{p,q}({}^{I}X)=X_{p,q}\text{~ with~ } d^{0}=d''\text{~ and~ }
E^{0}_{p,q}({}^{II}X)=X_{q,p}\text{~ with~ } d^{0}=d',
$$
and the $E^{1}$ terms are the partial homology modules of the double
complex with respect to $d''$ and $d$ respectively. The $d^{1}$
differentials are induced by $d'$ and $d''$ respectively, and the
$E^{2}$ term is the homology of $(E^{1},d^{1})$, and 
$$
{}^{I}E^{2}_{p,q}=H_{p}(H_{q}(X,d''),d')\q\text{and}\q
{}^{II}E^{2}_{p,q}=H_{q}(H_{p}(X,d'),d''). 
$$

These considerations in this section are used in the full development
of cyclic homology, and they are included here for the sake of
completeness. 
\end{remark}


