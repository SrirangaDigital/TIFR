\thispagestyle{empty}

\begin{center}
\textbf{Lectures on}\\[5pt] 
\textbf{Numerical Methods For Time Dependent Equations}\\[5pt]
\textbf{Applications to Fluid Flow Problmes}
\vfill

{\bf By}
\medskip

{\large\bf P. Lascaux}
\vfill


{\bf Tata Institute of Fundamental Research}

{\bf Bombay}

{\bf 1976}
\end{center}

\eject

\thispagestyle{empty}

\begin{center}
\textbf{Lectures on}\\[5pt] 
\textbf{Numerical Methods For Time Dependent Equations}\\[5pt]
\textbf{Applications to Fluid Flow Problmes}
\vfill

{\bf By}
\medskip

{\large\bf P. Lascaux}
\vfill





{\bf Notes By}
\medskip

{\large\bf S. Kesavan, M. Vanninathan}
\vfill

{\bf Tata Institute of Fundamental Research}

{\bf Bombay}

{\bf 1976}
\end{center}

\eject

\thispagestyle{empty}
~\vfill

\begin{center}

{\bf\copyright  Tata Institute of Fundamental Research, 1976}
\vskip 1cm

\parbox{0.7\textwidth}{No part of this book may be reproduced in any form by print.
  microfilm or any othere means without written permission from the
  Tata Institute of Fundamental Research. Colaba, Bombay 400 005}
\vfill

Printed in India By\\
Anil D. Ved At Prabhat Printers. Bombay 4000 004\\[10pt]
And Published By\\
The Tata Institute of Fundamental Research.
\end{center}



\chapter{Foreword}

The solution of time dependent equations of hydrodynamics is a subject
of great importance. Except for some very particular cases, the
solution cannot be obtained in an analytic form which, in passing,
causes difficulties when onw wishes to test a numerical method because
one only has very few solutions, chiefly related to 2-dimensional
problems.

For the 1-dimensional problems, the numerical methods studied in these
notes are the method of characteristics and the method of finite
differences. Unfortunately, we had not much time to treat
2-dimensional problems. But the last chapter is an introduction of the
method of finite elements which one can utilise for solving them.

Here we have essentially restricted ourselves to non-viscous fluids
and we have especially studied the cases of propagation of shocks. To
resolve this problem, we have presented a method of ``shock-fitting''
if one uses the method of characteristics and a method of
pseudo-viscosity if one wants to use the method of finite differences.

The course is mainly concentrated on the study of the stability of the
various schemes. We have considered only the stability for linearised
problems. A rigorous analysis in the nonlinear case is impossible at
the present moment.

In the first part of the course, we have chosen to study the schemes
for the three particularly simple model equations (in one-space
variable and in one time-variable): the heat equation, the wave
equation, and the advection equation. 

To do this, we have first introduced the mathematical notions of the
hyperbolic system of equations, weak solutions of the equations, energy
inequalities and the boundary conditions for the problem to be
well-posed.

Next, we have studied the consistency and stability of sufficiently
large number of schemes by obtaining energy inequalities using the
Fourier transform.

In the second part of the course, we have shown the practical
application of these numerical methods to the solution of the
equations of hydrodynamics.

Thereforem, this course covers only a very small portion of the vast
subject of the Numerical approximation of the equations of Fluid
Mechanics. The interested reader can refer to the various articles and
works given in the references. He can also find a very complete
bibliography in the book of ROACHE. Every two years, a Congress on the
Numerical methoda of Fluid mechanics takes place whose Proceedings
furnish precise details. Numerous articles on this subject are
published in the following 2 reviews:

Physics of Fluids and especially Journal of Computational Physics.

I thank all those who have enabled me to deliver this course, in
particular Professor K.G. Ramanathan and Professor R. Sridharan of
T.I.F.R., Bombay. My thanks go to Messrs. Kesavan and Vanninathan who
have written these notes with great clarity and within a very short
time. I appreciate the discussions I had with them and with some of
their collegues at the Indian Institute of Science, Bombay.

I conclude by saying that India is a wonderful country which has won
my heart and I hope to have another opportunity to visit it again.
\bigskip

\hfill {\large\bf P. Lascaus}

