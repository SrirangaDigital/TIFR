\thispagestyle{empty}

\chapter*{Introduction}\pageoriginale


1.~An interesting but still open problem in algebraic geometry is the following:

\medskip
\noindent
{\bf ZARISKI'S PROBLEM.}~{\em If $X$ is an affine algebraic variety
  over an algebraically closed field $k$ such that $X \times
  \bA^{1}_{k}\cong \bA^{3}_{k}$, where $\bA^{n}_{k}$ denotes the
  $n$-dimensional affine space over $k$, is $X$ isomorphic to
  $\bA^{2}_{K}$?} 
\medskip

In considering this problem it seems important and indispensable to
have algebraic (or topological) characterizations of the affine plane
$\bA^{2}_{k}$ as an algebraic variety. Several attempts have been made
toward this direction (\cf \cite{45}, \cite{32}), though the obtained
characterizations are not good enough to answer the Zariski's
Problem. A main motivation in writing these notes is to put together
the results which have been obtained so far surrounding this problem. 

The said assumption $X\times \bA^{1}_{k}\cong \bA^{3}_{k}$ implies
the following:
\begin{enumerate}
\renewcommand{\theenumi}{\arabic{enumi}}
\renewcommand{\labelenumi}{(\theenumi)}
\item {\em $X$ is a nonsingular affine unirational surface,}

\item {\em the\pageoriginale affine coordinate ring $A$ of $X$ is a
  unique factorization domain whose invertible elements are
  constants,} i.e., $A^{\ast}=k^{\ast}$,

\item {\em there lie sufficiently many rational (not necessarily
  nonsingular) curves with only one place at infinity on $X$.}
\end{enumerate}

In looking for a criterion for $X$ to be isomorphic to
$\mathbb{A}^{2}$ it will be reasonable to assume that $X$ satisfies
the above two conditions (1) and (2), though the third condition has
to be made more precise (or improved). A precision of the third
condition above is the next condition:

\eject

\begin{itemize}
\item[$(3')$] {\em $X$ has a nontrivial action of the additive group
  scheme $G_{a}$.}
\end{itemize}

Then the conditions (1), (2) and $(3')$ are necessary and sufficient
for $X$ to be isomorphic to $\mathbb{A}^{2}$ (\cf Theorem 3.1, Chapter
I). When $G_{a}$ acts on an affine scheme $X=\Spec(A)$, the
$G_{a}$-action can be interpreted in terms of a locally finite
iterative higher derivation on $A$. Indeed, several problems
concerning the $G_{a}$-action, \eg to find the subring $A_{0}$ of
invariants in $A$ and to investigate the properties of $A_{0}$ and the
canonical morphism $\Spec (A)\to \Spec(A_{0})$ induced by the
injection $A_{0}\hookrightarrow A$, become easier to treat by
observing the locally finite iterative higher derivation on $A$
associated with the $G_{a}$-action. The first two sections of Chapter
I are devoted to the study of locally finite (iterative) higher
derivations on $k$-algebras.

Instead of the condition $(3')$ one may consider the next milder condition
\begin{itemize}
\item[$(3'')$] {\em $X$\pageoriginale has an algebraic family
  $\mathscr{F}$ of closed curves on $X$ parametrized by a rational
  curve such that a general member of $\mathscr{F}$ is an affine
  rational curve with only one place at infinity and that two distinct
general members of $\mathscr{F}$ have no intersection on $X$.}
\end{itemize}

If $\Char(k)=0$ and $X$ satisfies the conditions (1) and (2), the
conditions $(3')$ and $(3'')$ are equivalent to each other (\cf
Theorem 2.3, Chapter I); indeed, a general member of $\mathscr{F}$ is
isomorphic to $\mathbb{A}^{1}$. However, if either one of the
conditions (1) and (2) is dropped the equivalence of $(3')$ and
$(3'')$ no longer holds (\cf 2.4, Chapter I).

In connection with the condition $(3'')$ we are interested in an
algebraic family $\mathscr{F}$ on a nonsingular affine surface, whose
general members are isomorphic to $\mathbb{A}^{1}$. We have the
following result (\cf Theorem 4.1.2, Chapter I):

{\em Let $S$ be a nonsingular variety over $k$ and let $f:X\to S$ be a
  faithfully flat, affine morphism of finite type such that every
  fiber of $f$ is irreducible. Assume that the general fiber of $f$
  are isomorphic to $\bA^{1}$. Then there exist a nonsingular
  variety $S'$ over $k$ and a faithfully flat, finite, radical
  morphism $S'\to S$ such that $\fprod{X}{S'}{S}$ is an
  $\bA^{1}$-bundle over $S'$. Thus, if $\Char(k)=0$ then $X$ is an
  $\bA^{1}$-bundle over $S$, and if $\Char(k)>0$ the generic fiber $f$
  is a purely inseparable $k(S)$-form of $\mathbb{A}^{1}$.} (\cf 4.6,
Chapter I).

We\pageoriginale are interested especially in the case where $S$ is
the projective line $\mathbb{P}^{1}$ over $k$. Affine
$\mathbb{A}^{1}$-bundles over $\mathbb{P}^{1}$ are classified (\cf
Theorem 5.5.4, Chapter I), while the case where the generic fiber of
$f$ is a purely inseparable $k(\mathbb{P}^{1})$-form of
$\mathbb{A}^{1}$ will be studied more closely in Chapter III in
connection with unirational (irrational) surfaces defined over $k$.

The Zariski's problem is generalized as follows:

\medskip
\noindent
{\bf CANCELLATION PROBLEM.}~{\em Let $A$ and $B$ be $k$-algebras such
  that $A[x_{1},\ldots,x_{n}]$ is $k$-isomorphic to
  $B[y_{1},\ldots,y_{n}]$, where $x_{1},\ldots, x_{n}$\break  and
  $y_{1},\ldots,y_{n}$ are indeterminates. Is $A$ then $k$-isomorphic
  to $B$?}
\medskip

A $k$-algebra $A$ is said to be strongly $n$-invariant if $A$
satisfies the condition: If any $k$-algebra $B$ and a $k$-isomorphism
$\theta$:
$A[x_{1},\ldots,x_{n}]\xrightarrow{\sim}B[y_{1},\ldots,y_{n}]$ then
$\theta(A)=B$. The property that $A$ is strongly $1$-invariant is
closely related to the property that $A$ is not birationally ruled
over $k$ (\cf. Lemma 6.2, Chapter I), and the strong $1$-invariance of
a $k$-algebra $A$ is studied via locally finite (iterative) higher
derivation on $A$ (\cf Lemma 6.3, Proposition 6.6.2, etc., Chapter I).


2.~The significance of studying a family of (nonsingular) rational
curves with only one place at infinity on a nonsingular affine
rational surface may be gathered from the foregoing
discussions. Several important results have been obtained in this line
(\cf Abhyankar-Moh \cite{2}, Moh \cite{38} and Abhyankar-Singh
\cite{3}).

Let\pageoriginale $k$ be an algebraically closed field of
characteristic $p$. Let $C_{0}$ be an irreducible curve with only one
place at infinity on $\mathbb{A}^{2}:=\Spec(k[x,y])$ defined by
$f(x,y)=0$. Embed $\mathbb{A}^{2}$ into the projective plane
$\mathbb{P}^{2}$ as the complement of a line $\ell_{0}$. Let $C$ be
the closure of $C_{0}$ in $\mathbb{P}^{2}$, let
$C\cdot\ell_{0}=d_{0}\cdot P_{0}$ and let $d_{1}=\mult_{P_{0}}C$. Let
$C_{\alpha}$ be the curve on $\mathbb{A}^{2}$ defined by
$f(x,y)=\alpha$ for $\alpha\in k$ and let $\Lambda(f)$ be the linear
pencil on $\mathbb{P}^{2}$ spanned by $C$ and $d_{0}\ell_{0}$. Then
the results are stated as follows:
\begin{enumerate}
\renewcommand{\theenumi}{\roman{enumi}}
\renewcommand{\labelenumi}{(\theenumi)}
\item IRREDUCIBILITY THEOREM (Moh \cite{38}; \cf Section 1, Chapter
  II).

{\em Assume that $p\times d_{0}$ or $p\times d_{1}$. Then the curve
  $C_{\alpha}$ is an irreducible curve with only one place at infinity
for an arbitrary constant $\alpha$ of $k$.}

\item EMBEDDING THEOREM (Abhyankar-Moh \cite{2}; \cf Section 1,
  Chapter II). {\em Assume that $p\times d_{0}$ or $p\times d_{1}$,
    and that $C_{0}$ is nonsingular and rational. Then there exists a
    biregular automorphism of $\mathbb{A}^{2}$ which maps $C_{0}$ into
    the $y$-axis.}

\item FINITENESS THEOREM (Abhyankar-Singh \cite{3}; \cf Section 4,
  Chapter II). {\em Assume that} $p=0$. By an embedding of $C_{0}$
  into $\mathbb{A}^{2}$ we mean a biregular mapping $\epsilon$ of
  $C_{0}$ into $\mathbb{A}^{2}$; two embeddings $\epsilon_{1}$ and
  $\epsilon_{2}$ of $C_{0}$ into $\mathbb{A}^{2}$ are said to be
  equivalent to each other if there exists a biregular automorphism
  $\rho$ of $\mathbb{A}^{2}$ such that $\epsilon_{2}=\rho\cdot
  \epsilon_{1}$. {\em Then there exist only finitely many equivalence
    classes of embeddings of $C_{0}$ into $\mathbb{A}^{2}$.}
\end{enumerate}

In their proofs the main roles are played by the theory of
approximate\pageoriginale roots of polynomials, i.e., the theory of
generalized Tschirnhausen transformations. We shall present more
geometric proofs of these theorems (though we could not prove the
third theorem in full generality), which are based on the notions of
admissible data and the Euclidean transformations (as well as the
$(e,i)$-transformations) associated with admissible data. Roughly
speaking, our idea of proof is explained as follows.

Let $X$ be a nonsingular affine rational surface defined over $k$ and
let $C_{0}$ be an irreducible closed curve on $X$ such that $C_{0}$
has only one place at infinity. Suppose that there exists an
admissible datum $\mathscr{D}=\{V,X,C,\ell_{0},\Gamma,d_{0},d_{1},e\}$
be an admissible datum for $(X,C_{0})$ (\cf Definition 1.2.1, Chapter
II). $C$ is then linearly equivalent to $d_{0}(e\ell_{0}+\Gamma)$ on
$V$, and the linear pencil $\Lambda$ on $V$ spanned by $C$ and
$d_{0}(e\ell_{0}+\Gamma)$ has base points centered at $P_{0}:=C\cap
\ell_{0}$ and its infinitely near points. If $p\times (d_{0},d_{1})$
the Euclidean transformation or the $(e,i)$-transformation of $V$
associated with $\mathscr{D}$ plays a role of producing a new
admissible datum
$\widetilde{\mathscr{D}}=\{\widetilde{V},X,\widetilde{C},\widetilde{\ell}_{0},\widetilde{\Gamma},\widetilde{d}_{0},\widetilde{d}_{1},\widetilde{e}\}$
for $(X,C_{0})$ such that either $\widetilde{d}_{0}<d_{0}$ or
$\widetilde{d}_{0}=d_{1}$ and $\widetilde{d}_{1}<d_{1}$ and that
$p\times (\widetilde{d}_{0},\widetilde{d}_{1})$. After the Euclidean
transformations or the $(e,i)$-transformations associated with
admissible data repeated finitely many times we reach to an admissible
datum
$\widehat{\mathscr{D}}=\{\widehat{V},X,\widehat{C},\widehat{\ell}_{0},\widehat{\Gamma},\widehat{d}_{0},\widehat{d}_{1},\widehat{e}\}$
for $(X,C_{0})$ such that $\widehat{d}_{0}=\widehat{d}_{1}=1$. Then,
by the $(\widehat{e},\widehat{e})$-transformation of $\widehat{V}$
associated with $\widehat{\mathscr{D}}$, we obtain a nonsingular
projective surface $V'$ such that the proper transform $\Lambda'$ of
$\Lambda$ on $\Lambda'$ is free from\pageoriginale base points, that
if $\Delta'$ is the member of $\Lambda'$ corresponding to
$d_{0}(e\ell_{0}+\Gamma)$ of $\Lambda$ then $\Delta'$ is the unique
irreducible member of $\Lambda'$, that the fibration of $V'$ defined
by $\Lambda'$ has a cross-section $S$ and $V'-S\cup \Supp(\Delta')$ is
isomorphic to $X$, and that if $C'$ is the proper transform of $C$ on
$V'$ then $C'-C'\cap S$ is isomorphic to the given curve
$C_{0}$. Retaining the notations $C_{0}$, $C$, $\ell_{0}$, $d_{0}$ and
$d_{1}$ as before the statement of the irreducibility theorem,
$\{\mathbb{P}^{2}_{k},\mathbb{A}^{2}_{k},
C,\ell_{0},\phi,d_{0},d_{1},1\}$ is an admissible datum for
$(\mathbb{A}^{2}_{k},C_{0})$. Hence by the foregoing arguments we know
that the curve $C_{\alpha}:f(x,y)=\alpha$ is an irreducible curve with
only one place at infinity for every $\alpha\in k$, and that if
$C_{0}$ is isomorphic to $\mathbb{A}^{1}_{k}$ then $C_{\alpha}$ is
isomorphic to $\bA^{1}_{k}$ for every $\alpha\in k$. The theorems (i)
and (ii) can be proved in this fashion. The foregoing process of
eliminating the base points of $\Lambda(f)$ in conjunction with
Artin-Winters's theorem \cite{7} on degenerate fibers of a curve of
genus $g$ and the Kodaira vanishing theorem by Ramanujam \cite{46}
proves the weakened version of the finiteness theorem. Sections 1 and
4 of Chapter II are devoted to the proofs of these theorems.

Furthermore, we can give a new proof of the structure theorem on the
automorphism group $\Aut_{k}k[x,y]$ over a field of arbitrary
characteristic, which is based on the foregoing arguments of
eliminating the base points of the pencil $\Lambda(f)$ and an easy
lemma on reducible fibers of a fibration by rational curves (\cf
Sections 2 and 3 of Chapter II).

In Sections 2 and 6 of Chapter II, some related topics are
discussed. Let $C_{0}$ be a nonsingular rational curve on
$\bA^{2}_{k}:=\Spec(k[x,y])$\pageoriginale defined by $f(x,y)=0$;
$C_{0}$ may have 
one or more places at infinity. Let $C_{\alpha}$ be the curve on
$\bA^{2}_{k}$ defined by $f(x,y)=\alpha$ for $\alpha\in k$, and let
$\Lambda(f)$ be the linear pencil on $\bP^{2}_{k}$ defined by the
inclusion of fields $k(f)\hookrightarrow k(x,y)$, where $\bA^{2}_{k}$
is embedded into $\bP^{2}_{k}$ as the complement of a line
$\ell_{0}$. Then the generic member of $\Lambda(f)$ is a rational
curve if and only if $f$ is a field generator, i.e., $k(x,y)=k(f,g)$
for some $g\in k(x,y)$ (\cf Lemma 2.4.1, Chapter II), and if $f$ is a
field generator then $C_{0}$ has at most two points (including
infinitely near points) on the line $\ell_{0}$ at infinity (\cf Lemma
2.4.2, Chapter II). In Section 6 of Chapter II the following theorem
is proved:

\newpage

{\em Assume that the characteristic of $k$ is zero. With the above
  notations, $f=c(x^{d}y^{e}-1)$ after a suitable change of
  coordinates $x$, $y$ of $k[x,y]$, where $c\in k^{\ast}$ and $d$ and
  $e$ are positive integers with $(d,e)=1$, if and only if the
  following conditions are met:}
\begin{itemize}
\item[(a)] {\em $f$  is a field generator,}

\item[(b)] {\em $C_{\alpha}$ has exactly two places at infinity for
  almost all $\alpha\in k$,}

\item[(c)] {\em $C_{\alpha}$ is connected for every $a\in k$.}
\end{itemize}

In Section 5 of Chapter II, we shall study the structure of the affine
coordinate ring $A:=k[x,y,f/g]$ of a hypersurface on $\bA^{3}_{k}$ of
the type: $gz-f=0$, where $f$, $g\in k[x,y]$ and $(f,g)=1$. Namely, we
shall show that the divisor class group $C\ell(A)$ and the
multicative\pageoriginale group $A^{\ast}$ are completely determined
if $\Spec(A)$ has only isolated singularities, and that, in case of
$\Char(k)=0$, $\Spec(A)$ has nontrivial $G_{a}$-action if and only if
$g\in k[y]$ after a suitable change of coordinates $x$, $y$ of
$k[x,y]$.

3.~Let $k$ be an algebraically closed field of characteristic
$p>0$. Let $X$ be a nonsingular projective surface, and let $f:X\to
\mathbb{P}^{1}$ be a surjective morphism such that a general fiber of
$f$ is an irreducible rational curve with a single cusp as its
singularity. Then the generic fiber $X_{\mathscr{R}}$ of $f$ with the
unique singular point deleted off is a purely inseparable form of
$\mathbb{A}^{1}$ over the function field
$\mathscr{R}:=k(\mathbb{P}^{1})$, and $X$ is a unirational surface
over $k$. In Chapter III, we shall describe the structure of such a
surface $X$ in the case where the arithmetic genus $g$ of $X$ is
either $1$ or $2$, under the additional assumption that $f$ has a
rational cross-section. When $g=1$ then $p$ is either $2$ or $3$ and
$X$ is a unirational quasi-elliptic surface; there exist $K3$-surfaces
and surfaces with canonical dimension $\kappa=1$ besides rational
surfaces (\cf Theorems 2.1.1 and 2.1.2, Chapter III). When $g=2$ then
$p$ is either 2 or 5; if $p=5$ there exist $K3$-surfaces and surfaces
of general type besides rational surfaces (\cf Section 3, Chapter
III).

\newpage

\begin{center}
{\bf Notations and conventions}\pageoriginale
\end{center}
\medskip

Notations and conventions of the present notes conform to the general
current practice. Therefore we shall make some additional notes below.
\begin{enumerate}
\item Let $A$ be an algebra over a field $k$. Then $A^{\ast}$ denotes
  the multiplicative group of invertible elements of $A$; thus
  $k^{\ast}$ denotes $k-(0)$. If $A$ is an integral domain $Q(A)$
  denotes the quotient field of $A$. A unique factorization domain $A$
  is sometimes called a factorial domain (or ring). If
  $A_{\mathscr{P}}$ is factorial for every prime ideal $\mathscr{P}$
  of $A$ then $A$ is called locally factorial. For an affine
  $k$-variety $X$, the affine coordinate ring of $X$ is denoted by
  $k[X]$ if there is no fear of confusing $k[X]$ and a polynomial ring
  over $k$.

\item The $n$-dimensional affine space and projective space defined
  over $k$ are denoted respectively by $\mathbb{A}^{n}_{k}$ (or
  $\mathbb{A}^{n}$) and $\mathbb{P}^{n}_{k}$ (or $\mathbb{P}^{n}$). We
  denote by $\mathbb{A}^{1}_{\ast}$ the $k$-scheme isomorphic to the
  underlying $k$-scheme of the multiplicative $k$-group scheme
  $G_{m}:=\Spec (k[t,t^{-1}])$. The additive $k$-group scheme is
  denoted by $G_{a}$ (or $G_{a,k}$).

\item Let $V$ be a nonsingular projective surface defined over an
  algebraically closed field. Then we use the following notations:

%\begin{longtable}{l@{\;:\;}p{8cm}}
\begin{description}
\item[$K_{V}$:] a canonical divisor (or the canonical divisor class)
  of $V$.

\item[$\omega_{V}$:] the dualizing invertible sheaf on $V$, i.e.,
  $\omega_{V}\cong \mathscr{O}_{V}(K_{V})$.\break  $\chi(\mathscr{O}_{V})$
  (or $\chi(V,\mathscr{O}_{V})$): the Euler-Poincare characteristic of
  $V$, $\chi(\mathscr{O}_{V})=\sum\limits^{2}_{i=0}(-1)^{i}\dim
  H^{i}(V,\mathscr{O}_{V})$.\pageoriginale

\item[$P_{r}(V)$:] the $r$-genus of $V$ for a positive integer $r$,
  i.e., $P_{r}(V)=\dim H^{0}(V,\omega^{\otimes r}_{V})$.

\item[$P_{g}(V)$:]  the geometric genus of $V$.

\item[$P_{a}(V)$:] the arithmetic genus of $V$, i.e.,
  $p_{a}(V)=\chi(\mathscr{O}_{V})-1$.

\item[$\kappa(V)$ (or $\kappa$):]  the canonical dimension of $V$,
  i.e., $\kappa(V)={\displaystyle{\mathop{\Sup}_{r>0}}} \dim\break
  \rho_{r}(V)$ where $\rho_{r}$ is the $r$-th canonical mapping of $V$
  defined by $|rK_{V}|$.
\end{description}

Let $D$ and $D'$ be divisors on $V$. Then we use the notations:

\medskip
\begin{description}
\item[$\mathscr{O}_{V}(D)$:] the invertible sheaf attached to $D$.

\item[$p_{a}(D)$:] the arithmetic genus of $D$, i.e.,
  $p_{a}(D)=\dfrac{1}{2}(D\cdot D+K_{V})+1$.

\item[$(D\cdot D')$:] the intersection multiplicity of $D$ and $D'$.

\item[$(D^{2})$:] the self-intersection multiplicity of $D$.

\item[$D\sim D'$:]  $D$ is linearly equivalent to $D'$.

\item[$D>0$:] $D$ is an effective divisor.

\item[$|D|$:] the complete linear system defined by $D$.

\item[$|D|-\sum m_{i}p_{i}$:] the linear subsystem of $|D|$ consisting of
  members of $|D|$ which pass through the points $p_{i}$'s with
  multiplicities $\geqq m_{i}$, where $m_{i}$'s are positive integers.
\end{description}

Let $C$ be an irreducible curve on $V$ and let $P$ be a point on
$C$. Then $\mult_{P}C$ denotes the multiplicity of $C$ at $P$.

\item Let\pageoriginale $f:W\to V$ be a birational morphism of nonsingular
  projective surfaces. If $D$ is an effective divisor on $V$ then
  $f^{\ast}(D)$ denotes the total transform (or the inverse image as a
  cycle) of $D$ by $f$; $f'(D)$ denotes the proper transform of $D$ by
  $f$. If $C$ is an irreducible curve on $V$, $f^{-1}(C)$ denotes the
  set-theoretic inverse image of $C$ by $f$. On the other hand, if
  $D'$ is an effective divisor on $W$ then $f_{\ast}(D')$ denotes the
  direct image of $D'$ by $f$ as a cycle. If $\Lambda$ is a linear
  pencil on $V$ consisting of effective divisors then $f'\Lambda$
  denotes the proper transform of $\Lambda$; namely, if
  $f^{\ast}\Lambda$ is the linear pencil on $W$ consisting of the
  total transforms $f^{\ast}D$ of members $D$ of $\Lambda$ then
  $f'\Lambda$ is the pencil $f^{\ast}\Lambda$ with all fixed
  components deleted off. If $\Lambda'$ is a linear pencil on $W$
  consisting of effective divisors then $f_{\ast}\Lambda'$ denotes the
  linear pencil on $V$ consisting of the direct images $f_{\ast}D'$ of
  members $D'$ of $\Lambda'$.

\item If $f:W\to V$ is a finite morphism of nonsingular projective
  surfaces then the notations $f^{\ast}(C)$, $f^{-1}(C)$ and
  $f_{\ast}(C')$ conform to those in the case where $f$ is a
  birational morphism.

\item Let $\Lambda$ be an irreducible linear pencil of effective
  divisors on a nonsingular projective surface $V$. An irreducible
  curve $S$ on $V$ is called a quasi-section if $S$ is not contained
  in any member of $\Lambda$ and $\Lambda$ has no base points on
  $S$. A quasi-section $S$ of $\Lambda$ is called a cross-section of
  $\Lambda$ if $(S\cdot D)=1$ for a general member $D$ of $\Lambda$.

\item A\pageoriginale surface $V$ defined over a field $k$ is said to
  be unirational over $k$ if there exists a dominating rational
  mapping $f:\mathbb{P}^{2}_{k}\to V$.

\item The present notes consist of three chapters. When we refer to a
  result stated in the same chapter we only quote the number of the
  paragraph (\eg (\cf Theorem 1.1) or (\cf 1.1)); when we refer to a
  result stated in other chapters we quote it with the number of
  chapter (\eg (\cf Theorem (I.1.1)) or (\cf I.1.1)).
\end{enumerate}
