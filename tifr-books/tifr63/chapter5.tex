
\chapter[Review of the Error Estimates for the...]{Review of the Error
  Estimates for the Finite Element  Method}\label{chap5}   

THE\pageoriginale PURPOSE OF this chapter is to state the theorems on
error estimates which are useful for our future analysis. The proof of
the theorems can be found in CIARLET \cite{key9}.

\begin{def*}
Let $\Omega\subset\mathbb{R}^n$ be an open subset, $m\geq 0$ be an
integer and $1\leq p\leq +\infty$. Then the Sobolev Space
$W^{m,p}(\Omega)$ is defined by 
$$
W^{m,p}(\Omega)=\{v\varepsilon L^p(\Omega):\partial^\alpha
v\varepsilon L^p(\Omega),\quad\text{for all } |\alpha |\leq m\}.
$$

On the space $W^{m, p}(\Omega)$ we define a norm $\parallel\centerdot
\parallel_{m,p,\Omega}$ by
$$
\parallel v\parallel_{m,p,\Omega}=\left(\int\limits_\Omega
\sum\limits_{|\alpha |\leq m}|D^\alpha v|^p\,dx\right)^{1/p},
$$
and a semi norm $|\centerdot |_{m,p,\Omega}$ by 
$$
|v|_{m,p,\Omega}=\left(\int\limits_\Omega \sum\limits_{|\alpha
|=m}|D^\alpha v|^p\,dx\right)^{1/p}.
$$

If $k$ is an integer, then we consider the quotient space
$$
W^{-k+1,p}(\Omega)=W^{k+1,p}(\Omega)/p_k(\Omega)
$$
with the quotient norm
$$
\parallel \tilde{v}\parallel_{k+1,p,\Omega}=\underset{\ell\varepsilon
\mathbb{P}_k}{\Inf}\parallel v+\ell\parallel_{k+1,p,\Omega},
$$
where\pageoriginale $\tilde{v}$ is the equivalence class containing
$v$.

We introduce a semi norm in $\tilde{W}^{k+1,p}(\Omega)$ by 
$$
|\tilde{v}|_{k+1,p,\Omega}=|v|_{k+1,p,\Omega}.
$$

Then we have 
\end{def*}

\setcounter{THM}{0}
\begin{THM}\label{chap5:THM1}
{\bf (CIARLET - RAVIART).} In $\tilde{W}^{k+1,p}(\Omega)$ the semi norm
$|\tilde{v}|_{k+1,p,\Omega}$ is a norm equivalent to the quotient norm
$\parallel v\parallel_{k+1,p,\Omega}$.
\end{THM}

Using this theorem it is easy to prove 

\begin{THM}\label{chap5:THM2}
Let $W^{k+1,p}(\Omega)$ and $W^{m,q}(\Omega)$ be such that
$W^{k+1,p}(\Omega)\hookrightarrow W^{m,q}(\Omega)$ (continuous
injection). Let 
$$
\pi\in\mathscr{L}(W^{k+1,p}(\Omega),\;W^{m,q}(\Omega))
$$
be such that for each $p\varepsilon\mathbb{P}_k$, $\pi p=p$. Then
there exists a $c=c(\Omega,\pi)$ such that for each $v\in
W^{k+1,p}(\Omega)$
$$
|v-\pi v|_{m,q,\Omega}\leq c|v|_{k+1,p,\Omega}.
$$
\end{THM}

\begin{def*}
Two open subsets $\hat{\Omega}, \Omega$ of $\mathbb{R}^n$ are said to
be affine equivalent if there exists an affine map $F$ from
$\hat{\Omega}$ onto $\Omega$ such that $F(\hat{x})=B\hat{x}+b$, where
$B$ is a $n\times n$ non singular matrix and
$b\varepsilon\mathbb{R}^n$.

We have 
\end{def*}

\begin{THM}\label{chap5:THM3}
Let $\hat{\Omega}, \Omega$ be affine equivalent with $F$ as their
affine map. Then there exist constants $\hat{c},c$ such that for all
$v\varepsilon W^{m,p}(\Omega)$,\pageoriginale
$$
|\hat{v}|_{m,p,\hat{\Omega}}\leq c\parallel B\parallel^m\;|\det
B|^{-1/p}|v|_{m,p,\Omega},
$$
and for all $\hat{v}\varepsilon W^{m,p}(\hat{\Omega})$,
$$
|v|_{m,p,\Omega}\leq\hat{c}\parallel B^{-1}\parallel^m\;|\det
B|^{1/p}\;|\hat{v}|_{m,p,\hat{\Omega}},
$$
where
$$
\hat{v}=v\bullet F.
$$ 
\end{THM}

If $h$ ($\resp \hat{h}$) is the diameter of $\Omega$ ($\resp
\hat{\Omega}$) and $p$ ($\resp \hat{p}$) is the supremum of the
diameters of all balls that can be inscribed in $\Omega$ ($\resp
\hat{\Omega}$), then we have 

\begin{THM}\label{chap5:THM4}
$\parallel B\parallel \leq h/\hat{\rho}$ and $\parallel B^{-1}
\parallel \leq \hat{h}/\rho$.
\end{THM}

\begin{def*}
Two finite elements $(\hat{K}, \hat{\Sigma}, \hat{P})$ and $(K, \Sigma,
P)$ are said to be affine equivalent if there exists an affine map
$F\hat{x}= B\hat{x}+b$ on $\mathbb{R}^n$, where $B$ is an $n \times n$
non singular matrix, and $b\varepsilon\mathbb{R}^n$ such that 

\begin{enumerate}
\item [(i)] $F(\hat{K})= K$
\item [(ii)] $\hat{p} =\{\hat{p}=p \circ F:p\varepsilon P\}$,
\item [(iii)] $\hat{\Sigma} =\{\hat{\phi}=F^{-1}\circ\phi:\phi\varepsilon
\Sigma\}$ 
\end{enumerate}
where
$$
F^{-1}\circ\phi(\hat{p})=\phi(\hat{p}\circ F^{-1}).
$$
\end{def*}

\begin{def*}
Let\pageoriginale $(K,\Sigma, P)$ be a finite element and
$v:K\to\mathbb{R}$ be a smooth function on $K$. Then by virtue of the
$P$-unisolvency of $\Sigma$ there exists a unique element, say,
$\pi_Kv\in P$, such that $\phi(\pi_Kv)=\phi(v)$ for all
$\phi\in \Sigma$. The function $\pi_Kv$ is called the
$P$-interpolate function of $v$ and the operator $\pi_K:C^\infty(K)\to
P$ is called the $P$-interpolation operator.
\end{def*}

Now we state an important theorem which is often used.

\begin{THM}\label{chap5:THM5}
Let $(\hat{K},\hat{\Sigma},\hat{P})$ be a finite element. Let $s(=0,
1, 2)$ be the maximal order of derivatives occurring in
$\Sigma$. Assume that 

\begin{enumerate}
\item [(i)] $W^{k+1,p}(\hat{K})\hookrightarrow C^s(\hat{K})$,
\item [(ii)] $W^{k+1,p}(\hat{K})\hookrightarrow W^{m,q}(\hat{K})$,
\item [(iii)] $P_k\subset\hat{P}\subset W^{m,q}(\hat{K})$,
\end{enumerate}

Then there exists a constant $C=C(\hat{K},\hat{\Sigma},\hat{P})$ such
that for all affine equivalent finite elements $(K,\Sigma,P)$ we have 
$$
|v-\pi_Kv|_{m,q,K}\leq C\;(\meas K)^{1/q-1/p}\frac{h_K^{k+1}}
{\rho_K^m}|v|_{k+1,p,K}
$$
for all $v\in W^{k+1,p}(K)$, where $\pi_K$ is a
$P$-interpolate operator, $h_K$ is the diameter of $K$ and $\rho_K$ is
the supremum of diameter of all balls inscribed in $K$.
\end{THM}

\begin{def*}
A family $(T_h)$ of triangulations of $\Omega$ is regular if 
\begin{enumerate}
\item [(i)] for\pageoriginale all $h$ and for each $K\varepsilon T_h$ the 
finite elements $(K,\Sigma, P)$ are all affine equivalent to a single
  finite element $(\hat{K},\hat{\Sigma},\hat{P})$;
\item [(ii)] there exists a constant $\sigma$ such that for all $T_h$
  and for each $K\varepsilon T_h$ we have 
$$
\frac{h_K}{\rho_K}\leq\sigma;
$$
\item [(iii)] for a given triangulation $T_h$, if 
$$
h=\max\limits_{K\varepsilon T_h}h_K,
$$
\end{enumerate}
then $h\to 0$.
\end{def*}

\setcounter{exercise}{0}
\begin{exercise}\label{chap5:exr1}
Prove that there exists a constant $C$ independent of $h$ such that 
$$
|p|_{1,k}\leq C/h\;|p|_{\circ,K}\quad\text{for all } p\varepsilon
\mathbb{P}_k. 
$$

A theorem which gives a global error bound is the following.
\end{exercise}

\begin{THM}\label{chap5:THM6}
Let us assume that 

\begin{enumerate}
\item [(i)] $\pi_h:H^{k+1}(\Omega)\to V_h$, the restriction of $\pi_h$
  to $V_h$ being the identity,
\item [(ii)] $V_h\subset\underset{K\varepsilon
T_h}{\pi}\mathbb{P}_k(K)$,
\item [(iii)] $V_h\subset H^m(\Omega)$,\pageoriginale
\item [(iv)] $u\in H^m(\Omega)\quad$ (regularity assumption), 
\end{enumerate}
Then we have 
$$
\parallel u-\pi_hu\parallel_{m,\Omega}\leq
C\;h^{k+1-m}|u|_{k+1,\Omega},
$$
where $C$ is a constant independent of $h$ and $(T_h)$ is a regular
family of triangulations.
\end{THM}

For stating a theorem on $L_2$ -error estimates we need the definition
of a regular adjoint problem.

\begin{def*}
Let $V=H^1(\Omega)$ or $H_\circ^1(\Omega)$, $H=L^2(\Omega)$. The
adjoint problem:

\begin{equation*}
\begin{cases}
\text{Find}\quad\phi\varepsilon V\quad\text{such that}\\
a(v,\phi)= (g, v)\quad\text{for all } v\varepsilon V,
\end{cases}
\end{equation*}
is said to be regular if 

\begin{enumerate}
\item [(i)] for all $g\varepsilon L^2(\Omega)$, the solution $\phi$ of
the adjoint problem for $g$ belongs to $H^2(\Omega)\cap V$;
\item [(ii)] there exists a constant $C$ such that 
$$
\parallel\phi\parallel_{2,\Omega}\leq C|g|_{\circ,\Omega}.
$$
\end{enumerate}

We now have
\end{def*}

\begin{THM}\label{chap5:THM7}
Let $(T_h)$ be a regular family of triangulations on $\Omega$ with
reference finite element $(\hat{K},\hat{\Sigma},\hat{P})$. Let $s=0$
and $n\leq 3$. Suppose\pageoriginale there exists an integer $k\geq 1$
such that $u\varepsilon H^{k+1}(\Omega) \; P_k\subset\hat{p}\subset H^1
(\hat{K})$. Assume further that the adjoint problem is regular. Then
there exists a constant $C$ independent of $h$ such that 
$$
|u-u_h|_{\circ,\Omega}\leq C \;h^{k+1}\;|u|_{k+1,\Omega}.
$$
\end{THM}
