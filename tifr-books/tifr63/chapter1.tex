
\chapter{Sobolev Spaces}\label{chap1}

IN THIS CHAPTER\pageoriginale the notion of Sobolev space $H^1(\Omega)$ is
introduced. We state the Sobolev imbedding theorem, Rellich theorem,
and Trace theorem for $H^1(\Omega)$, without proof. For the proof of
the theorems the reader is referred to ADAMS \cite{key1}.

\section{Notations}\label{chap1:subsec1.1} 
Let $\Omega\subset \mathbb{R}^n(n=1, 2\; or\; 3)$ be an open set. Let $\Gamma$
denote the boundary of $\Omega$, it is assumed to be bounded and
smooth. Let 
\begin{align*}
L^2(\Omega)&=\left\{f:\int\limits_\Omega|f|^2\,dx <\infty\right\} \quad \text{and}\\
(f, g)&=\int\limits_\Omega fg \,dx
\end{align*}
Then $L^2(\Omega)$ is a Hilbert space with $(\cdotp,\cdotp)$ as the
scalar product.

\section{Distributions}\label{chap1:subsec1.2}
Let $\mathscr{D}(\Omega)$ denote the space of infinitely
differentiable functions with compact support in
$\Omega$. $\mathscr{D}(\Omega)$ is a nonempty set. If 
\begin{equation*}
f(x)= 
\begin{cases}
\exp\left(\frac{1}{|x|^2-1}\right) & \text{if}\quad |x|<1\\
0 & \text{if}\quad |x|\geq 1
\end{cases}
\end{equation*}
then $f(x) \epsilon\mathscr{D}(\Omega),\Omega=\mathbb{R}$. 

The topology chosen for $\mathscr{D}(\Omega)$ is such that a sequence
of elements $\phi_n$ in $\mathscr{D}(\Omega)$ converges to an element
$\phi$ belonging to $\mathscr{D}(\Omega)$ in $\mathscr{D}(\Omega)$ if
there exists a compact set $K$ such that 
$$
\supp \;\phi_n, \supp \;\phi\subset K
$$ 
$D^\alpha \phi_n\to D^\alpha \phi$\pageoriginale uniformly for each
multi-index $\alpha =(\alpha_1,\ldots,\alpha_n)$ where $D^\alpha\phi$
stands for 
$$
\frac{\partial^{{\alpha}_1+\ldots +{\alpha}_n}\phi}
     {\partial^{\alpha_1}x_1\cdots\partial^{\alpha_n}x_n}.
$$

A continuous linear functional on $\mathscr{D}(\Omega)$ is said to be
a {\it distribution}. The space of distributions is denoted by
$\mathscr{D}'(\Omega)$. We use $<\cdotp,\cdotp>$ for the duality
bracket between $\mathscr{D}'(\Omega)$ and $\mathscr{D}(\Omega)$.

\begin{exam}\label{chap1:exam1}
\begin{itemize}
\item [(a)] A square integrable function defines a distribution: If
$f\epsilon L^2(\Omega)$ then 
$$
\langle f, \phi \rangle=\int\limits_\Omega f\phi \,dx 
\quad \text{for all } \phi\epsilon \mathscr{D}(\Omega)
$$
can be seen to be a distribution. We identify $L^2(\Omega)$ as a space
of distribution, i.e.
$$
L^2(\Omega)\subset\mathscr{D}'(\Omega).
$$
\item [(b)] The dirac mass $\delta$, concentrated at the origin,
  defined by 
$$
\langle\delta,\phi \rangle=\phi(0)\quad\text{for all }
\phi\epsilon\mathscr{D}(\Omega)
$$
defines a distribution. 
\end{itemize}
\end{exam}

\begin{def*}
{\bf Derivation of a Distribution}

If $f$ is a smooth function and $\phi\epsilon\mathscr{D}(\Omega)$
then using integration by parts we obtain
$$
\int\limits_\Omega\frac{\partial f}{\partial x_i}\phi \,dx= -
\int\limits_\Omega f\frac{\partial\phi}{\partial x_i} \,dx.
$$
This\pageoriginale gives a motivation for defining the derivative of a
distribution.

If $T\epsilon\mathscr{D}'(\Omega)$ and $\alpha$ is a multi index
then $D^\alpha T\epsilon\mathscr{D}'(\Omega)$ is defined by
$$
\langle D^\alpha T, \phi \rangle=(-1)^{|\alpha|}\langle T, 
D^\alpha\phi \rangle \; \forall \phi\epsilon \mathscr{D}(\Omega).
$$
If $T_n, T\epsilon \mathscr{D}'(\Omega)$ then we say $T_n\to T$ in
$\mathscr{D}'(\Omega)$ if 
$$
\langle T_n, \phi \rangle\to 
\langle T, \phi \rangle\quad \text{for all }\phi\epsilon\mathscr{D}
(\Omega).
$$

The derivative mapping $D^\alpha :\mathscr{D}'\to\mathscr{D}'$ is
continuous since if $T_n\to T$ in $\mathscr{D}'$ then
\begin{align*}
\langle D^\alpha T_n, \phi \rangle &=(-1)^{|\alpha|} \langle T_n, D^\alpha \phi \rangle\\
&\to (-1)^{|\alpha|} \langle T, D^\alpha\phi \rangle\\
&= \langle D^\alpha T, \phi \rangle\quad \text{for all } \phi\epsilon \mathscr{D}
(\Omega). 
\end{align*}
\end{def*}

\section{Sobolev Space}\label{chap1:subsec1.3} 
The Sobolev space $H^1(\Omega)$ is defined by 
$$ 
H^1(\Omega)=\left\{v\epsilon L^2(\Omega):\frac{\partial v}{\partial
  x_i} \epsilon L^2(\Omega), \;1\leq i\leq n\right\}
$$
where the derivatives are taken in the sense of distribution.
$$
f\epsilon L^2(\Omega) \text{ need not imply }  
\frac{\partial f}{\partial x_i}\epsilon L^2(\Omega).
$$

\begin{exam}\label{chap1:exam2}
Let $\Omega =[-l, l]$
\begin{equation*}
f(x)=
\begin{cases}
 -1 & \text{if}\; x< 0\\
0 & \text{if}\; x\geq 0.
\end{cases}
\end{equation*}
Then $f\epsilon L^2[-l, l]$; but $df/dx =\delta$ is not given by a
locally integrable function and hence not by an $L^2$ function.  
\end{exam}

We\pageoriginale define an inner product $(\cdotp,\cdotp)_1$ in
$H^1(\Omega)$ as follows:
$$
(u, v)_1=(u, v)+\sum\limits_{i=1}^{n}\left(\frac{\partial u}{\partial x_i},
\frac{\partial v}{\partial x_i}\right)\quad \text{for all } u, v \epsilon
H^1(\Omega). 
$$
Let $\parallel \cdotp \parallel_1$ be the norm associated with this
inner product. Then 

\begin{lem}\label{chap1:lem1}
$H^1(\Omega)$ with $\parallel \cdotp\parallel_1$ is a Hilbert spaces.
\end{lem}

\begin{proof}
Let $u_j$ be a Cauchy sequence in $H^1(\Omega)$. This imply 
$$
\{u_j\},\left\{\frac{\partial u_j}{\partial x_i}\right\} 
i=1, 2, \ldots, n 
$$
are Cauchy in $L^2$. Hence there exists $v, v_i\epsilon L^2(\Omega)
1\leq i\leq n$ such that 
\begin{gather*}
u_j\to v \; \text{in} \; L^2(\Omega),\\
\frac{\partial u_j}{\partial x_i}\to v_i \; 
\text{in} \; L^2(\Omega), 1\leq i\leq n,
\end{gather*}
For any $\phi\epsilon\mathscr{D}(\Omega)$,
$$
\left\langle\frac{\partial u_j}{\partial x_i},\phi \right\rangle = -\left\langle u_j,
\frac{\partial\phi}{\partial x_i}\right\rangle\to - \left\langle u, \frac{\partial\phi}
{\partial x_i}\right\rangle=\left\langle\frac{\partial u}{\partial x_i},\phi\right\rangle.
$$
But 
$$
\left\langle<\frac{\partial u_j}{\partial x_i}, \phi\right\rangle\to 
\left\langle v_i, \phi \right\rangle.
$$
Hence
$$
v_i=\frac{\partial u}{\partial x_i}.
$$
Thus
\begin{align*}
u_j &\to u \quad\text{in} \quad L^2(\Omega)\\
\frac{{\partial u}_j}{\partial x_i}&\to \frac{\partial u}{\partial x_i}
\quad\text{in} \quad L^2(\Omega).
\end{align*}
This\pageoriginale proves $u_j\to u$ in $H^1(\Omega)$. 
\end{proof}

\section{Negative Properties of $H^1(\Omega)$}\label{chap1:subsec1.4}

\begin{itemize}
\item [(a)] The functions in $H^1(\Omega)$ need not be continuous
  except in the case $n=1$.

\begin{exam}\label{chap1:exam3} 
Let 
\begin{align*}
\Omega&=\left\{(x, y)\epsilon\mathbb{R}^2:x^2+y^2 < r_\circ^2\right\}, r_\circ < 1.\\
f(r)&= (\log 1/r)^k, k< 1/2 \quad \text{where}\\
r&= (x^2+y^2)^{1/2}.
\end{align*}
Then $f\epsilon H^1(\Omega)$ but $f$ is not continuous at the
origin. 


In the case $n=1$, if $u\epsilon H^1(\Omega), \Omega\subset\mathbb{R}^1$
then $u$ can be shown to be continuous using the formula 
$$
u(y)-u(x)=\int\limits_x^y\frac{du}{dx}(s)\,ds
$$
where $du/ds$ denotes the distributional derivative of $u$.
\end{exam}

\item [(b)] $\mathscr{D}(\Omega)$ is not dense in $H^1(\Omega)$. To
  see this let $u\epsilon (\mathscr{D}(\Omega))^\bot$ in
  $H^1$ and $\phi\epsilon \mathscr{D}(\Omega)$. We have
$$
(u, \phi)_1=(u, \phi)+\sum\limits_{i=1}^n \left(\frac{\partial
 u}{\partial x_i}, \frac{\partial \phi}{\partial x_i}\right)=0 
$$
$$
i.e.\qquad  \left\langle u, \phi \right\rangle +\sum\limits_{i=1}^n 
\left\langle -\frac{\partial^2u}{\partial x_i^2}, \phi \right\rangle=0 $$
Thus 
$$
\langle -\Delta u +u, \phi \rangle=0 \quad \text{for all}\quad \phi \epsilon
\mathscr{D}(\Omega).
$$
Hence
$$
-\Delta u +u=0 \quad \text{in} \quad \mathscr{D}'(\Omega).
$$
\end{itemize}

Let\pageoriginale $\Omega = \{x\epsilon \mathbb{R}^n:|x|< 1\}$,
\begin{align*}
u(x)&= e^{r.x} \quad \text{where} \quad r \epsilon \mathbb{R}^n,\\
\Delta u(x)&= |r|^2 e^{r.x} = |r|^2 u.\\
&= u \quad \text{if} \quad |r|=1.
\end{align*}

Thus when $n=1, u$ with $r=\pm 1$ belongs to
$\mathscr{D}(\Omega))^\bot$, when $n>1$ there are infinitely many
$r's(r\epsilon S^{n-1})$ such that $u\epsilon
(\mathscr{D}(\Omega))^\bot$. Moreover these functions for different
$r$'s are linearly independent. Therefore 
\begin{align*}
\text{dimension}\quad (\mathscr{D}(\Omega))^\bot\geq 2 \quad \text{if}
\quad n=1\\
\text{dimension} \quad (\mathscr{D}(\Omega))^\bot =\infty \quad
\text{if} \quad n>1.
\end{align*}
This proves the claim (b).

We shall define $H_\circ^1(\Omega)$ as the closure of
$\mathscr{D}(\Omega)$ in $H^1(\Omega)$. We have the following
inclusions
$$
\mathscr{D}(\Omega) \underset{dense}{\subset} H_\circ^1(\Omega)\subset
H^1(\Omega)\underset{dense}{\subset} L^2(\Omega).
$$

\section{Trace Theorem}\label{chap1:subsec1.5} 
Let $\Omega$ be a bounded open subset of $\mathbb{R}^n$ with a Lipschitz
continuous boundary $\Gamma :i.e.$ there exists finite
number of local charts $a_j, 1\leq j\leq J$ from $\{y'
\epsilon \mathbb{R}^{n-1}:|y'|<\alpha\}$ into $\mathbb{R}^n$ and a
number $\beta > 0$ such that 
\begin{gather*}
\Gamma = \bigcup\limits_{j=1}^J \left\{(y', y_n):y_n = a_j(y'), |y'|<\alpha \right\},\\
\left\{(y', y_n):a_j(y')<y_n < a_j(y')+\beta, |y'|<\alpha\right\}\subset\Omega,
1 \leq j \leq J,\\
\left\{(y', y_n):a_j(y')-\beta < y_n <a_j(y'), |y'| < \alpha\right\} \subset C
\overline{\Omega},1\leq j\leq J.
\end{gather*}

It\pageoriginale can be proved that $C^\infty (\overline{\Omega})$ is dense
in $H^1(\Omega)$. If $f \epsilon C^\infty (\overline{\Omega})$ we
define the trace of $f$, namely $\gamma f$, by 
$$
\gamma f = f|_\Gamma. \quad \text{Note} \quad \gamma f \epsilon
L^2(\Gamma)\quad \text{if } f \epsilon
C^\infty(\overline{\Omega})
$$

$\gamma : C^\infty (\overline{\Omega})\to L^2(\Gamma)$ is continuous
and linear with norm $\parallel \gamma u\parallel_{L^2(\Gamma)}\leq
C\parallel u \parallel_1$. Hence this can be extended as continuous
linear map from $H^1(\Omega)$ to $L^2(\Gamma)$. 
$$
H_\circ^1(\Omega) \quad\text{is characterised by}
$$
\setcounter{THM}{1}
\begin{THM}\label{chap1:THM1}
$$
H_\circ^1(\Omega) = \{ v \epsilon H^1(\Omega):\gamma V=0\}
$$
\end{THM}

\section{Dual Spaces of $H^1(\Omega)$ and 
$H_\circ^1(\Omega)$}\label{chap1:subsec1.6} 

The mapping 
\begin{align*}
I:H^1(\Omega)\to(L^2(\Omega))^{n+1} \quad \text{defined by}\\
I(v)=\left(v, \frac{\partial v}{\partial x_1},\ldots,\frac{\partial v}
{\partial x_n}\right)
\end{align*}
is easily seen to be an isometric isomorphism of $H^1(\Omega))$ into
subspace of $(L^2(\Omega))^{n+1}$. If $f \epsilon(H^1(\Omega))'$
then $F:I(H^1(\Omega))\to \mathbb{R}$ with $F(Iu)=f(u)$ is a continuous linear
functional on $I(H^1(\Omega))$. Hence by Hahn Banach theorem $F$ can be
extended to $(L^2(\Omega))^{n+1}$. Therefore, there exists $(v,
v_1,\ldots, v_n) \epsilon (L^2(\Omega))^{n+1}$ such that 
$$
f(u)=F(Iu)=(v, u)+\sum\limits_{i=1}^n (v_i, \partial u/\partial x_i).
$$
This representation is not unique since $F$ cannot be extended
uniquely to $(L^2(\Omega))^{n+1}$ For all $\phi \epsilon
\mathscr{D}(\Omega)$ we have 
$$
f(\phi)=\langle v, u \rangle -\sum\limits_{i=1}^n \left\langle\frac{\partial v_i} {\partial
x_i}, \phi \right\rangle ,
$$\pageoriginale
Thus
$$
f|_{\mathscr{D}(\Omega)}=v-\sum\limits_{i=1}^n\frac{\partial v_i}
{\partial x_i}. 
$$

Conversely if $T\epsilon \mathscr{D}'(\Omega)$ is given by 
$$
T=v-\sum\limits_{i=1}^n\frac{\partial v_i}{\partial x_i},
$$
where $v, v_i \epsilon L^2(\Omega), 1\leq i\leq n$ then $T$ can be
extended as a continuous linear functional on $H^1(\Omega)$ by the
prescription
$$
T(u)=(v, u)+\sum\limits_{i=1}^n \left(v_i,\frac{\partial u}{\partial x_i} \right)
\quad \text{for all } u \epsilon H^1(\Omega).
$$

The extension of $T$ to $H^1(\Omega)$ need not be unique. But we will
prove that the extension of $T$ to $H_\circ^1(\Omega)$ is unique. Let
$\tilde{T} \epsilon (H_\circ^1(\Omega))'$ be such that
$\tilde{T}|_{\mathscr{D}(\Omega)}=T$. 

Let $u \epsilon H_\circ^1(\Omega)$. Then there exists $u_m
\epsilon \mathscr{D}(\Omega)$ such that $u_m\to u$ in $H^1$. Now 
\begin{align*}
\tilde{T}(u) &= \tilde{T}(\lim\limits_{\text{in}\;H^1} u_m)= \lim_m
\tilde{T}(u_m)\\
&= \lim\limits_m T(u_m)\\
&= \lim\limits_m \left[(v, u_m) + \sum\limits_{i=1}^n \left(v_i, \frac{{\partial
u}_m}{\partial x_i}\right)\right]\\
&= (v, u)+\sum\limits_{i=1}^n \left(v_i, \frac{\partial u}{\partial x_i}\right) 
\end{align*}
Thus 
$$
\tilde{T}=T  \text{ on }  H_\circ^1(\Omega).
$$

Hence\pageoriginale we identify $(H_\circ^1(\Omega))'$ with a space of
distribution and we denote it by $H^{-1}(\Omega)$. That is 
$$
H^{-1}(\Omega)=\left\{v-\sum\limits_{i=1}^n\frac{\partial v_i}{\partial
x_i}: (v, v_1,\ldots,v_n) \epsilon (L^2(\Omega))^{n+1}\right\} \subset
\mathscr{D}' (\Omega).
$$

\begin{exer}\label{chap1:exer1}
Show that $\partial/\partial x_i:L^2(\Omega)\to H^{-1}(\Omega)$ is continuous.
\end{exer}

\section{Positive Properties of $H_\circ^1(\Omega)$ and
  $H^1(\Omega)$.}\label{chap1:subsec1.7}

\begin{THM}\label{chap1:THM3}
{\bf (Poincare's Inequality)}. Let $\Omega$ be an open bounded subset
of $\mathbb{R}^n$. Then there exists a constant $C(\Omega)$ such that 
$$
\int\limits_\Omega v^2\,dx \leq C(\Omega)\int\limits_\Omega|\nabla v|^2
\,dx \quad \text{for all } v \epsilon H_\circ^1(\Omega).
$$
\end{THM}

\begin{proof}
We shall prove the inequality for the functions in $\mathscr{D}
(\Omega)$ and use the density of $\mathscr{D} (\Omega)$ in
$H_\circ^1(\Omega)$.

Since $\Omega$ is bounded, we have 
$$
\Omega \subset[a_1, b_1] \times\ldots\times[a_n, b_n].
$$

For any $u(x) \epsilon \mathscr{D}(\Omega)$, we have 
$$
u(x)=\int\limits_{a_i}^{x_i}\frac{\partial u}{\partial
x_i}(x_1,\ldots, x_{i-1},t, x_{i+1},\ldots,x_n)\,dt.
$$
Thus 
$$
|u(x)|\leq (b_i-a_i)^{1/2} \left(\int\limits_{a_i}^{b_i}\left|\frac{\partial u}
{\partial x_i}\right|^2\,dt\right)^{1/2}
$$

Squaring both sides and integrating we obtain
$$
\int\limits_{a_1}^{b_1}\cdots \int\limits_{a_n}^{b_n} |u(x)|^2\,dx\leq
(b_i-a_i)^2 \int\limits_{a_1}^{b_1}\cdots \int\limits_{a_n}^{b_n}
\left|\frac{\partial u}{\partial x_i}\right|^2 \,dx.
$$
Thus\pageoriginale
$$
\int\limits_\Omega |u(x)|^2 \,dx \leq C(\Omega)\int\limits_\Omega
|\nabla u|^2 \,dx
$$
where
$$
C(\Omega)=(b_1-a_1)^2+\cdots +(b_n-a_n)^2.
$$

Let $v\epsilon H_\circ^1(\Omega)$. Then there exists $u_n
\epsilon \mathscr{D}(\Omega)$ such that $u_n\to v$ in $H^1$, which
implies $u_n\to v$ in $L^2$ and $\dfrac{\partial u_n}{\partial x_i} \to
\dfrac{\partial v}{\partial x_i}$ in $L^2$. Using this and the
inequality for smooth functions we arrive at the result.
\end{proof}

\begin{REM}\label{chap1:rem1}
The theorem is not true for functions in $H^1(\Omega)$. For example a
nonzero constant function belongs to $H^1(\Omega)$ but does not
satisfy the above inequality.

We state Sobolev imbedding theorem and Rellich's theorem, which have
many important applications. 
\end{REM}

\setcounter{sobolev}{3}
\begin{sobolev}\label{chap1:sobo4}
If $\Omega$ is an open bounded set having a Lipschitz continuous
boundary then we have the imbedding 
$$
H^1(\Omega)\hookrightarrow L^p(\Omega),
$$
$p < q$ or $p \leq q$ according as $n=2$ or $n>2$ where 
$$
1/q=1/2-1/n.
$$
\end{sobolev}

\setcounter{rellich}{4}
\begin{rellich}\label{chap1:rell5}
The above imbedding is compact for\break $p < q$. 
\end{rellich}

