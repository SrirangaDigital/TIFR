
\chapter{The Plancherel formula}\label{PartIII-chap3}

\setcounter{section}{3}
\setcounter{subsection}{0}
\subsection{Unitary
  algebras}\label{partIII-chap3-sec3.1}\pageoriginale%sec 3.1 

Consider the regular representation of a locally compact group $G$ in
the space  $L^2$ and decompose this into a continuous sum of
irreducible representations. Then we have an isometry $f\rightarrow
X_f$ of $L^2$ onto $L^2_\wedge$.  By the definition of the norm in
$L^2_\wedge$, we have $\int_G
|f|^2 d\lambda=\int_{\mathcal{Z}}||X_f||^2 d\mu (\mathcal{Z})$. The
map $f\rightarrow X_f$ is in a sense the Fourier transform for $G$,
and the above equality, the Plancherel formula. As a matter of fact, we
do get the classical Plancherel formula from this as a particular
case when $G$ is commutative. However, we have had many choices to
make in the  decomposition and as such this definition of a Fourier
transform is not sufficiently unique and consequently uninteresting. We now proceed
to obtain a Plancherel formula which is unique.   

Let $G$ be a separable, locally compact, unimodular group. We have seen
(Ch. \ref{partII-chap4}, Part \ref{partII}) 
that the regular representation of $G$ in $L^2$
gives rise to a representation of $L^1$ in $L^2$. In fact we have the
formula for every $f \in L^1$ and   $ g \in L^2 $.  
\begin{align*}
g*f(x) &= \int g(xy^{-1})f(y)dy\\
&= \tau y(a) \text{ where }  \check{g}(x)=f(x^{-1}) 
\end{align*}

If we take $x= e$, $g*f(e) = \int e(y)\check{f}(y) dy$ (Since the
group is unimodular) 
$$
= \langle g,f^* \rangle \text{ where  } f^* (x)= \overline{f(x^{-1})}
$$

By associativity of convolution product, we see that 
\begin{align*} 
\langle g*f, h \rangle &= g*f*h^*(e)\\
&= g*(h*f^*)^*(e)\\
&= \langle g, h*f^* \rangle \text{ for every } f,g,h \in L^1 \cap L^2
\end{align*}\pageoriginale

The  $*$ operation we have defined is an involution. Moreover $L^1$ acts
on $L^2$ or, what is the same, $L^2$ is a representations space for
$L^1$. The mapping $f\rightarrow T_f$  where $T-f(g)=g*f$ is
unitary. Thus $L^1$ is a self adjoint algebra of operator on
$L^2=\mathscr{H}$. $\mathcal{A}=L^1 \cap L^2$  is a subalgebra
$\mathscr{H}$ with an involution $*$. This satisfies the following
axioms:
\begin{itemize} 
\item[(a)] $\langle x,y \rangle   = \langle y^*,x^* \rangle$ for every
$x,y \in \mathcal{A}$

For,
\begin{align*}
\langle f,g \rangle  & = \int f(x)\overline{g(x)} dx\\
& = \int \overline{g(x^{-1})}f(x^{-1})dx\\
& = \int g^*(x) \overline{f^*(x)}dx\\
& = \langle g^*,f^*\rangle \text {~ for every~ }  f,g,\in L^1 \cap L^2.
\end{align*}

\item[(b)] $\langle x, yz \rangle  = \langle y^*x,z\rangle$,  or
  equivalently
 
$\langle yx,z\rangle  = \langle y,zx^* \rangle$  for every $x,y,z
\in \mathcal{A}$. 


\item[(c)] $\mathcal{A}$  is dense  in $\mathscr{H}$.

As a  consequence the operators  $V_x(y) = yx $  on  $\mathcal{A}$
can be extended to operators on $ \mathscr{H}$. 

\item[(d)] The identity operator is the strong limit of that $V_x$.
\end{itemize}

This is an immediate consequence of prop. \ref{partII-chap4-prop3},
Ch. \ref{partII-chap4-sec4.7}, Part \ref{partII}, 

\begin{defi*}%def
 Let $\mathcal{A}$ be subspace of the  Hilbert space $\mathscr{H}$. If
 $\mathcal{A}$ is an associative with involution satisfying conditions
 (a), (b), (c) and (d),  $\mathcal{A}$ is said to be a {\em unitary
   algebra} (Godement). (Ambrose with a slightly different definition
 calls it an $H^*$- algebra). 
\end{defi*}

\subsection{}\label{partIII-chap3-sec3.2}\pageoriginale%3.2

Associated with a given unitary algebra, we have a representation\break
$U_x(y)= xy$ and an antirepresentation $V_x(y)=yx$. Axiom (b)
asserts that these two representations are unitary. The map
$x\longrightarrow x^*$ is an isometry  by a (a) and consequently can
be extended to a map $S:\mathscr{H}\rightarrow\mathscr{H}$. The $U_x$
and the $V_x$ are related by means of the relations $V_x=SU_x * S$ for
every $x\in\mathcal{A}$. In fact, if $y,z\in \mathcal{A}$,  we have  
\begin{align*}
\langle V_x y,z \rangle &= \langle yx,z \rangle \\
&= \langle z^*, x^* y^* \rangle\\
&= \langle z^*, U_{x^*}y^*\rangle\\
&= \langle SU_{x*}Sy,z \rangle
\end{align*}

Hence $V_x=SU_x*S$ on $\mathcal{A}$ and hence on $\mathscr{H}$. We
shall denote by $\mathcal{U}$, $\mathcal{V}$ the uniformly closed
algebras generated by  the $U_x$, $V_x$ respectively. Let
$\mathcal{R}$ be the uniformly closed algebra generated by both the
$U_x$ and the $V_x$. 

\begin{defi*}%def
 An element $a \in\mathscr{H}$  is said to be {\em bounded} if the
 the linear map $x \rightarrow V_x a$ of
 $\mathcal{A}\rightarrow\mathscr{H}$ is continuous. 
 \end{defi*}
 
 The mapping shall be denoted $U_a$, and the set of bounded elements
 $\mathscr{B}$. 
 
 \begin{remark*}%rem
 To start with, one should have defined right -boundedness and
 left-boundedness of elements in $\mathscr{H}$. But if a is bounded in
 the above sense, a trivial computation shows that $U^*_a x = V_xSa$ for
 every $x\in\mathcal{A}$. So $Sa$ is bounded and $U_{Sa}=U_a^*$. Now we
 have $U_x a=SV_{x*}S_a=SU^*_a S_x$ and the map $x\rightarrow U_x a$
 is continuous and hence defines a continuous operator $V_a$ and we
 have $V_a=SU^*_aS$. 
\end{remark*}

\setcounter{proposition}{0}
\begin{proposition}\label{partIII-chap3-prop1}%pro 1
 $\mathscr{M}=\{U_a:a\in \mathscr{B}\}$\pageoriginale 
is a self adjoint ideal which is weakly
  dense is $\mathcal{V}'$. 
\end{proposition}

In fact, for every$ x,y\in \mathcal{A}$, $U_a V_x y=U_a(yx)=V_{yx}a= V_x
V_ya= V_xU_ay$; therefore $U_a$ commutes with $V_x$; hence $U_a\in
\mathcal{V}'$. Moreover if $T\in \mathcal{V}'$, we have $TU_a x =
TV_xa=V_xTa$; hence $Ta$ is bounded and $U_{Ta}=TU_a$ and consequently
$\mathscr{M}$ is an ideal in $\mathcal{V}'$. Since $U_a^*=U_{Sa}$, it is
self-adjoint. Since $\mathcal{V}'$ is weakly closed, it only remains to
show that $\mathcal{V}' \subset \mathscr{M}''$, or again that $T \in
\mathcal{V}'$, $X \in \mathscr{M}'$ implies that $TX=XT$. But we have
seen that, for $x\in 
\mathcal{A}$, $TU_x \in \mathscr{M}$. Hence $TU_x X=XTU_x$ and we may now allow
$U_x$ to tend to 1 in the strong topology to obtain $TX=XT$. 
 
 
From this follows at once the 

\setcounter{thm}{0} 
 \begin{thm}[Godement-Segal]\label{partIII-chap3-thm1}%the 1
 In the notations, $\mathcal{U}'=\mathcal{V}''$, or equivalently
 $\mathcal{V}'=\mathcal{U}''$. 
\end{thm}
  
In fact, since $\mathcal{V} \subset \mathcal{U}'$, $\mathcal{V}\supset
\mathcal{U}''$. We have only to show 
that $\mathcal{V}'\subset \mathcal{U}''$. In other words, we have to
prove that every element of $\mathcal{V}'$ commutes with every element
of $\mathcal{U}'$. Since the $U_a$, $a \in \mathscr{B}$
and similarly $V_{a}$, $a\in\mathscr{B}$ are dense in $\mathcal{V}'$,
$\mathcal{U}'$ respectively, it suffices to establish the
commutativity of $V_{a}$, $V_{b}$, $a,b\in\mathscr{B}$. 
First we assert
that $U_c b=V_b c$ for every $c\in\mathscr{B}$. 

For,
\begin{align*}
\langle U_c b,x\rangle &= \langle b,U_{Sc}x\rangle  = \langle b,V_xSc\rangle\\
&= \langle V^*_xb,Sc\rangle  = \langle c,SV^*_xb \rangle\\
&= \langle c,U_x Sb \rangle =\langle c,V_{Sb}x\rangle = \langle V_b
c,x\rangle 
\end{align*}

Now, $U_aV_bx=U_aU_x b = U_{U_a x} b=V_{b}(U_{a}x)$ by the above
calculation and the proof of theorem \ref{partIII-chap3-thm1} is complete. 
 
\subsection{Factors}\label{partIII-chap3-sec3.3}\pageoriginale%sec 3.3

A weakly closed self-adjoint subalgebra of operators on a Hilbert space
$\mathscr{H}$ is said to be a {\em factor} if its centre reduces to
the scalar operators. 

If in the above discussion we assume $\mathscr{R}$ to be irreducible,
then $\mathscr{R}'= \mathcal{U}\cap \mathcal{V}'= \mathcal{U}'\cap
\mathcal{U}''$ = Centre of $\mathcal{U}''$. Since $\mathscr{R}$ is
irreducible, by cor. to Schur's lemma (Ch. 5.4, Part II),
$\mathscr{R}'$ = scalar operators. Hence $\mathcal{U}''$ is a factor. 

\begin{examples*}
\begin{enumerate}
\renewcommand{\labelenumi}{(\theenumi)}
\item The set of all bounded operators on $\mathscr{H}$ is  a factor.

\item The set of bounded operators on $\mathscr{H}$ which is isomorphic
  to\break $\Hom(\mathscr{H}_1,\mathscr{H}_1)$ where $\mathscr{H}_1$ is
  another Hilbert space is also a factor. This is said to be a {\em
    Factor of type I}. If $\mathscr{H}_1$ is of dimension $n$, this is
  said oe be if type $I_n$. 
\end{enumerate}
\end{examples*}

\subsection{Notion of a trace}\label{partIII-chap3-sec3.4}%sec 3.4

If we consider only the operators on $\mathscr{H}$ which are of finite
rank, then we have the notion of a trace defined by
$\sum_{n} \langle Te_n, e_n \rangle$  where the $e_n$ form
an orthonormal basis. In the general case, we may define trace
axiomatically in the following way: 

\begin{defi*}%def
If $\mathscr{P}$ is the set of positive operators on $\mathscr{H}$,
{\em trace} is a map of $\mathscr{P}$ into $[0,\infty]$ satisfying  
\begin{enumerate}
\renewcommand{\theenumi}{\alph{enumi}}
\renewcommand{\labelenumi}{(\theenumi)}
\item $\Tr(UPU^{-1})= \Tr P$ for every unitary operator  $U$, and 

\item If $P$ is a positive operator = $\sum T_\alpha$, where the
  $T_{\alpha}$ are also positive operators, and the series in strongly
  convergent, then   $\Tr P=\sum \Tr T_\alpha$. 
\end{enumerate}
\end{defi*}

In\pageoriginale particular, (b) implies that for every positive $\lambda$,
$\Tr(\lambda P)=\lambda \Tr P$. It is obvious that this is true if
$\lambda$  is rational and since the rational numbers are dense in $R$,
by (b), it is also true for all  $\lambda\in R^+$. If $A$ is any operator
on $\mathscr{H}$ with a minimal decomposition into positive operators,
then trace can be defined on $A$ by extending by linearity. 

Now, instead of $\Hom (\mathscr{H},\mathscr{H})$, we may consider any 
$*$-subalgebra $F$ of $\Hom (\mathscr{H},\mathscr{H})$ and define the notion
of a trace as above.  However, for arbitrary $*$-subalgebras, neither
the existence of a non-trivial trace nor its uniqueness is assured.
For instance, if $\mathscr{H}=\mathscr{H}_1+\mathscr{H}_2$ is the
direct sum of the Hilbert spaces $\mathscr{H}_1$ and $\mathscr{H}_{2}$
and $F$ is the subalgebra $\Hom
(\mathscr{H}_1,\mathscr{H}_{1})+\Hom(\mathscr{H}_{2},\mathscr{H}_{2})$
  of $\Hom(\mathscr{H},\mathscr{H})$, then the function $\varphi$
  defined by $\varphi(T_{1}+T_{2})=\lambda_{1}\Tr T_{1}+\lambda_{2}\Tr
  T_{2}$ (where $\lambda_1$
and $\lambda_2 $ are arbitrary positive constants) is a trace.
However, when $F$ is a factor, the nontrivial trace, if it exists,
is unique.  Those factors which do not possess a nontrivial trace are
said to be of type III.  

A nontrivial trace on $F$ can be proved to have the following properties:
\begin{enumerate}
\item If $H\in F \cap \mathscr{P}$, $Tr H=0$ of and only if $H=0$; and 

\item For every positive $H \in F$, there exists $H^1\in F$ such that
  $0<H^{1}\leq H$ and $\Tr H^1<\infty$. 
\end{enumerate} 

\begin{defi*}%def
 An element $A$ of $a*$-algebra  $F$ operators on $\mathscr{H}$, is
 said to be {\em normed} (or of {\em Hilbert-Schmidt type}) with
 respect to a trace on $F$, if $\Tr(A^*A)<\infty$. 
\end{defi*}

Let\pageoriginale $F_0$ be the set of operators of finite trace and
$F_1$ the set of 
normed operators in $F$. Than if $A$, $B\in F_1$, we have $B^*A\in F_0$
and $\Tr (B*A)$ is a scalar product on $F_1$. 

In the case of the algebra $\mathcal{U}''$, one can prove  

\begin{thm}\label{partIII-chap3-thm2}%thm
There exists on $\mathcal{U}''$ one and only one trace such that 
\begin{enumerate}
\renewcommand{\theenumi}{\alph{enumi}}
\renewcommand{\labelenumi}{\rm(\theenumi)}
\item  $A\in \mathcal{U}''$ is normed if and only if $A=U_a$ for some
  $a \in  \mathscr{B}$. 

\item If $A= U_a$ and $B=U_b$ are normed, then $\Tr(B*A)= \langle
  a,b \rangle$. 
\end{enumerate}
\end{thm}

The proof may be found in \cite{key23} or \cite{key12}, Ch. I, \S\ 6,
n$^\circ$ 2.  


In particular, if $R$ is irreducible, then the factor $\mathcal{U}''$ is  
\textit{not} of type III. 


\subsection{}\label{partIII-chap3-sec3.5}%sec 3.5


We now assume that two more conditions are satisfied by $\mathcal{A}$,
viz.
\begin{enumerate}
\renewcommand{\labelenumi}{(\theenumi)}
\item $\mathcal{A}$ is separable i.e. there exists a $*$-subalgebra
  everywhere dense in $\mathcal{A}$, which has a countable basis (in
  the algebraic sense). 

\item There exists an element $e\in \mathcal{A}$ such that $e^*=e$ and
  $\mathscr{R}$ is dense in $\mathcal{A}$. 
\end{enumerate}

Condition (2), however, is, as in the case of general decomposition,
not a real restriction, and if it is not satisfied, $\mathcal{A}$ can
be split up into a discrete sum of algebras each of which satisfies
this condition. 

We shall now perform the decomposition of $\mathscr{H}$ into a
continuous sum of Hilbert spaces with reference to the uniformly
closed algebra $\mathscr{R}$ of operators on $\mathscr{H}$. Let
$\mathcal{A}_1$ be the set $\{Re:R \in \mathscr{R}\}$. 

The\pageoriginale fundamental family of vector fields in $L^2_{\wedge}$
is given by 
$a=Re\rightarrow \tilde{a}(\zeta)=X_R(\zeta) \in
\mathscr{H}(\zeta)$. We have already seen (Ch. 2) that this map is an
isometry. Consider for every $\zeta$ the set  
$\mathcal{A}(\zeta)=\{\tilde{a}(\zeta): a \in \mathcal{A}_1\}$. $U_a,a
\in \mathcal{A}$ is a decomposed operator
$=\int_{\mathcal{Z}}U_a(\zeta)d \mu(\zeta)$. 

Our object now will be to put on $\mathcal{A}(\zeta)$ the structure of
an algebra and an involution with respect to which
$\mathcal{A}(\zeta)$ becomes a unitary algebra, To this end, we define
for $\xi =\tilde{a}_1(\zeta)$ and $\eta=\tilde{a}_2(\zeta),a_1,a_2 \in
\mathcal{A}_1$, $\xi\cdot \eta=U_{a_1}(\zeta)\tilde{a}_2(\zeta)$ and 
$\xi^*\tilde{a}_1^*(\zeta)$. Of course we have to prove that these
definitions are independent of the particular $a_1$, $a_2$ we choose. In
other words we have to verify that if  
$\tilde{a}_1(\zeta)=\widetilde{a'_1}(\zeta)$ and
$\tilde{a}_2(\zeta)=\widetilde{a'_2}(\zeta)$, then 
$ U_{a_1}(\zeta)\tilde{a}_{2} (\zeta)  = U_{a_{1}}(\zeta)
\widetilde{a'_2}$ and that $\tilde{a}_{1}^{*}
(\zeta)=\tilde{a}'_{1}{^*}(\zeta)$. In order to prove the former, we
show that
$U_{a_1}(\zeta)\tilde{a}_2(\zeta)=V_{a_2}(\zeta)\tilde{a}_1(\zeta)$. If
$a_1= R_1 e$ and $a_2=R_2e$, we have
\begin{align*}
U_{a_1}(\zeta)\tilde{a}_2(\zeta)  = \widetilde{(U_{a_1}a_2)}(\zeta) &
= X{_U}_{a_1}R_2(\zeta)\\  
=\tilde{b}(\zeta)\text{~ where~ }b & =U_{a_1}R_2e\\
=(R_1e)(R_2e) & =V_{a_2}(R_1e)
\end{align*}

Hence $\tilde{b}
(\zeta)=X_{Va_2R_1}(\zeta)=V_{a_2}(\zeta)\tilde{a}_1(\zeta)$. Therefore,
we have proved that the vector fields
$U_{a_1}(\zeta)\tilde{a}_2(\zeta)$ and
$V_{a_2}(\zeta)\tilde{a}_1(\zeta)$ are equal and so we have
$U_{a_1}(\zeta)\tilde{a}_2(\zeta)=V_{a_2}(\zeta)\tilde{a}_1(\zeta)$
for almost every $\zeta$. But since $\mathcal{A}_1$ has a countable
basis, we can find a set $N$ of measure zero such that for
\textit{every} $a_1,a_2 \in \mathcal{A}_1$, we have
$U_{a_1}(\zeta)\tilde{a}_2(\zeta)=V_{a_2}(\zeta)\tilde{a}_1(\zeta)$
for $\zeta \not\in N$. Now the left hand side is unaltered if we
replace $a_2$ by $a'_2$ while the right hand side remains
the\pageoriginale same if we replace $a_1$ by $a'_1$. 

It only remains to prove that $\xi^*$ is well-defined for almost
all $\zeta'$. It is enough to prove that for every $a_1, a_2 \in
\mathcal{A}_1$, the set of $\zeta$ such that $\widetilde {a_1}
(\zeta)$ = $\widetilde{a_2} (\zeta)$ and $\widetilde {a_1*} (\zeta)$
$\neq$ $\widetilde {a_2*} (\zeta)$ is of measure zero. Let $E =  \{
\zeta : \widetilde {a_1} (\zeta)$ = $\widetilde {a_2} (\zeta) \}
$. The operator field A defined by 
$$ 
A(\zeta) = 
\begin{cases} 
\text{Identity if } \zeta \in E \\ 0
  ~\text{ if~ } \zeta \notin E 
\end{cases} 
$$
is a Hermitian scalar decomposed operator in $\mathscr{H}$ and we
have $Aa_1 = Aa_2$.  
Hence we have $Aa^*_1 = Aa^*_2$, i.e. $A(\zeta)$ $\widetilde {a^*_1}
(\zeta) = A(\zeta) \widetilde a^*_2 (\zeta)$ almost everywhere and
$\widetilde{a^*_1} (\zeta) = \widetilde{a^*_2} (\zeta)$ for almost
every $\zeta \in E$. 

We shall now prove that with the above operations, $A(\zeta)$ is a
unitary algebra. 
\begin{itemize}
\item[(a)] $\langle \xi, \eta \rangle = \langle \eta^*,\xi^*  \rangle$

We have, in our usual notation, $\langle \xi, \eta \rangle =
\varphi_{R^*_2R_1} (\zeta)$. 
$a^*_1 = SR_1 Se =SR_1Se$ (since $e^* = e$) and $a^*_2 = SR_2 Se$. We
have now to prove 
that $\varphi _{R^*_2R_1} (\zeta) = $ $\varphi _{(SR_1S) *(SR_2S)}
(\zeta)$. We assert that $M^* = SMS$ for  every $M \in
\mathscr{R}$. In fact, if 
$x \in\mathcal{A}$, $U_x M = M U_x = U_{Mx}$ and $U_{M*Sx} = M^*U_{Sx} = U_{(Mx)^*}
= U_{SMx}$. Since the map $x \rightarrow U_x$ is one-one, we have
$M^*SX = SMx$ or $M^* = SMS$. Therefore 
\begin{align*}
\langle  M (SR_1) * (SR_2 S)e, e \rangle &=  \langle MSR_2Se,  SR_1e
  \rangle\\
&=  \langle R_1e, R_2 SMSe \rangle\\ 
& = \langle R_1e, R_2 M^*e  \rangle. 
\end{align*}

In other words, 
$$
\langle M(SR_1 S) * (SR_2)e,e \rangle = \langle M R^*_2 R_1e,e \rangle
$$
for every $M \in A$. By the definition of the spectral measure,
$$
d \mu _{(SR_1S)}*(SR_2S)e,e = d \mu _{R^*_2 R_1 e,e}  
$$\pageoriginale 
and hence $\varphi (SR_1S)*(SR_2S)  = \varphi _{R_2^* R_1}$. 

\item[(b)] $\langle \xi_1 \xi _2, \xi _3 \rangle = \langle \xi_1,
  \xi_3, \xi^*_2 \rangle$.   

We have to show that $\varphi_{R^*_3 R_1 R_2} = \varphi_{(R_3
  R^*_2)^*R_1}$  which is obvious.   

\item[(c)] $A(\zeta)$ is dense in $\mathscr{H}(\zeta)$. 

This is again evident.
 
\item[(d)] Regarding the existence of sufficiently many operators, we cannot
 assert that is true for all $\zeta$. However, this is true for almost
 all $\zeta$ For this we need a  
\end{itemize}

\setcounter{lem}{0}
\begin{lem}\label{partIII-chap3-lem1}%Lem 1
 In a self adjoint algebra $\mathcal{A}$ of operators on $\mathscr{H}$,
 the set $\{ Ax: A \in \mathcal{A}, x \in \mathscr{H}\}$  is dense in
 $\mathscr{H}$ if and only if the Identity is the strong limit of $A
 \in \mathcal{A}$. 
 \end{lem}

In fact, if $x$ is an element of $\mathscr{H}$, we denote
$\overline{\{Ax\}}$ by $F$. Let $F^\perp$ be its orthogonal complement. If
$x$ is not in $F$, let $x = x_1 + x_2$ with $x_1 \in F$, $x_2 \in
F^\perp$. Then $Ax = Ax_1 + Ax_2$ with $Ax_1 \in F$, $Ax_2 \in F^\perp$ 
(since $\mathcal{A}$ is a self-adjoint algebra). But $Ax \in F$. Hence
$Ax_2 \in f^{\perp}$ as well as $F$. Since the sum is direct, $Ax_2=0 $
for every $A \in \mathcal{A}$. If $x_2 \neq 0$, this contradicts the
assumption that $\{Ax: A \in \mathcal{A}, x\in\mathscr{H}\}$ is dense
in $\mathscr{H}$. For, space $E = \{ x : A x = 0 $ for every
$A \in \mathcal{A} \} $is non-zero. $E$ is invariant under $\mathcal
{A}$ and therefore $E^\perp$  is invariant under
$\mathcal{A}$. Consequently $\{ \mathcal {A} a : a \in \mathcal {H}\}$
is  contained in $E^ \perp$.  Now, $\{U _{x}y : x, y \in \mathcal {A}
\}$ is dense in $\mathcal {H}$ and hence  $\{ x y : x, y \in
\mathcal{A_1} \}$   is dense in $\mathcal{A}_1$. On the other hand,
there  exists a sequence ${Y_n}$ is $\mathcal{A}_1$ such that the
  $\tilde{Y_n} (\zeta)$  are dense in $\mathcal {H} (\zeta)$ for every
  $\zeta$. Each ${Y_n}$ can be  approximated\pageoriginale by a sequence
  $X_{n,p}Y_{n,p}$ with $X_{n,p}, Y_{n,p} \in A_{1}$.  Hence we have
  $\tilde{Y_n} (\zeta) = \underset{p \rightarrow \infty}{\lim}
  U_{X_{n,p}} (\zeta) \tilde{Y_{n,p}} (\zeta)$ for almost every $\zeta$
  and for each $n$, since we have only a countable family
  $Y_n$. Therefore $ \{ \xi \eta : \xi, \eta \in A(\zeta) \}$ is dense
  in $\mathcal{A}$ $(\zeta)$ for almost every $\zeta$. 

Thus we have shown that the algebra $\mathcal{A} (\zeta)$ is unitary
for  almost every $\zeta$. By Mautner's theorem, almost all these
algebras are irreducible. Thus $U(\zeta)''$ is a factor. We can apply
Theorem \ref{partIII-chap3-thm2} 
to this factor. Thus the scalar product in the space of
bounded elements is given by a trace. More precisely, we have a trace
function on $U (\zeta)''$ such that $\langle \tilde{a_1} (\zeta),\break
\tilde{a_2}(\zeta) \rangle = Tr(U_{a_2}* (\zeta) U_{a_{1}})
(\zeta))$. But we know  that the correspondence $a \rightarrow
\tilde{a}$ is an isometry and hence one gets  
$$ 
\langle a_{1}, a_{2} \rangle = \int_{\mathcal{Z}} Tr\bigg[ u_{a_{2}}^* (\zeta)
  U_{a_{1}} (\zeta) \bigg] d \mu (\zeta) 
$$

This is the Plancherel formula which is in a certain sense unique.
However this is obviously not absolutely unique as the trace function
is unique only upto a constant multiple. 

In the case of a locally compact, separable, unimodular group $G$, we
take $\mathcal{A}= L^1 \cap L^{2}$ and $\mathscr{H} = L^{2}$. In this
case Plancherel formula can be rewritten as  
$$
 \int_{G} f(x) \overline{g (x)dx} = \int_{\mathcal{Z}} \Tr (U_{g}^*
 (\zeta) U_{f}(\zeta)) d \mu (\zeta). 
 $$

Or again $g^* * f(e) = \int_{\mathcal{Z}} \Tr (U_{g* * f} (\zeta) ) d
\mu (\zeta)$. 

If we write $g^{*}*f = h$, we get 
$$
h (e) = \int_{z} \Tr (U_{h} (\zeta)  d \mu (\zeta) 
$$

This is the generalisation of the \textit{Fourier inversion
  formula}. At any point $x$, the value of $h(x)$ is given by  
$$
h(x), = \int_{\mathcal{Z}} \Tr (U_{h} (\zeta) U_{x}(\zeta)) d \mu
 (\zeta) 
$$ 

This\pageoriginale is of course true not for all functions, but only
for function of 
the type $g^**f$ with $g,f \in L^1 \cap L^{2}$ as in the classical case.  


\subsection{A particular case}\label{partIII-chap3-sec3.6} % sec 3.6

We have obtained a Plancherel formula in terms of the factorial
representations of the group $G$ and it would be more desirable to have
a formula in terms of the irreducible representations of the
group. This is however possible only in the following particular
case. 

\begin{defi*}
A locally compact group is said to be  {\em type I} if every factorial
representation of the group is of type I. 
\end{defi*}
 
 This definition implies that every factorial representation is a
 discrete multiple of an irreducible representation. In fact, if $F$
 is the factor corresponding to a factorial representation of $G$,
 then $F$ is isomorphic (algebraically) to
 $\Hom{(\mathscr{H}_2,\mathscr{H}_2)}$ where $\mathscr{H}_2$ is a
 Hilbert  space.  If $F$ is of type I, it can be proved (see, for instance
 \cite{key1}, \S\ 8, Ch. I) that  $F'$ is also of type I. by the
 isomorphism between $\Hom {(\mathscr{H}_1,\mathscr{H}_1)}$ and $F'$,
 we therefore get that for every projection $P$ in $F'$, there exists
 a minimal projection $<P$. in other words, every invariant subspace
 $\neq 0$ of ${\mathscr{H}}$ contains a \textit{minimal} invariant
 subspace. The restriction of the operators of $F$ to any such minimal
 invariant subspace gives rise to an irreducible unitary
 representation and since the minimal projections in a Hilbert space are
 conjugate by unitary isomorphisms, these irreducible unitary
 representations are \textit{equivalent}. On the other hand, the
 family of invariant subspaces of $\mathscr{H}$ which are direct sums
 of minimal invariant subspace, partially ordered by inclusion, is
 obviously inductive. By Zorn's lemma, there exists in it a maximal
 element\pageoriginale say $\mathscr{H}^1$. If $\mathscr{H}^1\neq
 \mathscr{H}$, $\mathscr{H}^{1\perp}$ is nonempty and consequently
 contains a minimal invariant subspace $\mathscr{H}^1_1$. Then
 $\mathscr{H}^1\oplus\mathscr{H}^1_1$ again belongs  to the family,
 thereby contradicting the maximality of $\mathscr{H}^1$. 
 
Thus if $x \in \mathcal{A}$, the operator of the factorial representation is
decomposed into irreducible $U^0_x$ which are all equivalent. The map
$U_x \rightarrow U^0_x$ is an isomorphism. Hence in a group of type
1, we have the formula 
$$  
\int_G f(x)\overline{g(x)}dx=\int_ {\mathcal{Z}}
Tr(U_g^*(\zeta)U_f(\zeta))d\mu(\zeta)  
$$
where the $U (\zeta)$ are irreducible representations and \textit{not}
merely factorial representations, and the trace is the \textit{usual}
trace. 
 
The definition of a group of type I seems a little inoccuous but is
of importance since all semisimple lie groups are of type I. The
problem remains however to give an explicit Plancherel measure, etc. 

It is known in the case of complex semisimple lie groups (see \cite{key26})
and in the case of $SL(2,R)$ (\cite{key2}, \cite{key25}), but not in
the general case.  

\subsection{Plancherel formula for commutative
  groups}\label{partIII-chap3-sec3.7}% sec 3.7 
  
Let $\Omega$ be the spectrum of $\mathscr{R}'$ in this case. We have
however to pass to a quotient $\mathcal{Z}$ by means of an equivalence
relation. It can be proved that $\mathcal{Z}$ is actually the one
point compactification of the spectrum of $L^1$. We now assert that
every representation of $L^1$ in a space $E$ arises from a  
 representation of the group $G$. In fact if $a=U_f b$, with $b \in
 \mathscr{H}$, $f \in L^1$, we put $U_x a=U_{\epsilon_x * f} b$.
It can be proved (see for instance \cite{key22}, \cite{key6}) that the
$U_x$ are well defined.   

This\pageoriginale establishes a one-one correspondence between
characters of $G$ 
and one dimensional representation of $L^1$. Hence the spectrum of
$L^1$ is only the character group of $GU \{ 0 \}$ to compactify it. In
this case, every factor consists only of scalar operators and hence any
factorial representation is a discrete multiple of irreducible
representations of dimension 1. If the character group of $G$ is denoted by
$\hat{G}$, we have, since $\Tr \chi (f) =\chi (f) =\int f(x) \chi (x)
dx$, 
$$
\int_G || f ||^2 dx =\int_{\hat{G}}|\chi(f) |^2 d\mu(\chi).  
$$

It only remains to prove that the Plancherel measure in this case is
the Haar measure on $\hat{G}$. But this is obvious since each
character is of norm 1 and multiplication of $\chi(f)$ by another
character leaves the integral invariant. Thus in this case, we have
the  classical plancherel formula 
$$
\int_G || f ||^2 dx =\int_{\hat{G}} \mid \chi (f) \mid^2 d\chi.
$$
 
Again the Fourier inversion formula becomes in this case
$$
f(x)=\int_{\hat{G}} \chi (f) \overline{\chi(X)} d \chi.
$$


\backmatter

\chapter{Bibliography}

For\pageoriginale Part \ref{partI}, see particularly \cite{key3},
\cite{key8}, \cite{key9}, 
\cite{key10}, \cite{key43}, \cite{key49};
for Part \ref{partII}, \cite{key4}, \cite{key5}, \cite{key6}, \cite{key7},
\cite{key19}, \cite{key22}, \cite{key29}, \cite{key31},
\cite{key44}, \cite{key49}; and Part \ref{partIII}, \cite{key12}, \cite{key20},
\cite{key23}, \cite{key33}, \cite{key34},
\cite{key36}, \cite{key40}, \cite{key46}, \cite{key47},


\begin{thebibliography}{99}
\bibitem{key1}  W. Ambrose  -  The $L^2$-system of a unimodular group I.
   Trans. Am. Math. Soc. 65, 1949, p.27-48.

\bibitem{key2}  V. Bargmann  - Irrejucible unitary representations of the
   Lorentz group.   Annals of Math, 48, 1947, p. 568-640.

\bibitem{key3}  N. Bourbaki  -  Topologie Generale, Ch. III-IV, Paris 1951.

\bibitem{key4} N. Bourbaki  -  Espaces Vectoriels Topologiques, Ch.I-V
   Paris 1953 and 1955.

\bibitem{key5} N. Bourbaki  -  Integration, Ch.$I-V$, paris, $1952$ and $1956$.

\bibitem{key6} F. Bruhat  -   Sur Ies representations induites des groupes de
    Lie
    Bull. Soc. Math. France, 84, 1956, p. 97-205.

\bibitem{key7} H. Cartan and R. Godement  -  Theorie de la dualite et
  Analyse Harmonique 
    dans les groupes abeliens localement compacts,
    Annales Scientifiques, E.N.S., 64, 1947, p. 79-99.

\bibitem{key8} C. Chevalley\pageoriginale  -  Theory of Lie groups, 
  Princeton 1946. 

\bibitem{key9} J. Dieudonn\'e  -   Groupes de Lie et hyperalg\`ebres de Lie sur 
                          un corps de caract\'eristiqur $p > 0,$   
                          Comm. Math. Helv., 28, 1954, p. 87-118.

\bibitem{key10} J. Dieudonn\'e  -  Lie groups and Lie hyperalgebras over a 
     field of characteristic $p > 0-II$.
     Am. Journ. Math., 77, 1955, p.218-244.

\bibitem{key11} J. Dixmier  -  Algebres quasi-unitaires.
     Comm. Math. Helv., 26, 1952, p. 275-322.

\bibitem{key12} J. Dixmier  -   Les algebres d'operateurs dans l'espace 
      Hilbertien, Paris 1957.

\bibitem{key13} I. Gelfand  -   Normierte Ringe
      Rec. Math[Math.Sbornik] N.S. 9 [51], 3-24.

\bibitem{key14} I. Gilfand - M. Naimark    -  Unitary Representations of the
  group of the linear transformations of the straight line.
        Doklady Akad. Nauk S.S.S.R 63, 1948, p. 609-612.

\bibitem{key15} I. Gilfand - M. Naimark  -  Unit are Darstellungen der Klassischen 
        gruppen, Berlin, 1957.

\bibitem{key16} I. Gilfand - M. Naimark  -   The analogue of
  Plancherel's Formula for  Complex unimodular group.    
         Doklady Akad. Nauk S.S.S.R 63, 1948, p. 609-612.

\bibitem{key17} I. Gelfand - D. Raikoy -  Irreducible unitary representation
  of locally 
         bicompact group.

\bibitem{key18}  A.M. Gleason\pageoriginale  -  Groups without small subgroups.
   Annals of Math., 56, 1952, p. 193-212.

\bibitem{key19}  R. Godement  -  Les fonctions de type positif et le theorie
    des groupes.
   Trans. Am. Math. Soc., 63, 1948, p. 1-84.

\bibitem{key20} R. Godement -  Sur la theorie des representations unitaires
   Annals of Math., 53 (1952), p. 68-124.

\bibitem{key21} R. Godement -  Memoire sur la theorie des caracteres.
   Journal de Math.Pures et Appl.
  30, 1951, p.1-110.

\bibitem{key22} R. Godement -  A theory of spherical functions-$I$.
   Trans. Am. Math. Soc., 73, 1952, p. 496-556.

\bibitem{key23} R. Godement -  Theorie des caracters I Algebres Unitaires.
  Annals of Math., 59, 1954, p. 54-62.

\bibitem{key24} R. Godement -  Theorie des caracteres II. Definition et
  proprietes generales des caracteres.
  Annals of Math., 59, 1954, p. 63-85.

\bibitem{key25} Harishchandra -  Plancherel formula for the $2\times
  2$ real uni -  modular group.
    Proc.Nat.Acad. Sc. 38, p.337-342.

\bibitem{key26}  Harishchandra -   The Plancherel formula for complex semi  - 
   simple lie groups.
   Trans.Am.Math.Soc., 76, 1954, p. 485-528.

\bibitem{key27} I. Kaplansky  -  A theorem on rings of operators.
    Pacific Journ. of Math., 1, 1951, p. 227-232.

\bibitem{key28} G.W. Mackey\pageoriginale  -  Induced representations
  of groups.  Amer. Journ. of Math., 73, p. 576-592.

\bibitem{key29} G.W. Mackey  -  Induced representations of locally compact
    groups I.
   Annals of Math., 55, 1952, p. 101-140.

\bibitem{key30} G.W. Mackey  -  Induced representations of locally compact
   groups II. Annals of Math., 58, 1953, p. 193-221

\bibitem{key31} G.W. Mackey -  Functions on locally compact groups.
   Bull. Amer. Math. Soc., 56, 1950, p. 385-412.

\bibitem{key32}  F.I. Mautner  -  The completeness of the irreducible unitary
   representations of a locally compact group.
   Proc. Nat. Acad. Sc., 34, 1948, p. 52-54.

\bibitem{key33}  F.I. Mautner   -  Unitary representations of locally compact
   groups I.
   Annals of Math., 51, 1950, p.1-25.

\bibitem{key34} F.I. Mautner -  Unitary representations of locally compact
   groups II.
   Annals of Math., 52, 1950, p. 528-556.

\bibitem{key35} F.I. Mautner   -  On the decomposition of unitary representa-
   tions of Lie groups.
  Proc. Amer. Math. Soc., 2, 1951, p. 490-496.

\bibitem{key36} F.I. Mautner -  Note on the Fourier inversion formula for groups.
   Trans. Am. Math. Soc., 78, 1955, p. 371-384.

\bibitem{key37}  M.A. Naimark\pageoriginale   -  Normed rings - Moscow 1956.

\bibitem{key38}  M.A Naimark -  S.V.Fomin  -  Continuous direct sum of
  Hilbert spaces  and some of their applications. Uspehi Mat. Nauk.,
  10. 1955, p.111-142 and  Transl. Amer. Math. Soc., 2, Vol. 5,
   1957, p.35-65.

\bibitem{key39} {J. von Neumann} -  On certain topology for rings of operators.
   Annals of Math., 37, 1936, p. 111-115.

\bibitem{key40}  J. von Neumann  -  On rings of operators- Reduction Theory.
   Annals of Math., 50, 1949, p. 401-485.

\bibitem{key41} {R. Pallu de la Barriere}  -  Algebres unitaires et
  espaces D'Ambrose. 
   Annals Scientifiques, E.N.S., 70, 1953, p. 381-401.

\bibitem{key42}  {L. Pontrjagin} - Topolgical groups, Princeton, 1946.

\bibitem{key43}  L. Pontrjagin  - Continuous group, Moscow, 1954.

\bibitem{key44}  {L. Schwartz}  -  Theorie des Distributions, Paris 1950
  / 1951.

\bibitem{key45}  {I.E. Segal} - The group algebras of a locally compact group.
   Trans. Amer. Math. Soc., 61, 1947, p. 69-105.

\bibitem{key46}  I.E. Segal  - An extension of Plancherel's formula to
   separable unimodular groups.
   Annals of Math., 52, 1950, p. 272-292.

\bibitem{key47}  I.E. Segal  -  Decomposition of operator algebras.
   Memoirs of Amer. Math. Soc., 9, 1951.

\bibitem{key48} {I.E. Segal}-\pageoriginale The two - sided regular
  representations of 
    a unimodular locally compact group.
    Annals of Math., 51, 1950, p. 293-298.

\bibitem{key49}  {A. Weil} - $L'$ Integration dans les groupes
    topologiques et ses applications.
    Paris, 1940 and 1953.
\end{thebibliography}
