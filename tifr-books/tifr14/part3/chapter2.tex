
\chapter{Continuous sum of Hilbert spaces - II}\label{PartIII-chap2}

\setcounter{section}{2}
\setcounter{subsection}{0}
\subsection{}\label{partIII-chap2-sec2.1}\pageoriginale

In the last chapter, given a family $\mathscr{H} (\zeta)$ of Hilbert
spaces 
indexed by elements $\zeta$ of a locally compact space $\mathcal{Z}$,
we constructed the continuous sum $\mathscr{H} = L^2_\wedge $. Now we
shall decompose a Hilbert space $\mathscr{H}$ into a continuous sum
with reference to a given comutative, weakly closed 
*-subalgebra $m$ of $\Hom \mathscr{(H,H)}$.

$\mathscr{M}$ satisfies Gelfand's conditions and is hence isomorphic and
isometric to the space $\mathscr{C} (\Omega)$ of continuous complex
valued functions on a compact space $\Omega$ which is called the
\textit{spectrum} of $\mathscr{M}$. By this isomorphism, every continuous linear
form on $\mathscr{M}$ is transformed into a continuous liner form on
$\mathscr{C}(\Omega)$ or, what is the same, a measure on the  
space $\Omega$. In particular, the continuous linear form $ \langle
Mx,y \rangle $ where $x,y \in \mathscr{H}$ gives rise to a measure
which we shall denote by $d \mu _{x,y}$ i.e. we have $\langle Mx,y
\rangle  = \int_\Omega \hat{M} (\chi) d \mu_{x,y}(\chi)$. This measure
is called the 
\textit{spectral measure} associated to $x$, $y$. This depends linearly
on $x$ and anti linearly on $y$.  

Let $\mathscr{M}'$ be the commutator of $\mathscr{M}$. We now assume
that there  exists an element  $a\in \mathscr{H}$ such that the set
$\{ Aa : A \in \mathscr{M}'\}$ is dense in $\mathscr{H}$. 
This assumption however is \textit{not} a real
restriction on our theory. For otherwise, we can decompose
$\mathscr{H}$ into a discrete sum of  Hilbert spaces each of which
satisfies the above 
condition. Let $a_1$ be any element of $\mathscr{H}$ and
$\mathscr{H}_1$ the closed subspace generated by $Aa_1$, $A \in
\mathscr{M}'$. $\mathscr{H}_1$ is invariant\pageoriginale under both
$\mathscr{M}$ and $\mathscr{M}'$. If 
$\mathscr{H}^{\perp}_{1}$ is the orthogonal complement of
$\mathscr{H}_1$, we can carry out the same process for
$\mathscr{H}^{\perp}_{1}$, and so on. Thus  
$\mathscr{H} = \mathscr{H}_1 \oplus \mathscr{H}_2 \oplus$.. where all
the $\mathscr{H}_i$ satisfy the above condition. 

Now we show that the spectral measure $d \mu_{a,a}$ has $\Omega$ as
its 
support. In fact, if $f$ is a positive continuous function on $\Omega$
such 
that its integral is $0$, then $f=0$. We can write $f$ as $|\hat{M}|^2$
and we  
have $\int|\hat{M}|^2 d \mu_{(a,a)} = \langle M*Ma,a \rangle =
||Ma||^2 $. Hence if $\int | \hat{M}|^2 d \mu_{(a,a)} = 0,         
     Ma = 0$ or $AMa = 0$ for every $A \in \mathscr{M}'$, or again
     $M(Aa)=0$. Since  
the As are dense in $\mathscr{H}$, it follows that $M=0$. Moreover,
this shows that the measure $d \mu_{a,a}$ is positive. 

\setcounter{proposition}{0}
\begin{proposition}\label{partIII-chap2-prop1}
Corresponding to any operator $A \in \mathscr{M}'$ there exits one and
only one continuous function $\varphi_{A}$ on $\Omega$ such that $d \mu_{(Aa,a)}
=\varphi_{A} d\mu_{a,a}$ and we have $|| \varphi _A|| \le || A || $. 
\end{proposition} 
 
In fact, it is enough to prove the proposition when $A$ is a 
positive Hermitian operator since any operator is a finite linear
combination of them. Under this assumption we have
\begin{align*}
 \int | \hat{M}|^2 d \mu_{Aa,a}   & = \langle  M^*MAa,a \rangle \\ 
 & = \langle AMa,Ma \rangle \\
 &  = \le || A || \langle Ma, Ma \rangle \\
 & = || A || \int | \widehat{M}|^2 d \mu_a, a
\end{align*}

Now  $d\mu_{Aa,a}$ is positive and $d \mu_{Aa,a} \le k$ $d
\mu_{a,a}$. Therefore, by Lebesgue - Nikodym theorem, there exist a
measurable function $\psi \in L^{\infty} (\mu_{a,a})$ such that $d
\mu_{Aa,a} = \psi_A d \mu_{a,a}$. We also have $|| \psi_{A}||_{\infty}
\le || A ||$. 

The proof is complete if we prove the 

\begin{lemma*}
For\pageoriginale every bounded measurable function $\psi$  on  $\Omega$ there
exists one and only one continuous function $\varphi$ such that $
\varphi =\psi $ a.e. 
\end{lemma*}

By the definition of $\mu _{x,y},$ we have $|| \mu _{x,y} || \le || x
||~ || y ||$.  
Therefore $\int \psi{(\chi)} d \mu_{x,y}(\chi) \le || \psi ||_\infty
|| x ||~ || y ||$. 


Thus $\int \psi{(\chi)} d \mu_{x,y}(\chi)$  is a sesquilinear map
which is continuous in each of the variables $x,y$. As a consequence
of Riesz representation theorem, there exits a linear operator $T$
such that $\int \psi{(\chi)} d \mu_{x,y}(\chi)  = \langle Tx,y
\rangle$. We now show that $T$ commutes with every element of
$\mathscr{M}'$. For, if $M \in \mathscr{M}'$, we have 
\begin{align*}
\langle M T x,y \rangle &= \langle Tx, M^*y \rangle \\  
&=
\int \psi{(\chi)} d \mu_{X,M^*y}(\chi)\\ 
&=
\int \psi{(\chi)} d \mu_{Mx ,y}(\chi) (\text { since } M \in
\mathscr{M}') \\ 
&=
\langle TMx,y\rangle 
\end{align*}

Since $\mathscr{M}$ is weakly closed, $T \in \mathscr{M}$.

We have $\int \psi (\chi) \hat{M}d \mu_{a,a} = \int \hat{T}\hat{M}d
\mu _{a,a}$  and hence  $\hat{T} = \psi (\chi) a.e$ - This proves the
lemma. 

The function $\varphi_A \in \mathscr{C}(\Omega)$ thus constructed
satisfies the following properties:
\begin{description}
\item[(a)] $\varphi_{\lambda A+ \mu B}  = \lambda \varphi_A +
  \mu \varphi_B$.
 
 For, 
\begin{align*}
\int \varphi_{\lambda {A} + \mu B} d \mu_{a,a} &=
  \langle(\lambda A + \mu B) a,a \rangle = \lambda \langle Aa, a
  \rangle + \mu \langle Ba, a \rangle\\
&= \int (\lambda \varphi_A + \mu \varphi_B) d \mu_{a,a}
\end{align*}

\item[(b)] $\varphi_{A^*} = \bar{\varphi}_A$.

 For, 
\begin{align*}
\int \varphi_{A^*} d \mu_{a,a} &= \langle A^*a,a \rangle =
 \overline {\langle Aa,a \rangle}\\ 
&= \int \bar{\varphi}_A d \mu_{a,a}.
\end{align*}

\item[(c)] $\varphi_{MA} = \hat{M} \varphi_A$\pageoriginale
\begin{align*}
 \int \varphi_{MA} d \mu_{a,a} &= \langle MAa,a \rangle\\
& = \int \hat{M}d \mu_{Aa,a} = \int \hat{M} \varphi_A d \mu_{a,a} 
\end{align*}

\item[(d)] $\varphi_{A^*A} \ge 0$ 

In fact, if f is a positive function, we can express it as 
$ | \hat{M} |^2$. 
 $$
 \int \hat{M}^2 \varphi_{A^*A} d \mu_{a,a} = || AMa ||^2  \ge 0. 
 $$

\item[(e)]  $| \varphi_A | \le || A || $
\end{description} 

This has been proved in the Prop. \ref{partIII-chap2-prop1},
Ch. \ref{partIII-chap2-sec2.1}. 

\subsection{}\label{partIII-chap2-sec2.2}

In order to get a decomposition of $\mathscr{H}$ into a continuous
sum, we need the following 

\begin{lemma*}
Let $\mathcal{B}$ be any *-subalgebra of $\Hom (\mathscr{H},
\mathscr{H})$ which 
is uniformly closed. Let $\varphi$ be a positive continuous linear
form on $\mathcal{B}$ such that $\varphi(A^*)= \overline{\varphi(A)},
\varphi(A^*A) \ge 0$  and  $| \varphi(A) | \le k || A ||$.  To
$\varphi$ we can make correspond a canonical unitary representation of
the algebra $\mathcal{B}$. 
\end{lemma*} 

In fact, because of the conditions we have imposed on $\varphi$,
$\varphi(B^*A)$ is a positive Hermitian form on $\mathscr{B}$. Hence we 
have by the Cauchy-Schwarz inequality $| \varphi(B^*A) ||^2 \le
\varphi(B^*B) \varphi(A^*A)$. Therefore, $\varphi(B^*B)=0$ if and only 
if $\varphi(B^*A)=0 $ for every $A \in \Hom (\mathscr{H}, \mathscr{H})$. Hence
$\varphi(B^* B)=0 \Rightarrow \varphi ((AB)^* AB)=0$. It follows that
the set $N$ of elements $B$ such that $\varphi (B^\ast B)=0$ is a left
ideal. On the space $B/N$, $\varphi$ is transformed into a positive
definite Hermitian from and consequently $\varphi$ gives rise to a
scalar product. The completion of this space under this norm shall be
denoted $\mathscr{H}_\varphi$. The canonical map
$\mathscr{B}\rightarrow \mathscr{H}_\varphi$\pageoriginale is
continuous since $\varphi(B^\ast 
B)\leq k|| B ||^2$. On the other hand we have also a map $f :
\mathscr{B} \rightarrow   \Hom
(\mathscr{H}_\varphi,\mathscr{H}_\varphi)$ defined by $f(A)= U_A$
where  $U_A(\dot{B})=\dot{\widehat{AB}}$. We show that this as also
continuous. Consider $B^\ast || A ||^2 B-B^\ast A^\ast AB=B^\ast(|| A
||^2 - A^\ast A) B; (|| A ||^2 - A^\ast A)$  is positive Hermitian and
hence so is $H =B^\ast || A ||^2 B-B^\ast A^\ast AB$. $H^{\dfrac{1}{2}}$
is the uniform limit of polynomials in $H$ and since $\mathscr{B}$ is
uniformly closed, $H^{\dfrac{1}{2}}\in \mathscr{B}$. 
Therefore we have $\varphi_H =
\varphi_{H^{\frac{1}{2}} H^{\frac{1}{2}}} \geq 0$ by assumption. We
have now proved that $\varphi(B^\ast A^\ast AB)\leq || A ||^2 \varphi
(B^\ast B)$. This shows that the map $U_A$ is continuous. This can be
extended to an operator of $\mathscr{H}$. We have also shown that $||
U_A || \leq || A ||$. It remains to prove that this is a unitary
representation. Let $A$, $B$, $C \in \mathscr{B}$. Then $\langle U_A
\dot{B},\dot{C}\rangle= \varphi((A^\ast C)^\ast B)=\langle \dot{B},U_A
\ast C \rangle$. Hence $U^\ast_A = U_A$ which shows that this is
unitary. 

In fact all the above considerations hold for a Banach algebra with
involution. 

\subsection{}\label{partIII-chap2-sec2.3}

After this lemma in the general set-up, we revert to our decomposition
of $\mathscr{H}$ into a continuous sum. For every fixed $\chi \in \Omega$,
$\varphi_A (\chi)$ is a positive continuous linear from on
$\mathscr{M}'$ which 
satisfies all the conditions of the lemma. Hence we have a unitary
representation of $\mathscr{M}'$ in the Hilbert space $\mathscr{H}_\chi
=\dfrac{\mathscr{M}'}{N_\chi}$ where $N_\chi =\{A : \varphi_{A^\ast
  A}(\chi)=0\}$. In other words, to each point $\chi \in \Omega$ we have
assigned a Hilbert space $\mathscr{H}_\chi$. If $M \in \mathscr{M}$, since
$\varphi_{MA}=\hat{M}\varphi_A$, we have $U_M (\chi)=\hat{M}(\chi)$ 
identity. For,
$\langle U_M (\chi)\dot{B},\dot{C}\rangle = \varphi_{C^\ast
  MB}(\chi)=\hat{M}(\chi) \varphi_{C^\ast B}(\chi)=\hat{M}(\chi) \langle
\dot{B},\dot{C}\rangle$. We have now all the data necessary for the
construction of a continuous sum except\pageoriginale the fundamental
family of 
vector fields. We have so far operated with $\mathscr{M}'$, but, in practice, $
\mathscr{M}'$ is very large. For instance it is not in general separable in the
norm. So we assume given a subalgebra $\mathcal{A}$ of $\mathscr{M}'$ such that 
\begin{enumerate}
\renewcommand{\theenumi}{\alph{enumi}}
\renewcommand{\labelenumi}{(\theenumi)}
\item $\mathcal{A}$ is uniformly closed.

\item There exists a sequence $A_n \in \mathcal{A}$ such that
  $\mathcal{A}$ is generated by the $A_n$ and $\mathcal{A}\cap \mathscr{M}$. 

\item There exists  $a \in \mathscr{H}$ such that $\{ Aa:a \in
  \mathscr{H} \}$ is dense in $\mathscr{H}$. It is actually this
  algebra $\mathcal{A}$ which is in general given and the problem will
  then be to find an $\mathscr{M}\subset \mathcal{A}'$ such that the
  above conditions are satisfied. 
\end{enumerate}

We have a map $\mathcal{A}\to \Hom (\mathscr{H}_\chi,
\mathscr{H}_\chi)$. However it is possible that there are more functions
$\varphi$ than would be absolutely necessary. 
That is, there may exist elements $\chi$, $\chi'$ in $\Omega$ such that
$\varphi_A(\chi)=\varphi_A(\chi')$ for every $A \in \mathcal{A}$. In
this case, we have two points $\chi$, $\chi'$ in the base space $\Omega$
which are in some sense equivalent with respect to
$\mathcal{A}$. Therefore we introduce an equivalence relation $R$ in
$\Omega$ by setting $\chi \sim \chi'$ whenever $\varphi_A (\chi)
=\varphi_A(\chi')$ for every $A \in \mathcal{A}$. This is a closed
equivalence relation and $\Omega/R=\mathcal{Z}$ is a compact Hausdorff
space. The same procedure for $\mathcal{Z}$ and $\mathcal{A}$ as for
$\Omega$ and $\mathscr{M}'$ gives a Hilbert space $\mathcal{H}_\zeta$ at each
point $\zeta \in \mathcal{Z}$  and a continuous representation of the
algebra $\mathcal{A}$ in $\Hom (\mathscr{H}_\zeta,
\mathscr{H}_{\zeta})$. The
image of the measure $d \mu_{a, a}$ by the canonical map $\Omega \to
\mathcal{Z}$ is denoted by $\mu$. At each point $\zeta$ we have a map
$\mathcal{A} \to \mathcal{H}_\zeta$ and hence for a  fixed $A \in
\mathcal{A}$ we obtain a vector field. This family of vector fields is
the \textit{fundamental family} we sought to construct. In fact,  
\begin{enumerate}
\renewcommand{\theenumi}{\alph{enumi}}
\renewcommand{\labelenumi}{(\theenumi)}
\item They constitute a vector space, since $\mathcal{A}$ is an algebra.

\item $||
  X_A(\zeta)||=\varphi_{A*A}(\zeta)^{\frac{1}{2}}$\pageoriginale and
  hence $|| X_A(\zeta)||$ is continuous. 

\item For each $\zeta$, $X_A(\zeta)$ in everywhere dense since
  $\mathscr{H}_\zeta$ is only the completion of the space of
  $X_A(\zeta)$. 

\item[(c$'$)] Consider $\sum M_i B_i$ where $M \in \mathcal{A}\cap
  \mathscr{M}$ and $B$ is a finite product $Ai_1 Ai_2 \ldots Ai_p$ of
  the $A_n$. Then 
  $X_{\sum M_{i} B_{i}} =\sum U_{M_{i}} X_{B_{i}}$ with $U_M$ being
  scalars. $\{X_{B_{i}}\}$ is only a countable family and the vectors
  $X_{\sum M_i B_i}(\zeta)$ are dense in
  $\mathscr{H}(\zeta)$. Consequently the countable family $X_{\sum
    \alpha_i B_i}(\zeta)$  where the $\alpha_i$ are complex numbers
  with real and imaginary parts rational is also dense in
  $\mathscr{H}(\zeta)$. 
\end{enumerate}

Thus we have now all the data for the construction of a continuous sum
$\mathscr{L}^{2}_{\wedge}$. of course we have still to establish that
$L^{2}_{\wedge}=\mathscr{H}$. In fact, since $\mathcal{Z}$ is compact,
every continuous vector field is square summable. Therefore, we have 
$$
||X_A||^2=\int||X_A(\zeta)||^2d \mu (\zeta)=\int
\varphi_{A*A}(\chi)d\mu_{a,a}(\chi)=||Aa||^2 
$$

Also $Aa=0$ implies $X_A=0$. Therefore, the map $Aa\to X_A$ is an
isometry of a dense subspace of $\mathscr{H}$ and hence can be
extended to an isometry $J :\mathscr{H}\to L^2_{\wedge}$. It remains
to prove that this map is surjective. We have seen
(Prop. \ref{partIII-chap1-prop2}, Ch. \ref{partIII-chap1-sec1.2})  
that vector fields of the form $\sum \varphi_i (\zeta) Y_i(\zeta)$
where $\varphi_i(\zeta)$ are continuous functions of $\zeta$ are dense
in $L^2_{\wedge}$. Therefore it is enough to prove that
$J(\mathscr{H})$ contains all such elements (the image of
$J(\mathscr{H})$ being closed in $L^2_{\wedge}$). Let $\mathscr{M}_0$ be the
self adjoint subalgebra of $\mathscr{M}$ consisting of elements $M$ such that
$\hat{M}$ is constant on cosets modulo the equivalence relation
$R$. Consequently, $\hat{M}$ may be considered as a continuous map on
$\mathcal{Z}$. But we have $||\sum M_i A_i^a||^2= \int_{\mathcal{Z}}
||\sum \hat{M}_{i} X_{A} \|^2 d\mu$ and therefore the map $J^1 : \sum
M_i A_i^a \to \sum \hat{M}_i X_{A_i}$ is an isometry which
coincides\pageoriginale 
with $J$ on the elements $Aa$. This shows that $\sum
\hat{M}_{i}X_{A_{i}}\in J(\mathscr{H})$ and hence
$L^{2}_{\wedge}=J(\mathscr{H})$.  

Hereafter we shall identify $L^2_{\wedge}$ with $\mathscr{H}$. We now
assert that $\mathcal{A}$ is contained in the space of decomposed
operators on $\mathscr{H}$. In fact, we will show that $A= \int U_A
(\zeta)$. We have already proved (Ch. \ref{partIII-chap2-sec2.2}) that
$||U_A (\zeta) |||\le 
||A||$ and $U_A(\zeta)$ is hence bounded. Also $U_A
(\zeta)X_B(\zeta)=X_{AB}(\zeta)$ is again a continuous vector field
and by Prop. \ref{partIII-chap1-prop5},
Ch. \ref{partIII-chap1-sec1.6}, $U_A(\zeta)$ is a 
continuous operator 
field. If now $\bar{A} = \int U_A (\zeta)$, then $ \bar{A}
Ba=U_A(\zeta)X_B(\zeta)=X_{AB}(\zeta)=ABa$ by our identification for
every $B\in \mathcal{A}$.  Since  $\{Ba : B \in \mathcal{A}\}$ is
dense in $\mathscr{H}$, we have $\bar{A}=A$. In other words, every
operator $A \in \mathcal{A}$. is decomposable into a
\textit{continuous} operator field. 

Now, we have another algebra $\mathscr{M}$ of operators on
$\mathscr{H}$. It is natural to expect then $\mathscr{M}$ consists of scalar
decomposed operators. It 
is of course true, but the proof is not obvious. As before, let
$\mathscr{M}_0$ be the subalgebra of $\mathscr{M}$ composed of
elements $M$ such that $\hat{M} \in \mathscr{C} (\mathcal{Z)}$. We
first prove that $\mathscr{M}_0$ consists of 
scalar decomposed operators. Let $B, C \in \mathcal{A}$.  Then 

\vfill\eject
\begin{align*}
\langle MBa, Ca \rangle & = \langle MC^*Ba, a \rangle = \int_\Omega
\bar{M}(\chi) d \mu_{C^*Ba, a}(\chi)\\ 
& = \int  \hat{M}(\chi) \varphi_{C^*B}(\chi)d\mu_{a,a}(\chi)\\ 
& = \int_{\mathcal{Z}}  \hat{M}(\zeta) \varphi_{C^*B}(\zeta) d
\mu(\zeta )\quad \text{ (since the continuous}\\[-4pt]
&\hspace{4.5cm} \text{ functions are constant}\\
& \hspace{4.5cm}\text{ on the equivalence classes) }\\ 
& = \int_\mathcal{Z}' \hat {M}(\zeta) \langle X_B (\zeta), X_C (\zeta)
\rangle d\mu(\zeta) \; \text { by definition of the norm}.  
\end{align*}

That is to say that $M  = \int_{\mathcal{Z}} \hat{M}(\zeta)$. Identity. 
We\pageoriginale 
now extend this result to every element $M \in \mathscr{M}$. Since we know
that the space of scalar operators is weakly closed, it suffices to
prove that $\mathscr{M}\subset \mathscr{M}''_0$. Again by the
Hahn-Banach theorem, it is enough to 
show that any weakly continuous linear form which is zero on $\mathscr{M}_0$
(and hence on $\mathscr{M}_0''$) is also zero on $\mathscr{M}$. 
But any weakly continuous
linear form on $\Hom_S(\mathscr{H},\mathscr{H}_W)$ is of the form $U
\to \sum\limits^n_{i=1} \langle U X_i, Y_i \rangle$. if  $\sum \langle
MX_i, Y_i \rangle = 0$ for every $M \in \mathscr{M}_0$, then 
$\sum \int_{{\mathcal{Z}}}\hat{M}(\zeta) \langle X_i (\zeta), Y_i
(\zeta) \rangle d\mu (\zeta) = 0$. Hence  
$\sum \langle X_i (\zeta), Y_i (\zeta) \rangle = 0$ for almost every
$\zeta$ in $\mathcal{Z}$, or again $\sum \langle X_i (\pi (\chi)), Y_i
(\pi(\chi))\rangle= 0 a. e. $ on $\Omega$ where $ \pi$ is the
canonical map $\Omega \rightarrow \mathcal{Z}$. Therefore, $\sum
\langle MX_i, Y_i \rangle  
= \int_{\Omega}\hat{M}(X) \sum \langle X_i (x), Y_i (x) \rangle d
\mu_{a, a}= 0$ for every $ M \in \mu$. This completes the proof of our
assertion. 

\subsection{Irreducibility of the components - Mautner's
  theorem}\label{partIII-chap2-sec2.4}% sec 2.4 

Finally it remains to show that the unitary representations of the
algebra $\mathcal{A}$ in the $\mathscr{H}_\zeta$ are irreducible. The
algebra $\mathscr{M}$ is at our choice and we are interested in taking it as
large as possible. Thus we assume that $\mathscr{M}$ is a maximal commutative
subalgebra of $\mathcal{A'}$ and obtain the  

\setcounter{thm}{0}
\begin{thm}[Mautner]\label{partIII-chap2-thm1}
{Let} $\mathcal{A}$ be any uniformly closed *-subalgebra of $\Hom
\mathscr{(H, H)}$  such that 
\begin{enumerate}
\renewcommand{\theenumi}{\alph{enumi}}
\renewcommand{\labelenumi}{\rm(\theenumi)} 
\item there exists a sequence $A_n$ which generates
  $\mathcal{A}$; 

\item there exists an element $a \in \mathscr{H}$ such that the
  set $\{ Aa : a \in \mathcal{A}\}$ is dense in $\mathscr{H}$. 
\end{enumerate}

{Let} $\mathscr{M}$ {be any maximal commutative} *-{subalgebra of}
$\mathcal{A}'$. {Then in the decomposition of} $\mathscr{H}$ {into
  the continuous sum of the} $\mathscr{H}_{\zeta}$ {with respect
  to } $\mathscr{M}$\pageoriginale 
{and} $\mathcal{A}$, {almost every representation}
$U_A (\zeta)$ {of} $\mathcal{A}$ in $\mathscr{H}(\zeta)$ {is irreducible}. 
\end{thm}

Let $\{e_n\}$ be the orthogonal basics given in
Ch. \ref{partIII-chap1-sec1.5} with respect 
to the fundamental sequence $\wedge_0$ and let $\mathscr{M}_0$ be the subset
$\{M \in \mathscr{M}:\hat{M}\in \mathscr{C} (\mathcal{Z})\}$ of
$\mathscr{M}$. If 
$\mathscr{B}$ be the algebra generated by $\mathcal{A}$ and $\mathscr{M}_0$,
then $\mathscr{M}=\mathscr{B}'$. In fact, we have seen that
$\mathscr{M} \subset \mathscr{M}''_0$ (Ch. \ref{partIII-chap2-sec2.3})
and therefore 
$\mathscr{M}'=\mathscr{M}'_0$. Hence $\mathscr{B}'=\mathcal{A'} 
\cap \mathscr{M}'_0 =\mathcal{A}'\cap \mathscr{M}'$ and $\mathcal{A}'
\cap \mathscr{M}'=\mathscr{M}$, $\mathscr{M}$ being a
maximal subalgebra. We define for any two integers $p$, $q$ as in theorem
\ref{partIII-chap1-thm3}, Ch. \ref{partIII-chap1-sec1.8}, Hermitian
decomposed operators $H_{p,q}$ on 
$\mathscr{H}$ such that 
\begin{align*}
H_{p,q}(\zeta) e_n (\zeta) & = 0 \text{~ if~ } n \neq p \text{~ or~ } q;\\
H_{p,q}(\zeta) e_p(\zeta) & = || e_p (\zeta ) ||^2 e_q (\zeta),
\text{~ and}\\  
H_{p,q}(\zeta) e_q (\zeta) & = || e_q (\zeta ) ||^2 e_p (\zeta ).
\end{align*}

We have already seen (Ch. \ref{partIII-chap1-sec1.8}) that any
operator which commutes with 
all the $H_{p,q}(\zeta)$ is a scalar operator. Now, $H_{p,q}$ is
bounded since we have $|| H_{p,q} (\zeta)|| \leq || e_p
(\zeta)||~|| e_q (\zeta)|| \leq || e_p ||~ ||e_q ||$. $H_{p,q}$ is
continuous since it transforms every vector filed of type
$e_{j}(\zeta)$ into another continuous vector field
(Prop. \ref{partIII-chap1-prop5}, Ch. \ref{partIII-chap1-sec1.6}). Now
$H_{p,q}=\int_{\mathcal{Z}} H_{p,q}(\zeta) 
d \mu(\zeta)$ and this commutes with every element of
$\mathscr{M}$. Therefore   
$$ 
H_{p,q}=\int_{\mathcal{Z}}H_{p,q} (\zeta)d\mu (\zeta)\in
\mathscr{M}'=\mathscr{B}''. 
$$

Let $Y_n = e_n/|| e_n ||^{1/n}$. Then we have $\sum || Y_n ||^2 <
\infty$ and by theorem \ref{partII-chap5-thm2},
Ch. \ref{partII-chap5-sec5.6} Part \ref{partII}, there exist Hermitian 
operators $B_k \in \mathscr{B}$ such that $\sum_{n}||
H_{p,q}Y_n - B_k Y_n ||^2 < 1/k^2$. Therefore $|| H_{p,q} e_n - B_k
e_n || \leq \dfrac{|| e_n ||^n}{k}\rightarrow 0$ as $k \rightarrow
\infty$., i.e., $\int \sum_{n}|| H_{p,q}(\zeta)Y_n (\zeta)-
B_k (\zeta)Y_{n}(\zeta)||^2 d\mu (\zeta)\rightarrow 0$
as\pageoriginale $k \rightarrow \infty$. As in Riesz-Fisher theorem,
we can find a subsequence 
$B_{k_i}$ such that $H_{p,q} Y_n - B_{k_i} Y_n \rightarrow 0$ as $k_i
\rightarrow \infty$ outside a set of measure zero. Since the $H_{p,q}$
are only countable in number, we can pass to the diagonal sequence and
get a sequence $B_{k_j}$ such that \textit{for every}  $p,q$, $H_{p,q}
Y_n-B_{k_{j}}Y_{n} \rightarrow 0$ as $k_j \rightarrow \infty$ outside
a set $N$ measure zero. 

Let $\zeta \not\in N$ and $L$ be a subspace invariant under
$\mathcal{A}(\zeta)$ or again under $\mathcal{A}(\zeta)''$. Let $S$ be
any element of $\mathcal{A}(\zeta)'$. Then $S$ commutes with every
$U_B (\zeta)$ where $B \in \mathscr{B}$ (where $U_B (\zeta)=
\sum_{i} M_i (\zeta) U_A (\zeta)$ whenever
$B=\sum_{i} M_i A_i$). 

So,
\begin{align*}
\langle SH_{p,q}(\zeta) e_n (\zeta), e_m (\zeta) \rangle & = \lim
\langle SU_{B_k}(\zeta), e_n (\zeta), e_m (\zeta)\rangle\\  
& = \lim \langle U_{B_k}(\zeta), Se_n (\zeta), e_m (\zeta)\rangle \\
& = \lim \langle S_{e_n}(\zeta), U_{B_k} (\zeta), e_m (\zeta)\rangle\\
& = \langle S_{e_n}(\zeta), H_{p,q} (\zeta), e_m (\zeta)\rangle.
\end{align*}
or $H_{p,q}(\zeta)$ commutes with every $S \in \mathcal{A}(\zeta)'$
for almost every $\zeta$. Hence $\mathcal{A}(\zeta)'$ consists only of
scalar  operators for almost every $\zeta$. Hence the only invariant
subspaces of $\mathcal{A}(\zeta)''$ are the trivial ones and the
representation $A \rightarrow U_A (\zeta)$ of  $\mathcal{A}$  in
$\mathscr{H}(\zeta)$  is irreducible. 

The following corollary is more or less immediate:

\begin{coro*}
 Let $U$ be unitary representation of a separable, locally compact
 group $G$ in a Hilbert space $\mathscr{H}$ such that there exists $a \in
 \mathscr{H}$ with the minimal closed invariant subspace containing
 $a=\mathscr{H}$. Then $U$ is a continuous sum of unitary
 representations which are {\em almost all} irreducible. 
 \end{coro*}

In\pageoriginale fact, in Mautner's theorem, we have only to take for
$\mathcal{A}$ 
the uniform closure of the algebra generated by elements of the form
$U_x, x \in G$. 

\subsection{}\label{partIII-chap2-sec2.5}%2.5

We have already said that the space $\Omega $ could have been  used in
much  the same way as the space $\mathcal{Z}$. We now give an
illustration to explain our remark that $\Omega$ is too large for
practical purposes and that in the decomposition with respect to
$\Omega$ the same representations may repeat `too often' (which is
what we sought to avoid by our equivalence relation). Let
$\mathscr{H}$ be the space $\mathscr{L}^2 (R)$ and $U$ the regular
representation $\sigma_x$ of $R$ in $L^2 (R)$. Let $\mathcal{A}$ be
the algebra of operators $\sigma_f$ for $f \in L ^1$. It can easily be
seen that $\mathcal{A}''$ contains all operators $\sigma_\mu$ on $L^2$
with $\mu$, a bounded measure. $\mathcal{A}$ and $\mathcal{A}''$ are of
course commutative. Therefore $\mathcal{A}' \supset \mathcal{A}''$. We
say take $m'$ to be $\mathcal{A}''$ itself. If $\Omega$ is the spectrum
of $\mathscr{M}'$ it contains the spectrum of $\mathscr{M}^1$.  This
is the much larger 
than the spectrum of $L^1$, whereas a `good' decomposition of $L^2(R)$
into a Fourier transform is given by $f \rightarrow \hat{f}
(y)=\dfrac{1}{\sqrt{2\pi}} \int f(x)e^{ixy} dx$. The Fourier in
version formula will be $f=\dfrac{1}{\sqrt{2\pi}} \int \hat{f}(y)
e^{ixy}dy$.  $L^2$ is the direct integral of Hilbert spacers of
dimension $1$. If $\mathscr{M}^1$ denotes the set of bounded measures, then $L^1$
is an ideal in $\mathscr{M}^1$ and is hence contained in a maximal ideal of
$\mathscr{M}^1$. Thus two different characters of $\mathscr{M}^1$ give
rise to the same 
character of $L^1$. Thus a decomposition with $\Omega$ consists of
unnecessarily repeated representations while that with $\mathcal{Z}$
(spectrum of $L^1$ in our example) economises them and reduces the
decomposed representations to a minimum. 

\subsection{Equivalence of
  representations}\pageoriginale\label{partIII-chap2-sec2.6} %2.6

 The decomposition into continuous sum is obviously not unique,
 because the process depends on the choice of  $a \in\mathscr{H}$ such
 that $\{ Aa:A \in \mathcal{A} \}$ is dense in $\mathscr{H}$ and on this
 choice of $\mathscr{M}$. The question therefore arises whether all these
 decompositions are equivalent in some sense. 

\begin{defi*}
 Two decompositions $L^2 _{\wedge_1}, (\mathcal{Z}_1,\mu_1, \wedge_1)$
 and  $L^2 _{\wedge_2} (\mathcal{Z}_2, \mu_2, \wedge_2 ) $ of
 $\mathscr{H}$ are said to be equivalent if there exists a measurable
 one-one map $t : \mathcal{Z}_1\rightarrow \mathcal{Z}_2$ and a map
 $U_\zeta  $ of  $\mathscr{H}(\zeta)$ onto $\mathscr{H}(t(\zeta))$
 such that, the correspondence to every vector field $X$ on
 $\mathcal{Z}_1$ of a vector  field on $\mathcal{Z}_2$ defined  by
 $Y(t(\zeta))=  U_\zeta X (\zeta)$, is an isomorphism of
 $L^2_{\wedge_1}$ onto $L^2_{\wedge_2}$. 
 \end{defi*}

It is almost immediate that if the vector a is changed, we get
equivalent decompositions. However, it is not true that if $\mathscr{M}$ is
chosen  in different ways the corresponding decompositions are
equivalent.  
