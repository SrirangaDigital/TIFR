
\chapter{Regular Representations}\label{partII-chap4}

\setcounter{section}{4}
\setcounter{subsection}{0}
\subsection{General
  notions}\label{partII-chap4-sec4.1}\pageoriginale%sec 4.1  

Let $G$ be a locally compact group and $E$ a locally convex
topological vector space. 

\begin{defi*} %def
 A {\em continuous representation} of $G$ in $E$ is a map $x\rightarrow U_x$
 of $G$ into $\Hom (E,E)$ such that this is a representation in the
 algebraic sense (i.e. $U_{xy}= U _x  U_y$ and   $U_e =$ identity)
 and such that the map $(a,x)\rightarrow U_xa$ of $E\times G
 \rightarrow E$ is continuous. 
\end{defi*}

The latter of these conditions, which we denote by $R$, is equivalent
to the following: 

$R'_1$: For every compact subset $K$ of $G$, the set $\{U_x :
x \in K\}$ is equicontinuous, and $R'_2$: for every $a\in E$, the
map $x \rightarrow U_xa$  of $G \rightarrow E$ is continuous. 

In fact, $R \Rightarrow R'_2$ trivially. Let $V$ be a neighbourhood
of $0$ in $E$.  For every $x\in K$, there exists an neighbourhood $A_x$
of $x$ in $G$ and $W_x$  of $0$ in $E$ such that for every $y\in A_x$
and $ a\in W_x$, $U_y$ $a \in V$. Since $K$ is compact, we can choose
$A_{x_1},\ldots A_{x_n}$ which cover $K$ and let $W =\bigcap
W_{x_j}$. Now, $x\in K$, $b\in W \Rightarrow U_x b\in V$. Hence the set
$\{ U_x\}$ is equicontinuous. We proceed to prove the
converse; in fact, we show more generally that $R'_1$ with the
following condition $R''_2$: There exists a dense subset $F$ of $E$
such that for every $a \in F$, the map $x\rightarrow U_x a$ of
$G\rightarrow E$ is continuous, implies $R$. It is required to show
that the map $(x,a)\rightarrow U_x a$ is continuous at any point
$(x,a)$. Let $V$ be any convert neighbourhood of $0$ in $E$. We seek a
neighbourhood $W$ of $0$ in $E$ and a neighbourhood $A$ of $x$ in $G$
such that $b\in a + W$, $y\in A\Rightarrow U_y b\in U_x$ $a
+ V$. Let $K$ be a compact neighbourhood of $x$ in $G$. Then there
exists a neighbourhood $V_1$  of $0$ such that $y\in K$, $b\in V_1
\Rightarrow U_y b\in V/4$ (by $R'_1$). Let $b \in (a+V_1) \bigcap
F$. Then we can find a   
neighbourhood\pageoriginale $A$ of $x$ in $C$ contained in $K$ such
that $U_yb - U_xb\in V/4$ for every $y \in A$ (by $R''_2$). Now, if $c \in
a+V_1$, $y \in A$, we have  
$$ 
U_yc - U_xa = U_y (c-a) + (U_y-U_x)b + (U_y - U_x)(a-b) \in V. 
$$ 

This completes the proof of the equivalence.

Moreover, if $E$ is a barrelled space (or in particular a Banach
space) then axiom $R'_2$ itself $\Rightarrow R$. For, the map $x
\rightarrow U_x$ is continuous from $G$ to $\Hom(E,E)$ with the
topology of simple convergence and hence the image of a compact subset
is again compact and, $E$ being barrelled, equicontinuous. 

\subsection{Examples of
  representations}\label{partII-chap4-sec4.2}%sec 4.2 

\begin{description}
\item[(i)] \textbf{Unitary representations.} Let $U$ be a
  representation of $G$ in a Hilbert space $H$ such that $U_x$ is a
  unitary operator, i.e. $U_x-1 = U^*_x$  for every $x \in G$. Then
  $V$ is called {\em a unitary representation.} 

\item[(ii)] \textbf{Bounded representations.} A representation $U$ of
  $G$ in a Banach space $B$ is said to be \textit{bounded} if there
  exists $M$ such that $|| U_x ||< M$ for every $x\in G$. It should be
  remarked here that in general representations in Banach spaces are
  not bounded as for instance the representation $x\rightarrow
  e^x$. Id of $R$ in itself. However, such representations are bounded
  on every compact subset. 

\item[(iii)] \textbf{Regular representations.} Left and right
  translations in $G$, as we have seen before, give rise to
  representations of $G$ in the space $\mathscr{C}_G, L^p$ etc. In
  fact they give rise to representations of $G$ in any function space
  connected with $G$ with a reasonably good definition and a
  convenient topology.  


The space $\mathscr{C}_G$ with the usual topology can easily be seen
to be barrelled. Therefore, in order to verify that $\tau$ is a
continuous representation,\pageoriginale we have only to prove that
$y\rightarrow\tau_y f$ is continuous. It is again sufficient to
establish continuity at the point $y = e$. The function $f$ is
uniformly continuous and hence when $y \rightarrow e$,
$\tau_y f\rightarrow  f$ uniformly having its support contained in a
fixed compact set. 

In the case of $L^p (1\leq p \leq \infty)$ with respect to the
right Haar measure, $||\tau_y|| = 1$ by the invariance of the measure
and hence the $\tau_y$ form an equicontinuous set for $y\in G$. Also
the map $y \rightarrow\tau_y f$ is continuous for the topology on
$\mathscr{C}_G$ which is finer than that of $L^p$ and since
$\mathscr{C}_G$ is dense in $L^p$ is a representation of $G$ in
$L^p$. On the other hand, if we consider $\sigma_y$, we have
$||\sigma_yf|| = (\int |f(y^{-1}x)|^p dx)^{1/p} =
(\Delta(y))^{1/p}||f||_p$ and the continuity of
$y\rightarrow\sigma_yf$ follows as a consequence of the continuity of
$\Delta(y)$. Note that the proof is not valid when $p$ is
infinite as $\mathscr{C}_G$ is not dense in $L_\infty$. In fact the
map $y\rightarrow \tau_y f$ of $G\rightarrow L_\infty$ is continuous if
and only if $f$ is uniformly continuous on $G$. 

\item[(iv)] \textbf{Induced representations.} Let $H$ be a closed
  subgroup of a topological group $G$, and $L$ a continuous
  representation of $H$ in a locally convex space $E$. Let
  $\mathscr{C}^L$ be the space of functions on $G$ with values in $E$
  which are continuous with compact support modulo $H$ (i.e. their
  supports are contained in the saturation of a compact set), and
  which satisfy the following equality: 


$f(\xi x) =
\left(\dfrac{\delta(\xi)}{\Delta(\xi)}\right)^{\frac{1}{2}}
L_\xi f(x)$ for every $x\in G$ and $\xi \in H$. The factor
$\left( \dfrac{\delta(\xi)}{\Delta(\xi)}\right)^{\frac{1}{2}}$, it will be
noted, occurs purely for technical reasons and can without much
trouble be done away with. The above equality, in essence, expresses
only the condition of covariance of $f$ with respect to left
translations by $\xi$. On this space $\mathscr{C}^L$ we can, as we
have more than once done before, introduce the topology of direct
limit of those on $\mathscr{C}^L_K$,  the\pageoriginale 
latter being the space of
functions in $\mathscr{C}^L$ whose supports are contained in $HK$,
with the topology of uniform convergence on $K$. This is again a
locally convex space and right translations by elements of $G$ give
rise to a (regular) representation of $G$ in $\mathscr{C}^L$. This is
called the \textit{representation} of $G$ \textit{induced} by $L$. 

\item[(v)] Let us again assume $L$ to be a unitary representation of
  $H$ in a Hilbert space $E$. Let $\mathscr{C}^L$ be the space of
  continuous functions on $G$ with values in $E$ having compact
  support modulo $H$ such that $f(\xi x) =
  \left(\dfrac{\delta(\xi)}{\Delta(\xi)}\right)^{\frac{1}{2}}
  L_\xi f(x)$. Naturally one tries to introduce a scalar product in
  $\mathscr{C}^L$ in the usual way, but the possibility of $f(x)$ not
  being square integrable (which 
it is not in general) foils the attempt. However, though $f(x)$ may
not be square integrable, we are in a position to assert that
$\int_{G/H}||f(x)||^2(\rho(x))^{-1} dx<\infty$ ($\rho$ of course
is the  function defined in Lemma \ref{partII-chap3-lem2},
Ch. \ref{partII-chap3-sec3.4} and $dx$ the quasiinvariant 
measure on $G/H$). In fact, the function $||f(x)||^2(\rho(x))^{-1}$
is invariant modulo $H$ and consequently can be considered as a
continuous function on $G/H$ with compact support. Hence we can define
$||f||^2_L = \int_{G/H}||f(x)||^2\left((\rho(x))\right)^{-1}
dx$. Let $\mathscr{H}^L$ be the completion of $\mathscr{C}^L$ under
this norm. As usual, $\mathscr{H}^L$ is the space of measurable
functions $f$ which satisfy the condition of covariance and are such
that $\int_{G/H}||f(x)||^{2}(\rho(x))^{-1}dx<\infty$.
This 
is a Hilbert space in which the scalar product is given by $\langle
f,g \rangle  = \int\limits_{G/H} \langle f(x), g(x) \rangle (\rho
(x))^{-1}  dx$. The right regular representation of $G$ in
$\mathscr{H}^L$ is unitary. For,  
\begin{align*}
||\tau_yf||^2 & = \int_{G/H}||f(xy)||^2
\left((\rho(x))\right)^{-1}d\dot{x}\\ 
& =\int_{G/H}||f(x)||^2
\left((\rho(xy^{-1}))\right)^{-1}\frac{\rho(xy^{-1})}{\rho(x)}dx
     \text{~ by quasi invariance}\\ 
& =\int_{G/H}||f(xy)||^2 \left ( (\rho(x))\right )^{-1}d\dot{x}\\ 
& =||f||^2_L \text{~ for every~ }f\in\mathscr{H}^L
\end{align*}
\end{description}

The\pageoriginale same proof as in (iv) gives the continuity of the
representation. 

Thus one can define induced representations in many ways, in each case
the representation space being so chosen as to reflect the particular
properties of the representation one wishes to study. We shall not
dwell on induced representations any longer, but only give the
following general definition which, it is needles to say, includes the
last two cases. 

\begin{defi*}%def
 A representation $U$ of $G$ in $F$ is said to be {\em induced} by
 representation $L$ of $H$ in $E$ if there exists a linear continuous
 map $\eta$ of $\mathscr{C}^L$ into $F$ such that 
\begin{itemize}
\item[(i)] $\eta$ is injective with its image in $F$ everywhere dense,
 and  

\item[(ii)] $\eta$ commutes with the representation in the sense that
$U_x \circ \eta= \eta \circ \tau_x$ for every $x\in G$. 
\end{itemize}
\end{defi*}

\subsection{Contragradient representation}\label{partII-chap4-sec4.3}

Let $U$ be a continuous representation of $G$ in a locally convex
space $E$. For every $x \in G$ consider $t_{U_{x}}\in \Hom(E',
E')$ which is continuous for any `good' topology on $E'$ (weak, strong
of convex-compact). We denote by $\check{U}$ the map $x\rightarrow
{}^t U_{x}^{-1}$. Regarding this map we have the following 

\setcounter{proposition}{0} 
\begin{proposition}\label{partII-chap4-prop1}%pro 1
 If $U$ is a continuous representation of $G$ in a quasi complete
 locally convex space $E$, then $\check{U}$ is also a continuous
 representation of $G$ in $E'_C$ (convex compact topology). We need
 here the following formulation of {\em Ascoli's theorem}. Let $X$ be
 a locally compact topological space, $F$ a uniform Hausdorff space
 and $\mathscr{C}(X,F)$ the space of continuous functions from $X
 \rightarrow F$. Let $\Lambda$ be an equicontinuous subset of
 $\mathscr{C} (X,F)$ such that the set $\{ \lambda(x) : \lambda \in
 \Lambda \}$ is relatively compact in $F$ for every $x \in X$. Then
 (i) $\Lambda$ is relatively compact in $\mathscr{C} (X,F)$ with the
 topology of compact convergence,\pageoriginale and (ii) on $\Lambda$
 the topology 
  of compact convergence coincides with every Hausdorff weaker topology
 (in particular, with the topology of simple convergence). 
\end{proposition}

\medskip
\noindent
{\bf Proof of proposition~\ref{partII-chap4-prop1}.}~
Let $K$ be a compact of $G$. The set $\{ U_x : x\in K \}$ is
equicontinuous and for every $a\in E$, $\{U_xa : x\in K \}$ is compact
as the map $x \rightarrow U_xa$  is continuous. Hence by Ascoli's
theorem, the topology of simple convergence and the topology  of
compact convergence are one and the same on this subset of
$\mathscr{C} (E,E)$ i.e. $x\rightarrow e\Rightarrow U_xa
\rightarrow a$ uniformly for a in a compact subset $H$ of $E$. 

Let $a'$ be an element of $E'_C$. We wish to prove that $x\rightarrow
\check{U}_x a'$ is continuous. It is enough to prove the continuity at
the unit element $e$. Let $x \rightarrow e$. Then $U_{x^{-1}} a\rightarrow
a$ uniformly on a compact set $H$ and hence $\langle U_{x^{-1}}a,a'\rangle
= \langle a,\check{U}_x a'\rangle \rightarrow \langle a,a' \rangle$
uniformly on $H$. This shows that $x \rightarrow \check{U}_x a'$ is
continuous for the convex-compact topology on $E'$. We have to show
moreover that the set$\{\check{U}_x : x\in K \}$ is
equicontinuous. If $H$ be a convex compact subset of $E$, we seek to
prove the existence of another convex compact set $H'$ such that
$a' \in (H')^0 \Rightarrow \check{U}_xa' \in H^0$ for every
$x\in K$ where $A^{0}$ denotes the polar of $A$. But
$\check{U}_{x}a'\in H^{0}$ for every $x\in K$ if and only if $\Big
|\langle U_{x^{-1}} a,a' \rangle \Big|\leq 1$ 
for every $x\in K$ and $a \in H$. Let $H'$ be the closed convex
envelope of the compact set description by $U_{x^{-1}}a$. It is obvious
that $a^1 \in (H')^0 \Rightarrow \big| \langle b,a' \rangle \big|
\leq 1$ for every $b\in H' \Rightarrow \big| \langle U_{x^{-1}}a,a'
\rangle \big|\leq 1$ for every $x\in K$ and $a\in H$. It only remains
to show that $H'$ is compact. But $H'$ is precompact, and being a
closed bounded set, also complete. This shows that $H'$ is compact. 

This representation in $E'$ is called the \textit{contragradient} of $U$.

\begin{remark*}%rem
 It will be noted that we have used the quasi completeness of the space
 $E$\pageoriginale only to prove that the closed  convex  envelope  of
 a compact 
 set is also  compact. Hence the proposition  is valid for  the more 
 general class of locally convex spaces which satisfy the above
 condition. 
\end{remark*}
 
\begin{example*}
 We have seen that the right and left  translations give
 representations  of $G$  in the function spaces
 $\mathscr{C}_{G}$, $\mathscr{E}^{0}$, $\bar{\mathscr{C}}_{G}$,
 etc. By our proposition above,we see that $\sigma$ is continuous 
 in $\mathscr{M}$, $\mathscr{M}^{C}$, $\mathscr{M}^{1}$ 
(with the convex compact topology).
\end{example*}
 
\begin{remark*}
The regular  representation of $G$ in $\mathscr{M}^1$ is {\em not continuous}
with respect  to the {\em strong} topology. For,  
 $\tau_{x} \epsilon_{e} = \epsilon_{x}$ and $|| \epsilon_{x}
-\epsilon_{e}|| = 2$ if $x\neq  e$. Hence as $x \rightarrow e$,
$\epsilon_{x}$ does not tend to $\epsilon_{e}$. 
\end{remark*}
 
\subsection{Extension of a  representation to
  $\mathscr{M}^{C}$}\label{partII-chap4-sec4.4}%sec 4.4  
 
 Let $U$ be a continuous  representation of $G$ in a locally convex
 quasi complete space $E$. Let $\mu$  be any  measure on $G$ with
 compact support. Then we write $U_{\mu}  a = \int_{G} U_{x} ad \mu
 (x)$. (The function  $x \rightarrow  U_{x} a$ is a vector-valued
 function and $\int U_{x} ad \mu(x)$ has been defined in
 Ch. \ref{partII-chap1-sec1.5}). 
  
\setcounter{thm}{0}
\begin{thm}\label{partII-chap4-thm1}%the
 \begin{enumerate} 
\renewcommand{\labelenumi}{\rm(\theenumi)}
\item $U_{\mu}$ {is a linear continuous  function of $E$ in itself}.

\item $\mu \rightarrow U_{\mu}$ {is an algebraic representation
  of} $\mathscr{M}^{C}$ {in} $E$. 

\item {If} $U$ {is a bounded representation in a Banach space,
  then this representation  can be extended to a continuous
  representation of the Banach algebra} $\mathscr{M}^{1}$ {in to  the Banach
  algebra} $\Hom(E,E)$. 
\end{enumerate}
\end{thm}
  
\begin{enumerate}
\renewcommand{\labelenumi}{(\theenumi)}
\item  In fact, as  $a \rightarrow  0$, $U_{x} a  \rightarrow 0$
  uniformly on the compact support of $\mu$ and hence $U_{\mu}$ is
  continuous. Its linearity is trivial. 

\item Again  the linearity of the map $\mu  \rightarrow  U_{\mu}$ is
  obvious.  
\begin{align*}
U_{\mu * \nu } a & = \int U_{x} a d (\mu * \nu ) (x)\\
& =\iint U_{xy}ad \mu (x)d  \nu (y)\quad \text{(see Remark 2,
  Ch. \ref{partII-chap2-sec2.2})}\\ 
& =\int d \mu(x)U_{x}(U_{\nu} a) \text{(Remark, Ch. \ref{partII-chap1-sec1.5})}\\
& =U_{\mu}U_{\nu} a. 
\end{align*}\pageoriginale

\item If $\mu \in \mathscr{M}^{C}$, we have
\begin{align*}
|| U_\mu a ||   & = || \int U_{x} ad_\mu(x)||\\
& \leq \int  || U_{x}a   ||d| \mu |\\ 
& \leq  k|| a  ||~     || \mu  ||
\end{align*}
\end{enumerate}

This proves that the map $(a,\mu)\rightarrow U_{\mu}a$ of $E\times
\mathscr{M}^{C} \rightarrow E$  is continuous. Since $\mathscr{M}^{1}$
is only the 
completion  of $\mathscr{M}^{C}$  with the  topology of the norm, this
map can be 
extended to a continuous map $E \times \mathscr{M}^{1}\rightarrow E$, which
proves all that was asserted. 

\subsection{Convolution of measures}\label{partII-chap4-sec4.5}%sec
                                %4.5 

We can get, in particular,representations of the  space $\mathscr{M}$ of
measures on a group $G$ by considering regular representations 
of $G$. We define $\sigma_\mu \nu  = \int
\sigma_{x}(\nu)d\mu(x)$. $\mathscr{M}$ is the  dual of the barrelled
space $\mathscr{C}_{G}$ and is hence quasi complete. We therefore have 
\begin{align*}
\langle f,\sigma\mu(\nu)\rangle  & = \int \langle
f,\sigma_{x}(\nu)\rangle d\mu(x) \\ 
& = \int d\mu(x)\int f(y)d\nu(x^{-1}y)\\
& = \iint f(xy)d\mu(x) d \nu (y)\\
& =  \langle f,\mu * \nu \rangle 
\end{align*}

In other words, $\sigma_{\mu}(\nu) =\mu *\nu$  and $\tau_{\mu}(\nu) =
\nu *\check{\mu}$ 
where $d  \check{\mu} (x)  =  d\mu(x^{-1})$.  We have not imposed
any conditions on $\nu$, and $\mu$ has been assumed to have compact
support. Thus convolution of two measures could have been defined as
$\sigma_\mu (\nu)= \int \sigma_{x}(\nu)d \mu (x)$ straightaway.  

\subsection{}\label{partII-chap4-sec4.6}\pageoriginale

\begin{proposition}%pro 2
Let $E$ be a subspace of $\mathscr{M}$ with a finer topology such that 
\begin{enumerate}
\renewcommand{\theenumi}{\alph{enumi}}
\renewcommand{\labelenumi}{\rm(\theenumi)}
\item $E$ is invariant by $\tau$; 

\item  $\tau$ restricted to $E$ is continuous;

\item  $E$ is  quasi complete.
\end{enumerate}
\end{proposition}

Then  for every $\mu \in \mathscr{M}^{C}$ and  $a \in E$, we have $a* \mu
\in E$ and $a*\mu = \tau_{\check{\mu}} a$. If  moreover $\tau$ is
bonded on $E$, then this true for $\mu \in \mathscr{M}^{1}$.  

The proposition is immediate in view  of our remarks in
Ch. \ref{partII-chap4-sec4.5}. 

In particular, if $a\in L^{P}(p < \infty)$, $\mu \in \mathscr{M}^{1}$,
then $a * \mu \in L^{p}$, and 
$ || a*\mu ||_{p} \leq || a ||_{p} || \mu ||$. Again, we may take an
 integrable function $f$ instead  of $ \mu $ and get  $a* f\in L^{p}$
 and $|| a*f ||_{p}\leq   || a ||_{p}||f||_{1}$. Otherwise stated, $L^{1}$ is
 represented as an algebra of operators in the Banach space $L^{p}(p<
 \infty)$. Another case is that of $E =\mathscr{C}_{G}$. If  $f$ is a
 function and $\mu$ a measure both with compact supports, then $f*
 \mu \in \mathscr{C}_{G}$. 

\subsection{Process of regularisation.}\label{partII-chap4-sec4.7}%sec4.7

Let $V$ be any neighbourhood of $e$  in $G$. Let $A_{V}$  be  the set
$\big\{f \in \mathscr{C}_{v}: f \geq 0$ and $ \int f
(x)dx=1\big\}$. As $V$ describes the neighbourhood filter at $e$,
$A_{V}$ also describes a filter $\Phi$ in the function space
$\mathscr{C}_{G}$. 

\begin{proposition}\label{partII-chap4-prop3}%pro 3
 If $U$ is a continuous representation of $G$ in a quasi complete space
 $E$, then $U_{f}a \rightarrow a $ following $\Phi$  for every 
 $a \in  E$. 
\end{proposition} 
  
In fact $U_{f}a =  \int U_{x}af(x)dx$. If $W$ is a closed convex
neighbourhood of $0$ in $E$, then by the continuity of $U_{f}a$ one
can find a neighbourhood  $V$ of $e$ such that $ U_{x}a \in a + W $
for every $x \in V$. Now $U_{f}a -a =\int(U_{x}a-a)f(x)dx \in W$
whenever $f \in A_{V} $ by convexity of $W$. 

\begin{remark*}%rem
$U_{f}a$\pageoriginale has certain properties of continuity stronger than that of
  $a$. For instance if we take for $U$ the regular representation of
  $G$ in $\mathscr{M}$,  $\tau_f \mu \in \mathscr{E}^0$. When $\Phi \rightarrow
  e$, $\tau_{f} \mu \rightarrow \mu$. This is a process of approximation 
 of a measure, as it were by continuous  functions. If $G$  satisfies
 the first axiom of countability, we can find a sequence  $\{f_{n} \}$
 of continuous  functions generating the filter $\Phi$. In particular,
 if $G$ is a Lie group, we have thus an approximation 
 of measures by sequences of continuous  functions. Finally we  remark
 in passing that the same procedure  can be adopted in the case of Lie
 groups for distributions  instead of measures. Thus a distribution on
 a Lie  group  can be approximated by a sequence of indefinitely
 differentiable  functions. 
\end{remark*}
