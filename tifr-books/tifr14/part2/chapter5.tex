
\chapter{General  theory of representations}\label{partII-chap5}
 
\setcounter{section}{5}
\setcounter{subsection}{0}
\subsection{Equivalence of
  representations}\label{partII-chap5-sec5.1}\pageoriginale%sec5.1 

\begin{defi*}%def
A representation $U$ of  a topological group  $G$ in locally convex
space $E$ is said  to be {\em equivalent} to another representation
$U'$ in $E'$ if there  exists  an isomorphism $T$ of $E$  onto  $E'$
such  that $TU_{x} = U'_{x}T$ for every  $x \in G$. 
\end{defi*}

This  is evidently  a very strong requirement  which fails to
characterise as equivalent certain representations  which are
equivalent in the   intuitive  sense. However, we  are interested  in
the case  of unitary  representations  in Hilbert  spaces and the
definition is good enough for our purposes. 
 
\begin{defi*}%def
Two representations  $U$ in $H$, $U'$ in $H'$ are {\em unitarily
  equivalent} if there exists a unitary isomorphism   $T:H
\rightarrow H'$ such that 
$TU_{x}=U'_{x}T$  for  every $x \in  G$. 
\end{defi*}

\setcounter{proposition}{0}
\begin{proposition}\label{partII-chap5-prop1}%pro 1
 Two equivalent unitary representations are unitarily\break equivalent. In
 fact, $TT^*U'_{x}= TU_{x}T^*=U'_{x}TT^*$ i.e. $U_{x}$ commutes  with the
 positive Hermitian operator  $TT^*$ and hence also with
 $H=\sqrt{TT^*}$. It  can be easily seen that  $H^{-1}T$ is  a unitary
 operator which  transforms  $U$ into $U'$. 
\end{proposition} 

\subsection{Irreducibility  of  representations}\label{partII-chap5-sec5.2}%sec5.2 

\begin{defn}[algebraic  irreducibility]\label{partII-chap5-defi1} 
A representations  $U$ of  a  group
  $G$ in a vector  space  $E$  is  said to be  {\em algebraically
    irreducible}  if  there  exists  no proper  invariant  subspace
  of $E$. 
\end{defn}
 
\begin{defn}[topological  irreducibility]\label{partII-chap5-defi2} 
A representation  $U$ of a
   topological  group  $G$ in  a locally  convex  space $E$  is  said
   to be  {\em topologically  irreducible}  if there  exists  no
   proper {\em closed}  invariant  subspace. 
\end{defn}
 
 \begin{defn}[complete  irreducibility]\label{partII-chap5-defi3}%def 3
  A\pageoriginale representation  $U$ of  a topological
   group  $G$ in a locally convex space  $E$ is said to  be {\em
     completely irreducible}  if  any operator  in $\Hom (E,E)$ (with
   the topology  of simple convergence)  can be approximated by
   finite linear combinations  of the $U_{x}$. 
\end{defn}

It is at once obvious  that (i) $\Rightarrow$ (ii)  and that
(iii) $\Rightarrow$ (ii).  
It can be proved that when $E$ is a Banach space, (i) $\Rightarrow$
(iii) (Proof can be found in Annals  of
Mathematics, 1954, Godement). For unitary representations, (ii) and
(iii) are equivalent (due to von Neumann's density  theorem,
Th. \ref{partII-chap5-thm2}. Ch. \ref{partII-chap5-sec5.6}). Finally,
all the three   
  definitions   are  equivalent  for  finite dimensional
  representations ((ii) $\Rightarrow$  (iii) due to  Burnside's
  theorem, Th. \ref{partII-chap5-thm1}, Ch. \ref{partII-chap5-sec5.5}). 

\subsection{Direct  sum of  representations}\label{partII-chap5-sec5.3}%sec5.3 

\begin{defi*}%def 
 A representation $U$ of $G$ in $E$ is said to be the {\em direct sum}
 of representations $U_{i}$ of $G$  in $E_{i}$ if $E_i$ are invariant
 closed subspaces of $E$ such that the sum $\sum E_{i} $  is direct
 and is  everywhere  dense in $E$, and if $U_{i} $ is the restriction
 of $U$ to $E_{i}$. Moreover, if $U$ is a unitary  representation in
 Hilbert space, $U$ is said to be the  {\em Hilbertian direct sum} of
 the $U_{i}$ if  $E_{i}$ is  orthogonal  to  $E_{j}$  whenever $i \neq
 j$.  
\end{defi*}
 
\begin{defi*} %def
A  representation is {\em completely reducible}  if it can be expressed
as a direct  sum of irreducible  representations. 
\end{defi*}
 
\subsection{Schur's lemma*}\label{partII-chap5-sec5.4}%sec 5.4
   
We give here two formulations (Prop. \ref{partII-chap5-prop2}
and \ref{partII-chap5-prop3}) of Schur's lemma, the
first being trivial and  the second more suited to our purposes. 
 
 \begin{proposition}\label{partII-chap5-prop2}%pro 2
 Let $U$ and $U'$ be two algebraically irreducible representations in
 $E$, $E'$ respectively. If $T$ is a linear map: $E \rightarrow  E'$
 such that $T U_{x}= U'_{x}T$\pageoriginale 
 for every $x \in  G$, then either $T=0$
 or an algebraic isomorphism. 
\end{proposition}  
      
From this, we immediately deduce the following

\begin{coro*} %coro     
Let $U$ be an algebraically irreducible finite - dimensional
representation of  a group  $G$ in $E$. The only endomorphisms  of $E$
which commute with  all the $U_{x}$  are scalar multiples of the
identity. 
\end{coro*}
 
In fact, if $\lambda$ is  an eigenvalue  of $T$, $T-\lambda I$  is not
an isomorphism and is, by Schur's lemma, $=0$. 
 
\begin{proposition}\label{partII-chap5-prop3}%pro 3
Let $U$, $U'$ be two  unitary topologically irreducible representations in
$H$, $H'$ respectively. If $T$ is a continuous linear map $H
\rightarrow H'$ such that $TU_{x} = U'_{x}T$ for every $x \in G$,
then either $T=0$ or an isomorphism of Hilbert spaces. 
\end{proposition}

In fact, $T^*$ is a continuous  operator  with $UT^*=T^*U'$. $H=T^*T$  is a
Hermitian  operator commuting with every $U_{x}$. Hence $U_{x}$
commutes with every $E_{\lambda}$ in the spectral decomposition  $H=
\int \lambda dE_{\lambda}$ and consequently leaves every spectral  
 subspace invariant. Therefore, the spectral sub-spaces reduce to
 $\{0\}$ or $E$. i.e.  $H$ is a scalar $=\lambda I$.  Similarly $H' =
 TT^*= \lambda I$.  Hence $T$ is either $0$ or an isometry  up to
 a constant. 
 
The proof  of Prop. \ref{partII-chap5-prop3} implicitly contains the
following 
 
\begin{coro*}%coro
 Let $U$ be a unitary topologically irreducible representation  of a
 group  $G$ in a Hilbert space $E$. The only operators of $E$ which
 commute with all the $U_{x} $  are scalar multiples of the
 identity. This is immediate since any operator can be expressed  as a
 sum of Hermitian operators  for which the corollary  has been proved
 in prop. \ref{partII-chap5-prop3}.  
\end{coro*} 
 
\subsection{Burnside's theorem}\label{partII-chap5-sec5.5}%sec5.5 

\setcounter{thm}{0}
\begin{thm}[Burnside]\label{partII-chap5-thm1}%the 1
Let $U$ be an algebraically irreducible representation
of\pageoriginale $G$ into $E$ 
of  finite  dimension. Then every operator in $E$ is a linear
combination of the $U_{x}$. 
\end{thm}

Consider the algebra $\mathcal{A}$ of  finite linear combinations of
$U_{x}$. Let $\mathscr{B}$ be the subset of $\Hom(E,E)$ consisting  of
elements   $B$ such that $\Tr(AB)=0$  for every $A \in
\mathcal{A}$. Obviously it is enough  to  show that
$\mathscr{B}=\{0\}$.  Now  we  define a representation  $V$ of $G$ in
$\mathscr{B}$ by defining $V_{x} (B)  =  U_{x}\circ B$  for every $B \in
\mathscr{B}$. This  is not in general an irreducible
representation. However, if $ \mathscr{B} \neq \{0\}$, we can find  a
non-zero irreducible subspace $\mathscr{C}$ of $\mathscr{B}$. Now the
map $ \lambda_{a}: B \rightarrow  B _{a}$  of ${\mathscr{C}}$  into
$E$ is a linear map which transforms  the representation $V$ into
$U$. For, 
$$
\lambda _{a} \circ V_{x}(B) =  \lambda_{a}\circ U_{x}\circ B  =U_{x}
\circ B a  = U_{x}\circ
\lambda B \text{~ for every~ } B \in \mathscr{C}. 
$$

Hence by Schur's lemma  (prop. \ref{partII-chap5-prop2},
Ch. \ref{partII-chap5-sec5.4}), $\lambda_{a} =0$ or 
is an isomorphism. If $\lambda_{a}=0$ for every $a \in E$, $B=0$ for
every $B \in \mathscr{C}$ and  hence $\mathscr{C} = \{0\}$. But this
is contradictory to our assumption that $\mathscr{C}$ is
non-zero. Therefore, there exists $a_{1} \in E $ such that $\lambda
_{a_1}  \neq 0$. So, $\lambda _{a_1}$ is an isomorphism of
$\mathscr{C}$ onto $E$. Let $(a_{1},a_{2}\ldots)$ be a basis  of
$E$. Then $\lambda^{-1}_{a_1}  \circ \lambda_{a_2}$ is an operator on
$\mathscr{C}$. This obviously commutes with  every $V_{x}$. Hence  by
cor. to prop. \ref{partII-chap5-prop2}, Ch. \ref{partII-chap5-sec5.4}
$\lambda^{-1}_{a_1}\circ \lambda_{a_2}= 
\mu_{2} I$ where $\mu_2$ is a scalar. We shall thus write
$\lambda_{a_j} = \mu_{j} \lambda_{a_1}$ with $\mu_{1} =1$. Now,  one
can introduce  a scalar  product in $E$ such that $\Tr(U_{x}B)$ =
$\sum\limits_{j} \langle U_{x}Ba_{j},a_{j}\rangle=0$ for every $x \in
E$. But 
\begin{align*}  
\sum\limits_{j}\langle U_{x} Ba_{j}, a_{j} \rangle & = \sum\limits_{j} \mu_{j} \langle U_{x}Ba_{1}, a_{j}\rangle \\ 
  & = \langle U_{x}Ba_{1}, \sum\limits_{j} \mu_{j} a_{j} \rangle  =  0 
 \end{align*} 

Since the $ U_{x}Ba_{1}$ generate $E$, $\sum\limits_{j} \mu_{j} a_{j}
= 0$ or again $\mu_{j} = 0 $ for every $j$. 
But\pageoriginale 
$\mu_1 = 1$. This gives us a contradiction and it follows that $B=
\{0\}$. 

\subsection{Density theorem of von
  Neumann}\label{partII-chap5-sec5.6}%sec 5.6 

Let $U$ be a unitary representation of a topological group $G$ in a
Hilbert space $H$. We denote by $\mathcal{A}$.the subset of the set
$\Hom(H,H)$ of operators on $H$ consisting of finite linear
combinations of the $U_x$. This is a self adjoint subalgebra of
$\Hom(H,H)$, but is not in general closed in it. Let $\mathcal{A}'$ be
the 
set of operators which commute with every element of
$\mathcal{A}$. Obviously $\mathcal{A}'$ is also self-adjoint. 

\begin{proposition}\label{partII-chap5-prop4}%pro
 $\mathcal{A}'$ is {\em weakly} closed in $\Hom(H,H)$.
\end{proposition}

In fact, if $A = \lim A_i$ (in the weak topology) with
$A_i\in\mathcal{A}'$, then  
\begin{align*}
\langle BAx,y \rangle &= \lim  \langle BA_ix,y \rangle = \lim  \langle
A_iBx,y \rangle\\ 
  & =  \langle ABx,y \rangle \text{~ for every~ } B \in \mathcal{A}.
\end{align*}
 
Hence $A \in \mathcal{A}'$.

Let $\mathcal{A}''$ be the commutator of the algebra
$\mathcal{A}'$. Then $\mathcal{A}''$ is a weakly closed self-adjoint
subalgebra containing $\mathcal{A}$ and hence it contains the weak
closure of $\mathcal{A}$. We can in fact assert 

\begin{thm}[von Neumann]\label{partII-chap5-thm2}%the 2
$\mathcal{A}''=$ weak closure of $\mathcal{A}$.
\end{thm}
 
We actually prove a stronger assertion, viz. Let $(x_n)$ be a sequence
of elements in $E$ such that $\sum|| x_n||^{2}< \infty$ and $T$ an
operator in $\mathcal{A}''$. Then for every $\epsilon > 0$, there
exists $A \in\mathcal{A}$ such that $\sum|| Tx_n- Ax_n ||^{2} <
\epsilon$. (This in particular implies that $\mathcal{A}'' =$ strong
closure of $\mathcal{A}$ or even the strongest closure of $\mathcal{A}$, in the
sense of von Neumann).  
 
We first show that for every $x \in E$, $Tx$ is in the closure of
$\{Ax: A \in \mathcal{A}\}$.\pageoriginale In fact, $F =
\overline{\{Ax\}}$ is a closed invariant subspace of $E$ and let
$F^{\perp}$  be its 
orthogonal complement. $F^{\perp}$  is also invariant under the self
adjoint algebra $\mathcal{A}$ and hence the orthogonal projection $P$
of $E$ onto $F$ commutes with every element of $\mathcal{A}$. $P$
therefore belongs to $\mathcal{A}'$ and $T$ leaves $F$
invariant. Since $x \in F$, $Tx$ is also in $F$. 

Now consider the space $E_1 = E \oplus E  \oplus \cdots
\cdots$ (Hilbertian sum). Every element $x \in E_1$ is of the form
$(x_1,\ldots x_n \cdots)$ with $\sum|| x_n||^{2} < \infty$. Let
$\tilde{A}$ be the operator on $E_1$ defined by $Ax = (Ax_1,  \ldots
Ax_n,  \ldots)$. The map $A \rightarrow\tilde{A}$ is an isomorphism
of $\Hom(H,H)$ into $\Hom(H_1,H_1)$. Denote the image of
$\mathcal{A}$ by $\tilde{\mathcal{A}}$. If $B$ is any operator on
$E_1$, it can be expressed in the form $B(x_1, \ldots x_n \ldots )
= (y_1, \ldots y_n \ldots)$ where $y_n = \sum\limits_{p}
b_{n,p} x_p$ and $b_{n,p}$ is an operator on $E$. We now show that
$B \in (\tilde{\mathcal{A}})'$ if and only if $b_{n,p} \in
\mathcal{A}'$ for every $n$ and $p$. For, $B\tilde{A} (x_1,\ldots
x_n, \ldots) = \tilde{A}B (x_1, \ldots x_n, \ldots)$ for every
$x \in E_1$ implies that $b_{n,p}  A = Ab_{n,p}$ by taking all
$x_n$, $n \neq p$ to be zero. Conversely, if this is
satisfied, $\tilde{A}B (x_1,.. x_n ..)$ 
\begin{align*}
 & = \tilde{A}(\ldots, \sum\limits_{p} b_{n,p} x_p, \ldots) =
  (\ldots, \sum\limits_{p} b_{n,p}  Ax_p, \ldots)\\ 
 & = B( Ax_1, \ldots Ax_n, \ldots) = B \tilde{A}(x_{1}, \ldots
  x_n, \ldots) 
 \end{align*}

Again, $B \in (\tilde{A})''$ if and only if the diagonal elements of
the infinite matrix $b_{n,p}$ are equal and in $\mathcal{A}''$, and the
rest of the elements are zero. In fact, if $ C \in (\tilde{A})''$, we
have $(BC)_{m,n} = (CB)_{m,n}$ for every $B \in \mathcal{A}'$, or
$\sum\limits_{p} b_{m,p}  c_{p,n} = \sum\limits_{q}  c_{m,q}
b_{q,n}$  for every $b_{i,j} \in \mathcal{A}'$. Putting $b_{i,j} =
\delta_{n,i} \delta_{n,j}$, we get $c_{i,j} = 0$ if $i \neq j$ and
$c_{m,m} = c_{n,n}$ for every\pageoriginale $m$ and $n$. 
So we have $(\tilde{A})'' =
(\widetilde{A''})$. Therefore there exists $A \in \mathcal{A}$ such that
$\sum|| Tx_n - Ax_n||^{2} < \epsilon$.  
  
Moreover, in theorem \ref{partII-chap5-thm2}, if $T$ is Hermitian we
can find a \textit{Hermitian} operator $A$ such that $\sum|| Tx_n - Ax_n||^{2}<
\epsilon$. As before it is enough to prove this for one vector $x$. In
other words, we have to show the existence of Hermitian operator $A$
such that $|| Tx - Ax || < \epsilon$. We know that $T$ is the strong limit
of $A \in \mathcal{A}$. Hence $T= T^*$ is the \textit{weak} (and not
strong, in general) limit of $A^*$ or again the weak limit of
$\dfrac{A + A^*}{2}$. Now $\dfrac{A+A^*}{2}$ is a Hermitian operator
in $\mathcal{A}$. Hence $T$ is weakly adherent to this convex set. In
this case, weak adherence is the same as the weak adherence in the
sense of topological vector spaces, but weak and strong topologies are
the same in a convex space. 
 
 
In particular, if we have a unitary topologically irreducible
representation, by Schur's lemma (Prop. \ref{partII-chap5-prop3},
Ch. \ref{partII-chap5-sec5.4}) $\mathcal{A}' 
= \{\lambda I \}$ and hence $\mathcal{A}''  = \Hom(E,E)$. Therefore
every operator is strongly adherent to $\mathcal{A}$. This is the
analogue of Burnside's theorem (Th. \ref{partII-chap5-thm1},
Ch. \ref{partII-chap5-sec5.5}) for unitary 
representations.
