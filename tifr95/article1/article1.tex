\title{Fundamental Group of the Affine Line in Positive Characteristic}
\markright{Fundamental Group of the Affine Line in Positive Characteristic}

\author{By~ Shreeram S. Abhyankar\footnote{Invited Lecture delivered on 8 January 1992 at the International Colloquium on Geometry and Analysis in TIFR in Bombay. This work was partly supported by NSF Grant DMS-91-01424}}
\markboth{Shreeram S. Abhyankar}{Fundamental Group of the Affine Line in Positive Characteristic}

\date{}
\maketitle

\section{Introduction}\pageoriginale

I have known both Narasimhana and Seshadri since 1958 when I had a nice meal with them at the Student Cafeteria in Cit\'e Universitaire in Paris. So I am very pleased to be here to wish them a Happy Sixtieth Birthday. My association with the Tata Institute gore back even further to 1949-1951 when, as a college student, I used to attend the lectures of M. H. Stone and K. Chandraseksharan, first in Pedder Road and then at the Yacht Club. Then in the last many years I have visited the Tata Institute numerous times. So this Conference is a nostalgic homecoming to me.     

To enter into the subject of Fundamental Groups, let me, as usual, make a.

\section{High- School Beginning}

So consider a polynomial
$$
f=f(Y) = Y^{n}+a_{1}Y^{n-1}+\cdots+a_{n}
$$
with coefficients $a_{1}, \ldots,a_{n}$ in some field $K$; for example, $K$ could be the field of rational numbers. We want to solve the equation $f=0$, i.e., we want to find the roots of $f$. Assume that $f$ is irreducible and has no multiple roots. Suppose somehow we found a root $y_{1} $ of $f$. Then to make the problem of finding the other roots easier, we achieve a decrease in degree by ``throwing away" the root $y_{1}$ to get
$$
f_{1}=f_{1}(Y)=\dfrac{f(Y)}{Y-y_{1}}=Y^{n-1}+b_{1}Y^{n-2}+\cdots+b_{n-1}.
$$
If $f_{1}(Y)$ is also irreducible and if somehow we found a root $y_{2}$ of $f_{1}$, then  ``throwing away" $y_{2}$ we get
$$
f_{2} = f_{2}Y = \dfrac{f_{1}(Y)}{Y-y_{2}}=Y^{n-2}+c_{1}Y^{n-3}+\cdots+c_{n-2}.
$$
Note that the coefficients $b_{1}\ldots, b_{n-1}$ of $f_{1}$ do involve $y_{1}$ and hence they are note in $K$, but they are in $K(y_{1})$. So, although we assumed $f$ to be irreducible in $K[Y]$, when we said ``if  $f_{1}$ is irreducible," we clearly meant ``if $f_{1}$ is irreducible in $K(y_{1})[Y]"$. Likewise, irreducibility of $f_{2}$ refers to its irreducibility in $K(y_{1}, y_{2})[Y]$. And so on. In this way we get a sequence of polynomials $f_{1}, f_{2}, \ldots, f_{m}$ of degrees $n-1, n-2, \ldots, n-m$ in $Y$ with coefficients in $K(y_1), K(y_{1}, y_{2}), \ldots, K(y_{1},\ldots,y_{m})$, where $f_{i}$ is irreducible in $K(y_{1}, \ldots, y_{i})[Y]$ for $i=1, 2, \ldots, m-1$. If $f_{m}$ is reducible in $K(y_(1), \ldots, y_{m})[Y]$ then we stop, otherwise we proceed to get $f_{m+1}$, and so on. Now we may ask the following. 

\noindent
{\bf Question:} Given any positive integers $m<n$, doers there exist an irreducible polynomial $f$ of degree $n$ in $Y$ with coeffcients in some field $K$ such that the above sequence terminates exactly after $m$ steps, i.e., such that  $f_{1}, f_{2}, \ldots, f_{m-1}$ are irreducible but $f_{m}$ is reducible?  

I presume that most of us, when asked to respond quickly, might say: ``Yes, but foe large $m$ and $n$ it would be time consuming to write down concrete examples". However, the {\bf SURPRISE OF THE CENTURY}  is that the {\bf ANSWER} is {\bf NO}. More precisely, it turns out that

\subsection{$f_{1}, f_{2}, f_{3}, f_{4}, f_{5}$ irreducible $\Rightarrow$
 $f_{6}, f_{7},\ldots, f_{n-3}$ irreducible.}\label{art1-sec2.1}

In other words, if $f_{1}, \ldots, f_{5}$ are irreducible then $f_{1}, \ldots, f_{n-1}$ are all irreducible except the $f_{n-2}$, which is a quadratic, may or may not be irreducible. This answers the case of $m \geq 6$. Going down the line  to $m \leq 5$ and assuming $m<n-2$, for the case of $m=5$ we have that

\subsection{$f_{1}, f_{2}, f_{3}, f_{4}$ irreducible but $f_{5}$ reducible $\Rightarrow$ $n=24$ or $12$ and for the case of $m=4$ we have that}\label{art1-sec2.2}

\subsection{$f_{1}, f_{2}, f_{3}$ irreducible but $f_{4}$ reducible $\Rightarrow$ $n=23$ or $11$.}\label{art1-sec2.3}

Going further down the line, for the case of $m=3$ we have that

\subsection{$f_{1}, f_{2}$ irreducible but $f_{3}$ reducible $\Rightarrow$ Refined FT of Proj Geom}\label{art1-sec2.4}

i.e., if $f_{1}, f_{2}$ are irreducible but $f_{3}$ is reducible,then there are only a few possibilities and they are suggested by the Fundamental Theorem  of  Projective Geometry, which briefly says that ``the underlying division ring of a synthetically defined desarguestion projective plane is a field in and only if any three point of a projective line can be mapped to any other three points of that projective line by a unique projectivity." Going still further down the line for the case of $m=2$ we have that   

\subsection{$f_{1}$ irreducible but $f_{2}$ reducible $\Rightarrow$ known but too long}\label{art1-sec2.5}

i.e., if $f_{1}$ is irreducible but $f_{2}$ is reducible, the answer is known but the list of possibilities is too long to write down here. Finally, for the case of $m=1$ we have that

\subsection{$f_{1}$ has exactly two irreducible factors $\Rightarrow$ Pathol proj Geom $+$ Stat}\label{art1-sec2.6}

i.e., if $f_{1}$ has exactly two irreducible factors, then again a complete answer is known, which depends on Pathological Projective Geometry and Block Designs from Statistics! Here I am reminded of the beautiful course on Projective Geometry which I took from Zariski (in 1951 at Harvard), and in which I learnt the Fundamental Theorem mentioned in (2.4). At the end of that course, Zariski said to me that ``Projective geometry is a beautiful dead subject, so don't try to do research in it" by which he implied that the ongoing research in tha subject at that time was rather pathological and dealt with non- desarguesian planes and such. But in the intervening thirty or forty years, this ``pathological" has made great strides in the hands of pioneers from R. C. Bose \cite{art1-key21} and S. S. Shrikhande \cite{art1-key58} to P. Dembowski \cite{art1-key29} and D. G. Higman \cite{art1-key34}, and has led to a complete classification of Rank 3 groups, which from our view-point of the theory of equations is synonymous to case \ref{art1-sec2.6}. So realizing how even a great man like Zariski could be wrong occasinally, I have learnt to drop one of my numerous prejudices, namely my prejudice against Statistics.  

Note that a permutation group is said to be {\it transitive} if any point (of the permuted set) can be sent to any other, via a permutation in the group. Likewise, a permutation group is {\it m-fold transitive} (briefly: {\it m-fold transitive}) if any $m$ points can be sent to any other $m$ points, via a permutation in that group. $Doubly Transitive = 2-transitive$, $Triply Transitive = 3-transitive$, and so on. By the {\it one point stabilizer} of a transitive permutation group we mean the subgroup consisting of those permutations which keep a certain point fixed; the {\it orbits} of that subgroup are the minimal subsets of the permuted set which are mapped to themselves by every permutation in that subgroup; of the nontrivial orbits are called the {\it sub degrees} of the group, so that the numbers of sub degress is one less than than rank. Thus a {\it Rank 3} group is transitive permutation group whose one point stabilizer has three orbits; the lengths of the two nontrivial orbits are the sub degrees. Needless to say that a Rank 2 group is nothing but a Doubly Transitive permutation group. At any rate, in case \ref{art1-sec2.6}, the degrees of the two irreducible factors of $f_{1}$ correspond to the sub degrees of the relevant Rank 3 group. Now CR3(= the Classification Theorem of Rank 3 groups) implies that very few pairs of integres can be the sub degrees  of Rank 3 groups, very few nonisomorphic Rank groups can have the same sub degrees; see Kantor-Liebler \cite{art1-key42} and Liebeck \cite{art1-key44}. Hence \ref{art1-sec2.6} says that if $f_{1}$ has exactly two irreducible factors then their degrees (and hence also $n$) can have only certain very selective values.

Here, by the relevant group we mean the {\it Galois group} of $f$ over $K$, which we donate by Gal$(f, K)$ and which, following Galois, we define as the group of those permutations of the roots $y_{1}\ldots, y_{n}$ which retain all the polynomial relations between them wiht coefficients in $K$. This definition makes sense without $f$ being irreducible but still assuming $f$ to have no multiple roots. Now our assumptio of $f$ being irreducible is equivalent to assuming that Gal$(f, K)$ is transitive. Likewise, $f_{1}, \ldots, f_{m-1}$ are irreducible iff Gal$(f, K)$ in $m$-transitive. Moreover, as already indicated, $f_{1}$ has exactly two irreducible factors iff Gal$(f, K)$ has rank 3. To match this definition of Galois with the modern definition, let $L$ be the {\it splitting field} of $f$ over $K$, i.e., $L=K(y_{1}\ldots, y_{n})$. Then according to the modern definition, the {\it Galois group} of $L$ over $k$, denoted by Gal$(L, K)$, is defined to be the group of all automorphisms of $L$ which keep $K$  point wise fixed. Considering Gal$(f, K)$ as a permutations group of the subscripts $1,\ldots, n$ of $y_{1}, \ldots, y_{n}$ for every $\tau \in {\rm Gal}(L, K)$ we have a unique $\sigma \in {\rm Gal}(f, K)$ such that $\tau(y_{i}) = y_{\sigma(i)}$ for $1 \leq i \leq n$. Mow we get an isomorphism of Gal$(L, K)$ onto Gal$(f, K)$ by sending each $\tau$ to the corresponding $\sigma$.

Having sufficiently discussed case\ref{art1-sec2.6}, let us note that \ref{art1-sec2.5} is equivalent to CDT (=Classification Theorem of Doubly Transitive permutation groups) fr which we manu refer to Cameron \cite{art1-key22} and Kantor \cite{art1-key41}. At any rate, CDT implies that if $f_{1}$ is irreducible but $f_{2}$ is reducible then we must have: either $n=q$ for some prime power $q$, or $n=(q^{l}-1)/(q-1)$ for some integer $l>1$ and some prime power $q$, or $n=2^{2l-1}-2^{l-1}$ for some integer $l>2$, or $n=15$, or $n=176$, or $n=276$. Likewise \ref{art1-sec2.4} is equivalent to CTT(= Classification of Triply Transitive permutation groups) which is subsumed in CDT, and as a consequence of it we can say that if $f_{1}, f_{2}$ are irreducible but $f_{3}$ is reducible then we must have: either $n=2^{l}$ for some positive integer $l$, or $n-q+1$ for some prime powder $q$, or $n=22$.
       
similarly, \ref{art1-sec2.3} is equivalent to CQT (= Classification of Quadruply Transitive permutation groups) which is subsumed in CTT, and as a consequence of it we can  say that if $f_{1}, f_{2}, f_{3}$ are irreducible but $f_{4}$ is reducible then we must have: eithee $n=23$ and Gal$(f, K)=M_{23}$ or $n=11$ and Gal$(f, K)=M_{11}$, where  $M$ stands for Mathieu. Likewise, \ref{art1-sec2.2} is equivalent to CFT(= Classification of Fivefold Transitive permutation groups) which is subsumed in CQT, and as a consequence of it we can say that if $f_{1},f_{2}, f_{3}, f_{4}$ are irreducible but $f_{5}$ is reducible then we must have: either $n=24$ and Gal$(f, K)=M_{24}$ and Gal$(f, K)=M_{12}$.  

Note that, $M_{24}$ and $M_{24}$ are the only 5-told but not 6-fold transitive permutation groups other than the {\it symmetric group} $S_{5}$ (i.e., the group of all permutations on 5 letters) and the {\it alternating group} $A_{7}$ (i.e., the  sub-group of $s_{7}$ consisting of all {\it even} permutations). Moreover, $M_{23}$ and $M_{11}$ are the respective one point stabilizers of $M_{24}$ and $M_{12}$ and they are the only 4-fold but not 5-fold transitive permutation groups other than $s_{4}$ and $A_{6}$. Here the subscript denotes the degree, i.e., the number of letters being permuted. The four groups $M_{24}, M_{23}, M_{12}, M_{11}$, were constructed by Mathieu \cite{art1-key46} in 1861 as examples of highly transitive permutation groups. But the fact that they are the only 4-transitive permutation groups others than the symmetric groups and the alternating groups, was proved only in 1981 when CDT, and hence also CTT, CQT, CFT and CST, were deduced from  CT(= Classification Theorem of finite simple groups); see Cameron \cite{art1-key23} and Cameron-Cannon \cite{art1-key24}. Recall that a group is {\it simple} if it has no nonidentity normal subgroup other than itself; it turns out that the five {\it Mathieu groups} $M_{24}, M_{23}, M_{22}, M_{12}, M_{11}$ and $M_{22}$ is the point stabilizer of $M_{23}$, are all simple. Now CST refers to the Classification Theorem of Sixfold Transitive permutation groups, according to which the symmetric groups and the alternating groups are the only 6-transitive permutation groups; note that $S_{m}$ is $m$-transitive but not $(m+1)$-transitive, whereas $A_{m}$ is $(m-2)$-transitive but not $(m-1)$-transitive for $m\geq 3$. In \ref{art1-sec2.2} to \ref{art1-sec2.6} we had assumed $M<n-2$ to avoid including the symmetric and alternating groups; dropping this assumption, \ref{art1-sec2.1} is equivalent to CST with the clarification that, under the assumption of \ref{art1-sec2.1}, the quadratic $f_{n-2}$ is irreducible or reducible according as Gal$(f, K) = s_{n}$ or $A_{n}$.

We have already hinted that CR3 was also deduced as a consequence of CT; Liebeck \cite{art1-key44}. The proof of CT itself was completed in 1980 (see Gorenstein \cite{art1-key32}) with {\it staggering statistics:} 30 years; 100 authors; 500 papers; 15,000 pages! Add some more pages for CDT and CR3 and so on.

All we have done above is to translate this group theory into te language of theory of equations where $K$ is ANY field. So are still talking High-School? Not really, unless we admint CT into High-School!
  
\section{Galois} 

Summarizing, to compute the Galois group Gal$(f, K)$, say when $K$ is the field $k(X)$ of univariate rational functions over an algebraically closed ground field $k$, by throwing away roots and using some algebraic geometry we find some multi-transitivity and other properties of the Galois groups and fedd there into the group theory machine. Out comes a list of possible groups. Reverting to algebraic geometry, sometimes augmented by High-School manipulations, we successively eliminate various members from that list until, hopefully, one is-left. That then is the answer. I say hopefully because we would have a contradiction in which the ultimate reality {\bf (Brahman)} is described by {\it Neti Neti}, not this, not that. If you practice pure {\it Advaita}, then nothing is left, which is too austere. So we fall back on the kinder {\it Dvaita} according to which the unique {\bf God} remains. 

\section{Riemann and Dedekind}

In case $K=\bC(X)$ and $a_{i}=a_{i}(X)\in \bC[X]$ for $1\leq i \leq n$, where {\bf C} is the field  of comples numbers, following Riemann \cite{art1-key53} we can consider the {\it monodromy group} of $f$ thus.

Fix a non discriminant point $\mu$, i.e., value $mu \in \bC$ of $X$ for which the equation $f=0$ has $n$ distinct roots. Then, say by the Implicit Function Theorem, we can solve the equation the equation $f=0$ near $\mu$, getting $n$ analytic solutions $\eta _{1}(X), \ldots, \eta _{n}(X)$ near $\mu$. To find out how there solutions are intertwined, mark a finite number of values $\alpha_{1}, \ldots, \alpha_{w}$ of $X$ which are different from $\mu$ but include all the discriminant points, and let $\bC_{w}$ be the complex $X$-plane minus these $w$ points. Now by making analytic continuations along any closed path $\Gamma$ in $\bC_{w}$ starting and ending at $\mu$ so that $\eta_{i}$ continues into $\eta_{j}$ with $\Gamma'(i)= j$ for $1 \leq i \leq n$. As $\Gamma$ varies over all closed paths in $\bC_{w}$ starting and ending at $\mu$, the permutations $\Gamma'$ span a subgroup of $S_{n}$ called the {\it monodromy group} of  $f$ which we denote bye $M(f)$.   
 
By indentifying the analytic solutions $\eta_{1}, \ldots, \eta_{n}$ with the algebraic roots $y_{1}, \ldots, y_{n}$, the monodromy group $M(f)$ gets identified with the Galois group Gal$(f, \bC(X))$, and so these two groups are certainly isomorphic as permutation groups.
