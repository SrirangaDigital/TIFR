\title{Fundamental Group of the Affine Line in Positive Characteristic}
\markright{Fundamental Group of the Affine Line in Positive Characteristic}

\author{By~ Shreeram S. Abhyankar\footnote{Invited Lecture delivered on 8 January 1992 at the International Colloquium on Geometry and Analysis in TIFR in Bombay. This work was partly supported by NSF Grant DMS-91-01424}}
\markboth{Shreeram S. Abhyankar}{Fundamental Group of the Affine Line in Positive Characteristic}

\date{}
\maketitle

\section{Introduction}\pageoriginale

I have known both Narasimhana and Seshadri since 1958 when I had a nice meal with them at the Student Cafeteria in Cit\'e Universitaire in Paris. So I am very pleased to be here to wish them a Happy Sixtieth Birthday. My association with the Tata Institute gore back even further to 1949-1951 when, as a college student, I used to attend the lectures of M. H. Stone and K. Chandraseksharan, first in Pedder Road and then at the Yacht Club. Then in the last many years I have visited the Tata Institute numerous times. So this Conference is a nostalgic homecoming to me.     

To enter into the subject of Fundamental Groups, let me, as usual, make a.

\section{High- School Beginning}

So consider a polynomial
$$
f=f(Y) = Y^{n}+a_{1}Y^{n-1}+\cdots+a_{n}
$$
with coefficients $a_{1}, \ldots,a_{n}$ in some field $K$; for example, $K$ could be the field of rational numbers. We want to solve the equation $f=0$, i.e., we want to find the roots of $f$. Assume that $f$ is irreducible and has no multiple roots. Suppose somehow we found a root $y_{1} $ of $f$. Then to make the problem of finding the other roots easier, we achieve a decrease in degree by ``throwing away" the root $y_{1}$ to get
$$
f_{1}=f_{1}(Y)=\dfrac{f(Y)}{Y-y_{1}}=Y^{n-1}+b_{1}Y^{n-2}+\cdots+b_{n-1}.
$$
If $f_{1}(Y)$ is also irreducible and if somehow we found a root $y_{2}$ of $f_{1}$, then  ``throwing away" $y_{2}$ we get
$$
f_{2} = f_{2}Y = \dfrac{f_{1}(Y)}{Y-y_{2}}=Y^{n-2}+c_{1}Y^{n-3}+\cdots+c_{n-2}.
$$
Note that the coefficients $b_{1}\ldots, b_{n-1}$ of $f_{1}$ do involve $y_{1}$ and hence they are note in $K$, but they are in $K(y_{1})$. So, although we assumed $f$ to be irreducible in $K[Y]$, when we said ``if  $f_{1}$ is irreducible," we clearly meant ``if $f_{1}$ is irreducible in $K(y_{1})[Y]"$. Likewise, irreducibility of $f_{2}$ refers to its irreducibility in $K(y_{1}, y_{2})[Y]$. And so on. In this way we get a sequence of polynomials $f_{1}, f_{2}, \ldots, f_{m}$ of degrees $n-1, n-2, \ldots, n-m$ in $Y$ with coefficients in $K(y_1), K(y_{1}, y_{2}), \ldots, K(y_{1},\ldots,y_{m})$, where $f_{i}$ is irreducible in $K(y_{1}, \ldots, y_{i})[Y]$ for $i=1, 2, \ldots, m-1$. If $f_{m}$ is reducible in $K(y_(1), \ldots, y_{m})[Y]$ then we stop, otherwise we proceed to get $f_{m+1}$, and so on. Now we may ask the following. 

\noindent
{\bf Question:} Given any positive integers $m<n$, doers there exist an irreducible polynomial $f$ of degree $n$ in $Y$ with coeffcients in some field $K$ such that the above sequence terminates exactly after $m$ steps, i.e., such that  $f_{1}, f_{2}, \ldots, f_{m-1}$ are irreducible but $f_{m}$ is reducible?  

I presume that most of us, when asked to respond quickly, might say: ``Yes, but foe large $m$ and $n$ it would be time consuming to write down concrete examples". However, the {\bf SURPRISE OF THE CENTURY}  is that the {\bf ANSWER} is {\bf NO}. More precisely, it turns out that

\subsection{$f_{1}, f_{2}, f_{3}, f_{4}, f_{5}$ irreducible $\Rightarrow$
 $f_{6}, f_{7},\ldots, f_{n-3}$ irreducible.}\label{art1-sec2.1}

In other words, if $f_{1}, \ldots, f_{5}$ are irreducible then $f_{1}, \ldots, f_{n-1}$ are all irreducible except the $f_{n-2}$, which is a quadratic, may or may not be irreducible. This answers the case of $m \geq 6$. Going down the line  to $m \leq 5$ and assuming $m<n-2$, for the case of $m=5$ we have that

\subsection{$f_{1}, f_{2}, f_{3}, f_{4}$ irreducible but $f_{5}$ reducible $\Rightarrow$ $n=24$ or $12$ and for the case of $m=4$ we have that}\label{art1-sec2.2}

\subsection{$f_{1}, f_{2}, f_{3}$ irreducible but $f_{4}$ reducible $\Rightarrow$ $n=23$ or $11$.}\label{art1-sec2.3}

Going further down the line, for the case of $m=3$ we have that

\subsection{$f_{1}, f_{2}$ irreducible but $f_{3}$ reducible $\Rightarrow$ Refined FT of Proj Geom}\label{art1-sec2.4}

i.e., if $f_{1}, f_{2}$ are irreducible but $f_{3}$ is reducible,then there are only a few possibilities and they are suggested by the Fundamental Theorem  of  Projective Geometry, which briefly says that ``the underlying division ring of a synthetically defined desarguestion projective plane is a field in and only if any three point of a projective line can be mapped to any other three points of that projective line by a unique projectivity." Going still further down the line for the case of $m=2$ we have that   

\subsection{$f_{1}$ irreducible but $f_{2}$ reducible $\Rightarrow$ known but too long}\label{art1-sec2.5}

i.e., if $f_{1}$ is irreducible but $f_{2}$ is reducible, the answer is known but the list of possibilities is too long to write down here. Finally, for the case of $m=1$ we have that

\subsection{$f_{1}$ has exactly two irreducible factors $\Rightarrow$ Pathol proj Geom $+$ Stat}\label{art1-sec2.6}

i.e., if $f_{1}$ has exactly two irreducible factors, then again a complete answer is known, which depends on Pathological Projective Geometry and Block Designs from Statistics! Here I am reminded of the beautiful course on Projective Geometry which I took from Zariski (in 1951 at Harvard), and in which I learnt the Fundamental Theorem mentioned in (2.4). At the end of that course, Zariski said to me that ``Projective geometry is a beautiful dead subject, so don't try to do research in it" by which he implied that the ongoing research in tha subject at that time was rather pathological and dealt with non- desarguesian planes and such. But in the intervening thirty or forty years, this ``pathological" has made great strides in the hands of pioneers from R. C. Bose \cite{art1-key21} and S. S. Shrikhande \cite{art1-key58} to P. Dembowski \cite{art1-key29} and D. G. Higman \cite{art1-key34}, and has led to a complete classification of Rank 3 groups, which from our view-point of the theory of equations is synonymous to case \ref{art1-sec2.6}. So realizing how even a great man like Zariski could be wrong occasinally, I have learnt to drop one of my numerous prejudices, namely my prejudice against Statistics.  

Note that a permutation group is said to be {\it transitive} if any point (of the permuted set) can be sent to any other, via a permutation in the group. Likewise, a permutation group is {\it m-fold transitive} (briefly: {\it m-fold transitive}) if any $m$ points can be sent to any other $m$ points, via a permutation in that group. $Doubly Transitive = 2-transitive$, $Triply Transitive = 3-transitive$, and so on. By the {\it one point stabilizer} of a transitive permutation group we mean the subgroup consisting of those permutations which keep a certain point fixed; the {\it orbits} of that subgroup are the minimal subsets of the permuted set which are mapped to themselves by every permutation in that subgroup; of the nontrivial orbits are called the {\it sub degrees} of the group, so that the numbers of sub degress is one less than than rank. Thus a {\it Rank 3} group is transitive permutation group whose one point stabilizer has three orbits; the lengths of the two nontrivial orbits are the sub degrees. Needless to say that a Rank 2 group is nothing but a Doubly Transitive permutation group. At any rate, in case \ref{art1-sec2.6}, the degrees of the two irreducible factors of $f_{1}$ correspond to the sub degrees of the relevant Rank 3 group. Now CR3(= the Classification Theorem of Rank 3 groups) implies that very few pairs of integres can be the sub degrees  of Rank 3 groups, very few nonisomorphic Rank groups can have the same sub degrees; see Kantor-Liebler \cite{art1-key42} and Liebeck \cite{art1-key44}. Hence \ref{art1-sec2.6} says that if $f_{1}$ has exactly two irreducible factors then their degrees (and hence also $n$) can have only certain very selective values.

Here, by the relevant group we mean the {\it Galois group} of $f$ over $K$, which we donate by Gal$(f, K)$ and which, following Galois, we define as the group of those permutations of the roots $y_{1}\ldots, y_{n}$ which retain all the polynomial relations between them wiht coefficients in $K$. This definition makes sense without $f$ being irreducible but still assuming $f$ to have no multiple roots. Now our assumptio of $f$ being irreducible is equivalent to assuming that Gal$(f, K)$ is transitive. Likewise, $f_{1}, \ldots, f_{m-1}$ are irreducible iff Gal$(f, K)$ in $m$-transitive. Moreover, as already indicated, $f_{1}$ has exactly two irreducible factors iff Gal$(f, K)$ has rank 3. To match this definition of Galois with the modern definition, let $L$ be the {\it splitting field} of $f$ over $K$, i.e., $L=K(y_{1}\ldots, y_{n})$. Then according to the modern definition, the {\it Galois group} of $L$ over $k$, denoted by Gal$(L, K)$, is defined to be the group of all automorphisms of $L$ which keep $K$  point wise fixed. Considering Gal$(f, K)$ as a permutations group of the subscripts $1,\ldots, n$ of $y_{1}, \ldots, y_{n}$ for every $\tau \in {\rm Gal}(L, K)$ we have a unique $\sigma \in {\rm Gal}(f, K)$ such that $\tau(y_{i}) = y_{\sigma(i)}$ for $1 \leq i \leq n$. Mow we get an isomorphism of Gal$(L, K)$ onto Gal$(f, K)$ by sending each $\tau$ to the corresponding $\sigma$.

Having sufficiently discussed case\ref{art1-sec2.6}, let us note that \ref{art1-sec2.5} is equivalent to CDT (=Classification Theorem of Doubly Transitive permutation groups) fr which we manu refer to Cameron \cite{art1-key22} and Kantor \cite{art1-key41}. At any rate, CDT implies that if $f_{1}$ is irreducible but $f_{2}$ is reducible then we must have: either $n=q$ for some prime power $q$, or $n=(q^{l}-1)/(q-1)$ for some integer $l>1$ and some prime power $q$, or $n=2^{2l-1}-2^{l-1}$ for some integer $l>2$, or $n=15$, or $n=176$, or $n=276$. Likewise \ref{art1-sec2.4} is equivalent to CTT(= Classification of Triply Transitive permutation groups) which is subsumed in CDT, and as a consequence of it we can say that if $f_{1}, f_{2}$ are irreducible but $f_{3}$ is reducible then we must have: either $n=2^{l}$ for some positive integer $l$, or $n-q+1$ for some prime powder $q$, or $n=22$.
       
similarly, \ref{art1-sec2.3} is equivalent to CQT (= Classification of Quadruply Transitive permutation groups) which is subsumed in CTT, and as a consequence of it we can  say that if $f_{1}, f_{2}, f_{3}$ are irreducible but $f_{4}$ is reducible then we must have: eithee $n=23$ and Gal$(f, K)=M_{23}$ or $n=11$ and Gal$(f, K)=M_{11}$, where  $M$ stands for Mathieu. Likewise, \ref{art1-sec2.2} is equivalent to CFT(= Classification of Fivefold Transitive permutation groups) which is subsumed in CQT, and as a consequence of it we can say that if $f_{1},f_{2}, f_{3}, f_{4}$ are irreducible but $f_{5}$ is reducible then we must have: either $n=24$ and Gal$(f, K)=M_{24}$ and Gal$(f, K)=M_{12}$.  

Note that, $M_{24}$ and $M_{24}$ are the only 5-told but not 6-fold transitive permutation groups other than the {\it symmetric group} $S_{5}$ (i.e., the group of all permutations on 5 letters) and the {\it alternating group} $A_{7}$ (i.e., the  sub-group of $s_{7}$ consisting of all {\it even} permutations). Moreover, $M_{23}$ and $M_{11}$ are the respective one point stabilizers of $M_{24}$ and $M_{12}$ and they are the only 4-fold but not 5-fold transitive permutation groups other than $s_{4}$ and $A_{6}$. Here the subscript denotes the degree, i.e., the number of letters being permuted. The four groups $M_{24}, M_{23}, M_{12}, M_{11}$, were constructed by Mathieu \cite{art1-key46} in 1861 as examples of highly transitive permutation groups. But the fact that they are the only 4-transitive permutation groups others than the symmetric groups and the alternating groups, was proved only in 1981 when CDT, and hence also CTT, CQT, CFT and CST, were deduced from  CT(= Classification Theorem of finite simple groups); see Cameron \cite{art1-key23} and Cameron-Cannon \cite{art1-key24}. Recall that a group is {\it simple} if it has no nonidentity normal subgroup other than itself; it turns out that the five {\it Mathieu groups} $M_{24}, M_{23}, M_{22}, M_{12}, M_{11}$ and $M_{22}$ is the point stabilizer of $M_{23}$, are all simple. Now CST refers to the Classification Theorem of Sixfold Transitive permutation groups, according to which the symmetric groups and the alternating groups are the only 6-transitive permutation groups; note that $S_{m}$ is $m$-transitive but not $(m+1)$-transitive, whereas $A_{m}$ is $(m-2)$-transitive but not $(m-1)$-transitive for $m\geq 3$. In \ref{art1-sec2.2} to \ref{art1-sec2.6} we had assumed $M<n-2$ to avoid including the symmetric and alternating groups; dropping this assumption, \ref{art1-sec2.1} is equivalent to CST with the clarification that, under the assumption of \ref{art1-sec2.1}, the quadratic $f_{n-2}$ is irreducible or reducible according as Gal$(f, K) = s_{n}$ or $A_{n}$.

We have already hinted that CR3 was also deduced as a consequence of CT; Liebeck \cite{art1-key44}. The proof of CT itself was completed in 1980 (see Gorenstein \cite{art1-key32}) with {\it staggering statistics:} 30 years; 100 authors; 500 papers; 15,000 pages! Add some more pages for CDT and CR3 and so on.

All we have done above is to translate this group theory into te language of theory of equations where $K$ is ANY field. So are still talking High-School? Not really, unless we admint CT into High-School!
  
\section{Galois} 

Summarizing, to compute the Galois group Gal$(f, K)$, say when $K$ is the field $k(X)$ of univariate rational functions over an algebraically closed ground field $k$, by throwing away roots and using some algebraic geometry we find some multi-transitivity and other properties of the Galois groups and fedd there into the group theory machine. Out comes a list of possible groups. Reverting to algebraic geometry, sometimes augmented by High-School manipulations, we successively eliminate various members from that list until, hopefully, one is-left. That then is the answer. I say hopefully because we would have a contradiction in which the ultimate reality {\bf (Brahman)} is described by {\it Neti Neti}, not this, not that. If you practice pure {\it Advaita}, then nothing is left, which is too austere. So we fall back on the kinder {\it Dvaita} according to which the unique {\bf God} remains. 

\section{Riemann and Dedekind}

In case $K=\bC(X)$ and $a_{i}=a_{i}(X)\in \bC[X]$ for $1\leq i \leq n$, where {\bf C} is the field  of comples numbers, following Riemann \cite{art1-key53} we can consider the {\it monodromy group} of $f$ thus.

Fix a non discriminant point $\mu$, i.e., value $mu \in \bC$ of $X$ for which the equation $f=0$ has $n$ distinct roots. Then, say by the Implicit Function Theorem, we can solve the equation the equation $f=0$ near $\mu$, getting $n$ analytic solutions $\eta _{1}(X), \ldots, \eta _{n}(X)$ near $\mu$. To find out how there solutions are intertwined, mark a finite number of values $\alpha_{1}, \ldots, \alpha_{w}$ of $X$ which are different from $\mu$ but include all the discriminant points, and let $\bC_{w}$ be the complex $X$-plane minus these $w$ points. Now by making analytic continuations along any closed path $\Gamma$ in $\bC_{w}$ starting and ending at $\mu$ so that $\eta_{i}$ continues into $\eta_{j}$ with $\Gamma'(i)= j$ for $1 \leq i \leq n$. As $\Gamma$ varies over all closed paths in $\bC_{w}$ starting and ending at $\mu$, the permutations $\Gamma'$ span a subgroup of $S_{n}$ called the {\it monodromy group} of  $f$ which we denote bye $M(f)$.   
 
By indentifying the analytic solutions $\eta_{1}, \ldots, \eta_{n}$ with the algebraic roots $y_{1}, \ldots, y_{n}$, the monodromy group $M(f)$ gets identified with the Galois group Gal$(f, \bC(X))$, and so these two groups are certainly isomorphic as permutation groups.

To get generators for $M(f)$, given any $alpha \in \bC_{w}$, let $\Gamma_{\alpha}$ be the path in $\bC_{w}$ consisting of a line segment from $\mu$ to a point very near $\alpha$ followed by a small circle around $\alpha$ and then back to $\mu$ along the said line segment. Let us write the corresponding permutation $\Gamma'_{\alpha}$ as a product of disjoint cycles, and let $e_{1}, \ldots, e_{h}$ be the lengths of these cycles. To get a tie-up between  these Riemannian considerations and the  thought of Dedekind \cite{art1-key28}, let $v$ be the valuation of $\bC(X)$ corresponding to $\alpha$, i.e., $v(g)$ is the order of zero at $alpha$ for every $g\in \bC[X]$. Then, as remarked in my 1957 paper \cite{art1-key3}, the cycle lengths $e_{1}, \ldots, e_{n}$ coincide with the ramification exponents of the various extensions of $v$ to the \textit{root field} $\bC(X)(y_{1})$, and their {\bf LCM} equals the ramification exponent of any extension of $v$ to the splitting field $\bC(X)(y_{1}, \ldots,y_{n})$. In particular, $\Gamma'_{\alpha}$ is the identity permutation iff $\alpha$ is not a \textit{branch point}, i.e., if and only if the ramification exponents of the various extensions of $v$ to the root field $\bC(X)(y_{1})$ (or equivalently to the splitting field $\bC(X)(y_{1},\ldots, y_{n})$) are all 1. At any rate, a branch point is always a dicriminant point but not conversely. Indeed, the difference between the two is succinctly expressed by Dedekind's Theorem according to which the ideal generated by the $Y$-derivative of $f$ equals the products of the \textit{different} and the \textit{conductor}. In this connection you may refer to pages 423 and 438 of any my Monthly Article\cite{art1-key5} which costitutes some of my \textit{Ramblings} in the woods of algebraic geometry. You may also refer to pages 65 and 169 of my recent book \cite{art1-key6} for Scientists and Engineers inti which these Ramblings have now been expanded.
  
Having given a tie-up between the ideas of Riemanna and Dedekind(both of whom wre pupils of Gauss) concerning branch points, ramification exponents, and so on, it is time to say that these things actually go back to Newton \cite{art1-key47}. For an excellent discussion of the seventeenth century work of Newton on this matter, see pages 373-397 of Part II of the 1886 Textbook of Algebra by Chrystal \cite{art1-key27}. For years having recommended Chrystal as the best book to learn algebra from, from time to time I decide to take my own advice a wealth of information it contains! At any rate, \`am la Newton, we can use fractional power series in $X$ to factor $f$ into linear factors in $Y$, and then combine conjugacy classes to get a factorization $f= \prod^{h}_{i=1} \phi_{i}$ where $\phi_{i}$ is an irreducible ploynomail of degree $e_{i}$ in $Y$ whose coefficients are power series in $X-\alpha$. If the field $\bC$ were not algebraically closed then the degree of $\phi_{i}$ would be $e_{i}f_{i}$ with $f_{i}$ being certain ``residuce degrees" and we would get the famous formula $\sigma^{h}_{i=1} e_{i}f_{i}=n$ of Dedekind- Domain Theory. See Lectures 12 and 21 of Scientists \cite{art1-key6}.

Geometrically speaking, i.e., following the ideas of Max Noether \cite{art1-key48}, if we consider the curve $f=0$ in the discriminant points correspond to vertical lines which meet the curve in less than $n$ point, the branch points correspond to vertical tangents, and the `` conductor points" are the singularities. See figure 9 on page 429 of Ramblings \cite{art1-key5}.

 Getting back to finding generators for $M(f)$, with the refinement of discriminant points into branch points and conductor points in hand, it suffices to stipulate that $\alpha_{1}, \ldots, \alpha_{w}$ inculdue all the branch points rathaer than all the discrimant points. Now bye choosing the \textit{base point} $\mu$ suitably, we may asume that the line from $mu$ to $\alpha_{1}, \ldots, \alpha_{w}$ do not meet each other except at $\mu$.  Now it will turn out that the permutations $\Gamma'_{\alpha_{1}}, \ldots, \Gamma'_{\alpha_{w}}$ generate $M(f)$. This follows from the
\textit{Monodromy Theorem} together with the fact that the \textit{(topological) fundamental group} $\pi_{1}(\bC_{w})$ of $\bC_{w}$ (also called the Poincar\`e group of $\bC_{w}$) is the \textit{free group} $\bF_{w}$ on $w$ generators. Briefly speaking, the monodromy Theorem says that two paths, which can be continuously deformed into each other, give rise to the same analytic continuations. The fundamental group itself may heuristically be described as that incarnation of the monodromy

 group which works for all functions whose branch points are amongest $\alpha_{1}, \ldots, \alpha_{w}$. More precisely, $\pi_{1})(\bC_{w})$ consists of the equivalent means they can be continuously deformed inti each other. Now the (equivalence classes of the) paths $\Gamma_{\alpha_{1}}, \ldots, \Gamma_{\alpha_{w}}$ are free generators of $\pi_{1}(\bC_{w})$, and we have an obvious epimorphism of $\pi_{1}(\bC_{w})= \bF_{w}$ onto $M(f)$ and hence the permutations $\Gamma_{\alpha_{1}}, \ldots, \Gamma_{\alpha_{w}}$ generate $\bM(f)$; for relevant picture etc., you may see pages 442-443 of Ramblings \cite{art1-key5} or pages 171-172 of Scientists \cite{art1-key6}. So the curve $f=0$, or equivalently the \textit{Galois extension} $L=\bC(X)(y_{1}, \ldots, y_{n})$, is an \textit{ramified covering} of $\bC_{w}$, and tha Galois Group Gal$(L, C(X)) = {\rm Ga}(f, C(X))$ is generated by $w$ generators. Surprisingly, to this day \textit{there is no algebraic proof of this algebraic fact}.

The Riemann Existence Theorem says that conversely, every finite homomorphic image of $\pi_{1}(\bC_{w})$ can be realized as $\bM(f)$ for some $f$. Thus be defining the \textit{algebraic fundamental growp} $\pi_{A}(\bC_{w})$ as the set of all finite groups which are the Galois groups of finite unramified coverings of $\bC_{w}$, we can say that $\pi_{A}(\bC_{w})$ coincides with the set of all  finite groups generated by $w$ generators. \textit{Needless to say that, a fortiori, there is no algebraic proof of the converse part of this algebraic fact either}.

Now, in the complex $(X, Y)$-plane, $f=0$ is a curve $C_{g}$ of some \textit{genus} $g$, i.e., if from $C_{g}$ we delete a finite number of points including all its singularities, then what we get is homeomorphic to a sphere with $g$ handles minus a finite number of points. For any nonnegative integer $w$, let $C_{g,w}$ be obtained by adding to $C_{g}$ its \textit{points at infinity,} then \textit{desingularizing} it, and finally removing $w+1$ points from the desingularized verison. Then $C_{g,w}$ is homomorphic to a sphere with $g$ handles minus $w+1$ points, and hence it can be seen that $\pi_{1}(C_{g,w})=\bF_{2g+w}$; for instance see the excellent topology book of Seifert and Threlfall \cite{art1-key54}. The above monodromy and existence considerations generalize fromn the genus zero case to the case of general $g$, and we get the \textit{result} that the \textit{algebraic fundamental group} $\pi_{A}(C_{g,w})$ coincides with the set of all finite groups generated be $2g+w$ generators, where $\pi_{A}(C_{g,w})$ is \textit{defined} to be the set of all finite groups which are the Galois groups of finite unramified coverings of $\bC_{g,w}$.

\section{Chrystal and Forsyth}
just as Chrystal excels in explaining Newtonian (and Eulerian) ideas, Forsyth's 1918 book on Function Theory \cite{art1-key31} is highly recommended for getting a good insight into Riemannian ideas. Thus it was by absorvbing parts of Forsyth that, in my recent papers \cite{art1-key8} and \cite{art-key10}, I could algebracize some of the monodromy considerations to formulate certain ``Cycle Lemmas" which say that under such and sucn conditions the Galois group contains permutations having such and such cycle structure.

Now the Rirmann Existence Theorem was only surmised be Riemann \cite{art1-key52} by appealing to the \textit{Principle} of his teacher Dirichlet which, after Weierstrass Criticism was put on firmee ground by Hilbert in 1904 \cite{art1-key35}. In the meantime another classical treatment of the Riemann Existence Theorem was carried out culminating in the Klein-Poincar\'e-Koebe theory of automorphic functions, for which again Forsyth's book is a good source. A modern treatment of the Riemann Existence Theorem using coherent analytic sheaves was finally given by Serre in his famous GAGA paper \cite{art1-key55} of 1956.

\section{Serre}

Given any algebraically closed ground field $k$ of any nonzero characteristic $p$, in my 1957 paper \cite{art1-key3}, all this led me to \textit{define} and \textit{algebraic fundamental group} $\pi_{A}(C_{g,w})$ of $C_{g,w} = C_{g}$ minus $w+1$ points, where $w$ is a nonnegative integer and $C_{g}$ is a nonsigular projetive curve of genus $g$ over $k$, to be the set of all finite groups which can be realized as Galois groups of finite unramified coverings of $C_{g,w}$. In tha paper, I went on to \textit{conjecture} that $\pi_{A}(C_{g,w})$ coincides with the set of all finite groups $G$ for which $G/p(G)$ is generated by $2g+w$ generators, where $p(G)$ is the subgroup of $G$ generated by all its $p$-Sylow subgroups. The $g=w=0$ case of this conjecture, which may be called the \textit{quasi p-group conjecuture}, says that for the affine line $L_{k}$ over $k$ we have $\pi_{A}(L_{k})=Q(p)$ where $Q(p)$ denotes the set of all quasi $p$-groups, i.e., finite groups which are generated by their $p$-Sylow subgroups. It may be noted every finite simple group whose order is divisible bey $p$ is obviously a quasi $p$-group. Hence in particular the alternating group $A_{n}$ is a quasi $p$-group whenever either $n\geq p > 2$ or $n-3\geq p=2$. Likewise the symmetric group $S_{n}$ is a quasi $p$-group provided $n\geq p=2$.   

In support of the quasi $p$-group conjecture,in the 1957 paper  I wrote down several equations giving unramified coveing of the affine line $L_{k}$ and suggested that their Galois groups be computed. This included the equation $\overline{F}_{n,q,s,a}=0$ with
$$
\overline{F}_{n,q,s,a}=Y^{n}-aX^{s}Y^{t} + 1 \quad {\rm and} \quad n=q+t
$$
where $0\neq a \in k$ and $q$ is a positive power of $p$ and $s$ and $t$ are positive integers with $t \nequiv 0(p)$, and we want to compute its Galois group $\overline{G}_{n,q,s,q} = {\rm Gal}(\overline{F}_{n,q,s,q}, k(X))$.

By using a tiny amount of the information contained in the above equation, I showed that $\pi_{A}(L_{k})$ contains many unasolvable groups, and indeed by taking homomorphic imagtes of subgroups of members of $\pi_{A}(\bL_{k})$ we get all finite groups; see Result 4 and Remark 6 on pages 841-842 of \cite{art1-key3}. This was somewhat of a surpise because the comples affine line is simply connected, and although $\pi_{A}(\bL_{k})$ was known to contain $p$-cyclic groups (so called Artin-Schreier equations), it was felt that perhaps it does not contain much more. This feeling, which turned out to be wrong, might have been based on the facts that $L_{k}$ is a ``commutative group variety" and the fundamental group of a topological group is always abelian; see Proposition 7 on page 54 of Chevalley \cite{art1-key26}.

To algebracize the fact that the comples affine line is simply connected, be the genus formula we deduce that the affine line over an algebraically closed ground field of characteristic zero has no nontrivial unramified coverings. In our case of characteristic $p$, the same formula shows that every membed of $\pi_{A}(\bL_{k})$ is a squasi $p$-group; see Result 4 on page 841 of \cite{art1-key3}.

Originally I found the above equation $\overline{F}_{n,q,s,a}=0$ by taking a section of a surface which I had constructed in my 1955 Ph.D. Thesis \cite{art1-key1} to show that jung's classical method \cite{art1-key40} of surface desigularization doed not work for nonzero charactheristic because the local fundamental group above a normal crossing of the branch locus need not be solvable, while in the comples case it is always abelian. This failure of Jung's method led me ti devise more algoprithmic techniques fr desingularizing surfaces in nonzero characteristic, and this formed the positive part of my Ph.D. Thesis \cite{art1-key2}.

Soon after the 1957 paper, I wrote a series of articles \cite{art1-key4} on ``tame coverings" of higher dimensional algebraic varieties, and took note of Grothendieck \cite{art1-key33} proving the ``tame part" of the above conjecture which says that the members of $\pi_{A}(C_{g,w})$ whose order is prime to $p$ are exavtly all the finite gropus of order prime to $p$ generated by $2g+w$ generators.


%%%%%%%%19%%%%%%%%%%%
