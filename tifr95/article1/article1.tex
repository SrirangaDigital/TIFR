\title{Fundamental Group of the Affine Line in Positive Characteristic}
\markright{Fundamental Group of the Affine Line in Positive Characteristic}

\author{By~ Shreeram S. Abhyankar\footnote{Invited Lecture delivered on 8 January 1992 at the International Colloquium on Geometry and Analysis in TIFR in Bombay. This work was partly supported by NSF Grant DMS-91-01424}}
\markboth{Shreeram S. Abhyankar}{Fundamental Group of the Affine Line in Positive Characteristic}

\date{}
\maketitle

\section{Introduction}\pageoriginale

I have known both Narasimhana and Seshadri since 1958 when I had a nice meal with them at the Student Cafeteria in Cit\'e Universitaire in Paris. So I am very pleased to be here to wish them a Happy Sixtieth Birthday. My association with the Tata Institute gore back even further to 1949-1951 when, as a college student, I used to attend the lectures of M. H. Stone and K. Chandraseksharan, first in Pedder Road and then at the Yacht Club. Then in the last many years I have visited the Tata Institute numerous times. So this Conference is a nostalgic homecoming to me.     

To enter into the subject of Fundamental Groups, let me, as usual, make a.

\section{High-School Beginning}

So consider a polynomial
$$
f=f(Y) = Y^{n}+a_{1}Y^{n-1}+\cdots+a_{n}
$$
with coefficients $a_{1}, \ldots,a_{n}$ in some field $K$; for example, $K$ could be the field of rational numbers. We want to solve the equation $f=0$, i.e., we want to find the roots of $f$. Assume that $f$ is irreducible and has no multiple roots. Suppose somehow we found a root $y_{1} $ of $f$. Then to make the problem of finding the other roots easier, we achieve a decrease in degree by ``throwing away" the root $y_{1}$ to get
$$
f_{1}=f_{1}(Y)=\dfrac{f(Y)}{Y-y_{1}}=Y^{n-1}+b_{1}Y^{n-2}+\cdots+b_{n-1}.
$$
If $f_{1}(Y)$ is also irreducible and if somehow we found a root $y_{2}$ of $f_{1}$, then  ``throwing away" $y_{2}$ we get
$$
f_{2} = f_{2}Y = \dfrac{f_{1}(Y)}{Y-y_{2}}=Y^{n-2}+c_{1}Y^{n-3}+\cdots+c_{n-2}.
$$
Note that the coefficients $b_{1}\ldots, b_{n-1}$ of $f_{1}$ do involve $y_{1}$ and hence they are note in $K$, but they are in $K(y_{1})$. So, although we assumed $f$ to be irreducible in $K[Y]$, when we said ``if  $f_{1}$ is irreducible," we clearly meant ``if $f_{1}$ is irreducible in $K(y_{1})[Y]"$. Likewise, irreducibility of $f_{2}$ refers to its irreducibility in $K(y_{1}, y_{2})[Y]$. And so on. In this way we get a sequence of polynomials $f_{1}, f_{2}, \ldots, f_{m}$ of degrees $n-1, n-2, \ldots, n-m$ in $Y$ with coefficients in $K(y_1), K(y_{1}, y_{2}), \ldots, K(y_{1},\ldots,y_{m})$, where $f_{i}$ is irreducible in $K(y_{1}, \ldots, y_{i})[Y]$ for $i=1, 2, \ldots, m-1$. If $f_{m}$ is reducible in $K(y_(1), \ldots, y_{m})[Y]$ then we stop, otherwise we proceed to get $f_{m+1}$, and so on. Now we may ask the following. 

\noindent
{\bf Question:} Given any positive integers $m<n$, doers there exist an irreducible polynomial $f$ of degree $n$ in $Y$ with coeffcients in some field $K$ such that the above sequence terminates exactly after $m$ steps, i.e., such that  $f_{1}, f_{2}, \ldots, f_{m-1}$ are irreducible but $f_{m}$ is reducible?  

I presume that most of us, when asked to respond quickly, might say: ``Yes, but foe large $m$ and $n$ it would be time consuming to write down concrete examples". However, the {\bf SURPRISE OF THE CENTURY}  is that the {\bf ANSWER} is {\bf NO}. More precisely, it turns out that

\subsection{$f_{1}, f_{2}, f_{3}, f_{4}, f_{5}$ irreducible $\Rightarrow$
 $f_{6}, f_{7},\ldots, f_{n-3}$ irreducible.}\label{art1-sec2.1}

In other words, if $f_{1}, \ldots, f_{5}$ are irreducible then $f_{1}, \ldots, f_{n-1}$ are all irreducible except the $f_{n-2}$, which is a quadratic, may or may not be irreducible. This answers the case of $m \geq 6$. Going down the line  to $m \leq 5$ and assuming $m<n-2$, for the case of $m=5$ we have that

\subsection{$f_{1}, f_{2}, f_{3}, f_{4}$ irreducible but $f_{5}$ reducible $\Rightarrow$ $n=24$ or $12$ and for the case of $m=4$ we have that}\label{art1-sec2.2}

\subsection{$f_{1}, f_{2}, f_{3}$ irreducible but $f_{4}$ reducible $\Rightarrow$ $n=23$ or $11$.}\label{art1-sec2.3}

Going further down the line, for the case of $m=3$ we have that

\subsection{$f_{1}, f_{2}$ irreducible but $f_{3}$ reducible $\Rightarrow$ Refined FT of Proj Geom}\label{art1-sec2.4}

i.e., if $f_{1}, f_{2}$ are irreducible but $f_{3}$ is reducible,then there are only a few possibilities and they are suggested by the Fundamental Theorem  of  Projective Geometry, which briefly says that ``the underlying division ring of a synthetically defined desarguestion projective plane is a field in and only if any three point of a projective line can be mapped to any other three points of that projective line by a unique projectivity." Going still further down the line for the case of $m=2$ we have that   

\subsection{$f_{1}$ irreducible but $f_{2}$ reducible $\Rightarrow$ known but too long}\label{art1-sec2.5}

i.e., if $f_{1}$ is irreducible but $f_{2}$ is reducible, the answer is known but the list of possibilities is too long to write down here. Finally, for the case of $m=1$ we have that

\subsection{$f_{1}$ has exactly two irreducible factors $\Rightarrow$ Pathol proj Geom $+$ Stat}\label{art1-sec2.6}

i.e., if $f_{1}$ has exactly two irreducible factors, then again a complete answer is known, which depends on Pathological Projective Geometry and Block Designs from Statistics! Here I am reminded of the beautiful course on Projective Geometry which I took from Zariski (in 1951 at Harvard), and in which I learnt the Fundamental Theorem mentioned in (2.4). At the end of that course, Zariski said to me that ``Projective geometry is a beautiful dead subject, so don't try to do research in it" by which he implied that the ongoing research in tha subject at that time was rather pathological and dealt with non- desarguesian planes and such. But in the intervening thirty or forty years, this ``pathological" has made great strides in the hands of pioneers from R. C. Bose \cite{art1-key21} and S. S. Shrikhande \cite{art1-key58} to P. Dembowski \cite{art1-key29} and D. G. Higman \cite{art1-key34}, and has led to a complete classification of Rank 3 groups, which from our view-point of the theory of equations is synonymous to case \ref{art1-sec2.6}. So realizing how even a great man like Zariski could be wrong occasinally, I have learnt to drop one of my numerous prejudices, namely my prejudice against Statistics.  

Note that a permutation group is said to be {\it transitive} if any point (of the permuted set) can be sent to any other, via a permutation in the group. Likewise, a permutation group is {\it m-fold transitive} (briefly: {\it m-fold transitive}) if any $m$ points can be sent to any other $m$ points, via a permutation in that group. $Doubly Transitive = 2-transitive$, $Triply Transitive = 3-transitive$, and so on. By the {\it one point stabilizer} of a transitive permutation group we mean the subgroup consisting of those permutations which keep a certain point fixed; the {\it orbits} of that subgroup are the minimal subsets of the permuted set which are mapped to themselves by every permutation in that subgroup; of the nontrivial orbits are called the {\it sub degrees} of the group, so that the numbers of sub degress is one less than than rank. Thus a {\it Rank 3} group is transitive permutation group whose one point stabilizer has three orbits; the lengths of the two nontrivial orbits are the sub degrees. Needless to say that a Rank 2 group is nothing but a Doubly Transitive permutation group. At any rate, in case \ref{art1-sec2.6}, the degrees of the two irreducible factors of $f_{1}$ correspond to the sub degrees of the relevant Rank 3 group. Now CR3(= the Classification Theorem of Rank 3 groups) implies that very few pairs of integres can be the sub degrees  of Rank 3 groups, very few nonisomorphic Rank groups can have the same sub degrees; see Kantor-Liebler \cite{art1-key42} and Liebeck \cite{art1-key44}. Hence \ref{art1-sec2.6} says that if $f_{1}$ has exactly two irreducible factors then their degrees (and hence also $n$) can have only certain very selective values.

Here, by the relevant group we mean the {\it Galois group} of $f$ over $K$, which we donate by Gal$(f, K)$ and which, following Galois, we define as the group of those permutations of the roots $y_{1}\ldots, y_{n}$ which retain all the polynomial relations between them wiht coefficients in $K$. This definition makes sense without $f$ being irreducible but still assuming $f$ to have no multiple roots. Now our assumptio of $f$ being irreducible is equivalent to assuming that Gal$(f, K)$ is transitive. Likewise, $f_{1}, \ldots, f_{m-1}$ are irreducible iff Gal$(f, K)$ in $m$-transitive. Moreover, as already indicated, $f_{1}$ has exactly two irreducible factors iff Gal$(f, K)$ has rank 3. To match this definition of Galois with the modern definition, let $L$ be the {\it splitting field} of $f$ over $K$, i.e., $L=K(y_{1}\ldots, y_{n})$. Then according to the modern definition, the {\it Galois group} of $L$ over $k$, denoted by Gal$(L, K)$, is defined to be the group of all automorphisms of $L$ which keep $K$  point wise fixed. Considering Gal$(f, K)$ as a permutations group of the subscripts $1,\ldots, n$ of $y_{1}, \ldots, y_{n}$ for every $\tau \in {\rm Gal}(L, K)$ we have a unique $\sigma \in {\rm Gal}(f, K)$ such that $\tau(y_{i}) = y_{\sigma(i)}$ for $1 \leq i \leq n$. Mow we get an isomorphism of Gal$(L, K)$ onto Gal$(f, K)$ by sending each $\tau$ to the corresponding $\sigma$.

Having sufficiently discussed case\ref{art1-sec2.6}, let us note that \ref{art1-sec2.5} is equivalent to CDT (=Classification Theorem of Doubly Transitive permutation groups) fr which we manu refer to Cameron \cite{art1-key22} and Kantor \cite{art1-key41}. At any rate, CDT implies that if $f_{1}$ is irreducible but $f_{2}$ is reducible then we must have: either $n=q$ for some prime power $q$, or $n=(q^{l}-1)/(q-1)$ for some integer $l>1$ and some prime power $q$, or $n=2^{2l-1}-2^{l-1}$ for some integer $l>2$, or $n=15$, or $n=176$, or $n=276$. Likewise \ref{art1-sec2.4} is equivalent to CTT(= Classification of Triply Transitive permutation groups) which is subsumed in CDT, and as a consequence of it we can say that if $f_{1}, f_{2}$ are irreducible but $f_{3}$ is reducible then we must have: either $n=2^{l}$ for some positive integer $l$, or $n-q+1$ for some prime powder $q$, or $n=22$.
       
similarly, \ref{art1-sec2.3} is equivalent to CQT (= Classification of Quadruply Transitive permutation groups) which is subsumed in CTT, and as a consequence of it we can  say that if $f_{1}, f_{2}, f_{3}$ are irreducible but $f_{4}$ is reducible then we must have: eithee $n=23$ and Gal$(f, K)=M_{23}$ or $n=11$ and Gal$(f, K)=M_{11}$, where  $M$ stands for Mathieu. Likewise, \ref{art1-sec2.2} is equivalent to CFT(= Classification of Fivefold Transitive permutation groups) which is subsumed in CQT, and as a consequence of it we can say that if $f_{1},f_{2}, f_{3}, f_{4}$ are irreducible but $f_{5}$ is reducible then we must have: either $n=24$ and Gal$(f, K)=M_{24}$ and Gal$(f, K)=M_{12}$.  

Note that, $M_{24}$ and $M_{24}$ are the only 5-told but not 6-fold transitive permutation groups other than the {\it symmetric group} $S_{5}$ (i.e., the group of all permutations on 5 letters) and the {\it alternating group} $A_{7}$ (i.e., the  sub-group of $s_{7}$ consisting of all {\it even} permutations). Moreover, $M_{23}$ and $M_{11}$ are the respective one point stabilizers of $M_{24}$ and $M_{12}$ and they are the only 4-fold but not 5-fold transitive permutation groups other than $s_{4}$ and $A_{6}$. Here the subscript denotes the degree, i.e., the number of letters being permuted. The four groups $M_{24}, M_{23}, M_{12}, M_{11}$, were constructed by Mathieu \cite{art1-key46} in 1861 as examples of highly transitive permutation groups. But the fact that they are the only 4-transitive permutation groups others than the symmetric groups and the alternating groups, was proved only in 1981 when CDT, and hence also CTT, CQT, CFT and CST, were deduced from  CT(= Classification Theorem of finite simple groups); see Cameron \cite{art1-key23} and Cameron-Cannon \cite{art1-key24}. Recall that a group is {\it simple} if it has no nonidentity normal subgroup other than itself; it turns out that the five {\it Mathieu groups} $M_{24}, M_{23}, M_{22}, M_{12}, M_{11}$ and $M_{22}$ is the point stabilizer of $M_{23}$, are all simple. Now CST refers to the Classification Theorem of Sixfold Transitive permutation groups, according to which the symmetric groups and the alternating groups are the only 6-transitive permutation groups; note that $S_{m}$ is $m$-transitive but not $(m+1)$-transitive, whereas $A_{m}$ is $(m-2)$-transitive but not $(m-1)$-transitive for $m\geq 3$. In \ref{art1-sec2.2} to \ref{art1-sec2.6} we had assumed $M<n-2$ to avoid including the symmetric and alternating groups; dropping this assumption, \ref{art1-sec2.1} is equivalent to CST with the clarification that, under the assumption of \ref{art1-sec2.1}, the quadratic $f_{n-2}$ is irreducible or reducible according as Gal$(f, K) = s_{n}$ or $A_{n}$.

We have already hinted that CR3 was also deduced as a consequence of CT; Liebeck \cite{art1-key44}. The proof of CT itself was completed in 1980 (see Gorenstein \cite{art1-key32}) with {\it staggering statistics:} 30 years; 100 authors; 500 papers; 15,000 pages! Add some more pages for CDT and CR3 and so on.

All we have done above is to translate this group theory into te language of theory of equations where $K$ is ANY field. So are still talking High-School? Not really, unless we admint CT into High-School!
  
\section{Galois} 

Summarizing, to compute the Galois group Gal$(f, K)$, say when $K$ is the field $k(X)$ of univariate rational functions over an algebraically closed ground field $k$, by throwing away roots and using some algebraic geometry we find some multi-transitivity and other properties of the Galois groups and fedd there into the group theory machine. Out comes a list of possible groups. Reverting to algebraic geometry, sometimes augmented by High-School manipulations, we successively eliminate various members from that list until, hopefully, one is-left. That then is the answer. I say hopefully because we would have a contradiction in which the ultimate reality {\bf (Brahman)} is described by {\it Neti Neti}, not this, not that. If you practice pure {\it Advaita}, then nothing is left, which is too austere. So we fall back on the kinder {\it Dvaita} according to which the unique {\bf God} remains. 

\section{Riemann and Dedekind}

In case $K=\bC(X)$ and $a_{i}=a_{i}(X)\in \bC[X]$ for $1\leq i \leq n$, where {\bf C} is the field  of comples numbers, following Riemann \cite{art1-key53} we can consider the {\it monodromy group} of $f$ thus.

Fix a non discriminant point $\mu$, i.e., value $mu \in \bC$ of $X$ for which the equation $f=0$ has $n$ distinct roots. Then, say by the Implicit Function Theorem, we can solve the equation the equation $f=0$ near $\mu$, getting $n$ analytic solutions $\eta _{1}(X), \ldots, \eta _{n}(X)$ near $\mu$. To find out how there solutions are intertwined, mark a finite number of values $\alpha_{1}, \ldots, \alpha_{w}$ of $X$ which are different from $\mu$ but include all the discriminant points, and let $\bC_{w}$ be the complex $X$-plane minus these $w$ points. Now by making analytic continuations along any closed path $\Gamma$ in $\bC_{w}$ starting and ending at $\mu$ so that $\eta_{i}$ continues into $\eta_{j}$ with $\Gamma'(i)= j$ for $1 \leq i \leq n$. As $\Gamma$ varies over all closed paths in $\bC_{w}$ starting and ending at $\mu$, the permutations $\Gamma'$ span a subgroup of $S_{n}$ called the {\it monodromy group} of  $f$ which we denote bye $M(f)$.   
 
By indentifying the analytic solutions $\eta_{1}, \ldots, \eta_{n}$ with the algebraic roots $y_{1}, \ldots, y_{n}$, the monodromy group $M(f)$ gets identified with the Galois group Gal$(f, \bC(X))$, and so these two groups are certainly isomorphic as permutation groups.

To get generators for $M(f)$, given any $alpha \in \bC_{w}$, let $\Gamma_{\alpha}$ be the path in $\bC_{w}$ consisting of a line segment from $\mu$ to a point very near $\alpha$ followed by a small circle around $\alpha$ and then back to $\mu$ along the said line segment. Let us write the corresponding permutation $\Gamma'_{\alpha}$ as a product of disjoint cycles, and let $e_{1}, \ldots, e_{h}$ be the lengths of these cycles. To get a tie-up between  these Riemannian considerations and the  thought of Dedekind \cite{art1-key28}, let $v$ be the valuation of $\bC(X)$ corresponding to $\alpha$, i.e., $v(g)$ is the order of zero at $alpha$ for every $g\in \bC[X]$. Then, as remarked in my 1957 paper \cite{art1-key3}, the cycle lengths $e_{1}, \ldots, e_{n}$ coincide with the ramification exponents of the various extensions of $v$ to the \textit{root field} $\bC(X)(y_{1})$, and their {\bf LCM} equals the ramification exponent of any extension of $v$ to the splitting field $\bC(X)(y_{1}, \ldots,y_{n})$. In particular, $\Gamma'_{\alpha}$ is the identity permutation iff $\alpha$ is not a \textit{branch point}, i.e., if and only if the ramification exponents of the various extensions of $v$ to the root field $\bC(X)(y_{1})$ (or equivalently to the splitting field $\bC(X)(y_{1},\ldots, y_{n})$) are all 1. At any rate, a branch point is always a dicriminant point but not conversely. Indeed, the difference between the two is succinctly expressed by Dedekind's Theorem according to which the ideal generated by the $Y$-derivative of $f$ equals the products of the \textit{different} and the \textit{conductor}. In this connection you may refer to pages 423 and 438 of any my Monthly Article\cite{art1-key5} which costitutes some of my \textit{Ramblings} in the woods of algebraic geometry. You may also refer to pages 65 and 169 of my recent book \cite{art1-key6} for Scientists and Engineers inti which these Ramblings have now been expanded.
  
Having given a tie-up between the ideas of Riemanna and Dedekind(both of whom wre pupils of Gauss) concerning branch points, ramification exponents, and so on, it is time to say that these things actually go back to Newton \cite{art1-key47}. For an excellent discussion of the seventeenth century work of Newton on this matter, see pages 373-397 of Part II of the 1886 Textbook of Algebra by Chrystal \cite{art1-key27}. For years having recommended Chrystal as the best book to learn algebra from, from time to time I decide to take my own advice a wealth of information it contains! At any rate, \`am la Newton, we can use fractional power series in $X$ to factor $f$ into linear factors in $Y$, and then combine conjugacy classes to get a factorization $f= \prod^{h}_{i=1} \phi_{i}$ where $\phi_{i}$ is an irreducible ploynomail of degree $e_{i}$ in $Y$ whose coefficients are power series in $X-\alpha$. If the field $\bC$ were not algebraically closed then the degree of $\phi_{i}$ would be $e_{i}f_{i}$ with $f_{i}$ being certain ``residuce degrees" and we would get the famous formula $\sigma^{h}_{i=1} e_{i}f_{i}=n$ of Dedekind- Domain Theory. See Lectures 12 and 21 of Scientists \cite{art1-key6}.

Geometrically speaking, i.e., following the ideas of Max Noether \cite{art1-key48}, if we consider the curve $f=0$ in the discriminant points correspond to vertical lines which meet the curve in less than $n$ point, the branch points correspond to vertical tangents, and the `` conductor points" are the singularities. See figure 9 on page 429 of Ramblings \cite{art1-key5}.

 Getting back to finding generators for $M(f)$, with the refinement of discriminant points into branch points and conductor points in hand, it suffices to stipulate that $\alpha_{1}, \ldots, \alpha_{w}$ inculdue all the branch points rathaer than all the discrimant points. Now bye choosing the \textit{base point} $\mu$ suitably, we may asume that the line from $mu$ to $\alpha_{1}, \ldots, \alpha_{w}$ do not meet each other except at $\mu$.  Now it will turn out that the permutations $\Gamma'_{\alpha_{1}}, \ldots, \Gamma'_{\alpha_{w}}$ generate $M(f)$. This follows from the
\textit{Monodromy Theorem} together with the fact that the \textit{(topological) fundamental group} $\pi_{1}(\bC_{w})$ of $\bC_{w}$ (also called the Poincar\`e group of $\bC_{w}$) is the \textit{free group} $\bF_{w}$ on $w$ generators. Briefly speaking, the monodromy Theorem says that two paths, which can be continuously deformed into each other, give rise to the same analytic continuations. The fundamental group itself may heuristically be described as that incarnation of the monodromy

 group which works for all functions whose branch points are amongest $\alpha_{1}, \ldots, \alpha_{w}$. More precisely, $\pi_{1})(\bC_{w})$ consists of the equivalent means they can be continuously deformed inti each other. Now the (equivalence classes of the) paths $\Gamma_{\alpha_{1}}, \ldots, \Gamma_{\alpha_{w}}$ are free generators of $\pi_{1}(\bC_{w})$, and we have an obvious epimorphism of $\pi_{1}(\bC_{w})= \bF_{w}$ onto $M(f)$ and hence the permutations $\Gamma_{\alpha_{1}}, \ldots, \Gamma_{\alpha_{w}}$ generate $\bM(f)$; for relevant picture etc., you may see pages 442-443 of Ramblings \cite{art1-key5} or pages 171-172 of Scientists \cite{art1-key6}. So the curve $f=0$, or equivalently the \textit{Galois extension} $L=\bC(X)(y_{1}, \ldots, y_{n})$, is an \textit{ramified covering} of $\bC_{w}$, and tha Galois Group Gal$(L, C(X)) = {\rm Ga}(f, C(X))$ is generated by $w$ generators. Surprisingly, to this day \textit{there is no algebraic proof of this algebraic fact}.

The Riemann Existence Theorem says that conversely, every finite homomorphic image of $\pi_{1}(\bC_{w})$ can be realized as $\bM(f)$ for some $f$. Thus be defining the \textit{algebraic fundamental growp} $\pi_{A}(\bC_{w})$ as the set of all finite groups which are the Galois groups of finite unramified coverings of $\bC_{w}$, we can say that $\pi_{A}(\bC_{w})$ coincides with the set of all  finite groups generated by $w$ generators. \textit{Needless to say that, a fortiori, there is no algebraic proof of the converse part of this algebraic fact either}.

Now, in the complex $(X, Y)$-plane, $f=0$ is a curve $C_{g}$ of some \textit{genus} $g$, i.e., if from $C_{g}$ we delete a finite number of points including all its singularities, then what we get is homeomorphic to a sphere with $g$ handles minus a finite number of points. For any nonnegative integer $w$, let $C_{g,w}$ be obtained by adding to $C_{g}$ its \textit{points at infinity,} then \textit{desingularizing} it, and finally removing $w+1$ points from the desingularized verison. Then $C_{g,w}$ is homomorphic to a sphere with $g$ handles minus $w+1$ points, and hence it can be seen that $\pi_{1}(C_{g,w})=\bF_{2g+w}$; for instance see the excellent topology book of Seifert and Threlfall \cite{art1-key54}. The above monodromy and existence considerations generalize fromn the genus zero case to the case of general $g$, and we get the \textit{result} that the \textit{algebraic fundamental group} $\pi_{A}(C_{g,w})$ coincides with the set of all finite groups generated be $2g+w$ generators, where $\pi_{A}(C_{g,w})$ is \textit{defined} to be the set of all finite groups which are the Galois groups of finite unramified coverings of $\bC_{g,w}$.

\section{Chrystal and Forsyth}
just as Chrystal excels in explaining Newtonian (and Eulerian) ideas, Forsyth's 1918 book on Function Theory \cite{art1-key31} is highly recommended for getting a good insight into Riemannian ideas. Thus it was by absorvbing parts of Forsyth that, in my recent papers \cite{art1-key8} and \cite{art-key10}, I could algebracize some of the monodromy considerations to formulate certain ``Cycle Lemmas" which say that under such and sucn conditions the Galois group contains permutations having such and such cycle structure.

Now the Rirmann Existence Theorem was only surmised be Riemann \cite{art1-key52} by appealing to the \textit{Principle} of his teacher Dirichlet which, after Weierstrass Criticism was put on firmee ground by Hilbert in 1904 \cite{art1-key35}. In the meantime another classical treatment of the Riemann Existence Theorem was carried out culminating in the Klein-Poincar\'e-Koebe theory of automorphic functions, for which again Forsyth's book is a good source. A modern treatment of the Riemann Existence Theorem using coherent analytic sheaves was finally given by Serre in his famous GAGA paper \cite{art1-key55} of 1956.

\section{Serre}

Given any algebraically closed ground field $k$ of any nonzero characteristic $p$, in my 1957 paper \cite{art1-key3}, all this led me to \textit{define} and \textit{algebraic fundamental group} $\pi_{A}(C_{g,w})$ of $C_{g,w} = C_{g}$ minus $w+1$ points, where $w$ is a nonnegative integer and $C_{g}$ is a nonsigular projetive curve of genus $g$ over $k$, to be the set of all finite groups which can be realized as Galois groups of finite unramified coverings of $C_{g,w}$. In tha paper, I went on to \textit{conjecture} that $\pi_{A}(C_{g,w})$ coincides with the set of all finite groups $G$ for which $G/p(G)$ is generated by $2g+w$ generators, where $p(G)$ is the subgroup of $G$ generated by all its $p$-Sylow subgroups. The $g=w=0$ case of this conjecture, which may be called the \textit{quasi p-group conjecuture}, says that for the affine line $L_{k}$ over $k$ we have $\pi_{A}(L_{k})=Q(p)$ where $Q(p)$ denotes the set of all quasi $p$-groups, i.e., finite groups which are generated by their $p$-Sylow subgroups. It may be noted every finite simple group whose order is divisible bey $p$ is obviously a quasi $p$-group. Hence in particular the alternating group $A_{n}$ is a quasi $p$-group whenever either $n\geq p > 2$ or $n-3\geq p=2$. Likewise the symmetric group $S_{n}$ is a quasi $p$-group provided $n\geq p=2$.   

In support of the quasi $p$-group conjecture,in the 1957 paper. I wrote down several equations giving unramified coveing of the affine line $L_{k}$ and suggested that their Galois groups be computed. This included the equation $\overline{F}_{n,q,s,a}=0$ with
$$
\overline{F}_{n,q,s,a}=Y^{n}-aX^{s}Y^{t} + 1 \quad {\rm and} \quad n=q+t
$$
where $0\neq a \in k$ and $q$ is a positive power of $p$ and $s$ and $t$ are positive integers with $t \nequiv 0(p)$, and we want to compute its Galois group $\overline{G}_{n,q,s,q} = {\rm Gal}(\overline{F}_{n,q,s,q}, k(X))$.

By using a tiny amount of the information contained in the above equation, I showed that $\pi_{A}(L_{k})$ contains many unasolvable groups, and indeed by taking homomorphic imagtes of subgroups of members of $\pi_{A}(\bL_{k})$ we get all finite groups; see Result 4 and Remark 6 on pages 841-842 of \cite{art1-key3}. This was somewhat of a surpise because the comples affine line is simply connected, and although $\pi_{A}(\bL_{k})$ was known to contain $p$-cyclic groups (so called Artin-Schreier equations), it was felt that perhaps it does not contain much more. This feeling, which turned out to be wrong, might have been based on the facts that $L_{k}$ is a ``commutative group variety" and the fundamental group of a topological group is always abelian; see Proposition 7 on page 54 of Chevalley \cite{art1-key26}.

To algebracize the fact that the comples affine line is simply connected, be the genus formula we deduce that the affine line over an algebraically closed ground field of characteristic zero has no nontrivial unramified coverings. In our case of characteristic $p$, the same formula shows that every membed of $\pi_{A}(\bL_{k})$ is a squasi $p$-group; see Result 4 on page 841 of \cite{art1-key3}.

Originally I found the above equation $\overline{F}_{n,q,s,a}=0$ by taking a section of a surface which I had constructed in my 1955 Ph.D. Thesis \cite{art1-key1} to show that jung's classical method \cite{art1-key40} of surface desigularization doed not work for nonzero charactheristic because the local fundamental group above a normal crossing of the branch locus need not be solvable, while in the comples case it is always abelian. This failure of Jung's method led me ti devise more algoprithmic techniques fr desingularizing surfaces in nonzero characteristic, and this formed the positive part of my Ph.D. Thesis \cite{art1-key2}.

Soon after the 1957 paper, I wrote a series of articles \cite{art1-key4} on ``tame coverings" of higher dimensional algebraic varieties, and took note of Grothendieck \cite{art1-key33} proving the ``tame part" of the above conjecture which says that the members of $\pi_{A}(C_{g,w})$ whose order is prime to $p$ are exavtly all the finite gropus of order prime to $p$ generated by $2g+w$ generators.

But after these two things,for a long time I forgot all about covering and fundamental groups.

Then suddenly, after a lapse of nearly thirty years, Serre pulled me back into the game in October 1988 by writing to me a series of letters in which be briefly said: ``I can now show that if $t=1$ then $\overline{G}_{n,q,s,a}=PSL(2,q)$.
Can you compute $\overline{G}_{n,q,s,a}$ for $f\geq 2$? Also, can you find unramified $A_{n}$ coverings of $L_{k}$?"

Strangely, the answers to both these questions turned out to be almost the same. Namely, with much prodding and prompting by Serre (hundred e-mails and a dozen s-mails=snail-mails) augmented by groups theory lessons first from Kantor and Feit and then Cameron and O'Nan, and by using the method of throwing away roots, CT in the guise of CDT, the Cycle Lemmas, the Jordan-Marggraff Theorems on limits of transitivity (see Jordan \cite{art1-key39} and Marggraff \cite{art1-key45} or Wielandt \cite{art1-key60}), and finally some High-School type factorizations, in the papers \cite{art1-key8} to \cite{art1-key10} I proved that: 
\begin{align}
t&=1 \Rightarrow \overline{G}_{n,q,s,a} = PSL(2,q).\label{art1-eq6.1}\\
q&=p > 2 \leq t\; {\rm and} \;(p,t)\neq (7,2) \Rightarrow \overline{G}_{n,q,s,a} = A_{n}.\label{art1-eq6.2}\\
q&=p > 2 \leq t \;{\rm and} \;(p,t) = (7,2) \Rightarrow \overline{G}_{n,q,s,a} = PSL(2,8).\label{art1-eq6.3}\\
q&=p = 2 \Rightarrow \overline{G}_{n,q,s,a} = S_{n}.\label{art1-eq6.4}\\
q&=p > 2 > {\rm and}\; (p,t) \neq (7,2) \Rightarrow \overline{G}_{n,q,s,a} = A_{n}.\label{art1-eq6.5}\\
p&=2 < q < t \Rightarrow \overline{G}_{n,q,s,a} = A_{n}.\label{art1-eq6.6}\\
p&=2 < q = 4 {\rm and} \; t=3 ({\rm and} n=7) \Rightarrow \overline{G}_{n,q,s,a} = PSL(3,2).\label{art1-eq6.7}\\
p&=3 < q = 9 {\rm and} \; t=2 ({\rm and} n=11) \Rightarrow \overline{G}_{n,q,s,a} =M_{11}.\label{art1-eq6.8}
\end{align}
Note that $PSL(m.q)=SL(m,q)/{\rm(scalar matrices)}$ where $SL(m,q)=$ The group of all $m$ by $m$ matrices whose determinant is 1 and whose entries are in the field GF$(q)$ of $q$ elements. Now my proof of \eqref{art1-eq6.1} uses the Zassenhaus-Feit-Suzuki Theorem which characterizes doubly transitive permutation groups for which no 3 points are fixed by a nonidentity permutation; see Zassenhaus \cite{art1-key62}, Feit \cite{art1-key30} and Suzuki \cite{art1-key59}. As Serre has remarked, his proof of \eqref{art1-eq6.1} may be called a ``descending" proof as opposed to my ``ascending" proof. Serre's proof may be found in his November 1990 letter to me which appears as an Appendix to my paper \cite{art1-key8}. Actually, when \cite{art1-key8} was already in press, Serre found that a proof somewhat similar to his was already given by Carlitz \cite{art1-key25} in 1956.  

Throwing away one root of $\overline{F}_{n,q,s,a}$ and then applying Abhyankar's Lemma (see pages 181-186) of Part III of \cite{art1-key4}) and deforming things conveniently, we get the monic polynomial $\overline{F}_{n,q,s,a,b,u}$ of degree $n-1$ in $Y$ with coefficient in $k(x)$ given by
$$
\overline{F}'_{n,q,s,a,b,u}=t^{-2}\left[(Y+t)^t-Y^{t}\right](Y+b)^q-aX^{-s}Y^{u}
$$
with $0 \neq b \in k$ and positive integer $u,n-1$. Now upon letting
$$
r=(q+t){\rm LCM} \left(t,\dfrac{q-1}{\GCD (q-1, q+t)}\right)
$$
and $\overline{G}'_{n,q,s,a,b,u} = \Gal(\overline{F}'_{n,q,s,a,b,u}, k(X))$, in the papers \cite{art1-key8} and \cite{art1-key10} I also proved that, in the following cases, $\overline{F}'_{n,q,s,a,b,u}=0$ gives an unramified covering of $L_{k}$ with the indicated Galois group:
\begin{itemize}
\item[{\rm 6.1$'$}]$b = u = t > 2 \neq q = p$ and $s \equiv 0(p-1)$ and $s \equiv 0(t) \Rightarrow$ $\overline{G}_{n,q,s,a,b,u} = A_{n-1}$.
\item[{\rm 6.2$'$}] $b = u = t = 2$ and $q = p \neq 7$ and $s\equiv 0(p-1) \Rightarrow$
 $\overline{G}'_{n,q,s,a,b,u} = A_{n-1}$.
\item[{\rm 6.3$'$ }] If $t =2$ and $q = p > 5$ then {\it u} can be chosen so that $1<u<(p+1)/2$ 
and $\GCD(p+1,u)=1$, and for any such {\it u} upon assuming
$b=u/(u-1)$ and $s \equiv 0(u(p+1-u))$, we have $\overline{G}'_{n,q,s,a,b,u} =A_{n-1}$.
\item[{\rm 6.4$'$}] $b=u=t$ and $q=p=2$ and $s\equiv 0(t) \Rightarrow \overline{G}'_{n,q,s,a,b,u} = S_{n-1}$.
\item[{\rm 6.5$'$}] $b = u = t > q$ and $p > 2$ and $s \equiv 0(r)\Rightarrow \overline{G}'_{n,q,s,a,b,u} = A_{n-1}$.
\item[{\rm 6.6$'$}] $b = u = t > q = p = 2$ and $s \equiv 0(r)\Rightarrow \overline{G}'_{n,q,s,a,b,u} = S_{n-1}$.
\item[{\rm 6.7$'$}] $b = u = t > q > p = 2$ and $s \equiv 0(r)\Rightarrow \overline{G}'_{n,q,s,a,b,u} = A_{n-1}$.
\end{itemize}

Another equation written down in the 1957 paper giving an unramified covering of $L_{k}$ is $\widetilde{F}_{n,t,s,a}=0$ where $n,t,s$ are positive integers with
$$
t < n \equiv 0(p)\; {\rm and}\; \GCD(n,t)=1\; {\rm and}\; s \equiv 0(t) 
$$
and $\widetilde{F}_{n,t,s,a}$ in the polynomial given by
$$
\widetilde{F}_{n,t,s,a}=Y^{n}-aY^{t}+X^{s}\; {\rm with}\; 0\neq a \in k. 
$$
Again upon letting $\widetilde{F}_{n,t,s,a}=\Gal(\widetilde{F}_{n,t,s,a}, k(X))$, in the papers \cite{art1-key8} and \cite{art1-key10} I proved that:
\begin{itemize}
\item[{\rm (6.1*)}] $1 < t < 4$ and $p\neq 2 \Rightarrow \widetilde{G}_{n,t,s,a}= A_{n}$.
\item[{\rm (6.2*)}] $1 < t < n-3$ and $p \neq 2 \Rightarrow \widetilde{G}_{n,t,s,a}= A_{n}$.
\item[{\rm (6.3*)}] $1 < t = n-3$ and $p \neq 2$ and $11\neq p \neq 23 \Rightarrow \widetilde{G}_{n,t,s,a}= A_{n}$.
\item[{\rm (6.4*)}] $1 < t < 4 < n$ and $p = 2 \Rightarrow \widetilde{G}_{n,t,s,a}= A_{n}$.
\item[{\rm (6.5*)}] $1 < t < n-3$ and $p = 2 \Rightarrow \widetilde{G}_{n,t,s,a}= A_{n}$.
\end{itemize} 
I  proposition 1 of the 1957 paper I discussed the polynomial

$Y^{hp+t} +aXY^{hp} + 1 + \sum\limits_{i=1}^{h-1}a_{i}Y^{(h-i)p}$ with $t\equiv 0(p)$ and $0 \neq a \in k$  and $a_{i} \in k $ giving an unramified covering of $L_{k}$. The polynomial $\overline{F}$ studied in (6.1) to (6.6) is the $hp=q $ and $ a_{1}= \ldots = a_{h-1} =0$ ace of this after ``reciprocating" the roots and changing $X$ to $X^{s}$. Considering the $p=2 =h = t-1$ and $a = a_{1} = 1$ case this we get the polynomial. 
$$
F^{\circ} =Y^{7} +xY^{4}+Y^{2}+1
$$
and it can be shown that:

$(6.1^{\circ})$ For $p=2$ the equation $F^{\circ}=0$ gives an unramified coveing of $L_{k}$ with $\Gal(F^{\circ}, k(X))=A_{7}$.

By throwing away a root of $f^{\circ}$ and then invoking Abhyankar's Lemma we obtain the polynomial
$$ 
F'^{\circ} =Y^{6} +X^{27}Y^{5}+ x^{54}Y^{4} +(X^{18}+X^{36})Y^{3}+X^{108}Y^{2} +(X^{90}+X^{135})Y + X^{162}
$$
and therefore by $(6.1^{\circ})$ we see that:

$(6.2^{\circ})$ for $p=2$ the equation  $F'^{\circ}=0$ gives an unramified covering of $L_{k}$ with $\Gal(F'^{\circ}, k(X))=A_{6}$.

Now it was Serre who first propted me to use CT in calculating the various Galois groups discsussed above. But after I had done this, agin it was Serre who groups discussed above. Bit after I had done this, agin it was Serre who prodded me to try to get around CT. So, as described in the papers \cite{art1-key8} and \cite{art1-key10}, by traversing as suitable path in items (6.1) to (6.2$^{\circ}$) we get a \textit{complete equational proof} of the following Facts \textit{without CT:}

\medskip
\noindent
\textbf{Facts.} (6.i) For all $n \geq p > 2$ we have $A_{n}\in \pi_{A}(L_{k})$ . (6.ii) For all $n\geq p = 2$ we have $S_{n} \in \pi_{A}(L_{k})$. (6.iii) For $n \geq p =2$ a with $3\neq n \neq 4$ we have $A_{n}\in\pi_{A}(L_{k})$: (note that $A_{3}$ and $A_{4}$ are not quasi 2-groups).

While attempting to circumvent CT, once I got a very amusing e-mail from Serre saying `About the \textbf{essential} removal of CT from your $A_{n}$-determinations: what does essential mean? (Old story: a noble man had a statue of himself made be a well-known sculptor. The sculptor asked: do you want an equestrian statue or not? The noble man did not understand the word. He said: oh, yes, equestrian if you want, but not too much$\ldots$). This is what I feel about \textit{non essential use of CT.}"

 In any case, learnings and adopting (or adapting) all this group theory has certainly been very rejuvenating to me. To state CT very briefly: $Z_{p}$ (= the cyclic group of prime order $p$), $A_{n}$ (excluding $n\leq 4$), $\PSL(n+1, q)$
(excluding $n=1$ and $q \leq 3$) together with 15 other related and reincarnated infinite families, and the 26 \textit{sporadics} including the 5 Mathieus is a complete list of finite simple groups; for details see Abhyankar \cite{art1-key8} and Gorenstein \cite{art1-key32}.

\section{Jacobson and Berlekamp}

Concerning items (6.6), (6.7'), (6.4*) and (6.5*), when I said that I proved them in \cite{art1-key8} and \cite{art1-key10}, what I actually meant was that I proved their weaker version asserting that the Galois group is the alternating group of the symmetric group, and then thanks to Jacobson's Criterion,the symmetric group possibility was eliminated in my joint paper Ou and Sathaye \cite{art-key14}. What I am saying is that the classical criterion, according to which the Galois group of an equation is contained in the alternating two. A version of such a criterion which is valid for all characteristics including two was given by Jacobason in this Algebra books published in 1964 \cite{art1-key37} and 1974 \cite{art1-key38}; an essentially equivalent version may also be found in the 1976 paper \cite{art1-key20} if Berlekamp with some preliminary work in his 1968 book \cite{art1-key19}; both these criteria have a bearing on the Arf invariant of a quadratic form \cite{art1-key18} which itself was inspired by some work of Witt \cite{art1-key61}. This takes care of $A_{n}$ coverings for characteristic two provided $ 6 \neq n \neq 7$. 

This leaves us with the $A_{6}$ and $A_{7}$ coverings for characteristic two described in items (6.2$^\circ$) and (6.1$^\circ$). Again using the Jacobson's Criterion, these are dealt with in my joint paper with Yie \cite{art1-key17}.

Thus, although Facts (6.i) and (6.ii) are indeed completely proved in my papers \cite{art1-key8} and \cite{art1-key10}, but for Fact (6.iii) I was lucky to enjoy the active collaboration of my former (Sathaye) and present (Ou and Yie) students.

Likewise, item (6.7) is not proved in my papers \cite{art1-key8} to \cite{art1-key10}, but was communicated to me by Serre (e-mail of October 1991) and is included in my joint paper \cite{art1-key17} with Yie. Similarly, item (6.8) is not in my papers \cite{art1-key8} to \cite{art1-key10}, but is proved in my joint paper \cite{art1-key15} with Popp and Seiler.

\medskip
In \cite{art1-key8} I used polynomial $\widetilde{F}$ into the polynomial  $\overline{F}$ and thereby get a proof of a stronger version of (6.1*) to (6.5*) without CT. Let us start by modifying the Third Irreducibility Lemma i Section 19 of \cite{art1-key8} thus [in the proof of that lemma,once $\xi_{\lambda}(1, Z)$ has been mistakenly printed as $\xi_{\lambda(1, Z)}$ ]:

\subsection{}

Let $k$ be any field which need not be algebraically closed and whose characteristic $\chc$ $k$ need not be positive. Let $n > t > 1$ be integers such that $\GCD(n,t)=1$ and $t\equiv 0(\chc k)$, and let $\Omega(Z)$ be the monic polynomial of degree $n-1$ in $Z$ with coefficients in $k(Y)$ obtained by putting
$$
\Omega(Z)= \dfrac{\left[z+Y)^{n}-Y^{n}\right]-\left[Y^{n-t}+Y^{-t}\right]\left[(Z+Y)^{t}-Y^{t}\right]}{Z}.   
$$
Then $\Omega(Z)$ is irreducible in $k(Y)[Z]$.

\noindent
\textbf{Proof.} Since $\nequiv 0(\chc, k)$, upon letting
$$
\xi'_{\lambda'}(Y,Z)=\dfrac{\left[(Z+Y)^n-Y^{n}\right]}{Z} \; {\rm and}\; \eta_{\mu}(Y,Z)=\dfrac{-\left[(Z+Y)^t-Y^{t}\right]}{Z}
$$
by the proof of the above cited lemma we see that: $\xi'_{\lambda'}(Y,Z)$ and $\eta_{\mu}(Y,Z)$ are homogeneous polynomials of degree $\lambda'=n-1$ and $\mu = t-1$ respectively, the polynomials $\xi'_{\lambda'}(1,Z)$ and $\eta_{\mu}(Y,Z)$ have no nonconstant common factor in $k[Z]$, and the polynomial $\eta_{\mu}(Y,Z)$ has a has a nonconstant irreducible factor in $k[Z]$ which does not divide $\xi'_{\lambda'}(1,Z)$ and whose square does not divide $\eta_{\mu}(Y,Z)$. Upon letting 
$$
\xi_{\lambda}(Y,Z)=\dfrac{Y^{t}\left[(Z+Y)^{n}-Y^{n}\right]-Y^{n}\left[(Z+Y)^{t}-Y^{t}\right]}{Z}
$$
we see that $\xi_{\lambda}(Y,Z)$ is a homogeneous polynomail of degree $\lambda =n+t-1$ and $\lambda(1,Z)= \xi'_{\lambda'}(Y,Z)+\eta_{\mu}(1,Z)$ and therefore: the polynomials $\xi_{\lambda}(Y,Z)$ and $\eta_{\mu}(1,Z)$ have no  nonconstant commot fator in $K[Z]$, and the polynomial $\eta_{\mu}(1,Z)$ has na nonconstant irreducible factor in $k[Z]$ which does not divide $\xi_{\lambda}(1,Z)$ and whose square does not divide $\eta_{\mu}(1,Z)$. The proof of the Second Irreducibilitiy Lemma, of Section 19 of \cite{art1-key8} clearly remains valid if only one of the polynomials $\xi_{\lambda}(Y,Z)$ and $\eta_{\mu}(Y,Z)$ is assumed to be regular in $Z$, and in the present situation $\eta_{\mu}(Y,Z)$ is obviously regular in $A$. Therefore by the said lemma, the polynomial $\xi_{\lambda}(Y,Z) + \eta_{\mu}(Y,Z)$  is irreducible $k(Y)[Z]$. Obviously $\Omega(Z)=Y^{-t}\left[\xi_{\lambda}(Y,Z) + \eta_{\mu}(Y,Z)\right]$ and hence $\Omega(Z)$ is irreducible in $k(Y)[Z]$.

By using (7.1) we shall now prove:

\subsection{}

Let $k$ be any field which need not be algebraically closed and whose characteristifc $\chc$ $k$  need not be positive. Let $0 \neq a\in k$ and let $n, t, s$ be positive integers such that $1 < t\nequiv 0(\chc k)$ and $1 < n-t \nequiv 0(\chc k)$ and $\GCD(n, t) =1$. Then the polynomial $\Phi(Y)=Y^{n}-aX^{s}Y^{t}+1$ is irreducible in $k(X)[Y]$, its $y$-discriminant is nonzero, and for its Galois group we have: $\Gal(\Phi(Y), k(X))=A_{n}$ or $S_{n}$. Similarly, the polynomial $\Psi(Y)=Y^{n}-aY^{t}+X^{x}$ is irreducible in $k(X)[Y]$, its $Y$-discriminant is nonzero, and for its Galois group we have: $\Gal(\Psi(Y), k(X))=A_{n}$ or $S_{n}$.

\noindent
\textbf{Proof:} In view of the Basic Extension Principle and Corollaries (3.2) and (3.5) of the Substitutional Principle of Sections 19 of \cite{art1-key8}, without loss of generality we may assume that $k$ is algebraically closed and $a=1 =s$. Since $\Phi$ and $\Psi$ are linear in $X$, they are irreducible. By the discriminant calculation in Section 20 of \cite{art1-key8} we see that their $Y$-discriminants are nonzero. As in the beginning of section 21 of \cite{art1-key8} we see that the valuation $X=\infty$ of $k(X)/k$ splits into two valuations in the rood field of $\Psi(Y)$ and their reduced  ramification exponents are $t$ and $n-t$. Now $t$ and $n-t$ are both nondivisible by $\chc k$ and $\GCD(t,n-t)=1$, and hence by the Cycle Lemma of Section 19 of \cite{art-key8} we conclude that $Gal(\Phi(Y), k(X))$ contains a $t$-cycle and an $(n-t)$-cycle. By throwing away a root of $\Phi(Y)$ we get $\left[\Phi(Z+Y)-\Phi(Y)\right]/Z$ which equals $\Omega(Z)$ because by solving $\Phi(Y)=0$ we get $X=Y^{n-t}+Y^{-t}$. Consequently by (7.1) we Conclude that $\Gal(\Phi(Y)mn k(X))$ is double transitive. Clearly either $1<t<(n/2)$ or $1<n-t<(n/2)$, and hence by Marggraff's Second Theorem as stated in Section 20 of \cite{art1-key8} we get $\Gal(\Phi(Y), k(X)) = A_{n}$ or $S_{n}$. By Corollaries (3.2) and (3.5) of the Substitutional Principle of section 19 of \cite{art1-key8} it now follows that $\Gal(Y^{n}-X^{t-n}Y^{t}+1, k(X))=A_{n}$ or $S_{n}$. By multiplying throughout by $x^{n}$, we obtain the polynomial $Y^{n}-Y^{t}+X^{n}$ whose Galois group must be the same as the Galois group of $Y^{n}-X^{t-n}Y^{t}+1$. Therefore $\Gal(Y^{n}-Y^{t}+X^{n}, k(X))=A_{n}$ or $S_{n}$, and hence again by Corollaries (3.2) and (3.5) of the Substitutional Principle of Section 19 of \cite{art1-key8} we conclude that $\Gal(\Psi(Y), k(X)) =A_{n}$ or $S_{n}$.  

To get back to the polynomial $\widetilde{F}_{n,t,s,a},$ let us return to the assumption of $k$ being an algebraically closed field of nonzero characteristic $p$. Let $0\neq a \in k$ and let $n, t, s$ be positive integers with
$$
t < n \equiv 0(p)\; {\rm and}\; \GCD(n,t) = 1\; {\rm and}\; s\equiv 0(t)
$$
and recall that $\widetilde{F}_{n,t,s,a} = 0$ gives an unramified covering of $L_{k}$ where
$$
\widetilde{F}_{n,t,s,a} =Y^{n}-aY^{t}+X^{s}
$$
and we want to consider the Galois group $\widetilde{G}_{n,t,s,a} = \Gal(widetilde{F}_{n,t,s,a}, k(X))$. Since every member of $\pi_{4}(L_{k})$ is a quasi $p$-group and since $S_{n}$ in not a quasi $p$-group for $p geq 3$, in view of (2.28) of \cite{art1-key14}, by the $\Psi$ case of (7.2) we get the following sharper version of (6.1*) to (6.5*):
\begin{itemize}
\item[{\rm (7.1*)}] $l < t < n-1 \Rightarrow \widetilde{G}_{n,t,s,a} =A_{n}$.
\end{itemize}

Just as the samall \textit{border values} of $t$ play a \textit{special role} for the bar polynomial in (6.1) to (6.8), likewise the \textit{condition} $1\neq t \neq n-1$ in (7.1*) in \textit{not accidental} as shown by the following four assertions:
\begin{itemize}
\item[{\rm (7.2*)}] $1 =t =n-5$ and $ p=2\Rightarrow \widetilde{G}_{n,t,s,a} =PSL(2,5) \approx A_{5}$.
\item[{\rm (7.3*)}] $5=t=n-1$ and $ p=2\Rightarrow \widetilde{G}_{n,t,s,a} =PSL(2,5) \approx A_{5}$.
\item[{\rm (7.4*)}] $1=t=n-11$ and $p=3 \Rightarrow \widetilde{G}_{n,t,s,a} = \widehat{M}_{11}\approx M_{11}$.
\item[{\rm (7.5*)}] $1=t$ and $n=p^{m} \Rightarrow \widetilde{G}_{n,t,s,a} =(Z_{p})^m$. 
\end{itemize}

Out of these four assertions, (7.2*) and (7.3*) may be found in my joint paper \cite{art1-key17} with Yie, and (7.4*) may be found in my joint paper \cite{art1-key15} with Popp and Seiler. If may be noted that $PSL(2,5)$ and $A_{5}$ are isomorphic as abstract groups but not as permutation groups. Likewise $\widehat{M}_{11}$ found by taking the image of the $M_{11}$ found by taking the images of the $M_{11}$ under a noninner automorphism of $M_{11}$ found by taking the image of the $M_{11}$ found by taking the image of the $M_{11}$ under a noninner automorphism of $M_{12}$; see my paper \cite{art1-key8} or volume III of the encyclopedic groups theory book \cite{art1-key36} of Huppert and Blackburn. 

Concering assertion (7.5*), multiplying the roots by a suitable nonzero element of $k$ we can reduce to the case of $t=1=a$ and $n=p^{m}$ and then, remembering that $(Z_{p})^m =$ the $m$-fold direct product of the cyclic group $Z_{p}$ of order $p=$ the underlying additive group of $\GF(p^m)$, our claim follows from the following remark: 

(7.1**) If $Y^{p^{m}}-Y+x$ is irreducible over a field $k$ of characteristic $p$, with $x\in K$ and $\GF(p^m) \subset K$, then by taking a root $y$ of $Y^{p^{m}}-Y+x$ we have $Y^{p^{m}}-Y+x = \prod_{i\in \GF(p^m)}[Y-(y+i)]$ and hence exactly as in the $p$-cyclic case we get $\Gal\left(Y^{p^{m}}-Y +x, K\right) = (z_{p})^m$. 

By throwing away a root of $\widetilde{F'}_{n,t,s,a}$ of degree $n-1$ in $Y$ with coefficients in $k(X)$ given by
$$
\widetilde{F'}_{n,t,s,a} = Y^{-1}\left[(Y+1)^{n} -1\right]- aX^{-s}Y^{-1}\left[(Y+1)^{t}-1\right]
$$
and for its Galois group $\widetilde{G'}_{n,t,s,a} = \Gal\left(\widetilde{F'}_{n,t,s,a}, k(X)\right)$, in my joint paper \cite{art1-key15} with Popp and Seiler it is shown that:

(7.1$'$) $1=t=n-11$ and $p=3$ and $S\equiv 0(n-1) \Rightarrow \widetilde{G'}_{n,t,s,a} = PSL(2,11)$, where, for the said values of the parameters, the equation $ \widetilde{G'}_{n,t,s,a}=0$ gives an unramified covering of $L_{k}$.

Finally let
$$
\overline{F}^{(d)}_{n,q,s,a,b} = Y^{dn}-aX^{x}Y^{dt} + b\; {\rm with\; positive\; integer}\; d\nequiv 0(p)
$$ 
where once again $a,b$ are nonzero elements of $k$ and $n,t,s$ are positive integers with $t<n$ and $\GCD(n,t)=1$ and
$n-t=q=a$ positive power of $p$. For the Galois group $\overline{G}^{(d)}_{n,q,s,a,b} = \Gal(\overline{F}^{(d)}_{n,q,s,a,b}, k(X))$, in my joint paper \cite{art1-key15} with Popp and Seiler it is shown that:

(7.2$'$) $d=t=2=n-9$ and $q=9$ and $p=3 \Rightarrow \overline{G}^{(d)}_{n,q,s,a,b} =M^{*}_{11}\approx M_{11}$, where, for the said values of the parameters, the equation  $\overline{G}^{(d)}_{n,q,s,a,b} = 0$ gives an unramified covering of $L_{k}$.

Agin note that $M^{*}_{11}$ and $M_{11}$ are isomorphic as abstract groups but not asa permutations groups; here $M^{*}_{11}$ is the transitive but not 2-transitive incarnation of $M_{11}$ obtained by considering its cosets according to the index 22 subgroup $PSL(2,9)$.

\section{Grothendieck}

Having dropped my prejudice against Statistics, it is high time to show my appreciation of Grothendieck.

For example by using the (very existential and highly nonequational) work of Grothendieck \cite{art1-key33} on tame coverings of curves, in \cite{art1-key7} and \cite{art1-key11} I have shown that for any pairswise nonisomorphic nonabelian finite simple groups $D_{1},\ldots, D_{u}$ with $|\Aut D_{u}| \nequiv 0(p)$ the \textit{wreath product} $(D_{1}\times\cdots\times D_{u})$ Wr $Z_{p}$ belongs to $\pi_{A}(L_{k})$. For instance we may take $D_{1}= A_{m_{1}}, \ldots D_{u} =A_{m_{u}}$ with $4 < M_{1} < M_{2} < \ldots < M_{u} < p$. 

Actually, using Grothendieck \cite{art1-key33} I first prove an Enlargement Theorem and then from i deduce the above result about wreath products as a group theoretic consequence. The \textit{Enlargements Theorem} asserts that if $\Theta$ is any $\pi_{A}(L_{k})$, then some enlargement of $\Theta$ by $J$ belongs to $\pi_{A}(L_{k})$. 

Now enlargement is a generalization of group extensions. Namely, an \textit{enlargement} of an group $\Theta$ by a group $J$ is group $G$ together with an exact sequence $1 \rightarrow H \rightarrow J \rightarrow 1 $ and a normal subgroup $\Delta$ of $\lambda(H)$, where $\lambda$ is the given map of $H$ into $G$, such that $\lambda(H)/\Delta$ is isomorphic to  $\Theta$ and no nonidentity normal subgroup of $G$ is contained in $\Delta$. Note that here $G$ is an \textit{extension} of $H$ by $J$. The motivation behind enlargements is the fact that a Galois extensions of A Galois extension need not be Galois and if we pass to the relevant least Galois extension then its Galois group is an enlargement of the second Galois group by the first.

Talking of group extensions, as a striking consequence of CT it can be seen that the direct product of two finite nonabelian simple groups is the only extensions of one by the other. Here the relevant direct consequence of CT is the \textit{Schreier Conjecture} which which says that the outer automorphism group of any finite nonabelian simple group is solvable; see Abhyankar \cite{art1-key11} and Gorenstein \cite{art1-key32}.

As another interesting result, in \cite{art1-key12} I proved that $\pi_{A}(L_{k})$ is closed with respect to direct products. it should also be noted that Nori \cite{art1-key49} has shown that $\pi_{A}(L_{k})$ contains $\SL(n,p^m)$ and some other Lie type simple groups of characteristic $p$.

Returning to Grotherndieckian techniques,Serre \cite{art1-key56} proved that if $\pi_{A}(L_{k})$ contains a group $H$ then it contains every quasi $p$-group which is an extension of $H$ by a solvable group $J$. 

Indeed it appears that the ongoing work of Harbarter and Raynaud using Grothendieckian techniques is likely to produce \textit{existential} proofs of the quasi $p$-group conjecture.

But it seems worthwhile to march on with the equational concrete approach at least because it gives results over the prime field $\GF(p)$ and also because we still have no idea what tha \textit{complete algebraic fundamental group} $\pi^{C}_{A}(L_{k})$ looks like where $\pi^{C}_{A}(L_{k})$ is the Galois group over $k(X)$ of the compositum of all finite Galois extensions of $k(X)$ which are ramified only at infinity and which are contained in a fixed algebraic closure of $k(X)$. 

\section{Ramanujan}

In the equational approach . ``modula" things seem destined to plays a significant role. For instance the Carlitz-Serre construction $\PSL(2,q)$ coverings and Serre's alternative proof \cite{art1-key57} that $\overline{G}_{n,q,s,q} = \PSL(2,8)$ for $q=p=7$ and $n=9$, are both modular. Similarly my joint paper \cite{art1-key16} with Popp and Seiler which uses the Klein and Macbeath curves for writing down $\PSL (2,7)$ and $\PSL (2,8)$  coverings for small characteristic is also modular in nature. 

Inspired by all this, I am undertaking the project of browsing in the 2 volume treatise of Klein and Fricke \cite{art1-key43} on Elliptic Modular Functions to prepare mysely for understanding Ramanujan himself who may be called the king of Things Modular, where Things = Funcitons, equations, Mode of Thought or what have you; see Ramanujan's Collected Papers \cite{art1-key50} and Ramanujan Revisited \cite{art1-key15}. 

To explain what are moduli varieties and modular funtions in a very naive but friendly manner: The discriminant $b^{2}-4ac$ of a quadratic $aY^{2}+bY+c$ is the oldest known invariant. Coming to cubics or quartics $aY^{a} + bY^{3} +cY^{2} +e$ we can, as in books on theory of equations, consider \textit{algebraic invariants}, i.e., polynomial functional of $a,b,c,d,e$ which do not change (much) when we change $Y$ by a fractional linear transformation (see my Invariant Theory Paper \cite{art1-key13}), or we may consider \textit{transcendenatal invariants} and then essentially we ge ellipatic modular functions. More generally we may consider several (homogeneous) polynomials in several (set-of) variables; when thought of as funcitions of the variables they give us algebraic varieties or multi-periodic functions or abelian varieties and so on; but as functions of the coeficients we get algebraic invariants or moduli varieties or modular functions. Modular functions and their transforms are related by \textit{modular equations}; thinks of the expansion of $\sin n\theta$ in terms in $\sin \theta$!

But postponing this to another lecture on another day, let me end with a few equational questions suggested by the experimental data presented in this lecture.
\begin{question}\label{art1-qus9.1}
Which quasi $p$-groups can be obtained by \textit{coverings of a line by a line}? In other words, which quasi $p$-groups are the Galois groups of $f$ over $k(X)$ for some monic polynomial $f$ in $Y$ with coefficients in $k[X]$ such that $f$ is linear in $X$, no valuation of $k(X)/k$ is ramified in the splitting field of $f$ other than the valuations $X=0$ and $X=\infty$, and the may even allow the given quasi $p$-group to equal $p(\Gal(F, k(X)))$ ; note that $\Gal(f, k(X))/p(\Gal(f,k(X)))$ is necessarily a cyclic group of order prime to $p$. In any case this prime to $p$ cyclic quotient as well as the tame branch point at $X-0$ can be removed be Abhyankar's Lemma. [Hoped for Answer: many quasi $p$-groups if not all].
\end{question}

\begin{question}\label{art1-qus9.2}
Do \textit{fewnomilas} suffice for all simple quasi $p$-groups? Etymology: binomial, trinomial, \ldots, fewnomial. In other words,is there a positive integer $d$ (hopefully small) such that every simple quasi $p$-group can be realized as the Galois group of a polynomial $f$ containing at most $d$ terms in $Y$ (more precisely, at most $d$ monomials i $Y$)
 with coefficients in $k[X]$ which gives an unramified covering of the $X$-axis $L_{k}$? Indeed,do fewnomias suffice for most (if not all) quasi $p$-groups (without requiring them to be simpel)? If not, then do \textit{sparanomials} suffice for most (if not all) quasi $p$-groups? Etymology: sparnomial = spare polynomial in $Y$ plus a polynomial in $Y^{p}$.
\end{question}

\begin{question}\label{art1-qus9.3}
Which quasi $p$-groups can be realized as Galois groups of polynomials in $Y$ whose coeffcients are polynomials in $X$ over the \textit{prime field} $GF(p)$ such that no valuation of $\GF(p)[X]$ is ramified in the relevant splitting field and such tha $\GF(p)$ is relatively algebraically closed in the splitting field and such that $\GF(p)$ is relatively. algebraiclally closed in that splitting field? Same question where we drop the condition of $\GF(p)$ begin relatively algebraically closed but where we repalce quasi $p$-groups by finite groups  $G$ for which $G/p(G)$ is cyclic. For instance, given any positive power $q'$ of any prime $p'$ such that the order of $\PSL(2,q')$ is divisible by $p$, we may ask whether there exists an unramified covering of the affine line over $\GF(p)$ whose Galois group is the \textit{semidirect} product of $PSL(2,q')$ with $\Aut(\GF(q'))$, i.e., equivalently, whether there exists a polynomial in $Y$ over $\GF(p)[X]$, with Galois group the said semidirect product, such that no valuation of $\GF(p)[X]$ is ramified in the relevant splitting field (without requiring $\GF(p)$ to be relatively algebraically closed in that splitting field). Note that, in \cite{art1-key9}, this last question has been answered affirmatively for $p=7$ and $q' =8$. Also note that for even $q'$ the said semidirect product is the \textit{projective semilinear group} $P\Gamma L(2,q')$, whereas for odd $q'$ it is an index 2 subgroup of $P\Gamma L(2,q')$; for definitions see \cite{art1-key8}.
\end{question}

\begin{question}\label{art1-qus9.4}
Do we get fewer Galois groups if we replace branch locus by \textit{discriminant locus}? For instance, can every quasi
$p$-group be realized as the Galois group of a monic polynomial in $Y$ over $k[X]$ whose $Y$-discriminant is a nonzero element of $k$? Likewise which members of $\pi_{A}(L_{k,w})$, where $L_{k,w} = L_{k}$ minus $w$ points, can be realized as Galois groups of monic polynomials in $Y$ over $k[X]$ whose $Y$-discriminants have no roots other than the assigned $w$ points. We may ask the same thing also for ground fields of characteristic zero.
\end{question}

\begin{question}\label{art1-qus9.5}
Concerining the bar and tilde equations discussed in (6.1) to (6.8) and (7.1*) to (7.5*) respectively, what further interesting groups do we out of the \textit{border values} of $t$?
\end{question}

\begin{question}\label{art1-qus9.6}
Can we describe the \textit{complete algebraic fundamental group} $\pi^{C}_{A}(L_{k})$? More generally, for a nonsigular projective curve $C_{g}$ of genus $g$ minus $w+1$ points, can we describe the complete algebraic fundamental $\pi^{C}_{A}(C_{g,w})$?
\end{question}

\begin{question}\label{art1-qus9.7}
Descriptively speaking, can the ``same" equation give unramified coverings of the affine line for all quasi $p$-groups in the ``same family" of groups? For instance, $Y^{q+1}-XY+1 =0$ gives an unramified covering of the affine line, over a field of characteristic $p$, with Galois group $PSL(2,q)$ for every power $q$ every prime $p$. Now thinking of the larger family of groups $PSL(m,q)$, can be find a `single" equation with integer coefficients, ``depending" on the parameters $m$ and $q$, giving an unramified covering of the affine line, over a field of characteristic $p$, whose Galois group is $PSL(m,q)$ for every integer $m>1$ and every power $q$ of every prime $p$? Can we also arrange that the ``same" equation gives an unramified covering of the affine line, over every field whose characteristic divides the order of $\PSL(m,q)$, whose Galois group is $\PSL(m,q)$? Even more, can we arrange that at the same time the Galois group of that equation over $\overline{\bQ}(X)$ is $\PSL(m,q)$ (but no condition on ramification) where $\overline{\bQ}$ is the algebraic closure of $\bQ$?
\end{question}

\setcounter{note}{7}
\begin{note}\label{art1-note9.8}
Two or more of the above questions can be combined in an obvious manner to formulate more questions.
\end{note}

\begin{thebibliography}{99}
\bibitem{art1-key1} S.S. Abhyankar, \textit{On the ramification of algebraic functions}, Amer. Jour. Math. {\bf 77} (1955) 572-592.
\bibitem{art1-key2} S.S. Abhyankar, \textit{Local uniformization on algebraic surfaces over ground fields of characteristics} $p\neq 0$, Ann. Math. {\bf 63} (1956) 491-526.
\bibitem{art1-key3} S.S. Abhyankar, \textit{Coverings of algebraic curves,} Amer. Jour. Math. {\bf 79} (1957) 825-856.
\bibitem{art1-key4} S.S. Abhyankar, \textit{Tame coverings and fundamental groups of algebraic varieties, Parts I to VI,} Amer.Jour.Math. {\bf 81, 82} (1959-60).
\bibitem{art1-key5} S.S. Abhyankar, \textit{Historical ramblings in algebraic geometry and related algebra,} Amer. Math. Monthly {\bf 83} (1976) 409-448.
\bibitem{art1-key6} S.S. Abhyankar, \textit{Algebraic Geometry for Scientists and Engineers,} Mathematical Surveys and Monographs {\bf 35}, Amer. Math. Soc., Providence, 1990.
\bibitem{art1-key7} S.S. Abhyankar, \textit{Group enlargements,} C. R. Acad. Sci. Paris {\bf 312} (1991) {\bf 763-768}.
\bibitem{art1-key8} S.S. Abhyankar, \textit{Galois theory on the line in nonzero characteristic,} Bull. Amer. Math. Soc. {\bf 27} (1992) 68-133.
\bibitem{art1-key9} S.S. Abhyankar, \textit{Square-root parametrization of plane curves,} Algebraic Geometry and Its Applications, Springer-Verlag (1994) 19-84.
\bibitem{art1-key10} S.S. Abhyankar, \textit{Alternating group coverings of the affine line in characteristic greater than two,} Math. Annalen {\bf 296} (1993) 63-68.
\bibitem{art1-key11} S.S. Abhyankar, \textit{Wreath products and enlargements of groups,} Discrete Math. {\bf 120} (1993) 1-12.
\bibitem{art1-key12} S.S. Abhyankar, \textit{Linear disjointness of polynomials,} Proc. Amer. Math. Soc. {\bf 116} (1992) 7-12.
\bibitem{art1-key13} S.S. Abhyankar, \textit{Invariant theory and enumerative combinatories of Young tableaux,} Geometric Invariance in computer Vision, Edited by J. L. Mundy and A. Zisserman, MIT Press (1992) 45-76.
\bibitem{art1-key14} S.S. Abhyankar, J. Ou and A. Sathaye, \textit{Alternating group coverings of the affine line in characteristic two,} Discrete Mathematics, (To Appear).
\bibitem{art1-key15} S.S. Abhyankar, H. Popp and W. K. Seiler, \textit{Mathieu group coverings of the affine line,} Duke Math. Jour. {\bf 68} (1992) 301-311.
\bibitem{art1-key16} S.S. Abhyankar, H. Popp and W. K. Seiler, \textit{Construction techniques for Galois coverings of the affine line,} Proceedings of the Indian Academy of Sciences {\bf 103} (1993) 103-126.
\bibitem{art1-key17} S.S. Abhyankar and I. Yie, \textit{Small degree coverings of the affine line in characteristic two,} Discrete Mathematics (To Appear).
\bibitem{art1-key18} C. Arf, \textit{Untersuchungen iiber quadratische Formen in K\"orpern der Characteristik 2(Teil I.),} Crelle Journal {\bf 183} (1941) 148-167. 
\bibitem{art1-key19} E. Berlekamp, \textit{Algebraic Codign Theory} McGraw-Hill, 1968.
\bibitem{art1-key20} E. Berlekamp, \textit{An analog to the discriminant over fields of characteristic 2,} Journal of Algebra {\bf 38} (1976) 315-317.
\bibitem{art1-key21} R. C. Bose, \textit{On the construction of balanced incomplete block designs,} Ann, Eugenics {\bf 9} (1939) 353-399.
\bibitem{art1-key22} P.J. Cameron, \textit{Finite permutation groups and finite simple groups,} Bull. Lond. Math. Soc. {\bf 13} (1981) 1-22.
\bibitem{art1-key23} P.J. Cameron. \textit{Permutation groups,} Handbook of Combinatorics, Elsevier, Forthcoming.
\bibitem{art1-key24} P.J. Cameron and J. Cannon, \textit{Fast recongnition of doubly transitive groups,} Journal of Symbolic Computations {\bf 12} (1991) 459-474.
\bibitem{art1-key25} L. Carlitz, \textit{Resolvents of certain linear groups in a finite field,} Canadian Jour. Math.{\bf 8} (1956) 568-579.
\bibitem{art1-key26} C. Chevalley, \textit{Theory of lie Groups,} Princeton University Press, 1946.  
\bibitem{art1-key27} G. Chrystal, \textit{Textbook of Algebra,} Parts I and II, Edinburgh, 1886.
\bibitem{art1-key28} R, Dedekind and H. Weber, \textit{Theorie der Algebraischen Funktionen einer Ver\"anderlichen,} Crelle Joournal {\bf 92} (1882) 181-290.
\bibitem{art1-key29} P. Dembowski, \textit{Finte Geometries}, Spirnger-Verlag, 1968.
\bibitem{art1-key30} W, Feit \textit{On a class of doubly transitive permutation groups,} Illinois Jour. Math. {\bf 4} (1960) 170-186.
\bibitem{art1-key31} A. R. Forsyth, \textit{Theory of Functions of a Complex Variable,} Cambridge University Press, 1918.
\bibitem{art1-key32} D. Gorenstein, \textit{Classifying the finite simple groups,} Bull. Amer. Math. Soc. {\bf 14} (1986) 1-98.
\bibitem{art1-key33} A. Grothendieck, \textit{Rev\^etements \'Etales et Groupe Fondamental (SGA 1),} Lecture Notes in Mathematics {\bf 224} Springer-Verlag, 1971.
\bibitem{art1-key34} D.G. Higman, \textit{Finite permutation groups of rank 3,} Math. Z. {\bf 86} 145-156.
\bibitem{art1-key35} D. Hilbert, \textit{\"Uber das Dirichletesche Prinzip,} Mathematische Annalen {\bf 59} (1904) 161-186.
\bibitem{art1-key36} B. Huppert and N. Blackburn, \textit{Finite Groups I, II, III,} Springer-Verlag, New York, 1982.
\bibitem{art1-key37} N. Jacobson, \textit{Lectures in Abstract Algebra, Vol III,} Van Nostrand, Princeton, 1964.
\bibitem{art1-key38} N. Jacobson, \textit{Basic Algebra, Vol I,} W. H. Freeman and Co. San Francisco, 1974.
\bibitem{art1-key39} C. Jordan, \textit{Sur la limite de transitivit\'e des groupes non altern\'es,} Bull. Soc. Math. France {\bf 1} (1973) 40-71.
\bibitem{art1-key40} H. W. E. Jung,\textit{Darstellung der Funktionen eines algebraischen K\"orperszweier unabh\"angigen Ver\"angigen Ver\"anderlichen in der Umgebung einer Stelle,} Crelle Journal {\bf 133} (1908) 289-314.
\bibitem{art1-key41} W. M. Kantor, \textit{Homogeneous designs and geometric lattices,} Journal of Combinatorial Theory, Series A {\bf 38} (1985) 66-74.
\bibitem{art1-key42} W. M. Kantor and R. L. Liebler, \textit{The rank  3 permutation representations of the finite classical groups,} Trans. Amer. Math. Soc. {\bf 271} (1982) 1-71.
\bibitem{art1-key43} F. Klein and R. Fricke, \textit{Vorlesungen \"uber die Theorie der Elliptischen Modulfunctionen,} 2 volumes, Teubner, Leipzig, 1890.
\bibitem{art1-key44} M. W. Liebeck, \textit{The affine permutation groups of rank three,} Proc. Lond. Math. Soc. {\bf 54} (1987) 447-516.
\bibitem{art1-key45} B. Marggraff, \textit{\"Uber primitive Gruppern mit transitiven Untergruppen geringeren Grades,} Dissertation, Giessen, 1892.
\bibitem{art1-key46} E. Mathieu, \textit{M\'emoire sur l'\'etude des fonctions de plusieurs quantit\'es, sur la mani\`ere de les former, et sur les substitutions qui les laissent invariables,} J. Math Pures Appl. {\bf 18} (1861) 241-323.
\bibitem{art1-key47} I. Newton, \textit{Geometria Analytica} 1680.
\bibitem{art1-key48} M. Noether, \textit{\"Uber einen Satz aus der Theorie der algebraischen Funktionen,} Math. Annalen {\bf 6} (1873) 351-359.
\bibitem{art1-key49} M .V. Nori, \textit{Unramified coverings of the affine line in positive characteristic} Algebraic Geometry and Its Applications, Springer-Verlag (1994) 209-212.
\bibitem{art1-key50} S. Ramanujan, \textit{Collected Papers,} Cambridge University Press, 1927.
\bibitem{art1-key51} S. Ramanujan, \textit{Ramanujan Revisited,} Proceedings of the Centenary Conference at University of Illinois, Edited by G. E. Andrews, R. A. Askey, B. C. Berndt, K. G. Ramanathan and R. A. Rankin, Academic Press, 1987.
\bibitem{art1-key52} B. Riemann, \textit{Grundlagen f\"ur eine allgemeine Theorie der Functionen einer ver\"anderlichen complexen Gr\"osse,} Inauguraldissertation, G\"ottingen (1851). Collected Works, Dover (1953) 3-43.
\bibitem{art1-key53} B. Riemann, \textit{Theorie der Abelschen Funktionen,} Jour. f\"ur die reine undang. Mathematik {\bf 64} (1865) 115-155.
\bibitem{art1-key54} H. Seifert and W. Threlfall, \textit{Lehrbuch der Topologie,} Teubner, 1934.
\bibitem{art1-key55} J-P. Serre, \textit{G\'eom\'etrie alg\'ebrique et g\'eom\'etrie analytique,} Ann. Inst. Fourier {\bf 6} (1956) 1-42.
\bibitem{art1-key56} J-P. Serre, \textit{Construction de rev\^etments \'etales de la droite affine en charact\'eristic p,} C. R. Acad. Sci. Paris {\bf 311} (1990) 341-346.
\bibitem{art1-key57}J-P. Serre, \textit{A letter as an appendix to the squar-root parametrization paper of Abhyankar,} Algebraic Geometry and Its Applications, Springer-Verlag (1994) 85-88.
\bibitem{art1-key58} S. S. Shrikhande, \textit{The impossibility of certain symmetrical balanced incomplete block designs,} Ann. Math. Statist. {\bf 21} (1950) 106-111.
\bibitem{art1-key59} M. Suzuki. \textit{On a class of doubly transitive groups, Ann. Math. {\bf 75} (1962) 104-145}.
\bibitem{art1-key60} H. Wielandt. \textit{Finite Permutation Groups,} Academic Press, New York, 1964.
\bibitem{art1-key61} E. Witt. \textit{Theorie der quadratischen Formen in beliebigen K\"orpern,} Crelle Journal {\bf 176} (1937) 31-41.
\bibitem{art1-key62} H. Zassenhaus, \textit{Kennzeichung endlicher linearen Gruppen als Premutationgruppen,} Abh. Mth. Sem. Univ. Hamburg {\rm (11)} (1936) 17-744.
\end{thebibliography}

\bigskip
\begin{flushleft}
Mathematics Department\\
Purade University,\\
West Lafayette, IN 47907\\
U.S.A.
\end{flushleft}
