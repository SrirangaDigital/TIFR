\title{Some quantum analogues of solvable Lie groups}
\markright{Some quantum analogues of solvable Lie groups}

\author{By~ C. De Concini, V.G. Kac and C. Procesi}
\markboth{C. De Concini, V.G. Kac and C. Procesi}{Some quantum analogues of solvable Lie groups}

\date{}
\maketitle

\section*{Introduction}
In the\pageoriginale papers \cite{art4-keyDK1} \cite{art4-keyDK2}, \cite{art4-keyDKP1} \cite{art4-keyDKP2}the quantized enveloping algebras introduced by Drinfeld and Jimbo have been studied in the case $q=\varepsilon$, a primitive $l$-th root of 1 with $l$ odd (cf. calx for basic definitions). Let us only recall for the moment that such algebras are canonically constructed starting from a Cartan matrix of finite type and in praticular we can talk of the associated classical objects (the root system, the simply connected algebraic group $G$. etc.) For such a algebra tha generic (resp. any) irreducible representation has dimesion equal to (resp. bounded by) $l^{N}$ where $N$ is the number of positive roots and the set of irreducible representations has a canonical map to the big cell of the corresponding group $G$.

In this paper we analyze the structure of some subalgebras of quanrized enveloping algebras corresponding to unipotent and solvable subgroups of $G$. These algebras have the non-commutative structure of iterated algebras of twisted polynomials with a derivation, an object which has often appeared in the general theory of non-commutative rings (see e.g. \cite{art4-keyKP}, \cite{art4-keyGL} and references there). In  pariticular, we find maximal demensions of their irreducible representations. Our results confirm the validity of the general philosophy that the representation theory is intimately connected to the Poisson geometry.

\section{Twisted polynomial rings}

\subsection{}
In this section we will collect some well knownn definitions and properties of twisted derivations.\label{art4-subsec-1.1}

Let $A$ be an algebra and let $\sigma$ be an automorphism of $A$. A \textit{twisted derivation} of $A$ realtive ot $\sigma$ is a linear map $D:A\rightarrow A$ such that:
$$
D(ab)=D(a)b+ \sigma(a)D(b).
$$
\begin{example*}
An element $a\in A$ induces an inner twisted derivation $ad_{\sigma}a$ relative to $\sigma$ defined by the formula:
$$
(ab_{\sigma}a)b=ab-\sigma(b)a.
$$

The following well-known fact is very useful in calculations with twisted derivations. (Hre and further we use ``box" notation:
$$ [n] = \dfrac{q^{n}-q^{-n}}{q-q^{-1}}, [n]! = [1][2]\ldots[n],
\begin{bmatrix}
m\\
n
\end{bmatrix}
= \dfrac{[m][m-1]\ldots[m-n+1]}{[n]!}
$$
One also writes $[n]_{d}$, etc. if $q$ is replaced by $q^{d}$.)
\end{example*}

\begin{prop*}
Let $a\in A$ and let $\sigma$ be an automorphism of $A$ such that $\sigma(a) = q^{2}a$, where $q$ is a scalar. Then
$$
(ad_{\sigma}a)^{m}(x)=\sum\limits_{j=0}^{m}(-1)^{j}q^{j(m-1)}
\begin{bmatrix}
m\\
j
\end{bmatrix}
a^{m-j}\sigma^{j}(x)a^{j}.
$$
\end{prop*}

\begin{proof}
Let $L_{a}$ and $R_{a}$ denote the operators of left and right multiplications by $a$ in $A$. Then
$$
ad_{\sigma}a= L_{a}-R_{a}\sigma.
$$
Since $L_{a}$ and$R_{a}$ commute, due to the assumption $\sigma(a)=q^{2}a$ we have
$$
L_{a}(R_{a}\sigma)=q^{-2}(R_{a}\sigma)L_{a}.
$$
Now the proposition is immediate from the following well-known binomial formula applied to the algebra End $A$.
\end{proof}

\begin{lemma*}
suppose that $x$ andy $y$ are elements of an algebra such that $yx=q^{2}xy$ for some scalar $q$. Then
$$
(x+y)^{m} = \sum\limits_{j=0}^{m}
\begin{bmatrix}
m\\
j
\end{bmatrix}
q^{j(m-j)}x^{j}y^{m-j}.
$$
\end{lemma*}

\begin{proof}
in by induction on $m$ using
$$
\begin{bmatrix}
m\\
j-1
\end{bmatrix}
q^{m+1} + 
\begin{bmatrix}
m\\
j
\end{bmatrix}
=
\begin{bmatrix}
m+1\\
j
\end{bmatrix}
q^{j},
$$
which follows from
$$
q^{b}[a]+ a^{-a}[b] = [a+b].
$$
\end{proof}

Let $\ell$ be a positive integer and let $q$ be a primitive $\ell$-th root of 1. Let $\ell' = \ell$ if $\ell$ is odd  and =$\frac{1}{2}\ell$ if $\ell$ is even. Then, by definition, we have
$$
\begin{bmatrix}
\ell'\\
j
\end{bmatrix}  
=0\; \text{for all}\; j\; \text{such that} \;0 < j < \ell'.
$$
This together with Proposition \ref{art4-subsec-1.1} implies

\begin{coro*}
Under the hypothesis of Proposition \ref{art4-subsec-1.1} we have:
$$
(ad_{\sigma}a)^{\ell'}(x)= a^{\ell'}x-\sigma^{\ell'}(x)a^{\ell'}\; \text{\rm if}\; q\; \text{\rm is a primitive}\; \ell-\text{\rm th root of}\; 1.
$$ 
\end{coro*}

\begin{remark*}
Let $D$ be a twisted derivation associated to an automorphism $\sigma$ such that $\sigma D=q^{2}D\sigma$. Then by induction on $m$ one obtains the following well-known $q$-analogue of the Leibniz formula:
$$
D^{m}(xy) =\sum\limits_{j=0}^{m}
\begin{bmatrix}
m\\
j
\end{bmatrix}
q^{j(m-j)}D^{m-j}(\sigma^{j}x)D^{j}(y).
$$
It follows that if $q$  is  a primitive $\ell$-th root of 1, then $D^{\ell'}$ is a twisted derivation associated to $\sigma^{\ell'}$
 \end{remark*}

\subsection{}
Given an automorphism $\sigma$ of $A$ and a twisted derivation $D$ of $A$ relative to $\sigma$ we define the \textit{twisted polynomial algebra} $A_{\sigma, D}[x]$ in the indeterminate $x$ to be the $\bbF$-module $A\otimes_{\bbF}\bbF[x]$
thought as formal polynomials with multiplication defined by the rule:\label{art4-subsec-1.2}
$$
xa = \sigma(a)x+ D(a).
$$
When $D=0$ we will also denote theis ring by $A_{\sigma}[x]$. Notice that the definition has been chosen in such a way that in the new ring the given twisted derivation becomes the inner derivation $ad_{\sigma}x$. 

Let us notice that if $a,b\in A$ and $a$ is invertible we can perform the change of variables $y:=ax+b$ and we see that $A_{\sigma, D}[x]=A_{\sigma', D'}[y]$. It is better to make the formulas explicit separately when $b=0$ and when $a=1$. In the fist case $yc=axc=a(\sigma(c)x+ D(c)) =a(\sigma(c))a^{-1}y+aD(c)$ and we see that the new automorphism $\sigma'$ is the composition $(Ada)\sigma$, so that $D' :=aD$ is a twisted derivation relative to $\sigma'$. Here and further $Ada$ stands for the inner automorphism:
$$
(Ada)x = axa^{-1}.
$$
In the case $a=1$ we have $yc=(x+b)c =\sigma(c)x+ D(b)+bc =\sigma(c)y+D(b)+bc-\sigma(c)b$, so that $D'=D+ad_{\sigma}b$. Summarizing we have

\begin{prop*}
Changing $\sigma, D$ to $(Ada)\sigma, aD$ (resp. to $\sigma, D+D_{b}$) does not change the twisted polynomial ring up to isomorphism.
\end{prop*}

We may express the previous fact with a definition: For a ring $A$ two pairs $(\sigma, D)$ and $(\sigma', D')$ are \textit{equivalent} if they are obtained one from the other by the above moves.

If $D=0$ we can also consider the twisted Laurent polynomial algebra $A_{\sigma}[x, x^{-1}]$. It is clear that if $A$ has no zero divisors, then the algebras $A_{\sigma, D}[x]$ and  $A_{\sigma}[x, x^{-1}]$ also have no zero divisors.

The importance for us of twisted polynomial algebras will be clear in the section on quantum groups.

\subsection{}\label{art4-subsec-1.3}
We want to study special cases of the previous construction.

Let us first consider a finite dimensional semisimple algebra. $A$ over and algebraically closed field $\bbF$, let
$\bigoplus_{i}\bbF e_{i}$ be the fixed points of the center of $A$ under $\sigma$ where the $e_{i}$ are central idempotents. We have $D(e_{i}) =D(e_{i}^{2}) =2D(e_{i})e_{i}$ hence $D(e_{i}) = 0$ and, if $x=xe_{i}$, then $D(x)=D(x)e_{i}$. It follows that, decomposing $A \bigoplus_{i}Ae_{i}$, each component $Ae_{i}$ is stable under $\sigma$ and $D$ and thus we have
$$
A_{\sigma, D}[x]= \bigoplus\limits_{i}(Ae_{i})_{\sigma,D}[x].
$$ 
This allows us to restrict our analysis to the case in which 1 is the only fixed central idempotent.

The second special case is described by the following:

\begin{lemma*}
Consider the algebra $A=\bbF^{\oplus k}$ with $\sigma$ the cyclic permutation of the summands, and let $D$ be a twisted derivation of this algebra relative to $\sigma$. Then $D$ is an inner twisted derivation.
\end{lemma*}

\begin{proof}
Compute $D$ on the idempotents: $D(e_{i}) = D(e_{i}^{2}) =D(e_{i})(e_{i}+e_{i+1})$. Hence we must have $D(e_{i})=a_{i}e_{i}-b_{i}e_{i+1}$ and from $0=D(e_{i}e_{i+1}) = D(e_{i})e_{i+1}D(e_{i+1})$ we deduce $b_{i} =a_{i+1}$. Let now $a = (a_{1},a_{2}, \ldots, a_{k})$; an easy computation shows that $D= ad_{\sigma}a$. 
\end{proof}

\begin{prop*}
Let $\sigma$ be the cyclic permutation of teh summands of the algebra $\bbF^{\oplus k}$. Then
\begin{enumerate}[{\it (a)}]
\item $\bbF_{\sigma}^{\oplus k}\left[x, x^{-1}\right]$ is an Azumaya algebra of degree $k$ over its center\break $\bbF\left[x^{k}, x^{-k}\right]$.\label{art4-enum-a}
\item $R:=\bbF_{\sigma}^{\oplus k}\left[x, x^{-1}\right]\otimes_{\bbF[x^{k}, x^{-k}]}\bbF\left[x, x^{-1}\right]$ is the algebra of $k\times k $ matrices over $\bbF\left[x, x^{-1}\right]$.\label{art4-enum-b}
\end{enumerate}
\end{prop*}

\begin{proof}
It is enough to prove (\ref{art4-enum-b}). Let $u:=x\otimes x^{-1}$, $e_{i}:=e_{i}\otimes 1$; we have $u^{k} = x^{k}\otimes x^{-k} = 1$ and $ue_{i}=e_{i+1}u$. From these formulas it easily follows that the elements
$e_{i}u^{j}(i,j =1,\ldots, k)$ span a subalgebra $A$ and that there exists an isomorphism $A\widetilde{\longrightarrow}(\bbF)$ mapping $\bbF^{\oplus k}$ to the diagonal matrices and $u$ to the matrix of the cyclic permutation. Then $R=A\otimes_{\bbF}\bbF\left[x, x^{-1}\right]$. 
\end{proof}

\subsection{}\label{art4-subsec-1.4}

Assuem now that $A$ is semi-simple and that $\sigma$ induces a cyclic permutation of the central idempotents.
\begin{lemma*}
~

\begin{enumerate}[\it (a)]
\item $A= M_{d}(\bbF)^{\oplus k}$ \label{art4-enum-(a)}
\item Let $D$ be a twisted derivation of $A$ realtive to $\sigma$. Then the pair $(\sigma, D)$ is equivalent to the pair $(\sigma', 0)$ where\label{art4-enum-(b)}
\end{enumerate}
\end{lemma*}
\begin{equation}\label{art4-subseceq-1.4.1}
\sigma'(a_{1}, a_{2},\ldots, a_{k}) = (a_{k},a_{1}, a_{2}, \ldots, a_{k-1})
\end{equation}

\begin{proof}
Since $\sigma$ permutes transitively the simple blocks they must all have the same degree $d$ so that $A=M_{d}(F)^{\oplus k}$. Furthermore we can arrange the identifications of the simple blocks with matrices so that: 
$$
\sigma'(a_{1}, a_{2},\ldots, a_{k}) = (\tau(a_{k}),a_{1}, a_{2}, \ldots, a_{k-1}),
$$
where $\tau$ is an automorphism of $M_{d}(\bbF)$. Any such automorphism in inner, hence after composing $\sigma$ with an inner automorphism, we any assume in the previous formula that $\tau =1$, Then we think of $A$ as $M_{d}(\bbF)\otimes \bbF^{\oplus k}$, the new automorphism being of the form $1\otimes \sigma'$ where $\sigma':\bbF^{\oplus k} \rightarrow
 \bbF^{\oplus k}$ is given by (\ref{art4-subseceq-1.4.1}).

 We also have that $M_{d}(\bbF) = A^{\sigma}$ and $\bbF^{\oplus k}$ is the centralizer of $A^{\sigma}$. Nest observe that $D$ restricted to $A^{\sigma}$ is a derivation of $M_{d}(\bbF)$ with values in $\oplus_{i=1}^{k}M_{d}(\bbF)$, i.e., $D(a)= (D_{1}(a), D_{2}(a), \ldots, D_{k}(a))$ where each $D_{i}$ is a derivation of $M_{d}(\bbF)$. Since for $M_{d}(\bbF)$. all derivations are inner we can find an element $u \in A$ such that $D(a) =[u,a]$ for all $a\in M_{d}\bbF$. So $(D-ad_{\sigma}u)(a) = [u,a]-(ua-\sigma(a)u)=0$ for $a\in A^{\sigma}$. Thus, changing $D$ by adding $-ad_{\sigma}u$ we may assume that $D=0$ on $M_{d}(\bbF)$.

 Now consider $b \in \bbF^{\oplus k}$ and $ac\in M_{d}(\bbF)$; we have $D(b)a = D(ba) = D(ab) =aD(b)$. Since $\bbF^{\oplus k}$ is the centralizer of $M_{d}(\bbF)$ we have $D(b)\in \bbF^{\oplus k}$ and $D$ induces a twisted derivation of $\bbF^{\oplus k}$. By Lemma \ref{art4-subsec-1.3} this last derivation is inner and the claim is proved.
\end{proof}

Summarizing we have

\begin{prop*}
Let $A$ be a finite- dimensional semisimple algebra over an algebraically closed field $\bbF$. Let $\sigma$ be an automorphism of $A$ which induces a cyclic permutation of order $k$ of the central idempotents of $A$. Let $D$ be a twisted derivation of $A$ relative to $\sigma$. Then:
\begin{gather*}
 A_{\sigma,D}[x] \cong M_{d}(\bbF)\otimes \bbF_{\sigma}^{\oplus k}[x],\\
 A_{\sigma,D}[x, x^{-1}]\cong M_{d}(\bbF)\otimes \bbF_{\sigma}^{\oplus k}[x, x^{-1}].
\end{gather*}
This last algebra is Azumaya of degree dk.
\end{prop*}

\subsection{}\label{art4-subsec-1.5}
We can now globalize the previous constructions. Let $A$ be a prime algebra (i.e. $aAb =0$, $a, b\in A$, implies that $a=0$ or $b=0$) over a field $\bbF$ and let $Z$ be the center of $A$. Then $Z$ is a domain and $A$ is torsion free module over $Z$. Assume that $A$ is a finite module over $Z$. Then $A$ embeds in a finite-dimensional central simple algebra $Q(A) = A\otimes_{Z}Q(Z)$, where $Q(Z)$ is the ring of fractions of $Z$. If $\overline{Q(Z)}$ denotes the algebraic closure of $Q(Z)$ is the ring of fractions of $Z$. If $\overline{Q(Z)}$ denotes the algebraic closure of $Q(Z)$ in the ring of fractions of $Z$. If $\overline{Q(Z)}$ denotes the algebraic closure of $Q(Z)$ we have that $A\otimes_{z}\overline{Q(Z)}$ is the full ring $M_{d}\overline{(Q(Z))}$ of $d \times d$ matrices over $\overline{Q(Z)}$. Then $d$ is called the \textit{degree} of $A$.

Let $\sigma$ be an automorphism of the algebra $A$ and let $D$ be a twisted derivation of $A$ relative to $\sigma$. Assume that
\begin{enumerate}[(\rm a)]
\item There is subalgebra $Z_{0}$ of $Z$, such that $Z$ in finite over $Z_{0}$.
\item $D$ vanishes on $Z_{0}$ and $\sigma$ restricted to $Z_{0}$ is the identity. 
\end{enumerate}

These assumptions imply that $\sigma $ restricted to $Z$ is an automorphism of finite order. Let $d$ be the degree of $A$ and let $k$ be the order of $\sigma $ on the center $Z$. Assume that the field $\bbF$ has characteristic 0. The main result of this section is:

\begin{theorem*}
Under the above assumptions the twisted polynomial algebra $A_{\sigma, D}[x]$ is an order in a central simple algebra of degree $kd$.
\end{theorem*}

\begin{proof}
Let $Z^{\sigma}$ be the fixed points in $Z$ of $\sigma$. By the definition, it is cleat that $D$ restricted to $Z^{\sigma}$ is derivation. Since it vanishes on a subalgebra over which it is finite hence algebraic and since we are in characteristic zero it follows that $D$ vanishes on $Z^{\sigma}$. Let us embed $Z^{\sigma}$ in an algebraically closed field $\bbL$ and let us consider the algebra $A\otimes_{Z^{\sigma}}\bbL = \bbL^{\oplus k}$ and $A \times_{Z}\bbL = M_{d}(\bbL)$. Thus we get that $A\otimes_{Z^{\sigma}}\bbL=\oplus_{i=1}^{k}M_{d}(\bbL)$. The pair $\sigma, D$ extends to $A\otimes_{Z^{\sigma}}\bbL$ and using the same notations we have that $(A\otimes_{Z^{\sigma}}\bbL)_{\sigma, D}[x] = (A_{sigma, D}[x])\otimes_{Z^{\sigma}} \bbL$. We are now in the situation of a semisimple algebra which we have already studied and the claim follows.
\end{proof}

\begin{coro*}
Under the above assumptions, $A_{\sigma, D}[x]$ and $A_{\sigma}[x]$ have the same degree.
\end{coro*}
\begin{remark*}
The previous analysis yields in fact a stronger result. Consider the open set of Spec $Z$ where $A$ is an Azumaya algebra; it is clearly $\sigma$-stable. In it we consider the open part where $\sigma $ has order exactly $k$. Every orbit of $k$ elements of the group generated by $\sigma$ gives a point $F(p)$ in Spec $Z^{\sigma}$ and $A\otimes_{Z}Z\otimes _{Z^{\sigma}}F(p) = \oplus_{i=1}^{k}M_{d}(F(p))$. Thus we can apply the previous theory which allows us to describe the fiber over $F(p)$ of the spectrum of $A_{\sigma, D}[x]$.
\end{remark*}

\subsection{}\label{art4-subsec-1.6}
Let $A$ be a prime algebra over a field $\bbF$ of characteristic 0, let\break $x_{1}, \ldots, x_{n}$ be a set of generators of $A$ and let $Z_{0}$ be a central subalgebra of $A$. For each $i =1, \ldots, K$, denote by $A^{i}$ the subalgebra of $A$ generated by $x_{1}, \ldots, x_{i}$ and let $Z_{0}^{i} = Z_{0}\cap A^{i}$. We assume that the following three conditions hold for each $i=1,\ldots, k$:
\begin{enumerate}[{\rm (a)}]
\item $x_{i}x_{j} = b_{ij}x_{j}x_{i} + P_{ij}$ if $i>j$. where $b_{ij}\in \bbF$, $P_{ij} \in A^{i-1}$.\label{art4-enum_(a)}
\item $A^{i}$ is a finite module over $Z_{0}^{i}$.\label{art4-enum_(b)}
\item Formulas $\sigma_{i}(x_{j}) = b_{ij}x_{j}$ for $j< i$  define a automorphism of $A^{i-1}$ which is the identity on $Z_{0}^{i-1}$.\label{art4-enum_(c)}
\end{enumerate}

Note that letting $D_{i}(x_{j}) = P_{ij}$ for $J < i$, we obtain $A^{i} = A_{\sigma_{i}, D_{i}}^{i-1}[x_{i}]$, so that $A$ is an iteratated twisted polynomial algebra, Note also that each triple $(A^{i-1}, \sigma_{i}, D_{i})$ satisfies assumptions \ref{art4-subsec-1.5} (\ref{art4-enum-(a)}) and (\ref{art4-enum-(b)}). 

We may consider the twisted polynomial algebras $\overline{A}^{i}$ with zero derivations, so that the relations are $x_{i}x_{j} = b_{ij}x_{j}x_{i}$ for $j < i$. We call this the \textit{associated quasipolynomial algebra} (as in \cite{art4-keyDK1}).

We can prove now the main theorem of this section.

\begin{theorem*}
Under the above assumptions, the degree of $A$ is equal to the degree of the associated quasipolynomial algebra $\overline{A}$. 
\end{theorem*}

\begin{proof}
We use the following remark. If there is an index $h$ such that the element $P_{ij} =0$ for all $i> h$ and all $j$, then monomials in the variables different from $x_{h}$ form as subalgebra $B$ and the algebra $A$ is a twisted polynomial ring $B_{\sigma, D}[X_{h}]$. The associated ring $B_{\sigma}[X_{h}]$ is obtained by setting $p_{hj} = 0$ for all $j$. Having made this remark we see that can inductively modify the relations \ref{art4-subsec-1.6}(\ref{art4-enum_(a)}) so that at the $h$-th step we have an algebra $A_{h}^{n}$ with the same type of relations but $P_{ij} = 0$ for all $i>n-h$ and all $j$. Since $A_{h}^{n}$ and $A_{h-1}^{n}$ are of type $B_{\sigma, D}[x]$ and $B_{\sigma}[x]$ respectively we see, by Corollary \ref{art4-subsec-1.5}, that they have all the same degree. 
\end{proof}

\section{Quantum groups}

\subsection{}\label{art4-subsec-2.1}
Let $(a_{ij})$ be an indecomposable $n\times n $ Cartan matrix and let $d_{1}, \ldots,\break d_{n}$ be relatively prime positive integers such tha $d_{i}a_{ij} = d_{j}a_{ji}$. Recall the associated notions of the weight, coroot and root lattices $p,Q\spcheck$ and $Q$, of the root and coroot systems $R$ and $R\spcheck$, of the Weyl group $W$, the $W$-invariant bilinear form $(.|.)$, etc.:  

Let $P$ be a lattice over $\bbZ$ with basis $\omega_{1},\ldots, \omega_{n}$ and let $Q\spcheck = \Hom_{\bbZ}\break (P, \bbZ)$ be the dual lattice with the dual basis $\alpha_{1}{\spcheck}, \ldots, \alpha_{n}{\spcheck}$, i.e. $\langle \omega_{i}, \alpha_{n}{\spcheck}\rangle = \delta_{ij}$. Let $P_{+}= \sum_{i=1}^{n} \bbZ_{+}\omega_{i}$. Let 
 $$
 \rho = \sum\limits_{i=1}^{n} \omega_{i},\quad \alpha_{j} = \sum\limits_{i=1}^{n}a_{ij}\omega_{i}\;(j=1,\ldots, n),
$$
and let $Q=\sum_{j=1}^{n}\bbZ\alpha_{j} \subset P$, and $Q_{+} = \sum_{j=1}^{n}\bbZ_{+}\alpha_{j}$.

Define automorphisms $s_{i}$ of $p$ by $s_{i}(\omega_{j}) = \omega_{j}-\delta_{ij}\alpha_{j}$ $(i,j = 1, \ldots, n)$.
Then $s_{i}(\alpha_{j}) = \alpha_{j}-a_{ij}\alpha_{i}$. Let $W$ be the subgroup of $GL(p)$ generated by $s_{1}, \ldots, s_{n}$. Let
\begin{align*}
\Pi &= \{\alpha_{1}, \ldots, \alpha_{n}\}, \quad \Pi\spcheck = \left\{\alpha_{1}\spcheck, \ldots, \alpha_{n}\spcheck\right\},\\
 R &= W\Pi,\quad R^{+} = R\cap Q_{+},\quad R\spcheck = W\Pi\spcheck . 
\end{align*}
The map $\alpha_{i}\longmapsto \alpha_{i}\spcheck$ extends uniquely to a bijective $W$-equivariant map $\alpha \longmapsto \alpha_{i}\spcheck$ between $R$ and $R\spcheck$. The reflection $s_{\alpha}$ defined by $s_{\alpha}(\lambda) = \lambda-\langle \lambda, \alpha\spcheck\rangle\alpha$ lies in $W$ for each $\alpha \in R$, so that $s_{\alpha_{i}}=s_{i}$.

Define a bilinear pairing $P\times Q \rightarrow \bbZ$ by $(\omega_{i}|\alpha_{j})=\delta_{ij}d_{j}$. Then $(\alpha_{i}|\alpha_{j}) = d_{i}a_{ij}$, giving a symmetric $\bbZ$-valued $W$-invariant bilinear form on $Q$ such that $(\alpha |\alpha) \in 2\bbZ$. We may indentify $Q\spcheck$ with a sublattice of the $\bQ$-span of $P$ (containing $Q$) using this form. Then: 
$$
\alpha_{i}\spcheck = d_{i}^{-1}\alpha_{i}, \; \alpha\spcheck= 2\alpha/(\alpha|\alpha).
$$

One defines the \textit{simply connected quantum group} $\calU$ associated to the matrix $(a_{ij})$ as analgebra over the ring $\calA:= \left[q, q_{-1}, (q^{d_{i}}-q^{-d_{i}})^{-1}\right]$. with generators $E_{i}, F_{i}(i = 1, \ldots, n)$, $K_{\alpha}(\alpha \in P)$ subject to the following relations (this is simple variation of the construction of Drinfeld and Jimbo): 
\begin{align*}
 &K_{\alpha}K_{\beta} = k_{\alpha+\beta}, k_{0}=1,\\
&\sigma_{\alpha}(E_{i}) = q^{(\alpha | \alpha_{i})}E_{i}, \sigma_{\alpha}(F_{i}) = q^{-(\alpha | \alpha_{i})}F_{i,}\\
&[E_{i}, F_{j}] = \delta_{ij}\dfrac{k_{\alpha_{i}}-K_{-\alpha_{i}}}{q^{d_{i}}-q^{-d_{i}}},\\
&(ad_{\sigma_{-\alpha_{i}}}E_{i})^{1-a_{ij}}E_{j} = 0, \; (ad_{\sigma_{\alpha_{i}}}F_{i})^{1-a_{ij}}F_{j} = 0\; (i \neq j),
\end{align*}
where $\sigma_{\alpha} = Ad$ $K_{\alpha}$. Recall that $\calU$ has a Hopf algebra structure with comultiplications $\Delta$, antipode $S$ and counit $\eta$ defined by: 
\begin{align*}
\Delta E_{i} &= E_{i}\otimes 1 + K_{\alpha_{i}} \otimes E_{i}, \; \Delta F_{i} = F_{i}\otimes K_{-\alpha_{i}} + 1 \otimes F_{i},\; \Delta k_{\alpha} = K_{\alpha}\otimes K_{\alpha},\\
SE_{i}& = -K_{-\alpha_{i}}E_{i}, \; SF_{i} =-F_{i}K_{i}, \; Sk_{\alpha} = K_{-\alpha},\\
\eta E_{i} &= 0, \; \eta F_{i} = 0,\; \eta K_{\alpha} = 1. 
\end{align*}

 Recall that the braid group $\calB_{W}$ (associated to $W$), whose canonical generators one denotes buy $T_{i}$, acts as a group of automorphisms of the algebra $\calU$ (\cite{art4-keyL}):

\begin{align*}
T_{i}K_{\alpha} &= K_{s_{i}(\alpha)}, \; T_{i}E_{i} = -F_{i}K_{\alpha_{i}}\\
T_{i}E_{j} &= \dfrac{1}{[-a_{ij}]_{d_{i}}}! (ad_{\sigma_{\alpha_{i}}}(-E_{i}))^{-a_{ij}}E_{j},\\
 T_{i}k &= kT_{i},
\end{align*}
where $k$ is a conjugate-linear anti-automorphism of $\calU$, viewed as an algebra over $\IC$, defined by:
$$
kE_{i} =F_{i}, \; kF_{i}=E_{i},\; kK_{\alpha} = K_{\alpha}, \; kq =q^{-1}.
$$

\subsection{}\label{art4-subsec-2.2}
Fix a reduced expression $\omega_{0} = S_{i_{1}}\ldots s_{i_{N}}$ of the longest element of $W$, and let
$$
\beta_{1} =\alpha_{i_{1}},\; \beta_{2} =s_{i_{1}}(\alpha_{i_{2}}), \ldots, \beta_{N} = s_{i_{1}} \ldots s_{i_{N-1}}(\alpha_{i_{N}}) 
$$
be the corresponding convex ordering of $R^{+}$. Introduce the corresponding \textit{root vectors} $(m=1,\ldots, N)$ (\cite{art4-keyL}):
$$
E_{\beta_{m}} = T_{i_{1}}\ldots, T_{i_{m-1}}E_{i_{m}}, \; E_{\beta_{m}} = T_{i_{1}}\ldots T_{i_{m-1}}F_{i_{m}} = kE_{\beta}
$$
(they depend on the choice of the reduced expression).

For $k=(k_{1}\ldots, k_{N})\in \bbZ_{+}^{N}$ we let
$$
E^{k}=E_{\beta_{1}}^{k_{1}}\ldots E_{\beta_{N}}^{}k_{N}, \; F^{k}= kE^{k}.
$$

\begin{lemma*}
~
\begin{enumerate}[(a)]
\item $[L]$ The elements $F^{k}K_{\alpha}E^{r}$, where $k,r \in \bbZ_{+}^{N}$, $\alpha \in P$, from a basis of $\calU$ over $\calA$.\label{art4-enumL-(a)}
\item $[LS]$ For $i < j$ one has: \label{art4-enumL-(b)}
\end{enumerate}
\begin{equation}
E_{\beta_{i}}E_{\beta_{j}}- q^{(\beta_{i} | \beta_{j})} E_{\beta_{j}}E_{\beta_{i}} =\sum\limits_{k\in\bbZ_{+}^{N}}c_{k}E^{k},\label{art4-eq2.2.1}
\end{equation}
\end{lemma*}

\noindent
where $c_{k}\in \IC \left[q, q^{-1}\right]$ and $c_{k} \neq 0$ only when $k=(k_{1}, \ldots, k_{N})$ is such that $k_{s} =0 $ for $s\leq i$ and $s\geq j$.

An immediate corollary is the following:

Let $w = s_{i_{1}}\ldots s_{i_{k}}$ which we complete to a reduced expression $\omega_{0} = s_{i_{1}}\ldots s_{i_{N}}$ of the longest element of $W$. Consider the elements $E_{\beta_{j}}$, $j = 1, \ldots, k$. Then we have:

\begin{prop*}
~
\begin{enumerate}[(a)]
\item The elements $E_{\beta_{j}}$, $j=1, \ldots, k$, generate a subalgebra $\calU^{w}$ which is independent of the choice of the reduced expression of $w$.\label{art-enum-p-(a)}
\item If $w' = w s$ with $s$ $a$ simple reflection and $l(w') = l(w) +1 = k+1$, then $\calU^{w'}$ is a twisted polynomial algebra of type $\calU_{\sigma, D}^{w}[E_{\beta_{k+1}}]$, where the formulas for $\sigma$ and $D$ are implicitly given in the formulas (\ref{art4-eq2.2.1}).\label{art-enum-p-(b)}
\end{enumerate}
\end{prop*}

\begin{proof}
(\ref{art-enum-p-(a)}) Using the face that once can pass from one reduced expression of $w$ to another by braid relations one reduces to the case of rank 2 where one repeats the analysis made by Luszting (\cite{art4-keyL}). (\ref{art-enum-p-(b)}) is clear by Lemma \ref{art4-subsec-2.2}. 
\end{proof}

\noindent
The elements $K_{\alpha}$ clearly normalize the algebras $\calU^{w}$ and when we add them to these algebras we are performing an iterated construction of Laurent twisted polynomials. The resulting algebras will be called $\calB^{w}$.

Since the algebras $\calU^{w}$ and $\calB^{w}$ are iterated twisted polynomial rings with relations of the type \ref{art4-subsec-1.6}(\ref{art4-enum_(a)}) we can consider the associated quasipolynomial alagebras, and we will denote them by $\overline{\calU}^{w}$ and $\overline{\calB}^{w}$. Notice that the latter algebras depend on the reduced expression chosen for $w$. Of course the defining relations for these algebras are obtained from (\ref{art4-eq2.2.1}) by replacing the right-hand side by zero. We could of coures also perform the same construction with the negative roots but this is not strictly necessary since we can simply apply the anti-automorphism $k$ to define the analogous negative objects.

\section{Degrees of algebras $\calU_{\calE}^{w}$ and $\calB_{\calE}^{w}$}\label{art4-sec-3}

\subsection{}\label{art4-subsec-3.1}
We specialize now the previous sections to the case $q=\calE$, a primitive $\ell$-th root of 1. Assuming that $\ell' > \max\limits_{i} d_{i}$. we may consider the specialized algebras:
$$
\calU_{\calE} = \calU/(q-\calE), \; \calU_{\calE}^{w} = \calU^{w}/(q-\calE), \; \calB_{\calE}^{w} = \calB^{w}/(q-\calE), \;{\rm etc}.
$$

\noindent
We have obvious subalgebra inclusions $\calU_{\calE}^{w} \subset \calB_{\calE}^{w} \subset \calU_{\calE}$.

First, let us recall and give a simple proof of the following crucial fact \cite{art4-keyDK1}:

\begin{prop*}
Elements $E_{\alpha}^{\ell}(\alpha \in R)$ and $K_{\beta}^{\ell}(\beta\in P)$ lie in the centre $z_{\calE}$ of $\calU_{\calE}$ if $\ell'> \max\limits_{i,j} |a_{ij}|$ (for any generalized Cartan matrix $(a_{ij}))$.
\end{prop*}

\begin{proof}
The only non-trivial thing to check is that $[E_{i}^\ell, E_{j}]= 0$ for $i \neq j$. From the ``Serre relations" it is immediate that $(ad_{\sigma_{-\alpha_{i}}}E_{i})^{\ell'}E_{j} =0$. Due to Corollary \ref{art4-subsec-1.1}, this can be rewritten as
$$
E_{i}^{\ell'}E_{j}=\calE^{-\ell'(\alpha_{i} | \alpha_{j})}E_{j}E_{i}^{\ell'},
$$
proving the claim. 
\end{proof}

As has been alreadu remarked, the algebras $\calU_{\calE}^{w}$ and $\calB_{\calE}^{w}$ are iterated twisted polynomial algebras with relations of the type \ref{art4-subsec-1.6}(\ref{art4-enum_(a)}) Proposition \ref{art4-subsec-3.1} shows that they satisfy conditions \ref{art4-subsec-1.6}(\ref{art4-enum_(b)}) and (\ref{art4-enum_(c)}). Hence Theorem \ref{art4-subsec-1.6} implies

\begin{coro*}
Algebras $\calU_{\calE}^{w}$ and $\overline{\calU}_{\calE}^{w}$ (resp.$\calB_{\calE}^{w}$ and $\overline{\calB}_{\calE}^{w}$) have the same degree.
\end{coro*}

\subsection{}\label{art4-subsec-3.2}
We proceed to calculate the degrees of algebras $\overline{\calU}_{\calE}^{w}$ and $\overline{\calB}_{\calE}^{w}$. Recal that these algebras are, up to inverting some variables, quasipolynomial algebras whose generations satisfy relations of type $x_{i}x_{j} = b_{ij}x_{j}x_{i}$, $i,j = 1, \ldots, s$, where the elements $b_{ij}$ have tha special form $b_{ij}= \calE^{c_{ij}}$, the $c_{ij}$ being entries of a skew-symmetric integral $s \times s$ matrix $H$. As we have shown in
[\cite{art4-keyDKP2}, Proposition 2.2] considering $H$ as the matrix of a linear map $\bbZ^{s} \rightarrow (\bbZ/(\ell))^{s}$, the degree of the corresponding twisted polynomial algebra is $\sqrt{h}$, where $h$ is the number of element of the image of this map.

Fix $w \in W$ and its reduced expression $w = s_{i_{1}}\ldots s_{i_{k}}$. We shall denote the matrix $H$ for the algebras $\overline\calU_{\calE}^{w}$ and $\overline\calB_{\calE}^{w}$ by $A$ and $S$ respectively. First we describe explicitly these matrices.

Let $d=2$ unless $(a_{ij})$ is of type $G_{2}$ in which case $d=6$, and let $\bbZ' = \bbZ\left[d^{-1}\right]$. Consider the roots $\beta_{1}, \ldots \beta_{k}$ as in Section \ref{art4-subsec-2.2}, and consider the free $\bbZ'$-module $V$ with basis $u_{1}, \ldots, u_{k}$. Define on $V$ a skew-symmetric bilinear form by
$$
\langle u_{i} | u_{j}\rangle = (\beta_{i} | \beta_{j})\; {\rm if} \; i < j.
$$

\noindent
Then $A$ is the matrix of this bilinear form in the basis $\{u_{i}\}$. Identifying $V$ with its dual $V^{*}$ using the give basis, we may think of $A$ as a linear operator from $V$ to itself.

Furthermore,
$$
S =
\begin{pmatrix}
A  & -^{t}C\\
C & 0
\end{pmatrix}
$$
where $C$ is the $n\times k$ matrix $((\omega_{i} | \beta_{j}))_{1 \leq i \leq n, 1 \leq j \leq k}$. We may think of the matrix $C$ as a linear map from the module $V$ with the basis $u_{1}, \ldots, U_{k}$ to the module $Q\spcheck \otimes _{\bbZ}\bbZ'$ with the basis $\alpha_{1}\spcheck, \ldots, \alpha_{n}\spcheck$. Then we have:
\begin{equation}
C(u_{i}) = \beta_{i}, \; i =1,\ldots, k.\label{art4-eq3.2.1}
\end{equation}

To study the matrices $A$ and $S$ we need the following

\begin{lemma*}
Given $\omega = \sum_{i=1}^{n}\delta_{i}\omega_{i}$ with $\delta_{i} = 0$ or $1$, set
$$
I_{\omega} = \{t \in {1,\ldots, k} | s_{i_{t}}(\omega) \neq \omega\}.
$$
Then
$$
w(\omega) = \omega - \sum\limits_{j\in i_{w}} \beta_{j}.
$$
\end{lemma*}

\begin{proof}
by induction on the length of $w$ . Write $w=w' s_{i_{k}}$. If $k \not\in I_{\omega}$ then $w(\omega)= w'(\omega)$ and we are done by induction. Otherwise $w(\omega) = w'(\omega-\alpha_{i_{k}}) = w'(\omega)-\beta_{k}$ and again we are done by induction.
\end{proof}

Note that ${1,2,\ldots, k} = \coprod\limits_{i=1}^{n}I_{\omega_{i}}$. 

\subsection{}\label{art4-subsec-3.3}
Consider the operators: $M=(A -{}^{t}C)$ and $N=(C O)$ so that $S= M\oplus N$.

\begin{lemma*}
~
\begin{enumerate}[(a)]
\item The operator $M$ is surjective.\label{art4-enum_l_(a)}
\item The vectors $v_{\omega}:=\left(\sum_{t \in I_{w}} u_{t}\right)-\omega-w(\omega)$, as $\omega$ runs through the fundamental weights, form a basis of the kernel of $M$.\label{art4-enum_l_(b)}
\item $N(v_{\omega}) = \omega - w(\omega) = \sum_{t\in I_{\omega}}\beta_{t}$.\label{art4-enum_l_(c)}
\end{enumerate}
\end{lemma*}

\begin{proof}
(a) We have by a straightforward computation:
$$
S(u_{i} + \beta_{i}) = - (\beta_{i} | \beta_{i})u_{i}-2 \sum\limits_{j>i}(\beta_{i}|\beta_{j})u_{j}- \beta_{i},
$$
and
$$
M(u_{i} + \beta_{i}) = - (\beta_{i} | \beta_{i})u_{i}-2 \sum\limits_{j>i}(\beta_{i}|\beta_{j})u_{j}
$$
Since $(\beta_{i} | \beta_{i})$ is invertible in $\bbZ'$ the claim follows.

(b) Since the vectors $v_{\omega}$ are part of a basis and, by (\ref{art4-enum_l_(a)}), the kernle of $M$ is a direct summand, it is enough to show that these vectors lier in the kernel. Now to check that $M(v_{\omega_{i}}) =0$ is equivalent to seeing tha $v_{\omega_{i}}$ lies in the kernel of the corresponding skew-symmetric form, i.e. $\langle u_{j} | v_{\omega_{i}}\rangle = 0$ for all $j = 1,\ldots, k$:

Using Lemma \ref{art4-subsec-3.2}, we have
\begin{equation}
\langle u_{j} | v_{\omega_{i}}\rangle = -2\sum\limits_{t > j}(\beta_{j} | \beta_{t}) + 2(\beta_{j}| w(\omega_{i})) + a_{j},\label{art4-eq3.3.1}
\end{equation}
where $a_{j} = 0$ if $j \not\in I_{\omega_{i}}$ and $a_{j} = (\beta_{j} | \beta_{j})$ otherwise.

We proceed by inductions on $k= l(w)$. Let us write $v_{\omega_{i}}(w)$ to stress the dependence on $w$. For $k=0$ there is nothing to prove. Let $w=w's_{i_{k}}$ with $l(w')=l(w)-1$. We distinguish two cases according to whether $i=i_{k}$ or not. If $i=\neq i_{k}$, i.e. $k\not\in I_{\omega_{i}}$, we have that $v_{\omega_{i}} = v_{\omega_{i}}(w')$ hence the claim follows by induction if $j < k$. For $j =k$ we obtain from (\ref{art4-eq3.3.1}):
$$
\langle u_{k} | v_{\omega_{i}}\rangle = -2(\beta_{k} | w(\omega_{i})) = -2(w'(\alpha_{i_{k}}) | w'(\omega_{i})) = -2(\alpha_{i_{k}} | \omega_{i}) = 0.
$$
Assume now that $i_{k} = i$ so that $w=w's_{i}$. Then $v_{\omega_{i}}(w)=v_{\omega_{i}}(w') + u_{k} -\beta_{k}$. For $j < k$ by induction we get:
$$
\langle u_{j} | v_{\omega_{i}}\rangle = \langle u_{j} | u_{k}\rangle -\langle u_{j} | \beta_{k}\rangle = -(\beta_{j} | \beta_{k}) + (\beta_{j} | \beta_{k}) = 0.
$$
Finally if $j=k$ we have:
$$
2(\beta_{k}|w(\omega_{i})) + (\beta_{k}|\beta_{k}) = 2(w' \alpha_{i} | w' (\omega_{i}-\alpha_{i})) + (\alpha_{i} | \alpha_{i}) = 2(\alpha_{i} | \omega_{i})-(\alpha_{i}| \alpha_{i}) = 0.
$$

Finally using (\ref{art4-eq3.2.1}), we have: $N(v_{\omega_{i}}) = \sum_{t\in I_{\omega_{i}}} \beta_{t}$, hence (\ref{art4-enum_l_(c)}) follows form Lemma \ref{art4-subsec-3.2}.
\end{proof}

\subsection{}\label{art4-subsec-3.4}
In order to compute the kernel of $S$ we need to compute the kernel of $N$ on the submodule spanned by the vectors $v_{\omega_{i}}$. Let us identify this module with the weight lattice $p$ by identifying $v_{\omega_{i}}$ with $\omega_{i}$. By Lemma \ref{art4-subsec-3.3}(\ref{art4-enum_l_(c)}), we see that $N$ in identified with map $1-w$ : $P\rightarrow Q$. At this point we need the following fact:

\begin{lemma*}
Consider the highest root $\theta =\sum_{i=1}^{n}a_{i}\alpha_{i}$ of the root system $R$. Let $\bbZ' = \bbZ''\left[a_{1}^{-1}, \ldots, a_{n}^{-1}\right]$, and let $M' = M \otimes_{\bbZ}\bbZ'$, $M'' = M\otimes_{\bbZ}\bbZ''$ for $M=p$ or $Q$. Then for any $w \in W$, the $\bbZ''$-submodule $(1-w)P''$ of $Q''$ is a direct summand.
\end{lemma*}

\begin{proof}
Recall that one can represent $w$ in the form $w=S_{\gamma_{1}}\ldots s_{\gamma_{m}}$ where $\gamma_{1} \ldots \gamma_{m}$ is a linearly independent set of roots (see e.g. \cite{art4-keyC}). Since in the decomposition
$\gamma\spcheck =\sum_{i}r_{i}\alpha_{i}\spcheck$ one of the $r_{i}$ is 1 or 2, it follows that $(1-s_{\gamma})P' = \bbZ'_{\gamma}$. Since $1-w = (1-s_{\gamma_{1}}\ldots s_{\gamma_{m-1}})s_{\gamma_{m}} + (1-s_{\gamma_{m}})$, we deduce by induction that
\begin{equation}
(1-w)P' = \sum\limits_{i=1}^{m}\bbZ'\gamma_{m}\label{art4-eq3.4.1}
\end{equation}
Recall now that any sublattice of $Q$ spanned over $\bbZ$ by some roots is a $\bbZ$-span of a set of roots obtained from $\Pi$ by iterating the following procedure: add a highest root to the set of simple roots, then remove several other roots form this set. The index of the lattice $M$ thus obtained in $M\otimes_{\bbZ}\bQ \cap Q$ is equal to the product of coefficients of removed roots in the added highest root. Hence it follows from (\ref{art4-eq3.4.1}) that
$$
((1-w)P'')\otimes_{\bbZ} \bQ \cap Q'' = (1-w)P",
$$
proving the claim.
\end{proof}

We call $\ell > 1$ a \textit{good} integer if it is relatively prime to $d$ and to all the $a_{i}$.

\begin{theorem*}
If $\ell$ is a good integer, then
$$
\deg\calB_{\calE}^{w} = \deg\overline{\calB}_{\calE}^{w} = ell^{\frac{1}{2}(\ell(w)+\rank(1-w))}.
$$
\end{theorem*}

\begin{proof}
From the above descussion we see that $\deg\overline{\calB}_{\calE}^{w} = \ell^{s}$, where $s = (\ell(w) +n) -(n -\rank(1-w))$, which together with Corollary \ref{art4-subsec-3.1} proves the claim.
\end{proof}

\subsection{}\label{art4-subsec-3.5}
We pass now to $\calU_{\calE}^{w}$. For this we need to compute the image of the matrix $A$. Computing first its kernel, we have that $K$ $er$ $A$ is identified with the set of linear combinations $\sum_{i}c_{i}v_{\omega_{i}}$ for which $\sum_{i}c_{i}(\omega_{i}+w(\omega_{i}))=0$ i.e. $\sum_{i}c_{i}\omega_{i}\in \ker(1 + w)$. This requires a case by case analysis. A simple case is when $w_{0} = -1$, so that $1+w = w_{0}(-1 + w_{0}w)$ and one reduces to the previous case. Thus we get

\begin{prop*}
If $w_{0} =-1$ (i.e. for types different from $A_{n}$, $D_{2n+1}$ and $E_{6}$) and if $\ell$ is a good integer, we have:
$$
\deg\calU_{\calE}^{w} = \deg\overline\calU_{\calE}^{w} = \ell^{\frac{1}{2}(\ell(w)+\rank(1+w)-n)}.
$$
\end{prop*}

Let us note the special case $w=w_{0}$. Remark that defining ${}^{t}\omega : = -w_{0}(\omega)$ we have an involution $\omega \rightarrow {}^{t}\omega$ on the set of fundamental weights. let us denote by $s$ the number of orbits of this involution.

\begin{theorem*}
If $\calE$ is a primitive $\ell$-th root of $1$, where $\ell$ is an integer greater than $1$ and relatively prime to $d$, then algebra $\calU_{\calE}^{w_{0}}$ and $\calB_{\calE}^{w_{0}}$ have degrees $\ell^{frac{N-s}{2}}$ and $\ell^{\frac{N+s}{2}}$ respectively.
\end{theorem*}

\begin{proof}
In this case $l(w_{0}) = N$ and the maps $\omega \rightarrow \omega +w_{0}(\omega)$ and $\omega \rightarrow \omega-w_{0}(\omega)$ are $\omega \rightarrow \omega - {}^{t}\omega$ and $\omega \rightarrow \omega + {}^{t}\omega$ and so their ranks are clearly $n-s$ and $s$ respectively.
\end{proof}

\section{Poisson structure}\label{art4-sec-4}

\subsection{}\label{art4-subsec-4.1}
Before we revert to the discussion of our algebras we want to make a general remark. Assume that we have a manifold $M$ and a vector bundle $V$ of algebras with $1$ (i.e. $1$ and the multiplication map are smooth sections). We identify the functions on $M$ with the sections of $V$ which are multiples of $1$.
Let $D$ be a derivation of $V$, i.e. a derivation of the algebra of sections which maps the algebra of functions on $M$ into itself and  let $X$ be the corresponding vector field on $M$. 

\begin{prop*}
For each point $p\in M $ there exists a neighborhood $U_{p}$ and a map $\varphi_{t}$ defined for $|t|$ sufficiently small on $V|U_{p}$ which is a morphism of vector bundles covering the germ of the $1$-parameter group generated by $X$ and is also an isomorphism of algebras.
\end{prop*}

\begin{proof}
The hypotheses on $D$ imply that it is a vector field on $V$ linear on the fibers, hence we have the existence of a local lift of the 1-parameter group as a morphism of vector bundles. The condition of being a derivation implies that the lift preserves the multiplications section i.e. it is a morphism of algebras.
\end{proof}

We will have to consider a variation of this: suppose $M$ is a Poisson manifold and assume furthermore that the Poisson structure lifts to $V$ i.e. for each (local) functions $f$ and section $s$ we have a  Poisson bracket which is a derivation. This means that we have a lift of the Hamiltonian vector fields as in the previous proposition. We deduce: 

\begin{coro*}
Under the previous hypotheses, the fibers of $V$ over points of a given symplectic leaf of $M$ are all isomorphic as algebras. 
\end{coro*}

\begin{proof}
The proposition implies that in a neighborhood of a point  in a leat the algebras are isomorphic but since the notion of isomorphism is transitive this implies the claim.
\end{proof}

\subsection{}\label{art4-subsec-4.2}
Let us recall some basic facts on Poisson groups (we refer to \cite{art4-keyD}, \cite{art4-keySTS}, \cite{art4-keyLW} for basic definitions and properties). Since a Poisson structure on a manifold $M$  is a special type of a section of
$\Lambda^{2}T(M)$ it can be viewed as a linear map from $T^{*}(M)$ to $T(M)$. The image of this map is thus a distribution on the manifold $M$, It can be integrated so that we have a decomposition of $M$ into symplectic leaves which are connected locally closed submanifolds whose tangent spaces are the spaced of the distribution. In fact in our case the leaves will turn out to be Zarisko open sets of algebraic of algebraic subvarieties.

For a group $H$ the tangent space at each point can be identified to the Lie algebra $\hbar$ by left translation and thus a Poisson structure on $H$ can be given as a family of maps $\gamma_{h} : \hbar^{*} \rightarrow \hbar$ as $h \in H$. Let $G$ be an algebraic group and $H, K \subset G$ algebraic groups if their corresponding Lie non-algebras $(g, \hbar, k)$
form a Manin triple , i.e. (cf. \cite{art4-keyD}, \cite{art4-keyLW}) if $g$ has a non-degenerate,symmetric invariant bilinear form with respect to which the Lie subalgebras $\hbar$ an $k$ are isotropic and $g=\hbar \oplus k$ (as vector spaces). Then it follows that we have a canonical isomorphism $\hbar^{*} =k$. Having identified $\hbar^{*}$ with $k$, the Poisson structure on $H$ is thus described by giving for every $h\in H $ a linear map $\gamma_{h} : k \rightarrow \hbar$.  

Let $x\in k$, consider $x$ as an element of $g$, set $\pi : g \rightarrow \hbar$ to be the projection with kernel $k$. Set:
$$
\gamma_{h}(x) = (Adh)_{\pi}(Adh)^{-1}(x).
$$
Then one can verify (aw in \cite{art4-keyLW}) that the corresponding tensor satisfies the required properties of a Poisson structure. (In fact any Poisson structure on $H$ can be obtained in this way.)

Notice now that the (restriction of the) canonical map:
$$
\delta: H\rightarrow G/K
$$
is an \'etale covering of some open set is $G/K$. Thus for every point $h\in H$ we can identify the tangent space to $H$ in $h$ with the tangent space of $G/K$ at $delta(h)$. By using right translation we can then identify the tangent space to $G/K$ at $\delta(h)$ with $g/(Adh)K$, the tangent space at $h\in H$ with $\hbar$ by right translation and the isomorphism between them with the projection $\hbar \rightarrow g/(Adh)k$. 

Using all these identifications once verifies that the map $\gamma_{h}$ previously considered is the map induced by differentiating the left $K$-action on $G/K$. From this it follows.

\begin{prop*}
The symplectic leaves for the symplectic structure on $H$ coincide with the connected components of the preimages under $\delta$ of $K$-orbits under the left multiplications of $G/K$.
\end{prop*}

consider now a quotient Poisson groups $S$ and $H$, that is $S$ is a quotient group of $H$ and the ring $\IC[S] \subset \IC[H]$ is a Poisson subalgebra. Let $U$ be the kernel of the quotient homomorphism $\varphi: H \rightarrow S$, let $s= Lie\; S$, $u = Lie\; U$ and $d\varphi : \hbar \rightarrow s$ the Lie algebra quotient map. Then $u$ is an ideal in $\hbar$ and we identify $s^{*}$ with a subspace of $\hbar^{*} =k$ by taking $u^{\perp} \subset g$ under the invariant form and intersecting it with $k$. Then for $p\in S$ the linear mar: $\overline{\gamma_{p}}: s^{*} \rightarrow s$ giving rise to the Poisson structure is given by:
$$
\overline{\gamma}_{p} = (d\varphi) \cdot (\gamma_{\overline{p}} | s^{*})
$$
where $tilde{p} \in H$ is any representative of $p$($\overline{\gamma}_{p}$ is independent of the choice of $\tilde{p}$).

The construction of the Manin triple corresponding to the Poisson manifold $S$ is obtained from the following simple fact:

\begin{lemma*}
Let $(g, \hbar, k)$ be a Manin triple of Lie algebras, and let $u \subset \hbar$ be an ideal such that $u^{\perp}(in g)$ intersected with $k$ is a subalgebra of the Lie algebra $k$. Then
\begin{enumerate}[(a)]
\item $u^{\perp}$ is a subalgebra of $g$ and $u$ is an ideal of $u\perp$.\label{art4-enum-l(a)}
\item $(u^{\perp}/u, \hbar/u, k\cap u^{\perp})$ is Manin triple, where the bilinear form on $u^{\perp}$ is induced by that of $g$.\label{art4-enum-l(a)}
\end{enumerate}
\end{lemma*}

\begin{proof}
Straightforward.
\end{proof}

\subsection{}\label{art4-subsec-4.3}
In the remaining sections we will apply the above remarks to the Poisson groups associated to the Hopf algebra $\calU_{\calE}$ and its Hopf subalgebra $\calB_{\calE}: = \calB_{\calE}^{w_{0}}$, and will derive some results on representations of the algebra $\calB_{\calE}$. From now on $\calE$ is a primitive $\ell$-th root of 1 where $\ell > 1$ is relatively prime to $d$.

Let $Z_{0}$ (resp. $Z_{0}^{+}$) be the subalgebra of $\calU_{\calE}$ (resp. $\calB_{\calE}$) generated by the elements $E_{\alpha}^{\ell}$ with $\alpha \in R$ (resp. $\alpha \in R^{+}$) and $K_{\beta}$ with $\beta \in P$. (We assume fixed a reduced expression of $w_{0}; Z_{0}$ and $Z_{0}^{+}$ are independent of this choice \cite{art4-keyDK1}.) Recall that they are central subalgebras (Proposition \ref{art4-subsec-3.1}).

It was shown in \cite{art4-keyDK1} that $Z_{0}$ and $Z_{0}^{+}$ are Hopf subalgebras, hence Spec $Z_{0}$ and Spec $Z_{0}^{+}$ have a canonical structure of an affine algebraic group. Furthermore. since $\calU_{\calE}$ is a specialization of the algebra $\calU$ at $q = \calE$, the center $Z_{\calE}$ of $\calU_{\calE}$ possesses a canonical Poisson bracket given by the formula:
$$
{a, b} =\dfrac{\left[\hat{a}, \hat{b}\right]}{2\ell^{2}(q-\calE)}\mod (q-\calE), a, b \in Z_{\calE},
$$
where $\hat{a}$ denotes the preimage of a under the canoncila homomorphism $\calU \rightarrow \calU_{\calE}$. The algebras $Z_{0}$ and Spec $Z_{0}^{+}$ have a canonical structure of Poisson algebraic groups, Spec $Z_{0}^{+}$ begin a quotient Poisson group of Spec $Z_{0}$.

In \cite{art4-keyDKP1} an explicit isomorphism was constructed between the Poisson grout Spec $Z_{0}$ and a Poisson group $H$ which is described below. We shall identify these Poisson groups.

Let $G$ be the connected simply connected algebraic group associated to the Cartan Matrix $(a_{ij})$ and let $g$ be its (complex) Lie algebra. We fix the triangular decomposition $g=u_{-} + t + U_{+}$, let $b_{\pm} = t + u_{\pm}$, and denote by $(. | .)$ the invariant bilinear form on $g$ which on the set of roots $R\subset t^{*}$ coincides with that defined in Section \ref{art4-subsec-2.1}. Let $U_{\pm}$, $B_{\pm}$ and $T$ be the algebraic subgroups of $G$ corresponding to Lie algebras $u_{\pm}$, $B_{\pm}$ and $t$. Then as an algebraic group, $H$ is the following subgroup of $G\times G$:
$$
H = \left\{(tu_{+}, t^{-1}u_{-}) | t \in T, u_{\pm} \in \calU_{\pm}\right\}.
$$ 
The Poisson structure on $H$ is given by the Manin triple $(g \oplus g, \hbar, k)$, where 
\begin{align*}
\hbar &= \{(t +u_{+}, -t+ u_{-}) | t\in t, u_{\pm} \in u_{\pm}\},\\ 
k &= \left\{(\mathsf{g},\mathsf{g}) |  \mathsf{g} \in g\right\},
\end{align*}
and the invariant bilinear form of $g\oplus g$ is 
$$
((x_{1}, x_{2})| (y_{1}, y_{2})) = -(x_{1}| y_{1}) + (x_{2} | y_{2}).
$$

We identify the group $B_{+}= H/\{(1,u_{-} | u_{-}\in U_{-})\}$. The Manin triple generating Poisson structure on $B_{+}$ is obtained from $(a \oplus g, \hbar, k)$ by taking the ideal $u=\{(0, u_{-}, u_{-} \in u_{-})\}$ and applying the construction given by Lemma \ref{art4-subsec-4.2}. We clearly obatina the triple $(q \oplus t, b_{+}, b_{-})$, where we used identifications
$$
b_{\pm} = \{(u_{\pm}-t, \pm t) | u_{\pm} \in u_{\pm}, t \in t\}.
$$

According to the general recipe of Proposition \ref{art4-subsec-4.2}, the symplectic leaves of the Poisson group $B_{+}$ are obtained as follows. We identify the groups $B_{\pm}$ with the following subgroups of $G\times T$:
$$
B_{\pm} =\left\{(t^{-1}u_{\pm}, t^{\pm 1}) | t \in T, u_{\pm} \in U_{\pm}\right\}.
$$
The inclusion $B_{+} \subset G\times T$ induces an \'etale morphism
$$
\delta: B_{+}\rightarrow (G\times T)/B_{-}.
$$
Then the symoplectic leaves of $B_{+}$ are the connected components of the preimages under the map under the map $\delta$ of $B_{-}$-orbits on $G\times T/B_{-}$ under the left multiplication .

In order to analyze the $B_{-}$-orbits on $g\times T/B_{-}$, let $\mu_{\pm}$ : $B_{\pm} \rightarrow T$ denote the canonical homomorphisms with kernels $U_{\pm}$ and consider the equivariant isomorphism of $B_{-}$-varieties $\gamma$ : $G/U_{-}\rightarrow (G\times T)/B_{-}$ given by $\gamma(\mathsf{g}\calU_{-})= (\mathsf{g}, 1)B_{-}$, where $B_{-}$ acts on $G/U_{-}$ by  
\begin{equation}
b(\mathsf{g}U_{-}) = b\mathsf{g}\mu_{-}(b)U_{-}.\label{art4-eq4.3.1}
\end{equation}

\noindent
Then the map $\delta$ gets identified with the map $\delta : B_{+} \rightarrow G/U_{-}$ given by
$$
\delta(b)= b\mu{+}(b)U_{-}.
$$ 

We want to study the orbits of the action (\ref{art4-eq4.3.1}) of $B_{-}$ on $G/U_{-}$. Consider the action of $B_{-}$ on $G/B_{-}$ by left multiplication. Then the canonical map $pi: G/U_{-} \rightarrow G/B_{-}$ is $B_{-}$-equivariant, hence $\pi$ maps every $B_{-}$-orbit $\calO$ in $G/U_{-}$ to a $B_{-}$-orbit in $G/B_{-}$, i.e. a Schubert cell $C_{w}= b_{-}wB_{-}/B_{-}$ for some $e \in W$. We shall say that the orbit $\calO$ is associated to $w$.   

\begin{remark*}
We have a sequence of maps:
$$
B_{+} \xrightarrow{\delta}(G\times T)/B_{-} \xrightarrow{\gamma^{-1}}G/U_{-}\xrightarrow{\pi} G/B_{+}.
$$

Let $\psi = \pi \;\circ \; \gamma^{-1}\; \circ \; \delta$ and $X_{w} = B_{+} \cap B_{-}w B_{-}$. Then:
$$
\pi^{-1}(C_{w})= b_{-}wB_{-}/U_{-} \;\text{\rm and}\; \psi^{-1}(C_{w})= X_{w}.
$$
\end{remark*}

We can prove now the following

\begin{prop*}
Let $\calO$ be $a$ $B_{-}$-orbit in $G/U_{-} $ under the action (\ref{art4-eq4.3.1}) associated to $w \in W$. Then the morphism:
$$
\pi|_{\calO} :\calO \rightarrow C_{w}
$$
is a principal torus bundle with structure groups:
$$
T^{w}:= \left\{w^{-1}(t)t^{-1},\; \text{\rm where} \; t\in T\right\}.
$$
In particular:
$$
\dim\calO = \dim C_{w} + \dim T^{w} = l(w) + \rank(I -w).
$$
\end{prop*}

\begin{proof}
For $\mathsf{g} \in G$ we shall write $[\mathsf{g}]$ for the coset ${}_{\mathsf{g}}U_{-}$. The morphism $\pi$ is clearly a principle $T$-bundle with $T$ acting on the right by $[\mathsf{g}]t := [\mathsf{gt}]$. The action (\ref{art4-eq4.3.1}) fo $B_{-}$-orbits. Each of $B_{-}$ commutes with the right $T$-action so that $T$ permutes the $B_{-}$-orbits. Each $B_{-}$-orbit is a principal bundle whose structure group is the subtorus of $T$ which stabilizes the orbit. This subtorus is independent of the orbit since $T$ is commutative. In order to compute it we proceed as follows. Let $[\mathsf{g_{1}}]$, $[\mathsf{g_{2}}]$ be two elements in $\calO$ mapping to $w \in C_{w}$. We may assume the $\mathsf{g_{1}} = nh$, $\mathsf{g_{2}} = nk$ with $h, k \in T$ uniquely determined, where $n\in N_{G}(T)$ is representative of $w$. Suppose that $b[nh] = b[nk]$, $b \in B_{-}$. We can first reduce to the case $b=t \in T$; indeed, writing $b = ut$ we see that $u$ must fix $w \in C_{w}$ hence $un = nu'$ with $u' \in U_{-}$ and hence $u$ acts trivially on $t[nh]$. Next we have that, by definition of the $T$-action (\ref{art4-eq4.3.1}),
$$
[nk] = \left[tnht^{-1}\right] = \left[n(n^{-1}tnht^{-1})\right]
$$
hence $k=n^{-1}tnht^{-1}$ or $k=h\left(h^{-1}n^{-1}tnht^{-1}\right) = h\left(n^{-1}tnt^{-1}\right)$ as required.
\end{proof}

\begin{lemma*}
Let $\calO \subset B_{+}$ be a symplectic leaf associated to $w$. Then $\calO T = X_{w}$.
\end{lemma*}

\begin{proof}
From our proof we know that the map $\delta$ is a principal $T$-bundle and $T$ permutes transitively the leaves lying over $C_{w}$
\end{proof}

We thus have a canonical stratification of $B_{+}$, indexes by the Weyl group, by the subsets $x_{w}$. Each such subset is a union of leaves permuted transitively by the right multiplications of the group $T$.

We say that a point $a\in \Spec Z_{0}^{+}= B_{+}$ lies over $w$ if $\psi(a) \in C_{w}$. 

\subsection{}\label{art4-subsec-4.4}
Recall that $T=\bbZ^{\times}\otimes_{\bbZ}A\spcheck$ and therefore any $\lambda \in P = \Hom_{\bbZ}(Q^{\spcheck}, \bbZ^{\times})$ defines a homomorphism (again denoted by)$\lambda: T\rightarrow \bbZ^{\times}$ For each $t\in T$ we define and automorphism $\beta_{t}$ of the algebra $\calB_{\calE}$ by:
$$
\beta_{t}(K_{\alpha}) = \alpha(t)K_{\alpha}, \quad \beta_{t}(E_{\alpha}) = \alpha(t)E_{\alpha}.
$$
Note that the automorphisma $B_{t}$ leave $Z_{0}^{+}$ invariant and permute transitively the leaves of each set $\psi^{-1}(C_{w})\subset B_{+}$.

Given $a\in B_{+}=\Spec Z_{0}^{+}$, denote by $m_{a}$ the corresponding maximal ideal of $Z_{0}^{+}$ and let 
$$
A_{a}=B_{\calE}/m_{a}\calB_{\calE}.
$$
These are finite-dimensional algebras and we may also consider these algebras as algebras with trace in order to use the techniques of \cite{art4-keyDKP2}.

\begin{theorem*}
If $A,b \in \Spec Z_{0}^{+}$ lie over the same element $w \in W$, then the algebras $A_{a}$ and $A_{b}$ are isomorphic (as algebras with trace).
\end{theorem*}

\begin{proof}
We just apply Proposition \ref{art4-subsec-4.1} to the vector bundle of algebras $A_{a}$ over a symplectic leat and the group $T$ of algebra automorphisms which permutes the leaves in $\psi^{-1}(C_{w})$ transitively.
\end{proof}

\subsection{}\label{art4-subsec-4.5}
Let $B^{w}: =B_{+}\cap w B_{-}w^{-1}$ and $U^{w} : =U_{+}\cap wU_{-}w^{-1}$ so that  $B^{w} = U^{w}T$. Set also $U_{w}:=U_{+} \cap wU_{+}w^{-1}$. One knows that $\dim B^{w} = n+l(w)$ and that the multiplication map:
$$
\sigma : U_{w} \times B^{w} \rightarrow B_{+}
$$
is an isomorphism of algebraic varieties. We define the map
$$
p_{w} : B_{+} \rightarrow B^{w}
$$
to be the inverse of $\sigma$ followed by the projection on the second factor.

\begin{prop*}
The map
$$
p_{w}|x_{w} : X_{w}\rightarrow B^{w}
$$
is birational.
\end{prop*}

\begin{proof}
We need to exhibit a Zarisko open set $\Omega \subset B^{w}$ such that for any $b \in \Omega$ there is a unique $u \in U_{w}$ with $ub \in x_{w}$.

Let $n \in N_{G}(T)$ be as above a representative for  $w$ so that:
$$
X_{w} =\{b \in B_{+} |b = b_{1}nb_{2},\; \text{\rm where} \; b_{1}, b_{2} \in B_{-}\}.
$$
Consider the Bruhat cell $B_{-}n^{-1} B_{-}\subset G$. Every element in $B_{-}n^{-1} B_{-}$ can be written uniquely in the form:
$$
bn^{-1}u, \;\text{\rm where}\;  b\in n^{-1}B^{w}n, \; u \in U_{-}.
$$
The set $B_{+}U_{-} = B_{+}B_{-}$ is open dense and so it intersects $B_{-}n^{-1}B_{-}= n^{-1}b^{w}U_{-}$ in a non-empty open set which is clearly $B_{-}$-stable for the right multiplication, hence

\noindent
$B_{+}B_{-}\cap B_{-}n^{-1}B_{-} = n^{-1}\Omega U_{-}$ for some non empty open set $\Omega \subset B^{w}$. In particular $\Omega \subset nB_{+}B_{-} =nU_{w}B^{w}U_{-}$. Take $b\in \Omega$ and write it as $b=nxcv$ with $x\in U_{w}$, $c\in B^{w}$, $v \in U_{-}$. By the remarks made above this decomposition is unique; furthermore, $nxn^{-1}\in U_{w}$, $ncn^{-1} \in B_{-}$. For the element $n^{-1}b = xcv$ we have by construction that $xcv \in B_{-}n^{-1}B_{-}$ and $nx^{-1}n^{-1}b=\left(ncn^{-1}\right)nv \in B_{-}n B_{-}$ and $nx^{-1}n^{-1} \in U_{w}$. Thus setting $u : = nx^{-1}n^{-1}$we have found $u \in U_{w}$ such that $ub \in X_{w}$. This $u$ is unique since the element $x$ is unique.
\end{proof}

We are ready now for the concluding theorem which is in the spirit of the conjecture formulated in \cite{art4-keyDKP1}.

\begin{theorem*}
Let $p\in X_{w}$ be a point over $w \in W $ and let $A_{p}$ be the corresponding algebra. Assume that $l$ is good integer. Then the dimension of each irreducible representation of $A_{p}$ is divisible by =$l^{\frac{1}{2}(l(w)+ \rank(1-w))}$.
\end{theorem*}

\begin{proof}
Consider the algebra $B_{\calE}^{w}$ for which we know by Theorem \ref{art4-subsec-3.5} that
$$
\deg B^{w} = l^{\frac{1}{2}(l(w)+\rank(1-w))}.
$$
The subalgebra $Z_{0, w}$ of $Z_{0}$ generated by the elements $K_{\lambda}^{l}$ and $E_{\alpha}^{l}$, where $\lambda \in P$ and $\alpha \in R^{+}$ is such that $-w^{-1}\alpha \in R^{+}$, is isomorphic to the coordinate ring of $B^{w}$, and $\calB_{\calE}^{w}$ is a finite free module over $Z_{0, w}$. Thus by \cite{art4-keyDKP2} there is a non empty open set $\calA$ of $B^{w}$ such that for  $p\in \calA$ any irreducible representation of $\calB^{w}$ lying over $p$ is of maximal dimension, equal to the degree of $\calB_{\calE}^{w}$. Now the ideal $I$ defining $X_{w}$ has intersection $0$ with $Z_{0, w}$ and so when we restrict a generic representation of $\calB_{\calE}$ laying over points of $X_{w}$ to the algebra $\calB_{\calE}^{w}$ we have, as a central character of $Z_{0, w}$, a point in $\calA$ Thus the irreducible representation restricted to $\calB^{w}$ has all its composition factors irreducible of dimension equal to $\deg\calB^{w}$. This proves the claim.  
\end{proof}

It is possible that the dimension of any irreducible representation of $B_{\calE}$ whose central character restricted to $Z_{0}^{+}$ is  a point of $x_{w}$ is exactly $\ell^{\frac{1}{2}(\ell(w)+\rank(1-w))}$. This fact if true would require a more detailed analysis in the spirit of Section \ref{art4-subsec-1.3}.

We would like, in conclusion, to propose a more general conjecture, similar to one of the results of \cite{art4-keyWK} on solvable Lie algebras of characteristic $p$.

Let $A$ be an algebra over ${\IC}\left[q, q^{-1}\right]$ on generators $x_{1}, \ldots, x_{n}$ satisfying the following relations:
$$
x_{i}x_{j}= q^{h_{ij}}x_{j}x_{i}+ P_{ij}\; \text{if} \; i > j,
$$
where $(h_{ij})$ is a skew-symmetric matrix over $\bbZ$ and $P_{ij}\in \IC\left[q, q^{-1}\right]\break\left[x_{1}, \ldots, x_{n}\right]$. Let $\ell > 1$ be an integer relatively prime to all elementary divisors of the matrix $(h_{ij})$ and let $A_{\varepsilon} = A/(q -\varepsilon)$ and assume that all elements $x_{i}^{\ell}$ are central. Let $Z_{0} = \IC\left[x_{1}^{\ell}, \ldots, x_{n}^{\ell}\right];$ this algebra has a canonical Poisson structure.

\begin{conjecture*}
Let $\pi$ be an irreducible representation of the algebra $A_{\varepsilon}$ and let $\calO_{\pi} \subset \Spec Z_{0}$ be the symplectic leaf containing the restriction of the central character of $\pi$ to $Z_{0}$. Then the dimension of this representation is equal to $\ell^{\frac{1}{2}\dim \calO_{\pi}}$.

This conjecture of course holds if all $P_{ij}$ are 0, and it is in complete agreement with Theorems \ref{art4-subsec-1.6}, \ref{art4-subsec-3.5} and \ref{art4-subsec-4.6}.
\end{conjecture*}

\begin{thebibliography}{99}
\bibitem[C]{art4-keyC} R.W. Carter, \textit{Conjugacu classes in the Weyl group}, Compositio Math. {\bf 25} (1972) 1-59.
\bibitem[D]{art4-keyD} V.G. Drinfeld, \textit{Quantum groups, Proc.}ICM Berkeley {\bf 1} (1986) 789-820.
\bibitem[DK1]{art4-keyDK1} C. De Concini, V.G. Kac, \textit{Representations of quantum groups at roots of}1, in progress in Math. {\bf 92}, Birksh\"auser, (1990) 471-506.
\bibitem[DK2]{art4-keyDK2} C. De Concini, V.G. Kac, \textit{Representations of quantum groups at roots of $1$ : reduction to the exceptional case}, in Infinite Analysis, World Scientific, (1992) 141-150. 
\bibitem[DKP1]{art4-keyDKP1}C. De Concini, V.G. Kac, C. Procesi, \textit{Quantum coadjoint action}, Journal of AMS, {\bf 5} (1992) 151-190.
\bibitem[DKP2]{art4-keyDKP2} C. De Concini, V.G. Kac, C. Procesi, \textit{Some remarkable degenerations of quantum groups}, preprint, (1991).
\bibitem[GL]{art4-keyGL} K.R. Goodearl, E.S. Letzter, \textit{Prime ideals in skew and q-skew polynomial rings}, preprint (1991).
\bibitem[KP]{art4-keyKP} V.G.KAc, D.H. Peterson, \textit{Generalized invariants of groups generated by reflections}, in Progress in Math. {\bf 60}, Birkh\"auser, (1985) 231-250.
\bibitem[L]{art4-keyL}G. Lusztig, \textit{Quantum groups at roots of} 1, Geom. Ded. {\bf 35} (1990) 89- 114.
\bibitem[LS]{art4-keyLS} S.Z. Levendorskii, Ya. S. Soibelman, \textit{Algebras of functions on compact quatum groups, Schubert cells and quantum tori}, Comm. Math. Physics {\bf 139} (1991) 141-170.
\bibitem[LW]{art4-keyLW}J.-H. Lu, A Weinstein, \textit{Poisson Lie groups, dressing transformations and Bruhat decompositions}, J. Diff. Geom {\bf 31} (1990) 501-526.
\bibitem[STS]{art4-keySTS} M.A. Semenov-Tian-Shansky, \textit{Dressing transformations and Poisson group actions}, Publ. RIMS {\bf 21} (1985) 1237-1260.
\bibitem[WK]{art4-keyWK}B.Yu, Weisfeiler, V.G. Kac, \textit{On irreducible representations of Lie p-algebras}, Funct. Anal. Appl. {\bf 5:2} (1971) 28-36.
\end{thebibliography}

\begin{tabular}{l @{\hspace{1.8cm}} l}
\text{Classe di Scienze} & \text{Department di mathematics}\\
\text{Scuda Normale Superiore}  &\text{Instituto Guide Gastelnuovo}\\
\text{56100 Pisa} & \text{Univrsita di Ronala Sapienza}\\
\text{Piazza dei Cavalieri} & \text{Piazzale Aldo Moro, 5}\\
\text{Italy} & \text{1-00185 Rome, Italy}\\
\end{tabular}
