\title{Some quantum analogues of solvable Lie groups}
\markright{Some quantum analogues of solvable Lie groups}

\author{By~ C. De Concini, V.G. Kac and C. Procesi}
\markboth{C. De Concini, V.G. Kac and C. Procesi}{Some quantum analogues of solvable Lie groups}

\date{}
\maketitle

\section*{Introduction}
In the\pageoriginale papers \cite[DK]{art4-key1-2}, \cite[DKP]{art4-key1-2} the quantized enveloping algebras introduced by Drinfeld and Jimbo have been studied in the case $q=\varepsilon$, a primitive $l$-th root of 1 with $l$ odd (cf. $\calx$ for basic definitions). Let us only recall for the moment that such algebras are canonically constructed starting from a Cartan matrix of finite type and in praticular we can talk of the associated classical objects (the root system, the simply connected algebraic group $G$. etc.) For such a algebra tha generic (resp. any) irreducible representation has dimesion equal to (resp. bounded by) $l^{N}$ where $N$ is the number of positive roots and the set of irreducible representations has a canonical map to the big cell of the corresponding group $G$.

In this paper we analyze the structure of some subalgebras of quanrized enveloping algebras corresponding to unipotent and solvable subgroups of $G$. These algebras have the non-commutative structure of iterated algebras of twisted polynomials with a derivation, an object which has often appeared in the general theory of non-commutative rings (see e.g. \cite{art4-keyKP}, \cite{art4-keyGL} and references there). In  pariticular, we find maximal demensions of their irreducible representations. Our results confirm the validity of the general philosophy that the representation theory is intimately connected to the Poisson geometry.

\section{Twisted polynomial rings}

\subsection{}
In this section we will collect some well knownn definitions and properties of twisted derivations.

Let $A$ be an algebra and let $\sigma$ be an automorphism of $A$. A \textit{twisted derivation} of $A$ realtive ot $\sigma$ is a linear map $D:A\rightarrow A$ such that:
$$
D(ab)=D(a)b+ \sigma(a)D(b).
$$
