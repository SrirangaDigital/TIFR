\title{Algebraic Representations of Reductive Groups over Local Fields}
\markright{Algebraic Representations of Reductive Groups over Local Fields}

\author{By~ William J. Haboush\footnote{This research was founded in part by the National Science Foundation}}
\markboth{William J. Haboush}{Algebraic Representations of Reductive Groups over Local Fields}

\date{}
\maketitle



\section*{Introduction}

This paprr is an extended study of the behaviour of simplicial co- sheaves in the buildings associated to algebraic groups, both finite and infinite dimensional. Recently the theory of simplicial sheaves and co-sheaves has found a number of important applications to the representation theory and cohomology theory of finite theory of finite groups (see \cite{art6-keyT}, art6-keyRS), the computations of teh cohomology of arithmetic groups and the problem of admissible representatiosn of P-adic groups and teh Langlands classification (\cite{art6-keyCW}, \cite{art6-keyBW}). My interest has been, for the most part, the representation theory of semi-simple groups over fields of positive characteristic. In this area, of course, the driving force of much recent work has been the so-called Lusztig characteristic $p$ conjecture \cite{art6-keyL1} (so called to distinguish it from a number of other equally intersecting Lusztig conjectures). In contemplating this conjecture one is struck by certain resonances with work in admissible representations etc.

The line of argument I am hoping to achieve is something like this. One should attempt to use the homoligical algebra of simplicial co-sheaves to construct a category of representations of something like the loop group associated to th semisimple group, $G$, which have computable character theory. Then one should attempt to express the finite dimensional representations of $G$ as virtual representations in the category. Then presumably the ``generic decomposition patterns" should be formulae expressing the character of a dual Weyl module in terms of these computable characters. The hope of constructing such a theory has led me to conduct the rather extended exploration below.  

One is immediately tempted to replace harmonic analysis with a purely algebraic theory and to use this theory to do the representation-theoretic computations necessary. My replacement for harmonic analysis is this. Let $G$ be algebraic of simple type over $\bbZ$. Let $K$ be a field complete with respect to a discrete rank one valuations, let $\calO$ be the valuation ring and let $L$ be its residue field. Then consider the Bruhat-Tits building of $G(K)$. It is a simplicial complex. Let $\bG =G(K)$. Let $\calI$ be its Bruhat-Tits building. Then $\calI$ is $\bG$ equivariant. Let $k$ be another field. My idea is to consider the category of $\bG$ equivariant co-sheaves of $k$-vector spaces which are, in some sense made precise within, locally algebraic. Then in a manner entirely analogous to the classical notion of rational of rational representative functions, this category admits an injective co-generator. The endomorphism ring of this canonical co-generator is a certain algebra. Let it be denoted $\calH$. \cite{art6-keyT} shows that for suitably finite $bG$ is a certain algebra. Let it be denoted $\calH$-module. Then one may sent the class of a finite dimensional $G$- representation to the alternating sum of the left derived functors of the co-limit of its induced $\bG$-co-sheaf in the Grothendieck ring of $\calH$. In this context that one would hope to obtain interesting identities relating finite dimensional representation theory to the representation theory of $\calH$.

I have made certain choices in this discussion. As I am discussing sheaves and co-sheaves on simplicial complexes, I have decided to use the word carapaces for co-sheaves. There are three reasons: the first is that the word, co-sheaves, seems rather cobbled together, the second is that a carapace really would look like a lobster shell or such if one were to draw one and the third is the Leray used the word for sheaves and I don't like to see such a nice word go to waste.  am working for the most part with carapaces rather than sheaves because I am working on an infinite simplicial complex and in moving form limit to co-limit, one in moving from an infinite direct product sort of thing to an infinite co-product sort of thing. Thus,in using limits rather than co-limits, one loses structure just as one does in taking an adic completion of a commutative ring. I have also decided to include a discussion of the homological algebra of carapaces. Now all of this material is some sort of special case of certain kinds of sheaves on sites or the homological algebra of abelian group valued functors, but with all due apologies to those who have worked on those topics, I would prefer to formulate this material in a way which anticipates my intentions. A number of mathematicians have done this sort of thing. Tits and Solomon \cite{art6-keyS}, \cite{art6-keyT}, Stephen Smith and mark Ronan \cite{art6-keyRS} immediately come to mind. But again, working on a infinite complex has dictated that I reformulate many of the elements for this situation.

\section{Carapaces and their Homology}\label{art6-sec-1}

Basic references for this section are \cite{art6-keyMac} and \cite{art6-keyGr}. Basic notions and definition all follow those two sources. A simiplicial complex, $X$, will be a set $ver(X)$ together with a collection of finite subsets of $ver(X)$ such that when ever $\sigma \in X$ every subset of $\sigma$ is in $X$. These finite subsets of $ver(X)$ are the simplices of $X$. The dimension of $\sigma $ is its cardinal less one and the dimension of $X$, if it exists,is the maximum of the dimensions of simplices in $X$. We shall view $X$ as a category, the morphisms being the inclusions of simlices. We identify $ver(X)$ with the zero simplices of $X$. If $X$ and $Y$ are simplicial complexes a morphism of complexes from $X$ to $Y$ is just a convariant functor from $X$ to $Y$; a simplicial morphism is a morphism of complexes taking vertices to vertices.

Let $R$ be a commutative ring with unit fixed for teh remainder of this work and let $M od(R)$ denote the category of $R$-modules. Let $X$ be a simplicial complex.

\begin{definition}\label{art6-def-1.1}
\textit{An $R$-carapace} on $X$ is a convariant functor from $X$ to $M od(R)$. If $A$ and $B$ are two $R$-carapaces on $X$, \textit{a morphism} of $R$-carapaces from $A$ to $B$ is a natural transformation of functors. 
\end{definition}
 If $A$ is an $R$-carapace on $X$ and $\sigma \in X$ is a simplex, then $A(\sigma)$ is called the segment of $A$ along $\sigma$. A sequence of morphisms of $R$-carapaces will be called exact if and only if the corresponding sequence of segments is exact for each simplex in $X$. The product (respectively co-product) of a family of $R$-carapaces is the $R$-carapace whose segments and morphisms are the products (respectively co-products) of those in  the family of carapaces. If $\sigma \subseteq \tau$ is a pair of simplices in $X$ write $e_{A, \sigma}^{\tau}$ or $e_{\sigma}^{\tau}$ when there is no possibility of confusion for the map from $A(\sigma)$ to $A(\tau)$. I will call it the expansion of $A$ form $\sigma$ to $\tau$. Finally, if $A$ and $B$ are two $R$-carapaces on $X$, write $\Hom_{R, X}(A, B)$ for the $R$-module of morphisms of $R$-carapaces from $A$ to $B$.

Let $M$ be an $R$-module. Then let $M_{X}$ denote the constant carapace with value $M$. That is, its segment along any simplex is $M$ and its expansions are all the identity map. In addition for any given simplex, $\sigma$, there are two dually defined carapaces, $M \uparrow_{\sigma}$ and $M \downarrow^{\sigma}$ defined by:

%\setcounter{equation}{1}
\begin{equation}
\begin{aligned}\label{art6-eq-1.2}
M \uparrow_{\sigma} (\tau) &= M(\sigma \subseteq \tau)\\
M \uparrow_{\sigma} (\tau) &= (0)(\sigma \nsubseteq \tau)
\end{aligned}
\end{equation}

\begin{equation}
\begin{aligned}\label{art6-eq-1.3}
M \downarrow^{\sigma} (\tau) &= M\quad(\sigma \supseteq \tau)\\
M \downarrow^{\sigma} (\tau) &= (0)\quad(\sigma \nsupseteq \tau)
\end{aligned}
\end{equation}

 In $M \uparrow_{\sigma}$, the expansions are the identity for paris $\tau$, $\gamma$ such that $\sigma \subseteq \tau \subseteq \gamma$ and 0 otherwise. In $M \downarrow^{\sigma}$ they are the identity for $\tau$, $\gamma$ such that $\tau \subseteq \gamma \subseteq \sigma$ and 0 otherwise.

\begin{lem}\label{art6-lemma-1.4}
Let $X$ be a simplicial complex and let $M$ be an $R$-module and let $A$ be an $R$-carapace on $X$. Then,
\begin{enumerate}[1.]
\item $\Hom_{R, X}(M \uparrow_{\sigma}, A) = Hom_{R}(M, A(\sigma))$\label{art6-enum-lemma1.4-(1)}
\item $\Hom_{R, X}(A , M \downarrow^{\sigma}) = Hom_{R}( A(\sigma), M)$\label{art6-enum-lemma1.4-(2)}
\item If $M$ is $R$-projective, then $M \uparrow_{\sigma}$ is projective in teh category of $R$-carapaces on $X$.\label{art6-enum-lemma1.4-(3)}
\item If $M$ is $R$-injective, then $M \downarrow^{\sigma}$ is injective in the category of $R$-carapaces on $X$.\label{art6-enum-lemma1.4-(4)}
\end{enumerate}
\end{lem}

\begin{proof}
Statements (\ref{art6-enum-lemma1.4-(1)}) and (\ref{art6-enum-lemma1.4-(2)}) require no proof. Note that the functor which assigns to an $R$-carapace on $X$ its segment along $\sigma$ is an exact functor This observation together with (\ref{art6-enum-lemma1.4-(1)}) and (\ref{art6-enum-lemma1.4-(2)}) implies (\ref{art6-enum-lemma1.4-(3)}) and (\ref{art6-enum-lemma1.4-(4)}). 
\end{proof}

\begin{prop}\label{art6-prop-1.5}
Let $X$ be a simplicial complex. Then the category of $R$-carapaces on $X$ has enough projectives and enough injectives.
\end{prop}

\begin{proof}
Let $A$ be any $R$-carapace on $X$. We must show that there is a surjective map from a projective $R$-carapace to $X$ and an injective map from $A$ into an injective $R$-carapace. For each simplex, $\sigma$ in $X$ choose a projective $R$-module, $P_{\sigma}$, with a surjective map $\pi_{\sigma}$ mapping $P_{\sigma}$ onto $A(\sigma)$. Let
$$
P = \coprod\limits_{\sigma \in X}P_{\sigma} \uparrow_{\sigma} 
$$
For each $\sigma$, let $\overline{\pi}_{\sigma}$ be the morphism of carapaces corresponding to $\pi_{\sigma}$ given by \ref{art6-lemma-1.4}, \ref{art6-enum-lemma1.4-(1)}. Define $\pi$ by:
$$
\pi = \coprod\limits_{\sigma \in X}\overline{\pi}_{\sigma}.
$$
Then $\pi$ is a surjective map from a projective to $A$.

To construct an injective, choose an injective module and an inclusion, $j_{\sigma}:A(\sigma) \rightarrow I_{\sigma}$. Then define an injective carapace, $I$, and an inclusion, $j$, as products of the carapaces $I_{\sigma} \downarrow^{\sigma}$ and inclusions $\overline{j}_{\sigma}$ defined dually to the corresponding objects in the projective case. 
\end{proof}

\begin{prop}\label{art6-prop-1.6}
Let $X$ be a simplicial complex. Then
    \begin{enumerate}[(1)]
        \item If $P$ is a projective $R$-caparapace on $X$, then $P(\sigma)$ is $R$ projective for each $\sigma\in X$.\label{art6-prop1.6-enum-(1)}
        \item If $I$ is an injective $R$-carapace on $X$, then $I(\sigma)$ is $R$ injective for each $\sigma \in X$.\label{art6-prop1.6-enum-(2)}
    \end{enumerate}
\end{prop}

\begin{proof}
Suppose $P$ is projective. For each $\sigma$ let $\pi_{\sigma}: F_{\sigma}\rightarrow P(\sigma)$ be a surjective map from a projective $R$-module onto  $P(\sigma)$. Then $Q= \coprod_{\sigma \in X}F_{\sigma} \uparrow_{\sigma}$ is projective by \ref{art6-lemma-1.4} and $\coprod_{\sigma \in X}\pi_{\sigma}$ maps $Q$ onto $P$. Since $P$ is projective, it is a direct summand of $Q$. But then $P(\sigma)$ is a  direct summand of $Q(\sigma) = \coprod_{\tau \subseteq \sigma}F_{\tau}$ which is clearly projective. This establishes the first statement.

To prove the second statement, for each $\sigma$ choose and embedding,, $j_{\sigma} : I(\sigma) \rightarrow J_{\sigma}$ where $J_{\sigma}$ in $R$-injective. Then use $\prod_{\sigma\in X}$ to embed $I$ in $\prod_{\sigma \in X}J_{\sigma}$ and reason dually to the previous argument replacing, at each point where it occurs, the co-product with the product. 
\end{proof}

It is clear that there are natural homology and cohomology theories on the category of $R$-carapaces on $X$. The most natural functors to consider are the limit and the co-limit over $X$. To simplify the discussion, \textit{$C ar_{R}(X)$ denote the category of $R$-carapaces on $X$.}

\begin{definition}\label{art6-prop-1.7}
Let $U$ denote any sub-category of $X$, and let $A$ be an $R$-carapace on $X$. Then
\begin{align*}
\Sigma (U, A) &= \lim\limits_{\substack{\longrightarrow \\ \sigma \in U}}A(\sigma)\\
\Gamma (U, A) &= \lim\limits_{\substack{\longleftarrow \\ \sigma \in U}}A(\sigma).
\end{align*}
We will refer to $\Sigma(U, A)$ as the \textit{segment} of $A$ over $U$ and to $\Gamma(U, A)$ as the \textit{sections} of $A$ over $U$.
\end{definition}

Certain observation are in order. Since subcategories of $X$ are certainly not in general filtering the functor, $\Sigma(U, ?)$, is right exact on $C ar_{R}(X)$. Similarly $\Gamma(U, ?)$ is left exact. Furthermore $\Gamma(U, A) = \Hom_{R, U}(R_{U}, A)$, On the other hand, $\Sigma(U, A)$ cannot be represented as a homomorphism functor in any obvious way but its definition as a direct limit allows us to conclude that:
$$
\Hom_{R}(\Sigma(X, A), M) = \Hom_{R, X}(A, M_{X})
$$ 

\begin{definition}\label{art6-definition-1.9}
For any $R$-carapace on $X$, A, let
$$
H_{n}(X, A)= L_{n}\Sigma{X, A}
$$
and let
$$
H^{n}(X, A) = R^{n}\Gamma(X, A)
$$
the left and right derived functors of $\Sigma(X, -)$ and $\Gamma(X, -)$ respectively. These groups shall be referred to as the \textit{expskeletal homology and cohomology groups} of $X$ with coefficients in $A$.
\end{definition}

\begin{example}
\textbf{The Koszul Resolution of the Constant Carapace.}

Choose an ordering on the vertices of $X$. For each $r \geq 0$, let $X(r)$ denote the set of simplices of dimension $r$. Let $K_{q}(X, R) = \coprod _{\sigma \in X_{(q)}} R \uparrow_{\sigma}$. If $A$ is any $R$-carapace on $X$, then $\bigwedge_{R}^{q}A$ is understood to be the carapace whose segment along $\sigma$ is $\bigwedge_{R}^{q}(A(\sigma))$. Then, it is not at all difficult to see that $\bigwedge_{R}^{q+1}K_{0}(X, R) = K_{q}(X, R)$ Furthermore, when
$\sigma \subseteq \tau$ there is always a natural map from $M\uparrow_{\tau}$ to $M \uparrow_{\sigma}$ obtained by applying 4,1 to $M\uparrow_{\tau}$ and noting that since $M\uparrow_{\sigma}(\sigma) = M = M\uparrow(\tau)$ there is a map in $\Hom_{R, X}(M\uparrow_{\tau}, MM\uparrow_{\sigma})$ corresponding to the identity. Make use of the ordering on the vertices of $X$ to define an alternating sum of the maps corresponding to the faces of a simplex. It is easy to see the that this gives a complex of carapaces:
$$
\ldots \rightarrow (X, R)\rightarrow K_{q-1}(X, R)\rightarrow\ldots \rightarrow K_{0}(X, R)\rightarrow R_{X}\rightarrow(0)
$$ 
Then check that the sequence of segments on $\sigma$ is just the standard Koszul resolution of the unit ideal which begina with a free module of rank $\dim(\sigma) + 1$ and the map which sends each of its generators to one. Consequently, this construction gives a resolution of the constant carapace by projectives. On the other hand it is evident that $\sigma(X, K_{q}(X, R))$ is just the $R$-module of simplicial $q$-chains on $X$ with coefficients in $R$ and that the boundaries are the standard simplicial boundaries. In this way, one verifies that the exoskeletal homology and cohomology with coefficients in a constant carapace is just the simplicial homology and cohomology. This phenomenon has been observed and exploited by Casselman and Wigner in their work on admissible representations and the cohomology of artihmetic groups \cite{art6-keyCW}.
\end{example}

\section{Operations on Carapaces}\label{art6-sec-2}

In the category of $R$-carapaces on $X$ there is a self-evident notion of tensor product:

\begin{definition}\label{art6-definition-2.1}
Let $A$ and $B$ be two $R$-carapaces on $X$. Let:
$$
(A \otimes_{R} B)(\sigma) = A(\sigma) \otimes_{R}B(\sigma)
$$
Then $A\otimes_{R}B$ is a carapace which we will refer to as \textit{the tensor product} of $A$ and $B$.

The properties of the tensor product are for the most part clear. Most of them are stated in the following.
\end{definition}

\begin{prop}\label{art6-prop-2.2}
Let $X$ be a simplicial complex. Then the tensor product of $R$-carapaces on $X$ is an associative, symmetric, bi-additive functor right exact in both variables. Furthermore:
   \begin{enumerate}[(1)]
   \item For any $R$-carapace, $A$, $R_{X}\otimes_{R}A \simeq A$\label{art6-prop2.2-enum-(1)}
   \item If $P$ is a projective $R$-carapace then tensoring with $P$ on either side is 
   an exact functor.\label{art6-prop2.2-enum-(2)}
   \item For any two $R$-carapaces, $A$ and $B$, there is a natural map:
      $$
      t_{A, B}:\Sigma(X, A\otimes_{R}B)\rightarrow \Sigma(X, A)\otimes_{R}\Sigma(X, B)
      $$
      Moreover, $t_{A, B}$ is a natural transformation in $A$ and $B$ and it is functorial in $X$ as well.\label{art6-prop2.2-enum-(3)}
   \end{enumerate}
\end{prop}

\begin{proof}
The first statement is self-evident; the second follows trivially from \ref{art6-prop-1.6} but the third might require some comment. To construct $t_{A, B}$ let $e_{A, \sigma} : A(\sigma) \rightarrow \Sigma(X, A)$ and $e_{B, \sigma} : B(\Sigma) \rightarrow \Sigma(X, B)$ be the expansions. Then $e_{A, \sigma} \otimes_{R} e_{B, \Sigma}$ maps $A\otimes_{R} B(\sigma)$ into $\Sigma(X, A) \otimes_{R} \Sigma(X, B)$ compatibly with respect to expansion. Since $\Sigma(X, A \otimes_{R} B)$ is a colimit, this defines $t_{A, B}$ uniquely and ensures that it is functorial as asserted. 
\end{proof}

The construction of a tensor product is thus quite straightforward but the construction of an internal homomorphism functor with the requisite adjointness properties presents certain technical difficulties. For any $\sigma \in X$ let $x(\sigma)$ denote the subcategory of $X$ whose objects are the simplices $\tau$ such that $\tau \supseteq \sigma$. The morphisms of $X(\sigma)$ are inclusions of simplices. Then $X(\sigma)$ is not a subcomplex of $X$. If $\sigma \subseteq \tau$ then $X(\tau) \subseteq X(\sigma)$. There is a natural restriction map from the group $\Hom_{R, X (\sigma)}(A|_{X(\sigma)}, B|_{X(\sigma)})$ to $\Hom_{R, X(\tau)}(A|_{X(\tau)}, B|_{X(\tau)})$. We will will write $e_{\calH, \sigma}^{\tau}$ for this restriction map.

\begin{definition}\label{art6-definition-2.3}
Let $A$ and $B$ be two $R$-carapaces on $X$. \textit{The carapace of local homomorphisms} Fro  $A$ to $B$ will be written $\calH om_{R, X}(A, B)$. Its value on $\sigma$ is
$$
\calH om_{R, X}(A, B)(\sigma) = \Hom_{R, X(\sigma)}(A|_{X}(\sigma), B|_{X}(\sigma))
$$
Its expansions are the maps $e_{\calH, \sigma}^{\tau}$
\end{definition}

For want of a direct reference, we include some discussion of the basic properties of $\calH om$.

\begin{theorem}\label{art6-thm-2.4}
The local homorophism functor, $\calH om_{R, X}(A, B)$, is additive, covariant in $B$ and contravariant in $A$ and left exact in both variables. Moreover
    \begin{enumerate}[(1)]
    \item $\calH om_{R, X}(R_{X}, A)\simeq A$, \text{\rm functorially in} $A$\label{art6-thm2.4-enum-(1)}
    \item $\Gamma(X, \calH om_{R,X}(A, B)) = \Hom_{R,X}(A, B)$\label{art6-thm2.4-enum-(2)}
    \item There is a canonical isomorphism functorial in $A, B$, and $C$,\label{art6-thm2.4-enum-(3)}
        $$
        \phi : \Hom_{R, X}(A, \calH om_{R, X}(B, C))\rightarrow \Hom_{R, X}(A \otimes_{R} B, C)
        $$
    \item There is a functorial isomorphism:\label{art6-thm2.4-enum-(4)}
    $$
    \Hom_{R, X}(R_{X}, \calH om_{R, X}(A, B)) \simeq \Hom_{R, X}(A, B)
    $$  
    \end{enumerate}
\end{theorem}

\begin{proof}
Of the three preliminary statements, only left exactness requires comment. What must be shown is the left exactness of segments of $\calH om_{R, X}(A, B)$ as $A$ and $B$ vary over short exact sequences. But\break $\calH om_{R, X}(A, B)(\sigma) = \Hom_{R,X(\sigma)}(A|_{X(\sigma)}, B_{X(\sigma)})$. Restriction to $X(\sigma)$ is exact and $\Hom_{R, X}(\sigma)$ is left exact in both of its arguments. The requisite left exactness follows immediately.

To establish (\ref{art6-thm2.4-enum-(1)}), we must establish the isomorphism on segments. But
$$
\calH om_{R, X}(R_{X}, A)(\sigma) = \Hom_{R,X(\sigma)}(R_{X}, A|_{X(\sigma)}),
$$
but this last expression is equal to:
$$
\lim_{\substack{\longleftarrow \\ \tau \in X(\sigma)}}A(\tau).
$$
However the category, $X(\sigma)$ has an initial element and so the projective limit is just $A(\sigma)$.

For Statement (\ref{art6-thm2.4-enum-(2)}), write:
$$
\Gamma(X, \calH om_{R,X}(A, B)) = \lim_{\substack{\longleftarrow \\ \tau \in X}}(A|_{X(\sigma)}, B|_{X(\sigma)})
$$
There is a natural map from $\Gamma(X, \calH om_{R,X}(A, B))$ to this projective limit. Just send $f$ to the element in the limit whose component at $\sigma$ is the restriction (the pull-back actually) of $f$ to the sub-category, $X(\sigma)$. It is a triviality to verify that this map is an isomorphism.

Rather than giving a fully detailed proof of (\ref{art6-thm2.4-enum-(3)}), we will give complete definitions of $\phi$ and a map $\psi$ which is inverse to it. The necessary verifications, though numerous and quite technical, contain no surprises and so we leave them to the reader. First suppose\break $f \in \Hom_{R,X}(A, \calH om_{R,X}(B, C))$. Then for
$a \in A(\sigma)$, $f_{\sigma}(a)$ is a family, $f_{\sigma}(a)  =\{f(a)_{\tau}\}_{\tau \supseteq \sigma}$ where
$f(a)_{\tau} \in Hom_{R}(B(\tau), C(\tau))$. The following equations express the facts that $f$ is a carapace morphism and that $f(a)$ is carapace morphism from $B|_(\sigma)$ to $C|_{x(\sigma)}$:
\begin{equation}\label{art6-eq-5}
f_{\tau}(e_{A,\sigma}^{\tau}(a))_{\gamma} = f_{\sigma}(a)_{\gamma}%\tag{5}\label{eq:proof-5}
\end{equation}
\begin{equation}\label{art6-eq-6}
e_{C, \tau}^{\gamma} \circ f_{\tau}(a)_{\tau} = f_{\tau}(a)\gamma \circ e_{B,\tau}^{\gamma}
\end{equation}
Then we may define $\phi : \Hom_{R,X}(A, \calH om_{R,X}(B,C))\rightarrow \Hom_{R,X}(A \otimes_{R}B, C)$ and $\psi$ opposite to it by the equation:
\begin{equation}\label{art6-eq-7}
\phi(f)_{\sigma}(a\otimes b) =[f_{\sigma}(a)_{\sigma}](b) \quad a\in A(\sigma), b\in B(\sigma)
\end{equation}
\begin{equation}\label{art6-eq-8}
[\phi (F)_{\sigma}(a)]_{\tau}(b) = f_{\tau}([e_{A, \sigma}^{\tau}(a)\otimes b) \quad a\in A(\sigma), b \in B(\tau)
\end{equation}
These are the two maps, inverse to one another, which establish (\ref{art6-thm2.4-enum-(3)}).

The last statements is obtained by applying the third to the left hand side and observing that $R_{X}\otimes_{R}A\simeq A$.
\end{proof}

\section{Derived Functors}\label{art6-sec-3}

In this section we introduce the most elementary derived functors on $C ar_{R}(X)$. These include the derived functors of both the module valued and the carapace valued tensor and homomorphism functors and the relation between the two. First recall that by \ref{art6-lemma-1.4}, whenever $P$ is a projective $R$-module and $\sigma$ is a simplex in $X$ then $P\uparrow_{\sigma}$ is a projective $R$-carapace. Consequently a coproduct of carapaces of the form $F \uparrow_{\sigma}$, where $F$ is a free $R$-module, is a projective $R$-carapace. We will refer to such carapaces as elementary projectives. Furthermore notice that if $M$ is any $R$-module then $\sigma(X, M\uparrow_{\tau})=M$. Hence if $Q$ is an elementary projective, $\sigma(X, Q)$ is a direct sum of free modules and hence free.

\begin{lemma}\label{art6-lemma-3.1}
Let $X$ be any simplicial complex.
    \begin{enumerate}[(1)]
    \item Every $R$-carapace on $X$ is a surjective image of an elementary projective.\label{art6-lemma3.1-enum-1}
    \item If $P$ is a projective $R$-carapace on $X$, then $\sigma(X, P)$ is $R$-projective.\label{art6-lemma3.1-enum-2}
    \item A tensor product of elementary projectives is an elementary projective.\label{art6-lemma3.1-enum-3}
    \item A tensor product of projectives is projective.\label{art6-lemma3.1-enum-4}
    \end{enumerate}
\end{lemma}

\begin{proof}
Let $A$ be an $R$-carapace on $X$. For each $\sigma \in X$ let $F_{\sigma}$ be a free $R$-module and let $q_{\sigma} : F_{\sigma} \rightarrow A(\sigma)$ be surjective morphism. Then, just as in \ref{art6-prop-1.5}, $\coprod_{\sigma \in X}F_{\sigma} \uparrow_{\sigma}$ is an elementary projective and $\coprod_{\sigma \in X}Q_{\sigma}$ maps in onto $A$. Thus \ref{art6-lemma3.1-enum-1}) is established.

To prove \ref{art6-lemma3.1-enum-2}), choose an elementary projective, $F$, and a surjective map, $q: F\rightarrow P$. Since $P$ is projective, it is a direct summand of $F$ and so $\sigma(X, P)$ is direct summand of $\sigma(X, F)$ which is free. This establishes \ref{art6-lemma3.1-enum-2}.

The fourth statement follows from the third because, by \ref{art6-lemma3.1-enum-1}, every projective is a direct summand of an elementary projective. Hence we must prove \ref{art6-lemma3.1-enum-3}). But this reduces to proving that is $\sigma$ and $\tau$ are two simplices, then $R\uparrow_{\sigma} \otimes_{R}R \uparrow_{\tau}$ is elementary projective. But $R\uparrow_{\sigma}\otimes_{R}R\uparrow_{\tau}(\gamma) \neq 0$ if and only if $\gamma \supseteq \sigma$ and $\gamma \supseteq \tau$. Thus $R\uparrow \otimes_{R}R \uparrow_{\tau}\neq (0)$ if and only if $\sigma \cup \tau = \alpha$ is a simplex and then $R\uparrow_{\sigma}\otimes_{R}R \uparrow_{\alpha}$. This establishes the result. 
\end{proof}

There are at least four very obvious homological bifunctors on\break $C ar_{R}(X)$

\begin{definition}\label{art6-definition-3.2}
Let $A$ and $B$ be $R$-carapaces on $X$. Write $\Ext_{R, X}^{r}$ for the $r$'th right derived functor of the left exact module valued bifunctor, $\Hom_{R, X}(a, B)$. Write $\calE xt_{R, X}^{q}$ for the carapace valued $q$'th right derived functor of the carapace valued local homomorhism functor. Write $\calT or_{q}^{R,X}(A, B)$ for the $q$'the carapace valued left derived functor of the carapace valued tensor product, $A\otimes_{R}B$ and write $\Tor_{q}^{R, X}(A, B)$ for the $q$'th left derived functor of the right exact bifunctor $\sigma(X, A\otimes_{R}B)$. 
\end{definition}

\begin{lemma}\label{art6-lemma-3.3}
If $P$ is a projective $R$-carapace on $X$, then for any $R$-carapace, $A$, $P\otimes_{R} A$ is $\sigma$-acyclic and $\calH om_{R, X}(P, A)$ is $\Gamma$-acyclic.
\end{lemma}

\begin{proof}
Let $\{Q_{j}\}$, $j \leq 0$ be  a projective resolution of $A$. Then $\{P \otimes_{R} Q_{j}\}$ is a projective resolution of $P \otimes_{R} A$. Apply the functor $\sigma$ to obtain $\Tor_{j}^{R,X}(A, P) = H_{j}(A\otimes_{R}P)$. But $\Tor_{j}^{R, X}(A, P) = (0)$ because $P$ is projective. This establishes the $\sigma$-acyclicity of $A\otimes_{R}P$. For the other acyclicity, let $\{K_{j}\}$ be a projective resolution of $R_{X}$. Then\break $\Hom_{R, X}(K_{j}, \calH om_{R, X}(P, A)) = \Hom_{R, x}(K_{j}\otimes_{R}P, A)$. But $K_{j} \otimes P$ is a projective resolution of $P$. Thus $H^{j}(X, \calH om_{R,X}(P, A)) = \Ext_{R, X}^{j}(P, A)$ which is $(0)$ because $P$ is projective.
\end{proof}

The elementary properties of these four functors are described in the following.

\begin{prop}\label{art6-lemma-3.4}
Let $A$ and $B$ be $R$-carapaces on $X$, Then
    \begin{enumerate}[(1)]
    \item $\calT or_{q}^{R,X}(A, B)(\sigma) = Tor_{q}^{R}(A(\sigma), B(\sigma))$\label{art6-lemma3.4-enum-1}
    \item $\Tor_{q}^{R, X}(R_{X}, A) = H_{q}(X, A)$\label{art6-lemma3.4-enum-2}
    \item $\Ext_{R,X}^{q}(R_{x}, A) = H^{q}(X,A)$\label{art6-lemma3.4-enum-3}
    \item There is a spectral sequence with $E_{2}^{p,q}$ term:\label{art6-lemma3.4-enum-4}
    $$
    E_{2}^{p,q} = H^{p}(X, \calE xt_{R,X}^{q}(A, B))
    $$
    and whose abutment is:
    $$
    \Ext_{X,R}^{p+q}(A, B)
    $$
    \item There is a spectral sequence with $E_{p,q}^{2}$ term:\label{art6-lemma3.4-enum-5}
    $$
    E_{p,q}^{2} = H_{p}(X, \calT or_{q}^{R, X}(A, B))
    $$
    and with abutment:
    $$
    \Tor_{p+q}^{R,X}(A, B)
    $$
    \end{enumerate}
\end{prop}

\begin{proof}
To prove \ref{art6-lemma3.4-enum-1}), let ${P_{j}}$ be a projective resolution of $A$. Then for each $\sigma$, $\{P_{j}(\sigma)\}$ is a projective resolution of $A(\sigma)$. Moreover the segment of $P_{j}\otimes_{R}B$ along $\sigma$ is $P_{j}\otimes_{R}b(\sigma)$. But the segment along $\sigma$ is an exact functor on $C ar_{R}(X)$ and so the $\sigma$-segment of the homology of the complex, $P_{j}\otimes_{R}B$ is the homology of the complex $P_{j}(\sigma) \otimes_{R}B(\sigma)$. This is just the result desired.

Statements \ref{art6-lemma3.4-enum-2} and \ref{art6-lemma3.4-enum-3} are both essentially trivial. Just note that\break $\Tor_{q}^{R, X}(R_{X}, A)$ (respectively $\Ext_{R,X}^{q}(R_{X}, A)$) is a connected sequence of homological functors acyclic on projectives (respectively injectives) and that $\Tor_{0}^{R,X}(R_{X},A)=\Sigma(X, A)$ (respectively $\Ext_{R,X}^{0}(R_{X},A) = \Gamma(X,A))$. The two statement follow.

The local global spectral sequences in \ref{art6-lemma3.4-enum-4}) and \ref{art6-lemma3.4-enum-5}) ar just composition of two functor sequences as in \cite{art6-keyGr}. Let $F_{A}(B) = A\otimes_{R}B$ and let $G_{A}(B) = \calH om_{R,X}(B,A)$. By Lemma \ref{art6-lemma-3.3}, $F_{A}$ carries projectives to $\Sigma$-acyclics and $G_{A}$ carries projectives to $\Gamma$-acyclics. The left derived functors of $F_{A}$ are the functors, $\calT or_{q}^{R,X}(A,-)$ while the right derived functors of$G_{A}$ are the functors, $\calH om_{R,X}^{q}(A, -)$. The construction of the spectral sequences is standard.
\end{proof}

\section{Homological Dimension}\label{art6-sec-4}

In this section we will determine the homological dimension of $C ar_{R}(X)$ for a finite dimensional simplicial complex, X. Write $X^{r}$ for the set of simplices in $X$ of dimension $r$ and write $X_{n}$ for $\bigcup_{r\geq n}X^{(r)}$. If $M$ is an $R$-module,write $pd(M)$ for the projective dimension of $M$ and write $hd(R)$ for the homological dimension of $R$. 

\begin{definition}\label{art6-defin-4.1}
Let $A$ be an $R$-carapace on $X$.
    \begin{enumerate}[(1)]
    \item Write $s(A)\leq n$ if $A(\sigma) = (0)$ whenever $dim(\sigma) < n$ and let $s(A) = inf\{n: s(A) \leq n\}$. Then $s(A)$ is called the \textit{support dimension} of $A$.\label{art6-defin1-enum-1}
    \item We say that $A$ is \textit{locally bounded} if $pd(A(\sigma)) \leq q$ for some fixed $q \geq 0$. In that case let $ld(A)=sup\{pd(A(\sigma)) : \sigma \in X\}$. When it exists, $ld(A)$ is called the \textit{local projective dimension} of $A$.\label{art6-defin1-enum-2}
    \end{enumerate}
\end{definition}

\begin{prop}\label{art6-proposition-4.2}
Suppose that $A$ is an $R$-carapace on $X$ of support dimension at least $n$ and local projective dimension $r$. Then  there is an exact sequence:
%\setcounter{equation}{2}
\begin{equation}\label{art6-equation-4.3}
0 \rightarrow A^{1} \rightarrow P_{r}\rightarrow\ldots \rightarrow P_{0} \rightarrow A \rightarrow 0
\end{equation}
so that:
    \begin{enumerate}[(1)]
    \item $ld(A^{1}) = 0$\label{art6-equation-enum-1}
    \item $P_{i}$ is projective\label{art6-equation-enum-2}
    \item $s(A^{1})\geq n+1$\label{art6-equation-enum-3}
    \item $s(P_{i}) \geq n$\label{art6-equation-enum-4}
    \end{enumerate}
\end{prop}

\begin{proof}
For each $\sigma \in X_{n}$ choose a projective module, $Q^{\sigma}$ and a surjective morphism, $\phi^{\sigma} : Q^{\sigma} \rightarrow A(\sigma)\rightarrow 0$. For each $\sigma$, let $\phi_{\sigma} : Q^{\sigma}\uparrow_{\sigma} \rightarrow A$ be the morphism of carapace induced by $\phi^{\sigma}$. Let $Q_{0}= \coprod_{\sigma\in X_{n}}(Q^{\sigma})\uparrow_{\sigma}$ and let $d_{0}=\coprod_{\sigma \in X_{n}}\phi_{\sigma}$. Since $s(A)\leq n$, $d_{0}$ in surjective. Let $N_{0} =ker(d_{0})$. Then, clearly $s(N_{0}) \leq n$ but $ld(N_{0})\leq r-1$. Thus we may repeat the process the process with $A$ replaced by $N_{0}$ and continue inductively until we obtain an exact sequence:
$$
0 \rightarrow N_{r-1} \rightarrow Q_{r-1} \rightarrow\ldots \rightarrow Q_{0} \rightarrow A \rightarrow 0.
$$
In this sequence, the $Q_{i}$ are projective, the support dimensions of the $Q_{i}$ and of $N_{r-1}$ are at least $n$ and $ld(N_{r-1})$ are at least $n$ and $ld(N_{r-1})=0$. That is $N_{r-1}(\sigma)$ is projective for each $\sigma$. Now let $Q_{r}=\coprod_{\sigma \in X_{n}}(N_{r-1}(\sigma))\uparrow_{\sigma}$. Clearly, $Q_{r}$ maps onto $N_{r-1}$ and the map is an isomorphism on segments over simplices of dimension $n$. Let $d_{r}$ be the composition of the map onto $N_{r-1}$ with the inclusion into $Q_{r-1}$ and let $A^{1} =ker(d_{r})$. Clearly, $A^{1}$ and the $Q_{i}$ answer the requirements of the proposition.
\end{proof}

\begin{theorem}\label{art6-thm-4.4}
Let $X$ be a simplicial complex of dimension $d$ and let $A$ be a locally bounded $R$-carapace on $X$ of local projective dimension $r$. Then $pd(A)\leq d+r$.
\end{theorem}

\begin{proof}
 Apply Proposition 4.2 with $n=0$. The result is the exact sequence:
 $$
 0 \rightarrow A^{1} \rightarrow P_{r}\rightarrow \ldots \rightarrow P_{0}\rightarrow A \rightarrow 0
 $$
 Then apply 4.2 to $A^{1}$ observing that $ld(A^{1})= 0$ and $s(A^{1})\leq 1$. The result is a short exact sequence, $0 \rightarrow A^{2} \rightarrow Q_{1} \rightarrow A^{1} \rightarrow 0$ where $Q_{1}$ is projective, $ld(A^{2})=0$ and $s(A^{2})\leq 2$. We may continue until we reach $s(A^{d})\leq d$. But for any $B$, if the segments of $B$ are projective and $s(B)\leq dim(X)$ then $B$ is projective. We may thus assemble these short sequences and the sequence of $P_{i}$ to obtain a sequence:
 $$
 0\rightarrow A^{d} \rightarrow Q_{d-1} \rightarrow \ldots \rightarrow Q_{1} \rightarrow P_{r} \rightarrow \ldots \rightarrow
 P_{0}\rightarrow A \rightarrow 0
 $$
 This exact sequence is the projective resolution establishing the result.
 \end{proof}

\begin{coro}\label{art6-coro-4.5}
If $dim(X) =d$ and if $M$ is an $R$-module of projective dimension $r$, then $pd(M_{X})\leq r+d$. If $M$ is projective then $pd(M_{X}) \leq d$. 
\end{coro}

\begin{coro}\label{art6-coro-4.6}
If $R$ is of homological dimension $r$ and $dim(X)= d$ then the homological dimension of $C ar_{R}(X)$ is at most $d+r$.
\end{coro}

Neither of these corollaries requires so much as one word of proof.

\section{Carapaces and Morphisms of Complexes}\label{art6-sec-5}
Recall that a morphism of complexes from $X$ to $Y$ is just a covariant functor from the category of simplices in $X$ to the category of simplices in $Y$; it is simplicial if it carries vertices to vertices. If $S$ is a subset of the vertex set of $X$ then it admits a simplicial complex structure by taking as its set of simplices the set of simplices in $X$ all fo whose vertices lie in $S$. We will write $\tilde{S}$ fof this complex. When we speak of a subcategory of $X$ we will always, unless otherwise indicated, mean a full subcategory of the simplex category of $X$. If $U$ and $V$ are subcategories if $X$  we will write $U \subset V$ to indicate that $U$ is a full subcategory of $V$. In this case there is always a functorial map from $\Sigma(U, A)$ to $\Sigma(V, A)$ for any carapace, $A$.

Suppose that $f$ is a morphism of complexes from $X$ to $Y$ and that $Z$ is a simplicial sub-complex of $Y$. Let $f^{-1}(Z)$ denote the simplicial sub-complex of $X$ which has as its set of simplices the set $\{\sigma \in X : f(\sigma)\in Z \}$. Clearly, $f^{-1}$ is a functor from the subcomplexes in $Y$ to those in $X$.

\begin{definition}\label{art6-definition-5.1}
Let $X$ and $Y$ be simplicial complexes, let $A$ be a  $R$-carapace on $X$ and let $B$ be one on $Y$. Let $F:X \rightarrow Y$ be a morphism of complexes.

    \begin{enumerate}[(1)]
    \item Let $(f^{*}(B))(\sigma)=B(f(\sigma))$ and let $ e_{f^{*}(B), \sigma}^{\tau} = e_{B, f(\sigma)}^{f(\tau)}$. Then $f^{*}(B)$ is an $R$-carapace on $X$ and $f^{*}$ is a covariant functor from $C ar_{R}(Y)$ to $C ar_{R}(X)$. The carapace $f^{*}(B)$ is called the \textit{inverse image} of $B$ under $f$. \label{art6-definition5.1-enum-(1)}
    
    \item Let $(f_{*}(A))(\sigma) = \Sigma(f^{-1}(\tilde{\sigma}), A)$ and let $e_{f_{*}(A), \sigma}^{\tau}$ be the natural map of segments induces by the inclusion of categories, $f^{-1}(\tilde{\sigma})\subset f^{-1}(\tilde\tau)$ when $\sigma \subset \tau$. Then $f_{*}$ is a covariant functor from carapaces on $X$ to carapaces on $Y$. The carapaces, $f_{*}(A)$ is called the \textit{direct image} of $A$ by $f$.\label{art6-definition5.1-enum-(2)}
    \end{enumerate}
    
    Both $f^{*}$ and $f_{*}$ are additive. In addition they satisfy the adjointness properties expected.
\end{definition}

\begin{theorem}\label{art6-thm-5.2}
Let $X$ and $Y$ be simplicial complexes, let $f$ be a morphism of complexes from $X$ to $Y$, let $A$ be an $R$-carapace on $X$ and let $B$ be one on $Y$.
    \begin{enumerate}[(1)]
    \item $f^{*}$ is exact.\label{art6-thm5.2-enum-1}
    \item $f_{*}$ is right exact.\label{art6-thm5.2-enum-2}
    \item $f_{*}$ is left adjoint to $f^{*}$. That is,
    $$
    \Hom_{R, A}(A, f^{*}(B))\simeq \Hom_{R,Y}(f_{*}(A), B)
    $$
    functorially in $A$ and $B$.\label{art6-thm5.2-enum-3}
    \end{enumerate}
\end{theorem}

\begin{proof}
The first statement is a triviality. The second statement in nothing more than the right exactness of co-limits. Thus only the last statement requires attention.

To prove \ref{art6-thm5.2-enum-3}), we will give morphisms,
$$
\psi : \Hom_{R,X}(A, f^{*}(B))\rightarrow \Hom_{R,Y}(f_{*}(A), B)
$$
and $\phi$ inverse to it. Begin with $\psi$. If $\alpha \in  \Hom_{R,X}(a, f^{*}(B))$, write $\alpha =\{\alpha_{\sigma}\}_{\sigma \in X}$. Then $\alpha_{\sigma}$ maps $A(\sigma)$ to $B(f(\sigma))$ for each $\sigma$ compatibly with respect to $\sigma$. Then  $\sigma \in f^{-1}(\tilde{\rho})$ if and only if $f(\sigma)\subseteq \rho$. Thus the set of maps, $e_{B, f(\sigma)}^{\rho} \circ \alpha_{\sigma}$ is direct system of maps giving a morphism from $[f_{*}(A)](\rho)= \Sigma(f^{-1}(\tilde{\rho}), A)$ to $B(\rho)$. For each $\rho$ call this map $\beta_{\rho}$. Then since $\beta_{\rho}$ is functorial in $\rho$, the family $\{\beta_{\rho}\}_{\rho \in Y}$ is a morphism, $\beta$, from $f_{*}(A)$ to $B$. Let $\psi (\alpha) = \beta$.

Now we wish to define $\phi$ . If $\beta \in \Hom_{R,Y}(f_{*}(A), B)$ then $\beta$ is a family $\{B_{\rho}\}_{\rho in Y}$ where $\beta_{\rho}$ maps $\Sigma(f^{-1}(\tilde{\rho}), A)$ to $B(\rho)$. For any $\sigma$ in $X$, let $\rho = f(\sigma)$ and let $a_{\sigma}$ be the natural map from $A(\sigma)$ to $\Sigma(f^{-1}(\tilde{\rho}), A)$. Then $\beta_{\rho} \circ a_{\sigma}$ is a map from $A(\sigma)$ to $B(f(\sigma))$ for each $\sigma$. Let $\alpha_{\sigma} = \beta_{f(\sigma)}$ for each $\sigma$ and let $\phi(\beta)$ be the map $\alpha = \{\alpha_{\sigma}\}_{\sigma \in X}$. We leave the task of verifying that $\psi$ and $\phi$ are maps of the requisite type and that they are inverse to one another to the reader.
\end{proof}

It is entirely expected the $f^{*}$ has a left adjoint. It is a bit surprising, though not at all subtle, that is also has a right adjoint. Let $f : X \rightarrow Y$ be a morphism of complexes. For $\rho \in Y$ let $f^\dagger(\rho)$ denote the sub-category of $X$ consisting of all $\sigma \in X$ such that $f(\sigma) \supseteq \rho$.

\begin{definition}\label{art6-definition-5.6}
Let $X$ and $Y$ be simplicial complexes, let $f : X\rightarrow Y$ be a morphism of complexes and let $A$ be an $R$-carapace on $X$. Define an $R$-carapace on $Y$ by the equation:
$$
f_{\dagger}(A)(\rho) = \Gamma(f^{\dagger}(\rho), A)
$$
This is clearly an $R$-module valued functor on the simplex category of $Y$ and so it is an $R$-carapace on $Y$. We will call it the \textit{right direct image} of $A$ under $f$.
\end{definition} 

\begin{prop}\label{art6-prop-5.4}
Let $f: X\rightarrow Y$ be a morphism of complexes, let $A$ be an $R$-carapace on $X$ and let $B$ be one on $Y$. Then $f_{\dagger}$ is left exact and right adjoint to $f^{*}$. That is,
$$
\Hom_{R,X}(f^{*}(B), A)\simeq \Hom_{R,Y}(B, f_{\dagger}(A))
$$
functorially in $A$ and $B$.
\end{prop}

\begin{proof}
Left exactness follows from the left exactness of $\Gamma$ and so we only need to establish the adjointness. We give the two morphisms. Let
$$
\mu : \Hom_{R, Y}(B, f_{\dagger}(A)) \rightarrow \Hom_{R,X}(f^{*}(B), A)
$$
be one of the two morphisms and let $\zeta$ be its inverse.

Choose $\delta$ in $Hom_{R,Y}(B, f_{\dagger}(A))$. For each $\rho \in Y$, $\delta$ takes each element, $b \in B(\rho)$ to a compatible family, $\{[\delta_{\rho}(b)]\sigma\}_{f(\sigma)\supseteq \rho}$ where $[\delta_{\rho}(b)]_{\sigma} \in A(\sigma)$. For each $\sigma$ we must give a map $\eta (\delta)_{\sigma}: B(f(\sigma)) \rightarrow A(\sigma)$. Let
$$
\left[\eta (\delta)_{\sigma}\right](b) = \left[\delta_{f(\sigma)}(b)\right]_{\sigma}
$$
This defines $\eta$.

To define $\zeta$, choose $b \in B(\rho)$ adn suppose that $beta \in \Hom_{R,X}(f^{*}(B), A)$. If $f(\sigma)\supseteq \rho$ let $a_{\sigma} = \beta_{\sigma}(e_{B, \rho}^{f(\sigma)}(b))$. Then let
$$
\zeta(\beta)_{\rho}(b) = \{a_{\sigma}\}_{f(\sigma)\supseteq \rho}
$$
We leave the verifications involved to the reader.
\end{proof}

\begin{coro}\label{art6-corollary-5.5}
Let $X, Y$ and $f$ be as above. Then:
    \begin{enumerate}[(1)]
    \item $f_{*}$ carries projectives on $X$ to projectives on $Y$.\label{art6-corollary5.5-enum-1}
    \item For any $R$-carapace $A$ on $X$, $\Sigma(Y, f_{*}(A)) = \Sigma(X, A)$.\label{art6-corollary5.5-enum-2}
    \item $f_{\dagger}$ carries injectives on $X$ to injectives on $Y$.\label{art6-corollary5.5-enum-3}
    \item For any $A$ on $X$, $\Gamma(Y, f_{\dagger}(A))= \Gamma(x, A)$\label{art6-corollary5.5-enum-4}
    \end{enumerate}
\end{coro}

\begin{proof}
For the first statement, let $P$ be a projective on $X$ and let $M \rightarrow N \rightarrow 0$ be a surjective map in $C ar_{R}(X)$. Consider the map, $\Hom_{R, Y}$ $(f_{*}(P), M) \rightarrow \Hom_{R,Y}(f_{*}(P), N)$. By the adjointness statement in \ref{art6-thm-5.2}, \ref{art6-thm5.2-enum-3}), this is the same as the map $\Hom_{R, X}(P, f^{*}(M))\rightarrow \Hom_{R,X}$ $(P, f^{*}(N))$. But now $f^{*}$ is exact and $P$ is projective on $X$ and so this map is surjective. This takes care of \ref{art6-thm5.2-enum-1}).

In general, if $M$ and $N$ are $R$-modules and there is an isomorphism $Hom_{R}(M,Q) \simeq Hom_{R}(N,Q)$ functorial in $Q$, then $M \simeq N$. Apply this to \ref{art6-corollary5.5-enum-2}) using the definition of the functor $\Sigma(X, ?)$ and
\ref{art6-thm5.2-enum-3}) to obtain:
\begin{align*}
Hom_{R}(\Sigma(Y, f_{*}(A)), M) &= \Hom_{R,Y}(f_{*}(A), M_{Y})\\
= \Hom_{R,X}(A, M_{X}) &= Hom_{R}(\Sigma(X, A), M)
\end{align*}
Statement \ref{art6-corollary5.5-enum-2}) follows.

The proof of \ref{art6-thm5.2-enum-3}) is precisely dual to the proof of \ref{art6-thm5.2-enum-1}). To establish
\ref{art6-corollary5.5-enum-4}), apply \ref{art6-prop-5.4} and \ref{art6-thm-2.4}, \ref{art6-thm2.4-enum-(2)}). Write:
$$
\Gamma (Y, f_{\dagger}(A)) = \Hom_{R,Y}(R_{Y}, f_{\dagger}(A)) = \Hom_{R,X}(R_{X}, A) = \Gamma(X,A)
$$
Thus \ref{art6-corollary5.5-enum-4}) is also proven.
\end{proof}

Corollary \ref{art6-corollary-5.5} establishes exactly what is necessary for two composition of functor spectral sequences. Many are possible but we content ourselves with the two most obvious.


\begin{prop}\label{art6-prop-5.6}
Let $X$ and $Y$ be simplicial complexes, let $A$ be an $R$-carapace on $X$ and let $f:X\rightarrow Y$ be a morphism of complexes.
\begin{enumerate}[(1)]
\item There is a spectral sequence with $E_{p,q}^{2}$ term:
    $$
    E_{p,q}^{2} = H_{p}(Y, L_{q}fA)
    $$
    and abutment:
    $$
    H_{r}(X, A)
    $$
    \item There is a spectral sequence with $E_{2}^{p,q}$ term:
        $$
        E_{2}^{p, q} =H^{p}(Y, R^{q} f_{\dagger}A)
        $$
        and abutment:
        $$
        H^{r}(X, A)
        $$
\end{enumerate}

These spectral sequences are sufficiently standard that no proof is required. The proofs in \cite{art6-keyGr}, for example, apply.
\end{prop}

\section{Certain Special Carapaces}\label{art6-sec-6}
This section will be devoted to the study of certain acyclic carapaces. We will need certain conventions. If $X$ is a simplicial, a complement in $X$ is a full subcategory of its simplex category such that the complement of its collection of simplices is a simplicial complex.  The reader may verify that $C$ is a complement in $X$ if, whenever $\sigma \in C$ and $\tau \supseteq \sigma$ then $\tau \in C$. Alternatively $C$ is a complement in $X$ if and only if whenever $\sigma \in C$, then $X (\sigma) \subseteq C$. These two conditions apply to arbitrary subcollections of the simplex set of $X$ and we will use the term complement in this sense. Clearly arbitrary unions and intersetions of complements are complements.

\begin{definition}\label{art6-definition-6.1}
Let $X$ be a simplicial complex.
    \begin{enumerate}[(1)]
    \item An $R$-carapace, B, is called \textit{brittle} if for every sub-complex of $X$, $Z$, the natural map, $\Sigma(Z, B) \rightarrow \Sigma(X, B)$ is injective.\label{art6-definition6.1-1}
    \item An $R$-carapace, $F$, is called \textit{flabby} if for every complement in $X$, $C$, the natural map $\Gamma(X, F) \rightarrow \Gamma (C, F)$ in surjective.\label{art6-definition6.1-2}
    \end{enumerate}
\end{definition}

Our development follows standard treatments of flabbyness for\break sheaves. On occasion something more is called for in the brittleness arguments. Flabbyness will be an entirely familiar concept, but brittleness might be a bit strange. We will begin with some descriptive comments. First notice that if $dim(X)>0$ then $R_{X}$ is not brittle. Suppose that $\sigma$ is positive dimensional simplex in $X$ and that $x$ and $y$ are distinct vertices in it. Let $Z=\{x, y\}$. That is, $Z$ is the disconnected two point complex. Then clearly, $\Sigma(X, R_{X})= R\oplus R$ and, since $\sigma \in X$ and $Z \subseteq \sigma$, the map, $\Sigma(Z, R_{X})\rightarrow \Sigma(X, R_{X})$ is not injective since it factors through $R_{X}(\sigma) = R$. 

If $\sigma \in X$ and $B$ is brittle then by definition, $B(\sigma) \subseteq \Sigma(X, B)$. But brittleness also forces the relation, $B(\sigma)\cap B(\tau) = B(\sigma \cap \tau)$ where the intersection is taken in $\Sigma(X, B)$. To see this just note that, because $\Sigma(Z, A)$ is nothing  but the inductive limit over $Z$, there is an exact sequence,
$$
0 \rightarrow B(\sigma \cap \tau) \rightarrow B(\sigma) \coprod B(\tau) \rightarrow \Sigma(\sigma \cup \tau, B)\rightarrow 0
$$
and, by brittleness, an inclusion $\Sigma(\sigma \cup \tau, B) \subseteq \Sigma(X, B)$.

Before proceeding a convention is necessary. If $\sigma$ is a simplex in $X$ then write $\hat{\sigma}$ for the complex whose vertex set is $\sigma$ but whose simplex set is the set of all proper subsets of $\sigma$. That is $\sigma$ is not a simplex in $\hat{\sigma}$ which is a simplicial sphere. Then $\hat{\sigma} \subset \tilde{\sigma}$.

We will also require the following. Let $f: M \rightarrow N$ be a morphism of $R$-modules. Then $f$ is injective if and only if, for each injective $R$-module, $J$, the induced map $Hom_{R}(N, J)\rightarrow  Hom_{R}(M, J)$ is surjective.

Finally suppose that $Z$ is a simplicial sub-complex of $X$. Let $C$ be the set the simplices of $X$ which are not siplices of $Z$. For any $R$-module, $M$, define $R$-carapaces $M_{Z}^{*}$ and $M_{*}^{C}$ by the equation:

%\setcounter{equation}{1}
\begin{equation}
\begin{aligned}\label{art6-eq-6.2}
M_{z}^{*}(\sigma) &= M \quad \text{if} \quad \sigma \in Z \\
M_{z}^{*}(\sigma) &= (0)\quad \text{if} \quad \sigma \notin Z
\end{aligned}
\end{equation}

Then $M_{*}^{C}$ is defined by exactly the same equations, replacing $M_{Z}^{*}$ by $M_{*}^{Z}$ and $Z$ by $C$. As $Z$ is a complex $M_{Z}^{*}$ in naturally a quotient of $M_{X}$ while $M_{*}^{C}$ is naturally a subobject. In fact, the following is exact:
$$
0 \rightarrow M_{*}^{C}\rightarrow M_{X}\rightarrow M_{Z}^{*} \rightarrow 0
$$

In addition, the following hold
\begin{equation}
\begin{aligned}\label{art6-eq-6.3}
\Hom_{R, X}(A, M_{Z}^{*}) &= Hom_{R}(\Sigma(Z, A), M)\\
\Hom_{R, X}(M_{*}^{C}, A) &= Hom_{R}(M, \Gamma (C, A))
\end{aligned}
\end{equation}


\begin{lem}\label{art6-lemma-6.4}
Let $X$ be a simplicial complex, let $Z \subseteq X$ be a subcomplex of $X$ and let $C$ be a complement in $X$. Then if $A$ is brittle on $X$, $A|_{Z}$ is brittle on $Z$. If $A$ is flabby on $X$, then $A|_{C}$ is flabby on $C$.
\end{lem}

\begin{proof}
If $A$ is brittle and $Z'$ is subcomplex of $Z$ then the composition, $\Sigma(Z', A)\rightarrow \Sigma(Z, A) \rightarrow \Sigma(X, A)$ is the map, $\Sigma(Z', A) \rightarrow \Sigma(X, A)$. If a composition in injective, each map in it injective. This proves the first statement. The proof of the second statement i precisely dual to it and so we leave in to the reader.
\end{proof}

%\setcounter{prop}{4}
\begin{theorem}\label{art6-thm-6.5}
Let $X$ be a simplicial complex and let
$$
0 \rightarrow A' \rightarrow A \rightarrow A'' \rightarrow 0
$$
be exact.
\begin{enumerate}[(1)]
    \item If $A''$ is brittle, then
    $$
    0 \rightarrow \Sigma(x, A') \rightarrow \Sigma(X, A)\rightarrow\Sigma(X, A'') \rightarrow 0
    $$
    is exact.\label{art6-thm6.5-enmu-1}
    
    \item If $A'$ is flabby, then
    $$
    0\rightarrow \Gamma(X, A') \rightarrow \Gamma (X,A)\rightarrow \Gamma(X, A'')\rightarrow 0
    $$
    is exact.\label{art6-thm6.5-enum-2}
\end{enumerate}
\end{theorem}

\begin{proof}
To prove (\ref{art6-thm6.5-enmu-1}) we need only show that $\Sigma(X, A') \rightarrow \Sigma(X, A)$ is injective. By the observation above, it would suffice to show that\break $Hom_{R}(\Sigma(X, A), J) \rightarrow Hom_{R}(\Sigma(X, A'), J)$ is surjective for an injective, $J$. But $Hom_{R}(\Sigma(X, A), J) = \Hom_{R, X}(A, J_{X})$ and the same for $A'$. Thus, to establish (\ref{art6-thm6.5-enmu-1}), it suffices to prove that every carapace morphism, $f:A'\rightarrow J_{X}$ extends to a morphism. $\tilde{f}: A\rightarrow J_{X}$.

Let $j : A' \rightarrow A$ be the injection and let $\pi : A\rightarrow A''$ be the surjection. Let $f: A' \rightarrow J_{X}$ be a morphism of carapaces. Let $\calF$ be the family of paris, $(Z, f_{Z})$ where $Z$ is a subcomplex and $f_{Z}:A|_{Z} \rightarrow J_{Z}$ is a morphism such that $f_{Z}\circ J = f|_{Z}$. Order these by inclusion on $Z$ and extension on $f_{Z}$. This orders $\calF$ inductively and so Zorn's Lemma yields a maximal element, $(W, f_{W})$. If $W \neq X$ there is some $\sigma \in X$ such that $\sigma \notin W$. If $\sigma \cap W = \emptyset$ we may trivially extend $f_{W}$ to $W \cup \{t\}$ where $t$ is any vertex in $\sigma$. This contradicts maximality. Thus we may assume that $\sigma \cap W \neq \emptyset$. Let $\tilde{\sigma} \cap W =Y$. Consider $f_{\sigma} : A'(\sigma)\rightarrow J$. By the injectivity of $J$, we may choose $f_{\sigma}^{1}: A(\sigma) \rightarrow J$ such  that $f_{\sigma}^{1} \circ j_{\sigma} = f_{\sigma}$.  If $\gamma \subseteq \sigma$ let $f_{\gamma}^{1}=f_{\gamma}^{1} \circ e_{A, \gamma}^{\sigma}$. Since j is morphism, the following commutes:
$$
\xymatrix{ 
\ar@{}[dr]|{}
A'(\sigma) \ar[r]^-{j_{\sigma}} & A(\sigma)  \\
 A'(\gamma)\ar[u]^-{e_{A', \gamma}^{\sigma}}\ar[r]_-{j_{\gamma}} & A(\gamma)\ar[u]_-{e_{A, \gamma}^{\sigma} A}  }
$$
Hence $f_{\gamma}^{1} \circ j_{\gamma} = f_{\sigma}^{1}\circ e_{A, \gamma}^{\sigma}\circ j_{\gamma} = f_{\sigma}^{1}\circ e_{A', \gamma}^{\sigma} = f_{\gamma}$. That is, $f^{1} \circ j = f$ on $\sigma$. But on
$\tilde{\sigma} \cap W = Y$, $f_{W}\circ j = f$. Thus on $Y$, $(f_{W}-f^{1})\circ j =0$. It follows that $f_{W}-f^{1}$ induces a map form $A''|_{Y}$ to $J_{Y}$. But $\Sigma(Y, A'')\rightarrow (X, A'')$ is brittle. Hence $\Hom_{R, X}(A'', J_{x})\rightarrow \Hom_{R, Y}(A''|Y, J_{Y})$ is surjective. Thus, there is and $f_{2} \in \Hom_{R, X}(A'', J_{X})$ such that $\pi \circ f_{2}|_{Y} =  (f_{W}-f^{1})|_{Y}$. Consequently, $(f^{1}+ \pi \circ f_{2})|_{\tilde{\sigma}\cap W}=f_{W}|_{\tilde{\sigma}\cap W}$. Hence $f_{W}$ can be extended to $W\cup \tilde{\sigma}$ contradicting the maximality of $(W, f_{W})$. That is $W = X$ and so \ref{art6-thm6.5-enmu-1}) is established.

To prove \ref{art6-thm6.5-enum-2}) we must prove that $\Gamma(X, A)\rightarrow \Gamma(X, A'')$ is surjective. An element $a\in \Gamma (z, A)$ is a function on $Z$ such that $a(\sigma) \in A(\sigma)$ and $e_{A, \sigma}^{\tau}(a(\sigma)) = a(\tau)$. Suppose $a'' \in \Gamma (X, A'')$ is given. Order the pairs $(C, a_{C})$, where $C$ is a complement and $a_{C} \in \Gamma(C, A)$, $\pi(a_{c}) = a''|_{C}$, by inclusion and extension. This being an inductive order, there is a  maximal element, $(U, a_{U})$. If $U \neq X$, there is simplex, $\tau$ not in $U$. Choose $\tilde{a}_{r} \in A(\tau)$ such that $\pi_{\tau}(\tilde{a}_{r}) = a''(\tau)$. Define $\tilde{a}_{1}$ in $X(\tau)$ by $\tilde{a}_{1}(\sigma) = e_{A, \tau}^{\sigma}(\tilde(a)_{r})$. If $X(\tau) \cap U = \emptyset$ then $\tilde{a}_{1}$ extends $a_{U}$ contradicting maximality of $(U,a_{U})$, and so we may assume that $X(\tau)\cap U \neq \emptyset$. This intersection is a complement. Consider the difference $\tilde{a}_{1}-a_{U}$ on this intersection. Now $\pi(\tilde{a}_{1}-a_{U})= 0$ on $X(\tau) \cap U$ whence $^{}\tilde{(}a_{1}-a_{U})|X(\tau)\cap U \in \Gamma (X(\tau)\cap U, A')$ Since $A'$ is flabby there is an element $a' \in \Gamma (X, A')$ such that $a'|X(\tau) \cap U = (a_{1}-a_{U})|X(\tau)\cap U$. Clearly $a_{1}-(a'|X(\tau))$ extends $a_{U}$ contradicting maximality. Thus $U =X$ and we have established \ref{art6-thm6.5-enum-2})
\end{proof} 

\begin{coro}\label{art6-corollary-6.6}
Let
$$
0 \rightarrow A' \rightarrow A \rightarrow A'' \rightarrow 0
$$
be an exact sequence of $R$-carapaces on $X$.
    \begin{enumerate}[(1)]
    \item If $A$ and $A''$ are brittle, then $A'$ is also.\label{art6-corollary6.6-enum-1}
    \item If $A$ and $A'$ are flabby, then $A''$ is also.\label{art6-corollary6.6-enum-2}
    \end{enumerate} 
\end{coro}

\begin{proof}
We prove \ref{art6-corollary6.6-enum-1}). Suppose $Z$ is a subcomplex of $X$. Then, by \ref{art6-lemma-6.4}, $A|_{Z}$ and $A''|_{Z}$ are both brittle and hence, $0 \rightarrow \Sigma(Z, A')\rightarrow \Sigma(Z, A)$ is exact. Thus the following diagram, which has exact rows and columns, commutes:
 $$
 \xymatrix{
 & & 0\ar[d]\\
 0 \ar[r] & \Sigma(Z, A') \ar[r]\ar[d] & \Sigma(Z, A)\ar[d]\\
 0 \ar [r] & \Sigma(X, A')\ar[r] & \Sigma(X, A)
 } 
 $$                          
It is immediate the $\Sigma(Z, A')\rightarrow \Sigma(X, A')$ is monic. As for Statement \ref{art6-corollary6.6-enum-2}), noting that $Z$ must be replaces by a complement, the proof is both well known and strictly dual to the proof of \ref{art6-corollary6.6-enum-1}).
\end{proof}

\begin{prop}\label{art6-proposition-6.7}
Let $X$ and $Y$ be simplicial complexes let $f: X \rightarrow Y$ be a morphism of complexes and let $A$ be an $R$-carapace on $X$. Then
\begin{enumerate}[(1)]
\item If $A$ is brittle, then $f_{*}(A)$ is brittle.\label{art6-proposition6.7-enum-1}
\item If $A$ is flabby then $f_{\dagger}(A)$ is flabby. \label{art6-proposition6.7-enum-2}
\end{enumerate}
\end{prop}

\begin{proof}
To prove \ref{art6-proposition6.7-enum-1}), let $U\subseteq Y$ be a subcomplex. Then, $f^{-1}(U)$ is a subcomplex of $X$ and so, if $A$ is brittle, then $\Sigma(f^{-1}(U), A) \rightarrow \Sigma(X, A)$ is injective. But $\Sigma(f^{-1}(U), A)=\Sigma(U, f_{*}A)$ and $\Sigma{X, A} = \Sigma(Y, f_{*}A)$ by definition. That proves the first statement. The proof of \ref{art6-proposition6.7-enum-2}) is completely parallel except that it uses \ref{art6-corollary-5.5},
\ref{art6-corollary5.5-enum-4} in place of the corresponding properties of $f_{*}$
\end{proof}

\begin{prop}\label{art6-proposition-6.8}
Let $X$ be a simlicial complex.
\begin{enumerate}[(1)]
\item Projective carapaces are brittle; injective carapaces are flabby.\label{art6-proposition6.8-enum-1}
\item a coproduct of brittle carapaces is brittle; a product of flabby carapaces is flabby.\label{art6-proposition6.8-enum-2}
\item For any simplex, $\sigma \in X $ and any $R$-module, $M$, $M \uparrow_{\sigma}$ is brittle and $M\downarrow^{\sigma}$ is flabby.  \label{art6-proposition6.8-enum-3}
\end{enumerate}
\end{prop}

\begin{proof}
Let $Z$ be any subcomplex of $X$. let $C$ be its complement and let $M$ any $R$-module. Then $0 \rightarrow M_{*}^{C} \rightarrow M_{X} \rightarrow M_{Z}^{*} \rightarrow 0$ is exact. Thus, if $P$ is projective, $Hom_{R,X}(P, M_{X}) \rightarrow \Hom_{R, X}(P, M_{Z}^{*})$ is surjective, But this is the map, $Hom_{R}(\Sigma(P, X), M) \rightarrow Hom_{R}(\Sigma(P, Z), M)$. But this map will be surjective for every $M$ if and only if the map $\Sigma(Z, P) \rightarrow \Sigma(X, P)$ is injective (in fact, it must be split).

If $I$ in injective, we need only consider the case, $M =R$. Then $\Hom_{R, X}(R_{X}, I) \rightarrow \Hom_{R, X}(R_{*}^{C}, I)$ is surjective. This is the sequence,  $\Gamma (X, I) \rightarrow \Gamma(C, I)$ and hence
(\ref{art6-corollary6.8-enum-2}) is established.

To prove \ref{art6-corollary6.8-enum-2}), let $\{A_{i}\}_{i \in I}$ be a family of $R$-carapaces on $X$. Since inductive limits of arbitrary co-products are co-products and projective limits of products are products, we may write:

%\setcounter{equation}{8}
\begin{equation}
\begin{aligned}\label{art6-eq-6.9}
\Sigma \left(Z, \coprod_{i \in I} A_{i}\right) &= \coprod_{i \in I} \Sigma (Z, A_{i})\\
\Gamma \left(C, \prod_{i \in I}A_{i}\right) &= \prod_{i \in I}\Gamma (C, A_{i})
\end{aligned}
\end{equation}
Since a co-product of monomorphisms is monic and a product of surjections is surjective, \ref{art6-corollary6.8-enum-2}) follows at once.

Statement \ref{art6-corollary6.8-enum-3}) is quite clear.
\end{proof}


%\setcounter{prop}{9}
\begin{prop}\label{art6-proposition-6.10}
If $A$ a brittle $R$-carapace on $X$, then $H_{i}(X, A) = 0$ for all $i > 0$. If $F$ is flabby, then $H^{i}(X, F) = 0$ for all $i > 0$.
\end{prop}

\begin{proof}
First choose a projective, $P$ and a surjective map so that there is an exact sequence:
$$
0 \rightarrow A_{0} \rightarrow P \rightarrow A \rightarrow 0
$$

The acyclicity of $P$ the Theorem \ref{art6-thm-6.5} together imply that for any brittle $A$, $H_{1}(X, A) = 0$. Then choose a projective resolution of $A$. Break this into a series of short exact sequences, use Corollary
\ref{art6-corollary-6.6} and apply induction. The same technique, applied dually, gives the second statement.

We conclude with local criteria for which there no immediate applications but which are somewhat interesting.
\end{proof}

\begin{prop}\label{art6-proposition-6.11}
Let $A$ a be an $R$-carapace on $X$.
    \begin{enumerate}[(1)]
    \item If for each simplex $\sigma \in X$ the map, $\Sigma(\hat{\sigma}, A)\rightarrow \Sigma(\hat{\sigma}, A)$, is injective then $A$ is brittle.\label{art6-prop6.11-enum-1}
    \item If for each simplex $\sigma \in X$ the restriction $A|{X}(\sigma)$ is flabby, then $A$
     is flabby.\label{art6-prop6.11-enum-2}
    \end{enumerate}
\end{prop}

\begin{proof}
Proofs of these statements use Zorn's lemma as its was used in Theorem \ref{art6-thm-6.5} First we prove
\ref{art6-prop6.11-enum-1}). Let $Z$ be an arbitary subcomplex of $X$. We must show that $\Sigma(Z, A)\rightarrow \Sigma{X, A}$ is injective. As in the proof of theorem \ref{art6-thm-5}, this comes to proving that for any injective module, $J$, any morphism, $f: A|Z \rightarrow J_{Z}$, admits and extension, $f:A\rightarrow J_{X}$. Applying Zorn one finds a maximal subcomplex on which $f$ admits and extension and so, replacing $Z$ by this maximal subcomplex, we may assume that $f$ does not extend to any subcomplex contating $Z$. If the vertex $x$ is not in $Z$ then $f$ clearly extends  to the disconnected union and so we may assume that very vertex is in $Z$. Choose a simplex, $\sigma$ of minimal dimension among the simplices not in $Z$.  Then $\hat{\sigma} \subseteq Z$. Making use of the condition in
(\ref{art6-prop6.11-enum-1}), we obtain a diagram:
$$
\xymatrix{
0 \ar[r] &\Sigma(\hat{\sigma}, A)\ar [d] \ar[r]& \Sigma(\tilde{\sigma}, A)\\
         & \Sigma(Z, A)\ar[d]^{f_{Z}} & \\
         & J                           &
}
$$

Hence there is a map, $f_{i} : \Sigma(\tilde{\sigma}, A) \rightarrow J$ extending $f_{Z}$ and so one may extend $f$ to $Z\cup \sigma$ contradicting maximality. It follows that it must be that $Z=X$.

The proof of \ref{art6-prop6.11-enum-2}), by duality, in entirely straightforward and so we omit it.
\end{proof}

\section{Canonical Resolutions}\label{art6-sec-7}
In this section we give canonical chain and co-chain complexes which can be used to compute the exoskeletal homology and cohomology\break groups. They arise from canonical resolutions and they are sufficiently canonical that they will be seen to be equivariant when there is a group action involved.

Let $A$ be an $R$-carapace on $X$. Then by \ref{art6-lemma-1.4}, \ref{art6-enum-lemma1.4-(1)}) and \ref{art6-enum-lemma1.4-(2)}), the identity map on $A(\sigma)$ induces a map, $\pi_{\sigma} : (A(\sigma))\uparrow_{\sigma} \rightarrow A$ and a map $j_{\sigma} : A\rightarrow (A(\sigma))\downarrow^{\sigma}$.  

\begin{definition}\label{art6-definition-7.1}
Let $X$ be a simplicial complex and let $A$ be an $R$-carapace on $X$.
    \begin{enumerate}[(1)]
    \item Let
        $$
        \calT_{0}(A) = \coprod_{\sigma \in X}(A(\sigma))\uparrow_{\sigma} \quad \text{and let} \quad \pi_{A}= \coprod_{\sigma \in X}\pi_{\sigma} 
        $$\label{art6-definition7.1-enum-1}
        
    \item Let
        $$
        \calS^{0}(A) = \prod_{\sigma \in X}(A(\sigma))\downarrow^{\sigma} \quad \text{and let} \quad j_{A}= \prod_{\sigma \in X}j_{\sigma}.
        $$\label{art6-definition7.1-enum-2}

     \item Let $\calK_{0}(A) = Ker(\pi_{A})$.\label{art6-definition7.1-enum-3}
     \item Let $\calC^{0}(A) = Coker(j_{A})$.\label{art6-definition7.1-enum-4}  
    \end{enumerate}

    This definition has certain immediate consequences.
\end{definition}

%\setcounter{prop}{1}
\begin{lemma}\label{art6-lemma-7.2}
 Let $X$ be a simplicial complex and let $A$ be an $R$- carapace on $X$.
\begin{enumerate}[(1)]
\item The four functors, $\calT_{0}$, $\calK_{0}$, $\calS^{0}$, and $\calC^{0}$ are exact additive functors.\label{art6-lemma7.2-enum-1}
\item Both $\pi_{A}$ and $j_{A}$ are natural transformations in the argument $A$. Further $\pi_{A}$ is always surjective and $j_{A}$ is always monic.\label{art6-lemma7.2-enum-2}

\item For all $A$, $\calT_{0}(A)$ is brittle and $\calS^{0}(A)$ is flabby.\label{art6-lemma7.2-enum-3}
\item If $A$ is brittle, then $\calK_{0}(A)$ is brittle; if $A$ is flabby, $\calC^{0}(A)$ is flabby.\label{art6-lemma7.2-enum-4}
\end{enumerate}
\end{lemma}

\begin{proof}
That $\calT_{0}$ and $\calS^{0}$ are exact and additive is a trivial observation. Since $\calT_{0}(A)$ is a coproduct of carapaces of the form $M\uparrow_{\sigma}$, proposition \ref{art6-proposition-6.8}, \ref{art6-lemma7.2-enum-2}) and \ref{art6-lemma7.2-enum-3}) guarantee that it is brittle. The flabbyness of $\calS^{0}(A)$ follows similarly from the fact that it is a product of carapaces of the form $M\downarrow^{\sigma}$. That $\calK_{0}$ and $\calC^{0}$ are exact is littel more thant the snake lemma. Statement \ref{art6-lemma7.2-enum-2}) is a triviality and so only
\ref{art6-lemma7.2-enum-4})  remains to be proven. This follows from \ref{art6-lemma7.2-enum-3}) and Corollary
\ref{art6-corollary-6.6}.  
\end{proof}

Definition \ref{art6-definition-7.1} and lemma \ref{art6-lemma-7.2} are just what is necessary to construct standard resolutions.

\begin{definition}\label{art6-definition-7.3}
Let $A$ be an $R$-carapace on $X$. Let $\calK_{n}(A) = \calK_{0}(\calK_{n-1}(A))$ and let $\calC^{n}(A) = \calC^{0}(\calC^{n-1}(A))$. That is $\calK_{n}$ is the $(n+1)'$st iterate of $\calK_{0}$ and the same, mutatis mutandis, is ture for $\calC^{n}$. Let $\calT_{n+1}(A) = \calT_{0}(K_{n}(A))$ and let $\calS^{n+1}(A) = S^{0}(\calC^{n}(A))$ for $n\leq 0$. Define maps, $\delta_{n} : \calT_{n+1}(A) \rightarrow \calT_{n}(A)$ and $\delta^{n} : \calS^{n+1}(A)$ as follows. The map, $\delta_{n}$ is the composition of the natural surjection, $\calT_{n+1}(A)\rightarrow \cal_{n}(A)$, with the inclusion, $\calK_{n}(A) \hookrightarrow \calT_{n}$. Similarly $\delta^{n}$ is the composition of the surjection, $\calS^{n}(A) \rightarrow C^{n}(A)$, and the inclusion, $\calC^{n}(A) \hookrightarrow \calS^{n+1}(A)$. Then $\{\calT_{n}(A), \delta_{n}\}$ is called the \textit{canonical brittle resolution} and $\{\calS^{n}(A, \delta^{n})\}$ is called the \textit{canonical flabby resolution} of $A$.  

Some remarks are in order. First of all, since each of the functors, $\calT_{n}$ and $\calS^{n}$, are compositions of exact functors, they are themselves exact functors. Further, by Lemma \ref{art6-lemma-7.2}, for any $A$, each of the carapaces $\calT_{n}(A)$ is brittle while the $\calS^{n}(A)$ are flabby. Thus, letting $C_{n}(X, A) = \Sigma(X, \calT_{n}(A))$ and $C^{n}(X, A)= \Gamma(X, \calS^{n}(A))$, whenever $0 \rightarrow A' \rightarrow A \rightarrow A'' \rightarrow 0$ is exact,
$$
0 \rightarrow C_{n}(X, A') \rightarrow C_{n}(X, A)\rightarrow C_{n}(X, A'')\rightarrow 0
$$
and
$$
0 \rightarrow C^{n}(X, A') \rightarrow C^{n}(X, A) \rightarrow C^{n}(X, A'')\rightarrow 0
$$
are exact. Abusing language, use $\delta_{n}$  and $\delta_{n}$ for the maps of segments and sections respectively as well as maps of carapaces, the homology groups of the complexes, $\{C_{n}(S, A), \delta_{n}\}$ and $\{C^{n}(X, A), \delta^{n}\}$ are connected sequences of homological functors. 
\end{definition}

\begin{definition}\label{art6-definition-7.4}
Let $A$ be an $R$-carapace on $X$. The complex, $\{C_{n}(X, A), \delta_{n}\}$ will be called the \textit{complex of Alexander chains} on $X$ with coefficients in $A$; $C^{n}(X, A), \delta^{n}\}$ will be called the Alexander co-chains. The homology of the complex of Alexander chains will be called the \textit{Alexander homology} and it will be written, $H_{n}^{a}(X, A)$. The homology of the Alexander co-chain complex will be called the \textit{Alexander cohomology} and it will be written $H_{a}^{n}(X, A)$. 
\end{definition}

\begin{prop}\label{art6-proposition-7.5}
The Alexander homology and cohomology of the simplicial complex, $X$, with coefficients in $A$ are isomorphic, respectively, to the exoskeletal homology and cohomology of $X$ with coefficients in $A$, functorially in $A$.
\end{prop}

\begin{proof}
By Proposition \ref{art6-proposition-6.10}, the exoskeletal homology groups vanish on brittle carapaces while the cohomology groups vanish on flabby carapaces. Hence the Alexander groups are the homology groups of the segments (respectively sections) over an acyclic resolution. The proposition follows.
\end{proof}

The following is an interesting footnote.

\begin{prop}\label{art6-proposition-7.6}
If $A$ is projective, the canonical brittle resolution of $A$ consists of projective carapaces. If $A$ in injective, each term in the canonical flabby resolution in injective.
\end{prop}

\begin{proof}
It suffices to prove that if $A$ is projective then $\calT_{0}(A)$ and $\calK_{0}(A)$ are projective and the corresponding statement for an injective $A$ and $\calS^{0}$ and $\calC^{0}$. Suppose that $P$ is projective and that $I$ in injective. Then by Proposition \ref{art6-prop-1.6}, $P(\sigma)$ is projective and $I(\sigma)$ is injective for each $\sigma \in X$. But then, by \ref{art6-lemma-1.4}, $(P(\sigma))\uparrow_{\sigma}$ is projective and $(I(\sigma))\downarrow^{\sigma}$ in injective. By the definition of $\calT_{0}$ and $S_{0}$ and because coproducts of projective are projective and products of injectives are injective, $\calT_{0}(P)$ is projective and $\calS^{0}(I)$ is injective. But then
$$
0 \rightarrow \calK_{0}(P) \rightarrow \calT_{0}(P) P^{\pi_{P}} \rightarrow P \rightarrow 0
$$
and
$$
0 \rightarrow I \xrightarrow{j_{I}} \calS^{0}(I) \rightarrow \calC^{0}(I)\rightarrow 0
$$
are exact. The last two terms of the first sequence are projective while the first two terms of the second sequence are injective. Hence $\calK_{0}(P)$ in projective and $\calC^{0}(I)$ is injective. An iterative application of these facts establishes the result.
\end{proof}

\section{G-carapaces and their Homology}
In this section we consider a simplicial complex, $X$, with a $G$-action for some group, $G$. Then there is a corresponding notion of $G$-carapace and several ways of constructing $G$-representations on the homology and cohomology of a $G$-carapace. One of our main purpose in this section is to show that all of these representations are the same. The method is standard ``relative homological algebra".

If $X$ is a simplicial complex and $G$ is group, a simplicial action of $G$ on $X$ is an action of $G$ in the vertex set of $X$ which carries simplices to simplicies. If $\sigma$ is a simplex in $X$, write $G_{\sigma}$ for the setwise stabilizer of $\sigma$ and $\hat{G}_{\sigma}$ for the pointwise stabilizer of $\sigma$. We will usually write $t_{\mathsf{g}}$ for the translation map, $t_{\mathsf{g}}(x) =\mathsf{g}x$. Then, if $A$ is an $R$-carapace on $X$, the iverse image of $A$ under $t_{\mathsf{g}}$ is the carapace, $[t_{\mathsf{g}}(A)](\sigma) = A(\mathsf{g}\sigma)$. When space does not permit otherwise, write $\mathsf{g}^{*}A$ for $t_{\mathsf{g}}^ {*}(A)$. Recall that the expansions on $t_{\mathsf{g}}^{*}(A)$ are the maps, $e_{\mathsf{g}^{*} A, \sigma}^\tau = e_{A, \mathsf{g}\sigma}^{\mathsf{g}\tau}$.
Notice that $t_{g}^{*}(t_{h}^{*}A) = g_{h\mathsf{g}}^{*}(A)$.

\begin{definition}
A $G$-carapace on $X$ is an $R$-carapace, $A$ together with a family of isomorphisms,
$\Phi = \{\Phi_{\mathsf{g}}\}_{\mathsf{g} \in G}$, $\Phi_{\mathsf{g}} : \rightarrow t_{\mathsf{g}}^{*} A$, such that for any pair, $\mathsf{g}, h \in G$, the diagram: 
\end{definition}

%\setcounter{equation}{1}
\begin{equation}\label{art6-eq-8.2}
\vcenter{
\xymatrix{
A  \ar[r]^{\Phi_{\mathsf{g}}} \ar[d]_{\Phi_{h\mathsf{g}}} & t^*_{\mathsf{g}} A \ar[d]^{t_{\mathsf{g}}^{*} \Phi_{h}}\\
t_{h\mathsf{g}}^{*}A  & t_{\mathsf{g}}^{*}(t_{h}^{*}A)
}}
\end{equation}
commute. That is, $t_{\mathsf{g}}^{*}(\Phi_{h}) \circ \Phi_{\mathsf{g}} = \Phi_{h\mathsf{g}}$.

If $(A, \Phi)$ and $(B, \Psi)$ are two $G$-carapaces a $G-morphism$ $f: A\rightarrow B$ is just a morphism such that $\Psi_{\mathsf{g}} \circ  f = t_{\mathsf{g}}^{*}(f) \circ \Phi_{\mathsf{g}}$.  Clearly $G$-carapaces are an Abelian category.

We now apologize for a digression which some would perfer to conceal in an ``obviously". Write $\prod A$ for the functor $\prod_{\sigma \in X}A(\sigma)$. An element, $a$, in $\prod A$ is a function such that $a(\sigma) \in A(\sigma)$. If $\alpha$ is an automorphism of $X$, there is an isomorphism, $I_{A}^{\alpha} : \prod \alpha^{*}A \rightarrow \prod A$. It is defined by $\left[I_{A}^{\alpha}(a)\right](\sigma) = a(\alpha^{-1}(\sigma))$. Now the coproduct, $\coprod_{\sigma \in X} A(\sigma)$, the module of relations, $N$, such that $\coprod_{\sigma \in X}A(\sigma)/N = \Sigma(X, A)$ and $\Gamma(X, A)$ are all submodules of $\prod A$ preserved by $I_{A}^{\alpha}$ and hence $I_{\alpha}^{\alpha}$ induces two other maps, both of which we will denote $I_{A}^{\alpha}$, from $\Sigma(X, \alpha^{*} A)$ to $\Sigma{X, A}$ and from $\Gamma(X, \alpha^{*} A)$ to $\Gamma(X, A)$. These maps are functorial in the same ways and so we will describe their properties for $\prod A$ and consider them established for all three functors. Let $A$ and $B$ be two $R$-carapaces and let $\alpha$ and $\beta$ be automorphisms of $X$. Let $\phi : A\rightarrow B$ be a morphism. The following equations, whose proof we leave to the reader, are the properties of interest. We emphasize that we shall use these equation for $\Sigma$ and $\Gamma$ rather than $\prod$.
\begin{equation}
\begin{aligned}\label{art6-eq-8.3}
I_{A}^{\alpha} \circ I_{\alpha^{*} A}^{\beta} &= I_{A}^{\alpha \circ \beta}\\
\left(\prod \phi\right) \circ I_{A}^{\alpha} &= I_{B}^{\alpha} \circ \prod \alpha^{*}\phi
\end{aligned}
\end{equation}

In general, if $f: A\rightarrow B$ is a morphism, write $f^{\Sigma}$ and $f^{\Gamma}$ for the induced morphisms on the segments and the sections respectively. Let $(A, \Phi)$ be a $G$-carapace on $X$. Then there are natural representations of $G$ on $\Sigma(X, A)$ and $\Gamma(X, A)$ respectively denoted $\Phi^{\Sigma}$ and $\Phi^{\Gamma}$ defined by the equations:
\begin{equation}
\begin{aligned}\label{art6-eq-8.4}
\left(\Phi^{\Sigma}\right)_{\mathsf{g}} &= I_{A}^{\mathsf{g}} \circ (\Phi_{\mathsf{g}})^{\Sigma}\\
\left(\Phi^{\Gamma}\right)_{\mathsf{g}} &= I_{A}^{\mathsf{g}} \circ (\Phi_{\mathsf{g}})^{\Gamma}
\end{aligned}
\end{equation}

To see that these are representation, we just apply \ref{art6-eq-8.3} Then, $(\Phi^{\Sigma})_{\mathsf{g}h} = I_{A}^{\mathsf{g} h} \circ (\Phi_{\mathsf{g} h})^{\Sigma} = I_{A}^{\mathsf{g}} \circ I_{h}^{\mathsf{g}^{*} A}
\circ ((h^{*} \Phi_{\mathsf(g)})^{\Sigma}) \circ \circ \Phi_{h}^{\Sigma} = I_{A}^{\mathsf{g}} \circ (I_{\mathsf{g}^{*} A} \circ (h^{*} \Phi_{\mathsf{g}})^{\Sigma}) \circ \Phi_{h}^{\Sigma} = I_{A}^{\mathsf{g}} \circ (\Phi_{\mathsf{g}}^{\Sigma} \circ I_{A}^{h}) \circ \Phi_{h}^{\Sigma} = (\Phi^{\Sigma})_{\mathsf{g}} \circ (\Phi^{\Sigma})_{h}$ The computation for $\Phi^{\Gamma}$ is virtually identical.

It is also clear that this argument gives canonically determined representations of $G$ on the left derived functors of $\Sigma(X, ?)$ and right derived functors of $\Gamma(X, ?)$. We give the argument for $\Sigma(X, ?)$. Let $\alpha$ be an automorphism of $X$. Then $\alpha^{*}$ is an automorphism of $Car_{R}(X)$ and so it carries projectives to projectives and injectives. It is moreover an exact functor. Let $\ldots \rightarrow P_{r} \rightarrow P_{r-1} \rightarrow P_{0}\rightarrow  A \rightarrow 0$ be a projective resolution of $A$. Then $\ldots  \rightarrow \alpha^{*}P_{r} \rightarrow \alpha^{*}P_{r-1} \rightarrow \ldots \rightarrow \alpha^{*}P_{0} \rightarrow \alpha^{*}A \rightarrow 0$ is projective resolution of $A$ and, applying $\Sigma(X, -)$, deleting $\alpha^{*}A $ and taking homology yields the left derived functors of $\Sigma(X, \alpha^{*}(-))$. Making use of the functoriality expressed by the second equation of \ref{art6-eq-8.3}), we obtain a commutative diagram.
$$
\xymatrix{
\cdots \ar[r] & \Sigma(X, \alpha^{*}P_{r})\ar[r]\ar[d]_{I_{p_{r}}^{\alpha}} &
\Sigma(X, \alpha^{*}P_{r-1})\ar[r]\ar[d]_{I_{p_{r-1}}^{\alpha}} & \cdots \ar[r]&\Sigma(X, \alpha^{*} P_{0})\ar@{}[d]_{I_{P_{0}}^{\alpha}}&\\
\cdots \ar[r] & \Sigma(X, P_{r})\ar[r] & \Sigma(X, P_{r-1})\ar[r] & \cdots \ar[r] &\Sigma(X, P_{0})&
}
$$
Passing to the homology of these complexes, we obtain unique, canonically defined morphisms, $L_{r}I_{A}^{\alpha}: H_{r}(X, \alpha^{*}A) \rightarrow H_{r}(X, A)$. Clearly, the dual construction will yield canonical morphisms, $R^{q}I_{A}^{\alpha} : H^{q}(X, \alpha^{*}A) \rightarrow H^{q}(X, A)$.

Suppose that $G$ acts on $X$ and that $A$ is a $G$-carapace with $G$-structure, $\Phi$. It is now clear that there is a canonical representation of $G$ on the exoskeletal homology and cohomology groups of $X$ in $A$. Write $L_{r}\Phi_{\mathsf{g}} : H_{r}(X, A) \rightarrow H_{r}(X, \mathsf{g}^{*} A)$ and $R^{q}\Phi_{\mathsf{g}} : H^{q}(X, A)\rightarrow H^{q}(X, \mathsf{g}^{*} A)$ for the map induced by $\Phi_{\mathsf{g}}$ on the homology and the cohomology respectively. Let $\Phi_{\mathsf{g}}^{q} = R^{q}I_{A}^{\mathsf{g}} \circ R^{q}\Phi_{\mathsf{g}}$ and $\Phi_{r}^{\mathsf{g}} = L_{r}I_{a}^{\mathsf{g}} \circ L_{r}\Phi_{\mathsf{g}}$. Then $\Phi_{r}^{\mathsf{g}}$ and $\Phi_{\mathsf{g}}^{q}$ are easily seen to give the unique representations on the homology and cohomology groups making them into homological functors with values in the category of $G$-modules.

What remains in the question of natural $G$-structure on carapace valued functors applied to $G$-carapaces. Let $A$ be a carapace with $G$-structure, $\Phi$, and let $B$ be one with $G$-structure, $\psi$. First consider the tensor product, $A \otimes_{R}B$. Inverse image preserves tensor product. That is, $t_{\mathsf{g}}(A \otimes_{R}B)\simeq t_{\mathsf{g}}^{*}(A)\otimes_{R} t_{\mathsf{g}}^{*}(B)$. Consequently, the family of isomorphisms, $\{\Phi_{g} \otimes_{R} \psi_{\mathsf{g}} : \mathsf{g} \in G \}$, is a $G$-structure on $A \otimes_{R} B$. observe that the commutativity ex-pressed by diagram \ref{art6-eq-8.2}) can also be described by the equation:
\begin{equation}\label{art6-eq-8.5}
\Phi_{h,\mathsf{g}\sigma} \circ \Phi_{g, \sigma} = \Phi_{h \mathsf{g}, \sigma}
\end{equation}
Consider the carapace of local homomorphisms. By definition\break $\calH om_{X,R}(A, B)(\sigma) = \Hom_{X(\sigma), R}(A|_{X(\sigma)}, B|_{X}(\sigma))$. We define a map, $\Theta_{\mathsf{g}, \sigma} : \Hom_{X (\sigma),R}(A|_{X(\sigma)}, B|_{X(\sigma)})\rightarrow \Hom_{X(\mathsf{g}\sigma),R}(A|_{X(\mathsf(g)\sigma)}, B|_{X(\mathsf{g}\sigma)})$. The translation, $t_{\mathsf{g}}$ maps $X(\sigma)$ to $X(\mathsf{g}\sigma)$. Hence $\Phi_{\mathsf{g}}$ maps $A|_{X(\sigma)}$ to 
$t_{\mathsf{g}^{*}}(A|_{X(\mathsf{g}\sigma)})$ and similarly for $B$. Consequently, for any $f \in \Hom_{X(\sigma),R}(A|_{X(\sigma)}, B|_{X(\sigma)})$, $\psi_{\mathsf{g},,\sigma} \circ f \circ (\Phi_{\mathsf{g}, \sigma})^{-1}$ maps $t_{\mathsf{g}^{*}}(A|_{X(\mathsf{g}\sigma)})$ to $t_{\mathsf{g}^{*}}(B|_{X(\mathsf{g}\sigma)})$. Hence, $t_{\mathsf{g}^{-1}}^{*}(\psi_{\mathsf{g}, \sigma} \circ f \circ (\Phi_{\mathsf{g},\sigma})^{-1}) \in \Hom_{X(\mathsf{g}\sigma), R}(A|_{X(\mathsf{g}\sigma)}, B|_{X(\mathsf{g}\sigma)})$. But, this last group is just $\calH om_{X, R}(A, B)(\mathsf{g}\sigma)$. Hence, define the $G$-structure on $\calH om_{X, R}$ by the equation:
\begin{equation}\label{art6-eq-8.6}
\Theta_{\mathsf{g}, \sigma}(f) = t_{\mathsf{g}^{-1}}^{*}(\psi_{\mathsf{g}}|_{X}({\sigma}) \circ f \circ (\Phi_{\mathsf{g}}|_{X}{(\sigma)})^{-1})
\end{equation} 

This equation is to be understood in the following sense. The map $\Phi_{\mathsf{g}}|_{X(sigma)}$ maps the restriction of $A$ to $X(\sigma)$ to the corresponding restriction of $t_{\mathsf{g}}^{*} A$ while $\psi_{\mathsf{g}}|_{X(\sigma)}$ does the same for $B$. Hence the composition in parentheses takes $t_{\mathsf{g}}^{*}(A)|_{x(\sigma)}$ to $t_{\mathsf{g}}^{*}(B)|_{x(\sigma)}$. Thus the inverse image of this map under $t_{\mathsf{g}^{-1}}$ yields an element of $\calH om_{X, R}(\mathsf{g}\sigma) = \Hom_{X(\mathsf{g}\sigma), R}\break (A|_{X(_\mathsf{g}\sigma)}, B|_{X(\mathsf{g}\sigma)})$ which is what i needed.

We check that $\Theta$ is a $G$-structure by establishing \ref{art6-eq-8.5} for it by direct computation. The computation is:
\begin{align*}
\Theta_{h, \mathsf{g}, \sigma}(f) &= t_{(h \mathsf{g})^{-1}}^{*}(\psi_{h \mathsf{g}} \circ f \circ (\Phi_{h \mathsf{g}})^{-1})\\
& = t_{h^{-1}}^{*}(t_{\mathsf{g^{-1}}}^{*}(t_{\mathsf{g^{-1}}}^{*} t_{\mathsf{g}}^{*} (\psi_{\mathsf{g}}) \circ f \circ \Phi_{\mathsf{g}}^{-1} \circ t_{\mathsf{g}}^{*}(\Phi_{h}^{-1})))\\
& = t_{h^{-1}}^{*}(\psi_{h} \circ t_{\mathsf{g}^{-1}} f \circ \Phi_{\mathsf{g}}^{-1}) \circ \Phi_{h}^{-1})\\
& = \Theta_{h,\mathsf{g}\sigma}(\Theta_{\mathsf{g}, \sigma}(f))
\end{align*}
Thus $\Theta$ is a natural $G$-structure on $\calH om_{X, R}(A, B)$. But\break $\Gamma(X, \calH om_{X, R}(A, B)) = \Hom_{X, R}(A, B)$. Hence \ref{art6-eq-8.4} determines a representation of $G$ on $\Hom_{X, R}(A, B)$. This is what we will call the natural representation of $G$ on $\Hom_{X, R}(A, B)$. The explicit description of this action is:
\begin{equation}\label{art6-eq-8.7}
\mathsf{g} \cdot f = t_{\mathsf{g}^{-1}}^{*}(\psi_{\mathsf{g}} \circ f \circ (\Phi_{\mathsf{g}})^{-1}) \quad \mathsf{g} \in G \quad f\in \Hom_{x, R}(A, B)
\end{equation}

Establish this as follows. If $T$ is and $G$-carapace with $G$-structure $\Theta$, then if $\tau \in \Gamma (X, T)$ the action of $G$ on $\Gamma(X, T)$ defined by \ref{art6-eq-8.4} is described by the equation:
\begin{equation}\label{art6-eq-8.8}
(\mathsf{g}. \tau)_{\sigma} = \Theta_{\mathsf{g}, \mathsf{g}^{-1}\sigma}(\tau_{\mathsf{g}^{-1}\sigma})
\end{equation}
When $T=\calH om_{X, R}(A, B)$, this becomes $(\mathsf{g \cdot f})_{\sigma} = \Theta_{\mathsf{g}, \mathsf{g}^{-1}\sigma}(f_{\mathsf{g}^{-1}\sigma})$. Under the identification of $\Hom_{X, R}(A, B)$ with
$\Gamma(X, \calH om_{X, R}(A, B))$, the $\sigma$ component of the map, $f$, is $f|_{X(\sigma)}$. Using this, apply
\ref{art6-eq-8.6} to compute the right hand side of \ref{art6-eq-8.8} for $T= \calH om_{X, R}(A, B)$. We obtain:
$$
(\mathsf{g} \cdot f)_{\sigma} = t_{\mathsf{g}^{-1}}^{*}(\psi_{\mathsf{g}}|_{X(g^{-1}\sigma)} \circ (f|_{X(\mathsf{g}^{-1}\sigma)})\circ (\Phi_{\mathsf{g}}|_{X(\mathsf{g}^{-1}\sigma)})^{-1}).
$$
By the definition of inverse image, this is $(t_{\mathsf{g}^{-1}}^{*}(\psi_{\mathsf{g}^{-1}} \circ f \circ (\Phi_{\mathsf{g}})^{-1}))|_{X(\sigma)}$ and this is just the right hand side of \ref{art6-eq-8.7} restricted to $X(\sigma)$. This proves the truth of \ref{art6-eq-8.7}.

%\setcounter{prop}{8}
\begin{prop}\label{art6-proposition-8.9}
Let $A, B$ and $C$ be $G$-carapaces with $G$-structures, $\Phi$, $\psi$ and $\Upsilon$ respectively. Then, the natural adjointness isomorphism
$$
\phi : \Hom_{X, R}(A, \calH om_{X, R}(B, C)) \rightarrow \Hom_{X, R}(A\otimes_{R}B, C)
$$
is a $G$-morphism. 
\end{prop}

\begin{proof}
Fix $\sigma \in X$, $A \in A(\sigma)$, $b \in B(\sigma)$ and $f \in \Hom_{X, R}(\calH om_{x, R}(B, C))$ and recall the definition of $\phi$. It is $\phi(f)_{\sigma}(a\otimes b) =\left[f_{\sigma}(a)\right]_{\sigma}(b)$ and in interpreting this formula one must remember that $f_{\sigma}(a) \in \Hom_{X, (\sigma), R}\break (B|_{X(\sigma)}, C|X(\sigma))$. We will show that $\phi(\mathsf{g} \cdot f) = \mathsf{g} \cdot \phi(f)$ and we will prove this by evaluating both sides of this equation on a $a\otimes b \in A(\sigma) \otimes B(\sigma)$. Starting with the left hand side:
\begin{align*}
[\phi(\Gg \cdot f)_{\sigma}](a\otimes b) &= [(t_{\Gg^{-1}}^{*}(\Theta_{\Gg} \circ f \circ (\Phi_{\Gg})^{-1}))_{\sigma}(a)]_{\sigma}(b) \quad \text{by} \;\ref{art6-eq-8.7}\\
& = [(\Theta_{\Gg, \Gg^{-1} \sigma} \circ f_{\Gg^{-1}\sigma} \circ (\Phi_{\Gg, \Gg^{-1}\sigma})^{-1})(a)]_{\sigma}(b)\\
& = \Upsilon_{\Gg, \Gg^{-1}\sigma}([f_{g^{-1}\sigma}(\Phi_{\Gg, \Gg^{-1}\sigma}^{-1}(a))_{\Gg^{-1}\sigma}](\psi_{\Gg, \Gg^{-1}\sigma}(b)))\\
& = \Upsilon_{\Gg, \Gg^{-1}\sigma}[\phi(f)(\Phi_{\Gg, \Gg^{-1}\sigma}(a)\otimes \phi_{\Gg, \Gg^{-1}\sigma}(b))] \\
&\qquad \qquad \qquad \qquad \qquad \qquad \text{by the def. of}\; \psi\\
& =[\Gg \cdot \phi (f)](a \otimes b) \; \text{by} \;\ref{art6-eq-8.7})
\end{align*}
That is $\phi(\Gg \cdot f)_{\sigma}(a\otimes b) = [\Gg \cdot \phi(f)]_{\sigma}(a \otimes b)$ as asserted. This proves the result.
\end{proof}

We conclude this section by showing that the canonical brittle and flabby resolutions of a $G$-carapace, $A$ are naturally equivariant. A cosideration of the definition of these resolutions shows that if suffices to show that there are canonical $G$-structure on $\calT_{0}(A)$ and $\calS^{0}(A)$ so that the maps, $\calT_{0}(A \rightarrow A)$ and $A \rightarrow \calS^{0}(A)$ are $G$-equivariant.

\begin{lemma}\label{art6-lemma-8.10}
Let $\alpha : X \rightarrow X$ be an automorphism. Let $M$ be an $R$-module, let $A$ be an $R$-carapace and let $\Phi : A
\rightarrow \alpha^{*} A$ be an isomorphism.
    \begin{enumerate}[(1)]
        \item $\alpha^{*}(M \uparrow_{\sigma}) = M \uparrow_{\alpha^{-1}\sigma}$.\label{art6-lemma8.10-enmu-1}
        \item $\alpha^{*}(M \downarrow^{\sigma}) = M\downarrow^{\alpha^{-1}\sigma}$.\label{art6-lemma8.10-enmu-2}
        \item There is a natural equality $\calT_{0}(\alpha^{*}A) = \alpha^{*f}(\calT_{0}(A))$.\label{art6-lemma8.10-enmu-3}
        \item There is a natural equality $\calS^{0}(\alpha^{*} A) = \alpha^{*}\calS^{0}(A)$.\label{art6-lemma8.10-enmu-4}
    \end{enumerate} 
\end{lemma}

\begin{proof}
The first two statements are trivially true. As for the third and fourth statements, the proofs are nearly identical and so we prove only the first. Write:
\begin{align*}
\calT_{0}(\alpha^{*}A) &= \coprod_{\sigma \in X} \alpha^{*}(A)(\sigma )\uparrow_{\sigma}\\
& = \coprod_{\sigma \in X}(A(\alpha \sigma ))\uparrow_{\sigma}\\
& = \coprod_{\sigma \in X}(A(\alpha)\uparrow_{\alpha^{-1}\sigma}\\
& = \coprod_{\sigma \in X} \alpha^{*}(A(\sigma)\uparrow_{\sigma})\\
& = \alpha^{*}\calT_{0}(A)
\end{align*}
Thus $\alpha^{*}(\calT_{0}(A) = \calT_{0}(\alpha^{*}A))$. That is, the two apparently different construction applies to $A$, $\alpha^{*}\calT_{0}(A)$ and $\calT_{0}(\alpha^{*}(A))$ result in identically the same object.
\end{proof}

\begin{prop}\label{art6-proposition-8.11}
Let $A$ be a $G$-carapace on $X$ with $G$-structure $\Phi$. For each $\Gg \in G$, let $\Phi_{\Gg}^{\calT} = \calT_{0}(\Phi_{\Gg})$ and let $\Phi_{\Gg}^{\calS} = \calS^{0}(\Phi_{\Gg})$. Then $\Phi^{\calT}$ and $\Phi^{\calS}$ are $G$-structures. Furthermore the surjection, $\pi_{A} : \calT_{0}(A)\rightarrow A$ and the injection, $j_{A}: A \rightarrow \calS^{0}(A)$ are $G$-equivariant.
\end{prop}

\begin{proof}
First, we show that $\Phi^{\calT}$ and $\Phi^{\calS}$ are $G$-structures. For $\Phi^{\calT}$ the calculation is:
\begin{align*}
\Phi_{\Gg}^{\calT} &= \calT_{0}(t_{h}^{*}(\Phi_{\Gg}) \circ \Phi_{h})\\
&= \calT_{0}(t_{h}^{*}(\Phi_{\Gg})) \circ \calT_{0}(\Phi_{h})\\
& = t_{h}^{*}(\calT_{0}(\Phi_{\Gg})) \circ \calT_{0}(\Phi_{h})
\end{align*}
To prove that $\pi_{A}$ and $\CMjmath_{A}$ are equivariant just note that $\pi$ is a natural transformation from $\calT_{0}$ to the indentity functor while $j$ is one form the identity functor to $\calS^{0}$. Then note that $\Phi_{\Gg}^{\calT}$ and $\Phi_{\Gg}^{\calS}$ are just the values of $\calT_{0}$ and $\calS^{0}$ on the morphism, $\Phi_{\Gg}$.
\end{proof}

\begin{coro}\label{art6-corollary-8.12}
Let $A$ be a $G$-carapace on $X$. Then there are canonical $G$-structures on the canonical brittle and flabby resolutions of $A$ so that the natural morphisms are $G$-equivariant.
\end{coro}

\begin{proof}
This is nothing more than an interative application of \ref{art6-proposition-8.11}). The details are left to the reader.
\end{proof}

\section{Induced and Co-Induced Carapaces}\label{art6-sec-9}
Let $X$ be a simplicial complex. Recall that $X(r)$ denotes the set of simplices of dimension $r$. We will use $X_{n}$ to denote the collection of simplices of dimension at least $n$. That is, $X_{n} = \bigcup_{r \geq n}X(r)$.

Let $G$ be a group acting simplicially on $X$. Recall that for any simplex, $\sigma \in X$, $G_{\sigma} = \{\Gg : \Gg \in G, \Gg \sigma = \sigma\}$ and $\hat{G}_{\sigma}= \{ \Gg : \Gg \in G_{\sigma \Gg x} = x \quad \forall \in \sigma \}$. We will call the action of $G$ on $X$ \textit{separated} if whenever $\tau \subseteq \sigma$ and $\Gg\tau \subseteq \sigma$ for some $\Gg \in G$, then $\Gg\tau = \tau$.

If $G$ acts on $X$, let $Y(0)= x(0)/G$ be the orbit space and let $\pi : X(0)\rightarrow Y(0)$ be the quotient map. Construct a simplicial complex, $Y$, with vertex set, $Y(0)$, by taking as simplices in $Y$ all finite subsets, $\tau \subseteq Y$ such that $\tau = \pi(\sigma)$ for some simplex $\sigma$ in $X$. If the action of $G$ on $X$ is separated, then for each simplex $\sigma$ in $X$, $G_{\sigma} \subseteq \hat{G}_{\sigma}$ and the dimension of $\pi(\sigma)$ is equal to the dimension of $\sigma$.

If $G$ is acting on $X$ so that the action is separated and if $Y$ is the quotient with quotient map $\pi : X\rightarrow Y$, then a section to $\pi$ is a simplicial map, $s: Y \rightarrow X$, such that $\pi \circ s = id_{Y}$. A separated action admitting a section, $s : Y \rightarrow X$, will be called an \textit{excellent} action. If the action of $G$ on $X$ is excellent with section, $s: Y\rightarrow X$, we will identify $Y$ with its image, $s(Y)$, in $X$ and we shall refer to $\pi$ as the retraction onto $Y$. We will describe this situation by saying that $(X, G)$ is an excellent pair with retraction $\pi : \rightarrow Y$. Notice that the action of $G$ on $X$ is separated if and only if whenever $X(\sigma) \cap X(\Gg\sigma)\neq \emptyset$ then $\Gg\sigma = \sigma$.

If $\calC$ is a category and $X$ is a simplicial complex, then a $\calC$-valued sheaf on $X$ is just a contravariant functor from $X$ to $\calC$. If $\sigma \subseteq \tau$ and $\bS$ is a sheaf on $X$ write $r_{\bS, \tau}^{\sigma} : \bS(\tau) \rightarrow \bS(\sigma)$ for the corresponding map and call it the restriction. If $G$ operates simplicially on $X$, then the assignment, $\hat{G}(\sigma) = \hat{G}_{\sigma}$ is a sheaf of groups on $X$. If the action is separated, then $G_{\sigma} = \hat{G}_{sigma}$, and so this also is a sheaf of groups. In any case we will refer to $\hat{G}$ as teh stabilizer of the action.

If $f : X \rightarrow Y$ is a morphism of complexes and $\bS$ is a sheaf on $Y$ then $f^{*}\bS$, defined by the equation $f^{*}\bS(\sigma) = \bS(f(\sigma))$ with the corresponding restrictions is called the inverse image of $\bS$. When $f$ is the inclusion of a subcomplex, we call $f^{*}\bS$ the restriction of $\bS$ to $X$ and we may on occasion write it, $\bS_{X}$.

Suppose now that $G$ is a group and that $H$ is a subgroup. We wish to fix notation for induced and co-induced modules. Write $R[G]$ for the group algebra of $G$ over $R$ and write $R[G]^{\ell}$ for the free rank one $R[G]$-module isomorphic to $R[G]$  as an $R$-module but with $R[G]$ structure defined by the equation, $\Gg \cdot x = x\Gg^{-1}$ for $x \in R[G]$ and $\Gg \in G$. We simply write $\Gg x$ for the product in $R[G]$. Let $M$ be an $H$-module. Then the $G$-module induced by $M$ is $R[G] \otimes_{R[H]}M$; the $G$-module co-induced by $M$ is $Hom_{R[H]}(R[G]^{\ell}, M)$. The $G$-structure on the induced module is just that obtained from left multiplication on $R[G]$. The structure on the co-induced module is just the structure described by $(\Gg f)(x)= f(\Gg^{-1} x)$. Write $I_{G/H}M$ for the induced module and $C_{G/H}M$ for the co-induced module.

Choose a complete set of coset representatives $Q\subset G$ for the space of left cosets, $G/H$. Then $I_{G/H}M =\coprod_{\Gg \in Q}\Gg \otimes M$. Write $\Gg M$ for $\Gg \otimes M$. The $R$-module, $\Gg M$ depends only on the coset $\Gg H$ and not on the particular representative, $\Gg$.

Suppose now that $H= G_{\sigma}$ for some simplex, $\sigma$. If $\gamma \in G_{\sigma}$ write $M^{\gamma}$ to denote $xM$ for any $x$ such that $x\sigma =\gamma$. Then $M^{\gamma}$ depends only on $\gamma$ for $x\sigma = y\sigma = \gamma$ if and only if $x \in yG_{\sigma}=yH$. If $\Gg \gamma =\tau$ then $\Gg M^{\gamma} = M\tau$. Dually, $R[G]^{\ell} = \coprod_{x\in Q}R[H] \cdot x =\coprod_{x\in Q}xR[H]$. Hence $C_{x\in Q}(M) = \prod_{x\in Q}Hom_{R[H]}(xR[H], M)$. If $\gamma \in G\sigma$ let $M_{\gamma} =Hom_{R[H]}(xR[H], M)$ for any $x$ such that $x\sigma = \gamma$. This is well defined. Moreover, if $y\gamma = \lambda$ then left translation by $y$ carries $M_{\gamma}$ to $M_{\lambda}$.

%\setcounter{equation}{1}
\begin{definition}\label{art6-definition-9.1}
Suppose $G$ acts on $X$, that $\sigma$ is a simplex in $X$ and that $M$ is a $G_{\sigma}$ representation over $R$. Let:
\begin{equation}
\begin{aligned}\label{art6-eq-9.2}
T_{\sigma}(M) &= \coprod_{r\in G_{\sigma}}M^{\tau}\uparrow_{\tau}\\
S^{\sigma}(M) &= \prod_{r\in G_{\sigma}}M^{\tau}\downarrow^{\tau}
\end{aligned}
\end{equation}
Then $T_{\sigma}(M)$ is called the \textit{carapace induced by} $M$ and $S^{\sigma}(M)$ is called the \textit{carapace coinduced by} $M$.
\end{definition}   

\begin{prop}\label{art6-proposition-9.3}
Let $G$ act on $X$ and let $M$ be a representation of $G_{\sigma}$ over $R$. Then
\begin{enumerate}[(1)]
\item $T_{\sigma}M$ and $S^{\sigma}M$ both admit canonical $G$-structures.\label{art6-proposition9.3-enum-1}
\item Let $A$ be any $G$-carapace on $X$. Then \label{art6-proposition9.3-enum-2}
    $$
    \Hom_{X,R}(T_{\sigma}M, A)^{G} = Hom_{G_{\sigma}}(M, A(\sigma))
    $$
    and
    $$
    \Hom_{x, R}(A, S^{\sigma} M)^{G} = Hom_{G_{\sigma}}(A(\sigma), M)
    $$
\item $T_{\sigma}(M)$ is brittle; $S^{\sigma}(M)$ is flabby. \label{art6-proposition9.3-enum-3}
\end{enumerate}
\end{prop}

\begin{proof}
First observe that \ref{art6-proposition9.3-enum-3}) is just a consequence of Definition \ref{art6-definition-9.1} and Proposition \ref{art6-proposition-6.8}. We pass to the construction of $G$-structures on $T_{\sigma}(M)$ and $S^{\sigma}(M)$.

First we evaluate these carapaces on a typical simplex, $\tau$. Recall that $\tilde{\tau}$ denotes the full simplicial complex underlying $\tau$. Then:
$$
T_{\sigma}(M)(\tau) = \coprod_{\gamma \in \tilde{\tau}\cap G_{\sigma}}M^{\gamma}
$$
\begin{equation}\label{art6-eq-9.4}
S^{\sigma}(M)(\tau) = \prod_{\gamma \in X (\tau) \cap G_{\sigma}} M_{\gamma}
\end{equation}
\end{proof}

If $\Gg \in G$, then $\Gg$ carries distinct simplicies in $\tilde{\tau} \cap G\sigma$ to distinct simplices in $\widetilde{\Gg \tau}\cap G\sigma$. Hence left multiplication by $\Gg$ carries separate summands in $T_{\sigma}(M)(\tau)$ to the corresponding summands in $T_{\sigma}(M)(\Gg \tau)$. Let $\Phi_{\Gg, \tau}$ be the sum of left multiplication by $\Gg$ on the separate components of $T_{\sigma}(M)(\tau)$. Similarly define a map, $\psi_{\Gg, \tau}$, from $S^{\sigma}(M)(\tau)$ to $S^{\sigma}(M)(\Gg \tau)$ by taking it to be left translation by $\Gg$ on each of the factors. Then, using Equation 8.5, one verifies that $\Phi$ and $\psi$ are $G$-structures.

Only \ref{art6-proposition9.3-enum-2}) remains to be proved. We compute directly.
\begin{align*}
\Hom_{X, R}(T_{\sigma}(M), A) &= \Hom_{X, R}(\coprod_{\tau \in G{\sigma}} M^{\tau}, A)\\
& = \prod_{\tau \in G\sigma}\Hom_{X, R}(M^{\tau} \uparrow_{\tau}, A)\\
& = \prod_{\tau \in G_{\sigma}} Hom_{R}(M^{\tau}, A(\tau)) \quad \text{by} 4, \eqref{art6-proposition9.3-enum-1} \text{of} \textsection \ref{art6-proposition9.3-enum-1}
\end{align*}
Let $\Upsilon = \{\Upsilon_{\Gg}\}_{\Gg \in G}$ be the $G$-structure on $A$ and let $\{f_{\tau}\}_{\tau \in G_{\sigma}}$
be an element of $\prod_{\tau \in G_{\sigma}}Hom_{R}(M^{\tau}, A(\tau))$. The element $f_{\tau}$ may be thought of as the $\tau$ segment of a morphism, $f: T_{\sigma}(M)\rightarrow A$. Moreover these segments may be chosen freely because $T_{\sigma}(M)$ is the direct sum of the carapaces, $M^{\gamma}\uparrow_{\gamma}$ ad $\gamma$ ranages over $G_{\sigma}$. Then, by equation \ref{art6-eq-8.7}, $\Gg \cdot\{f_{\tau}\}_{\tau\in G_{\sigma}}$ is the element of the same product whose $\tau$-component is $\Upsilon_{\Gg, \Gg^{-1}\tau} \circ f_{\Gg^{-1}\tau} \circ \Gg^{-1}$. This element of the product is $G$-stable if and only if
\begin{equation}\label{art6-eq-9.5}
\Upsilon_{\Gg, \sigma} \circ f_{\sigma} \circ \Gg^{-1} = f_{\Gg \sigma}.
\end{equation}
That is, if the family, $\{f_{\tau}\}$ is $G$-invariant, each component is uniquely determined by $f_{\sigma}$. Conversely, given $f_{\sigma} \in  Hom_{G_{\sigma}}(M , A(\sigma))$ one may use Equation \ref{art6-eq-9.5} to define $f_{\Gg \sigma}$ for eac, $\Gg$, chosen that carries $\sigma$ to $\Gg\sigma$. Any other such element is of the form $\Gg h$ for some $h \in G_{\sigma}$. But then replacing $\Gg$ by $\Gg h$ in \ref{art6-eq-9.5}) yields the same result because $f_{\sigma}$ is, by hypothesis, a $G_{\sigma}$-morphism. The proof for $S^{\sigma}(M)$ is too similar to bear repetition.

\begin{prop}\label{art6-prop-9.6}
Let $G$ act on $X$ and let $M$ be a representation of $G_{\sigma}$ over $R$. Then:
\begin{align*}
\Sigma(X, T_{\sigma}(M)) &= I_{G/G_{\sigma}}(M)\\
\Gamma(X, S^{\sigma}(M)) &= C_{G/G_{\sigma}}(M)
\end{align*}
\end{prop}

\begin{proof}
Begin by observing that for any $R$-module, $N$, $\Sigma(X, N\uparrow_{\sigma}) = N$ and $\Gamma (X, N downarrow^{\sigma}) = N$. Let $U$ be some complete set of representatives of the cosets, $\Gg G_{\sigma}$. Then,
\begin{align*}
\Sigma(x, T_{\sigma}(M)) &= \Sigma(X, \coprod_{\gamma \in G\sigma}M^{\gamma} \uparrow_{\gamma}) \quad \text{by Definition} \;\ref{art6-definition-9.1}\\
&= \coprod_{\gamma \in G\sigma} \Sigma(X, M^{\gamma} \uparrow_{\gamma})\\
& = \coprod_{x \in U} M^{x\sigma}\\
& = \coprod_{x \in U} \otimes M = I_{G/G_{\sigma}}(M)
\end{align*}

The corresponding compution for $S^{\sigma}(M)$ is:
\begin{align*}
\Gamma (X, S^{\sigma}(M)) &= \Gamma(X, \prod_{\gamma \in G\sigma}M_{\gamma}\downarrow^{\gamma})\\
& = \prod_{x \in U} \Gamma(X, M_{x\sigma} \downarrow^{x\sigma})\\
& = \prod_{x \in U} M_{x\sigma}\\
&= \prod_{x \in U}Hom_{R[G_{\sigma}]}(x R[G_{\sigma}], M)\\
& = Hom_{R[G_{\sigma}]}(\coprod_{x \in U} xR[G_{\sigma}], M)\\
& = Hom_{R[G_{\sigma}]}(R[G]^{\ell}, M ) = C_{G/G_{\sigma}(M)}.
\end{align*}
\end{proof}

\begin{prop}\label{art6-proposition-9.7}
Let $G$ act excellently on $X$ with retraction $\pi : X\rightarrow Y \subseteq X$. Then
\begin{align*}
T_{\sigma}(A(\sigma)) &= \coprod_{\gamma \in G\sigma}A(\gamma )\uparrow_{\gamma}\\
S^{\sigma}(A(\sigma)) &= \prod_{\gamma \in G\sigma}A(\gamma)\downarrow^{\gamma}
\end{align*}
\end{prop}

\begin{proof}
Let $\Phi$ be the $G$-structure on $A$. Then $\coprod_{\gamma \in G\sigma} A(\gamma)$ admits a natural representation of $G$. If $\Gg \in G$  let $\phi_{\Gg} = \prod_{\gamma \in G\sigma} \Phi_{\Gg,\gamma}$. It is understood that application of $\phi_{g}$ must be followed by reindexing of components. Furthermore, the natural injection, $j_{\sigma} : A(\sigma) \rightarrow \coprod_{\gamma \in G\sigma} A(\gamma)$ is $G_{\sigma}$ equivariant. Thus, $j_{\sigma}$ extends to a $G$-morphism, $j_{A} : I_{G/G_{\sigma}} \rightarrow A(\gamma)$. Then $j_{A}(\Gg \otimes A(\sigma)) = \phi_{\Gg}(j_{A}(1 \otimes A(\sigma))) =\Phi_{\Gg, \sigma}(A(\sigma))= A(\Gg\sigma)$. Since $I_{G/G_{\sigma}}(A(\sigma))$ is a coproduct of the $R$-submodules, $\Gg \otimes A(\sigma)$, it is clear that $j_{A}$ is an isomorphism. Since $A(\sigma)^{\gamma} = \Gg \otimes A(\sigma)$ one sees that $j_{A}$ restrict to isomorphisms $j_{\gamma} : A(\sigma)^{\gamma} \rightarrow A(\gamma)$ which comprise an equivariant family in the sense that $\phi_{\Gg} \circ j_{\gamma} = j_{\Gg \gamma} \circ h_{\Gg}$ is left homothety by $g$.

Now consider $A(\sigma)_{\gamma}$ By definition it is $Hom_{R[G_{\sigma}]}(\Gg r[G_{\sigma}], A(\sigma))$ with $G_{\sigma}$-action, $xf(u) = f(x^{-1} \cdot u) = f(ux)$ and where $\Gg$ is any element carrying $\sigma$ to $\gamma$. For some choice of $\Gg$, let $\beta_{\gamma}(f) = \Phi_{\Gg, \gamma}(f(\Gg))$. Choose $x\in G_{\sigma}$. Then  $\Phi_{\Gg x, \gamma}f(\Gg x) = \Phi_{\Gg x, \gamma}(X^{-1} f)(\Gg) = \Phi_{\Gg, \sigma} \circ \Phi_{x, \sigma}(X^{-1} f)(\Gg) = \Phi_{\Gg, \sigma}f(\Gg)$. Consquently, $\beta_{\gamma}$ is independent of the choice of $\Gg$ and the maps $\{\beta_{\gamma}\}$ are an equivariant family as above.

To prove the two statements of the proposition, note that, by definition, $T_{\sigma}(A(\sigma))= \coprod_{\gamma \in G\sigma} A(\sigma)_{\gamma} \uparrow_{\gamma}$ and $S^{\sigma}(A(\sigma)) = \prod_{\gamma \in G_{\sigma}}A(\sigma)^{\gamma} \downarrow^{\gamma}$. The isomorphisms in question, then, are $\coprod_{\gamma \in G\sigma}j_{\gamma} \uparrow_{\gamma}$ and $\prod_{\gamma \in G\sigma}\beta_{\gamma}\downarrow^{\gamma}$.
\end{proof}

\begin{prop}\label{art6-proposition-9.8}
Let $(X, G)$ be excellent with quotient, $Y \subseteq X$, and retraction, $\pi$. Let $A$ be any $G$-carapace of $R$-modules on $X$. Then:
\begin{enumerate}[(1)]
\item $\calT_{0}(A)= \coprod_{\sigma \in Y}T_{\sigma}(A(\sigma))$.\label{art6-proposition9.8-enum-1}
\item $\calS^{0}(A) = \prod_{\sigma \in Y} S^{\sigma}(A(\sigma))$.\label{art6-proposition9.8-enum-2}
\item $\Sigma(X, \calT_{0}(A)) = \coprod_{\sigma \in Y} I_{G/G_{\sigma}}(A(\sigma))$. That is $\Sigma(X, \calT_{0}(A))$ is a coproduct of induced modules.\label{art6-proposition9.8-enum-3}
\item $\Gamma(X, \calS^{0}(A))= \prod_{\sigma \in Y}C_{G/G_{\sigma}}(A(\sigma))$. That is $\Gamma(X, \calS^{0}(A))$ is a product of coinduced modules. \label{art6-proposition9.8-enum-4}
\end{enumerate}
\end{prop}

\begin{proof}
Statements \ref{art6-proposition9.8-enum-3}) and \ref{art6-proposition9.8-enum-4}) follow form \ref{art6-proposition9.8-enum-1}) and  \ref{art6-proposition9.8-enum-2}) by a direct apllication of proposition \ref{art6-prop-9.6} and so we need only prove \ref{art6-proposition9.8-enum-1})  and \ref{art6-proposition9.8-enum-2}).

To prove \ref{art6-proposition9.8-enum-1}) and \ref{art6-proposition9.8-enum-2}), first notice that since $G$ acts excellently, we may write the collection of simplices in $X$ as the disjoint union $\bigcup_{\sigma \in Y}G_{\sigma}$. Then,just apply Proposition \ref{art6-proposition-9.7}. One obtains:
\begin{align*}
\calT_{0}(A) &= \coprod_{\tau \in X}A(\tau)\\
& = \coprod_{\sigma \in Y} \coprod_{\tau \in G\sigma}A(\tau)\\
 & = \coprod_{\sigma \in Y} T_{\sigma}(A(\sigma)) \quad \text{by Proposition}\; \ref{art6-proposition-9.7}
\end{align*}
For $calS^{0}(A)$, the proof is:
\begin{align*}
\calS^{0}(A) &= \prod_{\sigma \in Y} \prod_{\tau \in G\sigma}A(\tau)\\
& = \prod_{\sigma \in Y}S^{\sigma}(A(\sigma))
\end{align*}
We leave the $G$-structures to the reader.
\end{proof}

Before proceeding note that if $Y \subseteq X$, if $\sigma \in Y$ and if $M$ is an $R$-module, we may perform the constructions $M \uparrow_{\sigma}$ and $M downarrow^{\sigma}$ both in $X$ and in $Y$ and that the results might differ. Rather than lingering upon unnecessary distinctions or unnecessarily complicating notation, we caution the reader to maintain a certain vigilance in reading the next proof.

\begin{lem}\label{art6-lemma-9.9}
Let $G$ act excellently on $X$ with retraction $\pi : X\rightarrow Y$ and section, $s$. Then for any $\sigma \in Y$, and $G_{\sigma}$-module, $M$, $s^{*}(T_{\sigma}(M)) = M \uparrow^{\sigma}$.
\end{lem}

\begin{proof}
By definition, $T_{\sigma}(M) = \coprod_{\gamma \in G\sigma}M^{\gamma} \uparrow_{\gamma}$. Evaluating,
$$
T_{\sigma}(M)(\lambda) = \coprod_{\gamma \in \lambda \cap G{\sigma}} M^{\gamma}.
$$
Suppose that two simplices, $\Gg \sigma$ and $h\sigma$ both  lie in $\lambda$. The action is excellent and so separated. It followed that $\Gg \sigma = h \sigma$. If $\lambda \in Y$ then $G\sigma \cap \tilde{\lambda}$ is just one simplex and if $\sigma \in Y$ that simplex must be $\sigma$. That is, for any $\lambda \in Y$, $T_{\sigma}(M)(\lambda) = M\uparrow_{\sigma}(\lambda)$ which establishes the result.

Suppose $X$ is a simplicial complex and that $M$ is a sheaf of groups on $X$. Let $V$ be a carapace of $R$-modules on $X$. Suppose that for each $\sigma \in X$, we are given a representation, $\rho_{\sigma}:M(\sigma) \rightarrow Aut_{R}(V)$. Then for each pair,  $\sigma \subseteq \tau$ notice that $V(\sigma)$ is naturally an $M(\tau)$ module simply by pulling back by the restriction, $r_{M, \tau}^{\sigma}$.
\end{proof}

\begin{definition}\label{art6-definition-9.10}
Let $X$ be a simplicial complex and let $M$ be a sheaf of groups on $X$. Then a \textit{carapace of representations} of $M$ on $X$ over $R$ is a carapace of $R$-modules, $V$, together with a family of representations, $\rho_{\sigma}: M(\sigma) \rightarrow Aut_{R}(V)$ such that for any pair, $\sigma \subseteq \tau$, the expansion, $e_{V, \sigma}^{\tau}$, is an $M(\tau)$-morphism. If $A$ and $B$ are two carapaces of representations of $M$ on $X$ over $R$, a \textit{morphism} of carapaces of representations is morphism of carapaces which is an $M(\sigma)$-morphism for ecah simplex, $\sigma$.
\end{definition}

Notice that carapaces of representations of $M$ are clearly an Abelian category. Generally, when there is no danger of confusion, we will just say $M$-carapace to denote a carapace of representations of $M$ on $X$ over $R$.

Let $G$ operate excellently on $X$ with retraction, $\pi : X\rightarrow Y$. Write $s : Y \rightarrow X$ for the inclusion. Since excellent actions are separated, $G_{\sigma} = \hat{G_{\sigma}}$, is a sheaf of groups on $X$ which we will write $G_{*}$. If $A$ is a $G$-carapace of $R$-modules on $X$, then $A$ is certainly a carapace of representations over the sheaf of groups, $G_{*}$. Finally notice that if $f : X\rightarrow Y$ is a morphism of complexes and if $V$ is a carapace of representations over the sheaf of groups, $M$ on $Y$ then it is purely formal to check that $f^{*}(M)$ is a carapace of representations over the sheaf of groups, $f^{*}M$.

\begin{definition}\label{art6-definition-9.11}
Suppose that $G$ operates excellently on $X$ with retraction, $\pi : X\rightarrow Y$, and section, $s: Y \rightarrow X$. Let $A$ be a $G$-carapace of $R$-modules on $X$. Then the\textit{prototype} of $X$ on $Y$ is $s^{*}(A)$. It is a carapace of representations over $s^{*}G_{*}$, the restriction of the sheaf of stabilizers.
\end{definition}

It is patently obvious that $s^{*}$ is ana exact functor from the category of $G$-carapaces to the category of carapaces of representations of $s^{*}G_{*}$ on $Y$ over $R$. More can be said. We will write $G_{*}$ for the restriction of the stabilizer sheaf to $Y$ if there is no danger of confusion.

\begin{theorem}\label{art6-thm-9.12}
Let $G$ act excellently on $X$ with retraction $\pi : X\rightarrow Y$ and section, $s$. Then $s^{*}$ is an isomorphism of categories from the category of $G$-carapaces of $R$-modules on $X$ to the category of carapace of representations of $s^{*}G_{*}$ on $Y$ over $R$.
\end{theorem}

\begin{proof}
In this discussion the functors $\calT_{i}$ and $\calK_{i}$ on $X$ as well as the corresponding functors associated to $Y$ occur. Consequently we will use $\calT_{i}$ and $\calK_{i}$ exclusively for the functors associated to $X$. The corresponding functors on $Y$ will be written $\overline{\calT}_{i}$ and $ \overline{\calK}_{i}$.

Write $C ar_{R}(G, X)$ for the category of $G$ carapaces of $R$-modules on $X$ and $C ar_{R}(G_{*}, Y)$ for the category of carapaces of representations of $G_{*}$ in $R$-modules on $Y$. Then $s^{*}$ is an exact functor from $C ar_{R}(G, X)$ to $C ar_{R}(G_{*}, Y)$. Define a functor from $C ar_{R}(G_{*}, Y)$ to $C ar_{R}(G, X)$ by the equation:
\begin{equation}\label{art6-equation-9.13}
T_{Y}^{0}(A) = \coprod_{\sigma \in Y}T_{\sigma}(A(\sigma))
\end{equation}
Then $s^{*}(T_{Y}^{0}(A)) = \prod_{\sigma \in Y}s^{*}(T_{\sigma}(A(\sigma))) = \coprod_{\sigma \in Y}A(\sigma) \uparrow_{\sigma}$ by Lemma \ref{art6-lemma-9.9}. But then by the definition of the functor $\overline{\calT}_{0}$, this says that $s^{*}(T_{Y}^{0}(A)) = \overline{\calT}_{0}(A)$. Let $T_{Y}^{1}(A) = T_{Y}^{0}(\overline{\calK}_{0}(A))$. For each $\sigma \in Y$, there is a natural inclusion, $\overline{\calK}_{0}(A)(\sigma) \rightarrow T_{Y}^{0}(A)(\sigma)$. By Proposition \ref{art6-proposition-9.3}, \ref{art6-proposition9.3-enum-2}), this inclusion gives a unique map $q_{\sigma} : T_{\sigma}(\overline{\calK}_{0}(A)(\sigma))\rightarrow T_{Y}^{0}(A)$. Let $q_{A}= \coprod_{\sigma \in Y} q_{\sigma}$. Then $q_{A}$ maps $T_{Y}^{1}(A)$ to $T_{Y}^{0}(A)$ and its restriction to $Y$ is just the natural map from $\overline{\calT}_{1}(A)$ to $\overline{\calT}_{0}(A)$. Define a functor on $C ar_{R}(G_{*}, Y)$ to $C ar_{R}(G, X)$ by the equation:
\begin{equation}\label{art6-eq-9.14}
\calI_{Y}(A)= Coker(q_{A})
\end{equation}

We will show that $\calI_{Y}$ is inverse to $s^{*}$. First suppose that $A$ is in $Car_{R}(G^{*}, Y)$. Then the sequence,
$$
T_{Y}^{1}(A) > q_{a}>>T_{Y}^{0}(A) \rightarrow \calI_{Y}(A)\rightarrow 0
$$
is exact. Apply the exact functor, $s^{*}$. The result is a commutative diagram:
$$
\xymatrix{
s^{*}T_{Y}^{1}(A)\ar@{=}[d]\ar[r]^{qA} & s^{*}T_{Y}^{0}\ar@{=}[d] \ar [r] & s^{*}(\calI_{Y}(A)) \ar[r]\ar[d] & 0\\
\overline{\calT}_{1}(A) \ar [r]_{\delta_{1}} & \overline{\calT}_{0}(A) \ar[r]_{\pi_{A}} & A \ar [r] & 0
}
$$
By exactness of the rows and commutativity, $s^{*}\calI_{Y}(A)$ is isomorphic to $A$.

Now consider $B$ is $Car_{R}(G, X)$. First notice that proposition \ref{art6-sec-3}, \eqref{art6-proposition-2} gives canonical maps, $T_{\sigma}(B(\sigma)) \rightarrow B$. Sum  to obtain a canonical map, $d_{0}T_{Y}^{0}(s^{*} B) \rightarrow B \rightarrow 0$. This map clearly vanishes on the image of $T_{Y}^{1}(s^{*}B)$. Hence it induces a mapping, $\xi_{B} : \calI_{Y}(s^{*}B) \rightarrow B$. Apply the exact functor, $s^{*}$ and use what we have just proven. Then $s^{*}(\calI_{Y}(S^{*}B)) = s^{*}(B)$ and $\xi_{B}$ induces the identity on segments in $Y$. Now finally note that a $G$-morphism of $G$-carapaces which is an isomorphism on $Y$ is an isomorphism. That is $\calI_{Y}(s^{*}(B))\simeq B$ functorially in $B$. It follows that $\calI_{Y}$ is inverse to $s^{*}$.
\end{proof}

In what follows, Theorem \ref{art6-thm-9.12} will be a very essential and fundamental tool for analyzing $Car_{R}(G, X)$.

\section{Recollections and Fundamentals; Buildings}\label{art6-sec-10}

For the most part we follow the notation and conventions of \cite{art6-keyBT-I} and \cite{BT-II}. Our purpose here is a brief review which will establish notation and emphasize one or two differences. Throughout $K$ is a field complete with respect to the discrete rank one valuation, $\omega : K^{*} \rightarrow \bbZ$. Then $\calO$  will be the center of $\omega$, $\overline{k}$ will be the residue field of $\calO$ and $\xi : \calO \rightarrow \overline{k}$ is the natural map.

Let $\bG$ be be a Chevalley group scheme defined over $\bbZ$. Assume it to be split, simply connected, connected and of simple type. Let $\bT$ be a maximal torus, let $\bN$ be its Cartan subgroup and let $\bB$ be a Borel subgroup containing $\bT$  all given as group subschemes of $\bG$ defined over $\bbZ$.  

Each of these group schemes being a functor, applying any one of them to $\xi$ gives a morphism which, in all cases, we will also call $\xi$ from $\bG(\calO)$ to $\bG(\overline{k})$, $\bB(\calO)$ to $\bB(\overline{k})$, etc.. let $G =\bG(K)$, let $G_{0} = \bG(\cal0)$ and let $\overline{G} = G(\overline{k})$. Let $\calB = \bB(K)$, let $\overline{B} = \bB(\overline{k})$ but let $B= \{x \in G_{0} : \xi(x) \in \overline{B}\}$. Let $N = \bN(K)$, let $T= \bT{K}$ and let $H = N\cap B$.

Let $X$ denote the finite free $\bbZ$-module of characters of $\bT$, let $\Phi$ denote the roots of $\bG$ with respect to $\bT$;  let $\Phi_{+}$ denote those positive with respect to $\bB$; let $\Delta$ be a basis of simple roots in $\Phi_{+}$ and let $\tilde{\alpha}$ denote the largest root. Let $\Gamma = Hom_{\bbZ}(\bX, Z)$ be the group of one parameter subgroups of $\bT$ and let $\Phi \subseteq \Gamma$ be the set of co-roots.

Let $\{\bU_{\alpha} : \alpha \in \Phi\}$ be a set of root subgroups let $U_{\alpha} = \bU_{\alpha}(K)$ and let $x_{\alpha} : \bU_{\alpha}\rightarrow G_{a, Z}$ be the natural isomorphism. Let $U_{\alpha,n} = \{u: \omega(x_{\alpha}(u)) \geq n \}$ and observe that $U_{\alpha, 0}=\bU(\calO)$.

Let $N_{0} = N \cap G_{0}$. Then $\Gamma$ can be identified with the group scheme morphisms, $\gamma : G_{m, Z} \rightarrow \bT$. Write $<\gamma, \chi>$ for the value, $\gamma(\chi)$ when $\gamma  \in \Gamma$ and $ \chi \in \bX$. We  may always think of $\gamma$ as a map, $\map: K^{*}\rightarrow T$ and so we may write $\chi(\gamma(y)) = y^{<\gamma, \chi>}$. Following this convention, we may define the action of $N$ on $\Gamma$ by the equation, $(^{n}\gamma)(x) = n(\gamma(x))n^{-1}$. For $t \in T$ write $t^{\chi}$ for $\chi(t)$. Then $N$ acts on $\bX$ by the equation $\chi^{n}(t) = \chi(ntn^{-1})$. Then $N_{0}$  normalizes $T$ and so acts on it by conjugation. Consider the semi-direct product, $T\triangleleft N_{0}$. For any $(t, n)\in T \\triangleleft N_{0}$, define a mapping, $\tau_{t, n} : \Gamma \in \Gamma$ by the equation,
$$
< \tau_{(t, n)}(\gamma), \chi > = -\omega(t^{\chi}) + <^{n}\gamma, \chi>
$$
Then $\tau_{t, n}$ is an affine transformation and $(t, n)\mapsto \tau_{(t,n)}$ is an action of $T \triangleleft N_{0}$ on $\Gamma$. If $n\in T \cap N_{0}$ it is clear that $\tau_{n,n^{-1}}= id_{\Gamma}$ and so this action reduces to an affine action of $T \cdot N_{0} = N$ on $\Gamma$. If $n \in N$, $\gamma \in \Gamma$ write $n=tn_{0}$ and let $^{n}\gamma =\tau_{(t, n_{0})}(\gamma)$. It is straightforward to verify that if $^{n}\gamma = \gamma$ for all $\gamma \in \Gamma$ then $n \in H$. 

Let $\bA = \Gamma \otimes_{Z} R$ and extend the action of $N$ to $\bA$ by linearity. Let $\bX_{R} = \bX \otimes_{Z} R $ and choose a form on $\bX_{R}$ invariant under the vector Weyl group. For $\lambda, \chi \in \bX_{R}$ write the form, $(\lambda|\chi)$. With this form, identify $\bA$ with $\bX_{R}$ and write $\check{\alpha} = \frac{2\alpha}{(\alpha | \alpha)}$ for each $\alpha \in \Phi$.

Since $\bH$ acts trivially under $\tau$, and since $N/H$ is naturally isomorphic to the affine Weyl group $\tau$ induces an action of the affine Weyl group on $\bA$ and it is a triviality that this is the canonical action. 

For any pair, $(\alpha, r) \in \Phi \times \bbZ$, let $\alpha^{*} = \{x \in \bA : \alpha(x) + r \geq 0\}$. These are closed half spaces and they are in bijective correspondence with $\Phi \times \bbZ$. If the half space, $\alpha^{*}$ corresponds to the pair $(\alpha, r)$, write $U_{\alpha^{*}}$ for the group, $U_{\alpha, r}$ defined above. Write $\partial\alpha^{*}$. The closed half spaces, $\alpha^{*}$, are called the affine roots of $G$ in $\bA$ and we write $\Sigma$ for the set of all affine roots of $G$ in $\bA$.

Define an equivalence on $\bA$ by saying that $x\sim y$ if and only if $x \in \alpha^{*}$ if and only if $y \in \alpha^{*}$ for all $\alpha \in \Sigma$. These equivalence classes are the facets of $\bA$ and they are the interiors of simplies (because $\bG$ is of simple type). Their closures give a simplicial decomposition of $\bA$. The maximal dimensional facets are called chambers. They are the connected components of $\bA\ \bigcup_{\alpha^{*} \in \Sigma}\partial \alpha^{*}$.

Let $\tilde{\Delta}$ denote the set of affine roots $(\alpha, 0)$, $\alpha \in \Delta$ and $(-\beta, 1)$ where $\beta$ is the unique largest root relative to the dual Weyl chamber. Let $S$ denote the set of reflections thorugh the hyperplanes, $\partial\alpha^{*}$ for all $\alpha^{*} \in \tilde{\Delta}$. These reflections afford a Coxeter presentation of the affine Weyl group, $N/H$, so that (G, B, N, S) is a Tits system in $G$. Then $N/H$ acts transitively on the chambers of $\bA$.

Let $\overline{C}_{0} = \bigcap_{\alpha^{*} \in \tilde{\Delta}} \alpha^{*}$ and let $C_{0}$ be its interior. Let $F$ be any contained in $\overline{C}_{0}$ and let $S_{F} = \{ s \in S : s(F) = F \}$. In pariticular $S_{C_{0}} = \emptyset$. For any $F$, let $P_{F}$ be the subgroup of $G$ generate by $B$ and some arbitararily chosen set of representatives of $S_{F}$ chosen and it is called the parahoric subgroup of $G$ associated to $F$. By definition, the parahoric subgroup of $G$ relative to the Tits system, $(G, B, N, S)$, are the conjugates of the subgroups, $P_{F}$. We shall call the facets, $F \subseteq \overline{C}_{0}$ the types of $G$ with respect to the Tits system, $(G, B, N, S)$. If $P$ is a parahoric subgroup of $G$, then $P$ is conjugate to a unique group of the form $P_{F}$ for some $F$. Then we call $F$ the type of $P$ and we write $F= \tau{P}$. Notice that our terminology differs slightly from \cite{art6-keyBT I} in which the subsets, $S_{F}$ are called the types. 

We may now describe $\calI = \calI(G, B, N, S)$, the building of $G$ with respect to the Tits system, $(G, B, N, S)$. As a point set, $\calI(G, B, N, S)$ is the set of pairs, $(P, x)$ where $P$ parahoric and $x \in \tau(P)$. Now $G$ acts on this set by the equation, $\Gg (P, x) = (\Gg P \Gg^{-1}, x)$. 

Let $x$ be any point of $\bA$. Then for some $n \in N$, $nx \in \overline{C}_{0}$. Then the point $nx$ is uniquely determined. Let $F$ be the smallest open facet containing $nx$.  Let $a(x) = (n^{-1} P_{F}n, nx)$. The group, $n^{-1}P_{F}n$ is uniquely determined just as $nx$ is and so $a$ maps $bA$ into $\calI(G, B, N, S)$. (This map is called $j$ in \cite{art6-keyBT I}.) We may now make $\calI(G, B, N, S)$ into a geometric simplicial complex. Its simplices are just $G$ translates of closed facets in $a(\bA)$ and its vertices are translates of the special points. The $G$-translates of $a(\bA)$ are called the apartments of $\calI(G, B, N, S)$.

There are several structures on $\calI = \calI(G, B, N, S)$. First there is what is called an affine structure in \cite{art6-keyBT I}. If $x$ and $y$ are any two points in $\calI$ they are contained in one apartment. Thus for any $\lambda \in [0,1]$, there is a  point $\lambda x + (1- \lambda)y$ determined by the apartment. This point is, however, independent of the apartment chosen and the operation which assigns to each pair, $(x, y)$ together with a real $\lambda \in [0,1]$, the point $\lambda x + (1-\lambda)y$ is the affine structure. There is a $G$ invariant metric whose restriction to any apartment is the metric is written $d(x, y)$. Finally there is a ``bornology" on subsets of $G$. $A$ set is called bounded if it is contained in a finite union of double cosets, $BwB$ for $w \in N$. Now let us record some statements particularly useful to us. They are for the most part simple rearrangements of statements in \cite{BT I} and \cite{BT II}. 

\begin{lem}\label{art6-lemma-10.1}
The action of $G$ on $\calI(G, B, N, S)$ is excellent with section, $\overline{C}_{0}$.
\end{lem}

\begin{proof}
First,m $\overline{C}_{0}$ is a fundamental domain for the $G$ action and $G$ is transitive on maximal dimensional simplices. This implies that the action is separated. The coled chamber, $\overline{C}_{0}$, is isomorphic to the quotient if $\calI$ by the $G$ action and the inclusion of $\overline{C}_{0}$ into $\calI$ is clearly a section.
\end{proof}

\begin{lem}\label{art6-lemma-10.2}
$G$ acts transitively on the paris $(F, \bC)$ where $F$ is a facet and $\bC$ is an apartment containing $F$. 
\end{lem}

\begin{proof}
This is just 2.26, p. 36 in \cite{art6-keyBT I}.
\end{proof}

\begin{prop}\label{art6-proposition-10.3}
The action of $B$ on $\calI(G, B, N, S)$ is excellent with section.

$a: \bA \rightarrow \calI$.
\end{prop}

\begin{proof}
The action of $G$ on $\calI(G, B, N, S)= \calI$ is separated and so, a fortiori, the action of $B$ is as well. Let $C_{0}$ be the chamner associated ot $B$. Recall the definition of the retraction of $\calI$ on $\bA$ with center, $C_{0}$ (\cite{art6-keyBT I} 2.3.5, P. 38). As we remarked, given any two facets, there is an apartment containing them. Thus for any fact, $F\subseteq \calI$, there is an apartment $\bA_{1}$ containing $F$ and $C_{0}$. By \ref{art6-lemma-10.2}, there is an element $\Gg$ in $G$ so that $\Gg(C_{0}, \bA_{1}) = (C_{0}, \bA)$ whence $\Gg F\subseteq \bA$.
Let $\rho_{C_{0},\bA}(F') =\Gg F$. We show that $\rho_{C_{0},\bA}^{-1}(F') =B \cdot F$.

Suppose that $\rho_{C_{0},\bA}(F') = F$. Then, by definition, there is an apartment, $\bA'$, containing $C_{0}$ and $F'$ and an element $\Gg \in G$, so that $\Gg\rm A' = \rm A$, $GgF' = F$ and $\Gg C_{0} = C_{0}$. Since $\Gg C_{0} =C_{0}$, $\Gg \in B$ and $\Gg F' = F$. Hence $\rho_{C_{0},\bA}^{-1}(F') \subseteq B \cdot F$. The opposite inclusion is clear.

To complete the proof that $\calI/B = \bA$. we use, nearly unmodified, the proof of 2.3.2, p. 37 of \cite{art6-BT I}. Clearly, for any facet, $F$, the orbit, $B \cdot F$ meets $\bA$. What remains to be shown is that $B \cdot F$ contains exactly one facet in $\bA$.

Suppose that $F$ and $F'$ are in $\bA$ and that $F=bF'$, $b \in B$. Then, $F= n F_{0}$, $F' =n'F_{0}$ for some $F_{0} \subseteq \overline{C}_{0}$. The stabilizer of $F$ (respectively, $F'$) is $nP_{F_{0}}n^{-1}(n'P_{F_{0}}(n')^{-1}$ respectively). Then $bn'P_{F_{0}}(n')^{-1}b^{-1} =nP_{F_{0}}n^{-1}$. Since parahoric subgroups are self normalizing, $n^{-1}bn' \in P_{F_{0}}$ and hence, $b^{-1}n \in n'P_{F_{0}}$. Let $X = S_{F_{0}}$ and let $W_{x}$ be the subgroup of $W$ generated by $X$. Then $P_{F_{0}} = BW_{X}B$. By \cite{art6-NB I}, IV, \ref{art6}, Proposition \ref{art6-sec-2}, $n'P_{F_{0}} \subseteq Bn'W_{X}B$. If $\nu : N \rightarrow W$ is teh natural surjection, this implies that $\nu(n) \in \nu(n')w_{X}$ and so $nP_{F_{0}}n^{-1} = n'P_{F_{0}}(n')^{-1}$. But $F$ is the fixed point set of $nP_{F_{0}}n^{-1}$ and $F'$ is the fixed point set of $n'P_{F_{0}}(n')^{-1}$. Consequently, since the two groups are equal, $F =F'$. thus for any facet, $B \cdot F$ contains exactly one facet in $\bA$. As this is true, in particular, for vertices, $\bA$ is the quotient of $\calI$ by the action of $B$ and so the proposition is proven.  
\end{proof}

\section{Mounmental Complexes}\label{art6-sec-11}

In this section the term monumental is thought of as meaning resembling or having the scale of a building ad it is used in the history of art. We use it to describe certain $G$ actions which have all the properties of the natural actions on buildings which are of interest to us. If $X$ is a finite dimensional simplicial complex and if $G$ is a group acting on it then a subcomplex, $Y \subseteq X$ will be  called homogeneous if, Whenever $Y$ contains two simplices, $\tau$ and $\gamma$ such that $\tau = \Gg \gamma$ for some $\Gg \in G$, there is an element $s\in G$ such that $sY = Y$ and $\tau = s \gamma$.

\begin{definition}\label{art6-definition-11.1}
Let $G$ be a group and let $X$ be a finite dimensional simplicial complex on  which $G$ is acting with a separared action. Then the action of $G$ on $X$ will be called \textit{monumental} if and only if the following conditions hold:
\begin{enumerate}[\rm (1)]
\item Every simplex in $X$ is contained in a maximal dimensional simplex.\label{art6-}\label{art6-definition11.1-enum-1}
\item $G$ acts transitively on the maximal dimensional simplices.\label{art6-definition11.1-enum-2}
\item For any simplex, $\sigma$ in $X$ the stabilizer, $G_{\sigma}$ is self normalizing in $G$.\label{art6-definition11.1-enum-3}
\item If $\sigma$ is a maximal dimensional simplex, the stabilizer, $G_{\sigma}$, acts excellently and it admits a homogeneous section, $Y \subseteq X$.\label{art6-definition11.1-enum-4}
\end{enumerate}
\end{definition}

suppose that $G$ is a group acting monumentally on $X$. Then a maximal dimensional simplex will be called a closed chamber; its interior will be called a chamber. Let $\overline{C}$ be a closed chamber in $X$. Then separation of the action implies the for any simplex $\sigma$ in $X$, the orbit, $G\sigma$ contains at most one simplex which is a face of $\overline{C}$ while homogeneity implies that there is always at least one. Consequently the action of $G$ is excellent with quotient $\overline{C}$ and section equal to the inclusion of $\overline{C}$ in $X$. Let $B = b(\overline{C})$ denote the $G$ stabilizer of $\overline{C}$ and let $Y$ be a homogeneous subcomplex of $X$ mapping isomorphically onto the quotient, $X/B$ Then any translate of $Y$, $\Gg Y$ will be called an apartment of type $Y$. In general we will think of the type, $Y$ as chosen and fixed once and for all and so we will often speak merely of apartments.  

Let $F$ be a field. Let $H$ be a commutative Hopf algebra over $F$ with comultiplication, $mu_{H} : H \rightarrow H \otimes_{F} H$, augmentation, $\epsilon_{H} : H \rightarrow F$, and antipode, $s_{H} : H \rightarrow H$. Then $H$ will be called \textit{proalgebraic} if it is reduced and a direct limit of sub-Hopf algebras finitely generated over $F$. Then $Spec(H)$ is a proalgebraic group scheme over $F$.

If $H$ is proalgebraic over $F$ let $\calG_{H}$ denote the group of $F$-valued points of $H$. That is $\calG_{H}$ is the group of $F$-homomorphisms from $H$ to $F$. Let $\calF(\calG_{H}, F)$ denote the ring of $F$ functions on $\calG_{H}$ and let $\gamma_{H}$ be the natural map from $H$ to $\calF(\calG_{H}, F)$, namely $\gamma_{H}(a)(\phi) = \phi(a)$. We will say that $H$ separates $k$-points if $\gamma_{H}$ is injective.

\begin{definition}\label{art6-definition-11.2}
Let $G$ be a group. A \textit{monumental $G$-complex}is a simplicial complex, $X$, on which $G$ is acting monumentally together with a $G$-carapace of commutative $F$-algebras with unit, $(\calA, \{\Phi_{\Gg}\}_{\Gg \in G})$, and $G$-morphisms, $\mu_{\calA} : \calA \rightarrow \calA \otimes_{F} \calA$, $s_{\calA} : \calA \rightarrow \calA$ and $\epsilon_{A} \rightarrow F_{X}$ so that that following conditions are satisfied:
\begin{enumerate}[{\rm(1)}]
\item Each of the morphisms, $\mu_{\calA}$, $s_{\calA}$ and $\epsilon_{\calA}$ is a morphism of $G$-carapaces of commutative F-algebra and for any simplex, $\sigma$, $\calA(\sigma)$, $\mu_{\calA, \sigma}$, $s_{\calA, \sigma}$, $\epsilon_{\calA, \sigma}$ is a profinite Hopf algebra over $F$.\label{art6-definition11.2-enum-1}

\item For each pair of simplices, $\sigma \subseteq \tau$, the expansion, $e_{\calA, \sigma}^{\tau}$ is a surjective morphism of proalgebraic Hopf algebras.\label{art6-definition11.2-enum-2}

\item The $G$ structure, $\{\Phi_{\Gg}\}_{\Gg \in G}$ acts by isomorphisms of carapraces of proalgebraic Hopf algebras. That is $\Phi_{\Gg, \sigma}$ is a Hopf algebra isomorphism for each $\Gg$ and $\sigma$.\label{art6-definition11.2-enum-3}

\item For each $\sigma$ let $\calG(\sigma) =\calG_{\calA(\sigma)}$. Then $\calG$ is naturally a sheaf of groups with the $G$ structure induced by $\Phi$. Then $\calA(\sigma)$ is reduced for each $\sigma$ and there is an isomorphism of $G$-sheaves, $\alpha : G_{*} \rightarrow \calG$.\label{art6-definition11.2-enum-4}
\end{enumerate}
\end{definition}

The first three conditions of Definition \ref{art6-definition-11.2} are self explanatory but the third requires some amplification. First of all, if $M$ is a simplicial sheaf on $X$ and $\psi$ is an automorphism of $X$, $\psi^{*}M(\sigma) = M(\psi(\sigma))$. The $G$-structure on $G_{*}$ is that arising from conjugation. That is, define $c_{\Gg, \sigma} : G_{\sigma} \rightarrow G_{\Gg \sigma}$ by the equation:
\begin{equation}\label{art6-eq-11.3}
c_{\Gg,\sigma}(x) = \Gg x \Gg^{-1}
\end{equation}

Now we explain the $G$-structure on $\calG$ induced by $\Phi$. The functor $G$ is contravariant and so $\calG(\Phi_{\Gg, \sigma})$ maps $\calG(\Gg \sigma)$ to $\calG(\sigma)$. Thus define a $G$-structure, $\{\Gamma_{\Gg}\}_{\Gg \in G}$ by the equation:
\begin{equation}\label{art6-eq-11.4}
\Gamma_{\Gg, \sigma} \calG(\Phi_{\Gg^{-1}, \Gg \sigma})
\end{equation}
Recalling that elements of $\calG(\sigma)$ are the $F$-homomorphisms from $\calA(\sigma)$ to $F$, this map can be more explicitly written, $\Gamma_{\Gg, \sigma}(x) - x \circ \Phi_{\Gg^{-1}, \Gg \sigma}$.

It is customary to write $a(x)$ for $x(a)$ when $a$ is in a ring and $x$ is a $F$-point of the ring. We may use $\alpha$ to identify $G_{\sigma}$ with $\calG(\sigma)$, writing, for $a\in \calA(\sigma)$ and $\Gg \in G_{\sigma}$, $a(\Gg)$ to denote $[\alpha_{\sigma}(\Gg)](a)$. With these conventions, Condition \ref{art6-definition11.2-enum-3}) is nothing more than the equation: 
\begin{equation}\label{art6-eq-11.5}
[\Phi_{\Gg, \sigma}(a)](x) = a(\Gg^{-1}x \Gg)
\end{equation}

Henceforth of $X$ is a monumental $G$-complex with carapace of F-Hopf algebras $\calA$ we will just say that $(G, X, \calA)$ is a monumental $G$-complex over $F$. Further we will use $\alpha$ to identify $G_{*}$ with $\calG$ and we will always view $\calA(\sigma)$ as a ring of function of $G_{\sigma}$. The definitions of this section contain all the properties of buildings which will be used lated on. The next section explains how the affine building of the group of $K$-valued points of some semi-simple group for some valued field, $K$ satisfies all of these conditions for an appropriate choice of the carapace $\calA$.  

\section{The Main Examples}\label{art6-sec-12}
 In this section we will discuss three monumental complexes. The first two are quite straightforward by the third requires a short dicussion of some classical results of M. Greenberg. Such symbols as $K, \omega, \overline{k}, \calO, \bG, \bG, \calB, \overline{B}$ etc. mean just what they did in the previous section.  

\section*{The Admissible Complex}
Let $K$ be a locally compact, non-Archimedean field and let $F$ be any (discrete) field. Let $\calI(G, B, N, S)$ be the buildings associated to $G$. For each $\sigma \calI$, $G_{\sigma}$, the parahoric subgroup associated to the facet, $\sigma$, is a profinite group. For each $\sigma$ let $\calA_{F}^{0}(\sigma)$ denote the ring of locally constant $F$-valued functions on $G_{\sigma}$. For any set $T$, $T^{F}$ will denote the set of all $F$-valued functions on $T$. Then
$$
\calA_{F}^{0}(\sigma) = \lim\limits_{\substack{\longrightarrow \\ M}}(G_\sigma/M)^{F} 
$$
where $M$ varies over the open normal subgroups of $G_{\sigma}$. Since each of the algebras, $(G_{\sigma}/M)^{F}$ is in fact a finite dimensional Hopf algebra with augmentation and antipode and since the inclusions$(G_{\sigma}/M_{1})^{F}\subseteq (G_{\sigma}/M_{2})^{F}$ when $M_{2} \subseteq M_{1}$, is a Hopf morphism, $\calA_{F}^{0}(\sigma)$ is a Hopf algebra with antipode and augmentation.It is clearly profinite and it is also clear that the expansions are surjective Hopf morphisms. Let $\calG_{F}$ be the sheaf of $F$-points of the carapace $\calA_{F}^{0}$ as in \S\ref{art6-definition-11.2}, \ref{art6-definition11.2-enum-4}). Then $\calG_{F}(\sigma)$ is the set of algebra homomorphisms,
\begin{align*}
Hom_{F}^{al}(\lim\limits_{\substack{\longrightarrow \\ M}}(G_\sigma/M)^{F}) &= \lim\limits_{\substack{\longrightarrow \\                                                  M}} Hom_{F}^{al}((G_{\sigma}/M)^{F}, F)\\
&=\lim\limits_{\substack{\longrightarrow \\ M}} G_{\sigma}/M\\
&=G_{\sigma}
\end{align*}
That is, there is a canonical isomorphism of sheaves of groups, $\alpha : G_{*} \rightarrow \calG_{F}$. Finally define the $G$ structure $\{\Phi_{\Gg}\}_{\Gg \in G}$ by the equation, $[\Phi_{\Gg, \sigma}(a)](x) = a(\Gg ^{-1}x\Gg)$. Since $\calI$ is the building of $G$ \ref{art6-definition11.2-enum-1}),\ref{art6-definition11.2-enum-2}) and \ref{art6-definition11.2-enum-3}) of \S \ref{art6-sec-11} show that $G$ acts monumentally on it. We have just observed that $\calA_{F}^{0}$ and $\calI$ satisfy Conditions \ref{art6-definition11.2-enum-1}) through \ref{art6-definition11.2-enum-4}) of Definition \ref{art6-lemma-10.2}. Hence $(\calI, \calA_{F}^{0})$ is a monumental $G$ complex. We shall call it the admissible complex of $G$ over $F$. We note that the group. $G$, may be replaced by a central extension, $\tilde{G}$. 

\section*{The Spherical Complex}
 For this example we depart somewhat from usual  terminology. Let $F$ be an algebraically closed field and let $G_{F} = \bG(F)$ be the group of $F$ points of the Chevalley scheme, $\bG$ which we assume to be of simple type. Construct a complex as follows. The vertices of $\calS$ are the proper reduced maximal parabolic subgroup schemes of $G_{F}$ which we regard as the base extension of $\bG$ to $F$. The set $P_{1},ldots P{n}$ is a simplex in $\calS$ if and only if the intersection  $P_{1}\cap \ldots \cap P_{n}$ is parabolic. Let $G_{F}$ act on $\calS$ by conjugation. For any $\sigma \in \calS$, we write $G_{\sigma}$ for the stabilizer. Then $G_{\sigma}$ is just the intersection of the groups corresponding to the vertices of $\sigma$. A chamber is the set of maximal parabolics containing a maximal torus. We must first establish that the action of $G$ on $\calS$ is monumental. The first three conditions of Definition
 \ref{art6-definition-11.1} are quite well  known. The proof of the fourth condition is again a modification of the proof of 2.3.2 of \cite{BT I}. Let $B$ be a Borel subgroup containing the maximal torus, $T$, and let $N$ be the normalizer of $T$. Let $P$ be any parabolic subgroup of $G$. By \cite{art6-keyH}, \cite{art6-definition-6.1}, the intersection of any two parabolic subgroups contains a maximal torus. Thus there is maximal torus, $S$, in $P \cap B$. Hence there is some element, $b \in B$, so that $bsb^{-1} = T$. Hence $bPb^{-1}$ contains $T$. Let $\bA(T)$ denote the apartment corresponding to $T$.m We have shown that for any facet, $\tau$, the $B$ orbit, $B\tau$  meets $\bA(T)$.  

Now suppose that $P$ and $Q$ are two $B$-conjugate parabolics both of which contain $T$. Then there are element $m$ and $n$ in $N$ and $b$ in $B$ and a parabolic $P_{0}$ containing $B$ so that $P=nP_{0}n^{-1}$, $Q=m P_{0}m^{-1}$ and $bPb^{-1}=Q$. Thus, $bnP_{0}n^{-1}b^{-1} = mP_{0}m^{-1}$ and so, since $P_{0}$ is self normalizing, $bn \in m P_{0}$. By \cite{art6-keyBo}, IV, 2.5.2, $bn \in BMW_{0} B$ where $W_{0}$ is the Weyl group of $P_{0}$. Hence $nP_{0}n^{-1} = mP_{0}m^{-1}$, That is any two $B$ conjugate parabolics containing $T$ are necessarily equal. It follows that any $B$ orbit $B\tau$ meets $\bA(T)$ in exactly one facet. Condition \ref{art6-definition11.2-enum-4}) of the definition in hence established.

For any $\sigma \in \calS$ let $\calA(\sigma)$ be the coordinate ring of the parabolic subgroup, $G_{\sigma}$. It is an elementary exercise in the theory of algebraic groups to see that $(G_{F}, \calS, \calA)$ is a monumental $G_{F}$ complex. 

\section*{The Affine Complex}
To describe this complex, we must recall some classical results of M. Greenberg. Let $\overline{k}$ be a perfect field and let $R$ be a ring scheme over $\overline{k}$. Let $X$ be any $\overline{k}$ scheme. Define a ringed space, $\tilde{R}(X)$ as follows. Its topological space is the underlying topological space of $X$ and we denote it $b(X)$. If U is open in $b(X)$, let $\calG_{R, X}(U)= R(Spec(\calO_{X}(U)))$. As this functor is representible, it is a sheaf of rings on $b(X)$. Call $R$ a scheme of local rings if $\calG_{R, X}$ is a sheaf of local rings for each scheme, $X$ and assume this to be the case. 

Let $V=R(\overline{k})$. Then for each $X$, $\calG_{R, X}$ is a sheaf of $V$ algebras. Let $\tilde{R}(X) = (b(X), \calG_{R, X})$. Then $\tilde{R}$ is a covariant functor from  $\overline{k}$ schemes to $V$ local of $Spec(V)$ schemes. Though we call the following the first theorem of Greenberg, it is not given as one theorem in \cite{art6-keyMG I} but it largely summarizes the content of \S\ref{art6-definition11.2-enum-4} of that work, especially Propositions \ref{art6-equation-enum-1} to \ref{art6-equation-enum-4} of \S \ref{art6-sec-4} and the extensions thereof in  \S\ref{art-sec-6}.

\subsection{First Theorem of Greenberg}\label{art6-subsec-12.1}
\textit{Suppose the ring scheme, $R$, is a projective limit of schemes each isomorphic to affine $n$ space over $\overline{k}$ for varying $n$. Then there is a right adjoint to the functor $\calG_{R, X}$ from the category of $\overline{k}$ schemes to the category of $V$-schemes. That is, there is a functor $\bF$ so that for any $\overline{k}$ scheme, $X$, and any $Spec(V)$ scheme, $Y$, the following holds:
$$
Hom_{V}(\calG_{R}(X), Y) = Hom_{k}-(X, \bF(Y))
$$
Moreover $\bF$ satisfies:}
\begin{enumerate}[(1)]
{\it
\item If $Y$ is of finite type over $Spec(V)$, then $\bF(Y)$ is a projective limit of schemes of finite type over $\overline{k}$.\label{art6-greenberg12.1-enum-1}

\item If $Y$ is affine then so is $\bF(Y)$.\label{art6-greenberg12.1-enum-2}

\item If $Y$ is a group scheme over $Spec(V)$, then $\bF(Y)$ is group scheme over $\overline{k}$ in such a way that the adjointness isomorphism of (\ref{art6-greenberg12.1-enum-2}) is a group morphism functorially in $X$.\label{art6-greenberg12.1-enum-3}}
\end{enumerate} 


This brings us to what we will call the second theorem of Greenberg. In this case we are assemblings parts of \S \ref{art6-sec-6}, Proposition \ref{art6-sec-1} of \cite{art6-keyMG I} and Proposition \ref{art6-prop2.2-enum-(2)} and the structure theorem of \cite{art6-keyMG II}, \S \ref{art6-sec-2}. Assume that $R= \lim\limits_{\substack{\longrightarrow \\n}}$ where $R_{n}$ is a ring scheme over $\overline{k}$ which is $\overline{k}$. isomorphic to $\bA\frac{n}{k}$, affine $n+1$ space over $\overline{k}$. Let $I_{n}$ be the scheme of ideals in $R$ corresponding to the kernel of the projection $R \rightarrow R_{n}$. Let $I_{n}^{r}$ be the scheme of ideals in $R_{r}$ such that $o\rightarrow I_{n}^{r} \rightarrow R_{r}\rightarrow 0$ is an exact sequence of group schemes for the additive structure. Assume that $I_{n}^{r}$ is affine $n-r$ space over $\overline{k}$ and that $R_{r}$ is a locally trivial fiber space over $R_{n}$ with fibre $I_{n}^{R}$ and the it is in fact a vector bundle over $R_{n}$. Let $\calG_{n} = \calG_{R_{n}}$ and let $\bF_{n}$ be its right adjoint. Let $\bU_{n}(Y)$ be the kernel of $\bF(Y) \rightarrow \bF_{n}(Y)$ and let $\bU_{n}^{r}(Y)$ be the kernel of $\bF_{r}(Y) \rightarrow \bF_{n}(Y)$, $(r > n)$. 

\subsection{Second Theorem of Greenberg}\label{art6-subsec-12.2}
\textit{Let $Y$ be a smooth group\break scheme with connected fibres over $Spec(V)$. If $r \geq n \geq 0$, $\bU_{n}^{r}$ is a finite dimensional unipotent group scheme.  Moreover, $\bF_{r}(Y)$ is $Spec(V)$ isomorphic to the total space of a vector bundale over $\bF_{0}(Y)$.}

We will refer to $bU_{n}(Y)$ (respectively, $\bU_{n}^{r}(Y)$) as the congruence subscheme of $bF(Y)$ (respectively $\bF_{r}(Y))$ of level $n$, and we shall call $\bF(Y)$ the realization of $Y$ over $\overline{k}$.

Now we return to the notation and conventions, of \S \ref{art6-sec-10}. Assume that $\overline{k}$ in perfect. Let $X = \calI(G, B, N, S)$. By Proposition \ref{art6-proposition-10.3}, the $G$ action os monumental. We will show that $X$ is a monumental complex over $G$. One of the more astonishing results i \cite{art6-keyBT II} is that for any $\sigma \in X$, there is group scheme over $\calO$, $M_{\sigma}$, so that the generic fiber of $M_{\sigma}$ is the base extension to $K$ of $\bG$ and such that $M_{\sigma}(\calO) = G_{\sigma}$. Since $\bG$ is split and simply connected and $\omega$ is discrete, $M_{\sigma}$ can be assumed to be smooth with connected fibers. These group schemes are determined up to isomorphism if one requires that they admit Bruhat decompositions of a particular type. (see section \ref{art6-coro-4.6} of \cite{art6-keyBT II}) We may assume that the schemes $M_{\sigma}$ are carried to each other by the conjugation action on the generic fibre. Finally we will assume that $\calO$ is the ring of $\overline{k}$ points of a ring scheme which is a projective limit of affine spaces over $\overline{k}$. This is the case when $\calO$ is the ring of Witt vectors of $\overline{k}$ or the ring the of formal power series in one variable over $\overline{k}$. Then for any $\sigma \in X$, $G_{\sigma}$ is the set of $\overline{k}$ points of $\bF M_{\sigma}$, the Greenberg realization of $M_{\sigma}$ over $\overline{k}$.  

Now we may describe the affine complex. Choose $\calO$ as above to be the $\overline{k}$ points of a ring scheme isomorphic to an inverse limit of affine spaces. Choose $\bG, \omega, B$ etc. as in \S \ref{art6-sec-11} and let $X = \calI(G, B, N, S)$. For each $\sigma in X$, le $\calA(\sigma)$ be the $\overline{k}$ coordinate ring of $\bF M_{\sigma}$. The verification that $(G, X, \calA)$ is a monumental complex is now an  entirely routine affair. This monumental complex over $G$ is what we shall call the affine complex of $G$ over $\overline{k}$.

\section{Locally Rational Carapaces}\label{art6-sec-13}
 In this section and for the remainder of this discussion $(G, X, \calA)$ will always denote a monumental $G$-complex over the field , $k$, Recall that the action of the $G$-structure on $\calA$ can be described by the equation $\Phi_{\Gg, \sigma}(f) = f \circ c_{\Gg^{-1}}$ where $c_{y}$ denotes conjugation by $y$. Recall also that if $M$ is any proalgebraic group with coordinates ring, $A$, over $k$, and if $V$ is a rational representation of $M$ with structure map $\beta : V \rightarrow V \otimes_{k} A$, then if $M$ acts on $A$ be conjugation,  $\gamma_{\Gg}(f) = f \circ c_{g^{-1}}$, then $\beta$ is equivariant as a map from the representation, $V$, to the representation $V\otimes_{k} A$, where the actions on $V$ and $\calA$ are determined by $\beta$ and $\gamma$ respectively.

\begin{definition}
Let $(G, X, \calA)$ be a monmental complex and let $\Phi$ be the $G$-structure on $\calA$. A \textit{locally rational carapace of representations} on $(G, X, \calA)$ is a $G$-carapace of $k$-vector spaces on $X$, $V$, with $G$ structure, $\psi$ and a morphism of $G$-carapaces, $beta : V \rightarrow V \otimes_{k}\calA$ so the for each $\sigma$, the map $\beta_{\sigma} : V(\sigma) \rightarrow V(\sigma)\otimes_{k}\calA(\sigma)$ is a comodule structure map in such a way that for each $\Gg, \psi_{\Gg, \sigma}$ is morphism of comodules.
\end{definition}

A morphism of locally rational carapaces is a $G$-morphism of $G$-carapaces, $f : V\rightarrow U$ which is a comodule morphism on each segment. It is clear that the locally rational carpaces on $(G, X, \calA)$ are an abelian category.

The reader is cautioned to note that that the $G$-structures on $\calA$ and $V$ exist quite apart from the local comodule structure map $\beta$. In particular $\calA$ admits several local comodule structures corresponding to left translation,  right translation and conjugation. These are non-isomorphic locally rational structures on the same $G$-carpace. The comultiplication $\mu : \calA \rightarrow \calA\otimes_{k}\calA$ may be regarded as the local structure corresponding locally to right translation. We will write $\calA^{r}$ for $\calA$ with this local comodule structure. 

The reader is cautioned to note that that the $G$-structures on $\calA$ and $V$ exist quite apart from the local comodule structure map, $\beta$. In particular $\calA$ admits several local comodule structure corresponding to left translation, right translation and conjugation. These are non-isomorphic locally rational structures on the same $G$-carapace. The comultiplication, $\mu :\calA \rightarrow \calA \otimes_{k} \calA$ may be regarded as the local structure corresponding locally to right translation. We will write $\calA^{r}$ for $\calA$ with this local comodule structure.

\begin{prop}\label{art6-proposition-13.2}
Let $(G, X, \calA)$ be a monumental complex and let $V$, $\beta$ be a locally rational carapace of representation on $(G, X, \calA)$. Let $mu, \epsilon, s$ denote the structure morphisms on $\calA$. Then
\begin{enumerate}[(1)]
\item The structures on $\calA$ endow the segment, $\Sigma(X, \calA)$, with the structure of a co-algebra with co-unit and antipode.\label{art6-proposition13.2-enum-1}

\item The structure morphism, $\beta$, makes the segment, $\Sigma(X, V)$ into a comodule
 over $\Sigma(X, \calA)$.\label{art6-proposition13.2-enum-2}

 \item Both the co-algebra structure data on $\Sigma(X, \calA)$ and the comodule structure map on $\Sigma(X, V)$ are $G$ morphisms of representations.\label{art6-proposition13.2-enum-3}
\end{enumerate}
\end{prop}

\begin{proof}
Inn this proof, we will make use of two functional properties of the map, $t_{A, B}$ of Proposition
\ref{art6-prop-2.2}, \ref{art6-prop2.2-enum-(3)}) which have not been established. Let $A, A', B, B'$ and $C$ be $k$-carapaces on $X$ and let $\alpha : A \rightarrow A'$ and $ \beta : B \rightarrow B'$ be morphisms. The two functorial properties, whose proofs we leave to the reader, are these:
\begin{align}\label{art6-eq-13.3}
t_{A' B'} \circ \Sigma(\alpha \otimes \beta) &= (\Sigma \alpha \otimes \Sigma \beta) \circ t_{A, B}\nonumber\\
(t_{A, B}\otimes id_{\Sigma C}) \circ t_{A \otimes B,C} &= (id_{\Sigma A} \otimes t_{B, C}) \circ t_{A, B \otimes C}
\end{align}

Now define structure data on $\Sigma \calA$ as follows. Let $\mu^{\Sigma} = t_{\calA, \calA} \circ \Sigma\mu$, let $e^{\sigma} = \Sigma e$ and let $s^{\Sigma} = \Sigma s$.

The proof, for example, of co-associativity is the following computation:
\begin{align*}
(\mu^{\Sigma} \otimes id_{\Sigma \calA}) \circ \mu^{\Sigma} &= (t_{\calA, \calA} \otimes id_{\Sigma \calA}) \circ ((\Sigma\mu) \otimes id_{\Sigma \calA}) \circ t_{\calA, \calA} \circ \Sigma\mu \\
&= (t_{\calA, \calA} \otimes id_{\Sigma \calA}) \circ  t_{\calA \otimes \calA, \calA} \circ \Sigma(\mu \otimes id_{\calA}) \circ \Sigma_{\mu}\\
&= (t_{\calA, \calA} \otimes id_{\Sigma\calA}) \circ t_{\calA \otimes, \calA, \calA} \circ \Sigma(id_{\calA} \otimes \mu) \circ \Sigma_{\mu}
\end{align*}
by the co-associativity of $\mu$
\begin{align*}
& = (id_{\Sigma\calA} \otimes t_{\calA, \calA}) \circ t_{A\calA, \calA \otimes \calA} \circ \Sigma(id_{\calA} \otimes \mu) \otimes \Sigma\mu \quad \text{by} (\ref{art6-proposition13.2-enum-3})\\
&= (\id_{\Sigma \calA} \otimes t_{\calA, \calA}) \circ (id_{\Sigma \calA} \otimes \Sigma\mu) \circ t_{\calA, \calA} \circ \Sigma\mu\\
&=[id_{\Sigma \calA} \otimes (t_{\calA, \calA} \circ \Sigma\mu)]\circ (t_{\calA, \calA} \circ \Sigma\mu)\\
&= (id_{\Sigma \calA} \otimes \mu^{\Sigma}) \circ \mu^{\Sigma}
\end{align*}

The proof of the remaining axioms making $\Sigma \calA$ into a coalgebra with co-unit and antipode are similar and are, for that reason, left to the reader.

Define the co-action on $\Sigma V$ by the equation, $\beta^{\sigma} = t_{V, \calA} \circ \Sigma \beta$. The proof that this is co-associative in truly indentical to the above computation with the first $\calA$'s in the expression there replaced by $V$'s.

The last numbered assertion just follows from functoriality.
\end{proof}

\begin{definition}\label{art6-definition-13.4}
Let $(G, X, \calA)$ be monumental complex. The \textit{algebra of measures} on $(G, X, \calA)$, which we write $\calL^{1}(G, X, \calA)$ or, more briefly as $\calL^{1}(X)$ when no confusion will result, is the algebra $(\Sigma \calA)^{*}$, the $k$-linear dual of $\Sigma\calA$. 
\end{definition}

 Much of what follows depends on a natural $\calL^{1}(G, X, \calA)$-structure on the exoskeletal homology groups of a locally rational carapace. To construct this action we must examine the functor, $\calT_{0}$ of \S \ref{art6-sec-7},
 (\ref{art6-definition7.1-enum-1}), the first term in the canonical brittle resolution. Recall that for any carapace $\calA, [\calT_{0}(A)](\sigma)= \coprod_{\tau \subseteq \sigma}A(\tau)$ and the expansions are the natural inclusions . Moreover, the boundary map, $\delta_{0}$ naturally maps $\calT_{0}(A)$ into $A$.

\begin{prop}\label{art6-proposition-13.5}
Let $(G, X, \calA)$ be a monumental complex and let $V$ be a locally rational carapace on $X$ with structure map $\beta$. Then
\begin{enumerate}[(1)]
\item The carapace, $\calT_{0}(V)$, admits a natural structure map, $\beta_{0}$, making it into a
locally rational carapace.\label{art6-proposition13.5-enum-1}

\item The structuremap , $\beta_{0}$ is uniquely determined by the requirement that $\delta_{0}$ be a morphism of locally rational carapaces.\label{art6-proposition13.5-enum-2}

\item If $V(\sigma)$ is finite dimensional for each $\sigma$ then the same is
 true of $\calT_{0}(V)$.\label{art6-proposition13.5-enum-3}
\end{enumerate}
\end{prop}

\begin{proof}
Suppose that $(V, \beta)$ is locally rational and that $\tau \subseteq \sigma$ are two simplices in $X$. Then, $(id_{V(\tau)}\otimes e_{\calA, \tau}^{\sigma})\circ \beta_{\tau}= \beta_{\tau, \sigma}$ makes $V(\tau)$ inti a $\calA(\sigma)$ comodule in such a way that  the diagram,
\begin{equation*}
\vcenter{
\xymatrix{
V(\tau)\ar[d]_-{V, T^{\sigma}}\ar[r]^-{\beta_{\tau, \sigma}} & V(\tau \otimes \calA(\sigma))\ar[d]^-{e_{V, \tau}^{\sigma}}\\
V(\sigma)\ar[r]^-{\beta \sigma}& V(\sigma)\otimes \calA(\sigma)
}}
\end{equation*}
commutes. Thus $(\beta_{0})_{\sigma} = coprod_{\tau \subseteq \sigma} \beta_{\tau, \sigma}$ is comodule structure map on $[calT_{0}(V)](\sigma)$ Then $(*)$ may be applied twice, once to prove that $calT_{0}(V)$ is a carapace of co-modules over $\calA$ and again to prove that $\beta_{0}$ commutes with the two comodule structure morphisms. One verifies directly that $\beta_{0}$ is a $G$-morphism. The uniqueness statement is clear as is the finiteness statement.
\end{proof}

\begin{prop}\label{art6-proposition-13.6}
Let $(G, X, \calA)$ be a monumental complex. Then for each $r geq 0$ the r'th exoskeletal cohomology, $H_{r}(X, -)$, is covariant functor from the category of locally rational carapaces on $X$ to the category of left $\calL^{1}(G, X, \calA)$-modules. 
\end{prop}

\begin{proof}
By Proposition \ref{art6-proposition-13.2}, $\Sigma$ is a functor from the category of locally rational carapaces to the category of $\Sigma(\calA)$ co-modules. The identity functor takes $\Sigma(\calA)$-co-modules to
$\calL^{1}(X)= (\Sigma(\calA))^{*}$-modules. Thus $\Sigma$ is a covariant functor to the category of left $\cal^{1}(X)$ modules.

By Proposition \ref{art6-proposition-13.5} and the definition of the canonical brittle resolution (Definition \ref{art6-definition-7.1}), the canonical brittle resolution of $V$ is a resolution by locally rational carapaces with boundary maps which are morphisms of locally rational carapaces. Thus the Alexander chains are a complex of $\calL^{1}(X)$-modules. The result follows immediately. 
\end{proof}

We shall be working with subalgebras of $\calL^{1}(X)$. In consequence, a ``working description" of it might be of use. First notice that $\calL^{1}(X) = Hom_{k}(\Sigma \calA, k) = \Hom_{X, k}(\calA, k_{X})$. Thus a typical element of $\calL^{1}(X)$ is a family, $\{\partial_{\sigma}\}_{\sigma \in X}$ where $\partial_{\sigma} \in \calA(\sigma)^{*}$, the linear dual of $\calA(\sigma)$. The coherence condition on the family, $\{\partial_{\sigma}\}_{\sigma \in X}$ is the commutativity of:
\begin{equation}\label{art6-eq-13.7}
\vcenter{
\xymatrix{
\calA(\sigma) \ar[d]_-{e_{\calA, \sigma}^{\tau}} \ar[r]^-{\partial_{\sigma}} & k \ar@{=}[d]\\
\calA(\tau) \ar[r]_{\partial_{\tau}} & k
}}
\end{equation}
for every pair, $\sigma \subseteq \tau$. This should be understood in the following sense. For any pro-algebraic group, the linear dual of its coordinates ring is an algebra under convolution. If one group contains another as a closed subgroup, then the dual of its coordinate ring contains the dual of the coordinate ring of the subgroup. Thus,
\ref{art6-eq-13.7}) says that whenever $\sigma \subseteq \tau$ the $\partial_{\sigma} \in (\calA(\tau))^{*}$. In particular $\partial_{\sigma} \in \bigcap_{\gamma \supseteq \sigma}$ as $\gamma$ ranges over the chambers containing $\sigma$. In each of the main examples, this means that $\partial_{\sigma}$ is in linear dual of th coordinate ring of the radical of $G_{\sigma}$ . This prompts us to define the $X$-radical of a stabilizer, $G_{\sigma}$ as the intersection, $\calR G(\sigma) = \bigcap_{\gamma \supseteq \sigma}G_{\gamma}$ where the intersection is taken over all chambers, $\gamma$, which contain $\sigma$ .

Henceforth we will write $\calA^{*}(\sigma)$ for the linear dual of $\calA(\sigma)$. Recall that for any commutative Hopf algebra $A$, the dual algebra, $A^{*}$ can be identified with the algebra of $k$-linear endomorphisms of $A$ which commute with left translation or alternatively with those that commute with right translation. That is, $\omega : A\rightarrow A$ is an endomorphism commuting with left translations, then $\phi = e_{A} \circ \omega \in A^{*}$ is an element of the dual such that $\phi * a = \omega (a)$ for all $a\in A$ and where the operation of $\phi$ is by right convolution. There is a similar statement with left and right interchanged. With this in mind, it is clear that $\calL^{1}(X)$ is the algebra of $k$ endomorphisms of $\calA$ which are co-module morphisms for left translation (but not necessarily $G$-morphisms).

The $G$-action on $\calL^{1}$ can now be described. If $\Gg \in G_{\sigma}$ then, $\Gg$ can be thought of as a $k$-homomorphism form $\calA(\sigma)$ to $k$ and so as an element of $\calA^{*}(\sigma)$ and $G_{\sigma}$ can be thought of as a subgroup of the unit group of $\calA^{*}(\sigma)$. Hence in this case $\Gg$ acts by true conjugation, $\Gg \cdot \partial = \Gg \partial\Gg^{-1}$. More generally the action is $(\Gg \cdot \partial)_{\sigma} = \partial_{\Gg \sigma} \circ \Phi_{\Gg \sigma}$.

\section{The Injective Co-generator}\label{art6-sec-14}

Let $(G, X, \calA)$ be a monumental complex. In this section we show that the category of locally rational carapaces admits an injective cogenerator. We give a particular injective cogenerator and we use it to construct a module category containing an image of the locally rational carapaces.

Let $C$ denote a chamber in $X$ and let $A$ denote an apartment containing $C$. The unadorned symbols, $\tau_{i}$ and $\calS^{i}$ will denote the canonical brittle and flabby resolutions on $X$ (see \ref{art6-eq-8.3}) while $\calT_{i}^{C}$ and $\calS_{C}^{i}$ will denote those on $C$ and $\calT_{i}^{A}$ and $\calS_{A}^{i}$ those on $A$.
 Let $\iota : C \rightarrow X$ be the inclusion. By Theorem \ref{art6-thm-9.12}, $\iota^{*}$ is an isomorphism of categories from $G$-carapaces on $X$ to $G_{*}$-carapaces on $C$. Recall that $\calI_{C}$ denotes its inverse. 

The locally rational carapaces on $X$ are a subcategory of the $G$-carapaces on $X$ ans so $\iota^{*}$ carries them to a subcategory on $C$. The reader can work out their properties when necessary; we will frequently argue in that category rather than the category of locally rational carapaces on $X$. For a $G$-carapace on $X, V$, we will $V_{C}$ to denote its restriction to $C$.

If $R$ is the coordinate ring of the proalgebraic group, $H$, it admits several structures as a rational representation.
Let $\mu, e$ and $s$ be the structural data for $R$. Write $R^{\ell}$ for the left translation module, $R^{\tau}$ for the right  translations module and $R$ unadorned for the conjugating action, $\gamma_{x}(a)(g) = a(x^{-1} gx)$. Notice that $s$ establishes an isomorphism between $R^{\ell}$ and $R^{\tau}$. Consequently except when the particular features of a calculation or proof demand other wise we will always write $R^{\tau}$ for this representation.

For a representation of $H$, $M$, write $M^{\flat}$ to mean the vector space, $M$ refurbished with the trivial representation. If $\beta : M \rightarrow M \otimes_{k}R$ is the coaction on $M$, then the coassociativity of $\beta$ is precisely equivalent to the statement that $\beta$ is an $H$ morphism from $M$ to $M^{\flat} \otimes_{k}R^{\tau}$ It is also true that $\beta $ is an $H$-morphism from $M$ to $M \otimes_{k} R$ where the tensor product is with respect to the given structure on $M$ and the conjugating representation on $R$.

If $M$ is any vector space equipped with the trivial representation, then $M\otimes_{k} R^{\tau}$ is $H$-injective, (It is, in fact, co-free.) To prove it, let $f: N \rightarrow M\otimes_{k}R^{\tau}$ be an $H$-map, and let $j : N \rightarrow Q$ be an $H$-monomorphism. Let $\xi : Q \rightarrow Q \otimes_{k} R$ be the coaction. Just choose $\phi : Q \rightarrow M$ so the $\phi \circ j = (id_{M}\otimes e) \circ f$. Then it is straightforward to verify that $(\phi \otimes id_{R}) \circ \xi $ is a comodule map from $Q$ to $M \otimes_{k} R$ whose composition with $j$ is just $f$.

Suppose that $P$ is a closed subgroup of $H$ of finite codimension. If $N$ is a representation of $P$ then the induced algebraic representation is a representation of $H$, $I_{H/P}(N)$ together with a $P$ morphism,
$\epsilon_{N} : I_{H/P}(N)\rightarrow N$ inducing the Frobenius reciprocity isomorphism, $Hom_{H}(W, I_{H/P}(N))= Hom_{P}(W|_{P}, N)$. To construct $I_{H/P}(N)$ consider $N \otimes_{k} R$. The subgroup $P$ acts on this product diagonally through the given representation on $N$ and right translation on $R$. In addition, $H$ acts by left translation on $R$ and the trivial action on $N$. The actions of $H$ and $P$ on $N \otimes_{k} R$ commute and so $(N \otimes_{k} R)^{P}$ is an $H$ sub-representation of $N^{\flat} \otimes_{k} R^{\ell}$. Then, $I_{H/P}(N) = (N \otimes_{k} R)^{P}$, and the map $\epsilon_{N}$ is the restriction of $id_{N}\otimes e$. Composing $id_{N}\otimes s$ with the inclusion of $I_{H/P}(N)$ in $N\otimes_{k}R$, we obtain a functorial map: 
\begin{equation}\label{art6-eq-14.1}
\imath : I_{H/P}(N) \hookrightarrow N^{\flat} \otimes_{k} R^{\tau}
\end{equation}
The following is crucial.

\begin{lem}\label{art6-lemma-14.2}
Let $H$ be profinite with structural data as above. Let $V$ be a rational representation of $H$. There is a vector space, $U$, and exact sequence,
$$
0 \rightarrow V \rightarrow V\otimes_{k} R^{\tau} \rightarrow U \otimes_{k}R^{\tau}
$$
\end{lem}

\begin{proof}
Let $\alpha : V \rightarrow V^{\flat} \otimes_{k}R^{r}$ be the coaction viewed as an $H$-morphism. We will construct a morphism $\psi : V^{\flat} \otimes_{k}R^{\tau} \rightarrow (V \otimes_{k} R)^{\flat} \otimes_{k}R^{\tau}$ so that $ker(\psi) = im(\alpha)$. Let $nu(v \otimes a) = v \otimes 1 \times a $ and let $(\lambda = id_{V} \otimes m \otimes id_{R})\circ (\alpha \otimes id_{R} \otimes id_{R}) \circ id_{V}[(s \otimes id_{R}) \circ \mu]$. These are both $H$ morphisms for notice that if $\tau(v \otimes r \otimes s) = v \otimes s \otimes r$, then $\tau \circ \lambda$ is a comodule structure on $V^{\flat} \otimes_{k} R^{\tau}$. With this interpretation, the kernel of $\tau \circ (\nu \lambda)$ is just the space of invariants for this action. But then $V \otimes_{k} R$ is the module of sections for a homogeneous bundle on $H$ and the image of $\alpha$ is the subspace of invariant sections. Applying $\tau$ again. we have shown that the kernel of $\nu-\lambda$ is the image.

Notice that as a $G_{*}$-carapace, $\calA_{C}$ is equipped with  the conjugating representation. If $V$ is a rational $G_{*}$-carapace, then $V \otimes_{k} \calA_{c}$ always denotes the tensor product with respect to this structure. The comodule structure, $\alpha : V\rightarrow V \otimes_{k} \calA_{C}$ is a $G_{*}$-morphism with respect to this structure. is not a $G_{*}$-morphism with respect to the right translation action.

Define the carapace, $A_{C}^{\tau}$ by the equation, $calA_{C}^{r}(\sigma) = \calA(\sigma)^{\tau}$. Let $\calA^{r} = \calI_{C}(\calA_{C}^{r})$.  Clearly, $\calA$ and $\calA^{\tau}$ are not isomorphic as  $G$ carapaces.

Let $\calS_{C}^{0}(k_{c}) = \bI_{C}$. Since $k$ is field, $\bI_{C}$ is injective. Let $\bI\calA_{C}^{r}= \bI_{C}\otimes_{k}\calA_{C}^{r}$. Then $id_{\bI_{c}} \otimes \mu$ makes $\bI\calA_{C}^{\tau}$ into a locally rational carapaces of $G_{*}$-modules. Tensoring the natural inclusion, $k_{C}\hookrightarrow \bI_{C}$, with $\calA^{\tau}$ yields the natural inclusion:
$$
\jmath: \calA_{C}^{\tau} \rightarrow \bI\calA_{C}^{\tau}
$$

In general we will call a carapace locally finite if each of its segments is a finite module.
\end{proof}

\begin{prop}\label{art6-prop-14.3}
Let $(G, X, \calA)$ be monumental with chamber, $C$. Let $J$ be an injective $k$-carapace on $C$. Then $J\otimes_{k} \calA_{C}^{r}$ is injective in the category of rational $G_{*}$-carapaces.
\end{prop}

\begin{proof}
The corresponding proof for a proalgebraic group globalizes. Let $U$ and $V$ be two rational $G_{*}$ carapaces with co-actions, $\alpha : U \rightarrow U \otimes_{k}\calA_{C}$ and $\beta : V\rightarrow V  \otimes_{k} \calA_{C}$ respectively. Let $\nu:U \rightarrow V$ be a $G_{*}$-monomorphism and let $f : U \rightarrow J \otimes_{k} \calA_{C}^{r}$ be any $G_{*}$ map. Let $e_{\calA} : \calA \rightarrow k_{C}$ be the counit. Then $(id_{J} \otimes e_{\calA}) \circ f$ maps $U$ to the $k$ injective, $J$. Hence there is a map, $\phi : V \rightarrow J$ such taht $(id_{J} \otimes e_{\calA}) \circ f = \phi  \circ \mu$. Then $(\phi \otimes id_{\calA}) \circ \beta$ maps $V$ to $J \otimes_{k}\calA^{r}$ and  a routine computation shows it to be a $G_{*}$-morphism extending $f$. 
\end{proof}

\begin{definition}\label{art6-definition-14.4}
Let $(G, X, \calA)$ be monumental with chamber, $C$. Let $\sigma \subseteq C $ be a face and let $V$ be a rational $G_{\sigma}$-module. Let $\widetilde{V}^{\sigma}$ be the carapace on $C$ with values:
\begin{equation}
\begin{aligned}\label{art6-eq-14.5}
\widetilde{V}^{\sigma}(\tau) &= I_{G_{\tau}/G_{\sigma}}(V) \quad \text{if}\; \tau \subseteq \sigma\\
\widetilde{V}^{\sigma}(\tau) &= (0) \quad \text{otherwise}
\end{aligned}
\end{equation}
The expansions are the canonical maps corresponding to transitivity of induction. The locally rational carapace on $X, \calI_{C}(\widetilde{V}^{\sigma})$ will be written, $V_{\sigma}^{||}$.
\end{definition}

\begin{lem}\label{art6-lemma-14.6}
Let $(G, X, calA)$ be monumental with chamber, $C$. Let $V$ be a representation of $G_{\sigma}$ for some face $\sigma$ in $C$. Then if $U$ is any locally rational $G$-carapace,
$$
Hom_{X, G}(U,V_{\sigma}^{||})=Hom_{C, G}(\iota^{*} U, \widetilde{V}^{\sigma}) = Hom_{G_{\sigma}}(U(\sigma), V)
$$
\end{lem}

\begin{proof}
The first equality is just the isomorphism induced by the isomorphism of categories, $\iota^{*}$. To prove the secon, suppose the $f \in Hom_{C, G_{*}}\break (\iota U, \widetilde{V}^{\sigma})$ We must show that $f$ is uniquely determined by $f_{\sigma}$. If $\tau$ is not a face of $\sigma$ then, $f_{\tau} = 0$. Write $e_{V, \tau}^{\sigma}$ for the expansion in $\widetilde{V}^{\sigma}$. If $\tau \subseteq \sigma$, then $e_{V, \tau}^{\sigma} circ f_{\tau} = f_{\sigma} \circ e_{U, \tau}^{\sigma}$. But $e_{V, \tau}^{\sigma} = \epsilon_{V}$ and so $f_{\tau}$ is the $G_{\tau}$-morphism uniquely determined by Frobenius reciprocity. The result follows.
\end{proof}

\begin{lem}\label{art6-lemma-14.7}
Let $(G, X, \calA)$ be monumetal with chamaber, $C$. Then, for each face, $\sigma$,
$$
\widetilde{\calA(\sigma)^{\tau}}^{\sigma} = (k \downarrow^{\sigma})\otimes_{k} \calA_{C}^{\tau}
$$ 
\end{lem}

\begin{proof}
This is just the following observation. Let $H$ be a group and let $P$ be a closed subgroup with coordinate rings, $R$ and $S$ respectively. Then $I_{H/P}(S^{\tau}) = R^{\tau}$ and the identification is functorial.
\end{proof}

\begin{theorem}\label{art6-thm-14.8}
Let $(G, X, \calA)$ be monumental with chamber, $C$. Then:
\begin{enumerate}[(1)]
\item If $V$ is an injective rational $G_{\sigma}$-module then, $\widetilde{V}^{\sigma}$ (respectively, $V_{\sigma}^{||})$ is an injective rational  $G_{*}$ carapace on $C$ (respectively an injectively locally rational $G$-carapace on $X$).\label{art6-thm14.8-enum-1}

\item The carapace, $\bI_{X}\calA_{C}^{\tau} = \calI_{C}(\calI \cal_{C})$ is injective in teh category of locally rational carapaces on $X$.\label{art6-thm14.8-enum-2}

\item  Let $V$ be a locally rational carapace on $X$. Then there are $k$-vector spaces, $M$ and $N$ and an exact sequence of locally rational $G$-carapaces:\label{art6-thm14.8-enum-3}
$$
0 \rightarrow V \rightarrow M \otimes_{k}\bI_{X}\calA^{\tau} \rightarrow N \otimes_{k}\bI_{X}\calA^{\tau}
$$
If $V$ is locally finite, then $M$ may be taken to be finite dimensional.
\end{enumerate}
\end{theorem}

\begin{proof}
To prove \ref{art6-thm14.8-enum-1}),notice that it suffices to prove the statement for rational $G_{*}$ carapces on $C$. Notice that Lemma \ref{art6-lemma-14.6} implices that the functor, $\widetilde{V}^{\sigma}$, carrying rational $G_{\sigma}$ modules, $V$, to $G_{*}$-carapaces is right adjoint to the exact functor which associates to ca carapace its $\sigma$ segment. The statement is an immediate consequence.

To prove \ref{art6-thm14.8-enum-2}),notice that $\bI_{X}\calA_{C}^{\tau} = \calI_{C}(\bI_{C}) \otimes_{k} \calA^{\tau}$ and so by Proposition \ref{art6-prop-14.3} it is injective.

Now we establish \ref{art6-thm14.8-enum-3}). Making use of the canonical equivalence, we may prove the correspondig statement for rational $G_{*}$ carapaces on $C$. Thus let $V$ be rational on $C$.

Let $\alpha : V \rightarrow V \otimes_{k} \calA$ be the coaction on $V$. For each face in $C$, $\Sigma$, we may use the correspondence of lemma \ref{art6-sec-6} to construct the unique map, $\psi: V \rightarrow \widetilde{V(\sigma)}^{\sigma}$, such that $\psi_{\sigma} = id$. Now consider the canonical morphisms $\imath_{V(\sigma)}$ of
 (\ref{art6-thm14.8-enum-1}). These morphisms induce a monomorphism of $G_{*}$-carapaces, $\imath^{\sigma} : \widetilde{V(\sigma)}^{\sigma} \rightarrow [V(\sigma)^{\flat}\downarrow^{\sigma}]\otimes_{k}\calA^{\tau}$. Let $f^{\sigma} = \imath^{\sigma} \circ \psi$ and let $f = \prod_{\sigma \subseteq C}f^{\sigma}$. This is an injective map from $V$ to
 $$
 \prod_{\sigma \subseteq C}(V(\sigma)^\flat \downarrow^{\sigma} \otimes \calA^{\tau})
 $$

 Let $M= \coprod_{\sigma \subseteq C}V(\sigma)^{\flat}$ and let $j_{\sigma} : V(\sigma)^{\flat} \rightarrow M$ be the natural inclusion. Let $\pi_{\sigma}$ be the natural projection of
 $$
 \prod_{\sigma \subseteq C}(V(\sigma)^{\flat} \downarrow^{\sigma} \otimes \calA^{\tau})
 $$
 on $V(\sigma) \downarrow^{\sigma} \otimes \calA^{\tau}$. Then $\bI_{X}\coprod_{\sigma \subseteq C}(\pi_{\sigma} \circ f_{\sigma})$ is the required map. If $V$ is locally finite, $M$ is finite. To conclude the proof just note that this construction may be applied to the cokernel of the map we have just defined.
\end{proof}

\begin{coro}\label{art6-corollary-14.9}
The carapace $\bI_{X}\calA_{C}^{\tau}$ is an injective cogenerator for the category of locally rational $G$-carapaces on $X$.
\end{coro}

\begin{proof}
This is just Statement \ref{art6-thm14.8-enum-3}) of the theorem.
\end{proof}

\begin{coro}\label{art6-corollary-14.10}
Let $\calE_{G, X} = End_{X, G}(\bI_{X} \calA^{\tau})$. Then the functor $\calP_{X, B}$ defined by $\calP_{X, G}(V) = \Hom_{X, G}(V, \bI_{X} \calA^{\tau})$ is a fully faithfull contravariant embedding of the category of locally rational $G$-carpaces on $X$ in the category of left $\calE_{G, X}$ modules.
\end{coro}

\begin{proof}
This follows from Corollary \ref{art6-corollary-14.9} by a direct application of the Gabriel Mitchell embedding theorem.
\end{proof}

\section{Locally Rational Carapaces}\label{art6-sec-15}

In this section we will construct a section from a certain subcategory of the category of $\calE_{G, X}$ modules to the category of locally finite locally rational carapaces. We begin with a preliminary description of $\calE_{G, X}$. Recall first that is $R$ is the coordinate ring of a proalgebraic group then thelinear endomorphisms of $R$ commuting with right convolutions are just convolutions on the left with elements of the linear dual of $R$. If $\omega$ is such an involution, then let $\overline{\omega}$ be the element of the dual defined by the equation, $\overline{\omega}(f) = [\omega(f)](e)$. Then left convolution with $\overline{\omega}$ is $\omega$. Also notice that Frobenius reciprocity for the indentity subgroup is the equation, $Hom_{G}(V, M \otimes R^{\tau}) = Hom_{k}(V, M)$.

Now observe that the carapace (on $C$), $\bI\calA_{C}^{\tau}$ may be written as the finite product, $\prod_{\sigma \subseteq C}\widetilde{\calA(\sigma)^{\tau}}^{\sigma}$. Hence,
\begin{equation}\label{art6-eq-15.1}
\Hom_{X, G}(\bI \calA_{C}^{\tau}, \bI\calA_{C}^{\tau}) = \prod_{\sigma, \tau \subseteq C}\Hom_{X}, G(\widetilde{\calA(\sigma)^{\tau}}^{\sigma}, \widetilde{\calA(\sigma)^{\sigma}}^{\sigma})
\end{equation}
By Lemma \ref{art6-lemma-14.6}, $\Hom_{X}, G(\widetilde{\calA(\sigma)^\tau}^{\sigma}$, $\widetilde{\calA(\sigma)^\tau}^{\tau} = (0)$ whenever $\widetilde{\calA(\sigma)^{\tau}}^{\sigma}(\tau)= (0)$. That is, the $\Hom$ is null unless $\sigma \supseteq \tau$. When this condition does hold:
$$
\Hom_{X, G}(\widetilde{\calA(\sigma)^{\tau}}^{\sigma}, \widetilde{\calA(\tau)^\tau}^{\tau}) = Hom_{G(\tau)}(\calA(\tau)^{\tau}, \calA(\tau)^{\tau})
$$
because $\widetilde{\calA(\sigma)^{\tau}}^{\sigma} = \calA(\tau)^{\tau}$. Thus the $\sigma, \tau$ component of the homomorphism is an element $\partial \in \calA(\tau)^{*}$ operating by left convolution. Hence an element of $\calE_{G, X}$ can be represented as a matrix, $(\partial_{\tau, \sigma})_{\tau \subseteq \sigma}$, where $\partial_{\tau, \sigma} \in (\calA(\tau)^{*})$. Notice also that if $\tau \subseteq \sigma$ then $\calA(\sigma)^{*} \subseteq \calA(\tau)^{*}$. Notice also that if $\tau \subseteq \sigma$ then $\calA(\sigma)^{*} \subseteq \calA(\tau)^{*}$. Consequently, if $(\partial_{\tau, \sigma})_{\tau \subseteq \sigma}$ and $(\overline{\partial}_{\tau, \sigma})_{\tau \subseteq \sigma}$ correspond to two elements of $\calE_{X, G}$, for any $\tau, \gamma, \sigma$ such that$\tau \subseteq \gamma \subseteq \sigma$, the product, $\partial_{\tau, \gamma}\overline{\partial}_{\gamma, \sigma}$ is defined and is an element of $\calA(\tau)^{*}$.

For a locally algebraic carapace, $V, \calP_{X, G}(V)$ may also be calucilated directly: 
$$
\calP_{X, G}(V)=  \Hom_{X, G}(V, \bI \calA_{C}) = \prod_{\sigma \subseteq C}Hom_{G(\sigma)}(V(\sigma), \calA(\sigma))
$$
By the remarks above, $Hom_{G(\sigma)}(V(\sigma), \calA(\sigma))= V(\sigma)^{*}$, the contragredient. Hence, $\calP_{X, G}(V) = \prod_{\sigma \subseteq c}V(\sigma)^{*}$. Moreover, if $\partial_{\tau, \sigma}$ is an element of $\calE_{X, G}$ then $\partial_{\tau, \sigma}$ operates on $\left(e_{V, \tau}^{\sigma}\right)^{*}(u)$ for $u \in V(\sigma)^{*}$ and $(e_{v, \tau}^{\sigma})^{*}$ the adjoint of the expansion in $V$. Having established this much, we leave the remainder of the proof of the following to the reader.

\begin{prop}\label{art6-proposition-15.2}
Let $(G, X, \calA)$ be monumental and let $V$ be locally rational on $X$. Then:
\begin{enumerate}[(1)]
\item The ring $\calE_{X, G}$ is equal to the ring of matrices, $(\partial_{\tau, \sigma})_{\tau \subseteq \sigma}$. The product of two such matirces is given by $(\partial_{\tau, \sigma}) \cdot (\overline{\partial}_{\tau, \sigma}) = (\delta_{\tau,\sigma})$ where\label{art6-proposition15.2-enum-1}
$$
\delta_{\tau, \sigma} = \sum\limits_{\tau \subseteq \gamma \subseteq \sigma}\partial_{\tau, \gamma}\overline{\partial}_{\gamma, \sigma}
$$

\item The module $\calP_{X, G}(V)$ is isomorphic as an additive group to the set of vectors $(v_{\sigma})_{\sigma \subseteq C}$ where the component, $v_{\sigma}$ is in the contragredient module, $V(\sigma)^{*}$. The action of $(\partial_{\tau, \sigma})$ on $(v_{\sigma})$ yields $(u_{\sigma})$ where\label{art6-proposition15.2-enum-2}
$$
u_{\sigma} = \sum\limits_{\sigma \subseteq \subseteq C} \partial_{\sigma, \gamma}(E_{V, \sigma}^{\gamma})^{*}(v_{\gamma})
$$
\end{enumerate}
\end{prop}

For the remainder of this section, we will write $\bI$ to denote the canonical co-generator, $\bI\calA_{C}^{\tau}$. Then, $\calE_{X, G} = End_{X, G}(\bI)$. First, restriction to the chamber, $C$, is an isomorphism to the category of $G_{*}$-carapaces. On the category of locally algebraic $G_{*}$-carapaces on $C$, the segment over $\sigma$ is an exact functor to the category of algebraic representations of $G(\sigma)$. Consequently, for each $\sigma$, the segment $\bI(\sigma)$ evaluted as a carapace on $C$, is left $\calE_{X, G}$-module and the action commutes with the co-action, $\bI(\sigma) \rightarrow \bI(\sigma) \otimes \calA(\sigma)$. Evaluating the restriction of $\bI$ to $C$ on the facet, $\sigma$, we obtain the module of vectors, $(a_{\tau})_{\tau \supseteq \sigma} : a_{\tau} \in \calA(\sigma)$, By the description of $\calE_{X,G}$ above the action of the marix, $(\partial_{\gamma, \lambda})_{\gamma \subseteq \lambda}$, on the vector, $(a_{\tau})_{\tau \supset}$.
\begin{equation}
\begin{aligned}\label{art6-eq-15.3}
(\partial_{\gamma,\lambda}) \cdot (a_{\tau})_{\tau \supset \sigma} &= (b_{\tau})_{\tau \supseteq \sigma}\\
\text{where} \quad b_{\tau} &= \Sigma_{C \supseteq \gamma \supseteq \tau}\; \partial_{\tau, \gamma} \cdot a_{\gamma}
\end{aligned}
\end{equation}
Furthermore, since elements of $\calE_{X, G}$ act as morphisms of carapaces, the expansions, which are compositions of projection of functions, are left $\calE_{X,G}$-morphisms. Notice that the result would have been quite different if we had evaluated before restricting of $C$. We write $\bI_{C}(\sigma)$ to denote the $\calE_{X, G}$-module, $\bI|_{C}(\sigma)$. Since $X$ and $G$ ar fixed throughout this section, we will write $\calE$ for $\calE_{X, G}$ as long as no ambiguity will result.

\begin{lem}\label{art6-lemma-15.4}
Let $M$ be a finitely generated $\calE$-module. Let $\widetilde{\calM}$ be the $G$-carapace on $X$ whose prototype on $C$ has the $\sigma$-setment, $Hom_{\calE}(\calM, \bI_{C}(\sigma))$ and the expansions, $Hom_{\calE}(id_{\calM}, e_{\bI, \gamma}^{\lambda})$. Then $\widetilde{\calM}$ is a locally algebraic carapace on $X$. Furthermore $\widetilde{\calM}$ is a $G$ subcarapace of an finite direct sum of carapaces each isomorphic to $\bI$.
\end{lem}

\begin{proof}
As usual we need only consider the restriction of $\widetilde{M}$ to $C$. Being the composition of a convariant functor with a carapace, it is certainly a carapace. We propose to show that it is locally algebraic.

Write
\end{proof}
