\title{Algebraic Representations of Reductive Groups over Local Fields}
\markright{Algebraic Representations of Reductive Groups over Local Fields}

\author{By~ William J. Haboush\footnote{This research was founded in part by the National Science Foundation}}
\markboth{William J. Haboush}{Algebraic Representations of Reductive Groups over Local Fields}

\date{}
\maketitle



\section*{Introduction}

This paprr is an extended study of the behaviour of simplicial co- sheaves in the buildings associated to algebraic groups, both finite and infinite dimensional. Recently the theory of simplicial sheaves and co-sheaves has found a number of important applications to the representation theory and cohomology theory of finite theory of finite groups (see \cite{art6-keyT}, art6-keyRS), the computations of teh cohomology of arithmetic groups and the problem of admissible representatiosn of P-adic groups and teh Langlands classification (\cite{art6-keyCW}, \cite{art6-keyBW}). My interest has been, for the most part, the representation theory of semi-simple groups over fields of positive characteristic. In this area, of course, the driving force of much recent work has been the so-called Lusztig characteristic $p$ conjecture \cite{art6-keyL1} (so called to distinguish it from a number of other equally intersecting Lusztig conjectures). In contemplating this conjecture one is struck by certain resonances with work in admissible representations etc.

The line of argument I am hoping to achieve is something like this. One should attempt to use the homoligical algebra of simplicial co-sheaves to construct a category of representations of something like the loop group associated to th semisimple group, $G$, which have computable character theory. Then one should attempt to express the finite dimensional representations of $G$ as virtual representations in the category. Then presumably the ``generic decomposition patterns" should be formulae expressing the character of a dual Weyl module in terms of these computable characters. The hope of constructing such a theory has led me to conduct the rather extended exploration below.  

One is immediately tempted to replace harmonic analysis with a purely algebraic theory and to use this theory to do the representation-theoretic
