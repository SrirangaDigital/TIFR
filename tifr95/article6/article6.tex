\title{Algebraic Representations of Reductive Groups over Local Fields}
\markright{Algebraic Representations of Reductive Groups over Local Fields}

\author{By~ William J. Haboush\footnote{This research was founded in part by the National Science Foundation}}
\markboth{William J. Haboush}{Algebraic Representations of Reductive Groups over Local Fields}

\date{}
\maketitle



\section*{Introduction}

This paprr is an extended study of the behaviour of simplicial co- sheaves in the buildings associated to algebraic groups, both finite and infinite dimensional. Recently the theory of simplicial sheaves and co-sheaves has found a number of important applications to the representation theory and cohomology theory of finite theory of finite groups (see \cite{art6-keyT}, art6-keyRS), the computations of teh cohomology of arithmetic groups and the problem of admissible representatiosn of P-adic groups and teh Langlands classification (\cite{art6-keyCW}, \cite{art6-keyBW}). My interest has been, for the most part, the representation theory of semi-simple groups over fields of positive characteristic. In this area, of course, the driving force of much recent work has been the so-called Lusztig characteristic $p$ conjecture \cite{art6-keyL1} (so called to distinguish it from a number of other equally intersecting Lusztig conjectures). In contemplating this conjecture one is struck by certain resonances with work in admissible representations etc.

The line of argument I am hoping to achieve is something like this. One should attempt to use the homoligical algebra of simplicial co-sheaves to construct a category of representations of something like the loop group associated to th semisimple group, $G$, which have computable character theory. Then one should attempt to express the finite dimensional representations of $G$ as virtual representations in the category. Then presumably the ``generic decomposition patterns" should be formulae expressing the character of a dual Weyl module in terms of these computable characters. The hope of constructing such a theory has led me to conduct the rather extended exploration below.  

One is immediately tempted to replace harmonic analysis with a purely algebraic theory and to use this theory to do the representation-theoretic computations necessary. My replacement for harmonic analysis is this. Let $G$ be algebraic of simple type over $\bbZ$. Let $K$ be a field complete with respect to a discrete rank one valuations, let $\calO$ be the valuation ring and let $L$ be its residue field. Then consider the Bruhat-Tits building of $G(K)$. It is a simplicial complex. Let $\bG =G(K)$. Let $\calI$ be its Bruhat-Tits building. Then $\calI$ is $\bG$ equivariant. Let $k$ be another field. My idea is to consider the category of $\bG$ equivariant co-sheaves of $k$-vector spaces which are, in some sense made precise within, locally algebraic. Then in a manner entirely analogous to the classical notion of rational of rational representative functions, this category admits an injective co-generator. The endomorphism ring of this canonical co-generator is a certain algebra. Let it be denoted $\calH$. \cite{art6-keyT} shows that for suitably finite $bG$ is a certain algebra. Let it be denoted $\calH$-module. Then one may sent the class of a finite dimensional $G$- representation to the alternating sum of the left derived functors of the co-limit of its induced $\bG$-co-sheaf in the Grothendieck ring of $\calH$. In this context that one would hope to obtain interesting identities relating finite dimensional representation theory to the representation theory of $\calH$.

I have made certain choices in this discussion. As I am discussing sheaves and co-sheaves on simplicial complexes, I have decided to use the word carapaces for co-sheaves. There are three reasons: the first is that the word, co-sheaves, seems rather cobbled together, the second is that a carapace really would look like a lobster shell or such if one were to draw one and the third is the Leray used the word for sheaves and I don't like to see such a nice word go to waste.  am working for the most part with carapaces rather than sheaves because I am working on an infinite simplicial complex and in moving form limit to co-limit, one in moving from an infinite direct product sort of thing to an infinite co-product sort of thing. Thus,in using limits rather than co-limits, one loses structure just as one does in taking an adic completion of a commutative ring. I have also decided to include a discussion of the homological algebra of carapaces. Now all of this material is some sort of special case of certain kinds of sheaves on sites or the homological algebra of abelian group valued functors, but with all due apologies to those who have worked on those topics, I would prefer to formulate this material in a way which anticipates my intentions. A number of mathematicians have done this sort of thing. Tits and Solomon \cite{art6-keyS}, \cite{art6-keyT}, Stephen Smith and mark Ronan \cite{art6-keyRS} immediately come to mind. But again, working on a infinite complex has dictated that I reformulate many of the elements for this situation.

\section{Carapaces and their Homology}\label{art6-sec-1}

Basic references for this section are \cite{art6-keyMac} and \cite{art6-keyGr}. Basic notions and definition all follow those two sources. A simiplicial complex, $X$, will be a set $ver(X)$ together with a collection of finite subsets of $ver(X)$ such that when ever $\sigma \in X$ every subset of $\sigma$ is in $X$. These finite subsets of $ver(X)$ are the simplices of $X$. The dimension of $\sigma $ is its cardinal less one and the dimension of $X$, if it exists,is the maximum of the dimensions of simplices in $X$. We shall view $X$ as a category, the morphisms being the inclusions of simlices. We identify $ver(X)$ with the zero simplices of $X$. If $X$ and $Y$ are simplicial complexes a morphism of complexes from $X$ to $Y$ is just a convariant functor from $X$ to $Y$; a simplicial morphism is a morphism of complexes taking vertices to vertices.

Let $R$ be a commutative ring with unit fixed for teh remainder of this work and let $M od(R)$ denote the category of $R$-modules. Let $X$ be a simplicial complex.

\begin{definition}\label{art6-def-1.1}
\textit{An $R$-carapace} on $X$ is a convariant functor from $X$ to $M od(R)$. If $A$ and $B$ are two $R$-carapaces on $X$, \textit{a morphism} of $R$-carapaces from $A$ to $B$ is a natural transformation of functors. 
\end{definition}
 If $A$ is an $R$-carapace on $X$ and $\sigma \in X$ is a simplex, then $A(\sigma)$ is called the segment of $A$ along $\sigma$. A sequence of morphisms of $R$-carapaces will be called exact if and only if the corresponding sequence of segments is exact for each simplex in $X$. The product (respectively co-product) of a family of $R$-carapaces is the $R$-carapace whose segments and morphisms are the products (respectively co-products) of those in  the family of carapaces. If $\sigma \subseteq \tau$ is a pair of simplices in $X$ write $e_{A, \sigma}^{\tau}$ or $e_{\sigma}^{\tau}$ when there is no possibility of confusion for the map from $A(\sigma)$ to $A(\tau)$. I will call it the expansion of $A$ form $\sigma$ to $\tau$. Finally, if $A$ and $B$ are two $R$-carapaces on $X$, write $\Hom_{R, X}(A, B)$ for the $R$-module of morphisms of $R$-carapaces from $A$ to $B$.

Let $M$ be an $R$-module. Then let $M_{X}$ denote the constant carapace with value $M$. That is, its segment along any simplex is $M$ and its expansions are all the identity map. In addition for any given simplex, $\sigma$, there are two dually defined carapaces, $M \uparrow_{\sigma}$ and $M \downarrow^{\sigma}$ defined by:

\setcounter{equation}{1}
\begin{equation}
\begin{aligned}\label{art6-eq-1.2}
M \uparrow_{\sigma} (\tau) &= M(\sigma \subseteq \tau)\\
M \uparrow_{\sigma} (\tau) &= (0)(\sigma \nsubseteq \tau)
\end{aligned}
\end{equation}

\begin{equation}
\begin{aligned}\label{art6-eq-1.3}
M \downarrow^{\sigma} (\tau) &= M\quad(\sigma \supseteq \tau)\\
M \downarrow^{\sigma} (\tau) &= (0)\quad(\sigma \nsupseteq \tau)
\end{aligned}
\end{equation}

 In $M \uparrow_{\sigma}$, the expansions are the identity for paris $\tau$, $\gamma$ such that $\sigma \subseteq \tau \subseteq \gamma$ and 0 otherwise. In $M \downarrow^{\sigma}$ they are the identity for $\tau$, $\gamma$ such that $\tau \subseteq \gamma \subseteq \sigma$ and 0 otherwise.

\begin{lem}\label{art6-lemma-1.4}
Let $X$ be a simplicial complex and let $M$ be an $R$-module and let $A$ be an $R$-carapace on $X$. Then,
\begin{enumerate}[1.]
\item $\Hom_{R, X}(M \uparrow_{\sigma}, A) = Hom_{R}(M, A(\sigma))$\label{art6-enum-lemma1.4-(1)}
\item $\Hom_{R, X}(A , M \downarrow^{\sigma}) = Hom_{R}( A(\sigma), M)$\label{art6-enum-lemma1.4-(2)}
\item If $M$ is $R$-projective, then $M \uparrow_{\sigma}$ is projective in teh category of $R$-carapaces on $X$.\label{art6-enum-lemma1.4-(3)}
\item If $M$ is $R$-injective, then $M \downarrow^{\sigma}$ is injective in the category of $R$-carapaces on $X$.\label{art6-enum-lemma1.4-(4)}
\end{enumerate}
\end{lem}

\begin{proof}
Statements (\ref{art6-enum-lemma1.4-(1)}) and (\ref{art6-enum-lemma1.4-(2)}) require no proof. Note that the functor which assigns to an $R$-carapace on $X$ its segment along $\sigma$ is an exact functor This observation together with (\ref{art6-enum-lemma1.4-(1)}) and (\ref{art6-enum-lemma1.4-(2)}) implies (\ref{art6-enum-lemma1.4-(3)}) and (\ref{art6-enum-lemma1.4-(4)}). 
\end{proof}

\setcounter{prop}{4}
\begin{prop}\label{art6-prop-1.5}
Let $X$ be a simplicial complex. Then the category of $R$-carapaces on $X$ has enough projectives and enough injectives.
\end{prop}

\begin{proof}
Let $A$ be any $R$-carapace on $X$. We must show that there is a surjective map from a projective $R$-carapace to $X$ and an injective map from $A$ into an injective $R$-carapace. For each simplex, $\sigma$ in $X$ choose a projective $R$-module, $P_{\sigma}$, with a surjective map $\pi_{\sigma}$ mapping $P_{\sigma}$ onto $A(\sigma)$. Let
$$
P = \coprod\limits_{\sigma \in X}P_{\sigma} \uparrow_{\sigma} 
$$
For each $\sigma$, let $\overline{\pi}_{\sigma}$ be the morphism of carapaces corresponding to $\pi_{\sigma}$ given by \ref{art6-lemma-1.4}, \ref{art6-enum-lemma1.4-(1)}. Define $\pi$ by:
$$
\pi = \coprod\limits_{\sigma \in X}\overline{\pi}_{\sigma}.
$$
Then $\pi$ is a surjective map from a projective to $A$.

To construct an injective, choose an injective module and an inclusion, $j_{\sigma}:A(\sigma) \rightarrow I_{\sigma}$. Then define an injective carapace, $I$, and an inclusion, $j$, as products of the carapaces $I_{\sigma} \downarrow^{\sigma}$ and inclusions $\overline{j}_{\sigma}$ defined dually to the corresponding objects in the projective case. 
\end{proof}

\begin{prop}\label{art6-prop-1.6}
Let $X$ be a simplicial complex. Then
    \begin{enumerate}[(1)]
        \item If $P$ is a projective $R$-caparapace on $X$, then $P(\sigma)$ is $R$ projective for each $\sigma\in X$.\label{art6-prop1.6-enum-(1)}
        \item If $I$ is an injective $R$-carapace on $X$, then $I(\sigma)$ is $R$ injective for each $\sigma \in X$.\label{art6-prop1.6-enum-(2)}
    \end{enumerate}
\end{prop}

\begin{proof}
Suppose $P$ is projective. For each $\sigma$ let $\pi_{\sigma}: F_{\sigma}\rightarrow P(\sigma)$ be a surjective map from a projective $R$-module onto  $P(\sigma)$. Then $Q= \coprod_{\sigma \in X}F_{\sigma} \uparrow_{\sigma}$ is projective by \ref{art6-lemma-1.4} and $\coprod_{\sigma \in X}\pi_{\sigma}$ maps $Q$ onto $P$. Since $P$ is projective, it is a direct summand of $Q$. But then $P(\sigma)$ is a  direct summand of $Q(\sigma) = \coprod_{\tau \subseteq \sigma}F_{\tau}$ which is clearly projective. This establishes the first statement.

To prove the second statement, for each $\sigma$ choose and embedding,, $j_{\sigma} : I(\sigma) \rightarrow J_{\sigma}$ where $J_{\sigma}$ in $R$-injective. Then use $\prod_{\sigma\in X}$ to embed $I$ in $\prod_{\sigma \in X}J_{\sigma}$ and reason dually to the previous argument replacing, at each point where it occurs, the co-product with the product. 
\end{proof}

It is clear that there are natural homology and cohomology theories on the category of $R$-carapaces on $X$. The most natural functors to consider are the limit and the co-limit over $X$. To simplify the discussion, \textit{$C ar_{R}(X)$ denote the category of $R$-carapaces on $X$.}

\begin{definition}
Let $U$ denote any sub-category of $X$, and let $A$ be an $R$-carapace on $X$. Then
\begin{align*}
\Sigma (U, A) &= \lim\limits_{\substack{\longrightarrow \\ \sigma \in U}}A(\sigma)\\
\Gamma (U, A) &= \lim\limits_{\substack{\longleftarrow \\ \sigma \in U}}A(\sigma).
\end{align*}
We will refer to $\Sigma(U, A)$ as the \textit{segment} of $A$ over $U$ and to $\Gamma(U, A)$ as the \textit{sections} of $A$ over $U$.
\end{definition}

Certain observation are in order. Since subcategories of $X$ are certainly not in general filtering the functor, $\Sigma(U, ?)$, is right exact on $C ar_{R}(X)$. Similarly $\Gamma(U, ?)$ is left exact. Furthermore $\Gamma(U, A) = \Hom_{R, U}(R_{U}, A)$, On the other hand, $\Sigma(U, A)$ cannot be represented as a homomorphism functor in any obvious way but its definition as a direct limit allows us to conclude that:
$$
\Hom_{R}(\Sigma(X, A), M) = \Hom_{R, X}(A, M_{X})
$$ 

\begin{definition}\label{art6-definition-1.9}
For any $R$-carapace on $X$, A, let
$$
H_{n}(X, A)= L_{n}\Sigma{X, A}
$$
and let
$$
H^{n}(X, A) = R^{n}\Gamma(X, A)
$$
the left and right derived functors of $\Sigma(X, -)$ and $\Gamma(X, -)$ respectively. These groups shall be referred to as the \textit{expskeletal homology and cohomology groups} of $X$ with coefficients in $A$.
\end{definition}

\begin{example}
\textbf{The Koszul Resolution of the Constant Carapace.}

Choose an ordering on the vertices of $X$. For each $r \geq 0$, let $X(r)$ denote the set of simplices of dimension $r$. Let $K_{q}(X, R) = \coprod _{\sigma \in X_{(q)}} R \uparrow_{\sigma}$. If $A$ is any $R$-carapace on $X$, then $\bigwedge_{R}^{q}A$ is understood to be the carapace whose segment along $\sigma$ is $\bigwedge_{R}^{q}(A(\sigma))$. Then, it is not at all difficult to see that $\bigwedge_{R}^{q+1}K_{0}(X, R) = K_{q}(X, R)$ Furthermore, when
$\sigma \subseteq \tau$ there is always a natural map from $M\uparrow_{\tau}$ to $M \uparrow_{\sigma}$ obtained by applying 4,1 to $M\uparrow_{\tau}$ and noting that since $M\uparrow_{\sigma}(\sigma) = M = M\uparrow(\tau)$ there is a map in $\Hom_{R, X}(M\uparrow_{\tau}, MM\uparrow_{\sigma})$ corresponding to the identity. Make use of the ordering on the vertices of $X$ to define an alternating sum of the maps corresponding to the faces of a simplex. It is easy to see the that this gives a complex of carapaces:
$$
\ldots \rightarrow (X, R)\rightarrow K_{q-1}(X, R)\rightarrow\ldots \rightarrow K_{0}(X, R)\rightarrow R_{X}\rightarrow(0)
$$ 
Then check that the sequence of segments on $\sigma$ is just the standard Koszul resolution of the unit ideal which begina with a free module of rank $\dim(\sigma) + 1$ and the map which sends each of its generators to one. Consequently, this construction gives a resolution of the constant carapace by projectives. On the other hand it is evident that $\sigma(X, K_{q}(X, R))$ is just the $R$-module of simplicial $q$-chains on $X$ with coefficients in $R$ and that the boundaries are the standard simplicial boundaries. In this way, one verifies that the exoskeletal homology and cohomology with coefficients in a constant carapace is just the simplicial homology and cohomology. This phenomenon has been observed and exploited by Casselman and Wigner in their work on admissible representations and the cohomology of artihmetic groups \cite{art6-keyCW}.
\end{example}

\section{Operations on Carapaces}

In the category of $R$-carapaces on $X$ there is a self-evident notion of tensor product:

\begin{definition}\label{art6-definition-2.2}
Let $A$ and $B$ be two $R$-carapaces on $X$. Let:
$$
(A \otimes_{R} B)(\sigma) = A(\sigma) \otimes_{R}B(\sigma)
$$
Then $A\otimes_{R}B$ is a carapace which we will refer to as \textit{the tensor product} of $A$ and $B$.

The properties of the tensor product are for the most part clear. Most of them are stated in the following.
\end{definition}

\begin{prop}\label{art6-prop-2.2}
Let $X$ be a simplicial complex. Then the tensor product of $R$-carapaces on $X$ is an associative, symmetric, bi-additive functor right exact in both variables. Furthermore:
   \begin{enumerate}[(1)]
   \item For any $R$-carapace, $A$, $R_{X}\otimes_{R}A \simeq A$\label{art6-prop2.2-enum-(1)}
   \item If $P$ is a projective $R$-carapace then tensoring with $P$ on either side is an exact functor.\label{art6-prop2.2-enum-(2)}
   \item For any two $R$-carapaces, $A$ and $B$, there is a natural map:
      $$
      t_{A, B}:\Sigma(X, A\otimes_{R}B)\rightarrow \Sigma(X, A)\otimes_{R}\Sigma(X, B)
      $$
      Moreover, $t_{A, B}$ is a natural transformation in $A$ and $B$ and it is functorial in $X$ as well.\label{art6-prop2.2-enum-(1)}
   \end{enumerate}
\end{prop}

\begin{proof}
The first statement is self-evident; the second follows trivially from \ref{art6-prop-1.6} but the third might require some comment. To construct $t_{A, B}$ let $e_{A, \sigma} : A(\sigma) \rightarrow \Sigma(X, A)$ and $e_{B, \sigma} : B(\Sigma) \rightarrow \Sigma(X, B)$ be the expansions. Then $e_{A, \sigma} \otimes_{R} e_{B, \Sigma}$ maps $A\otimes_{R} B(\sigma)$ into $\Sigma(X, A) \otimes_{R} \Sigma(X, B)$ compatibly with respect to expansion. Since $\Sigma(X, A \otimes_{R} B)$ is a colimit, this defines $t_{A, B}$ uniquely and ensures that it is functorial as asserted. 
\end{proof}

The construction of a tensor product is thus quite straightforward but the construction of an internal homomorphism functor with the requisite adjointness properties presents certain technical difficulties. For any $\sigma \in X$ let $x(\sigma)$ denote the subcategory of $X$ whose objects are the simplices $\tau$ such that $\tau \supseteq \sigma$. The morphisms of $X(\sigma)$ are inclusions of simplices. Then $X(\sigma)$ is not a subcomplex of $X$. If $\sigma \subseteq \tau$ then $X(\tau) \subseteq X(\sigma)$. There is a natural restriction map from the group $\Hom_{R, X (\sigma)}(A|_{X(\sigma)}, B|_{X(\sigma)})$ to $\Hom_{R, X(\tau)}(A|_{X(\tau)}, B|_{X(\tau)})$. We will will write $e_{\calH, \sigma}^{\tau}$ for this restriction map.

\begin{definition}\label{art6-definition-2.3}
Let $A$ and $B$ be two $R$-carapaces on $X$. \textit{The carapace of local homomorphisms} Fro  $A$ to $B$ will be written $\calH om_{R, X}(A, B)$. Its value on $\sigma$ is
$$
\calH om_{R, X}(A, B)(\sigma) = \Hom_{R, X(\sigma)}(A|_{X}(\sigma), B|_{X}(\sigma))
$$
Its expansions are the maps $e_{\calH, \sigma}^{\tau}$
\end{definition}

For want of a direct reference, we include some discussion of the basic properties of $\calH om$.

\begin{theorem}\label{art6-thm-2.4}
The local homorophism functor, $\calH om_{R, X}(A, B)$, is additive, covariant in $B$ and contravariant in $A$ and left exact in both variables. Moreover
    \begin{enumerate}[(1)]
    \item $\calH om_{R, X}(R_{X}, A)\simeq A$, \text{\rm functorially in} $A$\label{art6-thm2.4-enum-(1)}
    \item $\Gamma(X, \calH om_{R,X}(A, B)) = \Hom_{R,X}(A, B)$\label{art6-thm2.4-enum-(2)}
    \item There is a canonical isomorphism functorial in $A, B$, and $C$,\label{art6-thm2.4-enum-(3)}
        $$
        \phi : \Hom_{R, X}(A, \calH om_{R, X}(B, C))\rightarrow \Hom_{R, X}(A \otimes_{R} B, C)
        $$
    \item There is a functorial isomorphism:\label{art6-thm2.4-enum-(4)}
    $$
    \Hom_{R, X}(R_{X}, \calH om_{R, X}(A, B)) \simeq \Hom_{R, X}(A, B)
    $$  
    \end{enumerate}
\end{theorem}

\begin{proof}
Of the three preliminary statements, only left exactness requires comment. What must be shown is the left exactness of segments of $\calH om_{R, X}(A, B)$ as $A$ and $B$ vary over short exact sequences. But\break $\calH om_{R, X}(A, B)(\sigma) = \Hom_{R,X(\sigma)}(A|_{X(\sigma)}, B_{X(\sigma)})$. Restriction to $X(\sigma)$ is exact and $\Hom_{R, X}(\sigma)$ is left exact in both of its arguments. The requisite left exactness follows immediately.

To establish (\ref{art6-thm2.4-enum-(1)}), we must establish the isomorphism on segments. But
$$
\calH om_{R, X}(R_{X}, A)(\sigma) = \Hom_{R,X(\sigma)}(R_{X}, A|_{X(\sigma)}),
$$
but this last expression is equal to:
$$
\lim_{\substack{\longleftarrow \\ \tau \in X(\sigma)}}A(\tau).
$$
However the category, $X(\sigma)$ has an initial element and so the projective limit is just $A(\sigma)$.

For Statement (\ref{art6-thm2.4-enum-(2)}), write:
$$
\Gamma(X, \calH om_{R,X}(A, B)) = \lim_{\substack{\longleftarrow \\ \tau \in X}}(A|_{X(\sigma)}, B|_{X(\sigma)})
$$
There is a natural map from $\Gamma(X, \calH om_{R,X}(A, B))$ to this projective limit. Just send $f$ to the element in the limit whose component at $\sigma$ is the restriction (the pull-back actually) of $f$ to the sub-category, $X(\sigma)$. It is a triviality to verify that this map is an isomorphism.

Rather than giving a fully detailed proof of (\ref{art6-thm2.4-enum-(3)}), we will give complete definitions of $\phi$ and a map $\psi$ which is inverse to it. The necessary verifications, though numerous and quite technical, contain no surprises and so we leave them to the reader. First suppose\break $f \in \Hom_{R,X}(A, \calH om_{R,X}(B, C))$. Then for
$a \in A(\sigma)$, $f_{\sigma}(a)$ is a family, $f_{\sigma}(a)  =\{f(a)_{\tau}\}_{\tau \supseteq \sigma}$ where
$f(a)_{\tau} \in Hom_{R}(B(\tau), C(\tau))$. The following equations express the facts that $f$ is a carapace morphism and that $f(a)$ is carapace morphism from $B|_(\sigma)$ to $C|_{x(\sigma)}$:
\begin{equation}
f_{\tau}(e_{A,\sigma}^{\tau}(a))_{\gamma} = f_{\sigma}(a)_{\gamma}\tag{5}\label{eq:proof-5}
\end{equation}
\begin{equation}
e_{C, \tau}^{\gamma} \circ f_{\tau}(a)_{\tau} = f_{\tau}(a)\gamma \circ e_{B,\tau}^{\gamma}\tag{6}\label{eq:proof-6}
\end{equation}
Then we may define $\phi : \Hom_{R,X}(A, \calH om_{R,X}(B,C))\rightarrow \Hom_{R,X}(A \otimes_{R}B, C)$ and $\psi$ opposite to it by the equation:
\begin{equation}
\phi(f)_{\sigma}(a\otimes b) =[f_{\sigma}(a)_{\sigma}](b) \quad a\in A(\sigma), b\in B(\sigma)\tag{7}\label{eq:proof-7}
\end{equation}
\begin{equation}
[\phi (F)_{\sigma}(a)]_{\tau}(b) = f_{\tau}([e_{A, \sigma}^{\tau}(a)\otimes b) \quad a\in A(\sigma), b \in B(\tau)\tag{8}\label{eq:proof-8}
\end{equation}
These are the two maps, inverse to one another, which establish (\ref{art6-thm2.4-enum-(3)}).

The last statements is obtained by applying the third to the left hand side and observing that $R_{X}\otimes_{R}A\simeq A$.
\end{proof}

\section{Derived Functors}

In this section we introduce the most elementary derived functors on $C ar_{R}(X)$. These include the derived functors of both the module valued and the carapace valued tensor and homomorphism functors and the relation between the two. First recall that by \ref{art6-lemma-1.4}, whenever $P$ is a projective $R$-module and $\sigma$ is a simplex in $X$ then $P\uparrow_{\sigma}$ is a projective $R$-carapace. Consequently a coproduct of carapaces of the form $F \uparrow_{\sigma}$, where $F$ is a free $R$-module, is a projective $R$-carapace. We will refer to such carapaces as elementary projectives. Furthermore notice that if $M$ is any $R$-module then $\sigma(X, M\uparrow_{\tau})=M$. Hence if $Q$ is an elementary projective, $\sigma(X, Q)$ is a direct sum of free modules and hence free.

\begin{lemma}\label{art6-lemma-3.1}
Let $X$ be any simplicial complex.
    \begin{enumerate}[(1)]
    \item Every $R$-carapace on $X$ is a surjective image of an elementary projective.\label{art6-lemma3.1-enum-1}
    \item If $P$ is a projective $R$-carapace on $X$, then $\sigma(X, P)$ is $R$-projective.\label{art6-lemma3.1-enum-2}
    \item A tensor product of elementary projectives is an elementary projective.\label{art6-lemma3.1-enum-3}
    \item A tensor product of projectives is projective.\label{art6-lemma3.1-enum-4}
    \end{enumerate}
\end{lemma}

\begin{proof}
Let $A$ be an $R$-carapace on $X$. For each $\sigma \in X$ let $F_{\sigma}$ be a free $R$-module and let $q_{\sigma} : F_{\sigma} \rightarrow A(\sigma)$ be surjective morphism. Then, just as in \ref{art6-prop-1.5}, $\coprod_{\sigma \in X}F_{\sigma} \uparrow_{\sigma}$ is an elementary projective and $\coprod_{\sigma \in X}Q_{\sigma}$ maps in onto $A$. Thus \ref{art6-lemma3.1-enum-1}) is established.

To prove \ref{art6-lemma3.1-enum-2}), choose an elementary projective, $F$, and a surjective map, $q: F\rightarrow P$. Since $P$ is projective, it is a direct summand of $F$ and so $\sigma(X, P)$ is direct summand of $\sigma(X, F)$ which is free. This establishes \ref{art6-lemma3.1-enum-2}.

The fourth statement follows from the third because, by \ref{art6-lemma3.1-enum-1}, every projective is a direct summand of an elementary projective. Hence we must prove \ref{art6-lemma3.1-enum-3}). But this reduces to proving that is $\sigma$ and $\tau$ are two simplices, then $R\uparrow_{\sigma} \otimes_{R}R \uparrow_{\tau}$ is elementary projective. But $R\uparrow_{\sigma}\otimes_{R}R\uparrow_{\tau}(\gamma) \neq 0$ if and only if $\gamma \supseteq \sigma$ and $\gamma \supseteq \tau$. Thus $R\uparrow \otimes_{R}R \uparrow_{\tau}\neq (0)$ if and only if $\sigma \cup \tau = \alpha$ is a simplex and then $R\uparrow_{\sigma}\otimes_{R}R \uparrow_{\alpha}$. This establishes the result. 
\end{proof}

There are at least four very obvious homological bifunctors on\break $C ar_{R}(X)$

\begin{definition}\label{art6-definition-3.2}
Let $A$ and $B$ be $R$-carapaces on $X$. Write $\Ext_{R, X}^{r}$ for the $r$'th right derived functor of the left exact module valued bifunctor, $\Hom_{R, X}(a, B)$. Write $\calE xt_{R, X}^{q}$ for the carapace valued $q$'th right derived functor of the carapace valued local homomorhism functor. Write $\calT or_{q}^{R,X}(A, B)$ for the $q$'the carapace valued left derived functor of the carapace valued tensor product, $A\otimes_{R}B$ and write $\Tor_{q}^{R, X}(A, B)$ for the $q$'th left derived functor of the right exact bifunctor $\sigma(X, A\otimes_{R}B)$. 
\end{definition}

\begin{lemma}\label{art6-lemma-3.3}
If $P$ is a projective $R$-carapace on $X$, then for any $R$-carapace, $A$, $P\otimes_{R} A$ is $\sigma$-acyclic and $\calH om_{R, X}(P, A)$ is $\Gamma$-acyclic.
\end{lemma}

\begin{proof}
Let $\{Q_{j}\}$, $j \leq 0$ be  a projective resolution of $A$. Then $\{P \otimes_{R} Q_{j}\}$ is a projective resolution of $P \otimes_{R} A$. Apply the functor $\sigma$ to obtain $\Tor_{j}^{R,X}(A, P) = H_{j}(A\otimes_{R}P)$. But $\Tor_{j}^{R, X}(A, P) = (0)$ because $P$ is projective. This establishes the $\sigma$-acyclicity of $A\otimes_{R}P$. For the other acyclicity, let $\{K_{j}\}$ be a projective resolution of $R_{X}$. Then\break $\Hom_{R, X}(K_{j}, \calH om_{R, X}(P, A)) = \Hom_{R, x}(K_{j}\otimes_{R}P, A)$. But $K_{j} \otimes P$ is a projective resolution of $P$. Thus $H^{j}(X, \calH om_{R,X}(P, A)) = \Ext_{R, X}^{j}(P, A)$ which is $(0)$ because $P$ is projective.
\end{proof}

The elementary properties of these four functors are described in the following.

\begin{prop}\label{art6-lemma-3.4}
Let $A$ and $B$ be $R$-carapaces on $X$, Then
    \begin{enumerate}[(1)]
    \item $\calT or_{q}^{R,X}(A, B)(\sigma) = Tor_{q}^{R}(A(\sigma), B(\sigma))$\label{art6-lemma3.4-enum-1}
    \item $\Tor_{q}^{R, X}(R_{X}, A) = H_{q}(X, A)$\label{art6-lemma3.4-enum-2}
    \item $\Ext_{R,X}^{q}(R_{x}, A) = H^{q}(X,A)$\label{art6-lemma3.4-enum-3}
    \item There is a spectral sequence with $E_{2}^{p,q}$ term:\label{art6-lemma3.4-enum-4}
    $$
    E_{2}^{p,q} = H^{p}(X, \calE xt_{R,X}^{q}(A, B))
    $$
    and whose abutment is:
    $$
    \Ext_{X,R}^{p+q}(A, B)
    $$
    \item There is a spectral sequence with $E_{p,q}^{2}$ term:\label{art6-lemma3.4-enum-5}
    $$
    E_{p,q}^{2} = H_{p}(X, \calT or_{q}^{R, X}(A, B))
    $$
    and with abutment:
    $$
    \Tor_{p+q}^{R,X}(A, B)
    $$
    \end{enumerate}
\end{prop}

\begin{proof}
To prove \ref{art6-lemma3.4-enum-1}), let ${P_{j}}$ be a projective resolution of $A$. Then for each $\sigma$, $\{P_{j}(\sigma)\}$ is a projective resolution of $A(\sigma)$. Moreover the segment of $P_{j}\otimes_{R}B$ along $\sigma$ is $P_{j}\otimes_{R}b(\sigma)$. But the segment along $\sigma$ is an exact functor on $C ar_{R}(X)$ and so the $\sigma$-segment of the homology of the complex, $P_{j}\otimes_{R}B$ is the homology of the complex $P_{j}(\sigma) \otimes_{R}B(\sigma)$. This is just the result desired.

Statements \ref{art6-lemma3.4-enum-2} and \ref{art6-lemma3.4-enum-3} are both essentially trivial. Just note that\break $\Tor_{q}^{R, X}(R_{X}, A)$ (respectively $\Ext_{R,X}^{q}(R_{X}, A)$) is a connected sequence of homological functors acyclic on projectives (respectively injectives) and that $\Tor_{0}^{R,X}(R_{X},A)=\Sigma(X, A)$ (respectively $\Ext_{R,X}^{0}(R_{X},A) = \Gamma(X,A))$. The two statement follow.

The local global spectral sequences in \ref{art6-lemma3.4-enum-4}) and \ref{art6-lemma3.4-enum-5}) ar just composition of two functor sequences as in \cite{art6-keyGr}. Let $F_{A}(B) = A\otimes_{R}B$ and let $G_{A}(B) = \calH om_{R,X}(B,A)$. By Lemma \ref{art6-lemma-3.3}, $F_{A}$ carries projectives to $\Sigma$-acyclics and $G_{A}$ carries projectives to $\Gamma$-acyclics. The left derived functors of $F_{A}$ are the functors, $\calT or_{q}^{R,X}(A,-)$ while the right derived functors of$G_{A}$ are the functors, $\calH om_{R,X}^{q}(A, -)$. The construction of the spectral sequences is standard.
\end{proof}

\section{Homological Dimension}\label{art6-sec-4}

In this section we will determine the homological dimension of $C ar_{R}(X)$ for a finite dimensional simplicial complex, X. Write $X^{r}$ for the set of simplices in $X$ of dimension $r$ and write $X_{n}$ for $\bigcup_{r\geq n}X^{(r)}$. If $M$ is an $R$-module,write $pd(M)$ for the projective dimension of $M$ and write $hd(R)$ for the homological dimension of $R$. 

\begin{defin}\label{art6-defin-1}
Let $A$ be an $R$-carapace on $X$.
    \begin{enumerate}[(1)]
    \item Write $s(A)\leq n$ if $A(\sigma) = (0)$ whenever $dim(\sigma) < n$ and let $s(A) = inf\{n: s(A) \leq n\}$. Then $s(A)$ is called the \textit{support dimension} of $A$.\label{art6-defin1-enum-1}
    \item We say that $A$ is \textit{locally bounded} if $pd(A(\sigma)) \leq q$ for some fixed $q \geq 0$. In that case let $ld(A)=sup\{pd(A(\sigma)) : \sigma \in X\}$. When it exists, $ld(A)$ is called the \textit{local projective dimension} of $A$.\label{art6-defin1-enum-2}
    \end{enumerate}
\end{defin}

\setcounter{propo}{1}
\begin{propo}
Suppose that $A$ is an $R$-carapace on $X$ of support dimension at least $n$ and local projective dimension $r$. Then  there is an exact sequence:
\setcounter{equation}{2}
\begin{equation}\label{art6-equation-4.3}
0 \rightarrow A^{1} \rightarrow P_{r}\rightarrow\ldots \rightarrow P_{0} \rightarrow A \rightarrow 0
\end{equation}
so that:
    \begin{enumerate}[(1)]
    \item $ld(A^{1}) = 0$\label{art6-equation-enum-1}
    \item $P_{i}$ is projective\label{art6-equation-enum-2}
    \item $s(A^{1})\geq n+1$\label{art6-equation-enum-3}
    \item $s(P_{i}) \geq n$\label{art6-equation-enum-4}
    \end{enumerate}
\end{propo}

\begin{proof}
For each $\sigma \in X_{n}$ choose a projective module, $Q^{\sigma}$ and a surjective morphism, $\phi^{\sigma} : Q^{\sigma} \rightarrow A(\sigma)\rightarrow 0$. For each $\sigma$, let $\phi_{\sigma} : Q^{\sigma}\uparrow_{\sigma} \rightarrow A$ be the morphism of carapace induced by $\phi^{\sigma}$. Let $Q_{0}= \coprod_{\sigma\in X_{n}}(Q^{\sigma})\uparrow_{\sigma}$ and let $d_{0}=\coprod_{\sigma \in X_{n}}\phi_{\sigma}$. Since $s(A)\leq n$, $d_{0}$ in surjective. Let $N_{0} =ker(d_{0})$. Then, clearly $s(N_{0}) \leq n$ but $ld(N_{0})\leq r-1$. Thus we may repeat the process the process with $A$ replaced by $N_{0}$ and continue inductively until we obtain an exact sequence:
$$
0 \rightarrow N_{r-1} \rightarrow Q_{r-1} \rightarrow\ldots \rightarrow Q_{0} \rightarrow A \rightarrow 0.
$$
In this sequence, the $Q_{i}$ are projective, the support dimensions of the $Q_{i}$ and of $N_{r-1}$ are at least $n$ and $ld(N_{r-1})$ are at least $n$ and $ld(N_{r-1})=0$. That is $N_{r-1}(\sigma)$ is projective for each $\sigma$. Now let $Q_{r}=\coprod_{\sigma \in X_{n}}(N_{r-1}(\sigma))\uparrow_{\sigma}$. Clearly, $Q_{r}$ maps onto $N_{r-1}$ and the map is an isomorphism on segments over simplices of dimension $n$. Let $d_{r}$ be the composition of the map onto $N_{r-1}$ with the inclusion into $Q_{r-1}$ and let $A^{1} =ker(d_{r})$. Clearly, $A^{1}$ and the $Q_{i}$ answer the requirements of the proposition.
\end{proof}

\setcounter{prop}{3}
\begin{theorem}\label{art6-thm-4.4}
Let $X$ be a simplicial complex of dimension $d$ and let $A$ be a locally bounded $R$-carapace on $X$ of local projective dimension $r$. Then $pd(A)\leq d+r$.
\end{theorem}

\begin{proof}
 Apply Proposition 4.2 with $n=0$. The result is the exact sequence:
 $$
 0 \rightarrow A^{1} \rightarrow P_{r}\rightarrow \ldots \rightarrow P_{0}\rightarrow A \rightarrow 0
 $$
 Then apply 4.2 to $A^{1}$ observing that $ld(A^{1})= 0$ and $s(A^{1})\leq 1$. The result is a short exact sequence, $0 \rightarrow A^{2} \rightarrow Q_{1} \rightarrow A^{1} \rightarrow 0$ where $Q_{1}$ is projective, $ld(A^{2})=0$ and $s(A^{2})\leq 2$. We may continue until we reach $s(A^{d})\leq d$. But for any $B$, if the segments of $B$ are projective and $s(B)\leq dim(X)$ then $B$ is projective. We may thus assemble these short sequences and the sequence of $P_{i}$ to obtain a sequence:
 $$
 0\rightarrow A^{d} \rightarrow Q_{d-1} \rightarrow \ldots \rightarrow Q_{1} \rightarrow P_{r} \rightarrow \ldots \rightarrow
 P_{0}\rightarrow A \rightarrow 0
 $$
 This exact sequence is the projective resolution establishing the result.
 \end{proof}

\begin{coro}\label{art6-coro-4.5}
If $dim(X) =d$ and if $M$ is an $R$-module of projective dimension $r$, then $pd(M_{X})\leq r+d$. If $M$ is projective then $pd(M_{X}) \leq d$. 
\end{coro}

\begin{coro}\label{art6-coro-4.6}
If $R$ is of homological dimension $r$ and $dim(X)= d$ then the homological dimension of $C ar_{R}(X)$ is at most $d+r$.
\end{coro}

Neither of these corollaries requires so much as one word of proof.

\section{Carapaces and Morphisms of Complexes}\label{art6-sec-5}
Recall that a morphism of complexes from $X$ to $Y$ is just a covariant functor from the category of simplices in $X$ to the category of simplices in $Y$; it is simplicial if it carries vertices to vertices. If $S$ is a subset of the vertex set of $X$ then it admits a simplicial complex structure by taking as its set of simplices the set of simplices in $X$ all fo whose vertices lie in $S$. We will write $\tilde{S}$ fof this complex. When we speak of a subcategory of $X$ we will always, unless otherwise indicated, mean a full subcategory of the simplex category of $X$. If $U$ and $V$ are subcategories if $X$  we will write $U \subset V$ to indicate that $U$ is a full subcategory of $V$. In this case there is always a functorial map from $\Sigma(U, A)$ to $\Sigma(V, A)$ for any carapace, $A$.

Suppose that $f$ is a morphism of complexes from $X$ to $Y$ and that $Z$ is a simplicial sub-complex of $Y$. Let $f^{-1}(Z)$ denote the simplicial sub-complex of $X$ which has as its set of simplices the set $\{\sigma \in X : f(\sigma)\in Z \}$. Clearly, $f^{-1}$ is a functor from the subcomplexes in $Y$ to those in $X$.

\begin{definition}\label{art6-definition-5.1}
Let $X$ and $Y$ be simplicial complexes, let $A$ be a  $R$-carapace on $X$ and let $B$ be one on $Y$. Let $F:X \rightarrow Y$ be a morphism of complexes.

    \begin{enumerate}[(1)]
    \item Let $(f^{*}(B))(\sigma)=B(f(\sigma))$ and let $ e_{f^{*}(B), \sigma}^{\tau} = e_{B, f(\sigma)}^{f(\tau)}$. Then $f^{*}(B)$ is an $R$-carapace on $X$ and $f^{*}$ is a covariant functor from $C ar_{R}(Y)$ to $C ar_{R}(X)$. The carapace $f^{*}(B)$ is called the \textit{inverse image} of $B$ under $f$. \label{art6-definition5.1-enum-(1)}
    
    \item Let $(f_{*}(A))(\sigma) = \Sigma(f^{-1}(\tilde{\sigma}), A)$ and let $e_{f_{*}(A), \sigma}^{\tau}$ be the natural map of segments induces by the inclusion of categories, $f^{-1}(\tilde{\sigma})\subset f^{-1}(\tilde\tau)$ when $\sigma \subset \tau$. Then $f_{*}$ is a covariant functor from carapaces on $X$ to carapaces on $Y$. The carapaces, $f_{*}(A)$ is called the \textit{direct image} of $A$ by $f$.\label{art6-definition5.1-enum-(2)}
    \end{enumerate}
    
    Both $f^{*}$ and $f_{*}$ are additive. In addition they satisfy the adjointness properties expected.
\end{definition}

\begin{theorem}\label{art6-thm-5.2}
Let $X$ and $Y$ be simplicial complexes, let $f$ be a morphism of complexes from $X$ to $Y$, let $A$ be an $R$-carapace on $X$ and let $B$ be one on $Y$.
    \begin{enumerate}[(1)]
    \item $f^{*}$ is exact.\label{art6-thm5.2-enum-1}
    \item $f_{*}$ is right exact.\label{art6-thm5.2-enum-2}
    \item $f_{*}$ is left adjoint to $f^{*}$. That is,
    $$
    \Hom_{R, A}(A, f^{*}(B))\simeq \Hom_{R,Y}(f_{*}(A), B)
    $$
    functorially in $A$ and $B$.\label{art6-thm5.2-enum-3}
    \end{enumerate}
\end{theorem}

\begin{proof}
The first statement is a triviality. The second statement in nothing more than the right exactness of co-limits. Thus only the last statement requires attention.

To prove \ref{art6-thm5.2-enum-3}), we will give morphisms,
$$
\psi : \Hom_{R,X}(A, f^{*}(B))\rightarrow \Hom_{R,Y}(f_{*}(A), B)
$$
and $\phi$ inverse to it. Begin with $\psi$. If $\alpha \in  \Hom_{R,X}(a, f^{*}(B))$, write $\alpha =\{\alpha_{\sigma}\}_{\sigma \in X}$. Then $\alpha_{\sigma}$ maps $A(\sigma)$ to $B(f(\sigma))$ for each $\sigma$ compatibly with respect to $\sigma$. Then  $\sigma \in f^{-1}(\tilde{\rho})$ if and only if $f(\sigma)\subseteq \rho$. Thus the set of maps, $e_{B, f(\sigma)}^{\rho} \circ \alpha_{\sigma}$ is direct system of maps giving a morphism from $[f_{*}(A)](\rho)= \Sigma(f^{-1}(\tilde{\rho}), A)$ to $B(\rho)$. For each $\rho$ call this map $\beta_{\rho}$. Then since $\beta_{\rho}$ is functorial in $\rho$, the family $\{\beta_{\rho}\}_{\rho \in Y}$ is a morphism, $\beta$, from $f_{*}(A)$ to $B$. Let $\psi (\alpha) = \beta$.

Now we wish to define $\phi$ . If $\beta \in \Hom_{R,Y}(f_{*}(A), B)$ then $\beta$ is a family $\{B_{\rho}\}_{\rho in Y}$ where $\beta_{\rho}$ maps $\Sigma(f^{-1}(\tilde{\rho}), A)$ to $B(\rho)$. For any $\sigma$ in $X$, let $\rho = f(\sigma)$ and let $a_{\sigma}$ be the natural map from $A(\sigma)$ to $\Sigma(f^{-1}(\tilde{\rho}), A)$. Then $\beta_{\rho} \circ a_{\sigma}$ is a map from $A(\sigma)$ to $B(f(\sigma))$ for each $\sigma$. Let $\alpha_{\sigma} = \beta_{f(\sigma)}$ for each $\sigma$ and let $\phi(\beta)$ be the map $\alpha = \{\alpha_{\sigma}\}_{\sigma \in X}$. We leave the task of verifying that $\psi$ and $\phi$ are maps of the requisite type and that they are inverse to one another to the reader.
\end{proof}

It is entirely expected the $f^{*}$ has a left adjoint. It is a bit surprising, though not at all subtle, that is also has a right adjoint. Let $f : X \rightarrow Y$ be a morphism of complexes. For $\rho \in Y$ let $f^\dagger(\rho)$ denote the sub-category of $X$ consisting of all $\sigma \in X$ such that $f(\sigma) \supseteq \rho$.

\begin{definition}\label{art6-definition-5.6}
Let $X$ and $Y$ be simplicial complexes, let $f : X\rightarrow Y$ be a morphism of complexes and let $A$ be an $R$-carapace on $X$. Define an $R$-carapace on $Y$ by the equation:
$$
f_{\dagger}(A)(\rho) = \Gamma(f^{\dagger}(\rho), A)
$$
This is clearly an $R$-module valued functor on the simplex category of $Y$ and so it is an $R$-carapace on $Y$. We will call it the \textit{right direct image} of $A$ under $f$.
\end{definition} 

\begin{prop}\label{art6-prop-5.4}
Let $f: X\rightarrow Y$ be a morphism of complexes, let $A$ be an $R$-carapace on $X$ and let $B$ be one on $Y$. Then $f_{\dagger}$ is left exact and right adjoint to $f^{*}$. That is,
$$
\Hom_{R,X}(f^{*}(B), A)\simeq \Hom_{R,Y}(B, f_{\dagger}(A))
$$
functorially in $A$ and $B$.
\end{prop}

\begin{proof}
Left exactness follows from the left exactness of $\Gamma$ and so we only need to establish the adjointness. We give the two morphisms. Let
$$
\mu : \Hom_{R, Y}(B, f_{\dagger}(A)) \rightarrow \Hom_{R,X}(f^{*}(B), A)
$$
be one of the two morphisms and let $\zeta$ be its inverse.

Choose $\delta$ in $Hom_{R,Y}(B, f_{\dagger}(A))$. For each $\rho \in Y$, $\delta$ takes each element, $b \in B(\rho)$ to a compatible family, $\{[\delta_{\rho}(b)]\sigma\}_{f(\sigma)\supseteq \rho}$ where $[\delta_{\rho}(b)]_{\sigma} \in A(\sigma)$. For each $\sigma$ we must give a map $\eta (\delta)_{\sigma}: B(f(\sigma)) \rightarrow A(\sigma)$. Let
$$
\left[\eta (\delta)_{\sigma}\right](b) = \left[\delta_{f(\sigma)}(b)\right]_{\sigma}
$$
This defines $\eta$.

To define $\zeta$, choose $b \in B(\rho)$ adn suppose that $beta \in \Hom_{R,X}(f^{*}(B), A)$. If $f(\sigma)\supseteq \rho$ let $a_{\sigma} = \beta_{\sigma}(e_{B, \rho}^{f(\sigma)}(b))$. Then let
$$
\zeta(\beta)_{\rho}(b) = \{a_{\sigma}\}_{f(\sigma)\supseteq \rho}
$$
We leave the verifications involved to the reader.
\end{proof}

\begin{coro}\label{art6-corollary-5.5}
Let $X, Y$ and $f$ be as above. Then:
    \begin{enumerate}[(1)]
    \item $f_{*}$ carries projectives on $X$ to projectives on $Y$.\label{art6-corollary5.5-enum-1}
    \item For any $R$-carapace $A$ on $X$, $\Sigma(Y, f_{*}(A)) = \Sigma(X, A)$.\label{art6-corollary5.5-enum-2}
    \item $f_{\dagger}$ carries injectives on $X$ to injectives on $Y$.\label{art6-corollary5.5-enum-3}
    \item For any $A$ on $X$, $\Gamma(Y, f_{\dagger}(A))= \Gamma(x, A)$\label{art6-corollary5.5-enum-4}
    \end{enumerate}
\end{coro}

\begin{proof}
For the first statement, let $P$ be a projective on $X$ and let $M \rightarrow N \rightarrow 0$ be a surjective map in $C ar_{R}(X)$. Consider the map, $\Hom_{R, Y}$ $(f_{*}(P), M) \rightarrow \Hom_{R,Y}(f_{*}(P), N)$. By the adjointness statement in \ref{art6-thm-5.2}, \ref{art6-thm5.2-enum-3}), this is the same as the map $\Hom_{R, X}(P, f^{*}(M))\rightarrow \Hom_{R,X}$ $(P, f^{*}(N))$. But now $f^{*}$ is exact and $P$ is projective on $X$ and so this map is surjective. This takes care of \ref{art6-thm5.2-enum-1}).

In general, if $M$ and $N$ are $R$-modules and there is an isomorphism $Hom_{R}(M,Q) \simeq Hom_{R}(N,Q)$ functorial in $Q$, then $M \simeq N$. Apply this to \ref{art6-corollary5.5-enum-2}) using the definition of the functor $\Sigma(X, ?)$ and
\ref{art6-thm5.2-enum-3}) to obtain:
\begin{align*}
Hom_{R}(\Sigma(Y, f_{*}(A)), M) &= \Hom_{R,Y}(f_{*}(A), M_{Y})\\
= \Hom_{R,X}(A, M_{X}) &= Hom_{R}(\Sigma(X, A), M)
\end{align*}
Statement \ref{art6-corollary5.5-enum-2}) follows.

The proof of \ref{art6-thm5.2-enum-3}) is precisely dual to the proof of \ref{art6-thm5.2-enum-1}). To establish
\ref{art6-corollary5.5-enum-4}), apply \ref{art6-prop-5.4} and \ref{art6-thm-2.4}, \label{art6-thm2.4-enum-(2)}). Write:
$$
\Gamma (Y, f_{\dagger}(A)) = \Hom_{R,Y}(R_{Y}, f_{\dagger}(A)) = \Hom_{R,X}(R_{X}, A) = \Gamma(X,A)
$$
Thus \ref{art6-corollary5.5-enum-4}) is also proven.
\end{proof}

Corollary \ref{art6-corollary-5.5} establishes exactly what is necessary for two composition of functor spectral sequences. Many are possible but we content ourselves with the two most obvious.


\begin{prop}\label{art6-prop-5.6}
Let $X$ and $Y$ be simplicial complexes, let $A$ be an $R$-carapace on $X$ and let $f:X\rightarrow Y$ be a morphism of complexes.
\begin{enumerate}[(1)]
\item There is a spectral sequence with $E_{p,q}^{2}$ term:
    $$
    E_{p,q}^{2} = H_{p}(Y, L_{q}fA)
    $$
    and abutment:
    $$
    H_{r}(X, A)
    $$
    \item There is a spectral sequence with $E_{2}^{p,q}$ term:
        $$
        E_{2}^{p, q} =H^{p}(Y, R^{q} f_{\dagger}A)
        $$
        and abutment:
        $$
        H^{r}(X, A)
        $$
\end{enumerate}

These spectral sequences are sufficiently standard that no proof is required. The proofs in \cite{art6-keyGr}, for example, apply.
\end{prop}

\section{Certain Special Carapaces}\label{art6-sec-6}
This section will be devoted to the study of certain acyclic carapaces. We will need certain conventions. If $X$ is a simplicial, a complement in $X$ is a full subcategory of its simplex category such that the complement of its collection of simplices is a simplicial complex.  The reader may verify that $C$ is a complement in $X$ if, whenever $\sigma \in C$ and $\tau \supseteq \sigma$ then $\tau \in C$. Alternatively $C$ is a complement in $X$ if and only if whenever $\sigma \in C$, then $X (\sigma) \subseteq C$. These two conditions apply to arbitrary subcollections of the simplex set of $X$ and we will use the term complement in this sense. Clearly arbitrary unions and intersetions of complements are complements.

\begin{definition}\label{art6-definition-6.1}
Let $X$ be a simplicial complex.
    \begin{enumerate}[(1)]
    \item An $R$-carapace, B, is called \textit{brittle} if for every sub-complex of $X$, $Z$, the natural map, $\Sigma(Z, B) \rightarrow \Sigma(X, B)$ is injective.\label{art6-definition6.1-1}
    \item An $R$-carapace, $F$, is called \textit{flabby} if for every complement in $X$, $C$, the natural map $\Gamma(X, F) \rightarrow \Gamma (C, F)$ in surjective.\label{art6-definition6.1-2}
    \end{enumerate}
\end{definition}

Our development follows standard treatments of flabbyness for\break sheaves. On occasion something more is called for in the brittleness arguments. Flabbyness will be an entirely familiar concept, but brittleness might be a bit strange. We will begin with some descriptive comments. First notice that if $dim(X)>0$ then $R_{X}$ is not brittle. Suppose that $\sigma$ is positive dimensional simplex in $X$ and that $x$ and $y$ are distinct vertices in it. Let $Z=\{x, y\}$. That is, $Z$ is the disconnected two point complex. Then clearly, $\Sigma(X, R_{X})= R\oplus R$ and, since $\sigma \in X$ and $Z \subseteq \sigma$, the map, $\Sigma(Z, R_{X})\rightarrow \Sigma(X, R_{X})$ is not injective since it factors through $R_{X}(\sigma) = R$. 

If $\sigma \in X$ and $B$ is brittle then by definition, $B(\sigma) \subseteq \Sigma(X, B)$. But brittleness also forces the relation, $B(\sigma)\cap B(\tau) = B(\sigma \cap \tau)$ where the intersection is taken in $\Sigma(X, B)$. To see this just note that, because $\Sigma(Z, A)$ is nothing  but the inductive limit over $Z$, there is an exact sequence,
$$
0 \rightarrow B(\sigma \cap \tau) \rightarrow B(\sigma) \coprod B(\tau) \rightarrow \Sigma(\sigma \cup \tau, B)\rightarrow 0
$$
and, by brittleness, an inclusion $\Sigma(\sigma \cup \tau, B) \subseteq \Sigma(X, B)$.

Before proceeding a convention is necessary. If $\sigma$ is a simplex in $X$ then write $\hat{\sigma}$ for the complex whose vertex set is $\sigma$ but whose simplex set is the set of all proper subsets of $\sigma$. That is $\sigma$ is not a simplex in $\hat{\sigma}$ which is a simplicial sphere. Then $\hat{\sigma} \subset \tilde{\sigma}$.

We will also require the following. Let $f: M \rightarrow N$ be a morphism of $R$-modules. Then $f$ is injective if and only if, for each injective $R$-module, $J$, the induced map $Hom_{R}(N, J)\rightarrow  Hom_{R}(M, J)$ is surjective.

Finally suppose that $Z$ is a simplicial sub-complex of $X$. Let $C$ be the set the simplices of $X$ which are not siplices of $Z$. For any $R$-module, $M$, define $R$-carapaces $M_{Z}^{*}$ and $M_{*}^{C}$ by the equation:

\setcounter{equation}{1}
\begin{equation}
\begin{aligned}\label{art6-eq-6.2}
M_{z}^{*}(\sigma) &= M \quad \text{if} \quad \sigma \in Z \\
M_{z}^{*}(\sigma) &= (0)\quad \text{if} \quad \sigma \notin Z
\end{aligned}
\end{equation}

Then $M_{*}^{C}$ is defined by exactly the same equations, replacing $M_{Z}^{*}$ by $M_{*}^{Z}$ and $Z$ by $C$. As $Z$ is a complex $M_{Z}^{*}$ in naturally a quotient of $M_{X}$ while $M_{*}^{C}$ is naturally a subobject. In fact, the following is exact:
$$
0 \rightarrow M_{*}^{C}\rightarrow M_{X}\rightarrow M_{Z}^{*} \rightarrow 0
$$

In addition, the following hold
\begin{equation}
\begin{aligned}\label{art6-eq-6.3}
\Hom_{R, X}(A, M_{Z}^{*}) &= Hom_{R}(\Sigma(Z, A), M)\\
\Hom_{R, X}(M_{*}^{C}, A) &= Hom_{R}(M, \Gamma (C, A))
\end{aligned}
\end{equation}


\begin{lem}\label{art6-lemma-6.4}
Let $X$ be a simplicial complex, let $Z \subseteq X$ be a subcomplex of $X$ and let $C$ be a complement in $X$. Then if $A$ is brittle on $X$, $A|_{Z}$ is brittle on $Z$. If $A$ is flabby on $X$, then $A|_{C}$ is flabby on $C$.
\end{lem}

\begin{proof}
If $A$ is brittle and $Z'$ is subcomplex of $Z$ then the composition, $\Sigma(Z', A)\rightarrow \Sigma(Z, A) \rightarrow \Sigma(X, A)$ is the map, $\Sigma(Z', A) \rightarrow \Sigma(X, A)$. If a composition in injective, each map in it injective. This proves the first statement. The proof of the second statement i precisely dual to it and so we leave in to the reader.
\end{proof}

\setcounter{prop}{4}
\begin{theorem}\label{art6-thm-6.5}
Let $X$ be a simplicial complex and let
$$
0 \rightarrow A' \rightarrow A \rightarrow A'' \rightarrow 0
$$
be exact.
\begin{enumerate}[(1)]
    \item If $A''$ is brittle, then
    $$
    0 \rightarrow \Sigma(x, A') \rightarrow \Sigma(X, A)\rightarrow\Sigma(X, A'') \rightarrow 0
    $$
    is exact.\label{art6-thm6.5-enmu-1}
    
    \item If $A'$ is flabby, then
    $$
    0\rightarrow \Gamma(X, A') \rightarrow \Gamma (X,A)\rightarrow \Gamma(X, A'')\rightarrow 0
    $$
    is exact.\label{art6-thm6.5-enum-2}
\end{enumerate}
\end{theorem}

\begin{proof}
To prove (\ref{art6-thm6.5-enmu-1}) we need only show that $\Sigma(X, A') \rightarrow \Sigma(X, A)$ is injective. By the observation above, it would suffice to show that\break $Hom_{R}(\Sigma(X, A), J) \rightarrow Hom_{R}(\Sigma(X, A'), J)$ is surjective for an injective, $J$. But $Hom_{R}(\Sigma(X, A), J) = \Hom_{R, X}(A, J_{X})$ and the same for $A'$. Thus, to establish (\ref{art6-thm6.5-enmu-1}), it suffices to prove that every carapace morphism, $f:A'\rightarrow J_{X}$ extends to a morphism. $\tilde{f}: A\rightarrow J_{X}$.

Let $j : A' \rightarrow A$ be the injection and let $\pi : A\rightarrow A''$ be the surjection. Let $f: A' \rightarrow J_{X}$ be a morphism of carapaces. Let $\calF$ be the family of paris, $(Z, f_{Z})$ where $Z$ is a subcomplex and $f_{Z}:A|_{Z} \rightarrow J_{Z}$ is a morphism such that $f_{Z}\circ J = f|_{Z}$. Order these by inclusion on $Z$ and extension on $f_{Z}$. This orders $\calF$ inductively and so Zorn's Lemma yields a maximal element, $(W, f_{W})$. If $W \neq X$ there is some $\sigma \in X$ such that $\sigma \notin W$. If $\sigma \cap W = \emptyset$ we may trivially extend $f_{W}$ to $W \cup \{t\}$ where $t$ is any vertex in $\sigma$. This contradicts maximality. Thus we may assume that $\sigma \cap W \neq \emptyset$. Let $\tilde{\sigma} \cap W =Y$. Consider $f_{\sigma} : A'(\sigma)\rightarrow J$. By the injectivity of $J$, we may choose $f_{\sigma}^{1}: A(\sigma) \rightarrow J$ such  that $f_{\sigma}^{1} \circ j_{\sigma} = f_{\sigma}$.  If $\gamma \subseteq \sigma$ let $f_{\gamma}^{1}=f_{\gamma}^{1} \circ e_{A, \gamma}^{\sigma}$. Since j is morphism, the following commutes:
$$
\xymatrix{ 
\ar@{}[dr]|{}
A'(\sigma) \ar[r]^-{j_{\sigma}} & A(\sigma)  \\
 A'(\gamma)\ar[u]^-{e_{A', \gamma}^{\sigma}}\ar[r]_-{j_{\gamma}} & A(\gamma)\ar[u]_-{e_{A, \gamma}^{\sigma} A}  }
$$
Hence $f_{\gamma}^{1} \circ j_{\gamma} = f_{\sigma}^{1}\circ e_{A, \gamma}^{\sigma}\circ j_{\gamma} = f_{\sigma}^{1}\circ e_{A', \gamma}^{\sigma} = f_{\gamma}$. That is, $f^{1} \circ j = f$ on $\sigma$. But on
$\tilde{\sigma} \cap W = Y$, $f_{W}\circ j = f$. Thus on $Y$, $(f_{W}-f^{1})\circ j =0$. It follows that $f_{W}-f^{1}$ induces a map form $A''|_{Y}$ to $J_{Y}$. But $\Sigma(Y, A'')\rightarrow (X, A'')$ is brittle. Hence $\Hom_{R, X}(A'', J_{x})\rightarrow \Hom_{R, Y}(A''|Y, J_{Y})$ is surjective. Thus, there is and $f_{2} \in \Hom_{R, X}(A'', J_{X})$ such that $\pi \circ f_{2}|_{Y} =  (f_{W}-f^{1})|_{Y}$. Consequently, $(f^{1}+ \pi \circ f_{2})|_{\tilde{\sigma}\cap W}=f_{W}|_{\tilde{\sigma}\cap W}$. Hence $f_{W}$ can be extended to $W\cup \tilde{\sigma}$ contradicting the maximality of $(W, f_{W})$. That is $W = X$ and so \ref{art6-thm6.5-enmu-1}) is established.

To prove \ref{art6-thm6.5-enum-2}) we must prove that $\Gamma(X, A)\rightarrow \Gamma(X, A'')$ is surjective. An element $a\in \Gamma (z, A)$ is a function on $Z$ such that $a(\sigma) \in A(\sigma)$ and $e_{A, \sigma}^{\tau}(a(\sigma)) = a(\tau)$. Suppose $a'' \in \Gamma (X, A'')$ is given. Order the pairs $(C, a_{C})$, where $C$ is a complement and $a_{C} \in \Gamma(C, A)$, $\pi(a_{c}) = a''|_{C}$, by inclusion and extension. This being an inductive order, there is a  maximal element, $(U, a_{U})$. If $U \neq X$, there is simplex, $\tau$ not in $U$. Choose $\tilde{a}_{r} \in A(\tau)$ such that $\pi_{\tau}(\tilde{a}_{r}) = a''(\tau)$. Define $\tilde{a}_{1}$ in $X(\tau)$ by $\tilde{a}_{1}(\sigma) = e_{A, \tau}^{\sigma}(\tilde(a)_{r})$. If $X(\tau) \cap U = \emptyset$ then $\tilde{a}_{1}$ extends $a_{U}$ contradicting maximality of $(U,a_{U})$, and so we may assume that $X(\tau)\cap U \neq \emptyset$. This intersection is a complement. Consider the difference $\tilde{a}_{1}-a_{U}$ on this intersection. Now $\pi(\tilde{a}_{1}-a_{U})= 0$ on $X(\tau) \cap U$ whence $^{}\tilde{(}a_{1}-a_{U})|X(\tau)\cap U \in \Gamma (X(\tau)\cap U, A')$ Since $A'$ is flabby there is an element $a' \in \Gamma (X, A')$ such that $a'|X(\tau) \cap U = (a_{1}-a_{U})|X(\tau)\cap U$. Clearly $a_{1}-(a'|X(\tau))$ extends $a_{U}$ contradicting maximality. Thus $U =X$ and we have established \ref{art6-thm6.5-enum-2})
\end{proof} 

\begin{coro}\label{art6-corollary-6.6}
Let
$$
0 \rightarrow A' \rightarrow A \rightarrow A'' \rightarrow 0
$$
be an exact sequence of $R$-carapaces on $X$.
    \begin{enumerate}[(1)]
    \item If $A$ and $A''$ are brittle, then $A'$ is also.\label{art6-corollary6.6-enum-1}
    \item If $A$ and $A'$ are flabby, then $A''$ is also.\label{art6-corollary6.6-enum-2}
    \end{enumerate} 
\end{coro}

\begin{proof}
We prove \ref{art6-corollary6.6-enum-1}). Suppose $Z$ is a subcomplex of $X$. Then, by \ref{art6-lemma-6.4}, $A|_{Z}$ and $A''|_{Z}$ are both brittle and hence, $0 \rightarrow \Sigma(Z, A')\rightarrow \Sigma(Z, A)$ is exact. Thus the following diagram, which has exact rows and columns, commutes:
 $$
 \xymatrix{
 & & 0\ar[d]\\
 0 \ar[r] & \Sigma(Z, A') \ar[r]\ar[d] & \Sigma(Z, A)\ar[d]\\
 0 \ar [r] & \Sigma(X, A')\ar[r] & \Sigma(X, A)
 } 
 $$                          
It is immediate the $\Sigma(Z, A')\rightarrow \Sigma(X, A')$ is monic. As for Statement \ref{art6-corollary6.6-enum-2}), noting that $Z$ must be replaces by a complement, the proof is both well known and strictly dual to the proof of \ref{art6-corollary6.6-enum-1}).
\end{proof}

\begin{prop}\label{art6-proposition-6.7}
Let $X$ and $Y$ be simplicial complexes let $f: X \rightarrow Y$ be a morphism of complexes and let $A$ be an $R$-carapace on $X$. Then
\begin{enumerate}[(1)]
\item If $A$ is brittle, then $f_{*}(A)$ is brittle.\label{art6-proposition6.7-enum-1}
\item If $A$ is flabby then $f_{\dagger}(A)$ is flabby. \label{art6-proposition6.7-enum-2}
\end{enumerate}
\end{prop}

\begin{proof}
To prove \ref{art6-proposition6.7-enum-1}), let $U\subseteq Y$ be a subcomplex. Then, $f^{-1}(U)$ is a subcomplex of $X$ and so, if $A$ is brittle, then $\Sigma(f^{-1}(U), A) \rightarrow \Sigma(X, A)$ is injective. But $\Sigma(f^{-1}(U), A)=\Sigma(U, f_{*}A)$ and $\Sigma{X, A} = \Sigma(Y, f_{*}A)$ by definition. That proves the first statement. The proof of \ref{art6-proposition6.7-enum-2}) is completely parallel except that it uses \ref{art6-corollary-5.5},
\ref{art6-corollary5.5-enum-4} in place of the corresponding properties of $f_{*}$
\end{proof}

\begin{prop}\label{art6-proposition-6.8}
Let $X$ be a simlicial complex.
\begin{enumerate}[(1)]
\item Projective carapaces are brittle; injective carapaces are flabby.\label{art6-proposition6.8-enum-1}
\item a coproduct of brittle carapaces is brittle; a product of flabby carapaces is flabby.\label{art6-proposition6.8-enum-2}
\item For any simplex, $\sigma \in X $ and any $R$-module, $M$, $M \uparrow_{\sigma}$ is brittle and $M\downarrow^{\sigma}$ is flabby.  \label{art6-proposition6.8-enum-3}
\end{enumerate}
\end{prop}

\begin{proof}
Let $Z$ be any subcomplex of $X$. let $C$ be its complement and let $M$ any $R$-module. Then $0 \rightarrow M_{*}^{C} \rightarrow M_{X} \rightarrow M_{Z}^{*} \rightarrow 0$ is exact. Thus, if $P$ is projective, $Hom_{R,X}(P, M_{X}) \rightarrow \Hom_{R, X}(P, M_{Z}^{*})$ is surjective, But this is the map, $Hom_{R}(\Sigma(P, X), M) \rightarrow Hom_{R}(\Sigma(P, Z), M)$. But this map will be surjective for every $M$ if and only if the map $\Sigma(Z, P) \rightarrow \Sigma(X, P)$ is injective (in fact, it must be split).

If $I$ in injective, we need only consider the case, $M =R$. Then $\Hom_{R, X}(R_{X}, I) \rightarrow \Hom_{R, X}(R_{*}^{C}, I)$ is surjective. This is the sequence,  $\Gamma (X, I) \rightarrow \Gamma(C, I)$ and hence
(\ref{art6-corollary6.8-enum-2}) is established.

To prove \ref{art6-corollary6.8-enum-2}), let $\{A_{i}\}_{i \in I}$ be a family of $R$-carapaces on $X$. Since inductive limits of arbitrary co-products are co-products and projective limits of products are products, we may write:

\setcounter{equation}{8}
\begin{equation}
\begin{aligned}\label{art6-eq-6.9}
\Sigma \left(Z, \coprod_{i \in I} A_{i}\right) &= \coprod_{i \in I} \Sigma (Z, A_{i})\\
\Gamma \left(C, \prod_{i \in I}A_{i}\right) &= \prod_{i \in I}\Gamma (C, A_{i})
\end{aligned}
\end{equation}
Since a co-product of monomorphisms is monic and a product of surjections is surjective, \ref{art6-corollary6.8-enum-2}) follows at once.

Statement \ref{art6-corollary6.8-enum-3}) is quite clear.
\end{proof}


\setcounter{prop}{9}
\begin{prop}\label{art6-proposition-6.10}
If $A$ a brittle $R$-carapace on $X$, then $H_{i}(X, A) = 0$ for all $i > 0$. If $F$ is flabby, then $H^{i}(X, F) = 0$ for all $i > 0$.
\end{prop}

\begin{proof}
First choose a projective, $P$ and a surjective map so that there is an exact sequence:
$$
0 \rightarrow A_{0} \rightarrow P \rightarrow A \rightarrow 0
$$

The acyclicity of $P$ the Theorem \ref{art6-thm-6.5} together imply that for any brittle $A$, $H_{1}(X, A) = 0$. Then choose a projective resolution of $A$. Break this into a series of short exact sequences, use Corollary
\ref{art6-corollary-6.6} and apply induction. The same technique, applied dually, gives the second statement.

We conclude with local criteria for which there no immediate applications but which are somewhat interesting.
\end{proof}

\begin{prop}\label{art6-proposition-6.11}
Let $A$ a be an $R$-carapace on $X$.
    \begin{enumerate}[(1)]
    \item If for each simplex $\sigma \in X$ the map, $\Sigma(\hat{\sigma}, A)\rightarrow \Sigma(\hat{\sigma}, A)$, is injective then $A$ is brittle.\label{art6-prop6.11-enum-1}
    \item If for each simplex $\sigma \in X$ the restriction $A|{X}(\sigma)$ is flabby, then $A$
     is flabby.\label{art6-prop6.11-enum-2}
    \end{enumerate}
\end{prop}

\begin{proof}
Proofs of these statements use Zorn's lemma as its was used in Theorem \ref{art6-thm-6.5} First we prove
\ref{art6-prop6.11-enum-1}). Let $Z$ be an arbitary subcomplex of $X$. We must show that $\Sigma(Z, A)\rightarrow \Sigma{X, A}$ is injective. As in the proof of theorem \ref{art6-thm-5}, this comes to proving that for any injective module, $J$, any morphism, $f: A|Z \rightarrow J_{Z}$, admits and extension, $f:A\rightarrow J_{X}$. Applying Zorn one finds a maximal subcomplex on which $f$ admits and extension and so, replacing $Z$ by this maximal subcomplex, we may assume that $f$ does not extend to any subcomplex contating $Z$. If the vertex $x$ is not in $Z$ then $f$ clearly extends  to the disconnected union and so we may assume that very vertex is in $Z$. Choose a simplex, $\sigma$ of minimal dimension among the simplices not in $Z$.  Then $\hat{\sigma} \subseteq Z$. Making use of the condition in
(\ref{art6-prop6.11-enum-1}), we obtain a diagram:
$$
\xymatrix{
0 \ar[r] &\Sigma(\hat{\sigma}, A)\ar [d] \ar[r]& \Sigma(\tilde{\sigma}, A)\\
         & \Sigma(Z, A)\ar[d]^{f_{Z}} & \\
         & J                           &
}
$$

Hence there is a map, $f_{i} : \Sigma(\tilde{\sigma}, A) \rightarrow J$ extending $f_{Z}$ and so one may extend $f$ to $Z\cup \sigma$ contradicting maximality. It follows that it must be that $Z=X$.

The proof of \ref{art6-prop6.11-enum-2}), by duality, in entirely straightforward and so we omit it.
\end{proof}

\section{Canonical Resolutions}\label{art6-sec-7}
In this section we give canonical chain and co-chain complexes which can be used to compute the exoskeletal homology and cohomology\break groups. They arise from canonical resolutions and they are sufficiently canonical that they will be seen to be equivariant when there is a group action involved.

Let $A$ be an $R$-carapace on $X$. Then by \ref{art6-lemma-1.4}, \ref{art6-enum-lemma1.4-(1)}) and \ref{art6-enum-lemma1.4-(2)}), the identity map on $A(\sigma)$ induces a map, $\pi_{\sigma} : (A(\sigma))\uparrow_{\sigma} \rightarrow A$ and a map $j_{\sigma} : A\rightarrow (A(\sigma))\downarrow^{\sigma}$.  

\begin{definition}\label{art6-definition-7.1}
Let $X$ be a simplicial complex and let $A$ be an $R$-carapace on $X$.
    \begin{enumerate}[(1)]
    \item Let
        $$
        \calT_{0}(A) = \coprod_{\sigma \in X}(A(\sigma))\uparrow_{\sigma} \quad \text{and let} \quad \pi_{A}= \coprod_{\sigma \in X}\pi_{\sigma} 
        $$\label{art6-definition7.1-enum-1}
        
    \item Let
        $$
        \calS^{0}(A) = \prod_{\sigma \in X}(A(\sigma))\downarrow^{\sigma} \quad \text{and let} \quad j_{A}= \prod_{\sigma \in X}j_{\sigma}.
        $$\label{art6-definition7.1-enum-2}

     \item Let $\calK_{0}(A) = Ker(\pi_{A})$.\label{art6-definition7.1-enum-3}
     \item Let $\calC^{0}(A) = Coker(j_{A})$.\label{art6-definition7.1-enum-4}  
    \end{enumerate}

    This definition has certain immediate consequences.
\end{definition}

\setcounter{prop}{1}
\begin{lemma}\label{art6-lemma-7.2}
 Let $X$ be a simplicial complex and let $A$ be an $R$- carapace on $X$.
\begin{enumerate}[(1)]
\item The four functors, $\calT_{0}$, $\calK_{0}$, $\calS^{0}$, and $\calC^{0}$ are exact additive functors.\label{art6-lemma7.2-enum-1}
\item Both $\pi_{A}$ and $j_{A}$ are natural transformations in the argument $A$. Further $\pi_{A}$ is always surjective and $j_{A}$ is always monic.\label{art6-lemma7.2-enum-2}

\item For all $A$, $\calT_{0}(A)$ is brittle and $\calS^{0}(A)$ is flabby.\label{art6-lemma7.2-enum-3}
\item If $A$ is brittle, then $\calK_{0}(A)$ is brittle; if $A$ is flabby, $\calC^{0}(A)$ is flabby.\label{art6-lemma7.2-enum-4}
\end{enumerate}
\end{lemma}

\begin{proof}
That $\calT_{0}$ and $\calS^{0}$ are exact and additive is a trivial observation. Since $\calT_{0}(A)$ is a coproduct of carapaces of the form $M\uparrow_{\sigma}$, proposition \ref{art6-proposition-6.8}, \ref{art6-lemma7.2-enum-2}) and \ref{art6-lemma7.2-enum-3}) guarantee that it is brittle. The flabbyness of $\calS^{0}(A)$ follows similarly from the fact that it is a product of carapaces of the form $M\downarrow^{\sigma}$. That $\calK_{0}$ and $\calC^{0}$ are exact is littel more thant the snake lemma. Statement \ref{art6-lemma7.2-enum-2}) is a triviality and so only
\ref{art6-lemma7.2-enum-4})  remains to be proven. This follows from \ref{art6-lemma7.2-enum-3}) and Corollary
\ref{art6-corollary-6.6}.  
\end{proof}

Definition \ref{art6-definition-7.1} and lemma \ref{art6-lemma-7.2} are just what is necessary to construct standard resolutions.

\begin{definition}\label{art6-definition-7.3}
Let $A$ be an $R$-carapace on $X$. Let $\calK_{n}(A) = \calK_{0}(\calK_{n-1}(A))$ and let $\calC^{n}(A) = \calC^{0}(\calC^{n-1}(A))$. That is $\calK_{n}$ is the $(n+1)'$st iterate of $\calK_{0}$ and the same, mutatis mutandis, is ture for $\calC^{n}$. Let $\calT_{n+1}(A) = \calT_{0}(K_{n}(A))$ and let $\calS^{n+1}(A) = S^{0}(\calC^{n}(A))$ for $n\leq 0$. Define maps, $\delta_{n} : \calT_{n+1}(A) \rightarrow \calT_{n}(A)$ and $\delta^{n} : \calS^{n+1}(A)$ as follows. The map, $\delta_{n}$ is the composition of the natural surjection, $\calT_{n+1}(A)\rightarrow \cal_{n}(A)$, with the inclusion, $\calK_{n}(A) \hookrightarrow \calT_{n}$. Similarly $\delta^{n}$ is the composition of the surjection, $\calS^{n}(A) \rightarrow C^{n}(A)$, and the inclusion, $\calC^{n}(A) \hookrightarrow \calS^{n+1}(A)$. Then $\{\calT_{n}(A), \delta_{n}\}$ is called the \textit{canonical brittle resolution} and $\{\calS^{n}(A, \delta^{n})\}$ is called the \textit{canonical flabby resolution} of $A$.  

Some remarks are in order. First of all, since each of the functors, $\calT_{n}$ and $\calS^{n}$, are compositions of exact functors, they are themselves exact functors. Further, by Lemma \ref{art6-lemma-7.2}, for any $A$, each of the carapaces $\calT_{n}(A)$ is brittle while the $\calS^{n}(A)$ are flabby. Thus, letting $C_{n}(X, A) = \Sigma(X, \calT_{n}(A))$ and $C^{n}(X, A)= \Gamma(X, \calS^{n}(A))$, whenever $0 \rightarrow A' \rightarrow A \rightarrow A'' \rightarrow 0$ is exact,
$$
0 \rightarrow C_{n}(X, A') \rightarrow C_{n}(X, A)\rightarrow C_{n}(X, A'')\rightarrow 0
$$
and
$$
0 \rightarrow C^{n}(X, A') \rightarrow C^{n}(X, A) \rightarrow C^{n}(X, A'')\rightarrow 0
$$
are exact. Abusing language, use $\delta_{n}$  and $\delta_{n}$ for the maps of segments and sections respectively as well as maps of carapaces, the homology groups of the complexes, $\{C_{n}(S, A), \delta_{n}\}$ and $\{C^{n}(X, A), \delta^{n}\}$ are connected sequences of homological functors. 
\end{definition}

\begin{definition}\label{art6-definition-7.4}
Let $A$ be an $R$-carapace on $X$. The complex, $\{C_{n}(X, A), \delta_{n}\}$ will be called the \textit{complex of Alexander chains} on $X$ with coefficients in $A$; $C^{n}(X, A), \delta^{n}\}$ will be called the Alexander co-chains. The homology of the complex of Alexander chains will be called the \textit{Alexander homology} and it will be written, $H_{n}^{a}(X, A)$. The homology of the Alexander co-chain complex will be called the \textit{Alexander cohomology} and it will be written $H_{a}^{n}(X, A)$. 
\end{definition}

\begin{prop}\label{art6-proposition-7.5}
The Alexander homology and cohomology of the simplicial complex, $X$, with coefficients in $A$ are isomorphic, respectively, to the exoskeletal homology and cohomology of $X$ with coefficients in $A$, functorially in $A$.
\end{prop}

\begin{proof}
By Proposition \ref{art6-proposition-6.10}, the exoskeletal homology groups vanish on brittle carapaces while the cohomology groups vanish on flabby carapaces. Hence the Alexander groups are the homology groups of the segments (respectively sections) over an acyclic resolution. The proposition follows.
\end{proof}

The following is an interesting footnote.

\begin{prop}\label{art6-proposition-7.6}
If $A$ is projective, the canonical brittle resolution of $A$ consists of projective carapaces. If $A$ in injective, each term in the canonical flabby resolution in injective.
\end{prop}

\begin{proof}
It suffices to prove that if $A$ is projective then $\calT_{0}(A)$ and $\calK_{0}(A)$ are projective and the corresponding statement for an injective $A$ and $\calS^{0}$ and $\calC^{0}$. Suppose that $P$ is projective and that $I$ in injective. Then by Proposition \ref{art6-prop-1.6}, $P(\sigma)$ is projective and $I(\sigma)$ is injective for each $\sigma \in X$. But then, by \ref{art6-lemma-1.4}, $(P(\sigma))\uparrow_{\sigma}$ is projective and $(I(\sigma))\downarrow^{\sigma}$ in injective. By the definition of $\calT_{0}$ and $S_{0}$ and because coproducts of projective are projective and products of injectives are injective, $\calT_{0}(P)$ is projective and $\calS^{0}(I)$ is injective. But then
$$
0 \rightarrow \calK_{0}(P) \rightarrow \calT_{0}(P) P^{\pi_{P}} \rightarrow P \rightarrow 0
$$
and
$$
0 \rightarrow I \xrightarrow{j_{I}} \calS^{0}(I) \rightarrow \calC^{0}(I)\rightarrow 0
$$
are exact. The last two terms of the first sequence are projective while the first two terms of the second sequence are injective. Hence $\calK_{0}(P)$ in projective and $\calC^{0}(I)$ is injective. An iterative application of these facts establishes the result.
\end{proof}

\section{G-carapaces and their Homology}

