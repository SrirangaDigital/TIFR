\title{Moduli of logarithmic connections}
\markright{Moduli of logarithmic connections}

\author{By~Nitin Nitsure}
\markboth{By~Nitin Nitsure}{Moduli of logarithmic connections}


\date{}
\maketitle

The talk was\pageoriginale based on the paper \cite{art14-keyN} which will, appear elsewhere. What follows is a summary of the results.

Let $X$ be a non-singular projective variety, with $S\subset X$ a divisor with normal crossings. A logrithmic connection $E=(\calE, \nabla)$ on $X$ with sigularity over $S$ is a torsion free coherent sheaf  $\calE$ together with a $\bC$-linear map $\nabla : \calE \rightarrow \Omega_{X}^{1}[\log S] \otimes \calE$ satisfying the Leibniz rule and having curvature zero, where $\Omega_{X}^{1}[\log S]$ is the sheaf of 1-forms on $X$ with logarithmic singularities over $S$. By a theorem of Deligne \cite{art14-keyD}, a connection with curvature zero on a non-nonsigular quasi-projective variety $Y$ is regular if and only if given any Hironaka completion $X$ of $Y$ (so that $X$ is non-sigular projective and $S=X-Y$ is a divisor with normal crossings), the connection extends to a logrithmic connections on $X$ with singularity over $S$.

Carlos Simpson has constructed in \cite{art14-keyS} a moduli scheme for non-singular connections (with zero curvature) on a projective variety. A simple example (see \cite{art14-keyN}) shows that a modulo scheme for regular connections on a quasiprojective variety does note in general exist. Therefore, we have to consider the moduli problem for loarithmic connections on a projective variety.

The main difference between non-singular connections and logarithmic  connections is that for logarithmic connections,  we have to define a notion of (semi)-stability, and restrict ourselves to these. We say that  a logarithmic connection is (semi-)stable, if usual inequality between normalized Hilbert polynomials is satisfied for any $\nabla$-invariant coherent subsheaf. In the case of non-sigular connections on a projective variety, the normalized Hilbert polynomial is always the same, so semi-stability is automatically fullfilled. Followigng Simpson,s method, with the extra feature of keeping track of (semi-)stability, we prove the exitance of coarse moduli scheme for ($\bS$-equivalen classes of) semistable-logarithmic connections which have a given Hilbert polynomial. We also show that the infinitesimal deformations of a locally free logarithmic connection $E$ are parametrized by the first hypercohomology of the logarithmic de Rham complex associated with End $(E)$.

A given regular connection on a quasi-projective variety $Y$ has infinitely many extensions as logarithmic connections on a given Hironaka completion $X$ of $Y$. A canonical choice of such an extension is given by the fundamental construction of Deligne \cite{art14-keyD}, which gives a locally free logarithmic extension. Using our description, we show that certain extensions of any given regular connections are \textit{rigid}, that is, they have no infinitesimal deformations which keep the underlying regular connections fixed. The cirterion for this is that no two distinct eigenvalues of the residue the logarithmic connection must differ by an integer. In paticular, this shows that Deligne's construction gives a rigid extension.

\begin{thebibliography}{99}
\bibitem[D]{art12-keyD} Deligne, P., \textit{Equations differentielles a points singuliers reguliers}, Lect. Notes in Math. {\bf 163} Springer (1970)

\bibitem[N]{art12-keyN} Nitsure, N., \textit{Moduli of semi-stable logarithmic connections}, Journal of Amer. Math. Soc.
 {\bf 6} (1993) 597-609.

\bibitem[S]{art12-keyS} Simpson, C., \textit{Moduli of representations of the fundamental group of a smooth projective variety}, Preprint (1990) Princeton University.
\end{thebibliography}

\begin{flushleft}
School of Mathematics

Tata Institute of Fundamental Research

Homi Bhabha Road

Bombay 400 005

India
\end{flushleft}
