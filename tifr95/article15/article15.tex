\title{The Borel-Weil theorem and the Feynman path integral}
\markright{The Borel-Weil theorem and the Feynman path integral}

\author{By~Kiyosato Okamoto}
\markboth{By~Kiyosato Okamoto}{The Borel-Weil theorem and the Feynman path integral}


\date{}
\maketitle

\section*{Introduction}

Let $p$ and $q$ be the canonical mometum and coordinate of a particle. In the operator method of quantization, corresponding ot $p$ and $q$ there are operators, $P, Q$ which in the coordinate representation have the form:
$$
A=q, \quad P=-\sqrt{-1}\dfrac{d}{dp}.
$$

The quantization means the correspondence between the Hamiltonian fucntion $h(p, q)$ and the Hamiltonian operator $\bH= h(P, Q)$, where a certain procedure for ordering noncommuting operator arguments $P$ and $Q$ is assmed. The path integral quantization is the method to compute the kernel function of the unitary operator $\exp(-\sqrt{-1}t \bH)$.

any mathematically strict definition of the path integral has not yet given. In one tries to compute path integral in general one may encounter the difficulty of divergence of the path integral. Many examples, however, show that the path integral is a very poweful tool to compute the kernel function of the operator explicitly.

The purpose of this lecture is to explain what kinds of divergence we have when we try to compute the path integral for complex polarizations of a connected semisimple Lie group which contains a compact Cartan subgroup and to show that we can regularize the path integral by the process of ``normal ordering" (cf. Chapter 13 in \cite{art15-key10}). The details and proofs of these results are given in the forthcoming paper \cite{art15-key7}.

Since a few in the audience do not seem to know about the Feynman path integral I would like to start with explaining it form a point of view of the theory of unitary representations, using the Heisenberg group which is most deeply related with the quantum mechanics.

The Feynman's idea of the path integral can be easily and clearly understood if one computes the path integral on the coadjoint orbits of the Heisenberg group:
$$
G= \left\{ \begin{pmatrix}
1 & p & r\\
  &  1 & q\\
  &    & 1
\end{pmatrix}
; p,q,r \in \bR \right\}.
$$
The Lie algebra of $G$ is given by
$$
\fg = \left\{ \begin{pmatrix}
0 & a & c\\
  &  0 & b\\
  &    & 0
\end{pmatrix}
; a,b,c \in \bR \right\}.
$$
The dual space of $\fg$ is identified with
$$
\fg^{*} = \left\{ \begin{pmatrix}
0 &  & \\
 \xi &  0 & \\
 \sigma & \eta   & 0
\end{pmatrix}
; \xi,\eta, \sigma \in \bR \right\}.
$$
by the pairing
$$
\fg \times \fg^{*} \ni (X, \lambda) \longmapsto {\rm tr}(\lambda X) \in \bR.
$$
Any nontrivial coadjoint orbit is given by an element
$$
\lambda_{\sigma} = \begin{pmatrix}
0 &  &  \\
 0 &  0 & \\
 \sigma & 0  & 0
\end{pmatrix}
\quad \text{for some} \; \sigma \neq 0
$$
Then the isotropy subgroup at $\lambda_{\sigma}$ is given by
$$
G_{\lambda_{\sigma}} = \left\{ \begin{pmatrix}
1 & 0 & r \\
  &  1 & 0\\
  &    & 1
\end{pmatrix}
; r \in \bR \right\},
$$
and the Lie algebra of $G_{\lambda_{\sigma}}$ is
$$
\fg_{\lambda_{\sigma}}= \left\{ \begin{pmatrix}
1 & 0 & c \\
  &  0 & 0\\
  &    & 0
\end{pmatrix}
; c \in \bR \right\}.
$$
We consider the real polarization:
$$
\fp= \left\{ \begin{pmatrix}
0 & a & c \\
  &  0 & 0\\
  &    & 0
\end{pmatrix}
; a, c \in \bR \right\}.
$$
Then the analytic subgroup of $G$ corresponding to $\fp$ is given by
$$
P= \left\{ \begin{pmatrix}
1 & p & r \\
  &  1 & 0\\
  &    & 1
\end{pmatrix}
; p, r \in \bR \right\}.
$$
Clearly the Lie algebra homomorphism
$$
\fp \ni \begin{pmatrix}
0 & a & c \\
  &  0 & 0\\
  &    & 0
\end{pmatrix}
\longmapsto - \sqrt{-1}\sigma c \in \sqrt{-1}\bR.
$$
lifts to the unitary character $\xi_{\lambda_{\sigma}}$:
$$
P \ni \begin{pmatrix}
1 & p & r \\
  &  1 & 0\\
  &    & 1
\end{pmatrix}
\longmapsto e^{-\sqrt{-1}\sigma r} \in U(1).
$$
Let $L_{\xi_{\lambda_{\sigma}}}$ denote the line bundle associated with $\xi_{\lambda_{\sigma}}$ over the homogenous space $G /P$. Then the space $C^{\infty}(L_{\xi_{\lambda_{\sigma}}})$ of all complex valued $C^{\infty}$ -sections of $L_{\xi_{\lambda_{\sigma}}}$ can be identified with
$$
\left\{ f \in C^{\infty}(G); F(gp) = \xi_{\lambda_{\sigma}}(p)^{-1}f(g) \quad (g \in G, p \in P)\right\}.
$$
For any $g \in G$ we define an operator $\pi_{\lambda_{\sigma}}^{\fp}(g)$ on $C^{\infty}(L_{\xi_{\lambda_{\sigma}}}):$
For $f \in \bC^{\infty}(L_{\xi_{\lambda_{\sigma}}})$
$$
(\pi_{\lambda_{\sigma}}^{\fp}(g)f)(x)=f(g^{-1}x) \quad (x \in G).
$$

Let $\calH_{\lambda_{\sigma}}^{\fp}$ be the Hilbert space of all square integrable sections of $L_{\xi_{\lambda}}$. Then
$\pi_{\lambda_{\sigma}}^{\fp}$ is a unitary representation of $G$ on $\calH_{\lambda_{\sigma}}^{\fp}$.

We put
$$
M =\left\{\begin{pmatrix}
1 & 0 & 0 \\
  &  1 & q\\
  &    & 1
\end{pmatrix}
; q \in \bR \right\}.
$$
Then as is easily seen the product mapping $M \times P \longrightarrow G$ is a real analytic isomorphism which is surjective. Let $f \in C^{\infty}(L_{\xi_{\lambda_{\sigma}}})$. Then, since
$$
f(g\begin{pmatrix}
1 & p & r \\
  &  1 & 0\\
  &    & 1
\end{pmatrix}
)= e^{\sqrt{-1}\sigma r} f(g) \quad\; \text{for}\; q \in G,  \begin{pmatrix}
1 & p & r \\
  &  1 & 0\\
  &    & 1
\end{pmatrix}
\in P,
$$
$f$ can be uniquely determined by its values on $M$. From this we obtain the following onto-isometry:
$$
\calH_{\lambda_{\sigma}}^{\fp}\ni f \longmapsto F \in L^{2}(\bR)
$$
where
$$
F(q)=f(\begin{pmatrix}
1 & 0 & 0 \\
  &  1 & q\\
  &    & 1
\end{pmatrix}
) \quad (q \in \bR).
$$
For any $g =\exp\begin{pmatrix}
0 & a & c\\
  & 0 & b\\
  &   & 0
\end{pmatrix}
\in G$, we define a unitary operater $U_{\lambda_{\sigma}}^{\fp}(g)$ on $L^{2}(\bR)$ such that the diagram below is commutative:
 $$
 \xymatrix{
\calH_{\lambda_{\sigma}}^{\fp}\ar[r]\ar [d]_-{\pi_{\lambda_{\sigma}}^{\fp}(g)} & L^{2}(\bR)\ar[d]^-{U_{\lambda_{\sigma}}^{\fp}(g)}\\
\calH_{\lambda_{\sigma}}^{\fp} \ar[r] & L^{2}(\bR).
} 
$$
Then we have
\begin{align*}
(U_{\lambda_{\sigma}}^{\fp}F)(q)&= f(g^{-1}\begin{pmatrix}
1 & 0 & 0\\
  & 1 & q\\
  &   & 1
\end{pmatrix})\\
& =f(\begin{pmatrix}
1 & -a & -c+\frac{ab}{2}\\
  & 1 & -b\\
  &   & 1
\end{pmatrix}
\begin{pmatrix}
1 & 0 & 0\\
  & 1 & q\\
  &   & 1
\end{pmatrix}
)\\
& =f(
\begin{pmatrix}
1 & 0 & 0\\
  & 1 & q-b\\
  &   & 1
\end{pmatrix}
\begin{pmatrix}
1 & -a & -c+\frac{ab}{2}-aq\\
  & 1 & 0\\
  &   & 1
\end{pmatrix}
) \\
&= e^{\sqrt{-1}\sigma(-c +\frac{ab}{2}-aq)}F(q-b).
\end{align*}
 Now we show that the above unitary operator is obtained by the path integral.

 In the following, for the definition of the connection form $\theta_{\lambda_{\sigma}}$, the hamiltonian $H_{Y}$ and the action $\int_{0}^{T}\gamma^{*}\alpha-H_{Y}dt$ in the general case the audience may refer to the introduction of the paper \cite{art15-key5}.

 We use the local coordinates $q, p,r $ of $g \in G$ as follows:
 $$
 G \ni g = \begin{pmatrix}
1 & 0 & 0\\
 & 1 & q\\
  & & 1
 \end{pmatrix}
\begin{pmatrix}
1 & p & r\\
 & 1 & 0\\
  & & 1
 \end{pmatrix}.
$$
Since the canonical 1-form $\theta$ is given by $g^{-1}dg$, we have
$$
\theta_{\lambda_{\sigma}}= < \lambda_{\sigma}, \theta> ={\rm tr}(\lambda_{\sigma} g^{-1}dg)= \sigma(dr-pdq).
$$
We choose
$$
\alpha_{\fp}= -\sigma pdq.
$$
Then
$$
\dfrac{d\alpha_{fp}}{2 \pi} = \dfrac{-\sigma dp \wedge dq}{2\pi}.
$$
For $Y \in \fg$, the hamiltonian $H_{Y}$ is given by
$$
H_{Y} = {\rm tr}(\lambda_{\sigma} g^{-1} Y g)= \sigma(aq-bp + c)
$$
where $Y = \begin{pmatrix}
0 & a & c\\
  & 0 & b\\
  &   & 0
\end{pmatrix}.
$ The action is given by
$$
\int_{0}^{T}\gamma^{*}\alpha-H_{Y}dt = \int_{0}^{T}\{-\sigma p(t)q(t)-\sigma(aq-bp + c)\}dt.
$$
We divide the time interval $[0,T]$ into $N$-equal small intervals $\left[\frac{k-1}{N}T,  \frac{k}{N}T\right]$
$$
\int_{0}^{T} \gamma^{*}\alpha-H_{Y}dt = \sum\limits_{k=1}^{N}\int_{\frac{k-1}{N}T}^{\frac{k}{n}T} \gamma^{*}\alpha-H_{Y}dt.
$$
The physicists' calculation rule asserts that one should take the ``ordering":
$$
\sum\limits_{k=1}^{N}\left\{-\sigma p_{k-1}(q_{k}-q_{k-1})-\sigma(a \frac{q_{k}+q_{k-1}}{2}-bp_{k-1} + c)\frac{T}{N} \right\}.
$$

This choice of the ordering can be mathematically formulated as follows.

We take the paths: for $t \in \left[\frac{k-1}{N}T, \frac{k}{N}T\right]$
\begin{align*}
q(t)&=q_{k-1}+(t-\frac{k-1}{N}T)\dfrac{q_{k}-q_{k-1}}{T/N},\\
p(t)&= p_{k-1},\\
q(0)&= q, \; \text{and} \; q(T)=q'.
\end{align*}
Then the action for the above path becomes
\begin{equation*}
\begin{split}
\sum\limits_{k=1}^{N}&\int_{\frac{k-1}{N}T}^{\frac{k}{N}T} \{-\sigma p(t)q(t)-\sigma(aq-bp +c)\}dt\\
  &= \sum\limits_{k=1}^{N}\left\{-\sigma p_{k-1}(q_{k}-q_{k-1})-\sigma(a \frac{q_{k}+q_{k-1}}{2}-bp_{k-1}+c)\dfrac{T}{N} \right\}
\end{split}
\end{equation*}

The Feynman path integral asserts that the transition amplitude between the point $q=q_{0}$ and the point $q'=q_{N}$ is given by the kernel function which is computed as follows:

\noindent
$K_{Y}^{\fp}(q',q; T)$
\begin{flalign*}
 &= \lim\limits_{N \rightarrow \infty} \int_{-\infty}^{\infty}\cdots \int_{-\infty}^{\infty} \int_{-\infty}^{\infty} \cdots
 \int_{-\infty}^{\infty} |\sigma| \dfrac{dp_{0}}{2\pi} \cdots |\sigma|\dfrac{dp_{N-1}}{2\pi}dq_{1}\cdots dq_{N-1}\\
 &\quad \times \exp \left\{\sqrt{-1}\sigma \sum\limits_{k=1}^{N}\left[-p_{k-1}(q_{k}-q_{k-1}) -(a\dfrac{q_{k}+q_{k-1}}{2}-bp_{k-1} + c)\dfrac{T}{N}\right] \right\}\\
 &=\lim\limits_{N\rightarrow \infty} \int_{-\infty}^{\infty}\cdots \int_{-\infty}^{\infty} dq_{1}\cdots dq_{N-1}\prod\limits_{k=1}^{N}\delta(-q_{k} +q_{k-1}+ b \dfrac{T}{N}) \\
 &\qquad \qquad \qquad \times \left\{-\sqrt{-1}\sigma \sum\limits_{k=1}^{N}(\dfrac{a(q_{k} +q_{k-1})}{2} + c)\dfrac{T}{N}\right\}\\
 &=\lim\limits_{N \rightarrow \infty}\delta(-q_{N} + q_{0} bT) \exp \left\{-\sqrt{-1}\sigma(aT(q_{0} + \dfrac{bT}{2})+ cT) \right\}\\
 &=\delta(-q' + q bT) \exp\left\{-\sqrt{-1}\sigma(aqT + \dfrac{abT^{2}}{2} + cT) \right\}.
\end{flalign*}
For $F \in C_{c}^{\infty}(\bR)$ we have

\medskip
$\int_{-\infty}^{\infty}K_{Y}^{\fp}(q',q;T)F(q)dq$
\begin{flalign*}
&= \exp \left\{-\sqrt{-1}\sigma\left(aq'T-\dfrac{abT^{2}}{2} + cT \right)\right\}F(q'-bT)\\
&=(U_{\lambda_{\sigma}}^{\fp}(\exp TY)F)(q').
\end{flalign*}
Thus the path integral gives our unitary operator.

In the above quantization, the hamiltonian function $H_{Y}(p,q)$ corresponds to
$$
\sqrt{-1}\dfrac{d}{dt}U_{\lambda_{\sigma}}^{\fp}(\exp tY)|_{t=0} = \sigma(aq+c) - \sqrt{-1}b\dfrac{d}{dq},
$$
which is slightly different form
$$
H_{Y}(-\sqrt{-1}\dfrac{d}{dq},q) = \sigma(aq + c +\sqrt{-1}b\dfrac{d}{dq}).
$$
This difference comes from the fact that we chose $\alpha_{\fp}= -\sigma pdq$ whereas physicists usually take $pdq$.

\section{Coherent representation}\label{art15-sec-1}

In this section, we compute the path integral for unitary representations realized by the Borel-Wiel theorem for the Heisenberg group. In other words, we compute the path integral for a complex polarization which is called by physicists the path integral for a complex polarization which is called by physicists the path integral for the coheretnt representation. We shall show that the path integral, also in this case, gives unitary operators of these representations.

The complexification $G^{\bC}$ of $G$ and $\fg^{\bC}$ of $\fg$ are given by
\begin{align*}
G^{\bC} &=\left\{ \begin{pmatrix}
1 & p & r\\
  &  1 & q\\
  &    & 1
\end{pmatrix}
; p,q,r \in \bC \right\},\\
\fg^{\bC} &=\left\{ \begin{pmatrix}
0 & a & c\\
  &  0 & b\\
  &    & 0
\end{pmatrix}
; a,b,c \in \bC \right\}.
\end{align*}
we consider the complex polarization defined by
$$
\fp =\left\{ \begin{pmatrix}
0 & \sqrt{-1}b & c\\
  &  0 & b\\
  &    & 0
\end{pmatrix}
; a,b,c \in \bC \right\}.
$$
We denote by $P$ the complex analytic subgroup of $G^{\bC}$ corresponding to $\fp$. We put $W=GP=G^{\bC}$. Then it is easy to see that Lie algebra homomorphism
$$
\fp \ni \begin{pmatrix}
0 & \sqrt{-1}b & c\\
  &  0 & b\\
  &    & 0
\end{pmatrix}
 \longmapsto -\sqrt{-1}\sigma c \in \bC
$$
lifts uniquely to the holomorphic character $\xi_{\lambda_{\lambda_{\sigma}}}$:
$$
P \ni \begin{pmatrix}
1 & \sqrt{-1}b & c+\frac{\sqrt{-1}}{2}b^{2}\\
  &  1 & b\\
  &    & 1
\end{pmatrix}
 \longmapsto e^{-\sqrt{-1}\sigma c} \in \bC^{*}.
$$

We denote by $L_{\xi_{\lambda_{\sigma}}}$ the holomorphic line bundle on $G^{\bC}/P$ associated with the character $\xi_{\lambda_{\lambda_{\sigma}}}$.

We denote by $\Gamma(L_{\xi_{\lambda_{\sigma}}})$ the space of all holomorphic sections of $L_{\xi_{\lambda_{\sigma}}}$ and by $\Gamma(\bC)$ the space of all holomorphic functions on $\bC$.

We use the coordinates of $g \in G$:
\begin{flalign*}
g &= \exp \begin{pmatrix}
0 & -\frac{\sqrt{-1}}{2}z & 0\\[0.3cm]
  &  0 & \frac{1}{2}z\\[0.3cm]
  &    & 0 
\end{pmatrix}
\exp
\begin{pmatrix}
0 & -\frac{\sqrt{-1}}{2}\overline{z} & r+\dfrac{\sqrt{-1}}{4}|z|^{2}\\[0.3cm]
  &  0 & \frac{1}{2}\overline{z}\\[0.3cm]
  &    & 0 
\end{pmatrix}
\\
 & = \begin{pmatrix}
   1 & -\frac{\sqrt{-1}}{2}\overline{z} & -\dfrac{\sqrt{-1}}{8}z^{2}\\[0.3cm]
  &  1 & \frac{1}{2}z\\[0.3cm]
  &    & 1
\end{pmatrix}
\begin{pmatrix}
1 & -\frac{\sqrt{-1}}{2}\overline{z} & r+\dfrac{\sqrt{-1}}{4}|z|^{2} + \dfrac{\sqrt{-1}}{8}\overline{z}^{2} \\[0.3cm]
  &  1 & \frac{1}{2}\overline{z}\\[0.3cm]
  &    & 1
\end{pmatrix}
\end{flalign*}
where $z \in \bC, r \in \bR$.

We have the isomorphism
$$
\Gamma\left(L_{\xi_{\lambda_{\lambda_{\sigma}}}}\right) \ni f \longmapsto \in \Gamma (\bC)
$$
where
$$
F(Z)= f(\begin{pmatrix}
   1 & -\frac{\sqrt{-1}}{2}\overline{z} & -\dfrac{\sqrt{-1}}{8}z^{2}\\[0.3cm]
  &  1 & \frac{1}{2}z\\[0.3cm]
  &    & 1
\end{pmatrix}
)
$$

 We denotes by $\Gamma^{2}(L_{\xi_{\lambda_{\sigma}}})$ the Hilbert space of all square integrable holomprphic sections of $L_{\xi_{\lambda_{\sigma}}}$ and by $\Gamma^{2}\left(\bC, \frac{|\sigma|}{2\pi}e^{-\frac{\sigma}{2}|z|^{2}}\right)$

For any $g \in G$ we define an operator $\pi_{\lambda_{\sigma}}^{\fp}(g)$ on $\Gamma^{2}(L_{\xi_{\lambda_{\sigma}}}):$ For $f \in \Gamma^{2}(L_{\xi_{\lambda_{\sigma}}})$
$$
(\pi_{\lambda_{\sigma}}^{\fp}(g)f)(x)=f(g^{-1}x) \quad (x \in G).
$$
Then $ \pi_{\lambda_{\sigma}}^{\fp}$ is a unitary representation of $G$ on $\Gamma^{2}(L_{\xi_{\lambda_{\sigma}}})$.  Since
\begin{align*}
\int_{G/G_{\lambda_{\sigma}}}|f(g)|^{2}\omega_{\lambda_{\sigma}} &= \int_{\bC}\left| e^{-\frac{\sigma|z|^{2}}{4}}F(Z)\right|^{2} |\sigma| \dfrac{dzd\overline{z}}{2\pi}\\
&= \int_{\bC}|F(z)|^{2} e^{-\frac{\sigma |z|^{2}}{2}}|\sigma| \dfrac{dzd\overline{z}}{2\pi},
\end{align*}
where we denote by $\omega_{\lambda_{\sigma}}$ the canonical symplectic form on the coadjoint orbit $\calO_{\lambda_{\sigma}}= G/G_{\lambda_{\sigma}}$ and we put
$$
dzd\overline{z}= \dfrac{\sqrt{-1}}{2}dz \wedge d\overline{z}.
$$
The above isomorphism gives an isometry of $\Gamma^{2}(L_{\xi_{\lambda_{\sigma}}})$ onto $\Gamma^{2}(\bC, \frac{|\sigma|}{2 \pi}e^{-\frac{\sigma}{2}|z|^{2}})$.

As is easily seen $\Gamma^{2}(\bC, \frac{|\sigma|}{2\pi}e^{-\frac{\sigma}{2}|z|^{2}}) \neq \{0\}$ if and only if $\sigma > 0$. If follows that
$$
\Gamma^{2}(L_{\xi_{\lambda_{\sigma}}})\neq \{0\}\; \text{if and only if}\; \sigma > 0.
$$

For the rest of the section we assume that  $\sigma > 0$.

For any $g=\exp \begin{pmatrix}
   0 & a& c
     &  0 & b
     &    & 0
\end{pmatrix}
\in G $, we define a unitary operator $U_{\lambda_{\sigma}}^{\fp}(g)$ on $\Gamma^{2}(\bC, \frac{\sigma}{2\pi}e^{-\frac{\sigma}{2}|z|^{2}})$ such that the diagram below is commutative:
$$
\xymatrix{
\Gamma^{2}(L_{\xi_{\lambda_{\sigma}}}) \ar[r] \ar[d]_-{\pi_{\lambda_{\sigma}}^{\fp}(g)}& \Gamma^{2}(\bC, \frac{\sigma}{2\pi}e^{-\frac{\sigma}{2}|z|^{2}})\ar[d]^-{U_{\lambda_{\sigma}}^{\fp}(g)}\\
\Gamma^{2}(L_{\xi_{\lambda_{\sigma}}}) \ar [r] & \Gamma^{2}(\bC, \frac{\sigma}{2\pi}e^{-\frac{\sigma}{2}|z|^{2}} )
}
$$
Then we have
$(U_{\lambda_{\sigma}}^{\fp}(g)F)(z)$
\begin{flalign*}
& = f(g^{-1} \begin{pmatrix}
   1 & -\frac{\sqrt{-1}}{2}\overline{z} & -\dfrac{\sqrt{-1}}{8}z^{2}\\[0.3cm]
  &  1 & \frac{1}{2}z\\[0.3cm]
  &    & 1
\end{pmatrix}) \\
&= f(\begin{pmatrix}
   1 & -a & -c+\frac{ab}{2}\\
  &  1 & -b\\
  &    & 1
\end{pmatrix}
\begin{pmatrix}
   1 & -\frac{\sqrt{-1}}{2}z & -\dfrac{\sqrt{-1}}{8}z^{2}\\[0.3cm]
  &  1 & \frac{1}{2}z\\[0.3cm]
  &    & 1
\end{pmatrix})\\
&=f(
\begin{pmatrix}
   1 & -\frac{\sqrt{-1}}{2}(z-\gamma) & -\dfrac{\sqrt{-1}}{8}(z-\gamma)^{2}\\[0.3cm]
     &  1 & \frac{1}{2}(z-\gamma)\\[0.3cm]
     &    & 1
\end{pmatrix}\\
 &\qquad \times \begin{pmatrix}
 1  &-\frac{\sqrt{-1}_{\overline{\gamma}}}{2} & -c +\frac{\sqrt{-1}}{4}|\gamma|^{2}-\frac{\sqrt{-1}}{2}\overline{\gamma}z +\frac{\sqrt{-1}}{8}\overline{\gamma}^{2}\\[0.3cm]
    &  1      & -\frac{\overline{\gamma}}{2}\\[0.3cm]
    &          &   1 
 \end{pmatrix})\\
 &= e^{\sigma(-\sqrt{-1}c-\frac{1}{4}|\gamma|^{2} + \frac{1}{2}\overline{\gamma}z)}F(z-\gamma),
\end{flalign*}
Where $\gamma = b + \sqrt{-1}a$.

It is well-known that $U_{\lambda_{\sigma}}^{\fp}$ is an irreducible unitary representation of $G$ on $\Gamma^{2}(\bC, \frac{\sigma}{2\pi}e^{-\frac{\sigma}{2}|z|^{2}})$.

 Using the parametrization:
 \begin{flalign*}
g =\begin{pmatrix}
   1 & -\frac{\sqrt{-1}}{2}z & -\dfrac{\sqrt{-1}}{8}z^{2}\\[0.3cm]
  &  1 & \frac{1}{2}z\\[0.3cm]
  &    & 1
\end{pmatrix}
\begin{pmatrix}
   1 & \frac{\sqrt{-1}}{2}\overline{z} & r+\frac{\sqrt{-1}}{4}|z|^{2}+\frac{\sqrt{-1}}{8}\overline{z}^{2}\\[0.3cm]
  &  1 & \frac{1}{2}\overline{z}\\[0.3cm]
  &    & 1
\end{pmatrix},
 \end{flalign*} 
We have
\begin{gather*}
\theta_{\lambda_{\sigma}} = {\rm tr}(\lambda_{\sigma}g^{-1}dg) = \sigma(dr + \sqrt{-1}\dfrac{zd\overline{z}-\overline{z}dz}{4}),\\
H_{Y}= \sigma\left(\sqrt{-1}\dfrac{\overline{\gamma}z-\gamma{\overline{z}}}{2} + c\right).
\end{gather*}

We choose $\alpha_{\fp}= -\frac{\sqrt{-1}\sigma}{2}\overline{z}dz$. Then we have
$$
\dfrac{1}{2\pi}d\alpha_{\fp} = \dfrac{\sqrt{-1}\sigma dz\wedge d\overline{z}}{4\pi}.
$$
For fixed $z, z' \in \bC$ we define the paths: For $t \in \left[\frac{k-1}{N}T, \frac{k}{N}T \right]$
\begin{align*}
\overline{z}(t)&= \overline{z}_{k-1},\\
z(t) &= z_{k-1} + \left(t -\dfrac{k-1}{N}T\right)\dfrac{z_{k}-z_{k-1}}{T/N},\\
z(0) &= z \; \text{and} \; z(T)= z'.
\end{align*}

Then the action becomes
\begin{flalign*}
&\int_{0}^{T} \left\{\dfrac{1}{2}\sigma \overline{z}(t)\dot{z}(t)- \sqrt{-1}\sigma \left( \dfrac{\sqrt{-1}\overline{\gamma} z (t)- \sqrt{-1}\gamma \overline{z}(t)}{2} + c \right)\right\}dt\\
&= \sum\limits_{k=1}^{N}\int_{\frac{k-1}{N}T}^{\frac{k}{N}T}\left\{\dfrac{1}{2}\sigma \overline{z}(t)\dot{z}(t)- \sqrt{-1}\sigma \left( \dfrac{\sqrt{-1}\overline{\gamma} z (t)- \sqrt{-1}\gamma \overline{z}(t)}{2} + c \right)\right\}dt\\
&=\sigma \sum\limits_{k=1}^{N}\left[\dfrac{1}{2}\overline{z}_{k-1}(z_{k}-z_{k-1})-\left(\dfrac{\gamma}{2}\overline{z}_{k-1}- \frac{\overline{\gamma}}{4}(z_{k} + z_{k-1}) + \sqrt{-1}c\right)\dfrac{T}{N}\right].
\end{flalign*}

The following lemma can be easily proved.

\medskip
\noindent
{\bfseries Lemma \thnum{1.} \label{art15-lemma-1}} \textit{We have the following formula for $c_{1}, c_{2} \in \bC$.}

\begin{flalign*}
\int_{\bC}\dfrac{\sigma dz'd\overline{z}'}{2\pi} &\exp \sigma \left\{-\dfrac{1}{2} |z|^{2} + z'\left(\dfrac{1}{2}\overline{z} + c_{1}\right) + \overline{z}'\left(\dfrac{1}{2}z''-c_{2}\right)\right\}\\
&= \exp\sigma \left\{z''\left( \dfrac{\overline{z}}{2} + c_{1}\right) - 2c_{2}\left(\dfrac{\overline{z}}{2} + c_{1}\right)\right\}.
\end{flalign*}

Using this lemma, we can compute the path integral explicitly as follows:
$K_{Y}^{\fp}(z', z; T)$
\begin{flalign*}
&= \lim\limits_{N \rightarrow \infty} \int_{\bC}\cdots \int_{\bC}\dfrac{\sigma dz_{1} d\overline{z}_{1}}{2 \pi} \cdots
\dfrac{\sigma dz_{N-1}d\overline{z}_{N-1}}{2 \pi}\\
&\times \exp\left\{\sigma \sum\limits_{k=1}^{N}\left[\dfrac{1}{2}\overline{z}_{k-1}(z_{k}-z_{k-1})-\left(\dfrac{\gamma}{2}\overline{z}_{k-1}- \frac{\overline{\gamma}}{4}(z_{k} + z_{k-1}) + \sqrt{-1}c\right)\dfrac{T}{N}\right]\right\}\\
&= \lim\limits_{N \rightarrow \infty} \int_{\bC}\cdots \int_{\bC}\dfrac{\sigma dz_{1} d\overline{z}_{1}}{2 \pi} \cdots
\dfrac{\sigma dz_{N-1}d\overline{z}_{N-1}}{2 \pi}\\
&\qquad\times \exp \left\{\sigma \sum_{k=1}^{N}\left(-\dfrac{1}{2}|z_{k-1}|^{2} + \overline{z}_{k-1}(\dfrac{z_{k}}{2}-\dfrac{\gamma T}{2N}) + z_{k-1}
\dfrac{\overline T}{2N} \right)\right.\\
 &\qquad \qquad \left.+ \sigma (z_{N}-z_{0})\dfrac{\overline{\gamma T}}{4N}-\sqrt{-1}\sigma cT \right\}\\[0.8cm]
&=\lim\limits_{N \rightarrow \infty} \int_{\bC}\cdots \int_{\bC}\dfrac{\sigma dz_{1} d\overline{z}_{1}}{2 \pi} \cdots
\dfrac{\sigma dz_{N-1}d\overline{z}_{N-1}}{2 \pi} \\
 &\qquad \times \exp \left\{\sigma \left(-\dfrac{1}{2}|z_{0}|^{2} - \overline{z}_{0}\dfrac{\gamma T}{2N} + z_{0}\dfrac{\overline{\gamma}T}{2N}
 \right)\right.\\
 &\qquad \qquad  + \sigma \left(-\dfrac{1}{2}|z_{1}|^{2} + \overline{z}_{1} \left(\dfrac{z_{2}}{2}- \dfrac{\gamma T}{2N}\right) + z_{1}\left(\dfrac{\overline{z}_{0}}{2}+\dfrac{\overline{\gamma}T}{2N}\right)\right)\\
 &\qquad \qquad + \sigma \left(-\dfrac{1}{2}|z_{2}|^{2} + \overline{z}_{2} \left(\dfrac{z_{3}}{2}- \dfrac{\gamma T}{2N}\right) + z_{2}\dfrac{\overline{\gamma}T}{2N}\right)\\
 &\qquad \qquad + \sigma \sum\limits_{k=4}^{N}\left(-\dfrac{1}{2}|z_{k-1}|^{2} + \overline{z}_{k-1} \left(\dfrac{z_{k}}{2}- \dfrac{\gamma T}{2N}\right) + z_{k-1}\dfrac{\overline{\gamma}T}{2N}\right)\\
 &\qquad \qquad \left. + \sigma (z_{N}-z_{0})\dfrac{\overline{\gamma}T}{2N}- \sqrt{-1}\sigma c T \right\}\\[0.8cm]
 &=\left(\lim\limits_{N \rightarrow \infty} \int_{\bC}\cdots \int_{\bC}\dfrac{\sigma dz_{2} d\overline{z}_{2}}{2 \pi} \cdots
\dfrac{\sigma dz_{N-1}d\overline{z}_{N-1}}{2 \pi}\right) \\
&\qquad \times \exp \left\{ \sigma \left(-\dfrac{1}{2}|z_{0}|- \overline{z}_{0}\dfrac{\gamma T}{2N} + z_{0}\dfrac{\overline{\gamma}T}{2N}\right)\right.\\
 &+ \sigma \left(-\dfrac{1}{2}|z_{2}|^{2} + \overline{z}_{2}\left(\dfrac{z_{3}}{2}- \dfrac{\gamma T}{2N}\right) + z_{2}\left(\dfrac{\overline{z}_{0}}{2}+ 2 \dfrac{\overline{\gamma} T}{2N}\right) -\dfrac{\gamma T}{N}\left(\dfrac{\overline{z}_{0}}{2} + \dfrac{\overline{\gamma}T}{2N}  \right) \right)\\
 &\qquad  + \sigma \sum\limits_{k=4}^{N}\left(-\dfrac{1}{2}|z_{k-1}|^{2} + \overline{z}_{k-1} \left(\dfrac{z_{k}}{2}- \dfrac{\gamma T}{2N}\right) + z_{k-1}\dfrac{\overline{\gamma}T}{2N}\right)\\
 &\qquad \left. +\qquad \sigma(z_{N}-z_{0})\dfrac{\overline{\gamma}T}{4N}-\sqrt{-1}\sigma c T \right\}
\end{flalign*}
repeating the above procedure,
\begin{flalign*}
&=\lim\limits_{N \rightarrow \infty} \exp \left\{ \sigma \left( -\dfrac{1}{2}|z_{0}|^{2} + z_{N}\left(\dfrac{\overline{z}_{0}}{2} + \dfrac{\overline{\gamma}T}{2}\right) -\gamma T \left(\dfrac{\overline{z}_{0}}{2}+\dfrac{1}{2}\dfrac{\overline{\gamma}T}{2}\right)-\sqrt{-1}c T \right)\right.\\
 &\qquad \qquad \qquad \left.+ \sigma \left(-\overline{z}_{0}\dfrac{\gamma T}{2N} + (z_{0}-z_{N})\dfrac{\overline{\gamma}T}{4N} + \dfrac{\gamma T}{4N}(\overline{z}_{0}  + \overline{\gamma} T)\right)\right\}\\
&= \exp \left\{ \sigma \left(-\dfrac{1}{2}|z|^{2} + z'\left(\dfrac{\overline{z}}{2} + \dfrac{\overline{\gamma}T}{2}\right) -\gamma T \left(\dfrac{\overline{z}}{2}+\dfrac{\overline{\gamma}T}{4}\right)-\sqrt{-1}c T \right)\right\} 
\end{flalign*}

Thus for any $Y = \begin{pmatrix}
0 & a & c\\
  & 0 & b\\
  &   & 0  
\end{pmatrix}
\in \fg
$, we have

\begin{flalign*}
&\int_{\bC}\dfrac{\sigma dzd\overline{z}}{2 \pi} K_{Y}^{\fp}(z',z: T)F(z)\\
&= \int_{\bC}\dfrac{\sigma dzd\overline{z}}{2 \pi} \left\{ \sigma \left( -\dfrac{1}{2}|z|^{2} + \dfrac{1}{2}\overline{z}(z-\gamma T) +
\dfrac{1}{2}z'\overline{\gamma}T\right. \right.\\
&\left. \left. \qquad \qquad \qquad \qquad \qquad -\dfrac{1}{4}|\gamma|^{2}T^{2}-\sqrt{-1}c T \right) \right\}F(z)\\
& = \exp \left\{\sigma \left(\dfrac{1}{2}z'\overline{\gamma}T -\dfrac{1}{4}|\gamma|^{2}T^{2}-\sqrt{-1}c T\right)\right\}F(z'-\gamma T)\\
&= (U_{\lambda_{\sigma}}^{\fp} (\exp TY)F)(z').
\end{flalign*}

\section{Borel-Weil theorem}\label{art15-sec-2}

In this section, we consider unitary representations realized by the Borel-Weil theorem for semisimple Lie groups.

Let $G$ be a connected semisimple Lie group such that there exists a complexification $G^{\bC}$ with $\pi_{1}(G^{\bC})= \{1\}$ and such that rank $G=\dim\; T, T$ a maximal torus of $G$. Let $K$ be a maximal compact subgroup of $G$ which contains $T$, and $\fk$ the Lie algebra of $K$. Note that $G$ can be realized as a matrix group. We denote the conjugation of $G^{\bC}$ with respect to $G$, and that of $\fg^{\bC}$ with respect to $\fg$, both by - Let $\fg$ and $\fh$ be the Lie algebras of $G$ and $T$. We denote complexifications of $\fg$ and $\fh$ by $\fg^{\bC}$ and $\fh^{\bC}$, respectively. Then $\fh^{\bC}$ is a Cartan subalgebra of $\fg^{\bC}$.

Let $\Delta$ denote the set of all nonzero roots and $\Delta^{+}$ the set of all positive roots. Then we have root space decomposition
$$
\fg^{\bC} = \fh^{\bC} + \sum\limits_{\alpha \in \Delta}\fg^{\alpha}.
$$
Define
$$
\fn^{\pm} = \sum\limits_{\alpha \in \Delta^{+}}\fg^{\pm \alpha}, \quad \fb= \fh^{\bC} + n^{-}.
$$
Let $N, N^{-}, B$ and $T^{\bC}$ be the analytic subgroups corresponding to $\fn^{+}, n^{-}, \fb$, and $\fh^{\bC}$, respectively.

we fix an integral form $\Lambda$ on $\fh^{\bC}$. Then
$$
\xi_{\Lambda} : T \longrightarrow U(1), \qquad \exp H \longmapsto e^{\Lambda(H)} 
$$ 
define a unitary character of $T$ And $\xi_{\Lambda}$ extends uniquely to a holomoprhic one-dimensional representation of $B$:
$$
\xi_{\Lambda}: B = T^{\bC}N^{-} \longrightarrow \bC^{\times}, \qquad \exp H \cdot n^{-} \longmapsto e^{\Lambda(H)}.
$$
Let $\tilde{L}_{\Lambda}$ be the holomorphic line bundle over $G^{\bC}/B$ associated to the holomorphic one-dimensional representation $\xi_{\Lambda}$ of $B$. We denote by $L_{\Lambda}$ the restriction of $\tilde{L}_{\Lambda}$ to the open submanifold $G/T$ of $G^{\bC}/B$:
$$
\xymatrix{
G \ar[d]\ar@{^{(}->}[r] & GB \ar[d] \ar@{^{(}->}[r] & G^{\bC}\ar[d]\\
G/T \ar@{}[r]|{=} & GB/B \ar@{^{(}->}[r] & G^{\bC}/B
}
$$
and
$$
\xymatrix{
L_{\Lambda}\ar[d]\ar@{^{(}->}[r]& \tilde{L}_{\Lambda}\ar[d]\\
G/T \ar@{^{(}->}[r] & G^{\bC}/B.
}
$$
Then we can indentify the space of all holomorphic sections of $L_{\Lambda}$ with
$$
\Gamma(L_{\Lambda})= \left\{f : GB \xrightarrow{\hol}\bC ; f(xb) = \xi_{\Lambda}(b)^{-1}f(x), x \in GB, b\in B \right\}.
$$
Let $\pi_{\Lambda}$ be a representation of $G$ on $\Gamma(L_{\Lambda})$ defined by
$$
\pi_{\Lambda} (g)f(x)= f(g^{-1}x) \quad \;\text{for}\; g \in G, x \in GB\; \text{and}\; f \in \Gamma(L_{\Lambda}).
$$
For any $f \in \Gamma(L_{\Lambda})$ we define
$$
||f||^{2} = \int_{G} |f(g)|^{2}dg,
$$
where $dg$ is the Haar measure on $G$. We put
$$
\Gamma_{2}(L_{\Lambda})= \{f \in \Gamma (L_{\Lambda}); ||f|| < + \infty\}.
$$
Then the Borel-Weil theorem asserts that $(\pi_{\Lambda}, \Gamma_{2}(L_{Lambda}))$ is an irreducible unitary representations of $G$ (Bott \cite{art15-key1}, Kostant \cite{art15-key8} and Harish-Chandra \cite{art15-key2} \cite{art15-key3} \cite{art15-key4}).

For the moment we assume that $G$ is noncompact.

We fix a Cartan decomposition of $\fg$:
$$
\fg = \fk + \fp.
$$
We denote complexification of $\fk$ and $\fp$ by $\fk^{\bC}$ and $fp^{\bC}$, respectively. Let $\Delta_{c}$ and $\Delta_{n}$ denote the set of all compact roots and noncompact roots, respectively.

 Now we assume that $\Gamma_{2}(L_{\Lambda}) \neq 0$. Then there exists an ordering in the dual space of $\fh_{\bR} = i\fh$ so that every positive noncompact root os larger than every compact positive root. The ordering determines sets of compact positive roots $\Delta_{c}^{+}$ and noncompact positive roots $\Delta_{n}^{+}$. Furthermore $\Lambda$ satisfies the following two conditions: 
 \begin{align*}
\langle \Lambda, \alpha \rangle &\geq 0 \quad \text{for}\; \alpha \in \Delta_{c}^{+},\\
\langle \Lambda+ \rho, \alpha \rangle &< 0 \quad \text{for}\; \alpha \in \Delta_{n}^{+},
 \end{align*}
where $\rho = \frac{1}{2}\sum_{\alpha \in \Delta^{+}}\alpha$. Then there exists a unique element $\psi_{\Lambda}$ in $\Gamma(L_{\Lambda})$ which satisfies the following conditions:
\begin{align*}
\pi_{\Lambda}(h)\psi_{\Lambda} &= \xi_{\Lambda}(h)\psi_{\Lambda} \quad \text{for}\; h\in T,\\
d\pi_{\Lambda}(X)\psi_{\Lambda} &= 0 \qquad \text{for}\; X \in \fn^{+},\\
\psi_{\Lambda}(e) &=1, 
\end{align*}
where $d\pi_{\Lambda}$ is the complexfication of the differential representation of $\pi_{\Lambda}$. One can show that $\psi_{\Lambda}$ is an element of $\Gamma(L_{\Lambda})$. We normalize $dg$ so that $\int_{G}|\psi_{\Lambda}(g)|^{2}dg=1$.

\noindent
Define $D$ to be an open subset $\fn^{+}$ which satisfies $\exp D \cdot B=GB \cap N B$, where $\exp$ is the exponential map of $\fn^{+}$ onto $N$:
$$
\xymatrix@R=.1cm@C=.1cm{
\exp:  &  \fn^{+} \ar[rrrr]^-{\sim} &&&& N\\
       & \cup  &&&& \cup\\
       & D\ar[rrrr] &&&& \exp D.
      }
$$
For each $\alpha \in \Delta$, we choose an $E_{\alpha}$ of $\fg^{\alpha}$ such that
$$
B(E_{\alpha}, E_{-\alpha})=1
$$
and
$$
E_{\alpha}-E_{-\alpha}, \quad \sqrt{-1}(E_{\alpha} + E_{-\alpha}) \in \fg_{u},
$$
where $B(\cdot, \cdot)$ is the Killing form of $\fg^{\bC}$ and $\fg_{u}= \fk + \sqrt{-1}\fp$, the compact real form of $\fg^{\bC}$. Note that
$$
\overline{E}_{\alpha}= \left\{\begin{matrix}
-E_{-\alpha} \quad \text{for}\; \alpha \in \Delta_{c},\\
E_{-\alpha}\quad \text{for}\; \alpha \in \Delta_{n}
\end{matrix}
\right.
$$
We put $m=\dim \fn^{+}$ and introduce holomorphic coordinate on $\fn^{+}$ and $\fn^{-}$ by
\begin{align*}
&\bC^{m} \rightarrow \fn^{+},\quad (z_{\alpha})_{\alpha \in \Delta^{+}} \longmapsto z = \sum\limits_{\alpha \in \Delta^{+}}z_{\alpha}E_{\alpha},\\
&\bC^{m} \rightarrow \fn^{-},\quad (w_{\alpha})_{\alpha \in \Delta^{+}} \longmapsto w = \sum\limits_{\alpha \in \Delta^{+}}w_{\alpha}E_{-\alpha}
\end{align*}
We put
\begin{align*}
n_{z} &= \exp \sum\limits_{\alpha \in \Delta^{+}} z_{\alpha}E_{\alpha} \in N,\\
n_{w}^{-}&= \exp \sum\limits_{\alpha \in \Delta^{+}} w_{\alpha}E_{-\alpha} \in N^{-}.
\end{align*}
Let $\Gamma(D)$ be the space of all holomorphic functions on $D$. The following correspondence gives an isomorphism of $\Gamma(L_{\Lambda})$ into $\Gamma(D)$:
$$
\Phi : \Gamma(L_{\Lambda}) \longrightarrow \Gamma(D), \quad f \longmapsto F,
$$
where
$$
F(z)=f(n_{z}) \quad \text{for} \; z \in D.
$$
We put $\calH_{\Lambda}= \Phi(\Gamma_{2}(L_{\Lambda}))$. Let us denote by $U_{\Lambda}(g)$ the representation of $G$ on $\calH_{\Lambda}$ such that the diagram
%~ $$
%~ \xymatrix{
%~ \Gamma_{2}(L_{\Lambda}) \ar[d]_-{\pi_{\Lambda}(g)}\ar[r] & \calH_{\Lambda}\ar[d]_-{U_{\Lambda}(g)}\\
%~ \Gamma_{2}(L_{\Lambda}) \ar[r] \calH_{\Lambda}
%~ }
%~ $$
is commutative for all $g \in G$.

We normalize the invariant measure $\mu$ on$G/T$ such that
$$
\int_{G}f(g)dg =\int_{G/T}\left(\int_{T}f(gh)dh\right)d\mu(gT) \quad \text{for any}\; f \in C_{c}^{\infty}(G),
$$
where $dh$ is the Haar measure on $T$ such that $\int_{T}dh =1$.

We denote the measure on $D$ also by $\mu$ which is induced by the complex analytic isomorphism:
$$
\phi : D\hookrightarrow G/T.
$$
By the definition of $D, \phi(D)$ is open dense in $G/T$. For any $x \in NT^{\bC}N^{-}$ we denote the $N-, T^{\bC}-$ and $N^{-}$-component by $n(x),h(x)$ and $n^{-}(x)$, respectively. Then, for any $f \in \Gamma(L_{\Lambda}), g \in G$ and $h \in T$ we have
$$
|f(gh)|= |f(g)| \quad \text{and} \quad |\xi_{\Lambda}(h(gh))|= |\xi_{\Lambda}(h(g))|.
$$ 
This shows that $|f(g)|$ and $|\xi_{\Lambda}(h(g))|$ can be regraded as functions on $G/T$.

We put
$$
J_{\Lambda}(z) = |\xi_{\Lambda}(h(\phi(z)))|^{-2}.
$$
Then we have
\begin{align*}
\int_{G}|f(g)|^{2}dg &= \int_{G/T}|f(g)|^{2}d\mu(gT)\\
& = \int_{D}|F(z)|^{2}J_{\Lambda}(z)d\mu (z).
\end{align*}
We define
$$
\Gamma_{2}(D)= \{ F \in \Gamma (D); ||F||< +\infty\},
$$
where
$$
||F||^{2} = \int_{D}|F(z)|^{2}J_{\Lambda}(z)d\mu(z).
$$

In case that  $G$ is compact, we remark in the above that
\begin{gather*}
G=K,\quad GB-G^{\bC},\quad D =\fn^{+},\quad \fg = \fk,\quad \fp=0,\\
\Delta_{c}= \Delta,\quad \Delta_{n}= \emptyset,\quad \Gamma_{2}(L_{\Lambda})=\Gamma(L_{\Lambda}),
\end{gather*}
and $\Gamma(L_{\Lambda})\neq \{0\}$ if and only if $\Lambda$ is dominant.

For the rest of this paper we assume that $\Gamma_{2}(L_{\Lambda})\neq \{0\}$.

Suppose that $G$ is noncompact. We put
$$
\fp_{\pm}= \sum\limits_{\alpha \in \Delta_{n}^{\pm}}\fg^{\alpha}.
$$
We denote by $K^{\bC}, P_{+}$ and $P_{-}$ be the analytic subgroups of $G^{\bC}$ corresponding to $\fk^{\bC}, \fp_{+}$ and $\fp_{-}$, respectively. Then there is a unique open subset $\Omega$ of $\fp_{+}$ such that $GB=GK^{\bC}P_{-}=\exp\Omega K^{\bC}P_{-}$. We put $W=P_{+}K^{\bC}P_{-}$. Then $\psi_{\Lambda}$ is uniquely extended to a holomorphic function on $W$

Henceforth, throughout the paper, the discussions are valid for the compact case as well as for the noncompact case.

Define a scalar function $\calK_{\Lambda}$ on $ GB \times G\overline{B}$ by
$$
\calK_{\Lambda}(g_{1}, \overline{g}_{2})= \psi_{\Lambda}(g_{2}^{*}g_{1}).
$$
Then $\calK_{\Lambda}(\cdot, \overline{g}_{2})$, with $g_{2}$ fixed, can be regarded as an elemetn of $\Gamma_{2}(L_{\Lambda})$.

We define a scalar function $K_{\Lambda}$ on $D \times \overline{D}$ by
$$
K_{\Lambda}(z', \overline{z}'') = \calK_{\Lambda}(n_{z'}, \overline{n}_{z''}).
$$
Note that $K_{\Lambda}(z', \overline{z}'')$ is holomorphic in the first variable and anti-holomorphic in the second and that it can be regarded, with $n_{z''}$ fixed, as an element of $\calH_{\Lambda}$.

Now we define operators $\calK_{\Lambda}$ and $K_{\Lambda}$ on $\Gamma_{2}(L_{\Lambda})$ and $\calH_{\Lambda}$ by
$$
(\calK_{\Lambda}f)(g'')= \int_{G}\calK_{\Lambda}(g'', \overline{g}')f(g')dg' \quad \text{for}\; f\in \Gamma_{2}(L_{\Lambda})
$$
and
$$
(K_{\Lambda}F)(z'') = \int_{D}K_{\Lambda}(z'', \overline{z}')F(z')J_{\Lambda}(z')d\mu(z') \quad \text{for} \quad F\in \calH_{\Lambda},
$$
where $dg'$ is the Haar measure on $G$. Then we have the following commutative diagram:
$$
\xymatrix{
\Gamma_{2}(L_{\Lambda})\ar[r]\ar[d]_-{\calK_{\Lambda}} &\calH_{\Lambda}\ar[d]^-{K_{\Lambda}}\\
\Gamma_{2}(L_{\Lambda})\ar[r] & \calH_{\Lambda}.
}
$$

The important fact which we use in the next section is that $K_{\Lambda}$ is the indentity operator.

\section{Path Integrals}\label{art15-sec-3}
We keep the notation of the previous section.

In \cite{art15-key5}, we tried to compute path integrals for the complex polarization of $SU(1,1)$ and $SU(2)$ and we encountered the curcial difficulty of divergence of the path integrals. In \cite{art15-key6}, by taking the operator ordering into account and the regularizing the path integrals by use of the explicit form of the integrand, we computed path integrals and proved that the path integral gives the kernel fucntion of the irreducible unitary representation of $SU(1,1)$ and $SU(2)$. In \cite{art15-key7}, we gave an idea how to regularize the path integrals for complex polarizations of any connected semisimple Lie group $G$ which contains a comnpact Cartan subgroup $T$ and showed, along this idea, that the path integral gives the kernel function of the irreducible unitary representation of $G$ realized by Borel-Wiel theory.

Our idea is, in a sense, nothing but to reularize the path integral using ``normal ordering" (cf. Chapter 13 in \cite{art15-key10}) and can be explined as follows.

Put $\lambda = \sqrt{-1}\Lambda$. We extend $\lambda$ to an element of the dual space of $\fg$ which vanishes on the orthgonal complement of $\fh$ in $\fg$ with respect to the killing form. Then for any element $Y$ of the Lie algebra of $G$, the Hamiltonian on the flag manifold $G/T$ is defined by
\begin{align*}
H_{Y}(g)&=\langle \Ad^{*}(g)\lambda, Y\rangle\\
&= \sqrt{-1}\Lambda(\Ad(g^{-1})Y).
\end{align*}
Since the path integral of this Hamiltonian is divergent we regularize it by replacing
\begin{align*}
e^{\sqrt{-1}H_{Y}(g)}&= e^{\Lambda(H(\Ad(g^{-1})Y))}\\
& = \xi_{\Lambda}(\exp(H(\Ad(g^{-1})Y)))
\end{align*}
by
$$
\xi_{\Lambda}(h(\exp(\Ad(g^{-1})Y))),
$$
where$H$ and $h$ denote the projection operators:
\begin{gather*}
H : \fn^{+} + \fh^{\bC}+ fn^{-}\longrightarrow \fh^{\bC},\\
h : \exp \fn^{+} \exp \fh^{\bC} \exp \fn^{-} \longrightarrow \exp \fh^{\bC} =T^{\bC}.
\end{gather*}

\medskip
\noindent
{\bfseries Remark \thnum{1.} \label{art15-remark-1}} For simplicity we assume that $G$ is realized by a linear group. For any $X\in \fg^{\bC}$, we decompose $X=X_{+}+X_{0}+X_{-}$, where $X_{+}\in \fn^{+}, X_{0} \in \fh^{\bC} \;and \; x_{-} \in \fn^{-}$. We define the ``normal ordering" : : by the rule such that the elements in $\fn^{+}$ appear in teh left, the elements in $\fh^{\bC}$ in the middle and the elements in $\fn^{-}$ in the right. Then we have
$$
: \exp (X_{+} + X_{0} + X_{-}): =\exp X_{-}\exp X_{0} \exp x_{-}.
$$ 
For any $X \in \fg^{\bC}$ we define $\xi_{\Lambda}(\exp X) = e^{\Lambda(X)}$. Then the above regularization means that we replace
$$
\xi_{\Lambda}(\exp(X_{+}+ X_{0}+ X_{-}))
$$
by
$$
\xi_{\Lambda}(\exp x_{-})\xi_{\Lambda}(\exp X_{0})\xi_{\Lambda}(\exp X_{-}).
$$
Before we start computing path integrals on the flag manifold $G/T$ we prepare several lemmas.

\medskip
\noindent
{\bfseries Lemma \thnum{2.} \label{art15-lemma-2}} \textit{For any $g \in N B$ we decompose it as}
$$
g =n_{z}n_{w}^{-}t \quad where \quad n_{z} \in N, n_{w}^{-} \in N^{-}, t \in T^{\bC}.
$$
\textit{Suppose that $g \in G \cap N B$. Then we can express $w$ in terms of $z$ and $\overline{z}$ which we denote $w$ by $w(z,\overline{z})$. And we have}
$$
K_{\Lambda}(z, \overline{z})= \xi_{\Lambda}(t^{*}t) \quad and \quad J_{\Lambda}(z)=K_{\Lambda}(z, \overline{z})^{-1}.
$$

For any $z \in D$, we put $g(z, \overline{z})=n_{z}n_{w(z, \overline{z})}^{-}$.

Let $d$ denote the exterior derivative on $D$. We decompose it as $d=\partial + \overline{\partial}$, where $\partial$ and $\overline{\partial}$ are holomorphic part and and anti-holomorphic part of $d$, respectively.

Define
$$
\theta= \lambda(g^{-1} dg)= \lambda(n_{w}^{-1} n_{z}^{-1}\partial n_{z}n_{w}^{-}) + \lambda(t^{-1}dt),
$$
where $g=n_{z}n_{w}^{-}t \in G$. And we choose
$$
\alpha = \lambda(n_{w}^{-^{-1}}n_{z}^{-1}\partial n_{z}n_{w}^{-}).
$$
For any $Y \in fg$, the Hamiltonian functions is given by
$$
H_{Y}(g)= \langle \Ad^{*}(g)\lambda, Y\rangle= \langle\Ad^{*}(g(z, \overline{z}))\lambda, Y\rangle,
$$
where $g=g(z, \overline{z})\in G$.

If we decompose $g = n_{z}n_{w}^{-}t \in G \cap N B$ as in Lemma \ref{art15-lemma-2}, then we can show that
\begin{align*}
\Lambda(n_{w}^{-^{-1}}n_{z}^{-1}\partial n_{z}n_{w}^{-}) &= -\Lambda((tt^{*})^{-1}\partial(tt^{*}))\\
&= -\partial \log K_{\Lambda}(z, \overline{z}).
\end{align*}
It follows that $\alpha = -\sqrt{-1}\partial \log K_{\Lambda}(z, \overline{z})$.

Now we consider the Hamiltonian part of the action. Let
$$
K_{\overline{w}}(z)=K_{\Lambda}(z, \overline{w})
$$
and regard it as an element of $\calH_{\Lambda}$.
