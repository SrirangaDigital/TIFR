\title{Numerically Effective line bundles which are not ample}
\markright{Numerically Effective line bundles which are not ample}

\author{By~V. B. Mehta and S. Subramanian}
\markboth{By~V. B. Mehta and S. Subramanian}{Numerically Effective line bundles which are not ample}


\date{}
\maketitle

\section{Introduction}\label{art13-sec-1}
In \cite{art13-key6}, there is a\pageoriginale construction of a line bundle on a complex projective nonsingular variety which is ample on very propersubvariety but which is nonnample on the ambient variety. The example is obtained as the projective bundle associated to a ``general" stable vector bundle of degree zero on a compact Riemann surface of genus $g \geq 2$. Now,w e can show by an an algebraic argument valid in any characteristic, the existence of a variety of dimension $\leq 3$ with a line bundle as above. The details of the proof will appear elsewhere.

\section{}\label{art13-sec-2}
Let $C$ be a complete  nonsingular curve defined over an uncountable algerbaically closed field (of any characteristic). Let $M_{r}^{s}$ denote the moduli space of stable bundles of rank $r$ and gegree zero on $C$ and $M_{r}^{ss}$ the moduli of semistable bundles of rank $r$ and degree zero. We assume throught that the curve $C$ is ordinary. We can show

\medskip
\noindent
{\bfseries \thnum{2.1} Proposition. \label{art13-prop-2.1}} \textit{Let the characteristic of the ground field be positive and $F$ the frobenius morphism on $C$. There is a proper closed subset of $M_{r}^{*}$ such that for any stable bundle $V$ in the complement of this closed set, $F^{*}V$ is also stable.}

\medskip
\noindent
{\bfseries Proof.} We use Artin's theorem on the algrbraisation of formal moduli space for proving the above proposition. We have as a corollary to the proof of Proposition \eqref{art13-prop-2.1}.

\medskip
\noindent
{\bfseries \thnum{2.1.1} Corollary. \label{art13-coro-2.1.1}} \textit{Let $C$ be an oridinary curve. Then the rational map $f: M_{r}^{ss} \rightarrow M_{r}^{ss}$ induces by the Frobenius $F: C \rightarrow C$, is etale on an open set, and in particular, dominant.}


\medskip
\noindent
{\bfseries \thnum{2.1.2} Corollary. \label{art13-coro-2.1.2}} \textit{Let $C$ be an ordinary curve, For any positive integer $k$, there is a nonempty open subset of $M_{r}^{s}$ such that for $V$ in this open set, $F^{m*}V$ is stable for
$1 \leq m \leq k$. }

\medskip
\noindent
{\bfseries Proof.} We apply Proposition \eqref{art13-prop-2.1} and Corollary \eqref{art13-coro-2.1.1} successively.

\hfill Q.E.D.

We have

\medskip
\noindent
{\bfseries \thnum{2.2} Proposition. \label{art13-prop-2.2}} \textit{Given a finite etale morphism $p : C_{1} \rightarrow C$, there eixsts a proper closed subset of $M_{r}^{s}$ such that any vector bundle in the complement of this closed set remains  stable on $C_{1}$.} 

We have

\medskip
\noindent
{\bfseries \thnum{2.3} Proposition. \label{art13-prop-2.3}} \textit{For a fixed positive ieteger $k$, there is a nonempty open subset of $M_{r}^{s}$  such for any stable bundle $V$  in this open set, there is no nonzero homomorphism from a line bundle of degree zero to $S^{k}V$.}

\textit{Using the above results, we can show
}

\medskip
\noindent
{\bfseries \thnum{2.4} Theorem. \label{art13-thm-2.4}} \textit{Let $C$ be nonsigular ordinary curve of genus $\geq 2$ over an uncountable algebrically closed field (of any characteristic). There is a dense subset of $M_{r}^{s}$ such that for any stable bundle $V$ in this dense set, we have}
\begin{enumerate}[\it 1)]
\item $F^{k^{*}}(V)$\textit{is stable for all} $K \geq 1$.\label{art13-thm2.4-enum-1}

\item \textit{For any separable finite morphism} $\pi : \tilde{C} \rightarrow C, \pi^{*}(V)$ \textit{is stable}.\label{art13-thm2.4-enum-2}

\item \textit{There is no nonzero homomorphism from a line bundle of degree zero to the symmetric power} $S^{k}(V)$
 \textit{for any} $K \geq 1$.\label{art13-thm2.4-enum-3}
\end{enumerate}

\medskip
\noindent
{\bfseries Remark.} If $C$ is a smooth curve defined over a finite field (of characteristic $p$) then any continuous irreducible represenation $\rho : \pi_{1}^{alg}(C)\rightarrow SL(r, \overline{\bF}_{p})$ of the algebraic fundamental group of $C$ of rank $r$ over the finite field defines a stable vector bundle $V$ on $C$ such that $F^{m*}V \simeq V$ for some $m \leq 1$ (see \cite{art13-key4}). Such a bundle $V$ statisfies $F^{k*}(V)$ is stable for all $K \geq 1$. We can construct such representations for any curve $C$ of genus $g \geq 2$ when $r$ is coprime to $p$, and for an ordinary curve $C$ when $p$ divides $r$.


\section{}\label{art13-sec-3}
Let $C$ be a nonsingular ordinary curve of genus $\geq 2$, and $V$ a stable vector bundle of rank 3 degree zero on $C$ satisfying the conditions of Theorem \eqref{art13-thm-2.4} above. Let $\pi : P(V)\rightarrow C$ be the projective bundle associated to $V$ and $L=\calO_{\bP (V)}(1)$ the universal line bundle on $\bP(V)$. Then we have 

\medskip
\noindent
{\bfseries \thnum{3.1} Theorem. \label{art13-thm-3.1}} \textit{The line bundle $L$ is ample on very proper subvariety of $\bP(V)$, but $L$ is not ample on $\bP(V)$.}

\medskip
\noindent
{\bfseries Proof.} We can check that $L.C > 0$ for any integral curve $C$, and that $L^{2}$. $D >0$ for any irreducible divisor $D$. This implies that $L | D$ is ample on $D$. This shows that $L$ is ample on divisors in $\bP(V)$ and hence on any proper subvariety of $\bP(V)$. Also, $L$, is not ample on $\bP(V)$. \hfill Q.E.D. 

\medskip
\noindent
{\bfseries \thnum{3.2} Remark. \label{art13-thm-3.2}} The case $r=2$ is covered by the first part of Theorem
\eqref{art13-thm-3.1}.

\begin{thebibliography}{99}
\bibitem{art13-key1} M. Artin,\textit{The implicit function theorem in  Algebraic Geometry}, Proceedings International Colloquium on Algebraic Geometry, Bombay 1968, Oxford University Press (1969) 13-34.

\bibitem{art13-key2} T. Fujita, \textit{Semipositive line bundles}, Jour. of Fac. Sc. Univ. of Tokyo, Vol. {\bf 30} (1983) 353-378.

\bibitem{art13-key3} R. Hartshorne, \textit{Ample subvarieties of algebraic varieties}, Lecture Notes No. {\bf 156} (Springer-Verlag) (1970)

\bibitem{art13-key4} H. Lange and U. Stuhler, \textit{Vecktor bundel and Kurven und Darstellungender algebraisechen Fundamental gruppe}, Math. z. {\bf 156} (1977) 73-83.

\bibitem{art13-key5} V. B. Mehta and A. Ramanathan, \textit{Restriction of stable sheaves and representations of the fundamental group}, Inv. Math {\bf 77} (1984) 163-172.

\bibitem{art13-key6} S. Subramanian, \textit{Mumford's example and a general construction}, Proc. Ind. Acad. of Sci. Vol. {\bf 99}, December (1989) 97-208.
\end{thebibliography}

\begin{flushleft}
School of Mathematcs

Tata Institute of Fundamental Research

Homo Bhabha Road, Colaba

Bombay 400 005

INDIA.
\end{flushleft}
