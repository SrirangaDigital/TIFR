\title{Bases for Quantum Demazure modules-I}
\markright{Bases for Quantum Demazure modules-I}

\author{By~ V. Lakshmibai\footnote{Partially supported by NSF Grant DMs 9103129 and alos by the Faculty Development Fund of Northeastern University.}}
\markboth{V. Lakshmibai}{Bases for Quantum Demazure modules-I}

\date{}
\maketitle

\begin{center}
(Dedicated to professores M. S. Narasimhan and C.S. Seshadri on their 60th birthdays)
\end{center}

\section{Introduction}\label{art9-sec-1}
Let $\fg$ be a semi-simple Lie algebra over $\bbQ$ of rank $n$. Let $U$ be the quantized enveloping algebra of $\fg$ as constructed by Drinfeld (cf \cite{art9-keyD}) and Jimbo (cf \cite{art9-keyJ}). This is an algebra over $\bbQ(v)$ ($v$ being a parameter)  which specializes to $U(\fg)$ for $v=1$, $u(g)$ being the universal enveloping algebra fo $\fg$. This algebra has agenerators $E_{i}, F_{i}, K_{i}, 1\leq i \leq n$, which satisfy \textit{the quantum Chevalley and Serre relations} (cf \cite{art9-keyL1}). Let $A= \bbZ[v, v^{-1}] $ and $U_{A}^{\pm}$ be the A-sumbalagebra of $U$ generated by $E_{i}^{r}$ (resp. $F_{i}^{r}$), $1\leq i \leq n$, $r \in \bbZ^{+}$, (here $E_{i}^{r}, F_{i}^{r}$ are the quantum divided powers (cf \cite{art9-keyJ})). Let $d=(d_{1}, ldot, d_{n}) \in (\bbZ^{+})^{n}$ and $V_{d}$ be the simple $U$-module with a non-zero vector e such that $E_{i}e=0$, $K_{i}e =v^{-d_{i}}e$ (recall that $V_{d}$ is unique up to isomorphism). Let us denote $V_{d}$ by just $V$. Let $W$ be the Weyl group of $\fg$.

Let $w \in W$ and let $w=s_{i_{1}}\ldots s_{i_{r}}$ be a reduced expression for $w$. Let $U_{w,A}^{-}$ denote the $A$-submodule of $U$ spanned by $F_{i_{1}}^{(a_{1})}\ldots F_{i_{r}}^{ar}, a_{i}\in \bbZ^{+}$. We observe (\cite{art9-keyL4}) that $U_{w,4}^{-}$ depends only on $w$ and not onth reduced expression chosen. For $w\in W$, let $V_{w,A}=U_{w,A}^{-}e$. We shall refer to $V_{w, A}$ as the \textit{Quantum Demazure module associated with $w$.} Let $w_{0}$ be the unique element in $W$ of maximal lenght. Then $V_{w_{0},A}$ is simply $U_{A}^{-}e$. In the sequel, we shall denote $V_{w_{0}, A}$ by just $V_{A}$.

Let $g=s\ell$ {(\bf 3)}. In this paper, we construct an A-basis for $V_{A}$, which is compatible with $\{V_{w,A}, w \in W\}$. The construction is done using the configuration of Schubert varieties in the Flag variety $G/B$. Let $Id=\mu_{0} < \mu_{1} < \mu_{2} < \mu_{3} = w_{0}$ be a chain  in $W(=S_{3})$. Let $\mu_{i-1} = s_{\beta_{i}}\mu_{i}$ for some positive root $\beta_{i}$. Let $n(\mu_{i-1},\mu_{i})= (\mu_{i-1}(\lambda), \beta_{i}^{*})$, where $\lambda = d_{1}\omega_{1} + d_{2}\omega_{2}, \omega_{1}$ and $\omega_{2}$ being the fundamental weights of $s\ell(3)$. Let denote $n(\mu_{i-1}, \mu_{i})$ by just $n_{i}$. Let $C = \{ (\mu_{0}, \mu_{1}, \mu_{2}, \mu_{3};m_{1}, m_{2}, m_{3}) : 1 \geq \frac{m_{1}}{n_{1}}\geq \frac{m_{2}}{n_{2}} \geq \frac{m_{3}}{n_{3}} \geq 0 \}$. Given $c=(\mu_{0}, \mu_{1}, \mu_{2}, \mu_{3};m_{1}, m_{2}, m_{3})$, Let $\tau_{c} = \mu_{r}$, where $r$ is the largest integer such that $m_{r}\neq 0$. Given two elements
$$
c_{1} = \{(\mu_{0}, \mu_{1}, \mu_{2}, \mu_{3};m_{1}, m_{2}, m_{3})\}, c_{2}=\{ \lambda_{0}, \lambda_{1}, \lambda_{2}, \lambda_{3} ; p_{1}, p_{2},p_{3}\}
$$
in $C$. let us denote $\frac{m_{i}}{n(\mu_{i-1},\mu_{i})}$ (resp. $\frac{p_{i}}{n(\lambda_{i-1}, \lambda_{i})}$) by just $a_{i}$ (resp. $b_{i}$). We say $c_{1}\sim c_{2}$, if
\begin{enumerate}[(1)]
\item $a_{i}= b_{i}$
\item either
   \begin{enumerate}[(a)]
    \item $a_{1} =a_{2} > a_{3}, \mu_{2} = \lambda_{2}$

    or 

    \item $a_{1} > a_{2} =a_{3}, \mu_{1} =\lambda_{1}$

    or

    \item $a_{1} = a_{2} = a_{3} $
    \end{enumerate}
\end{enumerate}
We shall denote $C/\sim $ by $\overline{C}$, For $\theta \in \overline{C}$, we shall denote $\tau_{\theta} = \tau{\theta}, c$ being a representative for $\theta$ (note that $\tau_{\theta}$ is well-defined). We have (Theorems \ref{art9-thm-6.7} and \ref{art9-thm-7.2})

\begin{theorem}
$V_{A}$ has an $A$-basis $B_{d}=\{v_{\theta}, \theta \in \overline{C}\}$ where $v_{\theta} = D_{\theta}e, D_{\theta}$
being a monomial in $F_{i}^{'}s$ of the form $F_{i_{1}}^{(n_{1})}\ldots f_{i_{r}}^{(nr)}$. Further, for
$w \in W, \{v_{\theta} | w \geq \tau_{\theta}\}$ is an $A$-basis for $V_{w, A}$.
 \end{theorem}

Let $\calB_{d}$ denote Lusztig's canonical basis for $V_{A}$ (cf \cite{art9-keyL2}). It turns out that the transition matrix from $B_{d}$ to $\calB_{d}$ is upper triangular. We also give a conjectural $A$-basis $B_{d}$ for $V_{A}$ for $\fg$ of other types. An element in $B_{d}$ is again of the form $F_{i_{1}}^{(n_{1})}\cdots F_{i_{r}}^{(nr)}e$. We conjecture that the trasition matrix from $B_{d}$ to $\calB_{d}$ is upper triangular. The sections are organized as follows. In \S2,
we recall some results pertaining to the configuration of Schubert varieties in $G/B$. In \S3, we describe a conjectural $A$-basis  for $V_{A}$ (and also for $V_{w, A}$). In \S4, we study $\overline{C}$ in detail and constuct $B_{d}$ and $\calB_{d}$. In \S6 and \S7, we prove the results for $G=SL(3)$. in \S8, an Appendix, we have explicitly established a bijection between the elements of $\overline{C}$ (the indexing set for $B_{d}$) and the classical standard Young tableaux on $SL(3)$ of type $(d_{1}, d_{2})$.

The author would like to express her gratitude to SPIC Science Foundation for the hospitality extended to het during her stay (Jan-Feb, 1992) there, when research pertatining to this paper was carried out.

\section{Preliminaries}\label{art9-sec-2}

Let $G$ be a semi-simple, simply connected Chevalley group defined over a field $k$. Let $T$ be a maximal $k$-split torus, $B$ a Borel subgroup$, B\supset T$. Let $W$ be the Weyl group, and $R$ the root system of $G$ relative to $T$ . Let $R^{+}$ (resp. $S$) be the system of positive (resp. simple) roots of $G$ relative to $B$. For $w \in W$, let $X(w) = \overline{BwB}(\mod B)$ be tghe Schubert variety in $G/B$ associated with $w$.

\begin{definition}\label{art9-definition-2.1}
Let $X(\varphi)$ be a Schubert divisor in $X(\tau)$. We say tht $X(\varphi)$ is \textit{moving divisor in} $X(\tau)$ \textit{moved by the simple root} $\alpha$, if $\varphi=s_{\alpha} \tau$.
\end{definition}
