\title{Moduli Spaces of Abelian Surfaces with Isogeny\footnote{Supported by DFG-contract La 318/4 and EC-contract SC1-0398-C(A)}}
\markright{Moduli Spaces of Abelian Surfaces with Isogeny}

\author{By~Ch. Birkenhake and H. Lange}
\markboth{By~Ch. Birkenhake and H. Lange}{Moduli Spaces of Abelian Surfaces with Isogeny}


\date{}
\maketitle

\begin{center}
To M.S. Narasimhan and C.S Seshadri on the occasion of their 60th birthdays
\end{center}

\noindent
Let $(X, L)$ be as polarized abelian surface or type $(1,n)$. An isogeny of type $(1,n)$ is an isogeny of polarized abelian surfaces $\pi : (X, L)\rightarrow (Y, P)$ such that $P$ defines a principle polarization on $Y$.
According to \cite{art11-keyH-W} the coarse moduli space $\calA_{1,n}$ of such triplets $(X, L, \pi)$ exists and is analytically isomorphic to the quotient of the Siegel upper half space of degree 2 by the action of $\Gamma = \{M \in Sp_{4}(Z): M= (m_{ij}) \;\text{with}\; n|m_{i4}, i= 1,2,3\}$. $\calA_{(1,n)}$ is a finite covering of themodulinpsace of principally ploarixed abelina surface as well as of the moduli space of polarized abelian surface of type $(1,n)$. On the other hand, the moduli space of polarized abelian surfaces with level $n$-structure is as finite covering of $\calA_{(1,n)}$. If for example $n$ is a prime, the degrees of these coverings are $(n+1)(n^{2}+1), (n+1)$ and $ n(n-1)$ respectively

The aim of the present papert is to give explicit algebraic descriptions of the moduli spaces $\calA_{1,2}$ (see Theorem \ref{art11-thm-3.1}) and $\calA_{1,3}$ (see Theorem \ref{art11-thm-6.1}). An immediate consequence is that the moduli spaces $\calA_{1,2}$ and $\calA_{1,3}$ are rational.

An essential ingredient of the proof is the fact that the moduli space $\calA_{1,n}$ is canonically isomorphic to the moduli space $\calC_{2}^{n}$ of cyclic \'etale $n$-fold coverings of curves of genus 2. This will be shown
in Section \ref{art11-sec-1}. The second important tool is the fact that the composition of every $C \rightarrow H$ in $\calC_{2}^{n}$ with the hyperlliptic covering $H \rightarrow P_{1}$ is Galois with the dihedral group $D_{n}$ as Galois group (see Section \ref{art11-sec-2}). Finally we need some results on duality of polarizations on abelian surfaces which we compile in Section \ref{art11-sec-4}.

We would like to thank W. Barth and W.D. Geyer for some valuable discussions.

\section{Abelian Surface with an Isogeny of Type $(1,n)$}\label{art11-sec-1}

In this section we show that there is a canonical isomorphism between the moduli space of polarized abelian surfaces with isogeny of type $(1,n)$ and the moduli space of cyclic \'etale $n$-fold coverings of curves of genus two.

Let $X$ be an abelian surface over the field of complex numbers. Any ample line bundle $L$ on $X$ defines a polarization on $X$. In the notation we do not distinguish between the line bundle $L$ and the corresponding polarization. Denote by $\widehat{X} = \Pic^{0}(X)$ the dual abelian variety. The polarization $L$ determines an isogeny
$$
\phi_{L}:x \rightarrow \widehat{X}, \quad x \mapsto t_{x}^{*}L \otimes L^{-1}
$$
where $t_{x}:X \rightarrow X$ is the translation map $y \map y +x$. The kernel $K(L)$ of $\varphi_{L}$ is isomorphic to $ (Z/n_{1} Z \times Z/n_{2}Z)^{2}$ for some positive integers $n_{1},n_{2}$ with $n_{1} |\; n_{2}$. We call $(n_{1}, n_{2})$ the \textit{type of the polarization}. Any polarization of type $(n_{1}, n_{2})$ is the $n_{1}$-th power of a unique polarizations of type $(1, \frac{n_{2}}{n_{1}})$. Hence for moduli problems it suffices to consider polarizations of type $(1,n)$.  

From now on let $L$ be a line bundle defining a polarization of type $(1,n)$. An \textit{isogeny of type} $(1,n)$ is by definition an isogeny of polarized abelian varieties $\pi : (X, L) \rightarrow (Y, P)$ whose kernel is cyclic of order $n$. Necessarily $P$m defines a principal polarization on $Y$ and $\ker p$ is contained in $K(L)$. Conversely, according to \cite{art11-keyL-B} Cor. \ref{art11-6.3.5} any cyclic subgroup of $K(L)$ of order $n$ defines an isogeny of type $(1,n)$ of $(X,L)$. In particular, if $n$ is a prime number, then $(X, L)$ admits exactly $n+1$ isogenies of type $(1,n)$. According to \cite{art11-keyL-B} Exercise \ref{art11-8.4} the moduli space $\calA_{1,n}$  of polarzed abelian sufaces with isgeny of type $(1,n)$ exists and is analytically isomorphic to the quotient of the Siegel upper half space $h_{2}$ of degree 2 by the group $\{ M \in Sp_{4}(Z) | M=(m_{ij})\; \text{with} \; n|m_{i4}, i=1,2,\}$.

In the sequel a curve of genus two means either a smooth projective curve of genus 2 or a union of two elliptic curves intersecting transversally at the origin. Note that such a union $E_{1}+ E_{2}$ is of arithmetic genus 2. Torelli's Theorem implies that the moduli space of principally polarized abelian surfaces can be considered as a moduli space for curves of genus two in this sense.

Let $f: C\rightarrow H$ be a cyclic \'etale covering of degree $n$ of a curve $H$ of genus 2. According to Hurwitz'formula formular $C$ has arithmetic genus $n+1$ . Every line bundle $l \in \Pic^{0}(H)$ of order $n$ determines such a
cyclic \'etale covering $f: C\rightarrow H$ (for an explicit description of the  covering see Section \ref{art11-sec-2}). Two such line bundles lead to the same covering, if they generate the same group in $\Pic^{0}(H)$. This implies that the (coarse) moduli space $\calC_{2}^{n}$ of cyclic \'etale $n$-fold coverings of curves of genus two is a finite covering of the moduli space $\calM_{2}$ of curves of genus two. In particular $\calC_{2}^{n}$ is an algebraic variety of dimension 3. The moduli spaces $\calA_{1,n}$ and $\calC_{2}^{n}$ are related as follows.

\medskip
\noindent
{\bfseries \thnum{1.1} Propostion\label{art11-Proposition-1.1}} \textit{There is a canonical biholomorphic map} $\calA_{(1,n)} \rightarrow \calC_{2}^{n}$

There seems to be no explicit construction of the moduli space $\calC_{2}^{n}$ in the literature. One could also interprete Proposition \ref{art11-Proposition-1.1} as a construction of $\calC_{2}^{n}$. However it is not difficult to show its existence in a different way and thus the proposition makes sense as stated.

\medskip
\noindent
{\bfseries Step I: The map $\calA_{1,n}\rightarrow \calC_{2}^{n}$.} Let $\pi : (X, L)\rightarrow (Y,P)$ be an isogeny of type $(1,n)$. We may assume that $\pi^{*} P\simeq L$ as line bundles. Since $(Y, P)$ is a principally polarized abelian surface there is curve $H$ of genus 2 (in above sense) such that $Y=J(H)$, the Jacobian pf $H$, and $P\simeq\calO_{Y}(H)$. Note that for $H=E_{1} + E_{2}$ with elliptic curves $E_{1}$ and $E_{2}$, $J(H) = \Pic^{0}(H) \simeq E_{1}\times E_{2}$. By assumption $C : \pi^{-1}H \in |L|$. The \'etale covering $\pi : X \rightarrow Y$ is given by a line bundle $l \in \Pic^{0}(Y)$ of order $n$ and the coverin $\pi |C:C \rightarrow H$ corresponds to $l|H$. Since the restriction map $\Pic^{0}(Y) \xrightarrow{\sim} \Pic^{0}(H)$ is an isomorphism, the line bundle $l|H$ is of order $n$ and thus $\pi|C:C \rightarrow H$ is an element of $\calC_{2}^{n}$.

\medskip
\noindent
{\bfseries Step II: The inverse map $\calA_{1,n}\rightarrow \calC_{2}^{n}$.} 
Let $f: C\rightarrow H$ be a cyclic \'etale covering in $\calC_{2}^{n}$ associated to the line bundle $l_{H} \in \Pic^{0}(H)$. Via the isomorphism $\Pic^{0}(J(H)) \xrightarrow{\sim} \Pic^{0}(H)$ the line bundle $l_{H}$ extends to a line bundle $l \in \Pic^{0}(J(H))$ of order $n$. Let $\pi : X \rightarrow Y = J(H)$ denote the cyclic \'eta;e $n$-fold covering associated to $l$. Then $L=\pi^{*}\calO_{Y}(H)$ defines a polarization of type $(1,n)$, since $K(L)$ is a finite group of order $n^{2}$ (by Riemann-Roch) and contains the cyclic group $\ker \pi$ of order $n$. Hence $\pi : (X, L) \rightarrow (Y, \calO_{Y}(H))$ is an element of $\calA_{(1,n)}$.

Obviously the maps $\calA_{(1,n)}\rightarrow \calC_{2}^{n}$ and $\cal_{2}^{n} \rightarrow \calA_{(1,n)}$ are inverse to each other. Finally, extending the above construction to families of morphisms of curves and abelian varieties one easily sees that the maps are holomorphic.

\section{Cycli \'Etale Coverings of Hyperelliptic Curves}\label{art11-sec-2}

Any curve $H$ of genus 2 (in the sense of section 1) admits a natural involution $\iota$ with quotient $H/\iota$ of arithmetic genus 0. The aim of this section is to show
