\title{Poncelet Polygons and the Painlev\'e Equations}
\markright{Poncelet Polygons and the Painlev\'e Equations}

\author{By~ N. J. Hitchin}
\markboth{N. J. Hitchin}{Poncelet Polygons and the Painlev\'e Equations}

\date{}
\maketitle

\begin{center}
Dedicated to M.S. Narasimhan and C. S. Seshadri on the occasion of their 60th birthdays
\end{center}

\section{Introduction}\label{art7-sec-1}
The celebrated theorem of Narasimhan and Seshadri \cite{art7-key13} relating stable vector bundles on a curve to unitary representations of its fundamental group has been the model for an enormous range of recent results intertwining algebraic geometry and topology. The object which meditates between the two areas geometry and topology. The object which mediates between the two areas in all of these generalizations is the notion of a \textit{connection}, and existence Theorems for various types of connection provide the means of establishing the theorems. In one sense, the motivation for this paper is to pass beyond the existence and demand more explicitness. What do the connections look like ? Can we write them down? This question is our point of departure. The novelty of our presentation here is that the answer involves a journey which takes us backwards in time over two hundred years form the proof of Narasimhan and Seshadri,s theorem in 1965.

For simplicity, instead of considering stable bundles on curves of higher genus we consider the analogous case of parabolically stable bundles, in the sense of Mehta and Seshadri \cite{art6-key11}, on the complex projective line $\bC\bP^{1}$. Such a bundle consists of a vector bundle with a weighted flag structure at $n$ marked points
$a_{1}, \ldots, a_{n}$. The unitary connection that is associated with it is flat and has singularities at the points. In the generic case, the vector bundle itself is trivial, and the flat connection we are looking for can be written as a meromorphic $m \times m$ matrix-valued 1-form with a simple pole at each point $a_{i}$. The parabolic structure can easily be read off form the residuces of the form. The other side of the equation is a representation of the fundamental group $\pi_{1}(\bC\bP^{1}\ \{a_{1}, ldots, a_{n}\})$ in $U(m)$. the holonomy of the connection, and this presents more problems. Such questions occupied the attention of Fuchs, K-lein and others in the last century under the alternative name of monodromy of ordinary differential equations. Now if we fix the holonomy, and ask for the corresponding 1-form for each set of distinct points $\{a_{1}, \ldots a_{n}\}\subset \bC\bP^{1}$, what in fact we are asking for is a solution of a differential equation, the so-called Schlesinger equation (1912) of isomonodromic deformation theory. To focus things even more, in the simple case where $m=2$ and $n =4$, and explicit form for the connection demands a knowledge of solutions to a single nonlinear second order differential equation. This equation, originally found in the context of isomonodromic deformations by $R$. Fuchs in 1907 \cite{art6-key4}, is nowadays called Painlev\'e's 6th equation
\begin{equation*}
\begin{split}
\dfrac{d^{2}y}{dx^{2}} & = 1/2 \left(\dfrac{1}{y} + \dfrac{1}{y-1} + \dfrac{1}{y-x}\right)\left(\dfrac{dy}{dx}\right)^{2} - \left(\dfrac{1}{x} + \dfrac{1}{x-1} + \dfrac{1}{y-x}\right) \dfrac{dy}{dx} \\
 &+ \dfrac{y(y-1)(y-x)}{x^{2}(x-1)^{2}} \left(\alpha + \beta\dfrac{x}{y^{2}} + \gamma\dfrac{x-1}{(y-1)^{2}} + \delta\dfrac{x(x-1)}{(y-x)^{2}}\right)
\end{split}
\end{equation*}
and in the words of Painlev\'e, the general solutions of this equation are ``transcendantes essentiellement nouvelles"  That, on the face of it, would seem to be the end of the quest for explicitness-we are faced with the insuperable obstacle of Painlev\'e transcendants.

Notwithstanding Painlev\'e's statement, for certain values of the constants $\alpha, \beta, \gamma, \delta$, there do exist solutions to the equation which can be written down, and even solutions that are \textit{algebraic}. One property of any solution to the above equation is that $y(x)$ can only have branch points at $x = 0,1,, \infty$. This is essentially the ``Painlev\'e property", that there are no movable singularities. If we find an algebraic solution, then this means we have an algebraic curve with a map to $\bC\bP^{1}$ with only three critical values. Such a curve has a number of special properties. On the one hand, it is defined by a subgroup of finite index in $\Gamma(2) \subset SL(2, \bZ)$, and also,by a well-known theorem of Weil,is defined over $\overline{\bQ}$. In this paper we shal construct solutions by considering the case when the holonomy group $\Gamma$ of the connection is \textit{finite}. In that case the solution $y(x)$ to the Painlev\'e equation is algebraic.

Our approach here is to consider, for a finite subgroup $\Gamma$ of $SL(X, \bC)$, the quotient space $SL(2 ,\bC)/\Gamma$ and an equivariant compactification $Z$. Thus $Z$ is a smooth projective threefold with an action of $SL(2, \bC)$ and a dense open orbit. The Maurer-Cartan form defines a flat connection on $SL(2, \bC)/\Gamma$ with holonomy $\Gamma$, which extends to a meromophic connection on $Z$. The idea is then to look for rational curves in $Z$ such taht the induced connection is of the required form. By construction the holonomy is $\Gamma$, and if we can find enough curves to vary the cross-ratio of the singular points $a_{1}, \ldots a_{4}$, then we have a solution to the Painlev\'e equation. The question of finding and classifying such equivarian compactifications has been addressed by Umemura and Mukai \cite{art6-key12}, but here we focus on one particular case. We take $\Gamma$ to be the binary dihedral group $\tilde{D}_{k} \subset SU(2)$. This might seem very restrictive within the context of parabolically stable bundales, but behind it there hides a very rich seam of algebraic geometry which has its origins further back in history than Painlev\'e.

In the case of the dihedral group, the construction of a suitable compactification is due to Schwarzenberger \cite{art6-key16}, who constructed a family of rank 2 vector bundles $V_{k}$ over $\bC\bP^{2}$. THe threefold corresponding to the dihedral group $D_{k}$ turns out to be the projectivizesd bundle $P(V_{k})$. There are two types of relevant rational curves. Those which project to a line in $\bC\bP^{2}$ yield the solution $y =sqrt{x}$ to the Painlev\'e equation with coefficients $(\alpha, beta, \gamma, \delta) = (1/8, -1/8, 1/2k^{2}, 1/2-1/2k^{2})$. Those which project to a conic lead naturally to another problem, and this one goes back at least to 1746 (see \cite{art7-key3}). It is the problem of \textit{Poncelet polygons}. We seek conics $B$ and $C$ in the plane such that there is a $k$-sided polygon inscribed in  $C$ and circumscribed about $B$. Interest in this problem is still widespread. Poncelet polygons occur in questions of stable bundales on projectives spaces\cite{art7-key14} and more recently in the workl of Barth and Michel \cite{art7-key1}. In fact, we can use their approach to find the modular curve giving the algebraic solution $y(x)$ of the Painlev\'e equation corresponding to $\Gamma = \tilde{D}_{k}$. This satisfies Painlev\'e equation with coefficients $(\alpha, \beta, \gamma, \delta) = (1/8, -1/8, 1/8, 3/8)$. It is essentially Cayley's solution in 1853 of the Poncelet problem which allows us to go further and produce explicit solutions. It is method which fits in well with the isomonodromic approach. 

There are a number of reasons why this is a fruitful area of study. One of them concerns solutions of Painlev\'e equation in general and their relation to integrable systems, another is the connection with self-dual Einstein metrics as discussed in \cite{art7-key6}. In the latter context, the threefolds constructed are essentially twistor spaces, and the rational curves twistor lines,but we shall not pursue this line of approach here. Perhaps the most intriguing challenge is to find \text{any} explicit solution to an equation to which Painlev\'e remark refers.

The structure of the paper is as follows. In Section \ref{art7-sec-2} we consider singular connections and the isomonodromic deformation problem, and in Section \ref{art7-sec-3} see how equivariant compactifications give solutions to the problem. In Section\ref{art7-sec-4} we look at the way the dihedral group fits in with the problem of Poncelet polygons. Section \ref{art7-sec-5} and \ref{art7-sec-6} discuss the actual solutions of the Painlev\'e equation, especially for small values of $k$. Only there can we see in full explicitness the connection which, in the context of the theorem of Narasimhan and Seshadri, relates the parabolic structure and the representation of the fundamental group, however restricted this example may be. In the final section we discuss the modular curve which describes the solutions so constructed.

The author wishes to thank M.F. Atiyah and A. Beauville for useful conversations.

\section{Singular connections}\label{rt7-sec-2}  
We intrduce here the basic objects of our study -flat meromorphic connections with singularities of a specified type. For the most part we follow the exposition of Malgrange \cite{art7-key10}.

\begin{defini}\label{art7-definition-1}
Let $Z$ be a complex manifold, $Y$ a smooth hypersurface and $E$ a holomorphic vector bundle over $Z$. let $\nabla$ be a flat holomorphic connection on $E$ over $Z \ Y$ with connection form $A$ in some local trivialization of $E$. Then on $U\subseteq Z$ we say that
\begin{enumerate}[(1)]
\item $\nabla$ is meromorphic if $\calA$ is meromorphic on $U$.

\item $\nabla$ has a logarithmic singularity along $Y$ if, in a local coordinate system $(z_{1}\ldots, z_{n})$ of $Z$, with $Y$ given by $z_{1} = 0$, $A$ has the form
$$
A = A_{1}\dfrac{dz_{1}}{z_{1}} + A_{2}dz_{2} + \ldots + A_{n}dz_{n}
$$
where $A_{i}$ is holomorphic on $U$.
\end{enumerate}
\end{defini}

One may easily check that the definition is independent of the choice od coordinates and local trivialization. The essential point about a logarithmic singularity is that the pole only occurs in the conormal direction to $Y$. In fact $\nabla$ defines a holomorphic connection on $E$ restricted to $Y$, with connection form
$$
A_{Y} = \sum\limits_{i=2}^{n}A_{i}(0, z_{2}, \ldots, z_{n})dz_{i}.
$$

If $Z$ is 1-dimensional, then such a connection is just a meromorphic connection with simple poles. Flatness is automatic because the holomorphic curvature is a (2,0) form which is identically zero in one dimension. If we take $Z=\bC\bP^{1}$, $Y = \{a_{1}, \ldots, a_{n}, \infty\}$ and the bundle $E$ to be trivial, then $A$ is a matrix-valued meromorphic 1-form with simple poles at $z=a_{1},\ldots, a_{n}, \infty$ and can thus be written as
$$
A = \sum\limits_{i =1}^{i}\dfrac{A_{i}dz}{z-a_{i}}
$$
