\title{Poncelet Polygons and the Painlev\'e Equations}
\markright{Poncelet Polygons and the Painlev\'e Equations}

\author{By~ N. J. Hitchin}
\markboth{N. J. Hitchin}{Poncelet Polygons and the Painlev\'e Equations}

\date{}
\maketitle

\begin{center}
Dedicated to M.S. Narasimhan and C. S. Seshadri on the occasion of their 60th birthdays
\end{center}

\section{Introduction}\label{art7-sec-1}
The celebrated theorem of Narasimhan and Seshadri \cite{art7-key13} relating stable vector bundles on a curve to unitary representations of its fundamental group has been the model for an enormous range of recent results intertwining algebraic geometry and topology. The object which meditates between the two areas geometry and topology. The object which mediates between the two areas in all of these generalizations is the notion of a \textit{connection}, and existence Theorems for various types of connection provide the means of establishing the theorems. In one sense, the motivation for this paper is to pass beyond the existence and demand more explicitness. What do the connections look like ? Can we write them down? This question is our point of departure. The novelty of our presentation here is that the answer involves a journey which takes us backwards in time over two hundred years form the proof of Narasimhan and Seshadri,s theorem in 1965.

For simplicity, instead of considering stable bundles on curves of higher genus we consider the analogous case of parabolically stable bundles, in the sense of Mehta and Seshadri \cite{art6-key11}, on the complex projective line $\bC\bP^{1}$. Such a bundle consists of a vector bundle with a weighted flag structure at $n$ marked points
$a_{1}, \ldots, a_{n}$. The unitary connection that is associated with it is flat and has singularities at the points. In the generic case, the vector bundle itself is trivial, and the flat connection we are looking for can be written as a meromorphic $m \times m$ matrix-valued 1-form with a simple pole at each point $a_{i}$. The parabolic structure can easily be read off form the residuces of the form. The other side of the equation is a representation of the fundamental group $\pi_{1}(\bC\bP^{1}\ \{a_{1}, ldots, a_{n}\})$ in $U(m)$. the holonomy of the connection, and this presents more problems. Such questions occupied the attention of Fuchs, K-lein and others in the last century under the alternative name of monodromy of ordinary differential equations. Now if we fix the holonomy, and ask for the corresponding 1-form for each set of distinct points $\{a_{1}, \ldots a_{n}\}\subset \bC\bP^{1}$, what in fact we are asking for is a solution of a differential equation, the so-called Schlesinger equation (1912) of isomonodromic deformation theory. To focus things even more, in the simple case where $m=2$ and $n =4$, and explicit form for the connection demands a knowledge of solutions to a single nonlinear second order differential equation. This equation, originally found in the context of isomonodromic deformations by $R$. Fuchs in 1907 \cite{art6-key4}, is nowadays called Painlev\'e's 6th equation
\begin{equation*}
\begin{split}
\dfrac{d^{2}y}{dx^{2}} & = 1/2 \left(\dfrac{1}{y} + \dfrac{1}{y-1} + \dfrac{1}{y-x}\right)\left(\dfrac{dy}{dx}\right)^{2} - \left(\dfrac{1}{x} + \dfrac{1}{x-1} + \dfrac{1}{y-x}\right) \dfrac{dy}{dx} \\
 &+ \dfrac{y(y-1)(y-x)}{x^{2}(x-1)^{2}} \left(\alpha + \beta\dfrac{x}{y^{2}} + \gamma\dfrac{x-1}{(y-1)^{2}} + \delta\dfrac{x(x-1)}{(y-x)^{2}}\right)
\end{split}
\end{equation*}
and in the words of Painlev\'e, the general solutions of this equation are ``transcendantes essentiellement nouvelles"  That, on the face of it, would seem to be the end of the quest for explicitness-we are faced with the insuperable obstacle of Painlev\'e transcendants.

Notwithstanding Painlev\'e's statement, for certain values of the constants $\alpha, \beta, \gamma, \delta$, there do exist solutions to the equation which can be written down, and even solutions that are \textit{algebraic}. One property of any solution to the above equation is that $y(x)$ can only have branch points at $x = 0,1,, \infty$. This is essentially the ``Painlev\'e property", that there are no movable singularities. If we find an algebraic solution, then this means we have an algebraic curve with a map to $\bC\bP^{1}$ with only three critical values. Such a curve has a number of special properties. On the one hand, it is defined by a subgroup of finite index in $\Gamma(2) \subset SL(2, \bZ)$, and also,by a well-known theorem of Weil,is defined over $\overline{\bQ}$. In this paper we shal construct solutions by considering the case when the holonomy group $\Gamma$ of the connection is \textit{finite}. In that case the solution $y(x)$ to the Painlev\'e equation is algebraic.

Our approach here is to consider, for a finite subgroup $\Gamma$ of $SL(X, \bC)$, the quotient space $SL(2 ,\bC)/\Gamma$ and an equivariant compactification $Z$. Thus $Z$ is a smooth projective threefold with an action of $SL(2, \bC)$ and a dense open orbit. The Maurer-Cartan form defines a flat connection on $SL(2, \bC)/\Gamma$ with holonomy $\Gamma$, which extends to a meromophic connection on $Z$. The idea is then to look for rational curves in $Z$ such taht the induced connection is of the required form. By construction the holonomy is $\Gamma$, and if we can find enough curves to vary the cross-ratio of the singular points $a_{1}, \ldots a_{4}$, then we have a solution to the Painlev\'e equation. The question of finding and classifying such equivarian compactifications has been addressed by Umemura and Mukai \cite{art6-key12}, but here we focus on one particular case. We take $\Gamma$ to be the binary dihedral group $\tilde{D}_{k} \subset SU(2)$. This might seem very restrictive within the context of parabolically stable bundales, but behind it there hides a very rich seam of algebraic geometry which has its origins further back in history than Painlev\'e.

In the case of the dihedral group, the construction of a suitable compactification is due to Schwarzenberger \cite{art6-key16}, who constructed a family of rank 2 vector bundles $V_{k}$ over $\bC\bP^{2}$. THe threefold corresponding to the dihedral group $D_{k}$ turns out to be the projectivizesd bundle $P(V_{k})$. There are two types of relevant rational curves. Those which project to a line in $\bC\bP^{2}$ yield the solution $y =sqrt{x}$ to the Painlev\'e equation with coefficients $(\alpha, beta, \gamma, \delta) = (1/8, -1/8, 1/2k^{2}, 1/2-1/2k^{2})$. Those which project to a conic lead naturally to another problem, and this one goes back at least to 1746 (see \cite{art7-key3}). It is the problem of \textit{Poncelet polygons}. We seek conics $B$ and $C$ in the plane such that there is a $k$-sided polygon inscribed in  $C$ and circumscribed about $B$. Interest in this problem is still widespread. Poncelet polygons occur in questions of stable bundales on projectives spaces\cite{art7-key14} and more recently in the workl of Barth and Michel \cite{art7-key1}. In fact, we can use their approach to find the modular curve giving the algebraic solution $y(x)$ of the Painlev\'e equation corresponding to $\Gamma = \tilde{D}_{k}$. This satisfies Painlev\'e equation with coefficients $(\alpha, \beta, \gamma, \delta) = (1/8, -1/8, 1/8, 3/8)$. It is essentially Cayley's solution in 1853 of the Poncelet problem which allows us to go further and produce explicit solutions. It is method which fits in well with the isomonodromic approach. 

There are a number of reasons why this is a fruitful area of study. One of them concerns solutions of Painlev\'e equation in general and their relation to integrable systems, another is the connection with self-dual Einstein metrics as discussed in \cite{art7-key6}. In the latter context, the threefolds constructed are essentially twistor spaces, and the rational curves twistor lines,but we shall not pursue this line of approach here. Perhaps the most intriguing challenge is to find \text{any} explicit solution to an equation to which Painlev\'e remark refers.

The structure of the paper is as follows. In Section \ref{art7-sec-2} we consider singular connections and the isomonodromic deformation problem, and in Section \ref{art7-sec-3} see how equivariant compactifications give solutions to the problem. In Section\ref{art7-sec-4} we look at the way the dihedral group fits in with the problem of Poncelet polygons. Section \ref{art7-sec-5} and \ref{art7-sec-6} discuss the actual solutions of the Painlev\'e equation, especially for small values of $k$. Only there can we see in full explicitness the connection which, in the context of the theorem of Narasimhan and Seshadri, relates the parabolic structure and the representation of the fundamental group, however restricted this example may be. In the final section we discuss the modular curve which describes the solutions so constructed.

The author wishes to thank M.F. Atiyah and A. Beauville for useful conversations.

\section{Singular connections}\label{rt7-sec-2}  
We intrduce here the basic objects of our study -flat meromorphic connections with singularities of a specified type. For the most part we follow the exposition of Malgrange \cite{art7-key10}.

\begin{defini}\label{art7-definition-1}
Let $Z$ be a complex manifold, $Y$ a smooth hypersurface and $E$ a holomorphic vector bundle over $Z$. let $\nabla$ be a flat holomorphic connection on $E$ over $Z \ Y$ with connection form $A$ in some local trivialization of $E$. Then on $U\subseteq Z$ we say that
\begin{enumerate}[(1)]
\item $\nabla$ is meromorphic if $\calA$ is meromorphic on $U$.

\item $\nabla$ has a logarithmic singularity along $Y$ if, in a local coordinate system $(z_{1}\ldots, z_{n})$ of $Z$, with $Y$ given by $z_{1} = 0$, $A$ has the form
$$
A = A_{1}\dfrac{dz_{1}}{z_{1}} + A_{2}dz_{2} + \ldots + A_{n}dz_{n}
$$
where $A_{i}$ is holomorphic on $U$.
\end{enumerate}
\end{defini}

One may easily check that the definition is independent of the choice od coordinates and local trivialization. The essential point about a logarithmic singularity is that the pole only occurs in the conormal direction to $Y$. In fact $\nabla$ defines a holomorphic connection on $E$ restricted to $Y$, with connection form
$$
A_{Y} = \sum\limits_{i=2}^{n}A_{i}(0, z_{2}, \ldots, z_{n})dz_{i}.
$$

If $Z$ is 1-dimensional, then such a connection is just a meromorphic connection with simple poles. Flatness is automatic because the holomorphic curvature is a (2,0) form which is identically zero in one dimension. If we take $Z=\bC\bP^{1}$, $Y = \{a_{1}, \ldots, a_{n}, \infty\}$ and the bundle $E$ to be trivial, then $A$ is a matrix-valued meromorphic 1-form with simple poles at $z=a_{1},\ldots, a_{n}, \infty$ and can thus be written as
$$
A = \sum\limits_{i =1}^{i}\dfrac{A_{i}dz}{z-a_{i}}
$$
The holonomy of a flat connection on $Z\backslash Y$ is obtained by parallel translation around closed paths and defines, after fixing a base point $b$,a representation of the fundamental group
$$
\rho : \pi_{1}(Z\backslash Y) \rightarrow GL(m ,\bC)
$$
In one dimension, the holonomy may also be considered as the effect of analytic continuation of solutions to the system of ordinary differential equations
$$
\dfrac{df}{dz} + \sum\limits_{i=1}^{n} \dfrac{A_{i}f}{z-a_{i}} = 0
$$
around closed paths through $b$. As such, one often uses the classical term \textit{monodromy} rather than the differential geometric \textit{holonomy}. Changing the basepoint to $b'$ effects an overall conjugation (by the holonomy along a path from $b$ to $b'$) of the holonomy representation.

For the punctured projective line above, we obtain a representation of the group $\pi_{1}(S^{2}\backslash \{a_{1}, \ldots, a_{n}, \infty\})$. This is a free group on $n$ generators, which can be taken as simple loops $\gamma_{i}$ from $b$ passing once around $a_{i}$. Moving $b$ close to $a_{i}$, it is easy to see that $ \rho(\gamma_{i})$ is conjugate to
$$
\exp(-2\pi iA_{i}).
$$
There is also a singularity of $A$ at infinity with residue $A_{\infty}$. Since the sum of the residues of a differential is zero, we must have
$$
A_{\infty} = -\sum\limits_{i=1}^{n}A_{i}
$$
and so $\rho(\gamma_{infty})$ is conjugate also to $\exp(-2\pi i A_{\infty})$. In the fundamental group itself $\gamma_{1} \gamma_{2}\ldots \gamma_{n}\gamma_{\infty} =1$ so that in the holonomy representation
\begin{equation}\label{art7-eq-1}
\rho(\gamma_{1})\rho(\gamma_{2})\ldots \rho(\gamma_{\infty})
\end{equation}
Thus the conjugacy classes of the residuces $A_{i}$ of the connection determine the conjugacy classed of
$\rho(\gamma_{i})$, and these must also satisfy \eqref{art7-eq-1}. 

This is partial information about the holonomy representation. However, the full holonomy group depends on the position of the poles $a_{i}$. The problem of particular interset to us here is the \textit{isomonodromic deformation problem} to determine the dependence of $A_{i}$ on $a_{1}, \ldots, a_{n}$ in order that the holonomy representation should remain the same up to conjugation. All we have seen so far is that the conjugacy class of $\exp(2\pi iA_{i})$ should remain constant.

One way of approaching the isomonodromic deformation problem, due to Malgrange, is via a universal deformation space. Let $X_{n}$ denote the space of ordered distinct points $(a_{1}, \ldots, a_{n}) \in \bC$, and $\tilde{X}_{n}$ its universal covering. It is well-known that this is a contractible space- the classifying space for the braid group on $n$ strands. Now consider the divisor
$$
Y_{m}= \{(z,a_{1}, \ldots , a_{n}) \in \bC\bP^{1} \times X_{n}: z \neq a_{m}\}
$$
and let $\tilde{Y}_{m}$ be its inverse image in $\bC\bP^{1} \times \tilde{X}_{n}$. Furthermore,
define $\tilde{Y}_{\infty} = \{(\infty, x) : x \in \tilde{X}_{n}\}$.

The projection onto the second factor $p : \bC\bP^{1} \backslash \{a_{1}, \ldots ,a_{n}, \infty\}$, and the contractibility of $\tilde{X}_{n}$ implies fro  the exact homotopy sequence that the inclusion $i$ of a fibre induces an isomorphism of fundamental groups
$$
\pi_{1}(\bC\bP^{1} \backslash \{a_{1}^{0}, \ldots, a_{n}^{0}, \infty\})\cong \pi_{1} (\bC\bP^{1} \times \tilde{X}_{n} \backslash \{\tilde{Y}_{1} \cup \ldots\cup \tilde{Y}_{n}\cup  \tilde{Y}_{\infty}\})
$$
Thus a flat connection on $\bC\bP^{1}\backslash \{a_{1}^{0}, \ldots, a_{n}^{0}, \infty\}$ extends to flat connection  with the same holonomy on $\bC\bP^{1} \times \tilde{X}_{n} \backslash \{\tilde{Y}_{1} \cup \ldots\cup \tilde{Y}_{n} \cup  \tilde{Y}_{\infty}\}$. Malgrange's theorem asserts that this flat connection has logarithmic singularities along $\tilde{Y}_{m}$ and $\tilde{Y}_{\infty}$.

 More precisely,

\begin{theorem}\label{art7-thm-1}
{\bf Malgrange \cite{art7-key10}} Let $\nabla^{0}$ be flat holomorphic connection on the vector bundle $E^{0}$ over $\bC\bP^{1}\backslash \{ a_{1}^{0}, \ldots, a_{n}^{0}, \infty\}$, with logarithmic singularities at $a_{1}^{0}, \ldots, a_{n}^{0}$. Then there exists a holomorphic vector bundle $E$ on $\bC \bP^{1} \times \tilde{X}_{n}$ with a flat connection $\nabla$ with logarithmic singularities at $\tilde{Y}_{1}, \ldots,\tilde{Y}_{n}, \tilde{Y}_{\infty}$ and an isomorphism $j : \i^{*} (E, \nabla) \rightarrow (E^{0}, \nabla^{0})$. Furthermore, $(E, \nabla, j)$ is unique up to isomorphism.
\end{theorem}

 Now suppose that $E^{0}$ is holomorphically trivial. The vector bundle $E$ will not necessarily be trivial on all fibres of the projection $p$, but for a dense open set $U \subseteq \tilde{X}_{n}$ it will be. Choose a basis $e_{1}^{0}, e_{2}^{0}, \ldots e_{m}^{0}$ of the fibre of $E^{0}$ at $z = \infty$. Now since $\nabla$ has a logarithmic singularity on $\tilde{Y}_{\infty}$, it induces a flat connection there, and since $\tilde{Y}_{\infty} \cong \tilde{X}_{n}$ is simply connected, by parallel translation we can unambiguously extend $e_{1}^{0}, e_{2}^{0}, \ldots, e_{m}^{0}$ to trivialization of $E$ over $\tilde{Y}_{\infty}$. Then since $E$ is holomorphically trivial on each fibre over $U$. we can uniquely extend $e_{1}^{0}, e_{2}^{0}, \ldots, e_{m}^{0}$ along the fibres to obtain a trivialization $e_{1}, \ldots, e_{m}$ of $E$ on $\bC\bP^{1} \times U$. It is easy to see that, relative to this trivialization, the connection form pf $\nabla$ can be written
\begin{equation}\label{art7-eq-2}
A = \sum\limits_{i=1}^{n}A_{i} \dfrac{dz-da_{i}}{z-a_{i}}
\end{equation}
where $A_{i}$ is a holomorphic function of $a_{1}, \ldots, a_{n}$.

The flatness of the connection can then be expressed as:
$$
dA_{i} + \sum\limits_{j\neq i}[A_{i}, A_{j}]\dfrac{da_{i}-da_{j}}{a_{i}-a_{j}} =0
$$
which is known as \textit{Schlesinger's equation} \cite{art7-key15}.

The gauge freedom in this equation involves only the choice of the initial basis $e_{1}^{0}, e_{2}^{0}, \ldots, e_{m}^{0}$ and consists therefore of conjugation of the $A_{i}$ by a constant matrix.

The case which interests us here in where the holonomy lies in $SL(2,\bC)$, (so that the $A_{i}$ are trace-free 2 $\times$ 2 matrices), and where there are 3 marked points $a_{1}, a_{2}, a_{3}$ which, together with $z=\infty$, are the singular points of the connection. By a projective transformation we can make these points 0,1, $x$, Then
$$
A(z)= \dfrac{A_{1}}{z} + \dfrac{A_{2}}{z-1} + \dfrac{A_{3}}{z-x}
$$
and Schlesinger's equation becomes:
\begin{align}\label{art7-eq-3}
\dfrac{dA_{1}}{dx} &= \dfrac{[A_{3}, A_{1}]}{x}\nonumber\\
\dfrac{dA_{2}}{dx} &= \dfrac{[A_{3}, A_{2}]}{x-1}\\
\dfrac{dA_{3}}{dx} &= \dfrac{-[A_{3}, A_{1}]}{x} - \dfrac{A_{3}, A_{2}}{x-1}\nonumber
\end{align}
where the last equation is equivalent to
$$
A_{1} + A_{2} +A_{3} = -A_{\infty} = \text{const}.
$$

The relationship with the Painlev\'e equation can best be seen by following \cite{art7-key8}. Each entry of the matrix $A_{ij}(z)$ is of the form $q(z)/z(z-1)(z-x)$ for some quadratic polynomial $q$. Suppose that $A_{\infty}$ is diagonalizable, and choose a basis such that
$$
A_{\infty} =
\left(\begin{matrix}
\lambda & 0 \\
0 & -\lambda 
\end{matrix}\right)
$$
then $A_{12}$ can be written
\begin{equation}\label{art7-eq-4}
A_{12}(z) = \dfrac{k(z-y)}{z(z-1)(z-x)}
\end{equation}

\begin{align}\label{art7-eq-5}
 \dfrac{d^{2}y}{dx^{2}} &= 1/2\left(\dfrac{1}{y} + \dfrac{1}{y-1} + \dfrac{1}{y-x} \right)\left(\dfrac{dy}{dx}\right)^{2}
 -\left( \dfrac{1}{x} + \dfrac{1}{x-1} + \dfrac{1}{y-x}\right)\dfrac{dy}{dx}\nonumber\\
 & +\dfrac{y(y-1)(y-x)}{x^{2}(x-1)^{2}} \left(\alpha + \beta\dfrac{x}{y^{2}} + \gamma\dfrac{x-1}{(y-1)^{2}} + \delta\dfrac{x(x-1)}{(y-x)^{2}} \right)
\end{align}
where
\begin{align}\label{art7-eq-6}
\alpha &= (2\lambda -1)^{2}/2\nonumber\\
\beta &= 2d\det A_{1}^{2}\nonumber\\
\gamma &= -2\det A_{2}^{2}\nonumber\\
\delta &= (1 + 4\det A_{3}^{2})/2
\end{align} 
For the formulae which reconstruct the connection from $y(x)$ we refer to \cite{art7-key8}, but essentially the entires of tha $A_{i}$ are rational functions of $x, y$ and $dy/dx$. For our purposes it is useful to  note the geometrical form of the definition of $y(x)$ given by \ref{art7-eq-4}:

\begin{prop}\label{art7-proposition-1}
The solution $y(x)$ to the Painlev\'e equation corresponding to an isomonodromic deformation $A(z)$ is the point $y \in \bC\bP^{1} \backslash \{ 0,1,\break x, \infty\}$ at which $A(y)$ and $A_{\infty}$ have a common eigenvector.
\end{prop}

Note that strictly speaking there are two Painlev\'e equations (with $\alpha = (\pm 2\lambda -1)^{2}/2)$ correspinding to the values of $y$ with this property.

\section{Equivariant compactifications}

Consider the three- dimensional complex Lie group $SL(2, \bC)$ and the Lie algebra-valued 1-form
$$
A = -(dg)a^{-1}.
$$
The form $A$ is the connection form for a trivial connection on the trivial bundle. It simply relates teh trivializations of the principle frame bundle by left and right translation.

Now let $\Gamma$ be a finite subgroup of $SL(2, \bC)$. Then $SL(2, \bC)/\Gamma$ is non-compact ciompelx manifold and since $A$ is invariant under right translations,it descentds to this quotient. Thus, on $SL(2, \bC)/\Gamma$, $A$ defines a flat connection on the trivial rank 2 vector bundle. Its holonomy is tatutologically $\Gamma$.

In this section, we shall consider an equivarient compactification of $SL(2, \bC)/\Gamma$,  that is to say, a compact compelx manifold $Z$ on which $SL(2, \bC)$ acts with a dense open orbit with stabilizer conjugate to $\Gamma$. Let $Z$ be such a compactification, then the action of the group embeds the lie algebra $\underline{g}$  in th space of holomorphic vector fields on $Z$. Equivalently, we have a vector bundle homomorphism
$$
\alpha: Z \times \underline{g} \rightarrow TZ
$$
which is generically an isomorphism.It fails to be an isomorphism on the union of the lower dimensional orbits of $SL(2, \bC)$, and this is where $\bigwedge^{3} \alpha \in H^{0}(Z, \Hom(\bigwedge^{3} \underline{g}, \bigwedge^{3}T)) \cong H^{0}(Z, K^{-1})$ is a section of the anticanonical bundle, so the union of the orbits of dimension less than three form an \textit{anticanonical divisor} $Y$, which may of course have several components or be singular.

 In the open orbit $Z/Y \cong SL(2, \bC)/\Gamma$, the action is equivalent to left multiplication, and the connection $A$ above is given by
 $$
 A = \alpha^{-1}: TZ \rightarrow  Z \times \underline{g}.
 $$
It is clearly meromorphic on $Z$, but more is true.

\begin{prop}\label{art7-proposition-2}
If $\bigwedge^{3}\alpha$ vanishes non-degenerately on the divisor $Y$, then the connection $A = \alpha^{-1}$ has a logarithmic singularity along $Y$.
\end{prop}

This is a local statement, and so it can always be applied to the smooth part of $Y$ even if there are singular points.

\begin{proof}
In local coordinates, $\alpha$ is represented by a holomorphic function $B(z)$ with values in the space of 3 $\times$ 3 matrices. The divisor $Y$ is then the zero set of $\det B$. If $\det B$ has a non-degenerate zero at $p \in Y$, then its null-space is one-dimensional at $p$, so the kernel of $\alpha$, the the Lie algebra of the stabilizer of $p$, is one-dimensional. Thus the $SL(2, \bC)$ orbit through $p$ is two-dimensional, and so $Y$ is the orbit.

Now for any quare matrix $B$, let $B\spcheck$ denote the transpose of the matrix of cofactors. Then it is well-known that
$$
BB\spcheck = (\det B)I
$$
Hence in local coordinates
$$
A = \alpha^{-1} = \dfrac{B\spcheck}{\det B}
$$
and so $A$ has a simple pole along $Y$. From Definition \ref{art7-definition-1}, we need to show that the residue in the conormal direction. For this consider the invariant description of $B\spcheck$. We have on $Z$
$$
\wedge^{2} \alpha : \wedge^{2}\underline{g} \rightarrow \wedge^{2}T
$$ 
and using the identifications $\wedge^{2}\underline{g} \cong \underline{g}^{*}$ and $\wedge^{2}T \cong T^{*} \otimes \wedge^{3}T, B\spcheck$ represents the dual map of $\wedge^{2}\alpha$:
$$
(\wedge^{2}\alpha)^{*} : T\rightarrow \underline{g} \otimes \wedge T.
$$

Now the image of $\alpha$ at $p$ is the tangent space to the orbit $Y$ at $p$ by the definition of $\alpha$. Thus the image of $\wedge^{2} \alpha$ is $\wedge^{2}TY_{p}$ which means that $ (\wedge^{2}\alpha)^{*}$ annihilates $TY$, which is the required result.

Note that the kernel of $\wedge^{2}\alpha$ is the set of two-vectors $v \wedge w$ where $w \in \underline{g}$ and $v$ is ion the Lie algebra of the stabilizer of $p$. Thus the residue at $p$ of the connection $A$ lies in the Lie algebra of the stabilizer.
\end{proof}

Now suppose that $P$ is a rational curve in $Z$ which meets $Y$ transversally at four points. Then the restriction of $A$ to $P$ defines a connection with logarithmic poles at the points and, from the map of fundamental groups
$$
\pi_{1}(P\backslash \{a_{1},\ldots, a_{4}\})\rightarrow \pi_{1}(Z\backslash Y)\rightarrow \Gamma \rightarrow SL(2, \bC),
$$
its holonomy is contained in $\Gamma$. A deformation of $P$ will define a nearby curve in the same homotopy class and hence the induced connection will have the same holonomy. To obtain isomonodromic deformations, we therefore need to study the deformation theory of such curves.

\begin{prop}\label{art7-proposition-3}
Let $p \subset Z$ be a rational curve meeting $Y$ transversally at four points. Then $P$ belongs to a smooth four-parameter family of rational curves on which the cross-ratio of the points is nonconstant function.
\end{prop}

\begin{proof}
Th proof is standard Kodaira-Spencer deformation theory. By hypothesis $P$ meets the anticanonical divisor $Y$ in four points, so the degree of $K_{Z}$ on $P$ is -4. Hence, in $N$ is the normal bundle of $P \cong \bC\bP^{1}$,
$$
\deg N = -\deg K_{Z} + \deg K_{P} = 2
$$
and so
$$
N \cong \calO(m) \oplus \calO(2-m)
$$
for some integer $m$. However, since $C$ is transversal to the 2-dimensional orbit $Y$ of $SL(2, \bC)$, the map $\alpha$ always maps \textit{onto} the normal bundle to $C$. We therefore have a surjective homomorphism of holomorphic vector bundles
$$
\beta : \calO \otimes \underline{g} \rightarrow N
$$
and this implies that $ 0 \leq m \leq  2$. As a consequence, $H^{1} (P ,N) = 0$ and $H^{0}(P, N)$ is four-dimensional, so the existence of a smooth family follows fro  Kodaira \cite{art7-key9}. 

Since $\beta$ is surjective, its kernel is a line bundle of degree-$\deg N = -2$, so we have an exact sequence of sheaves:
$$
\calO(-2)\rightarrow \calO \otimes \underline{g}\rightarrow N.
$$
Under $\alpha$, the kernel maps isomorphically to the sheaf of sections of the tragent bundle $TP$ which vanish at the four points $P\cap Y$. From the long exact cohomology sequence we have
$$
0 \rightarrow \underline{g}\rightarrow H^{0}(P, N)\xrightarrow{\delta} H^{1}(P, \calO(-2))\rightarrow 0
$$
and since $H^{0}(P, N)$ is 4-dimensional and $\underline{g}$ is 3-dimensional, the map $\delta$ id surjective. But $\alpha \delta$ is the Kodarira-Spencer map for deformations of the four points on $P$, so since it is non-trivial, the cross-ratio is non-constant.
\end{proof}

\begin{example*}
~

\begin{enumerate}[]
\item As the reader may realize, the situation here is very similar to the study of twistor spaces and twistor lines, and indeed there is a differential geometric context for this (see \cite{art7-key6}, \cite{art7-key7}). This is not the agenda for this paper, but it is a useful example to see the standard twistor space-$\bC\bP^{3}$ and the straight linex in it-within the current context.

Let $V$ be the 4-dimensional space of cubic polynomials
$$
p(z)=c_{0} + c_{1}z +c_{2}z^{2} + c_{3}z^{3}
$$
and consider $V$ as a representation space of $SL(2, \bC)$ under the action
$$
p(z)\mapsto p \left( \dfrac{az + b}{cz + d}\right)(cz + d)^{3}.
$$
This is the unique (up to isomorphism) 4-dimensional irreducible representation of $SL(2, \bC)$. Then $Z= P(V) = \bC\P^{3}$ is a compact threefold with an action of $SL(2, \bC)$ and moreover the open dense set of cubics with distinct roots in an orbit. This follows since given any two triplex of distinct ordered points in $\bC\bP^{1}$, there is a unique element of $PSL(2, \bC)$ which takes one to the other. However, the cubic polynomial determines an \textit{unordered} triple of roots, and hence the stabilizer in $PSL(2, \bC)$ is the symmetric group $S_{3}$. Thinking of this as the symmetries of an equilateral triangle, the holonomy group $\Gamma \subset SL(2, \bC)$ of the connection $A= \alpha^{-1}$ is the binary dihedral group $\tilde{D}_{3}$. The lower-dimensional orbits consist firstly of the cubics with one repeated root, which is
2-dimensional, and those with a triple root, which constitute a rational normal curve in $\bC\bP^{3}$. Together they form the discriminant divisor $Y$, the anticanonical divisor discussed above.

A generic line in $\bC\bP^{3}$, generated by polynomials $p(z), q(z)$ meets $Y$ at those values of $t$ for which the discriminant of $tp(z) + q(z)$ vanishes, i.e. where
\begin{align*}
tp(z) +q(z) &= 0\\
tp'(z)+q'(z) &= 0
\end{align*}
have a common root. This occurs for $t = -q(\alpha)/p(\alpha)$ where $\alpha$ is a root of the quartic equation
$$
p'(z)a(z)-p(z)q'(z) =0
$$
and so the line meets $Y$ in four generically distinct points. Thus the 4-parameters family of lines in $\bC\bP^{3}$ furnish an example of the above proposition.
\end{enumerate}
As we remarked above, this is an example of an isomonodromoc deformation, as would be any family of curves $P$ in Proposition \ref{art7-proposition-3}. It yields a solution of the Painlev\'e equation either by applying the argument of Theorem \ref{art7-thm-1} to the connection with logarithmic singularities on $Z$, or appealing to the universality of Malgrange's construction. We shall not derive the solution of the Painlev\'e equation here from $\bC\bP^{3}$, since it will appear via a different compactification in the context of Poncelet polygons. There we shall also see how a striaght line in $\bC\bP^{3}$ defines a pair on conics with the Poncelet property for triangles.
\end{example*}

\section{Poncelet polygons and projective bundles}\label{art7-sec-4}
In this section we shall study a particular class of equivariant compactifications, originally due to Schwarzenberger \cite{art7-key16}. Consider the complex surface $\bC\bP^{1}\times \bC\bP^{1}$ and the holomorphic involution $\sigma$ which interchanges the two factors. The quotient space is $\bC\bP^{2}$. A profitableway of viewing this is a the map which assigns to a pair of complex numbers the coefficients of the quadratic polynomial which has them as roots. In affine coordinates we have the quotient map
\begin{align*}
\pi : \bC\bP^{1} \times \bC\bP^{1} &\rightarrow \bC\bP^{1}\\
(w, z) &\mapsto (-(w +z), wz).
\end{align*}
