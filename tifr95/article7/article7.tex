\title{Poncelet Polygons and the Painlev\'e Equations}
\markright{Poncelet Polygons and the Painlev\'e Equations}

\author{By~ N. J. Hitchin}
\markboth{N. J. Hitchin}{Poncelet Polygons and the Painlev\'e Equations}

\date{}
\maketitle

\begin{center}
Dedicated to M.S. Narasimhan and C. S. Seshadri on the occasion of their 60th birthdays
\end{center}

\section{Introduction}\label{art7-sec-1}
The celebrated theorem of Narasimhan and Seshadri \cite{art7-key13} relating stable vector bundles on a curve to unitary representations of its fundamental group has been the model for an enormous range of recent results intertwining algebraic geometry and topology. The object which meditates between the two areas geometry and topology. The object which mediates between the two areas in all of these generalizations is the notion of a \textit{connection}, and existence Theorems for various types of connection provide the means of establishing the theorems. In one sense, the motivation for this paper is to pass beyond the existence and demand more explicitness. What do the connections look like ? Can we write them down? This question is our point of departure. The novelty of our presentation here is that the answer involves a journey which takes us backwards in time over two hundred years form the proof of Narasimhan and Seshadri,s theorem in 1965.

For simplicity, instead of considering stable bundles on curves of higher genus we consider the analogous case of parabolically stable bundles, in the sense of Mehta and Seshadri \cite{art6-key11}, on the complex projective line $\bC\bP^{1}$. Such a bundle consists of a vector bundle with a weighted flag structure at $n$ marked points
$a_{1}, \ldots, a_{n}$. The unitary connection that is associated with it is flat and has singularities at the points. In the generic case, the vector bundle itself is trivial, and the flat connection we are looking for can be written as a meromorphic $m \times m$ matrix-valued 1-form with a simple pole at each point $a_{i}$. The parabolic structure can easily be read off form the residuces of the form. The other side of the equation is a representation of the fundamental group $\pi_{1}(\bC\bP^{1}\ \{a_{1}, ldots, a_{n}\})$ in $U(m)$. the holonomy of the connection, and this presents more problems. Such questions occupied the attention of Fuchs, K-lein and others in the last century under the alternative name of monodromy of ordinary differential equations. Now if we fix the holonomy, and ask for the corresponding 1-form for each set of distinct points $\{a_{1}, \ldots a_{n}\}\subset \bC\bP^{1}$, what in fact we are asking for is a solution of a differential equation, the so-called Schlesinger equation (1912) of isomonodromic deformation theory. To focus things even more, in the simple case where $m=2$ and $n =4$, and explicit form for the connection demands a knowledge of solutions to a single nonlinear second order differential equation. This equation, originally found in the context of isomonodromic deformations by $R$. Fuchs in 1907 \cite{art6-key4}, is nowadays called Painlev\'e's 6th equation
\begin{equation*}
\begin{split}
\dfrac{d^{2}y}{dx^{2}} & = 1/2 \left(\dfrac{1}{y} + \dfrac{1}{y-1} + \dfrac{1}{y-x}\right)\left(\dfrac{dy}{dx}\right)^{2} - \left(\dfrac{1}{x} + \dfrac{1}{x-1} + \dfrac{1}{y-x}\right) \dfrac{dy}{dx} \\
 &+ \dfrac{y(y-1)(y-x)}{x^{2}(x-1)^{2}} \left(\alpha + \beta\dfrac{x}{y^{2}} + \gamma\dfrac{x-1}{(y-1)^{2}} + \delta\dfrac{x(x-1)}{(y-x)^{2}}\right)
\end{split}
\end{equation*}
and in the words of Painlev\'e, the general solutions of this equation are ``transcendantes essentiellement nouvelles"  That, on the face of it, would seem to be the end of the quest for explicitness-we are faced with the insuperable obstacle of Painlev\'e transcendants.

Notwithstanding Painlev\'e's statement, for certain values of the constants $\alpha, \beta, \gamma, \delta$, there do exist solutions to the equation which can be written down, and even solutions that are \textit{algebraic}. One property of any solution to the above equation is that $y(x)$ can only have branch points at $x = 0,1,, \infty$. This is essentially the ``Painlev\'e property", that there are no movable singularities. If we find an algebraic solution, then this means we have an algebraic curve with a map to $\bC\bP^{1}$ with only three critical values. Such a curve has a number of special properties. On the one hand, it is defined by a subgroup of finite index in $\Gamma(2) \subset SL(2, \bZ)$, and also,by a well-known theorem of Weil,is defined over $\overline{\bQ}$. In this paper we shal construct solutions by considering the case when the holonomy group $\Gamma$ of the connection is \textit{finite}. In that case the solution $y(x)$ to the Painlev\'e equation is algebraic.

Our approach here is to consider, for a finite subgroup $\Gamma$ of $SL(X, \bC)$, the quotient space $SL(2 ,\bC)/\Gamma$ and an equivariant compactification $Z$. Thus $Z$ is a smooth projective threefold with an action of $SL(2, \bC)$ and a dense open orbit. The Maurer-Cartan form defines a flat connection on $SL(2, \bC)/\Gamma$ with holonomy $\Gamma$, which extends to a meromophic connection on $Z$. The idea is then to look for rational curves in $Z$ such taht the induced connection is of the required form. By construction the holonomy is $\Gamma$, and if we can find enough curves to vary the cross-ratio of the singular points $a_{1}, \ldots a_{4}$, then we have a solution to the Painlev\'e equation. The question of finding and classifying such equivarian compactifications has been addressed by Umemura and Mukai \cite{art6-key12}, but here we focus on one particular case. We take $\Gamma$ to be the binary dihedral group $\tilde{D}_{k} \subset SU(2)$. This might seem very restrictive within the context of parabolically stable bundales, but behind it there hides a very rich seam of algebraic geometry which has its origins further back in history than Painlev\'e.

In the case of the dihedral group, the construction of a suitable compactification is due to Schwarzenberger \cite{art6-key16}, who constructed a family of rank 2 vector bundles $V_{k}$ over $\bC\bP^{2}$. THe threefold corresponding to the dihedral group $D_{k}$ turns out to be the projectivizesd bundle $P(V_{k})$. There are two types of relevant rational curves. Those which project to a line in $\bC\bP^{2}$ yield the solution $y =sqrt{x}$ to the Painlev\'e equation with coefficients $(\alpha, beta, \gamma, \delta) = (1/8, -1/8, 1/2k^{2}, 1/2-1/2k^{2})$. Those which project to a conic lead naturally to another problem, and this one goes back at least to 1746 (see \cite{art7-key3}). It is the problem of \textit{Poncelet polygons}. We seek conics $B$ and $C$ in the plane such that there is a $k$-sided polygon inscribed in  $C$ and circumscribed about $B$. Interest in this problem is still widespread. Poncelet polygons occur in questions of stable bundales on projectives spaces\cite{art7-key14} and more recently in the workl of Barth and Michel \cite{art7-key1}. In fact, we can use their approach to find the modular curve giving the algebraic solution $y(x)$ of the Painlev\'e equation corresponding to $\Gamma = \tilde{D}_{k}$. This satisfies Painlev\'e equation with coefficients $(\alpha, \beta, \gamma, \delta) = (1/8, -1/8, 1/8, 3/8)$. It is essentially Cayley's solution in 1853 of the Poncelet problem which allows us to go further and produce explicit solutions. It is method which fits in well with the isomonodromic approach. 

There are a number of reasons why this is a fruitful area of study. One of them concerns solutions of Painlev\'e equation in general and their relation to integrable systems, another is the connection with self-dual Einstein metrics as discussed in \cite{art7-key6}. In the latter context, the threefolds constructed are essentially twistor spaces, and the rational curves twistor lines,but we shall not pursue this line of approach here. Perhaps the most intriguing challenge is to find \text{any} explicit solution to an equation to which Painlev\'e remark refers.

The structure of the paper is as follows. In Section \ref{art7-sec-2} we consider singular connections and the isomonodromic deformation problem, and in Section \ref{art7-sec-3} see how equivariant compactifications give solutions to the problem. In Section\ref{art7-sec-4} we look at the way the dihedral group fits in with the problem of Poncelet polygons. Section \ref{art7-sec-5} and \ref{art7-sec-6} discuss the actual solutions of the Painlev\'e equation, especially for small values of $k$. Only there can we see in full explicitness the connection which, in the context of the theorem of Narasimhan and Seshadri, relates the parabolic structure and the representation of the fundamental group, however restricted this example may be. In the final section we discuss the modular curve which describes the solutions so constructed.

The author wishes to thank M.F. Atiyah and A. Beauville for useful conversations.

\section{Singular connections}\label{rt7-sec-2}  
We intrduce here the basic objects of our study -flat meromorphic connections with singularities of a specified type. For the most part we follow the exposition of Malgrange \cite{art7-key10}.

\begin{defini}\label{art7-definition-1}
Let $Z$ be a complex manifold, $Y$ a smooth hypersurface and $E$ a holomorphic vector bundle over $Z$. let $\nabla$ be a flat holomorphic connection on $E$ over $Z \ Y$ with connection form $A$ in some local trivialization of $E$. Then on $U\subseteq Z$ we say that
\begin{enumerate}[(1)]
\item $\nabla$ is meromorphic if $\calA$ is meromorphic on $U$.

\item $\nabla$ has a logarithmic singularity along $Y$ if, in a local coordinate system $(z_{1}\ldots, z_{n})$ of $Z$, with $Y$ given by $z_{1} = 0$, $A$ has the form
$$
A = A_{1}\dfrac{dz_{1}}{z_{1}} + A_{2}dz_{2} + \ldots + A_{n}dz_{n}
$$
where $A_{i}$ is holomorphic on $U$.
\end{enumerate}
\end{defini}

One may easily check that the definition is independent of the choice od coordinates and local trivialization. The essential point about a logarithmic singularity is that the pole only occurs in the conormal direction to $Y$. In fact $\nabla$ defines a holomorphic connection on $E$ restricted to $Y$, with connection form
$$
A_{Y} = \sum\limits_{i=2}^{n}A_{i}(0, z_{2}, \ldots, z_{n})dz_{i}.
$$

If $Z$ is 1-dimensional, then such a connection is just a meromorphic connection with simple poles. Flatness is automatic because the holomorphic curvature is a (2,0) form which is identically zero in one dimension. If we take $Z=\bC\bP^{1}$, $Y = \{a_{1}, \ldots, a_{n}, \infty\}$ and the bundle $E$ to be trivial, then $A$ is a matrix-valued meromorphic 1-form with simple poles at $z=a_{1},\ldots, a_{n}, \infty$ and can thus be written as
$$
A = \sum\limits_{i =1}^{i}\dfrac{A_{i}dz}{z-a_{i}}
$$
The holonomy of a flat connection on $Z\backslash Y$ is obtained by parallel translation around closed paths and defines, after fixing a base point $b$,a representation of the fundamental group
$$
\rho : \pi_{1}(Z\backslash Y) \rightarrow GL(m ,\bC)
$$
In one dimension, the holonomy may also be considered as the effect of analytic continuation of solutions to the system of ordinary differential equations
$$
\dfrac{df}{dz} + \sum\limits_{i=1}^{n} \dfrac{A_{i}f}{z-a_{i}} = 0
$$
around closed paths through $b$. As such, one often uses the classical term \textit{monodromy} rather than the differential geometric \textit{holonomy}. Changing the basepoint to $b'$ effects an overall conjugation (by the holonomy along a path from $b$ to $b'$) of the holonomy representation.

For the punctured projective line above, we obtain a representation of the group $\pi_{1}(S^{2}\backslash \{a_{1}, \ldots, a_{n}, \infty\})$. This is a free group on $n$ generators, which can be taken as simple loops $\gamma_{i}$ from $b$ passing once around $a_{i}$. Moving $b$ close to $a_{i}$, it is easy to see that $ \rho(\gamma_{i})$ is conjugate to
$$
\exp(-2\pi iA_{i}).
$$
There is also a singularity of $A$ at infinity with residue $A_{\infty}$. Since the sum of the residues of a differential is zero, we must have
$$
A_{\infty} = -\sum\limits_{i=1}^{n}A_{i}
$$
and so $\rho(\gamma_{infty})$ is conjugate also to $\exp(-2\pi i A_{\infty})$. In the fundamental group itself $\gamma_{1} \gamma_{2}\ldots \gamma_{n}\gamma_{\infty} =1$ so that in the holonomy representation
\begin{equation}\label{art7-eq-1}
\rho(\gamma_{1})\rho(\gamma_{2})\ldots \rho(\gamma_{\infty})
\end{equation}
Thus the conjugacy classes of the residuces $A_{i}$ of the connection determine the conjugacy classed of
$\rho(\gamma_{i})$, and these must also satisfy \eqref{art7-eq-1}. 

This is partial information about the holonomy representation. However, the full holonomy group depends on the position of the poles $a_{i}$. The problem of particular interset to us here is the \textit{isomonodromic deformation problem} to determine the dependence of $A_{i}$ on $a_{1}, \ldots, a_{n}$ in order that the holonomy representation should remain the same up to conjugation. All we have seen so far is that the conjugacy class of $\exp(2\pi iA_{i})$ should remain constant.

One way of approaching the isomonodromic deformation problem, due to Malgrange, is via a universal deformation space. Let $X_{n}$ denote the space of ordered distinct points $(a_{1}, \ldots, a_{n}) \in \bC$, and $\tilde{X}_{n}$ its universal covering. It is well-known that this is a contractible space- the classifying space for the braid group on $n$ strands. Now consider the divisor
$$
Y_{m}= \{(z,a_{1}, \ldots , a_{n}) \in \bC\bP^{1} \times X_{n}: z \neq a_{m}\}
$$
and let $\tilde{Y}_{m}$ be its inverse image in $\bC\bP^{1} \times \tilde{X}_{n}$. Furthermore,
define $\tilde{Y}_{\infty} = \{(\infty, x) : x \in \tilde{X}_{n}\}$.

The projection onto the second factor $p : \bC\bP^{1} \backslash \{a_{1}, \ldots ,a_{n}, \infty\}$, and the contractibility of $\tilde{X}_{n}$ implies fro  the exact homotopy sequence that the inclusion $i$ of a fibre induces an isomorphism of fundamental groups
$$
\pi_{1}(\bC\bP^{1} \backslash \{a_{1}^{0}, \ldots, a_{n}^{0}, \infty\})\cong \pi_{1} (\bC\bP^{1} \times \tilde{X}_{n} \backslash \{\tilde{Y}_{1} \cup \ldots\cup \tilde{Y}_{n}\cup  \tilde{Y}_{\infty}\})
$$
Thus a flat connection on $\bC\bP^{1}\backslash \{a_{1}^{0}, \ldots, a_{n}^{0}, \infty\}$ extends to flat connection  with the same holonomy on $\bC\bP^{1} \times \tilde{X}_{n} \backslash \{\tilde{Y}_{1} \cup \ldots\cup \tilde{Y}_{n} \cup  \tilde{Y}_{\infty}\}$. Malgrange's theorem asserts that this flat connection has logarithmic singularities along $\tilde{Y}_{m}$ and $\tilde{Y}_{\infty}$.

 More precisely,

\begin{theorem}\label{art7-thm-1}
{\bf Malgrange \cite{art7-key10}} Let $\nabla^{0}$ be flat holomorphic connection on the vector bundle $E^{0}$ over $\bC\bP^{1}\backslash \{ a_{1}^{0}, \ldots, a_{n}^{0}, \infty\}$, with logarithmic singularities at $a_{1}^{0}, \ldots, a_{n}^{0}$. Then there exists a holomorphic vector bundle $E$ on $\bC \bP^{1} \times \tilde{X}_{n}$ with a flat connection $\nabla$ with logarithmic singularities at $\tilde{Y}_{1}, \ldots,\tilde{Y}_{n}, \tilde{Y}_{\infty}$ and an isomorphism $j : \i^{*} (E, \nabla) \rightarrow (E^{0}, \nabla^{0})$. Furthermore, $(E, \nabla, j)$ is unique up to isomorphism.
\end{theorem}

 Now suppose that $E^{0}$ is holomorphically trivial. The vector bundle $E$ will not necessarily be trivial on all fibres of the projection $p$, but for a dense open set $U \subseteq \tilde{X}_{n}$ it will be. Choose a basis $e_{1}^{0}, e_{2}^{0}, \ldots e_{m}^{0}$ of the fibre of $E^{0}$ at $z = \infty$. Now since $\nabla$ has a logarithmic singularity on $\tilde{Y}_{\infty}$, it induces a flat connection there, and since $\tilde{Y}_{\infty} \cong \tilde{X}_{n}$ is simply connected, by parallel translation we can unambiguously extend $e_{1}^{0}, e_{2}^{0}, \ldots, e_{m}^{0}$ to trivialization of $E$ over $\tilde{Y}_{\infty}$. Then since $E$ is holomorphically trivial on each fibre over $U$. we can uniquely extend $e_{1}^{0}, e_{2}^{0}, \ldots, e_{m}^{0}$ along the fibres to obtain a trivialization $e_{1}, \ldots, e_{m}$ of $E$ on $\bC\bP^{1} \times U$. It is easy to see that, relative to this trivialization, the connection form pf $\nabla$ can be written
\begin{equation}\label{art7-eq-2}
A = \sum\limits_{i=1}^{n}A_{i} \dfrac{dz-da_{i}}{z-a_{i}}
\end{equation}
where $A_{i}$ is a holomorphic function of $a_{1}, \ldots, a_{n}$.

The flatness of the connection can then be expressed as:
$$
dA_{i} + \sum\limits_{j\neq i}[A_{i}, A_{j}]\dfrac{da_{i}-da_{j}}{a_{i}-a_{j}} =0
$$
which is known as \textit{Schlesinger's equation} \cite{art7-key15}.

The gauge freedom in this equation involves only the choice of the initial basis $e_{1}^{0}, e_{2}^{0}, \ldots, e_{m}^{0}$ and consists therefore of conjugation of the $A_{i}$ by a constant matrix.

The case which interests us here in where the holonomy lies in $SL(2,\bC)$, (so that the $A_{i}$ are trace-free 2 $\times$ 2 matrices), and where there are 3 marked points $a_{1}, a_{2}, a_{3}$ which, together with $z=\infty$, are the singular points of the connection. By a projective transformation we can make these points 0,1, $x$, Then
$$
A(z)= \dfrac{A_{1}}{z} + \dfrac{A_{2}}{z-1} + \dfrac{A_{3}}{z-x}
$$
and Schlesinger's equation becomes:
\begin{align}\label{art7-eq-3}
\dfrac{dA_{1}}{dx} &= \dfrac{[A_{3}, A_{1}]}{x}\nonumber\\
\dfrac{dA_{2}}{dx} &= \dfrac{[A_{3}, A_{2}]}{x-1}\\
\dfrac{dA_{3}}{dx} &= \dfrac{-[A_{3}, A_{1}]}{x} - \dfrac{A_{3}, A_{2}}{x-1}\nonumber
\end{align}
where the last equation is equivalent to
$$
A_{1} + A_{2} +A_{3} = -A_{\infty} = \text{const}.
$$

The relationship with the Painlev\'e equation can best be seen by following \cite{art7-key8}. Each entry of the matrix $A_{ij}(z)$ is of the form $q(z)/z(z-1)(z-x)$ for some quadratic polynomial $q$. Suppose that $A_{\infty}$ is diagonalizable, and choose a basis such that
$$
A_{\infty} =
\left(\begin{matrix}
\lambda & 0 \\
0 & -\lambda 
\end{matrix}\right)
$$
then $A_{12}$ can be written
\begin{equation}\label{art7-eq-4}
A_{12}(z) = \dfrac{k(z-y)}{z(z-1)(z-x)}
\end{equation}

\begin{align}\label{art7-eq-5}
 \dfrac{d^{2}y}{dx^{2}} &= 1/2\left(\dfrac{1}{y} + \dfrac{1}{y-1} + \dfrac{1}{y-x} \right)\left(\dfrac{dy}{dx}\right)^{2}
 -\left( \dfrac{1}{x} + \dfrac{1}{x-1} + \dfrac{1}{y-x}\right)\dfrac{dy}{dx}\nonumber\\
 & +\dfrac{y(y-1)(y-x)}{x^{2}(x-1)^{2}} \left(\alpha + \beta\dfrac{x}{y^{2}} + \gamma\dfrac{x-1}{(y-1)^{2}} + \delta\dfrac{x(x-1)}{(y-x)^{2}} \right)
\end{align}
where
\begin{align}\label{art7-eq-6}
\alpha &= (2\lambda -1)^{2}/2\nonumber\\
\beta &= 2d\det A_{1}^{2}\nonumber\\
\gamma &= -2\det A_{2}^{2}\nonumber\\
\delta &= (1 + 4\det A_{3}^{2})/2
\end{align} 
For the formulae which reconstruct the connection from $y(x)$ we refer to \cite{art7-key8}, but essentially the entires of tha $A_{i}$ are rational functions of $x, y$ and $dy/dx$. For our purposes it is useful to  note the geometrical form of the definition of $y(x)$ given by \ref{art7-eq-4}:

\begin{prop}\label{art7-proposition-1}
The solution $y(x)$ to the Painlev\'e equation corresponding to an isomonodromic deformation $A(z)$ is the point $y \in \bC\bP^{1} \backslash \{ 0,1,\break x, \infty\}$ at which $A(y)$ and $A_{\infty}$ have a common eigenvector.
\end{prop}

Note that strictly speaking there are two Painlev\'e equations (with $\alpha = (\pm 2\lambda -1)^{2}/2)$ correspinding to the values of $y$ with this property.

\section{Equivariant compactifications}

Consider the three- dimensional complex Lie group $SL(2, \bC)$ and the Lie algebra-valued 1-form
$$
A = -(dg)a^{-1}.
$$
The form $A$ is the connection form for a trivial connection on the trivial bundle. It simply relates teh trivializations of the principle frame bundle by left and right translation.

Now let $\Gamma$ be a finite subgroup of $SL(2, \bC)$. Then $SL(2, \bC)/\Gamma$ is non-compact ciompelx manifold and since $A$ is invariant under right translations,it descentds to this quotient. Thus, on $SL(2, \bC)/\Gamma$, $A$ defines a flat connection on the trivial rank 2 vector bundle. Its holonomy is tatutologically $\Gamma$.

In this section, we shall consider an equivarient compactification of $SL(2, \bC)/\Gamma$,  that is to say, a compact compelx manifold $Z$ on which $SL(2, \bC)$ acts with a dense open orbit with stabilizer conjugate to $\Gamma$. Let $Z$ be such a compactification, then the action of the group embeds the lie algebra $\underline{g}$  in th space of holomorphic vector fields on $Z$. Equivalently, we have a vector bundle homomorphism
$$
\alpha: Z \times \underline{g} \rightarrow TZ
$$
which is generically an isomorphism.It fails to be an isomorphism on the union of the lower dimensional orbits of $SL(2, \bC)$, and this is where $\bigwedge^{3} \alpha \in H^{0}(Z, \Hom(\bigwedge^{3} \underline{g}, \bigwedge^{3}T)) \cong H^{0}(Z, K^{-1})$ is a section of the anticanonical bundle, so the union of the orbits of dimension less than three form an \textit{anticanonical divisor} $Y$, which may of course have several components or be singular.

 In the open orbit $Z/Y \cong SL(2, \bC)/\Gamma$, the action is equivalent to left multiplication, and the connection $A$ above is given by
 $$
 A = \alpha^{-1}: TZ \rightarrow  Z \times \underline{g}.
 $$
It is clearly meromorphic on $Z$, but more is true.

\begin{prop}\label{art7-proposition-2}
If $\bigwedge^{3}\alpha$ vanishes non-degenerately on the divisor $Y$, then the connection $A = \alpha^{-1}$ has a logarithmic singularity along $Y$.
\end{prop}

This is a local statement, and so it can always be applied to the smooth part of $Y$ even if there are singular points.

\begin{proof}
In local coordinates, $\alpha$ is represented by a holomorphic function $B(z)$ with values in the space of 3 $\times$ 3 matrices. The divisor $Y$ is then the zero set of $\det B$. If $\det B$ has a non-degenerate zero at $p \in Y$, then its null-space is one-dimensional at $p$, so the kernel of $\alpha$, the the Lie algebra of the stabilizer of $p$, is one-dimensional. Thus the $SL(2, \bC)$ orbit through $p$ is two-dimensional, and so $Y$ is the orbit.

Now for any quare matrix $B$, let $B\spcheck$ denote the transpose of the matrix of cofactors. Then it is well-known that
$$
BB\spcheck = (\det B)I
$$
Hence in local coordinates
$$
A = \alpha^{-1} = \dfrac{B\spcheck}{\det B}
$$
and so $A$ has a simple pole along $Y$. From Definition \ref{art7-definition-1}, we need to show that the residue in the conormal direction. For this consider the invariant description of $B\spcheck$. We have on $Z$
$$
\wedge^{2} \alpha : \wedge^{2}\underline{g} \rightarrow \wedge^{2}T
$$ 
and using the identifications $\wedge^{2}\underline{g} \cong \underline{g}^{*}$ and $\wedge^{2}T \cong T^{*} \otimes \wedge^{3}T, B\spcheck$ represents the dual map of $\wedge^{2}\alpha$:
$$
(\wedge^{2}\alpha)^{*} : T\rightarrow \underline{g} \otimes \wedge T.
$$

Now the image of $\alpha$ at $p$ is the tangent space to the orbit $Y$ at $p$ by the definition of $\alpha$. Thus the image of $\wedge^{2} \alpha$ is $\wedge^{2}TY_{p}$ which means that $ (\wedge^{2}\alpha)^{*}$ annihilates $TY$, which is the required result.

Note that the kernel of $\wedge^{2}\alpha$ is the set of two-vectors $v \wedge w$ where $w \in \underline{g}$ and $v$ is ion the Lie algebra of the stabilizer of $p$. Thus the residue at $p$ of the connection $A$ lies in the Lie algebra of the stabilizer.
\end{proof}

Now suppose that $P$ is a rational curve in $Z$ which meets $Y$ transversally at four points. Then the restriction of $A$ to $P$ defines a connection with logarithmic poles at the points and, from the map of fundamental groups
$$
\pi_{1}(P\backslash \{a_{1},\ldots, a_{4}\})\rightarrow \pi_{1}(Z\backslash Y)\rightarrow \Gamma \rightarrow SL(2, \bC),
$$
its holonomy is contained in $\Gamma$. A deformation of $P$ will define a nearby curve in the same homotopy class and hence the induced connection will have the same holonomy. To obtain isomonodromic deformations, we therefore need to study the deformation theory of such curves.

\begin{prop}\label{art7-proposition-3}
Let $p \subset Z$ be a rational curve meeting $Y$ transversally at four points. Then $P$ belongs to a smooth four-parameter family of rational curves on which the cross-ratio of the points is nonconstant function.
\end{prop}

\begin{proof}
Th proof is standard Kodaira-Spencer deformation theory. By hypothesis $P$ meets the anticanonical divisor $Y$ in four points, so the degree of $K_{Z}$ on $P$ is -4. Hence, in $N$ is the normal bundle of $P \cong \bC\bP^{1}$,
$$
\deg N = -\deg K_{Z} + \deg K_{P} = 2
$$
and so
$$
N \cong \calO(m) \oplus \calO(2-m)
$$
for some integer $m$. However, since $C$ is transversal to the 2-dimensional orbit $Y$ of $SL(2, \bC)$, the map $\alpha$ always maps \textit{onto} the normal bundle to $C$. We therefore have a surjective homomorphism of holomorphic vector bundles
$$
\beta : \calO \otimes \underline{g} \rightarrow N
$$
and this implies that $ 0 \leq m \leq  2$. As a consequence, $H^{1} (P ,N) = 0$ and $H^{0}(P, N)$ is four-dimensional, so the existence of a smooth family follows fro  Kodaira \cite{art7-key9}. 

Since $\beta$ is surjective, its kernel is a line bundle of degree-$\deg N = -2$, so we have an exact sequence of sheaves:
$$
\calO(-2)\rightarrow \calO \otimes \underline{g}\rightarrow N.
$$
Under $\alpha$, the kernel maps isomorphically to the sheaf of sections of the tragent bundle $TP$ which vanish at the four points $P\cap Y$. From the long exact cohomology sequence we have
$$
0 \rightarrow \underline{g}\rightarrow H^{0}(P, N)\xrightarrow{\delta} H^{1}(P, \calO(-2))\rightarrow 0
$$
and since $H^{0}(P, N)$ is 4-dimensional and $\underline{g}$ is 3-dimensional, the map $\delta$ id surjective. But $\alpha \delta$ is the Kodarira-Spencer map for deformations of the four points on $P$, so since it is non-trivial, the cross-ratio is non-constant.
\end{proof}

\begin{example*}
~

\begin{enumerate}[]
\item As the reader may realize, the situation here is very similar to the study of twistor spaces and twistor lines, and indeed there is a differential geometric context for this (see \cite{art7-key6}, \cite{art7-key7}). This is not the agenda for this paper, but it is a useful example to see the standard twistor space-$\bC\bP^{3}$ and the straight linex in it-within the current context.

Let $V$ be the 4-dimensional space of cubic polynomials
$$
p(z)=c_{0} + c_{1}z +c_{2}z^{2} + c_{3}z^{3}
$$
and consider $V$ as a representation space of $SL(2, \bC)$ under the action
$$
p(z)\mapsto p \left( \dfrac{az + b}{cz + d}\right)(cz + d)^{3}.
$$
This is the unique (up to isomorphism) 4-dimensional irreducible representation of $SL(2, \bC)$. Then $Z= P(V) = \bC\P^{3}$ is a compact threefold with an action of $SL(2, \bC)$ and moreover the open dense set of cubics with distinct roots in an orbit. This follows since given any two triplex of distinct ordered points in $\bC\bP^{1}$, there is a unique element of $PSL(2, \bC)$ which takes one to the other. However, the cubic polynomial determines an \textit{unordered} triple of roots, and hence the stabilizer in $PSL(2, \bC)$ is the symmetric group $S_{3}$. Thinking of this as the symmetries of an equilateral triangle, the holonomy group $\Gamma \subset SL(2, \bC)$ of the connection $A= \alpha^{-1}$ is the binary dihedral group $\tilde{D}_{3}$. The lower-dimensional orbits consist firstly of the cubics with one repeated root, which is
2-dimensional, and those with a triple root, which constitute a rational normal curve in $\bC\bP^{3}$. Together they form the discriminant divisor $Y$, the anticanonical divisor discussed above.

A generic line in $\bC\bP^{3}$, generated by polynomials $p(z), q(z)$ meets $Y$ at those values of $t$ for which the discriminant of $tp(z) + q(z)$ vanishes, i.e. where
\begin{align*}
tp(z) +q(z) &= 0\\
tp'(z)+q'(z) &= 0
\end{align*}
have a common root. This occurs for $t = -q(\alpha)/p(\alpha)$ where $\alpha$ is a root of the quartic equation
$$
p'(z)a(z)-p(z)q'(z) =0
$$
and so the line meets $Y$ in four generically distinct points. Thus the 4-parameters family of lines in $\bC\bP^{3}$ furnish an example of the above proposition.
\end{enumerate}
As we remarked above, this is an example of an isomonodromoc deformation, as would be any family of curves $P$ in Proposition \ref{art7-proposition-3}. It yields a solution of the Painlev\'e equation either by applying the argument of Theorem \ref{art7-thm-1} to the connection with logarithmic singularities on $Z$, or appealing to the universality of Malgrange's construction. We shall not derive the solution of the Painlev\'e equation here from $\bC\bP^{3}$, since it will appear via a different compactification in the context of Poncelet polygons. There we shall also see how a striaght line in $\bC\bP^{3}$ defines a pair on conics with the Poncelet property for triangles.
\end{example*}

\section{Poncelet polygons and projective bundles}\label{art7-sec-4}

In this section we shall study a particular class of equivariant compactifications, originally due to Schwarzenberger \cite{art7-key16}. Consider the complex surface $\bC\bP^{1}\times \bC\bP^{1}$ and the holomorphic involution $\sigma$ which interchanges the two factors. The quotient space is $\bC\bP^{2}$. A profitableway of viewing this is a the map which assigns to a pair of complex numbers the coefficients of the quadratic polynomial which has them as roots. In affine coordinates we have the quotient map
\begin{align*}
\pi : \bC\bP^{1} \times \bC\bP^{1} &\rightarrow \bC\bP^{1}\\
(w, z) &\mapsto (-(w +z), wz).
\end{align*}
From this it is clear that $\pi$ is a double covring branched over the image of the diagonal, which is the
conic $b\subset \bC\bP^{2}$ with equation $4y =x^{2}$. Moreover the line $\{a\} \times \bC\bP^{1} \subset \bC\bP^{1} \times \bC\bP^{1}$ maps to a line in $\bC\bP^{2}$ which meets $B$ at the single point $\pi(a, a)$. The images of the two lines  $\{a\} \times \bC\bP^{1}$ and $\bC\bP^{1} \times \{b\}$ are therefore the two tangents to the conic $B$ from the point $\pi(a, b)$.

Now let $\calO(k, l)$ denote the unique holomorphic line bundle of bidegree $(k,l)$ on $\bC\bP^{1}\times \bC\bP^{1}$, and define the direct image sheaf $\pi_{*}\calO(k, 0)$ on $\bC\bP^{2}$. This is a locally free sheaf, a rank 2 vector bundle $V_{k}$, and we may form the  projective bundle $P(V_{k})$, a complex 3-manifold which fibres over $\bC\bP^{3}$
$$
p : P(V_{k})\rightarrow \bC\bP^{2}
$$
with fibres $\bC\bP^{1}$.

Clearly the diagonal action of $SL(2,\bC)$ on $\bC\bP^{1} \times \bC\bP^{1}$ induces an action on $P(V_{k})$. Take a point $z \in P(V_{k})$ and consider its stabilizer. If $p(z) \in \bC\bP^{2}\backslash B$, then $p(z) = \pi(a, b)$ where $a\neq b$. Consider the projective bundle pulled back to $\bC\bP^{1} \times \bC\bP^{1}$. The point $(a, b)$ is off the diagonal in $\bC\bP^{1}\times \bC\bP^{1}$. so the fibre of $p(V_{k}) = P(\pi_{*}\calO(k, 0))$ is
\begin{equation}\label{art7-eq-7}
P(\calO(k,0)_{a}\oplus \calO(k,0)_{b}).
\end{equation}
The stabilizer of $(a, b)$ in $SL(2, \bC)$ is on 3-dimensional, and acts on\break $(u, v) \in \calO (k, 0)_{a}\oplus \calO(k,0)_{b}$ as
$$
(u, v)\mapsto(\lambda^{k}u, \lambda^{-k}v).
$$
Thus, as long as $u \neq 0$ or $v\neq 0$, the stabilizer of the point represented by $(u, v)$ in the fibre in finite. Thus the generic orbit is three-dimensional.

We have implicitly just defined the divisor $Y$ of lower-dimensional orbits, but to be more prescise, we have the inverse image of the branch locus
$$
D_{1}= \pi^{-1}(B)
$$
as one component. The other arises from the direct image construction as follows.

Recall that by definition of the direct image, for any open set $U\subseteq \bC\bP^{2}$,
$$
H^{0}(U, V_{k}) \cong H^{0}(\pi^{-1}(U), \calO(k, 0))
$$
so that there is an evaluation map
$$
\ev: H^{0}(\pi^{-1}(U), \pi^{*}v_{k})\rightarrow H^{0}(\pi^{-1}(U), \calO(k,0)).
$$
The kernel of this defines a distinguished line sub-bundle of $\pi^{*}(V_{k})$ and thus a section of the pulled back projective bundle$P(V_{k})$. This copy oc $\bC\bP^{1} \times \bC\bP^{1}$ in $P(V_{k})$ is a divisor $D_{2}$.

Both divisors are components of the anticanonical divisor $Y$, and it  remains to check the multiplicity. Now let $U$ be the divisor class of the tautological line bundle over the projective bundle $P(V_{k})$..The divisor $D_{2}$ is a section of $P(V_{k})$ pulled back to $\bC\bP^{1} \times\bC\bP^{1}$, and from its definition it is in the divisor class $p^{*}(-U) + \calO(k,0)$. Thus in $P(V_{k})$,
\begin{equation}\label{art7-eq-8}
D_{2}\sim -2U + kH
\end{equation}
where $H$ is the divisor class of the pull-back by $\pi$ of the hyperplane bundle on $\bC\bP^{2}$. Clearly, since $B$ is a conic,
\begin{equation}\label{art7-eq-9}
D_{1} \sim 2H.
\end{equation}
Now from Grothendieck-Riemann-Roch applied to the projection $\pi$, we find $c_{1}(V_{k})=(k-1)H$, from which it is easy to see that the canonical divisor class is
$$
K\sim 2U -(k+2)H
$$
so since $-k\sim -2U +(k+2)H\sim D_{1} + D_{2}$, the multiplicity in 1 for each divisor and we can take $Z =P(V_{k})$ as an example of an equivariant compactification to which Proposition 2 applies.

The stabilizer of a point in $Z\backslash Y$ is in this case the binary dihedral group $\tilde{D_{k}}$, which is the inverse image in $SU(2)$ of the group of symmetries in $SO(3) \cong SU(2)/\pm 1$ of a regular plane polygon  with $k$ sides. Although this can be seen quite easily from the above description of the action, there is a direct way of viewing $Z\backslash Y - P(V_{k})\backslash D_{1}\cup D_{2}$ as the $SL(2, \bC)$ orbit of a plane polygon.

Note that a polygon centred on $0 \in \bC^{3}$ is described by a non-null axis orthogonal to the plane of the polygon, and by $k$ (if $k$ is odd) or $k/2$ (if $k$ is even) equally spaced axes through the origin in that plane. Now, given a point $z\in P(V_{k})\backslash D_{1}\cup D_{2}$, its projection $p(z)= x \in P(\bC^{3})\backslash B$ is a non-null direction in $\bC^{3}$ which we take to be the axis. To find the other axes we use two facts: 
\begin{itemize}
\item The map $s \mapsto s^{k}$ from $\calO(1,0)$ to $\calO(k,0)$ defines a rational map $m_{k} : P(V_{1})\rightarrow P(V_{k})$ of degree $k$.
\item The projective bundle $P(V_{2})$ is the projectivized tangent bundle $P(T)$ of $\bC\bP^{2}$.
\end{itemize}
The first fact is a direct consequence of the definition of the direct image sheaf:
$$
H^{0}(U, V_{k})\cong H^{0}(\pi^{-1}(U), \calO(k,0))
$$
for any open set $U \subseteq \bC\bP^{2}$. The second can be found in \cite{art7-ket16}.

 Given these two facts, consider the set of points
 $$
 m_{2}(m_{k}^{-1}(z))\subset P(V_{2}).
 $$
Depending on the parity of $k$ this consists of $k$ or $k/2$ points in $P(T)$ all of which project to $x \in \bC\bP^{2}$. In other words they are line through $x$ or, using the polarity with respect to the conic $B$. points on the polar line of $x$. Reverting to linear algebra, these are axes in the plane orthogonal to $x$.

We now need to apply Proposition 3 to this particular set of examples to find rational curves which meet the divisor $Y = D_{1}+D_{2}$ transversally in four points. Now if $P$ is such a curve, then the intersection number $P$. $D_{1}\leq 4$ so $p(P) =C $ is a plane curve of degree $d$ which meets the branch conic $B$ in $2d \leq 4$ points, hence $d = 1$ or 2. We consider the case $d= 2$ first. The curve $C$ is a coniv in $\bC\bP^{2}$. The set of all conics forms a 5-parameter family and we want to determine the 4-parameter family of conics which lift to $P(V_{k})$.

\begin{theorem}\label{art7-thm-2}
A conic $C\subset \bC\bP^{2}$ meeting $B$ transversally lifts to $P(V_{k})$ if and only if there exists a $k$-sided polygon inscribed in $C$ and circumscribed about $B$.
\end{theorem}

\begin{proof}
A lifting of $C$ is a section $P(V_{k})$ over $C$, or equivalently a line subbundle $M\subset V_{k}$ over $C$. Since $C$ is a conic, the hyperplane bundle $H$ is of degree 2 on $C$, so we can write $M\cong H^{n/2}$ for some integer $n$. The inclusion $M\subset V_{k}$ thus defines a holomorphic section $s$ of the vector bundle $V_{k}\otimes H^{-n/2}$ over $C$. But $V_{k}$ is the direct image sheaf of $\calO(k, 0)$, so we have an isomorphism
$$
H^{0}(C, V_{k}\otimes H^{-n/2}) \cong H^{0}(\tilde{C}, \calO(k,0) \otimes \pi^{*}(H^{-n/2}))
$$
where $\tilde{C}= \pi^{-1}(C)\subset \bC\bP^{1} \times \bC\bP^{1}$ is the double covering of the conic $C$ branched over its points of intersection with $B$. But it is easy to see that $\pi^{*}(H)\cong\calO(1,1)$ so on $\tilde{C}$ we have a holomorphic section $\tilde{s}$ of $\calO(k-n/2, -n/2)$.

We have more, though, for since the intesection number$-K.P= (D_{1} + D_{2}). P =4$ and $D_{1}.P = B.C =4$, $P$ lies in $P(V_{k})\backslash D_{2}$, where $D_{2}$ was given as the kernel of the evaluation map. It the section $\tilde{s}$ vanishes anywhere, then the section of $P(V_{k})$ will certainly meet $D_{2}$, thus $\tilde{s}$ is everywhere non-vanishing an $\calO(k-n/2, -n/2)$ is the trivial bundle. In particular, its degree is zero on $\tilde{C}$. Now $C$ is a conic, so $\tilde{C}$ is the divisor of a section of $\pi^{*}(H^{2}) \cong \calO (2,2)$ and so the degree of the line bundle is $2k-2n=0$ and thus $n-k$. Hence a conic in $\bC\bP^{2}$ lifts to $P(v_{k})$ if and only if it has the property that
$$
\calO(k/2, -k/2)\cong \calO \quad \text{on}\quad \tilde{C}.
$$

Now recall the Poncelet problem \cite{art7-key3}: to find a polygon with $k$ sides which is inscribed in a conic $C$ and circumscribed about a conic $B$. The projection
$$
\pi: \bC\bP^{1}\times \bC\bP^{1} \rightarrow \bC\bP^{2}
$$
we have already used is the correct setting for the problem.

Let $(a, b)$ be a point in $\bC\bP^{1} \times \bC\bP^{1}$ and consider the two lines $\{a\} \times \bC\bP^{1}$ and $\bC\bP^{1} \times \{b\}$ passing through it. The first line is a divisor of the linear system $\calO(1,0)$ and the second of $\calO(0,1)$. As we have seen, their images in $\bC\bP^{2}$ are the two tangents to the branch conic $B$ from the point $\pi(a, b)$. Now let $C$ be the conic which contains the vertices of the Poncelet polygon, and let $P_{1} =(a_{1}, b_{1}) \in \tilde{C} \subset \bC\bP^{1} \times \bC\bP^{1}$ be a point lying over an initial vertex. The line $\{a_{1}\} \times \bC\bP^{1}$ meets $\tilde{C}\sim \calO(2,2)$ in two points generically,. which are $P_{1}$ and a second point $P_{2} = (a_{1}, b_{2})$. The two points $\pi(P_{1})$ and $\pi(P_{2})$ lie on $C$, and the line joining them is $pi(\{a_{1}\} \times \bC\bP^{1})$ which is tangent to $B$, and hence is a side of the polygon. The other side of the polygon through $pi(P_{2})$ is $\pi(\bC\bP^{1} \times\{b_{2}\})$ which meets the conic $C$ at $\pi(P_{3}) = \pi(a_{2}, b_{2})$. We carry on this procedure using the two lines through each point, to obtain $P_{1}, \ldots, P_{k+1}$. Since the Poncelet polygon is closed with $k$ vertices, we have $\pi(P_{k+1}) = \pi(P_{1})$.

Consider now the divisor classes $P_{i} + P_{i+1}$. We have
\begin{align*}
P_{1}+ P_{2} &\sim \calO(1,0)\\
P_{2}+ P_{3} &\sim \calO(0,1)\\
P_{3}+ P_{4} &\sim \calO(1,0)\\
\ldots  \quad     & \quad \ldots
\end{align*}
and $P_{k} + P_{k+1} \sim =calO{1, 0}$ if $k$ is odd and $\sim \calO(0,1)$ if $k$ is even.

In the odd situation, taking the alternating sum we obtain
\begin{equation}\label{art7-eq-10}
P_{1} +P_{k+1} \sim \calO ((k+1)/2, -(k-1)/2)
\end{equation}
and since $\pi(P_{k+1} = \pi(P_{1}))$, then $P_{k+1} = P_{1}$ or  $\sigma(P_{1})$. However, in the former case, we would have
$$
P_{k}+P_{1}\sim P_{k+1}\sim \calO(1,0)\sim P_{1}+P_{2}
$$
and consequently $P_{2}\sim P_{k}$ on the elliptic curve $\tilde{C}$ which implies $P_{2} = P_{k}$. But $\pi(P_{k})$ and $\pi(P_{2})$ and $\pi(P_{2})$ are different vertices of the polygon, so we must have $P_{k+1} = \sigma P_{1}$. This that the divisor $P_{k+1} +P_{1} = \pi^{-1}(\pi(P_{1}))$ ans so in the notation above
$$
P_{k+1} + P_{1}\sim H^{1/2} = \calO(1/2, 1/2).
$$

From \eqref{art7-eq-10} we therefore obtain the constraint on $\tilde{C}$
\begin{equation}\label{art7-eq-11}
\calO(k/2, -k/2)\sim \calO
\end{equation}
which is exactly the condition for the conic to lift to $P(V_{k})$. A similar argument leads to the same condition for $k$ even, where in this case $P_{k+1} = P_{1}$.

In the case that $d =1$, $C$ is a line, but the argument in  very similar. Here $M\cong H^{n}$ for some $n$ and on $\tilde{C}$ we have a section $\xi$ of $\calO(k-n, -n)$. This time, since $P.D_{2} = 2$, the line bundle is of degree 2, so $k-2n =2$, and so a lifting is defined by a section of $\calO{1+k/2, 1-k/2}$ on $\tilde{C}$.
\end{proof}

\medskip
\begin{example*}

~

\begin{enumerate}[]
 \item Let us now compare this interpretation with the equivariant compactification $\bC\bP^{3}$ of $SL(2, \bC)/\tilde{D}_{3}$ discussed earlier. In the first place, consider the line bundle
    $$
    \tilde{U}= U-2H
    $$
    on $P(V_{3})$.Now since for any 2-dimensional vector space $V^{*}\cong V \otimes \wedge^{2}V^{*}$, $P(V_{k})=P(V_{k}^{*})$, but with different tautological bundles. The tautiological bundle for $P(V_{3}^{*})$ is actually $\tilde{U}$, and so there are canonical isomorphisms
    \begin{align*}
    H^{0}(P(V_{3}), -\tilde{U}) &\cong H^{0}(\bC\bP^{2}, V_{3})\cong (\bC\bP^{1} \times \bC\bP^{1}, \calO(3, 0))\\
       &\cong H^{0}(\bC\bP^{1}, \calO(3))\cong \bC^{4}.
    \end{align*}
    The linear system $|-\tilde{U}|$ therefore maps $P(V_{3})$ equivariantlu to $\bC\bP^{3}$. Since $P.D_{1} = 4$ and $P.D_{2} = 0$, it follows from \eqref{art7-eq-8} and \eqref{art7-eq-9}, that $P.H = 2$ and $P.U = 3$, and so $P.\tilde{U} = -1$, so under this mapping the curves $P$ map to projective lines.

    There is a more geometic way of seeing the relation of lines in $\bC\bP^{3}$ to Poncelet triangles. Recall that we are viewing  $\bC\bP^{3}$ as the space of cubic polynomials, and $\bC\bP^{2}$ as the space of quadratic polynomials. The quadraitcs with a fixed linear factor $z-\alpha$ describe,as we have seen, aline in $\bC\bP^{2}$ which is tangent to the discriminant conic at the quadratic $(z-\alpha)^{2}$. Thus the three linear factors of a cubic $(z-\alpha)$,$(z-\beta)$,$(z-\gamma)$, $(z-\gamma) (z-\alpha)$.

    Now consider a straight line of cubics $p_{t}(z)=tp(z)+q(z)$ with roots $\alpha{t}, beta(t)$ and $\gamma{t}$. We have a 1-parameter family of triangles and
    \begin{align*}
    tp(\alpha) + q(\beta) &= 0\\
    tp(\beta) +q(\beta) &=0.
    \end{align*}
    Now from these two equations
    $$
    0 = p(\alpha)q(\beta) -p(\beta)q(\alpha)= (\alpha-\beta)r(\alpha, beta)
    $$
    where $r(\alpha, \beta)$ is a symmetric polynomial in $\alpha,\beta$. It is in fact \textit{quadratic} in $\alpha  \beta$, $\alpha \beta$ and thus defines a conic $C$ in the plane.

    Hence, as $t$ varies, the vertices of the triangle lie on fixed conic $C$, and we have a solution of the Poncelet problem for $k=3$.
    \end{enumerate}
\end{example*}

\section{Solutions of Painlev\'e VI}\label{art7-sec-5}

To find more about the connection we have just defined on $Z=P(V_{k})$ entails descending to local coordinates, which we do next.

Consider the projective bundle $P(V_{k})$ pulled back to $\bC\bP^{1} \times \bC\bP^{1}$. At a point off the diagonal $(a, b) \in \bC\bP^{1} \times \bC\bP^{1}$, as in \eqref{art7-eq-7}, the fibre is
$$
P(\calO(k,0)_{a} \oplus \calO(k,0)_{b}) = P(\calO(k,0) \oplus \calO(0,k))_{a,b}
$$
and awat from the zero section of the second factor, this is isomorphic to
$$
\calO(k,-k)_{a,b}.
$$

Now choose standard affine coordinates $(w,z)$ in $\bC\bP^{1} \times \bC\bP^{1}$. Since $K_{\bC\bP^{1}} \cong \calO (-2)$, we have corresponding local trivializations $dw$ and $dz$ of $\calO(-2, 0)$ and $\calO(0, -2)$. These define a local trivialization $(dw)^{-k/2}(dz)^{k/2}$ of $calO{k, -k}$, and thus coordinates
$$
(w, z, s)\mapsto s(dw)^{-k/2}(dz)_{(w,z)}^{k/2}.
$$
Note that $Z$ is the quotient of this space by the involution $(w,z, s) \mapsto (z,w,s^{-1})$. From this trivialization, the natural action of $ SL(2, C)$ on differentials gives the action on $Z$:
$$
(w, z, s) \mapsto \left( \dfrac{aw+b}{cw+d},\quad  \dfrac{az+b}{cz+d},\quad \dfrac{(cz+d)^{k}}{(cw +d)^{k}} s\right).
$$
Differenting this expression at the identity gives the tangent vector\break $(w', z', s')$
corresponding to a matrix
$$
\left(
\begin{matrix}
a' & b'\\
c' & -a'
\end{matrix}
\right)
\in \underline{g}
$$
as
\begin{align*}
w' &= -c'w^{2} +2a'w +b'\\
z'&= -c'z^{2} + 2a'z +b'\\
s' &= -kc'(w-z)s
\end{align*}
This is $\alpha(a',b', c') \in TZ_{(w,z,s)}$. Solving for $(a',b',c')$ gives the entries of the matrix of 1-forms $A=\alpha^{-1}$ as 
\begin{align}\label{art7-eq-12}
A_{11} &= \dfrac{dw-dz}{2(w-z)} - \dfrac{(w+z)ds}{2ks(w-z)} \nonumber\\
A_{12} &= \dfrac{wdz-zdw}{(w-z)} + \dfrac{wzds}{ks(w-z)}\\
A_{21} &= -\dfrac{ds}{ks(w-z)}\nonumber
\end{align}

\begin{prop}\label{art7-proposition-4}
The resedue of the connection at a singular point is conjugate to
$$
\left(
\begin{matrix}
1/4 & 0\\
0 & -1/4
\end{matrix}
\right)
\quad 
on\; D_{1} \quad and \quad
\left(
\begin{matrix}
1/2k & 0\\
0 & -1/2k
\end{matrix}
\right)
\quad on \; D_{2}
$$
\end{prop}

 \begin{proof}
In these coordinates, $s=0$ is the equation of $D_{2}$. From \eqref{art7-proposition-2}, the residue of $A$ at $s=0$ is
\begin{equation}\label{art7-eq-13}
\left(
\begin{matrix}
-(w+z)/2k(w-z) & wz/k(w-z)\\
-1/k(w-z) & (w+z)/2k(w-z)
\end{matrix}
\right)
\end{equation}
which has determinant $-1/4k^{2}$ and therefore eigenvalues $\pm1/2k$.

To find the rediduce at $D_{1}$, we need different coordinates, since the above ones are invalid on the diagonal. Take the affine coordinates $x=-(w+z)$, $y=wz$ on $\bC\bP^{2}$. Since the holomorphic functions in $w, z$ form a module over the symmetrix functions generated by 1, $w -z$ we can use these to give  coordinates in the projectivized direct image $P(V_{k})$, which are valid for $w=z$. We obtain an affine fibre coordinate $t$ related to $s$ above by
$$
s = \dfrac{t+w-z}{t-w+z}.
$$
Using this ans local coordinates $x$ and  $u= (w-z)^{2} =x^{2}-4y$ on $\bC\bP^{2}$ the divisor $D_{1}$ is given by $u=0$ and the residue here is
\begin{equation}\label{art7-eq-14}
\left(
\begin{matrix}
1/4+x/2kt & x/4+x^{2}/4kt\\
-1/kt & -1/4-x/2kt
\end{matrix}
\right)
\end{equation}
This has determinant $-1/16$ and hence eigenvalues $\pm1/4$.
 \end{proof}

\begin{remark*}
Exponentiating the residues we see that the holonomy of a small loop around the divisor $D_{1}$ or $D_{2}$ is conjugate to:
$$
\left(
\begin{matrix}
i  & 0\\
0 & -i
\end{matrix}
\right)
\quad on \; D_{1} \quad
\left(
\begin{matrix}
e^{i\pi /k}  & 0\\
0 & e^{-i\pi/k}
\end{matrix}
\right)
\quad on \; D_{2}
$$
In the dihedral group $D_{k} \subset SO(3)$ the conjugacy classes are those of a reflection in the plane and a rotation by $2\pi/k$.

These facts tell us something of the structure of the divisor $D_{1}$. Since we know that the residue of the meromorphic connection at a singular point lies in the Lie algebra of the stabilizer of the point, and this is here semisimple,
 the orbit is isomorphic to
 $$
 SL(2, \bC)/\bC^{*} \cong \bC\bP^{1} \times \bC\bP^{1}\backslash \Delta
 $$
 where $\Delta$ is the diagonal. The projections onto the two factors must, by $SL(2, \bC)$ invariance, be the two eigenspaces of the residue corresponding to the eigenvalues $\pm1/4$.

 Now $D_{1}= \pi^{-1}(B)$ is a projective bundle over the conic $B\cong \bC\bP^{1}$. By invariance it must be one of the factors above. To see which, note that from \eqref{art7-eq-14}, eigenvectors for the eigenvalues 1/4 and -1/4 are respectively.
 $$
 \left(
 \begin{matrix}
x+kt\\
-2
 \end{matrix}
\right)
\quad \text{and}\quad
\left(
 \begin{matrix}
x\\
-2
 \end{matrix}
\right)
$$
and so, from the choice of coordinates above, clearly the second represents the projection ot $B$. Note, moreover, that on the diagonal $w=z$, the coordinate $x=-(w+z)=-2z$, so that $x$ is an affine parameter on $B\cong \Delta \cong \bC\bP^{1}$. Furthermore, when $x=\infty$ the vector
$
\left(
\begin{matrix}
1\\
0
\end{matrix}
\right)
$
is an eigenvector of the residue with eigenvalue -1/4.

Now let us use this information to determine the solution to tyhe Painlev\'e equation corresponding to a rational curve $P\subset Z$. Recall that the curve $C=\pi(P)$ is a plane curve of degree $d$, where $d=1$ or $d=2$. As we have seen,when $d=1$, any line is of this form, but when $d=2$, the conic must circumscribe a Poncelet polygon.

By the $SL(2, \bC)$ action, we can assume that $C$ meets the conic $B$ at the poitn $x=-\infty$. From the discussion above, if $A_{\infty}$ is the residue of the connection at this point, then
$$
A_{\infty}
\left(
\begin{matrix}
1\\
0
\end{matrix}
\right) 
= -\dfrac{1}{4}
\left(
\begin{matrix}
1\\
0
\end{matrix}
\right) 
$$
From Proposition \ref{art7-proposition-1}, the solution of the Painlev\'e equation is the point $y$ on the curve $P$ at which $A(y)$ has this same eigenvector, i.e. where
$$
A_{21}(y)=0
$$
\end{remark*}

\begin{prop}\label{art7-proposition-5}
A line in the plane defines a solution to Painlev\'e sixth equation with coefficients $(\alpha, \beta, \gamma, \delta) = (1/8, -1/8,1/2k^{2},1/2-1/2k^{2})$. A Poncelet conic in the plane defines a solution to the Painlev\'e equation with coefficients $(\alpha, \beta,\gamma, \delta) = (1/8, -1/8, 1/8, 3/8)$.
\end{prop}

\begin{proof}
The residuces on $D_{1}$ and $D_{2}$ are given by \eqref{art7-eq-14} and \eqref{art7-eq-13}. The lifting of a line meets $D_{1}$ and $D_{2}$ in two points each, so using \eqref{art7-eq-6} (and taking account of the fact that the roles of the two basis vectors ar interchanged), we obtain the first set of coefficients. The lifting of a Poncelet conic meets $D_{1}$ in  four points, which gives the second set, agin from \eqref{art7-eq-6}.
\end{proof}

\begin{prop}\label{art7-proposition-6}
The lifting of a line in $\bC\bP^{3}$ to $P(V_{k})$ defines the solution
$$
y = \sqrt{x}
$$
of Painlev\'e VI with $(\alpha, \beta, \gamma, \delta)= (1/8,-1/8, 1/2k^{2},1/2-1/2k^{2})$.
\end{prop}

\begin{proof}
Taking tha double covering $\tilde{C} \subset \bC\bP^{1} \times \bC\bP^{1}$ of the line $C$ and the coordinates $w,z,s$ on the corresponding covering of $Z$, the lifted curve $P$ id defined locally by a function $s$ on the curve
$\tilde{C}$. in fact , as we shall see next, $s$ is a meromorphic function on $\tilde{C}$ with certain properties.

From the comments following Theorem \ref{art7-thm-2}, the lifting is given by a holomorphic section $\xi$ of $\calO(1+k/2, 1-k/2)$. This line bundle has degree 2 on $\tilde{C}$, and so $\xi$ vanishes at two points. Applying the involution $\sigma$, the $\sigma^{*}\xi$ is a section of $\calO(1-k/2, 1+k/2)$. Considering $\xi$ as a section of $\calO(k,0)\otimes \calO(1-k/2, 1+k/2)$. Considerign $\xi$ as a section of $\calO(k, 0)\otimes \calO(1-k/2, 1-k/2)$, teh lifting of $C$ to $P(V_{k})$ is defined by $(\xi_{a,b}, \xi_{b,a})$, or in the coordinates $w,z,s,$
\begin{equation}\label{art7-eq-15}
s(dw)^{-k/2}(dz)^{k/2} = \xi/\sigma^{*}\xi.
\end{equation}
Since $dw^{-1/2}$ and $dz^{-1/2}$ are holomorphic sections of $\calO(1,0)$ and $\calO(0,1)$, it follows that on $\tilde{C}$, $s$ is a meromophic function. Now using the $SL(2, \bC)$ action, we may assume that the line $C$ is given by $x=0$, which means that $\tilde{C}$ has equation
$$
w= -z
$$
which defines as obvious trivialization of $\calO(k,-k)$ and from which we deduce that $s$ has  two simple zeros at $(a_{1,-a_{1}})$, $(a_{2}, -a_{2})$ and two poles at $(-a_{1}, a_{1})$, $(-a_{2}, a_{2})$. Using $z$ as an affine parameter on $\tilde{C}$, wer obtain, up to a constant multiple,
\begin{equation}\label{art7-eq-16}
s = \dfrac{(z+a_{1})(z+a_{2})}{(z-a_{1})(z-a_{2})}.
\end{equation}
Now $y = wz$ is an affine parameter on the line $C$, which meets the conic $B$ at $y= 0, \infty$. The lifting $P$ meets the divisor $D_{2}$ where $y=-a_{1}^{2}$ and $y=-a_{2}^{2}$, so putting $a_{1}=i$ and $a_{2}=\sqrt{-x}$ then $P$m is a projective line with a parametrization such that the singular points of the induced connection are $0,1,x, \infty$, as required for the Painlev\'e equation.

It ramians to determine the solution of the equation, which is given by $A_{21}(y)=0$. But from \eqref{art7-eq-12}, this is where $ds=0$, and from \eqref{art7-eq-16} this is equivalent to
$$
\dfrac{1}{z+a_{1}}- \dfrac{1}{z-a_{1}}+\dfrac{1}{z+a_{2}}-\dfrac{1}{z-a_{2}} =0
$$
which gives
$$
y=-z^{2} = -a_{1}a_{2}= \sqrt{x}
$$
with the above choices of $a_{1}, a_{2}$.
\end{proof}

\begin{remarks*}
~

\begin{enumerate}[1.]
\item By direct calculation, the fucntion $y=\sqrt{x}$ solves Painlev\'e VI for any coefficient satisfying $\alpha + \beta = 0$ and $\gamma +\delta =1/2$. From \eqref{art7-eq-6} this occurs when the residues are conjugate in pairs.

\item When $k=2$, we obtain $(\alpha, \beta, \gamma, \delta) = (1/8, -1/8, 1/8, 3/8)$ which are the coefficients arising from Poncelet conics. We shall see the same solution appearing in the next section in the context of Poncelet quadrilaterals.
\end{enumerate}
\end{remarks*}

\noindent
Naturally, the solutions corresponding to Poncelet conics are more complicated, and we shall give some explicitly in Section \ref{art7-sec-6}. Here we give the general algebraic procedure for obtaining them.

In the case of a conic $C$ in $\bC\bP^{2}$, we have a section $\xi$, in fact a trivialization, of the bundle $\calO(k/2, -k/2)$ on the elliptic curve $\tilde{C}$. As in \eqref{art7-eq-15}, we still define the lifting by
$$
s(dw)^{-k/2}(dz)^{k/2}=\xi\sigma^{*}\xi
$$
but in this case $\xi$ is non-vanishing. The section $(dw)^{-k/2}$ vanishes to order $k$ at $\infty \in \bC\bP^{1}$, and so at the two points $(\infty, infty)$, $(\infty, b)$ where $\tilde{C}$ meets $\{\infty\} \times \bC\bP^{1}$. Similarly $(dz)^{-k/2}$ vanishes at $(\infty, \infty)$, $(b, \infty)$. The meromorphic function $s$ can be regarded as a map oc curves
$$
s:\tilde{C}\rightarrow \bC\bP^{1}.
$$
It follows that $s$ is a meromorphic function on $\tilde{C}$ with a zero of order $k$ at $(b, \infty)$, a pole of order $k$ at $(\infty, b)$ and no other zeros or poles.

The derivative $ds$ in  invariantly defined as a section of $K_{\tilde{c}} \otimes s^{*}K_{\bC\bP^{-1}}^{-1} \cong s^{*}\calO(2)$ (since $\tilde{C}$ is an elliptic curve and hence has trivial canonical bundle). In particular $ds$ vanishes with total multiplicity $2k$. But since $(\infty, a)$ and $(a, \infty)$ are branch points of order $k$, $ds$ has a zero of order $k-1$ at each of these points, leaving two extra points as the remaining zeros. Since the involution $\sigma$ takes $s$ to $s^{-1}$, these points are paired by the involution, and give a single point $y \in \bC\bP^{1}$ which is our solution to the Painlev\'e equation.

Fortunately Cayley's solution in 1853 to the Poncelet problem gives us means to find $y$ algebraically. A usefull modern account of this is given by Griffiths and Harris in \cite{art7-key5}, but the following description I owe to $M. F.$ Atiyah.

Suppose the elliptic curve $\tilde{C}$ is described as a cubic in $\bC\bP^{2}$ given by $v^{2}=h(u)$, where $h(u)$ is a cubic in polynomial and $h(0)= c_{0}^{2} \neq 0$. We shall find the condition on the coefficients of $h$ in order that there should exist a polynomial $g(v,u)$ of degree $(n-1)$ (a section of $\calO(n-1))$ on the curve with a zero of order $(2n-1)$ at $(v,u)=(c_{0},0)$ and a pole of order $(n-2)$ at $u=\infty$. Given such a polynomial,

