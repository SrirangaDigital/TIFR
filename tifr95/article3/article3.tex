\title{Sur la cohomologie de certains espaces de modules de fibr\'es vectoriels}
\markright{Sur la cohomologie de certains espaces de modules de fibr\'es vectoriels}

\author{By~ Arnanud Beauville\footnote{Avec le support partiel du projet europe\'een Science ``Gremetry of Algebraic Varieties", Contrat $n^{\circ}$ SCI-0398-C(A).}}
\markboth{Arnaud Beauville}{Sur la cohomologie de certains espaces de modules de fibr\'es vectoriels}

\date{}
\maketitle

\begin{center}
D\'edi\'e \`a M.S. Narasimhan et C.S. Seshadri pour leur $60^{{\grave{e}}me}$ anniversaire
\end{center}

Soit $X$\pageoriginale une surface de Riemann compacte. Fixons des entiers $r$ et $d$ \textit{premiersm entre eux}, avec $r\geq 1$, et notons $\calM$ 1' espace des modules $\calU_{x}(r,d)$ des fibri\'es $d$. C'est une vari\'et\'e projective et lisse, et il existe un \textit{fibr\'e de Poinca\'e $\bf\calE$ sure $X \times \calM$}; cela signifie que pour tout point $e$ de $\calM$, correspondant \`a un fibr\'e $E$ sur $X$, la restriction de $\calE$ \`a $X \times{e}$ est isomorphe \`a $E$.  

Notons $p,q$ les projections de $X \times \calM$ sur $X$ et $\calM$ respectivement. Soit $m$ un entier $\leq r$; la 
classe de Chern $c_{m}(\calE)$ admet une d\'ecomposition de K\"unneth
$$
c_{m}(\calE) = \sum\limits_{i}p^{*}\xi_{i} \cdot q^{*} \mu_{i}, 
$$
avec $\xi_{i} \in H ^{*}(X, \bZ), \mu_{i} \in H^{*}(\calM, \bZ),\deg(\xi_{i}) + \deg(\mu_{i}) =2m$. 

Nous dirons que les classes $\mu_{i}$ sont les \textit{composantes de K\"unneth de $c_{m}{\calE}$}. Un des re\'sultats essientiels de \cite{art3-keyA-B} est la d\'etermination d'un ensemble de g\'en\'erateurs de l'alg\'ebre de cohomologie $H^{*}(\calM. \bZ)$; il a la cons\`equence suivante:

\begin{theoreme*}
L'alg\`ebre de cohomologie $H^{*} (calM, \bQ)$ est engendr\'ee par les composantes de K\"unneth des classes de Chern de ${\calE}$. 
\end{theoreme*}

Let but de cette note est de montrer comment la m\'ethode de la diagonale utlilis\'ee dans \cite{art3-keyE-S} fournit une d\'emonstration tr\'es simple de ce th\'eor\`eme. Celcui-ci r\'esulte de l'\'enonc\'e un peu plus g\'en\'eral que voice:

\begin{prop*}
Soient $X$ une vari\'et\'e complex projective et lisse, et $\calM$ un espace de modules espace de modules de faisceaux stables sur $x$ (par rapport \'a une polarisation fix\'ee, cf. \cite{art3-keyM}). On fait les hypoth\`eses suivantes:
\begin{enumerate}[(i)]
\item La vari\'et\'e $\calM$ est projective et lisse.\label{art3-enum(i)}
\item Il existe un faiseceau de Poincar\'e $\calE$ sur $\calM$.\label{art3-enum(ii)}
\item Pour $E$, $F$ dans $\calM$, on a $\Ext^{i}(E, F) =0$ pour $i\geq 2$.\label{art3-enum(iii)}

Alors l'alg\`ebre de cohomologie $H^{*}(\calM, \bQ)$ est engendr\'ee par les composantes de K\"unneth des classes de Chern de $\calE$.   
\end{enumerate}
\end{prop*}

La d\'emostration suit de pr\`es celle du th. 1 de \cite{art3-keyE-S}. Rappelons-enl' id\`ee fondamentale: soit $\delta$ la classe de cohomologie de la diagonale dans $H^{*}(\calM \times \calM, \bQ)$; notons $p$ et $q$ les deux projections de $\cal \times M$ sur $\calM$. Soit $\delta = \sum\limits_{i}p^{*}\mu_{i} \cdot q^{*} \upsilon_{i}$, la d\'ecompostion de K\"unneth de $\delta$; alors l'espace $H^{*}(\calM, \bQ)$ est engendr\'e par les $v_{i}$. En effect, pour $\lambda$ dans $H^{*}(\calM, \bQ)$, on a
$$
\lambda = q_{*}(\delta \cdot p^{*}\lambda) = \sum\deg(\lambda \cdot \mu_{i})v_{i}
$$
d'o\`u notre assertion. Il s'agint donce d' exprimer la classe $\delta$ en fonction des classes de Chern du fibr\'e universel.

Notons $p_{1}, p_{2}$ le deux projections de $C\times \calM \times \calM$ sur $C \times \calM$, et $\pi$ la projections sur  $\calM \times \calM$; d\'esignons par $\calH$ le faisceau Hom$(p_{1}^{*}\calE, p_{2}^{*}\calE)$. Vul',hypoth\`ese (\ref{art3-enum(iii)}), l'hypercohomologie $R\pi_{*}\calH$ est repr\`esent\'ee dans la cat\'egorie d\'eriv\'ee par un complexe de fibr\'es $K^{\bullet}$, nul en degr\'e diff\'erent de O rt 1. Autrement dit, il existe un morphisme de fibr\'es $u$:$K^{0}\longrightarrow K^{1}$ tel qu, on ait, pour tout point $x=(E, F)$ de $\calM$, une suite exacte
$$
0 \rightarrow \Hom(E, F) \rightarrow  K^{0}(x)\xrightarrow{u(x)}k^{1}(x)\rightarrow \Ext^{1}(E, F) \rightarrow 0.
$$

Come l'espace $Hom(E, F)$ est non nul si et seulement si $E$ et $F$ est non nul si et seulement si $E$ $F$ sont isomorphes, on voit que la diagonale $\Delta$ de $\calM \times \calM$ co\"lncide ensemblistement avecle lieu de d\'eg\'en\'erescence $D$ de $u$ (d\'efine par l'annulation des mineurs de rang maximal de $u$). On peut prouver comme dans \cite{art3-keyE-S} l' \'egalit\'e sch\'ematique, mais cela n'est pas n\'ecessaire pour d\'emontrer la proposition. 

Soit $E$ un e\'el\'ement de $\calM$. On a
$$
\rg(K^{0})-\rg(K^{1}) = \dim \Hom(E,F)-\dim \Ext^{1}(E, F)
$$
quel que soit le point $(E, F)$ de $\calM \times \calM$. Puisque $\Ext^{2}(E, E) = 0$, la dimension $m$ de $\calM$ est e\'gale \`a dim $\Ext^{1}(E, E)$; ainsi la sous-vari\'e\'e d\'eterminantale $D$ de $\calM \times \calM$ a la codimension attendue $\rg(K^{1}) - \rg(k^{0}) + 1$. Sa classe de cohomologie $\delta' \in H^{m}(\calM \times M, \bZ)$ est alors donn\'ee par la formule de Proteous  
$$
\delta' = c_{m}(K^{1}-K^{0}) = c_{m}(-\pi!\calH),
$$
o\`u $\pi!$ d\'esigne le foncteur image directe en K-th\'eorie. Cette classe \'etant multiple de la classe $\delta$ de la diagonale, on conclut avec le lemme suivant:

\begin{lemme*}
Soit $\calA$ la sous-$bQ$-alg\`bre de  $H^{*}(\calM, \bQ)$ engendr\'ee par les composantes de K\"unneth des classes de Chern de $\calE$, et soient $p$ et $q$ les deux projecutions de $\calM \times M$ sur $calM$. Les classes de Chern de $\pi!\calH$ sont de la forme $\sum P^{*}\mu_{i} \cdot q^{*}v_{i}$, avec $\mu_{i}, v_{i}\in \calA$. 
\end{lemme*}

Notons $r$ la projections de $C\times \calM \times M$ sur $C$. Tout polyn\^ome
en les classes de Chern de $p_{1}^{*}calE$ et de $P_{2}^{*}\calE$ est une somme de produits de la forme $r^{*}\gamma\cdot \pi^{*}p^{*}\mu \cdot \pi^{*}q^{*}v$,  o\`u $\mu$ et $v$ appartiennent \`a $\calA$. Le lemme r\'emme r\'sulte alors de la formule de Riemann-Roch
$$
ch(\pi!\calH) =\pi_{*}(r^{*}\Todd(\C) \ch(\calH)). 
$$

\begin{remarque*}
La condition (\ref{art3-enum(iii)}) de la proposition est \'evidenmment tr\`es con-traignante. Donnons deux exemples:

\begin{enumerate}[{\rm a)}]
\item $X$ est une surface rationnelle ou r\'egl\'ee, et la polarisation $H$ v\'erifie $H$ $\cdot K_{x}<0$. L'argument de \cite[cor. 6.7.3]{art3-keyM} montre que la condition (\ref{art3-enum(iii)}) est satisfaite. Si de plus les coefficients $a_{i}$ du polyn\^o me de Hilbert des \'el\'ements de $\calM$, \'ecrit sous la forme\label{art3-enmu(a))}
$
\calX(E)\otimes H^{m}) = \sum\limits_{i=0}^{2}a_{i}
\begin{pmatrix}
m+i\\
i
\end{pmatrix}
$
, sont premiers entereux, les conditions (\ref{art3-enum(i)}) \`a (\ref{art3-enum(iii)}) sont satisfaites \cite[$\calx$6]{art3-keyM}.

Dans le cas d' une surface \textit{rationnelle}, on obtient mieux. Pour toute vari\'et\'e $T$, d\'esignons par $C H^{*}(T)$ l'anneau de Chow de $T$; gr\^a ce \`a l'isomorphisme $C H^{*}(X times \calM) \cong C H ^{*}(X) \otimes C H^{*}(\calM)$, on peut remplacer dans la d\'emonstration de la proposition l'anneau de cohomologie par l'anneau de Chow. On en d\'eduit que\textit{la cohomologie rationnelle de $\calM$ est alg\'ebrique}, c'est-\`a-dire que l'application ``classe de cycles" de cycles" $C H^{*}(\calM) \otimes \bQ \longrightarrow H^{*}(\cal M , \bQ)$ est u isomphisme d' anneaux. Dans le cas $X= \bP^{2}$, ellingsrud et Str\o mme obtiennent le m\^ome r\'esultat $sur$ $\bZ$, plus le fait que ces groupes sont sans torsion, gr\^ace \`a l'outil suppl\'ementaire de la suite spectrale de Beilinson. 

\item $X$ est une vari\'ete\'e de Fano De dimension 3. Soit $S$ une surface lisse appartenant au syst\`eme lin\'eaire $|-K_{x}|$ (de sorte que $S$ est une surface $K3$). Lorsqu'elle est satisfaite, la condition (\ref{art3-enum(iii)}) a des cons\'equences remarquables \cite{art3-keyT}: elle entra\^ine que ''application de restriction $E \longmapsto E_{|S}$ d\'efinit un isomorphisme de $\calM$ \textit{une sous-vari\'et\'e largrangienne} d'un espace de modules $\calM_{S}$ de fibr\'es sur $S$ (muni de sa structure symplectique canonique). Il me semble int\'eressant de mettre en \'evidence des espace de modules de fibr\'es sur une vari\'et\'e de Fano (et d\'ej\`a sur $\bP^{3}$) poss\'edant la propri\'et\'e (\ref{art3-enum(iii)}). \label{art3-enmu(b))} 
\end{enumerate}
\end{remarque*}

\begin{thebibliography}{99}
\bibitem[A-B]{art3-keyA-B} M. Atiyah et R. Bott, \textit{Yang-Mills equations over Riemann surfaces}, Phil. Trans. R. Soc. London A $\mathbf{308}$ (1982) 523-615.
\bibitem[E-S]{art3-keyE-S} G. Ellingsrud et S.A. Str\o mme, \textit{Towards the Chow ring of the Hilbert scheme of} $\bP^{2}$, J. reine angew. Math. {\bf 441} (1993) 33-44.
\bibitem[M]{art3-keyM} M. Maruyama, \textit{Moduli of stable sheaves, II.} J. Math. Kyoto Univ. {\bf 18} (1978) 557-614.
\bibitem[T]{art3-keyT} A. N. Tyurin, \textit{The moduli space of vector bundles on threefolds, surfaces and surves I,} preprint (1990).
\end{thebibliography}

\begin{flushleft}
Arnaud Beaville

Universit\'e Paris-Sud

Math\'ematiques- B\^at. 425

91 405 Orsay Cedex, France
\end{flushleft}
