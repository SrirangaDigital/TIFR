\title{Instantons and Parabolic Sheaves}
\markright{Instantons and Parabolic Sheaves}

\author{By~M. Maruyama}
\markboth{By~Ch. Maruyama}{Instantons and Parabolic Sheaves}


\date{}
\maketitle

\section*{Introduction}

S. Donaldson\pageoriginale  \cite{art12-key1} found a beautiful bijection between the set of marked $SU(r)$-instantons and the set of couples of a rank-$r$  vector bundle on $\bP_{\bC}^{2}$ and a trivialization on a fixed line. Then, based on a fixed line. Then, based on Hulek's result in \cite{art12-key3}, he concluded that the moduli space of marked $SU(r)$-instantons with fixed instanton number is connected. Hulek's result is, however, insufficient to deduce the connectedness. In fact, a vector bundle $E$ on $\bP_{\C}^{2}$ is said to be $s$-stable in Hulek's sense if $H^{0}(\bP_{\bC}^{2}, E) =0$ and $H^{0}(\bP_{\bC}^{2}, E^{\vee}) =0$. Hulek \cite{art12-key3} proved that the set of $s$-stable vector bundles on $\bP_{\bC}^{2}$ wit $r(E) =r$, $c_{1}(E)=0$ and $c_{2}(E)=n$ is parametrized by an irreducible algebric set. There are vector bundles on $\bP_{\bC}^{2}$ that correspond to marked $SU(r)$-instatons but are not s-stable. For example, if $c_{2}(E) < r(E), E$ cannot be $s$-stable and we have, on the other hand, $SU(r)$-instantons with instanton number $n< r$.

In this article we shall show that we can regard the couple $(E,H)$ of  a vector bundle $E$ on $\bP_{\bC}^{2}$ and a trivialization $h$ of $E$ on a fixed line as a parabolic stable vector bundle. Then, the connectedness of the moduli space of marked $SU(r)$-instantons reduces to that of the moduli space of parabolic stable sheaves. It is rather complicated but not hard to prove the connectedness of the modulo space of parabolic stable sheaves. The author hopes that he could prove in this way the connectedness of the moduli space of marked $SU(r)$-instantons.

\noindent
{\bfseries Notation.} For a field $k$ and integers $m,n, M(m,n,k)$ denotes the set of $(m \times n)$-matrices over $k$ and $M(n,k)$ does the full matrix ring $M(n,n,k)$, If $f : x \rightarrow S$ is a morphism of schemes, $E$ is coherent sheaf on $X$ and if  $s$ is a point (or, geromtric point) of $S$, then $E(s)$ denotes the sheaf $E\otimes_{\calO_{S}} k(s)$. For a coherent sheaf $F$ on a variety $Y$, we denote the rank of $F$ by $r(F)$. Assuming $Y$ to be smooth and quasi-projective, we can define the $i$-th chern class $c_{i}(F)$ of $F$.

\section{A result of Donaldson}\label{art12-sec-1}
We shall here reproduce briefly the main part of Donaldson's Work \cite{art12-key1}. Fix a line $\ell$ in $\bP_{\bC}^{2}$. Let $E$ be vector bundle of rank $r$ on $\bP_{\bC}^{2}$ wiht the following properties:


\begin{equation}
E|\ell \simeq \calO_{\ell}^{\oplus r},\tag{1.1.1} \label{art12-eq-1.1.1}
\end{equation}
\begin{equation}
c_{1} (E) =0 \;{\rm and}\; c_{2}(E) =n.\tag{1.1.2}\label{art12-eq-1.1.2}
\end{equation}

\noindent
\eqref{art12-eq-1.1.1} implies that $E$ is $\mu$-semi-stable and hence $c_{2}(E)= n \geq 0$. Since the $mu$-semi-stability of $E$ implies $H^{0}(\bP_{bC}^{2}, E(-1))=0$, we see that $E$ is the cohomology sheaf of monad
$$
H\otimes_{\bC} \calO_{\bP^{2}}(-1)\xrightarrow{s} K \otimes_{\bC} \calO_{\bP^{2}} \xrightarrow{t} L \otimes_{\bC} \calO_{\bP^{2}}(1),
$$
where $H, K$ and $L$ are $\bC$-vector spaces of dimension $n,2n +r$ and $n$, respectively. $s$ is represented by a $(2n +r)\times n$ matrix $A$ whose entries are linear forms on $\bP_{\bC}^{2}$. Fixing a system of homogeneous coordinates
$(x : y : z)$ of $\bP_{\bC}^{2}$, we may write
$$
A =A_{x}x + A_{y}y +A_{z}Z,
$$
where $A_{x}, A_{y}, A_{z} \in M(2n + r, n, \bC)$. Similarly $t$ is represented by
$$
B =D_{x}x +B_{y}y +B_{z}z
$$
with $B_{x}, B_{y}m B_{z} \in M(n, 2n + r, \bC)$. The condition $ts =0$ is equivalent to $B_{x}A_{x}=B_{y}A_{y} = B_{z}A_{z}=0$, $B_{x}A_{y} = -B_{y}A_{z} = -B_{z}A_{y}$ and $B_{z}A_{x} = -B_{x}A_{z}$. $E$ is represented by such a monad uniquely up to the action of $GL(H) \times GL(K) \times GL(L)$.

We may assume that $\ell$ is defined by $z=0$. Then $E |\ell$ is trivial if and only if $\det B_{x}A_{y} \neq 0$. We can find bases of $H, K$ and $L$ so that $B_{x}A_{y} = I_{n}$, where $I_{n}$ is the identitiy matrix of degree $n$. Changing the basis of $L$, we have
$$
A_{x}= \begin{pmatrix}
I_{n}\\
0\\
0
\end{pmatrix}
\begin{matrix}
n\\
n\\
r
\end{matrix}
,\quad
A_{y}=\begin{pmatrix}
0\\
I_{n}\\
0
\end{pmatrix}
\begin{matrix}
n\\
n\\
r
\end{matrix}
$$
and
$$
\displaystyle B_{x}=(\mathop{0}^{n}, \mathop{I_{n}}^{n}, \mathop{0}^{r}), \quad \displaystyle B_{y}=(\mathop{-I_{n}}^{n}, \mathop{0}^{n}, \mathop{0}^{r}).
$$
Then, setting
$$
A_{z}=\begin{pmatrix}
\alpha_{1}\\
\alpha_{2}\\
a
\end{pmatrix}
$$
with $\alpha_{1} ,\alpha_{2} \in M(n, \bC)$ and $a \in M(r,n,\bC)$, the equations $B_{y}A_{z} = -B_{z}A_{y}$ and $B_{z}A_{x} = -B_{x}A_{z}$ imply $B_{z}=(-\alpha_{2}, \alpha_{1},b)$ with $b \in M(n,r,\bC)$. The last equation $B_{z}A_{z} = 0$ means
 
\subsection{}\label{art12-subsec-1.2}
$[\alpha_{1}, \alpha_{2}] + ba =0$.

On $\ell$ we have an exact sequence
$$
0 \rightarrow \calO_{\ell}(-1)\xrightarrow{u}\calO_{\ell}^{\oplus 2} \xrightarrow{v}\calO_{\ell}(1) \rightarrow 0.
$$
The restriction of our monad to $\ell$ is
$$
\calO_{\ell}(-1)^{\oplus n} \xrightarrow{\oplus^{n}u\oplus 0}(\oplus^{n}\calO_{\ell}^{\oplus 2}) \oplus \calO_{\ell}^{\oplus r} \xrightarrow{\oplus^{n}v\oplus 0} \calO_{\ell}(1)^{\oplus n}
$$
and the trivialization of $E|_{\ell}$ comes from the last term of the middle. The equivalence defined by the action of $GL(H) \times GL(K) \times GL(L)$ induces an action of the group
$$
\left\{(p,\left. \begin{pmatrix}
p^{-1} & 0 & 0\\
0 & p^{-1} & 0\\
0 &  0 & q
\end{pmatrix}, p) \right| p \in GL(n, \bC), q \in GL(r, \bC)\right\}
$$
on the above normalized matrices. The action of $q$ is nothing but changing the trivializations of $E|_{\ell}$.

The condition that the above normalized couple$(A, B)$ gives rise to a vector bundle is the for every $(\lambda, \mu, \nu) \in \bP_{\bC}^{2}$,
$$
\lambda A_{x}+ \mu A_{y} + \nu A_{z} = \begin{pmatrix}
\lambda I_{n} + \nu \alpha_{1}\\
\mu I_{n} +\nu\alpha_{2}\\
\nu a
\end{pmatrix}
$$
is an injection of $H$ to $K$ and
$$
\lambda B_{x} + \mu B_{y} + \mu B_{z} =(-\mu I_{n}-\nu \alpha_{2},\lambda I_{n} +\nu\alpha_{1}, \nu b)
$$
is a surjection of $K$ to $L$. If $\nu =0$, then these conditions are trivially satisfied. Thus we have the following.

\medskip
\noindent
{\bfseries \thnum{1.3} Proposition.\label{art12-prop-1.3}} \textit{The set $\{(E, h) | E$ {\rm has the
properties} \eqref{art12-eq-1.1.1} {\rm and} \eqref{art12-eq-1.1.2}. $h$ {\rm is a trivilization of} $E|\ell\}$ $/\simeq$ is in bijective correspondence with the set of quadruples $(\alpha_{1}, \alpha_{2}, a,b)$ of matrices with the following properties \eqref{art12-eq-1.3.1}, \eqref{art12-eq-1.3.2} and \eqref{art12-eq-1.3.3} modulo an action of $GL(n, \bC)$}:

\begin{equation}
\alpha_{1}, \alpha_{2} \in M (n, \bC), a \in M (r, n, \bC) \;and\; b\in M(n,r, \bC), \tag{1.3.1}\label{art12-eq-1.3.1}
\end{equation}
\begin{equation}
[\alpha_{1}, \alpha_{2}] + ba =0,\tag{1.3.2}\label{art12-eq-1.3.2}
\end{equation}
\begin{equation}
for \; all  \;(\lambda, \mu) \in \bC^{2},\tag{1.3.3}\label{art12-eq-1.3.3}
\end{equation}
$$
\begin{pmatrix}
\lambda I_{n} + \alpha_{1}\\
\mu I_{n} + \alpha_{2}\\
a
\end{pmatrix}
$$
\textit{in injective and} $(-\mu I_{n}-\alpha_{2}, \lambda I_{n} + \alpha_{1}, b)$ \textit{is surjective.}
\textit{Here} $p \in GL(n, \bC)$ \textit{sends} $(\alpha_{1}, \alpha_{2}, a, b)$ \textit{to} $(p\alpha_{1}p^{-1}, p\alpha_{2}p^{-1}, ap^{-1}, pb)$.

Let us embed $\bP_{\bC}^{2}$ into $\bP_{\bC}^{3}$ as the plane defined by $w =0$, where $(x : y : z : w)$ is a system of homogeneous coordinates of $\bP_{\bC}^{3}$. Now we look at the vector bundle defines by a monad
$$
H \otimes_{\bC} \calO_{P^{3}}(-1) \xrightarrow{c} K \otimes_{\bC}\calO_{P^{3}} \xrightarrow{d} L \otimes_{\bC}\calO_{P^{3}}(1),
$$
and trivialized on $\ell =\{z=w=0\}$, $c$ and $d$ are represented by
\begin{align*}
A &= A_{x}x + A_{y}y + A_{z}z + A_{w}w\\
B &=B_{x}x + B_{y}y +B_{z}z+B_{w}w
\end{align*}
with
$$
A_{x} = \begin{pmatrix}
I_{n}\\
0\\
0
\end{pmatrix}
,\quad A_{y}=
\begin{pmatrix}
0\\
I_{n}\\
0
\end{pmatrix}
,\quad A_{z}=
\begin{pmatrix}
\alpha_{1}\\
\alpha_{2}\\
a
\end{pmatrix}
,\quad A_{w} =
\begin{pmatrix}
\hat{\alpha}_{1}\\
\hat{\alpha}_{2}\\
\hat{a}
\end{pmatrix}
$$
and
$$
B_{x}= (0,I_{n}, 0), \quad B_{y}=(-I_{n}, 0,0), \quad B_{z} =(-\alpha_{2}, \alpha_{1}, b), \quad B_{w} =(\hat{\alpha_{2}}, \hat{\alpha_{1}}, \hat{b}).
$$
The condition $dc =0$ of the monad means

\begin{equation}
[\alpha_{1}, \alpha_{2}] +ba = 0,\tag{1.4.1}\label{art12-eq-1.4.1}
\end{equation}
\begin{equation}
[\hat{\alpha_{1}}, \hat{\alpha_{2}}] +\hat{b}\hat{a} = 0,\tag{1.4.2}
\end{equation}
\begin{equation}
[\alpha_{1}, \hat{\alpha_{2}}] +[\hat{\alpha_{1}}\alpha_{2}] + b \hat{a}+ \hat{b}a = 0.\tag{1.4.3}\label{art12-eq-1.4.3}
\end{equation}

Let $\bH =\bR +\bR_{i} + \bR_{j} + \bR_{k}$ be the algebra of quaternions over $\bR$. Regarading $\bC^{4}$ as $\bH^{2}$, $bH$ acts on $\bC^{4}$ from left. Hence we have a $j$-invariant real analytic map
$$
\pi : \bP^{3}_{\bC} = \{\bC^{4}\backslash \{0\}\}/{\bC}^{*} \rightarrow S^{4} \cong \bP_{\bH}^{2}.
$$
A vector bundle $E$ of rank $r$ on $\bP_{\bC}^{3}$is called an instanton bundle if it comes from as $SU(r)$-instanton on $S^{4}$ , or equivalently if the following conditions are satisfied:

%~ \begin{equation}
%~ \begin{aligned}
%~ &\text{there is and isomorphism} \;\lambda\; \text{of} \;E\; \text{to} \;\overline{j^{*}(E^{\vee})}\; \text{such that the composition}\\
%~ &\overline{j^{*}(^{t}\lambda^{-1})}.\; \lambda \;\text{is equal to} \;\id_{E} \;\text{when we identify} \overline{j^{*}(\overline{j^{*}(E^{\vee})^{\vee}})}\; \text{with}\; E,
%~ \end{aligned}
%~ \end{equation}

\begin{equation}
E|_{\pi^{-1}(x)} \text{is trivial for all} x \in S^{4}.\tag{1.5.2}\label{art12-eq-1.5.2} 
\end{equation}
Ab instanton bundle $E$ is the cohomology of a monad which appeared in the above. The instanton structure of $E$ provides this space $K$ with a Hermitian structure and an isomorphism of $L$ to $\overline{H}^{\vee}$.

For  $\lambda =(\lambda_{1}, \lambda_{2}, \lambda_{3}, \lambda_{4}) \in \bC^{4}$, we set
\begin{align*}
A(\lambda) &= A_{x}\lambda_{1} + A_{y}\lambda_{2} +A_{z}\lambda_{3} + A_{w}\lambda_{4}\\
b(\lambda) &= B_{x}\lambda_{1} + B_{y}\lambda_{2} +B_{z}\lambda_{3} + B_{w}\lambda_{4}
\end{align*}
Since $A(\lambda)^{*} = ^{t}\overline{A(\lambda)}$ defines a linear map of $K \cong\overline{K}^{\vee}$ to
$\overline{H}^{\vee} \cong L$, we have a linear map
$$
\calA(\lambda) = A(\lambda)^{*} \oplus B(\lambda) : K \rightarrow L \oplus L.
$$
The quaternion algebra $\bH$ acts on $L \oplus L$ by
$$
i \;\binom{u}{v} = \binom{-\sqrt{-1}u}{-\sqrt{-1}v}, \quad j\; \binom{u}{v}= \binom{v}{-v}
$$ 
Then the condition that
$$
\calA(q\lambda)(v)=q \calA(\lambda)(v)\quad \text{for all}\; q \in \bH, v \in K
$$
is equivalent to the condition that $(A, B)$ gives rise to an $SU(r)$-instanton bundle.

Now the above condition can be written down in the form
\begin{align*}
\calA(j(0,0,z,w)) &= \binom{-^{t}\overline{A}_{z}w +^{t}\overline{A}_{w}z}{-B_{z}\overline{w}+B_{w}\overline{z}}\\
 &= j \binom{^{t}\overline{A}_{z}\overline{z} +^{t}\overline{A}_{w}\overline{w}}{B_{z}z+B_{w}w} = \binom{B_{z}z+B_{w}w}{-^{t}\overline{A}_{z}\overline{z}-^{t}\overline{A}_{w}\overline{w}}\\
\end{align*}
that is,
$$
^{t}\overline{A}_{w} = B_{z}\; \text{and} \; ^{t}\overline{A}_{z}= -B_{w}. 
$$
Thus we have $\hat{\alpha_{1}} = -\alpha_{2}^{*}, \hat{\alpha}_{2} = \alpha_{1}^{*}, \hat{a}=b^{*}$ and $\hat{b} = -a^{*}$. On the other hand, $A(\lambda)$ is injective if and only if $A(\lambda)^{*}$ is surjective. Thus our monad defines a vector bundle if and only if $\calA(\lambda)^{*}$ is sujective. Thus our monad defines a vector bundle if and  only if $\calA(\lambda)$ is surjective for all $\lambda \in \bC^{4}\backslash \{0\}$. \eqref{art12-eq-1.3.3} implies that $\calA(\lambda)$ is surjectifve for $\lambda \in \bC^{3}\backslash\{0\}$ which gives our plane $w=0$ and then so is $\calA(q \lambda)$. Since $\{q \lambda | q \in \bH, \lambda \in \bC^{3} \backslash\{0\}\}$ sweeps $\bC^{4} \backslash \{0\}$, \eqref{art12-eq-1.3.3} is equivalent to the condition that $\calA(\lambda)$ is surjective for all $\lambda \in \bC^{4}\backslash\{0\}$ 

Now the condition \eqref{art12-eq-1.4.2} is the adjoint of \eqref{art12-eq-1.4.1} or \eqref{art12-eq-1.3.2} and the condition \eqref{art12-eq-1.4.3} becomes
$$
[\alpha_{1}, \alpha_{1}^{*}] + [\alpha_{2}, \alpha_{2}^{*}] + bb^{*}-a^{*}a=0.
$$
 Let $M(SU(r), n)$ be the set of isomorphism classes of marked $SU(r)$-instantons with instanton number $n$. $M(SU(r), n)$ is the set of the isomorphism classes of teh couples $(\nabla, g)$ of an $SU(r)$-instanton $\nabla$ and an element $g$ of the fiber over northpole of the $SU(r)$-principle fiber bundle where the instantion is defined. What we have seen in the above is stated as follows.
  
\medskip
\noindent
{\bfseries \thnum{1.6} Proposition. \label{art12-prop-1.6}}\textit{$M(SU(r),n)$ is in bijective correspondence with the $U(n)$-quotient of the set of quadruples $\{(\alpha_{1}, \alpha_{2},a,b)\}$ of matrices with the properties
\eqref{art12-eq-1.3.1}, \eqref{art12-eq-1.3.2}, \eqref{art12-eq-1.3.3} and}

\begin{equation}
[\alpha_{1}, \alpha_{1}^{*}] + [\alpha_{2}, \alpha_{2}^{*}] + bb^{*}-a^{*}a=0.\tag{1.6.1}\label{art12-eq-1.6.1}
\end{equation}

Assume that $G=GL(n,\bC)$ acts on $\bC^{N}$ with a fixed norm structure such that $U(n)$ does isometrically. Take a $G$-invariant subscheme $W$ of $\bC^{N}$ whose points are att stable. Let $W_{0}$ be the set points that are nearest to the origin in its $G$-orbit.

\medskip
\noindent
{\bfseries \thnum{1.7} Proposition. \label{art12-prop-1.7}} \textit{$W/G$ is isomorphic to $W_{0}/U(n)$}.

Let us look at the $\bC$-linear space $V=M(n,\bC) \times M(n,\bC) \times M(r,n \bC)\times M(n,r,\bC)$ where $GL(n,\bC)$ acts as in Proposition \ref{art12-prop-1.3}. Then$U(n)$ acts on $V$ isometrically with respect to the obvious norm of $V$. Let $W$ be the subscheme of $V$ defined by \eqref{art12-eq-1.3.2} and \eqref{art12-eq-1.3.3}. Then we have have the following key results. 

\medskip
\noindent
{\bfseries \thnum{1.8} Lemma. \label{art12-Lemma-1.8}} [1, p. 458] \textit{A point $(\alpha_{1}, \alpha_{2}, a, b)$ in $W$ is contained in $W_{0}$ if and only if it has the property} \eqref{art12-eq-1.6.1}.

\medskip
\noindent
{\bfseries \thnum{1.9} Lemma. \label{art12-Lemma-1.9}} [1, Lemma]. \textit{$W$ is contained in the set of stable points of $V$ with respect to the action of $GL(n,\bC)$}.

Therefore, we come to the main result of \cite{art12-key1}.

\medskip
\noindent
{\bfseries \thnum{1.10} Theorem. \label{art12-Thm-1.10}} \textit{The set $\{(\alpha_{1}, \alpha_{2}, a, b)\; | \;(1.3.1), (1.3.2)\; \text{and} \;(1.3.3)\break \text{are satisfied} \}$ $/GL(n,\bC)$ is in bijective correspondence with the set \break $\{(\alpha_{1}, \alpha_{2}, a, b) | (1.3.1), (1.3.2)$ {\rm and} $(1.1.2)$. $h$ {\rm is a trivialization of} $E|_{\ell}\}/\cong$  is isomorphic to the space $M(SU(r),n)$ }.

\section{Parabolic sheaves}\label{art12-sec-2}
Let $(x, \calO_{X}(1), D)$ be a triple of a non-singular projective variety $X$ over an algebraically closed field $k$, an ample line bundle $\calO_{X}(1)$ on $X$ and an effective  Cartier divisor $D$ on $X$. A coherent torsion free sheaf $E$ is said to be a parabolic sheaf if the following data are assigned to it:

\begin{equation}
\text{a filtration}\; 0 = F_{t+1} \subset F_{t} \subset \cdot \subset F_{1} = E \otimes_{\calO_{X}}\calO_{D}
\text{by coherent subsheaves}\tag{2.1.1}\label{art12-eq-2.1.1},
\end{equation}

\begin{equation}
\text{a system of weights} \; 0 \leq \alpha_{1} < \alpha_{2} < \cdots < \alpha_{t} < 1.\tag{2.1.2}\label{art12-eq-2.1.2}
\end{equation}
We denote the parabolic sheaf by $(E, F_{*}, \alpha_{*})$.

For a parabolic sheaf $(E, F_{*}, \alpha_{*})$, we define
$$
par-\chi(E(m))= \chi(E(-D)(m)) + \sum\limits_{i=1}^{t} \alpha_{i}\chi(F_{i}/F_{i+1}(m)).
$$ 
If $E'$ is a coherent shubsheaf of $E$ with $E/E'$ torsion free,then we have an induced parabolic structure. In fact, since $E'\otimes_{\calO_{X}}\calO_{D}$ can be regarded as a subsheaf of $E\otimes_{\calO_{X}}\calO_{D}$, we have a filtration $0 =F_{\delta+1}' \subset F_{\delta}' \subset F_{\delta}' \subset \cdots\subset F_{\delta}'=E' \otimes_{\calO_{X}}\calO_{D}$ such that $F_{j}' = E'\otimes_{\calO_{X}} \calO_{D} \cap F_{i}$ for some $i$. The weight $\alpha_{j}'$ of $F_{j}'$ is defined to be $\alpha_{i}$ with $i=\max\{k | F_{j}' =E'\otimes_{\calO_{X}}\calO_{D} \cap F_{k}\}$.

\medskip
\noindent
{\bfseries \thnum{2.2} Definition. \label{art12-definition-2.2}} A parabolic $(E, F_{*}, \alpha_{*})$ is said to be stable if for every coherent subsheaf $E'$ with $E/E'$ torsion free and with $1 \leq r(E')< r(E)$ and for all sufficiently large integers $m$, we have
$$
par-\chi(E'(m))/r(E') < par-\chi(E(m))/r(E),
$$
where the parabolic structure of $E'$ is the induces from that of $E$.

Let $S$ be a scheme of finite type over a universally Japanese ring and now let $(X, \calO_{X}(1),D)$ be a triple of a smooth, projective, geometrically integral scheme $X$ over $S$, an $S$-ample line bundle $\calO_{X}(1)$ on $X$ an effective relative Cartier divisor $D$ on $X$ over $S$.

\medskip
\noindent
{\bfseries \thnum{2.3} Lemma. \label{art12-Lemma-2.3}} \textit{If $T$ is a locally noetherian $S$-scheme and of $E$ is a$T$-flat coherent sheaf on $X \times_{s}T$ such that for every geometric point t of $T$, $E(t)$ is torsion free, then $E|_{D}=E\otimes_{\calO_{X}}\calO_{D}$ is flat over $T$}.

\medskip
\noindent
{\bfseries Proof.} For every point $y$ of $T$, $E(y)$ is torsion free and $D_{y}$ is a Cartier divisor on $X_{y}$. Thus the natural homomorphism $E(-D)(y)\rightarrow E(y)$ is injective. Then, since $E$ is $T$-flat, $E|_{D}=E|E(-D)$ is $T$ flat Q.E.D.

Let $E$ be a coherent sheaf on $X \times_{S}T$ which satisfies the condition in the above lemma. $E$ is said to be a $T$-family of parabolic sheaves if the following data are assigned to it:
\begin{align}\label{art12-eq-2.4.1}
 &\text{a filtration}\; 0=F_{t+1}\subset F_{t} \subset \cdots \subset F_{1}= E|_{D}\; \text{by coherent}\tag{2.4.1}\\
  &      \text{subsheaves such that for}\; 1 <i < t, E|_{D}/F_{i}\; \text{is flat over}\; T\nonumber, 
\end{align}

\begin{equation}\label{art12-eq-2.4.2}
\text{a system of weights}\; 0 \leq \alpha_{1} < \alpha_{2} < \cdot < \alpha_{t} < 1\tag{2.4.2}.
\end{equation}
As in teh absolute case we denote by $(E, F_{*}, \alpha_{*})$ the family of parabolic sheaves. For $T$-families of parabolic sheaves $(E, F_{X, \alpha_{*}})$ and $(E',F_{*}', \alpha_{*}')$, they are said to he equivalent and we denote $(E, F_{*}, \alpha_{*}) \sim (e', F_{*}', \alpha_{*}')$ if there is an invertible sheaf $L$ on $T$ such that $E$ is isomorphic to $E' \otimes_{\calO_{T}} L$, the filtration $F_{*}$ is equal to $F_{*}'\otimes_{\calO_{T}} L$ under this isomorphism and if the systems of weights are the same.

Fixing polynomials $H(x), H_{1}(x), \ldots H_{t}(x)$ and a system of weights
$0 \leq \alpha_{1} < \cdots < \alpha_{t} < 1$, we set

$$
par-\Sigma(H,H_{*}, \alpha_{*})(T)=\left\{(E, F_{*}, \alpha_{*})\left| \begin{matrix}
(E, F_{*}, \alpha_{*}) \text{is a} T-\text{family of}\\
  \text{ parabolic sheaves with the}\\
  \text{the properties}\; (2.5.1)\; \text{and}\\
  (2.5.2)            
\end{matrix}\right.\right\}\big/\sim
$$

\begin{align}
& \text{for every geometrix point}\; y \; \text{of}\; T \;\text{and} \; 1 \leq i \leq t, \chi(e(y)(m))=\tag{2.5.1}\\
& H(m)\; \text{and}\; \chi((E(y)/F_{i+1}(y))(m)) =H_{i}(m)\nonumber,
\end{align}

\begin{flalign}
\text{for every geometric point}\; y \;\text{of} \; T, (E(y), F_{*}(y), \alpha_{*}) \; \text{is stable}\tag{2.5.2}. 
\end{flalign}

\noindent
Obviously $par-\Sigma(H, H_{*}, \alpha_{*})$ defines a contravariant functor of the category $(\Sch/S)$ of locally notherian $S$-schemes to that of sets. Note that for every geometric point $s$ of $S$ and every member
$$
(E, F_{*}, \alpha_{*}) \in par-\Sigma(H, H_{*}, \alpha_{*})(s),
$$
we have
$$
par-\chi(E(m))= H(m)- \sum\limits_{i=1}^{t} \varepsilon_{i}H_{i}(m),
$$
where $\varepsilon_{i}=\alpha_{i+1}-\alpha_{i}$ with $\alpha_{t+1} =1$.

One of main results on parabolic stable sheaves is stated as follows.

\medskip
\noindent
{\bfseries \thnum{2.6} Theorem. \label{art12-thm-2.6}} \cite{art12-key7}\textit{ Assume that all the weights $\alpha_{1}, \ldots, \alpha_{t}$ are rational numbers. Then the functor $par-\Sigma(H, H_{*}, \alpha_{*})$ has a coarse moduli scheme $M_{X/S}(H, H_{*}, \alpha_{*})$ of locally of finite type over $S$. If $S$ is a scheme over of field of characteristic zero, then the coarse moduli scheme is quasi-projective over $S$.}

Now let us go back to the situation of the preceding section. We have a fixed line $\ell$ in $\bP_{\bC}^{2}$, a torsion free, coherent sheaf $E$ of rank $r$ on $\bP_{\bC}^{2}$ and a trivialization of $E|_{\ell}$. There is a bijective correspondence between the set of trivilzations of $E|_{\ell}$ and the set
$$
\calT_{E}=\{\varphi :E|_{\ell} \rightarrow \calO_{\ell}(r-1) |H^{0}(\varphi) : H^{0}(E|_{\ell}) \xrightarrow{\sim}H^{0}(\calO_{\ell}(r-1))\}/\cong
$$
where $\cong$ means isomorphism as quotient sheaves.

\medskip
\noindent
{\bfseries \thnum{2.7} Lemma. \label{art12-lemma.2.7}} \textit{Let $E$ be a torsion free, coherent sheaf of rank $r$ on $\bP_{\bC}^{2}$ such that $E|_{\ell}$ is a trivial vector bundle. For every element $(\varphi: E|_{\ell} \rightarrow \calO_{\ell}(r-1))$ of $\calT_{E}$, $\ker(\varphi)$ is isomorohic to $\calO_{\ell}(-1)^{\oplus r-1}$} 

\medskip
\noindent
{\bfseries Proof.} Since $\ker(\varphi)$ is a vector bundle of rank $r-1$ on the line $\ell$, it is isomorphic to a direct sum $\calO_{\ell}(a_{1}) \oplus \cdots \oplus \calO_{\ell}(a_{r-1})$ of line bundles. Our condition on $\varphi$ means that $H^{0}(\ell, ker(\varphi))=0$, Combining this and the fact that $\deg(\ker(\varphi))=1-r$, we see that $a_{1}=\cdots=a_{r-1}=-1$.\hfill Q.E.D. 

Fixing a system of weights $\alpha_{1} =1/3$, $\alpha_{2}=1/2$, every element $\varphi$ of $\calT_{E}$
gives rise to a parabolic structure of $E$:
$$
0 =F_{3} \subset F_{2} =\ker(\varphi) \cong \calO_{\ell}(-1)^{\oplus r-1} \subset F_{1}=E|_{\ell}.
$$

\medskip
\noindent
{\bfseries \thnum{2.8} Proposition. \label{art12-Prop.2.8}} \textit{If $E$ as in Lemma \ref{art12-lemma.2.7}. Assume $E$ has the properties \eqref{art12-eq-1.1.1} and \eqref{art12-eq-1.1.2}, then the above parabolic sheaf is stable.}

\medskip
\noindent
{\bfseries Proof.} Set $r=r(E)$. We know that

\begin{equation*}
\begin{split}
par-\chi(E(m))/r(E) &= \dfrac{(m-1)^{2}}{2} + \dfrac{3(m-1)}{2} + 1 -\dfrac{n}{r}\\
 & \qquad +\dfrac{(r-1)m}{2r} + \dfrac{(r+m)}{3r}\\
 &=\dfrac{m^{2}}{2}+ \left(1 -\dfrac{1}{6r}\right)m + \dfrac{1}{3}-\dfrac{n}{r}.
\end{split}
\end{equation*}
Pick a coherent subsheaf $E'$ of $E$ with $E/E'$ torsion free and write
$$
par-\chi(E'(m))/r(E')=\dfrac{m^{2}}{2} +a_{1}m +a_{0}.
$$
Then we see
$$
a_{1} = \mu(E') + \dfrac{1}{2} + \dfrac{1}{r(E')}\left(\dfrac{s}{2} +\dfrac{t}{3}\right),
$$
where $s+t =r(E')$ and $0 \leq t \leq 1$. Thus if $\mu(E') < 0$, then we have
$$
a_{1}\leq \dfrac{1}{r(E')} + q < \dfrac{-1}{r} + 1 < 1-\dfrac{1}{6r}
$$
and hence we obtain the desired inequality. We may assume therefore that $c_{1}(E')=0$. Since $E/E'$ is torsion free, $E'$ is locally free in a neighborhood of $\ell$ and $E'|_{\ell}$ is a subsheaf of $E|_{\ell}$. Thus the triviality of $E|_{\ell}$ implies tha $E'|_{\ell} \cong\calO_{\ell}^{\oplus r(E')}$ and hence $E'|_{\ell}$ is not contained in $F_{2}$. Since $E'|_{\ell}/(F_{2}\cap E'|_{\ell})$ is subsheaf of $F_{1}/F_{2} \cong \calO_{\ell}(r-1)$, it is of rank 1. We have two cases.

\medskip
\noindent
{\bfseries Case \thnum{1.} \label{art12-prop2.8-case-1}} $F_{2}\cap E'|_{\ell}=0$ and $r(E')=1$. Then the length of the filtration of $E'|_{\ell}$ is 1 and the weight is 1/3. Thus we see that
$$
a_{1}= \dfrac{1}{2} + \dfrac{1}{3} =1-\dfrac{1}{6} < 1 -\dfrac{1}{6r}.
$$

\medskip
\noindent
{\bfseries Case \thnum{2.} \label{art12-prop2.8-case-2}} $F_{2}\cap E'|_{ell} \neq 0$. In this case $F_{2}'=F_{2}\cap E|_{\ell}'$ is a subsheaf of $\calO_{\ell}(-1)^{\oplus (r-1)}$ and hence for $m \geq 0$,
$$
\chi(F_{2}'(m)) \leq (r(E')-1)m.
$$
On the other hand, since $E'|_{\ell}/F_{2}'$ is a subsheaf of $\calO_{\ell}(r-1)$, we have that for $m\geq 0$,
$$
\chi(E'|_{\ell}/F_{2}'(m))\leq m +r.
$$
Combining these, we get
$$
a_{1} \leq \dfrac{1}{2} + \dfrac{1}{2} -\dfrac{1}{2r(E')} +\dfrac{1}{3r(E')}=1-\dfrac{1}{6r(E')}< 1-\dfrac{1}{6r}.
$$
This completes our proof.

\section{Connectedness of the moduli of instantons}\label{art12-sec-3}
Let us set
\begin{flalign*}
H(x)&=\dfrac{rx^{2}}{2} + \dfrac{3rx}{2} + r-n&\\
H_{1}(x) &= x +r\\
H_{2}(x)&= rx + r\\
\alpha_{1} &= \dfrac{1}{3}\; \text{and}\; \alpha_{2} = \dfrac{1}{2}.
\end{flalign*}
For these invariants, we have the modulo space $M(r,n)=M(H, H_{*}, \alpha_{*})$ of parabolic stable sheaves on $(\bP_{\bC}^{2}, \calO_{\bP^{2}}(1), \ell)$. There is an open subscheme $M(r, n)$ of $\tilde{M}(r, n)$ consisting of $(E, F_{*}, \alpha_{*})$ with the properties

\begin{equation}
E|_{\ell} \cong \calO_{\ell}^{\oplus r} \; \text{and} \; E|_{\ell}(r-1)\tag{3.1.1}\label{art12-eq-3.1.1}, 
\end{equation}

\begin{equation}
\text{for the surjection}\; \varphi: E|_{\ell}\rightarrow E|_{\ell}/F_{2}, H_{(\varphi)}^{0}\; \text{is isomorphic.}\tag{3.1.2}\label{art12-eq-3.1.2}
\end{equation}
$M(r,n)$ contains a slightly smaller open subscheme $M(r,n)_{0}$ consisting of locally free sheaves. What we have seen in the above is $M(SU(r), n) \cong M(r,n)_{0}$.

\medskip
\noindent
{\bfseries \thnum{3.2} Proposition. \label{art12-prop-3.2}} $M(r,n)$ \textit{is smooth and of pure dimension 2rn.}

To prove this proposition we shall follow the way we used in \cite{art12-key5}. Let us start with the general setting in Theorem \ref{art12-thm-2.6}. Let $\Sigma$ be the family of the classes of parabolic stable sheaves on the fibres of $X$ over $X$ with fixed invariants. For simplicity we assume that $\Sigma$ is bounded (Proposition \ref{art12-prop-3.2} is the case). If $m$ is a sufficiently large integer and if $(E, F_{*}, \alpha_{*})$ is a representative of a member of $\Sigma$ over a geometric point $s$ of $S$, then we have that

\begin{flalign}
& \text{both}\; E(-D_{s})(m)\; \text{and}\; E(m)\; \text{are generated by their global sections}&\tag{3.3.1}\label{art12-eq-3.3.1}\\[-0.5cm]
&\text{and for all}\; i > 0,  H^{i}(E(-D_{s})(m))=H^{i}(E(m))=0,\nonumber
\end{flalign}

\begin{flalign}
&\text{for} \; 1 \leq j \leq t\; \text{and}\; i > 0, H^{i}(F_{j}/F_{j+1}(m))= 0 \;\text{and} \; F_{j}(m)\; \text{is  }\tag{3.3.2}\label{art12-eq-3.3.2}\\
&\text{generated by its global sections}.\nonumber
\end{flalign}
Replacing every member
$$
(E, F_{*},\alpha_{*})\in par-\Sigma(H, H_{*}, \alpha_{*})(T)
$$
by $(\calO_{X}(m)\otimes_{\calO_{S}}E, \calO_{X}(m) \otimes_{\calO_{S}}F_{*}, \alpha_{*})$, we may assume $m=0$. Setting $N=dim H^{0}(X(s), E)$ for a member $(E, F_{*}, \alpha_{*})$ of $\Sigma$ on a fiber $X_{s}$ over $s$, fixing a free $\calO_{S}$ -module $V$ of rank $N$ and putting $V_{X}=V\otimes_{\calO_{S}} \calO_{X}$, there is an open subscheme $R$ of $Q=\Quot_{V_{X}/X/S}^{H}$ such that for every algebraically closed field $k$,a $k$-valued point $z$ of $Q$ is in $R(k)$ if and only if the following conditions are satisfied:
\begin{flalign}
&\text{For the universal quotient}\; \varphi: V_{X}\otimes_{\calO_{S}}\calO_{Q}\rightarrow \tilde{E}\;
\text{the induced}\tag{3.3.3}\label{art12-eq-3.3.3} \\
&\text{the induced map}\; H^{0}(\varphi(Z)) : V\otimes_{\calO_{s}}k(s)\rightarrow H^{0}(X_{s}, \tilde{E}(Z))\nonumber\\
&\text{is an isomorphism, where} s \text{is the image of} z \text{in} \;S(k).\nonumber 
\end{flalign}
\begin{flalign}
&\text{For every} \; i > 0, H^{i}(X(s), \tilde{E}(z))=0.&\tag{3.3.4}\label{art12-eq-3.3.4}\\[0.2cm]
&\tilde{E}(z)\; \text{is torsion free}&\tag{3.3.5}\label{art12-eq-3.3.5}.
\end{flalign}

We denote the restriction of $\tilde{E}$ to $X_{R}=X \times_{S} R $ by the same $\tilde{E}$. Then, by Lemma
\ref{art12-Lemma-2.3} we see that $\tilde{E}|_{D_{R}}$ is flat over $R$. Let $R_{t}$ be the $R$-scheme $\Quot_{\tilde{E}|_{D_{R}}/X_{R}/R}$ and let
$$
\tilde{E}|_{D_{R}} \otimes_{\calO_{R}}\calO_{R_{t}} \rightarrow \tilde{E}_{t}
$$ 
be the universal quotent. Assume that we have a sequence
$$
R_{j}\rightarrow R_{j+1} \rightarrow \cdots \rightarrow R_{t} \rightarrow R
$$
of scheme $R_{j}$ and a sequence of sheaves $\tilde{E}_{j}, \tilde{E}_{j+1}\ldots, \tilde{E}_{t}, \tilde{E}|_{D_{R}}$ such that $\tilde{E}_{i}$ is $R_{i}$-flat coherent sheaf on $X_{R_{i}}$ and that there is a surjection
$$
\tilde{E}_{i}\otimes_{\calO_{R_{i}}}\calO_{R_{i-1}} \rightarrow \tilde{E}_{i-1}.
$$
Set $R_{j-1}$ to be $\Quot_{\tilde{E}_{j}/X_{R_{j}}/R_{j}}$ and take the universal quotient
$$
\tilde{E}_{j} \otimes_{\calO_{R_{j}}} \calO_{R_{j-1}} \rightarrow \tilde{E}_{j-1}.
$$
on $R_{j-1}$. By induction on $j$ we come to $R_{1}$ and we have a sequence of surjections
$$
\tilde{E}|_{D_{R}}\otimes_{\calO_{R}}\calO_{R_{1}} \rightarrow \tilde{E}_{t}\otimes_{\calO_{R_{t}}}\calO_{R_{1}} \rightarrow \cdots \rightarrow \tilde{E}_{2}\otimes_{\calO_{R_{2}}} \calO_{R_{1}}\rightarrow \tilde{E_{1}}.
$$
Setting
\begin{align*}
E&=\tilde{E}\otimes_{\calO_{R}} \calO_{R_{1}}\\
F_{i}&=\ker(\tilde{E}|_{D_{R}} \otimes\calO_{R} \calO_{R_{1}} \rightarrow \tilde{E}_{i}\otimes_{\calO_{R_{i}}} \calO_{R_{1}})
\end{align*}
we get an $R_{1}$-family of parabolic sheaves. There is an open subscheme $U$ of $R_{1}$ such that for every algebraically closed field $k$, a $k$-valued point $z$ of $R_{1}$ is in $U(k)$ if and only if $(E(x), F_{*}(z), \alpha_{*})$ is stable. We shall denote the restriction of $(E, F_{*}, \alpha_{*})$ to $U$ by the same $(E, F_{*}, \alpha_{*})$.

The $S$-group scheme $G=GL(V)$ naturally acts on $Q$ and $R$ is $G$-invariant. There is a canonical $G$-linearization on $\tilde{z}$ and hence $G$ so acts on $R_{t}$ that the natural morphism of $R_{t}$ to $R$ is $G$-invaraiant. Then $\tilde{E}_{t}$ carries a natural $G$-linearization. Tracing these procedure to $R_{1}$, we come eventually to a $G$-action on $U$ and a $G$-linearization of the family $(E, F_{*}, \alpha_{*})$. Obviously the center $\bG_{m,s}$ of $G$ acts trivially on $U$ and we have an action of $\overline{G} = G/\bG_{m,s}$. We can show that there exists a geomeric quotient of $U$ by $\overline{G}$. Then we see
\begin{equation}
\text{The quotient} \; U/\overline{G}\; \text{is the moduli scheme in Theorem \ref{art12-thm-2.6}.}\tag{3.4}\label{art12-eq-3.4}
\end{equation}
For a $T$-family of parabolic sheaves $(E, F_{*}, \alpha_{*})$, we put
$$
K_{i}=\ker(E-E|D_{r}/F_{i})
$$
and then we get a sequence of $T$-flat coherent subsheaves $K_{t+1}=E(-D_{T})\subset k_{t}\subset \cdots \subset K_{2} \subset K_{1} = E$. For a real number $\alpha$, there is an integer $i$ such that $1\leq i \leq t + 1$ and $\alpha_{i-1} < \alpha-[\alpha] \leq \alpha_{i}$, where $\alpha_{0} =\alpha_{t}-1$ and $\alpha_{t+1}= 1$. Then we set
$$
E_{\alpha} = K_{i}(-[\alpha]D_{T}).
$$
Thus we obtain a filtration $\{E_{\alpha}\}_{\alpha \in \bR}$ of $E$ parametrized by real numbers that has the following properties

\begin{flalign}
&E/E_{\alpha}\; is T-\text{flat and if}\; \alpha \leq \beta, \text{then}\; E_{\alpha}
 \supseteq E_{\beta}\tag{3.5.1}\label{art12-eq-3.5.1}.\\[0.2cm]
&\text{if}\; \varepsilon\; \text{is a sufficiently small positive real number, then}\;
E_{\alpha-\varepsilon}=E_{\alpha}.\tag{3.5.1}\label{art12-eq-3.5.2}.\\[0.2cm]
&\text{For every real number}\; \alpha, \; \text{we have} \; E_{\alpha + 1}=E_{\alpha}(-D_{T}).\tag{3.5.3}\label{art12-eq-3.5.3}\\[0.2cm]
&E_{0}=E.\tag{3.5.4}\label{art12-eq-3.5.4}\\[0.2cm]
& \text{The length of the filtration for}\; 0 \leq \alpha \leq 1\; if finite.\tag{3.5.5}\label{art12-eq-3.5.5} 
\end{flalign}

Convesely, if $E$ is a $T$-flat coherent sheaf on $X_{T}$ such that for every geometric point $y$ of $T$, $E(y)$ is torsion free that $E$ has a filtration parametrized by $\bR$ with the above properties, then we have a $T$-family of parabolic sheaves. Thus we may use the notation $E_{*}$ for a $T$- family of parabolic sheaves $(E, F_{*}, \alpha_{*})$.

\medskip
\noindent
{\bfseries \thnum{3.6} Definition. \label{art12-definition-3.6}}  Let $E$ and $E_{*}'$ be $T$-familes of parabolic sheaves. A homomorphism $f: E \rightarrow E$ of the undelying coherent sheaves is said to be a homomorphism of paraboic sheaves if for all $0 \leq \alpha < 1$, we have
$$
f(E_{\alpha}) \subset E_{\alpha}'.
$$

$\Hom^{\Par}(E_{*}, E_{*}')$ denotes the set of all homomorphisms of parabolic sheaves of $E_{*}$ to $E_{*}'$.

If one notes that for a stable parabolic sheaf $E_{*}$ on a projective varaiety, a homomorphism of $E_{*}$ to itself is the multiplication by an element of the ground field $k$ or $\Hom^{\Par}(E_{*}, E_{*}) \cong k$, then one can prove the following lemma by the same argument as the proof of Lemma 6.1 in \cite{art12-key5}.

\medskip
\noindent
{\bfseries \thnum{3.7} Lemma. \label{art12-Lemma-3.7}} \textit{Let $A$ be an artinian local ring with residuce field $k$ and let $E_{*}$ be a $\Spec(A)$-family of parabolic sheaves. Assume that the restriction $\overline{E}=E_{*}\otimes_{A} k$ to th closed fiber is stable. Then the natural homomorphism $A\rightarrow \Hom^{\Par}(E_{*}, E_{*})$ is an isomorphism.}

 Let us  go back to the situation of \eqref{art12-eq-3.4}. Replacing the role of Lemma 6.1 argument of Lemma 6.3 of
 \cite{art12-key5} by the above lemma, we get a basic result on the action of $\overline{G}$ in $U$.

\medskip
\noindent
{\bfseries \thnum{3.8} Lemma. \label{art12-Lemma-3.8}}\textit{The action of $\overline{G}$ on $U$ is free.}

 It is well-known that this lemma implies the following (see \cite{art12-key5}, Proposition 6.4).

\medskip
\noindent
{\bfseries \thnum{3.9} Proposition. \label{art12-prop-3.9}}\textit{The natural morphism
$\pi : U \rightarrow W = U/\overline{G}$ is a principal fiber bundle with group $\overline{G}$}

Now we come to our proof of Proposition \ref{art12-prop-3.2}.

\medskip
\noindent
{\bfseries  Proof of Proposition\thnum{3.2}. \label{art12-prf-of-prop-3.2}} There is universal space $U$ whose quotient by the group $\overline{G}$ is $\tilde{M}(r, n)=M(H, H_{*}, \alpha_{*})$ under the notatior before Proposition \ref{art12-prop-3.2}. Since our moduli space $M(r,n)$ is open in $\tilde{M}(r,n)$ we have a $=\overline{G}$-invariant open subsheme $P$ of $U$ which mapped onto $M(r, n)$. Tlanks to Proposition \ref{art12-prop-3.9} the natural quotient morphism $\pi : P \rightarrow M (r, n)$ is a principal fiber bundle with group $\overline{G}$ and hence we have only to show that $P$ is smooth and has the right dimension. Put $X=\bP_{\bC}^{2}$. Fix an integer $m$ which satisfies the conditions
\eqref{art12-eq-3.3.1} and \eqref{art12-eq-3.3.2} for our $M(r, n)$. We set $H[m](x) =H(x +m)$. Take a point $E$ of
$M(r, n)(\bC)$ and a $\bC$-vector space $V$ of dimension $N=H^{0}(X, E(m))$. we have an surjection
$$
\theta : V_{x}=V\otimes_{\bC} \calO_{X}\rightarrow E(m).
$$
Since the kernel $K$ of $\theta$ is locally free, $\Hom_{\calO_{x}}(K, ???????(m))$ is the tangent space of $Q=\Quot_{V_{x}/X/\bC}^{H[m]}$ at the point $q$ that is given by the above sequence and an obstraction of the smoothness of $Q$ at $q$ is in $H^{1}(X, K^{\vee} \otimes_{\calO_{X}}E(m))$. Since $\Ext_{\calO_{X}}^{2}(E, E)$ is dual space of $\Hom_{\calO_{X}}(E, E(-3))$, it vanishes for stable $E$. We can apply the same argument as ion the proofs of Propositions 6.7 and 6.9 in \cite{art12-key5} to our situation and we see that $Q$ is smooth and of dimension
$$
2rn-r^{2} + N^{2}
$$
at the point $q$. $P$ is a folber space over an open subscheme of $Q$ whose fibers are an open subscheme of $\Quot_{\calO_{\ell}^{\oplus r}/\ell/\bC}$ consisting of surjections
$$
\calO_{\ell}^{\oplus r} \rightarrow \calO_{\ell}(r-1)
$$ 
such that the induced map of global sections in bijective. By Lemma \ref{art12-lemma.2.7} the space of obstructions for the smoothness of $P$ over $Q$ is\break $H^{1} (\ell,\calO_{\ell}(r)^{\oplus r-1}) =0$. Thus $P$ is smooth over $Q$ and hence so is over $\Spec(\bC)$. Moreover, the dimension of the fibers is equal to\break $\dim H^{0}(\ell, \calO_{\ell}(r)^{\oplus r-1})=r^{2}-1 $. Combining this and the above result on the dimension of $Q$, we see that $dim P=2rm +N^{2}-1$. Since $\dim\overline{G}=N^{2}-1$, $M(r,n)$ is of dimension $2rn$ at every point.
\hfill Q.E.D.

Base on Proposition \ref{art12-prop-3.2} and Hulek's result stated in Introduction, we can prove the following.

\medskip
\noindent
{\bfseries \thnum{3.10} Theorem. \label{art12-thm-3.10}} \textit{$M(r,n)_{0}$ is connected.}

\medskip
\noindent
{\bfseries Proof.} Our proof is divided into several steps. We set as befor $X$ to be $\bP_{\bC}^{2}$.

\begin{itemize}
\item[\bf(I)] If $r=1$, then $M(r,n)$ is the moudli space of ideals with colenght $n$ which define $0$-dimentsional closed subshcemes in $X\backslash \ell$. It is well known that this is irreducible \cite{art12-key2}.

\item[\bf(II)] Assume that $r \geq 2$. By Hulek's result we see that
$$
U(0)= \{E_{*} \in M (r, n)_{0} | H^{0}(x, E)=H^{0}(X, E^{\vee})= 0\}
$$
is irreducible. In fact, we have a surjective morphism of a $PGL(r)$-bundle over Hulek's parameter space of $s$-stable bundles to $U(0)$. Let us set
$$
U(a)=\{E_{*} \in M (r, n)_{0} | \dim H^{0}(X, E)= a\},
$$
$$
U(a, b) = \{E_{*} \in U(a) | E\cong \calO_{X}^{\oplus b} \oplus E_{1}, E_{1} \ncong \calO_{X}\oplus E_{1}'\}.
$$
Then $U(a)$ is locally closed and $U(a,b)$ is constructible in $M(r,n)_{0}$.

\item[(III)] We shall compute the dimension of $U(a,b)$. For an $E_{*} \in U(a, b)$, there is an extension
$$
0 \rightarrow \calO_{X}^{\oplus a} \rightarrow E \rightarrow J \rightarrow 0
$$
because $E$ is $\mu$-semi-stable and $c_{1}(E) =0$[\cite{art12-key6}, Lemma 1.1]. We have moreover that $ J|_{\ell}\cong \calO_{x}^{\oplus a}$ is a direct summand of $E$ and put $E=\calO_{X}^{\oplus b} \oplus E_{1}$. Consider the exact sequence
$$
0\rightarrow \calO_{X}^{\oplus a-b} \rightarrow E_{1} \rightarrow J \rightarrow 0.
$$
For the double dual $J'$ of $J$, we set $T= J'/J$ and $c =\dim H^{0}(x, T)$. Then we see
$$
\calE xt_{O_{x}}^{2} (T, \calO_{X})\cong  \calE xt_{\calO_{X}}^{1} (J, \calO_{X})
$$
and $\dim H^{0}(x, \calE xt_{\calO_{X}}^{2} (T, \calO_{X}))=c$. On the other hand, since $c_{2}(J^{\vee})=n-c$, $J^{\vee}$ is locally free and since $H^{2}(X, J^{\vee})=0$, we have
$$
\dim H^{1}(x, J^{\vee}) = n-c-(r-a) +d,
$$
where $d=\dim H^{0}(X, J^{\vee})$. Thus we see
\begin{equation*}
\begin{split}
\dim \Ext_{\calO_{X}}^{1}(J, \calO_{X}^{\oplus (a-b)})& = \dim H^{1}(X, (J^{\vee})^{\oplus a-b}) +\\
            &\quad \dim H^{0}(X, \calE xt_{\calO_{x}}^{t}(J, \calO_{X})^{\oplus a-b})\\
            &= (a-b) \{n-(r-a) + d \}.            
\end{split}
\end{equation*}
If we change free bases of $\calO_{X}^{\oplus a-b}$ , then we obtain the same sheaf. Hence if $\dim \Ext_{\calO_{X}}^{1} (J, \calO_{X}) < a-b$, then every extension of $J$ by $\calO_{X}^{\oplus a-b}$ contians $\calO_{X}$ as a direct factor, which is not the case. Therefore, we get
$$
a-b \leq n-(r-a)+d \quad \text{or}
$$
$$
n-r + b + d \geq 0
$$
The extensions of $J$ by $\calO_{X}^{\oplus a-b}$ are now parametrized by a space of dimension $(a-b)\{n-(r-a) + d\}$ and each point of the space is contained in a subspace of dimension $(a-b)^{2} + \dim \End_{\calO_{X}}(J)-1$ whose points parmetrize the same extension.

\item[\bf(IV)] Let us fix a system of homogeneous coordinates $(x_{0} : x_{1})$ of $\ell$ and identity $J|_{\ell}$ with the free sheaf $\oplus_{i=1}^{r-q}\calO_{\ell e_{i}}$. If we have a surjection
$$
\varphi : J|_{\ell} = \oplus_{i=1}^{r-a} \calO_{\ell e_{i}}\rightarrow \calO_{\ell}(r-a-1)
$$
and if $f_{1}(x_{0}, x_{1}) = \varphi(e_{1}), \ldots, f_{r-a}(x_{0},x_{1}) =\varphi(e_{r-a})$ are liearly independent, then for general homogeneous forms $g_{1},\ldots, g_{r-a}$ of degree $a-1$ and $g_{r-q+1},\ldots, g_{r}$ of degree $r-1$, we define a map $\tilde{\varphi}$ of $\oplus$ of $\oplus_{i=1}^{r}\calO_{\ell  e_{i}}$ to $\calO_{\ell}(r-1)$ by
\begin{align*}
\tilde{\varphi}(e_{i}) &= g_{i}x_{0}^{r-a} + x_{1}^{a}f_{i} \qquad 1 \leq i \leq r-a\\
\tilde{\varphi}(e_{j}) &=g_{j} \qquad r-a + 1 \leq j \leq r.
\end{align*}
Choosing $g_{j}$ suitably, we obtain a surjective $\tilde{\varphi} : \oplus_{i=1}^{r}\calO_{\ell e_{i}} \rightarrow \calO_{\ell}(r-1)$ which induces a bijection between the spaces of global sections. Conversely, if we have a homomorphism $\psi$ of $\oplus_{i=1}^{r}\calO_{\ell e_{i}}$ to $\calO_{\ell}(r-1)$ such that $h_{1}= \psi(e_{1}), \ldots, h_{r}= \psi(e_{r})$ are linearly independent. The  we can write $h_{i}$ uniquely in the following way:
$$
h_{i}=h_{i}'x_{0}^{r-a} + x_{1}^{a}h_{i}^{"}
$$
where $h_{i}'$ (or, $h_{j}''$) is of degree $a-1$ (or , $r-a-1$, resp.). There is a permutations $\sigma$ of $\{ 1, \ldots r\}$ such that $h_{\sigma(1)}'', \ldots,h_{\sigma(r-a)}''$ are linearly independent. Thus, after a  permutation of indices, the above procedure produces $\psi$ from a homomorphism $J|_{\ell} \rightarrow \calO_{\ell}(r-a-1)$ which induces a bijection between the spaces of global sections. We see therefore that the parabolic structures on $E$ are parametrtized by a fiber space over the space of parabolic structures of $J$ whose fibres are a finite union of open subschemes of an affine space of dimension $2ra-a^{2}$.
$$
\dim \End^{J}(E) \geq \dim \End_{\calO_{X}}(E)-\dim\End_{\calO_{X}}(J).
$$
On the other hand, we see
\begin{align*}
\End_{\calO_{X}}(E) &= \End_{\calO_{X}}(\calO_{X}^{\oplus b}) \oplus \Hom_{\calO_{X}}(\calO_{X}^{\oplus b}, E_{1})\\
                    &\quad  \oplus \Hom_{\calO_{X}}(E_{1}, \calO_{X}^{\oplus b}) \oplus \End_{\calO_{X}}(E_{1}).
\end{align*}
Since $\dim H^{0}(X, J^{\vee})=d$, there is an exact sequence
$$
0 \rightarrow J_{1} \rightarrow  J \rightarrow G \rightarrow 0
$$
with $(G^{\vee})^{\vee} \cong \calO_{X}^{\oplus d}$. For a $\xi$ in $\Hom_{\calO_{X}}(G, \calO_{X}^{\oplus a-b})\cong H^{0}(x, \calO_{X}^{\oplus d(a-b)})$, we have a member of $\End(E_{1})$
$$
E_{1} \rightarrow J \rightarrow G \xrightarrow{\xi} \calO_{X}^{\oplus a-b} \rightarrow E_{1}.
$$
Thus $\dim \End_{\calO_{X}}(E_{1}) \geq d(a-b)+ 1$. Therefore, we get the following inequality

\begin{align*}
\dim \End^{J}(E) &\geq \dim \End_{\calO_{X}}(E)-\dim \End_{\calO_{X}}(J)\\
 &\geq b^{2} + b(a-b) + ab + d(a-b) + 1 -\dim \End_{\calO_{X}}(J)\\
 &= ab + ad + 1 - \dim\End_{\calO_{X}}(J)
\end{align*}

\item[\bf(VI)] There is a couple $(A, \tilde{J}_{*})$ of a scheme $A$ and an $A$-family $\tilde{J}_{*}$ of parabolic sheaves which parametrizes all parabolic stable sheaves $J_{*}$ with rank $r-a$ and $c_{2}(J)=n$ such that the restriction $J|_{\ell}$ is trivial vector bundle and $H^{0}(X, J)=0$. We may assume that $A$ is reduces and quasi-finite over $M(r-a, n)$, and hence $\dim A\leq 2n(r-a)$. By Proposition \ref{art12-Prop.2.8} the sheaf $J$ defined in the step (III) appears as the underlying sheaf of a parabolic sheaf parametrized by $\tilde{J}_{*}$. For the underlying sheaf $\tilde{J}$ of the family $\tilde{J}_{*}$, we have a resolution by a locally free sheaves
$$
0 \rightarrow B_{1} \rightarrow B_{0} \rightarrow \tilde{J} \rightarrow 0.
$$
Using this resolution and splitting out $A$ into the direct sum of suitable subschemes, we can construct a locally free sheaf $C$ on $A$ such that for every point $y$ of $A$,  we have a natural isomorphism
$$
C(y) \cong \Ext_{\calO_{X}}^{1} (\tilde{J}(y), \calO_{X}).
$$
Note that $C$ is not necessarily of constant rank. On $D= \bV((c^{\vee})^{\oplus a-b})$ we have a universal section $\Xi$ of the sheaf $g^{*}(C^{\oplus a-b})$, where $g : X\times D \rightarrow X\times A$ is the natural projection. Let $\xi_{i}$ be the projection of $\xi$ to the $i$-the direct factor of $C^{\oplus a-b}$. The subset $D_{0}\{y \in D | \xi_{1}(y), \ldots, \xi_{a-b}(y)\; \text{span a linear subspace of rank}\; a-b\; C(y)\}$
$$
0 \rightarrow \calO_{X\times D_{0}}^{\oplus a-b} \rightarrow \tilde{E_{1}} \rightarrow g^{*} (\tilde{J})\rightarrow 0,
$$
where we denote $g |D_{0}$ by $g$. Let $H$ be the maximal open subscheme of $D_{0}$ where $\tilde{E_{1}}$ is locally free. Set $\tilde{E} = \tilde{E}_{1}|_{X\times H} \oplus \calO_{X \times H}^{\oplus b}$. (III) tells us
$$
\dim D_{0}\leq 2n(r-a) + (a-b) \{n-(r-a) + d\}.
$$
Note here that $d$ may depend on connected components of $D_{0}$. Furthermore, by the result of (III) again, each point of $H$ is contained. in a subspace of dimension $(a-b)^{2} + \dim \End_{\calO_{X}}(J)-1$ where $\tilde{E}_{1}$ parmetrizes the same extensions.

\item[\bf(VII)] By breaking up $H$ into the direct sum of suitable subschemes, we may assume that $g^{*}(\tilde{J})|_{\ell \times H}$ has a constant trivialization and hence so is $\tilde{E}|_{\ell\times H}$.
By the result of (IV), the parabolic structure of $g^{*}(\tilde{J})$ provides us with a fiber space $p : Z \rightarrow H$ and a $Z$-family $\tilde{E}_{*}$ of parabolic stable vector bundles such that $(X, \tilde{E}_{*})$ parametrizes all the parabolic stable sheaves contained in $U(a, b)$ and that every fiber of $p$ is of dimension $2ra-a^{2}$. Thus we see that
\begin{align*}
\dim Z &\leq 2n(r-a) + (a-b) \{n -(r-a) + d\} + 2ra-a^{2}\\
 &= 2nr-na -nb +ra +rb -ab + ad-bd
\end{align*}
 Moreover, the conclusion of (V) shows that on the fiber of $p$, each point is contained in a closed subsheme of dimension at least $ab + ad + 1-\dim \End_{\calO_{X}}(J)$ whose points define the same parabolic sheaf.

\item[\bf(VIII)] The family $\tilde{E}_{*}$ gives rise to a morphism of $Z$ to $M(r, n)_{0}$ whose image is exactly our $U(a,b)$. Combining the results of (VI) and (VII) we get
\begin{equation*}
\begin{split}
\dim U(a, b) &\leq 2nr -na -nb + ra + rb -ab +ad-bd-(a-b)^{2}-\\
             & \quad \dim \End_{\calO_{X}}(J) + 1 -ab -ad -1 + \dim \End_{\calO_{X}}(J)\\
             &= 2nr -(n-r +a)a-(n-r +b + d)b 
\end{split}
\end{equation*}
Since for every member $E_{*}$ of $U(a, b)$, the underlying sheaf $E$ is $\mu$-semi-stable, Riemann-Roch implies
$$
n-r + a = \dim H^{1}(X, E) \geq 0.
$$
This and the inequality we obtained in (III) show that
$$
\dim U(a, b) \leq 2nr.
$$
Replacing $E$ by $E^{\vee}$ in the definition of $U(a)$ and $U(a, b)$, we  define $U^{\vee}(a)$ and $U^{\vee}(a, b)$. Then, by the same argument as above we come to a family $(Z'm E_{*}')$ of the dual bundles of the members of $U^{\vee}(a, b)$. By taking the dual basis of the trivial sheaf $E'|_{\ell \times Z'}$, we have an isomorphism $E'^{\vee}|_{\ell \times Z'} \rightarrow E'|_{\ell \times Z'}$. Combining this isomorphism and the parabolic structure of $E_{*}'$, we obtain a $Z'$-family $E_{*}^{'\vee}$ of parabolic sheaves which parametrizes all the members of $U^{\vee}(a, b)$. The dimension of $U^{\vee}(a, b)$ is the same as $U(a, b)$.

\item [\bf(IX)] Assume that $n \geq r$. In this case we see that
$$
M(r, n)_{0} = U(0)\bigcup\left(\bigcup\limits_{a \geq 1, b \geq 0} U(a, b)\right) \bigcup \left(\bigcup\limits_{a \geq 1, b \geq 0} U^{\vee}(a, b)\right).
$$
By our result in (VIII) we see that if $a \leq 1$, the both $\dim U(a, b)$ and $\dim U^{\vee}(a,b)$ are less than $2rn$. On the other hand, we know that $U(0)$ is irreducible by \cite{art12-key3}. This completes the proof of this case.

\item [\bf(X)] Assme that $n < r$ . Then, Riemann-Roch implies that for every member $E_{*}$ of $M(r, n)_{0}$, we have
 $\dim^{0}(X, E) \geq r-n$. Thus we see that
 $$
 M(R, n)_{0} = U(r-n)\bigcup \left(\bigcup\limits_{a>r-n,b\geq 0} U(a, b)\right).
 $$
As in (IX) if $a > r-n$, then $\dim U(a, b)< 2rn$. This  means that it is sufficient to prive that $U(r-n)$ is connected. For a member $E_{*}$ of $U(r-n)$, there is an exact sequence
$$
0 \rightarrow \calO_{X}^{\oplus r-n} \rightarrow E\rightarrow J \rightarrow 0.
$$
According to the type of $J, U(r-n)$ is deivided into three subschemes:
\begin{align*}
V_{0}&=\{E_{*} \in U(r-n) | J \text{is locally free}, H^{0}(X, J^{\vee})=0\}\\
V_{1}&=\{E_{*} \in U(r-n) | J \text{is locally free}, H^{0}(X, J^{\vee})\neq 0\}\\
V_{2}&=\{E_{*} \in U(r-n) | J \text{is locally free} \}.
\end{align*}
For $J$ of $E_{*} \in V_{0}$, we have that $H^{0}(X, J^{\vee})=0$, $c_{1}(J^{\vee})=0$, $c_{2}(J^{\vee})=n, r(J^{\vee})=n$ and $J^{\vee}$ is $\mu$-semi-stable. Hence Riemann-Roch implies that $\Ext_{\calO_{X}}^{1}(J, \calO_{X})=H^{1}(X, J^{\vee})=0$. Then $V_{0}$ is contained in $U(r-n, r-n)$ or the undelying sheaf $E$ of a member of $V_{0}$ is written in a form $\calO_{X}^{\oplus r-n} \oplus J$ with $J$  $s$-stable. Since these $J$'s are parametrized by an irreducible variety, so are the members of $V_{0}$. This proves the irreducibilty of $V_{0}$. Applying the argument before (VIII) to the set $\{J^{\vee} | E_{*} \in V_{1}\}$, we find that $\{J_{*} | E_{*} \in V_{1}\}$ is of dimension less the $2n^{2}$. Then the dimension of $A$ in (VI) for $V_{1}$ for $V_{1}$ is of dimension less the $2n^{2}$ and hence $\dim V_{1}< 2nr$. A similar argument tells us that our remaining problem is to prove that $L=\{J_{*} |E_{*} \in V_{2}\}$ is of dimension less than $2n^{2}$.

\item[\bf(XI)] For the underlying sheaf $J$ of a member $J_{*}$ of $L$, we set $J^{'}$ to be the double dulal of $J$. Since $J$ is locally free in a neighborhood of $\ell$, the parabolic structure of $J_{*}$ induces of $J'$. $J'/J$ is a torsion sheaf supported by a 0-dimensional subscheme of $X$. $L$ is the disjoint union of $L_{1}, \ldots, L_{n}$, where $L_{m}=\{x_{1}, \ldots x_{k}\}$ be the support of $J'/J$. If the length of the artinian module $(J'/J)_{x_{i}}$ is $a_{i}$, then there is a filtration $0 \subset T_{a_{i}}^{(i)}\subset \cdots \subset T_{1}^{(i)} = (J'/J)_{x_{i}}$ such that $T_{j}^{(i)}/T_{j+1}^{(i)} \cong k(x_{i})$. Let $J_{j}^{(1)}$ be the kernel of the sujection $J'\rightarrow T_{1}^{(1)} /T_{j+1}^{(1)}$ and $J_{j}^{(2)}$ be the kernel of $J_{a_{1}}^{(1)} \rightarrow T_{1}^{(2)} /T_{j+1}^{(2)}$. Thus we can define a filtration $J_{a_{k}}^{(k)}\subset \cdot \subset J_{1}^{(k)}\subset \cdots\subset J_{a_{1}}^{(1)} \subset \cdots \subset J_{1}^{(1)} \subset J'$ such that $J_{j}^{(i)} /J_{j+1}^{(i)} \cong k(x_{i})$ and $J_{a_{i}}^{(i)} /J_{1}^{(i +1)} \cong k(x_{i+1})$ and we see that $J_{a_{k}}^{(k)}= J$. Since $J_{a_{i-1}}^{(i-1)}(X_{i})$ is an $n$-dimensional vector space, the surjection of $J_{a_{i-1}}^{(i-1)}$ to $k(x_{i})$ is parametrized by an $(n-1)$-dimensional projective space. Since $\Tor_{1}^{\calO}(k(x_{i}), k(x_{i}))$ is isomorphic to $k(x_{i})^{\oplus 2}$, the exact sequence
$$
\Tor_{1}^{\calO}(k(x_{i})), k(x_{i}) \rightarrow J_{j}^{(i)}(x_{i}) \rightarrow J_{j-1}^{(i)}(x_{i}) \rightarrow k(x_{i})\rightarrow 0
$$
shows us that $\dim J_{j}^{(i)}(x_{i})\leq \dim J_{j-1}^{(i)}(x_{i}) + 1$. Therefore, surjections of $J_{j}^{(i)}$ to $k(x_{i})$ is parametrized by a projective space of dimension less than or equal to $n+j-2$, where $\calO = \calO_{X, x_{i}}$. Fixing $J_{*}'\{J_{*} \in L_{m} |(J^{\vee})^{\vee} \cong J'\}$ is parametrized by a space of dimemsion less than or equal to
\begin{align*}
\delta_{m}&= \sum\limits_{i=1}^{k}\sum\limits_{j=1}^{a_{i}} (n+j-2) + 2k\\
 &= nm + \sum\limits_{i=1}^{k}\dfrac{a_{i}(a_{i}-3)}{2} + 2k\\
 &= nm + \sum\limits_{i=1}^{k}\left(\dfrac{a_{i}^{2}}{2}-\dfrac{3a_{i}}{2} + 2 \right).
\end{align*}
On the other hand, the space $\{J_{*}' |J_{*} \in L_{m}\}$ is of dimension $2n(n-m)$. Therefore $L_{m}$ is of dimension at most $2n(n-m)+ \delta_{m}$. Now we have
\begin{equation*}
\begin{split}
2n(n-m)+ \delta_{m} & = 2n^{2}-nm + \sum\limits_{i=1}^{k}\left(\dfrac{a_{i}^{2}}{2}- \dfrac{3a_{i}}{2} + 2\right)\\
& \leq 2n^{2}-m^{2} + \sum\limits_{i=1}^{k}\left(\dfrac{a_{i}^{2}}{2}-\dfrac{3a_{i}}{2} + 2 \right)\\
&= 2n^{2}- \sum\limits_{i=1}^{k}\left(\dfrac{a_{i}^{2}}{2} + \dfrac{3a_{i}}{2}-2 +a_{i}\sum\limits_{j \neq 1}a_{j}\right).
\end{split}
\end{equation*}
It is easy to see that
$$
\sum\limits_{i=1}^{k}\left(\dfrac{a_{i}^{2}}{2} + \dfrac{3a_{i}}{2}-2 +a_{i} \sum\limits_{j \neq i}a_{j}\right)
$$
is non-negative and equal to $0$ if and only if $n=1$. If $n=1$, then both  $V_{0}$ and $V_{1}$ are empty and $V_{2}$ is exactly $\{J=I_{x}\; \text{with}\; I_{x}\;\break \text{the ideal of a point} \;x \in X \backslash \ell \}$. There is a unique locally free sheaf $G_{x}$ which is an extension of $I_{x}$ by $\calO_{x}$. Finally we see that in this case the set undelying sheaves of the members of $U(r-n)$ is
$$
\{\calO_{X}^{\oplus  r-2} \oplus G_{x}| x \in X \backslash \ell\}
$$
which is parametrized by the irreducible variety $X\backslash \ell$.\hfill Q.E.D.

 Now we come to the connectedness of the moduli space of marked $SU(r)$-instantons.
\end{itemize}

\medskip
\noindent
{\bfseries  \thnum{3.11} Corollary. \label{art12-coro-3.11}} \textit{The moduli space $M(SU(r), n)$ of marked $SU(r)$-instantons with instanton number $n$ is connected.}


\begin{thebibliography}{99}
\bibitem {art12-key1} S.K. Donaldson, \textit{Instantosn and geometric invariant theory}, Comm. Math. Phys. {\bf 93} (1984) 453-460.

\bibitem {art12-key2} J. Fogarty, \textit{Algebraic families on an algebraic surface}, Amet. J. Math {\bf 90} 511-521.

\bibitem {art12-key3} K. Hulek, \textit{On the classification of stable rank-r vector bundles on thte projective plane}, Proc. Nice conference on vector bundles and differential equation, Birkh\"auser (1983) 113-142.

\bibitem {art12-key4} G. Kempf and L. Ness, \textit{Lengths of vectors in  representation spaces}, Lect. Notes in Math.
{\bf 732}, Springer-verlag, Berlin-Heidelberg-New York-Tokyo, (1978) 233-243.

\bibitem{art12-key5} M. Maruyama, \textit{Moduli of stable sheaves II}, J. Math. Kyoto Univ. {\bf 18}
(1978) 557-614.

\bibitem{art12-key6} M. Maruyama, \textit{On a compatification of a moduli space of stable vector bundles on a rational surface}, Alg, Geom. and Comm. Alg. in Honor of Masayoshi NAGATA, Kinokuniya, Tokyo (1987) 233-260.

\bibitem{art12-key7} M. Maruyama and K. Yokogawa, \textit{Moduli of parabolic stable sheaves}, to appear in Math. Ann.

\bibitem{art12-key8} C. H. Taubes, \textit{Pathconnected Yang-Mills moduli spaces}, J. Diff. Geom. {\bf 19} (1984) 337-392.
\end{thebibliography}
\begin{flushleft}
Department of Mathmatics

Faculty of Sciense

Koyoto University

Saky-ku, Kyoto 606-01

Japan
\end{flushleft}




