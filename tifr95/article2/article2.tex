\title{Impact of geometry of the boundary on the positive solutions of a semilinear Neumann problem with Critical nonlinearity}
\markright{Impact of geometry of the boundary on the positive solutions of a semilinear Neumann problem with Critical nonlinearity}

\author{By~ Adimurthi}
\markboth{Adimurthi}{Impact of geometry of the boundary on the positive solutions of a semilinear Neumann problem with Critical nonlinearity}



\date{}
\maketitle

\begin{center}
Dedicated to M.S. Narasimhan and C.S. Seshadri on their 60th Brithdays
\end{center}

Let\pageoriginale $n\geq3$ and $\mathbb{P}^{n}$ be a bounded domain with smooth boundary. We are concerned with the problem
of existence of a function $u$ satisfying the nonlinear equation
\begin{align}
-\Delta u &= u^{p}-\lambda u \quad{\rm in}\quad \Omega \nonumber\\
u &> 0 \nonumber\\
\dfrac{\partial u}{\partial v} &= 0 \quad {\rm on}\quad \partial \Omega \label{art2-eq1} 
\end{align}
where $p=\frac{n+2}{n-2},\lambda > 0$. Clearly $u=\lambda^{1/(p-1)}$ is a solution \ref{art2-eq1} and we call it a trivial solution. The exponent $p=\frac{n+2}{n-2}$ is critical from the view point of Soblev imbedding. Indeed the solution of \ref{art2-eq1} corresponds to critical points of the functional
\begin{equation}
Q_{\lambda}(u) = \dfrac{\int_{\Omega}|\nabla_{u}|^{2}dx + \lambda\int_{\Omega}u^{2}dx}{\left(\int_{\Omega}|u|^{p+1}dx\right)^{2/p+1}}\label{art2-eq2}
\end{equation}
on the manifold
\begin{equation}
M =\left\{u\in H^{1} (\Omega) ; \int_{\Omega}|u|^{p+1}dx=1\right\}\label{art2-eq3}
\end{equation}

In fact, if $v\leq 0$ is a critical point of \ref{art2-eq2} on $M$, then $u-Q(v)^{1/(p-1)v}$ satisfies \ref{art2-eq1}. Note that $p+1 =\frac{2n}{n-2}$ is the limiting exponent for the imbedding $H^{1}(\Omega)\mapsto L^{2n/(n-2)}(\Omega)$. Since this imbedding is not compact, the manifold $M$ is not weakly closed and hence $Q_{\lambda}$ need not satisfy the Palais Smale condition at all levels. Therefore there are serious difficulties when trying to find critical points by th standard variational methods. In fact there is a sharp contrast between the sub critical case $p<\frac{n+2}{n-2}$ and the critical case $p=\frac{n+2}{n-2}$.
  
Our motivation for investigation comes from a question of Brezis \cite{art2-key11}. If we replace the Neumann condition by Dirichlet condition $u=0$ on $\partial\Omega$ in \ref{art2-eq1}, then the existence and non existence of solutions depends in topology and geometry of the domain (see Brezis-Nirenbreg \cite{art2-key14}, Bhari-Coron \cite{art2-key9}, Brezis \cite{art2-key10}). In view of this, Brezis raised the following problem

\begin{center}
``Under what conditions on $\lambda$ and $\Omega$, (\ref{art2-eq1}) admits a solution?"
\end{center}

The interest in this problem not only comes from a purely mathematical question, but it has application in mathematical biology, population dynamics (see \cite{art2-16}) and geometry.

In order to answer the above question let us first look at the subcritical case where the compactness in assured.

\medskip
\noindent\textbf{Subcritical case }$1 < p < \frac{n+2}{n-2}$. 

\medskip
This had been studied extensively in the recent past by Ni \cite{art2-key18}, and Lin-Ni-Takagi \cite{art2-key16}. In \cite{art2-key16}, Lin-Li-Takagi have proved the following

\begin{theorem}\label{art2-thm1}
There exist two positive constants $\lambda_{*}$ and $\lambda_{*}$ such
\begin{itemize}
\item[a)] If $\lambda < \lambda_{*}$, then (\ref {art2-eq1}) admits only trivila solutions.
 \item[b)] If $\lambda > \lambda_{*}$, then (\ref {art2-eq1}) admits non constant solutions.
\end{itemize}

Further Ni \cite{art2-key18} and Lin-Ni \cite{art2-key15} studied the radial case for all $1< p < \infty$ and proved the following 
\end{theorem}

\begin{theorem}\label{art2-thm1}
Let $\Omega = {x; |x| < 1 }$ is a ball. Then there exists two positive constants $\lambda_{*}$ and $\lambda^{*}$  such that 
$\lambda_{*} \leq \lambda^{*}$ such that $\lambda_{*} \leq \lambda^{*}$ and
\begin{itemize}
\item[a)]For $1< p < \infty$, $\lambda > \lambda^{*}$, (\ref{art2-eq1}) admits a radially increasing solution
\item[b)] if $p \neq \frac{n+2}{n-2}$, then for $0< \lambda <  \lambda_{*}$, (\ref{art2-eq1}) does not admit a non constant radial solution.
\item[c)] Let $\Omega = \{x; 0 < \alpha < |x| < \beta\}$ be an annuluar domain and $ 1 < p < \infty$. Then there exist two positive constants $\lambda_{*} \leq \lambda^{*}$ such that for $\lambda_{*} \geq \lambda^{*}$, (\ref{art2-eq1}) admits a non constant radial solution and if $\lambda \leq \lambda^{*}$, then (\ref{art2-eq1}) does not admit a non constant radial solution. 
\end{itemize}
\end{theorem} 

In view of these results Lin and Ni \cite{art2-key15} made the following

\begin{conjecture*}
Let $p \geq \frac{n+2}{n-2}$ then there exist two positive constants $\lambda \leq \lambda^{*}$ such that
\begin{enumerate}[\rm (A)]
\item For $0 < \lambda < \lambda_{*}$, (\ref{art2-eq1}) does not admit non constant solutions.\label{art2-enum(A)}
\item For $\lambda < \lambda_{*}$, (\ref{art2-eq1}) admits a non constant solution.\label{art2-enum(B)}
\end{enumerate}

 In this article we analyze this conjecture in the \textbf{critical case} $p = \frac{n+2}{n-2}$. Surprisingly enougn, the critical case is totally different from the subcritical. In fact the part (\ref{art2-enum(A)}) fo the conjecture in general is false. The following results of Adimurthi and Yadava \cite{art2-key4} and Budd, Knaap and Peletier \cite{art2-key12} gives a counter example to the Part (\ref{art2-enum(A)}) of the conjecture.
\end{conjecture*}

\begin{theorem}
Let $n= 4, 5, 6$ and $\Omega = \{X: |x|< 1\}$. Then there exist a $\lambda_{*} > 0$ such that for $0 < \lambda < \lambda_{*}$, (\ref{art2-eq1}) admits a radially decreasing solution.

Let us now turn our attention to part (\ref{art2-enum(B)}) of the conjecture. Let $S$ denote the best Sobolev constant for the imbedding $H^{1}(\bbR^{n}) \mapsto  L^{2n/(n-2)}(\bbR^{n})$ given by  
\begin{equation}
S = {\rm inf}\left\{ \int_{\bbR^{n}} |\nabla u|^{2}dx: \int_{\bbR^{n}} |u|^{2n/n-2}dx =1 \right \}\label{art2-eq4}
\end{equation}
Then $S$ is achieved and any minimizer in given by $U_{\varepsilon, x_{0}}$ for some $\varepsilon > 0$, $x_{0} \in \bbR^{n}$ where
\begin{align}
U(x) &= \left[ \dfrac{n(n-2)}{n(n-2)+|x|^{2}}\right]^{\frac{n-2}{2}}\label{art2-eq5}\\
U_{\varepsilon, x_{0}}(x) &= \dfrac{1}{\varepsilon^{\frac{n-2}{2}}}U \left(\dfrac{x-x_{0}}{\varepsilon}\right)\label{art2-eq6}
\end{align}   
\end{theorem}

In order to answer part ({\ref{art2-enum(B)}}) of the conjecture, geometry of the boundary play an important role. To see this, we look at a more general problem than (\ref{art2-eq1}), the mixed problem.

Let $\partial \Omega = \Gamma_{0}\cup  \Gamma_{1}, \Gamma_{0} \cap \Gamma_{1} = \phi, \Gamma_{i}$ are submanifolds of dimension $(n-1)$. The problem is to find function $u$ satisfying
 \begin{align}
-\Delta u + \lambda u &= u ^{\frac{n+2}{n-2}} \quad {\rm in}\quad \Omega\nonumber\\
u &> 0 \nonumber\\
u &= 0 \quad {\rm on}\quad \Gamma_{0}\label {art2-eq7}\\
\dfrac{\partial u}{\partial v} &= 0 \quad {\rm on}  \quad\Gamma_{1} \nonumber     
 \end{align}
Let
\begin{align}
H^{1}(\Gamma_{0}) &= \left\{u \in H^{1}(\Omega) : u = 0\;\; {\rm on}\;\; \Gamma_{0}\right\}\label{art2-eq8}\\[4pt]
S(\lambda, \Gamma_{0}) &= {\rm inf}\left\{Q_{\lambda(u) \;; \;u \;\in\; H^{1}\; (\Gamma_{0})\; \cap\; M}\right\}\label{art2-eq9}
\end{align} 

Clearly, if $S(\lambda, \Gamma_{0})$ is achieved by some $v$, then we can take $v\leq 0$ and $u=S(\lambda, \Gamma_{0})^{\frac{n-2}{4}}v$ satisfies (\ref{art2-eq7}). $u$ is called a \textit{minimal energy} solution. Existence of a minimal energy solution is proved in Adimurthi and Mancini \cite{art2-key1} (See also X.J. Wang \cite{art2-key22}) and have the following 

\begin{theorem}
Assume that there exist an $x_{0}$ belonging to the interior of $\Gamma_{1}$ such that the mean curvature $H(x_{0})$ at $x_{0}$ with respect to unit outward normal is positive. Then $S(\lambda, \Gamma_{0})$ is achieved.
\end{theorem}

\begin{sketch of the proof}
The proof consists of two steps.
\begin{itemize}
\item[{\rm\bf Step 1.}] Suppose $S(\lambda, \Gamma_{0}) < S/2^{2/n}$, then $S(\lambda,m \Gamma_{0})$ is achieved.

Let $v_{k} \in H^{1}(\Gamma_{0})\cap M$ be a minimizing sequence. Clearly $\{v_{k}\}$ is bounded in $H^{1}(\Omega)$. Let for subsequence of $\{v_{k}\}$ still denoted by $\{v_{k}\}$, coverges weakly to $v_{0}$ and almost everywhere in $\Omega$. We first claim that $v_{0}\nequiv 0$. Suppose $v_{0}\equiv 0$, then by Cherrier imbedding (See \cite{art2-key8}) for every $\varepsilon > 0$, there exists $C(\varepsilon) > 0$ such that
 $$
 1 = \left(\int_{\Omega}|v_{k}|^{p+1}dx\right)^{2/p+1} \leq \dfrac{2^{2/n}}{S}(1+\varepsilon) \int_{\Omega}|\Delta v_{k}|^{2}dx + C(\varepsilon)\int v_{k}^{2}dx.
$$
By Rellich's compactness, $v_{k} \rightarrow 0$ in $L^{2}(\Omega)$ and hence in the above inequality letting $k \rightarrow \infty$ and $\varepsilon \rightarrow 0$ we obtain
\begin{align*}
1 &\leq \lim_{\varepsilon \rightarrow 0}\dfrac{2^{2/n}}{S}(1+\varepsilon) \lim_{k \rightarrow \infty} Q_{\lambda}(v_{k})\\
&= \dfrac{2^{2/n}}{S}S(\lambda, \Gamma_{0})\\
&< 1
\end{align*}
which is a contradiction. Hence $v_{0}\nequiv 0$. Let $h_{k} = v_{k}-V_{0}$, then $h_{k}\rightarrow 0$ weakly in $H^{1}(\Omega)$ and strongly in $L^{2}(\Omega)$. Hence
\begin{align*}
S(\lambda, \Gamma_{0}) &= Q_{\lambda}(v_{k}) + 0(1)\\
&= Q_{\lambda}(v_{0})\left(\int_{\Omega} |v_{0}|^{p+1}\right)^{s/p+1} + \int_{\Omega} |\Delta h_{k}|^{2}dx + 0(1)
\end{align*}
Now by Brezis-Lieb Lemma, Cherrier imbedding, from the above inequality, and by the hypothesis, we have for sufficiently small $\varepsilon > 0$,
\begin{align*}
S(\lambda, \Gamma_{0}) &= S(\lambda, \Gamma_{0}) \left(\int_{\Omega}|v_{k}|^{p+1}dx\right)^{2/p+1}\\
&\leq S(\lambda, \Gamma_{0}) \left\{\left(\int_{\Omega}|v_{0}|^{p+1}dx\right)^{2/p+1} + \left(\int_{\Omega}|h_{k}|^{p+1}dx\right)^{2/p+1}\right\} \\
&\quad + 0(1)\\
&= S(\lambda, \Gamma_{0}) \left\{\left(\int_{\Omega}|v_{0}|^{p+1}dx\right)^{2/p+1} +\dfrac{2^{2/n}}{S}(1 +              \varepsilon)\times \right. \\
&\qquad\qquad\qquad\qquad \left.\int_{\Omega}|\nabla h_{k}|^{2}dx\right\} + 0(1)\\
&= S(\lambda, \Gamma_{0}) \left(\int_{\Omega}|v_{0}|^{p+1}dx\right)^{2/p+1} + \int_{\Omega}|\nabla h_{k}|^{2}dx + 0(1)\\
&= S(\lambda, \Gamma_{0}) \left(\int_{\Omega}|v_{0}|^{p+1}dx\right)^{2/p+1} + S(\lambda, \Gamma_{0})-Q_{\lambda}(v_{0}) \times \\
&\qquad \qquad \qquad \left(\int_{\Omega}|v_{0}|^{p+1}dx\right)^{2/p+1} 
\end{align*}it
this implies that $Q_{\lambda}(v_{0}) \leq S(\lambda,\Gamma_{0})$. Hence $v_{0}$ is a minimizer.
\item[\rm \bf Step 2.] $S(\lambda, \Gamma_{0}) < S/2^{2/n}$

Let $x_{0}$ belong to the interior of $\Gamma_{1}$ at which $H(x_{0}) > 0 $ and $r > 0$ that $B(x_{0}, r) \cap \Gamma_{0} = \phi$. Let $\varphi \in C_{0}^{\infty}(B(x_{0}, r))$ such that $\varphi = 1$ for $|x-x_{0}| < r/2$. Let $\varepsilon > 0$ and $v_{\varepsilon}  = \varphi U_{\varepsilon, x_{0}}$. Then $v_{\varepsilon} \in H^{1}(\Gamma_{0})$ and we can find positive constants $A_{n}$ and $a_{n}$ depending only on $n$ such that      
\begin{equation}
Q_{\lambda(v_{\varepsilon})} = \dfrac{S}{2^{2/n}} -A_{n}H(x_{0})\beta_{1}(\varepsilon) + a_{n}\lambda \beta_{2}(\varepsilon) + 0(\beta_{1}(\varepsilon) + \beta_{2}(\varepsilon))\label{art2-eq10}
\end{equation}
\end{itemize}

\end{sketch of the proof}
