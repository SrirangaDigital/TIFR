\title{Compact complex manifolds whose tangent bundles satisfy numerical effectivity properties}
\markright{Compact complex manifolds whose tangent bundles satisfy numerical effectivity properties}

\author{By~ Jean-Pierre Demailly\\(joint work with Thomas Peternell and Michael Schneider)}
\markboth{Jean-Pierre Demailly}{Compact complex manifolds whose tangent bundles satisfy numerical effectivity properties}

\date{}
\maketitle
\begin{center}
Dedicated to M. S. Narasimhan and C.S. Seshadri on their sixtith birthdays
\end{center}

\setcounter{section}{-1}
\section{Introduction}\label{art5-sec-0}
A compact Riemann surface always ha s hermitian metric with constant curvature, in particular the curvature sign can be taken to be constant: the negative sign corresponds to curves of general type (genus $\geq$ 2), while the case to zero curvature corresponds to elliptic curves (genus 1), positive curvature being obtained only for $\bbP^{1}$ (genus 0). In higher dimensions the situation is must more subtle and it has been a long standing conjecture due to Frankel to characterize $\bbP_{n}$ as the only compact K\"ahler manifold with positive holomorphic bisectional curvature. Hratshorne strengthened Frankel's conjecture and asserted that $\bbP_{n}$ is the only compact complex manifold with ample tangent bundle. In his famous paper \cite{art5-keyMo79}, Mori solved Hartshorne's conjecture by using characteristic $p$ methods. Around the same time Siu and Yau \cite{SY80} gave an analytic proof of the Frankel conjecture. Combining algebraic and analytic tools Mok \cite{art5-keyMk88} classfied all compact K\"ahler manifolds with semi-positive holomorphic bisectional curvature. 

From the point of view of algebraic geometry, it is natural to consider the class fo projective manifolds $X$ whose tangent bundle in numerically effective (nef). This has been done by Campana and Peternell \cite{art5-keyCP91} and - in case of dimension 3 -by Zheng \cite{art5-keyZh90}. In particular, a complete classification is obtained for dimension at most three.

The main purpose of this work is to investigate compact (most often K\"ahler) manifolds with nef tangent or anticanonical bundles in arbitrary dimension. We fist discuss some basic properties of nef vector bundles which will be needed in the sequel in the general context of compact complex manifolds. We refer to \cite{art5-keyDPS91} and  \cite{art5-keyDPS92} for detailed proofs. Instead, we put here the emphasis on some unsolved questions.

\section{Numerically effective vector bundles}\label{art5-sec-1}

In algebraic geometry a powerful and flexible notion of semi-positivity is \textit{numerical effectivity}(``nefness"). We will explain here how to extend this notion to arbitrary compact complex manifolds.

\begin{definition}\label{art5-definition-1.1}
A line bundle $L$ on a projective manifold $X$ is said to be \textit{numerically effective} (nef for short) if $L\cdot C \geq 0$ for all compact curves $C \subset X$.

It is cleat that a line bundle with semi-positive curvature is nef. The converse had been conjectured by Fujita \cite{art5-keyFu83}. Unfortunately this is not true; a simple counterexample can be obtained as follows:
\end{definition}

\begin{example}\label{art5-example-1.2}
Let $\Gamma$ be an elliptic curve and let $E$ be a rank 2 vector bundle over $\Gamma$ which is a non-split extension of $\calO$ by $\calO$; such a bundle $E$ can be described as the locally constant vector bundle over $\Gamma$ whose monodromy is given by the matrices
$$
\begin{pmatrix}
1 & 0\\
0 & 1
\end{pmatrix}
,\qquad
\begin{pmatrix}
1 & 1\\
0 & 1
\end{pmatrix}
$$
associated to a pair of generators of $\pi_{1}(\Gamma)$. We take $L=\calO_{E}(1)$ over the ruled surface $X= \bbP(E)$. Then $L$ is nef and it can be checked that, up to a positive constant factor, there is only one (possibly singular) hermitian metric on $L$ with semi-positive curvature; this metric is  unfortunately singular and has logarithmic poles along a curve. Thus $L$ cannot be semi-positive for any smooth hermitian metric. 
\end{example}

\begin{definition}\label{art5-definition-1.3}
A vector bundle $E$ is called \textit{nef} if the line bundle $\calO_{E}(1)$ is nef on $\bbP(E)$ (= projectivized bundle of hyperplanes in the fibres of $E$).

Again it is clear that vector bundle $E$ which admits a metric with semi-positive curvature (in the sense of Griffiths) is nef. A compact K\"ahler manifold $X$ having semi-positive holomorphic bisectional curvature has bu definition a tangent bundle $TX$ with semi-positive curvature. Again the converse does not hold. One difficulty in carrying over the algebraic definition of nefness to the Kahler case is the possible lack of curves. This is overcome by the following:
\end{definition}

\begin{definition}\label{art5-definition-1.4}
Let $X$ be a compact complex manifold with a fixes hermaitian metric $\omega$. A line bundle $L$ over $X$ in \textit{nef} if for every $\varepsilon > 0$ there exists a smooth hermitian metric $h_{\varepsilon}$ on $L$ such that the curvature satisfies
$$
\Theta_{h_{\varepsilon}} \geq -\varepsilon\omega.
$$

This means that the curvature of $L$ can have an arbitrarilly small negative part. Clearly a $\nef$ line bundle $L$ satisfies $L\cdot C \geq 0$ for all curves $C\subset X$, but the coverse in not true ($X$ may have no curves at all, as is the case for instance for generic complex tori). For projective algebraic $X$ both notions coincide; this is an easy consequence of Seshadri's ampleness criterion: take $L$ to be a $\nef$ line bundle in the sense of Definition
\ref{art5-definition-1.1} and let $A$ be an ample line bundle; then $L^{\otimes K}\otimes A$ is ample for every integer $k$ and thus $L$ has smooth hermitian metric with curvature form $\Theta(L) \geq -\frac{1}{k}\Theta(A)$.

Definition \ref{art5-definition-1.3} can still be used to define the notion of $\nef$ vector bundles over arbitrary compact manifolds. If $(E, h)$ is a hermitian vector bundle recall that the Chern curvature tensor
$$
\Theta_{h}(E) =\dfrac{i}{2\pi}D_{E, h}^{2} = i \sum\limits_{\substack{1 \leq j , k\leq n \\ 1 \leq \lambda, \mu \leq r}}
a_{jk\lambda\mu}dz_{j} \wedge d\overline{z}_{k}\otimes e_{\lambda}^{\star}\otimes e_{\mu}
$$
is a hermitian (1,1)-form with values in $\Hom(E, E)$. We say that $(E, h)$ is \textit{semi-positive in Griffiths' sence} \cite{art5-keyGr69} and write $\Theta_{h}(E) \geq 0$ if $\Theta_{h}(E)(\xi \otimes t) = \sum a_{jk\lambda \mu} \xi_{j}\overline{\xi}_{k}v_{\lambda}\overline{v}_{\mu}\geq 0$ for every $\xi \in T_{x}X$, $v\in E_{x}$, $x\in X$. We write $\Theta_{h}(E)> 0$ in case there is strict inequality for $\xi \neq 0$, $m v\neq 0$. Numerical effectivity can then be characterized by the following differential geometric criterion (see \cite{art5-keyDe91}).  
\end{definition}

\begin{criterion}\label{art5-definition-1.5}
Let $\omega$ be a fixed hermitian metric on $X$. A vector bundle $E$ on $X$ is $\nef$ if and only if there is a sequence of hermitian metrics $h_{m}$ on $S^{m}E$ and a sequence $\varepsilon_{m}$ of positive numbers decreasing to 0 such that
$$
\Theta_{h_{m}}(S^{m} E) \geq -m\varepsilon_{m}\omega\otimes \Id_{S^{m} E}
$$
in the sense of Griffiths.

The main functional properties of $\nef$ vector bundles are summarized in the following proposition.
\end{criterion}

\begin{prop}\label{art5-definition-1.6}
Let $X$ be an arbitrary compact complex manifold and let $E$ be a  holomorphic vector bundle over $X$.
\begin{enumerate}[(i)]
\item \footnote{We expect (\ref{art5-enum-(i)}) to hold whenever $f$ is surjective, but there are serious technical difficulties to overcome in the nonalgebraic case.} If $f: Y\rightarrow X$ is a holomorphic map with equidimensional fibres, then $E$ is nef if and only if $f^{\star}E$ is nef.\label{art5-enum-(i)}
\item Let $\Gamma^{a}E$ be the irreducible tensor representation of $Gl(E)$ of highest weight $a= (a_{1}, \ldots a_{r}) \in \bbZ^{r}$, with $a_{1}\geq\ldots\geq a_{r} \geq 0$, Then $\Gamma^{a}E$ is nef. In particular, all symmetric and exterior powers of $E$ are nef.\label{art5-enum-(ii)}
\item let $F$ be a holomorphic vector bundle over $x$. If $E$ and $F$ are nef, then $E\otimes F$ is nef.\label{art5-enum-(iii)}

\item If some symmetric power $S^{m}E$ is nef $(m > 0)$, then $E$ in nef. \label{art5-enum-(iv)}

\item Let $0 \rightarrow F \rightarrow E \rightarrow Q \rightarrow 0$ be an exact sequence of holomorphic vector bundles over $X$. Then\label{art5-enum-(v)}
    \begin{enumerate}
        \item[($\alpha)$] $E$ nef $\Rightarrow Q$ nef.
        \item [($\beta)$] $F$, $Q$ nef $\Rightarrow E$ nef.
        \item [($\gamma)$] $E$ nef, $(\det Q)^{-1}$ nef $\Rightarrow F$ nef.
    \end{enumerate}
\end{enumerate}
\end{prop}

The proof of these properties in the general analytic context can be easily obtained by curvature computations. The argumentsa are parallel to those of the algebraic case and will therefore be omitted (see \cite{art5-keyHa66} and \cite{art5-keyCP91} for that case). Another useful result which will be used over and over in the sequel is

\begin{prop}\label{art5-definition-1.7}
Let $E$ be a nef vector bundle over a connected compact n-fold $X$ let $\sigma \in H^{0}(X, E^{\star})$ be a non zero section. Then $\sigma$ does not vanish anywhere.
\end{prop}

\begin{proof}
We merely observe that if $h_{m}$ is a sequence of hermitian metrics in $S^{m}E$ as in criterion 5, then
$$
T_{m}=\dfrac{i}{\pi} \partial\overline{\partial} \dfrac{1}{m}\log||\sigma^{m}||_{h_{m}}
$$
has zero $\partial\overline{\partial}$-cohomology class and satisfies $T_{m}\geq -\varepsilon_{m}\omega$. It follows that $T_{m}$ converges to a weak limit $T\geq 0$ with zero cohomology class. Thus $T= i\partial\overline{\partial}\varphi$ for some global plurisubharamonic function $\varphi$ on $X$. By the maximum principle this implies $T=0$. However, if $\sigma$ vanishes at some point $x$, then all $T_{m}$ have Lelong number $\geq 1$ at $x$. Therefore so has $T$, contradiction.
\end{proof}

one of out key results is a characterizations of vector bumdles $E$ which are numerically flat, i.e. such that both $E$ and $E^{\star}$ are nef.

\begin{theorem}\label{art5-thm-1.8}
Suppose that $X$ is K\"ahler. Then a holomorphic vector bundle $E$ over $X$ in numerically flat iff $E$ admits a filtration
$$
\{0 \} = E_{0}\subset E_{1}\subset \ldots \subset E_{p} = E
$$
by vector subbundles such that the quotients $E_{k}/E_{k-1}$ are hermitian flat, i.e. given by unitary representations $\pi_{1}(X) \rightarrow U(r_{k})$.
\end{theorem}

\begin{sketch of proof}
It is clear by \ref{art5-definition-1.6} (\ref{art5-enum-(v)}) that every vector bundle which os filtrated with hermitian flat quotients is nef as well as its dual. Conversely, suppose that $E$ is numerically flat. This assumption implies$c_{1}(E) = 0$ Fix a K\"ahler metric $\omega$. If $E$ is $\omega$-stable, then $E$ is Hermite-Einstein by the Unlenbeck-Yau theorem \cite{art5-keyUY86}, Moreover we have $0 \leq c_{2}(E) \leq c_{1}(E)^{2}$ by Theorem \ref{art5-thm-1.9} below, so $c_{2}(E) = 0$. Kobayashi's flatness that $E$ is hermitian flat. Now suppose that $E$ is unstable and take $\calF \subset\calO(E)$ to be destabilizing subsheaf of minimal rank $p$. We then have by definition $c_{1}(\calF) = c_{1}(\det \calF) = 0$ and the morphism $\det\calF \rightarrow \Lambda^{p}E$ cannot have any zero curvature current on the line bundle $\det \calF$, contradiction). This implies easily that $\calF$ is locally free, and we infer that $\calF$ is also numerically flat. Since $\calF$ is stable by definition, $\calF$ must be hermitian flat. We set $E_{1} = \calF$, observe that $E' - E/E_{1}$ is again numerically flat and proceed by induction on the rank.
\end{sketch of proof}

Another key point, which has been indeed used in the above proof, is the fact that the Fulton-Lazarsfeld inequalities \cite{FL83} for Chern classes of ample vector bundles still hold for nef vector bundles over compact K\"ahler manifolds:

\begin{theorem}\label{art5-thm-1.9}
Let $(X, \omega)$ be a compact K\"ahler manifold and let $E$ be a nef vector bundle on $x$. Then for all positive polynomials $p$ the cohomology class $P(c(E))$ is numerically positive, that is, $\int_{Y}P(c(E)) \bigwedge \omega^{k} \geq 0$ for anu $k$ and any subvariety $Y$ of $X$.
\end{theorem}

By a positive polynomial in the Chern classes, we mean as usual a homogeneous weighted polynomial $P(c_{1}m \ldots, c_{r})$ with $\deg c_{i}= 2i$, such that $P$ is a positive integral combination of Schur polynomials:
$$
P_{a}(c)= \det (c_{a_{i}-i+j})_{1\leq i, j\leq r}, \quad r \geq a_{1} \geq a_{2} \geq \ldots \geq a_{r} \geq 0
$$
(by convention $C_{0}=1$ ana $c_{i} = 0$ if $i \neq [0,r]$, $r= \rank E$). The proof of Theorem \ref{art5-thm-1.9} is based essentially on the same artuments as the original proof of \cite{art5-keyFL83} for the ample case: the starting point is the nonnegativity of all Chern classes $c_{k}(E)$ (Bloch-Gieseker \cite{art5-keyBG71}); the general case then
follows from a formula of Schubert calculus known as the Kempf-Laksov formula \cite{art5-keyKL74}, which express any Schur ploynomial $P_{a}(c(E))$ as a Chern class $c_{k}(F_{a})$ of some related vector bundle $F_{a}$. The only change occurs in the proof of Gieseker's result, where the Hard Lefschetz theorem is needed for arbitrary K\"ahler metrics instead of hyperplane sections (fortunately enough, the technique then gets simlified, covering tricks being eliminated). Since $c_{1}c_{k-1}-c_{k}$
$$
0 \leq c_{k}(E) \leq c_{1} (E)^{k} \;\text{\rm for all}\; k
$$
Therefore all Chern monomials are bounded above by corresponding powers $c_{1}(E)^{k}$ of the same degree, and we infer:

\begin{coro}\label{art5-coro-1.10}
If $E$ in nef and $c_{1}(E)^{n} =0$, $n= \dim X$, then all Chern polynomials $P(c(E))$ of degree $2n$ vanish.
\end{coro}

\section{Compact K\"ahler manifolds with nef anti-canonical line bundle}\label{art5-sec-2}
 Compact K\"ahler manifolds with zero or semi-positive Ricci curvature have been investigated by various authors (cf. \cite{art5-keyCa57}, \cite{art5-keyKo61}, \cite{art5-keyLi67}, \cite{art5-key71}, \cite{art5-key72}, \cite{art5-key} \cite{art5-keyBo74a}, \cite{art5-keyb}, \cite{art5-keyBe83}, \cite{art5-keyKo81} and \cite{art5-keykr86}). The purpose of this section is to discuss the following two conjectures.

\begin{conjecture}\label{art5-conje-2.1}
Let $X$ be a compact K\"ahler manifold with numerically effective anticanonical bundle $K_{X}^{-1}$. Then the fundametal group $\pi_{1}(X)$ has polynomial growth.
\end{conjecture}

\begin{conjecture}\label{art5-conje-2.2}
Let $x$ be a compact K\"ahler manifold with $K_{X}^{-1}$ numerically effective. Then the Albanese map $\alpha : X\rightarrow \Alb(X)$ is a smooth fibration onto the Albanese torus. If this hold, one can infer that there is a finite
\'etale cover $\widetilde{X}$ has simply connected fibres. In particular, $\pi_{i}(X)$ would almost abelian (namely an extension of a finite group by a free abelian group).

These conjectures are known to be true if $K_{X}^{-1}$ is semi-positive. In both cases, the proof is based in differential geometric techniques (see e.g. \cite{art5-keyBi63}, \cite{art5-keyHK75} for Conjecture \ref{art5-conje-2.1} and \cite{art5-keyLi71} for Conjecture \cite{art5-conje-2.2}). However, the methods of proof are not so easy to carry over to the nef case. Our main contributions to these conjectures are derived from Theorem \ref{art5-thm-2.3} below.
\end{conjecture}

\begin{theorem}\label{art5-thm-2.3}
Let $X$ be a compact K\"ahler manifold with $K_{X}^{-1}$ nef. Then $\pi_{1}(X)$ is a group of subexponential growth.
\end{theorem}

The proof actually gives the following additional fact (this was already known before, see \cite{Bi63}).

\begin{coro}\label{art5-coro-2.4}
If morever $-K_{X}$ is hermitian semi-positive, then $\pi_{1}(X)$ has polynomial growth of degree $\leq 2 \dim X$, in particular $h^{1}(X, \calO_{X}) \leq \dim X$.
\end{coro}

As noticed by F. Campana (private communication), Theorem \ref{art5-thm-2.3} also implies the following consequences. 

\begin{coro}\label{art5-coro-2.5}
Let $X$ be a compact K\"ahler manifold with $K_{X}^{-1}$ nef. Let $\alpha : X\rightarrow \Alb(X)$ be the Albanese map and set $n=\dim X$, $d=\dim \alpha(X)$. If $d=0, 1$ od $n$, $\alpha$ is surjective. The same is true if $d = n-1$ and if $X$ is projective algebraic. 
\end{coro}

\begin{coro}\label{art5-coro-2.6}
Let $x$ be a K\"ahler surface or a projective 3-fold with $K_{X}^{-1}$ nef. Then the Albanese map $\alpha : X\rightarrow \Alb(X)$ is surjective.
\end{coro}

We now explain the main ideas required in the proof of Theorem \ref{art5-thm-2.3}. If $G$ is a finitely generated group with generators $\mathsf{g}_{1},\ldots, \mathsf{g}_{p}$, we denote by $N(k)$ the number of elements $\gamma \in G$ which can be written as words
$$
\gamma = \mathsf{g}_{i_{1}}^{\varepsilon_{1}}\ldots\mathsf{g}_{i_{k}}^{\varepsilon_{k}}, \quad \varepsilon_{j} = 0, 1 \; \text{\rm or} \; -1
$$
of length $\leq k$ in terms of the generators. The group $G$ is said to have \textit{subexponential growth} if for every $\varepsilon > 0$ there is a constant $C(\varepsilon)$ such that
$$
N(k) \leq C(\varepsilon)e^{\varepsilon k}\; \text{\rm for} \; k\geq 0.
$$
This notion is independent of the choice of generators. In the free group with two generators, we have $N(k) = 1+4(1+3+3^{2}+\cdots + 3^{k-1}) =2 \cdot 3^{k} - 1$, thus a group with subexponential growth cannot contain a non abelian free subgroup.

The first step consists in the construction of suitable K\"ahler metric on $X$. Since $K_{X}^{-1}$ in nef, for every $\varepsilon > 0$ there exists a smooth hermitian metric $h_{\varepsilon}$ on $K_{X}^{-1}$ such that
$$
u_{\varepsilon} = \Theta _{h_{\varepsilon}}(K_{X}^{-1}) \geq -\varepsilon\omega.
$$
By \cite{art5-keyY77} and \cite{art5-keyY78} there exists a unique k\"ahler metric $\omega_{\varepsilon}$ in the cohomology class ${\omega}$ such that
\begin{equation}
\Ricci(\omega_{\varepsilon}) = -\varepsilon\omega_{\varepsilon} + \varepsilon\omega + u_{\varepsilon}.\tag{+}\label{eq:simple}
\end{equation}
In fact $u_{\varepsilon}$ belongs to the Ricci class $c_{1}(K_{X}^{-1}) = c_{1}(X)$, hence so does the right hand side $-\varepsilon\omega_{\varepsilon} + \varepsilon\omega + u_{\varepsilon}$. In particular there exists a function $f_{\varepsilon}$ such that
$$
u_{\varepsilon} = \Ricci(\omega) + i \partial\overline{\partial}f_{\varepsilon}.
$$
If we set $\omega_{\varepsilon} = \omega + i \partial\overline{\partial}f_{\varphi}$ (where $\varphi$ depends on $\varepsilon$), equation \eqref{eq:simple} is equivalent to the Monge-Amp\`ere equation
\begin{equation}
\dfrac{\left(\omega + i \partial \overline\partial\right)^{n}}{\omega^{n}} = e^{\varepsilon\varphi-f_{\varepsilon}}\tag{++}\label{eq:simple1}
\end{equation}
because
\begin{align*}
i\partial\overline\partial\log(\omega + i \partial\overline\partial \varphi)^{n}/\omega^{n} &= \Ricci(\omega) - \Ricci(\omega_{\varepsilon})\\
&=\varepsilon(\omega_{\varepsilon} - \omega) + \Ricci(\omega) - u_{\varepsilon}\\
&= i\partial\overline\partial(\varepsilon \varphi - f_{\varepsilon}).
\end{align*}
It follows from the general results of \cite{art5-keyY78} that \eqref{eq:simple1} has a unique solution $\varphi$, thanks to the fact the right hand side of \eqref{eq:simple1} is increasing in $\varphi$. Since $u_{\varepsilon} \geq -\varepsilon\omega$, equation \eqref{eq:simple} implies in particular that $\Ricci(\omega_{\varepsilon}) \geq -\varepsilon \omega$.

Now, recall the well-known differential geometric technique for\break bounding $N(K)$ (this technique has been explained to us in a very efficient way by S.Gallot). Let $(M, \mathsf{g})$ be a compact Riemannian $m$-fold and let $E\subset \widetilde{M}$ be a fundamental domain for the action of $\pi_{1}(M)$ on the universal covering $\widetilde{M}$. Fix $a\in E$ and set $\beta - \diam E$ . Since $\pi_{1}(M)$ acts isometrically on $\widetilde{M}$ with respect to the pull-back metric $\overline{\mathsf{g}}$, we infer that
$$
E_{k} = \bigcup\limits_{\gamma\in \pi_{1}(M),\;\; \length(\gamma)\leq k} \gamma(E)
$$
has volume equal to $N(k)$ $\Vol(M)$ and is contained in the geodesic ball $B(a, \alpha k + \beta)$, where $\alpha$ is maximum of the length of loops representing the generators $\mathsf{g}_{j}$. Therefore
\begin{equation}
N(K) \leq \dfrac{\Vol(B(a, \alpha k + \beta))}{\Vol(M)} \tag{*}\label{eq:sample2}
\end{equation}
and it is enough to bound the volume of geodesic balls in $\widetilde{M}$. For this we use the following fundamental inequality due to R. Bishop \cite{art5-keyBi63}, Heintze-karcher \cite{art5-keyHK78} and M. Gage \cite{art5-keyGa80}.

\begin{lem}
Let
$$
\Phi : T_{a}\widetilde{M} \rightarrow \widetilde{M},\qquad \Phi(\zeta)= \exp_{a}(\zeta)
$$ 
\end{lem}
