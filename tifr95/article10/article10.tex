\title{An Appendix to Bases for Quantum Demazure modules-I}
\markright{An Appendix to Bases for Quantum Demazure modules-I}

\author{}
\markboth{}{An Appendix to Bases for Quantum Demazure modules-I}
\date{}
\maketitle

Let $g$ be symeetrizable Kac-Moody Lie algebra, and $U$ be the quantized enveloping of $g$ as constructed by Drinfeld (cf \cite{art10-keyD}) and Jimbo (cf \cite{art10-keyJ}). This is an algebra over $\bbQ(v)$ ($v$ being a parameter) which specializes ti $U(g)$ for $v=1, U(g)$ being the universal enveloping algebra of $g$. This algebra has agenerators $E_{i}, F_{i}, k_{i}, 1 \leq i \leq n$, which satisfy ``the quantum Chevalley and Serre relations". Let $U^{\pm}$ be the $\bbQ(v)$-sub algebra of $U$ generated by $E_{i}$(resp$F_{i}$), $1 \leq i \leq n$. Let $A = \bbZ[v, v^{-1}]$ and $U_{A}^{\pm}$ be the $A$-subalgebra of $U$ generated by $E_{i}^{(r)}$ (resp. $F_{i}^{r}$), $1 \leq i \leq n, r \in \bbZ^{+}$, (here $E_{i}^{(r)}, F_{i}^{(r)}$ are the quantum divided powers of (cf \cite{art10-keyJ})). Let $\lambda$ be a dominant, integral weight and $V_{\lambda}$ the associated simple $U$-module. Let us fix a highest weight vector $e$ in $V_{\lambda}$ and denote $V_{A}=U_{A^{e}}(=U_{A}^{-}e)$. Let $W$ be the Weyl group of $g$. For $w \in W$, let $e_{w}$ be the corresponding extremal weight vector in $V_{\lambda}$ of wight $w(\lambda)$. Let $V_{w} = U^{+}e_{w}, V_{w, A} = U_{A}^{+}e_{w}$. In \cite{art10-keyLa} (see also \cite{art9-keyLS}), we proposed a conjecture (which we recall below) towards the construction of an $A$-basis for $V_{A}$ compatible with $\{V_{w, A}, w \in W\}$. This conjecture consists of two parts. The first part givews a (conjectural) character formula for the  $U^{+}$-module $V_{w}$ in terms of certain weighted chains in $W$. The second part gives a conjectural $A$-basis $B_{\lambda}$ for $V_{A}$, compatible with
 $\{V_{w, A} w \in W \}$. We now state the conjecture.  

\subsection*{Part I $I_{\lambda}$, an indexing set for $B_{\lambda}$}
Let $\lambda = \Sigma d_{i}\omega_{i}, \omega_{i}$ being the fundamental weights.

\section*{Admissble weighted $\lambda$-chains}
~

Let $\underline{c}= \{ \mu_{0}, \mu_{1}, \ldots \mu_{r}\}$ be a $\lambda$-chain in $W$, i.e., $\mu_{i} \> \mu_{i-1}, \ell(\mu_{i}) = \ell(\mu_{i-1})+1$ (if $d_{t}=0$ for $t = i_{1}, \ldots, i_{s}$, then we shall work with $w^{Q}$, the set of minimal representatives of $W_{Q}$ in $W, W_{Q}$ being the weyl group of the parabolic subgroup $Q$, where $S_{Q}=(\{\alpha_{t}, t= i_{1}, \cdots, i_{s}\})$). 

Let $\mu_{i-1}= s_{i}\mu_{i}$, where $\beta_{i}$ is some positive real root. Let $(\mu_{i-1}(\lambda), \beta^{*}) = m_{i}$.

\noindent
{\bfseries A 1. Definition.}\label{A. 1. definition.} A $\lambda$ -chain $\underline{c}$ is called \textit{simple} if all $\beta_{i}'$s are simple. 

\medskip
\noindent
{\bfseries A 2. Definition.}\label{A. 2. definition.} m By a \textit{weighted $\lambda$ chain} we shall mean $(\underline{c}, \underline{n})$ where $\underline{c}= \{ \mu_{0}, \cdots, \mu_{r}\}$ is chain and $\underline{n} =\{n_{1}, \ldots, n_{r}\}, n_{i} \in \bbZ^{+}$.

\medskip
\noindent
{\bfseries A 3. Definition.}\label{A. 3. definition.}A weighted $\lambda$-chain $(\underline{c}, \underline{n})$ is said to be \textit{admissible} if $1 \geq \frac{n_{1}}{m_{1}}\geq \cdots \geq \frac{n_{r}}{m_{r}} \geq 0$.

Let $(\underline{c}, \underline{n})$ be admissible. Let us denote the unequal values in $\left\{\frac{n_{1}}{m_{2}}, \cdots \frac{n_{r}}{m_{r}}\right\}$ by $a_{1}, \cdots, a_{s}$ so that $1 \geq a_{1}> a_{2} > \cdots > a_{s} \geq 0$. Let $i_{0}\cdots, i_{s}$ be defined by
$$
i_{0}=0, i_{s}=r, \frac{n_{j}}{m_{j}} = a_{t}, i_{t-1}+1 \leq j \leq i_{t}.
$$
We set
$$
D_{\underline{c}, \underline{n}} = \{(a_{1}, \ldots, a_{s});(\mu_{i_{0}},\ldots,\mu_{i_{s}})\}
$$

\medskip
\noindent
{\bfseries A 4. Definition.}\label{A. 4. definition.} Let $(\underline{c}, \underline{n}), (\underline{c}', \underline{n}')$ be two admissible weighted $\lambda$-chains. Let $D_{\underline{c}, \underline{n}} =\{(a_{1}, \ldots, a_{s});(\mu_{i_{0}}\cdots,\mu_{i_{s}})\}$, and $ D_{\underline{c}', \underline{n}'} = \{(a_{1}', \cdots, a_{l}');\break (\tau_{j_{0}}, \ldots, \tau_{j_{l}})\}$. We say $(\underline{c}, \underline{n}) \sim (\underline{c}',\underline{n}')$, if $s =t$, and $a_{t}=a_{t}'$, $i_{t}=j_{t}$, $\mu_{i_{t}} = \tau_{j_{t}}$,$1 \leq t \leq s$.

Let $C_{\lambda}=\{$ all admissible weighted $\lambda$-chains$\}$, and $I_{\lambda} = C_{\lambda}/ \sim$. Let $ \pi \in I_{\lambda}$, and let $(\underline{c}, \underline{n})$ be as representative of $\pi$. With notations as above, we set
$$
\tau(\pi)= \mu_{i_{s}}, v(\pi) = \sum\limits_{t=0}^{s}(a_{t}-a_{t+1})\mu_{i_{t}}(\lambda)
$$
where $a_{0}=1$ and $ a_{s+1}=0$ (note that $\tau(\pi)$ and $ v(\pi)$ depend only on $\pi$ and not on the representative chosen). For $w \in W$, let
$$
I_{\lambda}(w) =\{\pi \in I_{\lambda} \;| \;w \geq \tau(\pi)\}.
$$

\medskip
\noindent
{\bfseries A 5. Conjecture.}\label{A. 5. Conjecture.}
$$
\Char V_{w}= \sum_{\pi \in I_{\lambda}(w)} e^{v(\pi)}
$$

\medskip
\subsection*{Part II: An A-basis for $V_{A}$ compatible with $\{V_{w,A}, w \in W\}$}
~

Let $\pi,(\underline{c}, \underline{n})$ etc. be as in Part I. To $(\underline{c}, \underline{n})$ there corressponds a canonical (not necessarily admissible) weighted chain $(\delta(\underline{c}), \delta(\underline{n}))$ with $\delta(\underline{c})$ \textit{simple} (cf \cite{art10-keyLa},3.8). Let $\delta(\underline{c}) = \{\theta = \tau_{o}, \ldots, \tau_{r}\}, \underline{n}= \{n_{1}, \cdots, n_{r}\}$, $\beta_{t} = \alpha_{i_{t}}$, $1 \geq t \geq r$ (note that $\beta_{t}$'s are simple). We set
$$
v_{\underline{c}, \underline{n}} = F_{i_{r}}^{(n_{r})}\cdots F_{i_{1}}^{(n_{1})}e_{\theta}
$$ 

\medskip
\noindent
{\bfseries A 6. Conjecture.}\label{A. 6. Conjecture.} For each $\pi \in I_{\lambda}$, choose a representative $(\underline{c}, \underline{n})$ for $\pi$. Then $\{v_{\underline{c}, \underline{n}}: w \geq \tau (\pi)\}$ is $A$-basis for $V_{w, A}$.

In \cite{art10-keyLi}, Littelmann proves Conjecture 1, and as a consequence gives a Littlewood-Richardshon type ``decomposition rule" for a symmetrizable KacMoody lie algebras $g$, and a `` restriction rule" for a Levi subalgebra $L$ of $g$ which we state below.

Let $\theta$ be a dominant integral weight and let $\pi \in I_{\theta}$. Let $(\underline{c}, \underline{n})$ be a representative of $\pi$ and let $D_{\underline{c}, \underline{n}}$ be as above. Let us denote
$$
p(\pi, \theta) = \left\{\sum\limits_{k=t}^{s}(a_{k}-a_{k+1})\mu_{i_{k}}(\theta), 0 \leq t \leq s \right\}
$$

\medskip
\noindent
{\bfseries A 7. Definition.}\label{A. 7. Definition.} Let $\lambda, \theta$ be two dominant, integral weights. Let $\pi \in I_{\theta}$. Then $\pi$ is said to be $\lambda$-dominant if $\lambda + p(\pi, \theta)$ is contained in the dominant Weyl chamber.

\medskip
\noindent
{\bfseries A 8. Definition.}\label{A. 8. Definition.} Let $L$ be a Levi subalgebra of $g$, and let $\pi \in I_{\lambda}$.  Then $\pi$ is said to be \textit{$L$-dominant} if $p(\pi, \lambda)$ is contained in the dominant Weyl chamber of $L$. 


\medskip
\noindent
{\bfseries Decomposition rule.} (\cite{art10-keyLi}) Let $\lambda, \mu$ be two dominant integral weights. Let $I(\lambda, \mu) = \{\pi \in I_{\mu} \; | \; \pi\; \text{is}\; \lambda-\text{dominant}\}$. Then
$$
V_{\lambda} \otimes V_{\mu} = \bigotimes\limits_{\pi \in I(\lambda, \mu)}V_{\lambda +v(\pi)}
$$

\medskip
\noindent
{\bfseries Restriction rule.} (\cite{art10-keyLi}) Let $L$ be a  Levi subalgebra of $g$ . Let $ I(\lambda, L) = \{\pi \in I_{\lambda}: \pi \; \text{is}\; \text{L-dominant}\}$. Then
$$
res_{L}V_{\lambda} = \bigoplus\limits_{\pi \in I(\lambda, L)}U_{v(\pi)}
$$
(here, for an integral weight $\theta$ contained in the dominant Weyl chamber of $L, U_{\theta}$ denotes the corresponding simple highest weight module of $L$)

In \cite{art10-keyLi}, Littelmann introduces operators $e_{\alpha}, f_{\alpha}$ on $I_{\lambda}$, (for $\alpha$ simple), and associates an oriented, colored (by the simples roots) graph $G(V_{\lambda})$ with $I_{\lambda}$ as the set of vertices, and $\pi \xrightarrow{\alpha} \pi'$ if $\pi' =f_{\alpha}(\pi)$. He conjectures that $ G(V_{\lambda})$ is  the crystal graph of $V_{\lambda}$ as constructed by Kashiwara (\cite{art10-keyK}). 

Using the decomposition rule, Littlemann gives in \cite{art10-keyLi} a new (and simple) proof of the Parthasarathy-Ranga Rao Varadarajan conjecture.

\begin{thebibliography}{99}
\bibitem[D]{art10-keyD} V.G. Drinfeld, \textit{Hopf algebras and the Yang-Baxter equation}, Soviet Math. Dokl. {\bf 32} (1985) 254-258.

\bibitem[J]{art10-keyJ} M. Jimbo, \textit{A q-difference analogue of $U(g)$ and the Yang-Baster equation}, Lett. Math. Phys. {\bf 10} (1985) 63-69.

\bibitem[K]{art10-keyK} M. Kashiwara, \textit{Crystalizing the Q-analogue of Universal Enveloping Algebras}, Commun. Math. Phys. {\bf 133} (1990) 249-260.
\bibitem[La]{art10-keyLa}V. Lakshmibai, \textit{Bases for Quantum Demazure modules}, (The same volume).
\bibitem[Li]{art10-keyLi} P. Litttelmann, \textit{A Littlewood-Richardson rule for symmetrizable Kac-Moody algebras}(preprint).
\bibitem[LS]{art10-keyLS} V. Lakshmibai and C.S Seshadri, \textit{Standard Monomial Theory}, ``Proceedings of Hyderbad conference on Algebraic Groups", Manoj Prakashan, Madras (1991) 279-323.
\end{thebibliography}

\begin{flushleft}
Northeastern University

Mathematics Department

Boston, MA 02115, U.S.A.
\end{flushleft}
