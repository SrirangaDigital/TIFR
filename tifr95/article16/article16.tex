\title{Geometric Super-rigidity}
\markright{Geometric Super-rigidity}

\author{By~Yum-Tong Siu\footnote{Partially supported by a grant from the National Science Foundation}}
\markboth{By~Yum-Tong Siu}{Geometric Super-rigidity}


\date{}
\maketitle

\section*{Introduction}

A more descriptive title for this talk should be: ``The superrigidity of Margulis as a consequence of the nonlinear Matsushima vanishing theorem". What is presented in this talk is the culmination of an investigation in the theoey of geometric superrigidity which Sai-Kee-Yeung and I started about two years ago.

We first used the method of averaging and invariants to obtain Bochner type formulas which yield geometrix superrigidity for the Grassmannians and some other cases. Finally we obtained a general Bochner type formula which includes the usual formulas of Bochner, Kodaira, Matsushima, and Corlette as well as those obtained by averaging so that all cases of geometric superrigidity in its most general form can be derived from such a general Bochner type formula I would like to point out that, for the difficult cases such as those with a Grassmannian of rank at least two as domain and a Riemannian manifold wity nonpositive sectional curvature as target, the formula form the Matsushima vanishing theorem does not yield geometric supperigidity. For those difficult cases one needs the cases of the general Bochner type formula motivated by the method of averaging and invariants. Even with the other simpler cases for which the formula from the Matsushima vanishing theorem yields geometric superrigidity, to get the result with only the assumption of nonnegative sectional curvature for the target manifold instead of the stronger assumption of nonnegative curvature operator condition, one needs the use of an averaging argument.

Geometric superrigidity means the Archimedian case of Margulis's superrigidity \cite{art16-keyMar} formulated geometrically by assuming the target manifold to be only a Riemannian manifold with nonpositive curvature condition instead of locally symmetric. The complex case of Mostow's strong rigidity theorem \cite{art16-keyMos} is a consequence of the nonlinear version of Kodaira's vanishing theorem which yields a stronger result requiring only the target manifold to be suitably nonpositively curved rather than locally symmetric \cite{art16-keySi}. It turns out that in the same way the Archimedian case of Margulis's superrigidity is a consequence of the nonlinear version of Matsushima's vanishing theorem for the first Betti number \cite{art16-keyMat}. Again the result is stronger in that the target manifold is required only to be suitably nonpositively curved instead of locally symmetric. Moreover, this approach provides a common platform for Margulis's supperrigidity for the case of rank at least two and the recentsupperigiduty result of Corlette \cite{art16-keyCo} for the hyperbolic spaces of the quaternions and the Cayley numbers. The reason for the such vanishing theorem is the holonomy group which explians why supperigidity works for rank at least two as well as the hyperbolic spaces of quaternions and the Cayley numbers. The curvature $R(X, Y)$ as an element of the Lie algebra of $End(T_{M})$ generates the Lie algebra of the holonomy group. The minimum condition is that the holonomy group is $\calO(n)$ which simply says that $R(X, Y)$ is skew-symmetric. To get a useful vanishing theorem one needs an additional condition to remove a term involving only the curvature of the domain manifold. The K\"ahler case is the same as the holonomy group being $U(n)$. Then $R(X, Y)$ is $\bC$-linear as an element of $End(T_{m})$. This additional condition enables one to obtain the Kodaira vanishing theorem for negative line bundles. Other holonomy groups help yield vanishing theorems for geometric superrgidity. One can also get vanishing theorems for some of the special holonomy groups.

The approach to geometric superrigidity as the nonlinear version of Matsushima's vanishing theorem is motivated by a remark which E.  Calabi made privately to me during the Arbeitstagung of 1981 when I delivered a lecture on the newly discovered approach to the complex case of Mostow's strong rigidity as the nonlinear version of Kodaira's vanishing theorem. Calabi remarked that there is another vanishing theorem, namely Matsushima's which one should look at. He also remarked that Kodaira's vanishing theorem involves the curvature tensor quadratically \cite{art16-keyCa}. Actually the early rigidity result of $A$. Weil \cite{art16-keyW} already depends on Calabi's idea of integrating the square of the
curvature [\cite{art16-keyMat}, p. 316] and this early rigidity result launched the theory of strong rigidity and superrigidity. ?`From this point of view it is not surprising that superrdidity cane be approached from Matsushima's vanishing theorem. We state first here the final result we obtained.

\medskip
\noindent
{\bfseries Theorem \thnum{1.} \label{art16-thm-1}} \textit{Let $M$ be a compact locally symmetric irreducible Riemanninan manifold of nonpositive curvature whose universal cover is not the real or complex hypebolic space. Let $N$ be a Riemannian manifold whose complexified sectional curvature is nonpositive. If $f$ is a nonconstant harmonic map from $M$ to $N$, then the map from the universal cover of $M$ to that of $N$ induced by $f$ is a totally geodesic isometric embedding.}

Here nonpositive complexified sectional curvature means that
$$
R^{N}(V, W; \overline{V}, \overline{W})\leq 0
$$
for any complexified tangent vectors $V, W$ at any $x \epsilon N$, where $R^{N}$ is the curvature tensor of $N$. In this talk we follow the convention in Matusushima's paper \cite{art16-keyMat} that $R_{ijij}$ is negative for a negative curvture tensor [\cite{art16-Mat}, p. 314, line 6].

\medskip
\noindent
{\bfseries Theorem \thnum{2.} \label{art16-thm-2}} \textit{In Theorem \ref{art16-thm-1} when the rank of $M$ is at least two, one can replace the curvature condition of $N$ by the weaker condition that the Riemananian sectional curvature of $N$ is nonpositive.}

When the universal cover of $M$ is bounded symmetric domain of rank at least two, Theorem \ref{art16-thm-1} was proved by Mok \cite{art16-Mo}. When the universal cover of $M$ is the hyperbolic space of the quaternions and the Cayley numbers, Corlette's result differs from Theorem \ref{art16-thm-1} only in that Corlette's result requires the stronger curvature condition that the quadratic form $(\xi^{ij}) \mapsto R_{ijkl}^{N}\xi^{ij}\xi^{kl}$ be nonpositive for skew-symmetric $(\xi^{ij})$.

\medskip
\noindent
{\bfseries Theorem \thnum{3.} \label{art16-thm-3}} \textit{In Theorems \ref{art16-thm-1} and \ref{art16-thm-2} let $X$ be the universal cover of $M$ and $\Gamma$ be the fundamental group of $M$. Then the conclusions of
Theorems\ref{art16-thm-1} and \ref{art16-thm-2} remain true when the harmonic map $f$ from $M$ to $N$ is replaced by a $\Gamma$-equivariant harmonic map $f$ from $X$ to $N$.}

\medskip
\noindent
{\bfseries Remark.} With the existence result for equivariant harmonic maps corresponding to the results of Eells-Sampson \cite{art16-keyE-S}, Theorem \cite{art16-thm-3} implies the following Archimedian case of the superrigidity theorem of Marugulis \cite{art16-keyMar}: For lattices $\Gamma$ and $\Gamma'$ extends to a homomorphism from $G$ to $G'$, when $G$ is noncompact simple of rank at least two and $\Gamma$ is cocompact. The general Archimedian case of the superrigidity theorem of Margulis would follow from the corresponding generalization of Theorem \ref{art16-thm-3}. In order not to distract from the key points of our arguments, we will not discuss such generalizations in this talk. Also we will focus only on Theorems \ref{art16-thm-1}and \ref{art16-thm-2}, because the modifications in the proofs of Theorems
\ref{art16-thm-1} and \ref{art16-thm-2} needed to get Theorem \ref{art16-thm-3} are straightforward. 

\section*{An Earlier Approach of Averaging}
After Corlette \cite{art16-keyCo} obtained the superrigidity for the case of the hyperbolic spaces of the quaternions and the Cayley numbers, Sai-Kee Yeung and I started to try to undersatand how Corlette's result could be fitted in a more complete global picture of geometric superrigidity. Corelette's method is to generalize the method of the nonlinear $\partial\overline{\partial}$-Bochner formula for the complex strong rigidity by replacing the K\"ahler form used there by the invariant 4-form in the case of the quaternionic hyperbolic space. That 4-form corresponds to the once on the quaternionic projective space whose restriction to a quaternionic line is its standard volume form. Later Gromov \cite{art16-keyG} introduced the method of foliated harmonic maps so that Corlette's result could be proved by applying the nonlinear $\partial\overline{\partial}$-Bochner formula to the leaves. In his proof of the case of Theorem\ref{art16-thm-1} when the universal cover of $M$ is a bounded symmetric domain of rank at least two, Mok \cite{art16-keyMo} remarked that, according to Gromov, one should be able to develop the foliation technique of Gromov \cite{art16-keyG} to extend Mok's proof to many Riemannian symmetric manifolds of the noncompact type with rank at least two by considering families of totally geodesic Hermitian symmetric submanifolds of rank at least two.

The earlier approach Sai-Kee Yeung and I adopted was motivated by Gromov's work on foliated harmonic maps. We started out by considering a totally geodesic Hermitian suymmetric submanifold $\sigma$ of the universal cover $X$ of $M$. We look at the nonlinear $\partial\overline{\partial}$-Bochner formula applied to the restriction of the Hermitian-symmetric submanifold $\sigma$ and the average over all such submanifolds under the action of the automorphism group of $X$.

More precisely, we let $X$ be the quotient of a Lie group $G$ by a maximum compact subgroup $K$ and let $\Gamma$ be the fundamental group of $M$. Choose a suitable subgroup $H$ of $G$ so that $H/(A\cap K)$ is a bounded symmetric domain of complex dimension at least two. We pull back the harmonic map $f: M \rightarrow N$ to a map $\tilde{f}$ from $G/K$ to $N$ and, for every $k$ is $K$, apply the nonlinear $\partial\overline{\partial}$-Bochner technique developed in [\cite{art16-keySi}, \cite{art16-keySa}] to the restriction of $\tilde{f}$ to $k \cdot(H/(H \cap K))$. Since the image of $k\cdot (H /(H \cap K))$ in $\Gamma\backslash X$ is noncompact, one has to introduce a method a averaging over $k$ to handle the step of integration by parts. As a result of averaging over $k$ the integrand of the gradient square term of the differential of the map $f$ in the formula is an averaged expression of the Hessian of $f$.

The difficult step in this approach is to determine under what condition this averaged expression of the Hessian of $f$ is positive definite in the case of a harmonic map. It turns out that in some cases when we use only one single subgroup $H$ of $G$ this averaging expression in general is not positive definite for harmonic maps. To overcome this difficulty we choose two subgroups $H_{1}$ and $H_{2}$ instead of a single $H$ and we sum the $\partial\overline{\partial}$-Bochner formulas for the two subgroups. For example, this is done in the case of $SO(p,q)/S(O(p) \times O(q))$ for $p > 2$ and $q>2$ $(\nu =1, 2)$ and the sum of the two averaged expressions of the Hessian of $f$ turns out to be positive definite for harmonic maps for this case.

In Cartan's classification of Riemannian symmetric manifolds, besides the ten exceptional ones there are only the following four series which are not Hermitian symmetric:$ SO(p,q)/S(O(p)\times O(q))$,\break $Sp(p,q)/Sp(p)\times Sp(q), sU(k)/SO(k)$, and $SU^{*}(2n)/Sp(n)$. We explicitly verified that for these four series the averaged expression of the Hessian of the Hessian of $f$ is positive definite in the case of a harmonic map so that both Theorem \ref{art16-thm-1} and Theorem \ref{art16-thm-2} hold for these four series.  

The method of verification is to use scalar invariants from the representation of compact groups and Cramer's rule. More precisely, let $V$ be a finite-dimensional vector space over $\bbR$ with an inner product $<\cdot, \cdot>$. Let $K$ be a compact subgroup of the special orthogonal group $SO(V)$ with respect to the inner product. Let $S$ be an element of $V^{\oplus 4}$. To compute the average $\int_{g \in K^{g}} \cdot S$, we first enumerate all the one-dimensional $K$-invariant subspaces $\bbR I_{\kappa} (1 \leq \kappa \leq k)$ of $V^{\oplus 4}$ so that $\int_{g \in k^{g}}\cdot S= \sum_{\kappa=1}^{k} c_{\kappa}I_{\kappa}$ for some constants $c_{\kappa}$. By taking the inner product of this equation with $I_{\Lambda}$, we have the system of linear equations $\sum_{\kappa=1}^{k} c_{\kappa}< I_{\kappa},I_{\Lambda}> =<S, I_{\Lambda}>$ from which we can use Cramer's rule to solve for the constants $c_{\kappa}$. 

For such verification it does not matter whether one uses the original Riemannian symmetric space or its compact dual and we will use its compact dual in the following description of the verification.

For the case of $G=SO(p,q)$ and $K=S(O(p)\times O(q))$ for $p > 2$ and $q > 2$ we use the two subgroups $H_{1}=SO(p,2)$ and $H_{2}=SO(2,q)$ of $G$ so that $H_{j}/(H_{j}\cap K)$ is a bounded symmetric domain of rank two. The tangent space of $G/K$ is given by a $p \times q$ matrix and we denote the second partial derivative of the map $f$ with the $(\alpha, \beta)^{th}$ entry and the $(\gamma, \delta)^{th}$ entry by $f_{\alpha \beta, \gamma \delta}$. (Similar notations are also used for the description of the other three seres without further explanation.) Then the avearaged expression $\Phi_{\sigma_{1}}$ of the Hessian of $f$
