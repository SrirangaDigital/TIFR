\title{A Problem of Zariski}\label{chap4}
\markright{A Problem of Zariski}

\author{By J.-L. Colliot-Thelene}
\markboth{J.-L. Colliot-Thelene}{A Problem of Zariski}

\date{}
\maketitle

\setcounter{page}{93}

\footnotetext[1]{Summary of results appearing in \cite{chap4-key4}.}
\setcounter{pageoriginal}{126}
IN\pageoriginale 1949, ZARISKI raised the following question. Let $X$
and $Y$ be two algebraic varieties over a field $k$, let $P^{r}_{k}$
be the projective $r$-space over $k$. Assume that
$X\times_{k}P^{r}_{k}$ is $k$-birational to
$Y\times_{k}P^{r}_{k}$. Does it follow that $X$ is $k$-birational to
$Y$? In particular, if a $k$-variety $X$ is stably $k$-rational
(i.e. $X\times_{k}P^{r}_{k}$ is $k$-birational to $P^{d+r}_{k}$), is
it $k$-rational (i.e. is $k$-birational to $P^{d}_{k}$)? 

Although a positive answer is known in special cases
(Segre \cite{chap4-key13}, Nagata \cite{chap4-key11}), the problem
remained open for many years. A positive answer in the particular case
would have been of interest for the rationality problem of some moduli
spaces of stable vector bundles over curves
(\cite{chap4-key12}, \cite{chap4-key1}), as was pointed out to me by
Seshadri at the Colloquium. Unfortunately, the answer to Zariski's
problem is negative (\cite{chap4-key4}):

There exist stably rational surfaces over a suitable non-algebraically
closed field $k$ which are not $k$-rational, and there exist stably
rational threefolds over the complex field $C$ which are not
rational. 

Over\pageoriginale a non-algebraically closed field $k$, with
char. $k\neq 2$, which admits a field extension $K/k$, Galois with
group $\mathfrak{s}_{3}$ (the symmetric group on three letters), our
examples are surfaces given by an affine equation:
\begin{equation}
y^{2}-az^{2}=P(x)\label{chap4-eq1}
\end{equation}
where $P(x)$ is a separable polynomial with coefficients in $k$, and
$a\in k^{*}$, and moreover:

(2)~ $P$ is irreducible of the third degree, and $a=\disc (P)\in
k^{*}$ is not a square.

That such a surface is not $k$-rational is a special case of a result
of Iskovskih (\cite{chap4-key9}). That it is stably $k$-rational
appeared in the course of arithmetic investigations
(\cite{chap4-key8}) on surfaces as in \eqref{chap4-eq1} with $P(x)$ of
degree at most 4. As a rule, such surfaces are $k$-unirational as soon
as they have a $k$-rational point, but they need not be stably
$k$-rational (e.g. if $P(x)$ is a polynomial of the third degree which
is split over $k$, and $a$ is not a square in $k$; this particular
case is the one originally considered by F. Ch\^atelet
(\cite{chap4-key5}). 

Iskovskih's result uses the method of linear systems with base points,
as was first done over a non-algebraically closed field by B. Segre
(\cite{chap4-key14}). As for the methods of (\cite{chap4-key8}), they
involve a close analysis of principal homogeneous spaces under tori
over (smooth compactifications of) surfaces of type (1) -- and more
generally $k$-surfaces which become rational after a finite extension
of the ground field (\cite{chap4-key7}).

Over\pageoriginale the complex field $C$, examples of stably rational
non-rational threefolds are given by affine equations:
\setcounter{equation}{2}
\begin{equation}
y^{2}-a(t)z^{2}=P(x,t)\label{chap4-eq3}
\end{equation}
where $P(x,t)$ is an irreducible polynomial, of degree 3 in $x$, and
where $a(t)=\disc_{x}P$ has no square factor and is of degree at least
5.

That such threefolds are stably rational follows immediately from the
stable $k$-rationality of surfaces of type (1) (2) (take
$k=C(t)$). That \eqref{chap4-eq3} is not rational uses intermediate
jacobians (\cite{chap4-key6}) and Prym varieties, as was first done by
Mumford (\cite{chap4-key10}). However, the discriminant locus of a
conic bundle defined by \eqref{chap4-eq3} is a reducible singular
curve, and the delicate analysis of \cite{chap4-key2}
and \cite{chap4-key3} is required to show that \eqref{chap4-eq3} is
not rational.

\begin{thebibliography}{}
\bibitem{chap4-key1} E.\@ Ballico : Stable rationality for the variety
of vector bundles over an algebraic curve, {\em J. Lond. Math. Soc.,}
II. Ser., 30 (1984), 21-26.

\bibitem{chap4-key2} A. Beauville:\pageoriginale Prym varieties and
the Schottky problem. {\em Invent. Math.,} 41 (1977), 149-196.

\bibitem{chap4-key3} A. Beauville: Vari\'et\'es de Prym et jacobiennes
interm\'ediaires. {\em Ann. Sci. Ecole Norm. Sup.,} 10 (1977), 309-391.

\bibitem{chap4-key4} A. Beauville, J.-L. Colliot-Th\'el\`ene
J.-J. Sansuc, Sir P. Swinnerton-Dyer: Vari\'et\'es stablement
rationnelles non rationnnelles, {\em Ann. of Math.} 121 (1985) 283-318.

\bibitem{chap4-key5} F. Ch\^atelet: Points rationnels sur certaines
courbes et surfaces cubiques, {\em Enseign. Math.,} 5 (1959), 153-170.

\bibitem{chap4-key6} H. Clemens, P. Griffiths: The intermediate
jacobian of the cubic threefold, {\em Ann. of Math.,} 95 (1972), 281-356.

\bibitem{chap4-key7} J.-L. Colliot-Th\'el\`ene, J.-J. Sansuc: La
descente sur les vari\'et\'es rationnelles, in {\em Journ\'ees de
g\'eom\'etrie alg\'ebrique} (Angers 1979), A. Beauville ed.,
Sijthoff \& Noordhoff, Alphen aan den Rijn. (1980), 223-237.

\bibitem{chap4-key8} J.-L. Colliot-Th\'el\`ene, J.-J. Sansuc, Sir
P. Swinnerton-Dyer: Intersections de deux quadriques et surfaces de
Ch\^atelen g\'en\'eralis\'ees (preprint Orsay; 1986 {\em
C. R. Acad. Sc. Paris,} t. 298, S\'erie I, n$^{\circ}$ 16 (1984), 377-380).

\bibitem{chap4-key9} V. A. Iskovskih: Birational properties of a
surface of degree\pageoriginale 4 in $P^{4}_{k}$, {\em Mat. Sb.,} 88
(130) (1972), 31-37 (= Math. USSR Sb., 17 (1972), 30-36).

\bibitem{chap4-key10} D. Mumford: Prym varieties $I$, in {\em
Contributions to Analysis} (L. Ahlfors et al. ed.), Academic Press,
New York (1974), 325-350.

\bibitem{chap4-key11} M. Nagata: A theorem on valuation rings and its
applications, {\em Nagoya Math. J}, 29 (1967) 85-91.

\bibitem{chap4-key12} P. E. Newstead : Rationality of moduli spaces of
stable bundles, {\em Math. Ann.,} 215 (1975) 251-268; ibid., 249
(1980) 281-282.

\bibitem{chap4-key13} B. Segre: Sur un probl\`eme de M. Zariski, {\em
Colloque international d'alg\`ebre et de th\'eorie des nombers} (Paris
1949), C.N.R.S. Paris (1950) 135-138.

\bibitem{chap4-key14} B. Segre: On the rational solutions of
homogeneous cubic equations in four variables, {\em Math. Notae
Univ. Rosario Argentina} 11 (1951) 1-68.
\end{thebibliography}

\vskip 1cm
\noindent
C.N.R.S. Mathematiques,\\
B\^at. 425, Universit\'e de Paris-Sud,\\
F-91405 Orsay

