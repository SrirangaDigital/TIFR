\title{Conditions D'Existence des Fibres Stables de Rang Eleve Sur $\mathbb{P}_{2}$}\label{chap5}
\markright{Conditions D'Existence des Fibres Stables de Rang Eleve Sur $\mathbb{P}_{2}$}

\author{By J.M. Drezet and J. Le Potier}
\markboth{J.M. Drezet and J. Le Potier}{Conditions D'Existence des Fibres Stables de Rang Eleve Sur $\mathbb{P}_{2}$}

\date{}
\maketitle

\setcounter{page}{97}

\setcounter{pageoriginal}{132}
EN\pageoriginale DEHORS DU paragraphe 9, le travail qui suit est un
resume de notre article a paraitre aux Annales de
1'E.N.S \cite{chap5-key4}, auquel le lecteur devra se reporter pour
des d\'emonstrations compl\`etes. 

Let corps de base est $\mathbb{C}$.

\section{Introduction}\label{chap5-sec1}

\footnotetext[1]{Present\'e par J. Le Potier}
Soient $X$ use surface projective lisse, $A(X)$ 'anneau de Chow de
$X$. On sait qu'etant donnes un entier $r\geqq 2$, des classes
$c_{i}\in A^{i}(X)$ il existe des fibres vectoriels algebriques sur
$X$, de rang $r$, de classes de Chern $c_{1}$ et $c_{2}$ c'est un
resultat qui remonte a Schwarzenberger \cite{chap5-key13}. 

Par\pageoriginale contre, pour obtenir l'existence de fibr\'es stables
(relativement \`a une polarisation donn\'ee sur $X$) de rang $r$, de
classes de Chern $c_{1}$ et $c_{2}$ on doit imposer des conditions \`a
$r$, $c_{1}$ et $c_{2}$: on conna\^it par exemple le r\'esultat de
Bogomolov selon lequel on doit avoir n\'ecessairement
$$
c_{2}\geqq (r-1)c^{2}_{1}/2r.
$$

Cette condition n'est pas suffisante pour assurer l'existence de
fibres stables. Pour le plan projectif $\mathbb{P}_{2}$, les
conditions necessaires et suffisantes d'existence s'ecrivent en rang
$r=2$:
$$
c_{2}-c^{2}_{1}/4 \geqq 
\begin{cases}
2\text{~ si~ } c_{1}\text{~ est pair}\\
3/4\text{~ si~ } c_{1}\text{~ est impair}
\end{cases}
$$

Ce r\'esultat est d\^u lui aussi \`a
Schwarzenberger \cite{chap5-key4}; on se propose ici de l'\'etendre en
rang quelconque.

\section{Formulaire}\label{chap5-sec2}

Sur le plan projectif $\mathbb{P}_{2}$, on a
$A^{1}(\mathbb{P}_{2})=Z$, $A^{2}(\mathbb{P}_{2})=\mathbb{Z}$; si $E$
est un $\mathscr{O}_{\mathbb{P}_{2}}$ - module coherent, les classes
de chern de $E$ seront consider\'ees comme des nombres. Si $E$ est de
rang $>0$, on definit la pente $\mu$ et le discriminant $\Delta$ de
$E$ par les formules 
\begin{align*}
& \mu=c_{1}/r\\
& \Delta=(1/r)(c_{2}-(r-1)c^{2}_{1}/(2r)).
\end{align*}

Avec\pageoriginale ces conventions, la formule de Riemann-Roch
s'\'ecrit
\begin{align}
\chi(E) &=\sum(-1)^{i}h^{i}(E)\notag\\
&= r(P(\mu)-\Delta)\label{chap5-eq2.1}
\end{align}
o\`u $P$ est le polyn\^ome $P(X)=1+3x/2+x^{2}/2$.

Plus g\'en\'eralement, si $E$ et $E'$ sont deux
$\mathscr{O}_{\mathbb{P}_{2}}$-modules coh\'erents de rangs $r$ et
$r'$, de pentes $\mu$, de discriminants $\Delta$ et $\Delta'$
respectivement, on pose
$$
\chi(E,E')=\sum_{i}(-1)^{i}\dim \Ext^{i}(E,E')
$$
et la formule de Riemann-Roch s'\'etend sous la forme suivante:
\begin{equation}
\chi(E,E')=rr'(P(\mu'-\mu)-\Delta-\Delta')\label{chap5-eq2.2}
\end{equation}

En particulier, $\chi(E,E)=r^{2}(1-2\Delta)$.

\smallskip
\noindent
{\bf DUALITE DE SERRE:} Soient $E$ et $E'$ des faisceaux alg\'ebriques\break
coh\'erents sur $\mathbb{P}_{2}$; soit d\'autre part
$K\cong \mathscr{O}_{\mathbb{P}_{2}}(-3)$ le fibr\'e canonique sur
$\mathbb{P}_{2}$. Il existe un accouplement
$$
\Ext^{i}(E,E')\times \Ext^{2-i}(E',E\otimes K)\to \mathbb{C}
$$
qui\pageoriginale fait d'un de ces espaces vectoriels le dual de
l'autre.

\section{Faisceaux semi-stables}\label{chap5-sec3}

La notion de semi-stabilit\'e dont nous aurons besoin est celle de
Gieseker \cite{chap5-key7} et Maruyama \cite{chap5-key12}.

\begin{definition}\label{chap5-defi1}
Un faisceau algebrique coherent $E$ sur $\mathbb{P}_{2}$ est dit {\em
semi-stable} (resp. {\em stable}) si $E$ est sans torsion, et si pour
tout sous-module $0\neq E'\subset E$, on a 
$$
\mu(E')\leqq \mu(E)
$$
et en cas d'\'egalit\'e $\Delta(E')\geqq \Delta(E)$
(resp. $\Delta(E')>\Delta(E)$). 
\end{definition}

{\em Soit $S$} une variete algebrique. Considerons le foncteur
$\underline{M}(r,\mu,\Delta)$ qui \`a $S$ associe l'ensemble des
classes d'isomorphisme de faisceaux alg\'ebriques coh\'erents $E$ sur
$S\times \mathbb{P}_{2}$, $S$-plats et tels que pour tout $s\in S$, le
faisceau $E(s)$ induit sur $\mathbb{P}_{2}$ soit semi-stable de rang
$r$, de pente $\mu$, de discriminant
$\Delta$. Gieseker \cite{chap5-key7} et Maruyama \cite{chap5-key12}
ont construit une vari\'et\'e alg\'ebrique $M(r,\mu,\Delta)$ et un
morphisme fonctoriel
$$
\underline{M}(r,\mu,\Delta)\to \Mor(\quad , \ M(r,\mu,\Delta))
$$
qui fait de $M(r,\mu,\Delta)$ un espace de modules grossier,
c'est-\`a-dire qu'il satisfait \`a la propri\'et\'e suivante : pour
toute vari\'et\'e alg\'ebrique $M$, et tout morphisme fonctoriel
$$
M(r,\mu,\Delta)\to \Mor(\quad , \ M')
$$
il\pageoriginale existe un et un seul morphisme de vari\'et\'es
$M(r,\mu,\Delta)\to M'$ rendant le diagramme suivant commutatif
\[
\xymatrix@R=.5cm{
 & \Mor (\quad, \ M(r,\mu,\Delta))\ar[dd]\\
M(r,\mu,\Delta)\ar[ur]\ar[dr] &\\
 & \Mor(\quad, \ M')
}
\]
Ceci caract\'erise le vari\'et\'e alg\'ebrique $M(r,\mu,\Delta)$. Dans
$M(r,\mu,\Delta)$ les points qui proviennent de faisceaux stables
forment un ouvert lisse $M_{s}$; cet ouvert s'identifie \`a l'ensemble
des classes d'isomorphisme de faisceaux stables; il est de dimension
$$
\dim M_{s}=r^{2}(2\Delta-1)+1
$$

\noindent
{\bf FILTRATION DE HARDER-NARASIMHAN:}
Si $E$ est un faisceau alg\'ebrique coh\'erent sans torsion sur
$\mathbb{P}_{2}$, il a une filtration
$$
0\subset F_{1}\subset F_{2}\subset\ldots\subset F_{\ell}=E
$$
par des sous-modules coher\'ents tels que
\begin{itemize}
\item[(1)] le gradu\'e $gr_{i}=F_{i}/F_{i-1}$ soit semi-stable

\item[(2)] $\mu(gr_{i})\geqq \mu(gr_{i+1})$, et en cas d;\'egalit\'e
$\Delta(gr_{i})<\Delta(gr_{i+1})$. 
\end{itemize}
Une telle filtration est d\'etermin\'ee de mani\`ere unique par ces
conditions; on l'appelle la filtration de
Harder-Narasimhan \cite{chap5-key8}. 

\section{Fibr\'es exceptionnels}\label{chap5-sec4}

\begin{proposition}\label{chap5-prop1}
Soit $E$ un faisceau stable de rang $r$, de pente\pageoriginale $\mu$,
de discriminant $\Delta$ sur $\mathbb{P}_{2}$. Les assertions
suivantes sont equivalentes:
\begin{itemize}
\item[\rm(1)] $\Ext^{1}(E,E)=0$

\item[\rm(2)] $\Delta<1/2$

\item[\rm(3)] $\chi(E,E)>0$
\end{itemize}
\end{proposition}

En effet, par dualit\'e de Serre, $\Ext^{2}(E,E)$ est le dual de
$$
\Hom(E,E(-3))
$$ 
et par suite est nul par d\'efinition de la
stabilit\'e. D'autre part, on a $\Hom(E,E)=\mathbb{C}$; compte-tenu de
la formule de Riemann-Roch, on obtient
$$
\chi(E,E)=r^{2}(1-2\Delta)=1-\dim \Ext^{1}(E,E)
$$
ce qui donne l'\'equivalence voulue.

\begin{definition}\label{chap5-defi2}
Un faisceau alg\'ebrique coherent sur $\mathbb{P}_{2}$ est dit
exceptionnel (resp. {\em semi-exceptionnel}) s'il est stable
(resp. semi-stable) et si son discriminant $C$ satisfait \`a la
condition
$$
\Delta<1/2.
$$
\end{definition}

\begin{examples*}
Les fibres de rang un sont exceptionnels. Le fibre tangent
$T(\mathbb{P}_{2})$ est un fibre exceptionnel de pente $3/2$. Le fibre
noyau du morphisme canonique d'\'evaluation
$$
\text{ev~:~}\mathbb{P}_{2}\times \Gamma(\mathscr{O}(2))\to \mathscr{O}(2)
$$
est\pageoriginale un fibr\'e exceptionnel de rang 5, de pente -
$2/5$. Si $E$ est un fibr\'e exceptionnel, il en est de m\^eme du
dual, et des fibr\'es $E(i)$ pour $i\in \mathbb{Z}$.
\end{examples*}

En un point de l'espace de modules $M(r,\mu,\Delta)$ d\'efini par un
faisceau exceptionnel, on a $\dim M(r,\mu,\Delta)$ defini par un
faisceau exceptionnel, on a $\dim M(r,\mu,\Delta)=0$. Plus
pr\'ecis\'ement, on a le r\'esultat suivant:

\begin{proposition}\label{chap5-prop2}
Pour tout $\alpha\in Q$, il existe au plus, \`a isomorphisme pr\`es,
un faisceau exceptionnel de pente $\alpha$. Ce faisceau est en fait
localement libre; son rang est le plus petit d\'enominateur
$r_{\alpha}>0$ de $\alpha$, et son discriminant est donn\'e par
$$
\Delta=1/2(1-1/r^{2}_{\alpha})
$$
\end{proposition}

En effet, montrons d'abord que la pente $\alpha$ d\'etermine le rang
et le discriminant. On a d\'ej\`a vu que si $E$ est exceptionnel de
rang $r$, de pente $\alpha$ et de discriminant $\Delta$, on a
$$
\chi(E,E)=1=r^{2}(1-2\Delta);
$$
c'est-\`a-dire, en revenant aux classes de Chern $c_{1}$ {\em et}
$c_{2}$
$$
1=r^{2}-2rc_{2}+(r-1)c^{2}_{1}.
$$
Ceci entra\^ine que $r$ et $c_{1}$ sont premiers entr'eux; ceci
signifie que $r$ est le plus petit denominateur $>0$ de $\alpha$. Le
rang $r$ est ainsi\pageoriginale d\'etermin\'e; la m\^eme formule
donne pour discriminant $\Delta=1/2(1-1/r^{2})$. 

Montrons maintenant que l'ouvert $M_{S}\subset M(r,\mu,\Delta)$ est
r\'eduit \`a un point si $2\Delta <1$. La m\'ethode ci-dessous,
qui \'evite le recours au r\'esultat d'Ellingsrud \cite{chap5-key6}
nous a \'et\'e signal\'ee par Muka\"i. Soient $E$ et $E'$ deux
faisceaux exceptionnels de meme pente $\mu$, de meme rang $r$ et de
m\^eme discriminant $\Delta$; d'apr\`es la formula de Riemann-Roch,
$$
\chi(E,E')=r^{2}(1-2\Delta).
$$
On a encore $\Ext^{2}(E,E')=0$, car cet espace est le dual de
$$
\Hom(E',E(-3)),
$$ 
et donc est nul par stabilit\'e. Il en r\'esulte que
$\Hom(E,E')\neq 0$; un homomorphisme $f:E\to E'$ est en fait un
isomorphisme s'il est non nul, par stabilit\'e. Par suite, $E$ est
isomorphe \`a $E'$. 

Soit $E$ un faisceau exceptionnel de pente $\alpha$; pour $g\in \Aut
(\mathbb{P}_{2})$, le faisceau $g^{*}(E)$ est encore exceptionnel de
pente $\alpha$, et par suite isomorphe \`a $E$. Par suite, l'ensemble
des points de $\mathbb{P}_{2}$ au voisinage desquels le faisceau $E$
est localement libre est invariant par le groupe
$\Aut(\mathbb{P}_{2})$, et ne peut \^etre que $\mathbb{P}_{2}$
lui-meme.

\begin{proposition}\label{chap5-prop3}
Soit $E$ un faisceau semi-exceptionnel de pente $\alpha$. Alors $E$
est somme directe de fibr\'es exceptionnels de pente $\alpha$.
\end{proposition}

En effet, les fibr\'es semi-stables de pente $\alpha$ et de
discriminant $\Delta$\pageoriginale forment une categorie ab\'elienne,
artinienne et noeth\'erienne; par suite, un faisceau semi-stable de
pente $\alpha$, de discriminant $\Delta$ a une filtration de
Jordan-H\"older 
$$
0\subset F_{1}\subset F_{2}\subset\cdots\subset F_{\ell}=E
$$
avec $gr_{i}=F_{i}/F_{i-1}$ stable de pente $\alpha$, de discriminant
$\Delta$. Si $\Delta<1/2$, les faisceaux $gr_{i}$ sont exceptionnels
de meme pente $\alpha$, donc isomorphes; la filtration ci-desses est
en fait scind\'ee, d'o\`u la proposition. 

\section{L'ensemble \texorpdfstring{$\mathfrak{E}$}{E}}\label{chap5-sec5}

Soit $\alpha$ un nombre rationnel. On appelle rang de $\alpha$ le plus
petit entier $r_{\alpha}>0$ tel que $r_{\alpha}\alpha\in Z$, et
discriminant de $\alpha$ le nombre rationnel
$$
\Delta_{\alpha}=1/2(1-1/r^{2}_{\alpha}).
$$
Soit $\mathfrak{E}$ l'ensemble des nombres rationnels $\alpha$ qui
sont pentes de fibr\'es exceptionnels; si $\alpha\in \mathfrak{E}$, et
si $E_{\alpha}$ est un fibr\'e exceptionnel de pente $\alpha$,
$r_{\alpha}$ est le rang de $E_{\alpha}$ et $\Delta_{\alpha}$ son
discriminant. La proposition suivante nous permettra de donner une
construction de $\mathfrak{E}$. 

\begin{proposition}\label{chap5-prop4}
Soit $\mu\in Q$ un rationnel de rang $r$ et de discriminant
$\Delta$. Let assertions suivantes sont \'equivalents: 
\begin{itemize}
\item[\rm(1)] $\mu\in\mathfrak{E}$

\item[\rm(2)] $r(P(\mu)-\Delta)\in \mathbb{Z}$ et pour tout
$\alpha\in \mathfrak{E}$ tel que $0<|\alpha-\mu|<3$, on\pageoriginale a
$$
P(-|\alpha-\mu|)\leq \Delta_{\alpha}+\Delta
$$

\item[\rm(3)] $r(P(\mu)-\Delta)\in \mathbb{Z}$ et pour tout
$\alpha\in \mathfrak{E}$ tel que $r_{\alpha}<r$ et $|\alpha-\mu|\leq
1$, on a
$$
P(-1|\alpha-\mu|)\leqq \Delta_{\alpha}+\Delta
$$
\end{itemize}
\end{proposition}

En effet, si $\mu\in \mathfrak{E}$, c'est la pente d'un fibr'e
exceptionnel $E_{\mu}$, et $r(P(\mu)-\Delta)$ doit \^etre la
caract\'eristique d'Euler-Poincar\'e, donc un entier. L'implication
(1) $\Rightarrow$ (2) repose sur la formule de Riemann-Roch (2.2) :
d\'esignons par $E_{\alpha}$ un fibr\'e exceptionnel de pente
$\alpha$; si $\alpha>\mu$, on a par stabilit\'e
$$
\Hom(E_{\alpha},E_{\mu})=0
$$
D'autre part, $\Ext^{2}(E_{\alpha},E_{\mu})$, dual de
$\Hom(E_{\mu},E_{\alpha}(-3))$, est nul par stabilit\'e d\`es que
$\alpha-\mu<3$ par suite $\chi(E_{\alpha},E_{\mu})\leq 0$. Si
$\alpha<\mu$, on voit de m\^eme que $\chi(E_{\mu},E_{\alpha})\leqq 0$
pourvu que $\mu-\alpha<3$. 

L'implication (2) $\Rightarrow$ (3) est triviale; l'implication (3)
$\Rightarrow$ (1) r\'esulte du th\'eor\`eme d'existence 3 (cf.~\S\ \ref{chap5-sec7}).

\smallskip
\noindent
{\bf CONSTRUCTION DE $\mathfrak{E}$~:}
Soient $\alpha$ et $\beta$ deux nombres rationnels Si
$3+\alpha-\beta\neq 0$, l'equation en $t$
$$
P(\alpha-t)-\Delta_{\alpha}=P(t-\beta)-\Delta_{\beta}
$$
a une solution unique, not\'ee $t=\alpha\cdot \beta$, donn\'ee par
$$
\alpha\cdot\beta
=\dfrac{\alpha+\beta}{2}+\dfrac{\Delta_{\beta}-\Delta_{\alpha}}{3+\alpha-\beta} 
$$\pageoriginale

Soit $\mathfrak{D}$ l'ensemble des nombres rationnels de la forme
$p/2^{q}$ o\`u $p\in \mathbb{Z}$ et o\`u $q$ est un entier $\geq
0$. Soit $\epsilon:\mathfrak{D}\to \mathbb{Q}$ l'application
construite par r\'ecurrence sur $q$ d\'efinie par
\begin{align*}
& \epsilon(n)=n\text{~~ si~~ } n\in \mathbb{Z}\\
& \epsilon(2p+1/2^{q+1})=\epsilon(p/2^{q})\cdot \epsilon((p+1)/2^{q}).
\end{align*}

\begin{proposition}\label{chap5-prop5}
\begin{itemize}
\item[\rm(1)] L'application $\epsilon:\mathfrak{D}\to \mathbb{Q}$ est
bien d\'efinie et strictement croissante.

\item[\rm(2)] Le nombre $\epsilon(p/2^{q})$ est de rang $\geqq 2^{q}$
si $p$ est impair.

\item[\rm(3)] Pour tout $\alpha\in \epsilon(\mathfrak{D})$, or a
$r_{\alpha}(P(\alpha)\cdot \Delta_{\alpha})\in \mathbb{Z}$

\item[\rm(4)] Pour $\rho\in \mathfrak{D}$, et $n\in \mathbb{Z}$,
$\epsilon(\rho+n)=\epsilon(\rho)+n$, et
$\epsilon(-\rho)=-\epsilon(\rho)$. 
\end{itemize}
\end{proposition}

Il en r\'esulte en particulier que la fonction
$\epsilon:\mathfrak{D}\to \mathbb{Q}$ est parfaitement d\'etermin\'ee
quand on conna\^it sa restriction \`a $\mathfrak{D}\cap [0,1/2]$. Le
calcul effectif donne par exemple:
\begin{center}
\begin{tabular}{l|ccccccccc}
$p$ & 0 & 1 & 2 & 3 & 4 & 5 & 6 & 7 & 8\\
\hline
&&\\[-10pt]
$\epsilon(p/2^{4})$ & 0 & $\dfrac{13}{34}$ & $\dfrac{5}{13}$ &
$\dfrac{75}{194}$ & $\dfrac{2}{5}$ & $\dfrac{179}{433}$ &
$\dfrac{12}{29}$ & $\dfrac{70}{169}$ & $\dfrac{1}{2}$ 
\end{tabular}
\end{center}

\medskip

\begin{theorem}\label{chap5-thm1}
On $a\in (\mathfrak{D})=\mathfrak{E}$. 
\end{theorem}

Ceci\pageoriginale d\'etermine donc, \`a isomorphisme pr\`es, tous les
fibr\'es exceptionnels. Indiquons le plan de la d\'emonstration: soit
$\mathfrak{D}_{q}$ l'ensemble des nombres de la forme $p/2^{q}$, o\`u
$p\in \mathbb{Z}$, et o\`u $q$, est un entier tel que $0\leq q'\leq
q$. Pour montrer que $\epsilon(\mathfrak{D})\subset \mathfrak{E}$, on
verifie que $\epsilon(\mathfrak{D}_{q})\subset \mathfrak{E}$ par
r\'ecurrence sur $q$. Il suffit de demontrer le lemme suivant, qui
repose sur la proposition \ref{chap5-prop4}: 

\begin{lemma}\label{chap5-lem1}
Soient $\alpha=\epsilon(p/2^{q})$,
$\beta=\gamma \epsilon((p+1)/2^{q})$ et $\mu=\alpha\cdot \beta$. On
suppose que $\alpha\in \mathfrak{E}$, et
$\beta\in \mathfrak{E}$. Alors
\begin{itemize}
\item[\rm(1)] Pour tout $\alpha'\in \mathfrak{E}$, tel que
$\alpha<\alpha'<\beta$, on a $r_{\alpha'}\geqq r_{\mu}$

\item[\rm(2)] On a $\mu\in \mathfrak{E}$.
\end{itemize}
\end{lemma}

Inversement, soit $\mu\in \mathfrak{E}$; choisissons $q$ assez grand
pour que $2^{q}\geqq r_{\mu}$, et $p$ de sorte que
$$
\alpha=\epsilon(p/2^{q})\leqq \mu <\epsilon(p+1)/2^{q}
$$

Comme $\alpha\cdot \beta$ est de rang $\geqq 2^{q+1}$ d'apr\`es la
proposition \ref{chap5-prop5}, on voit que la partie (1) du lemma
ci-dessus impose $\mu=\epsilon(p/2^{q})$. Par suite
$\mu\in \epsilon(\mathfrak{D})$. 

\section{Le th\'eor\`eme d'existence}\label{chap5-sec6}

L'\'etude des faisceaux semi-stables de discriminant $\Delta<1/2$
ayant deja \'et\'e faite, on se limite dans ce paragraphe au cas
$\Delta<1/2$. Remarquons d'autre part que pour $R$ r\'eel donne, 
l'ensemble\pageoriginale $\mathfrak{E}(R)$ des \'el\'ements de
$\mathfrak{E}$ de rang inf\'erieur ou egal \`a $R$ est localement
fini.

\begin{theorem}\label{chap5-thm2}
Soient $r$ un entier $\geqq 2$, $c_{1}$ et $c_{2}\in \mathbb{Z}$; on
pose $\mu=c_{1}/r$, $\Delta=(1/r)(c_{2}-(r-1)c^{2}_{1}/(2r))$, et on
suppose $\Delta\geqq 1/2$. 

On designe par: $\alpha(r,\mu)$ le plus grand des \'el\'ements
$\alpha\in \mathfrak{E}$ tels que $r_{\alpha}\leqq r/2$ et
$\alpha\leqq \mu$, et par $\beta(r,\mu)$ le plus petit
des \'el\'ements $\beta\in \mathfrak{E}$ tels que $r_{\beta}\leqq r/2$
et $\mu\leqq \beta$.

Let assertions suivantes sont \'equivalentes:
\begin{itemize}
\item[\rm(1)] Il existe un fibr\'e stable de rang $r$, de pente $\mu$,
de discriminant $\Delta$.

\item[\rm(2)] Il existe un faisceau semi-stable de rang $r$, de pente
$\mu$, de discriminant $\Delta$.

\item[\rm(3)] Pour tout $\alpha\in \mathfrak{E}$, tel que
$\alpha-\mu<3$, on a
$$
P(-\alpha-\mu|)\leqq \Delta_{\alpha}+\Delta
$$

\item[\rm(4)] $\Delta\geqq
P(\alpha(r,\mu)-\mu)-\Delta_{\alpha(r,\mu)}$ et $\Delta\geqq
P(\mu-\beta(r,\mu))-\Delta_{\beta(r,\mu)}$ 
\end{itemize}
\end{theorem}
En effet les implications (1) $\Rightarrow$ (2) et (3) $\Rightarrow$
(4) sont triviales. La d\'emonstration de l'implication (2)
$\Rightarrow$ (3) est tout \`a fait semblable \`a celle qui a \'et\'e
vue dans la proposition \ref{chap5-prop4} et repose sur le
th\'eor\`eme de dualit\'e de Serre et la formula de Riemann-Roch
(2.2). Reste \`a demontrer que (4) $\Rightarrow$ (1); pour ceci, on
commence\pageoriginale par v\'erifier de mani\`ere purement
arithm\'etique que l'assertion (4) entra\^ine l'assertion suivante:
\begin{itemize}
\item[(5)] $\Delta\neq 1/2$, {\em et pour tout
$\alpha\in \mathfrak{E}$ tel que $r_{\alpha}<r$ et $\alpha-\mu|\leq 1$}
$$
P(-1|\alpha-\mu|)\leqq \Delta_{\alpha}+\Delta
$$
\end{itemize}

L'implication (5) $\Rightarrow$ (1) r\'esultera du th\'eor\`eme \ref{chap5-thm3}
(cf. \S\ \ref{chap5-sec7}). 

Let th\'eor\`eme \ref{chap5-thm2} permet effectivement de d\'ecrire
quelles sont exactement les classes de Chern des fibr\'es
stables. Supposons par exemple $r=20$, $c_{1}=9$ et donc $\mu=0,45$;
le tableau du paragraphe 5 montre que $\alpha(r,\mu)=0,4$ et
$\beta(r,\mu)=0,5$. Par suite, les conditions (4) s'\'ecrivent 
$$
\Delta\geqq P(-0,05)-12/25\quad\text{et}\quad \Delta\geqq P(-0,05)-3/5
$$
et sont donc \'equivalentes a $\Delta\geqq 0,55125$, c'est-\`a-dire
$c_{2}\geqq 50$.

\section{La methode}\label{chap5-sec7}

Pour compl\'eter les d\'emonstrations de la
proposition \ref{chap5-prop4} (et donc du\break
th\'eor\`eme \ref{chap5-thm1}) et du th\'eor\`eme \ref{chap5-thm2}, il
suffit de v\'erifier l'\'enonc\'e suivant: 

\begin{theorem}\label{chap5-thm3}
Soient $r$ un entier $>1$, et deux rationnels tels que
$r\; \mu\in \mathbb{Z}$, $r(P(\mu)-\Delta)\in \mathbb{Z}$, $\Delta\neq
1/2$. On {\em suppose satisfaites les conditions suivantes:} 
\begin{itemize}
\item[(S)] Pour tout $\alpha\in \mathfrak{E}$ tel que $r_{\alpha}<r$
et $|\alpha-\mu|\leqq 1$, on a
$P(-|\alpha-\mu|)\leqq \Delta_{\alpha}+\Delta$ 

Alors\pageoriginale il existe un fibr\'e stable de rang $r$, de pente
$\mu$, de discriminant $\Delta$ sur $\mathbb{P}_{2}$. 
\end{itemize}
\end{theorem}

\subsection{Construction d'une grande famille de fibres vectoriels.}\label{chap5-sec7.1}

Soit $d\subset \mathbb{P}_{2}$ une droite fix\'ee. On commence par
construire une famille de fibr\'es vectoriels de rang $r$, de pente
$\mu$, de discriminant $\Delta$, param\'etr\'ee par une vari\'et\'e
alg\'ebrique lisse $S$:  
$$
E\to S\times \mathbb{P}_{2}
$$
satisfaisant aux conditions suivantes:

\begin{itemize}
\item[$(L)$] Pour tout $s\in S$, $\Ext^{2}(E(s),E(s))=0$

\item[$(KS)$] Le morphisme de d\'eformation infinit\'esimale de
Kodaira et Spen\-cer
$$
T_{s}S\to \Ext^{1}(E(s),E(s))
$$
est suriectif

\item[$(R)$] Pour tout $s\in S$, $E(s)|_{d}$ est rigide, c'est-\`a-dire
$$
\Ext^{1}_{d}(E(s)|_{d},E(s)|_{d})=0
$$
\end{itemize}

La construction s'inspire de \cite{chap5-key11}: consid\'erons le
polyn\^ome de\break Hilbert $H(m)=r(P(\mu+m)-\Delta)$; pour $m+\mu\geqq
-3/2$, la suite $m\to H(m)$ est croissante; elle est n\'egative pour
$-2\leqq m+\mu < 0$: on le v\'erifie pour $1\leqq m+\mu\leqq 0$ en
remarquant que $-m\in \mathfrak{E}$, et donc d'apr\`es la condition
$(S)$ $P(m+\mu)\leqq \Delta$ et d'autre part,
$P(m+\mu-1)=P(m+\mu)-(m+\mu+1)\leqq \Delta$.

Par\pageoriginale suite, il existe un entier $m_{0}$ tel que $N_{o}=H(m_{o})>0$,
$N_{1}=-H(m_{o}-1)\geqq 0$, $N_{2}=-H(m_{o}-2)\geqq 0$. Soit $Q$ le
fibr\'e canonique quotient de rang 2 sur $\mathbb{P}_{2}$; les
morphismes injectifs de fibr\'es vectoriels
$$
(\mathscr{O}(-1))^{N_{2}}\to (Q^{*})^{N_{1}}\oplus \mathscr{O}^{N_{o}}
$$
forment un ouvert non vide $\Omega$ de l'espace vectoriel de tous les
morphismes car $2N_{1}+N_{0}-N_{2}=r\geqq 2$. Pour $s\in \Omega$, le
conoyau de $s$ d\'efinit un fibr\'e vectoriel $F(s)$ de rang $r$, de
pente $\mu+m_{o}$, de discriminant $\Delta$. La famille de fibr\'es
vectoriels de rang $r$, de pente $\mu$, de discriminant $\Delta$
$$
E(s)=F(s)(-m_{o})
$$
satisfait aux conditions $(L)$ et $(KS)$. En fait, on a m\^eme la
condition plus forte que $(L)$
\begin{itemize}
\item[$(L')$] Pour tout $s\in \Omega$, $\Ext^{2}(E(s),E(s)(-1))=0$
\end{itemize}
qui permet de v\'erifier que l'ouvert $S\subset \Omega$ des points
$s\in S$ tels que le fibr\'e $E(s)_{d}$ soit rigide est non vide.

\subsection{Stratification de Shatz}\label{chap5-sec7.2}

On consid\`ere une famille $E\to S\times \mathbb{P}_{2}$ de fibr\'es
vectoriels sur\pageoriginale $\mathbb{P}_{2}$ para\-m\'etr\'ee par une
vari\'et\'e lisse $S$, et satisfaisant aux conditions $(L)$ et
$(KS)$. Soient $(H_{1},\ldots,H_{\ell})$ des polyn\^omes \`a
coefficients rationnels. On dit que $s\in S$ est de poids
$(H_{1},\ldots,H_{\ell})$ si la filtration de Harder-Narasimhan de
$E(s)$ 
$$
0\subset F_{1}\subset F_{2}\subset\cdots\subset F_{\ell}=E(s)
$$
est de longueur $\ell$, et si le gradu\'e associ\'e $gr_{i}(E(s))$ a
pour polyn\^ome de Hilbert $H_{i}$. Le rang $r_{i}$, la pente
$\mu_{i}$, le discriminant $\Delta_{i}$ de $gr_{i}(E(s))$ sont
d\'etermin\'es \`a partir de $H_{i}$ par la formule 
$$
H_{i}(m)=r_{i}(P(\mu_{i}+m)-\Delta_{i}).
$$

L'\'enonc\'e suivant \'etend \`a $\mathbb{P}_{2}$ les r\'esultats
obtenus par Atiyah et Bott dans le cadre des surfaces de
Riemann \cite{chap5-key1}: 

\begin{proposition}\label{chap5-prop6}
Sour les conditions $(L)$ et $(KS)$, {\em l'ensemble}
$$
Y(H_{1},\ldots,H_{\ell})
$$ 
des points $s\in S$ de poids
$(H_{1},\ldots,H_{\ell})$ est une sous-vari\'et\'e localement ferm\'ee
lisse de codimension
$$
\sum_{i<j}r_{i}r_{j}(\Delta_{i}+\Delta_{j}-P(\mu_{j}-\mu_{i}))
$$
Let sous-vari\'et\'es ainsi definies sont en nombre fini.
\end{proposition}

\subsection{Existence de fibr\'es semi-stables}

L'\'enonc\'e ci-dessus s'applique en particulier \`a la famille
construite au paragraphe \ref{chap5-sec7.1}. Pour montrer l'existence
de fibr\'es semi-stables\pageoriginale de rang $r$, de pente $\mu$, de
discriminant $\Delta$, il suffit de montrer que si $\ell>1$, les
sous-vari\'et\'es ci-dessus sont de codimension $>0$. Compte-tenu de
la formule de Riemann-Roch, ceci r\'esulte du lemme suivant. 

\begin{lemma}\label{chap5-lem2}
Soit $E$ un fibre vectoriel sur $\mathbb{P}_{2}$, de rang $r$, de
pente $\mu$ et de discriminant $\Delta$ satisfaisant \`a la condition
$(S)$. Soit
$$
0\subset F_{1}\subset F_{2}\subset\cdots\subset F_{\ell}=E
$$
la filtration de Harder-Narasimhan de $E$. de gradu\'e $gr_{i}(E)$. On
suppose en outre qu'il existe une droite $d\subset \mathbb{P}_{2}$
telle que $E|_{d}$ soit rigide. Alors si $\ell>1$
$$
\sum\limits_{i<j}\chi(gr_{i}(E),gr_{j}(E))<0
$$
\end{lemma}

En effet, soient $r_{i}$ le rang, $\mu_{i}$ la pente, $\Delta_{i}$ le
discriminant de $gr_{i}(E)$ l'existence d'une droite $d$ telle que
$E|_{d}$ soit rigide entra\^ine
$$
0\leqq \mu_{1}-\mu_{\ell}\leqq 1
$$
En particulier, pour $i<j$, on a $0\leqq \mu_{i}-\mu_{j}<3$ de la
semi-stabilit\'e de $gr_{i}(E)$ il d\'ecoule
$$
\chi(gr_{i}(E),gr_{j}(E))\leqq 0
$$

Supposons que pour tout $(i,j)$ tel que $i<j$,
$\chi(gr_{i}(E),gr_{j}(E))=0$, c'est-\`a-dire, d'apres la formule de
Riemann-Roch 
$$
P(\mu_{j}-\mu_{i})=\Delta_{i}+\Delta_{j}.
$$\pageoriginale
En particulier, si $\mu_{1}-\mu_{\ell}<0$, ceci entra\^ine
$\Delta_{1}+\Delta_{\ell}<1$, donc soit $\Delta_{1}<1/2$ soit
$\Delta_{\ell}<1/2$. Si $\mu_{1}=\mu_{\ell}$,
$\Delta_{1}<\Delta_{\ell}$ par d\'efinition de la filtration de
Harder-Narasimhan; par suite $\Delta_{1}<1/2$. Dans les deux cas, l'un
des faisceaux $gr_{1}(E)$, $gr_{\ell}(E)$ est
semi-exceptionnel. Supposons par exemple $gr_{1}(E)$
semi-exceptionnel; alors
\begin{align*}
\chi(gr_{1}(E),(E)
&= \chi(gr_{1}(E))+\sum_{i>1}\chi(gr_{1}(E),gr_{i}(E))\\[3pt]
&= \chi(gr_{1}(E),gr_{1}(E))
\end{align*}
D'apr\`es la proposition \ref{chap5-prop3} qui sonne la structure des
faisceaux semi-excep\-tionnels, cette quantit\'e est positive. On a
alors 
$$
P(\mu-\mu_{1})>\Delta+\Delta_{1}
$$
et $|\mu-\mu_{1}|\leqq 1$. Ceci contradit la condition $(S)$. Dans le
cau o\`u c'est $gr_{\ell}(E)$ qui est semi-exceptionnel, on
obtient \`a nouveau une contradiction en \'etudiant
$\chi(E,gr_{\ell}(E))$. 

\subsection{Construction de fibr\'es stables}\label{chap5-sec7.4}

Consid\'erons, dans la famille ci-dessus, l'ouvert
$\mathfrak{M}\subset S$ correspondant aux fibr\'es semi-stables. On
vient de voir que $\mathfrak{M}$ n'est pas vide.

\begin{lemma}\label{chap5-lem3}
Let points de $\mathfrak{M}$ correspondant aux fibr\'es stables
forment un ouvert $\mathfrak{M}_{s}$ partout dense dans
$\mathfrak{M}$. 
\end{lemma}

En\pageoriginale effet, si $\Delta<1/2$, pour $t\in \mathfrak{M}$,
chaque fibr\'e 
$E(t)$ est semi-exceptionnel; la condition $(S)$ impose en fait que
$E(t)$ soit exceptionnel d'apr\`es la proposition \ref{chap5-prop3}.

Si $\Delta>1/2$, soit $r_{i}$ une suite d'entiers tels que $\sum
r_{i}=r$, $r_{i}>0$. Let points $t\in \mathfrak{M}$ tels que $E(t)$
ait une filtration (dite de Jordan-H\"older)
$$
0\subset F_{1}\subset F_{2}\subset \cdots\subset F_{\ell}=E(t)
$$
dont le gradu\'e $gr_{i}$ soit stable de rang $r_{i}$, de pente $\mu$
et de discriminant $\Delta$, est un ferm\'e $Y(r_{1},\ldots,r_{\ell})$
dont on peut minorer la codimension, de mani\`ere tout \`a fait
semblable \`a ce qui a \'et\'e vu dans la
proposition \ref{chap5-prop6}, bien que ces ferm\'es n'aient ici
aucune raison d'\^etre lisses:
$$
\cosim Y (r_{1},\ldots,r_{\ell})\geq \sum r_{i}r_{j}(2-1)
$$
Par suite, $Y=\cup Y(r_{1},\ldots,r_{\ell})$ est un ferm\'e de
codimension $>0$; le compl\'ementaire de ce ferm\'e est exactement
l'ensemble des points $t\in \mathfrak{M}$ tels que $E(t)$ soit
stable. Cet ouvert est donc partout dense.

\section{Irreductibilit\'e}\label{chap5-sec8}

Un faisceau alg\'ebrique coh\'erent $E$ sur $\mathbb{P}_{2}$ est dit
$\mu$-stable s'il est sans torsion et si pour tout sous-module
$F\subset E$ de rang $r(F)<r(E)$, $F\neq 0$ on a 
$$
\mu(F)<\mu(E).
$$\pageoriginale

Let points de $M(r,\mu,\Delta)$ qui proviennent de faisceaux
localement libres $\mu$-stables forment un ouvert $M^{o}_{\mu s}$; par
une m\'ethode semblable a celle qui vient d'\^etre d\'ecrite au
paragraphe 7, on d\'emontre que cet ouvert est partout
dense. Compte-tenu du r\'esultat d'Ellingsrud donnant d\'ej\`a
l'irr\'eductibilit\'e de $M^{o}$ \cite{chap5-key6}\footnote[1]{Pour
$r=2$, l'irr\'eductibilit\'e de $M^{o}_{\mu s}$ est due \`a
Barth \cite{chap5-key2} et Hulek \cite{chap5-key9} Pour $\mu=0$, elle
est due a Hulek \cite{chap5-key10}. L'irr\'eductibilit\'e de
$M(2,\mu',\Delta)$ est connue de Maruyama \cite{chap5-key12}.}, on
obtient: 

\begin{theorem}\label{chap5-thm4}
L'espace de modules $M(r,\mu,\Delta)$ est irr\'eductible. 
\end{theorem}

\section{Groupe de Picard}\label{chap5-sec9}

Soient $r$ un entier $>1$, $\mu$ et $\Delta$ deux rationnels tels que
$c_{1}=r\mu$ et $\chi=r(P(\mu)-\Delta)$ soient entiers, et
$\Delta>1/2$. 

Dans le cas ou $r$, $c_{1}$ et $\chi$ sont premiers entr'eux, l'ouvert
$M_{s}$ de l'espace de modules $M(r,\mu,\Delta)$ est egal a
$M(r,\mu,\Delta)$: la vari\'et\'e $M(r,\mu,\Delta)$ est alors une
vari\'et\'e projective lisse de dimension $r^{2}(2\Delta-1)+1$; c'est
un espace de modules fin pour le foncteur quotient 
$$
S\to \underline{M}(r,\mu,\Delta)(S)/\Pic S
$$
o\`u $\Pic (S)$ op\`ere sur $\underline{M}(r,\mu,\Delta)(S)$
(cf. \S\ \ref{chap5-sec3}) par la formule
$$
(L,E)\to E\otimes pr_{1} \ {}^{*}(L) 
$$
ou\pageoriginale $L\in \Pic(S)$, et
$E\in \underline{M}(r,\mu,\Delta)(S)$. 

Consid\'erons l'ensemble $\mathfrak{E}(r/2)$ des \'el\'ements
$\alpha\in \mathfrak{E}$ tels que $r_{\alpha}\leq r/2$, et posons
\begin{gather*}
\delta(r,\mu)=\Sup P(-|\alpha-\mu|)-\Delta_{\alpha}\\
\alpha\in \mathfrak{E}(r/2)\\
|\alpha-\mu_{1}\leqq 1
\end{gather*}
Avec les notations du th\'eor\`eme \ref{chap5-thm2}, cette borne
superieure est en fait atteinte soit pour $\alpha=\alpha(r,\mu)$, soit
pour $\alpha=\beta(r,\mu)$. L'\'enonc\'e du
th\'eor\`eme \ref{chap5-thm2} peut en fait se lire:

{\em L'espace de modules $M(r,\mu,\Delta)$ n'est pas vide si et
seulement si}
$$
\Delta\geqq \delta(r,\mu)
$$

L'\'enonc\'e suivant est du au premier auteur; il donne, quand $r$,
$c_{1}$ et $\chi$ sont premiers entr'eux, le groupe de Picard de la
vari\'et\'e $M(r,\mu,\Delta):$ 

\begin{theorem}\label{chap5-thm5}
\cite{chap5-key3} On suppose que $\Delta>1/2$, et que $r$,
$c_{1}=r\mu$ et $\chi=r(P(\mu)-\Delta)$ sont des entiers premiers
entr'eux. Alors le groupe de Picard de la vari\'et\'e
$M(r,\mu,\Delta)$ est donn\'e par
$$
\Pic M(r,\mu,\Delta)=
\begin{cases}
\mathbb{Z} & \text{si~~ } \Delta=\delta(r,\mu)\\[4pt]
\mathbb{Z}^{2} & \text{si~~ }\Delta>\delta(r,\mu)
\end{cases}
$$
\end{theorem}

De\pageoriginale plus, en utilisant le fibr\'e universel, on peut
donner une base pour le groupe de Picard.

Consid\'erons par exemple le cas $r=4$, $\mu=1/2$, $\Delta=5/8$: alors
$c_{1}=2$, $\chi=5$, et par suite, l'espace de modules
$M(r,\mu,\Delta)$ correspondant est une vari\'et\'e projective lisse
de dimension 5; comme $\delta(4.1/2)=5/8$, son groupe de Picard est
$\mathbb{Z}$. On peut en fait v\'erifier que cet espace de modules est
isomorphe \`a $\mathbb{P}_{5}$. 

Signalons que le cas $r=1$ a \'ete \'etudi\'e par $G$. Elencwajg et
$P$. Le Barz \cite{chap5-key5}; le cas $r=2$ a \'egalement \'et\'e
trait\'e r\'ecemment par S.A. Str\o mme \cite{chap5-key16}.

Si $r$, $c_{1}$ et $\chi$ ne sont plus premiers entr\'eux la variete
$M(r,\mu,\Delta)$ est encore projective et normale, mais il appara\^it
en g\'en\'eral des singularit\'es dans $M(r,\mu,\Delta)$. Il est
encore possible de d\'ecrire le groupe de Picard de l'ouvert de
lissit\'e $M_{\text{reg}}(r,\mu,\Delta)$ de $M(r,\mu,\Delta)$ et le
r\'esultat est semblable au pr\'ec\'edent \cite{chap5-key3}: 
$$
\Pic M_{\reg}(r,\mu,\Delta)=
\begin{cases}
\mathbb{Z} & \text{si~~ } \Delta=\delta(r,\mu)\\[4pt]
\mathbb{Z}^{2} & \text{si~~ }\Delta>\delta(r,\mu).
\end{cases}
$$
Sauf pour quelques exceptions qu'il est possible d'\'enum\'erer, cet
ouvert de lissit\'e co\"incide en fait avec l'ouvert $M_{s}$
correspondant aux faisceaux stables. 

\begin{thebibliography}{}\pageoriginale 
\bibitem{chap5-key1} M.F. Atiyahet R. Bott. The
Yang-Mills equations over Riemann Surfaces, {\em
Phil. Trans. Roy. Soc. London,} A 308 (1982)

\bibitem{chap5-key2} W. Barth. Moduli of vector bundles on the
projective plane, {\em Invent. Math.,} 42 (1977), p. 63-91. 

\bibitem{chap5-key3} J.M. Drezet. Groupes de Picard des varietes de
modules de faisceaux stables sur $\mathbb{P}_{2}$, en preparation.

\bibitem{chap5-key4} J.M. Drezet et J. Le potier. Fibres stables et
fibres exceptionnels sur $\mathbb{P}_{2}$, a paraitre aux
Ann. Sc. Ec. Norm. Sup.

\bibitem{chap5-key5} G. Elencwajg et P. Le Barz. Picard groups of
polygons, Preprint, Nice (1983).

\bibitem{chap5-key6} G. Ellingsrud. Sur l'irr\'eductibilit\'e du
module des fibr\'es stables sur $\mathbb{P}_{2}$, Preprint, Nice (1982).

\bibitem{chap5-key7} D. Gieseker. On the moduli of vector bundles on
an algebraic surface, {\em Ann. of Math.,} 106 (1977), p. 45-60. 

\bibitem{chap5-key8} G. Harder\pageoriginale et M.S. Narasimhan. On
the cohomology groups of moduli spaces of vector bundles on curves,
{\em Math. Ann.} 212 (1975), p. 215-248.

\bibitem{chap5-key9} K. Hulek. Stable rank-2 vector bundles on
$\mathbb{P}_{2}$ with $c_{1}$ odd, {\em Math. Ann.} 242 (1979), p. 241-266.

\bibitem{chap5-key10} K. Hulek. On the classification of stable rank-r
vector bundles over the projective plane, {\em Vector Bundles and
Differential Equations,} Proceedings, Nice 1979, Progress in Math., 7
(1980), p. 113-144.

\bibitem{chap5-key11} J. Le Potier. Stabilit\'e et amplitude sur
$\mathbb{P}_{2}(\mathbb{C})$, {\em Vector Bundles and Differential
Equations}, Proceedings, Nice 1979, Progress in Math., 7 (1980),
p. 146-181. 

\bibitem{chap5-key12} M. Maruyama. Moduli of stable sheaves II,
Journal {\em of Mathematics of Kyoto University} 18(3) (1978),
p.~557-614. 

\bibitem{chap5-key13} R.L.E. Schwarzenberger. Vector bundles on
algebraic surfaces {\em Proc. London Math. Soc.,} (3) 11(1961),
p.~601-622. 

\bibitem{chap5-key14} R.L.E. Schwarzenberger. Vector bundles on the
projective plane, {\em Proc. London Math. Soc.,} (3) (1961),
p.~623-640. 

\bibitem{chap5-key15} S. Shatz.\pageoriginale The decomposition and
specialization of algebraic families of vector bundles, {\em
Compos. Math.,} 35 (2) (1977), p.~163-187.

\bibitem{chap5-key16} S.A. Str\o mme. {\em Ample divisors on fine
moduli spaces on the projective plane,} Preprint, Bergen (Norvege) 1984.
\end{thebibliography}

\vskip .5cm

\noindent
Universit\'e Paris 7\\
U.E.R. de Math\'ematiques et L.A. 212\\
Aile 45-55\\
2, Place Jussieu\\
75251 Paris Cedex 05
