\title{The Witt Group Of A Real Surface}
\markright{The Witt Group Of A Real Surface}

\author{By M. Ojanguren and G. Ayoub}
\markboth{M. Ojanguren and G. Ayoub}{The Witt Group Of A Real Surface}

\date{}
\maketitle

\setcounter{page}{337}
\setcounter{pageoriginal}{444}
THIS\pageoriginale IS A short report on some of the results obtained
by the first named author in his thesis(Universit\'{e} De Lausanne, 1984). Full details will appear elsewhere.

\section{The Witt group of a scheme}\label{s1}

Let $X$ be a noetherian scheme with structure sheaf $O_{\mathcal{X}}$. We assume that $2$ is invertible in $O_{\mathcal{X}}$. A \textit{vector bundle} over $X$ is, by definition, a locally free $O_{\mathcal{X}}$-module of finite rank. A \textit{quadratic space over} $X$ is a pair $(F,f)$ consisting of a vector bundle $F$ and an isomorphism $f:F\to \displaystyle\mathop{F}^{\vee}$ of $F$ into its dual such that $\displaystyle\mathop{f}^{\vee}=f$. Notice that we identify $F$ with its double dual through the canonical isomorphism. If $X= \Spec A$ is an affine scheme, $F$ can be identified with the projective $A$-module $F$ of its global sections  and $f$ defines on $F$ a symmetric bilinear scalar product $<x,y>=f(x)(y)$. In particular, if $F$ is free with basis  $e_1\ldots,e_r$, $f$ is defined by the invertible symmetric matrix $(<e_i,e_i>)$. Hence, if $A$ is a field, a quadratic space over $\Spec A$ is the same as a non-degenerate symmetric bilinear form over $K$. 

Given two quadratic spaces $(F, f)$ and $(G,g)$ over $X$, we define their \textit{orthogonal sum} $(F,f)\perp (G,g)$ as $((F\oplus G, f\oplus g)$. An \textit{isometry} of $(F,f)$ into $(G,g)$ is defined as a linear isomorphism $\phi:F\to G$ such that $\phi g\phi=f$. The set of isometries of $(F,f)$ into itself is the \textit{orthogonal group}  $O(F,f)$ of $F$. 

Given a vector bundle $P$ over $X$ we define a quadratic space $H(P)$ over $X$ - \textit{the hyperbolic space}  associated to $P$- by 
$$
H(P)=\left(P\oplus \displaystyle\mathop{P}^{\vee}\begin{bmatrix}
0 & 1\\
1 & 0
\end{bmatrix}\right)
$$
For\pageoriginale any submodule $L$ of $F$ we define its orthogonal $L^{\perp}$ as the kernel of $\displaystyle\mathop{i}^{\vee}\circ f$, where $\displaystyle\mathop{i}^{\vee}:F\to L$ is the dual of the inclusion $i:L\to F$ and $f$ is the isomorphism defining the quadratic structure on $F$. Clearly, the orthogonal of $P$ in $H(P)$ is $P$ itself. This motivates the following generalization of hyperbolic space. A space $(F,f)$ is said to be \textit{metabolic} if $F$ contains a subbundle (not just a submodule!) $L$ that coincides with its orthogonal: $L=L^{\perp}$. In this case, $L$ is called a \textit{lagrangian} of $(F,f)$. If $X$ is affine, a metabolic space with lagrangian $L$ is isometric to $H(L)$, but in general, a metabolic space need not be hyperbolic. 

The category of quadratic spaces over $X$ is a category with ``product'' in the sense of \cite[p. 344]{key3}. Let $G(X)$ be its Grothendieck group \cite[p. 346]{key3}. The Witt group of $X$ is the quotient $W(X)$ of $G(X)$ by the subgroup generated by the classes of metabolic spaces. If $X=\Spec A$, we put $W(A)=W(X)$. If $A$ is a field, $W(A)$ is the group of anisotropic quadratic forms over $K$ introduced by $E$. Witt in 1937 \cite{key16}. The Witt group of a scheme has been defined more recently by M. Knebusch. His lectures at the Queen's conference on a quadratic forms are an excellent introduction to the subject \cite{key10}.

It is easily checked that the tensor product $(F,f)\otimes (G,g)=(F\otimes G f\otimes g)$ induces a ring structure on $W(X)$. We shall use this fact in \S\ 2, but our main interest here is the group structure of $W(X)$. 

It is important to observe that, for any morphism of schemes $\phi:Y\to X$ and any quadratic space $(F,f)$ over $X$, the inverse image $\left(\phi^{\ast}F, \phi^{\ast}f\right)$\pageoriginale is a quadratic space over $Y$. Furthermore, $\phi^{\ast}$ induces a functorial ring homomorphism $W(\phi):W(X)\to W(Y)$. Hence $W$ is a contravariant functor from the category of the schemes under consideration to the category of commutative rings. In particular, if $X$ is reduced and irreducible, there is a canonical morphism $W(X)\to W(K)$, where $K$ is the field of rational functions of $X$. And if $X$ is a $k$-scheme, $k$ a ring, there is a canonical homomorphism $W(k)\to W(X)$.

\begin{EXPS}
If $k$ is a field (of characteristic $\neq 2$) and $\mathbb{A}^{n}_k$, $\mathbb{P}^{n}_k$ are, respectively, the $n$-dimensional affine and the $n$-dimensional projective space over $k$, then 
$$
W\left(\mathbb{A}^{n}_k\right)=W(k)=W\left(\mathbb{P}^{n}_k\right).
$$
The first equality is due to karoubi \cite{key9}. the second one to Arason \cite{key1}. 

In \cite{key11}, Knebusch asked if the Witt group of a finitely\pageoriginale generated $R$-scheme is finitely generated. For one-dimensional schemes this is indeed the case, as shown by Knebusch himself in the smooth case and by Dietel \cite{key5} in general. In his thesis, the first named author has proved the following results. 
\end{EXPS}

\begin{TM}
\textit{Let $X$ be an affine real surface and $X$ the normalization of $X_{\red}$. If the cokernel of the canonical homomorphism $\Pic X\to \Pic\\ \overline{X}$ is finitely generated, then the Witt group of $X$ is finitely generated.}
\end{TM}

\begin{corr}
\textit{The Witt group of a normal real surface is finitely generated.}
\end{corr}

\begin{ces}
There are real surfaces for which $W(X)$ is not fini\-tely generated. An easy example is the surface of $\mathbb{R}^{3}$ defined by the equation $z^{2}=x^{2}f(y)$, where $f$ is a square-free polynomial of degree at least $3$.

Most probably, the main theorem can be extended to quasi-projective surfaces.
\end{ces}

\section{The classical invariants}\label{s2}

We describe how the classical invariants of quadratic forms over fields have been extended to quadratic spaces over schemes. They will prove very useful for the study of the Witt group of a real surface. 

If a scheme $X$ is the disjoint union of two closed subschemes $X_1$, $X_2$, its Witt ring is simply $W(X_1)\times W(X_2)$. Hence we may (and do) assume that $X$ is connected. In particular, every vector bundle over $X$ has a well defined rank.  

\eject

\noindent
\textbf{(a) The rank homomorphism}

The rank of a metabolic space is even because a metabolic space is locally hyperbolic. Hence, the parity of the rank of $F$ only depends on the class of $(F,f)$ in $W(X)$ and yields a ring homomorphism 
$$
\rho:W(X)\to \dfrac{\mathbb{Z}}{2}
$$
called the \textit{rank}.

\begin{EXP}
$\rho$ induces an isomorphism 
$$
\rho:W(\mathbb{C})\xrightarrow{\sim} \dfrac{\mathbb{Z}}{2}
$$
We\pageoriginale denote by $I(X)$ the kernel of $\rho$: it is the \textit{ideal of even rank spaces}.
\end{EXP}

\medskip
\noindent
\textbf{(b) The signatures}

Let $K$ be a field. An \textit{ordering} of $K$ is a subset $P$ satisfying $P+P\subset P$, $PP\subset P$ and such that $K$ be the disjoint union of $P,-P$ and $\{0\}$. 

Let $(V,f)$ be a quadratic space over $K$. We can choose an orthogonal basis $e_1,\ldots,e_r$ of $V$. Given any ordering $P$ of $K$, we define the signature of $(V,f)$ with respect to $P$ as $\sigma_P(V,f)= (\text{number of}<e_i,e_i> \text{in} P)-(\text{number of} <e_i,e_i> \text{in} -P)$. By a well known-theorem of Sylvester, $\sigma_P(V,f)$ does not depend on the choice of the orthogonal basis. The signature of a hyperbolic space being zero, $\sigma_P$ defines a surjective ring homomorphism 
$$
\sigma_P:W(K)\to \mathbb{Z}
$$  

\begin{EXP}
The field $\mathbb{R}$ of real numbers has only one ordering given by $P$=set of non-zero squares. The homomorphism that it defines is in fact an isomorphism 
$$
W(\mathbb{R})\xrightarrow{\sim} \mathbb{Z}.
$$
It can be shown that, conversely, every surjective homomorphism $\rho:W(K)\to\mathbb{Z}$ coincides with $\sigma_P$ for a uniquely defined ordering $P$ of $K$. We therefore extend the notion of signature to any scheme $X$ by saying that a \textit{signature} of $K$ is surjective ring homomorphism 
$$
\sigma:W(X)\to \mathbb{Z}.
$$
\end{EXP}

Assume\pageoriginale now that $X$ is a real quasi-projective variety and let $X(\mathbb{R})$ denote the set of real (closed) points of $X$. For any $x\in X(\mathbb{R})$, the residue field $k(x)$ is $\mathbb{R}$, hence the canonical homomorphism 
$$
\sigma_X:W(X)\to W(k(x))=W(\mathbb{R})=\mathbb{Z}
$$
is a signature of $X$. Clearly, $\sigma_X$ only depends on the connected component of $X(\mathbb{R})$ containing $x$. Since $X(\mathbb{R})$ has only a finite set of connected components, we obtain in this way only a finite number of signatures. A very useful theorem of Knebusch \cite{key10} asserts that every signature of $X$ is a $\sigma_x$, $x\in X(\mathbb{R})$. In particular, a real variety has only finitely many signatures (and possibly none).

A word of caution may be appropriate. Assume that $X$ is integral and let $\mathbb{R}(X)$ be its field of rational functions. Every signature of $\mathbb{R}(X)$ defines, by composition with $W(X)\to W(\mathbb{R}(X))$, a signature of $X$. But, in general, not every signature of $X$ arises in this way. For example, if $X=\Spec \left(\dfrac{R[T_1,T_2]}{\left(T^{2}_1+T^{2}_2\right)}\right)$, $X(\mathbb{R})$ consists of one point, hence $X$ has one signature, whereas $\mathbb{R}(X)$ has no ordering because it contains an element of square $-1$. On the other hand, in spite of the fact that $X$ has only finitely many signatures, $\mathbb{R}(X)$ may have infinitely many. This happens, for instance, if $X=\mathbb{A}^{n}_{\mathbb{R}}$.

\medskip 
\noindent
\textbf{(c) The discriminant}

The tensor product of two quadratic spaces of rank $1$ is again a quadratic space of rank $1$ and the square of such a space is isometric to the ``unit space'' $(O_X,id)$. Hence the set of isometry classes of rank $1$ quadratic spaces has a natural structure\pageoriginale of abelian group of exponent $2$. We denote it by $Q(X)$. 

Let $(F,f)$ be a quadratic space over $X$, of rank $r$. The $r$-th exterior power $\displaystyle\mathop{\wedge}^{r}f$ maps $\displaystyle\mathop{\wedge}^{r}F$ to $\displaystyle\mathop{\wedge}^{r}\displaystyle\mathop{F}^{\vee}$. We identify $\displaystyle\mathop{\wedge}^{r}\displaystyle\mathop{F}^{\vee}$ with $\left(\displaystyle\mathop{\wedge}^{r}F\right)^{\vee}$, so that $d(F,f)=\left(\displaystyle\mathop{\wedge}^{r}F,(-1)^{r(r-1)/2}\displaystyle\mathop{\wedge}^{r}f\right)$ becomes a quadratic space of rank $1$. The discriminant of $(F,f)$ is, by definition, the class of $d(F,f)$ in $Q(X)$. The discriminant of a metabolic space is trivial but, in general, $d$ does not define a homomorphism from $W(X)$ to $Q(X)$. Only its restriction to the ideal of even rank spaces gives a group homomorphism 
$$
\delta:I(X)\to Q(X), 
$$
which we call the \textit{discriminant}.

Any global function $\alpha$ on $X$ gives rise to a rank one quadratic space $(O_X,\alpha)$. On the other hand, to any rank one quadratic space $(I,q)$ we can associate the isomorphism class of the self-dual line bundle $I$ in $\Pic X$. This yields an exact sequence 
$$
1\to \dfrac{\Gamma(X)^{\bigdot}}{(\Gamma(X)^{\bigdot})^{2}}\to Q(X)\xrightarrow{\pi}_2\Pic X\to 0, 
$$
where $\Gamma(X)$ is the ring of global sections of $O_X$ and, for any ring $\wedge$, $\wedge$ denotes its group of units. 

Clearly $\pi\delta$ depends only on the linear structure of the quadratic space over $X$ and is a homomorphism on the whole of $W(X)$. 

\begin{EXP}
Let $X=\Spec\left(\dfrac{\mathbb{R}[T_1,T_2]}{\left(T^{2}_1+T^{2}_2-1\right)}\right)$. The set of\pageoriginale real points of $X$ is a circle, hence $X$ has only one signature $\sigma$. It is easy to see that $\Pic  X=_2\Pic X\cong \mathbb{Z}/2$ and it can be shown that 
$$
\sigma \oplus \pi\delta:W(X)\to \mathbb{Z}\oplus \dfrac{\mathbb{Z}}{2}
$$
is a group isomorphism. 
\end{EXP}

\medskip
\noindent
\textbf{(d) The Clifford invariant}

Consider first an affine scheme $X=\Spec A$. As we remarked in \S\ 1, a quadratic space on $X$ is the same as a pair $(F,f)$ where $F$ is a projective $A$-module of finite type and $f$ a symmetric isomorphism of $F$ into its dual. Let $<,>$ denote the associated bilinear product. The Clifford algebra $C(F,f)$ of $(F,f)$ is the quotient of the tensor algebra $T=A\oplus F\oplus F\oplus F\oplus\ldots$ by its two-sided ideal generated by all elements $x\otimes x-<x,x>,x\in F$. We denote by $i:F\to C(F,f)$ the $A$-linear map induced by the canonical injection $F\to T$. The pair $(C(F,f),i)$ is uniquely defined by the following universal property: given any $A$-algebra $\wedge$ and $A$-linear map $\phi$: $F\to \wedge$ such that $\phi(x)^{2}=<x,x>1_{\wedge}$, there exists a unique homomorphism $\Phi:C(F,f)\to \wedge$ of $A$-algebras such that $\phi=\Phi\circ i$ 

The construction of $C(F,f)$ commutes with scalar extensions. Assume now that the rank of $F$ is an even integer $2r$. Then, locally for the \'{e}tale topology, $(F,f)$ is isometric to $H\left(A^{r}\right)$. A direct computation shows that $C(H\left(A^{r}\right))$. A direct computation shows that $C(H\left(A^{r}\right))\simeq M_{2r}(A)$; hence, locally for the \'{e}tale topology, $C(F,f)$ is a matrix algebra. This shows that $C(F,f)$ is an Azumaya algebra.

For a general (non affine) scheme $X$, the universal property of \\$C(F,f)$ allows to patch the various $C\left(F\mid U, f U\right)$ over a covering\pageoriginale consisting of affine open sets $U$. This patching yields an Azumaya algebra $C(E,f)$ over $X$. We denote by $\gamma(F,f)$ the class of $C(F,f)$ in the Brauer group of $X$. Notice that the opposite algebra of $C(F,f)$ satisfies the same universal property as $C(F,f)$ and is therefore canonically isomorphic to $C(F,f)$. From this it follows that $C(F,f)\simeq C(F,f)^{opp}$, and hence $\gamma(E,f)\in_2 Br(X)$. 

In general, $\gamma$ does not define a homomorphism from $W(X)$ to $_2Br(X)$, but if we restrict $\gamma$ to the subgroup $J(X)=\ker (I(X)\xrightarrow{\delta}Q(X))$, we get indeed a group homomorphism 
$$
\gamma:J(X)\to_2Br(X).
$$
We call $\gamma$ the \textit{Clifford invariant}.

\begin{EXP}
Let $X=\Spec\left(\dfrac{\mathbb{R}[T_1,T_2,T_3]}{\left(T^{2}_1+T^{2}_2+T^{2}_3-1\right)}\right)$. Since $X(\mathbb{R})$ is the real sphere, $X$ has only one signature $\sigma$. It is well known that $\Pic X=0$, hence $Q(X)=Q(\mathbb{R})=\dfrac{\overdot{\mathbb{R}}}{\overdot{\mathbb{R}}{}^{2}}\xrightarrow{\sim} \dfrac{\mathbb{Z}}{2}$. Choose now a real point $x: \Spec \mathbb{R}\to X$ on $X$. It defines a homomorphism $W(x):W(X)\to W(R)$. From the commutative diagram
$$
\xymatrix{W(X)\ar[r]^-{\sigma}\ar[d]_{W(x)}&\mathbb{Z}\ar@{=}[d]\\
W(\mathbb{R})\ar[r]^-{\sim}&\mathbb{Z}}
$$
and from $Q(X)=Q(\mathbb{R})$ it follows that $\ker \sigma =J(X)$. Hence $\gamma$ defines a homomorphism $\ker \sigma\to_2Br X$. It can be shown that 
\end{EXP}

$Br X=\dfrac{\mathbb{Z}}{2}\oplus \dfrac{\mathbb{Z}}{2}$,\pageoriginale where one copy of $\dfrac{\mathbb{Z}}{2}$ is represented by the usual (constant) quaternion algebra over $X$, the other copy being the kernel of $Br(x):Br(X)\to Br(\mathbb{R})$. An explicit construction of an algebra representing this kernel shows that $\gamma$ maps $J(X)$ isomorphically onto $\dfrac{\mathbb{Z}}{2}$. Hence $W(X)=\mathbb{Z}\oplus \dfrac{\mathbb{Z}}{2}$. 

\section{Regular affine surfaces}\label{s3}

We assume here that $X=\Spec A$ and that $A$ is a regular $2$-dimensional integral affine algebra over $\mathbb{R}$. We want to show that $W(X)$ is a finitely generated group. Let $K$ be the field of fractions of $A$. We recall a result proved in \cite{key13} and, earlier but \cite{key4} independently, by W. Pardon. 

\begin{thm}\label{thm1}
Let $A$ be a regular $2$-dimensional domain in which $2$ is invertible and $K$ its fields of fractions. The canonical homomorphism $W(A)\to W(K)$ is injective.
\end{thm}

A similar result holds for $3$-dimensional regular domains, but not in dimension $4$. It should not be too difficult to prove the analogous statement for smooth quasi-projective surfaces. 

We also need the following theorem of Elman and Lam \cite{key6}. 

\begin{thm}\label{thm2}
Let $K$ be a field of transcendence degree at most $2$ over $\mathbb{R}$. Then every element of $W(K)$ is determined by its classical invariants.
\end{thm}

Let now $X=\Spec A$ be an affine smooth real surface. It clearly suffices to show that $I(X)$ is finitely generated. 

Any signature $\sigma:W(K)\to \mathbb{Z}$ restricts to a signature $\sigma_i:W(X)\to \mathbb{Z}$.\pageoriginale Since the number of signatures of $X$ is finite, there is a finite number of different $\sigma_i$ which, together with the discriminant, give rise to a commutative diagram 
$$
\xymatrix{0\ar[r]&N(X)\ar[r]\ar[d]&I(X)\ar[d]\ar[r]^-{\pi\sigma_i\times \delta}&\mathbb{Z}X\ldots X\mathbb{Z}\ar[d]\times Q(X)\\
0\ar[r]&N(K)\ar[r]&I(K)\ar[r]_-{\pi \sigma \times \delta}& \pi\mathbb{Z}\times Q(K)}
$$
where the products in the bottom line extend over all the signatures of $K$. 

We show that $Q(X)$ and $_2Br(X)$ are finite groups. From the exact sequence of \'{e}tale sheaves 
$$
1\to \mu_2\to \mathbb{G}_m\xrightarrow{(-)^{2}}\mathbb{G}_m\to 1, 
$$
we get a cohomology exact sequence 
$$
\ldots\to H^{i}(X,\mathbb{G}_m)^{2}\to H^{i}(X,\mathbb{G}_m)\to H^{i+1}(X,\mu_2)\to H^{i+1}(X,\mathbb{G}_m)\to\ldots
$$
For $i=0$, this is the sequence relating $Q(X)$ to $\dfrac{\Gamma(X)}{(\Gamma(X)^{\cdot})^{2}}$ and $_2\Pic X$,\\hence $Q(X)=H^{1}(X,\mu_2)$. For $i=1$, this sequence shows that $_2Br(X)$ is a quotient of $H^{2}(X,\mu_2)$. Now, for a smooth real quasi-projective variety $X$, the groups $H^{i}(X,\mu_2)$ are finite. This can be shown as suggested in \cite[p. 244]{key12}, although the theorem stated there is obviously false for arbitrary ground fields. By the finiteness of $Q(X)$, we are reduced to show that $N(X)$ is finitely generated. In the commutative diagram 
$$
\xymatrix{N(X)\ar[d]\ar[r]^{\gamma}&_2Br(X)\ar[d]\\
N(K)\ar[r]^{\gamma}&_2Br(K)}
$$\pageoriginale
the vertical map on the left is injective, by Theorem\ref{thm1} and the bottom map is injective, by Theorem~\ref{thm2}; hence the top map is injective as well. Since $_2Br(X)$ is finite, $N(X)$ is finite and $W(X)$ is finitely generated. 

\section{Normal affine surfaces}\label{s4}

Assume now that $X=\Spec A$, where $A$ is an integrally closed affine domain over $\mathbb{R}$. As before, let $K$ be the field of fractions of $A$. In general, for normal surfaces, $W(X)\to W(K)$ is not injective.

\begin{EXP}
$A=\dfrac{[T_1,T_2,T_3]}{\left(T^{2}_1+T^{2}_2+T^{2}_3\right)}=\mathbb{R}[t_1,t_2,t_3]$. In this case, $X(\mathbb{R})$ consists of one point, hence $X$ has exactly one signature $\sigma$. The quadratic space $(A^{4}, id)$ represents a non-zero class $\xi$ in $W(X)$ because $\sigma(A^{4},id)=4$. The image of $\xi$ in $W(K)$ is the class of $\left(K^{4},id\right)$. Now, in $K$ we have $\left(\dfrac{t_1}{t_3}\right)^{2}+ \left(\dfrac{t_2}{t_3}\right)^{2}=-1$, hence $\left(K^{4},id\right)$ is isotropic and splits as $H(K)\perp(V,f)$. But the discriminant of $(V,f)$ must be $1$, hence $(V,f)\cong H(K)$ and $\left(K^{4},id\right)$ is hyperbolic. This shows that $\xi$ maps to zero in $W(A)$. 

Instead of Theorem~\ref{thm1}, we shall use the following result, proved in \cite{key13}. 
\end{EXP}

\begin{thm}\label{thm3}
Let $A$ be a normal $2$-dimensional domain, $K$ its field of fractions. Let $\xi$ be an element of $W(A)$. Assume that $\xi$ is in the\pageoriginale kernel of $W(A)\to W(K)$ and also in the kernel of $W(A)\to W\left(\dfrac{A}{m}\right)$ for every singular maximal ideal $m$ of $A$. Then, $\gamma(\xi)=0$ implies $\xi=0$. 
\end{thm}

\begin{corr}
\textit{Let $A$ be a normal $2$-dimensional affine algebra over $\mathbb{R},\\\sigma_1,\ldots\ldots \sigma_n$ the signatures of $A$ and $N(X)$ the kernel of the homomorphism }
$$
\sigma_1\times\ldots \times\sigma_n\times \delta:I(X)\to \mathbb{Z}\times\ldots\times \mathbb{Z}\times Q(X)
$$
Then, $\ker(N(X)\to W(K))\subset\ker \mid (_2Br(X)\to _2Br(K))$. 
\end{corr}

In fact, $N(X)\subset J(X)$, hence $\gamma$ defines a homomorphism $\gamma:\\N(X)\to_2 Br(X)$. Since $\sigma_i(\xi)=0$ for every $\sigma\mid_i$, for any maximal ideal $m$ of $A$, the image of $\xi$ in $W\left(\dfrac{A}{m}\right)=W(\mathbb{R})$ is zero. On the other hand, if $m$ is not real, $W\left(\dfrac{A}{m}\right)=W(\mathbb{C})=\dfrac{\mathbb{Z}}{2}$, hence $\xi\in I(X)$ maps to $I\left(\dfrac{A}{m}\right)=0$ This shows that $\xi$ maps to zero in every $W\left(\dfrac{A}{m}\right)$. By the theorem above, $\gamma$ is injective on $\ker(N(X)\to W(K))$. 

As in \S\ 3, we first show that $q(X)$ is finite. If $k$($=\mathbb{R}$ or $=\mathbb{C}$) is the algebraic closure of $\mathbb{R}$ in $K$, by \cite{key15} the quotient $\dfrac{A^{\bigdot}}{k^{\bigdot}}$ is finitely generated. Hence $\dfrac{A}{(k^{\bigdot})^{2}}$ is also finitely generated and its quotient $\dfrac{A^{\bigdot}}{{A^{\bigdot}}^{2}}$ is finite. To show that $_2\Pic X$ is finite, we consider the open set $U$ of all regular points of $X$. The group $\Pic X$ is a subgroup of the divisor class group of $X$, which in fact coincides with $\Pic U$. Since $_2$ $\Pic U=H^{1}(U,\mu_2)$ is finite, $_2\Pic X$ is finite. The exact sequence connecting $Q(X)$, $\dfrac{A^{\bigdot}}{{A^{\bigdot}}^{2}}$ and $_2 \Pic X$ shows that $Q(X)$ is finite. Proceeding as in \S $3$. we are reduced to showing that $N(X)$ is finitely generated. Let $U$ be any affine open set of $X$ consisting of smooth points. By \S $3$, $W(U)$ is finitely generated, hence it suffices to show that $\ker(N(X)\to W(U))$\pageoriginale is finitely generated. Clearly, this kernel is contained in $\ker(N(X)\to W(K))\subset\ker(_2Br(X)\to_2 Br(K))$. Hence we are reduced to showing that $\ker(_2Br(X)\to_2Br(K))$ is finite. According to a result of Grothendieck \cite[p.74]{key7}, this last group is contained in 
$$
\bigoplus\limits_{x\in S 2}\left(\dfrac{C\ell\left(A^{bs}_x\right)}{C\ell(A_x)}\right)
$$
where $C\ell$ denotes the divisor class group, $S$ denotes the (finite) set of singular points of $X$ and $A^{bs}_x$ is the strict henselization of the local ring $A_x$ of $X$ at $x$. 

Consider the scalar extension $B=\mathbb{C}\bigoplus\limits_{\mathbb{R}}A$ and put $Y=\Spec B$. Then $A^{bs}_x=B^{b}_y$, the henselization of $B_y$, where $y$ is a preimage of $x$ in $y$. By a general approximation theorem of Hironaka [8, p. 214], $C\ell \left(B^{bol}_y\right)=C\ell\left(\widehat{B}_y\right)$ where $B^{bol}_y$ is the local ring of holomorphic functions of the complex variety associated to $Y$ and $\widehat{B}_y$ is the completion of $B_y$. The group $C\ell\left(B^{hol}_y\right)$ has been computed by Prill \cite{key14} and turns out to be of the form $P\oplus \left(\dfrac{\mathbb{Q}}{\mathbb{Z}}\right)^{m}\oplus \mathbb{Q}^c$ where $P$ is a finitely generated group, $m$ is a finite integer and $c$ is either zero or the cardinality of $\mathbb{R}$. On the other hand, $C\ell(A_y)$ is the direct sum of finitely generated group and a divisible group. From these facts, it follows easily that$_2\left(\dfrac{C\ell\left(B^{bs}_y\right)}{C\ell(A_x)}\right)$ is finite. 

\section{The general case}\label{s5}

We now consider the case of an arbitrary real affine algebra $A$ of dimension $2$. Since $W(A)$ is the same as $W(A_{\red})$, we assume that $A$ is reduced. Let $K$ be its total ring of fractions and $\overline{A}$ the integral closure of $A$ in $K$. Since $\overline{A}$ is a finite product of normal domains,\pageoriginale it follows from the results of \S\ 4, that $W(\overline{A})$ is finitely generated. Let $\mathfrak{c}$ denote the conductor of $\overline{A}$ in $A$. We consider the cartesian diagram 
$$
\xymatrix{A\ar[r]\ar[d]&\overline{A}\ar[d]\\
\dfrac{A}{\mathfrak{c}}\ar[r]& \dfrac{\overline{A}}{\mathfrak{c}}}
$$
and the corresponding Mayer-Vietoris sequence of Grothendieck-Witt groups 
$$
\ldots\to KO_1\left(\dfrac{A}{\mathfrak{c}}\right)\to KO(A)\to\to KO\left(\overline{A}\right)\times KO\left(\dfrac{A}{\mathfrak{c}}\right)\to\ldots. 
$$

Here, for any scheme, $X$, $KO(X)$ is the quotient of the Grothendieck group of the category of quadratic spaces over $X$, modulo the subgroup generated by the difference $(F,f)-H(L)$, where $L$ is a lagrangian of the (metabolic) space $(F,f)$. A reasonable definition of $KO_1$ seems to be known only for an affine scheme $X=\Spec R:KO_1(X)=KO_1(R)=\dfrac{O(R)}{[O(R),O(R)]}$, where $O(R)$ denotes the inductive limit of the orthogonal groups $O_{2n}(R)$ of the spaces $H(R^{n})$. Associating to every bundle $P$ over $X$ the hyperbolic space $H(P)$ yields an exact sequence 
$$
K_0(X)\xrightarrow{H}KO(X)\to W(X)\to 0
$$
which, combined with the Mayer-Vietoris sequence, gives a commutative diagram 
$$
\xymatrix@=.3cm{
\ar[r]&K_0(A)\ar[r]\ar[d]&K_0\left(\overline{A}\right)\ar[d]\ar@{}[r]|{\times} & K_0\left(\dfrac{A}{\mathfrak{c}}\right)\ar[r]\ar[d]&\\
\ar[r]&KO(A)\ar[d]\ar[r]&KO\left(\overline{A}\right)\ar[d]\ar@{}[r]|{\times} & KO\left(\dfrac{A}{\mathfrak{c}}\right)\ar[d]\ar[r]&\\
\ar[r]&W(A)\ar[d]\ar[r]&W\left(\overline{A}\right)\ar[d]\ar@{}[r]|{\times} & W\left(\dfrac{A}{\mathfrak{c}}\right)\ar[d]\ar[r]\ar[d]&\\
&0 &0 &0 &}
$$
A\pageoriginale careful analysis of this diagram leads to a proof of the main theorem. Although the details are rather tricky, the gist of the argument can be sketched as follows: we know that $W\left(\overline{A}\right)$ and $W\left(\dfrac{A}{\mathfrak{c}}\right)$ are finitely generated; hence we only have to show that the kernel $N(A)$ of $W(A)\to W\left(\overline{A}\right)\times W\left(\dfrac{A}{\mathfrak{c}}\right)$ is finitely generated. A class in this kernel comes from an element $\xi$ in $KO(A)$ which becomes hyperbolic in $KO\left(\overline{A}\right)$ and in $KO\left(\dfrac{A}{\mathfrak{c}}\right)$. In general, $\xi \not\in M(A)=\ker (KO(A))\to KO\left(\overline{A}\right)\times KO\left(\dfrac{A}{\mathfrak{c}}\right)$, but the condition that coker $(\Pic A\to \Pic \overline{A})$ be finitely generated is precisely what is needed to reduce the finite generation of $N(A)$ to that of $M(A)$. To show that $M(A)$ is finitely generated, we consider any  $\xi \in M(A)$. By a quadratic analogue of Serre's theorem on projective modules \cite{key13}, we may assume that $\xi$ is represented by a space $(F,f)$ such that $(F,f)\otimes_A\overline{A}\cong H\left(\overline{A}^{2}\right)$ and $(F,f) \otimes_A \dfrac{A}{\mathfrak{c}}\cong H\left(\dfrac{A}{\mathfrak{c}}^{2}\right)$. This means that $\xi$ is a ``Minor patching'' of $\overline{H}(A^{2})$ and $H\left(\dfrac{A}{\mathfrak{c}}^{2}\right)$ over $\dfrac{A}{\mathfrak{c}}$ via an isometry of $H\left(\left(\dfrac{\overline{A}}{\mathfrak{c}}\right)^{2}\right)$. Denote $\dfrac{A}{\mathfrak{c}}$ by $B$ and let $S(B)$ be the subgroup of $GL_2(B)\times GL_2(B)$ consisting of the pairs $(\alpha, \beta)$ with $\det \alpha=\det \beta$. We identify $H(B^{2})$ with $(M_2(B),\det)$ and associate to $(\alpha, \beta)\in S(B)$ the isometry sending $\xi \in M_2(B)$ to $\alpha\xi \beta^{-1}$. This gives\pageoriginale a group homomorphism $S(B)\to O_4(B)$. It follows from results of Bass \cite{key2}, that the image of $S(B)$ is a normal subgroup of finite index in $O_4(B)$. On the other, hand, it is easy to check that patching $H\left(\overline{A}^{2}\right)$ and $H\left(\left(\dfrac{A}{\mathfrak{c}}\right)^{2}\right)$ via a matrix coming from $S(B)$ yield a stably hyperbolic space  $(F,f)$ over $A$. Hence, upto Witt equivalence, there are only a finite number of such $\xi$ and $M(A)$ is indeed finitely generated.

\begin{ack}
We thank W. Pardon for suggesting to us that the results of Elman and Lam could be used for our problem.
\end{ack}

\begin{thebibliography}{99}
\bibitem{key1}
{J. Arason}, Der Wittring projektiver R\"{a}ume, \textit{Math, Ann}., 253 (1980), 205--212.

\bibitem{key2}
{H. Bass}, \textit{Lectures on topics in algebraic K-theory}, Tata Institute of Fundamental Research, Bombay, 1961.

\bibitem{key3}
{H. Bass}, \textit{Algebraic K-theory}, Benjamin, New York, 1968.

\bibitem{key4}
{J. L. Borges}, El tiempo circular, \textit{Obras completas}, Ultramar, Madrid, 1977. 

\bibitem{key5}
{G. Dietel}, Wittringe reeller Kurven, I,II, \textit{Commun. Algebra} 11(1983), 2393--2494.

\bibitem{key6}
{R. Elman and T.Y. Lam},\pageoriginale Classification theorems for quadratic forms over fields, \textit{Comm. Math. Helv}., 49(1974), 373--381.

\bibitem{key7}
{A. Grothendieck}, Le groupe de Brauer, I, II, III, in \textit{Dix expos\'{e}s sur la cohomologie des sch\'{e}mas}, North Holland Amsterdam, 1968.

\bibitem{key8}
{H. Hironaka}, Formal line bundles along exceptional loci, in \textit{Algebraic Geometry}, Bombay Colloquim 1968, Bombay, 1969.

\bibitem{key9}
{M. Karoubi}, Localisation des formes quadratiques II, \textit{Ann.Sci. Ec Norm. Sup}., 8(1975), 99--155.

\bibitem{key10}
{M. Knebusch}, Symmetric bilinear forms over algebraic varieties, \textit{Queen's Papers in pure and Appl. Math.,} 46(1977), 103--283.

\bibitem{key11}
{M. Knebusch}, Some open problems, \textit{Queen's Papers in pure and Appl. Math}., 46(1977),

\bibitem{key12}
{J. S. Milne}, \textit{Etale cohomology}, Princeton, 1980.

\bibitem{key13}
{M. Ojanguren}, A splitting theorem for quadratic forms, \textit{Comm. Math. Helv}., 57(1982), 145--157.

\bibitem{key14}
{D. Prill}, The divisor class group of some rings of holomorphic functions, \textit{Math, Zeit}., 121 (1971), 58--80.

\bibitem{key15}
{P. Samuel}, A propos du th\'{e}or\`{e}me des unit\'{e}s, \textit{Bull. Sc. Math.,} 90(1966), 89--96.

\bibitem{key16}
{E. Witt},\pageoriginale Theorie der quadratischen Former in beliebigen K\"{o}rpern, \textit{Journal Reine Angew. Math.,} 176(1937), 31--44.
\end{thebibliography}

\vskip 1cm

\noindent
Institut de Mathematiques\\
Facult\'e des Sciences\\
Universit\'e de Lausanne\\
1015 Lausanne - Dorigny\\
Switzerland.

\newpage
~\phantom{a}
\thispagestyle{empty}
