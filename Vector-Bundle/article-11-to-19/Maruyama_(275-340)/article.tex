\title{Vector Bundles On $\mathbb{P}^{2}$ And Torsion Sheaves On The
  Dual Plane}
\markright{Vector Bundles On $\mathbb{P}^{2}$ And Torsion Sheaves On The
  Dual Plane}

\author{By M. Maruyama}
\markboth{M. Maruyama}{Vector Bundles On $\mathbb{P}^{2}$ And Torsion Sheaves On The
  Dual Plane}

\date{}
\maketitle

\setcounter{page}{213}
\setcounter{pageoriginal}{274}
\section*{Introduction}\pageoriginale

W. Barth found a beautiful relationship between rank-2 nets of
quadrics and stable vector bundles of rank-2 on the projective plane
$\mathbb{P}^{2}$ with the first Chern class zero (\cite{key2}). Then,
in \cite{key7}, K. Hulek defined a pre-stable Kronecker module and
succeeded in describing $s$-stable vector bundles on $\mathbb{P}^{2}$ in
terms of it. The notion of Kronecker modules is a generalization of
nets of quadrics. In fact, a net of quadrics is nothing but a
symmetric Kronecker module.

For a non-degenerate Kronecker module $\alpha$, we can define the
discriminant curve $\Delta(\alpha)$ in the dual plane and an
$\mathscr{O}_{\Delta(\alpha)}$-module $\mathscr{L}(\alpha)$. Moreover,
the couple ($\Delta(\alpha),\mathscr{L}(\alpha)$) determines
$\alpha$ (\cite[p. 124, 125]{key7}). When $\alpha$ is a net of quadrics
$\mathscr{L}(\alpha)$ becomes a $\theta$-characteristic on
$\Delta(\alpha)$, inheriting the symmetry of $\alpha$. Thus we come to
the known result that the classification of non-degenerate nets of
quadrics reduces to that of couples of a plane curve and a
generalized, ineffective $\theta$-characteristic (see, for example,
\cite{key4} and Corollary~\ref{cor2.12.4} of this article).

If\pageoriginale $\alpha$ is obtained from a vector bundle $F$ on
$\mathbb{P}^{2}$, the non degeneracy means that 

\eqref{eqn1.10.2} for general lines $\ell$ in $\mathbb{P}^{2}$,
$F\mid_{\ell}\simeq\mathscr{O}^{\oplus r(F)}_{\ell}$.

\smallskip

In this way, an $s$-stable vector bundle on $\mathbb{P}^{2}$ with the
property \eqref{eqn1.10.2} gives rise to a couple of a curve in the
dual plane and a coherent sheaf on it. The main purpose of this work
is to study the inverse of this process. A remarkable property of
$\mathscr{L}(\alpha)$ is 
$$
H^{0}(\Delta(\alpha),\mathscr{L}(\alpha))=H^{1}(\Delta(\alpha), \mathscr{L}(\alpha))=0. 
$$
Taking note of this, we shall start with, instead of a Kronecker
module, a coherent sheaf $L$ on the dual plane $P^{\ast}$ with the
following property:

\eqref{eqn2.3.1} $L$ is a torsion sheaf such that $H^{0}(P^{\ast},L)= H^{1}(P^{\ast},L)=0$.

Then we have the following resolution (Proposition~\ref{Prop2.5}):
$$\eqref{eqn2.5.1}
\quad 0\rightarrow\mathscr{O}_{P^{\ast}}(-1)^{\oplus
  n}\xrightarrow{\alpha}\mathscr{O}^{\oplus^{n}}_{P^{\ast}}\rightarrow
L(1)\rightarrow 0.
$$
$\alpha$ is represented by an $n\times n$-matrix $\alpha(L)$ whose
entries are linear forms on $P^{\ast}$. The $\alpha(L)$ is a Kronecker
module and the curve in $P^{\ast}$ defined by $\det \alpha(L)=0$ is
the discriminant curve $\Delta(\alpha(L))$ The given $L$ carries an
$\mathscr{O}_{\Delta(\alpha(L))}$-module structure which is the
$\mathscr{L}(\alpha(L))$ stated in the above.

Therefore,\pageoriginale we shall take, on one hand, the full subcategory
$\mathscr{C}$ of the category of coherent
$\mathscr{O}_{P^{\ast}}$-modules whose objects have the properties
\eqref{eqn2.3.1} and \eqref{eqn2.3.2}. The property \eqref{eqn2.3.2}
corresponds to a half of 1.2.2 of \cite{key7}; 
$\dim\left(\alpha^{T}(\varphi \otimes
V^{\ast})\right)\geqq 2$ (see Corollary~\ref{cor2.13.2}). On the other hand,
let $\mathscr{V}$ be the full subcategory of the category of coherent
sheaves on $\mathbb{P}^{2}$ such that an $F$ is contained in
$\mathscr{V}$ if $F$ is a vector bundle with the properties \eqref{eqn1.10.2},
$H^{0}\left(\mathbb{P}^{2},F\right)=0$ and $c_2(F)=r(F)$. Our main
result (Main Theorem~\ref{MT2.16}) is then stated as follows.

\begin{TM}
\textit{There is an equivalence between two categories $\mathscr{V}$ and $\mathscr{C}$}.
\end{TM}

Barth \cite{key2} and Hulek \cite{key7} used the monads to get vector
bundles from nets of quadrics and Kronecker modules. We shall
construct a vector bundle in $\mathscr{V}$ from a member of
$\mathscr{C}$ by exploiting a flag manifold and direct image
functors. An observant reader will find the idea of elementary
transformations behind our construction of vector bundles.

We know two types of good monads for $s$-stable vector bundles $E$ on
$\mathbb{P}^{2}$ with $c_1(E)=0$(\cite{key3}, \S 6).
\begin{align*}
M_1&:\mathscr{O}_P(-1)^{\oplus n}\rightarrow\mathscr{O}^{\oplus
  (2n+r)}_P\rightarrow \mathscr{O}_P(1)^{\oplus n}\\
M_2&: \mathscr{O}_P(-1)^{\oplus n}\xrightarrow{a}\Omega_P(1)^{\oplus
  n}\xrightarrow{b}\mathscr{O}_P^{\oplus (n-r)}
\end{align*}
where $P=\mathbb{P}^{2}$, $c_2(E)=n$ and $r(E)=r$. The former was used
by K. Hulek and W. Barth employed the latter in the case of $r=2$. If we
describe a vector bundle $F$ in $\mathscr{V}$ by $M_2$, then $b=0$ and
hence\pageoriginale $F$ is isomorphic to $\coker(\alpha)$ because
$n=r$. By using the resolution \eqref{eqn2.5.1} and
Corollary~\ref{cor3.6.1}, we also get to the monad.

For general $s$-stable vector bundles, the connection between our
results and the monadology will be revealed by Theorem~\ref{Theorem4.8}. Thus our results must be interpreted in terms of the monad of type
$M_2$ if one follows the work of W. Barth in \cite{key2}. The author
hopes, nevertheless, that his results will serve deeper understanding
of the relation between vector bundles on $\mathbb{P}^{2}$ and plane
curves. 

We have a nice application of our viewpoint to the study of moduli
spaces of stable vector bundles on $\mathbb{P}^{2}$ with the first
Chern class zero. It is easy to see that the second Chern class $c_2$
is greater than or equal to the rank $r$. If one fixes $c_2$, then the
extreme case $c_2=r$ is, in some sense, most important. In fact, we
shall show that a good part of the other moduli space is a subscheme
of that of the extreme case (Proposition~\ref{Prop1.7} and
Theorem~\ref{Theorem5.6}). The structure of the moduli space of the
extreme case will be studied in a forthcoming paper.

This work was completed while the author was staying at Tata Institute
of Fundamental Research. He wishes to express his hearty thanks to the
mathematicians at the Institute for their hospitality and their
stimulation given to him.

\smallskip
\noindent
\textbf{Notation and convention.}

A variety in this article is geometrically integral algebraic scheme
over a field. For a coherent sheaf $F$ on a variety $X$,
$h^{i}(X,F)$\pageoriginale denotes $\dim H^{i}(X,F)$ and $r(F)$ does
the rank of $F$, that is, 
$F\mid_U\simeq\mathscr{O}^{\oplus r(F)}_{u}$ on a non-empty open
set $U$. $F^{\ast}$ is the dual
$\Hom_{\mathscr{O}_X}(F,\mathscr{O}_X)$ of $F$. If $E$ is a coherent
sheaf on $\mathbb{P}^{2}$, we can define the first and the second
Chern classes $c_1(E)$, $c_2(E)$. Since all the cycles on
$\mathbb{P}^{2}$ are determined by their degrees up to the rational
equivalence, $c_1(E)$ and $c_2(E)$ can be regarded as integers.

\section{Vector bundles on \texorpdfstring{$\mathbb{P}^{2}$}{eq} with \texorpdfstring{$c_1=0$}{eq} and \texorpdfstring{$c_2=r$}{eq}.}\label{s1}

First of all, let us recall the definition of stable sheaves.

\begin{dfn}\label{dfn1.1}
Let $(X,\mathscr{O}_X(1))$ be a couple of a non-singular projective
variety $X$ over a field $k$ and an ample line bundle
$\mathscr{O}_X(1)$ on $X$. For a coherent sheaf $G$ on $X$ with
$r(G)>0$, $P_G(m)$ is the polynomial
$\chi (G(m))/r(G)$, where
$\chi (G(m))=\sum(-1)^{i}h^{i}(X, G(m))$. $\mu(G)$ is defined to
be the rational number $d(G,\mathscr{O}_X(1))/r(G)$, where
$d(G,\mathscr{O}_X(1))$ is the degree of $c_1(G)$ with respect to
$\mathscr{O}_X(1)$. For an overfield $K$ of $k$ and for a coherent
sheaf $G$ on $X_K=X\otimes_{k}K$, $P_G(m)$ and $\mu(G)$ denote those
with respect to $(X_K, \mathscr{O}_{X_K}(1))$, respectively. A
coherent sheaf $E$ on $X$ is stable (semi-stable, $\mu$-stable or
$\mu$-semi-stable) with respect to $\mathscr{O}_X(1)$ is (a) $E$ is
torsion free and (b) for every coherent subsheaf $F$ of $E\otimes_k
\overline{k}$ with $0<r(F)<r(E)$, we have $P_F(m)<P_E(m)$ for all
large $m(P_F(m)\leqq(P_E(m)$ for all large $m$, $\mu(F)<\mu(E)$ or
$\mu(F)\leqq\mu(E)$, resp.). When $E$ is locally free, a stable
(semi-stable, $\mu$-stable or $\mu$-semi-stable) sheaf is called a
stable (semi-stable, $\mu$-stable or $\mu$-semi-stable, resp.) vector
bundle. 
\end{dfn}

For a coherent sheaf on $\mathbb{P}^{2}$ which we are mainly concerned
with in this article, the above four notions are independent 
of the\pageoriginale choice of the ample line bundle.

\begin{notation}\label{notation1.2}
We shall fix a base field $k$ which is not necessarily algebraically
closed. $P$ denotes the projective plane $\mathbb{P}^{2}_k$. 

The following seems to be known.
\end{notation}

\begin{lemma}\label{lemma1.3}
Let $E$ be a torsion free coherent sheaf on $P$. If $c_1(E)=0$,
$c_2(E)\leqq 0$ and if $E$ is $\mu$-semi-stable, then $E$ is
isomorphic to $\mathscr{O}^{\oplus r}_P$.
\end{lemma}


\begin{Proof}
Let us prove the lemma by induction on $r=r(E)$. If $r=1$, then
$E'=(E^{\ast})^{\ast}$ is $\mathscr{O}_P$. Since
$c_2(E)=h^{0}\left(P,E'/E\right)$ is positive or zero
according as $E\neq E'$ or $E=E'$, we see that $E=E'\simeq
\mathscr{O}_P$ by our assumption that $c_2(E)\leqq 0$. Assume that
$r(E)=r>1$ and if $r(E)<r$, our assertion is true. By our assumption,
we see that for $E'=(E^{\ast})^{\ast}$, $c_1(E')=0$ and $c_2(E')\leqq
c_2(E)$ whose equality holds if and only if $E=E'$. Thus, replacing
$E$ by $E'$, we may assume that $E$ is locally free. The
$\mu$-semi-stability of $E$ implies that
$h^{2}(P,E)=h^{0}(P,E^{\ast}(-3))=0$ and hence $h^{1}(P,E)=-r+c_2(E)$
if $h^{0}(P,E)=0$. This is not the case because $c_2(E)\leqq
0$. Therefore, by using a non-zero section of $E$, we have an exact
sequence 
$$
0\rightarrow \mathscr{O}_P\rightarrow E\rightarrow F\rightarrow 0.
$$

By the assumption that $E$ is $\mu$-semi-stable we see that $F$ is
torsion free. Since $c_1(F)=c_1(E)$ and $c_2(F)=c_2(E)$, our induction
hypothesis implies that $F$ is isomorphic to
$\mathscr{O}^{\oplus(r-1)}_P$ and then $E\simeq \mathscr{O}^{\oplus
  r}_P$.
\enprf
\end{Proof}

The\pageoriginale above is said in other words:

\begin{cor}\label{cor1.3.1}
If $E$ is a $\mu$-semi-stable sheaf on $P$ with $c_1(E)=0$, then
$c_2(E)>0$ or $E\simeq \mathscr{O}^{\oplus r}_P$.
\end{cor}

The results of this section are based on the following.

\begin{lemma}\label{lemma1.4}
Let $E$ and $F$ be coherent sheaves on $P$ with
$c_1(E)=c_1(F)=0$. Assume there exists an exact sequence
$$
0\rightarrow E\rightarrow F\xrightarrow{\Psi}\mathscr{O}^{\oplus r}_P.
$$
\begin{enumerate}
\renewcommand{\labelenumi}{(\theenumi)}
\item If $F$ is stable, $c_2(F)=r(F)$ and if \ $\Psi$ is generically
  surjective, then $\Psi$ is surjective.

\item If \ $\Psi$ is surjective, $E$ is $\mu$-stable, $F$ is locally
  free and $H^0(P.F)=0$, then $F$ is stable.
\end{enumerate}
\end{lemma}

\begin{Proof}
\begin{enumerate}
\renewcommand{\labelenumi}{(\theenumi)}
\item Put $\im(\Psi)=E'$ and $\mathscr{O}^{\oplus r}_P/ E'=T$. Since
$\Psi$ is generically surjective, $T$ is a torsion sheaf. If $\dim
\Supp(T)=1$, then $d(E'\mathscr{O}_P(1))<0$ and hence $F$ is not
$\mu$-semi-stable, a fortiori, not stable. Thus $\dim \Supp(T)\leqq
0$ and then it is easy to see that $c_2(E')=h^{0}(P,T)=s$ and
$c_1(E')=c_1(E)=0$. Suppose that $s<r\cdot t=h^{0}(P,E')\geqq
h^{0}\left(P,\mathscr{O}^{\oplus
  r}_{P}\right)-h^{0}(P,T)=r-s>0$. Now $\Psi$ is surjective if and
only if $t=r$. Assume that $t<r$. Since $\mathscr{O}^{\oplus
  t}_{P}=H^{0}(P,E')\otimes_k\mathscr{O}_P$ is a subbundle of
$\mathscr{O}^{\oplus r}_P=H^{0}\left(P,\mathscr{O}^{\oplus
  r}_{P}\right)\otimes_k \mathscr{O}_P$, the $\mathscr{O}^{\oplus t}_P$
is a subsheaf of $E'$. By replacing $E$ by
$\Psi^{-1}\left(\mathscr{O}^{\oplus t}_P\right)$, we may assume that
$s\geqq r$ because $c_2\left(E'/\mathscr{O}^{\oplus
    t}_{P}\right)=c_2(E')=s$ and $r\left(E'/\mathscr{O}^{\oplus
  t}_{P}\right)=r-t$. By virtue of our assumption $c_2(E)=r(F)
-s$. Then Riemann-Roch Theorem shows that $P_E(m)=m^{2}/2
+3m/2-(r(F)-s)/r(E)+1$. On the other hand,
$P_F(m)=m^{2}/2+3m/2-r(F)/r(F)+1$. These
together with the inequality $-(r(F)-s)/r(E)\geqq
(-r(F)+r)/r(E)=-r(E)/r(E)=-1$\pageoriginale contradict the stability of $F$.

\item Let $F'$ be a coherent subsheaf of $F$ such that $0<r(F')<r(F)$
  and $F/F'$ is torsion free. Since $F$ is $\mu$-semi stable,
  $d(F',\mathscr{O}_P(1))\leqq 0$. If $d(F',\mathscr{O}_P(1))<0$, it
  is obvious that $P_{F'}(M)<P_F(m)$ for all large $m$. We may assume,
  therefore, that $c_1(F')=0$. It is well-known that $F'$ is locally
  free. For $E'=F'\cap E$, $F'/E'$ can be regarded as a
  subsheaf of $\mathscr{O}^{\oplus r}_P$. If $E'=0$ then $F'$ itself
  is a subsheaf of $\mathscr{O}^{\oplus r}_P$. Then $F'$ must be
  $\mathscr{O}^{\oplus s}_P$ because it is locally free and
  $c_1(F')=0$. This violates the assumption that $H^{0}(P,F)=0$. Hence
  we have that $r(E')>0$. Since $E$ is $\mu$-stable,
  $d(E',\mathscr{O}_P(1))<0$ or $r(E)=r(E')$. The former is not the
  case because $d(F', \mathscr{O}_P(1))=d(F'/E',
  \mathscr{O}_P(1))+d(E',\mathscr{O}_P(1))<0$ if that holds. In the
  latter case, the difference between $P_{F'}(m)$ and $P_F(m)$ is at
  most on constant terms which are $-c_2(F')/r(F')+1$ and
  $-c_2(F)/r(F)+1$, respectively. On the other hand,
  $E/E'$  is a subsheaf of $F/F'$ which is torsion
  free. This and $r(E')=r(E)$ imply that $E=E'$. Then, we see that
  $P_{F'}(m)<P_F(m)$ for all large $m$ because
  $$
\dfrac{-c_2(F')}{r(F')}=\frac{\left\{c_2(E)+c_2\left(\frac{F'}{E}\right)
  \right\}}{r(F')\leqq
    \frac{-c_2 (E)}{r(F')}}<\dfrac{-c_2(E)}{r(E)}=\dfrac{-c_2(F)}{r(F)}
=\frac{-c_{2}(F)}{r(F)}
$$
by virtue of Corollary~\ref{cor1.3.1}
\end{enumerate}
\enprf
\end{Proof}

For a vector bundle $F$ on $P$ with $c_1(F)=0$ and $c_2(F)=r(F)$,
Lemma~\ref{lemma1.4} provides us with a relation between the stability
and the $s$-stability (\cite{key7}).

\begin{cor}\label{cor1.4.1}
If $F$ is a stable vector bundle on $P$ with $c_1(F)=0$ and
$c_2(F)=r(F)$ and if $h^{0}(P,F^{\ast})=s$, then $F$ is an extension
of $\mathscr{O}^{\oplus s}_P$ by a vector bundle $E$ with
$H^{0}(P,E^{\ast})=0$. Moreover, $E$ is uniquely determined by $F$. 
\end{cor}

\begin{Proof}
Since\pageoriginale $F$ is stable,  $F^{\ast}$ is $\mu$-semi-stable. Then it is
easily seen that $\mathscr{O}^{\oplus
  s}_P=H^{0}(P,F^{\ast})\otimes_{k}\mathscr{O}_P$ is subsheaf of
$F^{\ast}$. The sheaf $E=\left(F^{\ast}/\mathscr{O}^{\oplus
    s}_P\right)^{\ast}$ is locally free and we have the following
exact sequence
$$
0\rightarrow E\rightarrow F\xrightarrow{\Psi}\mathscr{O}^{\oplus s}_P
$$
$\Psi$  is generically surjective and we can apply
Lemma~\ref{lemma1.4}, (1) to the above sequence. $F$ is, therefore, an
extension of $\mathscr{O}^{\oplus s}_P$ by $E$. The rest of our
assertion is obvious.

Let $E$ be a $\mu$-semi-stable vector bundle on $P$ such that
$H^{0}(P, E)=0$ and $c_1(E)=0$. Since $E^{\ast}$ is $\mu$-semi-stable,
$h^{2}(P,E)=h^{0}(P,E^{\ast}(-3))=0$. By Riemann-Roch Theorem, we have
$h^{1}(P,E)=r(E)-c_2(E)=t$. Pick a basis $\{\eta_1,\ldots,\eta_t\}$
of $H^{1}(P,E)$. Then $\eta=(\eta_1,\ldots,\eta_t)$ can be regarded as
an element of $\Ext^{1}_{\mathscr{O}_P}\left(\mathscr{O}^{\oplus t}_P,
E\right)\simeq H^{1}(P,E)^{\oplus t}\cdot \eta$ defines an extension
$$
0\to E\to F\to \mathscr{O}^{\oplus t}_P\to 0,
$$
where $F$ is locally free and $r(F)=c_2(E)=c_2(F)$. $F$ is uniquely
determined by $E$ up to isomorphisms so long as
$\{\eta_1,\ldots,\eta_t\}$ is a basis of $H^{1}(P,E)$. Moreover, by
the construction of $F$ we have $H^{0}(P,F)=0$.
\enprf
\end{Proof}

\begin{dfn}\label{dfn1.5}
$V(r,n)_k$ is the set of isomorphism classes of vector bundles $E$ on
  $P$ with the following properties:
\begin{equation}\label{eqn1.5.1}
E \text{~ is~ } \mu-\text{semi-stable.}
\end{equation}
\begin{equation}\label{eqn1.5.2}
H^{0}(P,E)=0.
\end{equation}
\begin{equation}\label{eqn1.5.3}
c_1(E)=0, \ c_2(E)=n\text{~ and~ } r(E)=r.
\end{equation}
\end{dfn}

The\pageoriginale above discussion shows that if $r>n$, then
$V(r,n)_k=\phi$ and that by mapping $E$ to the extension $F$, we
obtain a map $\sigma(r,n)_k$ of $V(r,n)_k$ to $V(n,n)_k$.

\begin{dfn}\label{dfn1.6}
$V(r,n)^{s}_k$ $\left(\text{or~ } V(r,n)^{\mu}_k\right)$ is the subset
  of consisting of stable (or, $\mu$-stable, resp.) vector bundles.
\end{dfn}

\begin{lem}
(2) means that $\sigma(r,n)_k=\left(V(r,n)^{\mu}_k\right)\subset
  V(r,n)^{s}_k$. Let $M(r,n)_0$ $\left(\text{or~ }
  M(r,n)^{\mu}_0\right)$ be the moduli space of stable vector bundles
  (or, $\mu$-stable vector bundles, resp.) $E$ of rank $r$ on $P$ such
  that $c_1(E)=0$ and $c_2(E)=n$. When $k$ is algebraically closed,
  $V(r,n)^{s}_k=M(r,n)_0(k)$ and $V(r,n)^{\mu}_k=M(r,n)^{\mu}_0(k)$ as
  sets. 
\end{lem}

\begin{Prop}\label{Prop1.7}
There exists an immersion $\Psi(r,n)$ of $M(r,n)^{\mu}$ to\break $M(n,n)_0$
such that for all algebraically closed fields $K$ containing $k$,
$\sigma(r,n)_k$ is induced by $\Psi(r,n)(K)$.
\end{Prop}

\begin{Proof}
Set $M(r,n)^{\mu}_0=M_1$. There exist a principal $PGL(N)$-bundle
$p:Q\rightarrow M_1$ and a vector bundle $\widetilde{E}$ on $P\times
Q$ with $GL(N)$-linearization (\cite[Proposition 6.4]{key10}, 
and \cite{key9}, \S 4) such that for all $x$ in $Q(K)$,
$\widetilde{E}(x)=\widetilde{E}\otimes k(x)$ is the vector bundle
corresponding to the point $p(x)$. For the projection $\pi:P\times
Q\rightarrow Q$, $G=R^{1}\pi_{\ast}(\widetilde{E})$ is a vector bundle
of rank $n-r$ on $Q$. Since
$\pi_{\ast}\left(\pi^{\ast}(G^{\ast}\otimes
\widetilde{E}\right)=G^{\ast}\otimes \pi \ast
\left(\widetilde{E}\right)=0$ and
$R^{1}\pi_{\ast}\left(\pi^{\ast}(G^{\ast})\otimes
\widetilde{E}\right)=G^{\ast}\otimes
R^{1}\pi_{\ast}\left(\widetilde{E}\right)=G\otimes G^{\ast}$,
$\Ext^{1}\mathscr{O}_{P\times
  Q}\left(\pi^{\ast}(G),\widetilde{E}\right)\xleftarrow{\sim}H^{1}\left(P\times
Q,\pi^{\ast}(G^{\ast})\otimes
\widetilde{E}\right)\to H^{0}(Q,G^{\ast}\otimes G)\simeq
\Hom_{\mathscr{O}_{Q}}(G,G)$. Thus $\Ext^{1}_{\mathscr{O}_{P\times
    Q}}\left(\pi^{\ast}(G),\widetilde{E}\right)$ contains a special
element $\zeta$ which corresponds to $\id_{G}$ of
$\Hom_{\mathscr{O}_Q}(G,G)$. Let us consider the extension defined by
$\zeta$:
$$
0\to \widetilde{E}\to \widetilde{F}\to \pi^{\ast}(G)\to 0.
$$\pageoriginale

It is easy to see that for all points $x$ of $Q(K)$, $F(x)$ is
$\sigma(r,n)_K (E(x))$. Moreover, this construction of $F$ is
compatible with base changes. By the universality of $M(n,n)0$, we
have a morphism $f'$ of $Q$ to $M(n,n)_0$. The $GL(N)$- linearization
of $E$ and the compatibility with base changes imply that there is a
morphism $f$ of $M_1$ to $M(n,n)_0$ such that $fp=f'$. For $M(n,n)_0$,
we have a principal $PGL(N')$-bundle $q':R\to M(n,n)_0$ and a vector
bundle $H'$ on $P\times R'$ with the same properties as in the case of
$M_1$. Set 
$R=\{ y\in R'|
(h^{0}(P_{k(y)},\widetilde{H}', \ (y)^{\ast})=n-r$
and
$$
\widetilde{H}'(y)\widetilde{H}^{0}(P_{k(y)},H'(y)^{\ast})\otimes
\mathscr{O}_{P_{k(y)}} \text{~ is $\mu$-stable$\}$.}
$$ 
Then $R$
is locally closed in $R'$ and $PGL(N')$-invariant. Hence $M_2=q'(R)$
is locally close in $M(n,n)_0$. If we endow $R$ and $M_2$ with the
reduced structure, then we have a principal $PGL(N')$-bundle: $q:R\to
M_2$. Since $M_1$ is smooth, $f$ is a morphism of $M_1$ to $M_2$. Put
$\widetilde{H'}\mid_{P\times
  R}=\widetilde{H}$. Corollary~\ref{cor1.4.1} shows that
$\left(\widetilde{H}^{\ast}/\tau^{\ast}\tau_{\ast}(\widetilde{H}^{\ast})\right)^{\ast}=\widetilde{D}$
is a vector bundle of rank $r$ such that for all $y$ of $R(K)$, $D(y)$
is a member of $V(r,n)^{\mu}_K$, where $\tau$ is the projection of
$P\times R$ to $R$. Thus we have a morphism $g'$ of $R$ to $M_1$. By
an argument similar to the case of $f'$, we obtain a morphism of $M_2$
to $M_1$ such that $gq=g'$. Take a point $x$ of $Q$. If one looks into
the construction of $f'$ (\cite{key9}, \S\ 5), then he will find a
morphism $f''$ of an open neighbourhood $U$ of $x$ to $R$ which covers
$f$. We also get an open neighbourhood $V$ of $f''(x)$ and a morphism
$g''$ to $U$ covering $g$. Furthermore, $(1\times f'')^{\ast}(D)$ is
isomorphic to $\widetilde{E}\mid_U$ and $(1\times
g'')^{\ast}(E\mid_U)\simeq D\mid_V$. Thus $1\times
g''f'')^{\ast}\left(\widetilde{E}\mid_U\right)$ is isomorphic to $E$
in a neighbourhood $U'$ of $x$. Since $Q$ is an open set of a
Quot-scheme and $E$ is the universal quotient sheaf, the universality
of the $\left(Q,\widetilde{E}\right)$ shows that $g''f''=\id$ on $U'$
after replacing\pageoriginale $F''$ by its composition with the action
of an element 
of $PGL(N)$. We have, therefore, that
$\pi=\pi g''f''=g'F''=g\tau f''=gf'=gf\pi$. Since $\pi$ is faithfully
flat, this means that $gf=\id$. Thus $f$ is birational. Since
$M(r,n)^{\mu}_0$ is smooth and $f$ is bijective, we deduce from this
and $ZMT$ that $f$ is isomorphic.
\enprf
\end{Proof}

Each member of $V(n,n)$ has beautiful properties.

\begin{lemma}\label{lemma1.8}
If $F$ is a member of $V(n,n)$, then $F$ is $1$-regular {\em (for the
definition of regularity, see \cite{key8})}.
\end{lemma}

\begin{Proof}
Since $F$ is $\mu$-semi-stable, so is $F^{\ast}$. This and the
assumption that $c_1(F)=0$ imply that
$h^{2}(P,F(-1))=h^{0}(P,F^{\ast}(-2))=0$. Then, as we have seen in the
above, we get that $h^{1}(P,F)=c_2(F)-r(F)$ which is equal to zero by
virtue of the assumption \eqref{eqn1.5.3} for $V(n,n)$.
\enprf
\end{Proof}

\begin{cor}\label{cor1.8.1}
If we write
$F\mid_\ell=\bigoplus\limits_{i=1}^{n}\mathscr{O}_{\ell}(a_i)$ for an
$F$ in $V(n,n)$ and a line $\ell$ in $P$, then we have $a_i\geqq-1$
for all $i$.
\end{cor}

\begin{Proof}
Since $F$ is $1$-regular, $F(1)$ is generated by its global sections
and then so is $F(1)\mid_{\ell}$. Thus $a_i+1\geqq 0$ for all $i$.
\enprf
\end{Proof}

Another remarkable property of $V(n,n)$ is 

\begin{lemma}\label{lemma1.9}
If $F$ is a member of $V(n,n)$, then $F$ is semi-stable. Moreover, if
$E$ is a locally free subsheaf of $F$ with $P_E(m)=P_F(m)$, then $E$
is contained $V(t,t)$.
\end{lemma}

\begin{Proof}
Since $F$ is $\mu$-semi-stable, the subsheaf $F'$ which may disturb
the semi-stability of $F$ is of degree zero. 
We may assume that\pageoriginale $F$ is
locally free. Since $F$ is $\mu$-semi-stable and since
$H^{0}(P,F')\subset H^{0}(P,F)=0$, we see that
$c_2(F')-r(F')=h^{1}(P,F')\geqq 0$. On the other hand,
$P_{F'}(m)=m^{2}/2+3m/2-c_2(F')/r(F')+1$ and
$P_{F'}(m)=m^{2}/2+3m/2$. Thus $P_{F'}(m)\leqq
P_F(m)$ for all large $m$ and $P_{F'}(M)=P_F(m)$ if and only if
$c_2(F')=r(F')$, that is, $F$ is a member of $V(r(F'),r(F'))$.
\enprf
\end{Proof}

\begin{cor}\label{cor1.9.1}
A vector bundle $E$ on $P$ is contained in $V(n,n)$ if and only if it
has the properties \eqref{eqn1.5.3} and (1.9.2) $E$ is semi-stable.
\end{cor}

\begin{Proof}
Assume that $E$ has the properties \eqref{eqn1.5.3} and
(1.9.2). Then, obviously it has the property
\eqref{eqn1.5.1}. If $H^{0}(P,E)\neq 0$, then $\mathscr{O}_P$ is a
subsheaf of $E$ and it violates the semi-stability of $E$. Thus we
obtain the property \eqref{eqn1.5.2}. 
The converse was proved in Lemma~\ref{lemma1.9}.
\enprf
\end{Proof}

The main aim of this article is to study the following category.

\begin{dfn}\label{dfn1.10}
$\mathscr{V}$ is the full subcategory of the category of coherent
  sheaves on $P$ whose objects are vector bundles $E$ on $P$ with the
  properties \eqref{eqn1.5.2} and 
\setcounter{equation}{0}
\begin{equation}\label{eqn1.10.1}
c_2(E)=r(E),
\end{equation}
\begin{equation}\label{eqn1.10.2}
\begin{aligned}
&\text{ for general lines } \ell \text{ in }
P_{\overline{k}},E\otimes_k\overline{k}\mid_{\ell} \simeq
\mathscr{O}_{\ell}^{\oplus r(E)}\\
&ob(\mathscr{V}) \text{ is a disjoint union of } \mathscr{V}(n)=\{E\in
\mathscr{V}\mid c_{2}(E)=n\}.
\end{aligned}
\end{equation}
\end{dfn}

\section{A class of torsion sheaves on \texorpdfstring{$\mathbb{P}^{2}$}{eq}.}\label{s2}

Let $V$ be a $3$-dimensional vector space over the field $k$ which
was\pageoriginale 
fixed in Notation~\ref{notation1.2}. Then $P$ is isomorphic to
$\Proj(S(V))$. For the dual space $V^{\ast}=\Hom_k(V,k)$ of $V$, {\em we set}
$P^{\ast}=\Proj(S(V^{\ast}))$. $P^{\ast}$ is the dual plane of $P$. Let
$\mathbb{F}$ be the flag manifold which defines the incidence
correspondence between $P$ and $P^{\ast}$. Then we have the following
diagram which is the most fundamental in the following:
\setcounter{equation}{0}
\begin{equation*}
\vcenter{
\xymatrix{&\mathbb{F}\ar[dl]_{P}\ar[dr]^{q}\\
P && P^{\ast}}}\tag{2.1}\label{eqn2.1}
\end{equation*}
where $p$ (or $q$) is isomorphic to the projective bundle
$\mathbb{P}(T_P(-1))$ (or, $\mathbb{P}(T_{P^{\ast}}(-1))$,
resp.) associated with the tangent bundle $T_P$ of $P$ (or,
$T_{P^{\ast}}$ of $P^{\ast}$, resp.). $p^{\ast}(\mathscr{O}_P(1))$ (or,
$q^{\ast}(\mathscr{O}_{P^{\ast}}(1))$) is the tautological line bundle of
$T_{P^{\ast}}(-1)$ (or $T_P(-1)$, resp.).

\setcounter{dfn}{1}
\begin{notation}\label{notation2.2}
For a coherent sheaf $G$ on $\mathbb{F}$, we denote 
$$
G\otimes p^{\ast}(\mathscr{O}_P(a))\otimes
q^{\ast}(\mathscr{O}_P^{\ast}(b))\text{~ by~ } G(a,b).
$$
\end{notation}

For a coherent sheaf $L$ on $P^{\ast}$, we shall consider the
following properties:
\setcounter{dfn}{3}
\setcounter{equation}{0}
\begin{equation}\label{eqn2.3.1}
\Supp (L)\neq P^{\ast}\text{~ and~ } H^{0}(P^{\ast},L)=H^{1}(P^{\ast},L)=0.
\end{equation}
\begin{equation}\label{eqn2.3.2}
q^{\ast}\left(\ext^{1}_{\mathscr{O}_{P^{\ast}}}(L,\mathscr{O}_{P^{\ast}}(-3))\right)(1,0)
\text{ is generated by its global sections.}
\end{equation}

In the first place, let us study \eqref{eqn2.3.1}. Assume that $L$ has
the property \eqref{eqn2.3.1}. The property that $\Supp(L)\neq
P^{\ast}$ implies that\pageoriginale $H^{2}(P^{\ast},L(a))=0$ for all integers
$a$. Then, by the property that $H^{1}(P^{\ast},L)=0$, we see that $L$
is $1$-regular.

\begin{lemma}\label{lemma2.4}
If a coherent sheaf $L$ on $P^{\ast}$ has the property
\eqref{eqn2.3.1}, then $L$ is $1$-regular.
\end{lemma}

Let $c_1(L)=n$. Since $\Supp(L)\neq P^{\ast}$ and
$H^{0}(P^{\ast},L)=0$, $\Ass(L)$ consists of codimension one points and
hence $n>0$. By Riemann-Roch Theorem, we have 
$$
\chi (L)=n(n+3)/2-c_2(L)
$$
because $r(L)=0$. Our assumption \eqref{eqn2.3.1} shows that
$c_2(L)=n(n+3)/2$. Using this, it is easy to see that
$c_1(L(1))=n$ and $c_2(L(1))=\left(n^{2}+n\right)/2$. Then, by
Riemann-Roch Theorem again and by the $1$-regularity of $L$, we have 
$$
h^{0}\left(P^{\ast},L(1)\right)=\chi (L(1))=\dfrac{n(n+3)}{2}-\dfrac{(n^{2}+n)}{2}=n.
$$

\begin{Prop}\label{Prop2.5}
Let $L$ be a coherent sheaf on $P^{\ast}$. $L$ has the property
\eqref{eqn2.3.1} and $c_1(L)=n$ if and only if there is an exact sequence
\setcounter{equation}{0}
\begin{equation}\label{eqn2.5.1}
0\to\mathscr{O}_{P^{\ast}}(-1)^{\oplus
  n}\xrightarrow{\lambda}\mathscr{O}_{P^{\ast}}^{\oplus n}\to L(1)\to 0.
\end{equation}
\end{Prop}

\begin{Proof}
``If'' part is obvious because $\Supp(L)=\{\det(\lambda)=0\}$ and
 $\det(\lambda)\notequiv 0$. Conversely, by Lemma~\ref{lemma2.4},
 $L(1)$ is generated by its global sections. Then the above
 computation gives rise to a surjection
 $\mathscr{O}_{P^{\ast}}^{\oplus n}\xrightarrow{\nu}L(1)$ which
 induces an isomorphism of
 $H^{0}\left(P^{\ast},\mathscr{O}_{P^{\ast}}^{\oplus n}\right)$\pageoriginale to
 $H^{0}\left(P^{\ast},L(1)\right)$ Let $K$ be the kernal of
 $\nu$. Since depth$_{\mathscr{O}_{P^{\ast},x}}L(1)_x=1$ at every
 closed point $x$ in $\Supp(L)$, $K$ is locally free. Now let us
 consider the following exact sequence
$$
0\to K(a)\to \mathscr{O}_{P^{\ast}}(a)^{\oplus
  n}\xrightarrow{\nu(a)}L(a+1)\to 0
$$

If $a\leqq -1$, then $H^{0}\left(P^{\ast},L(a+1)\right)=0$ and hence
$H^{1}\left(P^{\ast},K(a)\right)=0$. When $a=0$, $H^{0}(\nu (a))$ is
surjective by the construction of $\nu$. Thus
$H^{1}\left(P^{\ast},K\right)=0$. Since $H^{2}\left(P^{*},K(-1)\right)
H^{1}\left(P^{\ast},L\right)=0$, we see that $K$ is $1$-regular and hence
$H^{1}(P^{\ast},K(a))=0$ for all $a\geqq 0$. By a well-known result on
vector bundles on $\mathbb{P}^{2}$, $K$ is a direct sum of line
bundles. Moreover, $H^{0}(P^{\ast},K)=0$, $r(K)=n$ and
$c_1(K)=-n$. This $K$ is isomorphic to
$\mathscr{O}_{P}\ast(-1)^{\oplus n}$.
\enprf
\end{Proof}

The $\lambda$ in \eqref{eqn2.5.1} can be represented by an $n\times
n$-matrix $\alpha(L)$ whose entries are all linear forms on $P^{\ast}$,
that is, members of $H^{0}(P^{\ast},\mathscr{O}_{P^{\ast}}(1))\cdot
\alpha(L)$ is determined up to the choice of bases of
$\mathscr{O}_{P^{\ast}}^{\oplus n}$ and
$\mathscr{O}_{P^{\ast}}(-1)^{\oplus n}$, in other words, $\alpha(L)$
and $\alpha'(L)$ represent the same $\lambda$ if and only if there are
two elements $\beta$ and $\gamma$ of $GL(n,k)$  such that
$\alpha(L)=\beta\alpha'(L)\gamma$. Thus the curve defined by
$\det(\lambda)=\det\alpha(L)=0$ is independent of the choice of the
bases of $\mathscr{O}_{P^{\ast}}^{\oplus n}$ and
$\mathscr{O}_{P^{\ast}}(-1)^{\oplus n}$. 

\begin{dfn}\label{dfn2.6}
The discriminant of $L$ is the curve (effective Cartier divisor) in
$P^{\ast}$ defined by $\det\alpha(L)=0$ and it is denoted by $S(L)$. 
\end{dfn}

The definition is justified by the following which is proved by a
simple argument in linear algebra.

\begin{lemma}\label{lemma2.7}
$L$\pageoriginale is an $\mathscr{O}_{S(L)}$-module and $\Supp(L)$ is
  equal to the  support $|S(L)|$ of the divisor $S(L)$.
\end{lemma}

The proof of the next lemma is also obvious and hence we omit it. 

\begin{lemma}\label{lemma2.8}
For an exact sequence of coherent sheaves on $P^{\ast}$ 
$$
0\to L'\to L \to L''\to 0,
$$
if two of $L'$, $L$ and $L''$ have the property \eqref{eqn2.3.1}, then
so does the third. In that case, we have that $S(L)=S(L')+S(L'')$ as
Cartier divisors on $P^{\ast}$.
\end{lemma}

Let us list some of basic results on coherent sheaves on $P^{\ast}$
with the property \eqref{eqn2.3.1}

\begin{Prop}\label{Prop2.9}
Let $L$ be a coherent sheaf on $P^{\ast}$ with the property
\eqref{eqn2.3.1}.
\begin{enumerate}
\renewcommand{\labelenumi}{(\theenumi)}
\item If $S(L)$ is smooth at $x$, then $L$ is an invertible
  $\mathscr{O}_{S(L)}$ module at $x$.
\item If $S(L)$ is a non-singular curve, then $L$ is a line bundle on
  $S(L)$ with $\deg L=g-1$, where $g$ is the genus of $S(L)$. 
\item $\hom_{\mathscr{O}_{s(L)}}(L,\omega_{S(L)})=L'$ has the property
  \eqref{eqn2.3.1}, where $\omega_{S(L)}$ is the canonical sheaf of
  $S(L)$. Moreover, $\alpha(L')=t_{\alpha}(L)$ and hence $S(L)=S(L')$.

\item \begin{equation*}
\ext^{i}_{\mathscr{O}_{P^{\ast}}}(L,\mathscr{O}_{P^{\ast}})\simeq
\begin{cases}
\hom_{\mathscr{O}_{S(L)}}(L,\mathscr{O}_{S(L)}(n)) \text{~~ if~~ } i=1\\
0\text{~~ if~~ } i\neq 1
\end{cases}
\end{equation*}\pageoriginale
where $n=c_1(L)$. Hence
$\ext^{1}_{\mathscr{O}_{P^{\ast}}}(L,\mathscr{O}_{P^{\ast}}(-3))\simeq
L'$.

\item $\ext^{i}_{\mathscr{O}_{S(L)}} (L,\mathscr{O}_{S(L)})=0$ for all
  $i>0$.

\item Let $\delta:\mathscr{O}_{S(L)}\oplus n\to L(1)$ be the
  restriction of the map $\nu:\mathscr{O}_{P^{\ast}}^{\oplus n}\to
  L(1)$ to $S(L)$ and $M=\ker(\delta)$. Then
  $\ext^{i}_{\mathscr{O}S(L)}(M,\mathscr{O}_{S(L)})=$
  $\ext^{i}_{\mathscr{O}_{S(L)}}(M^{\ast},\mathscr{O}_{S(L)})=0$ for
  all $i>0$ and
  $\ext^{1}_{\mathscr{O}_{P^{\ast}}}(M,\mathscr{O}_{P^{\ast}})\simeq
      M^{\ast}(n)$, where
      $M^{\ast}=\hom_{\mathscr{O}_{S(L)}}(M,\mathscr{O}_{S(L)})$.

\item The canonical homomorphisms 
$$
L\to (L^{\ast})^{\ast}=
  \hom_{\mathscr{O}_{S(L)}}
  (\hom_{\mathscr{O}_{S(L)}}(L,\mathscr{O}_{S(L)}),\mathscr{O}_{S(L)})
$$ 
and $M\to  (M^{\ast})^{\ast}$ are isomorphism.
\end{enumerate}
\end{Prop}

\begin{Proof}
(1) is due to Barth. In fact, by \cite{key1} Lemma 7 and
  \eqref{eqn2.5.1}, $1=\text{\rm int}_x(S(L),\ell)=h^{0}(\ell, L_{\mid_{\ell}})$
  for a general line $\ell$ passing through $x$. Thus the minimal
  number of generators of $L$ at $x$ is $1$. This holds at every
  smooth point of $S(L)$. Then, as is well-known, $L$ is invertible on
  the open set of smooth points of $S(L)$. If $S(L)$ is non-singular,
  $L$ is invertible on $S(L)$ by\pageoriginale virtue of
  $(1)$. Riemann-Roch Theorem on $S(L)$ and \eqref{eqn2.3.1} provide
  us with $\deg L-g+1=0$ which proves $(2)$. For the proof of $(4)$,
  look at the sequence \eqref{eqn2.5.1}. It supplies us with a locally
  free resolution of $L(1)$. Thus
  $$
  \ext^{i}_{\mathscr{O}_{p^{*}}}(L,\mathscr{O}_{p^{*}})=\ext^{i}_{\mathscr{O}_{P^{\ast}}}(L(1),\mathscr{O}_{P^{\ast}})(1)=0\text{~~ if~~ } i\geqq 2.
  $$ 
Since $L$ is a torsion sheaf on $P^{\ast}$,
  $\hom_{\mathscr{O}_{P^{\ast}}}(L,\mathscr{O}_{P^{\ast}})=0$. By the
  exact sequence 
$$
0\to \mathscr{O}_{P^{\ast}}\to \mathscr{O}_{P^{\ast}}(n)\to
\mathscr{O}_{S(L)}(n)\to 0,
$$
we have the exact sequence
$$
0\to \hom_{\mathscr{O}_{P^{\ast}}}(L,\mathscr{O}_{S(L)}(n))\to
\ext^{1}_{\mathscr{O}_{P^{\ast}}}(L,\mathscr{O}_{P^{\ast}})\xrightarrow{\alpha}\ext^{1}_{\mathscr{O}_{P^\ast}}(L,\mathscr{O}_{P^{\ast}}(n))  
$$

Since $\ext^{1}_{\mathscr{O}_{P^{\ast}}}(L,\mathscr{O}_{P^{\ast}})$ is an
$\mathscr{O}_{S(L)}$-module and $\alpha$ is nothing but the
multiplication by an equation of $S(L)$, $\alpha$ is zero. Therefore,
we have that $\hom_{\mathscr{O}_{S(L)}}(L,\mathscr{O}_{S(L)}(n))\simeq
\hom_{\mathscr{O}_{P^{\ast}}}(L,\mathscr{O}_{S(L)}(n))\simeq \ext^{1}_{\mathscr{O}_{P^{\ast}}}(L,\mathscr{O}_{P^{\ast}}$. By
using the spectral sequence
$E^{p,q}_2=H^{p}(P^{\ast},\ext^{q}_{\mathscr{O}_{P^{\ast}}}(L,\mathscr{O}_{P^{\ast}}(-3)))\Rightarrow
E^{p+q}=\ext^{p+q}_{\mathscr{O}P^{\ast}}(L,\mathscr{O}_{P^{\ast}}(-3)))$
and (4), we get isomorphisms
$$
H^{i}(P^{\ast}\hom_{\mathscr{O}_{S(L)}}(L,\mathscr{O}_{S(L)}(n-3)))\simeq
\ext^{i+1}(L,\mathscr{O}_{P^{\ast}}(-3)), \ i=0,1.
$$ 
On the one hand,
$O_{S(L)}(n-3)$ is isomorphic to $\omega_{S(L)}$ and on the other
hand,
$\Ext^{i+1}_{\mathscr{O}_{P^{\ast}}}(L,\mathscr{O}_{P^{\ast}}(-3))$ is
a dual space of $H^{1-i}(P^{\ast},L)$ by virtue of Serre duality. We
infer from these and \eqref{eqn2.3.1} for $L$ that
$H^{1}(P^{\ast}, L')=0$ for $i=0,1$. It is obvious that
$L'$\pageoriginale is a torsion sheaf. Thus $L$ has the
property \eqref{eqn2.3.1}. To prove 
the latter half of (3), dualizing the sequence \eqref{eqn2.5.1} and
tensoring $\mathscr{O}_{P^{\ast}}(-1)$ to it, we have the exact
sequence 
\begin{align*}
0 &\to
\mathscr{O}_{P^{\ast}}(-1)^{\oplus(n)}\xrightarrow{t\alpha(L)}\mathscr{O}^{\oplus
  n}_{P^{\ast}}\to 
\ext^{1}_{\mathscr{O}_{P^{\ast}}}(L(1),\mathscr{O}_{P^{\ast}}(-1))\\
&\to \ext^{1}_{\mathscr{O}_{P^{\ast}}}(\mathscr{O}_{P^{\ast}},\mathscr{O}_{P^{\ast}}(-1))=0.
\end{align*}
On the other hand, we derive the following isomorphisms from (4)
\begin{align*}
& \ext^{1}_{\mathscr{O}_{P^{\ast}}}(L(1),\mathscr{O}_{P^{\ast}}(-1))\simeq
\hom(L,\mathscr{O}_{S(L)}(n-3))(1)\simeq\\
&\hom_{\mathscr{O}_{S(L)}}(L,\mathscr{O}_{S(L)})(1)=L'(1).
\end{align*}
Hence we see that $\alpha(L')=t\alpha(L)$. As for (5), since the
problem is local, it is enough to show the following if (4) is taken
into account.
\enprf
\end{Proof}

\begin{lemma}\label{lemma2.10}
Let $a$ be a non-zero divisor of a commutative ring $B$,
$A=B/aB$ and $N$ an $A$-module. Then 
$$
\ext^{i}_A(N,A)\simeq \ext^{i+1}(N,B).
$$
\end{lemma}

\begin{Proof}
The exact sequence $0\to B\xrightarrow{\times a} B \to A\to 0$
supplies us with the complex 
$$
\xymatrix{0\ar[r]& \Hom_{B}(B,B) \ar@{=}[d]\ar[r]^{\times
    a}&\Hom_{B}(B,B)\ar@{=}[d]\ar[r]&0\\
& B & B &}
$$
whose\pageoriginale cohomology is $\Ext^{i}(A,B)$. Thus
$\Ext^{1}_{B}(A,B)=A$ and $\Ext^{i}_B(A,B)=0$ if $i\neq 1$. Then,
making use of the spectral sequence (\cite[p.~349]{key5})
$$
E^{p,q}_2=\Ext^{p}_A(N,\Ext^{q}_B(A,B))\Rightarrow E^{p+q}=\Ext^{p+q}_{B}(N,B),
$$
our proof is completed.

Now let us come back to the proof of Proposition~\ref{Prop2.9}. It
remains to prove (6) and (7). From the exact sequence \eqref{eqn2.5.1}
we have the following exact commutative diagram: 
$$
\xymatrix{& 0\ar[d]&0\ar[d]&\\
0 \ar[r] &\mathscr{O}_{P^{\ast}}(-n)^{\oplus n}\ar@{=}[r]\ar[d]&\mathscr{O}_{P^{\ast}}(-n)^{\oplus n}\ar[d]& \\
0\ar[r]&\mathscr{O}_{P^{\ast}}(-1)^{\oplus
    n}\ar[r]\ar[d]&\mathscr{O}_{P^{\ast}}^{\oplus
    n}\ar[r]\ar[d]&L(1)\ar[r]\ar@{=}[d]&0\\
0\ar[r]&M\ar[r]\ar[d]&\mathscr{O}_{S(L)}^{\oplus n}\ar[r]\ar[d]&L(1)\ar[r]&0\\
& 0 & 0 &}
$$
By the same argument as in the case of $L$, we see that
$$
\ext^{1}_{\mathscr{O}_{P^{\ast}}}(M,\mathscr{O}_{P^{\ast}}(-n))\simeq
M^{\ast}.
$$ 
Then the left column of the above diagram gives rise to an
exact sequence: 
$$
0\to \mathscr{O}_{P^{\ast}}(-n+1)^{n}\to \mathscr{O}^{\oplus
  n}_{P^{\ast}}\to M^{\ast}\to 0
$$

Thus\pageoriginale
$\ext^{i}_{\mathscr{O}_{P^{\ast}}}(M,\mathscr{O}_{P^{\ast}})=\ext^{i}_{\mathscr{O}_{P^{\ast}}}(M^{\ast},\mathscr{O}_{P^{\ast}})=0$
for all $i>1$. Applying Lemma~\ref{lemma2.10} to the above, (6) is
proved. By (3), (5) and (6), we have the following exact commutative
diagram:
$$
\xymatrix{0\ar[r]&M\ar[r]\ar[d]&\mathscr{O}_{S(L)}^{\oplus
    n}\ar[r]\ar[d]^{\text{\rotatebox{90}{$\backsim$}}}&L\ar[r]\ar[d]&0\\
0\ar[r]&(M^{\ast})^{\ast}\ar[r]&\mathscr{O}_{S(L)}^{\oplus n}\ar[r]&(L^{\ast})^{\ast}\ar[r]&0}
$$

Thus the canonical homomorphism $\rho:L\to (L^{\ast})^{\ast}$ is
surjective. On the other hand, the exact commutative diagram 
$$
\xymatrix{0\ar[r]&\mathscr{O}_{P^{\ast}}(-1)^{\oplus
    n}\ar@{=}[d]\ar[r]^-{\alpha(L)}&\mathscr{O}_{P^{\ast}}^{\oplus
    n}\ar@{=}[d]\ar[r]&L(1)\ar[d]\ar[r] &0\\
0\ar[r]&\mathscr{O}_{P^{\ast}}(-1)^{\oplus
    n}\ar[r]^-{{}^t({}^t\alpha(L))}&\mathscr{O}_{P^{\ast}}^{\oplus
    n}\ar[r]&\hom\mathscr{O}_{S(L)}(L',\omega_{S(L)})(1)\ar[r]&0}
$$
provides us with an isomorphism $\theta$ of $L$ to
$$
\hom_{\mathscr{O}_{S(L)}}(L',\omega_{S(L)})\simeq
\hom_{\mathscr{O}_{S(L)}}(\hom_{\mathscr{O}_{S(L)}}(L,\omega_{S(L)}),
\omega_{S(L)}\simeq (L^{\ast})^{\ast}.
$$ 
Since $L$ is a coherent sheaf
on a noetherian scheme, $\rho$ must be isomorphism. Then, the natural
homomorphism of $M$ to $(M^{\ast})^{\ast}$ is isomorphic, too. 
\enprf
\end{Proof}

\begin{dfn}\label{dfn2.11}
Assume that $L$ has the property \eqref{eqn2.3.1}. $L$ is said to be
quadratic, if there is an isomorphism $\gamma:L\to
L'=\hom_{\mathscr{O}_{S(L)}}(L,\omega_{S(L)})$. A quadratic sheaf
$(L,\gamma)$ is called symmetric (or, symplectic) if ${}^t\gamma:
(L')'=\hom_{\mathscr{O}_{S(L)}} (\hom_{\mathscr{O}_{S(L)}}(L,\omega_{S(L)}),\omega_{S(L)})\simeq
L\to L'$ is equal to $\gamma$ (or, $-\gamma$, resp.).

The following is an interpretation of the definition.
\end{dfn}

\begin{Prop}\label{Prop2.12}
An\pageoriginale $L$ is symmetric (or, symplectic) if and only if $\alpha(L)$ is
symmetric (or, skew-symmetric, resp.) with respect to suitable bases
of $\mathscr{O}_{P^{\ast}}^{\oplus n}$ and
$\mathscr{O}_{P^{\ast}}(-1)^{\oplus n}$.
\end{Prop}

\begin{Proof}
Let $H$ be the vector space $H^{0}(P^{\ast}, L(1))$. Then the exact
sequence \eqref{eqn2.5.1} can be written in the form 
\setcounter{equation}{0}
\begin{equation}\label{eqn2.12.1}
0\to \mathscr{O}_{P^{\ast}}(-1)^{\oplus n}\to
\mathscr{O}_{P^{\ast}}\otimes H\to L(1)\to 0.
\end{equation}
Dualizing this sequence and tensoring $\mathscr{O}_{P^{\ast}}(-1)$ to
it, we have 
\begin{equation}\label{eqn2.12.2}
0\to \mathscr{O}_{P^{\ast}}(-1)\otimes H^{\ast}\to \mathscr{O}^{\oplus
  n}_{P^{\ast}}\to \hom_{\mathscr{O}_{S(L)}}(L,
\omega_{S(L)})(1)=L'(1)\to 0.
\end{equation}
Identifying $L'(1)$ with $L(1)$ through $\gamma(1)$, the middle of the
above can be regarded as $\mathscr{O}_{P^{\ast}}\otimes H$. For these
bases, \eqref{eqn2.12.1} and \eqref{eqn2.12.2} turn out 
\begin{equation}\label{eqn2.12.3}
\vcenter{
\xymatrix{0\ar[r]&\mathscr{O}_{P^{\ast}}(-1)\otimes
  H^{\ast}\ar[d]_{\beta}\ar[r]&\mathscr{O}_{P^{\ast}}\otimes H\ar[d]_{\delta}\ar[r]& L(1)\ar[d]^{\gamma}\ar[r]&0\\
0\ar[r]&\mathscr{O}_{P^{\ast}}(-1)\otimes
H^{\ast}\ar[r]&\mathscr{O}_{P^{\ast}}\otimes H\ar[r]&L'(1)\ar[r]&0}}
\end{equation}
By the choice of the basis of $H$, we see that $\delta=\id$. To see
$\beta$, let us make the dual diagram of \eqref{eqn2.12.3}:
$$
\xymatrix{0\ar[r]&\mathscr{O}_{P^{\ast}}(-1)\otimes
  H^{\ast}\ar@{=}[d]\ar[r]&\mathscr{O}_{P^{\ast}}\otimes H\ar[r]&
  \ext^{1}_{\mathscr{O}_{P^{\ast}}}(L(1),\mathscr{O}_{P^{\ast}})(-1)\\
  0\ar[r]&\mathscr{O}_{P^{\ast}}(-1)\otimes
  H^{\ast}\ar[r]&\mathscr{O}_{P^{\ast}}\otimes H\ar[u]\ar[r]&
  \ext^{1}_{\mathscr{O}_{P^{\ast}}}(L'(1),\mathscr{O}_{P^{\ast}})(-1)\ar[u]_{\epsilon}}
$$
\xymatrix{\ar[r]^-{\sim}&\hom_{\mathscr{O}_{S(L)}}(L,\omega_{S(L)})(1)\ar[r]&0\\
\ar[r]^-{\sim}&\hom_{\mathscr{O}_{S(L)}}(L',\omega_{S{L}})(1)\ar[u]^{t_{\gamma}}\ar[r]&0}

\noindent
where\pageoriginale $\epsilon=\ext^{1}_{\mathscr{O}_{P^{\ast}}}(\gamma,\mathscr{O}_{P^{\ast}})(-1)$. Though
the isomorphisms $\nu$ and $\nu'$ depend on the choice of the equation
of $S(L)$, the square of $\nu$, $\nu'$, ${}^t\gamma$ and $\epsilon$ is
commutative once we fix the equation of $S(L)$. Therefore,
${}^t\beta=\id$ or $-\id$ according as $\gamma$ is symmetric or
symplectic. The converse is obvious.
\enprf
\end{Proof}

This proposition and Proposition~\ref{Prop2.5} show the following
(cf. \cite[Proposition~6.23]{key4})
\setcounter{cor}{3}
\begin{cor}\label{cor2.12.4}
Giving a non-degenerate net of quadrics is  equivalent to giving a
coherent sheaf which has the property \eqref{eqn2.3.1} and is symmetric.
\end{cor}

Now we shall study the property \eqref{eqn2.3.2}. For a coherent sheaf
$L$ on $P^{\ast}$, $L$ has the properties \eqref{eqn2.3.1} and
\eqref{eqn2.3.2} if and only if $L\otimes_{k}\overline{k}$ has
them. Thus we assume, up to Remark~\ref{Remark2.14}, that the ground
field $k$ is algebraically closed. The following shows that the
condition \eqref{eqn2.3.2} is quite mild.

\begin{Prop}\label{Prop2.13}
Assume that a coherent sheaf $L$ on $P^{\ast}$ has the property
\eqref{eqn2.3.1}. Then $L$ has the property \eqref{eqn2.3.2} if and
only if for each line $M$ contained in $S(L)$ and for each point $x$
on $M$, there exists a point $y$ on $M$ different from $x$ such that
$$
Z(y)=\left\{s\in H^{0}(P^{\ast},L'(1))\mid s(y)=0\right\}
$$ 
generates $L'(x)=L\otimes k(x)$. 
\end{Prop}

\begin{Proof}
Let $D$ be the closed set $q^{-1}\left(|S(L)|\right)$ of $F$
(see the diagram \eqref{eqn2.1}). Pick a point $z$ on $D$ and put
$x=q(z)$. There exists a unique line $M$ passing through $x$ such that
$z$ is on the minimal\pageoriginale section $\Gamma$ of
$q^{-1}(M)\simeq F_1$. For a point $y$ of $M$ different from $x$,
$H=p^{-1}pq^{-1}(y)$ is isomorphic to $F_1$ and contains the $\Gamma$
as a fibre. We have the exact sequence on $\mathbb{F}$: 
$$
0\to \mathscr{O}_{\mathbb{F}}\to \mathscr{O}_{\mathbb{F}}(1,0)\to
\mathscr{O}_{H}(0,1)\otimes \mathscr{O}_H(-E)\to 0,
$$
where $E$ is the exceptional divisor on $H\simeq F_1$, that is,
$q^{-1}(y)$. Tensoring the above sequence with $q^{\ast}(L')$, we have 
$$
0\to q^{\ast}(L')\to q^{\ast}(L')(1,0)\to q^{\ast}(L')\otimes
\mathscr{O}_{H}(0,1)\otimes \mathscr{O}_H(-E)\to 0.
$$
Note that the equation of $H$ is a non-zero divisor of $q^{\ast}(L')$
and hence $q^{\ast}(L')\to q^{\ast}(L')(1,0)$ is injective. Now we
need 
\end{Proof}

\begin{LEM}\label{LEM2.13.1}
Let $S$ be a locally noetherian scheme over a field $k$ and $\pi:X\to
S$ a $\mathbb{P}^{n}$-bundle in the category of $k$-schemes. Then, for
every coherent $\mathscr{O}_S$-module $F$ and every line bundle $L$ on
$X$, we have a natural isomorphism of
$F\otimes_{\mathscr{O}_{s}}R^{i}\pi_{\ast}(L)$ to
$R^{i}\pi_{\ast}(\pi^{\ast}(F)\otimes_{\mathscr{O}_{X}}L)$.
\end{LEM}

\begin{Proof}
Since the problem is local with respect to $S$, we may assume that
$X=S\times_k\mathbb{P}^{n}$ and $S$ is a noetherian affine scheme. Let
$\pi'$ be the second projection. Then, by the base change theorem,
$L\otimes_{\mathscr{O}_{X}}{\pi'}^{\ast}(\mathscr{O}_{\mathbb{P}^{n}}(-m))\simeq
\pi^{\ast}(N)$ for some integer $m$ and a line bundle $N$ on $S$. For
these $N$ and $m$, we have that
$\pi^{\ast}(F)\otimes_{\mathscr{O}_{X}}L\cong
\pi^{\ast}(F\otimes_{\mathscr{O}_S}N)\otimes_{\mathscr{O}_{X}}{\pi'}^{*}(\mathscr{O}_{\mathbb{P}^{n}}(m))$. Since 
$H^{j}(S,F\otimes_{\mathscr{O}_{S}}N)=0$ for all $j>0$,
K\"{u}nneth's formula provides us with an isomorphism of
$H^{0}(S,F\otimes_{\mathscr{O}_{s}}N)\otimes_k
H^{i}(\mathbb{P}^{n},\mathscr{O}_{\mathbb{P}^{n}}(m))$ to
$H^{i}(X,\pi^{\ast}(F)\otimes_{\mathscr{O}_{\chi }}L)$.\pageoriginale This
means that
$$
(F\otimes_{\mathscr{O}_{S}}N)\otimes_{\mathscr{O}_{S}}R^{i}\pi_{\ast}({\pi'}^{\ast}(\mathscr{O}_{\mathbb{P}^{n}}(m)))\xrightarrow{\sim}R^{i}\pi_{\ast}(\pi^{\ast}(F)\otimes
_{\mathscr{O}_{\chi }}L)
$$
On the other hand, by the projection formula, we have that
$$
N\otimes_{\mathscr{O}_{S}}R^{i}\pi_{\ast}({\pi'}^{\ast}(\mathscr{O}_{\mathbb{P}^{n}}(m)))\simeq
R^{i}\pi_{\ast}(\pi^{\ast}(N)\otimes_{\mathscr{O}_{\chi }}{\pi'}^{\ast}(\mathscr{O}_{\mathbb{P}^{n}}(m)))\simeq
R^{i}\pi_{\ast}(L).
$$ 
\enprf
\end{Proof}

Let us come back to the exact sequence before the lemma. Thanks to
\eqref{eqn2.3.1} for $L'$ and the above lemma, we obtain that
$H^{0}(\mathbb{F},q^{\ast}(L')(1,0))\simeq H^{0}(H\cap D,
q^{\ast}(L')\otimes \mathscr{O}_{H}(0,1)\otimes \mathscr{O}(-E))$.

\begin{case}%1
Assume that $M\notsubset S(L)$, If $y$ is chosen so that it is not a
point on $S(L)$, then $q^{\ast}(L')\otimes\mathscr{O}_{H}(0,1)\otimes
\mathscr{O}(-E)$ can be identified with $L'(1)$. Since $L'(1)$ is
generated by its global sections, $(q^{\ast}(L')\otimes
\mathscr{O}_{H}(0,1)\otimes \mathscr{O}(-E))(z)=q^{\ast}(L')(0,1)(z)$
is generated by global sections of $q^{\ast}(L')\otimes
\mathscr{O}_{H}(0,1)\otimes \mathscr{O}(-E)$ which are the same as
those of $q^{\ast}(L')(1,0)$.
\end{case}

\begin{case}%2
Assume that $M\subset S(L)$. By applying Lemma~\ref{LEM2.13.1} to the
sequence before the lemma, we have the following exact sequence 
$$
0\to L'\xrightarrow{\sigma}L'\otimes T_{P^{\ast}}(-1)\to
q_{\ast}(q^{\ast}(L')\otimes \mathscr{O}_{H}(0,1)\otimes
\mathscr{O}(-E))\to 0
$$
The map $\sigma$ is indused by
$\mathscr{O}_{P^{\ast}}\xrightarrow{\times s}T_{P^{\ast}}(-1)$ with
$s$ the section defining $H$. Therefore, the last term of the sequence
is $L'(1)\otimes m_y$, where $m_y$ is the ideal of the point $y$ in
$O_{P^{\ast}}$. Since $H^{0}(P^{\ast},L'(1)\otimes m_y)=Z(y)$, we see
that $Z(y)$ generates $L'(x)$ if and only if $q^{\ast}(L')(1,0)$ is
generated by its global sections at $z$. 

Now\pageoriginale suppose that $L$ has the properties \eqref{eqn2.3.1} and
\eqref{eqn2.3.2} and that $S(L)$ contains a line $M$. Note that $L'$
is isomorphic to
$\ext^{1}_{\mathscr{O}_{P^{\ast}}}(L,\mathscr{O}_{P^{\ast}}(-3))$ by
Proposition~\ref{Prop2.9}, (3) and (4). Pick a point $x$ on $M$ and
let $z$ be the point on the minimal section of $q^{-1}(M)\cong F_1$
and on the fibre $q^{-1}(x)$. Applying the case II above to this
situation we see that $L$ must have the property of our proposition
for this $M$. Conversely, assume that the property of the proposition
holds for an $L$ with the property \eqref{eqn2.3.1}. Then, combining
the cases I and case II, it is easily seen that $L$ has the property \eqref{eqn2.3.2}.
\end{case}

An interpretation of the above proposition through $\alpha(L)$ is the
following.

\begin{cor}\label{cor2.13.2}
Let $L$ be a coherent sheaf on $P^{\ast}$ with the property
\eqref{eqn2.3.1} and $H'=H^{0}(P^{\ast},L'(1))$. Let us consider the
sequence in Proposition~\ref{Prop2.5} for $L'$
$$
0\to \mathscr{O}_{P^{\ast}}(-1)\otimes_k H\to
\mathscr{O}_{P^{\ast}}\otimes_k H'\to L'(1)\to 0
$$
where $H$ is a vector space of dimension $n$. Then $L$ has the
property \eqref{eqn2.3.2} if and only if the following holds: 
\end{cor}

\setcounter{subsection}{13}
\setcounter{subsubsection}{2}
\subsubsection{}\label{eqn2.13.3}
for each line  $M$ in $S(L)$, there is a couple of
points $x$ and $y$ of $M$ such that $H'=\alpha(x)(H)+\alpha(y)(H)$,
where for a point $z$ of $M$, $\alpha(z)=\alpha(L')\otimes k(z):H\to
H'$.

\begin{Proof}
In the first place, let us note the condition \ref{eqn2.13.3} is\pageoriginale
independent of the choice of the couple $x$ and $y$. In fact, if we
take another couple $(\lambda x+\mu y, \lambda'x+\mu'y)$, then
$\alpha(\lambda x+\mu y)=\lambda \alpha(x)+\mu \alpha(y)$  and
$\alpha(\lambda'x+\mu'y)=\lambda'\alpha(x)+\mu'\alpha(y)$. Let $h$ be
an element of $H'$. By \ref{eqn2.13.3}, there are $u$ and $v$ in $H$
such that $h=\alpha(x)(u)+\alpha(y)(v)$. For the matrix
$\left[\begin{smallmatrix}
t & s\\
t'& s'
\end{smallmatrix}\right]=\left[\begin{smallmatrix}
\lambda &\lambda'\\
\mu & \mu'
\end{smallmatrix}\right]^{-1}$, take $u'=tu+sv$ and $v'=t'u+s'v$. Then
$\alpha(\lambda x+\mu y) (u')+\alpha(\lambda'x+\mu'y)(v')=h$ as
required. Now assume that the condition in Proposition~\ref{Prop2.13}
holds. The exact sequence 
$$
H\xrightarrow{\alpha(y)} H'\to L'(1)\otimes k(y)\to 0
$$
shows that $\alpha(y)(H)=Z(y)$. Thus every element of $H'$ is
contained in $Z(y)$ module $\alpha(x)(H)$ by the condition of
Proposition~\ref{Prop2.13}. This is \ref{eqn2.13.3}. Conversely, it
is obvious that \ref{eqn2.13.3} implies that the condition of
Proposition~\ref{Prop2.13} for the given $x$ in \ref{eqn2.13.3} is
satisfied. On the other hand, as we have seen in the above,
\ref{eqn2.13.3} is independent of the choice of $(x,y)$.

Thus our proof is completed.
\enprf
\end{Proof}

\setcounter{cor}{3}
\begin{cor}
If the condition of Proposition \ref{Prop2.13} is satisfied by a point
an a line $M$ in $S(L)$, then it is satisfied by all the points on $M$.
\end{cor}

\begin{Remark}\label{Remark2.14}
It is easy to see that \ref{eqn2.13.3} corresponds to the $(\alpha 2)$
in \cite{key2}.

Now let us introduce a subcategory of the category of coherent\\ sheaves
on $P^{\ast}$.
\end{Remark}

\begin{dfn}\label{dfn2.15}
Let\pageoriginale $\mathscr{C}$ be the full subcategory of the
category of coherent sheaves on $P^{\ast}$ whose objects are coherent
sheaves with properties \eqref{eqn2.3.1}
and \eqref{eqn2.3.2}. ob($\mathscr{C}$) is a disjoint union of
$\mathscr{C}(n)$, where $\mathscr{C}(n)=\left\{L\in \mathscr{C}\mid
c_1(L)=n\right\}$.  
\end{dfn}

Our main result of this article is the following.

\begin{MT}\label{MT2.16}
For a member $F$ of $\mathscr{V}(n)$,
$R^{1}q_{\ast}(p^{\ast}(F)(-1,-1))$ is contained in $\mathscr{C}(n)$.

(2) The functor $\Phi:\mathscr{V}\ni F\to
R^{1}q_{\ast}(p^{\ast}(F)(-1,-1))\in \mathscr{C}$ gives rise to an
equivalence of categories of $\mathscr{V}$ to $\mathscr{C}$. 
\end{MT}

\begin{EXP}\label{EXP2.17}
Let $L$ be a coherent sheaf on $P^{\ast}$ with the property
\eqref{eqn2.3.1} and with $c_1(L)=1$. Then $S(L)$ is a line $M$ in
$P^{\ast}$ and $L\simeq \mathscr{O}_M(-1)$. Since $L'(1)\simeq
\mathscr{O}_M$, $L$ does not have the property
\eqref{eqn2.3.2}. Therefore, $\mathscr{C}(1)=\phi$. On the other hand,
$\mathscr{V}(1)$ is empty, too.
\end{EXP}

\section{From \texorpdfstring{$\mathscr{V}$}{eq} to \texorpdfstring{$\mathscr{C}$}{eq}.}\label{s3}

In this section, we shall study $R^{1}q_{\ast}(p^{\ast}(F)(-1,-1))$
for members $F$ of $\mathscr{V}$ and prove that (1) of our Main
Theorem holds and that $\Phi$ is fully faithful. Let $F$ be a vector
bundle on $P=\mathbb{P}^{2}_k$ with the properties \eqref{eqn1.5.2},
\eqref{eqn1.10.2} and 
$$
\eqref{eqn1.10.1}_{n} \ \ c_2(F)=r(F)=n, 
$$
that is, $F$ is a member of $\mathscr{V}(n)$. Take a general line
$\ell$ in $P$ so that\pageoriginale 
$F\mid_{\ell}\cong \mathscr{O}_{\ell}^{\oplus
  n}$. Then we have the following exact sequence: 
$$
0\to F(-1)\to F\to \mathscr{O}_{\ell}^{\oplus n}\to 0
$$
This supplies us with another exact sequence: 
$$
\begin{aligned}
&0\to q_{\ast}p^{\ast}(F(-1))\to q_{\ast}p^{\ast}(F)\to
\mathscr{O}_{P^{\ast}}^{\oplus n} \\
&{}\to R^{1}q_{\ast}(p^{\ast}(F(-1)))\to R^{1}q_{\ast}(p^{\ast}(F)), 
\end{aligned}
$$
where $p$ and $q$ are the same as in the diagram \eqref{eqn2.1}. The
leftmost term is, on one hand, torsion free because $F(-1)$ is so. On
the other hand, it is torsion, thanks to the property
\eqref{eqn1.10.2}. Hence it vanishes. Since $F$ is locally free,
$P^{\ast}$ is smooth and since $\dim P^{\ast}=2$,
$G=q_{\ast}p^{\ast}(F)$ is locally free. By \eqref{eqn1.10.2} again,
$R^{1}q_{\ast}(p^{\ast}(F(-1)))$ is torsion and hence $r(G)=n$. By
virtue of Corollary~\ref{cor1.8.1} and the base change theorem, we see
that $R^{1}q_{\ast}(p^{\ast}(F))=0$. Putting $L=\Phi(F)$, we have the
following exact sequence: 
$$
0\to G\to \mathscr{O}^{\oplus n}_{P^{\ast}}\to L(1)\to 0.
$$

\begin{lemma}\label{lemma3.1}
\begin{enumerate}
\renewcommand{\labelenumi}{(\theenumi)}
\item $L$ has the property \eqref{eqn2.3.1}.
\item $G\simeq \mathscr{O}_{P^{\ast}}(-1)^{\oplus n}$.
\end{enumerate}
\end{lemma}

\begin{Proof}
~Since $H^{0}(P^{\ast}, G)\simeq H^{0}(\mathbb{F}, p^{\ast}(F))\simeq
H^{0}(P,F)$, we see that\break $H^{0}(P^{\ast},G)=0$, by the property
\eqref{eqn1.5.2} for $F$. $H^{1}(P^{\ast},G)$ is a subspace of
$H^{1}(\mathbb{F},p^{\ast}(F))$ by a spectral sequence of Leray. On
the other hand, we have an exact sequence 
$$
H^{1}(P,F)\to H^{1}(\mathbb{F},p^{\ast}(F))\to H^{0}\left(P,R^{1}\mid p_{\ast}(p^{\ast}(F))\right).
$$
\pageoriginale
Since $p$ is a $\mathbb{P}^{1}$-bundle, we have that
$R^{1}p_{\ast}(p^{\ast}(F))=0$. By virtue of Lemma~\ref{lemma1.8},
$H^{1}(P,F)$ must vanish. These show that $H^{1}(P^{\ast},G)=0$. Thus
the map $H^{0}\left(P^{\ast},\mathscr{O}_{P^{\ast}}^{\oplus
  n}\right)\to H^{0}\left(P^{\ast},L(1)\right)$ is bijective. Since
$H^{0}(P^{\ast},R^{1}q_{\ast}(p^{\ast}(F)(0,-1))=0$ by
Corollary~\ref{cor1.8.1}, $H^{2}(P^{\ast},G(-1))$ is a subspace of
$H^{2}(\mathbb{F},p^{\ast}(F)(0,-1))$ by a spectral sequence of
Leray. As is easily seen, $R^{1}p^{\ast}(p^{\ast}(F)(0, -1))=0$ for
all $i$. Therefore, 
$$
H^{2}(\mathbb{F},p^{\ast}(F)(0,-1))=0
$$
and hence
$H^{2}(P^{\ast},G(-1))=0$. Similarly we have that
$H^{1}(P^{\ast}G(-1))=0$. Since $L$ is a torsion sheaf, the proof of
(1) is completed. The proof of (2) is completely the same as that of
Proposition~\ref{Prop2.5}, because the homomorphism of
$H^{0}(P^{\ast},\mathscr{O}^{\ast n}_{P^{\ast}})$ to
$H^{0}(P^{\ast},L(1))$ is isomorphic.
\enprf
\end{Proof}

The natural homomorphism of $q^{\ast}(G)$ to $p^{\ast}(F)$ is
generically isomorphic because of the property \eqref{eqn1.10.2} of
$F$ and the base change theorem. Since $G$ is locally free, the map is
injective; 
%\setcounter{dfn}{0}
\setcounter{equation}{1}
\begin{equation*}
0\to q^{\ast}(G)\to p^{\ast}(F)\to A\to 0.\tag{3.2}\label{eqn3.2}
\end{equation*}

Let us determine $A=\coker (q^{\ast}(G)\to p^{\ast}(F))$. If
$\ell$ is a sufficiently general line in $P$, then for
$p^{-1}(\ell)=H\simeq F_1$, $H$ is isomorphic to $P^{\ast}$ on
$\Supp(L)$ and $H\cap\Ass (A)=\phi$. We have the following exact
commutative diagram:
$$
\xymatrix{& 0\ar[d] & 0\ar[d] & 0\ar[d] &\\
0\ar[r]&q^{\ast}(G)(-1,0)\ar[d]\ar[r]&p^{\ast}(F(-1))\ar[d]\ar[r]&A(-1,0)\ar[d]\ar[r]&0\\
0\ar[r]&q^{\ast}(G)\ar[d]\ar[r]&p^{\ast}(F)\ar[d]\ar[r]&A\ar[d]\ar[r]&0\\
0\ar[r]&q^{\ast}(G)\mid_{H}\ar[d]\ar[r]&p^{\ast}(F)\mid_{H}\ar@{=}[d]\ar[r]&A\mid_{H}\ar[d]\ar[r]&0\\
& 0 &p^{\ast}(F\mid_{\ell})\ar[d]& 0 &\\
& & 0 & &}
$$\pageoriginale
Taking the direct image of the above by $q$, another exact commutative
diagram is obtained: 
$$
\xymatrix@=.55cm{&0\ar[d]& 0 \ar[d]& 0\ar[d]\\
0\ar[r]&G\ar[d]\ar[r]^-{\sim}&G=q_{\ast}p^{\ast}(F)\ar[d]\ar[r]&q_{\ast}(A)\ar[d]\ar[r]&0\\
0\ar[r]&G\ar[d]\ar[r]&\mathscr{O}_{P^{\ast}}^{\oplus
  n}\ar[d]\ar[r]^-{\sim}&q_{\ast}\left(A\mid_{H}\right)\simeq
A\mid_H\ar[d]\ar[r]&0\\
& 0\ar[r]&L(1)\ar[d]\ar[r]&R^{1}q_{\ast}(A(-1,0))\ar[d]\ar[r]&0\\
& & 0=R^{1}q_{\ast}p^{\ast}(fF)\ar[r]&R^{1}q_{\ast}(A)\ar[r]& R^{1}q_{\ast}q^{\ast}(G)=0}
$$
From the top row, we deduce that $q_{\ast}(A)=0$. We infer from the
bottom row that $R^{1}q_{\ast}(A)=0$. Thus we obtain an isomorphism 
\begin{equation*}\label{eqn3.1.3}
L(1)\cong A\mid_{H}.\tag{3.3}
\end{equation*}\pageoriginale
Note that we can regard $A\mid_H$ as a coh\d{e}rent sheaf on
$P^{\ast}$ because 
$$
\Supp(A)\subseteq q^{-1}(\Supp(L))
$$ 
and $H$ is
isomorphic to $P^{\ast}$ on $\Supp(L)$. Abusing the notation as in
\eqref{eqn3.1.3}, the following sequence is exact; 
$$
0\to A\to A(1,0)\to L(1)\otimes p^{\ast}(\mathscr{O}_P(1))\to 0
$$
on the support of $L(1)=A\mid_{H}$,
$p^{\ast}(\mathscr{O}_P(1))=\mathscr{O}_{P^{\ast}}(1)$. Taking this
into account, let us make the direct image by $q$ of the above
sequence; 
$$
0\to q_{\ast}(A)\to q_{\ast}(A(1,0))\to L(2)\to R^{1} q_{\ast}(A).
$$
Since \ $q_{\ast}(A)=R' q_{\ast}(A)=0$, we see 
\begin{equation*}
q_{\ast}(A(1,0))\simeq L(2).\tag{3.4}\label{eqn3.1.4}
\end{equation*}

From another exact sequence 
$$
0\to q^{\ast}(G)(1,0)\to P^{\ast}(F)(1,0)\to A(1,0)\to 0
$$
we have 
$$
0\to G\otimes T_{P^{\ast}}(-1)\to q_{\ast}p^{\ast}(F(1))\to
q_{\ast}(A(1,0))\to R' q_{\ast}(q^{\ast}(G)(1,0))
$$

Obviously the last term $R^{1}q_{\ast}(q^{\ast}(G)(1,0))$ is
zero. Putting the above\pageoriginale together, we have the following exact
commutative diagram: 
\begin{equation*}
\vcenter{
\xymatrix{0\ar[r]&q^{\ast}(G\otimes
  T_{P^{\ast}}(-1))\ar[d]_{u}\ar[r]&q^{\ast}q_{\ast}p^{\ast}(F(1))\ar[d]_{\nu}\ar[r]&\\
0\ar[r]&q^{\ast}(G)(1,0)\ar[r]&p^{\ast}(F(1))\ar[r]&}}\tag{3.5}\label{eq3.5}
\end{equation*}
$$
\xymatrix{q^{\ast}q_{\ast}(A(1,0)\ar[d]_{\omega}\ar[r]&0\\
A(1,0)\ar[r]&0}
$$
By the base change theorem $u$ is surjective. Thanks to
Corollary~\ref{cor1.8.1} and the base change theorem again we see that
$\nu$ is surjective, too. Thus $\omega$ is surjective. Setting
$K'=\ker(u)$, $K=\ker(\nu)$ and $K''=\ker(\omega)$, we get the
following exact sequence by the snake lemma: 
$$
0\to K'\xrightarrow{\zeta}K\to K''\to 0.
$$
Since $r(q^{\ast}(G\otimes
T_{P^{\ast}}(-1)))=2n=r(q^{\ast}q_{\ast}p^{\ast}(F(1)))$ and since both
$K'$ and $K$ are locally free, we know that $\Supp(K'')$ is of pure
codimension $1$, in fact, $\Supp(K'')$ is the divisor defined by
$\det(\zeta)=0$. The first Chern class of $q_{\ast}p^{\ast}(F(1))$
is equal to zero by the top row of \eqref{eq3.5} because $c_1(G\otimes
T_{P^{\ast}}(-1))=-n$ and $c_1(q_{\ast}(A(1,0)))=c_1(L(2))=n$ by
\eqref{eqn3.1.4}. Then we have
\begin{align*}
c_1(K') &
=c_1(\mathscr{O}_{\mathbb{F}}(0,-n))-c_1(\mathscr{O}_{\mathbb{F}}(n,-n))=c_1(\mathscr{O}_{\mathbb{F}}
(n,0))\quad\text{and}\\
c_1(K) &=0-c_1(\mathscr{O}_{\mathbb{F}}(n,0))=c_1(\mathscr{O}_{\mathbb{F}}(n,0)).
\end{align*}
\pageoriginale
These imply that $c_1(K'')=0$, which means that $K''=0$. We have therefore
\setcounter{dfn}{5}
\begin{Prop}\label{Prop3.6}
For a vector bundle $F$ in $\mathscr{V}(n)$, there exists an exact
sequence 
$$
0\to \mathscr{O}_{\mathbb{F}}(0,-1)^{\oplus n}\to
p^{\ast}(F)\xrightarrow{\pi}q^{\ast}(L)(-1,2)\to 0, 
$$
where~~ $L=R^{1}q_{\ast}(p^{\ast}(F)(-1))$.
\end{Prop}

An obvious corollary to the above is 

\begin{cor}\label{cor3.6.1}
$p_{\ast}(\pi)$ is an isomorphism of $F$ to
$p_{\ast}(q^{\ast}(L)(-1,2))$. 
\end{cor}

For the above $L$, let us put $C=S(L)$ and $D=q^{-1}(C)$. $C$ and $D$
are effective Cartier divisors on $P^{\ast}$ and $\mathbb{F}$,
respectively. Then the exact sequence in Proposition~\ref{Prop3.6} is
displayed in the following exact commutative diagram:
\setcounter{equation}{6}
\begin{equation*}
\vcenter{
\xymatrix@C=.5cm{&0 & 0&\\
0\ar[r]&N\ar[u]\ar[r]&p^{\ast}(F)\mid_{D}\ar[u]\ar[r]&q^{\ast}(L)(-1,2)\ar[r]&0\\
0\ar[r]&\mathscr{O}_{\mathbb{F}}(0,-1)^{\oplus
  n}\ar[u]^{\nu}\ar[r]&p^{\ast}(F)\ar[u]\ar[r]&q^{\ast}(L)(-1,2)\ar@{=}[u]\ar[r]&0\\
&p^{\ast}(F)(0,-n)\ar[u]\ar@{=}[r]& p^{\ast}(F)(0,-n)\ar[u]&&\\
&0\ar[u]& 0\ar[u] &&}}\tag{3.7}\label{eqn3.6.7}
\end{equation*}

The\pageoriginale canonical homomorphism
$$
q_C^{\ast}(\ext^{i}_{\mathscr{O}_C}(L,\mathscr{O}_C))\to
\ext^{i}_{\mathscr{O}_{D}}(q^{\ast}(L),\mathscr{O}_D)
$$ 
is an
isomorphism, thanks to the flatness of $q_C:D\to C$, and hence
$\ext^{i}_{\mathscr{O}_D}(q^{\ast}(L)(-1,2),\mathscr{O}_{D})=0$ for all
$i>0$. Indeed,
\begin{align*}
\ext^{i}_{\mathscr{O}_{D}}(q^{\ast}(L)(-1,2),\mathscr{O}_D) &\simeq
\ext^{i}_{\mathscr{O}_{D}}(q^{\ast}(L), \mathscr{O}_D)(1,-2)\\
&\simeq
q_C^{*}(\ext^{i}_C(L,\mathscr{O}_c))(1,-2)=0
\end{align*}
by Proposition~\ref{Prop2.9},
(5) because, as we have seen in Lemma~\ref{lemma3.1}, $L$ has the
property \eqref{eqn2.3.1}. We have therefore 
\begin{equation}
\ext^{i}_{\mathscr{O}_D}(N,\mathscr{O}_D)=0\text{~ for all~ } i>0.\tag{3.8}\label{eqn3.6.8}
\end{equation}
We shall determine $\ker(\nu_D)=T$, where $\nu_D$ is the restriction
of to $D$. The restriction of \ref{eqn3.6.7} to $D$ gives rise to the
following exact commutative diagram; 
\begin{equation*}
\xymatrix{& 0\ar[r]&N\ar[r]&p^{\ast}(F)\mid_D\\
\ar[r]&\Tor_1\mathscr{O}_{\mathbb{F}}(q^{\ast}(L)(-1,2),\mathscr{O}_D)\ar[r]&\mathscr{O}_D(0,-1)^{\oplus
  n}\ar[u]^{\nu_D}\ar[r]^-{\theta} & p^{\ast}(F)\mid_D\ar@{=}[u]}
\end{equation*}
From the diagram, we can deduce clearly that
$T=\ker(\nu_D)=r(\theta)=\Tor_1\mathscr{O}_{\mathbb{F}}(q^{\ast}(L)(-1,2),\mathscr{O}_D)$. On the other hand, the solution of $\mathscr{O}_D$ by locally free
sheaves 
$$
0\to \mathscr{O}_{\mathbb{F}}(-D)\simeq
\mathscr{O}_{\mathbb{F}}(0,-n)\xrightarrow{\delta}\mathscr{O}_{\mathbb{F}}\to
\mathscr{O}_D\to 0
$$
provides us with an isomorphism $\Tor_{1}\mathscr{O}_{\mathbb{F}}(q^{\ast}(L)(-1,2),\mathscr{O}_D)\simeq 
  (1\otimes \delta:q^{\ast}(L)(-1,2-n)\to q^{\ast}(L)(-1,2))$. Since
  $\delta$ is the multiplication by the local equation of $D$ which
  annihilates $L$), we know that $1\otimes \delta=0$, whence we have
  the following exact sequence: 
\begin{equation}
0\to q^{\ast}(L)(-1,2-n)\to \mathscr{O}_D(0,-1)^{\oplus n}\to N \to 0.\tag{3.9}
\label{eqn3.6.9}
\end{equation}
\pageoriginale
Taking the dual of the above sequence and tensoring with
$\mathscr{O}_D(0,-1)$, we get 
$$
\begin{aligned}
&0\to \hom_{\mathscr{O}_D}(N,\mathscr{O}_D(0,-1))=N^{\ast}(0,-1)\to
\mathscr{O}^{\ast n}_{D}\to\\
&\hom_{\mathscr{O}_D}(q^{\ast}(L),\mathscr{O}_D)(1,n-3)\to \ext^{1}_{\mathscr{O}_D}(N,\mathscr{O}_D)(0,-1).
\end{aligned}
$$

By \eqref{eqn3.6.8} and the fact that $\mathscr{O}_D(0,n-3)\simeq
q^{\ast}(\omega_C)$, we have the exact sequence 
\begin{equation}
0\to N^{\ast}(0,-1)\to \mathscr{O}^{\oplus n}_D\to q^{\ast}(L)(1,0)\to 0.\tag{3.10}\label{eqn3.6.10}
\end{equation}
In fact, the natural homomorphism 
$$
q^{\ast}(L)\simeq
q^{\ast}(\hom_{\mathscr{O}_C}(L,\omega_C))\to
\hom_{\mathscr{O}_D}(q^{\ast}(L),q^{\ast}(\omega_C))
$$ 
is isomorphic because of the flatness of $q_C$. Now we have 

\setcounter{dfn}{10}
\begin{Prop}\label{Prop3.11}
$L=R'q_{\ast}(p^{\ast}(F)(-1,-1))$ enjoys the property \ref{eqn2.3.2}.
\end{Prop}

\begin{Proof}
Since $\ext^{1}_{\mathscr{O}_{P^{\ast}}}(L,\mathscr{O}_{p^{*}}(-3))\simeq L'$ by
Proposition~\ref{Prop2.9}, (4), our assertion is obvious if one looks
at the exact sequence \eqref{eqn3.6.10}.

Combining Proposition~\ref{Prop3.6} with Proposition~\ref{Prop3.11},
we have a part of our Main Theorem.
\enprf
\end{Proof}

\begin{cor}\label{cor3.11.1}\pageoriginale
\begin{enumerate}
\renewcommand{\labelenumi}{(\theenumi)}
\item For a member $F$ of $\mathscr{V}(n)$,
  $R^{1}q_{\ast}(p^{\ast}(F)(-1,-1)$ is contained in
  $\mathscr{C}(n)$. 

\item The functor $\Phi$ in Main Theorem~\ref{MT2.16} is fully
  faithful. 
\end{enumerate}
\end{cor}

\begin{Proof}
\begin{enumerate}
\renewcommand{\labelenumi}{(\theenumi)}
\item is done Lemma~\ref{lemma3.1} and Proposition~\ref{Prop3.11}. Let
  $F_1$ and $F_2$ be two objects in $\mathscr{V}$ and set $L_i=\Phi
  (F_i)$. For a given homomorphism $f$ of $F_1$ to $F_2$, the exact
  sequences for $F_1$ and $F_2$ in Proposition~\ref{Prop3.6} provide
  us with the following commutative diagram: 
\setcounter{equation}{1}
\begin{equation*}
\vcenter{
\xymatrix{p^{\ast}(F_1)\ar[d]^{p^{\ast}(f)}\ar[r]^-{\pi_1}&q^{\ast}(L_1)(-1,2)\ar[d]^{\psi_{f}(-1,2)}\ar[r]&0\\
p^{\ast}(F_2)\ar[r]_-{\pi_2}&q^{\ast}(L_2)(-1,2)\ar[r]&0}}\tag*{(3.11.2)$_f$}\label{eqn3.11.2}
\end{equation*}
because the kernel of $\pi_i$ is $q_{\ast}p^{\ast}(F_i)$, where
$\psi_f$ is a homomorphism of $q^{\ast}(L_1)$ to
$q^{\ast}(L_2)$. Tensoring the above diagram with
$\mathscr{O}_F(-1,-1)$ and applying $R^{1}q_{\ast}$ to it, we get 
\setcounter{equation}{2}
\begin{equation}\label{eqn3.11.3}
\vcenter{
\xymatrix{\Phi(F_1)\ar[d]^{\Phi(f)}\ar[r]&L_1\otimes
R^{1}q_{\ast}(\mathscr{O}_{\mathbb{F}}(-2,1))\ar[d]^{\xi_f}\\
\Phi(F_2)\ar[r]& L_2\otimes R^{1}q_{\ast}(\mathscr{O}_{\mathbb{F}}(-2,1))}}
\end{equation}
by Lemma~\ref{LEM2.13.1}. Since
$R^{1}q_{\ast}(\mathscr{O}_{\mathbb{F}}(-2,1))\simeq
\mathscr{O}_{P^{\ast}}$, $\xi_f=q_{\ast}(\psi_f)$ and
$\psi_f=q^{\ast}(\xi_f)$. Now assume that for two elements $f$ and $g$
of $\Hom_{\mathscr{V}}(F_1,F_2)$,\pageoriginale we have $\Phi(f)=\Phi(g)$. Since 
\eqref{eqn3.11.3} is canonical, we get $\xi_f=\xi_g$. If one takes the
direct images of \ref{eqn3.11.2} and (3.11.2)$_g$ by
$p$, he obtains that $f=p^{\ast}(\pi_2)^{-1}
p_{\ast}(\psi_f(-1,2))\cdot p_{\ast}(\pi_1)=p_{\ast}(\pi_2)^{-1}\cdot
p_{\ast}(\psi_g(-1,2))\cdot p_{\ast}(\pi_1)=g$ because
$\psi_f=q^{\ast}(\xi_f)=q^{\ast}(\xi_g)=\psi_g$. Thus
$$
\Hom_{\mathscr{V}}(F_1,F_2)\to\Hom_{\mathscr{C}}(\Phi(F_1),\Phi(F_2))
$$
is injective. To prove the  
surjectivity of the map, let us pick a member $\xi$ of
$\Hom_{\mathscr{C}}(\Phi (F_1),\Phi(F_2))$ and set
$\psi=q^{\ast}(\xi)(-1,2)$. By Corollary~\ref{cor3.6.1},
$p^{\ast}(\pi_i)$ is an isomorphism of 
$$
p_{\ast}p^{\ast}(F_i)\text{~~ to~~ }
p_{\ast}(q^{\ast}(L_i)(-1,2)).
$$ 
Identifying $F_i$ with
$p_{\ast}p^{\ast}(F_i)$, set $f_{\xi}p_{\ast}(\pi_2)^{-1}\cdot
p_{\ast}(\psi)\cdot p_{\ast}(\pi_1)$. Then $f_{\xi}$ is an element of
$\Hom(F_1,F_2)$ and we have the following commutative diagram: 
$$
\xymatrix{&&p^{\ast}p_{\ast}(q^{\ast}(L_1)(-1,2))\ar[dl]^{k_{1}}\ar[ddd]^{p^{\ast}p_{\ast}(\psi)}\\
p^{\ast}(F_1)\ar[urr]_{\sim}^{p^{\ast}p_{\ast}(\pi_1)}\ar[r]_-{\pi_1}&q^{\ast}(L_1)(-1,2)\ar[d]^{\psi}&\\
p^{\ast}(F_2)\ar[drr]^{\sim}_{p^{\ast}p_{\ast}(\pi_2)}\ar[r]^-{\pi_2}&q^{\ast}(L_2)(-1,2)&\\
&&p^{\ast}p_{\ast}(q^{\ast}(L_2)(-1,2))\ar[ul]_{k_{2}}}
$$
Since $p^{\ast}(f_{\xi})=p^{\ast}p_{\ast}(\pi_{2})^{-1}\cdot
p^{\ast}p_{\ast}(\psi)\cdot p^{\ast}p_{\ast}(\pi_1)$, we see that
$\pi_2\cdot p^{\ast}(f_{\xi})=k_2\cdot
p^{\ast}p_{\ast}(\psi)\cdot p^{\ast}p_{\ast}(\pi_1)=\psi\cdot k_1\cdot
p^{\ast}p_{\ast}(\pi_1)=\psi\cdot\pi_1$. Then, as before, we have that
$\Phi(f_{\xi})=\xi$. 
\end{enumerate}
\enprf
\end{Proof}

\section{From \texorpdfstring{$\mathscr{C}$}{eq} to \texorpdfstring{$\mathscr{V}$}{eq}}\label{s4}

The remaining part of our proof of the Main Theorem is that if $L$ is
an object of $\mathscr{C}$, then there is an $F$ in $\mathscr{V}$ such
that $\Phi(F)$ is isomorphic\pageoriginale to $L$. By the
property \eqref{eqn2.3.2} for $L$ and Proposition~\ref{Prop2.9}, (4),
$q^{\ast}(L)(1,0)$ is 
generated by its global sections, where
$L'=\hom_{\mathscr{O}_{S(L)}}(L,\omega_{S(L)})$. Set $S(L)=C$ and
$q^{-1}(C)=D$ as in the preceding section. Assume that $L$ is a member
of $\mathscr{C}(n)$. Since $q^{\ast}(q_{\ast}(L')(1,0))\simeq L'\otimes
T_{P^{\ast}}(-1)$ by Lemma~\ref{LEM2.13.1}, we have that
$h^{0}(\mathbb{F},q^{\ast}(L')(1,0))=h^{0}(P^{\ast},L'\otimes
T_{P^{\ast}}(-1))$. By the exact sequence
$$
0\to L'\to {L'}^{\oplus 3}\to L'\otimes T_{P^{\ast}}(1)\to 0
$$
we see that $h^{0}(P^{\ast},L'\otimes
T_{P^{\ast}}(-1))=h^{1}(P^{\ast},L'(-1))$ because $L'$ has the property
\eqref{eqn2.3.1} by Proposition~\ref{Prop2.9}, (3). On the other hand,
the exact sequence 
$$
0\to \mathscr{O}_{P^{\ast}}(-3)^{\oplus n}\to
\mathscr{O}_{P^{\ast}}(-2)^{\oplus n}\to L'(-1)\to 0
$$
(Proposition~\ref{Prop2.9}, (3) and Proposition~\ref{Prop2.5}) implies
that $h^{1}(P^{\ast},
L'(-1))=h^{2}(P^{\ast},\mathscr{O}_{P^{3}}(-3)^{\oplus
  n})=n$. Thus we obtain that $h^{0}(\mathbb{F},q^{\ast}(L')(1,0))=n$.

Let $N$ be the kernal of a homomorphism $\mathscr{O}_{D}^{\oplus
  n}\xrightarrow{\tau} q^{\ast}(L')(1,0)$ such that $H^{0}(\tau)$
is isomorphic: 
\setcounter{equation}{0}
\begin{equation*}\label{eqn4.0.1}
0\to N\to \mathscr{O}^{\oplus n}_D\xrightarrow{\tau}
q^{\ast}(L')(1,0)\to 0.\tag{4.1}
\end{equation*}
Note here that $n>1$, by Example~\ref{EXP2.17}. Since
$\ext^{i}_{\mathscr{O}_{C}}(L',\mathscr{O}_C)=0$  for all $i>0$, we
see, as in the proof of (3.8), that 
\begin{equation}\label{eqn4.0.2}
\ext^{i}_{\mathscr{O}_D}(N,\mathscr{O}_D)=\ext^{i}_{\mathscr{O}_D}(q^{\ast}(L'),\mathscr{O}_D)=0
\text{~ for all~ } i>0.\tag{4.2}
\end{equation}
And\pageoriginale also, by the flatness of $q_C$ and
Proposition~\ref{Prop2.9}, (3) and (7), we see that 
\begin{align*}
&(q^{\ast}(L'))^{\ast}\simeq q^{\ast}({L'}^{\ast}) \text{ and the canonical
homomorphism}\\
&((q^{\ast}(L'))^{\ast})^{\ast}\to q^{\ast}(L') \text{ is
an isomorphism, where}\tag{4.3}\label{eqn4.0.3}
\end{align*}
$$
(q^{\ast}(R))^{\ast}=\hom_{\mathscr{O}_D}(q^{\ast}(R),\mathscr{O}_D).
$$
Dualizing \eqref{eqn4.0.1} and tensoring with
$\mathscr{O}_{\mathbb{F}}(0,n-1)$, we have, by \eqref{eqn4.0.2},
\eqref{eqn4.0.3} and Proposition~\ref{Prop2.9}, (3), 
\begin{equation*}\label{eqn4.0.4}
0\to q^{\ast}(L)(-1,2)\to \mathscr{O}_D(0, n-1)^{\oplus
  n}\xrightarrow{\sigma} N^{\ast}(0,n-1)\to 0.\tag{4.4}
\end{equation*}
Let $\widetilde{F}$ be the kernel of the homomorphism
$\mathscr{O}_{\mathbb{F}}(0,n-1)^{\oplus n}\to
\mathscr{O}_{D}(0,n-1)^{\oplus n}\xrightarrow{\sigma}
N^{\ast}(0,n-1)$. Then the following exact commutative diagram is
obtained 
\begin{equation*}\label{eqn4.0.5}
\vcenter{
\xymatrix@=.5cm{&0&0&&\\
0\ar[r]&q^{\ast}(L)(-1,2)\ar[u]\ar[r]&\mathscr{O}_D(0,n-1)^{\oplus n}\ar[u]\ar[r]&
N^{\ast}(0,n-1)\ar[r]&0\\
0\ar[r]&\widetilde{F}\ar[u]\ar[r]&\mathscr{O}_{\mathbb{F}}(0,n-1)^{\otimes
  n}\ar[u]\ar[r]& N^{\ast}(0,n-1)\ar@{=}[u]\ar[r]&0\\
&\mathscr{O}_{\mathbb{F}}(0,-1)^{\oplus
  n}\ar[u]\ar@{=}[r]&\mathscr{O}_{\mathbb{F}}(0,-1)^{\oplus n}\ar[u]&&\\
& 0\ar[u] & 0\ar[u] &&}}\tag{4.5}
\end{equation*}

\setcounter{dfn}{5}
\begin{Prop}\label{Prop4.6}
\begin{enumerate}
\renewcommand{\labelenumi}{(\theenumi)}
\item $\widetilde{F}$ is locally free.
\item There exists a vector bundle $F$ on $P$ such that
  $\widetilde{F}\simeq p^{\ast}(F)$.
\end{enumerate}
\end{Prop}

\begin{Proof}
Let\pageoriginale $x$ be a closed point of $D$ and $y=q(x)$. Since
$q_{c}:D\to C$ is a $\mathbb{P}^{1}$-bundle and since
depth$_{\mathscr{O}_{P^{\ast},y}}(L_y)=1$ (see
Proposition~\ref{Prop2.5})
depth$_{\mathscr{O}_{\mathbb{F},x}}(q^{\ast}(L')_x)=2$. On the other
hand, depth$_{\mathscr{O}_{\mathbb{F},x}}(\mathscr{O}_{D,x})=2$. Then
\eqref{eqn4.0.1} shows that depth$_{\mathscr{O}_{\mathbb{F},x}}$
$(N_x)=2$. Since $F$ is a non-singular,
projective three-fold, this implies that there is a locally free
resolution of $N$ of length $1$; $0\to E_1\to E_0\to N\to 0$. This gives
rise to an exact sequence 
$$
0\to E_0^{\ast}\to E_1^{\ast}\to
\ext^{1}_{\mathscr{O}_{\mathbb{F}}}(N,\mathbb{O}_{\mathbb{F}})\to 0
$$
On the other hand, as in the proof of Proposition~\ref{Prop2.9}. (4),
$N^{\ast}(0,n)\simeq
\ext^{1}_{\mathscr{O}_{\mathbb{F}}}(N,\mathscr{O}_{\mathbb{F}})$ is
proved. This and the above exact sequence show that
depth$_{\mathscr{O}_{\mathbb{F},x}}(N^{\ast})_x=2$ for all closed
points $x$ of $D$. Then our assertion (1) follows from the definition
of $\widetilde{F}$ or the middle row of \eqref{eqn4.0.5}.

Let $U$ be the open set $\left\{z\in P\mid p^{-1}(z)\notsubset
D\right\}$ in $P$. $U$ is the set of points $z$ such that there exists
a line $\ell$ passing through $z$ but not contained in $S(L)=C$. It is
clear that $P -U$ is a finite set of points. For $a z\in U$, pick
such a line $\ell$ and set $H=p^{-1}(\ell)$. For the $M$ in
Proposition~\ref{Prop2.9}, we have the following exact sequence 
$$
0\to\mathscr{O}_{P^{\ast}}(-n+1)^{\oplus n}\to \mathscr{O}^{\oplus
  n}_{P^{\ast}}\to M^{\ast}\to 0
$$
(see the proof of Proposition~\ref{Prop2.9} (6)). Since $n\geqq 2$,
$h^{0}(P^{\ast}, M^{\ast})=n$. Obviously, we see that 

\setcounter{subsection}{6}
\setcounter{subsubsection}{0}
\subsubsection{}\label{eqn4.6.1}
If $\psi:\mathscr{O}_{P^{\ast}}^{\oplus n}\to M^{\ast}$
is a homomorphism such that 
$H^{\ast}(\psi)$\pageoriginale 
is an isomorphism, then $\ker(\psi)$ is isomorphic to
$\mathscr{O}_{P^{\ast}}(-n+1)^{\oplus n}$.

No member of $\Ass(\mathscr{O}_D)$ does not contain the equation of $H$
and $\Ass(N)=\Ass(q^{\ast}(L'))=\Ass(\mathscr{O}_D)$. This and
\eqref{eqn4.0.1} provides us with the following exact sequence 
$$
0\to N\mid_{H}\to \mathscr{O}_{D}^{\oplus n}\mid_H\to
q^{\ast}(L')(1,0)\mid_{H}\to 0.
$$
From the choice of $l$, it turns out that $H$ is isomorphic to
$P^{\ast}$ in a neighborhood of $S(L)=C$ and hence we can identify
$D\cap H$ with $C$. Moreover, by this identification we have that
$L'(1)\simeq q^{\ast}(L')(1,0)\mid_H$. Therefore, the above exact
sequence can be regarded as that on $P^{\ast}$.
\setcounter{equation}{1}
\begin{equation}\label{eqn4.6.2}
0\to N\mid_H\to \mathscr{O}_{C}^{\oplus n}\xrightarrow{\delta}L'(1)\to 0.
\end{equation}
Since $H^{0}(\mathbb{F},q^{\ast}(L'))=H^{0}(P^{\ast},L')=0$ and
$H^{1}(\mathbb{F},q^{\ast}(L'))=H^{1}(P^{\ast},L')=0$, the exact
sequence $0\to q^{\ast}(L')\to q^{\ast}(L')(1,0)\to L'(1)\to 0$ gives
us an isomorphism $H^{0}(\mathbb{F},
q^{\ast}(L')(1,0))\xrightarrow{\sim}H^{0}(C,L'(1))$. The  map
$$
H^{0}(\mathbb{F},\mathscr{O}^{\oplus n}_D)\to
H^{\ast}(C,\mathscr{O}^{\oplus n}_C)
$$ 
is obviously isomorphic: 
$$
\xymatrix{H^{0}(\mathbb{F},\mathscr{O}^{\oplus
    n}_D)\ar[d]^{\wr}\ar[r]^-{\sim}&H^{0}(\mathbb{F}',q^{\ast}(L')(1,0))\ar[d]^{\wr}\\
H^{0}(C,\mathscr{O}^{\oplus n}_C)\ar[r]^-{H^{o}(\delta)}&H^{0}(C,L'(1))}
$$
$H^{0}(\delta)$\pageoriginale is, therefore, bijective. Then, by the
definition of $M$, we see 
\begin{equation}\label{eqn4.6.3}
N\mid_H\simeq M.
\end{equation}
Our next claim is 
\begin{equation}\label{eqn4.6.4}
N^{\ast}\mid_{H}\simeq
(N\mid_H)^{\ast}=\hom_{\mathscr{O}_C}(N\mid_{H'}\mathscr{O}_C). 
\end{equation}

Indeed, from the exact sequence 
$$
0\to \mathscr{O}_D(-1,0)\to \mathscr{O}_D\to \mathscr{O}_C\to 0,
$$
we obtain another 
$$
0\to N^{\ast}(-1,0)\to N^{\ast}\to
\hom\mathscr{O}_D(N,\mathscr{O}_C)\to \ext^{1}\mathscr{O}_D(q^{\ast}(L'),\mathscr{O}_D).
$$
\eqref{eqn4.0.2} and the exact sequence 
$$
\xymatrix@=.4cm{\ext^{1}\mathscr{O}_D(\mathscr{O}^{\oplus
    n}_D,\mathscr{O}_{D})\ar@{=}[d]\ar[r]&\ext^{1}\mathscr{O}_D(N,\mathscr{O}_D)\ar[r]&\ext^{2}\mathscr{O}_D(q^{\ast}(L'),\mathscr{O}_D(-1,0))\\
0}
$$
which is obtained from \eqref{eqn4.0.1} show that
$\ext^{1}_{\mathscr{O}_D}(N,\mathscr{O}_D(-1,0))=0$. Then our claim is
clear, because $\hom_{\mathscr{O}_D}(N,\mathscr{O}_C)\simeq
\hom_{\mathscr{O}_C}(N\mid_H,\mathscr{O}_C)=(N\mid_H)^{\ast}$. 

\eqref{eqn4.6.3} and \eqref{eqn4.6.4} yield an isomorphism
\begin{equation}\label{eqn4.6.5}
N^{\ast}\mid_H\simeq M^{\ast}
\end{equation}\pageoriginale

From the middle row of \eqref{eqn4.0.5} we also have an exact sequence 
\begin{equation}\label{eqn4.6.6}
0\to F\mid_{H}(0,1-n)\to \mathscr{O}^{\oplus n}_H
\xrightarrow{\eta}N^{\ast}\mid_H\to 0.
\end{equation}
For the $\eta$, we claim the following: 
\begin{equation}\label{eqn4.6.7}
H^{0}(\eta) \text{~ is isomorphic.}
\end{equation}

Since $q^{\ast}(L)(-1,2)\mid_{H}\simeq L(1)$, we get the following
exact commutative diagram by restricting \eqref{eqn4.0.5} to $H$ and
tensoring with $\mathscr{O}_F(0,1-n):$
$$
\xymatrix{0\ar[r]&L(2-n)\ar[r]&\mathscr{O}_C^{\oplus
    n}\ar[r]^-{\epsilon}&N^{\ast}\mid_H\ar[r]&0\\
& & \mathscr{O}_{H}^{\oplus
    n}\ar[u]^{\zeta}\ar[r]^{\eta}&N^{\ast}\mid_H\ar@{=}[u]\ar[r]& 0}
$$
By \eqref{eqn4.6.5}
$h^{0}\left(C,N^{\ast}\mid_H\right)=h^{0}(C,M^{\ast})$, which 
is equal to $n$, as we have seen in the first part of this
proof. Since $n\geqq 2$, $h^{0}(C,L(2-n))=0$. Hence $H^{0}(\epsilon)$
is isomorphic. $H^{0}(\zeta)$ is clearly an isomorphism. Thus our
claim is proved.

Since $\mathscr{O}^{\oplus n}_{H}\simeq
\left(q\mid_{H}\right)^{\ast}\left(\mathscr{O}_{P^{\ast}}^{\oplus
n}\right)$ and since $N^{\ast}\mid_H$ can be regarded as a sheaf on
$P^{\ast}$, there is a vector bundle $E$ on $P^{\ast}$ which is fitted
in an exact sequence  
$$
0\to E\to \mathscr{O}^{\oplus n}_{P^{\ast}}\xrightarrow{\zeta}
N^{\ast}\mid_H\to 0
$$
whose\pageoriginale pull-back to $H$ is \eqref{eqn4.6.6}. Since
$H^{0}(\xi)=H^{0}(\eta)$ is isomorphic, \ref{eqn4.6.1} and
\eqref{eqn4.6.5} imply that $E\simeq
\mathscr{O}_{P^{\ast}}(n-1)^{\oplus n}$ and hence $F\mid_H\simeq
\left(q\mid_H\right)^{\ast}(\mathscr{O}_{P^{\ast}}(1-n)^{\oplus n})(0,
n-1)\simeq \mathscr{O}_{H}^{\oplus n}$. What we have proved so far is
that for all points $z$ of $U$, $F(z)$ is a trivial bundle. Therefore,
$p_{\ast}\left(\widetilde{F}\right)=F$ is a vector bundle of rank $n$
and the natural homomorphism
$\lambda:p^{\ast}p_{\ast}\left(\widetilde{F}\right)\to \widetilde{F}$
isomorphic on $p^{-1}(U)$. The set where $\lambda$ is not isomorphic
is $\det(\lambda)=0$ which is pure codimension $1$ in $\mathbb{F}$. On
the other hand, $\mathbb{F}-p^{-1}(U)$ is at least codimension
$2$. Thus $\lambda$ is an isomorphism.
\enprf
\end{Proof}

The above proposition and the following lemma complete the proof of
Main Theorem~\ref{MT2.16}.

\begin{lemma}\label{lemma4.7}
The vector bundle $F$ in Proposition~\ref{Prop4.6} is contained in
$\mathscr{V}$ and $\Phi(F)\simeq L$. 
\end{lemma}

\begin{Proof}
Since $R^{i}q_{\ast}(\mathscr{O}_{\mathbb{F}}(0.-1)^{\oplus n})=0$ for
all $i>0$ and $q_{\ast}(\mathscr{O}_{\mathbb{F}}(0,-1)^{\oplus
  n})\simeq \mathscr{O}_{P^{\ast}}(-1)^{\oplus n}$, we see that
$H^{i}(\mathbb{F},\mathscr{O}_{\mathbb{F}}(0,-1)^{\oplus
  n})=H^{i}(P^{\ast},\mathscr{O}_{P^{\ast}}(-1)^{\oplus n})=0$ for all
$i$. All the cohomology groups of $q^{\ast}(L)(-1,2)$ vanish, too,
because $R^{i}q_{\ast}(q^{\ast}(L)(-1,2))\simeq L\otimes
R^{i}q_{\ast}(\mathscr{O}_{\mathbb{F}}(-1,2))=0$ for all $i$. By the
leftmost column of \eqref{eqn4.0.5}, these show that
$H^{i}\left(\mathbb{F},\widetilde{F}\right)=0$ for all $i$, Since
$H^{i}\left(\mathbb{F},\widetilde{F}\right)=H^{i}(P,F)$ for all $i$,
$F$ has the property \eqref{eqn1.5.2} and moreover $0=-c_2(F)+n$, by
Riemann-Roch Theorem for $F$ (Note that $c_1(F)=0$ and $r(F)=n$). Thus
$r(F)=c_2(F)$. Pick a point outside $S(L)$. Then, by the construction
of $\widetilde{F}$, $\widetilde{F}\mid_q-1_{(x)}$ is trivial. Since
$F\mid_{p q-1(x)}\simeq \widetilde{F}\mid_q-1_{(x)}$, we see that $F$
enjoys the property \eqref{eqn1.10.2}. The second assertion is easily
proved by using the leftmost column of \eqref{eqn4.0.5} again.
\enprf
\end{Proof}

For\pageoriginale an $L$ in $\mathscr{C}$,
$L'=\hom_{\mathscr{O}_{S(L)}}(L,\omega_{S(L)})$ is not necessarily
contained in $\mathscr{C}$. If both $L$ and $L'$ are members of
$\mathscr{C}$, the structure of corresponding bundles is clearer.

\begin{Theorem}\label{Theorem4.8}
Let $F$ be a member of $\mathscr{V}(n)$ and set $\Phi(F)=L$. Then $L'$
contained in $\mathscr{C}$ (i.e. $L'$ has the property
\eqref{eqn2.3.2}) if and only if there is an exact sequence of vector
bundles 
\setcounter{equation}{0}
\begin{equation}\label{eqn4.8.1}
0\to E\to F\to \mathscr{O}^{\oplus r}_P\to 0
\end{equation}
with $H^{0}(P,E^{\ast})=0$ Moreover $L'$ corresponds to the vector
bundle $\sigma(n-r,n)_k(E^{\ast})$ by $\Phi$.
\end{Theorem}

\begin{Proof}
Assume that we have the exact sequence \eqref{eqn4.8.1}. Since
$H^{0}(P,E^{\ast})$ and $E^{\ast}\mid_{\ell}\simeq \mathscr{O}^{\oplus
  (n-r)}_{\ell}$ for general lines $\ell$, the isomorphism class
$E^{\ast}$ is contained in $V(n-r, n)_k$. Let us consider a vector
bundle $F'$ whose isomorphism class is
$\sigma(n-r,n)_k(E^{\ast})$. Then $F'$ is fitted in an exact sequence 
$$
0\to E^{\ast}\to F'\to \mathscr{O}^{\oplus r}_P\to 0.
$$
Now it is obvious that 
$$
\Phi(F)\simeq
R^{1}q_{\ast}(P^{\ast}(E)(-1,-1))
$$ 
and 
$$
\Phi(F')\simeq
R^{1}q_{\ast}(P^{\ast}(E^{\ast})(-1,-1)).
$$ 
The dual sequence of
\eqref{eqn4.8.1} shows that $R^{1}q_{\ast}(p^{\ast}(E^{\ast})(-1,-1))$
is isomorphic to $R^{1}q_{\ast}(p^{\ast}(F^{\ast})(-1,-1))$. On the
other hand, dualizing the exact sequence of Proposition~\ref{Prop3.6},
we have 
\begin{equation}\label{eqn4.8.2}
0\to p^{\ast}(F^{\ast})\to \mathscr{O}_{\mathbb{F}}(0,1)^{\oplus n}\to
\ext^{1}_{\mathscr{O}_{\mathbb{F}}}(q^{\ast}(L)(-1,2),\mathscr{O}_{\mathbb{F}})\to 0.
\end{equation}
\pageoriginale
Since
$$\ext^{1}_{\mathscr{O}_{\mathbb{F}}}(q^{\ast}(L)(-1,2), \mathscr{O}_{\mathbb{F}})\simeq
q^{\ast}(\ext^{1}_{O_{P^{\ast}}}(L,\mathscr{O}_{P^{\ast}}(-3)))(1,1)\simeq
q^{\mathbb{F}_{\ast}}(L')(1,1),
$$ 
we get an isomorphism
$R^{1}q_{\ast}(p^{\ast}(F^{\ast})(-1,-1))\simeq L'$, by tensoring the
above sequence with $\mathscr{O}_{\mathbb{F}}(-1,-1)$ and then
applying the direct image functor of $q$ to it. Combining the above
results, we see that $\Phi(F')\simeq L'$ and hence $L'$ is contained
in $\mathscr{C}$. 

The proof of the converse consists of several steps. 

Step I. Let $F'$ be the bundle which corresponds to $L'$. We shall
construct a homomorphism of $\widetilde{F}^{\ast}=p^{\ast}(F^{\ast})$
to $\widetilde{F}'=p^{\ast}(F')$. Setting $H=p^{-1}(\ell)$ for a
general line $\ell$ in $P$, we have the following exact sequence: 
$$
0\to \widetilde{F}^{\ast}(-1,0)\to \widetilde{F}^{\ast}\to
\mathscr{O}^{\oplus n}_H\to 0.
$$ 
Since $R^{1}q_{\ast}\left(\widetilde{F}^{\ast}(-1,0)\right)\simeq
L'(1)$ as we have seen in the above, the above sequence provides us
with another exact sequence 
$$
0\to G'= q_{\ast}\left(\widetilde{F}^{\ast}\right)\to \mathscr{O}^{\oplus
  n}_{P^{\ast}}\to L'(1)\to
R^{1}q_{\ast}\left(\widetilde{F}^{\ast}\right)\to 0.
$$
Since $\widetilde{F}^{\ast}$ is locally free, so is $G'$. If one
denotes the cokernel of the natural map of $q^{\ast}(G')$ to
$\widetilde{F}^{\ast}$ by $T$, one gets the following exact commutative
  diagram (see \ref{eqn4.8.2})
\begin{equation}\label{eqn4.8.3}
\vcenter{
\xymatrix@=.5cm{&0&&&\\
&T\ar[u]&&0&\\
0\ar[r]& \widetilde{F}^{\ast}\ar[u]\ar[r]&
  \mathscr{O}_{\mathbb{F}}(0,1)^{\oplus
    n}\ar[r]&q^{\ast}(L')(1,1)\ar[u]\ar[r]&0\\
0\ar[r]&
q^{\ast}(G')\ar[u]\ar[r]&q^{\ast}q_{\ast}(\mathscr{O}_{\mathbb{F}}(0,1))^{\oplus
n}\ar@{=}[u]\ar[r]& q^{\ast}(B)\ar[u]\ar[r]&0\\
&&&R\ar[u]&\\
&&&0\ar[u]&}}
\end{equation}\pageoriginale
where $B$ is $\im(\beta)$ in the following exact sequence 
\begin{align*}
0 &\to G'\to \mathscr{O}_{P^{\ast}}(1)^{\oplus
  n}\xrightarrow{\beta}q^{\ast}(q^{\ast}(L')(1,1))\simeq L'\otimes
T_{P^{\ast}}\\
& \to R^{1}q_{\ast}\left(\widetilde{F}^{\ast}\right)\to
R^{1}q_{\ast}(\mathscr{O}_{\mathbb{F}}(0,1)^{\oplus n})=0.
\end{align*}
Let us consider another exact commutative diagram 
$$
\xymatrix{&0&0&&\\
&q^{\ast}(L')(1,1)\ar[u]\ar@{=}[r]& q^{\ast}(L')(1,1)\ar[u]& &\\
0\ar[r]&q^{\ast}(B)\ar[u]\ar[r]&q^{\ast}(L'\otimes
T_{P^{\ast}})\ar[u]\ar[r]&q^{\ast}\left(R^{1}q_{\ast}\left(\widetilde{F}^{\ast}\right)\right)\ar[r]&0\\
0\ar[r]&R\ar[u]\ar[r]&q^{\ast}(L')(-1,2)\ar[u]\ar[r]&Q\ar[u]\ar[r]&0\\
&0\ar[u]&0\ar[u]&&}
$$
The first diagram gives rise to an isomorphism of $T$ to
$R$. Composing this with the injection of the bottom row of the second
diagram, we obtain an injection $\delta_{0}:T\to
q^{\ast}(L')(-1,2)$. Note that we have the following isomorphism 
\begin{equation}\label{eqn4.8.4}
q^{\ast}(L')(-1,2)/\delta_0(T)\simeq q^{\ast}\left(R^{1}q_{\ast}\left(\widetilde{F}^{\ast}\right)\right).
\end{equation}
\pageoriginale
From the exact sequence in Proposition~\ref{Prop3.6} for $F'$, we
obtain the exact sequence 
\begin{align*}
&0\to
\Hom_{\mathscr{O}_{\mathbb{F}}}\left(\widetilde{F}^{\ast},\mathscr{O}_{\mathbb{F}}(0,-1)^{\oplus n}\right)\to
\Hom_{\mathscr{O}_{\mathbb{F}}}\left(\widetilde{F}^{\ast},\widetilde{F}'\right)\to\\
& \Hom_{\mathscr{O}_{\mathbb{F}}}\left(\widetilde{F}^{\ast},q^{\ast}(L')(-1,2)\right)\to
\ext^{1}_{\mathscr{O}_{\mathbb{F}}}\left(\widetilde{F}^{\ast},\mathscr{O}_{\mathbb{F}}(0,-1)^{\oplus n}\right). 
\end{align*}
As easily seen,
$\Hom_{\mathscr{O}_{\mathbb{F}}}\left(\widetilde{F}^{\ast},\mathscr{O}_{\mathbb{F}}(0,-1)^{\oplus
  n}\right)\simeq H^{0}\left(\mathbb{F},\widetilde{F}(0,-1)^{\oplus n}\right)=0$
and
$\Ext^{1}_{\mathscr{O}_{\mathbb{F}}}\left(\widetilde{F}^{\ast},\mathscr{O}_{\mathbb{F}}(0,-1)^{\oplus
  n}\right)\simeq H^{1}\left(\mathbb{F},\widetilde{F}(0,-1)^{\oplus
  n}\right)=0$. Thus there exists a unique homomorphism
$\widetilde{\delta}:F^{\ast}\to F'$ which covers
$\delta_{0}$. Obviously, for
$\delta=p_{\ast}\left(\widetilde{\delta}\right):F^{\ast}\to F'$,
$p^{\ast}(\delta)=\widetilde{\delta}$. 

Step II. Set $K=\ker(\delta)$. We shall prove that $K\simeq
\mathscr{O}_{P}^{\oplus r}$, where $r=h^{0}(P,F^{\ast})$. Since both
$F^{\ast}$ and $F'$ are $\mu$-semi-stable and since
$c_1(F^{\ast})=c_1(F')=0$, for $E'=\im(\delta)$, $c_1(E')=0$ and hence
$c_1(K)=0$. Let $S$ be the torsion part of $F'/E'$. Since
$0\leqq
c_1((F'/E')/S)=c_1(F'/E')-c_1(S)=-c_1(S)$,
we see that $S$ is supported by, at most, a finite set of points. This
implies that for a general $z$ of $P^{\ast}$,
$$
0\to \widetilde{K}\mid_q-1_{(z)}\to
\widetilde{F}^{\ast}\mid_q-1_{(z)}\to \widetilde{F}'\mid_q-1_{(z)}
$$
is exact, where $\widetilde{K}=p^{\ast}(K)$. From this, we deduce that 

\setcounter{subsection}{8}
\setcounter{subsubsection}{4}
\subsubsection{}\label{eqn4.8.5}
$\widetilde{K}\mid_{q^{-1}(z)}$\pageoriginale is a trivial bundle of rank $r$
for all general points $z$ in $P^{\ast}$ with $r=r(K)$
because both $\widetilde{F}^{\ast}\mid_{q^{-1}(z)}$ and
$\widetilde{F}'\mid_{q^{-1}(z)}$ are trivial for all points $z$ in
$P^{\ast}-S(L)$. If $\ell$ is a sufficiently general line in $P$, then
for $H= p^{-1}(\ell)$, we have the following exact commutative
diagram: 
\setcounter{equation}{5}
\begin{equation}\label{eqn4.8.6}
\vcenter{
\xymatrix@=.6cm{&0\ar[d]&0\ar[d]&0\ar[d]&\\
0\ar[r]&\widetilde{K}(-1,0)\ar[d]\ar[r]&
  \widetilde{K}\ar[d]\ar[r]&\widetilde{K}\mid_H\ar[d]\ar[r]& 0\\
0\ar[r]&\widetilde{F}^{\ast}(-1,0)\ar[d]\ar[r]& \widetilde{F}^{\ast}\ar[d]\ar[r]&
\widetilde{F}^{\ast}\mid_H\ar[d]\ar[r]&0\\
0\ar[r]&\widetilde{F}'(-1,0)\ar[r]& \widetilde{F}'\ar[r]& \widetilde{F}'\mid_{H}\ar[r]&0\\
}}
\end{equation}

On the other hand, the leftmost colum of \eqref{eqn4.8.3} yields the
exact sequence 
$$
0=R^{1}q_{\ast}(q^{\ast}(G')(-1,0))\to
R^{1}q_{\ast}\left(\widetilde{F}^{\ast}(-1,0)\right)\to
R^{1}q_{\ast}(T(-1,0))\to 0, 
$$
which implies that
$L'(1)=R^{1}q_{\ast}\left(\widetilde{F}^{\ast}(-1,0)\right)\simeq
R^{1}q_{\ast}(T(-1,0))$. Since
$R^{1}q_{\ast}(q^{\ast}R^{1}q_{\ast}\left(\widetilde{F}^{\ast}(-1,0))\right)\simeq
R^{1}q_{\ast}\left(\widetilde{F}^{\ast}(-1,0)\right)\otimes
R^{1}q_{\ast}\left(\mathscr{O}_{\mathbb{F}}\right)=0$, \eqref{eqn4.8.4}
shows that 
$$
R^{1}q_{\ast}(\delta_0(-1,0)):R^{1}q_{\ast}(T(-1,0))\to
R^{1}q_{\ast}(q^{\ast}(L')(-2,2))\simeq L'(1)
$$
is surjective and then it is an isomorphism because
$R^{1}q_{\ast}(T(-1,0))$ is isomorphic to $L'(1)$ as we have seen in
the above. In the commutative diagram 
$$
\xymatrix{R^{1}q_{\ast}\left(\widetilde{F}^{\ast}(-1,0)\right)\ar[d]^{u}\ar[r]& R^{1}q_{\ast}(T(-1,0))\ar[d]^{R^{1}q_{\ast}(\delta_0(-1,0))}\ar[r]&0\\
R^{1}q_{\ast}\left(\widetilde{F}'(-1,0)\right)\ar[r]& R^{1}q_{\ast}(q^{\ast}(L')(-2,2))\ar[r]&0}
$$\pageoriginale
we have showed that the maps except for $u$ were in bijective
correspondence, whence $u$ is bijective, too. Now, \eqref{eqn4.8.6}
supplies us with the following exact commutative diagram:
$$
\xymatrix@=.5cm{&0\ar[d]&0\ar[d]&&\\
0\ar[r]&
  q_{\ast}\left(\widetilde{K}\right)\ar[d]\ar[r]&q_{\ast}\left(\widetilde{K}\mid_H\right)\ar[d]\ar[r]& R^{1}q_{\ast}\left(\widetilde{K}(-1,0)\right)\ar[d]\\
0\ar[r]& G'\ar[d]\ar[r]&\mathscr{O}^{\oplus
  n}_{P^{\ast}}\ar[d]\ar[r]&L'(1)\ar[d]\ar[r]&0\\
0\ar[r]& \mathscr{O}_{P^{\ast}}(-1)^{\oplus
  n}\ar[r]&\mathscr{O}^{\oplus n}_{P^{\ast}}\ar[r]& L'(1)\ar[r]&0}
$$
Look at the two exact sequences 
$$
\xymatrix@R=.3cm{0\ar[r]&\widetilde{K}(-1,0)\ar[r]&\widetilde{F}^{\ast}(-1,0)\ar[r]&\widetilde{E}'(-1,0)\ar[r]&0\\
0\ar[r]& \widetilde{E}'(-1,0)\ar[r]&\widetilde{F}'(-1,0)}
$$
where $\widetilde{E}'$ is the torsion free $p^{\ast}(E')$. From these,
we have 
$$
\xymatrix{R^{1}q_{\ast}\left(\widetilde{F}^{\ast}(-1,0)\right)\ar[dr]_{u}\ar[r]^{\nu}&R^{1}q_{\ast}\left(\widetilde{E}'(-1,0)\right)\ar[d]\ar[r]&0\\
&R^{1}q_{\ast}\left(\widetilde{F}'(-1,0)\right)&}
$$
Since\pageoriginale $u$ is bijective, $\nu$ is injective, whence this is
isomorphic. By \ref{eqn4.8.5}, for general points $z$ of $P^{\ast}$,
$\widetilde{E}'\mid_{q^{-1}(z)}$ is a trivial bundle of rank $n-r$,
which implies that $q_{\ast}\left(\widetilde{E}'(-1,0)\right)$
vanishes. We see, therefore, that
$R^{1}q_{\ast}\left(\widetilde{K}(-1,0)\right)=0$. The above diagram
then shows that $q_{\ast}\left(\widetilde{K}\right)$ is isomorphic to
$q_{\ast}\left(\widetilde{K}\mid_{H}\right)$ which is a trivial bundle
thanks to the middle column of the diagram. The leftmost column gives
us that
$h^{0}\left(P^{\ast},q_{\ast}\left(\widetilde{K}\right)\right)=h^{0}\left(P^{\ast}.G'\right)=h^{0}\left(\mathbb{F},\widetilde{F}^{\ast}\right)=
h^{0}(P,F^{\ast})$. Therefore, $q_{\ast}\left(\widetilde{K}\right)\simeq
\mathscr{O}^{\oplus r}_{P^{\ast}}$ with $r=h^{0}(P,F^{\ast})$. By
\ref{eqn4.8.5}, the natural homomorphism $\omega$ of
$\mathscr{O}^{\oplus r}_{\mathbb{F}}\simeq
q^{\ast}q_{\ast}\left(\widetilde{K}\right)$ to $\widetilde{K}$ is
injective and $r=r\left(\widetilde{K}\right)$. Thus $\coker (\omega)$
is supported by the divisor $\det(\omega)=0$. On the other hand,
$c_1(\coker(\omega))=0$ because
$c_1\left(\widetilde{K}\right)=c_1\left(\mathscr{O}_{\mathbb{F}}^{\oplus
  r}\right)=0$. These mean that $\omega$ is an isomorphism, that is, $K\simeq
\mathscr{O}^{\oplus r}_{\mathbb{F}}$. Hence we obtain that $K\simeq
\mathscr{O}^{\oplus r}_P$ with $r=h^{0}(P,F^{\ast})$ as required. 

Step III. Let us prove that $E'=\im(\delta)$ is locally free. Set
$I=F'/E'$. Since $c_1(E')=0$, the torsion part $U$ of $I$ is
supported by a finite set of points. On the other hand, setting
$J=\coker(q_{\ast}(\delta):G'\to \mathscr{O}_{P^{\ast}}(-1)^{\oplus
  n})$ and taking \eqref{eqn4.8.4} into account, we have the following
exact commutative diagram: 
$$
\xymatrix@=.5cm{&&&0\ar[d]&\\
0\ar[r]& q^{\ast}(G')\ar[d]\ar[r]& \widetilde{F}^{\ast}\ar[d]\ar[r]&
  T\ar[d]\ar[r]&0\\
0\ar[r]&\mathscr{O}_{\mathbb{F}}(0,-1)^{\oplus
  n}\ar[d]\ar[r]&\widetilde{F}'\ar[d]\ar[r]& q^{\ast}(L')(-1,2)\ar[d]\ar[r]&0\\
0\ar[r]&q^{\ast}(J)\ar[d]\ar[r]&\widetilde{I}\ar[d]\ar[r]&q^{\ast}R^{1}q_{\ast}(F^{\ast})\ar[d]\ar[r]&0\\
&0&0&0&}
$$
The\pageoriginale bottom row Lemma~\ref{LEM2.13.1} assert that the
natural map of 
$q^{\ast}q_{\ast}\left(\widetilde{I}\right)$ to $\widetilde{I}$ is
bijective. Since $p$ and $q$ are flat, the torsion part of
$\widetilde{I}$ is, on 
one hand, $p^{\ast}(U)$ and, on the other hand, $q^{\ast}(V)$, where
$V$ is the torsion part of $q^{\ast}\left(\widetilde{I}\right)$. Thus
$p^{-1}(\Supp(U))=q^{-1}(\Supp(V))$, which implies that
$(\Supp U)$ is empty or equivalently $U=0$, that is, $I$ is torsion
free. This implies that $E'$ is locally free. 

Step IV. We shall complete the proof of the theorem. In the first
place,
$\chi (F')-\chi (I)=\chi (E')=\chi (F^{\ast})-\chi (\mathscr{O}_{P}^{\oplus
  r})$ and $\chi (F')=\chi (F^{\ast})=0$. Thus
$\chi (I)=r$. Since $c_1(I)=0$ and $r(I)=r$, we see, by
Riemann-Roch Theorem, that $c_2(I)=0$. Moreover, $I$ is
$\mu$-semi-stable because for a general line $\ell$ in $P$,
$I\mid_{\ell}\simeq \mathscr{O}_{\ell}^{\oplus r}$. Then applying
Lemma~\ref{lemma1.3} to $I$, we know that $I$ is isomorphic to
$\mathscr{O}^{\oplus r}_{P}$. Thus we have two exact sequences
\begin{align*}
&0\to \mathscr{O}_P^{\oplus r}\to F^{\ast}\to E'\to 0\\
&0\to E'\to F'\to \mathscr{O}^{\oplus r}_{P}\to 0.
\end{align*}
Setting $E={E'}^{\ast}$, we get our assertion. In fact, since
$H^{0}(P,F^{\ast})=r$, $H^{0}(P,E^{\ast})$ vanishes.
\enprf
\end{Proof}

\setcounter{cor}{6}
\begin{cor}\label{cor4.8.7}
If a member $L$ of $\mathscr{C}$ is quadratic, then we have an exact
sequence 
$$
0\to E\to F\to \mathscr{O}_P^{\oplus r}\to 0
$$ 
where $H^{0}(P,E)=0$, $E\simeq E^{\ast}$ and $r=h^{0}(P,F^{\ast})$.
\end{cor}

\begin{Proof}
We\pageoriginale have only to prove that $E$ is isomorphic to
$E^{\ast}$. Since $L\simeq L'$, $F'$ in the proof of
Theorem~\ref{Theorem4.8} is isomorphic to $F$. 
Thus we have two exact sequences:
\begin{align*}
&0\to \mathscr{O}_P^{\oplus r}\to F^{\ast}\to E^{\ast}\to 0\\
&0\to \mathscr{O}^{\oplus r}_{P}\to F^{\ast}\to E\to 0
\end{align*}
Since $h^{0}(P,F^{\ast})=r$,
$E^{\ast}=F^{\ast}/H^{0}(P,F^{\ast})\otimes \mathscr{O}_P$ by 
the first sequence and it is isomorphic to $E$ by the second.
\enprf
\end{Proof}

Let us show by an example that for an $L$ in $\mathscr{C}$, $L'$ is not
necessarily contained in $\mathscr{C}$. 

\begin{EXP}\label{EXP4.9}
Let $E$ be a member of $\mathscr{V}(2)$ and $I$ an ideal of a rational
point $x$ of $P$. By a well-known spectral sequence, we have the
following exact sequence: 
\begin{align*}
& 0\to H^{1}(P,\hom_{\mathscr{O}_P}(I,E))\to
  \Ext^{1}_{\mathscr{O}_P}(I,E)\to\\
& H^{0}(P,\ext^{1}_{\mathscr{O}_P}(I,E))\to H^{2}(P,\hom_{\mathscr{O}_P}(I,E)).
\end{align*}
Another exact sequence 
$$
0\to I\to \mathscr{O}_P\to k(x)\to 0
$$
shows that $\hom_{\mathscr{O}_P}(I,E)\simeq E$ and
$\ext^{1}_{\mathscr{O}_P}(I,E)\simeq \ext^{2}_{\mathscr{O}_P}(k(x),
E)\simeq k(x)^{\oplus 2}$. Therefore,
$\Ext^{1}_{\mathscr{O}_P}(I,E)\simeq
H^{0}(P,\ext^{1}_{\mathscr{O}_P}(I,E))\simeq k(x)^{\oplus
2}$\pageoriginale because
$E$ is a member of $\mathscr{V}(2)$. In a neighborhood $U$ of $x$,
$E\mid_U\simeq \mathscr{O}_U e_1\oplus \mathscr{O}_U e_2$. We may
assume that by the map $\ext^{1}\mathscr{O}_P(I,E)\to
\ext^{1}\mathscr{O}_U(I\mid_{U'}\mathscr{O}_U e_1\oplus \mathscr{O}_U
e_2)\to k(x)^{\oplus 2}$, $e_1$ is sent to the $(1,0)=\xi\cdot
\xi$. defines an extension 
$$
0\to E\to F\to I\to 0
$$
$F\mid_U$ is isomorphic to $\mathscr{O}_U e_2\oplus G$, where $G$ is
the extension of $I\mid_U$ by $\mathscr{O}_U e_1$ defined by a
generator of $\ext^{1}\mathscr{O}_U(I\mid_U,\mathscr{O}_U e_1)\simeq
k(x)$. By virtue of a result of Serre (\cite{key12}), $G$ is locally
free and hence so is $F$. Now it is clear that $H^{0}(P,F)=0$,
$c_2(F)=r(F)=3$ and for a general line $\ell$ in $P$, $F\mid_{\ell}$ is
trivial. Thus $F$ is a member of $\mathscr{V}(3)$. However,
$E'=F^{\ast}/H^{0} (P,F^{\ast})\otimes \mathscr{O}_P$ is a
subsheaf of $E^{\ast}$ but not equal to $E^{\ast}$ at $x$. Thus $E'$
is not locally free, this and Theorem~\ref{Theorem4.8} together show that
for $L=\Phi(F)$, $L'$ is not contained in $\mathscr{C}$. Note that the
discriminant curve $S(L)$ of $L$ is a union of non-singular conic and
a line. This $F$ also supplies us with an example to show that
Lemma~\ref{lemma1.4}, (1) does not hold without assuming the stability
of $F$. In fact, this $F$ is semi-stable but not stable.
\end{EXP}

\section{Applications and remarks}\label{s5}

Let us give some applications of our Main Theorem and
Theorem~\ref{Theorem4.8} and some remarks. The first remark is 

\begin{lemma}\label{lemma5.1}
\begin{enumerate}
\renewcommand{\labelenumi}{(\theenumi)}
\item A member $F$ of $\mathscr{V}$ is stable if and only if $\Phi(F)$
  is a simple object in $\mathscr{C}$.

\item Let\pageoriginale $L$ be a member of $\mathscr{C}$. If $S(L)$ is
irreducible and reduced, then a vector bundle corresponding to $L$ is
stable. 
\end{enumerate}
\end{lemma}

\begin{Proof}
\begin{enumerate}
\renewcommand{\labelenumi}{(\theenumi)}
\item If $\Phi(F)$ contains a proper subsheaf $L$ which is a member of
  $\mathscr{C}$, then $F$ contains a subsheaf $E$ corresponding to
  $L$. Since $E$ is a member of $\mathscr{V}(t)$ for some $t$, we see
  that $P_E(m)=m^{2}/2+3m/2=P_F(m)$, which implies
  that $F$ is not stable. Conversely, if $F$ is not stable, then it is
  semi-stable and contains an $E$ of a $\mathscr{V}(t)$ by
  Lemma~\ref{lemma1.9} $\Phi(E)$ is then a proper, non-zero subsheaf
  of $\Phi(F)$. 

\item Take an $F$ in $\mathscr{V}(n)$ which corresponds to $L$. Assume
  that $F$ is not stable. By virtue of Lemma~\ref{lemma1.9}, we see
  that $F$ is semistable and contains an $E$ in $\mathscr{V}(t)(t<n)$
  as a subsheaf. Then $M=\Phi(E)$ is a proper subsheaf of $L=\Phi(F)$
  and then $S(M)$ is a subsheme of $S(L)$ (Lemma~\ref{lemma2.8}) which
  violates our assumption. 
\end{enumerate}
\enprf
\end{Proof}

Which curve can be a discriminant curve of $L$ in $\mathscr{C}$ or the
curve of jumping lines of a vector bundle in $\mathscr{V}$? The
following answers partly this question.

\begin{lemma}\label{lemma5.2}
Let $D$ be an effective divisor on $P^{\ast}$ whose support does not
contain any line. Then there exists an $L$ in $\mathscr{C}$ such that
$S(L)=D$. 
\end{lemma}

\begin{Proof}
Since the support of $D$ contains no lines, we do not need to care
about the property \eqref{eqn2.3.2}, by virtue of
Proposition~\ref{Prop2.13}. Let $D$ be $\sum n_i D_i$, where $n_i$'s
are positive integers and $D_i$'s are mutually distinct irreducible,
reduced divisors. If we have\pageoriginale $L_i$ for each $D_i$ such that
$S(L_i)=D_i$, then  $L=\oplus L_i^{\oplus n_i}$ has the property
\eqref{eqn2.3.1} and $S(L)=D$ (see Lemma~\ref{lemma2.8}). Thus we may
assume  that $D$ is irreducible and reduced. Let $g:\widetilde{D}\to
D$ be the normalization of $D$. Then, as is easily seen, there is a
line bundle $\widetilde{L}$ on $\widetilde{D}$ such that
$H^{0}\left(\widetilde{D},
\widetilde{L}\right)=H^{1}\left(\widetilde{D},\widetilde{L}\right)=0$. Set
$L=g_{\ast}\left(\widetilde{L}\right)$. Since $g$ is a finite
morphism,
$H^{0}(P^{\ast},L)=H^{0}\left(\widetilde{D},\widetilde{L}\right)=0$
and
$H^{1}(P^{\ast},L)=H^{1}\left(\widetilde{D},\widetilde{L}\right)=0$. Thus
$L$ has the property \eqref{eqn2.3.1}. At smooth points of $D$, $g$ is
isomorphic. This means that $c_1(L)=D$. We see, therefore, that
$S(L)=D$. 
\enprf
\end{Proof}

The next question is on the existence of $\mu$-stable vector bundles
$F$ with $S(F)$ smooth. 

\begin{Prop}\label{Prop5.3}
Let $r$ and $c_2$ be integers with $c_2\geqq r\geqq 2$.
Then there exists a $\mu$-stable vector bundle $F$ on
$P=\mathbb{P}^{2}$ of rank $r$ with $c_1(F)=0$ and $c_2(F)=c_2$ such
that the curve of jumping lines $S(F)$ of $F$ is smooth.
\end{Prop}

\begin{Proof}
We shall fix a $c_2$ and prove our assertion by induction on $r$. In
the case of $r=2$, the proposition is well known (see \cite{key2}
or \cite{key11}). Suppose that Proposition~\ref{Prop5.3} is true up to 
$r-1$. Pick a $\mu$-stable vector bundle $E$ on $P$ of rank $r-1$ with
$c_1(E)=0$, $c_2(E)=c_2$ and $S(E)$ smooth. Since $H^{0}(P
E)=H^{2}(P,E)=0$, we have $h^{1}(P,E)=c_2-r+1>0$. Then we have a
non-trivial extension 
$$
0\to E\to F'\to \mathscr{O}_P\to 0
$$
which is stable by Lemma~\ref{lemma1.4}, (2). Since
$$
R^{1}q_{\ast}(p^{\ast}(E(-1)))\simeq R^{1}q_{\ast}(p^{\ast}(F'(-1)),
$$
$S(F')=S(E)$\pageoriginale which is a smooth curve in
$P^{\ast}$. Let $M$ be the irreducible component of the moduli space
of stable vector bundles on 
$P$ which contains the $F'$ in the above. Since $S(F')$ is smooth, for
general points $F$ of $M$ and general line $\ell$ in $P$,
$F\mid_{\ell}\simeq \mathscr{O}_{\ell}^{\oplus r}$ and $S(F)$ is
smooth. Assume that $F$ is not $\mu$-stable. Then $F$ is
$\mu$-semi-stable. Let $E'$ be a maximal proper subsheaf of $F$ with
$c_1(E')=0$. Then $E'$ is locally free, $E''=F/F'$ is torsion
free and $c_1(E')=0$. Moreover, both $E$ and $E'$ are
$\mu$-semi-stable. On the one hand, $S(F)$ is smooth and, on the other
hand, $S(F)=S(E')+S(E')$. This implies that one of $S(E')$ and $S(E')$
is zero; in other words, either $c_2(E')$ or $c_2(E')$ is zero. By
Lemma~\ref{lemma1.3}, we see that $E'\simeq \mathscr{O}_{P}^{\oplus
  s}$ or $E'\simeq \mathscr{O}_{P}^{\oplus(r-s)}$. If $E'$ is trivial,
$h^{0}(P,F)\geqq h^{0}(P,E')\neq 0$ which contradicts the stability of
$F$. Therefore $F$ is fitted in the extension
$$
0\to E'\to F\to \mathscr{O}_P^{\oplus t}\to 0
$$
Now we shall assume that our assertion is false for $r$ and show that
it leads us to a contradiction. Since $h^{0}(P,F^{\ast})$ is upper
semi-continuous on $M$ and $h^{0}(P,{F'}^{\ast})=1$, $t$ must be $1$ for
general $F$. Furthermore, by the openness of $\mu$-stability and the
fact that $E^{\ast}$ is $\mu$-stable, we see that
${E'}^{\ast}\simeq F^{*}/\mathscr{O}_{P}$ is $\mu$-stable for general
$F$. Thus there exists a non-empty open set $U$ of $M$ whose points
correspond to stable vector bundles $F$ such that $F$ is an extension of
$\mathscr{O}_P$ by a $\mu$-stable $E$: 
\setcounter{equation}{0}
\begin{equation}\label{eqn5.3.1}
0\to E\to F\to \mathscr{O}_P\to 0.
\end{equation}
The\pageoriginale dimension of the moduli space of $\mu$-stable vector
bundles $E$ with $c_1(E)=0$, $c_2(E)=c_2$ and $r(E)=r-1$ is 
$$
2(r-1)c_2-(r-1)^{2}+1(\cite[\text{Proposition~}6.9]{key10}).
$$
Since $h^{1}(P,E)=c_2-r+1$, the dimension of the space of the
extensions \eqref{eqn5.3.1} is $c_2-r$. By Proposition 6.9 of
\cite{key10} again, we have that $\dim U=2r c_2-r^{2}+1$. Then we get
$\dim U-\{2(r-1)c_2-(r-1)^{2}+1+c_2-r\}=c_2-r+1>0$. This means that
general members $F$ of $U$ cannot be fitted in the extension
\eqref{eqn5.3.1}. This is a contradiction.
\enprf
\end{Proof}

\begin{Remark}
The existence of $\mu$-stable vector bundles $E$ on $P$ such that
$c_1(E)=0$, $c_2(E)=c_2$ and $r(E)=r$ for the $c_2$ and $r$ given in
the above proposition is a very special case of the result by
J.M. Drezet and J. Le Potier \cite{key6}.

In view of Theorem~\ref{Theorem4.8}, we can say, on stable vector bundles
in $V(n,n)$, a little more than Corollary~\ref{cor1.4.1} under the
additional assumption \eqref{eqn1.10.2}.
\end{Remark}

\begin{Prop}\label{Prop5.5}
Let $F$ be a stable vector bundle in $\mathscr{V}$.
\begin{enumerate}
\renewcommand{\labelenumi}{(\theenumi)}
\item  Both $\Phi(F)$ and $\Phi(F)$ are contained in $\mathscr{C}$. 
\item  We have an exact sequence of vector bundles 
$$
0\to E\to F\to \mathscr{O}^{\oplus r}_P\to 0
$$
with\pageoriginale $E$ $\mu$-stable, where $r=h^{0}(P,F^{\ast})$. In
particular, if $H^{0}(P,F^{\ast})=0$, then $F$ is $\mu$-stable.
\end{enumerate}
\end{Prop}

\begin{Proof}
(1) is corollary to Corollary~\ref{cor1.4.1} and Theorem~\ref{Theorem4.8}. We
proved (2) in Corollary~\ref{cor1.4.1} except for the $\mu$-stability
of $E$. Assume that $E$ is not $\mu$-stable. Since $E$ is
$\mu$-semi-stable, there is a coherent subsheaf $E_1$ of $E$ such that
$c_1(E_1)=0$, $E_1$ is locally free and $r(E_1)<r(E)$. Set
$E_2=E/E_1$. Then $E_2$ is torsion free and for general lines
$\ell$ in $P$, we see that $E_{l}\mid_{\ell}$, $E\mid_{\ell}$ and
$E_{2}\mid_{\ell}$ are all trivial, which particularly implies that
$q_{\ast}(p^{\ast}(E_2))(-1,-1))$ is torsion. On the other hand, since
$p^{\ast}(E_2)$ is torsion free, thanks to the flatness of $p$, so is
$q_{\ast}(p^{\ast}(E_2)(-1,-1))$. Hence
$q_{\ast}(p^{\ast}(E_2)(-1,-1))=0$. Therefore, putting
$L_i=R^{1}q_{\ast}(p^{\ast}(E_i)(-1,-1))$ and
$L=R^{1}q_{\ast}(p^{\ast}(E)(-1,-1))=\Phi(F)$, we obtain the following
exact sequence: 
$$
0\to L_1\to L\to L_2 \to 0.
$$
Since $R^{i}P_{\ast}(p^{\ast}(E_j)(-1,-1))=0$ for all $i$ and $j$ (see
Lemma~\ref{LEM2.13.1} ), $H^{i}(\mathbb{F},p^{\ast}(E_j)(-1,-1))=0$
for all $i$ and $j$. Combining this and the fact that
$R^{i}q_{\ast}(p^{\ast}(E_j)(-1,-1))=0$ unless $i=1$, we know that
$L_1$ and $L_2$ have the property \eqref{eqn2.3.1}. By virtue of
Proposition~\ref{Prop2.9}, (4), we get the exact sequence 
$$
0\to L'_2\to L'\to L'_1\to 0.
$$
Then it is obvious that $L_1$ has the property \eqref{eqn2.3.2}
because so does $L$. By Lemma~\ref{lemma5.1}, $L$ is a simple object
in $\mathscr{C}$ and hence either\pageoriginale $L_1$ or $L_2$ vanishes. On the
other hand, it is not difficult to see that $\deg S(L_i)=c_2(E_i)$
(cf. \cite{key11} Proposition~\ref{Prop1.7}). Thus $c_2(E_1)=0$ or
$c_2(E_2)=0$. Applying Lemma~\ref{lemma1.3} to $E_i$, we get that
$E_1$ or $E_2$ is trivial. The former violates the condition that
$H^{0}(P,F)=0$ and the latter contradicts the fact that
$H^{0}(P,E^{\ast})=0$. This completes the proof.
\end{Proof}

Let $M'(r,n)^{\mu}_0$ be the moduli space of $\mu$-stable vector
bundles $E$ of rank $r$ and $P$ with $c_1(E)=0$ and $c_2(E)=n$ which
have the property \eqref{eqn1.10.2}. Then $M'(r,n)^{\mu}_0$ is an open
subscheme of $M(r,n)^{\mu}_0$. If the base field $k$ is of
characteristic zero, then $M'(2,n)^{\mu}_0=M(2,n)^{\mu}_0$, by the
theorem of GrauertM\"ulich. If we define $M'(n,n)_0$ to be the moduli
space of stable vector bundles $E$ with $c_1(E)=0$ and $c_2(E)=r(E)=n$
which have the property \eqref{eqn1.10.2}, then $M'(n,n)_0$ is also an
open subscheme of $M(n,n)_0$ and $\psi(r,n)(M'(r,n)^{\mu}_0)=
\psi(r,n)(M(r,n)^{\mu}_0)\cap M'(n,n)_0$ (see
Proposition~\ref{Prop1.7}). Let $\psi'(r,n)$ be the morphism of
$M'(r,n)^{\mu}_0$ to $M'(n,n)_0$ induced by $\psi(r,n)$ of
Proposition~\ref{Prop1.7}. Relations among $M'(r,n)^{\mu}_0$ are
clearer than those among $M(r,n)^{\mu}_0$ which we have seen in
Proposition~\ref{Prop1.7}.

\begin{Theorem}\label{Theorem5.6}
There are subschemes $Z(r,n)$ in $M'(n,n)_0 (r=2,3,\ldots, n)$ such
that {\rm(1)} $\psi'(r,n)$ induces an isomorphism of $M'(r,n)^{\mu}_0$ to
$Z(r,n)$, {\rm(2)} $M'(n,n)_0=\coprod\limits^{n}_{r=2}Z(r,n)$ and {\rm(3)}
$\overline{Z(r,n)}=\coprod\limits_{s\leqq r}Z(s,n)$, where
$\overline{\phantom{AAAA}}$ means the
closure in $M'(n,n)_0$.
\end{Theorem}

\begin{Proof}
By\pageoriginale Proposition~\ref{Prop1.7} and
Proposition~\ref{Prop5.5}, (2), our 
assertions are obvious except for (3). (3) is also easy if one
takes the irreducibility of the moduli space of stable bundles on $P$
(\cite[see, for example,]{key6}) into account. 
\end{Proof}

Let us close this article by the following question.

\begin{Ques}
Let $\overline{M}(n,n)$ be the moduli space of semi-stable sheaves $E$
of rank $n$ on $P$ with $c_1(E)=0$ and $c_2(E)=n$ (\cite{key10}). It
is known that $\overline{M}(n,n)$ is a projective, normal
variety. What is the closure of $Z(r,n)$ in $\overline{M}(n,n)$?
\end{Ques}


\begin{thebibliography}{99}
\itemsep=2pt
\bibitem{key1}
{W. Barth}, Some properties of stable rank-$2$ vector bundles
on $\mathbb{P}^{n}$, \textit{Math. Ann}., 226,1977, 125--150.


\bibitem{key2}
{W. Barth}, Moduli of vector bundles on the projective plane,
\textit{Invent. Math}., 42, 1977, 63--91.

\bibitem{key3}
{W. Barth and K. Hulek}, Monads and moduli of vector bundles,
\textit{Manuscr. Math}., 25, 1978, 323--347.

\bibitem{key4}
{A. Beauville}, Variet\'{e}s de Prym et jacobiennes
interm\'{e}diaires, \textit{Ann. Sci. \'{E}cole Norm. Sup}., 10, 1977,
304--391.

\bibitem{key5}
{H. Cartan and  S. Eilenberg},\pageoriginale \textit{Homological Algebra},
Princeton Univ. Press, Princeton, New Jersey, 1956.

\bibitem{key6}
{J. M. Drezet and  J. Le Potier}, Fibr\'{e}s stables et
fibr\'{e}s exceptionnels sur $\mathscr{P}_2$, \textit{Annales
Scient. \'{E}c. Norm. Sup.,} 18, 1985, 193--244.

\bibitem{key7}
{K. Hulek}, On the classification of stable rank-$r$ vector
bundles over the projective plane, Proceedings, \textit{Vector bundles
and differential equations}, Nice, 1979 (ed. A. Hirschowitz), progress
in Math. 7, Birkhauser, Boston-Basel-Stuttgart.

\bibitem{key8}
{S. Kleiman}, Les th\'{e}or\'{e}mes de finitude pour le foncteur
de Picard, S\'{e}minaire de Geom\'{e}trie Algebrique du Bois Marie
1966/67, Lect. Notes in Math., 225, Springer-Verlag,
Berlin-Heidelberg-New York-Tokyo, 1971.

\bibitem{key9}
{M. Maruyama}, Moduli of stable sheaves, I, \textit{J. Math. Kyoto
  Univ}., 17, 1977, 91--126.

\bibitem{key10}
{M. Maruyama}. Moduli of stable sheaves, II,
\textit{J. Math. Kyoto Univ}., 18, 1978, 557--614.

\bibitem{key11}
{M. Maruyama}, Singularities of the curves of jumping lines of a
vector bundle of rank $2$ on $\mathbb{P}^{2}$, \textit{Algebraic
  Geometry}, Proc. of Japan-France Conf., Tokyo and Kyoto, 1982,
Lect. Notes in Math., 1016, Springer-Verlag, Berlin-Heidelberg-New
York-Tokyo, 1983.

\bibitem{key12}
{J.-P Serre},\pageoriginale Sur les modules projectifts,
\textit{S\'{e}m. Dubreil Pisot} 1960/61.
\end{thebibliography}

\bigskip

\noindent
Department of Mathematics\\
Faculty of Science\\
Kyoto University\\
Kyoto 606, Japan


\newpage

~\phantom{a}
\thispagestyle{empty}





