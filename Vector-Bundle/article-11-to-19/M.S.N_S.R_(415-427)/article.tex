\title{$2\theta$-Linear Systems On Abelian Varieties}
\markright{$2\theta$-Linear Systems On Abelian Varieties}

\author{By M. S. Narasimhan and S. Ramanan}
\markboth{M. S. Narasimhan and S. Ramanan}{$2\theta$-Linear Systems On Abelian Varieties}

\date{}
\maketitle


\setcounter{page}{315}

\setcounter{pageoriginal}{414}
\section{Introduction and Statement of Main Theorem}\label{s1}\pageoriginale

We would like to consider the $2\theta$-linear system on an abelian
variety with a principal polarisation $\theta$. In the case when the
abelian variety is the Jacobian of a projective non-singular curve
$X$, there is a close relationship between semistable vector bundles
of rank $2$ and trivial determinant on $X$ and the $2\theta$-linear
system. On the one hand, one can describe the moduli $SU_X(2)$ of such
bundles on $X$ in terms of the $2\theta$-linear system. On the other,
classical questions regarding Kummer varieties or the Schottky
relation may be better understood in terms of this `new' variety
$SU_X(2)$. 

We first considered this relationship some fifteen years ago in the
case of genus $2$ and proved 

\begin{THM}[\cite{key5}]\label{THM1}
The variety $SU_X(2)$ is canonically isomorphic to the projective
space of divisors on the Jacobian, linearly equivalent to $2\theta$. 
\end{THM}

This result was somewhat of a surprise for the following reason. For
every line bundle $j$ on any curve $X$, consider the semistable bundle
$j\oplus j^{-1}$. This imbeds in $SU_X(2)$, the Kummer variety
$\mathscr{K}=j/i$, where $J=\pic^{\circ}(X)$ is the Jacobin of $X$ and
$i$ is the involution\pageoriginale $x\to x^{-1}$ of $J$. It is easy to see that
$SU_X(2)-\mathscr{K}$ is smooth. That $SU_X(2)$ is itself smooth in
the case of genus $2$ is surprising n view of the following  

\begin{THM}[\cite{key5}]\label{THM2}
The kummer variety $\mathscr{K}$ is precisely the singular locus of
$SU_X(2)$, is $g\geq 3$.
\end{THM}

Some of the ideas relating to Theorem~\ref{THM1} have been
generalised \cite{key2} to hyperelliptic curves of arbitrary genus
$g\geq 2$. In particular, we have 

\begin{THM}[\cite{key2}]\label{THM3}
If $X$ is hyperelliptic of genus $3$ and $i$ the involution of
$SU_X(2)$ induced by the hyperelliptic involution on $X$, then
$\dfrac{SU_X(2)}{i}$ is a quadric in $\mathbb{P}^{7}$. 
\end{THM}

The aim of this paper is the following generalisation of
Theorem~\ref{THM1}.

\begin{MT}
\textit{If $X$ is non-hyperlliptic of genus $3$, then $SU_X(2)$
  isomorphic to a quartic hypersurface in $\mathbb{P}^{7}$}
\end{MT}

In particular, of course, the Kummer variety is imbedded in
$\mathbb{P}^{7}$  and is the singular locus of a quartic
hypersurface. Suppose $f$ is the quartic polynomial in the homogenous
coordinates $(Z_1,\ldots,Z_8)$ defining\\ $SU_X(2)$. Then
Theorem~\ref{THM2} implies that $\mathscr{K}$ is defined by
$\dfrac{\partial f}{\partial z_{i}}=0,i=1,\ldots 8$. Thus we have, as
an application, the 

\begin{corr}
\textit{$\mathscr{K}$ can be defined by cubic polynomials}.
\end{corr}

Wirtinger \cite{key7} had shown that $\mathscr{K}$ can be defined by
quartics and\pageoriginale it was an open problem if cubics would suffice. (See
Coble (\cite[p. 106]{key1})). 

\section[The relationship between vector bundles......]{The relationship between vector bundles\\ and \texorpdfstring{$2\theta$}{eq}-linear
  systems}\label{s2}

Let us now make explicit the map of $SU_X(2)$ into the projective
linear system $P$ of $2\theta$, which exists for any $g$. If we denote
by $J^{d}$ the space of line bundles on $X$ of degree $d$, notice that
the natural divisor $\theta$ lives only in $J^{g-1}$ Hence the linear
system of $2\theta$ is a system of divisors in $J^{g-1}$. The map that
we have in mind associates to a vector bundle $E$ of rank $2$ with
trivial determinant, the subset $D_E=\left\{\xi \epsilon J^{g-1}:\Gamma
(\xi \otimes E)\neq 0\right\}$. If $E$ is not semistable, then it has a line
subbundle $L$ of positive degree. For every $\xi \epsilon J^{g-1}$,  we
have $\Gamma(\xi \otimes L)\neq 0$ so that $D_E=J^{g-1}$ in this
case. On the other hand, Raynaud \cite{key6} showed that if $E$ is
semistable, the $D_E$ is a proper subset of $J^{g-1}$ Then it is easy
to see that $D_E$ is the support of a divisor linearly equivalent to
$2\theta$. 

Actually, one can associate a divisor $D_E$ to $E$ (and not merely its
support) as follows. Take $x_1,\ldots x_N\epsilon X$, with $N$
sufficiently large. If $Z$ is the divisor $x_1+\cdots + x_N$,
tensorise the exact sequence 
$$
0\to \mathscr{O}(-Z)\to \mathscr{O}_{X}\to \mathscr{O}_Z\to 0
$$
with $\xi \otimes E$. If $N$ is large enough so that $\Gamma(\xi
\otimes E\otimes \mathscr{O}(-Z))=0$, then clearly $\Gamma(\xi \otimes
E)$ is the kernel of the connecting homomorphism 
$$
\sum\limits_{i=1}^{N}(\xi\otimes E)_{x_{i}}\to H^{1}(\xi\otimes
E\otimes \mathscr{O}(-Z))
$$
We\pageoriginale will now allow $\xi$ to vary over $J^{g-1}$. In others words,
taking $\xi$ to be the universal line bundle of degree $g^{-1}$
parametrised by $J^{g-1}$ we get two vector bundles on $J^{g-1}$, both
of rank $2N$ and a homomorphism. This defines a section of a line
bundle, namely the difference between their determinants. It is easy
to compute this line bundle and show that it is isomorphic to
$2\theta$. 

\section{Generalities on polarised abelian varieties}\label{s3}

Let $A$ be an abelian variety and $\tau$ a principal polarisation on
$4$. Consider $B=\pic ^{\tau}A$, namely the space of isomorphism
classes of line bundles on $A$ with Chern class $\tau$. Then $A$ acts
on $B$ by $a.\xi$=translation by $a$ of $\xi$. All line bundles $\xi
\epsilon B$ are ample and hence $H^{i}(\xi)=0$ for $i\geq 1$ (\cite[p. 150]{key3}). On the other hand, $\Gamma(\xi)$ is $1$-dimensional, since
the polarisation is principal. Hence the line bundles $\xi\epsilon B$
may be identified with their divisors in $A$. Let $\theta \subset B$
be defined by 
$$
\theta\{\xi \epsilon B: \text{ divisor of } \xi \text{ passes through
} 0\}
$$
This belongs to the principal polarisation $\tau$ in the sense, that
for any choice of a point in $B$, the natural identification $A\to
B$ leads to a divisor in $A$ whose class is $\tau$. Moreover,
$A\coprod B$ can be made into a group, using the involution $i:B\to B$
given by $\xi\to i^{\ast}\xi$ where $i$ is the morphism $x\to -x$ of
$A$. We have then a natural exact sequence 
$$
0\to A\to A\coprod B\to \mathbb{Z}/2 \to 0.
$$
Of\pageoriginale course, this sequence splits, but the point is that there is no
canonical splitting. Elements of $A_2=\{a\epsilon A: 2a=0\}$ are
called \textit{period characteristics} and elements of
$B_2=\{b\epsilon B:2b=0\}$ are called \textit{theta characteristics}. As a
group $A_2$ is isomorphic to $\mathbb{F}^{2}_2g$ and the principal
polarisation on $A$ gives a nondegenerate alternating form $\tau$ on
$A_2$. Moreover, $B_2$ can be identified with quadratic forms on $A_2$
whose associated alternating form is $\tau$. The Arf invariant then
distinguishes between odd and even theta characteristics. We have the
well-known \cite{key4}

\begin{TM}
\textit{The parity of the multiplicity of the divisor $\theta$ at
  $b\epsilon B_2$ determines the parity of the theta characteristic.}
\end{TM}

On $A\times B$, we may consider the divisor obtained by pulling back
$\theta$ by the natural group action $A\times B\to B$. This bundle
serves as a Poincar\'e bundle for line bundles on $A$ with Chern class
$\tau$, and also as one for line bundles on $B$ with Chern class
$\tau$. Indeed, the space $B$ and the divisor $\theta$ are
characterized by this property. It then follows that if $A$ is the
Jacobian of a curve $X$ of genus $g$, then $B$ may be identified with
$J^{g-1}$. 

Although there is no canonical $\theta$-divisor on $A$, there is a
divisor class $\Phi$ with Chern class $2\tau$. For every $b\epsilon
B$, $T^{\ast}_b\theta$, is the divisor in $A$, corresponding to
$b$. The linear equivalence class of
$T^{\ast}_b\theta+T^{\ast}_{-b}\theta$ is independent of $b\epsilon B$
and this defines $\Phi$. More elegantly, this may be constructed as
follows. Consider the map $\mu:A\times B\to B\times B$ given by
$(a,b)\mapsto (a+b,a-b)$. The pull back of tensor\pageoriginale product of
$P^{\ast}_1\mathscr{O}(\theta)\otimes P^{\ast}_2\mathscr{O}(\theta)$
on $B\times B$ to $A\times B$ may be looked upon as a family of line
bundles on $B$ parametrised by $A$. But then for any $a$, this gives
the isomorphism class of $\mathscr{O}(2\theta)$. Hence there is a line
bundle $\Phi$ on $A$ such that 
$$
P^{\ast}_1\phi \otimes P^{\ast}_2\mathscr{O}(2\theta)\approx
\mu^{\ast}\left(P^{\ast}_1\mathscr{O}(\theta)\otimes P^{\ast}_2\mathscr{O}(\theta)\right).
$$
Moreover, the situation comes with a section of the line
$P^{\ast}_1\Phi \otimes P^{\ast}_2\mathscr{O}(2\theta)$, so that we
have a canonical element of $\Gamma(A,\Phi)\otimes
\Gamma(B,\mathscr{O}(2\theta))$. This can be proved to give a perfect
pairing so that we have 

\begin{Theorem}\label{Theorem3.1}
There is a canonical duality between $\Gamma(A,\Phi)$ and\\
$\Gamma(B,\mathscr{O}(2\theta))$. 
\end{Theorem}

We have a natural morphism $A\to P\Gamma (A,\Phi)^{\ast}$ given by
linear system of $\Phi$. On the other hand, for $a\epsilon A$, we have
the divisor $T_a\theta+T_{a}\theta$ linearly equivalent to
$2\theta$. In other words, we have a morphism $A\to
P\Gamma(B,\mathscr{O}(2\theta))$. Under the duality of
Theorem~\ref{Theorem3.1}, these two morphism are compatible. 

When $A$ is the Jacobian and $B=J^{g-1}$ of the curve $X$, we may
interpret the morphism $A\to P\Gamma(B,\mathscr{O}(2\theta))$ defined
above as follows. For any $a\epsilon A$, consider the semistable
vector bundle $E=a\oplus (a)^{\ast}$. Then $E$ is of rank $2$ and
trivial determinant so that there is a divisor $D_E$ linearly
equivalent to $2\theta$ defined in \S\ 2. It is clear that this is
the same as the map given above. Thus one may say that the morphism
$SU_X(2)\to P\Gamma(J^{g-1},\mathscr{O}(2\theta))$ given by $E\mapsto
D_E$ restricts to the Kummer variety $\mathscr{K}$ as the morphism
given by the linear system of $\Phi$. It has been shown by Andreotti
and\pageoriginale Meyer over $\mathbb{C}$ that for $\theta$ irreducible, $\Phi$
imbeds the Kummer variety. This does not seem to have been proved in
the literature for characteristic $P>0$. It is easy enough to show
that if $\theta$ is irreducible, then the Kummer variety is mapped
injectively. We can prove (following ideas of Wirtinger\cite{key7})
over any field of characteristic $\neq 2$, the following 

\begin{TM}
\textit{Let $A$ be any abelian variety, $\tau$ a principal
  polarisation on it and $B=\pic^{\tau}A$. If no even theta
  characteristic lies on the canonical theta divisor in $B$,then
  $\mathscr{K}$ is imbedded by the canonical divisor class $\Phi$ in
  $A$ with Chern class $2\tau$.}
\end{TM}

\section{Proof of the Main Theorem}\label{s4}

We will sketch here the alinea of proof of our main theorem. The
detailed proofs will appear elsewhere. 

\begin{lemma}\label{lemma4.1}
Let $\xi$ be a line bundle of degree $1$ on $X$. Then $X$ may be
imbedded in $J$ by $x\to \xi^{-1}\otimes \mathscr{O}(x)$. The induced
map $\Gamma(J,\Phi)\to \Gamma(X,K\otimes \xi^{2})$ is onto. 
\end{lemma}


\begin{Proof}
Any divisor linearly equivalent to $K\otimes \xi^{2}$ may be written
as a sum of two divisors of degree $g$ each. Let these belong to
classes $K\otimes \xi \otimes \alpha^{-1}$ and $\xi \otimes \alpha$
for $\alpha \epsilon J^{g-1}$. It is easy to see that `most' sections
in $\Gamma(K\otimes \xi^{2})$ can be thus split up in which
$\Gamma\left(K\otimes \xi \otimes \alpha^{-1}\right)$ and
$\Gamma(\xi\otimes \alpha)$ are of dimension one each. It is clear
then that these divisors are images of elements in $\Gamma(\Phi)$
given by $\alpha \theta +\alpha^{-1}\theta \sim \Phi$. 

Consider the injective map 
$$
H^{1}\left(X,\xi^{-2}\right)\xrightarrow{\sim} \Gamma\left(X,K\otimes
\xi^{2}\right)^{\ast}\to
\Gamma(J,\Phi)^{\ast}\xrightarrow{\sim}\Gamma(J^{g-1},\mathscr{O}(2\theta)). 
$$
We\pageoriginale wish now to interpret the map $PH^{1}\left(X,\xi^{-2}\right)\to
P\Gamma(J^{g-1},\mathscr{O}(2\theta))$. 
\end{Proof}

\begin{lemma}\label{lemma4.2}
Let $\xi$ be a line bundle on $X$, of degree $1$. Any point of
$PH^{1}\left(X,\xi^{-2}\right)$ gives a nontrivial extension $0\to
\xi^{-1}\to E\to \xi \to 0$, where $E$ is a semi-stable vector bundle
of rank $2$ and trivial determinant. 
\end{lemma}

\begin{Proof}
See (\cite[Lemma~\ref{lemma5.1}]{key5}).

By allowing $\xi$ to vary over $J^{1}$, one can thus construct a bundle
$V$ on $J^{1}$ whose fibre over $\xi$ is identifiable with
$H^{1}\left(X,\xi^{-2}\right)$, and an exact sequence 
$$
0\to V\to \Gamma (J^{g-1},\mathscr{O}(2\theta))_J1\to W\to 0
$$
on $J^{1}$ It can be shown that the fibre $V_{\xi}$ is the subspace of
$\Gamma(J^{g-1},\mathscr{O}(2\theta))$ consisting of sections which
vanish on $\xi X$. 
\end{Proof}

\begin{lemma}\label{lemma4.3}
The map $PH^{1}\left(X,\xi^{-2}\right)\to P\Gamma(J^{g-1},(2\theta))$
can be interpreted as follows. If $E$ is the vector bundle associated
to $\upsilon \epsilon PH^{1}\left(X,\xi^{-2}\right)$, then the image of
$\upsilon$ is the divisor $D_E$. 
\end{lemma}

\begin{lemma}\label{lemma4.4}
Let $X$ be of genus $3$. If $E\epsilon SU_X(2)$ is stable and $\xi
X\subset D_E$ for some $\xi\epsilon J^{1}$, then $\Gamma(\xi\otimes
E)\neq 0$. 
\end{lemma}

These two lemmas ensure that the morphism $E\mapsto D_E$ is
injective. Firstly, it is easy to show (see 5.4 below for a slightly
stronger statement) that there exists $\xi \epsilon J^{1}$ with
$\Gamma(\xi \otimes E)\neq 0$. This implies of course that $\xi
X\subset D_E$. If then $D_E=D_E$ it follows from 4.4 that
$\Gamma(\xi\otimes E')\neq 0$ as well, which implies that both\pageoriginale $E$ and
$E'$ occur as extensions of $\xi$ by $\xi^{-1}$. Now 4.3 implies that
$E\mapsto D_E$ is injective.

We have also to show that the morphism $E\to D_E$ is an
imbedding. From the injectivity, we see that the image is a
hypersurface in $\mathbb{P}^{7}=\mathbb{P}\Gamma
\left(J^{g-1}\mathscr{O}(2\theta)\right)$. We will compute the differential
of the morphism $P(V)\to \mathbb{P}^{7}$. Let $E\epsilon SU_X(2)$ be a
stable bundle. Then we will show that in the diagram 
$$
\xymatrix{P(V)\ar[dr]\ar[rr]&&\mathbb{P}^{7}\\
&SU_{X}(2)\ar[ur]&}
$$
there exists a point of $P(V)$ lying over $E$ at which the
differential of the map $P(V)\to \mathbb{P}^{7}$ is injective. It
would follow that the hyper surface in question is normal, since the
set of nonstable points is of codim $3$ in $SU_X(2)$. Hence by
Zariski's Main Theorem, we are through. 

\section{Computation of differential}\label{s5}

But the computation of the differential of the morphism $P(V)\to
SU_X(2)$ turns out to be somewhat hard. The result is 


\begin{lemma}\label{lemma5.1}
The differential is injective at every point of
$PH^{1}\left(\xi^{-2}\right)$ corresponding to stable $E$, if
$\xi^{2}\notepsilon \theta$. 
\end{lemma}

Since the bundle $V$ is induced by a map $\varphi$ of the Jocobain
$J^{1}$ into the Grassmannian of $4$ dimensional subspaces of
$\Gamma(J^{2},\mathscr{O}(2\theta))$,\pageoriginale Lemma~\ref{lemma5.1} can be
proved by computing the differential of this map and a simple
computation of the differential of the natural map of the projective
bundle associated to the universal subbundle on the Grassmannian, into
the projective space. Thus it is easy to see that Lemma~\ref{lemma5.1}
would follow from the following statement regarding the differential
of $\varphi$ which is a map $H^{1}(X,\mathscr{O})\to
\hom\left(V_{\xi}\Gamma\left(K^{2}\otimes \xi^{-2}\right)\right)$. 

\begin{lemma}\label{lemma5.2}
Let $\upsilon\in T_{\xi}\left(J^{1}\right)$, $\xi^{2}\!\not\!\!\!\epsilon~
\theta$. (a) if $\upsilon$ does not belong to any $1$-dimensional
subspace of $H^{1}(X,\mathscr{O})$ corresponding to a point of $X$ in
its canonical imbedding, then the image of $\upsilon$ under the
differential of $J^{1}\to
Gr_4\left(\Gamma\left(J^{2},\mathscr{O}(2\theta)\right)\right)$ is an injective
map $V_{\xi}\to \Gamma \left(K^{2}\otimes \xi^{-2}\right)$. (b) if
$\upsilon$ belongs to such an $1$-dimensional sub-space, then the
image has $1$-dimensional kernel and this, as a point of $P(V_{\xi})$,
corresponds to a nonstable extension. 
\end{lemma}

The space $V_{\xi}$ can also be identified with the kernel of
$\Gamma\left(J^{2},\mathscr{O}(2\theta)\right)\\\to
\Gamma(X,K^{2}\otimes \xi^{-2})$ obtained by imbedding $X$ as $\xi X
\in J^{2}$. In other words, $V_{\xi}=\Gamma\left(J^{2},I_X\otimes
\mathscr{O}(2\theta)\right)$, where $I_X$ is the sheaf of ideals of
$\xi X$ in $J^{2}$. From the exact sequence 
$$
0\to I^{2}_X\to I_X\to N^{\ast}\to 0
$$
we obtain a natural map 
$V_{\xi}\to \Gamma\left(N^{\ast}\otimes K^{2}\otimes \xi^{-2}\right)$. It can be
shown that the required differential factors as follows. 
$$
\xymatrix{H^{1}(X,\mathscr{O})\ar[dr]\ar[r]& \hom\left(V_{\xi}\Gamma\ar[d]\left(K^{2}\otimes \xi^{-2}\right)\right)\\
&\hom\left(\Gamma\left(N^{\ast}\otimes K^{2}\otimes
\xi^{-2}\right),\Gamma\left(K^{2}\otimes \xi^{-2}\right)\right)}
$$
The\pageoriginale vertical map is induced by the map $V_{\xi}\to \Gamma
\left(N^{\ast}\otimes K^{2}\otimes \xi^{-2}\right)$ mentioned
above. The map $H^{1}(X,\mathscr{O})\to \hom \left(\Gamma
\left(N^{\ast}\otimes K^{2}\xi^{-2}\right)\right)$,\\ $\left(\Gamma\left(K^{2}\otimes
\xi^{-2}\right)\right)$ can be determined explicitly by using the
natural inclusion $N^{\ast}\to \Gamma (X,K)_X$. In fact, it is the
composite of the natural map 
$$
H^{1}(X,\mathscr{O})\to \hom\left(\Gamma(X,K)\otimes
\Gamma\left(K^{2}\otimes \xi^{-2}\right), \Gamma\left(K^{2}\otimes \xi^{-2}\right)\right)
$$
and the map $\hom\left(\Gamma(X,K)\otimes \Gamma\left(K^{2}\otimes
\xi^{-2}\right),\Gamma\left(K^{2}\otimes \xi^{-2}\right)\right)\to\\
\hom\left(\Gamma\left(N^{\ast}\otimes K^{2}\otimes
\xi^{-2}\right),\Gamma \left(K^{2}\otimes \xi^{-2}\right)\right)$
induced by $N^{\ast}\to \Gamma(X,K)_X$. In view of this,
Lemma~\ref{lemma5.2} will follow from 

\begin{lemma}\label{lemma5.3}
The natural map $V_{\xi}\to \Gamma\left(N^{\ast}\otimes K^{2}\otimes
\xi^{-2}\right)$ is an isomorphism if $\xi^{2}\notepsilon \Supp
\theta$. Geometrically speaking, there is no divisor linearly
equivalent to $2\theta$ which vanishes on $\xi X$ with multiplicity
$2$, f $\xi^{2}\notepsilon \Supp \theta$.
\end{lemma}

The proof that $E\mapsto D_E$ is an imbedding is completed by proving 

\begin{lemma}\label{lemma5.4}
Every stable bundle $E$ can be obtained as an extension $0\to
\xi^{-1}\to E\to \xi\to 0$, $\xi\epsilon J^{1}$ and
$\xi^{2}\notepsilon \Supp \theta$. 
\end{lemma}

To prove that the image is a quartic surface, we first observe that
the map $\mathbb{P}(V)\to \mathbb{P}^{7}$ is generically finite and is
of degree $8$. If $\tau_{V}$ is the tautological hyperplane bundle
along the fibres it suffices to show that
$[c_1(\tau_V)]^{6}[P(V)]=32$. This follows from 

\begin{lemma}\label{lemma5.5}
Let $c_i(i=1,2,3)$ denote the Chern classes of the bundle $V$ and let
$c(\theta)\epsilon H^{2}\left(J^{1}, \mathbb{Z}\right)$ be the
cohomology class defined by a $\theta$-divisor in $J^{1}$. Then we
have \pageoriginale
\begin{enumerate}
\renewcommand{\labelenumi}{(\theenumi)}
\item 
$$
\begin{aligned}
c_1(V)&=-4c(\theta),c_2(V)=8[c(\theta)]^{2}and\\
c_3(V)&{}=-\dfrac{16}{3}[c(\theta)]^{3}
\end{aligned}
$$
\item $[c_1(\tau_V)]^{6}[P(V)]=\left(-c^{3}_1+2c_1 c_2-c_3\right)\left[J^{1}\right]$. 
\end{enumerate}
\end{lemma}


\begin{thebibliography}{99}
\bibitem{key1}
{A. B. Coble.} Algebraic geometry and theta functions,
\textit{A.M.S Colloquium Publications} X, 1929.

\bibitem{key2}
{U. V. Desale and S. Ramanan.} Classification of vector bundles
of rank $2$ on hyperelliptic curves, \textit{Inventiones Math.}
38(1976) 161--185.

\bibitem{key3}
{D. Mumford.} Abelian varieties, \textit{T.I.F.R. Studies in
  Maths.} O.U.P, 1974.

\bibitem{key4}
{D. Mumford.} On the equations defining abelian varieties,
\textit{Inventions Math.} 1(1966) 287--354.

\bibitem{key5}
{M. S. Narasimhan and S. Ramanan} Moduli of vector bundles on a
compact Riemann surface. Ann. of Math. 89(1969)19--51.

\bibitem{key6}
{M. Raynaud.} Sections des fibr\'{e}s vectoriels sur une
courbe, \textit{Bull. Soc. Math. France} 110(1982)103--125.

\bibitem{key7}
{W. Wirtinger.}\pageoriginale Untersuchugen \"{u}ber Thetafunktionen,
\textit{Teubner}, 1895.
\end{thebibliography}

\vskip 1cm

\noindent
School of Mathematics,\\
Tata Institute of Fundamental Research,\\
Homi Bhabha Road, Colada,\\
Bombay - 400 005.\\
INDIA.
