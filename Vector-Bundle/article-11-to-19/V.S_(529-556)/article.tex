\title{Zero Cycles On A Singular Surface: An Introduction}
\markright{Zero Cycles On A Singular Surface: An Introduction}

\author{By V. Srinivas}
\markboth{V. Srinivas}{Zero Cycles On A Singular Surface: An Introduction}

\date{}
\maketitle


\setcounter{page}{401}
\setcounter{pageoriginal}{528}

\setcounter{section}{-1}
\section{}\pageoriginale
This article consists, for the most part, of an introduction
to the paper \cite{Srinivasb} (which will appear elsewhere), and an
outline of the proof of the infinite dimensionality theorem for zero
cycles which is proved there. The main object under consideration is
the Chow (cohomology) group of zero cycles on a normal
quasi-projective surface over an algebraically closed field. After
briefly surveying the known results for Chow groups of smooth
surfaces, we give the various equivalent definitions of the Chow group
of a singular surface, and mention results of Collino, Levine,
Pedrini, Weibel and the author giving generalisations to the singular
case of the above results for smooth surfaces. Next, we introduce
relative $K$-theory and use it to formulate a conjecture about the
structure of the Chow group of a normal surface. We then outline the
proof of the infinite dimensionality theorem. Finally, we make some
remarks about the related problems of the structure of $NK_0$ and
$K_{-1}$. We give a condition in terms of Picard groups which implies
the nonvanishing of $NK_0$ and $K_{-1}$ of a normal quasi-projective
surface over $\mathbb{C}$; we conjecture that this condition is in
fact necessary and sufficient for surfaces over any algebraically
closed field $k$.  

\section{The smooth case}\label{s1}\pageoriginale

Let $X$ be a smooth surface over an algebraically closed field $k$. The Chow group of zero cycles on $X$ is defined by $$
CH^{2}(X)=\dfrac{\text{ Free abelian group on points of } X}{\text{ cycles rationally equivalent to zero }}
$$
where the group of cycles rationally equivalent to zero is generated by cycles of the form $(f)_C$, where $C\subset X$ is a curve, $f$ a non-zero rational function on $C$ and $(f)_C$ denotes the divisor of $f$. 


The Grothendieck group of vector bundles $K_0(X)$, which also equals the Grothendieck group of coherent sheaves, has a filtration by codimension of support $\left\{F^{i}K_0(X)\right\}_{0\leq i \leq 2}$, where 

$F^{0}/F^{1}\simeq \mathbb{Z}, F^{1}/F^{2}\simeq \pic\ X$, and $F^{2}$ is the subgroup of $K_0(X)$ generated by classes of residue fields of points. We have the cycle map $\Psi:CH^{2}(X)\to F^{2}K_0(X)$, and the second Chern class $C_2:F^{2}K_0(X)\to CH^{2}(X)$; by the Riemann-Roch theorem (or directly, since $X$ is a surface) $\Psi$ and $-C_2$ are inverse isomorphisms. Thus $CH^{2}(X)$ has an alternate description as the sub-group of $K_0(X)$ generated by points of $X$. 

Next, $CH^{2}(X)$ can be interpreted in terms of algebraic $K$-theory. Let $\mathscr{K}_{2,X}$ be the Zariski sheaf associated to the pre-sheaf $U\to K_2(\Gamma (U,\\\mathscr{O}_U))$. Then by a result of \cite{Quillen}, $\mathscr{K}_{2,X}$ has a flasque resolution 
$$
0\to \mathscr{K}_{2, X}\to i_{\ast}K_2(k(X))\xrightarrow{T}\oplus \displaystyle\mathop{(i_C)_{\ast}}_{C: \text{ curves }} k(C)^{\ast}\xrightarrow{\partial}\oplus \displaystyle\mathop{(i_p)_{\ast}}_{P:\text{ points }}\mathbb{Z}\to 0
$$
Here\pageoriginale $K_2(k(X))$ is $K_2$ of the function field $k(X)$, which is given (by a result of Matsumoto - see \cite{Milnor}) by 
$$
K_2(k(X))=\dfrac{k(X)^{\ast}\otimes_{\mathbb{Z}}k(X)^{\ast}}{\langle a \otimes (1-a)|a\in k (X)^{\ast}, a\neq 1\rangle}.
$$ 
$i_{\ast} K_2k(X)$ is the constant sheaf $K_2(k(X))$ on $X$, and similarly $(i_C)_{\ast} k(X)^{\ast}$, $(i_p)_{\ast}\mathbb{Z}$ are the constant sheaves $k(C)^{\ast}$, $\mathbb{Z}$ on $C$ and $P$ respectively. The map $T$ can be explicitly described in terms of the above presentation, while $\partial$ is the sum of divisor maps 
$$
k(C)^{\ast}\to \displaystyle\mathop{\oplus}_{P \epsilon C} \mathbb{Z}[P]
$$
This resolution can be used to compute the cohomology of $\mathscr{K}_{2, X}$; in particular, we obtain the formula $H^{2}(X, \mathscr{K}_{2, X})=CH^{2}(X)$, This is the $K$-theoretic description of $CH^{2}(X)$. 

Suppose now that $X$ is a smooth projective surface over an algebraically closed field. By associating to each zero cycle $\sum n_i P_i$ its degree $\sum n_i$, we obtain a surjection $CH^{2}(X)\to \mathbb{Z}$; let $A_0(X)$ denote the kernel. If $S^{n}(X)$ denotes the $n^{th}$ symmetric product of $X$, so that points of $S^{n}(X)$ parametrize effective zero cycles of degree $n$, we have a natural map 
\begin{gather*}
r_n:S^{n}(X)\times S^{n}(X)\to A_0(X)\\
(A,B)\to [A] - [B]
\end{gather*}
Now assume that $k$ is uncountable i.e. a universal domain in the sense of Weil. We have the following definition of \cite{Mumford}.

\begin{dfn}
$A_0(X)$\pageoriginale is finite dimensional if some $r_n$ is surjective. If none of the $r_n$ is surjective, we say that $A_0(X)$ is \textit{infinite dimensional}. 
\end{dfn}

\begin{thm}[\cite{Mumford}]
Let $k=\mathbb{C}$, and suppose $\Gamma\left(X,\Omega^{2}_X\right)\neq 0$. Then $A_0(X)$ is infinite dimensional. 
\end{thm}

This result is false in characteristic $p>0$. In arbitrary characteristic, we have 

\begin{thm}[\cite{Blocha}]
Let $k$ be a universal domain of arbitrary characteristic, and suppose the cycle map 
$$
\pic\ X \otimes_{\mathbb{Z}}\mathbb{Q}_{\ell}\to H^{2}_{et}(X, \mathbb{Q}_{\ell}(1))
$$
is not surjective $(\ell\neq$ characteristic of $k$). Then $A_0(X)$ is infinite dimensional 
\end{thm}

(Bloch conjectures that the converse also holds). 

Next, we mention some results of Roitman on zero cycles. We have a natural map 
$$
\varphi:A_0(X)\to \Alb X 
$$
where $\Alb X$ is the Albanese variety of $X$; if $k=\mathbb{C}$ this can be described as follows: 
$$
\begin{aligned}
&\Alb X=\Gamma \left(X, \omega^{1}_X\right)^{\ast}/\text{ image } \left(H_1(X, \mathbb{Z})\right), \text{ and }\\
&{}\varphi\left(\sum([P_i]-[Q_i])\right)=\left(\omega \to \sum \int^{Q_i}_{P_i}\omega\right), \omega \epsilon \Gamma\left(X, \omega^{1}_X\right); \int^{Q_i}_{P_i}
\end{aligned}
$$
\pageoriginale
denotes the integral along any path joining $P_i$ and $Q_i$. 

\begin{thm}[\cite{Roitmana}, \cite{Roitmanb}]
\begin{enumerate}
\renewcommand{\labelenumi}{(\theenumi)}
\item If $A_0(X)$ is finite\break dimensional, then $\varphi:A_0(X)\to Alb\, X$ is an isomorphism.
\item $\varphi:A_0(X)\to Alb\, X$ is always an isomorphism on torsion subgroups.
\end{enumerate}
\end{thm}

(We note that the $p$-torsion statement in (2), for $p$ = characteristic $k$, is due to \cite{Milne}. Roitman's Proof of (1) has a small gap in characteristic $p$, which is however easily filled-see \cite{Srinivasb}). 

\section{The singular case:}\label{s2}

We now see to what extent this theory generalises to singular surfaces. Firstly, the three definitions of the Chow group in the smooth case all admit generalisations to the singular case. Let $X$ be a normal quasi-projective surface. The cycle theoretic definition is 
$$
\CH^{2}(X)=\dfrac{\text{ Free abelian group on smooth points of } X}{\left<(f)_C\mid C\subset X \text{ is a curve }, f\epsilon k(C)^{\ast}, \text{ and } C\cap X_{\sing}=\phi\right>}
$$
where $X_{\sing}$ denotes the singular locus of $X$, and $C$ runs over \textit{closed curves} in $X$ disjoint from $X_{\sing}$. 

The Grothendieck group of vector bundles $K_0(X)$ is now in general different from the Grothendieck group $G_0(X)$ of coherent\pageoriginale sheaves. However $K_0(X)$ is also the Grothendieck group of coherent sheaves $\mathscr{F}$ such that for each $x\epsilon X$ the stalk $\mathscr{F}_x$ has finite projective dimension over $\mathscr{O}_{x,X}$ (since $X$ is quasi-projective, such an $\mathscr{F}$ has a finite resolution by vector bundles). We have a filtration $F^{i}K_0(X)_{0\leq i\leq 2}$ with $\dfrac{F^{0}}{F^{1}}=\mathbb{Z}$, $\dfrac{F^{1}}{F^{2}}= \pic\ X$. and $F^{2}K_0(X)$ is the kernel of the determinant map $F^{1}K_0(X)\to \pic\ X$. Since any vector bundle is trivial in a neighbourhood of $X_{\sing}$, one checks that $F^{2}K_0(X)$ is also the sub-group generated by residue classes of smooth points. Thus we have a surjective cycle map $\Psi:CH^{2}(X)\to F^{2}K_0(X)$. 

\begin{thm}[\cite{Collino}]
$\Psi:CH^{2}(X)\to F^{2}K_0(X)$ is an isomorphism. 
\end{thm}

We note here that Fulton has defined a Chow (homology) group 
$$
CH_0(X)=
\dfrac{\text{ Free abelian group on all points }}{\left<(f)_C\mid C\subset X \text{ any curve }, f\epsilon k(C)^{\ast}\right>}
$$
This group is related to $G_0(X)$ is an analogous fashion. The natural
map $\CH^{2}(X)\to \CH_0(X)$ is always surjective, but in general not injective. 

Finally, the $K$-theoretic definition $H^{2}(X,\mathscr{K}_{2, X})$ still makes sense, though the resolution obtained in the smooth case is not valid here; however we have: 

\begin{thm1}
$H^{2}(X,\mathscr{K}_{2, X})=CH^{2}(X)$.
\end{thm1}

This result is due to \cite{Collino} when $X$ has $1$ singular point; the general case, along lines similar to Collino's proof, is due independently\pageoriginale to \cite{Levinea} and \cite{Pedrini} Next, we have a result of Levine generalizing Roitman's theorem on torsion $0$-cycles. 

\begin{thm}[\cite{Levineb}]
Let $X/k=\overline{k}$ be projective, $f:Y\to X$ a resolution of singularities. The composite 
$$
A_0(X)\xrightarrow{f^{\ast}}A_0(Y)\xrightarrow{\varphi} Alb\, Y
$$
is an isomorphism on torsion prime to the characteristic of $k$. 
\end{thm}

\section{}
Now suppose $X$ is a normal projective surface over a universal domain $k$. Let $U=X-X_{\sing}$; then there is a natural map $r_n:S^{n}(U)\times S^{n}(U)\to A_0(X)$ where $S^{n}(U)$ is the $nth$ symmetric product, and $r_n(A, B)=[A]-[B]$. 

\begin{dfn}
$A_0(X)$ is \textit{finite dimensional} if $r_n$ is surjective for some $n$; if none of the $r_n$ are surjective, then $A_0(X)$ is \textit{infinite dimensional}. 
\end{dfn}

We have the following straightforward generalisation of Roitman's finite dimensionality theorem. 

\begin{thm}[\cite{Srinivasb}]
$A_0(X)$ is infinte dimensional $\Leftrightarrow A_0(X)\simeq Alb\, Y$ where $Y\to X$ is a resolution of singularities. 
\end{thm}

As a consequence of this result and the Murthy-Swan cancellation theorem \cite{Murthy}, we see that there are $2$ possible situations for the Chow group over a universal domain: 
\begin{enumerate}
\renewcommand{\theenumi}{\roman{enumi}}
\renewcommand{\labelenumi}{(\theenumi)}
\item (finite dimensional case) for any affine open set $V\subset X$, every\pageoriginale vector bundle on $V$ is the direct sum of a trivial bundle and a line bundle. 
\item (infinte dimensional case) for an arbitrary neighbourhood $V$ of the singular locus $X_{\sing}, CH^{2}(V)$ has uncountable rank. 
\end{enumerate}

In case (ii), one can show that there exist uncountably many non-isomorphic indecomposable vector bundles of rank $2$; this is not immediately clear unless $\pic  X$ is finitely generated (see \cite{Srinivase}). 

We now state the infinite dimensionality theorems proved in \cite{Srinivasb}; the proofs are outlined below. 

\begin{thm1}
\begin{enumerate}
\renewcommand{\labelenumi}{(\theenumi)}
\item \textit{Let $k=\mathbb{C}$, and suppose $X$ is a normal projective surface with $H^{2}(X,\mathscr{O}_X)\neq 0$. Then $A_0(X)$ is infinite dimensional}.

\item \textit{Let $k$ be a universal domain of arbitrary characteristic. Let $X/k$ be a normal, projective surface, $\pi:Y\to X$ a resolution of singularities such that the reduced exceptional divisor $E=\pi^{-1}(X_{\sing})$ has smooth components and normal crossings. Suppose $\pic^{0} Y\to \pic^{0} E$ is not surjective. Then $A_0(X)$ is infinite dimensional.}
\end{enumerate}
\end{thm1}

We reformulate the hypothesis in slightly different terms. The interesting case of the theorem is when $A_0(Y)$ is finite. dimensional for a resolution $\pi: Y\to X$. If $X/\mathbb{C}$ is projective, this means $H^{2}(Y,\mathscr{O}_Y)=0$ by Mumford's result. Thus the Leray spectral sequence for $\pi$ yields an exact sequence 
$$
\xymatrix@R=.1cm{H^{1}(Y,\mathscr{O}_Y) \ar[r] & \Gamma \left(X,\ar@{=}[dd] R^{1}\pi_{\ast}\mathscr{O}_Y\right)\ar[r]&H^{2}(X,\ar@{=}[dd]\mathscr{O}_X)\ar[r]&0\\
 & & \rotatebox{20}{$\Bigg\backslash$} &\\
&\displaystyle\mathop{\lim}_{\overleftarrow{n}}H^{1}(E, \mathscr{O}_{n E}) &0}
$$
where\pageoriginale $n E$ is the subscheme of $Y$ with ideal sheaf $\mathscr{O}_Y(-nE)$. (the equality follows from the formal function theorem \cite{Hartshorne}). Since the maps of the inverse system $\left\{H^{1}(E, \mathscr{O}_{n E})\right\}_{n \geq 1}$ are all surjective, the hypothesis $H^{2}(X, \mathscr{O}_X)\neq 0$ is equivalent to the statement that $H^{1}(Y, \mathscr{O}_Y)\to H^{1}(E, \mathscr{O}_{n E})$ is not surjective for some $n\geq 1$. The hypothesis in 2) of the Theorem is that $H^{1}(Y,\mathscr{O}_Y)\to H^{1}(E, \mathscr{O}_E)$ is not surjective, if $\pic Y$ is reduced. 

\section{Examples}

(1) $X=\spec \dfrac{\mathbb{C}[x,y,z]}{\left(z^{2}+x^{3}+y^{7}\right)}$. Bloch and Murthy (unpublished) showed that there is a surjection $F^{2}K_0(X)\to \Omega^{1}_{\mathbb{C}/\mathbb{Z}}$, the module of absolute Kahler differentials of $\mathbb{C}$; since $\Omega^{1}_{\mathbb{C}/\mathbb{Z}}$ has uncountable rank, this gives a negative answer to a question of Murthy on Grothendieck groups of normal graded rings \cite{Bass}. If $A=\dfrac{\mathbb{C}[x,y,z]}{\left(z^{2}+x^{3}+y^{7}\right)}$, then $A\subset B=\mathbb{C}[u, v]$ is birational, where $u=\dfrac{x}{y^{2}}$, $v=\dfrac{z}{y^{3}}$. The maximal ideal at the origin in $A$ generates the ideal $\left(v^{2}+u^{3}\right)B$ in $B$, so that there is a resolution of singularities $Y\to X$ such that the exceptional divisor in $Y$ contains a rational curve with a cusp. If $\overline{X}\supset X$ is a projective surface where $\overline{X}-X$ consists only of non-singular points, and if $\overline{Y}\to \overline{X}$ is the corresponding resoluton, then $H^{1}(\overline{Y}, \mathscr{O}_{\overline{Y}})$ vanishes as $\overline{Y}$ is a smooth rational surface, while $H^{1}(E,\mathscr{O}_E)\neq \mathscr{O}$.\pageoriginale Hence from the theorem, $F^{2}K_0(\overline{X})$ is infinite dimensional, and $F^{2}K_0(X)$ has uncountable rank. 

(ii) \textbf{Cones.} Let $C\subset \mathbb{P}^{n}$ be a projectively normal curve, so that the cone $X\subset \mathbb{P}^{n+1}$ over $C$ is a normal projective surface whose only singularity is the vertex $P\epsilon X$. Let $\pi:Y\to X$ be the blow up of $X$ at $P$. Then $Y\simeq \mathbb{P}(\mathscr{O}_C\oplus \mathscr{O}_C(1))$ is a $\mathbb{P}^{1}$-bundle over $C$. If $f:Y\to C$, the exceptional divisor $E$ of $\pi$ is a reduced curve giving the (unique) section of $f$ with normal bundle $\mathscr{O}_c(-1)$. If $I=\mathscr{O}_Y(-E)$ is the ideal sheaf of $E$, then $\dfrac{I^{j}}{I^{j+1}}\cong\mathscr{O}_C(j)$. Thus 
\begin{enumerate}
\renewcommand{\theenumi}{\roman{enumi}}
\renewcommand{\labelenumi}{(\theenumi)}
\item $H^{1}(Y, \mathscr{O}_Y)\to H^{1}(E, \mathscr{O}_E)$ is an isomorphism 

\item for some $n>1$, $H^{1}(Y, \mathscr{O}_Y)\to H^{1}(E, \mathscr{O}_{n E})$ is not surjective $\Leftrightarrow$ for some $n>1$, $H^{1}\left(\dfrac{E, I}{I^{n}}\right)\neq 0\Leftrightarrow H^{1}\left(\dfrac{E, I}{I^{2}}\right)\neq 0 \Leftrightarrow H^{1}(C, \mathscr{O}_C(1))\neq 0$. Thus $H^{2}(X, \mathscr{O}_X)\neq 0 \Leftrightarrow H^{1}(C, \mathscr{O}_C(1))\neq 0 \Leftrightarrow H^{0}(C, \omega_C(-1))\neq 0 \Rightarrow \deg C\leq 2g -2$. Hence $A_0(X)$ is infinite dimensional if $C$ is a complete intersection of genus $\geq 3$ over $\mathbb{C}$ (eg. $C\subset \mathbb{P}^{2}$ of degree $\geq 4$), or $C$ is canonically embedded (see \cite{Srinivasa})

We remark that if $\deg C\geq 2g+1$, where $C\subset \mathbb{P}^{n}$ is embedded via a complete linear system, then $C$ is projectively normal and $A_0(X)$ is finite dimensional. If $C\subset \mathbb{P}^{n}_k$ is projectively normal, and $k=\overline{k}$ has characteristic $p>0$, then $A_0(X)$ is always finite dimensional \cite{Srinivasa}; in particular (1) in the Theorem is false if $\mathbb{C}$ is replaced by a universal domain in characteristic $p>0$. 

\item  Let $X\subset \mathbb{A}^{4}_{\mathbb{C}}$ be the surface $\begin{cases}
x^{3}+y^{3}+z^{3}=0\\
w^{2}+x^{2}+y^{2}+1=0
\end{cases}
$
\end{enumerate}

Then\pageoriginale $X$ is the double cover of the cubic cone $Y\subset \mathbb{A}^{3}$ given by $x^{3}+y^{3}+z^{3}=0$ branched along a smooth quadric section. Clearly $X\to Y$ is \'{e}tale over the vertex of $Y$, so that $X$ has $2$ singular points $P$, $Q$ each analytically isomorphic to the vertex of $Y$. Let $X_P\to X, X_Q\to X$ be the blow ups of $P$, $Q$ respectively. 

\begin{cl}
$CH^{2}(X_P)=CH^{2}(X_Q)=0$, while $CH^{2}(X)$ has infinite rank. 
\end{cl}

First we show that $CH^{2}(X)$ is infinite dimensional. Let $\overline{X}$ be a projective surface containing $X$ as an open set so that $\overline{X}-X$ consists only of smooth points. Let $Z\to \overline{X}$ be the blow up at $P$, $Q$. Then $Z$ is smooth and birationally ruled over $C\subset \mathbb{P}^{2}$ given by $x^{3}+y^{3}+z^{3}=0$ such that the exceptional curves $E_P$, $E_Q$ over $P$ and $Q$ respectively, both map isomorphically to $C$. Thus $\dim H^{1}(Z,\mathscr{O}_Z)=1$, while $\dim H^{1}(E, \mathscr{O}_E)=\dim H^{1}(E_P,\mathscr{O}_{E_P})+\dim H^{1}(E_Q, \mathscr{O}_{E_Q})=2$. Hence $CH^{2}(\overline{X})$ is infinite dimensional so that $CH^{2}(\overline{X})$ has uncountable rank. 

To show that $CH^{2}(X_P)=CH^{2}(X_Q)=0$, since the involution of $X/Y$ interchanges $P$, $Q$ it suffices to prove $CH^{2}(X_P)=0$. Now $\overline{X}\to \overline{Y}$ is branched along a smooth genus $4$ curve $(\overline{Y}\subset \mathbb{P}^{3}$ the projective cone over $C)$ which is tangent to $6$ rulings of $\overline{Y}$ (by Riemann-Hurwitz, for example). Hence $Z\to C$ is a non-minimal ruled surface with $6$ reducible fibers, each with $2$ components which are exceptional curve of the first kind, and $E_P$, $E_Q$ are sections. Blowing down the $6$ exceptional curves meeting $E_P$,we obtain a minimal ruled surface over $C$ with $2$ disjoint sections, namely the images of $E_Q$ and $E_P$ Since the normal bundle of $E_Q$ does not change under this map, the minimal ruled surface is $\mathbb{P}_C(\mathscr{O}_C\oplus \mathscr{O}_C(1))$ (the normal bundle of $E_Q$ is $\mathscr{O}_C(-1))$ i.e. $\overline{X}_P$ is isomorphic to the blow up of $\overline{Y}$ at $6$\pageoriginale points. Hence $A_0(\overline{X}_P)$ is finite dimensional, since $A_0(\overline{Y})$ is (see \cite{Srinivasa}). Thus $A_0(X_P)=0$ by an easy argument. 

\section{Relative K-theory}

Let $T$ be a scheme, $S\subset T$ a subscheme. One can define relative $K$-groups $K_i(T,S)$ for $i>0$ which fit into a natural long exact sequence (see \cite[appendix]{Coombes})
$$
\to K_i(T)\to K_i(S)\to K_{i-1}(T,S)\to K_{i-1}(T)\to\ldots  
$$
Let $X$ be a normal quasi-projective surface, $S\subset X$ a closed sub-scheme supported on $X_{\sing}$; let $Y\to X$ be a resolution of singularities, and $E\subset Y$ a subscheme such that we have a diagram 
$$
\xymatrix{E\ar[r]\ar[d]& Y\ar[d]\\
S\ar[r]&X}
$$
Then by naturality we have a diagram of long exact sequences 
$$
\xymatrix{K_1(Y)\ar[r]& K_1(E)\ar[r]&K_0(Y,E)\ar[r]& K_0(Y)\ar[r]& K_0(E)\\
K_1(X)\ar[u]\ar[r]& K_1(S)\ar[u]\ar[r]&K_0(X,S)\ar[u]\ar[r]& K_0(X)\ar[u]\ar[r]& K_{0}(S)\ar[u]}
$$
Further, it is shown (loc. cit.) that points of $X-S$ admit relative cycle classes in $K_0(X,S)$ which map to the usual classes in $K_0(X)$, and a similar claim holds for the pair $(Y,E)$. Let 

$F^{2}K_0(X,S)$=subgroups of $K_0(X,S)$ generated by points of\pageoriginale $Y-S F^{2}\\K_0(Y,E)$= subgroup of $K_0(Y,E)$ generated by points of $Y-E$. 

We have a diagram with surjective arrows (this is easy if $Y-E\simeq X -S$, but true even otherwise)
$$
\xymatrix{F^{2}K_0(Y, E)\ar[r]& F^{2}K_0(Y)\\
F^{2}K_0\ar[u](X,S)\ar[r]& F^{2}K_0(X)\ar[u]}
$$

\begin{cl}
$F^{2}K_0(X,S)\simeq F^{2}K_0(X)$. 
\end{cl}

\begin{ip}
Must show $(F)_C=0$ in $F^{2}K_0(X,S)$ for any $C\subset X$ with $C\cap X_{\sing}=0$, and $f\epsilon k(C)^{\ast}$. One checks that from the definition of the relative $K$-groups in loc. cit., if $\overline{C}$ is the normalisation of $C$, then the natural maps $K_i(\overline{C})\to K_i(X)$ factor through $K_(X, S)$; now the vanishing of $(f)_C$ in $K_0(X,S)$ follows from the vanishing of $(f)_{\overline{C}}$ in $K_0(\overline{C})$. 
\end{ip}

We can now state a conjecture due to Bloch and the author. Let $E\subset Y$ be the reduced exceptional divisor. 

\begin{con}
\begin{enumerate}
\renewcommand{\theenumi}{\roman{enumi}}
\renewcommand{\labelenumi}{(\theenumi)}
\item $F^{2}K_0(Y, nE)\to\to F^{2}K_0(Y, (n-1)E)$ is an isomorphism for all sufficiently large $n$. 
\item $F^{2}K_0(X)\simeq \displaystyle\mathop{\lim}_{\overleftarrow{n}} F^{2}K_0(Y, nE)$. 
\end{enumerate}
\end{con}

In particular  $\displaystyle\mathop{\lim}_{\overleftarrow{n}} F^{2}K_0(Y, nE)$ should be independent of the resolution $Y\to X$ We can verify this in general assuming a form of excision for $K_1$ for certain $1$-dimensional schemes, and in any case can verify this for resolutions where $E$ has smooth components and normal crossings. 

As\pageoriginale a special case of the conjecture, one expects that if $X$ has only singularities, then $CH^{2}(X)\simeq CH^{2}(Y)$. 

A closely related problem to that of computing $F^{2}K_0(X)$ for singular $X$ is the problem of computing $K_0(\mathscr{C}_R)$, where $R$ is the local ring of a singular point on a surface, $\mathscr{C}_R$ the category of $R$-modules of finite length and finite projective dimension. This problem is of independent interest to algebraists, but not much computation could be done by algebraic methods. Recently, $K$-theory has provided some results in this area; in \cite{Srinivasb} a method was introduced which enabled one to prove that 
$$
K_0(\mathscr{C}_R)=\mathbb{Z} \oplus N-(\text{ torsion })
$$
for some $N$, where $R$ is a rational double point in characteristic $p>0$ which occurs on a rational surface, or else $R$ is of type $E_8$ over an arbitrary algebraically closed field. Some further examples in characteristic $p>0$ were considered in \cite{Coombes}, and finally we have the results of \cite{Levinec} and the author \cite{Srinivasc} (independently). 

\begin{thm1}
\textit{Let $R$ be the local ring of a rational double point on a surface $X/\mathbb{C}$. Then $K_0(\mathscr{C}_R)=\mathbb{Z}$}. 
\end{thm1}

\begin{cor}
\textit{Let $X/k=\overline{k}$ be a normal quasi-projective surface with only quotient singularities. If $Y\to X$ is a resolution, then $CH^{2}(X)\simeq CH^{2}\\(Y)$}.
\end{cor}

We should point out that inspite of the explicit nature of the formula
$$
F^{2}K_0(X)=\displaystyle\mathop{\lim}_{\overleftarrow{n}} F^{2}K_0(Y, nE), 
$$
we\pageoriginale do not know a single example where we can prove that this formula is true, when the right side is different from $F^{2}K_0(Y)$. Thus we have only indirect evidence for the truth of the conjecture.\footnote{Some examples exist now.}

Levine has made  a conjecture related to ours. If $X$ is a normal (quasi-projective) surface, and $P\epsilon X$ the unique singular point, $Y\to X$ a resolution, $Z=Y\times_{X}\spec R$, $R=\mathscr{O}_{P,X'}$ and if $E\subset Y$ is the reduced exceptional divisor, then Levine constructs (loc. cit.) an exact sequence 
$$
H^{1}(Z, \mathscr{K}_{2, z})/N \xrightarrow{\Phi} K_0(\mathscr{C}_R)\xrightarrow{\Psi} F^{1}G_0(E)\to 0. 
$$
Here $G_0(E)$ is the Grothendieck group of coherent sheaves on $E, F^{1}\\ G_0(E)$ the subgroup generated by points; $H^{1}(Z, \mathscr{K}_{2, Z})$ is $H^{1}$ of the complex. 
$$
0\to K_2(k(Y))\xrightarrow{T} \displaystyle\mathop{\oplus}_{x \epsilon Z^{1}} k(x)^{\ast} \xrightarrow{\partial} \displaystyle\mathop{\oplus}_{x \epsilon Z^{2}}\to 0
$$
where $Z^{i}$ is the set of codimension $i$ points; $N$ is defined to be the subgroup of $H^{1}(Z, \mathscr{K}_{2, Z})$ generated by 
$$
\ker \left(\displaystyle\mathop{\oplus}_{\text{ exceptional\\curves }}k(x)^{\ast}\xrightarrow{\partial}\displaystyle\mathop{\oplus}_{z \epsilon Z^{2}} \mathbb{Z}\right).
$$

\footnotetext[2]{The author now has a proof of Levine's conjecture [23].}
\begin{conj}[{\bf(Levine)\boldmath$^{2}$}] \textit{$\phi:H^{1}(Z, \mathscr{K}_{2, Z})/N\to K_0(\mathscr{C}_R)$ is injective.}\pageoriginale

Levine shows that under the natural map $K_0(\mathscr{C}_R)\to K_0(X)$ we have image $SK_0(\mathscr{C}_R)=\ker (\Psi: K_0(\mathscr{C}_R)\to F^{1}G_0(E))$. We conjecture that 
$$
H^{1}(Z, \mathscr{K}_{2, Z})/N\to \displaystyle\mathop{\lim}_{\overleftarrow{n}} H^{1}(E, \mathscr{K}_{2, n E})/N
$$ 
is injective; this is closely related to our conjectural formula for $F^{2}K_0(X)$. 
\end{conj}

\section{Sketch of the proof}

In this section we sketch the proof of the infinite dimensionality theorem. We only consider the case $k=\mathbb{C}$; the other case uses similar methods. Let $\pi:Y\to X$ be a resolution of singularities. We may assume that $A_0(Y)$ is finite dimensional, since if $A_0(Y)$ is infinte dimensional then trivially $A_0(X)$ is infinite dimensional. Thus we may assume that for some $n\geq 1$, $H^{1}(Y, \mathscr{O}_Y)\to H^{1}(E, \mathscr{O}_{n E})$ is not surjective; fix the smallest such value of $n$. The idea is to show that $F^{2} K_0(Y, n E)$ is strictly larger than $F^{2}K_0(Y)$, so that $F^{2}K_0(X)\to F^{2}K_0(Y)\simeq Alb Y \times \mathbb{Z}$ is not an isomorphism. In particular, we want $K_0(Y, n E)\to K_0(Y)$ to have a kernel i.e. coker $(K_1(Y)\to K_1(n E))$ should be non-zero. One has a natural decomposition $K_1(T)=\Gamma(T, \mathscr{O}_T^{\ast})\oplus SK_1(T)$\pageoriginale for any ``reasonable'' scheme $T$, for a certain subgroup $SK_1(T)\to K_1(T)$; in fact $SK_1(T)$ is the intersection of the kernels of all the maps $K_1(T)\to K_1(\mathscr{O}_{t, T})$ for all points $t \epsilon T$. It turns out that $(SK_1 (nE)/\text{ image } SK_1(Y))\subset K_0(Y, n E)$ contains $\ker \left(F^{2}K_0(Y, n E)\to F^{2}K_0(Y)\right)$. Thus we will be mainly interested in the map $SK_1(Y)\to SK_1(n E)$. 

Suppose $E$ is a smooth curve. We have the localisation sequence of \cite{Quillen} for any open set $U \subset E$. 
$$
\begin{aligned}
&K_2(U)\to \displaystyle\mathop{\oplus}_{P \epsilon E - U}K_1(k(P))\to K_1(E)\to K_1(U)\to \displaystyle\mathop{\oplus}_{P \epsilon E-U}\mathbb{Z}\to K_0(E)\\
&{}\to K_0(U)\to 0
\end{aligned}
$$
Taking the direct limit over all open subsets $U\subset E$, we obtain 
$$
\begin{aligned}
&K_2(k(E))\to \displaystyle\mathop{\oplus}_{P \epsilon E} K_1(k(P))\to K_1(E)\to K_1(k(E))\to \displaystyle\mathop{\oplus}_{P \epsilon E} \mathbb{Z}\\
&{}\to K_0(E)\to \mathbb{Z}\to 0
\end{aligned}
$$
Here $k(E)$ is the function field, $K_1(E)=k(E)^{\ast}$, and $\partial:k(E)^{\ast}\to \oplus \mathbb{Z}$ is the divisor map. Thus $\ker\ \partial =\Gamma(E, \mathscr{O}^{\ast}_E)$ i.e. we have a presentation $(\text{since~ } K_1(k(P))=k(P)^{\ast})$$$
K_2(E)\to \displaystyle\mathop{\oplus}_{P \epsilon E} k(P)^{\ast}\to SK_1(E)\to 0. 
$$
From the resolution (for a smooth curve $E$ - see \cite{Quillen}).
$$
0\to \mathscr{K}_{2, E}\to i_{\ast} K_2(k(E))\to \displaystyle\mathop{\oplus}_{P \epsilon E}(i_P)_{\ast}k(P)^{\ast}\to 0
$$
we\pageoriginale thus have an isomorphism $SK_1(E)\simeq H^{1}(E, \mathscr{K}_{2, E})$. 

It turns out that even when $E$ has many components, the scheme $(n E)$ still satisfies $SK_1(nE)\simeq H^{1}(E, \mathscr{K}_{2. nE})$ by an analogous argument. Further, there exists a surjection $SK_1(Y)\to H^{1}(Y, \mathscr{K}_{2, Y})$ leading to a diagram 
$$
\xymatrix{SK_1(Y)\ar[d]\ar[r]& H^{1}(Y, \mathscr{K}_{2 Y})\ar[d]\\
SK_1(n E)\ar[r]^-{\simeq}& H^{1}(E, K_{2, n E})}
$$
So we are reduced to considering $\varphi:H^{1}(Y, \mathscr{K}_{2, Y})\to H^{1}(E, \mathscr{K}_{2,nE })$ We state a lemma which is a special case of a result of \cite{Blochb}, (which has already been implicitly used to say $SK_1(n E)\simeq H^{1}(E, \mathscr{K}_{2, nE})$; the case $n=2$ is due to \cite{Vander}

\begin{lem}
\textit{Let $R$ be a local $\mathbb{Q}$-algebra, $S_n=\dfrac{R[t]}{(t^{n})}$, $n>1$. Then there is an exact sequence}
$$
0\to \Omega^{1}_{R/\mathbb{Z}}\to K_2(S_n)\to K_2(S_{n-1})\to 0
$$
\end{lem}

This lemma allows us to compute the ``derivative'' of $\varphi$. Let $Y_{k[\epsilon]}$, $(n E)_{k[\epsilon]}$ denote the products (over $\spec k$) of $Y$, $(n E)$ respectively with $\spec (k[\epsilon])$, where $k[\epsilon]$ is the ring of dual numbers (i.e. $\epsilon^{2}=0$, $\epsilon \neq 0$). 

Then 
$$
H^{1}(Y_{k[\epsilon]},\mathscr{K}_2)=H^{1}(Y, \mathscr{K}_2)\oplus H^{1}\left(Y, \Omega^{1}_{Y/\mathbb{Z}}\right)
$$
$$
H^{1}((n E)_{k[\epsilon]}, \mathscr{K}_2)=H^{1}((n E), \mathscr{K}_2)\oplus H^{1}\left( E, \Omega^{1}_{(n E)/\mathbb{Z}}\right)
$$
\pageoriginale
Thus the derivative of $\varphi$ is 
$$
d\varphi: H^{1}\left(Y, \Omega^{1}_{Y/\mathbb{Z}}\right)\to H^{1}\left(E, \Omega^{1}_{(n E)/\mathbb{Z}}\right).
$$
For any ``reasonable'' scheme $T/k$, we have an exact sequence 
$$
0\to \Omega^{1}_{k/\mathbb{Z}}\otimes_k \mathscr{O}_T\to \Omega^{1}_{T/\mathbb{Z}}\to \Omega^{1}_{T/k}\to 0
$$
Thus we have a diagram
$$
\xymatrix{H^{O}(Y,\ar[d] \Omega^{1}_{Y/K})\ar[r]& H^{1}(Y,\ar[d] \mathscr{O}_Y)\otimes_{k}\Omega^{1}_{k/\mathbb{Z}}\ar[r]&H^{1}(Y,\ar[d] \Omega^{1}_{Y/\mathbb{Z}})\ar[r]&\\
H^{0}(E, \Omega^{1}_{(nE/k)})\ar[r]& H^{1}(E, \mathscr{O}_{n E})\otimes_k\Omega^{1}_{k/\mathbb{Z}}\ar[r]& H^{1}(E, \Omega^{1}_{(nE)/\mathbb{Z}})\ar[r]&}
$$
\xymatrix{H^{1}(Y,\ar[d] \Omega^{1}_{Y/k})\\
H^{1}(E, \Omega^{1}_{(nE)/k})\ar[r]& 0}

The cohomology groups of $\Omega^{1}_{y/k}$, $\Omega^{1}_{(n E)/k}$  are finite dimensional over $k$, while if $k$ is uncountable, $\Omega^{1}_{k/\mathbb{Z}}$ has uncountable dimension over $k$. Since 
$$
H^{1}(Y, \mathscr{O}_Y)\to H^{1}(E, \mathscr{O}_{n E})
$$
is not surjective, we see that $H^{1}\left(Y, \Omega^{1}_{Y/\mathbb{Z}}\right)\to H^{1}\left(E, \Omega^{1}_{(n E)/\mathbb{Z}}\right)$ has an uncountably generated cokernel (in fact the cokernel has uncountable dimension over $k$). 

To\pageoriginale get global results from the above computations, we let $k$ be an uncountable proper subfield of $\mathbb{C}$ which is algebraically closed, such that $X, Y, E$ are all defined over $k$. Then there is an embedding of $k$-algebras $A\subset \mathbb{C}$ where $A=k[t]^{n}_{(t)}$, the super-script $h$ denoting henselisation. Let $Y_A=Y\times_{\spec k}\spec A$, etc.; for $m\geq 1$ let $A_m=A/t^{m} A$ and $Y_{A_{m}}=Y\times _{\spec\ k}\spec A_m$, etc. We prove the following claims: 
\begin{enumerate}
\renewcommand{\theenumi}{\roman{enumi}}
\renewcommand{\labelenumi}{(\theenumi)}
\item $H^{1}((nE)_A, \mathscr{K}_2)\to H^{1}((nE)_{A_m}, \mathscr{K}_2)$ is surjective for any $m\geq 1$ 
\item $H^{1}((n E)_A, \mathscr{K}_2)\to H^{1}((n E)_K, \mathscr{K}_2)$ is injective, where $K=A[1/t]$ is the quotient field. 
\item coker $(H^{1}(Y_A, \mathscr{K}_2)\to H^{1}((n E)_A, \mathscr{K}_2))\otimes \mathbb{Q}$ has uncontable rank $\Rightarrow$ coker $(H^{1}(Y_K, \mathscr{K}_2)\to H^{1}((nE)_K, \mathscr{K}_2))\otimes \mathbb{Q}$ has uncountable rank (here we use the fact that $A_0(Y)$ is finite dimensional, so that $P_g=0$; if $q>0$, we use the fact that over $\mathbb{C}$ all surfaces with $p_g=0$, $q>0$ are classified). 
\item pass from $K$ to $\mathbb{C}$ by a transfer argument (i.e show that 
$$
K_0(Y_K, (nE)_K)\otimes \mathbb{Q}\to K_0(Y_{\mathbb{C}}, (n E)_{\mathbb{C}})\otimes \mathbb{Q}
$$ 
is injective). 
\end{enumerate}

These steps will show that $\ker (K_0(Y, nE)\to K_0(Y))\otimes \mathbb{Q}$ has uncountable rank; we want to get a similar result for $F^{2}K_0$. For this, we need some information about the boundary map. 
$$
\partial:SK_1(nE)\to K_0(Y, nE)
$$
in\pageoriginale the long exact sequence. If $n=1$, and $E$ is smooth, this map has relatively simple description. In this case 
$$
SK_1(E)=\dfrac{\displaystyle\mathop{\oplus}_{p \epsilon E}k(P)^{\ast}}{\text{ image } K_2(k(E))}.
$$ 
If $\sum\limits_{i=1}^{m}(\alpha_i)_{P_i}$ is a class in $SK_1(E)$, let $C\subset Y$ be a smooth curve meeting $E$ transversally at $P_1, \ldots\ldots, P_m$ and let $P_{m+1},\ldots\ldots,P_r$ be the other points of intersection which may also be taken to be transverse. Let $\alpha_{m+1},\ldots,\alpha_r$ all equal $1$; we can now find a rational function $f$ on $C$, regular at all the $P_i$ with $f(P_i)=\alpha_i$. Then 
$$
\partial\left(\sum(\alpha_i)_{P_i}\right)=(f)_C\epsilon F^{2}K_0(Y,E).
$$
Thus $\partial(SK_1(E))=\ker (F^{2}K_0(Y, E)\to F^{2}K_0(Y))$. 

In general, we obtain elements of $SK_1(nE)$ mapping to relative $0$-cycles in a similar way - if $C\subset Y$ is a smooth curve which is not a component of $E$, and $S=C\cap (nE)$ as schemes, then we have a diagram of relative $K$-theory sequences (see \cite{Coombes}, appendix). 
$$
\xymatrix{K_1(C)\ar[d]\ar[r]&K_1(S)\ar[d]\ar[r]^-{\partial}& K_0\ar[d](C,S)\ar[r]& K_0\ar[d](C)\\
K_1(Y)\ar[r]& K_1(nE)\ar[r]^-{\partial}& K_0(Y, nE)\ar[r]& K_0(Y)}
$$
where $K_1(S)=\Gamma(S, \mathscr{O}^{\ast}_S)$ since $\dim S=0$. The map $\partial:K_1(S)\to K_0(C,S)$ is as follows; given $g\epsilon \Gamma(S, \mathscr{O}^{\ast}_S)$, let $f$ be a rational function on $C$, which is a unit near $S$, such that $f\mid_{S}=g$. Then $\partial(g)=(f)_C\epsilon K_0(C,S)$. Thus image $(\Gamma(S, \mathscr{O}^{\ast}_S)\to SK_1(nE))$ gives\pageoriginale elements of $SK_1(nE)$ whose boundaries are algebraic cycles. 

If we assume $n>1$, but $E$ is smooth, then by the lemma of Bloch mentioned earlier we get a sequence ($I$ is the ideal sheaf of $E$ on $Y$)$H^{1}\\\left(E, \Omega^{1}_{E/\mathbb{Z}}\otimes \dfrac{I^{n-1}}{I^{n}}\right)\to H^{1}(E, \mathscr{K}_{2, nE})\to H^{1}(E, \mathscr{K}_{2, (n-1)E})\to 0$. If we choose a curve $C$ as above meeting $E$ transversally, and take $g\epsilon \Gamma\left(S, \mathscr{O}^{\ast}_S\right)$ such that $g\mid_{C \cap (n-1)E}=1$, then it turns out that the class in $SK_1(nE)$ obtained from $g\epsilon K_1(S)$ in fact lifts to $H^{1}\left(E, \Omega^{1}_{E/\mathbb{Z}}\otimes I^{n-1}/I^{n}\right)$. We have a flasque resolution of $\Omega^{1}_{E/\mathbb{Z}}\otimes I^{n-1}/I^{n}$ as follows: 

$0\to \Omega^{1}_{E/\mathbb{Z}}\otimes I^{n-1}/I^{n}\to i_{\ast}\Omega^{1}_{k(E)/\mathbb{Z}}\otimes I^{n-1}/I^{n}\to \oplus (i_P)_{\ast}A_P\to 0 $ where $A_P=\left(\Omega^{1}_{k(E)/\mathbb{Z}}/\Omega^{1}_{\mathscr{O},P,E/\mathbb{Z}}\right)\otimes I^{n-1}/I^{n}$.

Thus
$$
\dfrac{H^{1}\left(E, \Omega^{1}_{E/\mathbb{Z}}\otimes I^{n-1}/I^{n}\right)=(\oplus A_P)}{\text{ image }\left(\Omega^{1}_{k(E)/\mathbb{Z}}\otimes I^{n-1}/I^{n}\right).}
$$

If $h$ is a rational function on $Y$, which is a unit in the semi-local ring $\mathscr{O}_{S, Y}$, such that $h\mid_S=g$, and if $x\epsilon \mathscr{O}_{S, Y}$ such that $x\mathscr{O}_{S, Y}$ is the ideal of $C$ in $\mathscr{O}_{S, Y}$, then 
$$
\dfrac{dx}{x}\otimes (1-h)\epsilon \displaystyle\mathop{\oplus}_{P \epsilon S} A_P
$$
gives\pageoriginale the lift to $H^{1}\left(E, \Omega^{1}_{E/\mathbb{Z}} \otimes I^{n-1}/I^{n}\right)$, where we regard $1-h$ as a local section of $I^{n-1}/I^{n}$. Thus to check that $\partial(SK_1(nE))\subset F^{2}K_0(Y, nE)$, we have to see if groups like $H^{1}(E, \Omega^{1}_{E/ \mathbb{Z}}\otimes I^{n-1}/I^{n})$ are generated by such ``logarithmic'' principal parts. We could show this using residues for the quotient group $H^{1}(E, \Omega^{1}_{E/L}\otimes I^{n-1}/I^{n})$ where $L$ is any field of definition of $X$. Thus if $X, Y, E$ are defined over the field $\overline{\mathbb{Q}}$ of algebraic numbers, we would be done. In general, the argument is complicated by the fact that $E$ has many components, and $L$ may not equal $\overline{\mathbb{Q}}$. 

\section{\texorpdfstring{$NK_0$}{eq} and \texorpdfstring{$K_{-1}$}{eq}}

For any scheme $X/k$ we define 

$NK_0(X)=\text{ coker } (K_0(X)\to K_0(X\times \mathbb{A}^{1})) K_{-1}(X)=\text{ coker }\\ \left(K_0(X_{k[t]})\oplus K_0(X_{k[t^{-1}]})\to K_0(X_{k[t, t^{-1}]})\right)$

If $X$ is regular, $NK_0(X)=K_{-1}(X)=0$. Murthy had asked for examples of normal graded rings $A=\oplus A_n$ with $NK_0(A)\neq 0$. $(NK_0(A)=NK_0(\spec A))$. \cite{Weibel} showed that $NK_0(A)\neq 0$ for the local ring described in example 6 in the appendix to Nagata's book ``Local rings''; this is a $2$ dimensional normal domain which is analytically ramified. Swan (see \cite{Weibel}) gave some geometric examples in dimension $3$. Weibel and Swan also showed that if $A=\displaystyle\mathop{\oplus}_{n \geq 0}A_n$ satisfies 

$NK_0(A)=0$,\pageoriginale then $K_0(A)=K_0(A_0)$. Thus Bloch-murthy's example $\dfrac{\mathbb{C}[x,y,z]}{\left(z^{2}+x^{3}+y^{7}\right)}$ has $NK_0\neq 0$. 

Now suppose $X$ is a normal quasi-projective surface over an algebraically closed field $k$. Let $Y\to X$ be a resolution of singularities such that the reduced exceptional divisor $E$ has smooth components and normal crossings. 

\begin{con}
The following are equivalent: 
\begin{enumerate}
\renewcommand{\theenumi}{\roman{enumi}}
\renewcommand{\labelenumi}{(\theenumi)}
\item $NK_0(X)\neq 0$
\item\footnote{The conjecture for $K_{-1}$ has to be modified; see \cite{Srinivasd}} $K_{-1}(X)\neq 0$ 
\item $\pic nE\to \pic E$ is not an isomorphism for some $n>1$ 
\item $H^{1}(E, \mathscr{O}_{n E})\to H^{1}(E, \mathscr{O}_{E})$ is not an isomorphism for some $n>1$ 
\end{enumerate}

(iii) and (iv) are clearly equivalent. If $k=\mathbb{C}$ (or more generally, if $L$ is a field of definition of characteristic $0$, and $k$ has transcendence degree $\geq 1$ over $L$) then (iv)$\Rightarrow$ (i) and (iv) $\Rightarrow$ (ii); the details of the proof will appear elsewhere. Finally, if $X$ is the affine cone over $C\subset P^{n}$, with $H^{1}(C, \mathscr{O}_C(1))\neq 0$, then $NK_0(X)\neq 0$ in any characteristic $\neq 2$ (see \cite{Coombes}), giving examples of affine surfaces $X$ in characterisitic $P>0$ such that all vector bundles on $X$ are trivial, while $K_0(X\times \mathbb{A}^{1})\neq\mathbb{Z}$. Lastly, if $C\subset \mathbb{P}^{n}_k$ is projectively normal with $\deg C\geq 2g+1$, $g$=genus $C$, and $k$ has characteristic $0$ or $p\geq \dim H^{0}(C, \mathscr{O}_C(1))$, then $NK_0(X)=0$. 
\end{con}


\begin{thebibliography}{ABCDEFGHI}
\bibitem[Bass(1973)]{Bass}\pageoriginale
{H. Bass}. Some problems in``classical'' algebraic $K$-theory \textit{Lecture Notes in Math.} No. 342, Springer-Verlag, New-York (1973).

\bibitem[Bloch(1976)]{Blocha}
{S. Bloch}. Lectures on algebraic cycles, \textit{Duke Univ. Math. Ser.} IV, Duke Univ. Press, Durham (1976).

\bibitem[Bloch(1975)]{Blochb}
{S. Bloch}. $K_2$ of artinian $\mathbb{Q}$-algebras with applications to algebraic cycles, \textit{Comm. Alg}., 3(1975) 405--426.

\bibitem[Collino(1981)]{Collino}
{A. Collino}. Quillen's $\mathscr{K}$-theory and algebraic cycles on almost non-singular varieties, \textit{Ill. J. Math}., 25(1981) 654--666.

\bibitem[Coombes and Srinivas(1982)]{Coombes}
{K. R. Coombes} and V. Srinivas. Relative $K$-theory and vector bundles on certain singular varieties, \textit{Invert. Math}., 70(1982) 1--12.

\bibitem[Hartshorne(1977)]{Hartshorne}
{R. Hartshorne}. Algebraic Geometry, \textit{Grad. Texts in Math.} No. 52, Springer-Verlag, New York(1977).

\bibitem[Van der Kallen(1971)]{Vander}
{W. Van der Kallen}. Le $K_2$ des nombres duaux, \textit{C. R. Acad. Sci.} Paris 273A(1971), 1204--1207.

\bibitem[Levine(1985)]{Levinea}
{M. Levine}. Bloch's formula for singular surfaces, \textit{Topology},  24(2) (1985) 165--174.

\bibitem[Levine(1985)]{Levineb}
{M. Levine}. Torsion Zero cycles on singular
varieties, \textit{Amer. J. Math}., 107(1985), 737--757. 

\bibitem[Levine(1999)]{Levinec}
{M. Levine}.\pageoriginale On the $K$-groups of surface singularities, Preprint. 

\bibitem[Milne(1982)]{Milne}
{J. Milne}. Zero cycles on algebraic varieties in non-zero characteristic: Rojtman's theorem, \textit{Comp. Math}., 47(1982).

\bibitem[Milnor(1971)]{Milnor}
{J. Milnor}. \textit{Introducation to Algebraic K-theory}, Ann. Math. Studies No. 72, Prinecton Univ. Press, Princeton (1971).

\bibitem[Mumford(1968)]{Mumford}
{D. Mumford}. Rational equivalence of $0$-cycles on algebraic surfaces, \textit{J.Math. Kyota Univ}., 9(1968) 195--240.

\bibitem[Murthy(1976)]{Murthy}
{M. P. Murthy and R. G. Swan}. Vector bundles on affine surfaces, \textit{Invert. Math}., 36(1976) 125--165.

\bibitem[Perdrini and Weibel (1983)]{Pedrini}
{C. Pedrini and C. Weibel}. K-theory and Chow groups of singular varieties, to appear in proceedings of the AMS $K$-theory Conference, Boulder (1983). 

\bibitem[Quillen(1973)]{Quillen}
{D. Quillen}. Higher Algebraic $K$-theory, \textit{Lecture Notes in Math}. No. 341, Springer-Verlag, New York (1973). 

\bibitem[Roitman(1972)]{Roitmana}
{A. A. Roitman}. Rational equivalence of $0$-cycles, \textit{Math USSR Sbornik}, 18 (1972) 571--588. 

\bibitem[Roitman(1980)]{Roitmanb}
{A. A. Roitman}. The torsion of the group of $0$-cycles modulo rational equivalence, \textit{Ann. Math}., 111(1980) 553--569.

\bibitem[Srinivas(1982)]{Srinivasa}
{V. Srinivas}.\pageoriginale Vector bundels on the cone over a curve, \textit{Compos. Math}., 47(1982) 249--269.

\bibitem[Srinivas(1985)]{Srinivasb}
{V. Srinivas}. Zero cycles on a singular surface, I, II, \textit{J. Reine Ang. Math}., 359(1985) 90--105 and 362 (1985) 4--27.

\bibitem[Srinivas(1999)]{Srinivasc}
{V. Srinivas}. Modules of finite length and Chow groups of surfaces with rational double points, to appear in \textit{Ill. J. Math}. 

\bibitem[Weibel(1981)]{Weibel}
{C. Weibel}. $K$-theory and analytic isomorphisms, \textit{Invent Math}., 61(1981) 177.197.

\bibitem[Srinivas(1999)]{Srinivasd}
{V. Srinivas}. Grothendieck groups of polynomial and Laurent polynomial rings, to appear in \textit{Duke Math. J. }

\bibitem[Srinivas(1999)]{Srinivase}
{V. Srinivas}. Indecomposable projective modules on affine domains, to appear in \textit{Compos. Math.}
\end{thebibliography}

\vskip 1cm

\noindent
School of Mathematics\\
Tata Institute of Fundamental Research\\
Homi Bhabha Road,
Bombay 400 005\\
INDIA.
