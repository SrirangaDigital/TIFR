\title{On Projective Modules Over Positively Graded Rings}\label{chap10}
\markright{On Projective Modules Over Positively Graded Rings}

\author{By H. Lindel}
\markboth{H. Lindel}{On Projective Modules Over Positively Graded Rings}

\date{}
\maketitle

\setcounter{page}{195}

\setcounter{pageoriginal}{250}
\section*{Introduction}\pageoriginale

Let $A$ be a noetherian ring of finite Krull dimension $\dim(A)=d$.  A
fundamental result of Serre says that every projective $A$-module $P$
of $r=\rank (P)>d$ has a unimodular element $p$, i.e., an element $p$
such that $P$ splits into a direct sum $P=Q\oplus Ap$. This reduces
the study of the structure of projective modules over finite
dimensional noetherian rings to the case
$\displaystyle\mathop{=}^{\vee} d$. The maximal $r\in |N$ such that
there exists a projective $A$-module without unimodular elements is
called the {\em Serre dimension} Serre-dim$(A)$ of $A$. In 1979,
Plumstead showed that, for a polynomial ring $R=A[T]$ in one
indeterminate $T$, that Seree-dim $(A[T])\leq \dim(A)$, thus settling
a question of Eisenbud and Evans [\cite{chap10-Pl}). In 1981,
Bhatwadekar and Roy generalized Plumstead's result by proving that
Serre-dim $(A[T_{1},\ldots,T_{n}])\leq \dim(A)$ (\cite[Theorem
3.1]{chap10-Bh/Ro}, and Theorem \ref{chap10-thm2.4}
below). Recently, this result was extended to Laurent polynomial rings
$R=A[T_{1},\ldots,T_{n},U^{\pm 1}_{1},\ldots,U^{\pm 1}_{m}]$
establishing a positive answer to a question of Bass and Murthy
(\cite[Theorem 4.1]{chap10-Bh/Li/Ra}],  and
\cite[\S\ 9]{chap10-Ba/Mu}). To look for further cases of ring
extensions $R$ of $A$ with Serre-dim $(R)\leq \dim(A)$\pageoriginale
it seems to be sensible to consider rings $R$ that are birationally
equivalent to a polynomial ring $A[T_{1},\ldots,T_{n}]$. If, in this
case $n=1$, we know by a result of Rao (\cite[Theorem 1.1]{chap10-Ra})
that indeed Serre-dim $(R)\leq \dim (A)$.   

In studying the structure of projective modules, it can be useful to
consider projective modules over positively graded rings that we write
in the usual form $R=\bigoplus\limits_{i\geq 0}R_{i}$ (See, for
example, the use of Criterion $I$ in \cite{chap10-Bh/Li/Ra}). The
purpose of this paper is to give an elementary approach to some basic
theorems on projective modules over such rings. we give some
applications, but the approach itself is the main point. 

In \S\ \ref{chap10-sec1}, we generalise Quillen's patching theorem
(\cite[Theorem 1]{chap10-Qu}) from polynomial rings to positively graded rings
(cf. Theorem \ref{chap10-thm1.3}):

{\em Let $M$ be a finitely presented module over a positively graded
ring $R=\bigoplus\limits_{i\geq 0}R_{i}$, $A=R_{0}$. Then the set
$J(A,M)$ of all $u\in A$, for which $M_{u}$ is extended from $A_{u}$,
is an ideal in $A$.}

In the case of a projective $M$ or a flat extension $R/A$, Murthy has\break
observed (\cite[Theorem 3.6]{chap10-Mu}) that Quillen's
technique applies. Our proof is based on the crucial
Matrix-Lemma \ref{chap10-lem1.1} and does not depend on\break Quillen's
techniques. In particular, we need no flatness assumption. The above
result was obtained independently by Artamonow (\cite{chap10-Ar}).

As a first application, we derive a lovely result of Vorst
(\cite[Theorem 3.2]{chap10-Vo}, and
Theorem \ref{chap10-thm1.5} below).

{\em Let\pageoriginale $A$ be a noetherian ring such that all
projective $A[T_{1},\ldots,T_{n}]$-modules are extended from $A$. Then
every projective module over discrete Hodge $A$-algebras is extended
from $A$.}

In \S\ \ref{chap10-sec2}, we apply our matrix lemma to obtain in
correspondence with Criterion I in \cite{chap10-Bh/Li/Ra} (cf.~(2.1)).  

{\em Let $P$ be a projective module over a positively graded ring
$R=\bigoplus\limits_{i\geq 0}R_{i}$. Assume there exists an element
$q\in P$, whose canonical images in the localizations $P_{1+R}$ and
$P_{1+J}(A,R)$, $A=R_{0}$, are unimodular. Then $P$ has a unimodular
element $p$ with $p-q\in R^{+}P$, $R^{+}=\bigoplus\limits_{i\geq
1}R_{i}$.}

As a first application, we deduce the results of Plumstead and
Bhatwadekar/Roy on the Serre dimension of polynomial rings
(cf.~(2.4)). 

If $\dim(R)\geq \dim(A)+1$ and $R$ is an affine algebra over a field,
then in the situation of Theorem \ref{chap10-thm2.1} it suffices to
know that the canonical image $q_{1+JR^{+}}$ of $q$ in $P_{1+JR^{+}}$,
$J=J(A,P)$, is unimodular (cf.~(2.5)).

In \S\ \ref{chap10-sec2}, we apply our results to projective modules
over ``Segre extensions'' of a field $k$, i.e., to a $k$-algebra
$S_{mn}=k[x_{ij}]$, $1\leq i\leq m$, $1\leq j\leq n$ with
$x_{ij}x_{ls}-x_{is}x_{lj}=0$, $1\leq i\leq m$, $1\leq j\leq n$. We
obtain the following results (Theorem \ref{chap10-thm2.7} and
Theorem \ref{chap10-thm2.8}) 
\begin{itemize}
\item[(1)] {\em Every\pageoriginale stably free $S_{mm}$-module $P$ of
rank $(P)\geq 1+\min \{m,n\}$, is free.}

\item[(2)] {\em If $k$ is infinite, then every projective
$S_{mn}$-module $P$ of $\rank(P)\geq 1+\min\{m,n\}$ contains a
unimodular element.}
\end{itemize}

We have not made serious attempts to remove the additional assumption
on $k$ in the assertion 2. It remains an open question, whether all
projective $S_{mn}$-modules are free or at least stably free.

The Segre extensions can also be written as
$S_{mn}=A[\mathfrak{m}T_{2},\ldots,\mathfrak{m}T_{n}]$ where
$A=k[X_{1},\ldots,X_{m}]$ and $\mathfrak{m}$ is the canonical maximal
ideal $\sum\limits^{m}_{i=1}AX_{i}$ of $A$, $T_{2},\ldots,T_{n}$
indeterminates. If, in particular, $n=2$, $S_{mn}$ appears as the
blowing up of $A$ in $\mathfrak{m}$, i.e. as the Rees ring
$S_{m2}=\bigoplus\limits_{i\geq
1}\mathfrak{m}^{n}=A[\mathfrak{m}T]$. Assume, more generally that $A$
is a regular ring and let $\mathfrak{p}$ be a prime ideal in $A$ such
that $A/\mathfrak{p}$ is regular. We are interested in an example
where projective $A[T]$-modules and projective
$A/\mathfrak{p}[T]$-modules are extended from $A$ but projective
$A[\mathfrak{p}T]$-modules are not extended from $A$.

\section*{Acknowledgements}

We would like to thank the colleagues and officials from the Tata
Institute of Fundamental Research for their hospitality during the
International Colloquium on Algebraic Vector Bundles in January
1984. In particular, we are grateful to S.M.~Bhatwadekar\pageoriginale
and R.A.~Rao for many fruitful discussions.

\section{The basic matrix lemma and the graded patching
theorem}\label{chap10-sec1} 

Let $M$ be a module over a ring $R$ and let $A$ be a subring of
$R$. $M$ is called {\em extended from $A$}, if there exists an
$A$-module $N$ with $M=R\otimes_{A}N$. If $R=\bigoplus\limits_{i\geq
0}R_{i}$ is a positively graded ring with $A\cong R_{0}$ then
$N=M/R^{+}M$. In terms of relation modules, extendability can be
expressed by saying that $M=R^{n}/RK$, where $K$ is a submodule of
$A^{n}$ and $RK$ is the $R$-module generated by the vectors in $K$
considered as vectors in $R^{n}$. This is a consequence of right
exactness of the tensor product. Hence $M$ is extended from $A$ if and
only if it can be presented by a matrix with coefficients in $A$.

Now assume that there exist comaximal elements $u_{1}$, $u_{2}\in A$
such that $M_{i}=M_{u_{i}}$ is extended from $A_{i}=A_{u_{i}}$,
$i=1,2$. It is easy to see that there exist two submodules $N_{1}$,
$N_{2}$ of $M$ with the following properties: (1) $M=N_{1}+N_{2}$, (2)
$M_{i}=(N_{i})_{i}=(N_{i})_{u_{i}}$, $i=1,2$ (3) $N_{i}$ {\em has a
relation module $L_{i}\subset R$ such that $R_{i}L_{i}$,
$R_{i}=R_{u_{i}}$, is generated by vectors with components in the
canonical image of $A$ in $A_{i}$, $i=1,2$.} The epimorphisms
$R^{s_{i}}\to N_{i}$ with kernel $L_{i}$ give rise to a presentation
of $M$ of the form $R^{m}\xrightarrow{j}R^{s_1}\oplus R^{s_{2}}\to
M\to 0$, where $j(R^{m})$ is the fiber product of $R^{s_{1}}$ and
$R^{s_{2}}$ over $M$ with respect to the compositions of $R^{s_{i}}\to
N_{i}$ and the natural inclusion $N_{i}\to M$, $i=1$, $2$. It
follows\pageoriginale from (2) that $u^{l}_{i}M\subset N_{i}$ for a
suitable $l\in \mathbb{N}$, $i=1,2$, and (1) allows to assume that $j$
is given by a matrix $\underline{D}$ of the form
$$
\underline{D}=
\left(
\renewcommand{\arraystretch}{1.5}
\begin{array}{c|c}
\underline{V}_{2} & \underline{0}\\
\hline
u^{l}_{1}\underline{E} & \underline{B}_{1}\\
\hline
\underline{B}_{2} & u^{l}_{2}\underline{E}'\\
\hline
\underline{0} & \underline{V}_{1}
\end{array}
\right),\quad \underline{E}\text{~~ and~~ } \underline{E}'\text{~~
unit matrics.}
$$
The submatrices $\underline{V}_{i}$ present $N_{i}$, $i=1,2$. They can
be chosen in the form
$\underline{V}_{i}=\left(\dfrac{\underline{V}_{i}}{\underline{C}_{i}}\right)$,
where the canonical image of $\underline{V}'_{i}$ in $\Mat(A_{i})$
presents $(N_{i})_{i}$, $i=1,2$. This implies that, over $R_{1}$, the
row vectors of $\underline{D}$ are generated by the row vectors of
$$
\underline{D}_{1}=\left(
\renewcommand{\arraystretch}{1.5}
\begin{array}{c|c} 
\underline{V}'_{2} & \underline{0}\\
\hline
u^{l}_{1}\underline{E} & \underline{B}_{1}
\end{array}
\right),
$$
and that, over $R_{2}$, they are generated by those of 
$$
\underline{D}_{2}=
\left(
\renewcommand{\arraystretch}{1.5}
\begin{array}{c|c}
\underline{B}_{2} & u^{l}_{2}\underline{E}'\\
\hline
\underline{0} & \underline{V}_{1}
\end{array}
\right).
$$
Since $u_{1}$ and $u_{2}$ are comaximal, we see that in
$\underline{D}$ the matrices $\underline{V}_{i}$ can be replaced by
$\underline{V}'_{i}$, $i=1,2$. Therefore, we may assume that the
matrices $\underline{V}_{i}$ themselves have coefficients in
$A$. Notice that there exists a $t\in \mathbb{N}$ such that
$u^{t}(u^{l}_{1}u^{l}_{2}\underline{E}'-\underline{B}_{2}\underline{B}_{1})$
is a right multiple of $\underline{V}_{1}$ and\pageoriginale
$u^{t}(u^{l}_{1}u^{l}_{2}\underline{E}-\underline{B}_{1}\underline{B}_{2})$
is a right multiple of $\underline{V}_{2}$. To fix terminology, let us
call a matrix $\underline{D}$ of the described form {\em fibred} ({\em
over $A$ with respect to $u_{1}$, $u_{2}$}). We are going to show that
every fibred matrix over a positively graded ring
$R=\bigoplus\limits_{i\geq 0}R_{i}$ is equivalent to its ``constant
term'', where ``equivalence'' means that we need only elementary
transformations. 

Let us first fix some notation concerning positively graded rings
$R=\bigoplus\limits_{i\geq 0} R_{i}$. As usual, we set
$R^{+}=\bigoplus\limits_{i\geq 0}R_{i}$ and call a $r\in R_{i}$
homogenous of degree $i$. If $r=\sum\limits_{i\geq 0}r_{i}$, then
$r_{0}$ is the constant term of $r$. In correspondence with the
decomposition of $R$ into a direct sum of homogenous components
$R_{i}$, we have a decomposition of matrices
$\underline{B}\in \Mat(R)$ into a sum $B=\sum\limits_{i\geq 0}B_{i}$,
$\underline{B}_{i}\in \Mat(R_{i})$, which is canonically induced by
the decomposition of the coefficients. Every $a\in R_{0}$ induces a
$R_{0}$-algebra homomorphism $h_{a}:R\to R$ with
$h_{a}\left(\sum\limits_{i\geq 0}r_{i}\right)=\sum\limits_{i\geq
0}a^{i}r_{i}$. If $\underline{D}$ is a matrix over $R$, then
$h_{a}\cdot \underline{D}(0)$ denotes the constant term of
$\underline{D}$. 

The next lemma is the crucial observation in our approach.

\begin{sublem}\label{chap10-lem1.1}
Let $R=\bigoplus\limits_{i\geq 0}R_{i}$ be a positively graded ring
and let $D$ be a matrix with coefficients in $R$ of the form 
$$
\underline{D}=
\left(
\renewcommand{\arraystretch}{1.5}
\begin{array}{c|c}
u^{l}\underline{E} & \underline{B}_{1}\\
\hline
\underline{B}_{2} & \underline{B}_{3}\\
\hline
\underline{0} & \underline{V}
\end{array}
\right),
$$\pageoriginale
where $\underline{E}$ is a unit matrix, $\underline{V}$ has
coefficients in $R_{0}$, and $u\in R_{0}$. Assume that
$u^{t}(u^{l}B_{3}-B_{2}B_{1})=GV$, $G$ a matrix, $t$ a natural
number. Then, for all $n>t+l$ and all $a$, $c\in R_{0}$, the matrices
$h_{a}(\underline{D})$ and $h_{a+cu^{n}}(\underline{D})$ are
equivalent. 
\end{sublem}

\begin{proof}
We extend $R$ to a polynomial ring $R'=R[T]$, $T$ an indeterminate
$R'$ can be considered as a positively graded ring with homogeneous
components $R'_{i}=R_{i}[T]$, $i\geq 0$. The $R_{0}$-algebra
homomorphisms $h_{a}$, $a\in R_{0}$, extend canonically to
$R'_{0}$-algebra homomorphisms of $R'$. Set $h=h_{a+cu^{n}}$. The
matrix $h_{a}(\underline{D})$ is equivalent to a matrix 
$$
\underline{D}_{0}=
\left(
\renewcommand{\arraystretch}{1.5}
\begin{array}{c|c}
u^{l}\underline{E} & h(\underline{B}_{1})\\
\hline
h(B_{2}) & \underline{C}\\
\hline
\underline{O} & \underline{V}
\end{array}
\right)
$$
where $C$ is a submatrix, whose relations with $h(\underline{B}_{3})$
and $\underline{V}$ will now be described. There exist matrices
$\underline{L}$, $\underline{L}'$ with 
$$
\renewcommand{\arraystretch}{1.2}
\left(
\begin{array}{c|c}
\underline{E} & \underline{O}\\
\hline
\underline{L} & \underline{E}'
\end{array}
\right)
\left(
\begin{array}{c|c}
u^{l}\underline{E} & h_{a}(\underline{B}_{1})\\
\hline
h_{a}(\underline{B}_{2}) & h_{a}(\underline{B}_{3})
\end{array}
\right)
\left(
\begin{array}{c|c}
\underline{E} & \underline{L}\\
\hline
\underline{O} & \underline{E}
\end{array}
\right)=
\left(
\begin{array}{c|c}
u^{1}\underline{E} & h(\underline{B}_{1})\\
\hline
h(\underline{B}_{2}) & \underline{C}
\end{array}
\right)
$$
$\underline{E}'$ a unit matrix. A simple calculation leads to 
$$
u^{l}\underline{C}-h(\underline{B}_{2})h(\underline{B}_{1})-u^{l}h_{a}(\underline{B}_{3})-h_{a}(\underline{B}_{2})h_{a}(\underline{B}_{1}) 
$$
and hence to
$$
u^{l}(\underline{C}-h_{a}(\underline{B}_{3}))=h(\underline{B}_{2})h(\underline{B}_{1})-h_{a}(\underline{B}_{2})h_{a}(\underline{B}_{1}). 
$$
Since $h(r)-h_{a}(r)\in R'T$ for all $r\in R^{1}$, we obtain that
$\underline{C}-h_{a}(\underline{B}_{3})$ and
$h(\underline{B}_{3})-h_{a}(B_{3})$ are divisible by $T$. Hence
$\underline{C}-h(\underline{B}_{3})$ is divisible by $T$ and we have
$\underline{C}-h(\underline{B}_{3})=\underline{H}T$, $\underline{H}$ a
matrix. We have 
$$
u^{t}(u^{l}\underline{C}-h(\underline{B}_{2})h(\underline{B}_{1}))=u^{t}(u^{l}h_{a}(\underline{B}_{3})-h_{a}(\underline{B}_{2})h_{a}(\underline{B}_{1}))=h_{a}(G)V. 
$$
Applying $h$ to
$$
u^{t}(u^{l}\underline{B}_{3}-\underline{B}_{2}\underline{B}_{1})=\underline{G}\underline{V}
$$
we obtain  
$$
u^{t}(u^{l}h(\underline{B}_{3})-h(\underline{B}_{2})h(\underline{B}_{1})=h(\underline{G})\underline{V},
$$\pageoriginale
and hence
$$
u^{t+1}(\underline{C}-h(\underline{B}_{3}))=\underline{W}_{0}\underline{V}_{0},\underline{W}_{0}=h_{a}(\underline{G})-h(\underline{G}).
$$ 
Because $\underline{C}-h(\underline{B}_{3})=\underline{H}T$ and
$\underline{V}$ has coefficients in $R_{0}$, $T$ divides
$\underline{W}_{0}$ and we have $\underline{W}_{0}=\underline{W}T$ for
a matrix $\underline{W}$. This implies
$u^{t+l}\underline{H}=\underline{W}\underline{V}$. Now we return from
$R'$ to $R$ by the substitution $T\to cu^{n}$, $n\geq t+l$, $c\in
R_{0}$, the result of which we denote by a star ``$\ast$''. We have
$\underline{C}^{*}-h^{*}(\underline{B}_{3})=cu^{n}\underline{H}^{*}=\underline{S}\underline{V}$,
$\underline{S}=cu^{n-1-t}\underline{W}^{*}h^{*}=h_{a+cu^{n+1}}$. We
obtain 
$$
\underline{D}^{*}=
\renewcommand{\arraystretch}{1.5}
\left(
\begin{array}{c|c}
u^{l}\underline{E} & h^{*}(\underline{B}_{1})\\
\hline
h^{*}(\underline{B}_{2}) &
h^{*}(\underline{B}_{3})+\underline{S}\underline{V}\\
\hline
\underline{0} & \underline{V} 
\end{array}
\right).
$$
Hence $\underline{D}^{*}$ it is equivalent to
$h^{*}(\underline{D})$. Because $h_{a}(\underline{D})$ is equivalent
to $\underline{D}_{0}$ with coefficients in $R$, it is equivalent to
$\underline{D}^{*}_{0}$, whence to $h^{*}(\underline{D})$. This proves
the assertion.
\end{proof}

In terms of modules, Lemma \ref{chap10-lem1.1} implies the following

\begin{subthm}\label{chap10-thm1.2}
Let $M$ be a finitely presented module over a positively graded ring
$R=\bigoplus\limits_{i\geq 0}R_{i}$. Assume that $M=R\otimes_{R'}M'$,
$R'=h_{a}(R)$ for an $a\in R_{0}$ and $M'$ a finitely presented
$R'$-module. If $M'_{u}$ is extended from $(R_{0})_{u}$ for a $u\in
R_{0}$, then $M$ is extended from $h_{a+cu^{n}}(R)$ for all $c\in A$
and sufficiently high $n\in \mathbb{N}$. 
\end{subthm}

\begin{proof}
Since $M'_{u}$ is extended from $(R_{0})_{u}$' it can be presented by
a matrix of the form $h_{a}(\underline{D})$ where $\underline{D}$ has
the form as in Lemma \ref{chap10-lem1.1}.\pageoriginale By
Lemma \ref{chap10-lem1.1}, $M$ can be presented by every matrix
$h_{a+cu^{n}}(\underline{D})$, $c\in R_{0}$, $n\in \mathbb{N}$
sufficiently high. 
\end{proof}

Now it is easy to generalize Quillen's patching theorem (\cite[Theorem
1]{chap10-Qu}) from polynomial rings to positively graded rings.

\begin{subthm}\label{chap10-thm1.3}
Let $M$ be a finitely presented module over a positively graded ring
$R=\bigoplus\limits_{i\geq 0}R_{i}$ and let $J(A,M)$ be the set of all
$u\in A$, $A=R_{0}$ for which $M_{u}$ is extended from $A_{u}$. Then
$J(A,M)$ is an ideal in $A$. If, for all maximal ideals $\mathfrak{m}$
of $A$, the localizations
$M_{\mathfrak{m}}=M_{A\backslash \mathfrak{m}}$ are extended from
$A_{\mathfrak{m}}$, then $M$ is extended from $A$. 
\end{subthm}

\begin{proof}
It suffices to show that $M$ is extended from $A$, if there exist two
comaximal elements $u$, $v\in J(A,M)$. If $a=1$ it follows from
(\ref{chap10-thm1.2}) that $M$ is extended from $h_{a}(R)$,
$a=1+dv^{n}$, for all $d\in A$ and sufficiently high
$n\in \mathbb{N}$. Since $u$ and $v$ are comaximal, $M$ can be
presented by a fibred matrix $D$ with respect to $v$ and $u$. Then
$h_{a}(\underline{D})$ is also fibred with respect to $v$ and $u$ and
it presents $M$ by Lemma \ref{chap10-lem1.1}. Hence $M$ is not only
extended from $h_{a}(R)$ but also an extension of a $h_{a}(R)$-module
$M'$ such that $M'_{u}$ is extended from $A_{u}$. Hence $M$ is
extended from $h_{a+cu^{n}}(R)$ for all $c\in A$ and sufficiently high
$n\in \mathbb{N}$. Since $u$, $v$ are comaximal, it follows that $M$
is extended from $A=h_{0}(R)$. 
\end{proof}

\setcounter{subsection}{3}
\subsection{}\label{chap10-sec1.4}
The importance of Quillen's patching theorem in case of a polynomial
ring $R=A[T]$ comes from the possibility to generalize\pageoriginale
the ``local Horrocks theorem'' to the affine Horrocks
theorem. Avoiding the use of monic polynomials the local Horrocks
theorem can be formulated as follows: {\em Let $P$ be projective
module over a polynomial ring $A[T]$, where $A$ is local. Assume there
exists a free submodule $F$ of $P$ such that $P/F$ is a finite
$A$-module. Then $P$ is free.} Quillen's patching theorem shows that
this theorem remains valid for an arbitrary ring $A$. In the case of
positively graded rings, one can show the following generalized
version of Horrocks theorem: {\em Let $P$ be a projective module over
a positively graded ring $R=\bigoplus\limits_{i\geq 0}R_{i}$,
$A=R_{0}$ a noetherian ring of finite Krull dimension $d$ and
$R=A[x_{1},\ldots,x_{n}]$, $x_{i}$ homogeneous of positive
degree. Assume that $\dim R=d+1$ and that $(Rx_{1}:R^{+})\cap
R^{+}=Rx_{1}$. If there exists a projective submodule $F$ of $F$ such
that $F$ is extended from $A$ and $P/F$ is a finite $A$-module, then
$P$ is extended from $A$.} This result does not look very
satisfactory. We have not found really interesting applications and
omit the proof.

Let us add the remark that Roitman's converse of Quillen's patching
theorem also extends to the graded case: {\em Let $R$ be a positively
graded ring as above with $A$ as its zero component and let $S$ be any
multiplicative subset of $A$. If all projective $R$-modules are
extended from $A$ then all projective $R_{s}$-modules are extended
from $A$ then all projective $R_{s}$-modules are extended from
$A_{s}$.}

We finish this section with an interesting result of Vorst
(\cite[Theorem 1.1]{chap10-Vo}).

\setcounter{subprop}{4}
\begin{subthm}[Vorst]\label{chap10-thm1.5}
Let $A$ be a noetherian ring such that all\pageoriginale projective
$A[T_{1},\ldots,T_{n}]$-modules are extended from $A$. Then every
projective module $P$ over a discrete Hodge $A$-algebra is extended
from $A$.
\end{subthm}

\begin{proof}
Recall that $R$ is a discrete $A$-Hodge algebra, if $R$ is isomorphic
to a residue class ring $A[T_{1},\ldots,T_{n}]/\mathfrak{a}$, where
$\mathfrak{a}$ is generated by monomials. It is easy to see that it
suffices to treat the case that $R$ is reduced. Then $\mathfrak{a}$ is
generated by square free polynomials. We induct on $n$. If $n\leq 1$,
then $R=A$ or $R=A[T]$ if allow $\mathfrak{a}=0$. Suppose $n\geq
2$. The ideal $\mathfrak{a}$ can be written as
$\mathfrak{b}_{0}+\mathfrak{b}_{1}T$, $T=T_{n}$, where $b_{0}$ and
$b_{1}$ are generated by square free monomials in
$B=A[T_{1},\ldots,T_{n-1}]$; $C=B/\mathfrak{b}_{0}$ is a discrete
Hodge $A$-algebra and by the induction hypothesis all projective
$C$-modules are extended from $A$. We have $R=C[T]/\mathfrak{b}_{1}T$,
where the bar denotes residue class formation modulo
$\mathfrak{b}_{0}\cdot D=C/\mathfrak{b}_{1}\cong
B/(\mathfrak{b}_{0},\mathfrak{b}_{1})$, is a discrete Hodege
$A$-algebra and hence $D[T]$ is a discrete Hodge $A[T]$-algebra. By
the induction hypothesis, all projective $D$-modules are extended from
$A$. Now, we have the following result:
\end{proof}

\begin{sublem}\label{chap10-lem1.6}
Let $C[T]$ be a polynomial ring over a ring $C$ and let $\mathfrak{b}$
be an ideal in $C$ and $R=C[T]/\mathfrak{b}T$. If every projective
$C/\mathfrak{b}[T]$-module is extended from $C/\mathfrak{b}$ then
every projective $R$-module is extended from $C$.
\end{sublem}

\begin{proof}
Let $t$ denote the residue class of $T$ modulo $\mathfrak{b}T$. One
has $\mathfrak{b}t=0$ in $R=C[t]$ and hence
$R\mathfrak{b}=\mathfrak{b}$. Since
$R/R\mathfrak{b}=C/\mathfrak{b}[T]$, for every projective $R$-module
$P$, the factor module $P/\mathfrak{b}P$ is extended from
$C/\mathfrak{b}$. Notice that $R$ is a graded $C$-algebra,
$R=\bigoplus\limits_{i\geq 0}R_{i}$\pageoriginale with
$R_{i}=Ct^{i}_{t}$. By Theorem \ref{chap10-thm1.3}, it suffices to
handle the case that $C$ is local. Then $P/\mathfrak{b}P$ is free. Set
$r=\rank P/\mathfrak{b}P$. There exist a submodule
$F=\sum\limits^{r}_{i=1}Rf_{i}$ such that the residue classes $f_{i}$
of $f_{i}$ modulo, $\mathfrak{b}P$, $1\leq i\leq r$, form a basis of
$P/\mathfrak{b}P$. We have $P=F+\mathfrak{b}P$. Since
$R\mathfrak{b}=\mathfrak{b}$, there exists a $b\in \mathfrak{b}$ such
that $(1+b)P\subset F$. But $C$ is local and hence $1+b$ a unit. So
$P=F$. If $\sum r_{i}f_{i}=0$, then $r_{i}\in \mathfrak{b}$ for all
$1\leq i\leq n$, because $F/\mathfrak{b}F$ is free and
$R\mathfrak{b}=\mathfrak{b}$. This implies that $P$ is extended from
$C$.
\end{proof}

Due to this lemma applied to $D=C/\mathfrak{b}_{1}$, we obtain that
projective $R$-modules are extended from $C$, whence, from $A$, by the
induction hypothesis. This proves Vorst's result.

\section{Unimodular elements in projective modules over graded
rings}\label{chap10-sec2}

Let $P$ be a projective module over a commutative ring $R$, and let
$P^{*}=\Hom_{R}(P,R)$ be the dual of $P$. An element $P\in P$ is
called {\em unimodular} if the ideal $0_{P}(p)=\{\varphi(p),\varphi\in
aP^{*}\}$ equals the whole ring $R$. This is equivalent to the
property that $Rp$ is a direct summand of $P$. If $P$ is generated by
elements $p_{1},\ldots,p_{m}$ and $P^{*}$ by elements
$q^{*}_{1},\ldots,q^{*}_{n}$, the module $P$ is isomorphic to the
module that is generated by the rows
$(q^{*}_{1}(p_{i}),\ldots,q^{*}_{n}(p_{i}))$, $1\leq i\leq m$, of the
matrix $(q^{*}_{i}(p_{i}))$, $1\leq i\leq m$, $1\leq j\leq n$. We call
such a matrix a {\em presenting matrix} of $P$. In this ``matrix
picture'' an element $p$ is unimodular if and only if the coefficients
of the corresponding row generate the whole ring. 

Assume\pageoriginale that there exists an element $s$ in a subring $A$
of $R$ such that $P_{s}$ is free. Set $t=\rank P_{s}$. It is easy to
see that then there exist a submodule $F$ of $P$ with a system
$f_{1},\ldots,f_{t}$ of generators and homomorphisms
$f^{*}_{1},\ldots,f^{*}_{t}\in P^{*}$ and $l\in \mathbb{N}$ such that
$s^{l}P\subset F$ and $f^{*}_{i}(f_{j})=s^{l}\delta_{ii}$ where
$\delta_{ij}$ is the Kronecker symbol. If we include the $f_{i}$,
$1\leq i\leq t$, in a system of generators of $P$, we obtain a
presenting matrix $D$ of $P$ with a submatrix $s^{l}\underline{E}$,
$E \ a\ t\times t$-unit matrix, and with rank $\underline{D}_{s}=t$
where $\underline{D}_{s}$ denotes the canonical image of
$\underline{D}$ in $\Mat(R_{s})$. Let us call a matrix $D$ with this
porperty $s$-{\em distinguished.}

The following theorem corresponds to Criterion $I$ in
(\cite{chap10-Bh/Li/Ra}). It is the main result of this section.

\begin{subthm}\label{chap10-thm2.1}
Let $P$ be a projective module over a positively graded ring
$R=\bigoplus\limits_{i\geq 0}R_{i}$. Assume that there exists an
element $q\in P$ such that $q_{1+R}+\in Um(P_{1+R^{+}})$ and $q_{1+J}\in
Um(P_{1+J})$, $J=J(R_{0},P)$. Then $P$ contains a unimodular element
$p$ with $p-q\in JR^{+}P$. 
\end{subthm}

\begin{proof}
The assumptions on $q_{1+R^{+}}$ and $q_{1+J}$ mean that
$(0_{P}(q),R^{+})=R$ and $(0_{P}(q)\cap A,J)=A$, $A=R_{0}$. Hence it
follows that there exist a finitely generated ideal
$I=(u_{1},\ldots,u_{t})\subset J$ with $(0_{P}(q)\cap A,I)=A$. Because
$P$ is projective, we may assume, without loss of generality, that
$P_{u_{i}}$ is free. $1\leq i\leq t$. By the remark preceding the
theorem, we see that $P$ is presentable by a matrix $\underline{D}$
that is $u_{i}$-distinguished for all $i$, $1\leq i\leq t$. Let
$I_{j}=(u_{1},\ldots,u_{j})\subset I$,\pageoriginale $1\leq j\leq
t$. We show by induction that to every $j$ there exists a
$n\in \mathbb{N}$ such that $\underline{D}$ is equivalent to all
matrices $h_{a}(\underline{D})$, $a\in 1+I^{n}_{j}$. Since
$\underline{D}$ is $u_{1}$-distinguised it has the form described in
Lemma \ref{chap10-lem1.1} with $u=u_{1}$ (up to permutations of rows
and columns) including the additional assumption stated in the
lemma. So the assertion in case of $j=1$ follows from Lemma 
\ref{chap10-lem1.1} with $u=u_{j+1}$, $j<t$, and hence we obtain that
$h_{a}(\underline{D})$ and so $\underline{D}$ itself is equivalent to
all matrices $h_{a+cu^{n}_{j+1}}(\underline{D})$, $c\in A$ and $n$
suitably high. This finishes the induction. In case $j=t$, we have
that for a suitably high $n\in \mathbb{N}$ the matrix $\underline{D}$
is equialent to all $h_{a}(\underline{D})$, $a\in 1+I^{n}$. It follows
from $A=(0_{P}(q)\cap A,I)$ that there exists a $w\in I^{n}$ with
$1+w\in 0_{P}(q)$, and hence $\underline{D}$ is equivalent to
$\underline{D}'=h_{1+w}(\underline{D})$. If $q$ corresponds to a row
of the form $(q^{*}_{1}(q),\ldots,q^{*}_{l}(q))$, $P^{*}=\sum
Rq^{*}_{i}$, then the corresponding row of $D'$ has the form
$(v^{*}_{1}(p),\ldots,v^{*}_{1}(p))$ for a $p\in P$ and $P^{*}=\sum
Rv^{*}_{i}$ such that $v^{*}_{i}(p)=h_{1+w}(q^{*}_{i}(q))$, $1\leq
i\leq l$. Since $h_{1+w}(a)=a$ for $a\in A$ we obtain $0_{P}(q)\cap
A\subset 0_{P}(p)\cap A$, whence $1+w\in 0_{P}(p)$. Therefore
$0_{P}(q)$ contains the constant terms of the $v^{*}_{i}(p)$, $1\leq
i\leq l$, which are equal to the constant terms of $q^{*}_{i}(q)$ for
each $i$. But these constant terms generate $A$. This shows that $p$
is unimodular and that $p-q\in wR^{+}P\subset JR^{+}P$. 
\end{proof}

The following result of Eisenbud and Evans (\cite{chap10-E/E}) is
crucial for our applications of Theorem \ref{chap10-thm2.1}.

\begin{subthm}\label{chap10-thm2.2}
Let $P$ be a projective module of rank $r$ over a\pageoriginale
noetherian ring $A$ and let $p=(p_{1},a)$ be a unimodular element of
$P\oplus A$. Then there exists a $p_{2}\in P$ such that
$ht(0_{p}(p_{1}+ap_{2}))\geq \min \{r,ht(0_{p}\oplus_{A}(p)\}$. 
\end{subthm}

For example, this theorem implies a well known theorem of Serre:


\begin{subthm}\label{chap10-thm2.3}
Let $A$ be a noetherian ring of finite Krull dimension $d$. Every
projective $A$-module $P$ of rank $(P)\geq d+1$ has a unimodular
element. 
\end{subthm}

In 1979, Plumstead $[P]$ settled a question of Eisenbud and
Evans, which asked whether (\ref{chap10-thm2.3}) remains valid for a
polynomial rign $A[T]$. In 1981, Bhatwadekar and Roy generalized
Plumstead's result making available Plumstead's patching technique for
many indeterminates. As a kind of justification of our matrix
pictures, we now give a short deduction of their result from
Theorem \ref{chap10-thm2.1}. 

\begin{subthm}[Bhatwadekar/Roy]\label{chap10-thm2.4}
Let $P$ be a projective module over a polynomial ring
$R=A[T_{1},\ldots,T_{n}]$, $A$ a noetherian ring of $\dim A=d$. If
rank $(P)\geq d+1$, then $P$ has a unimodular element. 
\end{subthm}

\begin{proof}
We use induction on $n$. It suffices to handle the case that $R$ is
reduced. If $n=0$, the assertion follows from
(\ref{chap10-thm2.3}). Suppose $n\geq 1$. If $d=0$, then $A$ is a
direct product of fields and $P$ is even free, by the Theorem of
Quillen and Suslin (\cite{chap10-Qu}). Let $d\geq 1$. The localization
$P_{S}$ of $P$ at the set $S$ of non zero divisors\pageoriginale of $R$ is free
because $\dim R_{S}=0$. Therefore, there exists $s\in S$ such that
$P_{s}$ is free, and hence the Quillen ideal $J=J(A,P)$ has height
$\geq 1$. Following the procedure of Bhatwadekar and Roy, we consider
the factor module $\overline{P}=P/JTP$ over
$\overline{R}=R/(JT)=\overline{A}[T_{1},\ldots,T_{n-1}]$, where
$\overline{A}=A[T]/(JT)$, $T=T_{n}$. By the induction hypothesis, $P$
contains a unimodular element. This means that $P$ contains an element
$q$ with $(0_{P}(q),JT)=R$. Theorem \ref{chap10-thm2.2} allows to
assume that $ht(0_{P}(q))=d+1$. There exists a Nagata transformation
of indeterminate of the form $T'_{i}=T_{i}+T^{r_{i}}$, $1\leq i\leq
n-1$, $T'=T$, such that $0_{P}(q)$ contains a polynomial $f(T)$ that
is monic in $T$ over the ring
$R'=A[T_{1},\ldots,T'_{n-1}]$. Furthermore, $0_{P}(q)$ contains an
element $g$ of the form $g=1+hT$ with $h\in RJ$. The resultant res
$(f,g)$ of $f$ and $g$ in $R'$ has the form $1+h'$, $h'\in R'J$. Hence
$(0_{P}(q)\cap R',J(R',P))=R'$. Now Theorem \ref{chap10-thm2.1}
implies that $P$ contains a unimodular element.
\end{proof}

The following theorem generalizes Plumstead's result
((\ref{chap10-thm2.4}) with $n=1$):

\begin{subthm}\label{chap10-thm2.5}
Let $R=\bigoplus\limits_{i\geq 0}R_{i}$ be a positively graded ring,
$A=R_{0}$ noetherian of finite dimension $d$. Assume that $R$ is
finitely generated, $R=A[t_{1},\ldots,t_{n}]$, $t_{i}$ homogeneous in
$R^{+}$, $1\leq i\leq n$ with $\dim R=d+1$. Further assume that the
kernel of an $A$-epimorphism $\varphi$ from $A[T_{1},\ldots,T_{n}]$ to
$R$ with $\varphi(T_{1})=t_{i}$ has $ht(\ker \varphi)\geq n-1$. Let
$P$ be a projective $R$=module of rank $P\geq \dim R$. If
$P_{1+JR^{+}}$, $J=J(A,P)$, has a unimodular element, then $P$ has a
unimodular element. 
\end{subthm}

\begin{proof}
There\pageoriginale exist a $q\in P$ such that
$(0_{P}(q),JR^{+})=R$. By the result of Eisenbud and Evans
(see \ref{chap10-thm2.2}), we may assume that $ht(0_{p}(q))\geq
d+1$. Since $ht(\ker \varphi)\geq n-1$, the inverse image
$\mathfrak{b}=\varphi^{-1}(0_{P}(q))$ of $0_{P}(q)$ has
$ht(\mathfrak{b})\geq n+d=\dim B$. After suitable Nagata
transformations of indeterminates (cf.~the proof of
(\ref{chap10-thm2.4})) we obtain indeterminates $T'_{1},\ldots,T'_{n}$
with $B=A[T'_{1},\ldots,T'_{n}]$ and a sequence of polynomials
$f_{1}(T'_{1}),\ldots,f_{n}(T_{n})$ such that $f_{1}(T_{i})$ is monic
in $T_{i}$ with coefficients in $A[T_{1},\ldots,T_{i-1}]$, $1\leq
i\leq n$. It follows that $\overline{B}=B/\mathfrak{b}$ is a finite
$A/\mathfrak{b}\cap A$-module, and hence we conclude from
$\overline{B}J=\overline{B}$ that $(0_{P}(q)\cap A,J)=A$. Now the
assertion follows from Theorem \ref{chap10-thm2.1}.
\end{proof}

\begin{subremark}\label{chap10-rem2.6}
We do not know, if Theorem \ref{chap10-thm2.5}. remains valid without
the special assumption on $ht(\ker \varphi)$. But in case of affine
algebras over a field, this assumption is fulfilled. To have a simple
example, let us handle a well-known case: Let $R=k[x,y,z]$,
$z^{n}-xy=0$, $k$ a field. Murthy has shown that projective
$R$-modules are free. Since $R$ is graded and normal, we have
$\Pic(R)=0$. So it remains to show that a given projective $R$-module
$P$ of rank $P\leq 2$ has a unimodular element. $R$ is a graded
$A$-algebra with $A=k[x]$ and $\dim R=2=1+\dim A$. Moreover,
$R_{x}=A_{x}[z]$, $z$ algebraically independent over $A_{x}$. So
$P_{x}$ is free and $x\in J$. This implies $\dim R/JR^{+}=1$,
$R^{+}=(y,z)$, $\deg y=n$, $\deg z=1$. By (\ref{chap10-thm2.3})
$P/JR^{+}P$ has a unimodular element. Now we conclude from
(\ref{chap10-thm2.5}) that $P$ has a unimodular element. 
\end{subremark}

Now we shall consider projective modules over Segre extensions of a
field $k$. Let us first fix some notation. 

Let\pageoriginale $k$ be a field and let
$\underline{X}=(X_{ij})_{1\leq i\leq m, 1\leq j\leq n}$ be a matrix of
indeterminats $X_{ij}$ over $k$. Let $\mathfrak{d}_{mn}$ be the ideal
in the polynomial ring $k[X]$ which is generated by the $2\times
2$-minors of $\underline{X}$. We call the residue class ring
$S_{mn}=K[\underline{X}]/\mathfrak{d}_{mn}$ a {\em Segre extension} of
$k$. The residue classes of $X_{ij}$ will be denoted by $x_{ij}$,
$1\leq i\leq m$, $1\leq j\leq n$. So $S_{mn}$ is defined by the
relations $x_{ij}x_{lt}-x_{it}x_{lj}=0$, $1\leq i$, $l\leq m$, $1\leq
j$, $t\leq n$. The maximal ideal in $S_{mn}$ which is generated by the
$x_{ij}$ is denoted by $\mathfrak{n}(S_{mn})$. It is easy to see that
$ht(\mathfrak{d}_{mn})=(m-1)(n-1)$, $\dim S_{mn}=m+n-1$ and that
$S_{mn}$ is normal. One has an ascending chain $S_{m0}\subset
S_{m1}\subset\ldots \subset S_{mn}$ with  $\dim S_{m,i+1}=\dim
S_{mi}+1$ and $S_{mj}$ is in an obvious way, a positively graded
$S_{mi}$-algebra and birationally equivalent to a polynomial ring in
one variable over $S_{m,j-1}$, $1\leq i\leq j\leq n$. Notice that
$(S_{mn})_{x_{ij}}$ is isomorphic to a Laurent polynomial ring
$k[Y_{1},\ldots,Y_{m+n-1}, Y^{-1}_{1}]$. Hence the localizations
$P_{x_{ij}}$ of a projective $S_{mn}$-module $P$ at $x_{ij}$ are free
for $1\leq i\leq m$, $1\leq j\leq n$. This implies that the Quillen
ideal $J$ of $P$ in $S_{m,n-1}$ contains the maximal ideal
$\mathfrak{n}(S_{m,n-1})$. Therefore $P$ {\em is extended from
$S_{m,n-1}$, if $P_{z}$ is extended from $(S_{m,n-1})_{z}$ for a
$z\in \mathfrak{n}(S_{m,n-1})$.} By Rao's theorem (\cite[Theorem
1.1]{chap10-Ra}) we know that {\em every projective $S_{mn}$-module
$P$ of rank $(P)\geq \dim (S_{mn})$ has unimodular element and that
stably free $S_{mn}$-modules of rank equal to $\dim (S_{mn})$ are
free.}

In this special case, we can prove stronger results. At first, we
consider stably free modules.

\begin{subthm}\label{chap10-thm2.7}
Let\pageoriginale $R=S_{mn}$ be a Segre extension of a field $k$ with $m\leq
n$. Every stably free $R$-module $P$ of rank $(P)\geq m+1$ is free.
\end{subthm}

\begin{proof}
Let us first assume that $k$ is infinite. We proceed by induction on
$n$. If $n=1$, $P$ is even free, because $S_{m0}$ is a polynomial ring
in $m$ indeterminate over $k$. Suppose $n\geq 2$. We show that $P$ is
extended from $S_{m,n-1}$. It suffices to treat the case $P\oplus
R=R^{r+1}$, $r=\rank(P)$. Then $P$ can be presented by a unimodular
vector $\underline{v}\in Um_{r+1}(R)$. By known results of ideal
theory, we can assume that $ht(\mathfrak{a})=r$ and
$(\mathfrak{a},\mathfrak{n}(R))=R$ where $q$ is the ideal generated by
the first $r$ components of $v$. The inverse image $\acute{\mathfrak{a}}$ of
$\mathfrak{a}$ in the polynomial ring $R'=k[X]$ has height
$ht(\mathfrak{a}\text{'})=r+(m-1)(n-1)\geq (m-1)n+2$. This implies
that the contractions, $\mathfrak{b}_{i}=\grave{\mathfrak{a}}\cap
B_{i}$ of $\acute{\mathfrak{a}}$' to the subring
$B_{i}=k[X_{il},\ldots,X_{in}]$, have $ht(\mathfrak{b}_{i})\geq 2$,
because, by this contraction, $n(m-1)$ indeterminates are eliminated,
$1\leq i\leq m$. Since $ht(\mathfrak{b}_{i})\geq 2$, the
$\mathfrak{b}_{i}$ contain a {\em homogenous} polynomial $f_{i}$,
$1\leq i\leq m$. Because $k$ is infinite, there exist $c_{j}\in k$
such that after the homogenous linear transformation $X_{in}=X_{in}$,
$X_{ij}=X_{ij}+c_{j}X_{in}$, $1\leq j\leq n-1$, the $f_{i}$ are monic
in $X_{in}$, $1\leq i\leq m$. Since
$X_{ij}X_{lt}-X_{it}X_{lj}=X_{ij}X_{lt}-X_{it}X_{lj}$, $1\leq i$,
$1\leq m$, $1\leq j$, $t\leq n$, we may assume, without loss of
generality, that the $f_{i}$ are monic in $X_{in}$, $1\leq i\leq
m$. It follows that $(\mathfrak{a},R\mathfrak{n}(B))=R$,
$B=S_{m,n-1}$. Moreover, $R/\mathfrak{a}$ is a finite
$B/\mathfrak{a}\cap B$-module. Hence we obtain $(\mathfrak{a}\cap
B,\eta(B))=B$. Thus, we have shown that there exists a $z\in
B\backslash \mathfrak{n}(B)$ such that $P_{z}$ is free and hence
$J(B,P)\neq \mathfrak{n}(B)$. As remarked above, we have
$\mathfrak{n}(B)\subset J(B,P)$, whence $B=J(B,P)$. This shows that
$P$ is extended from $B$. If $k$ is finite, adjoin\pageoriginale an
indeterminate $U$ to $R$. Since $k(U)$ is infinite, $k(U)\otimes_{k}P$
is free, and hence $R[U]\otimes_{R}P$ is free by the affine Horrocks
theorem. Therefore $P$ is free.
\end{proof}

\begin{subthm}\label{chap10-thm2.8}
Let $P$ be a projective module over a Segre extension $R=S_{mn}$ of an
infinite field $k$, $m\leq n$. Every projective module $P$ of rank
$(P)\geq m+1$ has a unimodular element.
\end{subthm}

\begin{proof}
We proceed by induction on $n$. If $n=1$, $P$ is even free. Let $n=2$
and let $\mathfrak{p}$ denote the (prime) ideal in $R$ which is
generated by the coefficients of the $n$-th column of the matrix
$(x_{ij})_{1\leq i\leq m,1\leq j\leq n}$. $R=R/\mathfrak{p}$ is a
Segre extension isomorphic to $S_{m,n-1}$. By the induction hypothesis
$P=P/\mathfrak{p}P$ has a unimodular element, and hence $P$ contains
an element $q$ with $(0_{P}(q),\mathfrak{p})=R$. By the theorem of
Eisenbud and Evans (see (\ref{chap10-thm2.2})), we may assume that
$ht(0_{P}(q))\geq m+1$. As shown in the proof of
Theorem \ref{chap10-rem2.6}, we can assume without loss of generality
that $R/0_{P}(q)$ is a finite $B/B\cap 0_{P}(q)$-module,
$B=S_{m,n-1}$. Furthermore we may assume that
$(0_{P}(q),R\mathfrak{n}(B))=R$. This implies that $B=(0_{P}(q)\cap
B,\mathfrak{n}(B))$. Since $R$ is a positively graded $B$-algebra with
$R^{+}=\mathfrak{p}$ and since $J(B,P)\supset \mathfrak{n}(B)$, we
conclude from Theorem \ref{chap10-thm2.1} that $P$ has a unimodular
element. 
\end{proof}

\begin{subremarks}\label{chap10-rems2.9}
Theorem \ref{chap10-thm2.8} should be valid for arbitrary fields, but
we have no convincing argument to show this. We must leave it as an
open question, if all projective $S_{mn}$-modules are free. Because we
do not believe that this question has an affirmative answer, we are
interested in calculating $K_{0}(S_{mn})$. 
\end{subremarks}


\begin{thebibliography}{}
\bibitem{chap10-Ar} Artamonow,\pageoriginale W. A., Quillen's theorem for graded
algebras, Westnik MGU {\em Math., Nr.} 3, 59-61 (1983) (In Russian).

\bibitem{chap10-Ba/Mu} Bass, H., Murthy, M.~P., Grothendieck groups
and Picard groups of abelian group rings, {\em Ann. of Math.} 86,
16-73 (1967).

\bibitem{chap10-Bh/Li/Ra} Bhatwadekar, S. M., Lindel, H., Rao, R. A.,
The Bass-Murthy Question: Serre Dimension of Laurent Polynomial
Extensions, {\em Invent. Math.} 81, 189-203 (1985). 

\bibitem{chap10-Bh/Ro} Bhatwadekar, S. M., Roy, A., Some theorems
about projective modules over polynomial rings, {\rm J. of Alg.86}, (1983).

\bibitem{chap10-E/E} Eisenbud, D, Evans, G., Generating modules
efficiently -- Theorems from Algebraic $K$-theory, {\em J. of Alg.}
27, 278-305, (1973).

\bibitem{chap10-Mu} Murthy, M. P., {\em Lectures on Projective
Modules,} Notes by J. Yanik (1977). 

\bibitem{chap10-Pl} Plumstead, B. R., The Conjectures of Eisenbud and
Evans, {\em Amer. J. of Math.} 105, 1417-1433 (1983).

\bibitem{chap10-Qu} Quillen, D. Projective modules over polynomial
rings, {\em Invent. Math.} 36, 167-171 (1976).

\bibitem{chap10-Ra} Rao,\pageoriginale R. A., Stability theorems for overrings of
polynomial rings II, {\em J. of Alg.} 78, 437-444 (1982).

\bibitem{chap10-Vo} Vorst, A. C. F., The Serre Problem for Discrete
Hodge Algebras {\em Math. Zeit.,} 184, 425-433 (1983).
\end{thebibliography}


\vskip 1cm

\noindent
Mathematisches Institut\\
Universit\"at Munster\\
Einsteinstr. 62\\
44 M\"unster\\
Federal Republic of Germany

\newpage

~\phantom{a}
\thispagestyle{empty}


