\title{Bundles on $\mathbb{P}^{2}$ with a Quaternionic Structure}\label{chap9}
\markright{Bundles on $\mathbb{P}^{2}$ with a Quaternionic Structure}

\author{By M.-A. Knus}
\markboth{M.-A. Knus}{Bundles on $\mathbb{P}^{2}$ with a Quaternionic Structure}

\date{}
\maketitle

\setcounter{page}{173}

\setcounter{pageoriginal}{224}
\section*{Introduction}\pageoriginale

In this paper, we give a survey of some recent results on 2-bundles
over $\mathbb{P}^{2}_{\mathbb{C}}$, which have a quaternionic
structure. These bundles were studied in \cite{chap9-key7} in
connection with projective ideals of the polynomial ring in two
variables over the real quaternions $\mathbb{H}$.

We present some of the main results of \cite{chap9-key7} in a slightly
different form and give a few generalizations. In particular we
compute the Chern classes of bundles arising from an explicit family
of ideals in $\mathbb{H}[x,y]$ and we indicate how to calculate the
curves of jump lines for some of them. Finally, we mention connections
with bundles over quadrics and over $\mathbb{P}^{3}_{\mathbb{C}}$.

We do not mention any related results for bundles of higher rank
(see \cite{chap9-key6}, \cite{chap9-key10}, \cite{chap9-key12} and
report of Parimala at this Conference). 

Most\pageoriginale of the
results described here were obtained by M.~Ojanguren, R.~Parimala,
R.~Sridharan and the author, mostly as joint work (in many different
combinations), but the author is responsible for possible errors in
the presentation of the results given here.

We thank the Tata Institute for its hospitality and we are thankful to
many participants in the Colloquium for the useful discussions we had
with them in connection with this work.

\section{Quaternion algebras and quadratic bundles}\label{chap9-sec1}

\subsection{}\label{chap9-sec1.1}

Let $X$ be a scheme over $\mathbb{R}$ and let $\mathscr{H}$ be a sheaf
of algebras over $X$ which is a locally free $\mathscr{O}_{X}$-module
of rank 4 and such that the fiber at each real closed point is
isomorphic to the real quaternion division algebra $\mathbb{H}$. We
call such a sheaf of algebras a {\em quaternion algebra} over $X$. Any
quaternion algebra $\mathscr{H}$ has a quadratic structure given by
the {\em reduced norm $n$}. We recall that a {\em quadratic space} (or
a {\em quadratic bundle}) over a $K$-scheme $X$, ($K$ any field of
characteristic $\neq 2$) is a vector bundle $\mathscr{E}$ over $X$
with a symmetric bilinear form
$b:\mathscr{E}\otimes_{\mathscr{O}_{X}}\mathscr{E}\to \mathscr{O}_{X}$
which is nonsingular, i.e. which induces an isomorphism
$\mathscr{E}\xrightarrow{\sim}\mathscr{E}^{*}$. Composing $b$ with the
diagonal
$\mathscr{E}\to \mathscr{E}\otimes_{\mathscr{O}_{X}}\mathscr{E}$, we
get the quadratic form $q$ associated with $b$. We denote a quadratic
bundle by $(E,q)$ or simply $q$.

\subsection{}\label{chap9-sec1.2}

Let $\mathscr{E}$ be a locally free sheaf of rank 4 over $X$ which is
a right $\mathscr{H}$-module, $\mathscr{H}$ a quaternion algebra over
$X$. The restriction of $\mathscr{H}$ to any affine open set $U=\Spec
A$ is a separable $A$-algebra $H$. Hence any projective $A$-module,
which is also an $H$-module, projective as $H$-module. Thus
$\mathscr{E}$ is locally free of rank one as an
$\mathscr{H}$-module\pageoriginale 
(for the Zariski topology of $X$). We denote the category of such
$\mathscr{H}$-modules $\mathscr{E}$ by $P(\mathscr{H})$ and call them
simply $\mathscr{H}$-modules of rank one.

\subsection{}\label{chap9-sec1.3}

Let $\mathscr{H}$ be a quaternion algebra over $X$. We say that a
quadratic space $(E,q)$ over $X$ is of {\em type} $\mathscr{H}$ if
\begin{itemize}
\item[(a)] $\mathscr{E}\in P(\mathscr{H})$

\item[(b)] $q\circ m=q\otimes \mathscr{O}_{X}n$, where
$m:\mathscr{E}\otimes_{\mathscr{O}_{X}}\mathscr{H}\to \mathscr{E}$
defines the operation of $\mathscr{H}$ on $\mathscr{E}$ and $n$ is the
norm of $\mathscr{H}$. 
\end{itemize}

Let $X=\Spec A$. Then a quadratic space over $X$ is a pair $(P,q)$,
where $P$ is a finitely generated projective $A$-module and $q:P\to A$
is a nonsingular quadratic form. Condition (a) means that $P$ is a
projective right $H$-module of rank one for a quaternion algebra $H$
over $A$ and condition (b) means that $q(xh)=q(x)n(h)$, $x\in P$,
$h\in H$.

\subsection{}\label{chap9-sec1.4}

Let $X$ be an $\mathbb{R}$-scheme, let $\mathscr{H}$ be a quaternion
algebra over $X$ and let $(E,q)$ be a quadratic bundle over $X$ of
type $\mathscr{H}$. Then $q$ is locally similar to the norm $n$ of
$\mathscr{H}$. Hence we see that at the fiber of any real closed point
of $X$, the quadratic form $q$ is definite. We say that a quadratic
bundle is {\em positive definite} if the fiber at any real closed
point is positive definite. In the following, we always assume that a
quadratic bundle of type $\mathscr{H}$ is positive definite. This is
not a serious restriction since $q$ or $-q$ is positive definite if
$X$ is connected.

\setcounter{subprop}{4}
\begin{subprop}\label{chap9-prop1.5}
Let\pageoriginale $\mathscr{H}$ be a quaternion algebra over a real
scheme $X$ and 
let $\mathscr{E}\in P(\mathscr{H})$. Then there exists a universal
quadratic map $q:\mathscr{E}\to \mathscr{N}_{\mathscr{E}}$ (in the
category of coherent sheaves over $X$) satisfying property (b) of
(\ref{chap9-sec1.3}). The sheaf $\mathscr{N}_{\mathscr{E}}$ is
invertible and the pair $(q,\mathscr{N}_{\mathscr{E}})$ is uniquely
determined up to isomorphisms by (\ref{chap9-sec1.3}) {\em and} the
condition im $b=\mathscr{N}_{\mathscr{E}}$, where $b$ is the symmetric
bilinear map associated with $q$. 
\end{subprop}

\begin{proof}
If $X$ is affine, we construct $\mathscr{N}_{\mathscr{E}}$ in the
standard way by generators and relations. In order to prove
$\mathscr{N}_{\mathscr{E}}$ is an invertible sheaf, we may assume by
localization that $\mathscr{E}\simeq \mathscr{H}$. In this case the
pair $(n,\mathscr{O}_{X})$, where $n$ is the norm of $\mathscr{H}$, is
universal. Thus $\mathscr{N}_{\mathscr{H}}\simeq \mathscr{O}_{X}$ is
invertible. The construction clearly globalizes for any real scheme.
\end{proof}

We call the quadratic map $q:\mathscr{E}\to \mathscr{N}_{\mathscr{E}}$
the {\em norm} of $\mathscr{E}$. The following results are
consequences of the universal property of $\mathscr{N}_{\mathscr{E}}$.

\begin{subcoro}\label{chap9-coro1.6}
Let $(\mathscr{E},q)$ be a quadratic bundle of type $\mathscr{H}$ over
$X$. Then we have
$\mathscr{N}_{\mathscr{E}}\simeq \mathscr{O}_{X}$. Conversely, if
$\mathscr{E}\in P(\mathscr{H})$ is such that
$\mathscr{N}_{\mathscr{E}}\simeq \mathscr{O}_{X}$, then $\mathscr{E}$
is of type $\mathscr{H}$ for some quadratic structure $q$. The form
$q$ is uniquely determined up to a multiple $\lambda\in
H^{\circ}(X,G_{m})$ (and up to isometries).
\end{subcoro}

\begin{subcoro}\label{chap9-coro1.7}
Assume\pageoriginale that $H^{\circ}(X,G_{m})=\mathbb{R}^{X}$. Let
$(\mathscr{E},q)$ and $(\mathscr{E}',q')$ be two positive definite
quadratic bundles of type $\mathscr{H}$. If $\mathscr{E}$ and
$\mathscr{E}'$ are isomorphic as $\mathscr{H}$-modules, they are
isometric as quadratic spaces.
\end{subcoro}

\begin{subprop}\label{chap9-prop1.8}
Let $X$ be a real scheme such that
$H^{\circ}(X,\mathscr{O}_{X})=\mathbb{R}$ and let $(\mathscr{E},q)$ be
a positive definite quadratic bundle of type $\mathscr{H}$ over
$X$. If $\mathscr{E}\neq \mathscr{H}$, then
$H^{\circ}(X,\mathscr{E})=0$. 
\end{subprop}

\begin{proof}
Let $s\in H^{\circ}(X,\mathscr{E})$, then $q(s)\in
H^{\circ}(X,\mathscr{O}_{X})$ is a constant
$\lambda\in \mathbb{R}$. Since $q$ is positive definite, $\lambda$ is
positive if $s\neq 0$. The section $s$ defines a homomorphism
$\mathscr{H}\to \mathscr{E}$ which is an isometry if $\lambda\neq
0$. Thus $\mathscr{E}\simeq \mathscr{H}$ if $\mathscr{E}$ has a
nontrivial global section.
\end{proof}

The next result is again a consequence of the universal property of
$\mathscr{N}_{\mathscr{E}}$. 

\begin{sublem}\label{chap9-lem1.9}
Let $X$ be a real scheme, let $\mathscr{I}$ be an invertible sheaf on
$X$ and let $\mathscr{E}\in P(\mathscr{H})$ for some quaternion
algebra $\mathscr{H}$ over $X$. Then we have
$\mathscr{N}_{\mathscr{E}}\otimes_{\mathscr{O}_{X}}\mathscr{I}\cong \mathscr{N}_{\mathscr{E}}\otimes_{\mathscr{O}_{X}}\mathscr{I}^{2}$. 
\end{sublem}

\section{Bundles over projective spaces}\label{chap9-sec2}

\subsection{}\label{chap9-sec2.1}

Let $X=\mathbb{P}^{n}_{\mathbb{R}}$ be the real projective $n$-space,
let $\mathscr{H}$ be a quaternion algebra over $X$ and let
$\mathscr{E}\in P(\mathscr{H})$. In view of (\ref{chap9-prop1.5}), we
have $\mathscr{N}_{\mathscr{E}}\simeq \mathscr{O}(n)$ for some
$n\in \mathbb{Z}$. We claim that $n$ is even. By (\ref{chap9-lem1.9}),
we are reduced to the cases $n=0$ and $n=1$. For any real affine
$n$-space $U\subset X$, the restriction $\mathscr{N}_{\mathscr{E}U}$
is free, hence $\mathscr{E}|_{U}$ carries a quadratic form of type
$\mathscr{H}|_{U}$. As noticed in (\ref{chap9-sec1.4}) we may assume
that the form is positive definite. But if $n=1$, the form
would\pageoriginale change sings for a real closed point of some real
affine $n$-space $U$.

\setcounter{subprop}{1}
\begin{subprop}\label{chap9-prop2.2}
Let $\mathscr{H}$ be a quaternion algebra over
$\mathbb{P}^{n}_{\mathbb{R}}$ and let $\mathscr{E}\in
P(\mathscr{H})$. Then, for some $m\in \mathbb{Z}$, $\mathscr{E}(m)$
carries a quadratic form $q$ of type $\mathscr{H}$ which is positive
definite. The form $q$ is unique up to isometry.
\end{subprop}

\begin{proof}
The existence of $q$ follows from (\ref{chap9-sec2.1}). Let $q$ and
$q'$ be two quadratic structures on $\mathscr{E}$ of type
$\mathscr{H}$ and which are positive definite. By
(\ref{chap9-coro1.6}) we have $\lambda\in \mathbb{R}^{X}$ such that
$q'=\lambda q$. Since both forms are positive definite, $\lambda$ is
positive and the forms are isometric. 
\end{proof}


\begin{subremark}\label{chap9-rem2.3}
The uniqueness part of (\ref{chap9-prop2.2}) follows more generally
from the fact that a vector bunder over a real projective scheme
carries at most one positive-definite quadratic structure
(see \cite{chap9-key6}). 
\end{subremark}

\setcounter{subsection}{3}
\subsection{}\label{chap9-sec2.4}
We call the bundle $\mathscr{E}(m)$ of (\ref{chap9-prop2.2}) the {\em
normalization} of $\mathscr{E}$ and we say that $\mathscr{E}\in
P(\mathscr{H})$ is {\em normalized} if it carries a quadratic
structure of type $\mathscr{H}$. 

\subsection{}\label{chap9-sec2.5}

Let $\mathscr{H}_{0}$ be the constant sheaf
$H\otimes_{\mathbb{R}}\mathscr{O}_{X}$ of quaternion algebras over a
real scheme $X$. Let $\mathscr{E}$ be a locally free
$\mathscr{H}_{0}$-module. The embedding $\mathbb{C}\to \mathbb{H}$
induces a complex structure on $\mathscr{E}$. Hence we can associate
with $\mathscr{E}$ a complex bundle over
$X_{\mathbb{C}}=X\times_{\Spec \mathbb{R}}\Spec \mathbb{C}$. We denote
this bundle by $\mathscr{E}_{c}$. 

\setcounter{subprop}{5}
\begin{subprop}\label{chap9-prop2.6}
Let $X=\mathbb{P}^{n}_{\mathbb{R}}$ and let $\mathscr{E}\in
P(\mathscr{H}_{0})$ such that $\mathscr{E}(m)\; / \kern
-7pt \raisebox{-2pt}{{$\widetilde{\rightarrow}$}}\mathscr{H}_{0}$ for
all $m\in \mathbb{Z}$. Then $\mathscr{E}_{c}$ is a stable 2-bundle
over $\mathbb{P}^{n}_{\mathbb{C}}$.
\end{subprop}

\begin{proof}
We\pageoriginale may assume by (\ref{chap9-prop2.2}) that
$\mathscr{E}$ is a quadratic bundle of type $\mathscr{H}_{0}$. In view
of (\ref{chap9-prop1.8}), we have
$H^{0}(\mathbb{P}^{n}_{\mathbb{C}},\mathscr{E}_{c})=0$. Further,
$c_{1}(\mathscr{E}_{c})=0$, since $\mathscr{E}$ is a quadratic
bundle. Thus $\mathscr{E}_{c}$ is stable.
\end{proof}

\begin{subprop}\label{chap9-prop2.7}
Let $\mathscr{E}$ be a quadratic bundle of type
$\mathscr{H}_{0}$. Then the restriction of $\mathscr{E}$ to any real
line $L$ is trivial.
\end{subprop}

\begin{proof}
The restriction of $\mathscr{E}$ to $L$ gives an anisotropic bundle
over $\mathbb{P}^{1}_{\mathbb{R}}$ and the claim follows by a result
of Scharlau \cite{chap9-key14}. 
\end{proof}

\setcounter{subsection}{7}
\subsection{}\label{chap9-sec2.8}

Let
$\sigma:\mathbb{P}^{n}_{\mathbb{C}}\to \mathbb{P}^{n}_{\mathbb{C}}$ be
the usual real structure of $\mathbb{P}^{n}_{\mathbb{C}}$, given by
complex conjugation of the variables. For any $\mathscr{E}\in
P(\mathscr{H}_{0})$, {\em the} multiplication by $j\in \mathbb{H}$
induces an isomorphism
$\mathscr{E}_{c}\to \sigma\ast \mathscr{E}_{c}$, also denoted by $j$,
such that $\sigma\ast j\circ j=-1$. Conversely, any isomorphism
$j\cdot \mathscr{F}\to \sigma\ast\mathscr{F}$, $\mathscr{F}$ a
2-bundle over $\mathbb{P}^{n}_{\mathbb{C}}$, such taht $\sigma\ast
j\circ j=-1$ defines the structure of an $\mathscr{H}_{0}$-module of
rank one on $\mathscr{F}$ considered as a 4-bundle over
$\mathbb{P}^{n}_{\mathbb{R}}$. If $n$ is odd, then
$\mathbb{P}^{n}_{\mathbb{C}}$ has another real structure, without real
points. A stable 2-bundle $\mathscr{E}$ over
$\mathbb{P}^{3}_{\mathbb{C}}$ with a quaternionic structure $j$ with
respect to the nonstandard real structure of
$\mathbb{P}^{3}_{\mathbb{C}}$, such that $c_{1}(\mathscr{E})=0$ and
$\mathscr{E}$ is trivial over real lins is an instanton bundle. Over
$\mathbb{P}^{2}_{\mathbb{C}}$, there is only one real structure, the
usual one. Thus, quadratic bundles of type $\mathscr{H}_{0}$ are in
some sense instantons over $\mathbb{P}^{2}_{\mathbb{C}}$. 

\subsection{}\label{chap9-sec2.9}

Let $M(0,k)$ be the moduli space of stable 2-bundles over
$\mathbb{P}^{2}_{\mathbb{C}}$ with $c_{1}=0$ and $c_{2}=k$. By results
of Barth \cite{chap9-key1}, $M(0,k)$ is an irreducible complex variety
of dimension $4k-3$. Instantons correspond to real points of $M(0,k)$
for the real structure of $M(0,k)$ induced functorially by the real
structure of $\mathbb{P}^{2}_{\mathbb{C}}$. Thus, this set is a real
open smooth manifold of real dimension $4k-3$. Thus, the moduli space
of quadratic bundles of type $\mathscr{H}_{0}$ over
$\mathbb{P}^{2}_{\mathbb{R}}$\pageoriginale is a manifold of dimension
$4k-3$. An explicit description of this manifold is given in
section \ref{chap9-sec9} for $k=2$.

\section{Extensions of isometries}\label{chap9-sec3}

\subsection{}\label{chap9-sec3.1}

Let $K$ be a field of characteristic $\neq 2$ and let $X$ be a
$K$-scheme. We say that a quadratic bundle $(\mathscr{E},q)$ is {\em
anisotropic} if its restriction to any affine open set is
anisotropic. If $X=\Spec A$, then a quadratic space over $X$ is called
anisotropic if $q(x)=0$ implies $x=0$, $x\in P$. The property of
anisotropicity is birational. If $K=\mathbb{R}$, $X_{\mathbb{R}}$ is
connected and $X(\mathbb{R})$ not empty, then positive definite
quadratic bundles are anisotropic.

\setcounter{subprop}{1}
\begin{subprop}\label{chap9-prop3.2}
Let $X$ be an irreducible noetherian scheme over $K$ and let $Y$ be
the closed subscheme defined by a sheaf of locally principal
ideals. Let $(\mathscr{E},q)$, $(\mathscr{E}',q')$ be quadratic spaces
over $X$ with an isometry $\varphi:(\mathscr{E},q)\to
(\mathscr{E}',q')$ defined over $X-Y$. If $(\mathscr{E},q)$ and
$(\mathscr{E}',q')$ are anisotropic over $Y$, then $\varphi$ can be
extended to a unique isometry $(\mathscr{E},q)\to (\mathscr{E}',q')$
over $X$. 
\end{subprop}

\begin{proof}
In view of the asserted uniqueness, the question is local on
$X$. Thus, we may assume that $X$ is affine and the claim follows by
proposition (1.1) of \cite{chap9-key7}. 
\end{proof}

\begin{subexam}\label{chap9-exam3.3}
Let $X=\mathbb{P}^{n}_{\mathbb{R}}$ and let $U\subset X$ be a real
affine subspace of dimension $n$, i.e.,
$U=\mathbb{P}^{n}_{\mathbb{R}}-V(L)$, where $V(L)$ is a real
hyperplane in $\mathbb{P}^{n}_{\mathbb{R}}$. Let $(\mathscr{E},q)$,
$(\mathscr{E}',q')$ be positive definite quadratic bundles over
$\mathbb{P}^{n}_{\mathbb{R}}$. They are positive definite over $V(L)$,
hence anisotropic. Thus, any isometry $\varphi:(\mathscr{E},q)\to
(\mathscr{E}',q')$ over $U$ extends to a unique isometry over
$\mathbb{P}^{n}_{\mathbb{R}}$. 
\end{subexam}

\setcounter{subsection}{3}
\subsection{}\label{chap9-sec3.4}\pageoriginale

Let $\mathscr{E}$ and $\mathscr{E}'$ be $\mathscr{H}$ modules of rank
one over $\mathbb{P}^{n}_{\mathbb{R}}$, for $\mathscr{H}$ some
quaternion algebra over $\mathbb{P}^{n}_{\mathbb{R}}$. Let $U$ as in
(\ref{chap9-exam3.3}) and let $\varphi:\mathscr{E}\to \mathscr{E}'$ be
an isomorphism of $\mathscr{H}$ modules over $U$. By
(\ref{chap9-coro1.6}) $\mathscr{E}'|U$ and $\mathscr{E}'|_{U}$ are
quadratic spaces of type $\mathscr{H}|_{U}$ and by
(\ref{chap9-coro1.7}) $\varphi$ is an isometry
$\mathscr{E}|_{U}\xrightarrow{\sim}\mathscr{E}'|_{U}$. In view of
(\ref{chap9-exam3.3}) $\varphi$ extends to an isometry of the
normalizations $\mathscr{E}(m)$, resp. $\mathscr{E}'(m')$ of
$\mathscr{E}$ and $\mathscr{E}'$. This isometry is an isomorphism of
$\mathscr{H}$-modules. Thus, we see that if
$\mathscr{E}|_{U}\xrightarrow{\sim}\mathscr{E}'|_{U'}$ then
$\mathscr{E}(k)\xrightarrow{\sim}\mathscr{E}'$ for some
$k\in \mathbb{Z}$. 

\section{Bundles over the affine and the projective
plane}\label{chap9-sec4}

\subsection{}\label{chap9-sec4.1}

Let $\mathbb{P}^{2}_{\mathbb{R}}$ be the real projective plane with
coordinates $X_{1}$, $X_{2}$, $X_{3}$ and let
$\mathbb{A}^{2}_{\mathbb{R}}\subset \mathbb{P}^{2}_{\mathbb{R}}$ be
the affine plane $\mathbb{P}^{2}_{\mathbb{R}}-V(X_{3})$, with affine
coordinates $x=X_{1}/X_{3}$, $y=X_{2}/X_{3}$. We use the notation
$A=\mathbb{R}[x,y]$ and $H=H\otimes_{\mathbb{R}}A$. Let $P$ be a
finitely generated projective (right) $H$-module of rank $n$. Since
the ring $\mathbb{H}(x)[y]=\mathbb{H}\otimes_{\mathbb{R}}R(x)[y]$. is
a P.I.D., the module $P$ becomes a free module over
$\mathbb{H}(x)[y]$. Thus, there is a monic polynomial
$f\in \mathbb{R}[x]$ such that
$P\otimes_{A}A[1/f]=H^{n}\otimes_{A}A[1/f]$. Let $f_{X_{3}}$ be the
$X_{3}$-homogenization of $f$ preserving the degree of $f$. We extend
$P$ to $U=D(X_{3})\cup (D(X_{1})\cap D(f_{X_{3}}))$ by taking the free
$H$-module $H^{n}$ over $D(X_{1})\cap D(f_{X_{3}})$ and glueing with an
isomorphism $P\otimes_{A}A[1/f]\simeq H^{n}\otimes_{A}A[1/f]$ over
$D(X_{3})\cap D(X_{1})\cap
D(f_{X_{3}})=\Spec \mathbb{R}[x,y,x^{-1},f^{-1}]$. We thus have an
extension $\widetilde{P}$ of $P$ to
$U=\mathbb{P}^{2}_{\mathbb{R}}-(0,1,0)$. The direct image
$i\ast \widetilde{P}$ of $\widetilde{P}$ over
$\mathbb{P}^{2}_{\mathbb{R}}$ is a coherent sheaf and
by \cite[p.~110]{chap9-key2} is reflextive. Since a finitely generated
reflexive module over a regular local ring of dimension two is free,
the sheaf $i\ast \widetilde{P}$ is locally free. Thus we have a bundle
$\mathscr{E}=i\ast \widetilde{P}$ which extends $P$ to
$\mathbb{P}^{2}_{\mathbb{R}}$. 

\setcounter{subprop}{1}
\begin{subthm}\label{chap9-thm4.2}
Let\pageoriginale $\mathscr{H}_{0}$ be the constant sheaf of
quaternion algebras over $\mathbb{P}^{2}_{\mathbb{R}}$. Let
$\mathbb{A}^{2}_{\mathbb{R}}\subset \mathbb{P}^{2}_{\mathbb{R}}$ be
any real affine plane contained in $\mathbb{P}^{2}_{\mathbb{R}}$. Then
the embedding
$i:\mathbb{A}^{2}_{\mathbb{R}}\to \mathbb{P}^{2}_{\mathbb{R}}$ induces
a one-to-one correspondence between isomorphism classes of normalized
$\mathscr{H}_{0}$-bundles of rank one over
$\mathbb{P}^{2}_{\mathbb{R}}$ and isomorphism classes of projective
$H$-modules of rank one, where $H=\mathscr{H}_{0}(U)$.
\end{subthm}

\begin{proof}
By a linear change of coordinates, we may assume that
$\mathbb{A}^{2}_{\mathbb{R}}=\mathbb{P}^{2}_{\mathbb{R}}-V(X_{3})$. Then
the construction described in (\ref{chap9-rem4.3}) shows that any
projective $H$-module is the restriction of some
$\mathscr{H}_{0}$-module. The claim now follows from (\ref{chap9-sec3.4}). 
\end{proof}

\begin{subremark}\label{chap9-rem4.3}
A projective $\mathbb{H}[x,y]$-module $P$ gives a module over an
affine plane $\mathbb{A}^{2}_{\mathbb{R}}$. Its extension to
$\mathbb{P}^{2}_{\mathbb{R}}$ will in general depend on the particular
choice of the embedding
$\mathbb{A}^{2}\hookrightarrow \mathbb{P}^{2}$. Fixing the embedding
as $\mathbb{A}^{2}=\mathbb{P}^{2}-V(X_{3})$, we denote by
$\mathscr{E}(P)$ the normalized bundle which extends the projective
$\mathbb{H}[x,y]$-module $P$ of rank one. 
\end{subremark}

\setcounter{subsection}{3}
\subsection{The examples of Ojanguren-Sridharan}\label{chap9-sec4.4}

A construction of nonfree projective $D[x,y]$-modules of rank one, $D$
any division ring, was described in \cite{chap9-key9}. For
$D=\mathbb{H}$, the construction runs as follows: Let $f$,
$g\in \mathbb{R}[x,y]$ and let
$\varphi:\mathbb{H}[x,y]^{2}\to \mathbb{H}[x,y]$ be the
$\mathbb{H}[x,y]$-linear map given by $\varphi((1,0))=f+i$ and
$\varphi((0,1))=g+j$. Then the sequence 
$$
0\to P_{f,g}\to H^{2}\xrightarrow{\varphi}H\to 0
$$
with $P_{f,g}=\ker \varphi$ and $H=\mathbb{H}[x,y]$ is exact and
splits. Thus, $P_{f,g}$ is a projective $H$-module of rank one. It is
shown in \cite{chap9-key13} by explicit computations that the modules
$P_{x,y^{m}}$ are mutually\pageoriginale nonisomorphic. In contrast,
projective $\mathbb{H}[x,y]$-modules of rank $>1$ are free
(see \cite{chap9-key5} or for more general
results \cite{chap9-key15}). Thus, we can identify projective
$\mathbb{H}[x,y]$-modules or rank one with projective ideals of
$\mathbb{H}[x,y]$. 

\subsection{}\label{chap9-sec4.5}
The correspondence given by (\ref{chap9-thm4.2}) has application in
both directions. In one direction, we deduce results about projective
$\mathbb{H}[x,y]$-modules of rank one from results about stable
bundles over $\mathbb{P}^{2}_{\mathbb{C}}$. In the other, known
examples of projective $\mathbb{H}[x,y]$-modules give examples of
bundles over $\mathbb{P}^{2}_{\mathbb{C}}$. A first application is to
define the second Chern class of a projective $\mathbb{H}[x,y]$-module
$P$ of rank one as the second Chern class of the complex 2-bundle
$\mathscr{E}(P)_{c}$. With this definition, we deduce from
(\ref{chap9-sec2.9}) that the moduli space of projective
$\mathbb{H}[x,y]$-modules of rank one with fixed second Chern class
$k$ is a real manifold of dimension $4k-3$. Another application is to
define the curve $C(P)$ of jump lines of $P$ as the curve of jump
lines of $\mathscr{E}(P)_{c}$. 

Since the restriction of $\mathscr{E}(P)_{c}$ to any real line is
trivial, the curve $C(P)$ does not have any real closed points. Thus
its degree must be even. Since, by a result of
Barth \cite{chap9-key11}, the degree of the curve of jump lines is the
second Chern class, we see that the second Chern class of a projective
$\mathbb{H}[x,y]$-module of rank one is even. This integer is computed
for some examples in section \ref{chap9-sec7} of this paper. 

\section{Galois cohomology}\label{chap9-sec5}

\subsection{}\label{chap9-sec5.1}

Let
$\varphi:\mathbb{C}\otimes_{\mathbb{R}}\mathbb{H}\xrightarrow{\sim}M_{2}(\mathbb{C})$
be the fixed isomorphism of $\mathbb{C}$-algebras 
$$
s\otimes (u+vj)\to s
\begin{pmatrix}
u & v\\
-\overline{v} & \overline{u}
\end{pmatrix},\quad u, \ v\in \mathbb{C}.
$$\pageoriginale
Let $R=\mathbb{R}[x,y]$, $C=\mathbb{C}\otimes_{\mathbb{R}}R$ and
$H=H\otimes_{\mathbb{R}}R$. For any (right) projective ideal $P$ of
$H$, $\varphi(\mathbb{C}\otimes_{\mathbb{R}}P)$ is a projective ideal
of $M_{2}(C)$ of rank one, hence free. We choose $\gamma\in M_{2}(C)$
such that
\setcounter{equation}{1}
\begin{equation}
\varphi(\mathbb{C}\otimes_{\mathbb{R}}P)=\gamma\cdot M_{2}(C).\label{chap9-eq5.2}
\end{equation}
Let $\tau\otimes
1:\mathbb{C}\otimes_{\mathbb{R}}H\to \mathbb{C}\otimes_{\mathbb{R}}H$
the conjugation map on $\mathbb{C}$ and let
$\sigma=\varphi\circ \tau\otimes 1\circ \varphi^{-1}$ its transport on
$M_{2}(C)$ through $\varphi$. The element $\sigma(\gamma)$ is also a
generator of $\varphi(\mathbb{C}\otimes_{\mathbb{R}}P)$ and we have
$\sigma(\gamma)\alpha=\gamma$ for some $\alpha\in GL_{2}(C)$. The
element $\alpha$ satisfies the idenity $\sigma(\alpha)=\alpha^{-1}$,
i.e., $\alpha$ is a {\em 1-cocycle.} Two 1-cocycles $\sigma$ and
$\beta$ are cohomologous if $\sigma(\nu)\alpha=\beta\nu$ for some
$\nu\in GL_{2}(C)$. They correspond to isomorphic ideals.

\subsection{}\label{chap9-sec5.2}
It follows from $\sigma(\alpha)=\alpha^{-1}$ that
$\det\alpha=\overline{\det\alpha^{-1}}$. Thus, by Hilbert 90, there is
$\rho\in \mathbb{C}$ such that
$\det \alpha=\rho\overline{\rho}^{-1}$. Replacing $\alpha$ by the
cohomologous cocycle 
$$
\begin{pmatrix}
\overline{\rho} & 0\\
0 & 1
\end{pmatrix}
\alpha
\begin{pmatrix}
1 & 0\\
0 & \rho^{-1}
\end{pmatrix},
$$
we may assume that $\det\alpha=1$. We call such a cocycle {\em
normalized}. An explicit computation shows that 
$$
\sigma
\begin{pmatrix}
a & b\\
c & d
\end{pmatrix}
=
\begin{pmatrix}
\overline{d} & -\overline{c}\\
-\overline{b} & \overline{a}
\end{pmatrix},\quad a,b,c,d,\in C.
$$
Thus, if $\det \alpha=1$, the cocycle condition
$\sigma(\alpha)=\alpha^{-1}$ reduces to the condition
$\overline{\alpha}^{t}=\alpha$, that is $\alpha$ is a $2\times
2$-hermitian matrix. 

\setcounter{subprop}{2}
\begin{subexam}\label{chap9-exam5.3}
Let\pageoriginale $P_{f,g}$, $f$, $g\in \mathbb{R}[x,y]$, be a projective ideal of
$\mathbb{H}[x,y]$ as constructed in (\ref{chap9-rem4.3}). A generator
$\gamma_{f,g}$ for $\varphi(\mathbb{C}\otimes_{\mathbb{R}}P_{f,g})$
was computed by Parimala in \cite{chap9-key11} and \cite{chap9-key13}
(see also the computation in \cite{chap9-key4}):
$$
\gamma_{f,g}=
\begin{pmatrix}
1+g^{2} & g(f-i)\\
g(1+g^{2}) & g^{2}(f-i)-2i
\end{pmatrix}
$$
The corresponding hermitian $2\times 2$-matrix is
$$
\alpha_{f,g}=
\begin{pmatrix}
1+f^{2}g^{4} & -fg(1+g^{2})+ig(1+f^{2}g^{2})\\
-fg(1+g^{2})-ig(1+f^{2}g^{2}) & 4+g^{2}(1+f^{2})
\end{pmatrix}
$$
We remark that $\gamma_{f_{1},g}$ with $f_{1}\equiv
f\,\text{mod}\,(g^{2}+1)$ in $\mathbb{R}[x,y]$, is also a generator of
$\varphi\mathbb{C}\otimes_{\mathbb{R}}P_{f,g}$). This allows us
sometimes to simplify computations.
\end{subexam}

\setcounter{subsection}{3}
\subsection{}\label{chap9-sec5.4}
Let $\alpha\in GL_{2}(\mathbb{C}[x,y])$ be a hermitian matrix. We
identify $\mathbb{C}[x,y]$ with $\mathbb{R}[x,y]^{2}$ as
$\mathbb{R}[x,y]$-module and define a quadratic form $q$ on
$\mathbb{R}[x,y]^{4}=\mathbb{C}[x,y]^{2}$ by
$$
q(\xi)=\overline{\xi}^{\overline{t}}\alpha\xi,\quad \xi=
\binom{\xi_{1}}{\xi_{2}}\in \mathbb{C}[x,y]^{2}.
$$
If\pageoriginale $\alpha=a+ib$ is the decomposition of $a$ into real
and imaginary parts, the form $q$ is given by the real symmetric
$4\times 4$-matrix $S=\left(\begin{smallmatrix} a & b\\ b^{t} &
a\end{smallmatrix}\right)$. Thus, the matrix $S$ defines a quadratic
bundle of rank 4 over $\mathbb{A}^{2}_{\mathbb{R}}$. 

\setcounter{subsection}{4}
\subsection{}\label{chap9-sec5.5}

Let $P$ be a projective ideal of $\mathbb{H}[x,y]$ and let $\alpha$ be
a normalized cocycle corresponding to $P$. By Galois descent we have
$\varphi(P)\simeq \{\gamma \eta|\sigma(\gamma\eta)=\gamma\eta,\eta\in
M_{2}(C)\}$. $\gamma$ is a generator of
$\varphi(\mathbb{C}\otimes_{\mathbb{R}}P)$ such that
$\sigma(\sigma)\alpha=\gamma$. Thus
$\varphi(P)\simeq \{\gamma\eta\mid\sigma(\eta)=\alpha\eta\}$ and $P$
is isomorphic to $Q=\{\eta\in M_{2}(C)\mid\sigma(n)=\alpha\eta\}$,
where $\mathbb{H}[x,y]$ acts on $Q$ (as a right module) through
$\varphi$. Let
$$
f(\eta)=\overline{\eta}^{t}\alpha\eta,\quad \eta\in Q.
$$
We have
$f(\eta)=\overline{\eta}^{t}\sigma(\eta)=\det(\overline{\eta})=\det(\eta)$
since
$\det(\eta)=\det(\sigma(\eta))=\det(\alpha\eta)=\det(\bar{\eta})$. From
this, it is easy to check that $f$ is a quadratic form of type
$\mathbb{H}[x,y]$ on $Q$. Thus, we know by (\ref{chap9-coro1.6}) that
the quadratic form on $\mathbb{R}[x,y]^{4}$ given by a normalized
cocycle $\alpha$ of $P$ is isometric to the norm of $P$. 

\section{Transition maps}\label{chap9-sec6}

\subsection{}\label{chap9-sec6.1}

Let $X_{1}$, $X_{2}$, $X_{3}$ be the variables of
$\mathbb{P}^{2}_{\mathbb{R}}$ and let $\cup U_{i}=\mathbb{P}^{2}$ be
the covering given by $U_{1}=D(X_{i})=\Spec R_{i}$, where
$R_{i}=\mathbb{R}[X_{1},X_{i},X_{2}/X_{i},X_{3}/X_{i}]$ is a
polynomial ring in two variables over $\mathbb{R}$. Further, we
introduce the notations $H_{i}=\mathbb{H}\otimes_{\mathbb{R}}R_{i}$
and $C_{i}=\mathbb{C}\otimes_{\mathbb{R}}R_{i}$. For any module
$\mathscr{E}$ of rank one over the constant sheaf $\mathscr{H}_{0}$ of
quaternion algebras over $\mathbb{P}^{2}_{\mathbb{R}}$, the
restriction of $\mathscr{E}$ to $U_{i}$ defines a projective
$H_{i}$-module of rank one which is isomorphic to an ideal $P_{i}$ of
$H_{i}$. Let, as in \eqref{chap9-eq5.2}, $\gamma_{i}$ be\pageoriginale
a generator for $\varphi(C\otimes_{\mathbb{R}}P_{i})$ in
$M_{2}(C_{i})$. Restricting to $U_{i}\cap U_{j}$, we can choose the
image of $\gamma_{i}$ or $\gamma_{j}$ as generator. Thus we have
\setcounter{equation}{1}
\begin{equation}
\gamma_{j}=\gamma_{i}t_{ij},\quad t_{ij}\in GL_{2}(C_{ij})\label{chap9-eq6.2}
\end{equation}
where $U_{i}\cap U_{j}=\Spec R_{ij}$ and
$C_{ij}=\mathbb{C}\otimes_{\mathbb{R}}R_{ij}$. The family $\{t_{ij}\}$
is a family of transition maps for a complex 2-bundle
$\mathscr{F}_{c}$ over $\mathbb{P}^{2}_{\mathbb{C}}$. We claim that
$\mathscr{F}\simeq \mathscr{E}_{c}(\ell)$ for some
$\ell\in \mathbb{Z}$. For this we may first assume that the
$\gamma_{i}$ are such that the corresponding 1-cocyles $\alpha_{i}$
are hermitian $2\times 2$-matrices with determinant one. Further, we
may assume that the quadratic forms $q_{i}$ associated with the
$a_{i}$ as in (\ref{chap9-sec5.4}) are positive definite. We deduce
from \eqref{chap9-eq6.2} and the relation
$\sigma(\gamma_{i})a_{i}=\gamma_{i}$ that
$$
\det\overline{(t_{ij})}\alpha_{j}=\overline{t_{ij}}^{t}\alpha_{i}t_{ij}
$$
Since the $\alpha_{i}$ are positive definite, we have
$\det(\overline{t_{ij}})=\lambda^{2}_{ij}\in R^{X}_{ij}$ as in
(\ref{chap9-sec2.1}) and
\setcounter{equation}{2}
\begin{equation}
\alpha_{j}=\overline{u_{ij}}^{t}\alpha_{i}u_{ij}\label{chap9-eq6.3}
\end{equation}
with $u_{ij}=t_{ij}\cdot \lambda^{-1}_{ij}$. The $u_{ij}$ are
transition maps of a bundle $\mathscr{F}(n)$. The
relations \eqref{chap9-eq6.3} show that this bundle is a quadratic
bundle with quadratic form $q_{i}$ on $U_{i}$. Since $q_{i}$ is of
type $\mathscr{H}_{i}$ on $U_{i}$, we see that $F(n)$ is of type
$\mathscr{H}_{0}$ on $\mathbb{P}^{2}_{\mathbb{R}}$. The bundle
$\mathscr{E}(m)$ also is of type $\mathscr{H}_{0}$ for some
$m\in \mathbb{Z}$ (see (\ref{chap9-prop2.2})). Since the restrictions
of the quadratic bundles $\mathscr{F}(n)$ and $\mathscr{E}(m)$ to
$U_{i}$ are isometric, we conclude by (\ref{chap9-sec3.4}) that
$\mathscr{F}\simeq \mathscr{E}_{c}(\ell)$ for some $\ell$ as claimed. 

\setcounter{subprop}{3}
\begin{subremark}\label{chap9-rem6.4}
Assume\pageoriginale that the bundle $\mathscr{E}$ in
(\ref{chap9-sec6.1}) is normalised. Then $\mathscr{E}$ is isomorphic
to the bundle $\mathscr{F}(n)$ with the transition maps $u_{ij}$. The
relations \eqref{chap9-eq6.3} show that the quadratic structure of
$\mathscr{E}$ (or $\mathscr{F}(n)$) is induced by a {\em hermitian
structure} on $\mathscr{E}_{c}$ with respect to the real structure of
$\mathbb{P}^{2}_{\mathbb{C}}$. Such a structure (which is called
$\sigma$-{\em hermitian} in \cite{chap9-key6}, \cite{chap9-key7}
or \cite{chap9-key10}) is given on a complex bundle $\mathscr{G}$ by
an isomorphism $\varphi:\mathscr{G}\to \sigma^{*}\mathscr{G}^{*}$ such
that $(\sigma^{*}\varphi)^{t}=\varphi$. It can be shown, that for any
real scheme $X$, the quadratic structure of a bundle $\mathscr{E}$ of
type $\mathscr{H}_{0}$ is induced by a hermitian structure on
$\mathscr{E}_{c}$. 
\end{subremark}

\begin{subexam}\label{chap9-exam6.5}
Let $P_{x,y}$ be the projective ideal of $\mathbb{H}[x,y]$ given by
the Ojanguren-Sridharan construction for $f=x$ and $g=y$ and let
$\mathscr{E}_{x,y}=\mathscr{E}(P_{x,y})$ be the normalized extension
of $P_{x,y}$ to $\mathbb{P}^{2}_{\mathbb{R}}$ given by
(\ref{chap9-thm4.2}) (for the identification
$\mathbb{A}^{2}_{\mathbb{R}}=\mathbb{P}^{2}_{\mathbb{R}}-V(X_{3})$). The
restriction of $\mathscr{E}_{x,y}$ to $U_{3}$ has
$\gamma_{3}=\gamma_{x,y}$ (see \ref{chap9-exam5.3}) as generator,
where $x=X_{1}/X_{3}$ and $y=X_{2}/X_{3}$. Generators for the
restrictions $\mathscr{E}_{1}$ and $\mathscr{E}_{2}$ on $U_{1}$ and
$U_{2}$ were computed in \cite{chap9-key4}: we have
\begin{align*}
\gamma_{1} &=
\begin{pmatrix}
2y^{2}-iz(y^{2}+1) & -3yz-iy(2-z^{2})\\
-iy(1-y^{2}) & y^{2}+iz(2-y^{2})
\end{pmatrix}\\
\text{where}\quad y &= \frac{X_{2}}{X_{1}}\quad\text{and}\quad
z=\dfrac{X_{3}}{X_{1}}\\
\gamma_{2} &= 
\begin{pmatrix}
1-ixz & -x+iz\\
z(1+ixz) & -i(z^{2}+2)+zx
\end{pmatrix}\\
\text{where}\quad x &= \frac{X_{1}}{X_{2}}\quad\text{and}\quad
z=\dfrac{X_{3}}{X_{2}}. 
\end{align*}
\end{subexam}

\section{Chern Classes}\label{chap9-sec7}\pageoriginale

\begin{subprop}\label{chap9-prop7.1}
Let $\mathscr{E}_{x,y}$ be as in (\ref{chap9-rem6.4}). We have
$c_{1}(\mathscr{E}_{x,y})=0$ and $c_{2}(\mathscr{E}_{x,y})=2$. 
\end{subprop}

\begin{proof}
Let $E_{i}=\mathscr{E}_{x,y}|_{U_{i}}$ and let $\gamma_{i}$ be the
generator of $E_{i}$ given in (\ref{chap9-exam6.5}). Let $\mathscr{F}$
be the complex 2-bundle given by the transition maps $t_{ij}$
satisfying $\gamma_{j}\simeq \gamma_{i}t_{ij}$. We have
$c_{1}(\mathscr{F})=-2$, thus
$\mathscr{E}_{x,y}\simeq \mathscr{F}(1)$. Writing the $2\times
2$-matrix $\gamma_{i}$ as $\left(\begin{smallmatrix} a_{i} & b_{i}\\
c_{i} & d_{i}\end{smallmatrix}\right)$, we check that
$$
t_{ij}
\binom{d_{j}}{c_{j}}=
\binom{d_{i}}{-c_{i}}
\left(\dfrac{X_{i}}{X_{j}}\right)^{2}
$$
Thus $\binom{d_{i}}{-c_{i}}$ is a global section of
$\mathscr{G}=\mathscr{F}(2)$. The zeros of this section are $(1,0,0)$
and $(1,\pm 1,i)$ with multiplicity one. Thus we have
$c_{2}(\mathscr{G})=3$. It follows from the formula
$c_{2}(\mathscr{G}(m))=c_{2}(\mathscr{G})+mc_{1}(\mathscr{G})+m^{2}$,
that $c_{2}(\mathscr{E}_{x,y})=c_{2}(\mathscr{G}(-1))=3-2+1=2$ as
claimed. 
\end{proof}

\begin{subthm}\label{chap9-thm7.2}
Let $f$, $g\in R[x,y]$ be algebraically independent. Then we have
$c_{2}(\mathscr{E}_{f,g'})=2\cdot [\mathbb{C}(x,y):\mathbb{C}(f,g)]$. 
\end{subthm}

\begin{proof}
Let $f'(x,y)=f(x+y^{r},y)$, $g'(x,y)=g(x+y^{r},y)$. It can be shown
that $c_{2}(\mathscr{E}_{f',g'})=c_{2}(\mathscr{E}_{f,g})$. Thus, by
choosing $r$ big enough, we may assume that $f$ and $g$ have as terms
of higher degree monomials $y^{n}$ and $y^{m}$. Assume that $n\geq m$.

Let $x=X_{1}/X_{3}$, $y=X_{2}/X_{3}$ and let $F$, resp.~$G$ be the
$X_{3}$-homogeni\-zation of $f$, resp.~$g$. We define a rational map
$\Psi:\mathbb{P}^{2}_{\mathbb{R}}\to \mathbb{P}^{2}_{\mathbb{R}}$
which\pageoriginale extends the map $\varphi:\mathbb{A}^{2}_{\mathbb{R}}\to \mathbb{A}^{2}_{\mathbb{R}}$ given by $x\to f$, $y\to g$ by $\Psi(X_{1})=F$, $\Psi(X_{2})=GX^{n-m}_{3}$, $\Psi(X_{3})=X^{n}_{3}$. The map $\Psi$ is not defined at $(1,0,0)$. By a sequence of blow-ups at real closed points we construct a resolution of $\Psi$:
\[
\xymatrix@R=1.5cm{
 & Y\ar[dl]_{\pi}\ar[dr]^{\phi} &\\
\mathbb{P}^{2}_{\mathbb{R}}\ar[rr]_-{\Psi} && \mathbb{P}^{2}_{\mathbb{R}}
}
\]
Since $P_{f,g}=\varphi^{*}P_{x,y}$, we have $\pi^{*}\mathscr{E}_{f,g}\simeq \phi^{*}\mathscr{E}_{x,y}$ outside of 
$$
Z=\pi^{-1}(\{(1,0,0)\})\subset Y.
$$ 
The set $Z$ is a set of real lines. Therefore $\pi^{*}\mathscr{E}_{f,g}$ and $\phi^{*}\mathscr{E}_{x,y}$ are anisotropic over $Z$. In view of (\ref{chap9-prop3.2}), the two bundles $\pi^{*}\mathscr{E}_{f,g}$ and $\phi^{*}\mathscr{E}_{x,y}$ are isomorphic over $Y$. Thus we have $c_{2}(\mathscr{E}_{f,g})=c_{2}(\pi^{*}\mathscr{E}_{f,g})=c_{2}(\phi^{*}\mathscr{E}_{x,y})$. $\deg\phi=2$. $[\mathbb{C}(x,y):\mathbb{C}(f,g)]$. 
\end{proof}

\section{Curves of jump lines}\label{chap9-sec8}

\subsection{}\label{chap9-sec8.1}
Let $\mathscr{E}$ be a stable 2-bundle over
$\mathbb{P}^{2}_{\mathbb{C}}$ with $c_{1}(\mathscr{E})=0$ and let
$C(\mathscr{E})$ be the curve of jump lines of
$\mathscr{E}$. By \cite{chap9-key1} we know that the degree of
$C(\mathscr{E})$ is equal to $c_{2}(\mathscr{E})$. We compute the
equation of the curve of jump lines for the following classes of
bundles $\mathscr{E}_{f,g}$: 
\begin{itemize}
\item[(1)] $g=y$

\item[(2)] $f\in \mathbb{R}[x,y]$ is monic and has degree $n$ as a
polynomial in $x$. By the last remark of (\ref{chap9-exam5.3}) we may
assume that $f$ is linear in $y$.
\end{itemize}

By (\ref{chap9-thm7.2}), we know that the degree of
$C_{f,g}=C(\mathscr{E}_{f,g})$ is $2n$.

\subsection{}\label{chap9-sec8.2}\pageoriginale
Let $E_{3}$ be the restriction of $\mathscr{E}_{f,g}$ to
$U_{3}=D(X_{3})$. A generator $\gamma_{3}$ for
$\mathbb{C}\otimes_{\mathbb{R}}E_{3}$ is given in
(\ref{chap9-exam5.3}) using affine coordinates $x=X_{1}/X_{3}$ and
$y=X_{2}/X_{3}$. We have
$\det \gamma_{3}=(X^{2}_{2}+X^{2}_{3})/X^{2}_{3}$. Let
$V_{2}=D(X_{2})\cap D(X^{2}_{2}+X^{2}_{3})$ and let $H_{2}$ be the
restriction of the constant sheaf $\mathscr{H}_{0}$ to $V_{2}$. Since
$\gamma_{3}$ invertible on $U_{3}\cap V_{2}$, $\gamma_{3}$ induces an
isomorphism $E_{3}|_{U_{3}\cap
V_{2}}\xrightarrow{\sim}H_{2}|_{U_{3}\cap V_{2}}$. Thus we can extend
$E_{3}$ to a bundle $\mathscr{F}$ over $U=U_{3}\cap V_{2}$ by taking
$H_{2}$ on $V_{2}$ and glueing with $\varphi_{3}$ over $U_{3}\cap
V_{2}$. Since $\mathbb{P}^{2}-U=\{(1,0,0)\}$, the bundle $\mathscr{F}$
on $U$ has a unique extension to $\mathbb{P}^{2}$, also denoted by
$\mathscr{F}$. Since
$\mathscr{F}|_{U_{3}}\simeq \mathscr{E}|_{U_{3}}$, the bundles
$\mathscr{F}$ and $\mathscr{E}$ only differ by a twist on
$\mathbb{P}^{2}$. Hence they have the same curves of jump lines.

\subsection{}\label{chap9-sec8.3}
The equation $X_{1}=uX_{2}+vX_{3}$, $u$, $v\in \mathbb{R}$ gives a
real line $L$ contained in $U=U_{3}\cup V_{2}$ (see
(\ref{chap9-sec8.2})). We compute the lines $L$ which are jump lines
for $\mathscr{F}$. Using affine coordinates $x=X_{1}/X_{3}$ and
$y=X_{2}/X_{3}$ and substituting $x=uy+v$ in $\gamma_{3}$, we obtain a
transition map
$$
\Gamma=
\begin{pmatrix}
1+y^{2} & y(f_{1}-i)\\
y(1+y^{2}) & y^{2}(f_{1}-i)-2i
\end{pmatrix}
$$
for $\mathscr{F}|_{L}$ on $U_{3}\cap L\cup U_{2}\cap L$, where
$f_{1}(y)=f(uy+v,y)$. Since $U_{3}\cap L=\Spec \mathbb{R}[y]$ and
$V_{2}\cap L=\Spec \mathbb{R}[y^{-1},(1+y^{-2})^{-1}]$, we see that we
have to diagonalize $\Gamma$, operating on the right over
$\mathbb{R}[y]$ and on the lest over
$\mathbb{R}[y^{-1},(1+y^{-2})^{-2}]$, to determine the type of
$\mathscr{F}|_{L}$. In particular we can reduce $f_{1}$ modulo
$1+y^{2}$ by column operations. Thus we may assume that
$f_{1}=A(u,v)y+B(u,v)$\pageoriginale is linear in $y$. By further row
and column operations we check that a pair $(u,v)$ gives a jump line
if and only if $A(u,v)^{2}+B(u,v)^{2}+1=0$. Under the given hypothesis
for $f$, this is an equation of degree $2n$, which by
(\ref{chap9-thm7.2}) is the degree of the curve of jump lines. Thus we
have the affine part of the curve. Homogenizing, we obtain the full
curve of jump lines. 

\setcounter{subprop}{3}
\begin{subexam}\label{chap9-exam8.4}
Let $f=x$, then $f_{1}=uy+v$ and the equation of the curve of jump
lines of $\mathscr{E}_{x,y}$ is 
$$
U^{2}+V^{2}+W^{2}=0.
$$
\end{subexam}

\section{Bundles with \texorpdfstring{$c_{2}=2$}{c22} and \texorpdfstring{$c_{2}=4$}{c24}}\label{chap9-sec9}

\subsection{}\label{chap9-sec9.1}

Let $\mathscr{H}_{0}$ be the constant sheaf of quaternion algebras
over $\mathbb{P}^{2}_{\mathbb{R}}$ and let $\mathscr{E}$ be a
normalised $\mathscr{H}_{0}$-module of rank one with
$c_{2}(\mathscr{E})=2$. We know that the curve $C(\mathscr{E})$ of
jump lines of $\mathscr{E}$ is a real conic without real points. Such
a conic is induced by the conic (\ref{chap9-exam8.4}) through a real
projective transformation $\rho\in PGL_{3}(\mathbb{R})$. By a result
of Barth \cite{chap9-key1}, 2-bundles over
$\mathbb{P}^{2}_{\mathbb{C}}$ with $c_{1}=0$ and $c_{2}=2$ are
isomorphic if and only if they have the same curve of jump lines. Thus
$\rho^{*}\mathscr{E}_{x,y}$ and $\mathscr{E}$ are isomorphic as
$\mathscr{H}_{0}$-modules. Conversely, for any $\rho\in
PGL_{3}(\mathbb{R})$, $\rho^{*}\mathscr{E}_{x,y}$ is a normalized
$\mathscr{H}_{0}$-module. Thus the set of isomorphism classes of
$\mathscr{H}_{0}$-modules of rank one with $c_{1}=0$ and $c_{2}=2$ is
in bijection with the orbit of the conic $V(U^{2}+V^{2}+W^{2})$ under
the action of group $PGL_{3}(\mathbb{R})$. This gives an explicit
description of the 5-dimensional moduli space of these bundles (see
(\ref{chap9-sec2.9})). 

\subsection{}\label{chap9-sec9.2}

Let $\mathscr{E}$ be as in (\ref{chap9-sec9.1}) and let $\rho\in
PGL_{3}(\mathbb{R})$ such that
$\rho^{*}\mathscr{E}_{x,y}\simeq \mathscr{E}$.\pageoriginale By
modifying $\rho$ by an element in the isotropy group of
$V(U^{2}+V^{2}+W^{2})$, we may assume that $\rho$ fixes the line at
infinity $V(X_{3})$. Then we have
$\rho^{*}\mathscr{E}_{x,y}=\mathscr{E}_{f,g}$ with $f=ax+by+c$,
$g=a'x+b'y+c'$, both linear. The bundles $\mathscr{E}_{f,g}$ and
$\mathscr{E}_{x,y}$ are isomorphic if and only if $c=c'=0$ and
$\left(\begin{smallmatrix} a & b\\ a' & b'\end{smallmatrix}\right)$
belongs to the orthogonal group of the quadratic form 
$\left(\begin{smallmatrix} 1 & 0\\ 0 & 1\end{smallmatrix}\right)$. By
a similar argument, we see that the one-to-one correspondence between
normalized $\mathscr{H}_{0}$-modules of rank one over
$\mathbb{A}^{2}_{\mathbb{R}}$ and $\mathbb{P}^{2}_{\mathbb{R}}$ does
not depend on the embedding $\mathbb{A}\to \mathbb{P}^{2}$ for bundles
with $c_{2}=2$.

\setcounter{subprop}{2}
\begin{subexam}\label{chap9-exam9.3}
As noticed by parimala, we can replace $xy$ by $x$ in $\gamma_{x,y}$
(see (\ref{chap9-exam5.3})) and obtain a new matrix
$$
\gamma'=
\begin{pmatrix}
1+y^{2} & x-iy\\
y(1+y^{2}) & y(x-iy)-2i
\end{pmatrix}
$$
with corresponding 1-cocycle
$$
a'=
\begin{pmatrix}
1+x^{2}y^{2} & -x(1+y^{2})+iy(1+x^{2})\\
-x(1+y^{2})-iy(1+x^{2}) & 4+x^{2}+y^{2}
\end{pmatrix}
$$
This cocycle belongs to some projective ideal $P$ of $\mathbb{H}[x,y]$
with reduced norm given by the symmetric $4\times 4$-matrix.
$$
S=
\begin{pmatrix}
1+x^{2}y^{2} & -x(1+y^{2}) & 0 & y(1+x^{2})\\
-x(1+y^{2}) & 4+x^{2}+y^{2} & -y(1+x^{2}) & 0\\
0 & -y(1+x^{2}) & 1+x^{2}y^{2} & -x(1+y^{2})\\
y(1+x^{2}) & 0 & -x(1+y^{2}) & 4+x^{2}+y^{2}
\end{pmatrix}
$$\pageoriginale
The matrix $S$ gives the simplest known example of a nontrivial
quadratic bundle over $\mathbb{A}^{2}_{\mathbb{R}}$. We claim that $S$
is isometric to the norm on $P_{x,y}$ or that $P\simeq P_{x,y}$. Let
$\varphi:\mathbb{A}^{2}\to \mathbb{A}^{2}$ be the quadratic
transformation given by $x\to xy$, $y\to y$. We have
$P_{x,y}=\varphi^{*}P$. Since $\varphi$ is a birational map,
$c_{2}(P)=c_{2}(P_{x,y})=2$ and the curve of jump lines of $P$ is a
conic. By computations as in (\ref{chap9-sec8.3}), we obtain that
$C(P)=V(U^{2}+V^{2}+W^{2})$. Thus $P\simeq P_{x,y}$ as claimed.
\end{subexam}

\setcounter{subsection}{3}
\subsection{}\label{chap9-sec9.4}
Let $f\in \mathbb{R}[x,y]$ be monic and of degree 2 as polynomial in
$x$ and let $g=y$. As already noticed we may assume that $f$ is linear
in $y$. In view of (\ref{chap9-thm7.2}) the curve of jump lines of
$\mathscr{E}_{f,g}$ has degree 4.

We obtain by (\ref{chap9-sec8.3}) an affine equation for the curve of
jump lines of the form
$$
(u^{2}+v^{2})^{2}+p(u,v)=0
$$
where $p$ is of degree $3$. For example, $f=x^{2}+a$ gives the
equation
$$
(u^{2}+v^{2})^{2}+2a(u^{2}-v^{2})+(a^{2}+1)=0.
$$
These\pageoriginale curves which have the two cyclic points. $(1,\pm
i, 0)$ as double points, are called {\em bicircular quartics}. Since a
quartic with two double points depends on 12 parameters, we see that
it is impossible to obtain all bundles $\mathscr{E}$ with $c_{1}=0$
and $c_{2}=4$ which are $\mathscr{H}_{0}$-modules of rank one, as
translates of some $\mathscr{E}_{f,y}$, $f$ quadratic in $x$, for the
action of the group $PGL_{3}(\mathbb{R})$. 

\section{Bundles over a quadric and extensions to
\texorpdfstring{$\mathbb{P}^{3}$}{P3}}\label{chap9-sec10}

\subsection{}\label{chap9-sec10.1}
Another way to extend the projective $\mathbb{H}[x,y]$-module
$P_{x,y}$ to a bundle over $\mathbb{P}^{2}_{\mathbb{R}}$ is to
homogenize the sequence \ref{chap9-rem4.3}: We put
$H=\mathbb{H}[X_{1},X_{2},X_{3}]$ and define a map $\varphi:H^{2}\to
H$ by $\varphi((1,0))=X_{1}+iX_{3}$ and
$\varphi((0,1))=X_{2}+jX_{3}$. The kernal of the map $\varphi$ is a
graded module over $\mathbb{R}[X_{1},X_{2},X_{3}]$ and the associated
sheaf over $\mathbb{P}^{2}_{\mathbb{R}}$ is a bundle which extends
$P_{x,y}$. 

This suggests the following construction of a bundle over
$\mathbb{P}^{1}_{\mathbb{R}}\times \mathbb{P}^{1}_{\mathbb{R}}$: We
put $H=\mathbb{H}[X_{1},X_{2},U_{1},U_{2}]$ and define
$\varphi:H^{2}\to H$ by $\varphi((1,0))=X_{1}+iX_{2}$,
$\varphi((0,1))=U_{1}+jU_{2}$. The kernel defines a bundle
$\mathscr{E}$ over
$\mathbb{P}^{1}_{\mathbb{R}}\times \mathbb{P}^{1}_{\mathbb{R}}$. Assuming
that $\mathscr{E}$ is normalized, we have $c_{2}(\mathscr{E})=2$, and
the jump lines of $\mathscr{E}$ are the four lines
$X^{2}_{1}+X^{2}_{2}=0$, $U^{2}_{1}+U^{2}_{2}=0$. Let
$\mathbb{Q}\subset \mathbb{P}^{3}$ be the image of
$\mathbb{P}'\times \mathbb{P}^{1}$ by the Segre embedding. It follows
by results of Le Potier \cite{chap9-key8} that $\mathscr{E}$ as a
bundle over $\mathbb{Q}$ is the restriction of two instanton bundles
over $\mathbb{P}^{3}$. We can embed
$\mathbb{P}^{1}\times \mathbb{P}^{1}$ into $\mathbb{P}^{3}$ in such a
way that the nonstandard real structure on $\mathbb{P}^{3}_{C}$
induces the usual real structure on
$\mathbb{P}^{1}\times \mathbb{P}^{1}$. Thus the restriction of an
instanton bundle over $\mathbb{P}^{3}_{\mathbb{C}}$ to
$\mathbb{P}^{1}\times \mathbb{P}^{1}$ is a normalised
$\mathscr{H}_{0}$-bundle of rank one over
$\mathbb{P}^{1}\times \mathbb{P}^{1}$. By a result of Le potier
mentioned above, any normalized $\mathscr{H}_{0}$-bundle over
$\mathbb{P}^{1}\times \mathbb{P}^{1}$ with $c_{2}=2$\pageoriginale is
such a restriction. It would be interesting to know if this is true
for higher values of $c_{2}$.

\subsection{}\label{chap9-sec10.3}

Similarly, one can consider the bundles over $\mathbb{P}^{2}$ which
are restrictions of instanton bundles over $\mathbb{P}^{3}$. By
results of Donaldson \cite{chap9-key3}, these are stable bundles
which are trivial on the line at infinity. Hence
$\mathscr{H}_{0}$-modules of rank one over $\mathbb{P}^{2}$ are such
restrictions, since they are trivial over any real line. But there is,
{\em a priori}, no relation between the quaternionic structures on the
bundles over $\mathbb{P}^{2}$ and over $\mathbb{P}^{3}$. Let
$\pi:\mathbb{A}^{3}_{\mathbb{R}}\to \mathbb{A}^{2}_{\mathbb{R}}$ be
the projection given by $(x,y,z)\to (x,y)$. As communicated to us by
Parimala, nontrivial $\mathscr{H}_{0}$-modules of rank one over
$\mathbb{A}^{3}_{\mathbb{R}}$ of the form $\pi^{*}P$, $P$ an
$\mathscr{H}_{0}$-module over $\mathbb{A}^{2}_{\mathbb{R}}$, cannot be
extended as $\mathscr{H}_{0}$-modules to
$\mathbb{P}^{3}_{\mathbb{R}}$. 

\begin{thebibliography}{}
\bibitem{chap9-key1} Barth, W. : Moduli of vector bundles on the
projective plane, {\em Invent. Math.} 42 (1977), 63-91.


\bibitem{chap9-key2} Colliot-Th\'el\`ene, J:-L.; J.-J. Sansuc :
Fibr\'es quadratiques et composantes connexes r\'eelles, {\em
Math. Ann.} 244 (1979), 105-134.

\bibitem{chap9-key3} Donaldson, S. K. : Instantons and Geometric
Invariant Theory, Preprint.

\bibitem{chap9-key4} Knus, M. -A. : Quaternionic modules over
$\mathbb{P}^{2}_{\mathbb{R}}$, {\em Springer Lecture Notes} 917,
245-259, 1982,

\bibitem{chap9-key5} Knus, M. -A., M. Ojanguren: Modules and quadratic
forms over\pageoriginale polynomial algebras, {\em
Proc. Amer. Math. Soc.} 66 (1977), 223-226.

\bibitem{chap9-key6} Knus, M.-A., M. Ojanguren, R. Parimala: Positive
definite quadratic bundles over the plane, {\em
Comment. Math. Helvetici} 57 (1982), 400-411.

\bibitem{chap9-key7} Knus, M.-A., R. Parimala, R. Sridharan: Nonfree
projective modules over $\mathbb{H}[x,y]$ and stable bundles over
$\mathbb{P}^{2}(C)$, {\em Invent. Math.} 65 (1981), 13-27.
  
\bibitem{chap9-key8} Le Potier, J.: Sur l'espace des modules des
fibres de Yang et Mills, {\em Progress in Math.} 37; Mathematique et
Physique, Ed.: L. Boutet de Monvel, A. Douady, J.-L. Verdier,
Birkhauser, 65-137, 1983.

\bibitem{chap9-key9} Ojanguren, M., R. Sridharan: Cancellation of
Azumaya algebras, {\em J. Algebra} 18 (1971), 501-505.

\bibitem{chap9-key10} Ojanguren, M., R. Parimala, R. Sridharan:
Indecomposable quadratic bundles of rank $4n$ over the real affine
plane, {\em Invent. Math.} 71 (1983), 643-652.

\bibitem{chap9-key11} Parimala, R.: Failure of a quadratic analogue of
Serre's conjecture, {\em Amer. J. Math.} 100 (1978), 913-924.

\bibitem{chap9-key12} Parimala, R.: Indecomposable quadratic spaces
over the affine plane, to appear in {\em Advances in Mathematics.}

\bibitem{chap9-key13} Parimala, R., R. Sridharan: Projective modules
over polynomial\pageoriginale rings over division rings, {\em
J. Math. Kyoto Univ.} 15 (1975), 129-148.

\bibitem{chap9-key14} Scharlau, W.: Remark on symmetric bilinear forms
over euclidean domains, Preprint.

\bibitem{chap9-key15} Stafford, J. T.: Projective modules of
polynomial extensions of division rings, {\em Invent. Math.} 59
(1980), 105-117.
\end{thebibliography}

\vskip 1cm

\noindent
Mathematik, ETH-Zentrum.\\
CH-8092, Zurich.

\newpage

~\phantom{a}
\thispagestyle{empty}
