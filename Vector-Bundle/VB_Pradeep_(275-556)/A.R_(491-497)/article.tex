\title{Unitary Bundles and Restrictions of Stable Sheaves}
\markright{Unitary Bundles and Restrictions of Stable Sheaves}

\author{By A. Ramanathan}
\markboth{A. Ramanathan}{Unitary Bundles and Restrictions of Stable Sheaves}

\date{}
\maketitle


\setcounter{page}{373}
\setcounter{pageoriginal}{490}
\section{Introduction}\pageoriginale

This is a report on joint work with V. B. Mehta. Here we will state the main results and give a sketch of their proofs. Full details will appear in \cite{key4}.

Let $X$ be a smooth projective variety of dimension $n$ over an algebraically closed field $k$. Let $H$ be an ample line bundle on $X$. For any coherent sheaf $\mathscr{F}$ on $X$ let $c_1(\mathscr{F})$ denote its first Chern class. By $\deg \mathscr{F}$ we mean the intersection number $c_1(\mathscr{F})\cdot H^{n-1}$ and by $r k \mathscr{F}$ we mean the rank of the generic fibre of $\mathscr{F}$ over the function field of $X$. Let $V$ be a torsion free sheaf on $X$. Mumford has given the following definition of stability for $V$. For any proper subsheaf $W\subset V$ if we have $\deg \dfrac{W}{rk} V< \deg \dfrac{V}{rk} V$ then $V$ is said to be stable. If, on the other hand, only the weak inequality $\deg \dfrac{W}{rk} W\leq \deg \dfrac{V}{rk} V$ holds, $V$ is called \textit{semistable}. 

For a smooth projective curve $C$ stable bundles and their moduli have been studied in depth by Narasimhan, Ramanan and Seshadri (see \cite{key5}, \cite{key6}).

For\pageoriginale higher dimensional varieties $X$, clearly it would be of some use to know how the restriction of semi-stable or stable bundles on $X$ to suitable subvarieties of $X$ behave. In \cite{key3}, we proved that the restriction of a semistable bundle on $X$ to complete intersection sub-varieties in general position and of suitably high multi-degree remain semistable. The proof was based on an unpublished manuscript of Mumford. In this paper we extend the method to prove that stable bundles on $X$ also remain stable on such subvarieties. This has some interesting consequences which we describe now.

In \cite{key5}, it is proved that over the field of complex numbers for a curve $C$ of genus $\geq 2$, stable bundles on $C$ of degree zero are precisely those which come from irreducible unitary representations of the fundamental group $\pi_1(C)$. This means that a vector bundle $V\to C$ of rank $r$ is stable if and only if there is an irreducible unitary representation $\rho:\pi_1(C)\to U(r)\subset GL(r)$ such that $V$ is the associated bundle for the principal $\pi_1(C)$-bundle $\widetilde{C}\to C$ corresponding to $\rho$, where $\widetilde{C}$ is the universal covering of $C$ 

For higher dimensions, the right generalisation of this theorem was formulated by Kobayashi \cite{key2}. For a vector bundle to come from a unitary representation of the fundamental group it is necessary and sufficient that it admit a hermitian metric whose hermitian connection is flat. Generalising this, Kobayashi conjectured that a vector bundle is stable (more precisely a direct sum of stable bundles) if and only if it admits a hermitian metric for which the associated connection has curvature which satisfies the Einstein-Kahler condition. (See \cite{key2}\pageoriginale for precise statements). Kobayashi and Lubke proved that if a bundle admits such a metric then it is a direct sum of stable bundles. When all the Chern classes of the bundle vanish, this condition on curvature reduces to flatness, i.e. coming from a unitary representation of the fundamental group. 

In \cite{key1}, Donaldson proved that if $\dim X=2$ and $V\to X$ a stable vector bundle with  $c_1(V)=c_2(V)=0$ then $V$ comes from an irreducible representation of $\pi_1(X)$. We show here how this result combined with our restriction theorem yields the same result for higher dimensional $X$. 

\section{Restriction Theorem}\label{s2}

We assume, without loss of generality, that $H$ is very ample. If $s_1,\ldots,s_r$ are sufficiently general elements of $H^{0}\left(X,H^{m_r}\right)$ respectively, then their common zeroes defined by $s_1=\ldots=s_r=0$ is a complete intersection subvariety of $X$ of codimension $r$. We denote this subvariety by $Y(s_1\ldots s_r)$ or $Y_s$. 

\begin{thm}\label{thm1}
Let $V$ be a stable (resp. semistable) torsion free sheaf on $X$. Then there exists an integer $N$ (depending on $V$) such that for all $m\geq N$ and sufficiently general elements $s_i\in H^{0}\left(X,H^{2^{m}}\right)i=1,\ldots,r$ the restriction $V\mid Y_s$ is stable (resp. semistable) with respect to $H\mid Y_{s}$. 
\end{thm}

We sketch the proof. If $V\mid Y_s$ is stable, it is easy to see that $V$ is stable. Only the converse needs to be proved. First it is easy to see that we can reduce to the case where $Y_s$ is a curve. 

Suppose\pageoriginale $V\mid Y_s$ is not stable. Then one can associate a canonical subsheaf $W$ of $V\mid Y_s$ which contradicts stability. If the sheaf $W$ on $Y_S$ can be extended to a sheaf $\widetilde{W}$ on $X$ together with an inclusion $\widetilde{W}\subset V$ we would be through, for then $W$ would contradict the stability of $V$. To achieve such an extension two arguments due to Mumford are used. 

\begin{WL}
This essentially says that any line bundle on the generic $Y_0$ comes from a unique line bundle on $X_K$ where $K$ is the function field of the parameter variety $\{H^{0}(X,H^{2^{m}})\}^{n-1}$ through which $s$ varies and $Y_0$ is the corresponding complete intersection variety over $K$. 
\end{WL}

\begin{AD}
One can construct a $1$-parameter family $C\to S$ of smooth complete intersection curves in $X$ of degree $2^{m}$ degenerating to a reducible curve with two smooth components each of degree $2^{m-1}$. The sheaf $V$ gives, by pull back, sheaves on the curves of the family and one can compare the degree of instability of $V$ on the curves of $\deg 2^{m}$ with that of the curves of $\deg 2^{m-1}$. 
\end{AD}

Using $(A)$, one gets a line bundle $L_s$ on $X$ which  restricts to $\det W$ on $Y_S$. Using $(B)$, one can take $L_S$ to be $L$, independent of $s$. Then one sees that $L$ admits homomorphism $L\to \displaystyle\mathop{\wedge}^{r}V$ where $r$ is the rank of $W$. Using further a boundedness argument and the lemma of Enriques-Serveri we show that there is a $\widetilde{W}\to V$ with $\det \widetilde{W}=L$ which restricts to $W\to V$ on $Y_S$. Thus we are led to the contradiction $V$ is not stable. Hence $V\mid Y_S$ must have been stable to begin with. 

\section{Narasimhan-Seshadri Theorem for Higher Dimensions}\label{s3}\pageoriginale

\begin{thm}\label{thm2}
Let $X$ be a projective nonsingular variety of dimension $n$ over the field of complex numbers $\mathbb{C}$. Let $H$ be an ample line bundle on $X$. Let $V$ be a vector bundle on $X$ with $C_1(V)=0$ and $C_2(V)\cdot H^{n-1}=0$. Then $V$ comes from an irreducible unitary representation of the fundamental group $\pi_1(X)$ if and only if $V$ is stable with respect to $H$. 
\end{thm}

\begin{Proof}
Let $C\to X$ be a complete intersection curve. Then $\pi_1(C)\to \pi_1(X)$ is surjective. Hence an irreducible unitary representation of $\pi_1(X)$ gives by composition an irreducible unitary representation of $\pi_1(C)$. Hence a vector bundle on $X$ associated to such a representation of $\pi_1(X)$ gives on restriction to $C$ a stable bundle by the theorem of Narasimhan-Seshadri. Thus $V$ on $X$ itself must be stable (see Theorem~\ref{thm1} above). 

Let $V$ be a stable bundle. The set $S$ of all representations $\rho:\pi_1(X)\to GL(r)$ can obviously be parametrised by an algebraic variety. For example, if $a_1,\ldots,a_g$ are the generators of $\pi_1(X)$ with the relations $R_1,\ldots$ then the above set can be identified with the subvariety of the product $GL(r)^{g}$ satisfying the relations $R_1,\ldots$ Therefore we can find using the lemma of Enriques-Severi (cf.\cite{key3}) an $N$ such that for $m\geq N$ for a general complete intersection variety $Y$ in $X$ of degree $m$ the restriction $\Hom_X(V,W)\to \Hom_Y\left(V\mid Y, W\mid Y\right)$ is surjective for all $W\in S$. 

Further using Theorem~\ref{thm1} we can find a smooth complete intersection\pageoriginale surface $Y\subset X$ such that $V\mid Y$ is stable. Then by the result of Donaldson there is an irreducible unitary representation $\rho:\pi_1(Y)\to U(r)\subset GL(r)$ giving $V\mid Y$. Since, by Lefchetz, $\pi_1(Y)\xrightarrow{\approx}\pi_1(X)$, $\rho$ gives a representation of $\pi_1(X)$ as well and hence a unitary bundle $V_{\rho}$ on $X$. Since $Hom \left(V\mid Y, V_{\rho}\mid Y\right)\neq 0$ we have a nonzero map $\varphi:V\to V_{\rho}$ on $X$ which must be an isomorphism since the subvariety $\det\varphi=0$ does not intersect the surface $Y$. 
\end{Proof}


\begin{thebibliography}{99}
\bibitem{key1}
Donaldson, S. K. Anti self dual Yang-Mills connections over complex algebraic surfaces and stable vector bundles, \textit{Proc. Lond. Math. Soc}., in press (1984).

\bibitem{key2}
Kobayashi, S. Curvature and stability of vector bundles, \textit{Proc. Japan Acad}. 58 A (1982), 158--162.

\bibitem{key3}
Mehta V. B., and  Ramanthan, A.\pageoriginale Semistable sheaves on projective varieties and their restrictions to curves, \textit{Math. Ann.} 258(1982), 213--224.

\bibitem{key4}
Metha V. B., and  Ramanathan, A. Restriction of stable sheaves and representations of the fundamental group, \textit{Invent. Math.,} 77(1984), 163--172.

\bibitem{key5}
Narasimhan, M. S., and Seshadri, C.S. Stable and unitary bundles on a compact Riemann surface, \textit{Ann. Math}., 82 (1965), 540--567.

\bibitem{key6}
Ramanan, S. Vector bundles over algebraic curves, in \textit{Proceedings of the I.C.M} Helsinki, 1978.
\end{thebibliography}

\vskip 1cm

\noindent
School of Mathematics\\
Tata Institute of Fundamental Research\\
Homi Bhabha Road,\\
Bombay - 400 005.

\newpage
~\phantom{a}
\thispagestyle{empty}
