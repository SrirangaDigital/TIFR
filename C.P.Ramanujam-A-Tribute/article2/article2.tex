\title{The work of C.P. Ramanujam in Algebraic Geometry} 
\markright{The work of C.P. Ramanujam in Algebraic Geometry}

\author{By~ D. Mumford}
\markboth{D. Mumford}{The work of C.P. Ramanujam in Algebraic Geometry}

\date{}
\maketitle

\setcounter{page}{9}
\setcounter{pageoriginal}{7}
It\pageoriginale was a stimulating experience to know and collaborate with C.P. 
Ramanujam. He loved mathematics and he was always ready to take up a 
new thread or to pursue an old one with infectious enthusiasm. He was 
equally ready to discuss a problem with a first year student or a 
colleague, to work through an elementary point or to puzzle over a 
deep problem. On the other hand, he had very high standards. He felt 
the spirit of mathematics demanded of him not merely routine 
developments but the \emph{right} theorem on any given topic. He 
wanted mathematics to be beautiful and to be clear and simple. He was 
sometimes tormented by the difficulty of these high standards, but, in 
retrospect, it is clear to us how often he succeeded in adding to our 
knowledge, results both new, beautiful and with a genuinely original 
stamp.

Our lives and researches intertwined considerably. I first met him in 
Bombay in 1967--68, when he took notes on my course in Abelian 
Varieties and we worked jointly on refining and understanding better 
many points related to this theory. Later, in 1970--71, we were 
together in Warwick where he ran seminars on \'etale cohomology and on 
classification of surfaces. His excitement and enthusiasm was one of 
the main factors that made that ``Algebraic Geometry year'' a success. 
We discussed many topics involving topology and algebraic geometry at 
that time, and especially Kodaira's Vanishing Theorem. My wife and I 
spent many evenings together with him, talking about life, religion 
and customs both in India and the West and we looked forward to a warm 
and continuing friendship. His premature death was a great shock to 
all who knew him. I will always miss his companionship and 
collaboration in the enterprise of mathematics.

I will give a short survey of his contributions to algebraic geometry. 
Perhaps his most perfect piece of work is his proof that a smooth 
affine\pageoriginale complex surface $X$, which is contractable 
\emph{and} simply connected at $\infty$, is isomorphic to the plane 
$C^2$. The proof of this is not simple and uses many techniques; in 
particular, it shows how well he knew his way about in the classical 
geometry of surfaces! What is equally astonishing is his very striking 
counter-example showing that the hypothesis ``simply connected at 
$\infty$'' cannot be dropped. The position of this striking example in 
a general theory of 4-manifolds and particularly in a general theory 
of the topology of algebraic surfaces is yet to be understood. As 
mentioned above, the Kodaira Vanishing Theorem was an enduring 
interest of his. Both of us were particularly fascinated by this 
``deus ex machina'', an intrusion of analytic tools (\ie harmonic 
forms) to prove a purely algebraic theorem. His two notes on this 
subject went a long way to clarifying this theorem:
\begin{enumerate}[(a)]
\item he proves it by merely topological, not analytic, techniques and 
\item he finds a really satisfactory definitive extension of the 
theorem to a large class of non-ample divisors on surfaces. 
\end{enumerate}
This second point is absolutely essential for many applications and 
was used immediately and effectively by Bombieri in his work on the 
pluricanonical system $|nK|$ for surfaces of general type. His result 
is that if $D$ is a divisor on $X$, such that $(D^2)>0$ and $(D.C)\geq 
0$ for all effective curves $C$, then $H^1(X,\sO(-D))=(0)$. 

His earliest paper, on automorphisms group of varieties, is a 
definitive analysis of the way this group inherits an algebraic 
structure from the variety itself. This work employs the techniques of 
functors, e.g., families of automorphisms developed by Gr\"othendieck at 
about the same time. His paper ``On a certain purity theorem'' 
addresses itself to a question of Lang that puzzled almost all 
algebraic geometers at that time: \emph{given a proper surjective 
morphism $f:X\to Y$ between smooth varieties, is the set
$$
\set{y\in Y\mid f^{-1}(Y)\;\text{singular}}
$$
of codimension $1$ in $Y$?} Here he provides a topologico-algebraic 
analysis of one good case where it is true, and describes a counter 
example to the general case worked out jointly with me. We again see 
his fascination with the interactions between purely topological 
techniques and algebro-geometric ones.

This\pageoriginale interest comes out again in his joint paper with Le 
Dung-Trang, whose Main Result is described in the title: ``The 
invariance of Milnor's number implies the invariance of the 
topological type''. Here they are concerned with a family of 
hypersurfaces in $C^{n+1}:F_t(z_0,\ldots,z_n)=0$, with isolated 
singularities at the origin, whose coefficients are $C^\infty$ 
functions of $t\in[0, 1]$. They show that when Milnor's number 
$\mu_t$, giving the number of vanishing cycles at the origin, is 
independent of $t$, then if $n\neq 2$, the germs of the maps 
$F_t:C^{n+1}\to C$ near $0$ are independent of $t$, up to 
homeomorphism (if $n=2$, they get a slightly weaker result). A 
beautiful and intriguing corollary is that the Artin local ring
$$
C[[z_0,\ldots,z_n]]/\oset{F,\frac{\partial F}{\partial z_0},\ldots, 
\frac{\partial F}{\partial z_n}}
$$
already determines the topology of the map $F$ near $0$.

Finally, his paper ``On a geometric interpretation of multiplicity'' 
proves essentially the following elegant theorem: \emph{If $Y\subset X$ is a 
closed subscheme defined by $I_Y\subset\sO_X$, which blows up to a 
divisor $E\subset X'$, then
$$
\frac{(-1)^{n-1}(E^n)}{n!}=\cset{\overset{\text{leading coefficient of 
the polynomial}}{P(k)=X\oset{\sO_x/I_Y^k},k\gg 0}}.
$$}

In addition to these published papers, Ramanujam made many 
contributions to my book ``Abelian Varieties'', while writing up notes 
from my lectures. Reprinted here is the Appendix by him on Tate's 
Theorem on abelian varieties over finite fields; and the following 
extraordinary theorem: It had been proven by Weil that if $X$ is a 
projective variety and $m:X\times X\to X$ is a morphism, then if $m$ 
makes $X$ into a group, $m$ must satisfy the commutative law too. 
Ramanujam proved that if $m$ merely possessed a 2-sided identity 
$(m(x,e)=m(e,x)=x)$, then $m$ must also have an inverse and satisfy 
the associative law, hence make $X$ into a group!
