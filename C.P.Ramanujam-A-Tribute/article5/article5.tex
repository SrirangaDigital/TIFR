\title{Sums of $m$-th Powers in $p$-Adic Rings}\label{art5}
\markright{Sums of $m$-th Powers in $p$-Adic Rings}

\author{By~ C.P. Ramanujam}
\markboth{C.P. Ramanujam}{Sums of $m$-th Powers in $p$-Adic Rings}

\date{}
\maketitle

\setcounter{page}{59}

\setcounter{section}{-1}
\section{Introduction and notations}\label{art5-sec0}
\setcounter{pageoriginal}{44}
Let $A$\pageoriginale be a complete discrete valuation ring of characteristic zero with finite residue field, and for any integer $m>1$, let $J_m(A)$ be the subring of $A$ generated by the $m$-th powers of elements of $A$. We will prove that any element of $J_m (A)$ is a sum of at most $8m^5$ $m$-th powers of elements of $A$. We will also prove a similar assertion when the residue field of $A$ is only assumed to be perfect and of positive characteristic, with the number $\Gamma (m)$ of summands depending only on $m$ and not on $A$.

It follows from this and Theorem 2 of Birch \cite{art5-key1} (see also K\"orner \cite{art5-key2}) that any totally positive integer of sufficiently large norm in {\em any} algebraic number field belonging to the order generated by the $m$-th powers of integers of this field is actually a sum of at most max $(2^m + 1, \; 8 m^5)$ $m$-th powers of totally positive integers of this field. This answers a question raised by Siegel \cite{art5-key3} in the affirmative.

The author has been informed that Dr. B. J. Birch has also solved substantially the same problem by a different method, and that Dr. Birch's work will be published in the Mordell issue of {\em Acta Arithmetica}.

{\em Notations.} $A$ will denote a complete discrete valuation ring, $\fp$ its maximal ideal, and $\pi$ a generator of  $\fp$. The residue field $k = A/\fp$ is assumed to be perfect of characteristic $p>0$, and $q = p^f$ will denote the number of elements in $k$ when this is finite. Let $e$ be the ramification index of $A$, and $B$ the maximal unramified complete discrete valuation ring contained in $A$ with the same residue field $k$.

For any integer $m>1$, we write $m=n \cdot p^r$, with $r \geq 0$ and $(n,p) =1$. $J_m (A) = J_m$ will denote the subring of $A$ generated by the $m$-th powers of elements of $A$. When $k$ is finite, we denote by $f_1$ the least positive divisor of $f$ such that $\dfrac{p^{f} -1}{p^{f_1} -1}$ divides $n$, and we write $q_1 = p^{f_1}$. We put $\tau = r+ 1$ or $r+2$ according as $p$ is greater than or equal to 2.

$\gamma_1 (m)$, $\gamma_2 (m)$, $\Gamma (m)$ denote positive integers depending only on $m$ and not on $A$ or $p$.

\section{The unramified case.}\label{art5-sec1}
 We start with a known lemma (Lemma 5 of Birch \cite{art5-key1} or Theorem 7 of Stemmler \cite{art5-key4}), whose proof we reproduce for completeness.

\begin{lem}\label{art5-lem1}
When $k$ is infinite, $J_n = A$ and any element of $A$ is the sum of at most $\gamma_1 (n)$ $n$-th powers.

When $k$ is finite, let $k_1$ be its subfield with $q_1$ elements. Then $J_n$ is the
inverse\pageoriginale image of $k_1$ under the natural homomorphism of $A$ onto $k$, and any element of $J_n$ is the sum of at most $(n+1)$ $n$-th powers.
\end{lem}
\begin{center}
 [Mathematika 10 (1963), 137-146]
\end{center}

\begin{proof}
When $k$ is finite, $k_1$ is evidently the set of elements of $k$ which are sums of $n$-th powers of elements of $k$. By Theorem 1 of Tornheim \cite{art5-key5}, any element of $k_1$ is the sum of $n$ $n$-th powers. Hence any unit $\alpha$ of $A$ whose image in $k$ lies in $k_1$ can be written as
$$
\alpha \equiv \sum\limits^n_{i=1} x^n_i (\mod \fp), \quad x_i \in A.
$$
Since at least one $x_i$ is a unit and $(n,p) =1$, it follows from Hensel's lemma that $\alpha$ is a sum of $n$ $n$-th powers. If $\alpha$ were a non-unit, we write $\alpha = (\alpha -1) + 1^n$, and the assertion follows.

If $k$ is infinite, it follows from Theorem 2 and the proofs of Theorem 4 and its Corollary 1 of Tornheim \cite{art5-key5} that any element of $k$ is a sum of $\gamma'_1 (n)$ $n$-th powers of $k$, and our assertion follows as before.

The next lemma says that a congruence modulo a high power of $p$ can be refined to an equality, and is again well known (see Theorem 12 of Stemmler \cite{art5-key4}).
\end{proof}

\begin{lem}\label{art5-lem2}
If 
$$
\sum\limits^a_{i=1} x^m_i \equiv y (\mod p^\tau A)
$$
and $x_1$ is a unit, there is an $x'_1 \in A$, $x'_1 \equiv x_1 (\mod p A)$ with $y = (x'_1)^m + \sum\limits^a_2 x^m_i$.
\end{lem}

\begin{proof}
Suppose we have found a $z \in A$ with $z \equiv x_1  (\mod p A)$ satisfying 
$$
z^m \equiv y - \sum\limits^a _2 x^m_i (\mod p^s A)
$$ 
for some $s \geq \tau$. If we put $z_1 = z + \lambda p^{s-r}$, it follows easily that 
$$
z^m_1 \equiv z^m + z^{m-1}\cdot n \lambda p^s (\mod p^{s+1}),
$$
and since $nz^{m-1}$ is a unit, we can find $\lambda \in A$ such that 
$$
z^m_1 \equiv y - \sum\limits^{a}_2 x^m_i (\mod p^{s+1} A).
$$
Since $A$ is complete, the lemma follows.
\end{proof}

\begin{lem}\label{art5-lem3}
Let $C$ be a commutative ring, $\fa$ an ideal of $C$, and $s$ an integer $\geq 0$. We denote by $\fa^{(p^s)}$ the set of $p^{s}$-th powers of elements of $\fa$, and by $\fa_s$ the set of elements of $C$ of the form $a_0+ pa_1 + \ldots + p^s a_s$ with $a_i\in \fa^{(p^{s-i})}$. Then $C_s$ is a subring\footnote{$C_S$ is derived from $C$ as $\fa_S$ is from $\fa$,} of $C$, and $\fa_s$ an ideal of $C_s$. If $x, y \in C$, with $x \equiv y (\mod \fa)$, then $x^{p^s} \equiv y^{p^s} (\mod \fa_s)$.
\end{lem}

\begin{proof}
This\pageoriginale being trivial for $s = 0$, we may assume that $s \geq 1$, and that the lemma holds with $s-1$ instead of $s$. Since $C_s = C^{(p^s)} + p C_{s-1}$, the inclusions $C_s \cdot C_s \subset C_s$ and $p C_{s-1} + p C_{s-1} \subset p C_{s-1}$ are trivial, and only the inclusion $C^{(p^s)} + C^{(p^s)} \subset C_s$ remains to be verified. For $x$, $y \in C$, we have 
\begin{equation}
x^{p^s} + y^{p^s} = (x+y)^{p^s} - \sum\limits_{0< i < p^s} \binom{p^s}{i}  x^i y^{p^{s-i} }, \label{art5-eq1}
\end{equation}
and if $p^k || i$, $p^{s-k} || \binom{p^s}{i}$, and hence by induction hypothesis, $-\sum$ belongs  to $pC_{s-1}$, and the right side belongs to $C_s$. Thus $C_s$ is a ring, and $\fa_s$ a subring of $C_s$, and clearly $C_s \fa_s \subset \fa_s$. Finally, if $z = x + y$ with $y \in \fa$, it follows from \eqref{art5-eq1} that $x^{p^s} \equiv z^{p^s} (\mod \fa_s)$, which completes the proof of the lemma.
\end{proof}

It follows that the set 
$$
J'_m = \left\{ x_0^{p^r} + px_1^{p^{r-1}} + \ldots + p^r x_r | x_i \in J_n \right\}
$$
is a subring of $A$, and we have evidently the inclusion
$$
J'_m \subset J_m.
$$

\begin{prop}\label{art5-prop1}
Assume that $r \geq 1$ and that $A$ is unramified. If $k$ is infinite, $J_m = J'_m = A$ and any element of $A$ is a sum of at most $\gamma_2 (m)$ $m$-th powers.

If $k$ is finite, any element of $J_m$ is a sum of at most $\dfrac{p^\tau - 1}{p -1} n + \delta $ $m$-th powers, where $\delta=1$ if $n = p-1$ and $\delta =0$ otherwise. Moreover, $J_m = J'_m$ unless $p=2$ and $f_1 \neq f$.
\end{prop}


\begin{proof}
First consider the case when $k$ is finite and $p> 2$. We put $m_s = n \cdot p^s (0 \leq s \leq r)$, and we shall show that (with the notation of Lemma \ref{art5-lem3}) any element of $(J_n)_s$ is the sum of at most $\dfrac{p^{s+1} -1}{p-1} n$ $m_s$-th powers modulo $p^{s+1} A$. For $s=0$, this is the theorem of Tornheim quoted earlier. Suppose the result is true for $s$, and let $x = x_0^{p^{s+1}} + y \in (J_n)_{s+1}$, with $y \in p (J_n)_s$. By the theorem of Tornheim, there exist $y_1, \ldots, y_n \in A$ such that $x_0 \equiv \sum\limits^n_1 y^n_i$ ($\mod p A$), hence
{\fontsize{10pt}{12pt}\selectfont
$$
x_0^{p^{s+1}} \equiv \left( \sum\limits^n_1 y^n_i \right)^{p^{s+1}} (\mod p^{s+2} A) \quad \text{and} \quad \left(\sum\limits^n_1 y^n_i \right)^{p^{s+1}} \equiv \sum\limits^n_1 y^{m_{s+1}}_i (\mod p (J_n)_s),
$$}
as follows from (1) with $s+1$ instead of $s$. Thus we obtain by induction hypothesis,
$$
x \equiv \sum\limits^n_1 y^{m_{s+1}}_i  + p \sum\limits^N_1 z^m_j, \quad (\mod p^{s+2} A),
$$
where\pageoriginale $N = \dfrac{p^{s+1} -1}{p-1} \cdot n$. Since the residue field is perfect, we can find $t_j$ such that $t_j^p \equiv z_j (\mod p A)$, and hence $t_j^{m_{s+1}} \equiv z^{m_s}_j (\mod p^{s+1} A)$, which gives
$$
x \equiv \sum\limits^n_1 y^{m_{s+1}}_i + p \sum\limits^N_1 t^{m_{s+1}}_j \quad (\mod p^{s+2} A), 
$$ 
i.e., $x$ is a sum of $n+p N = n \cdot \dfrac{p^{s+2} -1}{p-1}\quad m_{s+1}$-th powers modulo $p^{s+2} A$.
 
This completes the induction, and we obtain for $s =r$ that any $x \in J'_m$ is the sum of at most $\dfrac{p^r -1}{p-1} \cdot n$ $m$-th powers modulo $p^{r+1}A$, and it is easily deduced from Lemma \ref{art5-lem2} that $x$ is actually the sum of this many $m$-th powers if $x \nequiv 0(p^{\tau})$. Further, if $n < p -1$ and $\zeta$ is a primitive $(p-1)$-th root of unity in $A$, $\sum\limits^{p-2}_{i=0} \zeta^{mi} = 0$, and if $n \geq p -1$, $0 \equiv \underbrace{1  + \ldots + 1}_{p^\tau \text{ times}} (\mod p^\tau)$. 

This proves the assertion when $x\equiv0(p^{\tau})$, in view of Lemma \ref{art5-lem2}.


Next consider the case when $J_n = A$, and suppose that every element of $A$ is a sum of $\lambda$ $n$-th powers modulo $pA$. Since the residue field is perfect, it follows that sums of $\lambda $ $m$-th powers form a system of representatives for the residue classes $\mod p A$. Since any element $x \in A$ can be represented as $x = x_0 + x_1 p+ \ldots + x_{\tau-1} p^{r-1} (\mod p^\tau A)$ with $x_i$ chosen from this system of representatives, it follows from Lemma \ref{art5-lem2} that any element of $A$ which is $\nequiv 0$ $(\mod p^\tau)$ is a sum of 
$$
(1+ p + \ldots + p^{\tau -1}) \lambda = \dfrac{p^{\tau} -1}{p-1} \cdot \lambda 
$$
$m$-th powers. Elements $x \equiv 0 (p^\tau)$ can be dealt with as before. This settles the case of the infinite residue field as well as the case of $p=2$ and $J_n = A$.

If $p=2$ and $J_n \neq A$, it follows from Lemma \ref{art5-lem1} that $f_1 \neq f$ and hence $f_1 \leq f/2$, $n \geq \dfrac{2^f-1}{2^{f_1} -1} \geq 2^{f/2}+1$, and $2^f \leq (n-1)^2$. We give an argument whose idea goes back to Siegel \cite{art5-key3}.

Let $\bar{x}^m_1, \ldots, \bar{x}_d^m$ be a minimal set of generators (as an additive group) of the image of $J_m (A)$ in $A/2^{r+2}A$, and $q_i$ the index of the subgroup generated by $\{\bar{x}^m_1, \ldots, \bar{x}^m_{i-1}\}$ in the subgroup generated by $\{\bar{x}^m_1,\ldots, \bar{x}^m_i\}$. Then $q_i \leq 2^{r+2}$, each $q_i$ is a positive power of 2 and $q_1 \ldots q_d \leq 2^{(r+2)f}$. Any element of $J_m (A)$ can therefore be written as a sum of at most $\sum\limits^d_{i=1} (q_i-1)$ $m$-th powers modulo $2^{r+2} A$. We shall find an upper bound for 
$$
\sum\limits^d_{i=1} (q_i -1) = \sum\limits^d_{i=1} \min (q_i -1, 2^{r+2} -1).
$$
Let $N$\pageoriginale be the supremum of all sums $S = \sum\limits^l_{i=1} \min (n_i -1, 2^{r+2} -1)$, where $l$, $n_1, \ldots, n_l$ vary subject to the following conditions: each $n_i (1\leq i \leq l)$ is a positive power of 2 and $n_1 \ldots n_l \leq 2^{(r+2)f}$. Suppose $l =d'$, $n_i = q'_i$ $(1 \leq i \leq d')$ is a choice of the variables for which $S$ attains its supremum, with $l$ minimal. We assert that for all but one $i$, $q'_i \geq 2^{r+2}$. If not, suppose for instance that $q'_{d'-1} < 2^{r+2}$, $q'_{d'} < 2^{r+2}$. Since all the $q'_i$ are powers of $2$, it follows that $q'_{d'-1} + q'_{d'} \leq 2^{r+2}$. If we define $q''_i = q'_i$ for $1 \leq i \leq d' - 2$, $q''_{d'-1} = q'_{d'-1} \cdot q'_{d'}$, we have
\begin{gather*}
\min (q'_{d'-1} - 1, 2^{r+2} -1) +\min (q'_{d'} -1, 2^{r+2} -1)\\
= (q'_{d'-1} -1) + (q'_{d'} -1) \leq \min (q''_{d'-1} - 1, 2^{r+2} -1 ),\\
N = \sum\limits^{d'}_{i=1} \min (q'_i -1, 2^{r+2} -1) \leq \sum\limits^{d'-1}_{i=1} \min (q''_{i} - 1, 2^{r+2} -1) \leq N,
\end{gather*}
since $q''_1 \ldots q''_{d'-1} = q'_1 \ldots q'_{d'} \leq 2^{(r+2)f}$, which contradicts the minimality of $d'$. Hence for all but one $i$, $q'_i \geq 2^{r+2}$, and all the $q'_i$ are $\geq 2$, and therefore 
$$
2^{(r+2)(d'-1)+1} \leq q'_1 \ldots q'_{d'} \leq 2^{(r+2)f}, d'\leq f.
$$
It follows that 
\begin{align*}
N & = \sum\limits^{d'}_1 \min (q'_i - 1, 2^{r+2} -1) \leq f (2^{r+2} -1)\\
& \leq \frac{2 \log (n-1)}{\log 2} (2^{r+2} -1) < n (2^{r+2} -1).
\end{align*}
It follows from Lemma \ref{art5-lem2} that any element of $J_m (A)$ is a sum of at most $(2^{r+2}-1)n$ $m$-th powers.
This completes the proof of Proposition \ref{art5-prop1}.
\end{proof}

\section{The case when the residue field is large}\label{art5-sec2}

\begin{prop}\label{art5-prop2}
When $k$ is infinite, $J_m = J'_m = A$, and any element of $A$ is a sum of at most $\gamma_3 (m)$ $m$-th powers.

When $k$ is finite, and $f_1 > r \geq 1$, any element of $J_m$ is a sum of at most $\{(r+1) (p^r+1) +1\} \left\{\dfrac{p^\tau -1}{p-1}  n + \delta\right\}$ $m$-th powers, where $\delta =1$ if $n=p-1$ and $\delta =0$ otherwise; and the equality $J_m = J'_m$ holds unless $p=2$ and $f \neq f_1$.
\end{prop}

\begin{proof}
When either $k$ is infinite or $f_1 > r$, we can find elements\break $u_0, \ldots, u_{p^r} \in J_m (B)$ such that their residue field images are distinct, by Lemma \ref{art5-lem1}. It follows that we can find $a_0, \ldots, a_{p^r} \in J_m (B)$ such that 
\begin{equation}
p^l x^{p^{r-1}} = \sum\limits^{p^r}_{i=0} a_i (u_i x + 1)^{p^r}. \label{art5-eq2}
\end{equation}
In fact,\pageoriginale writing $\binom{p^r}{p^{r-l}} = p^l \alpha_l$, where $\alpha_l$ is a unit in $J_m(B)$, we have only to solve the system of linear equations
\begin{equation*}
\sum\limits^{p^r}_{i=0} a_i u_i^v = 
\left\{
\begin{aligned}
& 0 \text{~ if ~} v \neq p^{r-l}, \quad 0 \leq v \leq p^r\\
& \alpha^{-1}_{l} \text{~ if ~}  v = p^{r-l} 
\end{aligned}
\right.
\end{equation*}
for the $a_i$ in $J_m(B)$. The determinant $\Delta = \prod\limits_{i < j} (u_i - u_j)$ is an element of $J_m(B)$ invertible in $B$, hence in $J_m(B)$, since $\Delta (\Delta^{m-1} (\Delta^{-1})^m) =1$, $\Delta^{m-1} (\Delta^{-1})^m \in J_m (B)$. Thus the $a_i$ can be solved for in $J_m (B)$.

Now for $x \in \fp$, $(1 + u_i x)^{p^r}$ is an $m$-th power by Hensel's lemma. It follows that if any element of $J_m (B)$ is a sum of $\lambda$ $m$-th powers, any element of $\fm_r$ (in the notation of Lemma \ref{art5-lem3}) is the sum of at most $\lambda (r+1) (p^r +1)$ $m$-th powers. But since any element of $A$ is congruent to an element of $B$ modulo $\fp$, it follows from Lemma \ref{art5-lem3} that any element of $J_m (A)$ or $J'_m(A)$ is congruent to an element of $J_m(B)$ or $J'_m(B)$ respectively modulo $\fp_r$.

Proposition \ref{art5-prop2} follows from this, if we substitute for $\lambda$ from Proposition \ref{art5-prop1}. 
\end{proof}

\begin{remarks*}%%% 
\begin{enumerate}
\item[(1)] It follows from the proof that the Proposition is true for any complete local ring with residue field finite and $f_1 > r$, or perfect and infinite of characteristic $p>0$, since such a ring contains a homomorphic image of an unramified complete discrete valuation ring of characteristic zero with the same residue field, by the structure theorems of Cohen.

\item[(2)] When $n =1$, that is when $m=p^r$, the passage from $J_m(A)$ to $J_m(B)$ [or from $J'_m(A)$ to $J'_m(B)$] is not necessary, and we can apply (2) to any element $x \in A$ directly. Thus if every element of $J_m(B)$ is a sum of at most $\lambda$ $m$-th powers, every element of $J_m(A)$ is a sum of at most $(r+1) (p^r +1)\lambda$ $m$-th powers. 

Moreover when $n=1$, the case $f =r$ can also be dealt with in the same manner. In fact if $\zeta$ is a primitive $(p^f-1)$-th root of unity in $B$, we have the identities
\begin{gather*}
(p^f -1) \binom{p^r}{p^k} x^{p^k} = \sum\limits^{p^f -2}_{i=0} \zeta^{-ip^k} (\zeta^i x +1)^{p^r}, \;\; 0 < k < r,\\
(p^f -1) (p^r x + x^{p^r}) = \sum\limits^{p^f -2}_{i=0} \zeta^{-i} (\zeta^{i} x+ 1)^{p^r},
\end{gather*}
and $\zeta$ is a $p^r$-th power in $B$.

\item[(3)] Since $p^{f_1} -1 \geq \dfrac{p^f-1}{n}$, the condition $f_1 >r$ is certainly fulfilled if $p^f>m$. Since the case $p^f =m$ is also covered by Remark (2), we may assume in what follows that $p^f < m$.
\end{enumerate}
\end{remarks*}

\section{The general case}\label{art5-sec3}
We\pageoriginale assume $A$ to have finite residue field, and that $f_1 \leq r$, $p^f < m$. The method is similar to that of \S \ref{art5-sec2}, but more complicated.

Let $\zeta$ be a primitive $(q_1-1)$-th root of unity in $B$. It is easily seen that $\zeta \in J_m (B)$. For any $k$ with $0 \leq k \leq r$, we wish to establish an identity of the form 
\begin{equation}
\sum\limits_j \sum\limits^{q_1 -2}_{i=0} a_{ij} (\zeta^i \lambda_j x + 1)^{p^r} = p^{r-k} (\lambda x)^{p^k} \label{art5-eq3}
\end{equation}
with $a_{ij}$, $\lambda_j$ and $\lambda$ suitably chosen. This identity is equivalent to the system of linear equations
\begin{equation*}
\binom{p^r}{p^k} \sum\limits_{i,j} a_{ij} \zeta^{iv} \lambda_j^v  = 
\left\{
\begin{aligned}
& 0 \text{~ if ~} v \neq p^k, \quad 0 \leq v \leq p^r,\\
& p^{r-k} \lambda^{p^k} \text{~ if ~} v = p^k.
\end{aligned}
\right.
\end{equation*}
If we put $\sum\limits_i a_{ij} \zeta^{iv} = b_{jv}$, then $b_{jv} = b_{jv'}$ if $v \equiv v' (\mod (q_1-1))$. We naturally choose $b_{jv}= 0$ if $v \not\equiv p^k (\mod (q_1 -1))$, and we will choose $b_{jv} =c_j$ for $v \equiv p^k (\mod (q_1 -1))$ later. The $a_{ij}$ are then determined by the $c_j$ by the equations $a_{ij} = u_i \cdot c_j$ with $u_i \in J_m (B)$. Let us write $p^k =\sigma (q_1 -1) + \rho_1, p^r - \rho_1 = \kappa (q_1 -1) +\rho_2$, with $0 \leq \rho_1$, $\rho_2 < q_1-1$; if $q_1 > 2$ and $l$ and $s$ are the least non-negative residues of $k$ and $r$ modulo $f_1$, it is easily checked that 
$$
\sigma = \dfrac{p^k - p^l}{q_1 -1}, \rho_1 = p^l, \text{ ~and~ } \kappa =\dfrac{p^r - p^s}{q_1 -1} \text{ ~or~ } = \dfrac{p^r - p^s}{q_1 -1} - 1
$$
according as $s \geq 1$ or $s < l$. We let the index $j$ vary through the range $0 \leq j \leq \kappa$. We write $\binom{p^r}{p^k} = p^{r-k} m_k$, where $m_k$ is a unit in $J_m (B)$. Putting $\lambda_j^{q_1-1} = \mu_j$, $m_k c_j \lambda_j^{\rho_1} = d_j$, so that $a_{ij} = m_k^{-1} \lambda_j^{-\rho_1} u_i d_j$, the equation \eqref{art5-eq3} becomes equivalent to the system of linear equations.
\begin{equation*}
\sum\limits^{\kappa}_{j=0} d_j \mu^t_j = 
\left\{
\begin{aligned}
& 0 \text{~ if ~} t \neq \sigma, \quad 0 \leq t \leq \kappa\\
& \lambda^{p^\kappa} \text{ ~ if ~ } t = \sigma.
\end{aligned}
\right.
\end{equation*}
We choose $\lambda_j = \pi^{jm} (0 \leq j \leq \kappa)$, so that $\mu_j = \pi^{jm(q_1-1)}$. If $S_\alpha (X_0, \ldots, X_\beta)$ denotes the elementary symmetric function of degree $\alpha$ in the variables $X_0,\ldots, X_\beta$, the solution of the above system of linear equations for the $d_j$ is
$$
d_j = \pm \lambda^{p^k} \dfrac{S_{\kappa -\sigma} (\mu_0, \ldots, \hat{\mu}_j, \ldots, \mu_\kappa)}{\prod\limits_{\alpha \neq j} (\pi^{jm(q_1-1)} - \pi^{\alpha m (q_1-1)})}
$$
and hence 
$$
a_{ij} = \pm \dfrac{m_k^{-1} u_i \lambda^{p^k} S_{\kappa -\sigma} (\mu_0, \ldots, \hat{\mu}_j,\ldots, \mu_\kappa)}{\pi^{jm\rho_1} \prod\limits_{\alpha \neq j} (\pi^{jm (q_1-1)} - \pi^{\alpha m (q_1-1)})}
$$
where the symbol `~$\hat{}$~' over a letter means that it is to be omitted. Now 
$$
S_{\kappa -\sigma} (\mu_0,\ldots, \hat{\mu}_j,\ldots, \mu_\kappa) = \pi^{m(q_1-1) (\kappa - \sigma -1) (\kappa-\sigma)/2} P_j(\pi^{m(q_1-1)}),
$$
where\pageoriginale $P_j$ is a polynomial of degree $\leq (\sigma +1) (\kappa - \sigma)$ with rational integral coefficients. The above expression for the $a_{ij}$ may be rewritten as 
$$
a_{ij} = \pm \dfrac{m_k^{-1} u_i\lambda^{p^k} \pi^{(\frac{1}{2}[(\kappa - \sigma -1)(\kappa -\sigma)] - \frac{1}{2} [(j-1)j] - (\kappa - j)j) m (q_1 -1) - jm \rho_1 } P_j (\pi^{m(q_1-1)})}{\prod^j_{\alpha =1} (1-\pi^{m\alpha (q_1-1)}) \prod\limits^{\kappa - j}_{\alpha =1} (1 -\pi^{m\alpha (q_1-1)})}
$$
We wish to choose $\lambda$ in such a way that it is divisible by as small a power of $\fp$ as possible, and such that the $a_{ij}$ are sums of as few a number of $m$-th powers as possible.

Let $\fp^X$ be the power of $\fp$ dividing $\lambda$. In order that the $a_{ij}$ belong to $A$ at all, we must have
\begin{multline*}
p^k X + \left( \dfrac{(\kappa - \sigma - 1)(\kappa- \sigma)}{2} - \dfrac{(j-1)j}{2} - (\kappa - j) j\right) m (q_1 -1) - jm\rho_1 \geq 0 \\
(0 \leq j \leq \kappa).
\end{multline*}
The minimum of the left side of the above inequality as $j$ varies in the range $0 \leq j \leq \kappa$ is attained for $j = \kappa$, so that we must have [on substituting $p^k = \sigma (q_1 -1)  \rho_1$]
$$
p^k (X-\kappa m) +m (q_1 -1) \dfrac{\sigma (\sigma +1)}{2} \geq 0.
$$
We choose $X$ in such a way that equality holds. We put 
$$
\lambda = \pi^X \prod\limits^\kappa_{j=1} (1 - \pi^{im(q_1 -1)}).
$$
It is easily checked that the polynomial $\prod\limits^\alpha_{j=1} (1-Y^j) \prod\limits^{\kappa -\alpha}_{j=1} (1-Y^j)$ divides $\prod\limits^\kappa_{j=1} (1 - Y^j)$, so that we obtain
$$
a_{ij} = \pi^{um} Q_j (\pi^{m(q_1 -1)}), \quad u = u (j) \geq 0 
$$
where $Q_j$ is a polynomial with coefficients in $J_m (B)$ of degree
\begin{align*}
& \leq (\sigma +1) (\kappa - \sigma) + p^k \dfrac{\kappa(\kappa+1)}{2} -\dfrac{j(j+1)}{2} - \dfrac{(\kappa - j) (\kappa - j +1)}{2}\\
& = (\sigma +1) (\kappa -\sigma) + (p^k -1) \dfrac{\kappa (\kappa +1)}{2} + j (\kappa - j).
\end{align*}
It follows from (3) that for any $x \in \fp^{X+1}$, $p^{r-k} x^{p^k}$ is a sum of at most 
$$
N (q_1-1) \left[ (\kappa +1) \left\{(\sigma +1) (\kappa -\sigma) + (p^k-1) \dfrac{\kappa (\kappa +1)}{2} \right\} + \sum\limits^\kappa_{j=0} j (\kappa - j) \right]
$$
$m$-th powers, where $N$ denotes the smallest integer such that any element of $J_m (B)$ is the sum of at most $N$ $m$-th powers. Let us put $\kappa_0 = \left[ \dfrac{p^r}{q_1 - 1}\right]$, so that $\kappa \leq \kappa_0$ for all $k$; the above integer is then $\leq N (q_1 -1) p^k \kappa_0 (\kappa_0 +1)^2 /2$.

Hence\pageoriginale if we put $Y=1+\sup\limits_{{0\le k\le r}} X$ and $\fa=\fp^Y$, any element of $\fa_r$ (in the notation of Lemma \ref{art5-lem3}) is the sum of at most
$$
\dfrac{N}{2} (q_1-1) \kappa_0 (\kappa_0+1)^2 \dfrac{p^{r+1} -1}{p-1}
$$
$m$-th powers. Since
$$
X = \kappa m + p^{r-k} n (q_1-1) \dfrac{\sigma (\sigma +1)}{2} = \kappa m + n \dfrac{(\sigma +1 )}{2} (p^r - \rho_1 p^{r-k}),
$$
and $\sigma$ and $(p^r - \rho_1 p^{r-k})$ are both non-decreasing as $k$ increases, we see that 
$$
\sup\limits_{0\leq k\leq r} X = \kappa_0 m + n (q_1-1) \dfrac{\kappa_0(\kappa_0+1)}{2}.
$$

Now, if $\bar J$ denotes the image of $J_m(A)$ in the ring $A_r/\fa_r + p^{\tau} A$, the order (as an abelian group) of $\bar J$ is of the form $p^Q$ where 
$$
Q \leq (r+1) f \left( \kappa_0 m + n (q_1 -1) \dfrac{\kappa_0 (\kappa_0 +1)}{2} + 1\right),
$$
by Lemma \ref{art5-lem3}. Choose a minimal set $\bar{x}_1^m, \ldots, \bar{x}_d^m $ of generators of $\bar{J}$ (as an abelian group), and denote by $q_i$ the index of the subgroup $\{\bar{x}^m_1, \ldots, \bar{x}^m_{i-1}\}$ in the subgroup $\{\bar{x}^m_1, \ldots, \bar{x}^m_i\}$ for $1 \leq i \leq d$, so that $p^r \geq q_i > 1$, each $q_i$ is a power of $p$ and $q_1 \ldots q_d = p^Q$. Any element of $\bar{J}$ is then a sum of at most $\sum\limits^d_1 (q_i -1)$ elements of the form $\bar{x}^m_i$. Arguing as in the last part of the proof of Proposition \ref{art5-prop2}, we obtain that 
$$
\sum\limits^d_{i=1} (q_i -1) \leq \sum\limits^{d'}_{j=1} \min (q'_j - 1, p^r -1),
$$
where each $q'_j$ is again a positive power of $p$, $\prod q'_j = p^Q$, and all but one of the $q'_j$ satisfy $q'_j \geq p^\tau$. Thus we obtain
\begin{gather*}
p^{(d'-1) \tau + 1} \leq p^Q,\\
d'-1 < (r+1) f \left(\kappa_0 m + n (q_1 -1) \dfrac{\kappa_0 (\kappa_0 + 1)}{2} + 1 \right) / \tau\\
\leq f \left( \kappa_0 m + n (q_1 -1) \dfrac{\kappa_0 (\kappa_0 +1)}{2} +1\right),\\
\text{i.e. } \hspace{2cm} d' \leq f \left(\kappa_0 m + n (q_1-1) \dfrac{\kappa_0 (\kappa_0+1)}{2} +1 \right), \hspace{2cm}
\end{gather*}
and hence
$$
\sum\limits^d_1 (q_i -1) \leq f (p^\tau -1) \left(\kappa_0 m +n (q_1 -1) \dfrac{\kappa_0 (\kappa_0 +1)}{2} +1 \right).
$$
It\pageoriginale follows that any element of $J_m(A)$ is a sum of at most 
$$
\dfrac{N}{2} (q_1-1) \kappa_0 (\kappa_0+1)^2 \dfrac{p^{r+1} -1}{p-1} + f (p^\tau -1) \left(\kappa_0 m + n (q_1-1) \dfrac{\kappa_0 (\kappa_0+1)}{2} +1 \right) 
$$
$m$-th powers. Now if $f = f_1$, then $f \leq r$ by assumption, and if $f \neq f_1$, then $f_1 \leq f/2$, so that 
$$
n \geq \dfrac{p^f -1}{p^{f_1} -1} \geq p^{f/2} +1,\quad f \leq \dfrac{2\log (n-1)}{\log p}.
$$
Using this and the value for $N$ as given by Proposition \ref{art5-prop1}, one easily deduces after a little calculation that the above integer is $< 8 m^5$. In view of Propositions \ref{art5-prop1} and \ref{art5-prop2}, we thus have 

\begin{prop}\label{art5-prop3}
Let $A$ be any complete discrete valuation ring of characteristic zero with finite residue field, and $m$ an integer $>1$. Then any element of the subring $J_m(A)$ of $A$ generated by $m$-th powers of elements of $A$ is a sum of at most $8m^5$ $m$-th powers.
\end{prop} 

\begin{remark*}
As is needless to remark, the bound $8m^5$ is quite rough, and it is possible to obtain better estimates for particular values of $m$. In view of the fact that the ``global to local'' reduction of the number field case to that of a $p$-adic field works when the number of variables is at least $2^m+1$ (Birch \cite{art5-key1} and K\"orner \cite{art5-key2}), and since $8m^5 < 2^m+1$ for $m \geq 27$, it might not be without interest to check that $2^m +1$ variables suffice for $m \leq 26$. For most values of this range, this is a consequence of our estimates, but the author has not checked this for all $m \leq 26$.
\end{remark*}

The author wishes to thank Professor K. G. Ramanathan for suggesting the problem and for his kind help. He is also indebted to Professor K. Chandrasekharan for several improvements.

\begin{thebibliography}{99}
\bibitem{art5-key1} B. J. Birch, ``Waring's problem in algebraic number fields'', {\em Proc. Cambridge Phil. Soc., 57 (1961)}.

\bibitem{art5-key2} O. K\"orner, ``\"Uber Mittelwerte trigonometrischer Summen und ihre Anwendung in algebraischen Zahlk\"orpern'', {\em Math. Annalen, 147 (1962)}.

\bibitem{art5-key3} C. L. Siegel, ``Generalisation of Waring's problem to algebraic number fields'', {\em American J. of Math., 66 (1944)}.

\bibitem{art5-key4} R. M. Stemmler, ``The easier Waring  problem in algebraic number fields'', {\em United States Office of Naval Research Technical Report N.R., 043-194 (1959)}.

\bibitem{art5-key5} L. Tornheim, ``Sums of $n$-th powers in fields of prime characteristic'', {\em Duke Math. J., 4 (1938)}.
\end{thebibliography}

Tata Institute of Fundamental Research, Bombay. 

\vskip 0.4cm
\hfill(Received on the 3rd of October, 1963)
