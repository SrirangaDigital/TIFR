\title{On a Certain Purity Theorem}\label{art7}
\markright{On a Certain Purity Theorem}

\author{By~ C.P. Ramanujam}
\markboth{C.P. Ramanujam}{On a Certain Purity Theorem}

\date{}
\maketitle

\begin{center}
[Received May 1, 1970]
\end{center}

\setcounter{page}{87}

\section{Statement of result}\label{art7-sec1}
\setcounter{pageoriginal}{64}
 Let\pageoriginale $X$ and $Y$ be connected complex manifolds and $f: X \to Y$ a proper and flat\footnote{Since $X$ and $Y$ are manifolds and $f$ is proper, the assumption of flatness is equivalent to the assumptions that $f$ is surjective and that all the fibres of $f$ are of the same dimension everywhere on $X$.} holomorphic map. Let $D \subset X$ be the analytic set where the differential $df$ is not of maximal rank and $E = f(D)$ the set of critical values of $f$. It has been conjectured that $E$ is pure of codimension one in $Y$. This has been proved by I. Dolgacev \cite{art7-key1} and R. R. Simha \cite{art7-key2} when the general fibre is a compact Riemann surface of genus $g \ge 1$. We prove the conjecture when the general fibre is the Riemann sphere, thereby establishing the conjecture when the relative dimension $\dim X - \dim Y$ is one. (When the general fibre is not connected, use Stein factorisation and purity of branch-locus).

Specifically, we prove the following 

\begin{theorem*}
Let $X$, $Y$ be connected complex manifolds, $f: X \to Y$ a proper flat holomorphic map such that the general fibre of $f$ is the Riemann sphere, $D$ the set points in $X$ where $df$ is not of maximal rank and $E = f(D)$. Then $E$ is pure of codimension one in $Y$.
\end{theorem*}

\section{Proof of Theorem}\label{art7-sec2}
Let us assume to the contrary that the analytic set $E$ is not pure of codimension one. We first achieve several reductions, which are valid even when the general fibre is only assumed of dimension one. 

Firstly, since the set of $(n, n +1)$ matrices not of maximal rank is an analytic subset pure of codimension 2 in the space of all $(n, n+1)$\pageoriginale matrices,  it follows that any irreducible component of $D$ is of codimension $\leqslant 2$ in $X$, hence any component of $E$ is of codimension $\leqslant 2$ in $Y$. Thus if $E$ is not pure of codimension  one, by removing a certain analytic set from $Y$, we may assume that $E$ is a connected sub-manifold of codimension two in $Y$.

Next, let $D' = \{x \in X \rank (df)_x \leqslant \dim Y -3  \}$. We assert that $\dim D' \leqslant \dim X -3$. If not, $D'$ has an irreducible component $Z$ of codimension $\leqslant 2$ in $X$,  and there would exist an open subset $U$ of $Z$ of smooth points $x$ where $\dim (\ker (df)_{x} \cap T_x (U)) \geqslant 2$, so that there would  exist a holomorphic vector field on a non-void open subset $U'$ of $U$, not tangential to the fibres of $f$ but mapped to zero by $df$. This is clearly impossible, since the restriction of $f$ to any integral curve of this vector field must be constant. Also, we see that if $Z$ is a component of $D'$ of dimension equal to $\dim X -3$, at any regular point of $Z$, $\ker d (f |Z) \neq 0$, so that dimension $f(Z)\leqslant \dim Y - 3$. Hence, by throwing out $f(D')$ from $Y$, we may assume that $\rank (df) \geqslant \dim Y - 2$ everywhere on $X$.

Let $D_i (1 \leqslant i \leqslant p)$ be the components of $D$, so that each $D_i$ is of codimension 2 in $X$ and $f(D_i) = E$. We can clearly choose a point $y \in E$ having the following properties: (i) every component of the fibre over $y$ of the map $f|D_i: D_i \to E$ meets the set $(D_i)_{\text{reg}}$ of smooth points of $D_i$, and (ii) on every component of the fibre over $y$ of the induced map $g_i:(D_i)_{\text{reg}} \to E$, $g_i$ is of maximal rank.

\begin{lemma*}
For a dense subset $\Omega$ of the Grassmannian of two dimensional subspaces of the tangent space $T_y (Y)$ to $Y$ at $y$, we have 
$$
Im (df)_x + V = T_y (Y), \forall x \in f^{-1} (y), \forall V \in \Omega.
$$
\end{lemma*}

\begin{proof}
Let $C_1, \ldots, C_q$ be the irreducible components of $f^{-1} (y) \cap D$. If for some $C_i$, $\rank (df)_x = \dim Y -2$ for all $x \in C_i$, by (i) and (ii) above, we see that $Im (df)_x = T_y(E)$ for all $x \in C_i$. Then define $\Gamma_i$ to be the set of 2-dimensional subspaces of $T_y(Y)$ meeting $T_y(E)$ in a space of dimension $\geqslant 1$. On the other hand, suppose $C$ is a component such that $\rank (df)$ is $\dim (Y)-1$ generically on $C_i$. Then there is a finite set of points $x_1, \ldots, x_r$ on $C_i$ such that $\rank (df)_{x_j} = \dim Y -2$ and $\rank (df)_x = \dim Y -1$ for $x \in C_i$, $x \neq x_j$ for any $j$.\pageoriginale Let $\sum_j$ be the set of 2-dimensional subspaces of $T_y(Y)$ meeting $Im(df)_{x_j}$ in a non-zero subspace. Let $\sum'$  be the set of 2-dimensional  subspaces of $T_y (Y)$ contained in $Im (df)_x$ for some $x \in C_i - \{x_1, \ldots, x_r\}$. Put $\Gamma_i = \sum' \cup \bigcup\limits_j \sum_j$.

A simple argument (counting constants) shows that each $\Gamma_i$ and hence $\Gamma = \bigcup \Gamma_i$ is a countable union of closed nowhere dense subsets of the Grassmannian of 2-dimensional subspaces of $T_y (Y)$. Hence if $\Omega$ is the complement of $\Gamma$ in this Grassmannian, $\Omega$ has the required property.

Now for the next reduction. Choose $y \in E$ as above, and a 2-dimensional subspace $V$ of $T_y(Y)$ transversal to $T_y(E)$ with $V \in \Omega$. Choose a submanifold $Y'$ of dimension two in a neighbourhood of $y$ meeting $E$ only at $y$ and having $V$ for its tangent space. Put $X' = f^{-1} (Y')$, so that $X'$ is a submanifold of $X$ proper and flat over $Y'$. Further, if $g: X' \to Y'$ is induced by $f,g$ is of maximal rank outside $g^{-1} (y)$. Now for any flat holomorphic map $\phi$ of complex manifolds, $\phi$ is of maximal rank along a fibre if and only if the ``correct'' fibre (as an analytic space, possibly with nilpotent elements in its structure sheaf, whose defining ideal sheaf is generated by the maximal ideal of the local ring of the point in the image space) is reduced and smooth. Now, the fibres of $f$ and $g$ over $y$ are the same, so that $g$ cannot be of maximal rank all along $g^{-1}(y)$.

We are thus reduced to the case where $Y$ is an open neighbourhood  in $C^2$ of the closed ball $D_1 = \{(z_1, z_2) \in C^2 || z_1|^2 + |z_2|^2 \leq 1\}$ and $E$ consists of the single point $0 \in C^2$. We now make use of the hypothesis that the general fibre (hence every fibre $f^{-1} (z)$ for $z \neq 0$) is holomorphically isomorphic to the Riemann sphere.

Now, we assert that the inclusion $f^{-1} (0) \hookrightarrow f^{-1} (D_1)$ is a homotopy equivalence. To see this, note that since $X$ can be triangulated such that $f^{-1} (0)$ is a sub-complex, there is a fundamental system of neighbourhoods $\{U_n\}_{n \geq 1}$ of $f^{-1} (0)$ such that $f^{-1} (0)$ is a strong deformation retract of each $U_n$. Further, if $D_\epsilon$ is the closed disc of radius $\epsilon < 1$ around 0 in $C^2$, $f^{-1} (D_\epsilon)$ is a strong deformation retract of $f^{-1} (D_1)$. In fact, one can retract $f^{-1} (D_1)$ to $f^{-1} (D_\epsilon)$ along a projectable\pageoriginale  vector field on $X$ whose projection to $Y$ is an inward radial vector field around 0. Now, we have
$$
D_1 \supset U_p \supset D_\epsilon \supset U_q
$$
for some integers $p$, $q>0$ and some $\epsilon$ with $0 < \epsilon < 1$. Since $U_p \supset U_q$ and $D_\epsilon \subset D_1$ are homotopy equivalences, the inclusion $U_q \hookrightarrow D_1$ induces isomorphisms of homotopy groups and is hence a homotopy equivalence. This proves the assertion.

Now, it is not difficult to show that $f^{-1} (D_1 - (0)) \to D_1 - (0)$ is a holomorphic fibre bundle with fibre the Riemann sphere and structure group the projective group $PGL(1,\mathbb{C})$. Since $D_1 - (0)$ is of the homotopy type of $S^3$ and $\pi_2 (PGL (1, \mathbb{C})) \approx \pi_2 (SL (2, \mathbb{C})) \approx \pi_2 (SU (2)) = \pi_2 (S^3) = (e)$, this fibration is topologically trivial. Let $M$ be the compact manifold $f^{-1} (D_1)$ with boundary $\partial M$ homeomorphic to $S^2 \times S^3$. Now, $f^{-1} (0)$ is the union of a number of irreducible one-dimensional compact analytic sets. Let $v$ be this number. Then by  Lefschetz duality, we have (with $F = f^{-1} (0)$\footnote{All homologies, cohomologies are with integer coefficients. Further, $H^*_c$ denotes cohomology with compact support.})
$$
Z^v \approx H_2 (F) \approx H_2 (M) \approx H^4 (M, \partial M) \approx H^3 (\partial M) \approx Z \Rightarrow v = 1,
$$
so that $f^{-1} (0)$ is irreducible. Also,
$$
H_1  (F) \approx H_1(M) \approx H^5 (M, \partial M) \approx H^4 (\partial M) = (0),
$$
so that $F$ is locally irreducible (i.e., without nodes) at every point and the normalisation of $F$ is the Riemann sphere.

Let $\fm$ be the sheaf of ideals on $Y$ defining the origin 0, and denote by $F = f^{-1} (0)$ the ``correct'' fibre with structure sheaf $\mathscr{O}_F = \mathscr{O}_X / \fm \mathscr{O}_X$. Since $\fm$ is defined by two elements and $F$ is of dimension one by the unmixedness theorem of Macaulay, the sheaf of ideals $\fm\mathscr{O}_X$ has no embedded components at any point, hence is primary with radical the sheaf of prime ideals $\fp$ defining the reduced fibre. Thus, if $\fm \mathscr{O}_X \neq \fp$, that is if $F$ is not reduced, we can find a series of ideals $\mathscr{O}_F = \fa_0 \supset \fa_1\supset \ldots \supset \fa_m = (0)$   such that each quotient $\fa_i / \fa_{i+1}$ is an $\mathscr{O}_X / \fp$-module and 
$$
\sum\limits_{i} \text{ Generic rank } \mathscr{O}_X/ \fp ~~(\fa_i, \fa_{i+1}) =r > 1.
$$

We now\pageoriginale  recall some ``well-known'' definitions. Let $X,$ $Y$ be connected complex manifolds and $f: X \to Y$ a holomorphic map. Let $Z$ be an irreducible closed analytic subset of $Y$ of codimension $p$ such that $f^{-1}(Z)$ is pure of codimension $p$. One then defines the {\em analytic cycle} ({\em i.e.,} formal linear combination of closed irreducible analytic subsets) $f^*(Z)$ in $X$ to be $\sum n_i C_i$ where $C_i$ are the components of $f^{-1} (Z)$ and $n_i$ is defined as follows. Let $\fq$ be the defining ideal of $Z$. Then the sheaves $Tor_r^{\mathscr{O}_Y} (\mathscr{O}_X, \mathscr{O}_Y/ \fq)$ are coherent sheaves on $X$ with support in $f^{-1} (Z)$. On an open subset of $C_i$, we can find coherent sheaves $F_{pj}$,
$$
Tor_r^{\mathscr{O}_Y} (\mathscr{O}_X, \mathscr{O}_Y/\fq) = F_{r0} \supset F_{r1} \supset \ldots \supset F_{rn_r} = (0)
$$
such that $F_{rj}/F_{rj+1}$ is an $\mathscr{O}_{C_i}$-module. Define the integer $n_{ir}$ as 
$$
n_{ir} = \sum \text{ ~~Generic rank on~~ } C_i  \text{ ~~of~~ } F_{rj}/F_{rj+1}
$$
and put 
$$
n_i = \sum (-1)^r n_{ir}
$$

Next, let $X$ be a connected complex manifold and $Z$ a {\em compact} irreducible analytic set on $X$ of dimension $r$. Then there is a unique generator $\xi \in H_{2r} (Z) \approx Z$ determined by the orientation of $Z$, since $Z$ can be triangulated and is an oriented compact connected pseudomanifold of dimension $2r$ (see Seifert-Threlfall \cite{art7-key3}). Let $\xi'$  be the image of $\xi$ in $H_{2r} (X)$ and $Cl (Z) \in H^{2(n-r)}_c (X)$, the Poincar\'e dual of $\xi'$. We call $Cl(Z)$ the cohomology class (with compact support) associated to $Z$. If $z^p_c(X)$ denotes the free abelian group on the irreducible compact analytic subsets of codimension $p$, we obtain a homomorphism $Cl: z^{p}_c(X) \to H^{2p}_c(X)$ on extending by linearity.

We then have the following (acceptable, ``well-known'' but nowhere explicitly proved) result (See remark at the end of the paper).

{\em If $f: X \to Y$ is a proper holomorphic map and $Z$, a compact analytic irreducible subset of $Y$ of codimension $p$ such that $f^*(Z)$ is defined, we have}
$$
H^{2p}_c (f) (Cl(Z)) = Cl (f^* (Z))
$$

Now,\pageoriginale let us return to our original situation. Let $\dot{D}_1$ be the interior of $D_1$ and $P$ any point of $\dot{D}_1$. Then $Cl (P) \in H^4_c (\dot{D}_1)$ is the fundamental cohomology class with compact support of $\dot{D}_1$. Put $\dot{M} = M - \partial M$, so that $f|\dot{M} : \dot{M} \to \dot{D}_1$ is a proper holomorphic map. Since $f$ is flat, $Tor^{\mathscr{O}_Y}_i (\mathscr{O}_X, \mathscr{O}_Y/\fm) = (0)$ for $i > 0$. Thus, if $\fm {\mathscr{O}}_X \neq \fp$, we see that $f^*(0)$ is the cycle $r F_0$ where $r$ is an integer $>1$ and $F_0$ is the (reduced) fibre. By the result stated, $H^4_c(f) (\xi) = r.Cl (F_0)$ with $r >1$. Now, consider the following commutative diagram
\[
\xymatrix{
Z \approx H^3 (\partial M) \ar[r]^{{}_\sim^{\beta_1}} &  H^4 (M, \partial M) & H^4_c (\dot{M}) \ar[l]_-{{}^{\beta_2}_{\sim}}\\Z \approx H^3 (S^3) \ar[r]^{{}^{\alpha_1}} \ar[u]^{\wr}_{f^*_1} & H^4 (D_1, S^3) \ar[u]_{f^*_2} & H^4_c (\dot{D}_1) \ar[l]_>>>>>>{{}_\sim^{\alpha_2}} \ar[u]_{f^*_3}
}
\]
where $f^*_1$ are the maps induced by $f$ in cohomology. Here, $\alpha_1$, $\alpha_2$ and $\beta_2$ are well-known to be isomorphisms, and $f^*_1$ is an isomorphism since $\partial M \to S^3$ is a trivial fibration with fibre $S^2$. Finally, $\beta_1$ is also an isomorphism since $H^3 (M) = H^4 (M) = (0)$. It follows that $f^*_3$ is an isomorphism. Hence it cannot take a generator $\xi$ of $H^4_c (\dot{D}_1)$ to an $r$-th multiple of an element of $H^4_c (M)$ where $r >1$.

This contradiction shows hat $\fm {\mathscr{O}}_X =\fp$, so that $F$ is a reduced (and irreducible) analytic set of dimension one. Since $\chi (\mathscr{O}_{f^{-1}(z)}) =1$ for $z \in Y$, $z \neq 0$ and $f$ is flat, we have $\chi(\mathscr{O}_F) =1$ and $H^1 (F, \mathscr{O}_F) = (0)$. This implies that $F$ is smooth (since singularities increase the arithmetic genus). By what was said earlier, $f$ is of maximal rank along $F$ which is a contradiction to our assumption that 0 is a critical value of $f$. The theorem is proved.
\end{proof}

\begin{remark*}
The result stated about the naturality of $Cl$ with respect to proper maps is certainly reasonable, but nowhere explicitly proved to our knowledge. Borel-Haefliger \cite{art7-key4} associate cohomology classes (with arbitrary support) to arbitrary (not necessarily compact) cycles, and prove naturality. Presumably their methods would also prove the above result.
\end{remark*}

However,\pageoriginale we prefer to give an $ad $ $hoc$ proof in our situation.

Let $P$ be a point of $f^{-1} (0)$ where the reduced fibre $F_{\red}$ is smooth and choose a locally closed submanifold $Z$ of dimension 2 through $P$ transversal to the reduced fibre having $P$ as the unique point of intersection with $F$. By shrinking $D_1$ if necessary, we may assume that $Z \cap M$ is proper over $D_1$ with finite fibres. Since $Z$ and $F$ have intersection number one in the topological sense (Seifert-Threlfall \cite{art7-key3}), using the well-known relationship between intersection of cycles and cup-product, we see that if $\alpha \in H^4_c (M) \approx H^4 (M, \partial M)$ is the class associated to the reduced fibre and $\zeta \in H_4 (M, \partial M)$ the cycle defined by $Z$, we have $\langle \alpha, \zeta \rangle =1$. Now, the  generator $\xi$ of $H^4 (D_1, S^3) \approx H^4_c (\dot{D}_1)$ can be represented by a real 4-form $\omega$ with compact support in $\dot{D}_1$ and 
$$
\int\limits_{\dot{D}_1} \omega = 1. 
$$
Since we have shown that $H^4 (D_1, S^3) \to H^4 (M, \partial M)$ is an isomorphism, $f^*(\omega)$ represents a generator of $H^4_c (\dot{M}) \approx H^4 (M, \partial M)$. Thus, we see that $\alpha$ is the $\mu$-th multiple of the class of $f^* (\omega)$ in $H^4 (M, \partial M)$ with $\mu \in Z$. We thus obtain
$$
1 = \langle \alpha, \zeta \rangle = \mu \int\limits_{Z} f^*(\omega).
$$
Now, if $\fm\mathscr{O}_X \neq \fp$, it is easy to see that $\fm$ together with the principal ideal in $\mathscr{O}_{P,X}$ defining $Z$ cannot generate the maximal ideal of $\mathscr{O}_{P,X}$. Hence, $f| Z: Z \to Y$ is not an isomorphism at $P$. By local analytic geometry, there is a neighborhood of 0 in $Y$, which we may assume to be $\dot{D}_1$, such that outside of a proper analytic subset of this neighborhood, $f|Z$ is a covering of degree $r >1$. Hence,
$$
1 = \mu \int\limits_{Z \cap \dot{M}} f^* (\omega) = \mu r \int\limits_{\dot{D}_1} \omega = \mu r
$$
which shows that $\mu = r =1$, a contradiction. This completes the proof.

\vskip 0.8cm

\noindent
\textbf{Addendum.} It was\pageoriginale pointed out to us in a letter by R. R. Simha that the appeal to the theorem on the continuity of the Euler characteristic at the end of the proof of the theorem is unnecessary, since, once we know that the special fibre is reduced, the set of points where $f$ is not of maximal rank is of codimension $\geqslant 3$, and hence is empty.

This enables us to extend the proof also to the case when the general fibre is a non-singular curve of arbitrary genus. In fact, in the notation of the above proof, we have only to show that $H^4_c (\dot{D}_1) \to H^4_c (\dot{M})$ is surjective (hence an isomorphism, since Rank $H^4_c (\dot{M})$ is $\geqslant 1$). The rest of the proof does not utilise the genus zero assumption. By the Wang sequence of the fibration $\partial M \to S^3$, we see that $H^3 (S^3) \to H^3 (\partial M)$ is surjective and since $H^4 (M) = 0$, $H^1 (\partial M) \to H^4 (M, \partial M) = H^4_c (\dot{M})$ is also surjective. This proves the assertion, and we get a uniform proof for all genera.

Added in Proof (Nov. 16, 1970)

The author is informed that the result has also been proved in the case of algebraic varieties when the genus of the general fibre is zero, by I. Dolgacev and M. Raynaud.

The following very simple counter-example to the conjecture when the dimension of the fibre is $\geqslant 2$ was pointed out by Professor David Mumford.

Let $X$ be a compact non-singular surface, $x_0 \in X$, $p_2 : X \times X \to X$ the second projection, and $\sum_1$ and $\sum_2$ the sections $\{x_0\} \times X$ and $\Delta$ the diagonal. Let $I$ be the sheaf of ideals defining $\sum_1 \cup \sum_2$. Blow up $X \times X$ with respect to the sheaf of ideals $I$, to obtain $Y$. Denote the composite $Y \xrightarrow{\sigma} X \times X \xrightarrow{p_2} X$ by $f$. Then $Y$ non-singular outside of $\sigma^{-1} (x_0, x_0)$ and $f$ is of maximal rank outside $\sigma^{-1} (x_0, x_0)$. If we can show that (i) $Y$ is non-singular, and (ii) $\sigma^{-1} (x_0, x_0)$ is of dimension 2, it would follow that $f^{-1} (x_0)$ has at least two components, and hence $f$ cannot be regular all along $f^{-1} (x_0)$.

To check these statements, we might as well blow up $\mathbb{C}^4$ with respect to the sheaf of ideals defined by $\{z_1 = z_2 = 0\} \cup \{z_3 = z_4 =0 \}$ i.e. with\pageoriginale respect to the sheaf of ideals generated by $\{z_1 z_3, z_1 z_4, z_2 z_3, z_2 z_4\}$. The resulting space is covered by four open subsets, all of which are isomorphic, a typical one being the affine algebraic variety with co-ordinate ring
$$
\mathbb{C} \left[z_1, z_2, z_3, z_4, \dfrac{z_4}{z_3}, \dfrac{z_2}{z_1}, \dfrac{z_2 z_4}{z_1 z_3} \right] = \mathbb{C} \left[ z_1, z_3, \dfrac{z_4}{z_3} , \dfrac{z_2}{z_1}\right] 
$$ 
which is isomorphic to $\mathbb{C}^4$. The intersection of this open set with the fibre over the origin is given by $z_1 = z_3 = 0$, hence is 2-dimensional.


\begin{thebibliography}{99}
\bibitem{art7-key1} I. Dolgacev: On the purity of the degeneration loci of families of curves. {\em Inv. Math.} 8(1969).

\bibitem{art7-key2} R. R. Simha: Uber die kritischen Werte gewisser holomorpher Abbildungen, {\em Manuscripta Mathematica} Vol. 3, Fasc, 1 (1970), 97-104.

\bibitem{art7-key3} Seifert und Threlfall: {\em Lehrbuch der Topologie,} Chelsea reprint.

\bibitem{art7-key4} A. Borel and A. Haefliger: La classe d'homologie fondamentale d'un \'espace analytique, {\em Bull. Soc. Math. France } 89 (1961).
\end{thebibliography}

\noindent
Tata Institute of Fundamental Research\\
Bombay 5, India.

\vfill\eject
~\phantom{a}
\thispagestyle{empty}
