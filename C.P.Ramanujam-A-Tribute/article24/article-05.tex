\title{Geometry of Hecke Cycles - I}\label{art05}
\markright{Geometry of Hecke Cycles - I}

\author{By~ M.S.~Narasimhan and S. Ramanan}
\markboth{M.S. Narasimhan and S. Ramanan}{Geometry of Hecke Cycles - I}

\date{}
\maketitle

\setcounter{page}{341}

\setcounter{pageoriginal}{290}
\section{Introduction}\label{art05-sec1}\pageoriginale

In the course of our study of vector bundles over a smooth projective
curve we considered \cite{art05-key5} a correspondence between the
space vector bundles of rank $n$ and degree $d$ on the one hand and
that of vector bundles of rank $n$ and degree $(1-d)$ on the
other. This is the geometric analogue of the Hecke Correspondence
which is defined in the case of a curve defined over a finite field. Let
$\xi$ be a line bundle on the curve $X$ and $U(n,\xi)$
(resp.~$U(n,\xi^{-1}X)$) be the moduli space of vector bundles on $X$
of rank $n$ with determinant isomorphic to $\xi$ (resp.~isomorphic to
$\xi^{-1}\otimes L_{x}$, $x\in X$), $n\geq 2$. For a general vector
bundle $E\in U(n,\xi)$, the subscheme corresponding to $E$ in
$U(n,\xi^{-1}X)$ under this correspondence is isomorphic to the
projective bundle $P(E^{*})$. We call these subschemes of
$U(n,\xi^{-1}X)$ good Hecke cycles. This identifies a suitable open
subset of $U(n,\xi)$ with an open subset of the Hilbert scheme of
$U(n,\xi^{-1}X)$. Results of this type were announced
in  [\ref{art05-sec5}\S~8] and are proved here in \S~\ref{art05-sec5},
which can be read independently of the rest of the paper.

In the case $n=2$, and $\xi$ is trivial we prove that the irreducible
component (which we shall call for convenience the Hecke component) of
the Hilbert scheme of $U(2,X)$ containing the good Hecke cycles is
smooth and provides a non-singular model for $U(2,\xi)$. This is the
main result of the paper (Theorem 8.14).

Now the Kummer variety associated to the Jacobian $J$ of $X$ is the
set of non-stable (and even singular if the genus $g\geq 3$) points of
$U(2,\xi)$. The possible elements of the Hecke component corresponding
to points of the Kummer variety can be listed (\S~\ref{art05-sec7})
and they all turn out to be conic bundles over $X$. The fibre in the
non-singular model over a non-nodal point $k$ of the Kummer variety is
isomorphic to $PH^1(X,j^{2})\times PH^{1}(X,j^{-2})$, where $j\in J$
lies above $k$, and the corresponding subschemes in $U(2,X)$ can be
described as the union of two projective line bundles on $X$
corresponding to non-trivial extensions
\begin{align*}
& 0\to j^{-1}\to E\to j\to 0\\
& 0\to j\to E'\to j^{-1}\to 0
\end{align*}\pageoriginale
identified along the sections given by $j^{-1}$ and $j$
respectively. Over a node of the Kummer variety, the elements are
trivial conic bundles contained in $X\times
P(H^{1}(X,\mathscr{O}))_{t}$ (imbedded in $U(2,X)$), where
$P(H^{1}(X,\mathscr{O}))_{t}$ is the thickening of
$PH^{1}(X,\mathscr{O})$ corresponding to the universal quotient bundle
$Q$ of $PH^{1}(X,\mathscr{O})$. (See \ref{art05-rems4.4} iii). Thus we get here not
only conics contained in $PH^{1}(X,\mathscr{O})$ but also lines in it
which are thickened within this $Q^{*}$-thickening. (The latter will
be referred to as `outside thickenings'). It is proved that the
Hilbert scheme is itself smooth at all these points except at 1) a
pair of intersecting lines in $PH^{1}(X,\mathscr{O}),2)$ double lines
contained schematically in $PH^{1}(X,\mathscr{O})$
(\S~\ref{art05-sec8}). At points of the above type another component
of the Hilbert scheme hits the Hecke component. We show that there is
a natural morphism of the union of these two components into the
Jacobian of $X$, which is constant on the Hecke component. (This
morphism should be thought of as playing the role of the Weil morphism
into the intermediary Jacobian). By studying the differential of this
morphism we show that the Hecke component is smooth also at these
points. (See Lemmas \ref{art05-lem8.10}, \ref{art05-lem8.11}).

The conics contained schematically in $PH^{1}(X,\mathscr{O})$ is a
$\bfP^{5}$ bundle over the Grassmannian of planes in
$PH^{1}(X,\mathscr{O})$. It will be shown in a later paper that the
non-singular model considered above can be blown down along these
fibrations (one for each node) to another non-singular model.

It turns out that all the subschemes described above consist of
bundles which are non-trivial extensions of line bundles of a
particular kind. Such (triangular) bundles are parametrised by a
projective bundle over $X\times J$. We have a morphism from this space
into $U(2,X)$, which may also be looked upon as a family
$\{P(D_{j})\}_{j\in J}$ of smooth subvarieties of $U(2,X)$
parametrised by $J$, where each $P(D_{j})$ is a projective bundle over
$X$ of dimension $(g-1)$. One of the essential points in the proof is
to study the first and second order differentials of this morphism and
in particular to compute its Hessian\pageoriginale at the critical
points (\S~\ref{art05-sec6}). The necessary preliminaries for this are
discussed in \S~\ref{art05-sec2}. This study is necessary to prove the
non-singularity of the Hilbert scheme at a point given by an `outside
thickening'. For this we need information about the conormal sheaf of
the thickening $X\times PH^{1}(X,\mathscr{O})_{t}$ of $X\times
PH^{1}(X\mathscr{O})$ in $U(2,X)$. Now this thickening is a special
case of thickenings which arise in the study of a family of smooth
subvarieties (in our case, the family $j\mapsto P(D_{j})$ mentioned
above). In this situation the conormal sheaf of the thickened scheme
can be described in terms of the Hessian (see Lemma 3.7 and Remark
3.9).

A different approach for the desingularisation of $U(2,\xi)$ has been
found by C.S.~Seshadri \cite{art05-key10}.

\medskip
\noindent
{\bf Notation.}~
All schemes will be of finite type over an algebraically closed field
$k$ of characteristic $\neq 2$. From \S~\ref{art05-sec5} on, $X$ will
denote a smooth projective curve of genus $g\geq 2$. If $S$ is a
subscheme of $\Pic X$, $U(n,S)$ will denote the subscheme of the
moduli scheme $U(n,d)$ of $S$-equivalence classes of semistable vector
bundles of rank $n$ of degree $d$ obtained as the inverse image of $S$
by the morphism $\det : U(n,d)\to \Pic$. The Jacobian of $X$ will be
denoted $J$. If $E$ is a family of semistable vector bundles over $X$,
then there is a canonical morphisms $\theta_{E}$ of the parameter
space into the moduli space.

If $X$ is a subscheme of $Y$, we denote by $\check{N}_{X,Y}$ the
conormal sheaf of $X$ in $Y$. In good cases, e.g. when $X$ is
regularly imbedded in $Y$, the sheaf $\check{N}_{X,Y}$ is locally free
and its dual is the normal bundle denoted $N_{X,Y}$ so that in that
case, $\check{N}_{X,Y}=N^{*}_{X,Y}$. If $\pi:X\to Y$ is a smooth
morphism, we denote by $T_{\pi}$, the tangent bundle along the fibres
of $\pi$. Its dual is sometimes denoted by $\Omega^{1}_{\pi}$ as
well. If $x$ is a (closed) point of $X$, the tangent space at $x$ is
denoted $T_{x}$. 

If $D$ is a Cartier divisor in a scheme $X$, then $L_{D}$ will denote
the line bundle defined by it. If $L$ is a line bundle generically
generated by sections, then the quotient sheaf of $H^{0}(X,L)^{*}_{X}$
by the subsheaf $L^{*}$ will be called the quotient sheaf of the
linear system defined by $L$. If $\mathfrak{F}$ is a coherent sheaf on
a closed subscheme $i:Y\to X$, the sheaf $i_{*}(F)$ will be denoted by
$\widetilde{\mathfrak{F}}$. 

If\pageoriginale $\pi:X\to Y$ is a projective morphism with a relatively ample sheaf
then $\Hilb(X,Y,P)$ will denote the relative Hilbert scheme over $Y$
with Hilbert polynomial $P$. If $Y=\Spec k$, we simply write
$\Hilb(X,P)$.

The projections from $X\times Y\times Z$ to the component schemes will
be denoted $p_{1}$, $p_{2}$, $p_{3}$, $p_{12}$, $p_{23}$,
$p_{1,3}$. When we are dealing with closed subschemes or closed
imbeddings we usually omit the word ``closed''.

\section{Deformations of Principal bundles}\label{art05-sec2}

We wish to recall certain facts concerning deformations of locally
trivial principal $G$-bundles, where $G$ is an algebraic group, and in
particular study the second order infinitesimal deformations of such
bundles. 

\begin{definition}\label{art05-defi2.1}
Let $P$ be a principal $G$-bundle on $X$ and $(S,s_{0})$ a pointed
scheme. A deformation of $P$ parametrised by $S$ is a principal
$G$-bundle $Q$ over $X\times S$, and an ismorphism $\Psi$ of
$Q|X~\times~{s_{0}}$ with $P$. Two deformations $Q_{1}$, $Q_{2}$ are
said to be equivalent if for every $s\in S$, there exists a
neighbourhood $U$ and an isomorphism $f:Q_{1}|X\times U\to
Q_{2}|X\times U$ such that the diagram
\[
\xymatrix@=1.5cm{
Q_{1}\ar[r]^-{\Psi_{1}}\ar[d]_-{f|X\times s_{0}} & P\\
Q_{2}\ar[ur]_-{\Psi_{2}} &
}
\]
commutes if $s_{0}\in U'$.
\end{definition}


\begin{subremark}\label{art05-rem2.2}
\begin{itemize}
\item[(i)] If the only automorphisms of $P$ are given by morphisms of
$X$ into the centre of $G$, then it is clear that the equivalence
class of a deformation $Q$ of $P$ is independent of $\Psi$.

\item[(ii)] If $P$ is a principal $G$-bundle on $X$, then we have
obviously the deformation functor $\delta_{P}$ on the category of
Artinian local algebras which associates to $A$, the equivalence class
deformations of $P$ parametrised by $A$. This functor can be seen to
satisfy the conditions $H_{1}$ and $H_{2}$ of
Schlessinger \cite[Theorem 2.11]{art05-key7}. If $X$ is proper $k$, it
satisfies $H_{3}$ as well. Moreover, if $P$ is such that
$H^{0}(X,Z_{X})\to H^{0}(X,\Ad P)$ is an isomorphism, where $Z$
denotes the centre of the Lie algebra of $G$, then $\delta_{P}$ also
satisfies $H_{4}$ and hence is prorepresentable. We proceed to
describe this functor a little more explicitly.
\end{itemize}
\end{subremark}

\begin{subprop}\label{art05-prop2.3}
Let\pageoriginale $S$ be an Artinian local algebra, and $s_{0}$ its
closed point. Then $\delta_{P}(S)$ is canonically bijective with the
set $H^{1}(X,N)$ where $N$ is the sheaf associated to the group scheme
$N=\ker p_{*}p^{*}G(\overline{P})\to G(\overline{P})$, where this map
is given by restriction to $X\times s_{0}$, $G(P)$ is the group scheme
$\Aut P$ and $p:X\times S\to X$ is the projection.
\end{subprop}

\begin{proof}
The set of principal $G$-bundles on $X\times S$ are in $(1,1)$
correspondence with the set $H^{1}(X\times S,p^{*}G(P))\simeq
H^{1}(X,p_{*}p^{*}G(P))$. Thus any deformation of $P$ gives rise to an
element in $H^{1}(X,p_{*}p^{*}G(P))$ which is in the `kernel' of
$H^{1}(X,p_{*}p^{*}G(P))\to H^{1}(X,G(P))$. But the prescription of
$\Psi$ allows one to describe the deformation as a 1-cocycle for
$p_{*}p^{*}G(P)$ and a 0-cochain for $G(P)$ of which its image in
$G(P)$ is the coboundary. This proves the assertion.
\end{proof}

\begin{subprop}\label{art05-prop2.4}
Let $(A,\bfm)$ be an Artinian local algebra with $\bfm^{2}=0$
(resp. $\bfm^{3}=0$). Then the group scheme $N$ of $2.3$ is isomorphic
to the vector bundle $\Ad P\otimes \bfm$ (resp. the group scheme $\Ad
P\otimes \bfm$, the group structure being defined by $(X,Y)\to
X+Y+\frac{1}{2}[X,Y]$). 
\end{subprop}

\begin{proof}
The exponential map yields a bijection of $\Ad P\times \mathfrak{m}$ onto $N$,
and the transfer of the group structure on $N$ to $\Ad P\otimes \bfm$
is as given above, by virtue of the Campbell-Hausdorff
formula \cite[SGA 3, Expose VII, 3.1]{art05-key9}.
\end{proof}

\begin{subremark}\label{art05-rem2.5}
If the field $k$ is of characteristic 0, we have an obvious
generalisation of \ref{art05-prop2.4} to all Artinian local algebras using the
Campbell-Hausdorff formula.
\end{subremark}

\begin{example}\label{art05-exam2.6}
Let $A$ be the algebra
$k[\epsilon_{1},\epsilon_{2}]/(\epsilon^{2}_{1},\epsilon^{2}_{2})$. Then
$H^{1}(X,\Ad P\otimes \bfm)$ can be described with respect to a
suitable open covering $(U_{i})$ of $X$ as follows. Any 1-cochain can
be described by
$f_{ij}\epsilon_{1}+g_{ij}\epsilon_{2}+h_{ij}\epsilon_{1}\epsilon_{2}$
where $f_{ij}$, $g_{ij}$, $h_{ij}$ belong to $H^{0}(U_{i}\cap
U_{j},\Ad P)$. The cocycle condition with respect to the above group
structure on $\Ad P\otimes \mathfrak{m}$ is that, for every $i$, $j$, $k$, we
have
{\fontsize{10pt}{11pt}\selectfont
\begin{align*}
&
f_{ik}\epsilon_{1}+g_{ik}\epsilon_{2}+h_{ik}\epsilon_{1}\epsilon_{2}=(f_{ij}\epsilon_{1}+g_{ij}\epsilon_{2}+h_{ij}\epsilon_{1}\epsilon_{2})(f_{jk}\epsilon_{1}+g_{jk}\epsilon_{2}+h_{jk}\epsilon_{1}\epsilon_{2})\\
&\quad
=(f_{ij}+f_{jk})\epsilon_{1}+(g_{ij}+g_{jk})\epsilon_{2}+\{\frac{1}{2}([f_{ij},g_{jk}]+[g_{ij},f_{jk}])+h_{ij}+h_{jk}\}\epsilon_{1}\epsilon_{2} 
\end{align*}}
or, what is the same, $f_{ij}$, $g_{ij}$ are 1-cocycles of $\Ad P$,
and $h_{ij}$ satisfies
\setcounter{equation}{6}
\begin{equation}
h_{ij}+h_{jk}=h_{ik}-\frac{1}{2}([f_{ij},g_{jk}]+[g_{ij},f_{jk}]).\label{art05-eq2.7} 
\end{equation}
On\pageoriginale the other hand, two cocycles $(f_{ij},g_{ij},h_{ij})$ and $(f'_{ij},g'_{ij},h'_{ij})$ are cohomologous if and only if there exist sections $(a_{i},b_{i},c_{i})$ over $U_{i}$ of ad $P$ with
\begin{align*}
& (f_{ji}\epsilon_{1}+g_{ij}\epsilon_{2}+h_{ij}\epsilon_{1}\epsilon_{2})(a_{j}\epsilon_{1}+b_{j}\epsilon_{2}+c_{j}\epsilon_{1}\epsilon_{2})\\
&\qquad\qquad = (a_{i}\epsilon_{1}+b_{i}\epsilon_{2}+c_{i}\epsilon_{1}\epsilon_{2})\times \times (f'_{ij}\epsilon_{1}+g'_{ij}\epsilon_{2}+h'_{ij}\epsilon_{1}\epsilon_{2}); 
\end{align*}
namely, $f'_{ij}-f_{ij}$ is the coboundary of $(a_{i})$ in $\Ad P$, $g'_{ij}-g_{ij}$ is the coboundary of $(b_{i})$, and 
\begin{equation}
h'_{ij}-h_{ij}=c_{j}-c_{i}+\frac{1}{2}([f_{ij},b_{j}]+[g_{ij},a_{j}]-[b_{i},f'_{ij}]-[a_{i},g'_{ij}])\label{art05-eq2.8}
\end{equation}
\end{example}

\setcounter{subsection}{8}
\subsection{Hessian.}\label{art05-sec2.9}
Let $f$ be a morphism of a smooth scheme $X$ into a scheme $Y$ and
$df:T_{x}\to T_{y}$, $y=f(x)$ be its differential. Then the {\em
Hessian} $h(f)$ of $f$ is defined to be a map $\ker df\otimes
T_{x}\to \coker df$. It is more convenient to define the dual map
$\check{h}(f):\ker (\check{df})\otimes
T_{x}\to \coker \check{df}$. Let  $(\mathscr{O}_{x},\bfm_{x})$,
$(\mathscr{O}_{y},\bfm_{y})$ be the local rings at $x$, $y$. If
$a\in \bfm_{y}$, with $a\circ f\in \bfm^{2}_{x}$ and $t\in T_{x}$,
then we get an element of $T_{y}$, by contracting with $t$ the element
in $S^{2}(\bfm_{x}/\bfm^{2}_{x})=\bfm^{2}_{x}/\bfm^{3}_{x}$ given rise
to by $a\circ f$. Its image in $\coker df$ depends only on the class
of $a$ modulo $\mathfrak{m}^{2}_{y}$ and is defined to be
$\check{h}(f)(a,t)$. 
 
\subsection{Functorial Description of Hessian.}\label{art05-sec2.10}

In terms of $A$-valued points, the Hessian can be described as
follows. Consider the ring
$$
A=k[\epsilon_{1},\epsilon_{2}]/(\epsilon^{2}_{1},\epsilon^{2}_{2})
$$
and the quotient rings $B_{1}$, $B_{2}$ and $C$ given by the ideals
$(\epsilon_{2})$, $(\epsilon_{1})$ and
$(\epsilon_{1}\epsilon_{2})$. Giving vectors $a$, $t$ in $\ker df$ and
$T_{x}$ respectively is the same as giving a $C$-valued point $p$ of
$X$ at $x$ such that the corresponding $B_{1}$-valued point is mapped
by $f$ on the 0-vector at $y$. Let $q$ be an $A$-valued point
extending $p$; there exists one such since $X$ is smooth. By
assumption, the image $f(q)$ actually yields a
$(k+k\epsilon_{2}+k\epsilon_{1}\epsilon_{2})$-valued point at $y\in
Y$. By restriction to $\Spec (k+k\epsilon_{1}\epsilon_{2})$, we get a
vector at $y$. Its image modulo Image $(df)$ is independent of the
extension $q$ of $p$ and gives $h(f)(a,t)$.

\setcounter{theorem}{10}
\begin{subremark}\label{art05-rem2.11}
The above definition of Hessian in terms of $A$-valued points enables
one to define the Hessian of a morphism of a smooth, prorepresentable
functor on the category of Artinian local algebras into another
prorepresentable functor. In particular, if\pageoriginale $P$ is a
principal $G$-bundle with $H^{2}(X,\Ad P)=0$, then it is easy to check
that the functor $\delta_{P}$ defined in 2.2, (ii) is smooth. Hence if
$P$ satisfies in addition the conditions of 2.2, (ii), then for any
homomorphism of $G$ into another algebraic group $H$, the notion of
Hessian of the morphism $\delta_{P}\to \delta_{Q}$ makes sense, where
$Q$ is the principal $H$-bundle associated to $P$. We proceed to
compute this in the case when $G\subset H$.
\end{subremark}

\begin{subprop}\label{art05-prop2.12}
\begin{itemize}
\item[\rm(i)] Let $P$ be a principal $G$-bundle on a\break scheme $X$,
proper over $k$. For  any homomorphism $G\to H$, the induced morphism
$e_{G,H}:\delta_{P}\to \delta_{Q}$ of the deformation functors has as
differential, the natural map $H^{1}(X,\Ad P)\to H^{1}(X,\Ad Q)$,
where $Q$ is the principal $H$-bundle associated to $P$.

\item[\rm(ii)] If $H^{2}(X,\Ad P)=0$ and $G\subset H$, then the
cokernel of $de_{G,H}$ is canonically isomorphic to $H^{1}(X,E)$ where
$E$ is the vector bundle associated to $P$ for the linear isotropy
action of $G$ on the tangent space at $e$ of $H/G$.

\item[\rm(iii)] Assume that $Q$ (and hence $P$) satisfies the
condition in 2.2, {\rm (ii)}. Under the identification in {\rm(ii)},
the Hessian of $e_{G,H}$ can be described as follows. Let
$t_{1}\in \ker de_{G,H}$ and $t_{2}\in H^{1}(X,\Ad P)$. Then $t_{1}$
is in the image under the boundary homomorphism of an element
$s_{1}\in H^{0}(X,E)$ associated to the exact sequence
$$
0\to \Ad P\to \Ad Q\to E\to 0.
$$
\end{itemize}
Then $h(t_{1},t_{2})=[s_{1},t_{2}]$, where the bracket is the cup
product associated to the natural action of $\Ad P$ on $E$.
\end{subprop}

\begin{proof}
\begin{itemize}
\item[(i)] is obvious from 2.4.

\item[(ii)] is a trivial consequence of (i) and the cohomology
sequence of $0\to \Ad P\to \Ad Q\to E\to 0$.

\item[(iii)] We will use the notation of 2.10. By the description of
the Hessian given there and of the functors $\delta_{P}$, $\delta_{Q}$
given in 2.4, $h(t_{1},t_{2})$ is obtained as follows. The vectors
$t_{1}$, $t_{2}$ give an element of $H^{1}(X,\Ad
P\otimes \mathfrak{m}/(\epsilon_{1},\epsilon_{2}))$ where
$\mathfrak{m}$ is the maximal ideal of $A$. This is the image of an
element in $H^{1}(X,\Ad P\otimes \mathfrak{m})$, the group structure
in $\Ad P\otimes \mathfrak{m}$ being given in 2.4. Its image in
$H^{1}(X,\Ad Q\otimes \mathfrak{m})$ goes to\pageoriginale zero in
$H^{1}(X,\Ad Q\otimes \mathfrak{m}/(\epsilon_{2}))$ and hence comes
from an element in $H^{1}(X,\Ad Q\otimes (\epsilon_{2}))$. The image
of this element in $H^{1}(X,E\otimes (\epsilon_{2})/(k\epsilon_{2}))$
is $h(t_{1},t_{2})$. In terms of cocycles for a suitable covering
$(U_{i})$ of $X$ (as in 2.6) this means the following. Let $f_{ij}$,
$g_{ij}$ be cocycles representing $t_{1}$, $t_{2}$, in $H^{1}(X,\Ad
P)$. Then there exists $h_{ij}\in H^{0}(U_{i}\cap U_{j},\Ad P)$ such
that $(f_{ij},g_{ij},h_{ij})$ satisfy 2.7. Reading these as sections
over $U_{i}\cap U_{j}$ of $\Ad Q$, we get a corresponding element of
$H^{1}(X,\Ad Q\otimes \mathfrak{m})$. We are given that $f_{ij}$ is a
coboundary for $\Ad Q$, namely, there exist $\lambda_{i}\in
H^{0}(U_{i},\Ad Q)$ with $f_{ij}=\lambda_{j}-\lambda_{i}$. Then the
cocycle $(f_{ij},g_{ij},h_{ij})$ is cohomologous by 2.8 to
$(0,g_{ij},h_{ij}+\frac{1}{2}([\lambda_{i},g_{ij}]-[g_{ij},\lambda_{j}])$,
taking $(a_{i},b_{i},c_{i})=(\lambda_{i},0,0)$. Thus the element in
$H^{1}(X,\Ad Q\otimes (k\epsilon_{2}+k\epsilon_{1}\epsilon_{2}))$ is
given by the cocycle
$g_{ij}\epsilon_{2}+h_{ji}+\frac{1}{2}([\lambda_{1}g_{ij}]-[g_{ij},\lambda_{j}])\epsilon_{1}\epsilon_{2}$. Finally,
the Hessian $h(t_{1},t_{2})$ is given by the cocycle
$\frac{1}{2}([\lambda_{i}g_{ij}]-[g_{ij},\lambda_{j})])$ of $E$, since
$h_{ij}$ is a section of $\Ad P$. Notice that since
$\lambda_{i}-\lambda_{j}$ is a section of $\Ad P$,
$\lambda_{i}=\lambda_{j}$ as sections of $E$ on $U_{i}\cap U_{j}$ and
hence $(\lambda_{i})$ determines a section $\lambda$ of $E$. Clearly
$\lambda$ maps on $(f_{ij})$ under the boundary homomorphism and the
cocycle
$\frac{1}{2}([\lambda,g_{ij}]-[g_{ij},\lambda])=[\lambda,g_{ij}]$
represents the cup product of $\lambda$ and the class of $g_{ij}$ as
was to be proved.
\end{itemize}
\end{proof}

\begin{remarks}\label{art05-rems2.13}
\begin{itemize}
\item[(i)] We will apply Proposition \ref{art05-prop2.12} to the case
when $G$ is $2\times 2$ triangular (Borel) subgroup of $H=GL(2,k)$,
and $P$ is a principal bundle over the curve $X$. Then $P$ is
described by an exact sequence
$$
0\to L_{1}\to W\to L_{2}\to 0,
$$
where $L_{1}$ and $L_{2}$ are line bundles. If $W$ is simple, then the
conditions in Prop. \ref{art05-prop2.12} are satisfied. The bundle
$\Ad P$ is the bundle $\Delta(W)$ of endomorphisms of $W$ leaving
$L_{1}$ invariant, $\Ad Q=\End W$, and the vector bundle $E$ can be
identified with $\Hom(L_{1},L_{2})$. Moreover, the bundle $\Ad P$
consisting of endomorphisms of trace $0$ is isomorphic to
$\Hom(L_{2},W)$. In particular, we have a map $\eta$ of $\Ad P$ onto
the sheaf $\mathscr{O}_{X}$, obtained by composing with the map $W\to
L_{2}$. With these identifications, the action of $\Ad P$ on $E$ is
simply multiplication by its image in $\mathscr{O}$. Thus if
$t_{1}\in \ker H^{1}(X,\Ad P)\to H^{1}(X,\Ad Q)$, and $t_{2}\in
H^{1}(X,\Ad P)$, then $h(t_{1},t_{2})\in H^{1}(X,E)$ is the cup
product of $\eta t_{2}\in H^{1}(X,\mathscr{O})$ and the element
$s_{1}\in H^{0}(X,\Hom(L_{1},L_{2}))$ of which\pageoriginale $t_{1}$ is the
image. In other words, $h(t_{1},t_{2})$ is simply the image of $\eta
t_{2}\in H^{1}(X,\mathscr{O})$ in $H^{1}(X,\Hom (L_{1},L_{2}))$ by the
map $\mathscr{O}\to \Hom(L_{1},L_{2})$ given by $t_{1}$.

\item[(ii)] If global fine moduli schemes for principal $G$ and $H$
bundles exist, then clearly Proposition \ref{art05-prop2.12} enables
one to compute the Hessian of the morphism $e_{G,H}$ induced on
the moduli schemes by extension of structure group.
\end{itemize}
\end{remarks}

\section{Thickenings}\label{art05-sec3}

Let $X$ be a scheme and $\mathscr{F}$ a coherent sheaf of
$\mathscr{O}$-Modules on $X$. Let
$\varphi:\mathscr{F}\to \Omega^{1}_{X}$ be a surjective
$\mathscr{O}_{X}$-homomorphism of $\mathscr{O}_{X}$-Modules. Then one
can define a ring structure on the subsheaf of
$\mathscr{O}\oplus \mathscr{F}$ consisting of $\{(f,x):df=\varphi
x\}$. This is a subsheaf $\mathscr{O}_{\varphi}$ of rings if we
consider $\mathscr{O}\oplus \mathscr{F}$ as an $\mathscr{O}$-Algebra
with $\mathscr{F}^{2}=0$.

\begin{lemma}\label{art05-lem3.1}
$(X,\mathscr{O}_{\varphi})$ is a scheme.
\end{lemma}

\begin{proof}
If $X=\Spec A$ is affine, then $(X,\mathscr{O}_{\varphi})\simeq \Spec
(M+{}_{\varphi}\Omega^{1})$ where $\widetilde{M}=\mathscr{F}$. This
shows in general that $(X,\mathscr{O}_{\varphi})$ is a prescheme and
since $(X,\mathscr{O}_{\varphi})_{\red}=(X,\mathscr{O})_{\red}$ is a
scheme, the lemma is proved.
\end{proof}

If $I=\ker \varphi$, then we have an exact sequence of
$\mathscr{O}_{\varphi}$-Modules ($I$ being considered as an
$\mathscr{O}_{\varphi}$-Module)
$$
0\to I\to \mathscr{O}_{\varphi}\to \mathscr{O}\to 0.
$$


\begin{lemma}\label{art05-lem3.2}
Let $X$ be a smooth irreducible scheme. The scheme
$X_{\varphi}=(X,\mathscr{O}_{\varphi})$ is Cohen-Macaulay if and only
if $I$ is a locally free $\mathscr{O}_{X}$-Module. In this case, the
dualising sheaf of $X_{\varphi}$ restricts to $X$ as
$\underline{\Hom}(I,\omega_{X})$. Moreover, $X_{\varphi}$ is a local
complete intersection if and only if $I$ is locally free of rank $1$
or $0$.
\end{lemma}

\begin{proof}
The problem being local, we may assume that $X=\Spec A$ and
$X_{\varphi}=\Spec A_{\varphi}$, and since $\Omega^{1}_{X}$ is locally
free, we may also assume that $A_{\varphi}=A\oplus I$, with
$I^{2}=0$. In the local case, an $A$-sequence is an
$A_{\varphi}$-sequence if and only if it is an $I$-sequence as
well. Hence the first assertion. Now, if $I$ is free, then
$A_{\varphi}=A\otimes (k\oplus V)$ with $V^{2}=0$, where $V$ is a
finite dimensional $k$-vector space. Thus, our assertion on the
dualising sheaf needs only to be proved when $X_{\varphi}=\Spec
(k\otimes V)$. In this case, the dualising sheaf is seen to be
actually $k\oplus V^{*}$, in which $V$ operates trivially on $k$ and
by duality\pageoriginale on $V^{*}$. To prove the last assertion, we
note that if $A_{\varphi}$ is a complete intersection, then so is
$k\oplus V$, and hence $k\oplus V^{*}$ is free over $k\oplus V$. This
clearly implies that $\dim V\leq 1$, while, on the other hand, it is
obvious that $\dim V=1$ implies $k\oplus V$ is a complete intersection. 
\end{proof}

\begin{subremark}\label{art05-rem3.3}
Let $X$ be a smooth subscheme of a scheme $Y$. Then we have an exact
sequence
$$
0\to \check{N}_{X,Y}\to \Omega^{1}_{Y}|_{X}\to \Omega^{1}_{X}\to 0.
$$
If $\check{N}_{X,Y}\to I$ is an $\mathscr{O}_{X}$-homomorphism into an
$\mathscr{O}_{X}$-Module $I$, then by the push-out construction, we
obtain a sheaf $\mathscr{F}$ and a surjective homomorphism
$\varphi:\mathscr{F}\to \Omega^{1}_{X}$. In particular, this gives
rise to the scheme $X_{\varphi}$. Moreover, there is a natural
morphism $X_{\varphi}\to Y$ making the diagram
\[
\xymatrix@R=1.5cm{
 & X_{\varphi}\ar[dl]\ar[dr] & \\
X\ar[rr] & & Y
}
\]
commutative. It is easy to see that this morphism $X_{\varphi}\to Y$
is a closed immersion if and only if $\check{N}_{X,Y}\to I$ is
surjective. (In this case, the thickened subscheme $X_{\varphi}$ of
$Y$ will be denoted by $X_{I}$). In other words, all the thickenings
of $X$ `inside' $Y$ are given by Quot $\check{N}_{X,Y}$. In the case
when $Y=Z_{\Psi}$, with $Z$ smooth and $\psi$ a surjective
homomorphism $\mathscr{F}\to \Omega^{1}_{Z}$, then $\check{N}_{X,Y}$
is the kernel of the map $\mathscr{F}|X\to \Omega^{1}_{X}$, obtained
as the composite of $\psi|X$ and the map
$\Omega^{1}_{Z}|X\to \Omega^{1}_{X}$.  
\end{subremark}

\begin{lemma}\label{art05-lem3.4}
Let $X$ be a smooth subscheme of a smooth scheme $Z$, and let $Q$ be a
locally free quotient of $N^{*}_{X,Z}$. If $\eta:N^{*}_{X,Z}\to Q$ is
the canonical map, then the restriction of $N_{X_{Q},Z}$ to $X$ is
locally free and fits in the exact sequence
$$
0\to S^{2}(Q)\to N_{X_{Q},Z}|X\to \ker \eta\to 0.
$$
\end{lemma}

\begin{proof}
Let $I$ (resp.~$J$) be the ideal sheaf of $X$ (resp.~$X_{Q}$) in
$Z$. We may choose local parameters
$(x_{1},\ldots,x_{l},y_{1},\ldots,y_{m},z_{1},\ldots,z_{n})$ at a
point of $X$ so that 
{\fontsize{10}{11}\selectfont
$$
I=\{x_{1},\ldots,x_{l},y_{1},\ldots,y_{m}\}\quad\text{and}\quad
J=\{x_{i}x_{j},y_{1},\ldots,y_{n}\}, i, j, =1,\ldots l.
$$}
From
this it is clear that $I^{2}/IJ$ is\pageoriginale locally free and the
natural map $S^{2}(Q)=S^{2}(I/J)\to I^{2}/IJ$ is an isomorphism. On
the other hand $N_{X_{Q},Z}|X\cong J|IJ$ fits in the exact sequence
$$
0\to I^{2}/IJ\to J|IJ\to J/I^{2}\to 0.
$$
Now $N^{*}_{X,Z}\cong I/I^{2}$ and the map $\eta$ is the natural
projection $I/I^{2}\to I/J$ and hence $\ker \eta=J|I^{2}$, proving the
lemma. 
\end{proof}
The following functorial remark on the extension 3.4 is an immediate
consequence of the definition.


\begin{lemma}\label{art05-lem3.5}
If $X$ is a smooth subscheme of a smooth scheme $Z$ and
$\eta_{1}:N^{*}_{X,Z}\to Q_{1}$, $\zeta:Q_{1}\to Q_{2}$ are
surjections, then the extensions \ref{art05-lem3.4} corresponding to
$X_{Q_{1}}$ and $X_{Q_{2}}$ fit in a commutative diagram
\[
\xymatrix{
0\ar[r] & S^{2}(Q_{1})\ar[d]\ar[r] & N_{X_{Q_{1}},Z}|X\ar[d]\ar[r]
& \ker \eta_{1}\ar[d]\ar[r] & 0\\
0\ar[r] & S^{2}(Q_{2})\ar[r] & \check{N}_{X_{Q_{2}},Z}|X\ar[r]
& \ker \zeta\circ \eta_{1}\ar[r] & 0
}
\]
\end{lemma}

\begin{subremark}\label{art05-rem3.6}
We will not determine the extension involved in \ref{art05-lem3.4} in
the following situation. Let $\varphi:Y\to Z$ be a morphism of smooth
schemes and $X$ a smooth subscheme of $Y$ on which $\varphi$ is an
imbedding. Then the Hessian (see \S\ref{art05-sec2.9}) of the map
$\varphi$ along $X$ goes down to a map $K\otimes N_{X,Y}\to \Coker
d\varphi$ where $K=\ker d\varphi$. Now $d{\varphi}$ induces a map
$N_{X,Y}\to N_{X,Z}$ and let $Q$ be the image of its transpose. We
wish to consider the thickening of $X$ in $Z$ given by
$\eta:N^{*}_{X,Z}\to Q$. It is easily seen that this thickening is in
fact the schematic image (by $\varphi$) of the total normal thickening
of $X$ in $Y$.
\end{subremark}

\begin{lemma}\label{art05-lem3.7}
In the situation of \ref{art05-rem3.6}, we further assume that $K=\ker
d\varphi$ has rank $1$ at all points of $X$. Then the extension in
Lemma \ref{art05-lem3.4} is the pull back by means of the transpose of
the Hessian : $\ker\eta\to K^{*}\otimes N^{*}_{X,Y}$ of the exact
sequence
$$
0\to S^{2}(Q)\to S^{2}(N^{*}_{X,Y})\to K^{*}\otimes N^{*}_{X,Y}\to 0
$$
obtained by symmetrising the exact sequence
$$
0\to Q\to N^{*}_{X,Y}\to K^{*}\to 0.
$$
\end{lemma}

\begin{proof}
We use the notation of \ref{art05-lem3.4}. Let $I'$ be the ideal of
$X$ in $Y$. Then notice that $f\in I\Rightarrow f\circ \varphi\in I'$
and $f\in J\Rightarrow f\circ \varphi\in {I'}^{2}$. Hence
we\pageoriginale have a map $J/_{IJ}\to
{I'}^{2}/{I'}^{3}=S^{2}(N^{*}_{X,Y})$ induced by $\varphi$. The
resulting map $J/{I'}^{2}=(N/Q)^{*}\to K^{*}\otimes N^{*}_{X,Y}$ is
easily seen to be the transpose of the Hessian.
\end{proof}

Putting together Lemmas \ref{art05-lem3.5} and \ref{art05-lem3.7}, we obtain

\begin{subprop}\label{art05-prop3.8}
In the situation of \ref{art05-lem3.7}, let $L$ be a locally free
quotient of $Q$. Then the pullback of the extension \ref{art05-lem3.4}
for $X_{L}$ by the inclusion $\ker \eta\to \ker (N^{*}_{X,Z}\to L)$ is
isomorphic to the pullback of the sequence
$$
0\to S^{2}(L)\to S^{2}(E)\to K^{*}\otimes E\to 0
$$
by the map $K^{*}\otimes N^{*}_{X,Y}\to K^{*}\otimes E$, where this
latter extension is obtained by symmetrising the sequence
$$
0\to L\to E\to K^{*}\to 0,
$$
$E$ being the push-out $N^{*}_{X,Y\coprod{Q}}L$.
\end{subprop}

\begin{proof}
From Lemma \ref{art05-lem3.5}, we obtain the commutative diagram
\[
\xymatrix{
0\ar[r] & S^{2}(L)\ar[r] & \check{N}_{X_{L},Z}|X\ar[r] & \ker
(N^{*}_{X,Z}\to L)\ar[r] & 0\\
0\ar[r] & S^{2}(Q)\ar[u]\ar[r] & \check{N}_{X_{Q},Z}|X\ar[r]\ar[u]
& \ker \eta\ar[r]\ar[u] & 0
}
\]
This shows that the required pullback is the push-out of the sequence
$$
0\to S^{2}(Q)\to \check{N}_{X_{Q},Z}|X\to \ker \eta\to 0
$$
by the map $S^{2}(Q)\to S^{2}(L)$. Now the proposition follows from
the description of this sequence in Lemma \ref{art05-lem3.7} and the
obvious commutative diagram
\[
\xymatrix{
0\ar[r] & S^{2}(L)\ar[r] & S^{2}(E)\ar[r] & K^{*}\otimes E\ar[r] & 0\\
0\ar[r] & S^{2}(Q)\ar[r]\ar[u] & S^{2}(N^{*}_{X,Y})\ar[r]\ar[u] &
K^{*}\otimes N^{*}_{X,Y}\ar[r]\ar[u] & 0
}
\]
\end{proof}

\begin{subremark}\label{art05-rem3.9}
In our applications, we will be only considering the special case of
(\ref{art05-rem3.6}), where we have $\psi:Y\to T$ is a proper smooth
morphism and $\varphi$ realises the fibres $\{X\}$ of $\psi$ as a
family of smooth subschemes of $Z$.
\end{subremark}

\section{Conics}\label{art05-sec4}

\begin{definition}\label{art05-defi4.1}
A scheme $C$ together with a very amply line bundle $h$ is said to be
a conic if its Hilbert polynomial is $(2m+1)$. A\pageoriginale scheme
$C$ over $T$ is said to be a {\em conic over} $T$ if with respect to a line
bundle on $C$ its fibres are conics and the morphism $C\to T$ is
faithfully flat. 
\end{definition}

We will show in \ref{art05-lem4.2} and \ref{art05-lem4.3} that this coincides with the usual notion
of a conic. 

\begin{lemma}\label{art05-lem4.2}
If $C$ is a reduced scheme and $h$ a very ample line bundle on $C$
with Hilbert polynomial of the form $2m+\delta$, $\delta\leq 1$, then
$\delta=1$, $C$ is isomorphic to a (reduced) conic in $\bfP^{2}$, and
$h$ is the restriction to $C$ of the hyperplane bundle on $\bfP^{2}$.
\end{lemma}

\begin{proof}
It is obvious that we may assume without loss of generality that $C$
is pure 1-dimensional. Clearly then, either $C$ is irreducible, or $C$
has two components $C_{1}$, $C_{2}$ with Hilbert polynomials
$m+d_{1}$, $m+d_{2}$ with $d_{1}+d_{2}-l=\delta$, where $l$ is the
length of the intersection $C_{1}\cap C_{2}$. In the former case the
normalisation of $C$ has to be rational; for, otherwise, a linear
system of degree 2 is at most one dimensional and cannot imbed
$C$. Moreover, if $C$ is not normal, the linear system of $h$ on $C$
will have dimenion $\leq 1$, which cannot imbed $C$. Thus if $C$ is
irreducible, it must be isomorphic to $\bfP^{1}$ and the bundle $h$ is
the square of the hyperplane bundle on $\bfP^{1}$. In the case when
$C$ has two components $C_{1}$ and $C_{2}$, we conclude as above that
$C_{1}$, $C_{2}$ are both isomorphic to $\bfP^{1}$ and $h$ restricts
to each of these as the hyperplane bundle. Thus we have
$d_{1}=d_{2}=1$. From the nature of the Hilbert polynomial, it is
clear that $C_{1}$ and $C_{2}$ intersect. But $\dim H^{0}(C,h)=4-1\geq
3$ since $h$ is very ample. Hence $l=1$, {\em i.e.} $C$ is a pair of
intersection lines in $\bfP^{2}$.
\end{proof}

\begin{lemma}\label{art05-lem4.3}
Let $C$ be a nonreduced scheme and $h$ an ample bundle on $C$ with
Hilbert polynomial of the form $2m+\delta$, $\delta\leq 1$. If
$h|C_{\red}$ is very ample, then $C$ contains a unique subscheme $C'$
which is obtained by thickening $\bfP^{1}$ by a line bundle $L$ of
degree $\leq -1$. If degree $L=-1$, then $L$ must be the dual of the
hyperplane bundle and $C=C'$. In otherwords, $C$ is isomorphic to the
total thickening of a line in $\bfP^{2}$. In particular, a nonreduced
conic (in the sense of \ref{art05-defi4.1}) is such a thickened line. 
\end{lemma}

\begin{proof}
Clearly,\pageoriginale we may assume $C$ has no zero-dimensional
components. Consider the exact sequence
$$
0\to I\to \mathscr{O}\to \mathscr{O}_{\red}\to 0
$$
on $C$, where $I$ is the ideal of nilpotents in $\mathscr{O}$. If $I$
has finite support, the Hilbert polynomial of $\mathscr{O}_{\red}$ is
of the same form and hence by \ref{art05-lem4.2}, must be $2m+1$. This
shows that $\mathscr{O}=\mathscr{O}_{\red}$, contrary to
assumption. Thus the support of $I$ is 1-dimensional. It follows that
the Hilbert polynomials of $I$ and $\mathscr{O}_{\red}$ are of the
form $m+d_{1}$, $m+d_{2}$. Since $h|C_{\red}$ is very ample, we
conclude as in \ref{art05-lem4.2} that $C_{\red}$ is isomorphic to
$\bfP^{1}$, and $h$ restricts to $\bfP^{1}$ as the hyperplane
bundle. Now consider the filtration
$$
\mathscr{O}\supset I\supset I^{2}\ldots\supset I^{n}=(0).
$$
Since the Hilbert polynomial $m+d_{1}$ of $I$ is the sum of the
Hilbert polynomials of the sheaves $I^{k}/I^{k+1}$ on $\bfP^{1}$, it
follows that $rk\ I/I^{2}=1$ and $I^{k}/I^{k+1}$ are all torsion
sheaves for $k\geq 2$. If $L$ is the line bundle on $\bfP^{1}$
obtained as the free quotient of $I/I^{2}$, we get an $L$-thickening
$C'$ of $\bfP^{1}$ as a subscheme of $C$. Since now the Hilbert
polynomial of $\mathscr{O}_{\red}$ is $m+1$, the Hilbert polynomial of
$I$ is $m+\delta-1$ and hence that of $L$ is $m+d$, with $d\leq 0$. In
particular, the degree of $L\geq -1$. If now $\deg L=-1$, then $L$ is
the dual of the hyperplane bundle on $\bfP^{1}$ whose Hilbert
polynomial is $m$, and hence that of $C'$ is $2m+1$. This proves that
$\delta=1$ and $C=C'$. It is clear that the $L$-thickening of
$\bfP^{1}$ is the normal thickening of a line in $\bfP^{2}$. Finally,
if $h$ is already very ample on $C$, then from the exact sequence
$$
0\to L\to \mathscr{O}_{C'}\to \mathscr{O}_{\red}\to 0
$$
tensored with $h$, we obtain
$$
0\to H^{0}(\bfP^{1},L\otimes h)\to H^{0}(C',h)\to H^{0}(\bfP^{1},h).
$$
Since $h$ is very ample on $C'$, we have $\dim H^{0}(C',h)\geq 3$, and
$H^{0}(\bfP^{1},h)$ being 2-dimensional, it follows that
$H^{0}(\bfP^{1},L\otimes h)\neq 0$, {\em i.e.} $\deg (L\otimes h)\geq
0$, proving that $\deg L\geq -1$.
\end{proof}

\begin{remarks}\label{art05-rems4.4}
\begin{itemize}
\item[(i)] From\pageoriginale what we have seen above, it is clear
that if $(C,h)$ 
is a conic as in \ref{art05-defi4.1}, then the linear system of $h$ in
$C$ is 2-dimensional and imbeds $C$ as a conic in $\bfP^{2}$. In
particular, if $C$ is a subscheme of $\bfP^{n}$ and $C$ is a conic
with respect to the restriction of the hyperplance bundle, then $C$ is
contained in a unique plane in $\bfP^{n}$ as a conic.

\item[(ii)] Also, if $C$ is a subscheme of the one-point union of two
projective spaces, and $C$ is a conic with respect to the restriction
of the natural very ample (hyperplane) bundle on this union, then $C$
is either a conic in one of the projective spaces, or is a pair of
lines, one in each of these spaces passing through their point of
intersection. 

\item[(iii)] Let $Y$ be the thickening of a projective space $Z$ by a
surjection $\varphi:\mathscr{G}\to \Omega^{1}$, with $\ker \varphi$
isomorphic to the dual of the universal quotient bundle $Q$ on $Z$. In
this case, the hyperplane bundle on $Z$ extends uniquely to a line
bundle $h$ on $Y$ which is very ample. If $C$ is a conic subscheme
(with respect to $h$) of $Y$, then either (a) $C$ is a subscheme of
$Z$ or (b) $C_{\red}$ is a line $l$ in $Z$ and $C$ is obtained as a
$\tau^{-1}$-thickening of this line in $Y$. By \ref{art05-rem3.3}, the
latter is obtained as a quotient of $N^{*}_{l,Z}$ which is isomorphic
to $\tau^{-1}$. Consider the diagram on $l$:
\[
\xymatrix{
 & & 0\ar[d] & 0\ar[d]  \\
0\ar[r] & Q\ar[r]\ar@{=}[d] & \mathscr{G}'\ar[r]\ar[d]
& \tau^{-1}\otimes \text{~trivial~}\ar[r]\ar[d] & 0\\
0\ar[r] & Q\ar[r] & \mathscr{G}\ar[r]\ar[d] & \Omega^{1}_{Z}\ar[r]\ar[d] & 0\\
 & & \Omega^{1}_{l}\ar@{=}[r]\ar[d] & \Omega^{1}_{l}\ar[d] & \\
 & & 0 & 0 &
}
\]
Now $H^{0}(l,\Hom(Q,\tau^{-1}))$ is 1-dimensional, since
$Q|l\simeq \tau^{-1}+$ trivial. Hence any map
$\mathscr{G}'\to \tau^{-1}$ when restricted to $Q$ must coincide (upto
a scalar factor) with the natural map $Q|l\to \tau^{-1}$. If this map
is zero, then the map $\mathscr{G}'\to \tau^{-1}$ factors through
$(\tau^{-1}\otimes \text{~trivial})$ and in this case, $C$ is actually
contained in $Z$ itself. All other $\tau^{-1}$ thickenings in $Y$ are
parametrised by an affine space of $\dim =\codim$ of $l$ in $Z$. 

\item[(iv)] If\pageoriginale $C$ is a conic over $T$, by \cite[SGA 6,
VII]{art05-key9}, the morphism $\pi:C\to T$ is a complete
intersection. Let $\omega$ be the relative dualising sheaf over
$T$. Then clearly $C$ is imbedded as a $T$-scheme in the projective
plane bundle $S$ over $T$ associated to
$E=R^{1}(\pi)_{*}\omega^{2}$. Moreover, if $C$ is imbedded as a
$T$-scheme in a projective bundle $P$ over $T$ such that the fibres of
$\pi$ are imbedded as conics, then the inclusion $C\hookrightarrow P$
factors through an imbedding $S\subset P$, linear on fibres. On $E$,
there is on everywhere nontrivial quadratic form with values in a line
bundle $L$ such that $C$ is the associated conic bundle over $T$.
\end{itemize}
\end{remarks}

\section{Good Hecke Cycles}\label{art05-sec5}

If $E$ is a vector bundle $(\neq 0)$ on $X$ and $k\in Z$, we denote by
$\mu_{k}(E)$ the rational number $(\deg E+k)/rk\ E$.

\begin{definition}\label{art05-defi5.1}
A vector bundle $E$ on $X$ is said to be $(k,l)$-stable
(resp.~$(k,l)$-semistable) if, for every proper subbundle $F$ of $E$,
we have
$$
\mu_{k}(F)<\mu_{-l}(E/F)(\text{resp.}~\mu_{k}(F)\leq \mu_{-l}(E/F)).
$$
\end{definition}

\begin{remark}\label{art05-rem5.2}
\begin{itemize}
\item[(i)] The condition \ref{art05-defi5.1} is equivalent to
$\mu_{k}(F)<\mu_{k-l}(E)$ or $\mu_{k-l}(E)<\mu_{-l}(E/F)$.

\item[(ii)] If $E$ is $(k,l)$-stable and $L$ any line bundle, then
$E\otimes L$ is also $(k,l)$-stable.

\item[(iii)] If $E$ is $(k,l)$-stable, then $E^{*}$ is $(l,k)$-stable.

\item[(iv)] A vector bundle of degree 0 is stable if and only if it is
$(0,1)$-stable. 

\item[(v)] A vector bundles of degree 1 is stable if and only if it is $(0,1)$-semi-stable.
\end{itemize}
\end{remark}

\begin{subprop}\label{art05-prop5.3}
In any family $E$ of vector bundles on $X$, para\-metrised by $T$, the
set of $(k,l)$-stable points is open in $T$.
\end{subprop}

\begin{proof}
Consider the Quot-scheme of $E$ over $T$. The set of
non-$(k,l)$-stable points is characterised as the union of the images
of the components of the Quot scheme whose Hilbert polynomials satisfy
an inequality which constrain them to vary over a finite set. Now our
assertion follows from the properness of the Hilbert scheme over $T$,
with a fixed Hilbert polynomial.
\end{proof}

\begin{subprop}\label{art05-prop5.4}
\begin{itemize}
\item[\rm(i)] Except\pageoriginale when $g=2$, $n=2$ and $d$ odd,
there always exist 
$(0,1)$-stable (and $(1,0)$-stable) bundles of rank $n$ and degree
$d$.

\item[\rm(ii)] There exist $(1,1)$-stable bundles of rank $n$ and
degree $d$, except in the following cases.
\begin{itemize}
\item[\rm(a)] $g=3$, and $d$ both even

\item[\rm(b)] $g=2$, $d\equiv 0$, $\pm 1(\text{\rm mod~}n)$

\item[\rm(c)] $g=2$, $n=4$, $d\equiv 2(\text{\rm mod~}4)$.
\end{itemize}
\end{itemize}
\end{subprop}

\begin{proof}
It is enough to estimate the dimension of the subvariety of $U(n,d)$,
consisting of non-$(0,1)$-stable (resp. non-$(1,1)$-stable) points and
prove that it is a proper subvariety. Clearly any such bundle $E$
contains a subbundle $F$ satisfying the inequality  
\begin{align*}
& \dfrac{\deg F}{rk\ F}\geq \dfrac{\deg E-1}{rk\ E}\\
\text{(resp.)}\qquad & \dfrac{\deg F+1}{rk\ F}\geq \dfrac{\deg E}{rk\ E}
\end{align*}
By using \cite[Proposition 2.6]{art05-key5} as in \cite[Lemma
6.7]{art05-key5} we may as well assume that $F$ and $E/F$ are stable
and compute the dimension of such bundles $E$. The dimension of a
component corresponding to a fixed rank $r$ and degree $\delta$ of
$F$, is majorised by $\dim U(r,\delta)+\dim U(n-r, d-\delta)+\dim
H^{1}(X,\Hom(E/F,F))-1=(g-1)(n^{2}-nr+r^{2})+1+(dr-\delta n)$. In
order to show this is $<\dim U(n,d)=n^{2}(g-1)+1$ we have only to
verify that $(g-1)r(n-r)>dr-\delta n$. But this is a simple
consequence of the inequality $dr-\delta n\leq r$ in the first case
and $\leq n$ in the second case, taking into account the exceptions
mentioned in the proposition.
\end{proof}

\begin{lemma}\label{art05-lem5.5}
Let $x\in X$ and
$0\to \underline{E'}\to \underline{E}\to \mathscr{O}_{X}\to 0$ be an
exact sequence of sheaves with $\underline{E}$, $\underline{E}'$
locally free. If $E$ is $(k,l)$-stable then $E'$ is $(k,l-1)$
stable. In particular, if $E$ is $(0,1)$-stable then $E'$ is
stable. Similar statements are valid when stable is replaced by
semistable. 
\end{lemma}

\begin{proof}
Let $F$ be a subbundle on $E'$ and $F'$ the subbundle of $E$ generated
by the map $F\to E$. Then $F\to F'$ is of maximal rank and
hence\pageoriginale $\deg F'\geq \deg F$. Now
$\mu_{k}(F)\leq \mu_{k}(F')<\mu_{k-l}(E)=\mu_{k-l+1}(E')$ as $\deg
E=\deg E'+1$. This proves that $E'$ is $(k,l-1)$ stable.
\end{proof}

\begin{lemma}\label{art05-lem5.6}
\begin{itemize}
\item[\rm(i)] Let $E$ be a $(0,1)$-stable vector bundle of rank $n$
and $E'$ a stable vector bundle of rank $n$ and determinant isomorphic
to $\det E\otimes L^{-1}_{x}$. If $f:E'\to E$ is a non-zero
homomorphism, we have an exact sequence
$$
0\to \underline{E}'\to \underline{E}\to \mathscr{O}_{x}\to 0.
$$

\item[\rm(ii)] Moreover $\dim H^{0}(X,\Hom(E',E))\leq 1$.
\end{itemize}
\end{lemma}

\begin{proof}
The map $f$ is of maximal rank. For, otherwise, let
\[
\xymatrix{
E'\ar[r] & G'\ar[d]\ar[r] & 0\\
E & G\ar[l] & 0\ar[l]
}
\]
be the canonical factorisation of $f$, where $rk\ G'<n$ and $G'\to G$
is of maximal rank. We then have
$\mu(G)\geq \mu(G')>\mu(E')=\mu_{-1}(E)$ which contradicts the
$(0,1)$-stability of $E$. Now the induced map
$\overset{n}{\wedge}f:\overset{n}{\wedge}E'\to \overset{n}{\wedge}E$
is non-zero and can vanish only at $x$ (with multiplicity 1). Thus $f$
is of maximal rank at all points except $x$ and is of rank $(n-1)$ at
$x$. This proves the first part.

If $f$ and $g$ are two linearly independent homomorphisms $E'\to E$,
then for $y\neq x$, we can find a suitable (nontrivial) linear
combination of $f$ and $g$ which is singular at $y$, see \cite[Lemma
7.1]{art05-key4}. This contradicts i) and completes the proof of
Lemma \ref{art05-lem5.6}. 
\end{proof}

Let now $W\to X\times S$ be a family of $(0,1)$-stable vector bundles
on $X$, with fixed determinant $\xi$, parametrised by $S$. Then we may
construct a family $H(W)$ of vector bundles on $X$, parametrised by
the associated projective bundle $\pi:P(W^{*})\to X\times S$ as
follows. It is easy to construct as in  [\ref{art05-sec5}\S\ 4] a
canonical surjection of the bundle $p^{*}_{2}\pi^{*}W$ on $X\times
P(W^{*})$ onto $p^{*}_{2}\tau\otimes \mathscr{O}_{P(W)^*}$, where
$P(W)^{*}$ is considered a divisor in $X\times P(W^{*})$, by the
inclusion $(p_{1}\circ \pi, Id)$. The kernel of this homomorphism is
the vector bundle $H(W)$. Let $K(W)$ be its dual.

\begin{remark}\label{art05-rem5.7}
This construction is the same as that in \cite{art05-key5}, except
that here we have allowed $x$ also to vary on the curve. From this
point\pageoriginale of view, the families constructed
in \cite{art05-key5} may properly be denoted $H_{x}(V)$, $K_{x}(V)$,
etc. 
\end{remark}

By Lemma \ref{art05-lem5.5}, $K(W)$ is a family of stable vector
bundles on $X$ with determinant of the form $\xi^{-1}\otimes L_{x}$,
$x\in X$. More precisely, we have a morphism
$\theta_{K(W)}:P(W^{*})\to U(n,\xi^{-1}X)$ with a commutative diagram
\setcounter{equation}{7}
\begin{equation}
\vcenter{
\xymatrix@C=1.8cm{
P(W^{*})\ar[d]\ar[r]^-{(\theta_{K(W)},p_{2}\circ \pi)} &
U(n,\xi^{-1}X)\times S\ar[d]^-{(\alpha,Id)}\\
X\times S\ar[r]^-{Id} & X\times S
}}\label{art05-eq5.8}
\end{equation}
where $\alpha:U(n,\xi^{-1}X)\to X$ is given by $L_{\alpha(E)}=(\det
E)\otimes \xi^{-1}$ for $E\in U(n,\xi^{-1}X)$. 

\setcounter{theorem}{8}
\begin{lemma}\label{art05-lem5.9}
The morphism $(\theta_{K(W)}, p_{2}\circ \pi)$ of $P(W)^{*}$
in \eqref{art05-eq5.8} is a closed immersion.
\end{lemma}

To prove this we will need the following

\begin{lemma}\label{art05-lem5.10}
Let $X$ be a scheme proper/$k$ and $\mathscr{F}$ a coherent sheaf of
$\mathscr{O}_{X}$-Modules. Let $S$ be a scheme and
$$
0\to \mathscr{H}\to p^{*}_{1}\mathscr{F}\to \mathscr{G}\to 0
$$
be an exact sequence of coherent sheaves over $X\times S$ with
$\mathscr{G}$ flat over $S$. Then $\mathscr{H}$ is a flat family of
$\mathscr{O}_{X}$-Modules parametrised by $S$ and its infinitesimal
deformation map at $s_{0}\in S$, is the negative of the composite of
the natural map $T_{s_{0}}\to
H^{0}(X,\Hom(\mathscr{H}_{0},\mathscr{G}_{0}))$ (the differential of
the morphism $S\to \Quot$) and the boundary homomorphism $H^{0}(X,\Hom
(\mathscr{H}_{0},\mathscr{G}_{0}))\to \Ext^{1}(X,\mathscr{H}_{0},\mathscr{H}_{0})$
associated to the sequence
$0\to \mathscr{H}_{0}\to \mathscr{F}\to \mathscr{G}_{0}\to 0$. Here
$\mathscr{H}_{0}$, $\mathscr{G}_{0}$ denote the restrictions of
$\mathscr{H}$, $\mathscr{G}$ to $X\times s_{0}$.
\end{lemma}

\begin{proof}
It is clearly sufficient to consider the case $S=\Spec
k[\epsilon]$. To give a sheaf $\mathscr{L}$ on $X\times S$ flat over
$S$ which extends a given sheaf $\mathscr{L}_{0}$ on $X\times s_{0}$,
is the same as giving an extension
$$
0\to \epsilon \mathscr{L}_{0}\to \mathscr{L}\to \mathscr{L}_{0}\to 0
$$
on $X$, where $\epsilon \mathscr{L}_{0}$ is another copy of
$\mathscr{L}_{0}$. This identifies the space of infinitesimal
deformations of $\mathscr{L}_{0}$ with
$\Ext^{1}(X,\mathscr{L}_{0},\mathscr{L}_{0})$. In particular,
$p^{*}_{1}\mathscr{F}$\pageoriginale is given by
$\epsilon \mathscr{F}\oplus \mathscr{F}$. Now the map
$p^{*}_{1}\mathscr{F}\to \mathscr{G}$ gives rise to a map
$\mathscr{H}_{0}\to \mathscr{F}\to \epsilon \mathscr{F}\oplus \mathscr{F}\to \mathscr{G}$. This
map goes into $\epsilon \mathscr{G}_{0}$ and hence induces an element
$t$ of $H^{0}(X,\Hom(\mathscr{H}_{0},\mathscr{G}_{0}))$ and thus
identifies this space with the fibre of
$\Quot_{\mathscr{F}}(k[\epsilon])\to \Quot_{\mathscr{F}}(k)$ over
$\mathscr{G}_{0}$. We have a map $\mathscr{H}\to \epsilon \mathscr{F}$
obtained by composing $\mathscr{H}\to p^{*}_{1}\mathscr{F}$ with the
projection $p^{*}_{1}\mathscr{F}\to \epsilon \mathscr{F}$. This fits
in a commutative diagram 
\[
\xymatrix{
0\ar[r] & \epsilon \mathscr{H}_{0}\ar[d]\ar[r]
& \mathscr{H}\ar[d]\ar[r] & \mathscr{H}_{0}\ar[d]\ar[r] & 0\\
0\ar[r] & \epsilon \mathscr{H}_{0}\ar[r] & \epsilon \mathscr{F}\ar[r]
& \epsilon\mathscr{G}_{0}\ar[r] & 0
}
\]
The last vertical map in the diagram is easily checked to be
$-t$. Hence the element of
$\Ext^{1}(X,\mathscr{H}_{0},\mathscr{H}_{0})$ giving the extension
$\mathscr{H}$ is the image of $-t$ under the connecting homomorphism
$$
\Hom(X,\mathscr{H}_{0},\mathscr{G}_{0})\to \Ext^{1}(X,\mathscr{H}_{0},\mathscr{H}_{0}).
$$ 
\end{proof}

\begin{remark}\label{art05-rem5.11}
Proposition 3.3 of \cite{art05-key5} is a particular case
of \ref{art05-lem5.10}, where $\mathscr{F}$ and $\mathscr{G}$ are
locally free, in which case
$\Ext^{1}(X,\mathscr{H}_{0},\mathscr{H}_{0})$ is the same as
$H^{1}(X,\End \mathscr{H}_{0})$.
\end{remark}

\noindent
{\bf Proof of Lemma \ref{art05-lem5.9}.} In view of the
diagram \ref{art05-eq5.8}, we may assume that $T$ is a point. Thus we
have a bundle $W$ on $X$ and the exact sequence
$$
0\to H(W)\to p^{*}_{1}W\to
p^{*}_{2}\tau \otimes \mathscr{O}_{P(W^{*})}\to 0
$$
on $X\times P(W^{*})$. Now it is easy to see that $P(W^{*})$ is a
component of $\Quot E$ and hence its tangent bundle can be identified
as in Lemma \ref{art05-lem5.10}, with 
$$
(p_{2})_{*}(\Hom
(H(W),\tau\otimes \mathscr{O}_{P(W^{*})})=\Hom(H(W)|P(W^{*}),\tau).
$$ 
Moreover, by Lemma \ref{art05-lem5.10}, the infinitesimal deformation map for
the family $H(W)$ of bundles on $X$ is given, upto sign, by the
connecting homomorphism 
$$
\Hom(H(W)|P(W^{*}),\tau)\to
R^{1}(p_{2})_{*}\End H(W).
$$ 
But this fits in the exact sequence
\begin{align*}
0\to & (p_{2})_{*}(\End H(W))\to
(p_{2})_{*}(\Hom(H(W),p^{*}_{1}W))\to\\
&\Hom (H(W)|P(W^{*}),\tau)\to R^{1}(p_{2})_{*}\End H(W).
\end{align*}
Now since $H(W)$ is a family of stable bundles,
$$
(p_{2})_{*}(\End (H(W))\simeq \mathscr{O}_{P(W^{*})}
$$
and by Lemma \ref{art05-lem5.6}, (ii), the same is true of
$$
(p_{2})_{*}(\Hom(H(W),p^{*}_{2}(W)). 
$$ 
This\pageoriginale proves that the differential of the map
$\theta_{K(E)}:P(W^{*})\to U(n,\xi^{-1}X)$ is injective. On the other
hand, $\theta_{K(E)}$ is itself injective by Lemma \ref{art05-lem5.6},
(ii), thus proving Lemma \ref{art05-lem5.9}.

Now, for each family of $(0,1)$-stable bundles we have a morphism of
the parameter space into $U(n,\xi^{-1}X)$, by
Lemma \ref{art05-lem5.9}. Since these morphisms are clearly functorial
we get a morphism $\Phi$ on the open subscheme of $U(n,\xi)$
consisting of $(0,1)$-stable points into the Hilbert scheme
$\Hilb(U(n,\xi^{-1}X))$. 

\begin{definition}\label{art05-defi5.12}
The cycles in $U(n,\xi^{-1}X)$ corresponding to $(0,1)$-stable points
of $U(n,\xi)$ will be called {\em good Hecke cycles}. Any subscheme in the
same irreducible component of the Hilbert\break scheme will be called a
{\em Hecke cycle}.
\end{definition}

\begin{theorem}\label{art05-thm5.13}
Let $\xi$ be a line bundle on $X$. The open subscheme of $U(n,\xi)$
consisting of $(0,1)$-stable points is isomorphic (by means of the map
associating to a bundle $W$ the corresponding good Hecke cycle
$\Phi(W)$) to an open subscheme of the Hilbert scheme of
$U(n,\xi^{-1}X)$. 
\end{theorem}

\begin{proof}
To prove the injectivity, we will check the stronger assertion that
for $W$, $W'\in U(n,\xi)$, both $(0,1)$-stable,
$\theta_{K_{x}(W)}(P(W^{*}_{x}))=\theta_{K_{x}(W')}(P({W'}_{x}^{*}))$ if
and only if $W$ and $W'$ are isomorphic. Indeed, by \cite[Lemma
2.5]{art05-key6}, such an assumption would yield an isomorphism
$\varphi:P(W^{*}_{x})\to P({W'}^{*}_{x})$ such that
$\varphi^{\sharp}K_{x}(W')\simeq K_{x}(W)\otimes p^{*}_{2}L$ on
$X\times P(W^{*}_{x})$, for some line bundle $L$ on
$P(W^{*}_{x})$. Now by \cite[Remark 4.7]{art05-key5}, we see, on
restricting this isomorphism to $P(W^{*}_{x})\times (y)$, $y\neq x$,
that $L$ is trivial. But then, by \cite[Lemma 4.3]{art05-key5} we have
$(p_{1})_{*}\varphi^{\sharp}K_{x}(W')^{*}\simeq
(p_{1})_{*}K_{x}(W')^{*}\simeq L_{x}^{-1}\otimes W'$ on the one hand,
and $(p_{1})_{*}K_{x}(W')^{*}\simeq L_{x}^{-1}\otimes W$ on the
other. This proves that $W\simeq W'$. 

Next we proceed to show that, if $W\in U(n,\xi)$ is $(0,1)$-stable,
the infinitesimal deformation map at $W$ of the family of good Hecke
cycles from $T_{W}\simeq H^{1}(X,\Ad W)$ to $H^{0}(P(W^{*}),N)$ is an
isomorphism, where $N$ is the normal bundle of $P(W^{*})$ in
$U(n,\xi^{-1}X)$. This would prove that the Hilbert scheme is smooth
at $\Phi(W)$ and that $\Phi$ is an isomorphism onto to an open subset,
as claimed. To complete the proof of the theorem we need two lemmas. 


\begin{lemma}\label{art05-lem5.14}
Let\pageoriginale $\pi_{1}:P_{1}\to U(n,\xi)\times X$ be the dual of
the projective Poincar\'e bundle on $U(n,\xi)\times X$. Let
$\pi_{2}:P_{2}\to U(n,\xi^{-1}X)$ be the restriction of the dual
projective Poincar\'e bundle on $U(n,\xi^{-1}X)\times X$ to the
divisor $U(n,\xi^{-1}X)$. Let $\Omega_{1}\subset P_{1}$ and
$\Omega_{2}\subset P_{2}$ be the open subsets of points at which the
families $K$ are stable. We then have an isomorphism
$\widetilde{\theta}:\Omega_{1}\to \Omega_{2}$ such that
$\pi_{2}\circ \widetilde{\theta}=\theta$. 
\end{lemma}

\begin{proof}
A point $p$ of $\Omega_{1}$ is described by a $(0,1)$-stable vector
bundle in $U(n,\xi)$, a point $x\in X$ and an element of
$P(E^{*})_{x}$. Now this can be looked upon as a map
$E\to \mathscr{O}_{x}$ with kernel $F$, where $F^{*}$ belongs to
$U(n,\xi^{-1}X)$. By looking at the map $F_{x}\to E_{x}$ of the fibres
at $x$, we can define a map $\Omega_{1}\to P_{2}$ by associating the
$\ker F_{x}$ to $p$. It is easy to see that this lift of $\theta$ maps
$\Omega_{1}$ isomorphically onto $\Omega_{2}$. In fact this map is
induced by the section $\sigma$ \cite[p. 402]{art05-key5}. 
\end{proof}

\begin{lemma}\label{art05-lem5.15}
Let $W\in U(n,\xi)$ be $(0,1)$-stable. Then the normal bundle
$N=N_{P(W^{*}),U(n,\xi^{-1}X)}$ of $P(W^{*})$ in $U(n,\xi^{-1}X)$ fits
into an exact sequence
$$
0\to p^{*}_{X}T_{X}\otimes T^{*}_{\pi}\to P(W^{*})\times T_{W}\to N\to 0.
$$
\end{lemma}

\begin{proof}
Since $W$ is $(0,1)$-stable, $P(W^{*})=(p_{2}\circ \pi_{1})^{-1}W$ is
contained in $\Omega_{1}$ and $N_{P(W^{*}),\Omega_{1}}\simeq
P(W^{*})\times T_{W}$. Moreover $N_{P(W^{*}),\Omega_{1}}\simeq
N_{\widetilde{\theta}P(W^{*}),\Omega_{2}}$. Since
$\pi_{2}\circ \widetilde{\theta}=\theta$, and $\theta|P(W^{*})$ is an
imbedding (Lemma \ref{art05-lem5.9}), we see that
$\widetilde{\theta}(P(W^{*}))$ is transversal to the fibres of
$\pi_{2}$. Hence we have an exact sequence (on $P(W^{*})$)
$$
0\to \widetilde{\theta}^{*}T_{\pi_{2}}\to
N_{\widetilde{\theta}(P(W^{*})),\Omega_{2}}\to
N_{(\theta(P(W^{*}))),U(n,\xi^{-1}X)}\to 0.
$$ 
But by Proposition 4.12 in \cite{art05-key5}, we have
$\widetilde{\theta}^{*}T_{\pi_{2}}\simeq
(p_{2}\circ \pi_{1})^{*}T_{X}\otimes T^{*}_{\pi}$. This proves the Lemma.
\end{proof}

\noindent
{\bf Completion of the Proof of Theorem \ref{art05-thm5.13}.}
Taking the direct image of the exact sequence in
Lemma \ref{art05-lem5.15}, and noting that, $H^{0}(X,T_{X})=0$, we see
that the natural map $T_{W}\to H^{0}(P(W^{*}),N)$ is an
isomorphism. But this is clearly the differential of $\Phi$ at $W$.

\end{proof}
Taking determinants in the exact sequence \ref{art05-lem5.15}, we
obtain

\begin{corollary}\label{art05-coro5.16}
Let $P(W^{*})$ be a good Hecke cycle in $U(n,\xi^{-1}X)$. Then the
line bundle $K_{\det}$ on $U_{X}$ restricts to $P(W^{*})$ as
$\pi^{*}K_{X}^{-(n-1)}\otimes K^{2}_{\pi}$. 
\end{corollary}

\begin{remarks}\label{art05-rems5.17}
\begin{itemize}
\item[\rm(i)] Although\pageoriginale the Hilbert scheme in
Theorem \ref{art05-rem5.11} is smooth, $H^{1}(P(W^{*}),N)\neq 0$
unlike the case discussed below.

\item[\rm(ii)] Let us now fix $x\in X$. The open subset $\Omega_{0}$
of points in the projective dual Poincar\'e bundle over
$U(n,\xi)\times (x)$, corresponding to the family $K_{x}$, is
isomorphic to a similar open subset of the projective Poincar\'e
bundle over $U(n,\eta)\times (x)$ where $\xi\otimes \eta=L_{x}$. Thus
we have two maps
\[
\xymatrix@C=.4cm{
 & \Omega_{0}\ar[dl]_-{\pi_{1}}\ar[dr]^{\pi_{2}} &\\
U(n,\xi) && U(n,\eta)
}
\]
which are projective fibrations over the open subsets of
$(0,1)$-stable poins of $U(n,\xi)$, $U(n,\eta)$ respectively. Now the
content of \cite[Proposition 4.12]{art05-key5} is that
$T^{*}_{\pi_{1}}\simeq T_{\pi_{2}}$. (See also \cite[Ch~ 5, \S\
1]{art05-key8a}). If $W$ is a $(0,1)$-stable point of $U(n,\xi)$, then
$\pi^{-1}_{1}(W)$ is imbedded by $\pi_{2}$, thus giving a family of
cycles in $U(n,\eta)$ parametrised by the $(0,1)$-stable points of
$U(n,\xi^{-1}X)$. For the normal bundle $N$ of the cycle $\Phi(W)$
corresponding to $W$ we have the exact sequence
$$
0\to \Omega^{1}_{P(W^{*}_{x})}\to H^{1}(X\Ad W)_{P(W^{*})}\to N\to 0.
$$
This shows that $H^{1}(\Phi(W),N)=0$ for $i\geq 1$ and 
$$
\dim
H^{0}(\Phi(W),N)=\dim H^{1}(X,\Ad W)+1.
$$ 
As in
Theorem \ref{art05-thm5.13} we see that the open subset of
$(0,1)$-stable points of $U(n,\xi^{-1}X)$ can be identified with an
open subset of a component of the Hilbert scheme of $U(n,\xi)$. 

\item[\rm(iii)] If $W$ is a $(1,1)$-stable bundle in $U(n,\xi)$, the
corresponding cycle $\Phi(W)$ in $U(n,\xi^{-1}X)$ which is isomorphic
to $P(W^{*})$, is contained entirely in the set of $(0,1)$-stable
points (Lemma \ref{art05-lem5.5} and Remark \ref{art05-rem5.2},
(iii)). By the reverse construction in (ii), $P(W^{*})$ is contained
in the Hilbert scheme of $U(n,\xi)$ consisting of good Hecke cycles
(this time, isomorphic to a projective space) passing through
$W$. Thus $P(W^{*})$ can be identified with this subscheme of the
Hilbert scheme. From this one may conclude that any continuous group
of automorphism of $U(n,\xi)$ lifts to the projective Poincar\'e
bundle over some open subset of $U(n,\xi)$ consisting of $(1,1)$
stable points.\pageoriginale Thus, if $(1,1)$-stable points exist,
then the automorphism group of $U(n,\xi)$ is finite. This was our
original approach to the Theorems 1, 2 of \cite{art05-key5}.
\end{itemize}
\end{remarks}

\section{Triangular bundles}\label{art05-sec6}

We will deal with the case of rank 2 bundles in the rest of the
paper. We will denote $U(2,\xi)$ (resp. $U(2,L_{x})$) by $U_{\xi}$
(resp. $U_{x}$). When $\xi$ is trivial we write $U_{\xi}=U_{0}$.

We wish to study the limits of good Hecke cycles, i.e., points in the
component of the Hilbert scheme, containing good Hecke cycles. It will
turn out (\S\ \ref{art05-sec7}) that these cycles consist of bundles
which are extensions of line bundles of a particular kind. These
bundles have been studied in \cite{art05-key6} and \cite{art05-key8}
and the computations made in this article result from a closer
analysis of the situation.

Let $R$ be a Poincar\'e bundle on $X\times J$, where $J$ is the
Jacobian of $X$, Consider on $X\times X\times J$ the bundle
$p^{*}_{13}R^{2}\otimes p^{*}_{12}L_{\Delta}^{-1}$, where $\Delta$ is
the diagonal in $X\times X$. Then the first direct image
$R^{1}(p_{23})_{*}(p^{*}_{13}R^{2}\otimes p^{*}_{12}L^{-1}_{\Delta})$
on $X\times J$ is clearly a vector bundle $D$. For each $j\in J$, the
restriction of $D$ to $X\times j$ gives rise to a bundle $D_{j}$ on
$X$.

\begin{lemma}\label{art05-lem6.1}
If $j^{2}\neq 1$, then the bundle $D_{j}$ fits in an exact sequence
$$
0\to j^{2}\to D_{j}\to H^{1}(X,j^{2})_{X}\to 0
$$
given by the identity element of $H^{1}(X\Hom
(H^{1}(X,j^{2})_{X},j^{2}))$. If $j^{2}=1$, then $D_{j}$ is
isomorphic, upto tensorisation by a line bundle, to
$H^{1}(X,\mathscr{O})_{X}$. In particular, $D_{j}$ is a semistable
vector bundle of degree $0$ and any nonzero subbundle of $D_{j}$ of
degree $0$ is the inverse image of a trivial subbundle of
$H^{1}(X,j^{2})_{X}$. 
\end{lemma}

\begin{proof}
By the base change theorem, the bundle $D_{j}$ is isomorphic to
$R^{1}(p_{2})_{*}(p^{*}_{1}j^{2}\otimes L^{-1}_{\Delta})$, where
$p_{1}$, $p_{2}$ are the projections $X\times X\to X$. Hence it is
dual to $(p_{2})_{*}(p^{*}_{1}(K\otimes j^{-2})\otimes
L_{\Delta})$. Consider on $X\times X$ the exact sequence
\setcounter{equation}{1}
\begin{equation}
0\to p^{*}_{1}(K\otimes j^{-2})\to p^{*}_{1}(K\otimes j^{-2})\otimes
L_{\Delta}\to \widetilde{j}^{-2}\to 0,\label{art05-eq6.2}
\end{equation}
where the sheaf $\widetilde{L}$ denotes the extension of a line bundle
$L$ on $\Delta$ to $X\times X$ and we have used the fact that
$L_{\Delta}|_{\Delta}\simeq K^{-1}_{X}$. We will show that
by\pageoriginale applying $(p_{2})_{*}$ to this sequence and
dualising, we get the canonical extension in \ref{art05-lem6.1}. Now
the pullback by $p^{*}_{2}j^{-2}\to \widetilde{j}^{-2}$ of this
sequence yields an extension of $p^{*}_{2}j^{-2}$ by
$p^{*}_{1}(K\otimes j^{-2})$. The corresponding element in
$H^{1}(X\times X,p^{*}_{2}j^{2}\otimes p^{*}_{1}(K\otimes
j^{-2}))=H^{0}(X,K\otimes j^{-2})\otimes H^{1}(X,j^{2})$ (since
$j^{2}\neq 1$) is the canonical element given by the duality
theorem. Thus it is clear that this pulled-back extension is also the
push-out by the evaluation map $H^{0}(K\otimes j^{-2})_{X\times X}\to
p^{*}_{1}(K\otimes j^{-2})$ of the extension
$$
0\to H^{0}(X,K\otimes j^{-2})_{X\times X}\to \ldots \to
p^{*}_{2}j^{-2}\to 0
$$
given by the canonical element of $H^{1}(X\times
X,p^{*}_{2}j^{2}\otimes H^{0}(X,K\otimes j^{-2}))$. Hence the exact
sequence obtained by taking direct image by $p_{2}$, namely,
$$
0\to H^{0}(X,K\otimes j^{-2})\to \ldots \to j^{-2}\to 0
$$
is again given by the canonical element. But this is also obtained by
taking the direct image of the sequence \eqref{art05-eq6.2}, thus
proving the first part of Lemma \ref{art05-lem6.1}. If $j^{2}=1$, then
$D_{j}=R^{1}(p_{2})_{*}(L^{-1}_{\Delta})\simeq
H^{1}(X,\mathscr{O})_{X}$ as is seen from the exact sequence
$$
0\to L^{-1}_{\Delta}\to \mathscr{O}\to \widetilde{\mathscr{O}}\to 0.
$$
If $F$ is a subbundle of degree 0, then $F\cap j^{2}=0$ or $F\supset
j^{2}$. In either case, the image of $F$ in $H^{1}(X,j^{2})_{X}$ is a
subbundle of degree 0 and hence trivial. If $F\cap j^{2}=0$, this
would imply that the extension $0\to j^{2}\to D_{j}\to
H^{1}(X,j^{2})_{X}\to 0$ splits over a trivial subbundle of
$H^{1}(X,j^{2})_{X}$ which is seen to be not possible, in view of our
description of this extension.
\end{proof}

\setcounter{theorem}{2}
\begin{lemma}\label{art05-lem6.3}
Let $C=\{(x,j)\in X\times J:j^{-2}\otimes L_{x}\in X\}$. Then
\begin{itemize}
\item[\rm(i)] the bundle $P(D)$ is trivial on $C$.

\item[\rm(ii)] $X\times j\subset C$ if and only if $j$ is an element of order $2$. 
\end{itemize}
\end{lemma}

\begin{proof}
\begin{itemize}
\item[(i)] On $X\times C$, the bundle $(p_{13})^{*}R^{2}\otimes
(p_{12})^{*}L^{-1}_{\Delta}$ is isomorpic to $p^{*}_{C}\zeta\otimes
(1\times \alpha)^{*}L^{-1}_{\Delta}$ where $\zeta$ is some line bundle
on $C$ and $\alpha:C\to X$ is the map defined by
$L_{\alpha(x,j)}=j^{-2}\otimes L_{x}$. Hence $D|C$ is isomorphic (by
base change theorem) to
$\zeta\otimes \alpha^{*}(R^{1}(p_{2})_{*}L^{-1}_{\Delta})$. But clearly
$R^{1}(p_{2})_{*}L^{-}_{\Delta}\simeq H^{1}(X,\mathscr{O})_{X}$. 

\item[(ii)] If $j$ is an element of order 2, clearly $X\times j\subset
C$. On the other hand, if $X\times j\subset C$, then
$D_{j}=D|_{X\times j}\simeq (\zeta|X\times j)\otimes$ trivial bundle,
and\pageoriginale hence by Lemma \ref{art05-lem6.1}, this is possible
only if $j$ is an element of order 2.
\end{itemize}

We now have a family $E$ of (triangular) vector bundles on $X\times
P(D)$ given by \cite[Lemma 2,4]{art05-key6}:
\setcounter{equation}{3}
\begin{equation}
0\to (1\times \pi)^{*}p^{*}_{13}R\otimes p^{*}_{2}\tau \to E\to
(1\times\pi)^{*}p^{*}_{13}R^{-1}\otimes
(1\times\pi)^{*}p^{*}_{12}L_{\Delta}\to 0\label{art05-eq6.4}
\end{equation}
where $\pi:P(D)\to X\times J$ is the projective fibration associated
to $D$.

The bundle $E$ on $X\times P(D)$ is a family of stable vector
bundles \cite[Lemma 10.1 (i)]{art05-key3} on $X$ of rank 2 and
determinants of the form $L_{x}$, $x\in X$. Thus there is an induced
classifying morphism $\theta_{E}:P(D)\to U_{X}$ fitting in an obvious
commutative diagram
\begin{equation}
\vcenter{
\xymatrix@=1.5cm{
P(D)\ar[d]_-{p_{1}\circ \pi}\ar[r]^-{\theta_{E}} &
U_{X}\ar[d]^-{\det}\\
X \ar[r]^-{\text{Id}} & X
}}\label{art05-eq6.5}
\end{equation}

We now wish to study the morphism $\theta_{E}$ and, in particular, its
differential. Clearly $E$ is a family of bundles with the triangular
group as structure group. As such, there is an infinitesimal
deformation map $T_{P(D)}\to R^{1}(p_{2})_{*}(\Delta(E))$ where
$\Delta(E)$ is the bundle of endomorphisms of $E$ preserving the exact
sequence \eqref{art05-eq6.4}. This is because the associated bundle
with the Lie algebra of the triangular group as fibre is
$\Delta(E)$. (Remark 2, 13, i).
\end{proof}

\setcounter{theorem}{5}
\begin{lemma}\label{art05-lem6.6}
The above infinitesimal deformation map induces an isomorphism of
$T_{p_{1}\circ \pi}$ with the kernel of
$R^{1}(p_{2})_{*}(\Delta(E))\to R^{1}(p_{2})_{*}(\mathscr{O})$ given
by the trace map $\Delta(E)\to \mathscr{O}$ and hence with
$R^{1}(p_{2})_{*}(S\Delta(E))$, where $S\Delta(E)$ is the subbundle of
$\Delta(E)$ consisting of endomorphisms of trace $0$.
\end{lemma}

\begin{proof}
In view of the diagram \ref{art05-eq6.5} it is clear that
$T_{p_{1}\circ \pi}$ is mapped into the kernel. From the diagram of
exact rows (on $P(D)$),
{\fontsize{9pt}{11pt}\selectfont
\[
\xymatrix@C=.35cm@R=.5cm{
0\ar[r] & T_{\pi}\ar[r]\ar[d] & T_{p_{1}\circ \pi}\ar[r]\ar[d]
& \pi^{*}p_{2}^{*}T_{J}\ar[r]\ar[d] & 0\\
0\ar[r] & R^{1}(p_{2})_{*}(E\otimes
(1\times \pi)^{*}(p^{*}_{13}R\otimes p^{*}_{12}L_{\Delta}^{-1}))\ar[r]
& R^{1}(p_{2})_{*}(\Delta(E))\ar[r] &
H^{1}(X,\mathscr{O})_{P(D)}\ar[r] & 0
}
\]}
we\pageoriginale see that it is enough to show that the infinitesimal
deformation map maps $T_{\pi}$ isomorphically on
$R^{1}(p_{2})_{*}(E\otimes R\otimes L_{\Delta}^{-1})$, since it is
obvious that the last vertical map is an isomorphism. Indeed, we have
\end{proof}

\begin{lemma}\label{art05-lem6.7}
Let $V$, $W$ be two simple vector bundles on $X$. Let
$P=PH^{1}(X,\Hom(W,V))$, and
$$
0\to p^{*}_{1}V\otimes p^{*}_{2}\tau\to E\to p^{*}_{1}W\to 0
$$
be the universal exact sequence on $X\times P$ (see \cite[Lemma
2.3]{art05-key6}). Then the infinitesimal deformation map of the
family $E$ of bundles with the evident parabolic group as structure
group, maps $T_{P}$ isomorphically onto the kernel of
$R^{1}(p_{2})_{*}(E\otimes p^{*}_{1}W^{*})\to R^{1}(p_{2})_{*}(\End\
W)$. 
\end{lemma}

\begin{proof}
From the definition of the universal extension, it follows that the
map $\mathscr{O}_{P}=(p_{2})_{*}(W\otimes W^{*})\to
R^{1}(p_{2})_{*}(V\otimes \tau \otimes W^{*})=H^{1}(X,V\otimes
W^{*})\otimes \tau$ is the natural inclusion. Hence its cokernel is
isomorphic to the tangent bundle $T_{P}$ of $P$. From the cohomology
exact sequence obtained from the given sequence tensored with $W^{*}$,
we obtain an isomorphism of $T_{P}$ with the kernel of
$R^{1}(p_{2})_{*}(E\otimes p^{*}_{1}W^{*})\to R^{1}(p_{2})_{*}(\End
W)$. We have to check that this identification is the infinitesimal
deformation map. This verification can be done by choosing splittings
for the given sequence over $U_{i}\times P$ where $(U_{i})$ is an open
covering of $X$ and expressing the infinitesimal deformation map in
terms of \v{C}ech cocycles with respect to this covering.
\end{proof}

\begin{proposition}\label{art05-prop6.8}
Recall that $C=\{(x,j)\in X\times J:j^{-2}\otimes L_{x}\in X\}$. The
differential $d\theta_{E}$ is injective outside $\pi^{-1}(C)$ and
$\ker d\theta_{E}$ is a line bundle on $\pi^{-1}(C)$. 
\end{proposition}

\begin{proof}
Consider the exact sequence obtained from \ref{art05-eq6.4}:
\setcounter{equation}{8}
\begin{equation}
0\to S\Delta(E)\to \Ad E\to (1\times \pi)^{*}(p^{*}_{13}R^{-2}\otimes
p^{*}_{12}L_{\Delta})\otimes p^{*}_{2}\tau^{-1}\to 0.\label{art05-eq6.9}
\end{equation}
Take the direct image on $P(D)$. The (zeroth) direct image of the
bundle on the right is zero since it is clearly torsion free and zero
outside $\pi^{-1}(C)$. Thus we get a short exact sequence (using
Lemma \ref{art05-lem6.6}) 
$$
0\to T_{p_{1}\circ\pi}\to R^{1}(p_{2})_{*}(\Ad E)\to
R^{1}(p_{2})_{*}(R^{-2}\otimes L_{\Delta})\otimes \tau^{-1}\to 0.
$$
From \cite[Lemma 2.6]{art05-key6}, we have $R^{1}(p_{2})_{*}(\Ad
E)=\theta^{*}_{E}T_{\det}$ and it is easy to see that the first map is
$d\theta_{E}$. Moreover, outside $\pi^{-1}(C)$, the last term in this
sequence is clearly locally free of rank $(g-2)$ while
its\pageoriginale restriction to $\pi^{-1}(C)$ is also locally free of
rank $(g-1)$. This proves the proposition.
\end{proof}

Identifying $\pi^{-1}(C)$ with $C\times P$ where
$P=PH^{1}(X,\mathscr{O})$, the restriction of $E$ to $X\times C\times
P$ can be described as follows.


\setcounter{theorem}{9}
\begin{lemma}\label{art05-lem6.10}
\begin{itemize}
\item[\rm(i)] There is a natural isomorphism $H^{1}(X\times C\times
P,p^{*}_{3}\tau\otimes p^{*}_{12}(1\times \alpha)^{*}L_{\Delta}^{-1})$
with $H^{1}(X,\mathscr{O})\otimes H^{1}(X,\mathscr{O})^{*}$. 

\item[\rm(ii)] The family $E\otimes p^{*}_{13}R^{-1}$ resticts to
$X\times C\times P$ as an extension 
$$
0\to p^{*}_{3}\tau\to E\otimes p^{*}_{13}R^{-1}\to
p^{*}_{12}(1\times \alpha)^{*}L_{\Delta}\to 0
$$
given by the canonical element in $H^{1}(X,\mathscr{O})\otimes
H^{1}(X,\mathscr{O})^{*}$, using the isomorphism in {\rm(i)}.
\end{itemize}
\end{lemma}

\begin{proof}
Note that $H^{1}(X\times C,(1\times \alpha)^{*}L^{-1}_{\Delta})\simeq
H^{0}(C,R^{1}(p_{2})_{*}(1\times \alpha)^{*}L^{-1}_{\Delta})\simeq
H^{0}(C, \alpha^{*}R^{1}(p_{2})_{*}L^{-1}_{\Delta})\simeq
H^{0}(C,\alpha^{*}R^{1}(p_{2})_{*}\mathscr{O}_{X\times X})\simeq
H^{1}(X,\mathscr{O})_{C}$. This proves (i). That the canonical element
gives the required extension follows from the definition of $E$ and
the base change theorem.
\end{proof}

\begin{proposition}\label{art05-prop6.11}
Let $0\to p^{*}_{2}\tau\to F'\to \mathscr{O}\to 0$ be the universal
extension of $\mathscr{O}$ by $\tau$ on $X\times
PH^{1}(X,\mathscr{O})$. Then the restriction of $T_{p_{1}\circ \pi}$
to $\pi^{-1}(C)=C\times P$ is isomorphic to the quotient of $F'\otimes
H^{1}(X,\mathscr{O})$ by the trivial subbundle contained in
$p^{*}_{2}\tau\otimes H^{1}(X,\mathscr{O})$.
\end{proposition}

\begin{proof}
We will apply Lemma \ref{art05-lem6.6}, and use the description of
$E|X\times C\times P$ given in Lemma \ref{art05-lem6.10}. Thus we have
the extension
$$
0\to p^{*}_{3}\tau\otimes
p^{*}_{12}(1\times \alpha)^{*}L^{-1}_{\Delta}\to
S\Delta(E)\to \mathscr{O}\to 0.
$$
This may be embedded in the commutative diagram
\setcounter{equation}{11}
\begin{equation}
\vcenter{
\xymatrix@C=.5cm@R=.5cm{
 & 0\ar[d] & 0\ar[d] & &\\
0\ar[r] & p^{*}_{3}\ar[d] \tau\otimes
(1\times\alpha)^{*}L^{-1}_{\Delta}\ar[r] & S\Delta(E)\ar[d]\ar[r]
& \mathscr{O}\ar@{=}[d]\ar[r] & 0\\
0\ar[r] & p^{*}_{3}\ar[d] \tau\ar[r] & F''\ar[d]\ar[r]
& \mathscr{O}\ar[r] & 0\\
 & p^{*}_{3}\tau\otimes \mathscr{O}_{\text{Graph~}\alpha}\ar@{}[r]|{=}\ar[d] &
 p^{*}_{3}\tau\otimes\mathscr{O}_{\text{Graph~}\alpha}\ar[d] & &\\
 & 0 & 0 & &
}}\label{art05-eq6.12}
\end{equation}
where the middle line is obtained from the top line as the push-out by
means of the map $p^{*}_{3}\tau \otimes
(1\times \alpha)^{*}L^{-1}_{\Delta}\to p^{*}_{3}\tau$. By
Lemma \ref{art05-lem6.6}, $T_{p_{1}\circ \pi}|C\times P$\pageoriginale is isomorphic
to the first direct image on $C\times P$ of $S\Delta(E)$. The middle
vertical line of the above diagram yields an isomorphism of this with
the first direct image of $F''$. For,
$R^{1}(p_{23})_{*}(p^{*}_{3}\tau \otimes \mathscr{O}_{\text{Graph~}\alpha})$
is clearly 0. Since $(p_{23})_{*}(p^{*}_{3}\tau)\to
(p_{23})_{*}(\tau\otimes \mathscr{O}_{\text{Graph~}\alpha})$ is
clearly an isomorphism, it follows that $(p_{23})_{*}F''\to
(p_{23})_{*}(p^{*}_{3}\tau\otimes \mathscr{O}_{\text{Graph~}\alpha})$
is surjective, proving our assertion above. It remains to compute
$R^{1}(p_{23})_{*}(F'')$. From the description of the extension in
Lemma \ref{art05-lem6.10} and hence that in the top horizontal line of
the diagram \ref{art05-eq6.12}, it can be checked that the element in
$H^{1}(X\times C\times P,p^{*}_{3}\tau)=H^{1}(X,\mathscr{O})\otimes
H^{1}(X,\mathscr{O})^{*}\oplus H^{1}(C,\mathscr{O})\otimes
H^{1}(X,\mathscr{O})^{*}$ given by the extension $F''$ is the element
$(Id, -\alpha^{*}(Id))$. Indeed, this follows on remarking that the
map $H^{1}(X,\mathscr{O})\simeq H^{1}(X\times
X,L_{\Delta}^{-1}\Delta)\to H^{1}(X\times X,\mathscr{O})\to
H^{1}(X,\mathscr{O})\oplus H^{1}(X,\mathscr{O})$ is given by
$(Id,-Id)$. Now our assertion follows from
\end{proof}

\setcounter{theorem}{12}
\begin{lemma}\label{art05-lem6.13}
Consider on $X\times X\times P$, the exact sequence
$$
0\to p^{*}_{3}\tau \to F_{0}\to \mathscr{O}\to 0
$$
given by the element $(Id,-Id)$ in $H^{1}(X\times
X,\mathscr{O})\otimes H^{0}(P,\tau)\simeq V\otimes V^{*}\oplus
V\otimes V^{*}$, with $V=H^{1}(X,\mathscr{O})$. The direct image on
$X\times P$ by $p_{23}$ of this sequence fits in a commutative diagram
{\fontsize{10pt}{11pt}\selectfont
\[
\xymatrix@C=.65cm@R=.7cm{
0\ar[r] & p^{*}_{2}T_{P}\ar[r] & R^{1}(p_{23})_{*}F_{0}\ar[r] &
H^{1}(X,\mathscr{O})_{X\times P}\ar[r] & 0\\
0\ar[r] & p^{*}_{2}\tau\otimes H^{1}(X,\mathscr{O})\ar[r]\ar[u] &
F'\otimes H^{1}(X,\mathscr{O})\ar[r]\ar[u] & H^{1}(X,\mathscr{O})_{X\times
P}\ar[r]\ar[u] & 0
}
\]}
where $0\to p^{*}_{2}\tau\to F'\to \mathscr{O}\to 0$ is the universal
extension on $X\times P$ and the first vertical map is induced by the
natural map $\tau\otimes H^{1}(X,\mathscr{O})\to T_{P}$ on $P$.
\end{lemma}

\begin{proof}
In the following Lemma, take $L=K_{X}$, $L'=\mathscr{O}$, $T=X\times
P$, and apply duality to prove the Lemma.
\end{proof}

\begin{lemma}\label{art05-lem6.14}
Let
$$
0\to p^{*}_{1}L\to M\to p^{*}_{1}L\otimes p^{*}_{2}L'\to 0
$$
be an exact sequence of vector bundles on the variety $X\times
T$. Assume that $\dim H^{0}(X\times t,M)$ is independent of $t$. Then
the extension
$$
0\to H^{0}(X,L)_{T}\to (p_{T})_{*}M\to \ker\delta\to 0,
$$
where\pageoriginale $\delta$ is the connecting homomorphism
$$
H^{0}(X,L)\otimes L'\to H^{1}(X,L),
$$ 
is given by the image under the
natural map 
$$
H^{1}(T,{L'}^{-1})\to H^{1}(T,(\ker \delta)^{*}\otimes
H^{0}(X,L))
$$ 
of the K\"unneth component in $H^{1}(T,{L'}^{-1})$ of the
element of $H^{1}(X\times T,\Hom(L,L)\otimes {L'}^{-1})$ determined by
the extension $M$.
\end{lemma}

\begin{proof}
Let $x_{1},\ldots,x_{N}$ be points of $X$ such that the evaluation map
$H^{0}(X,L)\to \sum\limits_{\oplus}L_{x_{i}}$ is an isomorphism. This 
fits in a commutative diagram
\[
\xymatrix{
0\ar[r] & H^{0}(X,L)_{T}\ar[d]\ar[r] & (p_{T})_{*}M\ar[d]\ar[r]
& \ker\delta\ar[d]\ar[r] & 0\\
0\ar[r] & \sum L_{x_{i}}\ar[r] & \sum M_{x_{i}\times T}\ar[r] &
L'\otimes \sum L_{x_{i}}\ar[r] & 0
}
\]
The latter sequence is clearly defined by the required element in
$H^{1}(T,{L'}^{-1})$. 
\end{proof}

\begin{proposition}\label{art05-prop6.15}
The differential $d\theta_{E}$ on $\pi^{-1}C=C\times P$ fits in an
exact sequence
$$
0\to \tau^{-1}\to
T_{p_{1}\circ \pi}|\pi^{-1}C\xrightarrow{d\Theta_{E}}\theta^{*}_{E}T_{\det}\to \tau^{-1}\otimes \pi^{*}\alpha^{*}F\to 0
$$
where $F$ is the vector bundle on $X$ defined by the exact sequence
$$
0\to T_{X}\to H^{1}(X,\mathscr{O})\to F\to 0
$$
corresponding to the canonical linear system and $\alpha:C\to X$ is
the map defined by $j^{-2}\otimes L_{x}=L_{\alpha(j,x)}$.
\end{proposition}

\begin{proof}
Note that if $\Delta$ denotes the diagonal divisor in $X\times X$,
then $(p_{2})_{*}(L_{\Delta})\\\simeq\mathscr{O}_{X}$ and
$R^{1}(p_{2})_{*}(L_{\Delta})\simeq F$. In fact, the first of these
isomorphisms is obvious and the second follows from the exact sequence
(on $X\times X$)
$$
0\to \mathscr{O}_{X\times X}\to L_{\Delta}\to \widetilde{T}_{X}\to 0
$$
where $\widetilde{T}_{X}$ is the sheaf on $X\times X$ supported on
$\Delta$ and inducing $T_{X}$ on it. Restricting \ref{art05-eq6.9} to
$X\times C\times P$, we get the sequence
$$
0\to S\Delta(E)\to \Ad E\to p^{*}_{3}\tau^{-1}\otimes
p^{*}_{12}(1\times \alpha)^{*}L_{\Delta}\to 0.
$$
The direct image of this sequence, using base change for $\alpha$,
yields the lemma, in view of Lemma \ref{art05-lem6.6}.

\end{proof}

We\pageoriginale will now compute the Hessian of the map $\theta_{E}$
restricted to $\pi^{-1}C=C\times P$. By
Proposition \ref{art05-prop6.8}, $\pi^{-1}C$ is the critical set for
$\theta_{E}$. Moreover, $\ker d\theta_{E}$ on $C\times P$ is
isomorphic to $\tau^{-1}$ by Proposition \ref{art05-prop6.15}. The
Hessian is thus a map $\tau^{-1}\otimes N_{C\times P,P(D)}\to \coker
d\theta_{E}|C\times P$. (See Remark \ref{art05-rem3.6}). Now
$N_{C\times P,P(D)}\simeq \pi^{*}N_{C,X\times
J}\simeq \pi^{*}\alpha^{*}N_{X,J^{1}}\simeq \pi^{*}\alpha^{*}F$ where
$F$ is the bundle defined in Proposition \ref{art05-prop6.15}. On the
other hand, by Proposition \ref{art05-prop6.15}, $\coker
d\theta_{E}|C\times P$ is isomorphic to
$\tau^{-1}\otimes \pi^{*}\alpha^{*}F$. With these identifications we have


\begin{proposition}\label{art05-prop6.16}
The Hessian on $\pi^{-1}C=C\times P$ of the map $\theta_{E}$
considered as a map
$\tau^{-1}\otimes \pi^{*}\alpha^{*}F\to \tau^{-1}\otimes \pi^{*}\alpha^{*}F$
is the identity map.
\end{proposition}

\begin{proof}
By Lemma \ref{art05-lem6.6}, the computation of the required Hessian
may be made using Remark \ref{art05-rems2.13}, i). From the description
of the family $E$ over $X\times \pi^{-1}C$ given in
Lemma \ref{art05-lem6.10}, we see that the Hessian of $\theta_{E}$ is
simply the map
{\fontsize{10pt}{12pt}\selectfont
$$
(p_{23})_{*}(\tau^{-1}\otimes \alpha^{*}L_{\Delta})\otimes
R^{1}(p_{23})_{*}(\alpha^{*}L^{-1}_{\Delta}\otimes E\otimes R^{-1})\to
R^{1}(p_{23})_{*}(\tau^{-1}\otimes \alpha^{*}L_{\Delta})
$$}
given by the cup product for the natural map
$$
(\tau^{-1}\otimes \alpha^{*}L_{\Delta})\otimes
(\alpha^{*}L^{-1}_{\Delta}\otimes E\otimes R^{-1})\to \tau^{-1}\otimes
E\otimes R^{-1}\to \tau^{-1}\otimes \alpha^{*}L_{\Delta}.
$$
From this it is clear that this map is the identity on $\tau^{-1}$
tensored with the induced map
$$
(p_{23})_{*}(\alpha^{*}L_{\Delta})\otimes
R^{1}(p_{23})_{*}(\mathscr{O})\to
R^{1}(p_{23})_{*}(\alpha^{*}L_{\Delta}).
$$ 
Now since
$(p_{23})_{*}(\alpha^{*}L_{\Delta})$ is canonically trivial, this is
simply the inverse image by $\alpha\circ p_{1}$ of the map
$H^{1}(X,\mathscr{O})_{X}\to R^{1}(p_{2})_{*}(L_{\Delta})$. This is
clearly the natural map $H^{1}(X,\mathscr{O})_{X}\to F$, proving our
assertion. 
\end{proof}

\begin{remark}\label{art05-rem6.17}
From Proposition \ref{art05-prop6.16}, we see that $\pi^{-1}C$ is a
`nondegenerate' critical manifold for $\theta_{E}$. 
\end{remark}

\begin{proposition}\label{art05-prop6.18}
For every $j\in J$, the map $\theta_{E}:\pi^{-1}(X\times j)\to U_{X}$
is an imbedding.
\end{proposition}

\begin{proof}
We first remark that in view of \ref{art05-eq6.5}, it is enough to
prove that $\theta_{E}:\pi^{-1}(x,j)\to U_{X}$ is an imbedding, for
every $x\in X$ and $j\in J$. Secondly, from \cite[Lemma
10.1]{art05-key3}, it follows that this map is injective. Moreover, if
$(x,j)\in C$, then by Proposition \ref{art05-prop6.8},
$\theta_{E}:\pi^{-1}(x,j)\to U_{x}$ is an\pageoriginale
imbedding. Finally, let $(x,j)\in C$. In order to show that
$d\theta_{E}|T_{\pi}$ is injective, it is enough to prove that the
composite of the inclusion $\tau^{1}\to T_{p_{1}\circ\pi}|_{C\times
P}$ with the differential $d\pi
:T_{p_{1}\circ\pi}\to \pi^{*}T_{p_{1}}=H^{1}(X,\mathscr{O})_{C\times
P}$ is injective. Consider the commutative diagram (on $\pi^{-1}C$)
{\fontsize{9pt}{11pt}\selectfont
\[
\xymatrix@C=.25cm{
0\ar[r] & S\Delta(E)\ar[d]\ar[r] & \Delta(E)\ar[d]\ar[r] &
(1\times \pi)^{*}(p^{*}_{13}R^{-2}\otimes
p^{*}_{12}L_{\Delta})\otimes p^{*}_{2}\tau^{-1}\ar@{=}[d]\ar[r] & 0\\
0\ar[r] & \mathscr{O}\ar[r] & E\otimes
(1\times \pi)^{*}p^{*}_{13}R^{-1}\otimes p^{*}_{2}\tau^{-1}\ar[r] &
''\ar[r] & 0
}
\]}
From this we conclude that this composite is given by the boundary
homomorphism of the lower sequence. Now our assertion follows from the
definition of the extension $E$.
\end{proof}

The nonsingular subvarieties $\theta_{E}(\pi^{-1}(X\times j))\subset
U_{X}$ will also be denoted $P(D_{j})$.

\begin{proposition}\label{art05-prop6.19}
If $j_{1}\neq j_{2}$ or $j^{-1}_{2}$, then $P(D_{j_{1}})$ intersects
$P(D_{j_{2}})$ in only finitely many poins. If $j^{2}\neq 1$, then
$P(D_{j})$ and $P(D_{j^{-1}})$ intersect transversally along the sections
given by the exact sequence in Lemma \ref{art05-lem6.1}. 
\end{proposition}

\begin{proof}
Let $j_{1}\neq j_{2}$, $j^{-1}_{2}$ and $W\in P(D_{j_{1}})\cap
P(D_{j_{2}})$. Then $W$ is given by an extension
$$
0\to j_{1}\to W\to j_{1}^{-1}\otimes L_{x}\to 0.
$$
and $W$ contains $j_{2}$ also. The existence of a nonzero homomorphism
$j_{2}\to j^{-1}_{1}\otimes L_{x}$ implies that $j_{1}\otimes
j_{2}=L_{x}\otimes L_{y}^{-1}$ for some $y\in X$. This proves that
there are (if at all) only finitely many solutions for $x$ and $y$. If
$(x,y)$ is one such, then $E$ has to be in the kernel of the map
$H^{1}(j^{2}_{1}\otimes L_{x}^{-1})\to H^{1}(j^{2}_{1}\otimes
L_{x}^{-1}\otimes L_{y})$ which is at most 1-dimensional. Thus $W$ is
uniquely determined by such a choice of $x$ and $y$. This proves our
first assertion.

Now let $W\in P(D_{j})\cap P(D_{j^{-1}})$, $j^{2}\neq 1$. As above, we
see that $W$ is the unique bundle given by the kernel of
$H^{1}(X,j^{2}\otimes L^{-1}_{x})\to H^{1}(X,j^{2})$. Clearly, this is
the kernel defined in Lemma \ref{art05-lem6.1}. It remains to prove
the transversality of the intersection of $P(D_{j})$ and
$P(D_{j^{-1}})$. Now $W$ is given simultaneously by two exact sequences 
\begin{align*}
 & 0\to j\to W\to j^{-1}\otimes L_{x}\to 0\\
\text{and}\qquad & 0\to j^{-1}\to W\to j\otimes L_{x}\to 0.
\end{align*}\pageoriginale
Consider the map $j\oplus j^{-1}\to W$. Since $j\neq j^{-1}$, this map
is a generic isomorphism and hence fits in an exact sequence
$$
0\to j\oplus j^{-1}\to W\to \mathscr{O}_{x}\to 0.
$$
Let $M\subset \Ad W$ be the bundle of endomorphisms (with trace $0$)
of $W$ taking $j$ into $j^{-1}$. Then we have the exact sequence
\setcounter{equation}{19}
\begin{equation}
0\to M\to \Ad W\to L_{x}\to 0.\label{art05-eq6.20}
\end{equation}
On the other hand, both $\Hom (j^{-1}\otimes L_{x},j)=j^{2}\otimes
L^{-1}_{x}$ and $\Hom(j\otimes L_{x},j^{-1})=j^{-2}\otimes L_{x}^{-1}$
are clearly subbundles of $M$. At any $y\neq x$, we have
$W_{y}=j_{y}\oplus j^{-1}_{y}$ and these two subspaces of $M_{y}$ may
be represented in matrix form by $\left(\begin{smallmatrix} 0 &
0\\ \ast & 0\end{smallmatrix}\right)$ and $\left(\begin{smallmatrix} 0 &
\ast\\ 0 & 0\end{smallmatrix}\right)$ and hence are linearly
independent. In other words, the map $j^{2}\otimes L^{-1}_{x}\oplus
j^{-2}\otimes L_{x}^{-1}\to M$ fails to be of maximal rank only at
$x$. Since $\det M=L^{-1}_{x}$, we have the exact sequence
$$
0\to j^{2}\otimes L^{-1}_{x}\oplus j^{-2}\otimes L^{-1}_{x}\to
M\to \mathscr{O}_{x}\to 0.
$$
In particular, the natural map $H^{1}(j^{2}\otimes L^{-1}_{x})\oplus
H^{1}(j^{-2}\otimes L_{x}^{-1})\to H^{1}(M)$ is
surjective. From \eqref{art05-eq6.20}, we conclude that the map
$H^{1}(M)\to H^{1}(\Ad W)$ has as cokernel $H^{1}(X,L_{x})$ which is
of dimension $(g-1)$. Thus the images of $H^{1}(j^{2}\otimes
L^{-1}_{x})$, $H^{1}(j^{-2}\otimes L^{-1}_{x})$ in $H^{1}(\Ad W)$ by
the natural inclusions have sum of dimension $(2g-2)$. But the image
of $H^{1}(j^{2}\otimes L^{-1}_{x})$ in $H^{1}(\Ad W)$ is the tangent
space at $W$ to the cycle $P(W^{*}_{x})$ in $U_{x}$. This proves the
transversality, since $\dim U_{x}=3g-3$.  
\end{proof}

\setcounter{theorem}{20}
\begin{lemma}\label{art05-lem6.21}
Let $j\in J$ with $j^{2}\neq 1$. Then identifying $X$ with the
intersection of $P(D_{j})$ and $P(D_{j^{-1}})$, we have the exact
sequence
$$
0\to T_{\pi j}\oplus T_{\pi j^{-1}}\to T_{\det}|X\to F\to 0
$$
where $T_{\pi j}$ is the restriction to $X$ of the tangent bundle
along the fibres of $P(D_{j})$ and $F$ is the quotient bundle
associated to the canonical linear system on $X$.
\end{lemma}

\begin{proof}
The exact sequence \ref{art05-eq6.20} when $x$ is also varied, yields
a sequence on $X\times X$
$$
0\to M\to \Ad W\to L_{\Delta}\to 0.
$$
Taking\pageoriginale direct image on $X$, we get
$$
R^{1}(p_{1})_{*}M\to T_{\det}|X\to F\to 0.
$$
On the other hand, the map $R^{1}(p_{1})_{*}(j^{2}\otimes
L^{-1}_{\Delta})\oplus R^{1}(p_{1})_{\ast}(j^{-2}\otimes
L^{-1}_{\Delta})\to R^{1}(p_{1})_{*}M$ has been proved to be
surjective. As in Proposition \ref{art05-prop6.19}, we see that the
image of $R^{1}(p_{1})_{*}(j^{2}\otimes L_{\Delta}^{-1})$ is $T_{\pi
j}$, proving our assertion.
\end{proof}

\begin{lemma}\label{art05-lem6.22}
\begin{itemize}
\item[\rm(i)] If $j^{2}\neq 1$, then we have the exact sequence on
$P(D_{j}):$
$$
0\to H^{1}(X,\mathscr{O})_{P(D_{j})}\to
N_{P(D_{j}),U_{X}}\to \tau^{-1}\otimes \pi^{*}F(j)\to 0
$$
where $F(j)$ is the quotient sheaf defined by the linear system
$K\otimes j^{2}$.

\item[\rm(ii)] If $j^{2}=1$, then we have the exact sequence on
$P(D_{j})=X\times P$ with $P=PH^{1}(X,\mathscr{O})$
$$
0\to \Iim d\theta_{E}|P(D_{j})\to
T_{\det}|P(D_{j})\to \tau^{-1}\otimes \pi^{*}F\to 0
$$
where $F$ is the quotient bundle on $X$ defined by the canonical
linear system. Moreover, we have a commutative diagram on $P(D_{j})$:
{\fontsize{10pt}{12pt}\selectfont
\[
\xymatrix@C=.5cm{
0\ar[r] & \tau\otimes H^{1}(X,\mathscr{O})\ar[d]\ar[r] & F'\otimes
H^{1}(X,\mathscr{O})\ar[d]\ar[r] & H^{1}(X,\mathscr{O})\ar[d]\ar[r] &
0\\
0\ar[r] & p^{*}_{2}T_{P}\ar[r] & \Iim d\theta_{E}\ar[r]
& \tau^{-1}\otimes p^{*}_{2}T_{p}\ar[r] & 0
}
\]}
where the extreme vertical maps are the natural ones and the first
extension is obtained by tensoring the universal extension on $X\times
P$ with $H^{1}(X,\mathscr{O})$.
\end{itemize}
\end{lemma}

\begin{proof}
\begin{itemize}
\item[\rm(i)] This follows from the sequence \eqref{art05-eq6.9} and
its direct image sequence noting that since $X\times j\nsubset C$ by
Lemma \ref{art05-lem6.3} ii) and Proposition \ref{art05-prop6.8}, the
map $T_{p_{1}\circ \pi}\to T_{U_{X}}$ restricted to $P(D_{j})$ is
still (sheaf theoretically) injective. The cokernel is clearly the
tensor product of $\tau^{-1}$ and the sheaf
$R^{1}(p_{1})_{*}(p^{*}_{2}j^{-2}\otimes L_{\Delta})$. From the exact
sequence associated to 
$$
0\to p^{*}_{2}j^{-2}\to p^{*}_{2}j^{-2}\otimes L_{\Delta}\to
j^{-2}\widetilde{\otimes} K^{*}_{X}\to 0
$$
we identify this sheaf with $F(j)$.

\item[\rm(ii)] This is an immediate consequence of
Propositions \ref{art05-prop6.15} and \ref{art05-prop6.11}.
\end{itemize}
\end{proof}

\begin{corollary}\label{art05-coro6.23}
For\pageoriginale any $j\in J$, we have
$$
K^{*}_{\det}|P(D_{j})\simeq \tau^{2}\otimes \pi^{*}(j^{4}\otimes K).
$$
\end{corollary}

\begin{proof}
From the exact sequence
$$
0\to K^{*}\otimes j^{-2}\to H^{1}(X,j^{-2})_{X}\to F(j)\to 0
$$
we see that $\det F(j)=K\otimes j^{2}$ and now from
Lemma \ref{art05-lem6.22}, i) we get $\det
T_{\det}|P(D_{j})=\tau^{-(g-2)}\otimes \det F(j)\otimes \det
T_{\pi}=\tau^{-(g-2)}\otimes K\otimes j^{2}\otimes \tau^{g}\otimes
j^{2}=\tau^{2}\otimes j^{4}\otimes K$, as claimed. The case $j^{2}=1$
follows similarly from  
\ref{art05-lem6.22}, ii). 
\end{proof}

\section{Limits of good Hecke cycles}\label{art05-sec7}

In this article, we study the limis of good Hecke cycles and their
normal bundles in $U_{X}$. We first wish to fix an ample line bundle
$\mathscr{O}(1)$ on $U_{X}$. 

\begin{lemma}\label{art05-lem7.1}
If $K_{\det}$ denotes the canonical line bundle along the fibres of
the fibration $\det : U_{X}\to X$, then the line bundle
$\mathscr{O}(1)=K^{*}_{\det}\otimes (\det)^{*}K_{X}$ is an ample
bundle on $U_{X}$.
\end{lemma}

\begin{proof}
Fix $\alpha\in J^{1}$. Consider the map $f:J\to J^{1}$ given by
$f(j)=j^{2}\otimes \alpha$. Then we have a commutative diagram
\[
\xymatrix{
f^{*}U(2,1)\ar[d]\ar[r] & U(2,1)\ar[d]\\
J\ar[r]^-{f} & J^{1}
}
\]
Now $f^{*}(U(2,1))\to J$ is isomorphic to the product
$U(2,\alpha)\times J\to J$. This gives in turn a commutative diagram
\[
\xymatrix{
U(2,\alpha)\times \widetilde{X}\ar[d]\ar[r]^-{\widetilde{f}} &
U_{X}\ar[d]\\
\widetilde{X}=f^{-1}(X)\ar[r]^-{f} & X
}
\]
Since $f$ is \'etale surjective, $X$ is nonsingular and
$f^{*}K_{X}=K_{X}$. On the other hand, $f^{*}K_{\det}\simeq
p^{*}_{1}K_{U_{\alpha}}$, so that our assertion follows
from \cite[Proposition 4.4]{art05-key2} and the ampleness of $K^{*}$
on $U_{\alpha}$ (See \cite[Theorem 1]{art05-key6}) and of $K$ on $X$.
\end{proof}

\begin{lemma}\label{art05-lem7.2}
Let\pageoriginale $Z$ be a good Hecke cycle. Then the polynomial
$P(m,n)=\chi(Z,K^{-m}_{\det}\otimes dp^{*}K^{n}_{X})$ is given by
$(4m+1)(2m+2n-1)\times (g-1)$. In particular, the Hilbert polynomial
of a good Hecke cycle (with respect to $\mathscr{O}(1)$) is
$P(n)=(4n+1)(4n-1)(g-1)$. 
\end{lemma}

\begin{proof}
If $W$ is the vector bundle to which the Hecke cycle $Z$ corresponds,
then by \ref{art05-coro5.16}, we have
$(K^{-m}_{\det}\otimes \det^{*}K^{n}_{X})|_{Z}\simeq
K^{-2m}_{\pi}\otimes \pi^{*}K^{m+n}_{X}$. Now
$$
\pi_{*}(K^{-2m}_{\pi}\otimes \pi^{*}K^{m+n}_{X})\simeq (\det
W)^{2m}\otimes S^{4m}(W^{*})\otimes K^{m+n}_{X}.
$$
Since the higher direct images are zero for $m\geq 0$, we have
$$
P(m,n)=\chi(X,(\det W)^{2m}\otimes S^{4m}(W^{*})\otimes K^{m+n}_{X})
$$
and the latter is computed to be as claimed, using the Riemann-Roch
theorem. 
\end{proof}

Let $p:H\to U_{X}$ be the restriction of the dual Poincar\'e bundle on
$X\times U_{X}$ to $U_{X}$ considered as a divisor by the map (det,
Id). By 5.2, v) and Lemma \ref{art05-lem5.5}, we obtain a map $h:H\to
U_{0}$ which is a projective fibration over the set of stable points
of $U_{0}$ by \ref{art05-rem5.2}, iv) and
Lemma \ref{art05-lem5.14}. The morphism $\theta_{E}:P(D)\to U_{X}$ (in
the notation of \S\ \ref{art05-sec6}) lifts to a map
$\widetilde{\theta}_{E}:P(D)\to H$ since $\theta^{*}_{E}H\simeq
P(E)^{*}$ and $E^{*}$ has a natural line subbundle by
construction. Let $\mathscr{K}$ be the variety of nonstable points of
$U_{0}$, namely the Kummer variety $\mathscr{K}=J/i$, where $i$ is the
involution $j\mapsto j^{-1}$.

\begin{lemma}\label{art05-lem7.3}
We have a commutative diagram
\[
\xymatrix@C=2.5cm@R=1.5cm{
P(D)\ar[d]_-{p_{2}\circ \pi}\ar[r]^-{\theta_{E}} &
h^{-1}(\mathscr{K})\subset H\ar[d]^-{h}\\
J\ar[r] & \mathscr{K}
}
\]
Moreover $\widetilde{\theta}_{E}$ is onto $h^{-1}(\mathscr{K})$.
\end{lemma}

\begin{proof}
A point of $P(D)$ is represented by an exact sequence
$$
0\to j\to E\to j^{-1}\otimes L_{x}\to 0
$$
Its\pageoriginale image in $H$ is given by the element $E\in U_{X}$
and the one-dimensional space of linear forms on $E_{x}$ vanishing on
$j_{x}$. Now the construction $K$ on $H$ which defines the morphism
$h$ associates to it the bundle $F$ obtained by
$$
0\to F\to E\to \mathscr{O}_{x}\to 0
$$
given by the above linear form. It is obvious $j$ is a subbundle of
$F$ and hence $F$ is $S$-equivalent to $j\oplus j^{-1}$ proving the
commutativity of the diagram. Now a point of $h^{-1}(\mathscr{K})$ is
given by a stable bundle $E\in U_{x}$, for some $x\in X$, and a linear
form on $E_{x}$ such that the bundle $F$ defined by the sequence
$$
0\to F\to E\to \mathscr{O}_{x}\to 0
$$
is non-stable. Let $j\subset F$. Clearly then $j$ is also contained in
$E$, for otherwise some line bundle of the form $j\otimes L_{D}$,
$D>0$ will be contained in $E$ contradicting its stability. Thus $E$
can be written as an extension
$$
0\to j\to E\to j^{-1}\otimes L_{x}\to 0
$$
proving the lemma.
\end{proof}

\begin{lemma}\label{art05-lem7.4}
\begin{itemize}
\item[\rm(i)] If $k\in \mathscr{K}$ is not a node, then the reduced
fibre of $h$ over $k$ is the push-out of $P(D_{j})$ and $P(D_{j^{-1}})$
along the natural section $X$, where $j$, $j^{-1}$ are points in $J$
over $k$.

\item[\rm(ii)] If $k\in \mathscr{K}$ is the image of an element $j\in
J$ of order $2$, then the reduced fibre of $h$ over $k$ is $X\times P$
where $P=PH^{1}(X,\mathscr{O})$. Moreover the schematic fibre of $h$
contains as a subscheme $X\times P_{t}$ where $P_{t}$ is the
thickening of $P$ by $0\to Q^{*}\to \mathscr{F}\to \Omega^{1}_{P}\to
0$, where $Q$ is the universal, quotient bundle of $P$.
\end{itemize}
\end{lemma}

\begin{proof}
\begin{itemize}
\item[(i)] follows from diagram \ref{art05-lem7.3} and the fact that
images by $\widetilde{\theta}_{E}$ of $(P(D_{j}))$ and $P(D_{j^{-1}})$
intersect transversally along $X$ since transversatility is true of
$\theta_{E}$ (Lemma \ref{art05-prop6.19}).

\item[(ii)] Clearly the total thickening $Z$ of $P(D_{j})$ in $P(D)$
is in the fibre over $k$ of the morphism
$h\circ \widetilde{\theta}_{E}$. Thus we have only to check that
$\theta_{E}$ induces an epimorphism of $Z$ onto $X\times P_{t}$ and
that the latter is a subscheme of $H$. But this follows from
Remark \ref{art05-rem3.6}. 
\end{itemize}
\end{proof}

\begin{remark}\label{art05-rem7.5}
The\pageoriginale reduced scheme defined in \ref{art05-lem7.4}, (i) will be denoted
$F_{k}$, if $k$ is not the image of an element of order 2. If it is,
then $F_{k}$ will denote the subscheme $X\times
PH^{1}(X,\mathscr{O})_{t}$ of $U_{X}$. These are contained in the
fibres of $H\to U_{0}$. Probably, these are precisely the schematic
fibres, and some of the technical difficulties in this article can be
obviated if one could prove this directly.
\end{remark}

\begin{lemma}\label{art05-lem7.6}
The natural morphism of the relative Hilbert scheme\break $\Hilb
(H,U_{0},P(n))$ into $\Hilb(U_{X},P(n))$ is injective.
\end{lemma}

\begin{proof}
Since $P(n)$ is of degree $2$ in $n$, any element of
$\Hilb(U_{X},P(n))$ represents a subscheme of $U_{X}$ of dimension
$2$. Hence our assertion follows from Proposition \ref{art05-prop6.19}.
\end{proof}

Let $\Hilb(H,U_{0},P(m,n))$ be the relative Hilbert scheme of the map
$H\to U_{0}$ of cycles $Z$ satisfying
$\chi(Z,p^{*}(K^{-m}_{\det}\otimes \det^{*}K^{n}_{X})=P(m,n)$. If
$H_{2}$ is the image of $\Hilb(H,U_{0},P(m,n))$ in $\Hilb
(U_{X},P(n))$ then $H_{2}$ contains good Hecke cycles. Now if $Z\in
H_{2}$ is not a good Hecke cycle, then $Z_{\red}\subset F_{k}$ for a
unique $k\in \mathscr{K}$. For a fixed $k\in \mathscr{K}$ the set of
elements of $H_{2}$ which are contained set theoretically in $F_{k}$
is denoted $H_{2,k}$. 

\setcounter{subsection}{6}
\subsection{Definition of The Varieties \texorpdfstring{$Q_{j}$, $Q_{k}$,
$R_{k}$}{Qj}.}\label{art05-sec7.7} 

Now we will define for $k$ not a node, three subvarieties $Q_{j}$,
$Q_{j^{-1}}$ and $R_{k}$ (with $j$, $j^{-1}\in J$ lying over $k$) of
$H_{2,k}$. Consider the Grassmannian of lines in $PH^{1}(X,j^{2})$;
the symmetric product of the projective line bundle over it will be
denoted $Q_{j}$. Also, we denote by $R_{k}$ the product of
$PH^{1}(X,j^{2})$ and $P(H^{1}(X,j^{-2}))$ for $j$ lying over
$k\in \mathscr{K}$. Now points of $Q_{j}$, $Q_{j^{-1}}$ and $R_{k}$ can
be considered as subschemes of $F_{k}$. In fact, consider the exact
sequences 
\begin{align*}
& 0\to j^{2}\to D_{j}\to H^{1}(X,j^{2})_{X}\to 0\\
& 0\to j^{-2}\to D_{j^{-1}}\to H^{1}(X,j^{-2})_{X}\to 0.
\end{align*}
Any line in $PH^{1}(X,j^{2})$ gives rise to a plane subbundle of
$P(D_{j})$ over $X$. A point of $Q_{j}$ gives rise a pair of
projective line subbundles (possibly identical) of this plane
bundle. Thus we obtain a subscheme of $F_{k}$. Similarly a point of
$R_{k}$ gives rise to projective line subbundles of $P(D_{j})$,
$P(D_{j^{-1}})$ containing the sections of $P(D_{j})$, $P(D_{j^{-1}})$ given
by\pageoriginale $j^{2}$ and $j^{-2}$ respectively. To compute the
Hilbert polynomials of these schemes, we may assume without loss of
generality that the scheme is obtained by gluing two projective line
bundles $P(F_{1})$, $P(F_{2})$ on $X$ where $F_{1}$ and $F_{2}$ are of
degree zero and contain either $j^{2}$ or $j^{-2}$ as line subbundles,
along the sections given by these subbundles. Now, from Corollary 
\ref{art05-lem6.22}, it follows that $\mathscr{O}(1)$ restricts to
$P(D_{j})$ as $K^{2}_{X}\otimes j^{4}\otimes \tau^{2}$. Its
restriction to the section of $P(D_{j})$ is therefore seen to be
$K^{2}_{X}$. Hence the Hilbert polynomial of the subschemes
corresponding to points of $Q_{j}$, $R_{k}$ is, on using the
Mayer-Vietoris sequence,
$$
2\chi (X,K^{m+n}_{X}\otimes
S^{2m}(F^{*}))-\chi(X,K^{m+n}_{X})=(4m+1)(2m+2n-1)(g-1). 
$$
Thus, we have identified the sets $Q_{j}$, $Q_{j^{-1}}$, $R_{k}$ with
subsets of $H_{2,k}$.

If $k$ is a node, $Q_{k}$ is defined to be the variety of subschemes
of the form $X\times C$, where $C$ is a conic in
$P=PH^{1}(X,\mathscr{O})$, while $R_{k}$ denotes those of the form
$X\times C$ where $C$ is a line thickened by $\tau^{-1}$ contained in
$P_{t}$ but not in $P$. As above $Q_{k}$, $R_{k}$ are checked to be
subsets of $H_{2,k}$. 

\setcounter{theorem}{7}
\begin{proposition}\label{art05-prop7.8}
If $k$ is not a node of $\mathscr{K}$, and $j$, $j^{-1}\in J$ are
points over $k$, then $H_{2,k}=Q_{j}\cup Q_{j^{-1}}\cup R_{k}$. If $k$ is
a node of $\mathscr{K}$, then $H_{2,k}=Q_{k}\cup R_{k}$. 
\end{proposition}

The rest of this section will be devoted to the proof of
Proposition \ref{art05-prop7.8}. 

\begin{lemma}\label{art05-lem7.9}
If $Z\in H_{2}$, then the map $p:Z\to X$ is surjective and for generic
$x\in X$, the fibre $Z_{x}$ has $(2m+1)$ as the Hilbert polynomial
with respect to the ample generator $h$ of $\Pic U_{x}$.
\end{lemma}

\begin{proof}
Since the polynomial $P(m,n)$ is not independent of $n$, it follows
that the map $p:Z\to X$ is surjective. For large $m$, we have
$$
P(m,n)=\chi(X,p_{*}(K^{-m}_{\det})\otimes K^{n}_{X}).
$$
In other words, $P(m,n)$ is the Hilbert polynomial of
$p_{*}(K^{-m}_{\det})$ with respect to the ample line bundle $K_{X}$
on $X$. Hence the rank of the sheaf $p_{*}(K^{-m}_{\det})$ is
$\dfrac{1}{\deg K}$ (the coefficient of $n$ in $P(m,n)$), namely
$\dfrac{1}{2g-2}(8m+2)(g-1)=4m+1$. But $K^{*}_{\det}$ restricts to
$U_{x}$ as $K^{*}_{U_{x}}$, which is twice the ample generator of
$\Pic U_{x}$. Hence the lemma.
\end{proof}

\begin{lemma}\label{art05-lem7.10}
Let\pageoriginale $S$ be a subscheme of $U_{x}$ having $2m+1$ as
Hilbert polynomial. If $S_{\red}\subset (F_{k})_{\red}$, for some
$k\in \mathscr{K}$, then $S$ is a conic contained schematically in
$F_{k}$. 
\end{lemma}

\begin{remark}\label{art05-rem7.11}
The proof of Lemma \ref{art05-lem7.10} given below can be simplified
if one knew a priori that $h|S$ is very ample. This would follow for
instance if we could show that $h$ itself is very ample which is very
likely to be true and indeed so at least if $X$ is
hyperelliptic \cite[5.10, II]{art05-key1}. However, since the
irreducible components of $(F_{k})_{\red}\cap U_{x}$ are projective
spaces, the restriction of $h$ to these is very ample, as ample
bundles on a projective space are very ample. From this it is easy to
see that $h|(F_{k})_{\red}\cap U_{x}$ is very ample and hence
$h|S_{\red}$ is very ample. This fact will be used in the proof.
\end{remark}

\noindent
{\bf Proof of Lemma \ref{art05-lem7.10}.}~
Let $S$ be a subscheme of $U_{x}$ having $2m+1$ as Hilbert
polynomial. If $S$ is reduced, our assertion follows from
Lemma \ref{art05-lem4.2} and Remark \ref{art05-rems4.4}. If $S$ is not
reduced, we shall use Lemma \ref{art05-lem4.3}. Let $S'$ be the
thickening of $S_{\red}=\bfP^{1}$ by the line bundle $L$ mentioned in
that lemma. We first show that $\deg L\geq -1$. Since $S'$ is a
subscheme of $U_{x}$, the bundle $L$ is a quotient of the conormal
bundle of $S_{\red}$ in $U_{x}$. Now we have the exact sequences
(Lemma \ref{art05-lem6.21})
$$
0\to \tau\otimes \text{~trivial~}\to
N^{*}_{P(D)_{(j,x)},U_{x}}\to \text{~trivial~}\to 0
$$
and
$$
0\to \tau\otimes \text{~trivial~}\to
N^{*}_{P(D)_{(j,x)},U_{x}} \to \tau\otimes \Omega^{1}\to 0
$$
according as $j^{2}\neq 1$ or $j^{2}=1$. On the other hand, we have
the exact sequence
$$
0\to N^{*}_{P(D)_{(j,x)},U_{x}}\to
N^{*}_{\bfP^{1},U_{x}}\to \tau^{-1}\otimes \text{~trivial~}\to 0.
$$
Since $\tau\otimes \Omega^{1}$ restricts to $\bfP^{1}$ as
$\tau^{-1}\oplus$ trivial, it follows that any quotient line bundle of
$N^{*}_{\bfP^{1},U_{x}}$ is of $\deg \geq -1$. Now if $j^{2}\neq 1$,
any quotient of $N^{*}_{\bfP^{1},U_{x}}$ isomorphic to $\tau^{-1}$ is
actually a quotient of $N^{*}_{\bfP^{1},P(D)_{(j,x)}}$ so that
$S\subset P(D)_{(j,x)}$. If $j^{2}=1$, we similarly conclude that
$S\subset F_{k}=PH^{1}(X,\mathscr{O})_{t}$. This proves the lemma.

\begin{lemma}\label{art05-lem7.12}
If $Z\in H_{2,k}$, then the morphism $p:Z\to X$ is flat.
\end{lemma}

\begin{proof}
Using\pageoriginale generic flatness, we see that on an open subset of
$X$, $p$ itself defines a section of a suitable relative Hilbert
scheme for the morphism $Z\to X$. This extends to a section over the
whole of $X$ and thus gives a closed subscheme $Z'$ of $Z$ flat over
$X$. Now from Lemma \ref{art05-lem7.9}, it follows that {\em all} the
fibres $Z'_{x}$ of $Z'$ have Hilbert polynomial $2m+1$ with respect to
$h$. Now by Lemma \ref{art05-lem7.10}, it follows that $Z'_{x}\subset
F_{k}$ for each $x\in X$. Since tha map $p:Z'\to X$ is flat, it
follows that $Z'\subset F_{k}$. Now suppose we show that
$K^{m}_{\det}|F_{k}$ has a direct image $V$ on $X$ of the form
$K^{4m}_{X}\otimes$ (a semistable vector bundle of degree $0$), at
least for large $m$. Then we would have $p_{*}(K^{m}_{\det}|Z')$ is a
quotient of $V$ for large $m$ and hence would be a vector bundle of
rank $4m+1$ and degree $\geq 2m(g-1)(4m+1)$. Hence
\begin{align*}
& \chi(Z',K^{m}_{\det}\otimes
p^{*}K^{n}_{X})=\chi(X,p^{*}(K^{m}_{\det})\otimes K^{n}_{X})\\
& \geq (4m+1)(2m+2n-1)(g-1)=P(m,n)
\end{align*}
by Riemann-Roch theorem. On the other hand, if $I$ is the sheaf of
ideals defining $Z'$ in $Z$, then from the exact sequence
$$
0\to I\to \mathscr{O}_Z\to \mathscr{O}_{Z'}\to 0,
$$
we conclude that $\chi(Z,\mathscr{O}(n))\geq \chi(Z',\mathscr{O}(n))$
for large $n$, and that equality holds only if $Z=Z'$. It remains to
prove 
\end{proof}

\begin{lemma}\label{art05-lem7.13}
The direct image of $K^{m}_{\det}|F_{K}$ on $X$ is of the form
$K^{m}_{X}\otimes$ (a semistable vector bundle of degree $0$).
\end{lemma}

\begin{proof}
First assume that $k$ is not a node. Then $F_{k}=P(D)_{j}\cup
P(D)_{j^{-1}}$ with $P(D)_{j}\cap (PD)_{j^{-1}}=X$. The restriction of
$K_{\det}$ to $P(D)_{j}$ (resp. $X$) is isomorphic to $K_{X}\otimes
j^{4}\otimes \tau^{2}_{D_{j}}$ (resp. $K_{X}$) (Corollary 6.22). Now
the direct image of $j^{4}\otimes \tau^{2}_{D_{j}}$ is isomorphic to
$j^{4}\otimes S^{2}(D^{*}_{j})$, and the restriction map to the direct
image of $j^{4}\otimes \tau^{2}_{D_{j}}$ on $X$ (namely, the trivial
bundle) is induced by the natural map $S^{2}(D^{*})\to j^{-4}$, given
by the inclusion $j^{2}\to D_{j}$. Hence, by the Mayer-Vietoris
sequence, we see that $p^{*}(K^{m}_{\det}|F_{k})$ is the tensor
product of $K^{m}_{X}$ and the fibre product of $j^{4m}\otimes
S^{2m}(D^{*}_{j})$ and $j^{-4m}\otimes S^{2m}(D^{*}_{j})$ over the
trivial bundle. Since $S^{2m}(D^{*}_{j})$ is obtainable
(\ref{art05-lem6.1}) as successive extension of line bundles of degree
0, it is semistable of degree 0. Hence so is the fibre product
mentioned above proving Lemma \ref{art05-lem7.13} in this case.


Now\pageoriginale let $k$ be a node. Then we have $\Pic(X\times
PH^{1}(X,\mathscr{O})_{t})\to \Pic(X\times PH^{1}(X,\mathscr{O}))$ is
an isomorphism. In particular from \ref{art05-lem6.22}, we see that
the restriction of $K^{m}_{\det}$ to $X\times
PH^{1}(X,\mathscr{O})_{t}$ is of the form $K^{m}_{X}\otimes$ a line
bundle coming from $PH^{1}(X,\mathscr{O})_{t}$. Hence its direct image
is of the form $K^{m}_{X}\otimes$ a trivial bundle, completing the
proof of Lemma \ref{art05-lem7.13}.

\end{proof}
\medskip
\noindent
{\bf Proof of Proposition 7.9.}~
Let $Z\in H_{2,k}$. By lemma \ref{art05-lem7.12}, the map $p:Z\to X$
is flat, and by Lemma \ref{art05-lem7.10}, $Z$ is schematically
contained in $F_{k}$ and all the fibres are conics. Let us first
consider the case when $Z$ is a subscheme of $P(D)_{j}$ for some $j$
over $k$. By (\ref{art05-rems4.4}, iv) there exists a vector subbundle
$E$ of $D_{j}$ of rank 3, such that $Z\subset P(E)$. Now $Z$ defines a
divisor in $P(E)$ with $L_{Z}\simeq \tau^{2}_{E}\otimes L$ for some
line bundle $L$ on $X$. We have the exact sequence
$$
0\to \tau^{-2}_{E}\otimes L^{*}\otimes K^{-m}_{\det}|_{P(E)}\otimes
K^{n}_{X}\to K^{-m}_{\det}|_{P(E)}\otimes K^{n}_{X}\to
K^{-m}_{\det}|_{Z}\otimes K^{n}_{X}\to 0.
$$
Substituting $K^{*}_{\det}|_{P(E)}\simeq \tau^{2}_{E}\otimes
K_{X}\otimes j^{4}$, and taking direct image on $X$, we obtain
$$
\chi(Z,K^{-m}_{\det}\otimes K^{n}_{X})=\chi(X,K^{m+n}_{X}\otimes
S^{2m}(E^{*}))-\chi (X,L^{*}\otimes K^{m+n}_{X}\otimes
S^{2m-2}(E^{*})). 
$$
The right side can be computed in terms of the degrees of $L$ and $E$
by the Riemann-Roch theorem. But the left side is given to be
$P(m,n)=(4m+1)(2m+2n-1)(g-1)$. Equating coefficients of these two
polynomials, we check that $\deg L=0$ and $\deg E=0$. In other words,
when $Z\subset P(D)_{j}$, we have shown that $Z$ is a divisor in
$P(E)$ given by a nonzero quadratic form $E\to E^{*}\otimes L$, where
$L$ is of degree 0 and $E$ a subbundle of $D_{j}$, also of degree
0. Now, if $k$ is a node, clearly $E$ is also trivial since $D_{j}$ is
trivial in this case. Moreover the quadratic form is a nonzero section
of $S^{2}(D^{*}_{j})\otimes L$, with $\deg L=0$, and hence it follows
(a) that $L$ is trivial and (b) that the quadratic form is a
constant. Thus if $Z\subset P(D)_{j}$ with $j^{2}=1$, then $Z\in
R_{k}$. On the other hand, if $k$ is not a node, by \S\ 6.1, $E$ is
the inverse image of a trivial vector subbundle of
$H^{1}(X,j^{2})_{X}$ of rank 2. Notice first that the map $E\to
E^{*}\otimes L$ cannot be an isomorphism since $E^{*}\otimes L$
contains a direct sum of two line bundles of degree $0$ while $E$ does
not. Moreover the kernel of $E\to E^{*}\otimes L$ is a proper
subbundle of degree 0 and, again by \S\ 6.1,\pageoriginale it must contain
$j^{2}$. Thus the quadratic form on $E$ is induced by an $L$-valued
quadratic form on the trivial subbundle $E/_{j^{2}}\subset
H^{1}(X,j^{2})_{X}$ viz. a nonzero section of
$S^{2}((E/j^{2})^{*}\otimes L$. As before, since $L$ is of degree 0,
it follows (a) that $L$ is trivial, and (b) that the section of
$S^{2}(E/j^{2})^{*}$ is constant. This proves that if $Z$ is contained
in $P(D)_{j}$ then $Z\in Q_{j}$.

It remains to consider the case when $Z\nsubset P(D)_{j}$ for any $j$
over $k$. Clearly if $k$ is not a node, $Z$ consists of two
irreducible components $Z_{1}$ and $Z_{2}$ with $Z_{1}\subset
P(D)_{j}$, $Z_{2}\subset P(D)_{j^{-1}}$. Moreover, $Z_{1}$
(resp. $Z_{2}$) defines a subbundle $E_{1}$ (resp. $E_{2}$) of $D_{j}$
(resp. $D_{j^{-1}}$) of rank 2, containing $j^{2}$ (resp. $j^{-2}$). Let
their degrees be $d_{1}$, $d_{2}$. By the Mayer-Vietoris sequence, we
see as before,
\begin{align*}
P(m,n)=\chi(Z,K^{-m}_{\det}\otimes K^{n}_{X})
&= \chi(X,S^{2m}(E^{*}_{1})\otimes K^{m+n}_{X}\otimes j^{4m})\\
&\quad +\chi(X,S^{2m}(E^{*}_{2})\otimes K^{m+n}_{X}\otimes j^{-4m})\\
&\qquad\hspace{2.45cm} -\chi (X,K^{m+n}_{X}).
\end{align*}
Computing the right side by the Riemann-Roch theorem, and substituting
for $P(m,n)$, we obtain $d_{1}=d_{2}=0$. Thus $E_{1}$, $E_{2}$ are
inverse images of line subbundle of the trivial bundles
$H^{1}(X,j^{2})_{X}$, $H^{1}(X,j^{-2})_{X}$ and since these subbundles
are of degree $0$, they must themselves be trivial. In other words,
$Z\in R_{k}$. 

Finally let $k$ be a node and $Z\in H_{2,k}$, $Z\nsubset
P(D)_{j}=(F_{k})_{\red}$. By \ref{art05-lem7.10}, $Z\subset F_{k}$. In
this case, $Z_{\red}$ is projective bundle associated to a rank 2
vector subbundle $E$ of $D_{j}=H^{1}(X,\mathscr{O})_{X}$. Moreover $Z$
is given by thickening within $X\times F_{k}$ by a line bundle of the
form $\tau^{-1}_{E}\otimes L$, where $L$ is a line bundle on $X$. From
the exact sequence
$$
0\to \tau^{-1}_{E}\otimes
L\to \mathscr{O}_{Z}\to \mathscr{O}_{Z_{\red}}\to 0
$$
and the fact that $K_{\det}|X\times
PH^{1}(X,\mathscr{O})\simeq \tau^{2}\otimes K_{X}$ we obtain
\begin{align*}
\chi(Z,K^{-m}_{\det}\otimes K^{n}_{X}) & =\chi(X,S^{2m}(E^{*})\otimes
K^{m+n}_{X})\\
&\quad +\chi(X,S^{2m-1}(E^{*})\otimes K^{m+n}_{X}\otimes L).
\end{align*}
Equating this expression with $P(m,n)$, we obtain 
$\deg E=0$, $\deg L=0$. 
Since $E$ is contained in the trivial bundle
$H^{1}(X,\mathscr{O})_{X}$, $E$ itself\pageoriginale is trivial. Now
we claim that if $L$ is nontrivial, $\Hom(\tau_{E}\otimes
L^{-1},N_{X\times P(E),U_{X}})=0$. This follows from sequence 
$$
0\to
N_{X\times P(E),X\times P}\to N_{X\times P(E),UX}\to N_{X\times
P,U_{X}}|_{X\times P(E)}\to 0
$$ 
and the sequences
in \ref{art05-lem6.21}, ii), since
\begin{itemize}
\item[(a)] $H^{0}(\Hom(\tau_{E}\otimes L^{-1},N_{X\times P(E),X\times
P}))=H^{0}(\Hom(\tau_{E}\otimes L^{-1},\tau\otimes \text{~trivial~}))=0$.

\item[(b)] $H^{0}(\Hom(\tau_{E}\otimes L^{-1},\tau^{-1}\otimes F))=0$.

\item[(c)] $H^{0}(\Hom(\tau_{E}\otimes L^{-1},\tau^{-1}\otimes
T_{P}))=0$, and

\item[(d)] $H^{0}(\Hom(\tau_{E}\otimes L^{-1},T_{P}))=0$.
\end{itemize}

It follows that $L$ is trivial, since $\tau^{-1}\otimes L\subset
N_{X\times P(E),U_{X}}$. Thus the scheme $Z$ is of the form $X\times
C$, where $C$ is a conic contained in $P_{t}$, and hence $Z\in
R_{k}$. This completes the proof of Proposition \ref{art05-lem7.9}.

\section{Nonsingularity of the Hecke component}\label{art05-sec8}

We will now prove that $H_{0}$, the irreducible component of
$\Hilb(U_{X})$ containing Hecke cycles, is nonsingular. Since $H_{0}$
is a variety of dimension $3g-3$, it is enough to show that the
Zariski tangent space at any point of $H_{0}$ is of dimension
$3g-3$. This is done in
Propositions \ref{art05-prop8.1}, \ref{art05-prop8.5}, \ref{art05-lem8.9}
and \ref{art05-prop8.12}. From the description in \S\ \ref{art05-sec7}
of the nature of the subschemes of $U_{X}$ occurring in $H_{2}$ (and
hence $H_{0}$), it follows that all these schemes are local complete
intersections and hence the Zariski tangent space at any $Z\in H_{0}$
to the Hilbert scheme itself is given by
$H^{0}(Z,N_{Z,U_{X}})$ \cite{art05-key9a}. Thus the nonsingularity of
$H_{0}$ at some $Z\in H_{0}$ would follow if $\dim
H^{0}(Z,N_{Z,U_{X}})\leq 3g-3$. However, it turns out that there are
points in $H_{0}$ at which the Hilbert scheme itself is not smooth. We
will first compute $H^{0}(Z,N_{Z,U_{X}})$ for $Z\in H_{0}$. 

We have to discuss several cases.

\begin{proposition}\label{art05-prop8.1}
Let $Z\in R_{k}$, $k$ not a node and $Z\nsubset P(D)_{j}$, for any $j$
over $k$. Then the Hilbert scheme is smooth at $Z$.
\end{proposition}

The proof is completed in \ref{art05-lem8.4}.

By the definition of $R_{k}$, there exist points $p$, $p'$ in
$$
PH^{1}(X,j^{2}), \ P(H^{1}(X,j^{-2}))
$$ 
such that if $E$, $E'$ are the
inverse images in $D_{j}$, $D_{j^{-1}}$ of the trivial line bundles
corresponding to these two points,\pageoriginale then $Z=P(E)\cap
P(E')$. Recall also that $P(E)\cap P(E')=X$ imbedded in $P(D_{j})$,
$P(D_{j^{-1}})$ by means of the subbundles $j^{2}$, $j^{-2}$ of $D_{j}$,
$D_{j^{-1}}$ respectively.

\begin{lemma}\label{art05-lem8.2}
Let $U$ be a nonsingular variety and $l_{1}$, $l_{2}$ be nonsingular
subvarieties intersecting (schematically) in a nonsingular subvariety
$X$ which is a divisor in both $l_{1}$ and $l_{2}$.
\begin{itemize}
\item[\rm(i)] Then the reduced scheme $l=l_{1}\cup l_{2}$ is a local
complete intersection.

\item[\rm(ii)] Moreover, we have the exact sequence
$$
0\to N^{*}_{l,U}\to \widetilde{N}_{1}\oplus \widetilde{N}_{2}\to
N^{*}_{l,U}|X\to 0
$$
where $N_{1}$, $N_{2}$ are the restrictions of $N^{*}_{l,U}$ to
$l_{1}$, $l_{2}$ respectively. 

\item[\rm(iii)] $N_{1}$ fits in an exact sequence
$$
0\to N_{1}\to N^{*}_{l_{1},U}\to N^{*}_{X,l_{2}}\to 0,\quad\text{while}
$$

\item[\rm(iv)] $N_{l,U}|X$ fits in the sequence
$$
0\to N_{X,U}/_{N_{X},l_{1}\oplus N_{X},l_{2}}\to N_{l,U}|X\to
N_{X,l_{1}}\otimes N_{X,l_{2}}\to 0.
$$
\end{itemize}
\end{lemma}     

\begin{proof}
At any point of $l$, we can choose a regular system of parameters
$(x_{1},\ldots,x_{r},y_{1},y_{2},z_{1},\ldots,z_{s})$ for
$\mathscr{O}_{U}$ with $(x_{1},\ldots,x_{r},y_{1})=I_{1}$ and
$(x_{1},\ldots,x_{r},y_{2})=I_{2}$ being the ideals defining $l_{1}$,
$l_{2}$ respectively. Since $l$ is then defined by $I_{1}\cap
I_{2}=(x_{1},\ldots,x_{r},y_{1},y_{2})$, it follows that $l$ is a
complete intersection. The sequence (ii) is obtained by tensoring with
$N^{*}_{l,U}$ the basic exact sequence
$$
0\to \mathscr{O}_{l}\to \widetilde{\mathscr{O}}_{l_{1}}\oplus \widetilde{\mathscr{O}}_{l_{2}}\to \widetilde{\mathscr{O}}_{X}\to 0.
$$
We have now the isomorphism $N_{l}=I_{1}\cap I_{2}/_{(I_{1}\cap
I_{2})^{2}}\simeq I_{1}\cap I_{2}/_{I_{1}(I_{1}\cap I_{2})}$ and an
exact sequence
$$
0\to I_{1}\cap I_{2}/_{I_{1}(I_{1}\cap I_{2})}\to
I_{1}/_{I^{2}_{1}}\to I_{1}/_{I^{2}_{1}+I_{1}\cap I_{2}}\to 0.
$$
But
$$
I_{1}/_{I^{2}_{1}+I_{1}\cap I_{2}}\simeq \dfrac{I_{1}/_{I_{1}\cap
I_{2}}}{(I_{1}/I_{1}\cap
I_{2})^{2}}\simeq \dfrac{I_{1}+I_{2}/_{I_{2}}}{(I_{1}+I_{2}/_{I_{2}})^{2}}\simeq N^{*}_{X,l_{2}}
$$
proving (iii).

Finally, restrict (iii) to $X$ to get the four term sequence
$$
0\to \underline{\Tor}^{1}_{l_{1}}(N^{*}_{X,l_{2}},\mathscr{O}_{X})\to
N^{*}_{l,U}|X\to N^{*}_{l_{1},U}|X\to N^{*}_{X,l_{2}}\to 0.
$$
From the resolution
$$
0\to L^{-1}_{X}\to \mathscr{O}_{l_{1}}\to \mathscr{O}_{X}\to 0
$$
and\pageoriginale the isomorphism $L^{-1}_{X}|X\simeq
N^{*}_{X,l_{1}}$, we evaluate the Tor-term to be
$N^{*}_{X,l_{2}}\otimes N^{*}_{X,l_{1}}$. On the other hand, it is
clear that $N^{*}_{l_{1},U}|X\to N^{*}_{X,l_{2}}$ has kernel $\simeq
(N_{X,U}/_{N_{X},l_{1}\otimes N_{X},l_{2}})^{*}$. Dualising, we
obtain (iv).
\end{proof}

We will apply \ref{art05-lem8.2} to the case $U=U_{X}$, $l_{1}=P(E)$,
$l_{2}=P(E')$ and $X=l_{1}\cap l_{2}$. In this case we have
$N_{X,l_{2}}\simeq \Hom(j^{-2},\mathscr{O})=j^{2}$ from the exact
sequence
$$
0\to j^{-2}\to E'\to \mathscr{O}\to 0.
$$
From the definition of $N_{1}$, we get 
$$
\det N_{1}\simeq \det N^{*}_{l_{1},U_{X}}\otimes
(\det \widetilde{N}^{*}_{X,l_{2}})^{-1}. 
$$
But $\det \widetilde{N}^{*}_{X,l_{2}}\simeq L_{X}$, where $L_{X}$ is
the line bundle on $l_{1}$ associated to the divisor $X$, as is seen
from the fact that the bundle $j^{-2}$ on $X$ extends to $l_{1}$ and
from the resolution
$$
0\to L^{-1}_{X}\to \mathscr{O}_{l_{1}}\to \mathscr{O}_{X}\to 0
$$
for the sheaf $\mathscr{O}_{X}$ on $l_{1}$. But $L_{X}$ is easily seen
to be $\tau_{E}$. Thus if $\omega^{l}$ is the dualishing sheaf on $l$,
we get $\omega_{l}|l_{1}\simeq \omega_{U_{X}}\otimes \det
N^{*}_{1}\simeq \omega_{U_{X}}\otimes \det N_{l_{1},U_{X}}\otimes
(\det N^{*}_{X,l_{2}})\simeq \omega_{l_{1}}\otimes \tau_{E}$. But
$\omega_{l_{1}}\simeq K_{X}\otimes K_{\pi}$, where $K_{\pi}$ is the
canonical bundle along the fibres of $P(E)\to X$ and hence
$\simeq \tau^{-2}_{E}\otimes \det E^{*}\simeq \tau^{-2}_{E}\otimes
j^{-2}$. This proves

\begin{lemma}\label{art05-lem8.3}
The dualising sheaf $\omega_{l}$ of $l=l_{1}\cup l_{2}$ restricts to
$l_{1}$ as $\tau^{-1}_{E}\otimes j^{2}\otimes K_{X}$. In particular,
$\omega_{l}|_{X}\simeq K_{X}$. 
\end{lemma}

\noindent
{\bf Proof of Proposition \ref{art05-prop8.1}.}~
We wish to compute $H^{2}(l,\omega_{l}\otimes N^{*}_{l,U_{X}})$ which
is dual to the space $H^{0}(l,N_{l,U_{X}})$ required. To this end we
first compute $H^{2}(l_{1},\omega_{l}\otimes N_{l})$. From the exact
sequence $0\to N_{1}\to N^{*}_{l_{1},U_{X}}\to \widetilde{j}^{-2}\to
0$, we obtain $H^{2}(l_{1}\omega_{l}\otimes N_{1})\simeq
H^{2}(l_{1},\omega_{l}\otimes N^{*}_{l_{1},U_{X}})$ since
$H^{i}(X,K_{X}\otimes j^{-2})=0$ for $i=1,2$. But
$H^{2}(l_{1},\omega_{l}\otimes
N^{*}_{l_{1},U_{X}})=H^{2}(l_{1},\omega_{l_{1}}\otimes \tau \otimes
N^{*}_{l_{1},U_{X}})$ by Lemma \ref{art05-lem8.3}, which in turn is
dual to $H^{2}(l_{1},\tau^{-1}\otimes N_{l_{1},U_{X}})$. To compute
this, we use the exact sequence
$$
0\to N_{l_{1},P(D_{j})}\to N_{l_{1},U_{X}}\to
N_{P(D_{j}),U_{X}}|_{l_{1}}\to 0
$$
and observe that $N_{l_{1},P(D_{j})}\simeq \tau\otimes V'$, where $V'$
is a vector space of rank $(g-2)$. Hence $\dim
H^{0}(l_{1},\tau^{-1}\otimes N_{l_{1},P(D_{j})})=g-2$. To compute
$H^{0}(l_{1},\tau^{-1}\otimes N_{P(D_{j}),U_{X}})$, we appeal to Lemma 
\ref{art05-lem6.22}, 1). Since $H^{0}(l_{1},\tau^{-1}\otimes
H^{1}(X,\mathscr{O}))=0$ and
$H^{0}(l_{1},\tau^{-2}\otimes \pi^{*}F(j))$ is also seen
to\pageoriginale be zero, we get $H^{0}(l_{1},\tau^{-1}\otimes
N_{P(D_{j}),U_{X}})=0$ and hence $\dim H^{2}(l_{1},\omega_{l}\otimes
N_{1})=g-2$. Now Proposition \ref{art05-prop8.1} will follow from the
exact sequence in \ref{art05-lem8.2} tensored with $\omega_{l}$, if we
prove

\begin{lemma}\label{art05-lem8.4}
$\dim H^{1}(X,N^{*}_{l,U_{X}}\otimes \omega_{l})\leq g+1$.
\end{lemma}

\begin{proof}
Since $\omega_{l}|_{X}\simeq K_{X}$ by Lemma \ref{art05-lem8.3}, we
have only to show by duality, that $\dim H^{0}(X,N_{l,U_{X}})\leq
g+1$. For this, we will use Lemma \ref{art05-lem8.2}, (iv). In fact,
$N_{X,l_{1}}\simeq j^{-2}$ and $N_{X,l_{2}}\simeq j^{2}$ and hence
$H^{0}(X,N_{X,l_{2}}\otimes N_{X,l_{2}})$ is of dimension $1$. To
compute 
$$
H^{0}(X,N_{X,U_{X}}/N_{X,l_{1}}\oplus N_{X,l_{2}}),
$$ 
we use
the exact sequence in \ref{art05-lem6.21}, namely
$$
0\to N_{X,P(D_{j})}\oplus N_{X,P(D_{j^{-1}})}\to N_{X,U_{X}}\to F\to 0.
$$
Thus we get
$$
0\to (N_{l_{1},P(D_{j})}\oplus N_{l_{2},P(D_{j^{-1}})})|X\to
N_{X,U_{X}/N_{X,l_{1}}\oplus N_{X},l_{2}}\to F\to 0.
$$
But $N_{l_{1},P(D_{j})}|X\simeq \tau\otimes (D_{j}/_{l_{1}})|X\simeq
j^{-2}\otimes$ trivial. Hence $H^{0}$ of the first term above is zero
while $H^{0}(X,F)$ is clearly of dimension $g$. This proves
Lemma \ref{art05-lem8.4}, and hence completes the proof of
Proposition \ref{art05-prop8.1}. 
\end{proof}

\begin{proposition}\label{art05-prop8.5}
Let $Z\in Q_{k}$, $k$ a node. Assume that $Z=X\times C$, where $C$ is
a conic in $P=PH^{1}(X,\mathscr{O})$. Then $\dim
H^{0}(Z,N_{Z,U_{X}})\leq 3g-3$, $3g-2$ or $3g-1$ according as $C$ is
nondegenerate, is a pair of intersecting lines, or is a double
line. In particular, if $C$ is nondegenerate, the Hilbert scheme is
smooth at $Z$.
\end{proposition}

\begin{proof}
We first recall \cite[SGA 6, VII, Proposition 1.7]{art05-key9} that if
$X_{1}\subset X_{2}\subset X_{3}$ with $X_{2}$, $X_{3}$ smooth and
$X_{1}$ a local complete intersection, then by pushing out the
restriction to $X_{1}$ of the exact sequence
$$
0\to T_{X_{2}}\to T_{X_{3}}|X_{2}\to N_{X_{2},X_{3}}\to 0
$$
by means of the map $T_{X_{2}}|X_{1}\to N_{X_{1},X_{2}}$, we obtain
the exact sequence
$$
0\to N_{X_{1},X_{2}}\to X_{X_{1},X_{3}}\to N_{X_{2},X_{3}}|X_{1}\to 0.
$$
We wish to apply this to the case $X_{1}=Z$, $X_{2}=X\times P$ and
$X_{3}=U_{X}$, and use the description of $T_{U_X}|X\times P$ given in 
\ref{art05-lem6.22}, (ii). Then we see that $N_{Z,U_{X}}$ fits in
exact sequence
\setcounter{equation}{5}
\begin{equation}
0\to N'\to N_{Z,U_{X}}\to \tau^{-1}\otimes F|Z\to 0\label{art05-eq8.6}
\end{equation}\pageoriginale
where $N'$ is obtained as the push-out of the sequence
$$
0\to T_{P}|Z\to \Iim d\theta_{E}|Z\to \tau^{-1}\otimes T_{P}|Z\to 0
$$
by means of the map $T_{P}|Z\to N_{Z,X\times P}=N_{C,P}$. Since the
direct image of $\tau^{-1}\otimes F|Z$ on $X$ is zero, it follows that
$H^{0}(Z,\tau^{-1}\otimes F)=0$ and hence that $H^{0}(Z,N')\to
H^{0}(Z,N_{Z,U_{X}})$ is an isomorphism. Again by \ref{art05-lem6.22},
(ii), we have the commutative diagram (on $Z$)
\begin{equation}
\vcenter{
\xymatrix@C=.6cm{
0\ar[r] & \tau\otimes H^{1}(X,\mathscr{O})\ar[r]\ar[d] & F'\otimes
H^{1}(X,\mathscr{O})\ar[r]\ar[d] & H^{1}(X,\mathscr{O})_{Z}\ar[r]\ar[d] & 0\\
0\ar[r] & N_{C,P}\ar[r] & N'\ar[r] & \tau^{-1}\otimes T_{P}\ar[r] & 0
}}\label{art05-eq8.7}
\end{equation}
where the top sequence is obtained by tensoring the universal
extension $F'$ on $X\times P$ with $H^{1}(X,\mathscr{O})$. Now
$H^{0}(N_{C,P})$ is easily computed to be $3g-4$, for instance by
imbedding $C$ as a divisor in a plane $\varpi$ and proving
$\dim H^{0}(C,N_{C,\widetilde{\omega}})=5$ and $\dim
H^{0}(C,N_{\widetilde{\omega}P})=3(g-3)$. It remains therefore to
compute the kernel of the boundary homomorphism $H^{0}(X\times
C,\tau^{-1}\otimes T_{P})\to H^{1}(X\times C,N_{C,P})$. But now it is
easy to see that $H^{0}(P,\tau^{-1}\otimes T_{P})\to
H^{0}(C,\tau^{-1}\otimes T_{P})$ is surjective by checking
$H^{0}(P,\tau^{-1}\otimes T^{P})\to
H^{0}(\varpi,\tau^{-1}\otimes T_{P})$ and
$H^{0}(\varpi,\tau^{-1}\otimes T_{P})\to
H^{0}(C,\tau^{-1}\otimes T_{P})$ are both surjective. Thus from 
\eqref{art05-eq8.7}, we see that the required boundary homomorphism is
the composite of that of the top sequence $H^{1}(X,\mathscr{O})\to
H^{1}(X\times P,H^{1}(X,\mathscr{O})\otimes \tau)$ and the natural map
$H^{1}(X\times P, H^{1}(X,\mathscr{O})\otimes \tau)\to
H^{1}(C,N_{C,P})$. Again we have the diagram (on $Z$).
{\fontsize{9pt}{11pt}\selectfont
\begin{equation}
\vcenter{
\xymatrix@C=.7cm@R=.5cm{
0\ar[r] & \tau\otimes H^{1}(X,\mathscr{O})\ar[r]\ar[dd] & F'\otimes
H^{1}(X,\mathscr{O})\ar[r]\ar[dd] &
H^{1}(X,\mathscr{O})_{Z}\ar[r]\ar[dd]\ar[dr] & 0\\
 & & & & \tau^{-1}\otimes T_{P}\ar[dl]\\
0\ar[r] & N_{C,P}\ar[r] & F'\otimes \tau^{-1}\otimes N_{C,P}\ar[r]
& \tau^{-1}\otimes N_{C,P}\ar[r] & 0
}}\label{art05-eq8.8}
\end{equation}}
where the lower sequence is obtained as the tensor product on $X\times
C$ of the universal extension $F'$ by $\tau^{-1}\otimes N_{C,P}$. From 
\eqref{art05-eq8.8} we conclude that the required map is the composite
of the natural map $H^{0}(C,\tau^{-1}\otimes T_{P})\to
H^{0}(C,\tau^{-1}\otimes N_{C,P})$ and the boundary homomorphism
$H^{0}(C,\tau^{-1}\otimes N_{C,P})\to H^{1}(X\times C,N_{C,P})$. From
the definition of the universal\pageoriginale extension, we conclude
that the latter map is injective. Thus finally we have
\begin{align*}
& \dim H^{0}(X\times C,N_{Z,U_{X}})=\dim H^{0}(X\times C,N')\\
&\quad =3g-4+\dim \ker H^{0}(C,\tau^{-1}\otimes T_{P})\to
H^{0}(C,\tau^{-1}\otimes N_{C,P})
\end{align*}
so that Proposition \ref{art05-prop8.5} would follow from
\end{proof}

\setcounter{theorem}{8}
\begin{lemma}\label{art05-lem8.9}
If $C$ is a conic in a projective space $P$, then 
$$
\ker
H^{0}(C,\tau^{-1}\otimes T_{P})\to H^{0}(C,\tau^{-1}\otimes N_{C,P})
$$
is of dimension $1$, $2$, or $3$ according as $C$ is nondegenerate, is
a pair of lines, or is a double line.
\end{lemma}

\begin{proof}
By duality, and noting that $\tau^{-1}|C$ is the dualising sheaf, we
have to compute the dimension of the cokernel of
$H^{1}(C,N^{*}_{C,P})\to H^{1}(C,\Omega^{1}_{P})$. From the exact
sequence
$$
N^{*}_{C,P}\to \Omega^{1}_{P}|_{C}\to \Omega^{1}_{C}\to 0
$$
we see that this cokernel is isomorphic to $H^{1}(C,\Omega^{1}_{C})$
since\break $H^{2}(C,\mathscr{F})=0$ for any coherent sheaf $\mathscr{F}$ on
$C$. For a nondegenerate conic, its dimension is clearly 1. On the
other hand, if $C$ is a pair of lines $l_{1}$, $l_{2}$, we have a
surjection
$\Omega^{1}_{C}\to \widetilde{\Omega}^{1}_{l_{1}}\oplus \widetilde{\Omega}^{1}_{l_{2}}$
with the kernel sheaf supported at the point of interesection of
$l_{1}$ and $l_{2}$. Hence $\dim H^{1}(\Omega^{1}_{C})=2$. If $C$ is
the $\tau^{-1}$-thickening of the projective line $l$, then we also
have a projection $p:C\to l$. From this we get the exact sequence
$$
0\to p^{*}\Omega^{1}_{l}\to \Omega^{1}_{C}\to \Omega^{1}_{p}\to 0.
$$
Now $\Omega^{1}_{p}$ is easily seen to be $i_{*}(\tau^{-1})$ where
$i:l\to C$ is the inclusion, and hence $H^{i}(C,\Omega^{1}_{p})=0$ for
all $i$. On the other hand,
$$
H^{1}(C,p^{*}\Omega^{1}_{l})\simeq H^{1}(l,\Omega^{1}_{l})\oplus
H^{1}(l,\tau^{-1}\otimes \Omega^{1}_{l})
$$
is of dimension 3. This proves Lemma \ref{art05-lem8.9}.

\end{proof}
We will in fact see that if $Z\in H_{2,k}$ corresponds to a degenerate
conic $C\subset P=PH^{1}(X,\mathscr{O})$, then $\Hilb(P(n))$ is not
smooth at $Z$.

To show that $H_{0}$ is nonsingular at such a $Z$, we proceed as
follows. Let $\widetilde{Z}\subset H_{2}\times U_{X}$ be the universal
subscheme representing $H_{2}$. Then the map $\widetilde{Z}\to
H_{2}\times X$ given by $(p_{H_{2}},\det)$ is flat. Since the fibres
are conics, it follows that this is a morphism of complete
intersection \cite[SGA 6, VII]{art05-key9}. Moreover, the direct image
on $H_{2}\times X$ of the dual of the relative dualising sheaf is
locally free of rank 3. The\pageoriginale determinant of this direct
image is a line bundle and hence induces a map $\varphi:H_{2}\to \Pic
X$. If $Z=P(E)$, is a good Hecke cycle, its image under $\varphi$ is
given as follows: If $\pi:P(E)\to X$ is the projective fibration,
$\varphi(Z)=\det \pi_{*}(T_{\pi})\simeq \det \Ad E$ is the trivial
bundle. Hence $\varphi$ is a constant on $H_{0}$. On the other hand,
if $Z\in Q_{j}$ is given by (rank 2) subbundles $F_{1}$, $F_{2}$ of
$D_{j}$ containing $j^{2}$ with $F_{1}/_{j^{2}}$, $F_{2}/_{j^{2}}$
trivial, then $\omega^{*}_{\pi}\simeq \tau_{E}\otimes (F_{1}\cap
F_{2})|_{Z}$ where $E=F_{1}+F_{2}$. Now from the exact sequence (on
$P(E)$)
$$
0\to \tau^{-1}_{E}\otimes F_{1}\cap F_{2}\to \tau_{E}\otimes F_{1}\cap
F_{2}\to \omega^{*}_{\pi}\to 0
$$
we get, on taking direct images on $X$,
\begin{align*}
\varphi(Z) &= \det \pi_{*}(\tau_{E}\otimes F_{1}\cap
F_{2})=\det(E^{*}\otimes F_{1}\cap F_{2})\\
&= (F_{1}\cap F_{2})^{3}\otimes \det E^{*}=j^{4}.
\end{align*}
since $F_{1}\cap F_{2}=j^{2}$ and $\det E=j^{2}$. Thus we have


\begin{lemma}\label{art05-lem8.10}
If $Z\in H_{0}$, then $\varphi(Z)$ is trivial, while if $Z\in Q_{j}$
is given by two subbundles $F_{1}$, $F_{2}$ of $D_{j}$ containing
$j^{2}$, then $\varphi(Z)=j^{4}$. 
\end{lemma}

Now in order to prove that $H_{0}$ is smooth at points $Z\in Q_{k}$,
$k$ a node, we will show

\begin{lemma}\label{art05-lem8.11}
$\Iim (d\varphi)_{Z}$ has dimension $\geq 1$ (\resp. $\geq 2$) if $Z$
is represented by a pair of lines (resp. a double line).
\end{lemma}

Assume the Lemma for the moment. Since $\varphi$ is a constant on
$H_{0}$, it follows that $(d\varphi)_{Z}$ is zero on
$T_{Z}(H_{0})$. But by Proposition \ref{art05-prop8.5}, $\dim
T_{Z}(H_{2})\leq 3g-2$ (resp. $3g-1$) in these cases. Now the map
$(d\varphi)_{Z}:T_{Z}(H_{2})/T_{Z}(H_{0})\to T_{\varphi(Z)}(J)$ has
image of dimension $\geq 1$ (resp. $\geq 2$). Hence $\dim
T_{Z}(H_{0})\leq (3g-3)$ in both these cases, proving $H_{0}$ is
nonsingular at these points.

\medskip
\noindent
{\bf Proof of Lemma \ref{art05-lem8.11}.}~
Let $P$ be a Poincar\'e bundle on $J\times X$. The sheaf
$\mathscr{F}=R^{1}(p_{j})_{*}(P^{2})$ on $J$ is locally free, outside
elements of order 2, of rank $(g-1)$. It is easily seen that
Grass$_{g-1}(\mathscr{F})$ is isomorphic to the blow up
$\pi:\widetilde{J}\to J$ at all elements of order 2. Hence on
$\widetilde{J}$ we have a surjection $\pi^{*}\mathscr{F}\to Q\to 0$
where $Q$ is the tautological quotient bundle or rank $(g-1)$. On the
other hand, on $X\times J$ we have a surjection of the\pageoriginale
vector bundle $D$ onto $p^{*}_{J}\mathscr{F}$. This gives rise to a
surjection $(1\times \pi)^{*}D\to p^{*}_{\widetilde{J}}Q$ (on
$X\times \widetilde{J}$) where now $Q$ is also locally free. Thus we
get a family of subschemes of $P(D)$, parametrised by $P(Q)$. These
subschemes are projective bundles associated to subbundles of $D_{j}$
of rank 2 containing the kernel of $D_{j}\to Q_{j}$ given by the
surjection above. Thus $P(Q)\times_{\widetilde{J}}P(Q)$ parametrises
two families of projective line subbundles. It is easy to see that
there is a flat family of schemes over
$P(Q)\times_{\widetilde{J}}P(Q)$ which is obtained as the union of
these two schemes and that this family is a family of subschemes of
$U_{X}$ with Hilbert polynomial $P(n)$. Thus we have a morphism
$P(Q)\times_{\widetilde{J}}P(Q)\to H_{2}$. By
Lemma \ref{art05-lem8.10}, we have the commutative diagram
\[
\xymatrix{
P(Q)\times_{\widetilde{J}}P(Q)\ar[d]\ar[r] & H_{2}\ar[dd]\\
\widetilde{J}\ar[d] & \\
J\ar[r]^-{4_{J}} & J
}
\]
where $4_{J}$ is the map $j\mapsto j^{4}$ of $J$. If $Z\in H_{2}$ is
represented by $X\times$ a pair of lines in $PH^{1}(X,\mathscr{O})$,
then it is the image of some point in $P(Q)\times_{\widetilde{J}}P(Q)$
over an element of order 2 of $J$. Since the map $\widetilde{J}\to J$
has nonzero differential at any point, and since
$P(Q)\times_{\widetilde{J}}P(Q)\to \widetilde{J}$ is a fibration, it
follows that $\Iim (d\varphi)_{Z}$ has dimension $\geq 1$. If $Z$ is
represented by $X\times$ a double line in $PH^{1}(X,\mathscr{O})$,
then $Z$ can be obtained as the image of two points of
$P(Q)\times_{\widetilde{J}}P(Q)$ whose images $x$, $x'$ in
$\widetilde{J}$ are two different points over the given node. In fact,
$x$ and $x'$ could be taken as any point on the line in
$PH^{1}(X,\mathscr{O})$, to which $Z$ corresponds. Hence
$$
\dim \Iim (d\varphi)_{Z}\geq \dim (\Iim(d\pi)_{x}+\Iim(d\pi)_{x})\geq 2.
$$
This completes the proof of \ref{art05-lem8.11} and hence the
nonsingularity of $H_{0}$ at $Z\in Q_{k}$, $k$ a node.

It remains to consider the case $Z\in R_{k}$, $k$ a node.

\begin{proposition}\label{art05-prop8.12}
Let\pageoriginale $Z\in R_{k}$, $k$ a node. Then $\dim H^{0}(Z,N_{Z,U_{X}})\leq
3g-3$. 
\end{proposition}

\begin{proof}
Let $Z=X\times l_{L}$, where $l$ is a projective line in
$PH^{1}(X,\mathscr{O})$ and $L$ is a subbundle of $N=N_{X\times
l,U_{X}}$ isomorphic to the hyperplane bundle on $l$. Then we have the
exact sequence
$$
0\to L^{-1}\otimes N_{Z,U_{X}}|X\times l\to N_{Z,U_{X}}\to
N_{Z,U_{X}}|X\times l\to 0.
$$
Now Proposition \ref{art05-prop8.12} follows from
\end{proof}

\begin{lemma}\label{art05-lem8.13}
\begin{itemize}
\item[\rm(i)] $\dim H^{0}(X\times l,N_{Z,U_{X}})\leq 2g-1$.

\item[\rm(ii)] $\dim H^{0}(X\times l,L^{-1}\otimes N_{Z,U_{X}})\leq g-2$.
\end{itemize}
\end{lemma}

\noindent
{\bf Proof of (i).}~
By \S~3.4 we have the exact sequence
\setcounter{equation}{13}
\begin{equation}
0\to N/L\to N_{Z,U_{X}}|_{X\times l}\to L^{2}\to 0.\label{art05-eq8.14}
\end{equation}
Since $H^{0}(X\times l, L^{2})\simeq H^{0}(l,\tau^{2})$ is of
dimension 3, (i) will be proved if we show that $\dim H^{0}(X\times
l,N/L)\leq 2(g-2)$. Now for this computation, we need the exact
sequences (Lemma \ref{art05-lem6.22}. (ii))
\begin{align}
& 0\to \tau\otimes V/W\to N'/L\to V/W\to 0\label{art05-eq8.15}\\
& 0\to N'/L\to N/L\to \tau^{-1}\otimes F\to 0\label{art05-eq8.16}
\end{align}
where $V=H^{1}(X,\mathscr{O})$, $W$ is the two dimensional subspace of
$V$ corresponding to $l$, and $N'$ is the kernel of the surjection
$N\to \tau^{-1}\otimes F$. From \eqref{art05-eq8.15}, we conclude that
$H^{0}(X\times l,N'/L)\simeq H^{0}(X\times l,N/L)$, since
$H^{0}(X\times l,\tau^{-1}\otimes F)=0$. To compute $H^{0}(X\times
l,N'/L)$, we note that $\dim H^{0}(X\times l,\tau\otimes V/W)=2(g-2)$
and hence i), Lemma \ref{art05-lem8.13} will be proved if we can show
that the boundary homomorphism $H^{0}(V/W)\to H^{1}(\tau\otimes V/W)$
is injective. From \ref{art05-lem6.22} (ii), we have the commutative
diagram
\[
\xymatrix{
V/W\simeq H^{0}(V/W)\ar[r] & H^{1}(\tau\otimes V/W)\simeq V\otimes
V^{*}\otimes V/W\\
V\simeq H^{0}(V)\ar[u]\ar[r] & H^{1}(\tau\otimes V)\simeq V\otimes
V^{*}\otimes V\ar[u] 
}
\]
where the lower map is the boundary homomorphism given by the
universal extension $F'$, and hence is the map
$v\mapsto \Id_{V}\otimes v$. From this it is easy to conclude that the
top horizontal map is injective. This proves (i).
        
\medskip
\noindent
{\bf Proof of (ii).}
We\pageoriginale
proceed as in (i). By tensoring \ref{art05-eq8.14} with $L^{-1}$, we
are reduced to computing a) $H^{0}(X\times l, L^{-1}\otimes N/L)$ and
b) the boundary homomorphism $H^{0}(X\times l,L)\to H^{1}(X\times l,
L^{-1}\otimes N/L)$. As for a), we have from \ref{art05-eq8.16},
$H^{0}(L^{-1}\otimes N/L)\simeq H^{0}(L^{-1}\otimes N'/L)$ and from 
\ref{art05-eq8.15}, $H^{0}(L^{-1}\otimes N'/L)\simeq H^{0}(V/W)$ which
has dimension $g-2$. Now (ii) will be proved if we can show that the
boundary homomorphism b) is injective. To prove this, we need a
description of the extension \ref{art05-eq8.14} (which is the dual of
the extension \ref{art05-lem3.4} for $l_{L}$). In other words, for
$N_{X\times P,U_{X}}$ we have the exact sequence
(\ref{art05-lem6.22}. (ii))
$$
0\to \tau^{-1}\otimes T_{P}\to N_{X\times P,U_{X}}\to \tau^{-1}\otimes
F\to 0.
$$
Now $L=\tau^{-1}\times T_{l}\subset \tau^{-1}\otimes T_{P}|X\times
l$. Any line subbundle of $N_{X\times l,U_{X}}$ mapping isomorphically
on $L$ gives rise to a thickening $X\times l_{L}$. From
Proposition \ref{art05-prop3.8}, we get the following information
about \ref{art05-eq8.14}. The map $d\theta_{E}|X\times l$ induces a
map $N_{X\times P,P(D)}=H^{1}(X,\mathscr{O})_{X\times
P}\to \tau^{-1}\otimes T_{P}$. Let $W$ be the inverse image of
$L=\tau^{-1}\otimes T_{l}$ so that we have the exact sequence (on
$X\times l$)
$$
0\to \tau^{-1}\to W\to \tau^{-1}\otimes T_{l}=L\to 0.
$$
Of course, $W$ is actually the trivial bundle with fibre = the
subspace of $P$ corresponding to $l$. Symmetrising this, we get the
sequence
\setcounter{equation}{16}
\begin{equation}
0\to\tau^{-1}\otimes W\to S^{2}(W)\to L^{2}\to 0.\label{art05-eq8.17}
\end{equation}
Taking into account the description of the Hessian of
$\theta_{E}|X\times P$, we see that the push-out
of \eqref{art05-eq8.14} by the map $N/L\to N_{X\times
P,U_{X}}/\tau^{-1}\otimes T_{P}=\tau^{-1}\otimes F$ is isomorphic to
the push-out of \eqref{art05-eq8.17}  by the natural map
$\tau^{-1}\otimes W\to \tau^{-1}\otimes
H^{1}(X,\mathscr{O})\to \tau^{-1}\otimes F$. In view of this, we have
a commutative diagram
{\fontsize{9pt}{11pt}\selectfont
\[
\xymatrix@C=-.2cm{
H^{0}(X\times l,L)\ar[r]\ar[d] & H^{1}(X\times l,
L^{-1}\otimes \tau^{-1}\otimes W)\ar[r] & H^{1}(X\times
l,L^{-1}\otimes \tau^{-1}\otimes H^{1}(X,\mathscr{O}))\ar[d]\\
H^{1}(X\times l, L^{-1}\otimes N/L)\ar[rr] && H^{1}(X\times l,
L^{-1}\otimes \tau^{-1}\otimes F)
}
\]}
We have to show that the left vertical map is injective. The second
top horizontal map is clearly injective, while the right vertical map
is even an isomorphism. Thus our assertion will be proved if the map
$$
H^{0}(l,L)\to H^{1}(l,L^{-1}\otimes \tau^{-1}\otimes W),
$$\pageoriginale
associated to \eqref{art05-eq8.17} tensored with $L^{-1}$, is
injective. But this is clear since $H^{0}(l,L^{-1}\otimes
S^{2}(W))=0$, thus proving Lemma \ref{art05-lem8.13} and hence
Proposition \ref{art05-prop8.12}.

We now have

\setcounter{theorem}{13}
\begin{theorem}\label{art05-thm8.14}
There is a natural morphism of the Hecke component $H_{0}$ into the
moduli space $U_{0}$ giving a non-singular model of $U_{0}$, which is
an isomorphism over the set of stable points. The reduced fibre over
a non-nodal point of the Kummer variety is isomorphic to
$\bfP^{g-2}\times \bfP^{g-2}$ while that over the node is isomorphic
to the union of a $\bfP^{5}$ bundle over the Grassmannian of planes in
$\bfP^{g-1}$ and a $\bfP^{g-2}$ bundle over the Grassmannian of lines
in $\bfP^{g-1}$. 
\end{theorem}

\begin{proof}
The morphism $\pi:\Hilb(H,U_{0},P(m,n))\to U_{0}$ is an isomorphism
over the open set $U$ of stable points. Moreover the restriction of
the morphism $\Hilb(H,U_{0},P(m,n))\to \Hilb(U_{X},P(n))$ to the
(schematic) closure of $p^{-1}(U)$ is an injective
(Lemma \ref{art05-lem7.6}) and birational
(Theorem \ref{art05-thm5.13}) morphism onto $H_{0}$. Since $H_{0}$ is
smooth, this morphism is an isomorphism onto $H_{0}$. This proves the
first part of the theorem.

The fibre over a non-nodal point of $\mathscr{K}$ corresponding to
$j\in J$, $j^{2}\neq 1$, is isomorphic to $PH^{1}(X,j^{2})\times
PH^{1}(X,j^{2})$ (Proposition \ref{art05-prop7.8}). The fibre over a
node is isomorphic to the union of the space of conics which are
schematically contained in $PH^{1}(X,\mathscr{O})$ and the space of
$\tau^{-1}$-thickenings of lines in $PH^{1}(X,\mathscr{O})$ which are
contained in the thickening $PH^{1}(X,\mathscr{O})_{t}$
(Proposition \ref{art05-prop7.8}). The first variety is a
$\bfP^{5}$-bundle over the Grassmannian of planes in
$PH^{1}(X,\mathscr{O})$ while the second is a $\bfP^{g-2}$ bundle over
the Grassmannian of lines in $PH^{1}(X,\mathscr{O})$ (See
Remark \ref{art05-rems4.4}. iii).
\end{proof}

\begin{thebibliography}{}
\bibitem[1]{art05-key1} U. V. Desale and S. Ramanan:\pageoriginale Classification of
vector bundles of rank 2 on hyperelliptic curves, {\em Inventiones
Math.,} 38 (1976) 161-185.

\bibitem[2]{art05-key2} R. Hartshorne: Ample subvarieties of albegraic
varieties, {\em Springer Lecture Notes in Mathematics,} No. 156.

\bibitem[3]{art05-key3} M. S. Narasimhan and S. Ramanan: Moduli of vector
bundles on a compact Riemann surface, {\em Ann. of Math.,} (89) (1969)
19-51. 

\bibitem[4]{art05-key4} M.S. Narasimhan and S. Ramanan: Vector bundles on
curves, {\em Proceedings of the Bombay Colloquium on Algebraic
Geometry}, 335-346, Oxford University Press, 1969.

\bibitem[5]{art05-key5} M.S. Narasimhan and S. Ramanan: Deformations of
the moduli space of vector bundles over an algebraic curve, {\em Ann
of Math.,} 101 (1975), 391-417.

\bibitem[6]{art05-key6} S. Ramanan: The moduli spaces of vector bundles
over an algebraic curve, {\em Math. Annalen,} 200 (1973), 69-84.

\bibitem[7]{art05-key7} M. Schelessinger: Functors on Artin rings, {\em
Trans. Amer. Math. Soc.,} 130 (1968), 208-222.

\bibitem[8]{art05-key8} A. Tjurin: Analog of Torelli's theorem for two
dimensional bundles over algebraic curves of arbitrary genus,
Iz. Akad. Nauk, SSSR, Ser Mat Tom 33, (1869) 1149-1170, {\em
Math. U.S.S.R. Izvestija,} Vol. 3 (1969) 1081-1101.

\bibitem[8a]{art05-key8a} A.N. Tjurin: The geometry of moduli of vector
bundles, {\em Uspekhi Mat. Nauk,} 29 : 6 (1974), 59-88, {\em Russian
Math. Surveys} 29 : 6 (1974), 57-88.

\bibitem[9]{art05-key9} S\'eminaire de G\'eom\'etrie Alg\'ebrique (SGA).

\bibitem[9a]{art05-key9a} Fondements de la G\'eom\'etrie Alg\'ebrique (FGA).

\bibitem[10]{art05-key10} C.S. Seshadri: Desingularisation of moduli
varieties of vector bundles on curves, to appear in the {\em
Proceedings of the Kyoto Conference on Algebraic Geometry,} 1977.

\end{thebibliography}

\vfill\eject
~\phantom{a}
\thispagestyle{empty}
