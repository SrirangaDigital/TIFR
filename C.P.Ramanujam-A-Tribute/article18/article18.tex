\title{Geometry of \texorpdfstring{$G/P$}{GP}-I\\ Theory of Standard Monomials for\\ Minuscule Representations}\label{art18}
\markright{Geometry of $G/P$-I - Theory of Standard Monomials......}

\author{By~ C.S. Seshadri}
\markboth{C.S. Seshadri}{Geometry of $G/P$-I - Theory of Standard Monomials......}

\date{}
\maketitle

\setcounter{page}{245}

\textsc{Introduction.}
\setcounter{pageoriginal}{206} Let\pageoriginale  $G$ be a semi-simple simply connected algebraic group defined over an algebraically closed field $k$. Let $P$ be a maximal parabolic subgroup in $G$ and $L$ the ample line bundle on $G/P$ which generates $\Pic G/P$. We say that $P$ is {\em minuscule} if the Weyl group $W$ (fixing of course a Borel subgroup, maximal torus etc.) acts transitively on the weight vectors of the irreducible $G$-module $H^0(G/P, L)$ ($k$ assumed to be of characteristic zero. Note then that $H^0(G/P,L)$ is the irreducible $G$-module associated to a fundamental weight). We can then index these weight vectors by $\{p_\tau\}$, $\tau \in W / W_{i(P)}$  (where $W_{i(P)}$ denotes the Weyl group of the maximal parabolic subgroup $i(P)$ which is the transform of $P$ under the Weyl involution $i$) so that $p_\tau$ is the highest weight vector, when $\tau \equiv$ Identity ($\mod W_{i(P)}$). We say that a monomial in $\{p_\tau\}$, say 
$$
p_{\tau_1} p_{\tau_2} \ldots p_{\tau_m} \in H^0(G/P, L^m)
$$
is {\em standard} if $\tau_1 \leqslant \tau_2 \leqslant \ldots \leqslant \tau_m$ (\cf Def. 1). Then the main result of this paper is the following (\cf. Th. 1):
\begin{align*}
& \text{{\em Standard monomials of length $m$, which are distinct, form a}}\\[-0.3cm]
& \tag{*}\\[-0.4cm]
& \text{{\em basis of $H^0 (G/P, L^m), m \geqslant 0$, $P$ minuscule.}}
\end{align*}
It is proved that this result holds also in arbitrary characteristic.

When $G = S L(n)$, every maximal parabolic subgroup is minuscule. In this case, when the base field $k$ is of characteristic zero, (*) is due to Hodge (\cf \cite{art18-key9}, \cite{art18-key10}) and is related to Young tableaux. When $k$ is of arbitrary characteristic, again for $G = S L(n)$, (*) was proved to be true by many, for example, by Musili (\cf \cite{art18-key13}). That the generalisation is to be sought for a minuscule $P$, was pointed out to the author by A. Borel.

An important consequence of (*) is that it allows one to determine the ideal\pageoriginale defining a Schubert variety in $G/P$, the scheme theoretic unions and intersections of Schubert varieties and the scheme theoretic hyperplane intersection of a Schubert variety, when this intersection decomposes into a union of Schubert varieties (\cf Th. 1). One knows (\cf \cite{art18-key12}) that these are the main technical difficulties in trying to prove the following:
\begin{equation*}
H^i (G/ B, L) = 0, \quad i > 0, \quad L \text{ in the dominant chamber.}
\tag{**}
\end{equation*}
The essential achievement of G. Kempf, in his proof of (**) (\cf \cite{art18-key11}), is that he is able to manage these technical difficulties for $G$ of type $E_6$, $E_7$, $E_8$ and $F_4$ and $P$ associated to what he calls a {\em distinguished weight}. For the cases $E_6$ and $E_7$, there exists a distinguished weight which is also minuscule. Hence from (*), a proof of (**) for $G$ of type $E_6$ and $E_7$ follows on the lines as in \cite{art18-key12}.

In another place, we shall given an application of (*) to the work of C. De Concini and C. Procesi (\cf \cite{art18-key7}) on their generalisation (to arbitrary characteristic) of the classical theorems on invariants, which are found, for example, in H. Weyl (\cf [24]).

The proof of Theorem \ref{art18-thm1} is related to the one in \cite{art18-key13}. The two new simple observations which have made the present generalisation possible and which the author missed for a long time, are as follows:
\begin{itemize}
\item[(i)] it suffices to do the standard monomial theory when the base field is of characteristic zero (\cf Remark \ref{art18-rem5});

\item[(ii)] it is not necessary to have, {\em a priori}, the quadratic relations defining $G/P$ in its canonical projective imbedding, in the specific form used in \cite{art18-key13}. Such quadratic relations follow, once one has the theorem on standard monomials (\cf Corollary \ref{art18-coro1}, Theorem \ref{art18-thm1}). 
\end{itemize}

The proof of Theorem \ref{art18-thm1} is somewhat indirect and uses Demazure's results (\cf \cite{art18-key8}) that Schubert varieties are Cohen-Macaulay when the ground field is of characteristic zero. One can give a more direct proof, closer to that of \cite{art18-key13}, if the following could be checked directly when the ground field is of characteristic zero (\cf Remark \ref{art18-rem8}).

$\dim H^0 (G/P, L^2) = \sharp$ (Distinct standard monomials of degree 2). This could perhaps be done by a clever handling of the Weyl's or Demazure's character or dimension formula.

\vskip 0.8cm

\begin{center}
{\bf Chapter 1}\\[0.2cm]
{\bf \large Standard Monomials}
\end{center}\pageoriginale

\section{Notations and preliminaries (\cf \texorpdfstring{\cite{art18-key1}, \cite{art18-key2}, \cite{art18-key3}, \cite{art18-key5}, \cite{art18-key12}}{Cite}).}\label{art18-sec1}

\begin{tabular}{r@{\;}l}
$G$  = & a semi-simple, simply-connected, split (Chevalley) group over\\
& a field $k$. Note that $G$ has also a $\bfZ$-form i.e., $G$ can be  considered\\
&  as $G' \otimes_{\bfZ} k$, where $G$ is split, semi-simple over $\bfZ$ etc.\\
$T$ = & a maximal torus of $G;B =$ a Borel subgroup of $G$, $B \supset T$.
Roots\\
& and weights indicated below are taken in the usual sense having\\
& fixed $T$ and $B$.\\
$\Delta$ = & system of roots, $\Delta^+$ = system of positive roots\\
$S$ = & simple roots: $\alpha_i$, $1 \leqslant i \leqslant l$, or the set $[1, \ldots, l]$ \\
$W$ = & Weyl group with simple reflections $s_i (=s_{\alpha_i})$, $1 \leqslant i \leqslant l$\\
 $\varpi_i$ = & fundamental weights, $1 \leqslant i \leqslant l$, i.e., $\langle \varpi_i, \check{\alpha_j}   \rangle  = \delta_{ij}$ where $\check{\alpha_j}$\\
&  denotes the coroot of $\alpha_j$ and $\langle , \rangle$ is as in \cite{art18-key4}.\\
$\mathscr{U}_\infty$ & = unipotent group $\approx \bfG_a$ associated to $\alpha \in\Delta$.\\
$\omega_0$ & = the element of maximal length in $W$\\
$i$ & = the Weyl involution - $\omega_0$.
\end{tabular}

Let $P$ be a parabolic subgroup of $G$, $P\supset B$. Then $P$ is {\em determined } by a subset $S_P$ of $S$ i.e. $P$ is generated by $B$ and $\mathscr{U}_{-\alpha}$, $\alpha = \sum n_i \alpha_i, n_i \geqslant 0$, $\alpha_i \in S_P$. Set
\begin{align*}
\Delta_P & = \left\{\alpha \in\Delta /\alpha = \sum n_i \alpha_i, \alpha_i \in S_P, n_i \in \bfZ \right\}\\
\Delta^+_P & = \left\{\alpha \in \Delta / \alpha = \sum n_i \alpha_i, \alpha_i \in S_P, n_i \geqslant 0, n_i \in \bfZ \right\}\\
\Delta_{P}^- & = \left\{  \text{ ~~ " ~~ " ~~ " ~~ \quad with}\quad n_i \leqslant 0, n_i \in\bfZ\right\}.
\end{align*}
We have
$$
P= M_P \cdot U_P
$$
$M_P$ being generated by $T$ and $\mathscr{U}_\alpha$, $\alpha \in \Delta_P$ and $U_P$ being generated by $\mathscr{U}_\alpha$, $\alpha \in \Delta^+$, $\alpha \not\in \Delta^+_P$. One knows that $U_P$ is the unipotent radical of $P$ and that $M_P$ is reductive so that $M_P$ (\resp $U_P$) is called the {\em reductive} (\resp {\em unipotent}) part of $P$.

The\pageoriginale subgroup $W_P$ of $W$ generated by the simple reflections $s_\alpha$, $\alpha \in S_P$, is called the {\em Weyl group of $P$}. One sees that $W_P$ is in fact the Weyl group of the reductive part $M_P$ of $P$. The radical of $M_P$ is a torus, whose dimension $= \sharp (S - S_P)$. Hence the {\em character group} (i.e. group of homomorphisms into $\bfG_m$) of $P$ (or of $M_P$), considered as an abelian group is free of rank $= \sharp (S - S_P)$. One sees  that the character group of $P$ can be identified with the group of characters $\chi$ of $T$ such that $\omega(\chi) =\chi \forall \omega \in W_P$. From this fact it is seen easily that $\{\varpi_i\}$, $i \in S - S_P$ form a basis of the character group of $P$.

In the sequel we will be working with a  {\em maximal parabolic subgroup $P$}. We write $i_0 = S - S_P$ and $\varpi = \varpi_{i_0}$, $i_0 \in [1, \ldots, l] =S$. We assume that the base field $k$ is algebraically closed.

Let $V$ denote the space of regular functions $f$ on $G$ such that 
$$
f(gb) = \chi_{\varpi} (b) f(g)
$$
where $\chi_{\varpi}$ denotes the character of $B$ defined by the fundamental weight ${\varpi}$. Then $V$ can be identified with the space of sections of the line bundle $L$ on $G/B$ associated to the character $\chi_{\varpi}^{-1}: B \to k$ of $B$. Now $\chi_{\varpi}$ extends to a character of $P$ and $V$ can in fact be identified with $H^0(G/P, L)$, where by the same $L$ we denote the line bundle associated to the character $\chi_{\varpi}^{-1}:P \to k$. Now $V$ has a unique $B$-fixed line and let us fix a generator $F$ of this line. We have 
$$
F(b_1 g b_2) = \chi^{-1}_{i(\varpi)} (b_1) F(g) \chi_{\varpi} (b_2), ~~ b_i \in B.
$$
If $f: G \to k$ is a regular function on $G$, we define 

$L_h f(g) = (h\circ f)(g) = f(h^{-1}g)$, $g$, $h \in G$. (left regular representation). We find 
$$
b_1 \cdot F(g) = F(b^{-1}_1 g) = \chi_{i(\varpi)} F(g)
$$
so that the weight of $F$ is $i(\varpi)$ i.e. the {\em highest weight of $V$} is $i(\varpi)$.

For $w \in W$, we denote by $X (w)$ the Schubert variety associated to $w$ i.e. the closure $\overline{B w B}$ of $BwB$ in $G$. Let $X_0$ denote the set of zeros of $F$. Then it is $P$-stable on the right and $B$-stable on the left. We denote by the same $X_0$ the subvariety of $G/P$ defined by $X_0$. Then $X_0$ is the unique codimension one Schubert variety in $G/P$. We claim that 
$$
X_0 = X (\omega_0 s) = X (i(s) \omega_0), s = s_{i_0}, i_0 = S - S_P, i(s) = \omega_0 s \omega_0.
$$\pageoriginale
To check this, since $X (\omega_0s)$ is of codimension one in $G/B$, it suffices to check that $X(\omega_0 s)$ is $P$-stable on the right. This is a consequence of Prop. 1.4, \cite{art18-key12} and by this proposition we have only to check that 
$$
(\omega_0 s) (\alpha) < 0 \quad \forall \alpha \in \Delta^+_P.
$$
Hence it suffices to check that 
$$
s(\alpha) \in \Delta^+, \alpha = \alpha_i, i\in S_P \text{ ~ i.e. ~ } i  \neq i_0, s = s_{i_0}.
$$
We have $s_{i_0} (\alpha_i) \in \Delta^+$, $i \neq i_0$ and thus the claim that $X_0 = X(\omega_0s)$ is proved.

We observe now that the set $hX_0$, $h \in G$, is precisely the zero set of $L_h F = F(h^{-1} g)$. We observe that for $t \in T$, $t$ leaves stable the set $X_0$ and hence the notation $\tau X_0$, $\tau \in W$ makes sense. Now if $\tau \in W$, $\tau \cdot F(g) = F(\tau^{-1} g)$ is well-defined only upto a scalar multiple; however we use this notation as it is convenient for us.

Note that the Weyl group $W$ does not operate even on the projective space $\bfP(V)$ associated to $V$ (as the torus operates with different weights on the weight vectors). Of course the normaliser $N(T)$ of $T$ operates on $V$.

\section{Preliminaries on the ideals of Schubert varieties in \texorpdfstring{$G/P$}{GP}.}\label{art18-sec2}
\begin{lem}\label{art18-lem1}
For $\tau \in W$, we have
$$
\tau X_0 = X_0 \Leftrightarrow \tau \in W_{i(P)}
$$
where $i(P)$ denotes the maximal parabolic group $w_0P\omega_0$ (i.e. the parabolic subgroup associated to the subset $i(S_P)$ of $S$).
\end{lem}

\begin{proof}
Observe that the action of $W$ on the characters $\chi$ of $T$ is defined as follows:
$$
\tau \cdot \chi (t) = \chi(\tau^{-1} t\tau)
$$
We have then $(\tau_1 \tau_2) \cdot \chi = \tau_1 \cdot (\tau_2 \chi)$. Define $H(g)$ by
$$
H(g) = F(\tau^{-1} g ) = L_\tau F (g), g\in G.
$$
The zero set of $H(g)$ is $\tau X_0$ as mentioned above. We have
\begin{align*}
t. H(g) & = H(t^{-1}g) = F(\tau^{-1} t^{-1}g) = F(\tau^{-1} t^{-1} \tau \tau^{-1} g)\\
& = \chi_{i(\overline{\omega)}} ((\tau^{-1}t^{-1} \tau)^{-1}) F(\tau^{-1}g) = \chi_{i(\omega)} (\tau^{-1} t \tau) H(g)\\
& = \chi_{\tau i (\overline{\omega})} (t) H (g).
\end{align*}
This\pageoriginale shows that $H(g)$ is a weight vector under $T$ with weight $\chi_{\tau i} (\overline{\omega})$. By the uniqueness of the codimension one Schubert variety in $G/P$ and the fact that $F$ is the unique (upto scalars) element in $V$ with weight $i(\overline{\omega})$, it follows that 
\begin{gather*}
\tau X_0 = X_0 \Leftrightarrow X_0 = \text{ zero set of } H(g)\\
\Leftrightarrow \tau i (\overline{\omega}) = i (\overline{\omega}) \text{~ i.e ~} \tau \in W_{i(P)}.
\end{gather*}
This completes the proof of Lemma \ref{art18-lem1}.
\end{proof}
        
\begin{lem}\label{art18-lem2}
The correspondence
$$
\tau \mapsto X (\tau w_0) = \overline{B \tau w_0 P}, \tau \in W
$$
induces a bijection of $W/W_{i(P)}$ onto the set of Schubert varieties in $G/P$.
\end{lem}

\begin{proof}
We have the following:
\begin{align*}
X(\tau_1 \omega_0) & = X (\tau_2 \omega_0) \Leftrightarrow \tau_1 \omega_0 = \tau_2 \omega_0 \theta, \theta \in W_P\\
& \Leftrightarrow \tau_1 = \tau_2 (\omega_0 \theta w_0) \Leftrightarrow \tau_1 = \tau_2 \lambda, \lambda \in W_{i(P)}.
\end{align*}
This proves Lemma \ref{art18-lem2}.
\end{proof}

\begin{definition}\label{art18-defi1}
We write $\tau_1 \leqslant \tau_2$ for $\tau_i \in W / W_{i(P)}$, whenever $X(\tau_1 \omega_0) \supseteq X (\tau_2 \omega_0)$. In a similar fashion we write $\tau_1 \geqslant \tau_2$, $\tau_1 \not\geqslant \tau_2$ or $\tau_1 \not\leqslant \tau_2$.
\end{definition}

We see that $\tau_1 \leqslant \tau_2$, $\tau_i \in W / W_{i(P)}$, if and only if we can find representatives of $\tau_i$ in $W$ (represented by the same letters $\tau_i$) such that a reduced expression of $\tau_2 \omega_0 $ (\resp $\tau_1$) can be obtained as a ``subword'' from a reduced expression of $\tau_1 w_0 \theta$  (\resp $\tau_2 \lambda$), $\theta \in W_P$ (\resp $\lambda \in W_{i(P)}$) (\cf p. 98, Prop. 1.7, \cite{art18-key12}).

\begin{lem}\label{art18-lem3}
We have the following:
\begin{itemize}
\item[(i)] $X(\tau_1 w_0) \not\subset \tau_2 X_0$ ~ if ~ $\tau_2 \geqslant \tau_1$

\item[(ii)] $X (\tau_2 w_0) \subset \tau_2 X_0$ ~ if ~ $\tau_2 \not\geqslant \tau_1$
\end{itemize}
i.e. (i) and (ii) can be written as
$$
X(\tau_1 w_0) \subseteq \tau_2 X_0 \Leftrightarrow \tau_2 \not\geqslant \tau_1.
$$
\end{lem}

\begin{proof}
\begin{itemize}
\item[(i)] We observe first that
$$
X (\tau_1 w_0) \not\subset \tau_1 X_0.
$$
To see this, suppose that this is not the case. Then we have
$$
\tau_1 w_0 \in \tau_1 X_0 \text{ which implies that } w_0 \in X_0.
$$
This\pageoriginale leads to a contradiction (for this would mean $X_0 =$ big cell).

Suppose that $\tau_2 \geqslant \tau_1$. Then $X (\tau_2 \omega_0) \subset X (\tau_1 w_0)$. We have 
$$
X (\tau_2 \omega_0) \not\subset \tau_2 X_0
$$
This implies, {\em a fortiori}, that $X(\tau_1 w_0) \not\subset \tau_2 X_0$.

\item[(ii)] We have to show that
$$
\tau^{-1}_2 X(\tau_2 w_0) \subset X_0 ~ \text{ if } ~ \tau_2 \not\geqslant \tau_1 .
$$
Let $C(w) = B w P$ (Bruhat or Schubert cell). It suffices to check that 
$$
\tau^{-1}_2 C(\tau_1 w_0) \subset X_0.
$$
Choose a representative of $\tau_2$ in $W$ and a reduced decomposition
$$
\tau_2 = s_1 \ldots s_p
$$
By the axioms (or properties) of a Tits system (axiom T-3 and its easy consequence, \cf \cite{art18-key4} pp. 22-23), we deduce that
\begin{equation*}
\tau^{-1}_2 C (\tau_1 w_0) \subset \bigcup\limits_{(s_{i_1} \ldots s_{i_k})} B (s_{i_k} \ldots s_{i_1}) \tau_1 w_0 P \tag{*}
\end{equation*}
where $(s_{i_1} \ldots s_{i_k})$ runs through all ``subwords'' of $s_1 \ldots s_p$ (i.e. a word consisting of a subset of $s_1, \ldots, s_p$ and written in the same order). Note that $s_{i_k}\ldots s_{i_1} = (s_{i_1} \ldots s_{i_k})^{-1}$. We claim that to prove (ii), it suffices to check that in (*) every $B(s_{i_k} \ldots s_{i_1}) \tau_1 w_0 P (\mod P)$
is not the big cell in $G/P$. For, if this property is true, then every $B(s_{i_k} \ldots s_{i_1}) \tau_1 w_0 P (\mod P) $ in (*) is contained in $X_0$ (the condimension one Schubert variety in $G/P$ contains any {\em proper} (i.e. $\neq G/P$) Schubert variety in $G/P$) and (ii) would then follow. Suppose then that 
$$
B(s_{i_k} \ldots s_{i_1}) \tau_1 w_0 P = B w_0 P (\text{big cell in }    G/P)
$$
where $s_{i_1} \ldots s_{i_k}$ is a subword of $(s_1 \ldots s_p)$. This implies that
$$
s_{i_k} \ldots s_{i_1} \tau_1 w_0 = w_0 \theta, \theta \in W_p
$$
\begin{align*}
\text{i.e. } \qquad s_{i_k} \ldots s_{i_1} \tau_1 = w_0 \theta w_0 = \theta', \theta' \in W_{i(P)}\\
\text{i.e. } \quad (s_{i_1} \ldots s_{i_k}) \theta' = \tau_1, \theta' \in W_{i(P)}.
\end{align*}
From Lemma \ref{art18-lem2} , it follows then that 
$$
X(\tau_1 w_0) = X(s_{i_1} \ldots s_{i_k} w_0).
$$
But\pageoriginale now $X(s_{i_1} \ldots s_{i_k} w_0) \supset X (\tau_2 w_0)$ since $s_{i_1} \ldots s_{i_k}$ is a subword of $\tau_2 = s_1\ldots s_p$. Hence $X(\tau_1 w_0) \supset X (\tau_2 w_0)$, which contradicts the assumption that $\tau_2 \not\geqslant \tau_1$. This concludes the proof of (ii) and hence of Lemma \ref{art18-lem3}.
\end{itemize}
\end{proof}

\begin{lem}\label{art18-lem4}
We have the following:

$X(\tau w_0) \cap \tau X_0 =$ union of all Schubert varieties in $X(\tau w_0)$ of codimension one = union of all proper Schubert subvarieties of $X(\tau w_0)$ ($X(\tau w_0)$ as in Lemma \ref{art18-lem2}).

Equivalently (because of Lemma \ref{art18-lem3})

(Big cell in $X(\tau w_0)$) $\cap \tau X_0$ is empty.
\end{lem}

(The intersections taken in this lemma are set theoretic).

\begin{proof}
Now $X(\tau w_0) \cap \tau X_0$ is of pure codimension one in $G/P$. Hence it suffices to prove that 

$X (\tau w_0) \cap \tau X_0 = $ union of all proper Schubert subvarieties of $X(\tau w_0)$. By Lemma \ref{art18-lem3}, if $\tau_1 \geqslant \tau$ and $\tau_1 \neq \tau$
$$
X(\tau_1 w_0) \subset \tau X_0 (\text{for } \tau \not\geqslant \tau_1).
$$
Hence $X(\tau w_0) \cap \tau X_0$ contains all the proper Schubert subvarieties of $X(\tau w_0)$ in $G/P$. Thus to prove Lemma \ref{art18-lem4}, it suffices to show that 
$$
(\text{Big cell in } X (\tau w_0)) \cap \tau X_0 ~ is empty.
$$
This is easily shown as follows (\cf Prop. A. 4, \cite{art18-key12}): we have to show that 
$$
B \tau w_0 P \cap \tau X_0 ~ \text{ is empty}.
$$
Take the action of $B$ on $G/B$ induced by left multiplication and let us compute the isotropy group at $\tau w_0 \in G / B$: 
$$
b \tau w_0 = \tau w_0 b' (b, b' \in B) \text{~ i.e. ~} (\tau w_0)^{-1} b (\tau w_0) \in B.
$$
Hence if $B_1$ is the isotropy sub-group at $(\tau w_0)$, we have 
\begin{align*}
B_1 &  = \{b \in B/ (\tau w_0)^{-1} b (\tau w_0) \in B\}\\
& = T \times \prod\limits_\alpha \mathscr{U}_\alpha / (\tau w_0)^{-1} (\alpha) > 0, \alpha > 0\\
& = T \times \prod\limits_{\alpha} \mathscr{U}_\alpha/ \alpha  >0, \tau^{-1} (-\alpha) > 0 \text{\quad i.e \quad} \tau^{-1} (\alpha) < 0.
\end{align*}
Now\pageoriginale $B = B_1 \times B_2$, where
$$
B_2 = \prod\limits_\alpha \mathscr{U}_\alpha
\begin{cases}
\tau^{-1} (\alpha) = \beta > 0, \alpha > 0\\
\text{i.e. ~ } \alpha  =\tau (\beta), \alpha > 0, \beta > 0
\end{cases}
$$
Hence
$$
T \cdot B_2 = B \cap \tau B \tau^{-1}.
$$
Therefore, if $y \in B \tau w_0 B$, we have 
$$
y = b_1 \tau w_0 b_2 , \quad b_i \in B, \quad b_1 = \tau c \tau^{-1}, c \in B.
$$ 
Hence
$$
y = \tau c w_0 b_2 ; \quad b_2, c \in B,
$$
This implies that 
$$
B \tau w_0 B \cap \tau X_0 \text{ is empty}
$$
for otherwise $\tau c w_0 b_2 = \tau \theta$, $\theta \in X_0$ which implies that $c w_0 b_2 \in X_0$ and this gives that $w_0 \in X_0$ which is a contradiction. Now it is immediate that 
$$
B \tau w_0 P \cap \tau X_0 \text{ ~ is empty}
$$
for otherwise
$$
x = b \tau w_0 p \in \tau X_0, ~ b \in B , ~ p \in P
$$
i.e. $x p^{-1} = b \tau w_0 \in \tau X_0$ i.e. $B \tau w_0 B \cap \tau X_0$ is empty-which is a contradiction. Thus Lemma \ref{art18-lem4} is proved. 
\end{proof}

\begin{lem}\label{art18-lem5}
Let $L$ be the line bundle on $G/P$ and $F$ the section of $L$ whose zero set is $X_0$ as in \S \ref{art18-sec1} (the space of sections of $L$ gives a realisation of the irreducible representation of $G$ with highest weight $i(\overline{\omega})$ in characteristic zero). Let $A_1$ be the linear subspace of $H^0(G/P, L)$ spanned by $\tau \cdot F$, $\tau \in W / W_{i(P)}$. Then the linear system $A_1$ has no base points.
\end{lem}

\begin{proof}
We have to show that given $x \in G/P$, there exists $\tau \in W / W_{i(P)}$ such that $x \not\in \tau X_0$. We can suppose that $x$ is in the big cell of a suitable Schubert variety $X(\tau w_0)$, as $G/P$ is the union of Schubert (or Bruhat)\pageoriginale cells and distinct Schubert cells are mutually disjoint. By the second part of Lemma \ref{art18-lem4}, it follows that $x \not\in \tau X_0$. This proves Lemma \ref{art18-lem5}.
\end{proof}

\begin{remark}\label{art18-rem1}
It is well-known that $\Pic G/P = \bfZ$, $L$ is ample and generates $\Pic G/P$. Since the linear system $A_1$ in Lemma \ref{art18-lem5} has no base points, $A_1$ defines a {\em morphism}
$$
\varphi : G/P \to \bfP (A^*_1), A^*_1 = \text{ dual of } A_1.
$$
We see that $\varphi$ is a {\em finite } morphism; this is a consequence of the fact that $L$ is ample, for on a fibre of $\varphi$ the restriction of $L$ is trivial and one would get a contradiction if the dimension of the fibre is strictly greater than zero. In particular we have 
$$
\dim. \text{ Image of } \varphi = \dim G / P.
$$
\end{remark}

\section{Standard monomials}\label{art18-sec3}
Let $L(\tau)$ denote the restriction of $L$ to the Schubert subvariety $X(\tau w_0)$ of $G/P$ (we endow $X(\tau w_0)$ with the canonical structure of a reduced closed subscheme of $G/P$), $L$ being the ample generator of $\Pic G/P$.

We set:
\begin{gather*}
R(\tau) = \bigoplus\limits^\infty_{n=0} H^0 (X (\tau w_0), L(\tau)^n); R (\tau)_n = H^0 (X (\tau w_0), L(\tau)^n)\\
R(e) = R = \bigoplus\limits^\infty_{n=0}H^0(G/P, L^n), R_n = H^0 (G/P, L^n)
\end{gather*}
($e$ = element of $W/W_{i(P)}$ corresponding to the class $W_{i(P)}$) 

\noindent
$I = W/ W_{i(P)}$, $I_r = \{\tau_1 \in I/ \tau_1 \geqslant \tau\}$.

\noindent
$p_r$ = the section $\tau \cdot F$ of $L$ on $G/P$ ($F$ as in \S \ref{art18-sec1} and Lemma \ref{art18-lem5}).

\noindent 
$A(\tau)$ = subalgebra of $R(\tau)$ generated by $P_\lambda,\lambda \in I_\tau$

\noindent
$A = A(e)$, $e$ = the element of $I = W/W_{i(P)}$ defined by $W_{i(P)}$.

\begin{definition}\label{art18-defi2}
An element $p \in R_m$ (\resp $R(\tau)_m$) of the form 
$$
p = p_{\tau_1} \ldots p_{\tau_m}, \tau_i \in I (\resp I_r), \tau_1 \leqslant \ldots \leqslant \tau_m)
$$  
is called a standard monomial in I (\resp $I_\tau$) of  degree or length $m$.
\end{definition}

\begin{prop}\label{art18-prop1}
Distinct\pageoriginale standard monomials in $I_\tau$ are linearly independent (as elements of $R(\tau)$). In particular distinct standard monomials in $I$ are linearly independent.
\end{prop}

\begin{proof}
We prove this by induction on $\dim X(\tau w_0)$. By definition $R(\tau)$ is a graded ring and hence it suffices to prove that standard monomials, say in $R(\tau)_m$ are linearly independent. Suppose now that $\dim X(\tau w_0)=0$ i.e. $X(\tau w_0)$ reduces to a point $x_0 \in G/P$. Then $I_\tau$ reduces to one element, namely $I_\tau = \{\tau\}$ and upto a constant multiple, there is only one standard monomial in $R(\tau)_m$ namely $p^m_\tau$. Further $p^m_\tau \neq 0$, for by Lemma \ref{art18-lem3}, $p_\tau (x_0) \neq 0$. Thus the proposition holds when $\dim X (\tau w_0) =0$.

Let us now pass to the general case. Observe that $R(\tau)$ is an integral domain. Since $p_\tau\neq 0$ in $R(\tau)$ (Lemma \ref{art18-lem3}), it follows in particular that 
$$
\lambda p_\tau = 0, ~~ \lambda \in R (\tau ) \Rightarrow \lambda = 0.
$$
Suppose now that $p\in R(\tau)_m$ can be expressed in the form 
\begin{equation*}
p = \sum\limits_{(\lambda)} a_{(\lambda)} p_{\lambda_1} p_{\lambda_{2}} \ldots p_{\lambda_m} \tag{*}
\end{equation*}
where $(\lambda) = (\lambda_1 \ldots, \lambda_m)$, $\lambda_1 \leqslant \ldots \leqslant \lambda_m$, runs over {\em distinct} elements of $I^m_\tau$ and $a_{(\lambda)} \neq 0$, $a_{(\lambda)} \in k$. We shall now show that $p=0$ leads to a contradiction.

\begin{romancase}\label{art18-romancasei}
Suppose that on the right side of (*) we have a (standard) monomial of the form $a_{(\beta)} p_{\beta_1} \ldots p_{\beta_m}$, $\beta_1 > \tau (\beta_1 \neq \tau)$ (of course $a_{(\beta)} \neq 0$).

Let $q$ denote the restriction of $p$ to $X (\beta_1 w_0)$. Then $q$ can be expressed as a sum of standard monomials as follows:
\begin{equation*}
q = \sum\limits_{(\lambda)} a_{(\lambda)} p_{\lambda_1} \ldots p_{\lambda_m}, (\lambda) \text{~ as in ~} (*) \text{ and } \lambda_1 \geqslant \beta_1, \tag{**}
\end{equation*}
i.e. $q$ is obtained from the right hand side of (*) by cancelling all the terms such that $\lambda_1 \not\geqslant \beta_1$ (by Lemma \ref{art18-lem3}, $p_{\lambda_1}$ vanishes on $X(\beta_1 w_0)$ when $\lambda_1 \not\geqslant \beta_1$). The right hand side of (**) contains at least one term, namely $a_{(\beta)} p_{\beta_1} \ldots p_{\beta_m}$; besides it consists of distinct standard monomials, as it is obtained from the right hand side of (*) by cancelling some terms. We have
$$
X (\beta_1 w_0) \subsetneqq  X (\tau w_0).
$$\pageoriginale
Hence by our induction hypothesis, $q \neq 0$, so that a fortiori $p \neq 0$. 
\end{romancase}

\begin{romancase}\label{art18-romancaseii}
We have only to consider the case, when in (*) $\lambda_1 = \tau$ for every ($\lambda$).

In this case we have $p = p_\tau p'$. Further $p'$ is a sum of distinct standard monomials, in fact in (*) we have only to cancel the first term $p_{\lambda_1} = p_\tau$ of every monomial. To prove that $p \neq 0$, it suffices to prove that $p' \neq 0$ since $R(\tau)$ is an integral domain. Now $p' \in R (\tau)_{m-1}$ and hence by induction on 
$m$ we can suppose that $p' \neq 0$. Thus $p\neq 0$.
\end{romancase}

This proves Proposition \ref{art18-prop1}.
\end{proof}

Let us recall that we denoted  by $A(\tau)$ (\resp $A$) the graded subalgebra of $R (\tau)$ (\resp $R$) generated by $p_\lambda$, $\lambda \in I_\tau$ (\resp $\lambda \in I$). Set 
$$
Y (\tau) = \text{Proj. } A (\tau), ~ Y = \text{Proj. } A (Y = Y (e)).
$$
Then $Y(\tau)$ is a variety and can be identified with the image of $X(\tau w_0)$ by the canonical morphism into a projective space defined by the linear system $A(\tau)_1 = \{p_\lambda / \lambda \in I_\tau\}$. By Lemma \ref{art18-lem5}, the canonical morphism $X(\tau w_0) \to Y(\tau)$ is a finite morphism and since it is surjective, it follows that $\dim Y(\tau) = \dim X (\tau w_0)$. The graded algebra $A(\tau)$ is generated by $A(\tau)_1$ (=homogeneous elements of degree one in $A(\tau)$). We have canonical {\em surjective } homomorphisms
$$
A \to A (\tau) \to 0, A (\tau_1) \to A (\tau_2) \to 0, \tau_2 \geqslant \tau_1
$$
induced by the canonical (restriction) homomorphisms
$$
R \to R (\tau), R (\tau_1) \to R (\tau_2), \tau_2 \geqslant \tau_1.
$$
We thus get canonical immersions
$$
Y(\tau) \hookrightarrow Y, Y (\tau_2) \hookrightarrow Y (\tau_1), \tau_2 \geqslant \tau_1.
$$
We see that $A(\tau)$ is the homogeneous coordinate ring of $Y(\tau)$ in the projective space associated to the linear system $A(\tau)_1$ (on $X(\tau w_0)$).

\begin{definition}\label{art18-defi3}
A subset $T$ of $I$ is said to be a `right half space' (\resp `left half space') if the following property holds:
$$
\lambda \in T, \mu \in I \text{~ and ~} \mu \geqslant \lambda (\resp \mu \leqslant \lambda) \Rightarrow \mu \in T.
$$
\end{definition}

We see\pageoriginale that intersections and unions of right (\resp left) half spaces are again right (\resp left) half spaces. The complement in $I$ of a right half space is a left half space. If $T$ is a right half space, then
$$
T = I_{\tau_1} \cup \ldots \cup I_{\tau_r}
$$
where $\tau_1, \ldots, \tau_r$ are the distinct minimal elements of $T$. We set $X(T)$ to be the {\em schematic union}
$$
X (T) = X (\tau_1 w_0) \cup \ldots \cup X ( \tau_r w_0).
$$
We have then 
$$
T = \{\tau \in I/ \text{restriction of } p_\tau \text{ to } X (T) \neq 0\}.
$$
Since $X(\tau_i w_0)$ is reduced, $X(T)$ is reduced. We speak of {\em standard monomials in a right half space $T$}, namely elements of the form 
$$
p_{\tau_1} p_{\tau_2} \ldots p_{\tau_m}, \tau_1 \in T (\text{then all } \tau_i \in T).
$$
From Proposition \ref{art18-prop1}, it follows immediately that distinct standard monomials in $T$ are linearly independent (in fact their restrictions to $X(T)$ are linearly independent). We define $Y(T)$ as the {\em schematic union}
$$
Y(T) = \cup Y (\tau_i), \tau_i \text{ minimal elements of $T$}.
$$
We see that $Y(T)$ is \textit{reduced} since $Y(\tau_i)$ is reduced. We note that 
$$
X(I_\tau) = X (\tau w_0), Y (I_\tau) = Y(\tau), \quad \tau \in I.
$$
We denote by $Y_0$ the subvariety of $Y$, which is the image of $X_0$ by the canonical morphism $X \to Y$ i.e. $Y_0 = Y(s)$, $s = S - S_P$. We denote by $\tau Y_0$ the subvariety of $Y$, which is the zero set of $p_\tau$, $\tau \in W / W_{i(P)}$. We denote by $R(T)$ the graded algebra
$$
R(T) =\bigoplus\limits^{\infty}_{n=0} H^0 (X(T), (L/ X (T))^n)
$$
and by $A(T)$ the graded subalgebra of $R(T)$ spanned by $\{p_\tau\}$, $\tau \in T$ (it suffices to take $\tau \in T$). We see that $A(T)$ is the homogeneous coordinate ring of $Y(T)$ (with respect to the projective imbedding of $Y$ considered above) and that 
$$
A (T) = A(I_\tau) = A(\tau) \text{~ when ~} T = I_\tau.
$$
We have canonical surjective homomorphisms 

$A \to A(T) \to 0$, \pageoriginale $A(T_1) \to A (T_2) \to 0$; $T_2 \subseteq T_1$ ($T_i$ right half spaces).

\begin{definition}\label{art18-defi4}
Let $T$ be a right half space. We set 

$\chi (T, m) \neq \sharp $ (Set of distinct standard monomials in $T$ of degree $m$)

$\chi(T,0) =1$ and $\chi(T, m) =0$, $m < 0$.
\end{definition}

\begin{lem}\label{art18-lem6}
\begin{itemize}
\item[(i)] $I_\tau - \{\tau\}$ is a right half space; in fact if $T$ is a right half space and $\tau_1, \ldots, \tau_r$ are some distinct element chosen from the minimal elements of $T$, then

$T - \{(\tau_1, \ldots, \tau_r)\}$ is again a right half space. We have 
$$
I_\tau - \{\tau\} =\bigcup\limits_{\tau_i} I_{\tau_i}
$$
where $\tau_i$ are precisely the minimal elements in $I_\tau$ such that $l(\tau_i) = l(\tau) +1$ ($l$ = length in $W/W_{i(P)}$ or equivalently $l(\tau)$ = codimension of $X(\tau w_0)$ in $G/P$).

\item[(ii)] We have (similar to Lemma \ref{art18-lem4})
$$
Y (\tau) \cap \tau Y_0 = Y(T), ~ T = I_\tau - \{\tau\}
$$
in the set theoretic sense.

\item[(iii)] $\chi (T_1 \cup T_2, m) = \chi(T_1, m) + \chi (T_2,m) - \chi (T_1 \cap T_2, m )$

\item[(iv)] $\chi (I_\tau , m) = \chi (I_r - \{\tau\}, m) + \chi (I_\tau, m -1)$

\end{itemize}
\end{lem}

\begin{proof}
The proof of (ii) is an immediate consequence of Lemma \ref{art18-lem4} and the definition of $Y(\tau)$. The proof of the other assertions is straightforward and immediate and is left as an exercise.
\end{proof}

\section{Consequences of the hypothesis of generation by standard monomials.}\label{art18-sec4}
Let $T_0$ be a right half space in $I$. If $T$ is a right half space such that $T \subseteq T_0$, we see that $(T_0 - T)$ is a  {\em left half space in $T_0$} i.e.
$$
\lambda \in (T_0 - T), \mu \leqslant \lambda \Rightarrow \mu \in T_0 - T.
$$
We can similarly define {\em right half spaces $T$ in $T_0$,} but this is equivalent to the fact that $T$ is a right half space in $I$, $T \subseteq T_0$. As in the case $T_0 =I$, we note that  

$T$ right half space in $T_0 \Leftrightarrow T_0 - T$ is a left half space in $T_0.$

\begin{prop}\label{art18-prop2}
Let\pageoriginale $T_0$ be a right half space in $I$. Suppose that $A(T_0)$ is spanned by standard monomials in $T_0$. Then we have:
\begin{itemize}
\item[\rm (i)] For every right half space $T$ in $T_0$, $A(T)$ is spanned by standard monomials in $T$ and the kernel of the canonical surjective homomorphism (of algebras)
$$
A(T_0) \to A (T)
$$
is the ideal in $A(T_0)$ generated by $\{p_\tau\}$, $\tau \in T_0 - T$.

\item[\rm (ii)] Let $S$ be a left half space in $T_0$, $T = T_0 - S$ and $J(S)$ the ideal in $A(T_0)$ generated by $\{p_\tau\}$ $\tau \in S$. Then the map
$$
S \mapsto J(S)
$$
from the set of left half spaces in $T_0$ into the set of ideals in $A(T_0)$ takes set intersection into ideal intersection, set union into ideal sum and  preserves distributivity properties.

\item[\rm (iii)] Consider the map
$$
T \mapsto Y (T)
$$
from the set of right half spaces in $T_0$ into the set of closed subschemes of $Y(T_0)$. Then this is a bijective map of the set of right half spaces in $T_0$ onto the set of closed subschemes of $Y(T_0)$, each member of which is a (scheme theoretic) union of $Y(\tau)$, $\tau \in T_0$. Further this map takes set union into scheme theoretic union, set intersection to scheme theoretic intersection and preserves distributivity properties. Since unions and intersections of right half spaces in $T_0$ are again right half spaces in $T_0$ and $Y(T)$ is reduced ($T$ right half space in $T_0$), it follows that unions and intersections of $Y(T)$ are again reduced.
\end{itemize}
\end{prop}

\begin{proof}
\begin{itemize}
\item[\rm (i)] Let $j$ be the canonical homomorphism
$$
j : A (T_0) \to A (T).
$$
Since $j$ is surjective, it is immediate that $A(T)$ is generated by standard monomials, since $A(T_0)$ has this property by hypothesis. Let $J$ be the ideal in $A(T_0)$ generated by $\{p_\tau\}$, $\tau \in T_0 - T$. By (ii) of Lemma \ref{art18-lem3}, it follows that $J\subset \Ker j$. Hence $j$ induces a canonical homomorphism
$$
j' : A (T_0) / J \to A (T).
$$
We have\pageoriginale to prove that $j'$ is an isomorphism. Let $\theta_1$ be a non-zero element of $A(T_0)/J$. Choose a representative $\theta \in A (T_0)$ of $\theta_1$. Then $\theta$ can be written as a linear combination of standard monomials in $T_0$, but since $p_\tau \in J$ for $\tau \in T_0 -T$, in this sum we can cancel those terms which involve $p_\tau$, $\tau \in T_0 - T$. Thus the representative $\theta$ of $\theta_1$ can be chosen so that it is a linear combination of standard monomials in $T$ and $\theta \neq 0$. Now by Prop. \ref{art18-prop1}, it follows that $j(\theta) \neq 0$. This implies that $j'(\theta_1) \neq 0$, which shows that $j'$ is injective. This proves (i).

\item[\rm (ii)] The map $J$ in (ii) of Prop. \ref{art18-prop2}, obviously takes set union to ideal sum. On the other hand, to show that $J(S_1) \cap J(S_2) = J(S_1 \cap S_2)$, $S_i$ left half space in $T_0$, we first observe $J(S_1 \cap S_2) \subset J (S_1) \cap J(S_2)$. If $T_1 = T_0 - S_1$ and $T_2 = T_0 - S_2$, we see immediately that $J(S_1) \cap J(S_2)$ vanishes on $X(T) (= X (T_1) \cup X(T_2))$, $T = T_1 \cup T_2$. On the other hand by (i) $J(S_1 \cap S_2)$ is the ideal of all elements in $A(T_0)$ vanishing on $X(T)$. Hence $J(S_1 \cap S_2) \supset J(S_1) \cap J (S_2)$. This proves the assertion $J(S_1 \cap S_2) = J(S_1) \cap J (S_2)$. The other assertions in (ii) follow easily.

\item[\rm (iii)] On account of (i), (iii) is just a reformulation of (ii).
\end{itemize}

This completes the proof of Prop. \ref{art18-prop2}.
\end{proof}

\begin{coro}\label{art18-coro1}
Let $T_0$ be a right half space in $I$ such that $A (T_0)$ is spanned by standard monomials. Let $\tau$ be an element of $W/W_{i(P)}$ such that $I_\tau \subset T_0$. Then the set theoretic intersection in (i), Lemma \ref{art18-lem6}, is in fact scheme theoretic, i.e.
$$
Y(\tau) \cap \tau Y_0 = Y(T), T = I_\tau - \{\tau\}
$$
where the intersection is scheme theoretic. In particular, the scheme theoretic intersection $Y(\tau) \cap \tau Y_0$ is reduced.
\end{coro}

\begin{proof}
This is an immediate consequence of Prop. \ref{art18-prop2}. The scheme theoretic intersection $Y(\tau) \cap \tau Y_0$ is defined by the ideal $J$ in $A(T_0)$ generated by $p_\lambda$, $\lambda \in T_0- I_\tau$ and $p_\tau$. We observe that $(T_0 - I_\tau) \cup \{\tau\}$ is a left half space in $T_0$, since its complement $I_\tau - \{\tau\}$ is a right half space (\cf (i), Lemma \ref{art18-lem6}). Hence $Y(T)$ is the scheme theoretic intersection $Y(\tau) \cap \tau Y_0$, $T = I_\tau -\{\tau\}$. This proves the corollary.
\end{proof}

\begin{remark}\label{art18-rem2}
Let\pageoriginale $M$ be the very ample line bundle on $Y$ (\cf \S \ref{art18-sec3} for the definition of $Y$) induced by the linear system $\{p_\tau\}$, $\tau \in I$. We denote this bundle by $\mathscr{O}_Y$ \ref{art18-eq1} and we denote the restriction of this line bundle to $Y(T)$ by $\mathscr{O}_{Y(T)}$ \ref{art18-eq1}, $T$ a right half space in $I$.

Suppose now that $T_0$ is a right half space in $I$ such that $A(T_0)$ is generated by standard monomials (in $T_0$). Let $I_\tau \subset T_0$, $\tau\in I$. Then Cor. \ref{art18-coro1}, Prop. \ref{art18-prop2} gives the exact sequence
\begin{equation*}
0 \to \mathscr{O}_{Y(\tau)} (-1) \to \mathscr{O}_{Y(\tau)} \to \mathscr{O}_{Y(T)} \to 0, T = I_\tau - \{ \tau\} . 
\tag*{(1)}\label{art18-eq1}
\end{equation*}
\end{remark}


\begin{coro}\label{art18-coro2}
Let $\{T_i\}$, $1 \leqslant i \leqslant r$, be a family of right half spaces in $I$, such that $A(T_i)$ is spanned standard monomials, $1 \leqslant i \leqslant r$. Then if $T = T_1 \cup \ldots \cup T_r$, $A(T)$ is also spanned by standard monomials.
\end{coro}

\begin{proof}
By induction on $r$, obviously it suffices to prove the proposition for the case $r = 2$. First we claim that 
$$
Y(T_1) \cap Y (T_2) = Y(T_1 \cap T_2) \text{ (scheme theoretically).}
$$
To prove this, consider the ideal $J$ in $A(T_1)$ which defines the scheme theoretic intersection $Y(T_1) \cap Y (T_2)$. Obviously $J$ contains all $p_\tau$ for $\tau \in (T_1 - T_1 \cap T_2)$. By Prop. \ref{art18-prop2}, since $A(T_1)$ is generated by standard monomials and $T_1 - T_1 \cap T_2$ is a left half space in $T_1$, it follows that $A(T_1)/ J = A (T_1 \cap T_2)$. This is precisely the above claim.

Let now $B$ be a commutative ring and $J_1$, $J_2$ two ideals in $B$. Then one sees easily that the homomorphism $B \to B / J_1 \oplus B / J_2$ defined by $b \mapsto $ ($b \mod J_1$, $b \mod J_2$) induces the following exact sequence of $B$-modules
$$
0 \to B/ J_1 \cap J_2 \to B / J_1 \oplus B / J_2 \xrightarrow{j} B / J_1 + J_2 \to 0
$$ 
where $j(b_1, b_2) = b_1 - b_2 (\mod J_1 + J_2)$ and $(b_1, b_2)$ in $B \oplus B$ represents an element of $B/ J_1 \oplus B / J_2$. If $J_1 \cap J_2 =0$, then this gives
\begin{equation*}
0 \to B \to B / J_1 \oplus B / J_2 \to B / J_1 + J_2 \to 0. \tag{*}
\end{equation*}
Let $J = \Spec B$ and $Z_i$ the closed subschemes of $Z$, $Z_i = \Spec B/J_i$, $i=1,2$. Then we have 
\begin{itemize}
\item[\rm (i)] $J_1 \cap J_2 = 0 \Leftrightarrow Z = Z_1 \cup Z_2$ (scheme theoretic union)

\item[\rm (ii)] $\Spec B/ J_1 + J_2 = Z_1 \cap Z_2$.\pageoriginale
\end{itemize}
Thus from (*), it follows that if $Z_i$, $i= 1,2$, are closed subschemes of $Z$ such that $Z = Z_1 \cup Z_2$ (scheme theoretic), then we have the following exact sequence of $\mathscr{O}_Z$-modules (patching up $Z_1$ and $Z_2$ along $Z_1 \cap Z_2$)
$$
0 \to \mathscr{O}_Z \to \mathscr{O}_{Z_1} \oplus \mathscr{O}_{Z_2}  \to \mathscr{O}_{Z_1 \cap Z_2} \to 0.
$$

From the above general remark, it follows that we have an exact sequence of $\mathscr{O}_{Y(T)}$-modules $(T = T_1 \cup T_2)$
\begin{equation*}
0 \to \mathscr{O}_{Y(T)} \to \mathscr{O}_{Y(T_1)} \oplus \mathscr{O}_{Y(T_2)} \to \mathscr{O}_{Y(T_1 \cap T_2)} \to 0. \tag*{(2)}\label{art18-eq2}
\end{equation*}
By hypothesis
$$
\dim A(T_i)_m = \chi (T_i, m), i= 1,2
$$
and by Prop. \ref{art18-prop2}, since $T_1 \cap T_2 \subset T_1$
$$
\dim A(T_1 \cap T_2)_m = \chi (T_1 \cap T_2, m).
$$
By (ii) of Lemma \ref{art18-lem6}, it follows that
$$
\chi (T,m) =\dim A(T)_m.
$$
On the other hand, by Prop. \ref{art18-prop1}, the subsapce in $A(T)_m$ spanned by standard monomials is of dimension $\chi(T,m)$. Hence it follows that $A(T)_m$ is spanned by standard monomials. This proves Corollary \ref{art18-coro2}, Proposition \ref{art18-prop2}.
\end{proof}

\begin{prop}\label{art18-prop3}
Let $T_0$ be a right half space in $I$ such that $A(T_0)$ is spanned by standard monomials. Then for every right half space $T$ in $T_0$, we have:
\begin{itemize}
\item[\rm (i)] $\dim H^0(Y (T), \mathscr{O}_{Y(T)} (m)) =\chi (T, m)$ or equivalently (because of Prop. \ref{art18-prop2})
$$
A(T)_m = H^0 (Y(T), \mathscr{O}_{Y(T)} (m)), m \geqslant 0.
$$
In particular, if $T_1$, $T_2$ are two right half spaces in $T_0$ such that $T_1 \subset T_2$, then the canonical homomorphism
$$
H^0 (Y(T_2), \mathscr{O}_{Y(T_2)} (m)) \to H^0 (Y(T_1), \mathscr{O}_{Y(T_1)}(m)), m \geqslant 0
$$
is surjective

\item[\rm (ii)] $H^i (Y (T), \mathscr{O}_{Y(T)}(m)) = 0$, $i > 0$, $m \geqslant 0$
\end{itemize}
 \end{prop}

\begin{proof}
We prove\pageoriginale the proposition by induction on $\dim Y(T)$ ($=\dim X(T)$, \cf Remark \ref{art18-rem1}). Let $n = \dim Y(T)$. We claim that it suffices to prove the proposition when $Y (T)$ is of the form $Y(\tau)$, $\tau \in I$. To prove this, since $Y(T)$ is the schematic union of $Y(\tau_i), T = \bigcup\limits^r_{i=1} I_{\tau_i} $ ($\tau_i$-the minimal elements of $T$), we see easily that it suffices to prove the following:
\begin{equation*}
\left\{
\begin{aligned}
& \text{Let }  T_1 = I_{\tau_1} \text{ and  } T_2 = \bigcup\limits^r_{i=2} I_{\tau_i}. \text{ Suppose that (i) and }\\
& \text{ (ii)  above hold for $Y(T_1)$ and $Y(T_2)$; then they hold for}\\
&  Y (T), \;   T = T_1 \cup T_2.\\
\end{aligned}
\right\} \tag{*}
\end{equation*}
The claim (*) is an easy consequence of the exact sequence 2 (\cf Cor. \ref{art18-coro2}, Prop. \ref{art18-prop2}), namely the exact sequence of $\mathscr{O}_{Y(T)}$ modules 
\begin{equation*}
0 \to \mathscr{O}_{Y(T)} \to \mathscr{O}_{Y(T_1)} \oplus \mathscr{O}_{Y(T_2)} \to\mathscr{O}_{Y (T_1 \cap T_2)} \to 0 \tag{**}
\end{equation*}
We see that $\dim Y(T_1 \cap T_2) \leqslant (n-1)$. Hence the proposition is true for $Y (T_1 \cap T_2)$. In particular,
$$
H^0 (\mathscr{O}_{Y(T_1 \cap T_2)} (m)) =A (T_1 \cap T_2)_m
$$
and it is spanned by standard monomials. Hence the canonical map
\begin{equation*}
H^0 (\mathscr{O}_{Y(T_1)} (m) \oplus \mathscr{O}_{Y(T_2)} (m)) \to H^0 (\mathscr{O}_{Y(T_1 \cap T_2)}(m)) \tag{***}
\end{equation*}
is surjective for $m \geqslant 0$. Writing the cohomology exact sequence for (**), we get
$$
\dim H^0 (\mathscr{O}_{Y(T_1 \cup T_2)} (m))  = \chi(T_1, m) + \chi (T_2,m) - \chi (T_1 \cap T_2, m),
$$ 
which implies (by Lemma \ref{art18-lem6}) that 
$$
\dim H^0(\mathscr{O}_{Y(T_1\cup T_2)}(m)) =\chi (T_1\cup T_2, m).
$$
Thus the assertion \ref{art18-eq1} of Prop. \ref{art18-prop3} follows.

Since the map (***) is surjective, we see that the sequence
\begin{equation*}
0 \to H^1 (\mathscr{O}_{Y (T_1 \cup T_2)}(m)) \to H^1 (\mathscr{O}_{Y(T_1)} (m)) \oplus H^1 (\mathscr{O}_{Y(T_2)} (m)), m \geqslant 0 \tag*{(3)}\label{art18-eq3}
\end{equation*}
induced by the cohomology exact sequence of (**) is exact. Further we get the exact sequence
$$
0 \to H^{i-1} (\mathscr{O}_{Y (T_1 \cap T_2)}(m)) \to H^i (\mathscr{O}_{Y(T)} (m)) \to H^i (\mathscr{O}_{Y(T_1)}(m)) \oplus H^i
$$
$(\mathscr{O}_{T(Y_2)}(m))$\pageoriginale for $i \geqslant 2$, $m \geqslant 0$. Since $\dim Y(T_1 \cap T_2) \leqslant (n-1)$, by the induction hypothesis, we have
$$
H^{i-1} (\mathscr{O}_{Y(T_1 \cap T_2)} (m)) =0, \; i \geqslant 2, \; m \geqslant 0.
$$ 
Thus we get the exact sequence
\begin{equation*} 
0 \to H^i (\mathscr{O}_{Y (T_1 \cup T_2)}(m)) \to H^i (\mathscr{O}_{Y(T_1)}(m)) \oplus H^i (\mathscr{O}_{Y(T_2)} (m)), 
\begin{cases}
i \geqslant 2\\
m \geqslant 0.
\end{cases}
\end{equation*} 
Combining this with \ref{art18-eq3} above, we get
$$
H^i (\mathscr{O}_{Y(T)}(m))= 0, \; i \geqslant 1, m \geqslant 0.
$$
This proves the claim (*) above i.e. it suffices to prove the proposition for $Y(\tau)$, $\tau\in I$.

Let  $T = I_\tau - \{\tau\}$. Tensoring the exact sequence \ref{art18-eq1} in Remark \ref{art18-rem2} by $\mathscr{O}_{Y(\tau)} (m)$, we get the exact sequence
\begin{equation*}
0 \to \mathscr{O}_{Y(\tau)} (m-1) \to\mathscr{O}_{Y(\tau)} (m) \to \mathscr{O}_{Y(T)} (m) \to0, m \geqslant 0. \tag*{(4)}\label{art18-eq4}
\end{equation*}
We have $\dim Y(T) = (n-1)$. By our induction hypothesis, it follows, in particular, that $H^0(\mathscr{O}_{Y(T)} (m)) = A(T)_m$ and that it is spanned by standard monomials. Thus implies that the canonical map
\begin{equation*}
H^0(\mathscr{O}_{Y(\tau)}(m)) \to H^0 (\mathscr{O}_{Y(T)} (m)), \tag*{(5)}\label{art18-eq5}
\end{equation*}
is surjective. If we set $\lambda_m = \dim H^0(\mathscr{O}_{Y(\tau)}(m))$, writing the cohomology exact sequence for \ref{art18-eq4} at the $H^0$ level, we get 
$$
\lambda_m - \lambda_{m-1} =\chi (T,m) \text{ for all } m (\text{note } \lambda_m = \chi (T, m) = 0, m < 0 ). 
$$
On the other hand, by Lemma \ref{art18-lem6}, we get 
$$
\chi (I_\tau, m) -\chi (I_\tau, m-1) = \chi (T,m) \text{ for all } m.
$$
For $m \leqslant 0$, we see immediately that $\chi (I_\tau, m) = \lambda_m$. From this it follows that $\chi(I_\tau, m) = \lambda_m$ i.e.
$$
\chi(I_\tau, m) = \dim H^0(\mathscr{O}_{Y(\tau)}(m)) \text{ for all }m.
$$
Writing the cohomology exact sequence for \ref{art18-eq4} and using the fact that the proposition holds for $\mathscr{O}_{Y(T)} (m)$ (on account of the inductive hypothesis) and the surjective map \ref{art18-eq5} above, we deduce easily the following:
$$
0 \to H^i (\mathscr{O}_{Y(\tau)}(m-1)) \to H^i (\mathscr{O}_{Y(\tau)}(m)) \text{ is exact, } i \geqslant 1, m \geqslant 0.
$$\pageoriginale
One knows that $H^i(\mathscr{O}_{Y(\tau)}(m)) = 0$, $i \geqslant 1$ and $m$ sufficiently large (Serre's theorem). Hence by decreasing induction on $m$, we deduce immediately that 
$$
H^i (\mathscr{O}_{Y(\tau)} (m)) = 0, \; i \geqslant 1, \; m \geqslant 0.
$$
This concludes the proof of Prop. \ref{art18-prop3}.
\end{proof}


\setcounter{coro}{0}
\begin{coro}%%% 1
Let $T_0$ be a right half space in $I$ such that $A(T_0)$ is spanned by standard monomials. Let
$$
\hat{Y} (T) = \Spec A(T), T \text{ ~ right half space in ~} T_0.
$$
Then we have the following:
\begin{itemize}
\item[(i)] $\hat{Y} (T)$ is reduced
\item[(ii)] Consider the map
$$
T \mapsto \hat{Y}(T)
$$
from the set of right half spaces in $T_0$ into the set of closed subschemes of $\hat{Y} (T_0)$. Then this is a bijective map of the set of right half spaces in $T_0$ onto the set of closed subschemes of $\hat{Y}(T_0)$, each member of which is a (schematic) union of $\hat{Y}(\tau)$, $\tau \in T_0$. Further this map takes set union into scheme theoretic union, set intersection to scheme theoretic intersection and preserves distributivity properties. Since unions and intersections of right half spaces in $T_0$ are again right half spaces in $T_0$ and $\hat{Y}(T)$ is reduced, it follows that unions and intersections of $\hat{Y}(T)$ are again reduced ($T$ right half spaces in $T_0$).

\item[(iii)] $\hat{Y}(\tau)$ is normal for $\tau \in I$ such that $I_\tau \subset T_0$. In particular, for such  a $\tau$, $Y(\tau)$ is normal. We refer to $\hat{Y}(T)$ as cones over $Y(T)$.
\end{itemize}
\end{coro}

\begin{proof}
Since $Y(T)$ is reduced and
$$
A(T) = \bigoplus\limits_{m \geqslant 0} H^0 (\mathscr{O}_{Y(T)} (m))  ~ \text{ (by Prop. \ref{art18-prop3})},
$$
it follows immediately that $\hat{Y}(T)$ is reduced. The assertion (ii) is merely the assertion (ii) of Prop. \ref{art18-prop2}.

Let\pageoriginale $\tau \in I$ be such that $I_\tau \subset T_0$. Hence as in the proof of Cor. \ref{art18-coro1}, Prop. \ref{art18-prop2}, it follows that 
$$
A(\tau)/_{p_\tau A(\tau)} = A(T), T = I_\tau - \{\tau\}.
$$
Let $\mathscr{O}_\tau$ be the local ring of $\hat{Y}(\tau)$  at its ``\textit{vertex}'' i.e. the point corresponding to the irrelevant maximal ideal of the graded ring $A(\tau)$. Since $A(T)$ is reduced by (i) above, it follows that $\mathscr{O}_\tau/{}_{p_\tau \mathscr{O}_\tau}$ is reduced. This implies that
$$
\text{depth ~} \mathscr{O}_\tau \geqslant 2.
$$
On the other hand, it has been shown by Chevalley \cite{art18-key6} that the singularity subset of $Y(\tau)$ is of Codim $\geqslant 2$ in $Y(\tau)$. It follows that the singular locus of $\mathscr{O}_\tau$ is of Codim $\geqslant 2$ in $\Spec \mathscr{O}_\tau$. It follows then that $\mathscr{O}_\tau$ is normal. i.e. the cone $\hat{Y}(\tau)$ is normal at its vertex. Hence $\hat{Y}(\tau)$ as well as $Y(\tau)$ are normal (note of course that $\hat{Y}(\tau)$ is the {\em cone over} $Y(\tau)$  in the usual sense for the imbedding of $Y(\tau)$ in the projective space corresponding to the linear system on $Y$ defined by $A_1$). This proves Cor. \ref{art18-coro1}, Prop. \ref{art18-prop3}. 
\end{proof}

\begin{remark}\label{art18-rem3}
Let $T_0$ and $\tau$ be as in Cor. \ref{art18-coro1}, Prop. \ref{art18-prop3} $(I \tau \subset T_0)$. It should be possible to prove that $\hat{Y}(\tau)$ is Cohen-Macaulay by showing that 
\begin{equation*}
H^i (\mathscr{O}_{Y(\tau)} (m)) =0, 0 \leqslant i < \dim Y (\tau), m < 0 \tag{*}
\end{equation*}
Let $T = I_\tau - \{\tau\}$ and $\tau_i$, $1\leqslant i \leqslant r$, be the minimal element of $T$ so that $Y(\tau_i)$ are the irreducible components of $Y(\tau) \cap \tau Y_0$. For proving (*), following the argument as in Prop. \ref{art18-prop3} for the case $m \geqslant 0$ (see for example \cite{art18-key13}), it is seen easily that one has to show that 
$$
Y (\tau_i) \cap Y (\tau_j)  (i \neq j) \text{ is of pure  Codim 1 in ~} Y(\tau_i)
$$
Equivalently one has to show that (\cf Remark \ref{art18-rem1})
$$
X(\tau_i w_0) \cap X(\tau_j w_0) (i \neq j) \text{~ is of Codim 1 in ~}  X (\tau_i w_0).
$$
When $P$ is {\em miniscule} (\cf \S \ref{art18-sec5}, below), this has been proved by Kempf (\cf \cite{art18-key11}) and a proof has also been shown to the author by Lakshmibai and Musili.
\end{remark}

\begin{remark}\label{art18-rem4}
Suppose that $Y(\tau) (\tau \in I)$ is of dimension $n$ and that $A(T)$\pageoriginale is spanned by standard monomials whenever $\dim Y(T) \leqslant (n-1)$ ($T$ a right half space in $I$). Then to show that $A(\tau)$ is spanned by standard monomials, {\em we claim that it suffices to prove that the scheme theoretic intersection $Y(\tau) \cap \tau Y_0$ is reduced}. For, one has the following exact sequence (\cf exact sequence \ref{art18-eq1} in Remark \ref{art18-rem2}, what is essential for having this exact sequence is that $Y(\tau) \cap \tau Y_0$ is reduced and in Remark \ref{art18-rem2} this follows from Cor. \ref{art18-coro1}, Prop. \ref{art18-prop2})
\begin{equation*}
0 \to \mathscr{O}_{Y(\tau)} (m-1) \to \mathscr{O}_{Y(\tau)} (m) \to \mathscr{O}_{Y(T)} (m) \to 0 \tag{*}
\end{equation*}
where $T = I_\tau - \{\tau\}$ (this exact sequence is obtained by tensoring the exact sequence \ref{art18-eq1} of Remark \ref{art18-rem2} by $\mathscr{O}_{Y(\tau)} (m)$). Then as in the proof of Prop. \ref{art18-prop3}, writing the cohomology exact sequence for (*), we deduce that $A(\tau)$ is spanned by standard monomials. In particular, if the scheme theoretic intersection $Y(\tau) \cap \tau Y_0$ is reduced for every $\tau \in I$, it will follow that $A(T)$ is generated by standard monomials for every right half space $T$ in $I$.
\end{remark}

\begin{remark}\label{art18-rem5}
Suppose that for a given $\tau \in I$ (or more generally a given right half space $T$ in $I$) {\em $A(\tau)$ is spanned by standard monomials when the ground field is of characteristic zero. Then we claim that $A(\tau)$ is spanned by monomials when the field is of arbitrary characteristic.}
\end{remark}

Let $D$ be a complete discrete valuation ring with quotient field $K$ of characteristic zero and residue field the (algebraically closed) ground field $k$, assumed to be of characteristic $p>0$. We know that $G = G' \otimes_D k$, where $G'$ is a {\em split} semisimple group scheme over $D$. One knows that the above considerations go through over the base $\Spec D$ (\cf \cite{art18-key15}, esp. Vol. III). To be more precise, let us note the following:
\begin{itemize}
\item[(i)] We have a maximal parabolic subgroup $P'$ in $G'$ such that $P'\otimes k = P$ and $G'/P'\otimes k = G/P$. Further $\Pic G'/P' \approx \bfZ$ and it has an ample generator $L'$ such that $L'\otimes k= L$, where $L$ is the ample generator of $\Pic G/P$. We have a canonical section $F' \in H^0(G'/P',L')$ whose zero set is the Schubert scheme $X'_0$ of codimension one in  $G'/P'$ with $X'_0 \otimes k = X_0$. We can also suppose that $T'$ is a maximal split torus in $G'$ such that $T' \otimes k = T$ and that in the $G'-D$ module $H^0(G'/P',L')$\pageoriginale (for the notation $G'-D$ module, see \cite{art18-key15}), the $D$ module spanned by $F'$ is $T'$ stable.

\item[(ii)] the Weyl group $W$ of $G$ is the Weyl group of the abstract root system associated to the split semi-simple group scheme $G'$. Then one knows that $W_D = N (T') / T'$, where $W_D$ is the constant group scheme over $\Spec D$ associated to the abstract group $W$  (\cf p.168-169, Vol. III, \cite{art18-key15}). The parabolic subgroup scheme $P'$ has a structure similar to that of $P$ (\cf \S \ref{art18-sec1}) and we can talk of the parabol-ic subgroup scheme $i(P')$, $i = - w_0$ (Weyl involution). The Weyl group of $i(P')$  is the constant group scheme (over $\Spec D$) associated to the abstract group $W_{i(P)}$.

\item[(iii)] the elements of $W$ can be canonically identified with the $D$-valued points of $W_D$ ($W_D$ being a constant group scheme) and these $D$-valued points of $W_D$ can be lifted (not canonically) to some $D$-valued points of $N(T')$ since the canonical morphism $N(T') \to W_D$ is smooth and $D$ is a complete discrete valuation ring with algebraically closed residue field. If $\tau \in W$, denote by the same $\tau$ the $D$-valued point of $N(T')$ obtained this way. If the $D$-valued point $\tau$ corresponds to an  element of the subgroup $W_{i(P')}$ of $W$, we  see that $\tau$ fixes the Schubert subscheme $X'_0$ or equivalently $\tau \cdot F' = \lambda F'$ ($\lambda$=unit in $D$) (\cf \S \ref{art18-sec2} and Lemma \ref{art18-lem1}). Let us denote by $p'_\tau$ the element $p'_\tau = \tau \cdot F'$, $\tau \in W$. Since the $D$-module spanned by $F'$ is $T'$ stable, $p'_\tau$ is well-defined upto a unit  (i.e. independent of the above lifting to $D$-valued points in $N(T')$ upto a unit in $D$) and thus upto a unit in $D$, the notation
$$
p'_\tau, \;\; \tau \in W/ W_{i(P)}
$$
is justified.
\end{itemize}

Let us now take up the proof of the assertion in Remark \ref{art18-rem5} above. Let us denote by $X'(\tau w_0)$ the Schubert subscheme in $G'/P'$ associated to $\tau w_0\in W$ and by $A'(\tau)$, $R'(\tau)$ the algebras over $D$ similar to $A(\tau)$, $R(\tau)$ (\cf \S \ref{art18-sec2}). We see that $A'(\tau)$ is the subalgebra of $R'(\tau)$ generated by $\{p_\tau\}$, $\tau \in W / W_{i(P)}$. Suppose that $q \in A'(\tau)_m$ is {\em not standard} (over $D$). By hypothesis, $A'(\tau) \otimes_D K$ is spanned (over $K$) by standard monomials. This implies that 
$$
q = \sum\limits_{(\alpha) \in I^m_\tau} \lambda_{(\alpha)} p_{\alpha_1} \ldots p_{\alpha_m}, \alpha_1 \leqslant \ldots \leqslant \alpha_m
$$\pageoriginale
where $\lambda_{(\alpha)} \in K$, $\lambda_{(\alpha)} \neq 0$ and $(\alpha)$ runs over {\em distinct elements} of $I^m_\tau$. We can write
\begin{equation*}
\lambda_{(\alpha)} = a_{(\alpha)} /p^n, 
\begin{cases}
a_{(\alpha)} \in D$, $n \geqslant 0,\\
p = \text{ generator of maximal ideal of } D \\
\text{ and at least one $a_{(\alpha)}$ is a {\em unit}  in $D$.}
\end{cases}
\end{equation*}
This gives
\begin{equation*}
p^n q = \sum\limits_{(\alpha) \in I^m_\tau} a_{(\alpha)} p_{\alpha_1} p_{\alpha_2} \ldots p_{\alpha_m},
\begin{cases}
\alpha_1 \leqslant \ldots \leqslant \alpha_m\\
a_{(\alpha)} \in D, a_{(\alpha)} \neq 0
\end{cases}
\tag*{(1)}\label{art18-addeq1}
\end{equation*}
where the summation on the right side runs over distinct standard monomials. Now $n \neq 0$, since otherwise $q$ is a sum of standard monomials. Hence $n>0$. Reduce \ref{art18-eq1} $\mod p$ i.e. read \ref{art18-eq1} by taking the images of the elements by the canonical homomorphism $A'(\tau) \to A(\tau) = A'(\tau) \otimes_D k$. Then we get
\begin{equation*}
0 = \sum \bar{a}_{(\alpha)} p_{\alpha_1} \ldots p_{\alpha_m}, \bar{a}_{(\alpha)} \neq 0 \text{ at least for one }(\alpha). \tag*{(2)}\label{art18-addeq2}
\end{equation*}
where $\bar{a}_{(\alpha)}$ denotes the image of $a_{(\alpha)}$ by the canonical homomorphism $D \to k$ and the right side runs over distinct standard monomials. This contradicts the fact that standard monomials are linearly independent over $k$ (\cf Prop. \ref{art18-prop1}). Thus we conclude that $A'(\tau)$ is spanned by standard monomials over $D$. This implies, a fortiori, that $A(\tau)$ is spanned by standard monomials (over $k$) as we have $A'(\tau)\otimes k = A(\tau)$. This proves the assertion in Remark \ref{art18-rem5}.

\section{Minuscule weights and the main theorem}\label{art18-sec5}
\begin{definition}\label{art18-defi5}
A fundamental weight $\varpi$ (or the associated parabolic group $P$) is said to be {\em minuscule} (\cf p. 226, exercise 24, \cite{art18-key4}) if any weight $\theta$ of the irreducible representation $V_1$ with highest weight $\varpi$ in characteristic zero (i.e. if $G'$ is the split group scheme over $\bfZ$ such that $G \otimes_\bfZ k = G$, then $V_1$ is the irreducible representation of $G'\otimes_\bfZ \bfC$ with highest weight $\varpi$) is of the form
$$
\theta = \tau (\varpi), \tau \in W, W = \text{ Weyl group}.
$$
\end{definition}

\begin{prop}[\cf exercise 24, p. 226, \cite{art18-key4}]\label{art18-prop4}
A fundamental\pageoriginale weight $\varpi$ is minuscule if and only if 
$$
\langle \varpi, \alpha\check  ~\rangle = 0, 1 \text{~ or ~} - 1, \forall \alpha \in \Delta: (\alpha\check   \quad coroot \text{ of }  \alpha).
$$
Further, if $\varpi$ is minuscule, $X(\tau\omega_0)$, $\tau \in W/ W_{i(P)}$ a Schubert variety in $G/P$ and $Y$ a Schubert variety which is an irreducible component of $X(\tau w_0) \cap \tau X_0$ (\cf Lemma \ref{art18-lem4}), then the multiplicity of this intersection along $Y$ is 1.
\end{prop}

\begin{proof}
Let $\alpha \in \Delta^+$ (positive root) and $\varpi$ be minuscule. One has to show that $\langle \varpi, \alpha\check ~\rangle =0$ or 1. One knows that $\langle \varpi, \alpha\check ~\rangle$ is an {\em integer } $\geqslant 0$ (\cf Th\'eor\`em 3, Chap. VII-9, \cite{art18-key14}); further by taking the 3-dimensional Lie algebra generated by $X_\alpha$, $Y_\alpha$ and $H_\alpha$ (notations as in \cite{art18-key14}) and using Theorem 1, Chap. IV-3 especially its Corollary 1, (b) (\cf \cite{art18-key14}), we see that $(\varpi-r\alpha)$ is also a weight of $V_1$ where $r$ an integer $0 \leqslant r \leqslant \langle\varpi, \alpha\check 
~\rangle$. Suppose now that $\langle \varpi, \alpha\check ~\rangle > 1$. Take $r$ such that 
$0 < r < \langle \varpi, \alpha \check ~\rangle $ Then $q = \varpi - r\alpha$ is a weight of $V_1$. We claim that 
$$
(q|q) < (\varpi / \varpi) (\text{notation as in  \cite{art18-key4}})
$$
for 
\begin{align*}
(\varpi - r \alpha / \varpi - r \alpha) & = (\varpi/ \varpi) + r^2 (\alpha / \alpha) -2 (\varpi/ \alpha) r\\
& = (\varpi/ \varpi) + r^2 (\alpha/ \alpha) -r(\alpha/\alpha) \langle \varpi, \alpha \check ~ \rangle \\
& \qquad  \left(\text{for }  \langle \varpi, \alpha \check ~ \rangle = \frac{2(\varpi/ \alpha)}{(\alpha, \alpha)}  \right)\\
& = (\varpi/ \varpi) + (\alpha/\alpha) [r^2 - r \langle \varpi, \alpha \check \rangle ].
\end{align*}
Since $0< r< \langle \varpi, \alpha \check ~\rangle $, we deduce that 
$$
r^2 - r \langle \varpi, \alpha \check ~ \rangle < 0
$$
This proves the claim that $(q/q) < (\varpi, \varpi)$. But since $\varpi$ is minuscule, $q = \tau(\varpi)$, $\tau \in W$, which implies, in particular, that $(q/q) = (\varpi, \varpi)$. This leads to a contradiction to the hypothesis that $\langle \varpi, \alpha \check ~\rangle > 1$. Thus we see that $\langle \varpi , \alpha \check ~\rangle  = 0$ or 1.

Suppose now that $\varpi$ is a fundamental weight, such that 
$$
\langle \varpi, \alpha \check ~\rangle  = 0, 1 \text{~ or ~}  -1 \quad \forall \alpha \in \Delta.
$$
Then\pageoriginale we see that for any $\tau \in W$, we have
$$
\langle \tau (\varpi), \alpha \check ~\rangle =0, 1 \text{ or } -1.
$$
Let $v \in V_1$ be the element with highest weight $\varpi$. Then one knows that 
$$
Y^{m_1}_{\beta_1} \ldots Y^{m_k}_{\beta_k} v, m_i \in N, \beta_i \in \Delta^+, (\cf Prop. 2, VII-3, \cite{art18-key14})
$$
generate $V_1$ and one has to show that if $Y^{m_1}_{\beta_1} \ldots Y^{m_k}_{\beta_k} v \neq 0$, then the weight of this element is of the form $\tau(\varpi) $, $\tau \in W$. By a simple induction argument, we see that it suffices to show that if $v' \in V_1$, $v' \neq 0$ and of weight $\tau(\varpi)$, $\tau \in W$ and if $Y_{\beta_i} v' \neq 0$, then $Y_{\beta_i } v'$ is of weight $\tau'(\varpi)$ for some $\tau' \in W$. Now $v'$ is the highest weight vector for a suitable conjugate of the Borel subalgebra and $\beta_i$ can be supposed to be positive with respect to this Borel subalgebra. Thus we can suppose without loss of generality, that $\tau$ = Identity i.e. $v = v'$. We see that $\langle \varpi, \beta_i\check ~ \rangle \neq 0$, for  if $\langle \varpi, \beta_i\check~\rangle =0 $, then by Theorem 1, Chap. IV-3, especially its Cor. 1, (b), \cite{art18-key5}, the 3-dimensional Lie algebra $X_{\beta_i}$, $Y_{\beta_i}$, $H_{\beta_i}$ annihilates $v$ (weight of $v = \lambda = \langle \varpi, \beta_i\check ~\rangle  =0$) and this contradicts the assumption $Y_{\beta_i} v \neq 0$. We see then that $\langle \varpi, \beta_i\check~\rangle  =1$ and then the weight of $Y_{\beta_i} v$ is $\varpi - \beta_i = \varpi - \langle\varpi, \beta_i \check~ \rangle  \beta_i = s_{\beta_i} (\varpi)$. This proves the first assertion of (Prop. \ref{art18-prop4}).

The last assertion of Prop. \ref{art18-prop4} is an immediate consequence of Chevalley's multiplicity formula (\cf \cite{art18-key8}, p. 78, Cor. 1, Prop. 1.2), since by this formula, this required multiplicity is of the form 
$$
\langle \varpi, \alpha \check ~\rangle  (\text{or perhaps ~ } \langle i(\varpi), \alpha \check ~ \rangle ), ~ \alpha \in \Delta^+
$$
and $\varpi$ is minuscule if and only if $i(\varpi)$ is minuscule ($i= $ Weyl involution).

This proves Prop. \ref{art18-prop4}.
\end{proof}

\begin{remark}\label{art18-rem6}
Suppose that the base field is of {\em characteristic zero} and $\varpi$ is minuscule. Then
$$
R(\tau) = A(\tau), ~~ \forall \tau \in W / W_{i(P)}.
$$
In particular, the canonical morphism $X(\tau w_0) \to Y(\tau)$ is an {\em isomorphism } (\cf beginning of \S \ref{art18-sec3} for the definition of $R(\tau)$, $A(\tau)$).
\end{remark}

\begin{proof}
Since\pageoriginale $R_m = H^0(G/P, L^m)$ is known to be irreducible with highest weight $mi(\varpi)$ (Borel-Weil theorem), the canonical $G$-homomorphism
$$
\underbrace{R_1 \otimes \ldots \otimes R_1}_{m \text{ times}} \to R_m
$$
is surjective. This shows that $R_1$ generates $R$. Since $i(\varpi)$ is minuscule, the weights of $R_1$ are of the form $\tau i (\varpi)$, $\tau \in W / W_{i(P)}$. Now $\tau i (\varpi)$ is the ``highest weight'' for a suitable conjugate of the Borel subgroup $B$ of $G$, which shows that the linear subspace of $R_1$ with weight $\tau i (\varpi)$ is of dimension one and hence coincides with the one dimensional subspace spanned by $p_\tau$. This shows that $R = A$. The general case follows from a result of Demazure  (\cf Theorem \ref{art18-thm1}, \cite{art18-key8}, p. 84), namely that the canonical map
$$
H^0(G/P, \mathscr{O}_{G/P} (m)) \to H^0(X(\tau w_0), \mathscr{O}_{X (\tau w_0)} (m))
$$
is surjective.
\end{proof}

\begin{remark}\label{art18-rem7}
We see that if $G$ is type $A_n$, every fundamental weight is minuscule. Looking at the tables in \cite{art18-key4}, we have the following:
\begin{itemize}
\item[(i)] $G$ of type $B_n$, $\varpi$ is minuscule $\Leftrightarrow \varpi = \varpi_n$ i.e. $\varpi$ corresponds to the ``right end root'' (in the Dynkin diagram)

\item[(ii)] $G$ of type $C_n$, $\varpi$ is minuscule $\Leftrightarrow \varpi = \varpi_1$ i.e. $\varpi$ corresponds to the ``left end root''.

\item[(iii)] $G$ of type $D_n$, $\varpi$ is minuscule $\Leftrightarrow \varpi = \varpi_1$ or $\varpi_{n-1}$ or $\varpi_n$ i.e. $\varpi$ corresponds to the ``extreme end roots''.

\item[(iv)] $G$ of type $E_6$, $\varpi$ is minuscule $\Leftrightarrow \varpi = \varpi_1$ or $\varpi_6$ i.e. $\varpi$ corresponds to the ``left or right end root''.

\item[(v)] $G$ of type $E_7$, $\varpi$ is minuscule $\Leftrightarrow \varpi = \varpi_7$ i.e. $\varpi$ corresponds to the ``right end root''.

\item[(vi)] there are no minuscule weights, when $G$ is of type $E_8$, $F_4$ or $G_2$.
\end{itemize}
\end{remark}

\begin{prop}\label{art18-prop5}
Suppose that the base field is of characteristic zero and $\varpi$ is minuscule. Then $R(\tau) = (A(\tau))$ is spanned by standard monomials in $I_\tau$, $\tau \in W / W_{i(P)}$.
\end{prop}

\begin{proof}
We prove\pageoriginale this by induction on $\dim X(\tau w_0)$. When $\dim X(\tau w_0) =0$, the required fact is immediate. Now by Remark \ref{art18-rem4}, it suffices to check that the scheme-theoretic intersection $X(\tau w_0) \cap \tau X_0$ is {\em reduced}. Now by Demazure's results, $X(\tau w_0)$ is Cohen-Macaulay (\cf Cor. 2, Theorem 1, \cite{art18-key8}, p. 84-85). Hence the scheme theoretic intersection $X(\tau w_0) \cap \tau X_0$ is also Cohen-Macaulay. Now by (ii) of Prop. \ref{art18-prop4}, the scheme theoretic intersection $X(\tau w_0) \cap \tau X_0$ is reduced at the generic points of its irreducible components. It follows that $X(\tau w_0) \cap\tau X_0$ is reduced and hence Prop. \ref{art18-prop5} follows.
\end{proof}

\begin{theorem}\label{art18-thm1}
Suppose that the fundamental weigh $\varpi$ is minuscule. Then if $T$ is a right half space in $I$, we have (the characteristic of the ground field $k$ being arbitrary):
\begin{itemize}
\item[(i)] $A(T)$ is spanned by standard monomials in $T$;

\item[(ii)] $A(T) = R(T)$;

\item[(iii)] the line bundle $L$ on $G/P$ is very ample;

\item[(iv)] $H^i (X (T), \mathscr{O}_{X(T)} (m)) =0$, $i>0$, $m \geqslant 0$;

\item[(v)] Consider the map
$$
T \mapsto X(T) (\resp \hat{X}(T) - \text{~ cone over ~} X(T))
$$
from the set of right half spaces in $I$ into the set of closed subschemes of $G/P$ (\resp $\hat{G}/P$). This is a bijective map of the set of right half spaces in $I$ onto the set of closed subschemes of $G/P$ (\resp $\hat{G}/P$), each member of which is a schematic union of $X(\tau)$ (\resp $\hat{X}(\tau)$), $\tau \in I$. Further this map takes set union into scheme theoretic union, set intersection into scheme theoretic intersection and preserves distributivity properties. Scheme theoretic unions and intersections of $X(T)$ (\resp $\hat{X}(T)$) are reduced;

\item[(vi)] $X(\tau w_0)$ (in fact the cone $\hat{X}(\tau w_0)$) is normal (in fact, $\hat{X}(\tau w_0)$ is also Cohen--Macaulay, \cf Remark \ref{art18-rem3} above);

\item[(vii)] $X(\tau w_0) \cap \tau X_0$ is reduced.
\end{itemize}
\end{theorem}

\begin{proof}
By Prop. \ref{art18-prop5}, $A(\tau)$ is spanned by standard monomials in $I_\tau$, $\tau \in W / W_{i(P)}$, when the ground field is of characteristic zero. Hence\pageoriginale by Remark \ref{art18-rem5}, the same holds when the ground field is of arbitrary characteristic. Now by Cor. \ref{art18-coro2}, Prop. \ref{art18-prop2}, it follows that $A(T)$ is spanned by standard monomials in $T$, $T$ being any right half space in $I$. This proves (i).

Now $H^1 (X(T), \mathscr{O}_{X(T)}(m))=0$, when $m$ is sufficiently large. Hence $R(T)_m = H^0 (X(T), \mathscr{O}_{X(T)}(m))$ is obtained by ``reduction $\mod p$'' of the same space in characteristic zero (note that $X(\tau w_0)$, $X(T)$ can be constructed as schemes over $\bfZ$), when $m$ is sufficiently large. In particular, we get
$$
\dim R(T)_m = \chi (T,m) , m \gg 0
$$
since by Prop. \ref{art18-prop5} and Cor. \ref{art18-coro2}, Prop. \ref{art18-prop2}, $R(T)_m$ is spanned by standard monomials in $T$ when the ground field is of characteristic zero. On the other hand 
$$
\dim A(T)_m = \chi (T, m), \forall m
$$
because of (i). Hence, we get 
$$
A(T)_m = R (T)_m, m \gg 0.
$$
Since $A(T)_1$ generates $A(T)$, it follows that the canonical morphism
$$
X(T) = \text{Proj} R(T) \to Y(T) = \text{Proj} A(T)
$$
is an isomorphism; in particular $L|_{X(T)}$ is very ample. Since (i) holds and $X(T) = Y (T)$, we deduce that 
$$
A(T)_m = R(T)_m
$$
by (i), Prop. \ref{art18-prop3}. This proves (ii) and (iii).


The assertions (iv), (v) and (vi) and (vii) are taken from Prop. \ref{art18-prop1}, its Cor. \ref{art18-coro1}, Prop. \ref{art18-prop3} and its Cor. \ref{art18-coro1}.
\end{proof}
\setcounter{coro}{0}
\begin{coro}%%% 1
Let $\varpi$ be minuscule. Suppose that $p_{\beta_1} p_{\beta_2}$ is a quadratic monomial which is not standard, $\beta_i \in I$. Then we have a unique relation
\begin{equation*}
p_{\beta_1} p_{\beta_2} = \sum\limits_{(\alpha ) \in I^2} \lambda_{(\alpha)} p_{\alpha_1} p_{\alpha_2}; \alpha_1 \leqslant \alpha_2 \text{ and } \lambda_{(\alpha)} \in k, \lambda_{(\alpha)} \neq 0 \tag{*}
\end{equation*}
where on the right side ($\alpha$) runs over distinct elements of $I^2$ of the form 
\begin{itemize}
\item[\rm (i)] $\alpha_1 \leqslant \beta_1$, $\alpha_1 \neq \beta_1$; $\alpha_1 \leqslant \beta_2$, $\alpha_1 \neq \beta_2$\pageoriginale 

\item[\rm (ii)] $\alpha_2 \geqslant \beta_1$, $\alpha_2 \neq \beta_2$; $\alpha_2 \geqslant \beta_2$, $\alpha_2 \neq \beta_2$.
\end{itemize}
\end{coro}

\begin{proof}
By Theorem \ref{art18-thm1}, $p_{\beta_1} p_{\beta_2}$ can be expressed uniquely as a sum of standard monomials. We shall now show that if a non-standard monomial $p_{\beta_1} p_{\beta_2}$ is expressed as a sum of standard monomials (even without assuming that $\varpi$ is minuscule), then the properties (i) and (ii) above are satisfied.

Suppose then $\alpha_1 \not\leqslant \beta_1$. Restrict (*) to the Schubert variety $X(\alpha_1 w_0)$. Then the restriction of $p_{\beta_1}$ to $X(\alpha_1 w_0)$ is zero (\cf Lemma \ref{art18-lem3}) and we get
$$
0 = \lambda_{(\alpha)} p_{\alpha_1} p_{\alpha_2} + \sum \ldots 
$$
which contradicts the linear independence of standard monomials (\cf Pr-op. \ref{art18-prop1}.). Thus we see that $\alpha_1 \leqslant \beta_1$. Suppose that $\alpha_1 = \beta_1$; since $\beta_1, \beta_2$ are { \em not } comparable, we deduce that $\beta_2 \not\geqslant \alpha_1$. Again the restriction of $p_{\beta_2}$ to $X(\alpha_1 w_0)$ is zero and restricting (*) to $X(\alpha_1 w_0)$ we get a contradiction. Thus we deduce that $\alpha_1 \leqslant \beta_1$, $\alpha_1 \neq \beta_1$. Similarly, we deduce that $\alpha_1 \leqslant \beta_2$; $\alpha_1 \neq \beta_2$.

To prove (ii), we observe that $\tau_i \cdot p_{\tau_2} = p_{\tau_1 \tau_2}, \tau_i \in W$, (this is well defined only upto a constant, see the discussion preceding Lemma \ref{art18-lem1}). Transforming (*) by left action by the element $w_0$ (or to be precise by a representative of $w_0$ in $N(T)$), we get 
\begin{equation*}
p_{w_0 \beta_1} p_{w_0 \beta_2} = \sum\limits_{(\alpha) \in I^2} \mu_{(\alpha)} p_{w_0 \alpha_2} p_{w_0 \alpha_1}, 
\tag{**}
\end{equation*}
where $(\alpha)$ runs over the same set of elements as in (*) and $\mu_{(\alpha)} \neq 0$. We see easily that
$$
\alpha_1 \leqslant \alpha_2 \Leftrightarrow   w_0 \alpha_1 \geqslant  w_0 \alpha_2 \text{~~ (\cf Definition \ref{art18-defi1}).}
$$
Hence the right side of (**) also runs over distinct standard monomials and the left side of (**) is not standard. Hence applying (i), we get for example
$$
w_0 \alpha_2 \leqslant w_0\beta_1; w_0  \alpha_2 \neq w_0 \beta_1
$$
and this implies $\alpha_2 \geqslant \beta_1$, $\alpha_2 \neq \beta_1$. This proves (ii) and completes the proof of the corollary.
\end{proof}

\begin{remark}\label{art18-rem8}
In the proof of Theorem \ref{art18-thm1}, the work of Demazure \cite{art18-key8}, especially his result that the Schubert varieties are Cohen-Macaulay\pageoriginale in characteristic zero, has been used. A more direct proof would be along the following lines: When $\varpi$ is minuscule and the base field is of characteristic zero, suppose one is able to check (probably using Weyl's or Demazure's character or dimension formula) directly that 
$$
\dim H^0(G/P,\mathscr{O}_{G/P}(2)) = \chi(I, 2).
$$
Then this implies that given a non-standard quadratic monomial $p_{\beta_1} p_{\beta_2}$, this can be expressed as in (*), Cor. \ref{art18-coro1}, Theorem \ref{art18-thm1} as a linear combination of standard monomials (base field is of characteristic zero) and we have seen in the proof of this corollary that the properties (i) and (ii) of Cor. \ref{art18-coro1}, Theorem \ref{art18-thm1}, follow then automatically. Then by the argument as in \cite{art18-key10} or (Prop. 3. 1, p. 153 \cite{art18-key13}), we conclude that the standard monomials span $R$, and by Remark \ref{art18-rem5} the same conclusion holds in arbitrary characteristic. Once we have this, Theorem \ref{art18-thm1}, follows from Prop. \ref{art18-prop2} and Prop. \ref{art18-prop3}.
\end{remark}

\begin{thebibliography}{99}
\bibitem{art18-key1} A. Borel: Linear algebraic groups, W. A. Benjamin, New York, (1969).

\bibitem{art18-key2} A. Borel: Linear representations of semi-simple algebraic groups, Proceedings of Symposia in Pure Mathematics, vol. 29, Algebraic geometry, {\em Amer. Math. Soc., (1975)}.

\bibitem{art18-key3} A. Borel et J. Tits: Groups r\'eductifs, {\em Publ. Math. I.H.E.S.,} vol. 27, (1965), p. 55-150. 

\bibitem{art18-key4} N. Bourbaki: {\em Groupes et alg\`ebres de Lie,} Chapitres 4, 5 et 6, Hermann, Paris, (1968).

\bibitem{art18-key5} C. C. Chevalley: {\em Classification de groupes de Lie alg\'ebriques } S\'eminaire, 1956-58, Secr\'etariat math\'ematique, 11, rue Pierre-Curie, Paris, (1958).

\bibitem{art18-key6} C. C. Chevalley: Sur les d\`ecompositions cellulaires des espaces $G/B$ (manuscrit non publi\'e), (1958).

\bibitem{art18-key7} C. De Concini and C. Procesi: A characteristic free approach to invariant theory, To appear.

\bibitem{art18-key8} M. Demazure: D\'esingularisation des vari\'et\'es de Schubert g\'en\'eralis\'ees, {\em Ann. Sc. \'Ec. Norm. Sup., } t. 7, (1974), pp. 58-88.

\bibitem{art18-key9} W. V. D. Hodge:\pageoriginale Some enumerative results in the theory of forms, {\em Proc. Camb. Phil. Soc.,} Vol. 39 (1943), pp. 22-30.

\bibitem{art18-key10} W. V. D. Hodge and D. Pedoe: {\em Methods of algebraic geometry, Vol. II, Camb. Univ. Press, 1952.}

\bibitem{art18-key11} G. Kempf: Linear systems on homogeneous spaces, {\em Annals of mathematics, 103 (1976), 557-591}.

\bibitem{art18-key12} Lakshmibai, C. Musili and C. S. Seshadri: Cohomology of line bundles on $G/B$, {\em Ann. Sc. Ec. Norm. Sup.,} t. 7, (1974), pp. 89-137.

\bibitem{art18-key13} C. Musili: Postulation formula for Schubert varieties, {\em J. Indian Math. Soc.,} 36 (1972), pp. 143-171.

\bibitem{art18-key14} J. P. Serre: {\em Alg\`ebres de Lie semi-simples complexes,} W. A. Benjamin, New York, 1966. 

\bibitem{art18-key15} SGA-3, {\em Sch\'emas en groups-S\'eminaire de g\'eom. alg. de l'I.E.H.S., Lecture notes in Mathematics, 153, Springer-Verlag.}

\bibitem{art18-key16} H. Weyl: {\em The classical groups,} Princeton, N. J., Princeton Univ. Press, (1946).

\end{thebibliography}


\vfill\eject
~\phantom{a}
\thispagestyle{empty}
