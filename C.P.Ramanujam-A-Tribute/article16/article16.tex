\title{On Euclid Algorithm}\label{art16}
\markright{On Euclid Algorithm}

\author{By~ Masayoshi Nagata}
\markboth{Masayoshi Nagata}{On Euclid Algorithm}

\date{}
\maketitle

\setcounter{page}{209}
\setcounter{pageoriginal}{174}
THROUGHOUT\pageoriginale THIS article, we shall mean by a ring a commutative ring with identity.

Roughly speaking, an integral domain which admits the Euclid algorithm is called a Euclid ring. But there are some distinctions among the ways conditioning  the Euclid algorithm, and we want to discuss the subject. Since the Euclid algorithm can be observed independently of the property that the ring is an integral domain, we dare define a Euclid ring without assuming the integrity. Thus:

\begin{defi*}
  A ring $E$ is called a $\underline{\text{Euclid ring}}$ if there is a mapping $\rho$ of $E-\{0\}$ into a suitable well-ordered set $W$ so that it holds that if $a$, $b \in E$, $a \neq 0$, then there are $q$, $r \in E$ such that 
\begin{equation*}
b = aq +r , \text{~~ and either ~~} \rho r < \rho a \text{~~ or ~~} r = 0. \tag{*}
\end{equation*}

In this case, we say also that $(E, W, \rho)$ is a $\underline{\text{Euclid ring.}}$
\end{defi*}

Of course, main importance is in the case of a usual Euclid ring, for which we add the assumption:
\begin{itemize}
\item[\rm (1)] $E$ is an integral domain.

Now, the distinction of the definitions of a Euclid ring comes from whether or not we assume some of the following two conditions:

\item[\rm (2)] If $b$ is a multiple of $a (a,b \in E, b \neq 0)$, then $\rho b\geqslant \rho a$.

\item[\rm (3)] $W$ is isomorphic to the set $\bfN$ of natural numbers. 
\end{itemize}

Our main results are the following:

\vskip 0.3cm

\noindent
{\bf Theorem 1.2.}
{\em If $(E, W, \rho)$ is a Euclid ring, then there is a mapping $\tau: E - \{0\} \to W$ such that $(E, W, \tau)$ is a Euclid ring satisfying the condition (2).}


Thus, though (2) restricts the choice of the mapping $\rho$, it does not restrict the class of Euclid rings.

\vskip 0.3cm

\noindent
{\bf Theorem 2.3.}
{\em There is a Euclid ring $(E, W, \rho)$ such that (i) $E$ is an integral\pageoriginale domain and (ii) for any mapping $\tau: E - \{0\} \to \bfN$, $(E, \bfN, \tau)$ cannot be a Euclid ring.}
     

Thus, even in the case of integral domains, the condition (3) reduces the class of Euclid rings. As will be seen by Theorem \ref{art16-thm4.2}, a structure theorem of a non-integral Euclid ring, there are easy examples of similar character if we do not assume $E$ to be an integral domain. The existence of such an integral domain was asked by P. Samuel, About Euclidean Rings, {\em J. Algebra}, 19 (1971), who defined an Euclid ring without assuming integrity. Hiblot [{\em C. R. Acad. Sc. Paris,} 281 (22 Sept. 1975), ser. A] claimed an example, but his proof contains some errors.

In order to make this article to be selfcontained, we repeat some of the results in the article of Samuel.

\section{The condition (2).}\label{art16-sec1}
 A local principal ideal ring is either an Artin local ring or an integral domain. This means that if $E$ is a principal ideal ring and if $Q$, $Q'$ are mutually distinct primary components of the zero ideal, then there is no maximal ideal containing both of $Q$, $Q'$. Thus:

\begin{lemma}\label{art16-lem1.1}
A principal ideal ring $E$ is the direct sum of certain principal ideal rings $E_l,\ldots, E_m$ such that each $E_i$ is either an Artin local ring or an integral domain.
\end{lemma}

Now we prove:

\begin{thm}\label{art16-thm1.2}
If $(E, W, \rho)$ is a Euclid ring, then there is a mapping $\tau: E - \{0\} \to W$ such that $(E, W, \tau)$ is a Euclid ring satisfying the condition (2) (with $\tau$ instead of $\rho$).
\end{thm}

\begin{proof}
As is well known, a Euclid ring is a principal ideal ring. Applying Lemma \ref{art16-lem1.1} to $E$, we obtain $E_l, \ldots, E_m$. Let $e_i$ be the identity of $E_i$. Now, let $U$ be the unit group of $E$ and define $\tau$ by: $\tau a =\min \{\rho (a u)| u \in U\}$. If $a$, $b \in E$, $a \neq 0$, then there are $u \in U$ and $q$, $r \in E$ such that $\tau a = \rho (a u)$, $b = auq +r$, $\rho r < \rho (au)$ or $r=0$. Thus $(E, W, \tau)$ is a Euclid ring. Let $a$ be a non-zero element of $E$, and set $S = \{c \in E | a c \neq 0\}$. Choose $d \in S$ so that $\rho(ad) \leqslant \rho (ac)$ for any $c \in S$. Then there are $q$, $r \in E$ such that $a = a dq+r$, either $\rho r < \rho (ad)$ or $r =0$.
Since\pageoriginale $r$ is a multiple of $a$, our choice of $d$ shows that $r=0$. Thus $a(1-dq) =0$. In each $E_i$, $(ae_i) (e_i - de_i qe_i) =0$. Therefore by the face that $E_i$ is either a local ring or an integral domain with identity $e_i$, we see that either $ae_i =0$ or $de_i$, $qe_i$ are units in $E_i$. If $ae_i=0$, then we may change $d$ by adding any element of $E_i$ and therefore there is a unit $u$ such that $ad = au$. By our assumption, $\tau a = \tau (a u)$. 
\end{proof}

Note that we have proved that if $b$ is a non-zero multiple of a and if $\tau a = \tau b$, then $b$ is an associate of $a$ (i.e., an element of $aU$). Therefore if $a$, $b \in E$ and if $b$ is a proper multiple of $a$ (i.e., $b$ is a non-zero multiple of $a$ and is not in $aU$), then $\tau b > \tau a$. So we have

\begin{corollary}\label{art16-coro1.3}
If $(E, W, \rho)$ is a Euclid ring satisfying the condition (2), and if $b$ is a proper multiple of a $(a, b \in E)$, then $\rho b > \rho a$.
\end{corollary}

\section{An example}\label{art16-sec2}
Before stating our example, we prove some preliminary results.

\begin{lemma}\label{art16-lem2.1}
Assume that $(A,M)$ is an Artin local ring, and $B$ is an Artin ring containing $A$, sharing the identity with $A$
 and is a finite $A$-module. Let $A^*$ and $B^*$ be the unit groups of $A$ and $B$, respectively. If $B \neq A$, then $\sharp (B^*/A^*) \geqslant \sharp (A/M)$.\footnote{If $M$ is a set, $\sharp (M)$ denotes the cardinality of $M$.}
\end{lemma}

\begin{proof}
First we consider the case where $B$ is not a local ring. Let $J$ be the radical of $B$ and consider the natural homomorphism $\varphi: B \to B / J$. The groups $B^*$, $A^*$ are mapped surjectively to the unit groups of $\varphi B$, $\varphi A$, respectively, and therefore we may assume that $J =0$. Then $B$ is the direct sum of fields $B_1, \ldots, B_s (s \geqslant 2)$. Since each $B_i$ contains $Ae_i$ ($e_i$ being the identity of $B_i$), we see the assertion easily in this case. Assume next that $(B,N)$ is a local ring. If $B/N \neq A/M$, then by taking an element $e$ of $B$ such that $(e \bmod N) \not\in A / M$ and considering cosets of $l+ue (u \in A^*)$, we see the assertion in this case. If $B/N = A / M$, then $N \neq MB$ by the lemma of Krull-Azumaya. Therefore, taking an element $e$ of $N$ outside of $MB$, we prove the assertion similarly.
\end{proof}

Next we introduce the notion of a canonical structure of a Euclid ring. First we make a well-ordered set $W$ to be canonical. Namely, we take the ordinal number $v$ which represents the order-type of $W$; $W$ is\pageoriginale isomorphic to the set $W'$ of ordinal numbers smaller than $v$. Therefore we identify $W$ with $W'$ at this step. We set $W'' = W \cup \{ v\}$. Note that the least member of $W'$ is 0 (order-type of the empty set). Now let $E$ be an arbitrary ring. For every $\mu \in W''$, we define $R_\mu$ and $S_\mu$ inductively as follows: (i) $R_0$ is the unit group, (ii) if $\mu \in W''$ and if $R_\lambda$ are defined for all $\lambda < \mu$, then $S_\mu = \{0\} \cup (\bigcup\limits_{\lambda < \mu} R_\lambda)$ and $R_\mu = \left\{a \in E - S_\mu | \right.$ for any $b\in E$, there are $q, r \in E$ such that $\left. b = aq+r , r\in S_\mu \right\}$. Now, our definition of a Euclid ring shows the following fact:

\begin{lemma}\label{art16-lem2.2}
There is a mapping $\rho: E - \{0\} \to W$ such that $(E, W, \rho)$ is a Euclid ring if and only if $S_v = E$. In this case, $(E, W, \tau)$ is a Euclid ring with $\tau$ such that $\tau x = \mu$ if and only if $x \in R_\mu$. Furthermore if $x \in E - \{0\} $, then $\tau x \leqslant \rho x$.
\end{lemma}

Of course, $W$ may be too big. The smallest possible one is $W^* = \tau (E - \{0\})$. This $(E, W^*,\tau)$ is called the {\em canonical structure} of the Euclid ring $E$.

Now we come to the construction of a Euclid integral domain by which the condition (3) cannot be satisfied.

Consider a set of algebraically independent elements $X_t$ indexed by the set $\bfR$ of real numbers and we set $A = P_S$ where $P$ is the polynomial ring in these $X_t$ over the ring $\bfZ$ of rational integers and $S = \left\{f \in P| ~~\text{ coefficients of $f$ have no proper common factor } \right\}$. Then every element $a$ of $A$ is expressed in the form $qs$ with $s$ in the multiplicative group generated by $S$ and $q$ a natural number or zero. This $q$ is uniquely determined by $a$, called the {\em absolute value} of $a$ and denoted by $|a|$.

We order prime numbers and denote them by $p_1, p_2, \ldots, (p_1 <p_2 < \ldots)$. For each $n \in\bfN$, we take natural numbers $a_n$ and $e_{in}$ (for $i =1,2,\ldots,n$) and also elements $b_0, b_n, b^*_n$ as follows:
\begin{itemize}
\item[(i)] Every $e_{in}$ is a multiple of $P_i$.

\item[(ii)] $a_n = \prod\limits_i p_i^{e_{in}}$, $b_0 = 1$, $b_n = \prod\limits_{i \leqslant n}a_i$  and $b^*_n = \sum^{n-1}_{i=0} b_i x_i$.
\end{itemize}

Next we consider the polynomial ring $K[T]$ in one variable $T$ over the field $K$ of quotients of $A$. In this ring, we define $T_n$ by $T= T_0 = b^*_n + b_n T_n$\pageoriginale and we set $B_n = A [T_n]$, $B = \cup_n B_n$. Note that $T_n= X_n + a_{n+1} T_{n+1}$.

Let $S^*$ be the multiplicatively closed subset generated by\break $T_0, T_1, \ldots, T_n, \ldots$ and set $E=B_{S^*}$. Now we claim:

\begin{thm}\label{art16-thm2.3}
The integral domain $E$ is a Euclid ring, but, not under the condition (3).
\end{thm}

The proof proceeds stepwise. First consider $f(T) = d_0 + d_1 T + \ldots + d_s T^s \in A [T] (d_i \in A, d_s \neq 0)$. For each $m \in \bfN$, $f$ is expressed as the product of a natural number $c_m$ and a primitive polynomial $f_m$ in $T_m$ over $A$. Obviously $c_m$ divides $c_{m+1}$. We  want to show:

\begin{lemma}\label{art16-lem2.4}
There is a natural number $m$ such that for every natural number $n \geqslant m$, it holds that (i) $c_n = c_m$ and (ii) { the constant term of $f_n$ is a unit in $A$}.
\end{lemma}

\begin{proof}
Multiplying some element of $S$, we may assume that all $d_i$ are in $P$. Then we can take a natural number $m$ such that there is no $X_t$, with $t \geqslant m-1$, appearing in some $d_i$. Then we consider $f = c_{m-1}f_{m-1}$, $f_{m-1} = e_0 + e_1 T_{m-1} + \ldots + e_s T_{m-1}^s$. Then the constant term of $f$ in its expression as a polynomial in $T_m$ is $c_{m-1}f_{m-1} (X_{m-1})$ because $T_{m-1} = X_{m-1} + a_m T_m$. Since $f_{m-1} (T_{m-1})$ is primitive and since $X_{m-1}$ does not appear in the coefficients, we see that $f_{m-1} (X_{m-1}) \in S$. Thus this $m$ is the required natural number.

$f_m$ obtained above enjoys the properties that (i) it is a primitive polynomial in some $T_m$ over $A$ and (ii) for every natural number $n$ not smaller than $m$, the constant term of $f_m$ as a polynomial in $T_n$ over $A$ is 
a unit in $A$. Such an element of $B$ is called a {\em primitive} element of $B$.

Lemma \ref{art16-lem2.4} can be applied to any non-zero element $f$ of $K[T]$, and $f = cf_m$ with a positive rational number $c$ and $a$ primitive element $f_m$. This $c$ is uniquely determined by $f$ and is called the {\em content} of $f$. Note the $f \in B$ if and only if the content is a natural number. Note also that $T_n$ are all primitive.
\end{proof}

\begin{lemma}\label{art16-lem2.5}
$B$ is a principal ideal domain. As ideal is maximal if and only if it is generated either by some $p_i (i \in \bfN)$ or by a primitive element\pageoriginale $f$ in $B$ which is irreducible and of positive degree as an element of $K[T]$. The unit group of $B$ is the group generated by $S$ and therefore every residue class of $B/p_i B$ is represented either by 0 or a unit.  
\end{lemma}

\begin{proof}
The last half is obvious. As for the first half, similar statement holds for every $B_m$ and a prime element $f$ in $B_m$ is not a prime element in $B_{m+1}$ if and only if $f$ is a primitive polynomial in $T_m$ but not in $T_{m+1}$.
\end{proof}

\begin{corollary}\label{art16-coro2.6}
$E$ is a principal ideal domain and the unit group of $E$ is generated by $SS^*$.
\end{corollary}
 
If $f \in E - \{0\}$, then with some $s\in S^*$, it holds that $sf \in B$ and we can apply the factorization given in Lemma \ref{art16-lem2.5} to $sf$ and we obtain $sf = cf_m$ with content  $c$ and a primitive element $f_m$.  $f_m$ may contain some factors which are in $S^*$; taking out all such factors, we have the following factorization:

$f = cpu$; where $c$ is the content, $p$ is primitive and having no factor in $S^*$ and $u$ a unit in $E$.

These factors are uniquely determined by $f$ (as for $p$, $u$, we observe them within unit factors in $A$) and therefore we denote them by $c(f)$, $p(f)$, $u(f)$, respectively, from now on.

Now we consider $W = \bfN \times \bfN$ (with lexicographical order), and we define a mapping $\rho: E  - \{0\} \to W$ by $\rho f = (1 + \deg p (f), c (f))$.

We claim:

\begin{lemma}\label{art16-lem2.7}
$(E, W, \rho)$ is a Euclid ring.
\end{lemma}

\begin{proof}
Let $f$, $g$ be non-zero elements of $E$ and we want to show the existence of $q$, $r \in E$ such that $g = fq+r$ and either $r =0$ or $\rho r < \rho f$. Since unit factors of $f, g$ can be disregarded (cf. the proof of Theorem \ref{art16-thm1.2}), we may assume that $f = c(f) p(f)$, $g = c(g) p (g)$. We use an induction on $\rho g$. If $\rho g < \rho f$, then $q = 0$, $r = g$. So we assume that $\rho g \geqslant \rho f$. Then either (i) $\deg g > \deg f$ or (ii) $\deg g = \deg f$ and $c(g) \geqslant c(f)$. In the case (i), if we express $f$ and $g$ as polynomials in $T_m$ with sufficiently large $m$, then the coefficient of the highest degree term in $f$ divides that of $g$, and we can reduce to the case where $\deg g$ is lower than before. Consider the case (ii). In $B_m$ with\pageoriginale sufficiently large $m$, the constant terms of $f$, $g$ are $c(f) \cdot$unit, $c(g)\cdot$unit. Therefore, by some $q$ which is a unit in $A$, the constant term of $r$ has absolute value $c(g) - c(f)$. If $c(f)\neq c (g)$, then $\deg r \leqslant \deg f$ and $c(r) < c(g)$, and this case is finished by our induction. If $c(f) = c(g)$, then since $T_m$ is a unit in $E$, we have $\deg p(r) < \deg p(f)$. 
\end{proof}

We consider the canonical structure $(E, W',\tau)$ of the Euclid ring $E$. In order to prove Theorem \ref{art16-thm2.3}, it suffices to show:

\begin{proposition}\label{art16-prop2.8}
$W' \simeq W = \bfN \times \bfN$.
\end{proposition}

In order to prove this, we adapt symbols $R_\mu$, $S_\mu$ as in the definition of a canonical structure. So, $R_0$ is the unit group of $E$. We prove the assertion stepwise.
\begin{itemize}
\item[(i)] {\em If $n$ is a natural number, then $R_n = \bigcup m R_0$ where $m$ runs through $M_n = \left\{\pi_{p_i}^{e_i} | \sum e_i = n (e_i \geqslant 0) \right\}$}.
\end{itemize}

\begin{proof}
In proving (i), we use induction arguments on $n$. First note that every prime number is in $R_1$ in view of the last half of Lemma \ref{art16-lem2.5}. Note also that if $f(\in E - \{0\})$ has a proper factor which is not in $S_n$ then $f$ cannot be in $R_n$ in view of Corollary \ref{art16-coro1.3}. By induction, no element of $M_n$ is in $S_n$. If $m \in M_n$, then in $E/ mE$, an element $f$ such that $c(f)$ is coprime to $m$ is represented by a unit. Therefore, in general, an arbitrary element $f$ is congruent to an element of the form $d\cdot\text{unit}$ with $d =$ (the $G.C.M.$ of $m$ and $c(f)$) and we see that $m \in R_n$. Thus $\bigcup_{m \in M_n} m R_0 \subseteq R_n $. Conversely, assume that $f \in R_n$. We may assume that $f = c(f) p (f)$. If $f= c(f) \cdot$unit, then the observation above and Corollary \ref{art16-coro1.3} imply that $c(f) \in M_n$. In order to show that $f = c(f)\cdot$unit, it suffices to show that if $f$ is a prime element of $E$ such that $f = p(f)$, then $f \not\in R_n$ (by virtue of Corollary \ref{art16-coro1.3}). Assume first that $\deg f =1$. Then $f = d_0+d_1 T_m$ with sufficiently large natural number $m(d_i \in A, |d_0|=1)$. Then in $E/fE$, by our choice of $a_n$, we see that every sufficiently large prime number $q$ appears as a factor of a (unit in $E$ modulo $f$) only in a power of $q^q$. Thus $q$ appears as a factor of residue classes of elements of $S_n - \{0\}$ only in the form
 $q^{sq-t}$ with $t$ such that $0 \leqslant t < n$. Therefore $f \not\in R_n$.\pageoriginale If $\deg f >1$, then Lemma \ref{art16-lem2.1} implies that $f \not\in R_n$. 
\end{proof} 

As a consequence, we have:
\begin{itemize}
\item[(ii)] $S_\omega = \bigcup u A$, where $\omega$ denotes the ordinal number corresponding to $\bfN$, and $u$ runs through the multiplicative group generated by $T_0, T_1,\ldots$.

\item[(iii)] For $n \in \bfN$ and $m \in \bfN \cup \{0\}$, it holds that
$$
R_{n\omega + m} = \{f \in E - \{0\} | \deg p (f) = n, c(f) \in R_m\}.
$$
\end{itemize}

\begin{proof}
By induction, $S_{n\omega} =\{0\} \cup (\cup fR_0)$ where $f$ runs through $F_n = \{f \in E |\deg p(f) < n\}$. Therefore, if $f$ is primitive and if $\deg p(f) = n$, then our proof of Lemma \ref{art16-lem2.7} shows that $f \in R_{n \omega}$. Conversely, if $f \in R_{n\omega}$, then $\deg p(f) \geqslant n$; if $\deg p(f) >n$, then our method in proving Lemma \ref{art16-lem2.1} shows that $f\not\in R_{n\omega}$. If $c (f) \neq 1$, then Corollary \ref{art16-coro1.3} shows that $f \not\in R_{n\omega}$. Thus we proved the case where $m=0$. For $m>0$, our proof (i) above is adapted and we complete the proof. 
\end{proof}

Now, this (iii) completes the proof of Proposition \ref{art16-prop2.8}.

\section{An operation on ordinal numbers}\label{art16-sec3}
For each ordinal number $\alpha$, we denote by $W_\alpha$ the set of ordinal numbers $\beta$ such that $\beta<\alpha$. For a given cardinality $\bfb$, we denote by $W_\bfb$ the set of ordinal numbers $\alpha$ such that $\sharp (W_\alpha) < \bfb$. Note that $\sharp (W_\bfb) = \bfb$ and the ordinal number representing the order-type of $W_\bfb$ is the beginning ordinal number for the cardinality $\bfb$ in case $\bfb$ is an infinite cardinality. 

Assuming that $\bfb$ is an infinite cardinality, we define an operation $*$ on $W_\bfb$ by the following properties:

(i)  \quad $\alpha * \beta = \beta * \alpha$, \quad (ii) \quad $0 * 0 = 0$, \quad (iii) \quad $(\alpha + 1) * \beta = (\alpha * \beta) + 1$

(iv) ~if~~ $(\alpha,\beta) \neq (0,0)$, ~then~ $\alpha * \beta =\sup (\{\alpha * \beta' + 1 | \beta'< \beta\} \break  \cup \{ \alpha' * \beta + 1 | \alpha' < \alpha\})$. 

One sees easily that this $*$ is well defined by these requirements, if we allow $\alpha * \beta$ to be in a bigger set of ordinal numbers, though we shall show that $\alpha * \beta$ belongs to $W_\bfb$ in Proposition \ref{art16-prop3.1} below. We shall\pageoriginale call $*$  {\em maximal addition}; $\alpha *\beta$ is called the {\em maximal sum} of $\alpha$ and $\beta$.\footnote{One can prove easily by virtue of Theorem \ref{art16-thm3.4} below that $\alpha *\beta$ is the largest among ordinal numbers which can be obtained in the form $\alpha_1 + \beta_1 + \ldots + \alpha_s + \beta_s$ with $\alpha_i$, $\beta_i$ such that $\alpha = \alpha_1 + \ldots + \alpha_s$, $\beta = \beta_1+ \ldots \beta_s$.}

The first remark is that $\alpha * \beta$ is independent of $W_\bfb$ containing $\alpha, \beta$ because of our definition.

We give further results on computing the value of maximal sums. From now on, Greek letters will denote ordinal numbers belonging to $W_\bfb$.

\begin{proposition}\label{art16-prop3.1}
\begin{itemize}
\item[(i)] $W_\bfb * W_\bfb = W_\bfb$. \quad (ii) $\alpha * 0 = \alpha$.

\item[(iii)] If $m$, $n$ are finite, then $(\alpha + m) (\beta +n) =(\alpha * \beta) + m + n$.

\item[(iv)] If $\alpha < \alpha'$, then $\alpha * \beta < \alpha' * \beta$.
\end{itemize}
\end{proposition}

\begin{proof}
(ii) $\sim$  (iv) are obvious by the definition. By (ii), we see that $W_\bfb \subseteq W_\bfb * W_\bfb$. By our definition, if $\alpha$, $\beta \in W_\bfb$, then $W_{\alpha * \beta} \subseteq (W_\alpha \cup \{\alpha\}) * (W_\beta \cup \{\beta\})$. Therefore $\sharp (W_{\alpha * \beta})\leqslant \sharp (W_\alpha \cup \{\alpha\}) \times \sharp (W_\beta \cup \{\beta\}) < \bfb$. Thus $W_{\alpha * \beta} \subset W_\bfb$ and $\alpha * \beta \in W_\bfb$.
\end{proof}

(iii) above shows that in order to compute the value of $\alpha * \beta$, it is essential to know the value in the case where $\alpha$, $\beta$ are limit ordinal numbers, and let us investigate the case.

For a limit ordinal number $\lambda$, we consider $\Gamma_{\gamma \lambda} = \{\mu \in W_\bfb |\gamma < \mu < \lambda\}$ for every $\gamma < \lambda$ and let $\tau_\lambda$ be min $\{$ordinal number representing the order-type of $\Gamma_{\gamma \lambda} | \gamma < \lambda \}$. We call $\tau_\lambda$ the {\em weight} of $\lambda$, and the symbol $\tau$ will maintain its meaning in this section.

\begin{lemma}\label{art16-lem3.2}
\begin{itemize}
\item[(i)] If $\lambda$ is the beginning ordinal number for an infinite cardinality, then the weight of $\lambda$ is $\lambda$ itself. 

\item[(ii)] If $\lambda$ is a limit ordinal number, then $\alpha + \lambda$ is also a limit ordinal number and $\tau_\lambda = \tau_{\alpha+ \lambda}$.

\item[(iii)] If $\alpha = \tau_\lambda$ for some $\lambda$, then $\tau_\alpha = \alpha$.
\end{itemize}
\end{lemma}

Proof is easy and we omit it.

\begin{thm}\label{art16-thm3.3}
Let $\alpha$ be an infinite ordinal number. Then the set $\Gamma =\{\tau_\lambda |\lambda$ limit ordinal $\leqslant \alpha\}$ has a maximal member, say $\delta$, and $\tau_\delta = \delta$. Furthermore,\pageoriginale $M = \{\beta | \beta$ limit ordinal $\leqslant \alpha$, $\tau_\beta = \delta\}$ is a finite set and $\delta$ is the smallest member in $M$.
\end{thm}

The maximal member of $M$ is called the {\em deepest limit ordinal number} in $\alpha$.

\begin{proof}
Consider $\delta = \sup \Gamma$. If $\tau_\delta < \delta$, then there is a $\gamma$ smaller than $\delta$ such that the order-type of $\Gamma_{\gamma\delta}$ is smaller than $\delta$. Then, with $\Gamma' = \{\tau_\beta \in \Gamma | \tau_\beta > \gamma\}$, we have $\delta = \sup \Gamma = \sup \Gamma' < \delta$, a contradiction. Thus $\delta = \tau_\delta$ and $\delta$ is the smallest member in $M$. If $M$ were infinite, then we have an ascending chain $\delta = \beta_0 < \beta_1 < \ldots < \beta_n < \ldots$ in $M$. Let  $\mu =\sup_{n \in \bfN} \beta_n$. For any $\epsilon < \mu$, $\Gamma_{\epsilon \mu}$ contains infinitely many $\beta_{i}$ which are in $M$; for $\epsilon' < \delta$, $\Gamma_{\epsilon'\mu}$ does not contain any member of $M$. Thus $\tau_\delta < \tau_\mu$, a contradiction.
\end{proof}

Maximal sums can be computed making use of the following:

\begin{thm}\label{art16-thm3.4}
For infinite ordinal numbers $\alpha, \beta$, let $\delta_\alpha, \delta_\beta$ be the deepest limit ordinal numbers in $\alpha$, $\beta$, respectively.
\end{thm}

Assume that $\tau_{\delta_\alpha} \geqslant \tau_{\delta_\beta}$. Write $\alpha = \delta_{\alpha} + \alpha'$, $\beta = \delta_\rho + \beta'$. Then
\begin{itemize}
\item[(i)] $\alpha * \beta = \delta_\alpha + (\beta * \alpha')$,

\item[(ii)] If $\tau_{\delta_\alpha} = \tau_{\delta_\beta}$, then $\alpha * \beta = (\delta_\alpha + \delta_\beta) +(\alpha'*\beta')$.
\end{itemize}

Thus the computation can be reduced to the case where $\tau_{\delta_\alpha}, \tau_{\delta_\beta}$ are smaller.

\begin{proof}
(ii) follows easily from (i), and it suffices to prove (i). We use an induction on $(\alpha, \beta)$; namely, we assume the validity of (i) for $(\alpha'', \beta'')$ such that $\alpha'' \leqslant \alpha$, $\beta'' \leqslant \beta$, $(\alpha'' ,\beta'') \neq (\alpha, \beta)$. If $\alpha > \delta_\alpha$, $\beta> \delta_\beta$, then (i) is shown easily by induction hypothesis. The most important case is the case where $\alpha = \tau_{\delta_\alpha}$, $\beta = \tau_{\delta_\beta}$. By induction, if $\beta'' < \beta$, then $\alpha * \beta'' = \alpha + \beta''$. If $\tau_{\delta_\alpha} = \tau_{\delta_\beta}$, then by the symmetry, we have $\alpha + \beta = \alpha * \beta$. If $\tau_{\delta_\alpha } > \tau_{\delta_\beta}$ and if $\alpha'' < \alpha$, then $\alpha'' * \beta < \tau_{\delta_\alpha}$ by induction. Therefore $\alpha * \beta = \alpha + \beta$ in this case. The remaining case can be proved similarly. 
\end{proof}

\begin{corollary}\label{art16-coro3.5}
Assume that $\alpha, \beta$ are limit ordinal numbers and that $\tau_\alpha \geqslant \tau_\beta$. Then $\alpha * \beta = \sup \{\alpha * \beta' | \beta' < \beta\}$ and $\tau_{\alpha * \beta} = \tau_\beta$.
\end{corollary}

\begin{thm}\label{art16-thm3.6}
The maximal\pageoriginale addition satisfies the associativity. Namely, $(\alpha * \beta) * \gamma = \alpha * (\beta * \gamma)$.
\end{thm}

\begin{proof}
We use induction on $(\alpha, \beta, \gamma)$. If some of $\alpha, \beta, \gamma$ is not a limit ordinal number, the assertion is obvious by induction and by (iii) in the definition. If all of $\alpha, \beta,\gamma$ are limit ordinal numbers, then we prove the assertion easily by induction and by Corollary \ref{art16-coro3.5} above. 
\end{proof}

\section{Structure of generalized Euclid rings}\label{art16-sec4}

We maintain the notation of \S \ref{art16-sec3} except for $\rho, \rho'$, $\rho_i, \varphi$ (which a\`re mappings) and $W_i (i = 1, \ldots, s)$.

In this section, we want to show the following three theorems.

\begin{thm}\label{art16-thm4.1}
Assume that a ring $E$ is the direct sum of rings $E_1, \ldots, E_s$. Then $E$ is a Euclid ring if and only if every $E_i$ is a Euclid ring. Consequently, a Euclid ring is the direct sum of Euclid rings $E_i, \ldots, E_s$ such that each $E_i$ is either an integral domain or an Artin local ring.
\end{thm}

\begin{thm}\label{art16-thm4.2}
Assume that $(E_i, W_i, \rho_i) (i, = 1, \ldots, s; W_i \subset W_{\bfb}$  with  a sufficiently large cardinality $\bfb$) are Euclid rings and let $\lambda_i$ be the ordinal number representing the ordertype of $W_i$. Extend $\rho_i$ so that $\rho_i 0 = \lambda_i$ and define $\rho: E = E_1 \oplus \ldots \oplus E_s \to W_{\mu +1}$ with $\mu =\lambda_1 * \ldots * \lambda_s$ by:
$$
\rho (a_1 + \ldots + a_s) = (\rho_1 a_1) * \ldots * (\rho_s a_s) \quad (a_i \in E_i).
$$
Then $(E, W_{\mu}, \rho)$ is a Euclid ring. 

If every $(E_i, W_i, \rho_i)$ is the canonical structure of $E_i$, then $(E, W_{\mu}, \rho)$ is the canonical structure of $E$.
\end{thm}

\begin{thm}\label{art16-thm4.3}
If $E$ is a Euclid ring, then (i) a ring which is a homomorphic image of $E$ is a Euclid ring and (ii) any ring of quotients of $E$ is a Euclid ring.
\end{thm}

As for the proof of Theorem \ref{art16-thm4.1}, the if part follows from Theorem \ref{art16-thm4.2}, its {\em only if } part follows from Theorem \ref{art16-thm4.3} (i) and its last part follows from Lemma \ref{art16-lem1.1}.

Theorem \ref{art16-thm4.2}\pageoriginale    follows from the definition of maximal addition, noting that the condition (*) in the definition of a Euclid ring is equivalent to the following:

(*) If $a$, $b\in E$, $a \neq 0$, then there are $q,r \in E$ such that $b = aq +r$, and $\rho r< \rho a $ or $r =a$.


Theorem \ref{art16-thm4.3}(i) follows from:

\begin{proposition}\label{art16-prop4.4}
Let $(E, W,\rho)$ be a Euclid ring. If $\varphi : E \to E'$ is a surjective ring homomorphism, then $(E', W, \rho')$ is a Euclid ring with $\rho'$ defined by:
$$
\rho'x = \min \{~\rho y ~|~ \varphi y = x\}.
$$
\end{proposition}

\begin{proof}
Assume that $a'$, $b' \in E'$, $a' \neq 0$. Take $a,  b \in E$ so that $\varphi a = a'$, $\rho a = \rho' a'$, $\varphi b = b'$. Then there are $q, r \in E$ such that $b = a q + r$, either $\rho r <\rho a$ or $r =0$. Then $b' = a' (\varphi q) +\varphi r$, either $\rho'(\varphi r) \leqslant \rho r < \rho a = \rho' a'$ or $\varphi r = 0$. 
\end{proof}

Theorem \ref{art16-thm4.3}(ii) follows from:

\begin{proposition}\label{art16-prop4.5}
Let $(E, W , \rho)$ be a Euclid ring and let $S$ be a multiplicatively closed subset of $E$ not containing 0. Then $(E_s, W, \rho')$ is a Euclid ring with $\rho'$ defined by:
$$
\rho' x = \min \{\rho y | y \in E, \text{ there are } s, s' \in S, ys /s' = x\}.
$$
\end{proposition}

This can be proved quite similarly to the proof of Theorem \ref{art16-thm1.2}. 

In closing this article, the writer likes to remark that:

Although, in order to state the condition (*) in the definition of a Euclid ring, it is natural to extend $\rho$ so that $\rho 0 < \rho a$ for any $a \neq 0$, in view of Theorem \ref{art16-thm4.2}, Proposition \ref{art16-prop4.4} etc., the writer feels it to be more natural to extend $\rho$ so that $\rho 0 > \rho a$ for any $a \neq 0$.

\vfill\eject
~\phantom{a}
\thispagestyle{empty}
