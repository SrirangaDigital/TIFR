\chapter{}\label{lec15}

\section*{Crystal Binding and elastic constants}

Structure $\to$ a game of Hard balls and/or mathematical puzzles.

$\Rightarrow$ Although the structure of crystals are described packing of hard balls. The real seenario is different. Otherwise one would as $k$ questions
\begin{itemize}
\item[(i)] Why all structures are not close packed?

\item[(ii)] Why $AB \ AB\ AB$ is stable giving $HCP$ and $ABC \ ABC\ldots$ FCC structures.

There could have been a random mixture!

\item[(iii)] What is that holds all the atoms together?

\item[(iv)] How does a crystal knows about axis system? 
\end{itemize}

There must be some attractive (may be electrostatic in nature) interactions that keeps the atoms together.

Gravitational force, Magnetic force affect the atoms significantly weekly to have effect of binding.

--- Bonds: exchange coupling, van der waals force, covalent bonding etc are under consideration.

\noindent
{\bf Cohesive energy:} Minimum energy need to be applied to the crystal to separate its' components into neutral free atoms at rest, at infinite separation.

\noindent
{\bf Lattice energy:} This term is used for ionic crystals - The minimum energy required to separate the components of a crystal to free atoms.
\begin{itemize}
\item[(a)] Inert gases are weakly bound: cohesive energy is a few percent of $c$, $S_{i}$, $Ge\ldots$ etc.

\item[(b)] Alkali metals have intermediate values of cohesive energy.

\item[(c)] Transition metals are strongly bound.
\end{itemize}

Melting point, Buck modulii depend/related to cohesive energy.

\section*{Inert gas crystals}
\begin{itemize}
\item Simplest crystal with electron distribution close to that of free atoms.

\item Transparent insulators, weakly bound, low melting point.

\item Atoms have very high ionization energy $\to$ electronic configuration is stable.

\item Outermost cells are completely filled, electron distribution is spherically symmetric.

\item Form in close packed structure (FCC) except $He^{3}$ and $He^{4}$
\begin{itemize}
\item[(i)] $He^{3}$ and $He^{4}$ do not solidify at zero pressure even at absolute zero. Zero-point motion is a quantum effect that plays dominant role for $He^{3}$ and $He^{4}$.

\item[(ii)] Average fluctuation at $OK$ is 30\% to 40\% of bond length. Excluding zero-point energy, molar volume = 90 cc/mole for solid $He$. Experimental value 27.5 cc/mole for liquid $He^{4}$ and 36.8 cc/mole for liquid $He^{3}$.
\end{itemize}
What holds the atoms together:
\end{itemize}

\section*{Van der Waals-London Interaction}

If the charge distribution is rigit, interaction would be zero $\to$ spherical distribution of electrons makes total charge zero outside neutral atom (electrons + nucleus). So cohesive energy $\hookrightarrow 0$. But in reality, atoms can induce dipole moment between them.
\begin{center}
{\bf Figure}
\end{center}

For similar atoms one can assume $q_{1}=a_{2}=q$.

$\therefore$ The Hamiltonian of the unperturbed system will be
\begin{equation*}
H_{0}=\dfrac{P^{2}_{1}}{2m}+\frac{1}{2}Cx^{2}_{1}+\dfrac{P^{2}_{2}}{2m}+\frac{1}{2}Cx^{2}_{2}\tag{1}\label{lec15-eq1}
\end{equation*}
\begin{quote}
$C=mw^{2}_{0} =$ force constant.

$W_{0} =$ strongest optical absorption if uncoupled atoms.
\end{quote}
Switch on interaction between them
\begin{equation*}
H_{1}=\dfrac{q^{2}}{R}+\frac{q^{2}}{R+x-x_{2}}-\frac{q^{2}}{R+x_{1}}-\frac{q^{2}}{R-x_{2}}\tag{2}\label{lec15-eq2}
\end{equation*}
For
\begin{equation*}
|x_{1}|,|x_{2}|<< R \fbox{$H_{1}=\dfrac{2q^{2}x_{1}x_{2}}{R^{3}}$}\tag{3}\label{lec15-eq3}
\end{equation*}
Consider normal mode transformation as
\begin{equation*}
x_{s}=\dfrac{1}{\sqrt{2}}(x_{1}+x_{2})\quad x_{a}=\dfrac{1}{\sqrt{2}}(x_{1}-x_{2})\tag{4}\label{lec15-eq4}
\end{equation*}
\begin{align*}
\therefore\quad &\fbox{$x_{1}=\dfrac{x_{s}+x_{a}}{\sqrt{2}}$}\quad \fbox{$x_{2}=\dfrac{x_{s}-x_{a}}{\sqrt{2}}$}\\
\therefore\quad &p_{1}=\dfrac{p_{s}+p_{a}}{\sqrt{2}}\quad p_{2}=\dfrac{p_{S}-p_{a}}{\sqrt{2}}
\end{align*}
$\therefore$ The total Hamiltonian will be $H=H_{0}+H_{1}$
$$
\therefore\quad H=\left[\dfrac{p^{2}_{s}}{2m}+\frac{1}{2}\left(C-\frac{2q^{2}}{R^{3}}\right)x^{2}_{s}\right]+\left[\frac{pa^{2}}{2m}+\dfrac{1}{2}\left(C+\frac{2q^{2}}{R^{3}}\right)x^{2}_{a}\right]
$$
$\therefore$ Frequencies of the complied oscillators are
\begin{align*}
w &= \left[\left(c\pm \dfrac{2q^{2}}{R^{3}}\right)/m\right]^{\frac{1}{1}}=w_{0}\left[1\pm \frac{1}{2}\left(\frac{2q^{2}}{CR^{3}}\right)-\frac{1}{8}\left(\frac{2q^{2}}{CR^{3}}\right)^{2}+\cdots\right]\\
w_{0} &= \sqrt{\frac{c}{m}}
\end{align*}
$\therefore$ The zero point energy $=\dfrac{1}{2}\hbar(w_{s}+w_{a})$ which is smaller than $2\cdot \frac{1}{2}\hbar w_{0}$ by
$$
\Delta U=\frac{1}{2}\hbar (\Delta w_{s}+\Delta w_{a})=-\dfrac{\hbar w_{0}}{8}\left(\frac{2q^{2}}{CR^{3}}\right)^{2}\simeq -\frac{A}{R^{6}}
$$
This is called the {\em vander waal's interaction} also known as {\em London interaction} or {\em induced dipole - dipole interaction}.
\begin{itemize}
\item[(i)] This is purely a quantum effect $\Delta U=0$ for $\hbar=0$.

\item[(ii)] Does not depend on overlap of charge distribution of the atoms.

\item[(iii)] Does not depend on details of charge distribution.
\end{itemize}
For identical atoms $A\sim \hbar w_{0}\alpha^{2}$\quad $\alpha=$ Polarizability.

\section*{Repulsive Interaction}

As the two atoms are brought together, charge distribution gradually overlap, thereby changing electrostatic energy of the system.

When the separation is sufficiently small, repulsion starts to appear due to {\em Pauli Exclusion Principle} : two electrons cannot have all quantum numbers same.
\begin{center}
{\bf Figure}
\end{center}

Empirical formula for repulive potential $\sim \dfrac{B}{r^{12}}$

$\therefore$ Total energy
$$
\fbox{$U(r)=\dfrac{B}{r^{12}}-\dfrac{A}{r^{6}}$}
$$
$A$, $B$ are constants can be determined by measurements.

This is called Lennard - Jones potential.
$$
\text{Force } = -\dfrac{dU}{dr}
$$

Sometimes people use repulsive interaction as $\sim \lambda e^{\frac{-r}{\rho}}$
$$
\rho \to \text{ range of interaction.}
$$

\section*{Equillibrium Lattice Constants}

Lets consider $N$ atoms in the crystal.
$$
\therefore\quad U_{\text{tot}}\frac{1}{2}N(4t)\left[{\sum\limits_{j}}'\left(\dfrac{\sigma}{a_{ij}r}\right)^{12}-{\sum\limits_{j}}'\left(\dfrac{\sigma}{a_{ij}r}\right)^{6}\right]
$$
`$\dfrac{1}{2}$' to avoid double counting.

$a_{ij}r$ is the distance between atom `$i$' and atom `$j$'

$r=$ nearest neighbour distance.

In FCC structure ${\sum\limits_{j}}'(a_{ij})^{-12}=12.13188$\quad ${\sum\limits_{j}}'(a_{ij})^{-6}=14.45392$

In FCC structure, no. of near neighbor $=12$ and the first term is close to 12.

In HCP structure ${\sum\limits_{j}}'(a_{ij})^{-12}=12.13229$\quad ${\sum\limits_{j}}'(a_{ij})^{-6}=14.45489$
%page 12

