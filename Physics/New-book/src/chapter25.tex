\chapter[Lecture 25]{}\label{lec25}

\begin{center}
{\bf Figure}
\end{center}

$\mu=-\dfrac{\partial U}{\partial B}$ Oscillates as a function of $\dfrac{1}{B}$ called $dH_{v}A$ oscillation.

The interval of oscillation $\Delta\left(\dfrac{1}{B}\right)=\dfrac{2\pi e}{\hbar \subset S}$

$S\to$ Extremal area of Fermi surface normal to $\overrightarrow{B}$.
\begin{center}
{\bf Figure}
\end{center}

\section*{Direct method - ARPES}

Discovery of electrons J.J. Thomson - 1897.

Drude (1900) put forward his theory of metallic conduction.

Solid consists of heavy (+ve) charge (immobile) and mobile electrons that makes them neutral solid.

Some fraction of electrons of $z$ electrons ($z=$ atomic no.) are strongly attached to (+ve) charge $\to$ don't move much that rest moves all over the solid and can be treated like classical electron gas, with velocity distance following Maxwell- Boltzmann Distance.
$$
f_{2}=n\left(\dfrac{m}{2\pi k_{B^{T}}}\right)^{\frac{3}{2}}e^{-m\nu^{2}/2k_{B^{T}}}
$$

This worked quite well $\to$ Wiedemann - Franz law.

Specific heat $\dfrac{3}{2}k_{B}$ per electron $\to$ this was not observed.

Later after Quantum theory came, Pauli's exclussion principle needs to be applied and Fermi-Dirac Distribution $\to$ {\em Sommerfeld's model.}

\section*{Free-electron approximation}

\subsection*{Drude's model}

\subsection*{Heat Capacity}
$$
e_{F}\sim k_{B}T_{F}\quad T_{F}\sim 10^{4}K
$$
$\therefore \ \dfrac{T}{T_{F}}$ is very small close to room temperature.

$\therefore$ The fraction of electrons taking part in heat capacity is $\sim \dfrac{T}{T_{F}}$
\begin{gather*}
\therefore \ U_{el}=\left(N\dfrac{T}{T_{F}}\right)K_{B}T\quad N= \text{ total no. of electrons.}\\
\therefore \ C_{el}=\dfrac{\partial U_{el}}{\partial T}\simeq N k_{B}T/T_{F}=\gamma T\\
\gamma \text{ is a const } \to \text{ measure of effective mass.}\\
\therefore \ \fbox{$C=\gamma T+AT^{3}$}\quad T^{3} \text{ is phonon part.}
\end{gather*}
at plot of $C/T$ with $T^{2}$ given $\gamma$ and $A$, and we can measure effective mass that way, $\gamma\to$ measure of effective mass.

A factor $\dfrac{\pi^{2}}{2}$ appears if calculated using Formi-Dirac distribution.

\section*{Force on an electron}
$$
F=-e(\overrightarrow{E}+\dfrac{1}{e}\overrightarrow{\nu}\times \overrightarrow{B})
$$
In an electric field
$$
\delta k=-\dfrac{eEt}{\hbar}\quad \nu=-\dfrac{eE\tau}{m}\Leftarrow m\nu=\hbar k
$$

Due to scattering of electrons, $t\sim \tau\to$ relaxation [The electric field acts till the electron gets scattered] time.
\begin{gather*}
\therefore\quad J=nq\overrightarrow{\nu}=ne^{2}\tau E/m\\
\therefore\quad \sigma =\dfrac{j}{E}=\dfrac{ne^{2}\tau}{m}\quad \rho=\dfrac{1}{r}=\text{ resistivity.}
\end{gather*}

Scattering can be due to phonons $(\tau_{i})$ or impurities $(\tau_{i})$
$$
\dfrac{1}{\widetilde{L}}=\dfrac{1}{\widetilde{L}_{L}}+\dfrac{1}{\widetilde{L}_{i}}\quad \text{Marthiessen's rule.}
$$
$\rho(T)/\rho(0)=$ residual resistivity ratio.

Impurity scattering usually independent of temperature $T$.

Phonon contribution is dependent on no. of phonons at a particular temperature $T$ usually \fbox{$\rho_{L}\alpha T$}.

\section*{Umklapp Scattering}
\begin{center}
{\bf Figure}
\end{center}

Scattering $\to$
$$
k'_{1}=k_{1}+q_{1}
$$
If this scattering involves a reciprocal lattice vector, it is called Umklapp scattering. e.g.,
$$
k'_{2}=k_{2}+q_{2}\pm G
$$
\begin{itemize}
\item a scattering of electron with wave vector $k_{2}$, results to a final wave vector $k'_{2}$ which involves lattice translation $G=AA'$.

\item The angle of scattering is large $\to \sim \pi$ strong scatterers.

\item If the Fermi surface does not intersect zone boundary, one needs minimum value of $q(=q_{0})$ to have Umklapp scattering.

\item At low temperature, Umklapp scattering goes as $e^{-\theta_{U}/T}$

$\theta_{U}=$ Characteristic temperature

For potassium $\theta_{U}=23K$\quad $\theta_{D}=91K$

$\therefore$ at low temperature, Umklapp scattering is almost {\em zero}.
\end{itemize}
$\therefore$ Resistivity is dominated by small angle scattering.

\section{Motion in Magnetic field}

Force on an electron $=F=\dfrac{dp}{dt}=\dfrac{\hbar dk}{dt}\quad P=\hbar k$

However, due to scattering, the fermi sphere will experience a friction at a rate $\dfrac{1}{\tau}$
$$
\therefore \ F=\hbar \left(\dfrac{d}{dt}+\dfrac{1}{\tau}\right)\delta k
$$
change in Momentum + Friction.
$$
\therefore\quad \hbar \dfrac{dk}{dt}+\dfrac{1}{\tau}\hbar \delta k=-e\left[\overrightarrow{E}+\dfrac{1}{C}\overrightarrow{\nu}\times \overrightarrow{B}\right] \text{ (Lorenz force)}
$$
if $\overrightarrow{B}=B\widehat{z}$.
\begin{align*}
m\left(\dfrac{d\nu_{x}}{dt}\right)+\dfrac{v_{n}}{\tau} &= -e\left(E_{x}+\dfrac{B}{C}\nu_{y}\right)\\[3pt]
m\dfrac{d\nu_{y}}{dt}+\dfrac{\nu_{y}}{\tau} &= -e\left(E_{y}-\dfrac{B}{C}\nu_{x}\right)\\
m\dfrac{d\nu_{z}}{dt}+\dfrac{\nu_{z}}{\tau} &= -eE_{z}
\end{align*}
In the steady state in a static electric field, derivatives are zero.
\begin{gather*}
\therefore\quad 
\fbox{
$
\begin{array}{l}
\nu_{x}=-\dfrac{e\tau}{m}E_{x}-w_{C}\tau \nu_{y}\\
\nu_{y}=-\dfrac{e\tau}{m}E_{y}+w_{C}\tau \nu_{x}\\
\nu_{z}=-\dfrac{e\tau}{m}E_{z}
\end{array}
$
}\\
w_{C}=\dfrac{eB}{m_{C}}=\text{ cyclotron frequency}
\end{gather*}
$w_{C}\tau$ is (cyclotron frequency multiplied by time) and an important number.

$w_{C}\tau$ is unity of larger, the effect of magnetic field on electron orbits is significant.

\section*{Hall effect}

Electric field developed across two faces of a conductor in the direction $\overrightarrow{j}\times \overrightarrow{B}$

If $\overrightarrow{J}=J\widehat{x}$; $\overrightarrow{B}=B\widehat{z}$
\begin{center}
{\bf Figure}
\end{center}

Then $\delta \nu_{y}=0$ as current is not flowing out in $y$ direction
$$
\therefore \ E_{y}=\dfrac{w_{C}m}{e}\nu_{x}=-w_{C}\tau E_{x}=-\dfrac{eB\tau}{m_{C}}\cdot E_{x}.
$$

The quantity $\dfrac{E_{y}}{J_{x}B}$ is called Hall co-efficient $R_{H}$.
\begin{gather*}
\therefore \ R_{H} = \left. -\dfrac{eB\tau E_{\lambda}/m_{C}}{ne^{2}\tau E_{x}B/m}=-\dfrac{1}{neC}\right| J=\dfrac{ne^{2}\tau E_{x}}{m}\\[7pt]
\therefore \ \fbox{$R_{H}=-\dfrac{1}{neC}$}\quad\text{or}\quad \fbox{$R_{H}=-\dfrac{1}{ne}$}\quad \text{SI unit.}
\end{gather*}
(-ve) for free electrons. $e$ is magnitude of electrons charge.
\begin{itemize}
\item[(i)] Get carrier concentration

\item[(ii)] Type of carriers.
\end{itemize}
Assumption, all relaxation times are equal.

If both electrons and holes conduct, the scenario is complex.

\section*{Thermal Conductivity of Metals}
$$
J_{Q}=-K\dfrac{dT}{dx}
$$
\begin{align*}
K &= \text{ thermal conductivity}\\
 &= \frac{1}{3}C\nu l
\end{align*}
\begin{quote}
$C$ = heat capacity

$\nu$ = Drift velocity (Avg. velocity)

$l$ = mean free path.
\end{quote}
\begin{align*}
C_{el} &= \frac{\pi^{2}}{2}Nk_{B}T/T_{F}\\
\epsilon_{F} &= k_{B}T_{F}=\frac{1}{2}m\nu^{2}_{F}
\end{align*}
\begin{align*}
\therefore\quad K_{el}=\frac{1}{3}\cdot \frac{\pi^{2}}{2}k_{B}T\cdot \dfrac{k_{B}}{\frac{1}{2}m\nu^{2}_{F}}\cdot \nu_{F}\cdot l &= \frac{\pi^{2}nk^{2}_{B}T}{3m}\tau\\
l &= \nu_{F}\tau
\end{align*}
$\therefore$ The ratio of $K_{el}$ and $\sigma$
$$
\fbox{$\dfrac{K}{\sigma}=\dfrac{\pi^{2}k^{2}_{B}Tn\tau/3m}{ne^{2}\tau/m}=\dfrac{\pi^{2}}{3}\left(\dfrac{k_{B}}{e}\right)^{2}T=LT$}
$$
Wiedemann - Franz law.
\begin{align*}
L = \text{ Lorentz number } =\dfrac{K}{\sigma T} &= \frac{\pi^{2}}{3}\left(\dfrac{k_{B}}{3}\right)^{2}\\
&= 2.72\times 10^{-13}(\text{erg}/\text{esu}-k^{2})\\
&= 2.45\times 10^{-8}\text{ watt-}\sigma hm/k^{2}
\end{align*}
This is remarkable that the ratio does not contain $n$ or $m$ of electrons !!

At high and low temperatures it may be okay.

But at intermediate temperature $\left(\dfrac{K}{\sigma T}\right)$ is temperature dependent !!

\section*{Density of states}
$$
D(E)=\dfrac{dN}{dE}\quad E=\dfrac{\hbar^{2}k^{2}}{2m}
$$
\begin{center}
{\bf Figure}
\end{center}

\section*{One dimension}

Fermi volume $=2k_{F}$
\begin{align*}
\therefore\quad N &= \frac{L}{2\pi}\cdot 2k_{F}=\frac{L}{2\pi}\cdot x\cdot \sqrt{\frac{2m}{\hbar^{2}}}\sqrt{E_{F}}=\sqrt{\dfrac{2m}{\hbar^{2}\pi^{2}}}\cdot \sqrt{E_{F}}\\
\therefore\quad D(E) &= \dfrac{dN}{dE}=\text{ const. } \sum\limits_{i}(E-\epsilon_{i})^{\frac{1}{2}}
\end{align*}
\begin{center}
{\bf Figure}
\end{center}

\section*{Two Dimension}

Formi volume $=\pi k^{2}_{F}=\dfrac{2m E_{F}}{\hbar^{2}}$
$$
\therefore\quad D(E)=\dfrac{dN}{dE}\simeq \text{ Const.}
$$
\begin{center}
{\bf Figure}
\end{center}

\section*{Three dimension}

Fermi volume $=\dfrac{4}{3}\pi k^{3}_F{}=\dfrac{4\pi}{3}\left(\sqrt{\dfrac{2mE_{F}}{\hbar^{2}}}\right)^{3}$
$$
\therefore\quad D(E)\sim \sqrt{E}
$$
\begin{center}
{\bf Figure}
\end{center}

\noindent
{\bf Zero Dimension :- } Atomic levels like.

\noindent
{\bf Absorption :-}

\section{Drude's model}

no. of electron per unit volume
$$
n=\dfrac{N}{V}\sim 6.02\times 10^{23}\dfrac{z\rho m}{A}
$$
\begin{quote}
$Z=$ atomic no.

$\rho_{m}=$ mass density.

$A=$ Atomic mass.
\end{quote}

Conduction electron density is a fraction of $n$ usually $n_{e}\sim 10^{22}/em^{3}$.

If $r_{S}$ is the radius of the volume per conduction electron
\begin{align*}
\therefore\quad \dfrac{V}{N} &=\dfrac{1}{n_{e}}=\frac{4}{3}\pi r^{3}_{S}\\
\therefore\quad r_{S} &= \left(\dfrac{3}{4\pi n_{e}}\right)^{\frac{1}{3}}
\end{align*}
$\dfrac{r_{S}}{a_{0}}\sim 2-3$ in most cases.

$a_{0}=$ Bohr radius.

$\simeq \dfrac{\hbar^{2}}{me^{2}}=0.529\times 10^{-8}$ cm.

In alkali metals $\dfrac{r_{S}}{a_{0}}\Rightarrow 3-6$

In some cases it can go upto $10$.

These are typically 1000 times greater than classical gas at NTP.

There can be strong electron-electron and electron ion interactions $\to$ still, in this model, we assume.
\begin{itemize}
\item[(i)] No. interaction between two collisions
\begin{itemize}
\item[(a)] Neglect of electron - electron interaction

$\to$ Independent electron approximation.

\item[(b)] Neglect of electron - ion interaction

$\to$ free electron approximation.
\end{itemize}
(a) often works well in real system but (b) is no good.

\item[(ii)] Collision of electrons with impenetrable ion core. Drude did not consider electron, electron scattering.

$\to$ One can simply talk about scattering effect without going into the detailed origin of scattering.

\item[(iii)] Probability of collision in a time duration, $dt\sim \dfrac{dt}{\tau}$, $\tau=$ relaxation time.

$\tau$ is independent of electrons position and velocity, it turns out that this approximation works well.

\item[(iv)] electrons achieve thermal equillibrium via collisions $\to$ Thermodynamic equillibrium via collisions.
\end{itemize}

\section*{DC Conductivity}

Ohm's law - $V=IR$

$\to E=\rho j$\quad $j=$ current density, $\rho=$ resistivity, $E=$ electric field.

$j=-neV$

$n\to$ electron density per unit volume.

$\nu\to$ average velocity.
\begin{center}
{\bf Figure}
\end{center}

Random motion does not contribute in current flow. So, after a collision, velocity gain due to electric field is
\begin{gather*}
m\dfrac{dv}{dt}=-eE\\
a, V=-\dfrac{eEt}{m}\to \nu_{\text{arg}}=-\dfrac{eE\tau}{m}
\end{gather*}
$\tau=$ relaxation time $=t_{\text{arg}}$.
$$
\therefore\quad J=\left(\dfrac{ne^{\nu}\tau}{m}\right)E
$$
$a$, $\sigma=$ conductivity $=\dfrac{ne^{2}\tau}{m}$\quad $\rho=\dfrac{1}{\sigma}=\dfrac{m}{ne^{2}\tau}$
$$
a,\quad \tau = \dfrac{m}{\rho ne^{2}}
$$
at $RT$, $\rho\sim\mu$ Ohm-cm $=10^{-18}$ stat ohm-cm (in atomic units)
$$
\therefore\quad \tau=\left(\dfrac{0.22}{\rho_{\mu}}\right)\left(\dfrac{r_{S}}{a_{0}}\right)^{3}\times 10^{-14}\text{sec.}
$$
$\rho_{m}$ in units of micro-ohm-cm.

at $RT$, $\tau=10^{-14}\text{sec}$ to $10^{-15}$ sec. $\sim Fs$.

$\lambda=\nu_{0}\tau$

assuming classical equipartition theorem
$$
\frac{1}{2}m\nu^{2}_{0}=\dfrac{3}{2}k_{B}T\to \nu_{0}\sim 10^{7}\text{ cm/sec.}
$$
$\therefore \ \lambda \sim 1$ to $10\pi^{0}$.

In real materials. $\nu_{0}$ is an order of magnitude smaller and temperature independent.

$\tau$ at low temperature is an order of magnitude langer than the value at $RT$.

$\lambda$ is about $10^{3}$ times larger $\to$ much larger than lattice spacings.

At sufficiently low temperature $\lambda\sim\text{ cm}$.

$\therefore$ Bumping at ions seems not correct.

\section*{Tight binding model}

Free electron and nearly free electron picture gives good overview of the electronic structure in a solid and their properties.
\begin{itemize}
\item[$\to$] Quasi continuous distribution of energies with a gap in between.

\item[$\to$] For an improved description, one can start with atomic states by constructing a Block wave function through a {\em linear combination of atomic orbitals} (LCAO) 

This is called tight binding approximation.

\item[$\to$] Nearly free electron approx works for metals.

\item[$\to$] For covalently bonded solid or metals with localized electrons ($d$ and $f$), tight binding model is a better starting point.
\end{itemize} 

Lets start with the simplest form $\to$
$$
H_{at}=-\dfrac{\hbar^{2}\nabla^{2}}{2m}+V_{at}(r)\quad\text{for an atom.}
$$

Eigenvalues $E_{n}$ Eigen functions $\phi_{n}(r)$

For a Solid $\to$
\begin{align*}
H &= -\dfrac{\hbar^{2}\nabla^{2}}{2m}+\sum\limits_{R}V_{at}(r-R)\\[3pt]
  &= -\dfrac{\hbar^{2}\nabla^{2}}{2m}+V_{at}(r)+\sum\limits_{R\neq 0}V_{at}(r-R)\\[3pt]
  &= -\dfrac{\hbar^{2}\nabla^{2}}{2m}+V_{at}(r)+\nu(r)\quad \nu(r)=\sum\limits_{R\neq 0}V_{at}(r-R)
\end{align*}
non-local contribution.

If the atoms are places far away from each other, the atomic wave functions may be a good starting point.
\begin{align*}
\therefore\quad & \int \phi^{*}_{n}(r)H\phi_{n}(r)dr\\
& =E_{n}+\int\phi^{*}_{n}\nu(r)\phi_{n}dr=E_{n}-\beta
\end{align*}
Small shift in energy due to other atoms becomes significant when $\nu(r)$ is appreciably large.

Now lets take a Bloch state to solve this problem.
$$
\psi_{k}(r)=\dfrac{1}{\sqrt{N}}\sum\limits_{R}e^{ik\cdot R}\phi_{n}(r-R)
$$
$\dfrac{1}{\sqrt{N}}$ is a normalization factor.
\begin{align*}
\therefore\quad E(k) &= \int \psi^{*}_{k}(r)H\psi_{k}(r)dr\\
&= \int \sum\limits_{R,R'}\dfrac{1}{N}e^{ik(R-R')}\phi^{*}_{n}(r-R)H\phi_{n}(r-R)dr
\end{align*}

All sums for a particular choice of $R'$ should be same as long as translational symmetry is preserved.
\begin{align*}
\sum\limits_{R'}f &= Nf(R'=0)\\
\therefore\quad E(k) &= \sum\limits_{R}\int \phi^{*}_{n}(r)H\phi_{n}(r-R)dr\\
&= E_{n}-\beta+\sum\limits_{R\neq 0}\int \phi^{*}_{n}(r)H\phi_{n}(r-R) dr\\
I &= E_{n}\int\phi^{*}_{n}(r)\phi_{n}(r-R)dr+\sum\limits_{R\neq 0}\int \phi^{*}_{n}(r)\nu(r)\phi_{n}(r-R)dr
\end{align*}
$0$ as $\phi$'s are centered at two different atoms.

`$X$' is also close to zero, but can be appreciable if $\nu(r)$ becomes large when approaching the neighbor.
\begin{gather*}
\to `X' = -\gamma(R)\\
\therefore\quad E(k)=E_{n}-\beta-\sum\limits_{R\neq 0}\gamma(R)e^{ik.R}
\end{gather*}
for an one dimensional solid, considering only nearest neighbors.

%page 48
