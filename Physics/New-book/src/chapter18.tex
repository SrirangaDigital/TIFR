\chapter[Lecture 18]{}\label{lec18}

{\bf Force Constant :} Effective forces in metals are quite long-range and are carried from ion to ion by conduction electrons.

The range can be as long as 20 planes!

Assume, we need $p$ planes for the dispersion relation
$$
\therefore\quad w^{2}=\dfrac{2}{M}\sum\limits_{k>0}C_{p}(1-\cos pka)
$$
Multiply both sides by $\cos nka$ and integrate over independent range of $k\to$
\begin{align*}
M\int\limits^{\pi/a}_{-\pi/a}d_{k}w^{2}_{k}\cos nka &= 2\sum\limits_{k>0}C_{p}\int\limits^{\pi/a}_{-\pi/a}dk(1-\cos pka)\cos nka\\
&= 2\pi \dfrac{C_{n}}{a}
\end{align*}
integral vanishes except $p=n$.
$$
\therefore\quad \fbox{$C_{p}=-\dfrac{Ma}{2\pi}\int\limits^{\pi/a}_{-\pi/a}dk w^{2}_{k}\cos pka$}
$$
$C_{p}$ is the force constant at range $pa$, for a structure with monatonic basis.

\section*{Crystal with a Basis}

Phonon dispersion shows new features in the presence of a basis.

Lets assume, there are two atoms in the basis; like NaCl crystal.

For convenience, consider direction of propagation such that the planes contain one kind of atom. [111] - direction 
\begin{center}
{\bf Figure}
\end{center}
The equation of motion will be
\begin{align*}
M_{1}\dfrac{d^{2}u_{s}}{dt^{2}} &= C(v_{s}+v_{s-1}-2u_{s})\\
M_{2}\dfrac{dv_{s}}{dt^{2}} &= C(u_{s+1}+u_{s}-2v_{s})
\end{align*}
The trial solution can be
$$
u_{s}=u\exp (iska) \exp (-iwt)\quad v_{s}=v\exp (iska)\exp(-iwt)
$$
On substitution, we get
\begin{equation*}
\left.
\begin{array}{l}
-w^{2}M_{1}u = Cv[1+\exp(-ika)]-2Cu\\
-w^{2}M_{2}v=Cu[\exp(ika)+1]-2Cv
\end{array}\right]\tag{A}\label{lec18-eqA}
\end{equation*}

This will have a solution for $u$ and $v$ if the determinant vanishes.
$$
w^{4}-2C\left(\frac{1}{M}+\frac{1}{M_{2}}\right)w^{2}+\dfrac{e^{2}k^{2}a^{2}}{M_{1}M_{2}}=0
$$
\begin{align*}
\Rightarrow \ w^{2} &= C\left(\dfrac{1}{M_{1}}+\dfrac{1}{M_{2}}\right)\pm \sqrt{C^{2}\left(\frac{1}{M_{1}}+\frac{1}{M_{2}}\right)^{2}-\dfrac{C^{2}k^{2}a^{2}}{M_{1}M_{2}}}\\
&= C\left(\dfrac{1}{M_{1}}+\dfrac{1}{M_{2}}\right)\pm C\sqrt{\left(\frac{1}{M_{1}}+\frac{1}{M_{2}}\right)^{2}-\dfrac{k^{2}a^{2}}{M_{1}M_{2}}}\\
&= C\left(\frac{1}{M_{1}}+\frac{1}{M_{2}}\right)\pm C\left(\dfrac{1}{M_{1}}+\frac{1}{M_{2}}\right)\left[1+\frac{k^{2}a^{2}M_{1}M_{2}}{(M_{1}+M_{2})^{2}}\right]^{\frac{1}{2}}\\
&= 2C\left(\frac{1}{M_{1}}+\frac{1}{M_{2}}\right)\quad\text{and}\quad \frac{1}{2}\cdot C\left(\frac{1}{M_{1}}+\frac{1}{M_{2}}\right)\frac{k^{2}a^{2}M_{1}M_{2}}{(M_{1}+M_{2})^{2}}=\frac{\frac{1}{2}C}{M_{1}+M_{2}}\cdot k^{2}a^{2}
\end{align*}
%page 29
