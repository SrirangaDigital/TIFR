\chapter{}\label{lec9}

\section*{Start from projection operator}

Multiply $\Gamma^{(i)}(R)^{*}_{\lambda'k'}$ from the left and employ great orthogonality theorem -
\begin{align*}
\sum\limits_{R}\Gamma^{(i)}(R)^{*}_{\lambda'k'}P_{R}\phi^{(j)}_{k} &= \sum\limits_{R}\sum\limits_{\lambda}\Gamma^{(i)}(R)^{*}_{\lambda'k'}\phi^{(j)}_{\lambda}\Gamma^{(j)}(R)_{\lambda k}\\
&= \frac{h}{l_{j}}\delta_{ij}\delta_{kk'}\phi^{(j)}_{\lambda'}\\
\therefore\quad \phi^{(j)}_{\lambda}=\frac{l_{j}}{h}\sum\limits_{R}\Gamma^{(j)}(R)^{*}_{\lambda k}P_{R}\phi^{(j)}_{k}=P^{(j)}_{\lambda k}\phi^{(j)}_{k}
\end{align*}
Here
$$
\fbox{$P^{(j)}_{\lambda k}=\frac{l_{j}}{h}\sum\limits_{R}\Gamma^{(j)}(R)^{*}_{\lambda k} P_{R}$}
$$
So if we know $\phi^{(j)}_{k}$ we can get all the partners by employing $P^{(j)}_{\lambda k}$ on $\phi^{(j)}_{k}$.
\begin{gather*}
P^{(j)}_{kk}\phi^{(j)}_{k}=\dfrac{l_{j}}{h}\sum\limits_{R}\Gamma^{(j)}(R)^{*}_{kk}P_{R}\phi^{(j)}_{k}-\mathfrak{g}_{h}\sum\limits_{R}\sum\limits^{l_{j}}_{\lambda=1}\Gamma^{(j)}(R)^{*}_{kk}\Gamma^{(j)}(R)_{\lambda k}\phi^{(j)}_{\lambda}\\
a, \ \fbox{$P^{(j)}_{kk}\phi^{(j)}_{k}=\phi^{(j)}_{k}$}
\end{gather*}

\begin{theorem*}
If $\Gamma^{(1)}$, $\Gamma^{(2)}\ldots \Gamma^{(l)}$ are all the distinct irreducible representation of a group operator, $P_{R}$, then any function $F$ in the space operated on by $P_{R}$ can be decomposed into the form
$$
\fbox{$F=\sum\limits^{C}_{j=1}\sum\limits^{l_{i}}_{k=1}f^{(j)}_{k}$}
$$
$f^{(j)}_{k}$ belongs to $k^{\text{th}}$ row of $j^{\text{th}}$ irred repr.
\end{theorem*}

\begin{proof}
Consider $F_{1}F'_{2}$, $F'_{3}\ldots F'_{n}$ are obtained via operation of $P_{E}$, $P_{A_{2}},P_{A_{3}}\ldots P_{A_{n}}$ on $F$. Discard all function not linearly independent of the others.

Say $F_{1}F_{2}\ldots F_{n}$ are `$n$' distinct functions remaining from above group. Orthogonalize them.

These functions form the basis of a unitary representation of the group \fbox{$P_{S}P_{R}F=P_{SR}F$} satisfie in the space spanned by the set by construction.

Say, $F_{1},F_{2},F_{3}\ldots F_{n}$ form $n$-dimensional representation $\Gamma$
$$
\therefore\quad P_{R}F_{k}=\sum\limits^{n}_{i=1}F_{i}\Gamma(R)_{i_{k}}
$$
\begin{itemize}
\item[(i)] If $\Gamma$ is irreducible representation $F_{k}$ belong to a particular row, `$k$' of the representation.

\item[(ii)] If $T$ is not irreducible, then $\Gamma$ can be brought to block form by a similarity {\bf transformation}.
\begin{equation*}
\alpha\Gamma(R)\alpha =
\left(
\begin{matrix}
T(R) & 0 & 0\\
0 & T^{(j)}(R) & 0\\
0 & 0 & \ldots
\end{matrix}
\right)\tag{A}\label{lec9-eqA}
\end{equation*}
\end{itemize}

The matrix $\alpha$ defines a set of functions $F''_{k}$ which transform as above and hence are the functions of the type $f^{(j)}_{k}$ for various value of `$j$' and `$k$'.

Using $\alpha^{-1}$ we may express $F_{k}$, and in particular $F$, a linear combination of $F''_{k}$ or $f^{(j)}_{k}$. (proved)

Now,
\begin{align*}
P_{R}f^{(j')}_{k'} &= \sum\limits^{n}_{i=1}f_{i}^{(j')}\Gamma^{(j')}(R)_{ik'}\\
P^{(j)}_{kk}f^{(j')}_{k'} &= \frac{l_{i}}{h}\sum\limits_{R}\Gamma^{(j)}(R)_{kk}P_{R}f_{k'}\\
&= \sum\limits^{n}_{i=1}\sum\limits_{R}\frac{l_{j}}{h}\Gamma^{(j)}(R)^{*}_{kk}f^{(j')}_{i}\Gamma^{(j')}(R)_{ik'}\\
&= \delta_{kk'}\delta_{jj'}f^{(j)}_{k}
\end{align*}
$$
\therefore\quad \fbox{$P^{(j)}_{kk}F=f^{(j)}_{k}$}\Rightarrow P^{(j)}_{kk}
$$
is the projection operator which projects out part of any function to the $k^{\text{th}}$ row of $j^{\text{th}}$ representation.

Such a projection is called {\bf idempotent}
$$
a_{1}\fbox{$P^{(j)}_{kk}P^{(j)}_{kk}=P^{(j)}_{kk}$}
$$
all powers of the operators are equal.
\end{proof}

\begin{theorem*}
Two functions belonging to two different representations (irreducible) or to different rows of same unitary representation are orthogonal.
\end{theorem*}

\begin{proof}
This is valid for any $R\in \mathfrak{g}$. $\therefore$ \ One can sum over, $R$ and divide by `$h$'
\begin{align*}
\left(\phi^{(j)}_{k},\psi^{(j')}_{k'}\right) &= \left(P_{R}\phi^{(j)}_{k},P_{R}\psi^{(j')}_{k'}\right)\\
                             &= \sum\limits_{\lambda\lambda'}\Gamma^{(j)}(R)^{*}_{\lambda k}\Gamma^{(j')}(R)_{\lambda'k'}\left(\phi^{(j)}_{\lambda},\psi^{(j')}_{\lambda'}\right)
\end{align*}
Take sum over $R$ and divide by `$h$' use great orthogonality theorem
$$
\fbox{$\left(\phi^{(j)}_{k},\psi^{(j')}_{k'}\right)=\delta_{jj'}\delta_{kk'}\sum\limits^{l_{i}}_{\lambda=1}\left(\phi^{(j)}_{\lambda},\psi^{(j)}_{\lambda}\right)/l_{j}$}
$$
(prooved) Here $(\phi^{(j)}_{k},\psi^{(j)}_{k})$ is independent of $k$.
\end{proof}

Take sum of projection operators $\to$ less detailed information but good enough in many cases.
$$
P^{(j)}=\sum\limits_{k}P^{(j)}_{kk}=\dfrac{l_{j}}{h}\sum\limits_{R}X^{(j)}(R)^{*}P_{R}
$$

Any function $f^{(j)}$ expressible as a sum of functions belonging only to rows within the $j^{\text{th}}$ representation will satisfy $P^{(j)}f^{(j)}=f^{(j)}$.
$$
\to \fbox{$P^{(j)}F=f^{(j)}$}
$$
$\therefore \ P^{(j)}$ projects out the part of an arbitrary function, $F$ belonging to $j^{\text{th}}$ representation.

\begin{example*}
Consider a group of $E$, $\sigma$, $\gamma \ \to$ reflectino $\to n$ to $-x$.

This has two classes and two one dimensional irreducible representation.
\begin{center}
\begin{tabular}{>{$}c<{$}|>{$}c<{$}>{$}c<{$}}
\hline
 & E & \sigma\\
\hline
\Gamma^{1} & 1 & 1\\
\Gamma^{2} & 1 & -1 
\end{tabular}
\end{center}
$P^{1}$ can be defined as $\dfrac{1}{2}(P_{E}+P_{\sigma})$ and $P^{(2)}=\dfrac{1}{2}(P_{E}-P_{\sigma})$
$$
\therefore\quad P^{(1)}F(x)=\dfrac{1}{2}\left[F(x)+F(-x)\right]\quad P^{(2)}F(x)=\dfrac{1}{2}(F(x)-F(-x))
$$
Evidently, these two function are `even' and `odd' similar to $\Gamma^{(1)}$ and $\Gamma^{(2)}$ respectively.
\end{example*}

\section*{Direct Product Group}
\begin{align*}
\mathfrak{g}_{a} &= E,A_{2},A_{3},\ldots A_{h_{a}}\\
\mathfrak{g}_{b} &= E, B_{2},B_{3},\ldots B_{h_{b}}
\end{align*}
Operators of one type commutes with the other types.

$\therefore \ \mathfrak{g}_{a}\times \mathfrak{g}_{b}=E,A_{2},A_{3}\ldots A_{h_{a}},B_{2},A_{2}B_{2},\ldots A_{h_{a}}B_{2},\ldots A_{h_{a}}B_{h_{b}}$

Order of the group $=h_{a}h_{b}$
\begin{itemize}
\item Both groups have one common element, $E$

\item Elements of direct product group follow closure property since $A_{k'}$ and $B_{l}$ commutes.
\begin{align*}
(A_{k}B_{l})(A_{k},B_{l'}) &= (A_{k}A_{k'})(BB_{ll'})\\
&= A_{j}B_{m}\to A_{j}\in \mathfrak{g}_{a}; \ B_{m}\in \mathfrak{g}_{b}\\
&\in \mathfrak{g}_{a}\times \mathfrak{g}_{b}
\end{align*}
$\therefore \ \mathfrak{g}_{a}\times \mathfrak{g}_{b}$ is closed and satisfy the properties of a group.
\end{itemize}

\section*{Representation}
\begin{itemize}
\item Irreducible representation of the component group also form irreducible representation of the Direct product group.

\item Group element representation requires 4 indices.

e.g., \fbox{$(A\times B)_{i_{k},j_{l}}=A_{ij}B_{kl}$}

\item They follow group multiplication property
\begin{align*}
& \Gamma^{(a\times b)}(A_{k}B_{l})\Gamma^{(a\times b)}(A_{k'}B_{l'})=\left[\Gamma^{(a)}(A_{k})\times \Gamma^{(b)}(B_{l})\right]\left[\Gamma^{a}(A_{k'})\times \Gamma^{(b)}(B_{l'})\right]\\
&\quad \Gamma^{(a)}(A_{k})\Gamma^{(a)}(A_{k'})\times \Gamma^{(b)}(B_{l})\Gamma^{(b)}(B_{l'})\\
&= \Gamma^{(a)}(A_{k}A_{k'})\times \Gamma^{(b)}(B_{l}B_{l'})\\
&= \Gamma^{(a\times b)}(A_{k}A_{k'}B_{l}B_{l'})
\end{align*}

\item Direct product of two irreducible representation form an irreducible representation of the direct product group.
\end{itemize}
$l^{(a)}_{1},l^{(a)}_{2}\ldots $ and $l^{(b)}_{1}$, $l^{(b)}_{2}\ldots$ are the dimensionalities of the irreducible representations of $\mathfrak{g}_{a}$ and $\mathfrak{g}_{b}$
$$
\therefore\quad \sum (l^{a}_{i})^{2}=h_{a}\quad\text{and}\quad \sum\limits_{i}(l^{b}_{i})^{2}=h_{b}
$$
In direct product group, the dimensionalities will be 
\begin{align*}
l_{ij} &= l^{a}_{i}l^{b}_{j}\\
\sum\limits_{ij}l^{2}_{ij} &= \sum\limits_{ij}(l^{a}_{i}l^{b}_{j})^{2}=\sum\limits_{i}(l^{a}_{i})^{2}\sum\limits_{j}(l^{b}_{j})^{2}=h_{a}\cdot h_{b}=h
\end{align*}
$\therefore$ \ There can be no other irreducible representation.

$\Rightarrow$ \ This shows, how an extra quantum number arised due to additional commuting symmetry operator.

\section*{Class Structure and Character}

Character of direct product representation is the product of the characters of component representation.
\begin{align*}
X^{(a\times b)}(A_{k}B_{l}) &= \sum\limits_{ij}\Gamma^{(a\times b)}(A_{k}B_{l})_{ij,ij}\\
&= \sum\limits_{ij}\Gamma^{(a)}(A_{k})_{ii}\Gamma^{(b)}(B_{l})_{jj}\\
&= \sum\limits_{i}\Gamma^{(a)}(A_{k})_{ii}\sum\limits_{j}\Gamma^{(b)}(B_{l})_{jj}=X^{(a)}(A_{k})X^{b}(B_{l})
\end{align*}

\begin{example*}
Consider a direct product group of
$$
\text{and}\quad 
\left.
\begin{array}{l}
D_{3} = E,A,B,C,D,F\\
S= E,\sigma_{h}
\end{array}
\right|\quad h=6\times 2=12
$$
product group is $D_{3h}=D_{3}\times S$ the elements commute as there is no difference is rotation is done first or reflection.

Multiplication table
\begin{center}
\begin{tabular}{>{$}c<{$}|>{$}c<{$}>{$}c<{$}}
S & E & \sigma_{h}\\
\hline
E & E & \sigma_{h}\\
\sigma_{h} & \sigma_{h} & E
\end{tabular}
\end{center}

Character table
\begin{center}
\begin{tabular}{>{$}c<{$}|>{$}c<{$}>{$}c<{$}}
S & E & \sigma_{h}\\
\hline
T^{(+)} & 1 & 1\\
T^{(+)} & 1 & -1
\end{tabular}
\end{center}

$\therefore$ \ Character table of $D_{3h}$
\begin{center}
\begin{tabular}{|>{$}c<{$}|>{$}c<{$}>{$}c<{$}>{$}c<{$}|>{$}c<{$}>{$}c<{$}>{$}c<{$}|}
\hline
D_{3h} & E & (ABC) & (DF) & \sigma_{h} & \sigma_{h}(ABC) & \sigma_{h}(DF)\\
\hline
\Gamma^{1+} & 1 & 1 & 1 & 1 & 1 & 1\\
\Gamma^{2+} & 1 & -1 & 1 & 1 & -1 & 1\\
\Gamma^{3+} & 2 & 0 & -1 & 2 & 0 & 1\\
\Gamma^{1-} & 1 & 1 & 1 & -1 & -1 & -1\\
\Gamma^{2-} & 1 & -1 & 1 & -1 & 1 & -1\\
\Gamma^{3-} & 2 & 0 & -1 & -2 & 0 & 1\\
\hline
\end{tabular}
\end{center}
Going from $D_{3}$ to $D_{3h}$ dimensionalities of representations has not changed.
\end{example*}

\section*{Direct product representation within a group}

Create a new representation $\Gamma$ from $\Gamma^{(1)}$ and $\Gamma^{(2)}$ of the same group.

Let the bases are $\phi_{1},\phi_{2},\phi_{3}\ldots \phi_{n}$ and $\psi_{1},\psi_{2},\psi_{3}\ldots\psi_{m}$ 

$T^{(1)}\to n$ dimensional and $\Gamma^{(2)} m$ dimensional.
\begin{align*}
P_{R}(\phi_{R}\psi_{\lambda}) &= \sum\limits_{k'}\phi_{k'}\Gamma^{(1)}(R)_{k'k}\sum\limits_{\lambda'}\psi_{\lambda'}\Gamma^{(2)}(R)_{\lambda'\lambda}\\
&=\sum\limits_{k'\lambda'}\phi_{k'}\psi_{\lambda'}\Gamma^{(1)}(R)_{k'k}\Gamma^{2}(R)_{\lambda'\lambda}\\
&=\sum\limits_{k',\lambda'}\phi_{k'}\psi_{\lambda'}\Gamma(R)_{k'\lambda',k\lambda}
\end{align*}
$\therefore \ \Gamma(R)=\Gamma^{(1)}(R)\times \Gamma^{(2)}(R)$ form a representation with basis functions $\phi_{k}\psi_{\lambda}$.
\begin{itemize}
\item $X(R)=X^{(1)}(R)X^{(2)}(R)$

\item {\bf This representation is reducible} $\leftarrow$ This is the difference with the previous case.
\end{itemize}

As total No. of distinct irreducible representation is fixed.









