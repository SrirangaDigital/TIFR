\chapter{}\label{lec13}

\section*{Reciprocal Lattice}

For a periodic potential as found in solid.

$a=$ lattice constant translation symmetry.
\begin{gather*}
\psi(x+a)=\psi(x)\\
\downarrow\\
\fbox{$\psi_{k}(x)=u_{k}(x)e^{ikx}$}
\end{gather*}
where $k=\dfrac{2\pi p}{L}$ \ $p$ is an integer and $k$ is a good quantum no. in Solid.

Largest value of $k=\dfrac{2\pi}{L/p}$ when $\dfrac{L}{p}$ is smallest for a finite solid of length `$L$'.

We can consider $L/p$ is the lattice constant `$a$'.

$\therefore$ In one dimension - $k$ can be written as \fbox{$k=\dfrac{2\pi}{a}$}. All other values of $k$ falls within the range $\fbox{0 to $(2\pi/a)$} \Rightarrow -\frac{\pi}{a}$ to $+\frac{\pi}{a}$.

If `$n(x)$' represents the property of the solid at $x$, it can be written as
$$
n(x)=\sum\limits_{p=\text{integers, (-ve, 0, +ve)}}n_{p}e^{ikx\cdot p}\quad \text{if}\quad k=\dfrac{2\pi}{a}
$$
Total contribution of $n_{P}(x)$ from all the sites.

$n_{P}$ are the Fourier Coefficients of the above expansion and can be expressed as 
$$
n_{P}=\frac{1}{a}\int\limits^{a}_{0}dx \ n(x)e^{-ikxp}
$$

In three dimension-
$$
\fbox{$n(r)=\sum\limits_{G}n_{G}e^{iG.r}$}
$$
and $n_{G}=\dfrac{1}{V_{c}}\int\limits_{\text{Cell}}dv \ n(r)e^{-iG.r}$

$V_{c}$ = volume of the unit cell.

Here, $\overrightarrow{G}$ or $\overrightarrow{k}$ are the Fourier conjugate of real space `$\overrightarrow{r}$' and hence, the space constituted by $k$ is call reciprocal space or momentum space.

For a lattice constituted by the real space vectors $a_{1}$, $a_{2}$ and $a_{3}$ one can define the reciprocal lattice vectors as $b_{1}$, $b_{2}$ and $b_{3}$
\begin{align*}
v^{*} &=b_{1}(b_{2}\times b_{3})=\dfrac{(2\pi)^{3}}{v^{3}}(a_{2}\times a_{3})\cdot \{(a_{3}\times a_{1})\times (a_{1}\times a_{2})\}\\
&=\dfrac{(2\pi)^{3}}{v^{3}}(a_{2}\times a_{3})\cdot a_{1}\cdot v=\dfrac{(2\pi)^{3}}{v}
\end{align*}
\begin{align*}
(a_{3}\times a_{1})\times (a_{1}\times a_{2}) &\Rightarrow e_{ijk}(a_{3}\times a_{1})_{j}(a_{1}\times a_{2})_{k}\\
&=e_{ijk}e_{jlm}a_{3}a_{1m}e_{knP}a_{1a}a_{2P}\\
&= e_{ijk}e_{jlm}e_{knp}a_{3l}a_{1m}a_{1n}a_{2p}\\
&=e_{lnp}a_{3l}a_{lm}a_{ln}a_{2p}-e^{k=m}_{mnp}a_{3L}a_{1m}a_{1n}a_{2p}
\end{align*}
$$
b_{1}=\frac{2\pi}{v}(a_{2}\times a_{3})\quad b_{2}=\frac{2\pi}{v}(a_{3}\times a_{1})\quad b_{3}=\frac{2\pi}{v}(a_{1}\times a_{2})
$$
\begin{align*}
e_{jkl}e_{jem}e_{knp}a_{3e}a_{1m}a_{1n}a_{2p} &= e_{knp}a_{3k}a_{1i}a_{1n}a_{2p}-e_{knp}a_{3i}a_{1k}a_{1n}a_{2p}\\
&= a_{1i}a_{3}(a_{1}\times a_{2})-a_{3i}a_{1}\cancel{(a_{1}\times a_{2})=0}\\
&= a_{1}v
\end{align*}

$v=a_{1}(a_{2}\times a_{3})$ =  volume of the primitive unit cell. (PUC)

$v^{*}=b_{1}(b_{2}\times b_{3})$ = volume in reciprocal space.

Reciprocal space is a Bravais Lattice and follows a general definition.

A Bravais lattice is a discrete set of vectors not all in a plane, closed under vector addition and subtraction.

All $a$'s and $b$'s follow orthogonality relation \fbox{$\overrightarrow{a_{i}}\cdot \overrightarrow{b_{j}}=2\pi \delta_{ij}$}.

Now, if $G$ is a reciprocal lattice vector, one can write
\begin{align*}
 &\overrightarrow{G}=\alpha_{1}b_{1}+\alpha_{2}b_{2}+\alpha_{3}b_{3}\quad \text{and}\quad \overrightarrow{R}=n_{1}a_{1}+n_{2}a_{2}+n_{3}a_{3}\\
\therefore\quad &\overrightarrow{G}\cdot \overrightarrow{R}=2\pi(\alpha_{1}n_{1}+\alpha_{2}n_{2}+\alpha_{3}n_{3})\Rightarrow \fbox{$\sum\limits_{i=1}^{3}2\pi \alpha_{i}n_{i}=\overrightarrow{G}\cdot \overrightarrow{R}$}
\end{align*}

Reciprocal lattice and Direct lattice are mutually Fourier spaces $\to$ Fourier transform of Reciprocal space is Real space and vice versa.

Reciprocal lattice and Real lattice of a crystal belong to same crystal system $\to$ seven crystal system.

All symmetry operations in a real system are also symmetry of the reciprocal space.

\begin{proof}
$A$ is a rotation vector in real space. $a'=A\cdot a$; \fbox{$a'_{i}=A_{ij}a_{j}$}.

If $a$ becomes $A$. $a$ is real space, $b$ will be $A\cdot b$ in reciprocal space.
\begin{align*}
[A(\overrightarrow{a}_{2}\times \overrightarrow{a}_{3})]_{i} &= A_{ii'}(a_{2}\times a_{3})_{i'}\\
&= A_{ii'}e_{i'j'k'}a_{2j'}a_{3k'}
\end{align*}
Einstein's Convention 

Sum implicit for repeated index.
\begin{align*}
(A\times B)_{i} &= \sum\limits_{jk}e_{ijk}A_{j}B_{k}\\
l_{ijk} &\to \text{Levi-civita symbol}\\
&= \text{when } i,j,k\text{ cyclic and } i\neq j\neq k\\
&= -1\text{ for non cyclic and } i\neq j\neq k\\
&= 0\text{ otherwise.}
\end{align*}
$$
\overrightarrow{A}\cdot \overrightarrow{B}=A_{i}B_{i}
$$
Sum on repented dummy index `$i$' is implicit.

Now,
\begin{align*}
[(A\cdot a_{2})\times (A\cdot a_{3})]_{i} &= e_{ijk}(A\cdot a_{2}); (A\cdot a_{3})_{k}\\
&= e_{ijk}A_{jj'}a_{2j'}A_{kk'}a_{3k'}\\
&= e_{ijk}A_{jj'}A_{kk'}a_{2j'}a_{3k'}
\end{align*}
Rotation vectors form unitary representation and orthogonal.

$\therefore \ \fbox{$A_{jj'}A_{jk'}=\delta_{j'k'}$}$\quad $A_{ij}A_{ik}=\delta_{jk}$ Kronecker delta
$$
\fbox{$e_{ijk}=e_{i'j'k'}A_{ii'}A_{jj'}A_{kk'}$}
$$
\begin{align*}
&= e_{lmn}A_{il}A_{jm}A_{jj'}A_{kn}A_{kk'}a_{2j'}a_{3k'}\\
&= e_{lmn}a_{il}\delta_{mj'}\delta_{nk'}a_{2j'}a_{3k'}=e_{lj'k'}A_{il}a_{2j'}a_{3k'}
\end{align*}
$\therefore \ $ \fbox{$A(a_{2}\times a_{3})=(A\cdot a_{2})\times (A\cdot a_{3})$}

$\therefore \ A$ is also a rotation vector in reciprocal space.

Angles between primitive vectors: $\alpha(\overrightarrow{a}_{2}-\overrightarrow{a}_{3})$, $\beta(\overrightarrow{a}_{3}-\overrightarrow{a}_{1})$, $\gamma(\overrightarrow{a}_{1}-\overrightarrow{a}_{2})$

Reciprocal space $\alpha^{*}(\overrightarrow{b}_{2}-\overrightarrow{b}_{3})$, $\beta^{*}(\overrightarrow{b}_{3}-\overrightarrow{b}_{1})$, $\gamma^{*}(\overrightarrow{b}_{1}-\overrightarrow{b}_{2})$
\end{proof}

Reciprocal primitive vectors for simple Bravais Lattices (7 crystal systems)
\begin{center}
{\fontsize{7}{9}\selectfont
\begin{tabular}{lcccc}
\hline
{\bf Crystal System} & {\boldmath$b_{1}$} & {\boldmath$b_{2}$} & {\boldmath$b_{3}$} & {\bf Angles}\\
\hline
Cubic, Tetra, Orth. & $2\pi/a_{1}$ & $2\pi/a_{2}$ & $2\pi/a_{3}$ & $\alpha^{*}=\beta^{*}=\gamma^{*}=90^{\circ}$\\
Rhombohedral, Triclinic & $2\pi a_{2}a_{3}\sin d/v$ & $2\pi a_{3}a_{1}\sin \beta/v$ & $2\pi a_{3}a_{1}\sin \gamma/v$ & Different\\
Monochinic & $2\pi/a_{1}\sin \beta$ & $2\pi/a_{2}$ & $2\pi/a_{3}\sin\beta$ & $\beta^{*}\pi-\beta$\\
Hexagonal & $2\pi/a_{1}\sin\gamma$ & $2\pi/a_{2}\sin\gamma$ & $2\pi/a_{3}$ & $\gamma^{*}=\pi-r$\\
\hline
\end{tabular}}\relax
\end{center}
\begin{itemize}
\item A crystal structure with a basis has reciprocal lattice a Bravais lattice.
\end{itemize}
{\bf FCC}
\begin{align*}
\overrightarrow{a}_{1} &= \frac{a}{2}[\widehat{x}+\widehat{y}]\quad \overrightarrow{a}_{2}=\frac{a}{2}[\widehat{y}+\widehat{z}]\quad \overrightarrow{a}_{3}=\dfrac{a}{z}(\widehat{z}+\widehat{x})\\
v &= \overrightarrow{a}_{1}(\overrightarrow{a}_{2}\times \overrightarrow{a}_{3})=\frac{a^{3}}{8}(\widehat{x}\times \widehat{y})\left[(\widehat{y}+\widehat{z})\times (\widehat{z}+\widehat{\lambda})\right]\\
&= \frac{a_{3}}{8}(\widehat{x}+\widehat{y})[\widehat{x}-\widehat{z}+\widehat{y}]=\frac{a_{3}}{4}
\end{align*}
\begin{align*}
b_{1} = \frac{2\pi}{v}(a_{2}\times a_{3}) &=\frac{2\pi}{v}[\widehat{x}+\widehat{y}-\widehat{z}]\frac{a^{2}}{4}\\
&= \frac{8\pi}{a^{3}}[\widehat{x}+\widehat{y}-\widehat{z}]\frac{a^{2}}{4}=\frac{4\pi}{a}\cdot\frac{1}{2}(\widehat{x}+\widehat{y}-\widehat{z})
\end{align*}
Similarly 
$$
b_{2}=\frac{4\pi}{a}\cdot \frac{1}{2}(\widehat{y}+\widehat{z}-\widehat{x})\quad b_{3}=\frac{4\pi}{a}\cdot \frac{1}{2}(\widehat{z}+\widehat{x}-\widehat{y})
$$
$\therefore$ Reciprocal lattice of a fcc lattice is a bcc lattice.

Similarly, one can show reciprocal lattice of a bcc lattice is a fcc lattice.
\begin{itemize}
\item Note, here $|a^{*}|=\dfrac{4\pi}{a}$ it is not $\dfrac{2\pi}{a}$ as in simple cubic lattice.

\item One can show that reciprocal lattice of a simple hexagonal lattice (Bravais) with lattice constants $c$ and $a$ is another simple hexagonal lattice with lattice constants $\dfrac{2\pi}{c}$ and $\dfrac{4\pi}{\sqrt{3}a}=\dfrac{2\pi}{a\sin \gamma}\to$ Rotated $30^{\circ}$ w.r.t. `$c$'-axis of direct lattice.
\end{itemize}

\section*{First Brillouin Zone}

The wigner-Seitz primitive cell of the reciprocal lattice is call first Brillouin zone. $L$-Brillouin (1930).
\begin{itemize}
\item[$\to$] 
\begin{itemize}
\item[$\bullet$] Find the nearest neighbours, find the places bisecting the line connected to the nearest neighbors.

\item[$\bullet$] The volume enclosed is the first Brillouin zone.
\end{itemize}
\item[$\to$] Same exercise with second nearest neighbor creates an enclosed volume. This volume excluding the first Brillouin zone is the second Brillouin zone.

..... so on.
\end{itemize}

\section*{Born-Karman Condition}

The wave in a crystal has periodicity at the boundary of the real lattice.

Lets consider primitive cell of the underlying Bravais lattice.

$-a$ parallelepiped space $(N_{1}\overrightarrow{a_{1}}, N_{2}\overrightarrow{a_{2}},N_{3}\overrightarrow{a_{3}})$

$\therefore$ \ Total no. of sites = $N=N_{1}\times N_{2}\times N_{3}$.

According to Born-Karman condition $\psi(\overrightarrow{r})=\psi(r+N_{j}\overrightarrow{a}_{j})j=1,2,3=\psi(r)e^{ik,N_{j}a_{j}}$ for all `$j$'.

$e^{i\overrightarrow{k}\cdot N_{j}\overrightarrow{a}_{j}}=1 \Rightarrow \overrightarrow{k}\cdot N_{j}\overrightarrow{a}_{j}=2\pi p$\quad $p=$ integer.

Now \fbox{$a_{i}\cdot b_{j}=2\pi \delta_{ij}$}
\begin{align*}
\overrightarrow{k} &= \alpha_{1}\overrightarrow{b}_{1}+\alpha_{2}\overrightarrow{b}_{2}+\alpha_{3}\overrightarrow{b}_{3}\\
\Rightarrow^{2\pi}\alpha_{1}N_{1} &=2\pi p_{1}\Rightarrow\alpha_{1}=\dfrac{m_{1}}{N_{1}}\\
2\pi \alpha_{2}N_{2} &= 2\pi p_{2}\Rightarrow d_{2}=\dfrac{m_{2}}{N_{2}}\\
2\pi\alpha_{3}N_{3} &= 2\pi p_{3}\Rightarrow \alpha_{3}=\frac{m_{3}}{N_{3}}
\end{align*}
$\therefore$ \ \fbox{$\overrightarrow{k}=\dfrac{m_{1}}{N_{1}}\overrightarrow{b}_{1}+\dfrac{m_{2}}{N_{2}}\overrightarrow{b}_{2}+\dfrac{m_{3}}{N_{3}}b_{3}$}\quad $m_{1},m_{2},m_{3}$ are integers.

$\therefore \ \overrightarrow{k}$ in a finite size crystal is descretized in reciprocal space, with the minimum unit spanned by $\left(\dfrac{\overrightarrow{b}_{1}}{N_{1}},\dfrac{\overrightarrow{b}_{2}}{N_{2}},\dfrac{\overrightarrow{b}_{3}}{N_{3}}\right)$.

$\therefore$ \ The volume $\Delta \overrightarrow{k}$ of $\overrightarrow{k}=\dfrac{\overrightarrow{b}_{1}}{N_{1}}\left(\dfrac{\overrightarrow{b}_{2}}{N_{2}}\times \dfrac{\overrightarrow{b}_{3}}{N_{3}}\right)=\dfrac{1}{N}\overrightarrow{b}_{1}(\overrightarrow{b}_{2}\times \overrightarrow{b}_{3})=\dfrac{v^{*}}{N}=\dfrac{(2\pi)^{3}}{V}$

$\dfrac{v^{*}}{N}$ is a tiny reciprocal volume compared to $v^{*}$ as $N=N_{1}N_{2}N_{3}$ is large.

$\therefore \ \overrightarrow{k}$ is a quasi-continuous function with decretization $\Delta \overrightarrow{k}$ inversely proportional to total volume $V-vN$ of the real lattice.

In first Brillouin zoen - $\overrightarrow{k}$ should stay within $-\dfrac{b_{j}}{2}$ to $+\dfrac{b_{j}}{2}$

$m_{j}$ lies between $-\dfrac{N_{j}}{2}$ to $+\dfrac{N_{j}}{2}$

Total No. of $k=N$, $N_{2}N_{3}=N$ within the first $BZ$.

\noindent
{\bf Energy spectrum:} No. of quantum states in a branch of the energy spectrum is related to the total No. of $\overrightarrow{k}$ in the $FBZ$, it is $N$ in phonon spectrum

$2N$ for electron energy bands. (up and down spin)

\section*{Lattice Planes}
%page 28
