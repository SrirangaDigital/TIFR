\chapter[Lecture 12]{}\label{lec12}

\section*{Irreducible Representation of the Point Group}
\begin{itemize}
\item From the order $(h)$ of the group and its class structure one can find the number and dimensionality of the irreducible representations, which is fixed by the rules discussed earlier.

\item One can of course work out the complete character table following the method discussed earlier.

\item These are available in various references

Margenau and Murphy -

Eyring, Walter and Kimball -
\end{itemize}

Example $\to$ equilater triangle $\to$ $D_{3}$ symmetry
\begin{center}
\begin{tabular}{|c|c|c|ccc|}
\hline
\multicolumn{3}{|c|}{$D_{3}$} & $E$ & $2C_{3}$ & $3C'_{2}$\\
\hline
$x^{2}+y^{2}$, $z^{2}$ & & $A_{1}$ & 1 & 1 & 1\\
\hline 
 & $R_{z}$, $Z$ & $A_{2}$ & 1 & 1 & $-1$\\
$\left.\begin{array}{l}
xy,yz\\
x^{2}-y^{2}, xy
\end{array}
\right\}$ & 
$\left.\begin{array}{l}
x,y\\
R_{x},R_{y}
\end{array}
\right\}$ 
& $E$ &  2 & $-1$ & 0\\
\hline
\end{tabular}
\end{center}
\begin{itemize}
\item $C'_{1}$ refer 2-fold axis perpendicular to principal 3-fold axis. If there is other inequivalent 2-fold axis, it would be $C''_{2}$.

\item Labels of irreducible representations :
\begin{quote}
$+1$ for principal rotation $\to \ A$

$-1$ for principal rotation $\to \ B$

$E$ for $2D$ representation

$T$ for $3D$ representation
\end{quote}
when inversion symmetry is present, one should use $g$ for `even' (gerade) and $u$ for `odd' (ungerade)

\item Second column $\to$ co-ordinates $[x,y,z]$ and rotation $[R_{x}, R_{y},R_{Z}]$ to indicate the representation according to which they transform.

\item First column $\to$ quadratic forms of co-ordinates for the same purpose as above.
\end{itemize}
The assignments can be checked by using projection operator $P^{(j)}$.

\medskip
\noindent
{\bf Example = $C_{2v}$}
\begin{center}
\begin{tabular}{|c|c|c|cccc|}
\hline
\multicolumn{3}{|c|}{$C_{2v}$} & $E$ & $C_{2}$ & $\sigma_{v}$ & $\sigma'_{v}$\\
\hline
$x^{2}$, $y^{2}$, $z^{2}$ & $z$ & $A_{1}$ & 1 & 1 & 1 & 1\\
$xy$ & $R_{2}$ & $A_{2}$ & 1 & 1 & $-1$ & $-1$\\
$xz$ & $R_{y}$, $x$ & $B_{1}$ & 1 & $-1$ & 1 & $-1$\\
$yz$ & $R_{x}$, $y$ & $B_{2}$ & 1 & $-1$ & $-1$ & 1\\
\hline 
\end{tabular}
\end{center}
For $\sigma_{v}$ mirror plane $xz$ plane.

$\sigma'_{v}$ minor plane $yz$ plane.
$$
P_{F}
\left(\begin{matrix}
x\\
y\\
z
\end{matrix}\right)
=
\left(\begin{matrix}
x\\
y\\
z
\end{matrix}\right)
P_{C_{2}}
\left(\begin{matrix}
x\\
y\\
z
\end{matrix}\right)
=
\left(\begin{matrix}
-x\\
-y\\
z
\end{matrix}\right)
\quad 
P_{\sigma_{v}}
\left(\begin{matrix}
x\\
y\\
z
\end{matrix}\right)
=
\left(\begin{matrix}
x\\
-y\\
z
\end{matrix}\right)
P_{\sigma_{v'}}
\left(\begin{matrix}
x\\
y\\
z
\end{matrix}\right)
=
\left(\begin{matrix}
-x\\
y\\
z
\end{matrix}\right)
$$
For $1D$ representation $X^{(i)}=\Gamma^{(i)}(R)$ 
\begin{align*}
P_{C_{2}}x &= \Gamma^{(B_{1})}(C_{2})x\quad \text{or more generally } P_{R}x=\Gamma^{(B_{1})}(R)x\\
P_{R}y &= \Gamma^{(B_{2})}(R)y\quad P_{R}z=\Gamma^{(A_{1})}(R)Z
\end{align*}
$\therefore$ \ $x$ transforms according to $B_{1}$, $y$ transform as $B_{2}$ and $Z$ transforms as $A_{1}$.
$$
\therefore\quad xy \text{ transforms as } B_{1}\times B_{2}=A_{2}
$$
use of projection operator -
\begin{align*}
P^{(A_{2})}xy &= \frac{l_{A_{2}}}{h}\sum x^{(A_{2})}(R)P_{R}xy\\
&= \frac{1}{4}\left\{xy+(-x)(-y)-x(-y)-(-x)y\right\}=xy
\end{align*}
$\therefore \ xy$ indeed belongs to $A_{2}$.

For the example of $D_{3} \to$
$$
\left(
\begin{matrix}
\cos\theta & \sin\theta & 0\\
-\sin\theta & \cos\theta & 0\\
0 & 0 & 1
\end{matrix}
\right)\quad\text{for rotation by $\theta$}
$$
Here for $D_{1}\theta=\dfrac{2\pi}{3}$.

Consider $Z$ as 3-fold axis, $x$ as $A$-axis and $y$ is perpendicular to these.
$$
P_{E}
\left(\begin{matrix}
x\\
y\\
z
\end{matrix}\right)
=
\left(\begin{matrix}
x\\
y\\
z
\end{matrix}\right)
\quad
P_{D}
\left(\begin{matrix}
x\\
y\\
z
\end{matrix}\right)=
\left(
\begin{array}{c}
-\frac{x}{2} -\frac{\sqrt{3y}}{2}\\[3pt]
+\frac{\sqrt{3}}{2}x -\frac{y}{2}\\[3pt]
Z
\end{array}
\right)
\quad 
P_{A}
\left(\begin{matrix}
x\\
y\\
z
\end{matrix}\right)
=
\left(\begin{matrix}
x\\
-y\\
-z
\end{matrix}\right)
$$

Here $D\neq D^{-1}$ \ $F=D^{-1}$ (anti-clockwise rotation)
$$
\therefore\quad P_{D}f(x)=f(D^{-1}x)
$$
clockwise rotation of an object is equivalent to counter-clockwise rotation of co-ordinate system.
\begin{itemize}
\item $Z$ transforms always into a multiple of itself $\Rightarrow Z$ forms a one dimensional representation.

Since $X(C'_{2})=-1$, $Z$ corresponds to $A_{2}$ from character table.

\item $x$ and $y$ have cross terms under $P_{D}\Rightarrow$ They transform as two-dimensional representation.
\begin{gather*}
P_{R}\phi^{(j)}_{k}=\sum\limits^{l_{j}}_{\lambda=1}\phi^{(j)}_{\lambda}\Gamma^{(j)}(R)_{\lambda k}\Rightarrow P_{D}x=x\Gamma(D)_{xx}+y\Gamma(D)_{yx}\\
\therefore\quad \Gamma(D)_{xx}=-\frac{1}{2}\Gamma(D)_{yx}=-\frac{\sqrt{3}}{2}
\end{gather*}
These matches with the representation of $D$ mentioned before.

\item Although simple functions like $x$, $y$ transforms in this way, there may be other pairs (can be infinit).

\item Since $x^{2}+y^{2}$, $z^{2}$ is invariant under all operations under $D_{3}$ they belong to $A_{1}\Rightarrow \therefore \phi (x^{2}+y^{2},z^{2})$ is invariant in $D_{3}$
\begin{itemize}
\item[$\to$] $x\phi$ and $y\phi$ for any such $\phi$, we have two functions that transform like $x$ and $y$.

\item[$\to$] Thus a relatively small number of matrices can describe transformation properties of all eigen functions for a given problem.
\end{itemize}
\end{itemize}

\section*{Three dimensional Rotation Group}

Rotations in $3D$ form an infinite group - the covering operations of a sphere.
\begin{quote}
No. of class = infinite [Rotation is an abelian group]

No. of irreducible representations = infinite.
\end{quote}

Lets consider spherical Harmonics as basis functions
$$
Y^{m}_{l}(\theta,\phi)\sim P^{m}_{l}(\theta)e^{im\phi}
$$
If we shift the polar axis (rotate to a new position), the new basis $Y^{m}_{l}(\theta',\phi')$ can be expressed in terms of $Y_{l}^{m'}(\theta,\phi)$ of the same `$l$'.
$$
\fbox{$P_{R}Y^{m}_{l}=\sum_{m'}\Gamma^{(l)}(R)_{m'm}Y^{m'}_{l}$}
$$
{\bf N.B.} If $m=0$, $\Gamma^{(l)}(R)_{m'0}$ can be derived by using addition theorem for spherical harmonics to be simply $\sim Y^{-m'}_{l}(\theta'',\phi'')$, where $\theta''$ and $\phi''$ are giving direction of new polar axis.
\begin{itemize}
\item $m$ goes from $-l$ to $+l\Rightarrow$ total $(2l+1)$ basis functions.

\item The representation is irreducible and complete.
\end{itemize}

\section*{Character}

e.g., If the contour is rotated by $\alpha$, it is equivalent to rotation of axis system by $-x$
$$
\therefore\quad P_{\alpha}Y^{m}_{l}(\theta,\phi)=Y^{m}_{l}(\theta,\phi-\alpha)=e^{-im\alpha}Y^{m}_{l}(\theta,\phi)
$$
$\therefore$ The representation of the rotation, $P_{R}$ can be, 
\begin{align*}
\Gamma^{(l)}_{(\alpha)}&=
\left(
\begin{matrix}
e^{-il\alpha} & 0 & \ldots & 0\\
0 & e^{-i(l-1)_{\alpha}} & \ldots & 0\\
\ldots & \ldots & \ldots & \ldots\\
0 & 0 & \ldots & e^{il\alpha}
\end{matrix}
\right)\\
\therefore\quad X^{(l)}(\alpha) &= \text{Trace } \Gamma^{(l)}(\alpha)=\sum\limits^{l}_{m=-l}e^{-im\alpha}\\
&= e^{-il\alpha}\left[1+e^{id}+\cdots+e^{i(2l)\alpha}\right]\\
&= e^{-il\alpha}\dfrac{e^{i(2l+1)\alpha}-1}{e^{i\alpha}-1}=\dfrac{e^{i(1+\frac{1}{2})\alpha}-e^{-i(1-\frac{1}{2})\alpha}}{e^{i\alpha/2}-e^{-i\alpha/2}}\\
&= \frac{\sin (l+Y_{2})\alpha}{\sin(\frac{\alpha}{2})}
\end{align*}
Since all rotations by $\alpha$ are in the same class (regardless of axis), they all have the character calculated above - same.
\begin{itemize}
\item There cannot be other inequivalent irreducible representation of odd order.
\end{itemize}

\begin{proof}
If it exists, it must be orthogonal to one found above. Therefore, they must be orthogonal to the difference $[X^{l}(\alpha)-\lambda^{(l-1)}(\alpha)]$ to
$$
X^{(e)}(\alpha)-X^{(e-r)}(\alpha)=2\cos l\alpha
$$
\begin{itemize}
\item[(i)] $\therefore$ The new representation is orthogonal to Fourier series of cosines.

\item[(ii)] Since rotation by $\pm \alpha$ are in the same class, $x$ must be an even function of $\alpha$.
\end{itemize}
This is not possible $\to$ a function is orthogonal to Fourier series of cosine and at the same time even.

$\therefore$ Spherical Harmonics of order $l$ form the basis of the only distinct irreducible representation of dimensionality $(2l+1)$ of the full rotation group.
\begin{itemize}
\item For even-dimensional representation (half integer angular momentum) it turns out that above formula for character holds in this case too - The new characters are orthogonal to the ones for odd-dimensional representation.

\item Here range of $\alpha$ changes from $\pm \pi$ to $\pm 2\pi$.

\item Rotation by $2\pi$ leads to a sign change and rotation by $4\pi$ is the identity.
\end{itemize}
\end{proof}

\section*{Crystal field splitting}

In any crystal environment, one can estimate the potential near the ion in question by replacing each neighboring ion by point charge.

Let's assume the central ion is at $(000)$

For a charge `$e$' at $(-a,0,0)$
$$
V(xyz)=\dfrac{e}{((x+a)^{2}+y^{2}+z^{2})^{\frac{1}{2}}}\leftarrow \frac{e}{r}
$$
Employing Tailor's expansion 
\begin{gather*}
f(x)=\sum\limits^{D}_{n=0}\frac{f^{(n)}}{n!}\text{~ $\leftarrow \ n^{\text{th}}$ derivation } (a)(n-a)^{N}\\
(1+x)^{\frac{-1}{2}}=1-\frac{1}{2}x+\frac{3}{8}x^{2}-\frac{5}{16}x^{3}+\frac{35}{128}x^{4}\ldots
\end{gather*}
\begin{itemize}
\item[(i)] $V_{c}$ symmetric w.r.t. $Z$-axis $V_{a}\sim rY^{0}_{2}(\theta)\sim 3z^{2}-r^{2}$

\item[(ii)] Orthorhombic symmetry $V_{r}\sim Ax^{2}+By^{2}-(A+B)z^{2}$
\end{itemize}
This is a mixture of $Y^{0}_{2}$ and $(Y^{2}_{2}+Y^{-2}_{2})\sim (x^{2}-y^{2})$ 
\begin{align*}
H &= \sum\limits_{el}\frac{P^{2}_{i}}{2m}-\sum\limits_{\text{nuc.}}\frac{P^{2}_{k}}{2M_{k}}-\sum\limits_{\text{nuc}_{1}el}\frac{Z_{k}e^{2}}{r_{ki}}+\sum\limits_{\text{nuc.}}\frac{Z_{k}Z_{L}e^{2}}{r_{KL}}+\sum\limits_{el}\frac{e^{2}}{r_{ij}}\\
&\quad +H_{LS}\text{ (spin-orbit term)} + H_{SS}\text{ (spin-spin term)}\\
&\quad +H_{nfs}\text{ (hyperfine structure)}+H_{\text{ext}}\text{ (external field coupling)}.
\end{align*}
\begin{align*}
V(xyz) &= \frac{e}{a}\left(1+\frac{2x}{a}+\frac{r^{2}}{a^{2}}\right)\\
&= \frac{e}{a}\left[1-\frac{x}{a}-\frac{r^{2}}{2a^{2}}+\frac{3x^{2}}{2a^{2}}+\cdots\right]
\end{align*}
Now, if we put another charge $e$ at $(a,0,0)$ we get same expression with odd power of `$x$' having reversed sign.

Similarly one can place a charge at $(0,\pm a,0)$ and $(0,0,\pm a)$ forming an octahedra.

$\therefore$ The leading non-spherical term will be
$$
V_{c}=\frac{35}{4}\frac{e}{a^{5}}\left(x^{4}+y^{4}+z^{4}-\frac{3}{5}r^{4}\right)
$$
This is invariant under symmetry operations of $O_{h}$.

\section*{Intermediate Field Case}

Spin-orbit coupling term: $\xi(r)$ L.S. $=E_{LS}$, $\xi(r)=\frac{1}{2}m^{2}c^{2}-\frac{1}{1}\cdot \frac{dv}{dr}$

$3d$ transition metal group.

Crystal field $O_{h}$ $\to$ consider $O$ first.

First column - Bouchaert, Smoluchowski and Wigner.

Second $\to$ Von der Lage and Bethe.

$D$ In both these prime $\to$ parity change/inversion Third column $\to$ Bethe.

$4^{\text{th}}$ Column $\to$ Conventional one.

Crystal field splitting in intermediate to spin-orbit energy and separation of L-S terms in the free-atom L\&S are good quantum no but $J$ is not.

\section*{Character Table for $O$}

\begin{center}
\begin{tabular}{|>{$}c<{$}|>{$}c<{$}|>{$}c<{$}|>{$}c<{$}|>{$}c<{$}>{$}c<{$}>{$}c<{$}>{$}c<{$}>{$}c<{$}|}
\hline
\multicolumn{4}{|c|}{} & E & 8C_{3} & 3C_{2} & 6C_{2} & 6C_{4}\\
\hline
\Gamma_{1} & \alpha & \Gamma_{1} & A_{1} & 1 & 1 & 1 & 1 & 1\\
\Gamma_{2} & \beta' & \Gamma_{2} & A_{2} & 1 & 1 & 1 & -1 & -1\\
\Gamma_{12} & \gamma & \Gamma_{3} & E & 2 & -1 & 2 & 0 & 0\\
\Gamma'_{15} & \delta' & \Gamma_{4} & T_{1} & 3 & 0 & -1 & -1 & 1\\
\Gamma'_{25} & \epsilon & \Gamma_{5} & T_{2} & 3 & 0 & -1 & 1 & -1\\
\hline
\end{tabular}
\end{center}
for an orbital angular momentum $L$, there are $2L+1$ spherical Harmonics, $Y^{M}_{L}$, which are degenerate in isotropic space.

Representation may be called $\Gamma^{(L)}$ or $D^{(L)}$ or (Darstelling)  $D_{L}$ following Wigner.
\begin{align*}
\therefore\quad x(C_{2}) &=x(\pi)=(-1)^{L}\quad\text{for all } L\\
x(C_{3}) &= x\left(\frac{2\pi}{3}\right)=\left\{
\begin{matrix}
1 & L=0,3,\ldots\\
0 & L=1,4,\ldots\\
-1 & L=2,5,\ldots
\end{matrix}
\right.\\
x(C_{4}) &= x\left(\frac{\pi}{2}\right)=
\left\{
\begin{matrix}
1 & L=0,1,4,5,\ldots\\
-1 & L=2,3,6,7,\ldots
\end{matrix}
\right.\\
\end{align*}
To reduce $D_{L}$ to smaller irreducible representations
$$
D_{L}=\sum\limits_{i}a_{i}\Gamma_{i}\quad a_{i}=\frac{1}{24}\sum\limits_{k}N_{k}x_{i}(c_{k})x_{L}(c_{k})
$$

\eject

Now for the group `$O$' the character table for different $L$'s
\begin{center}
\begin{tabular}{>{$}l<{$}>{$}l<{$}>{$}l<{$}>{$}l<{$}>{$}l<{$}>{$}l<{$}}
O & E(2L+1) & 8C_{3} & 3C_{2} & 6C_{2} & 6C_{4}\\[2pt]
D_{0}\quad s & 1 & 1 & 1 & 1 & 1\\[2pt]
D_{1}\quad p & 3 & 0 & -1 & -1 & 1\\[2pt]
D_{2}\quad d & 5 & -1 & 1 & 1 & -1\\[2pt]
D_{3}\quad f & 7 & 1 & -1 & -1 & -1\\[2pt]
D_{4}\quad d & 9 & 0 & 1 & 1 & 1
\end{tabular}
\end{center}
$\therefore \ D_{0}=A_{1}$ cannot split

$D_{1}=T_{1}$ remains a single irreducible representation

$D_{2}=E+T_{2}$
\begin{itemize}
\item[(i)] There is no five-dimensional representation so, it will split.

\item[(ii)] Splits into a two-fold and a three-fold degenerate level.
\end{itemize}

$D_{3}=A_{2}+T_{1}+T_{3}$ `$f$'-state split into 3 levels.

\smallskip

$D_{4}=A_{1}+E+T_{1}+T_{2}$ a `$g$' state splits into 4 levels.

\begin{example*}
For $D_{2}$  and $T_{1}$
\begin{align*}
a_{T_{1}} &= \frac{1}{24}[3.5+8.0.(-1)+3.(-1).1+6.(-1).1+6.1.(-1)]\\[2pt]
&= \frac{1}{24}[15+0-3-6-6]=0
\end{align*}
\end{example*}

\section*{Additional splitting due to lowering of symmetry}

\textbf{Trigonal distrotion}

Octahedron is distorted by elongation along a {\bf body-diagonal}.

$\to$ resulting field is $D_{3}$ (3-fold axis)

To reduce to irreducible representation first write character table of $D_{3}$ and corresponding characters of $O$ below that.
\begin{center}
\begin{tabular}{cc|ccc}
\hline
\multicolumn{2}{c|}{$D_{3}$} & $E$ & $2C_{3}$ & $3C_{2}$\\[2pt]
\hline
Irreducible & $A_{1}$ & 1 & 1 & 1\\[2pt]
representation of & $A_{2}$ & 1 & 1 & $-1$\\[2pt]
$D_{3}$ & $E$ & 2 & -1 & 0\\[2pt]
Irreducible & $A_{1}$ & 1 & 1 & 1\\[2pt]
representation of $0$ & $A_{2}$ & 1 & 1 & $-1$\\[2pt]
 & $E$ & 2 & $-1$ & 0\\[2pt]
 & $T_{1}$ & 3 & 0 & $-1$\\[2pt]
 & $T_{2}$ & 3 & 0 & 1\\[2pt]
\hline
\end{tabular}
\end{center}
$A_{1}$, $A_{2}$ and $E$ of $D_{3}$ remains irreducible under $D_{3}$.

For $O$ : From inspection one can see
\begin{align*}
& T_{1}\to E+A_{2}\\[2pt]
& T_{2}\to E+A_{1}
\end{align*}
One can use formula and find this too.

$A_{1}$ \& $A_{2}$ are one dimensional and could not split.

$E$ is non-trivial.

If triangle axis is not one of these axes, the residual symmetry in the presence of both fields would be smaller than either group and might contain only $E$. Then all levels will split to non-degenerate levels.

Splitting of $d$-band in $O_{h}$ field and then in $D_{4}h$.

$d\to t_{ig}+e_{g}\to e_{g}+b_{2g}+a_{1g}+b_{1g}\leftarrow D_{4h}$
$$
\underbrace{d_{a_{2}},d_{yz}\quad d_{xy}}_{t_{2g}(O_{h})}\quad \underbrace{d_{z^{2}}\quad d_{x^{2}=y^{2}}}_{l_{g}(O_{h})}
$$
