\chapter[Lecture 24]{}\label{lec24}

\section*{Fermi surface}
\begin{quote}
Extended zone scheme $\to$ Different bands in different zone.

Reduced zone scheme $\to$ All bands in first zone.

Periodic zone scheme $\to$ every band in every zone.
\end{quote}
\begin{center}
{\bf Figure}
\end{center}
\begin{itemize}
\item[(i)] Interaction of electrons with periodic potential creates energy gap at zone boundary.

\item[(ii)] Fermi surface intersects zoen boundary perpendicularly.

\item[(iii)] Crystal potentials round out sharp corners in FS.

\item[(iv)] Total volume enclosed by FS depends only on electron concentration and is independent of lattice interactions.
\end{itemize}

Draw spheres at every reciprocal lattice point $\to$ volume represent electron concentration.
\begin{itemize}
\item[(a)] If a $k$-print lies at least in one sphere $\to$ first zone

\item[(b)] $k$-point at least in two spheres $\to$ second zone.

\item[(c)] $k$-point at least in three spheres $\to$ third zone.
\end{itemize}
\begin{center}
{\bf Figure}
\end{center}

Experimental Determination of Fermi surface.
\begin{tabbing}
\phantom{N}Direct method \=$\to$ Phto emission (ARPES)\\[4pt]
Indirect method \>$\to$ de Haas - Van Alphen effect\\[4pt]
                \>$\to$ Shubnikow-de Haas effect\\[4pt]
                \>$\to$ Anomalous skin effect\\[4pt]
                \>$\to$ magneto resistance\\[4pt]
                \>$\to$ magneto-acoustic effect
\end{tabbing}

In a magnetic field, momentum of a particle with charge $q$ is
$$
P=P_{\text{kin}}+P_{\text{field}}=\hbar k+\dfrac{qA}{C}\quad \overrightarrow{B}=\overrightarrow{V}\times \overrightarrow{A}
$$

Following Onsagar and Lifshitz approach, we assume that orbits in a magnetic field is quantized by Bohr-Sommerfeld relation
$$
\oint p-d\overrightarrow{r}=(n+\gamma)2\pi \hbar
$$
$n\to$ integer, $\nu\to$ phase correction $=\dfrac{1}{2}$ for electrons.
$$
\oint p\cdot dr=\oint \hbar k\cdot dr+\dfrac{q}{c}\oint A\cdot dr
$$
Now $\hbar k=p=\dfrac{q}{c}\overrightarrow{r}\times \overrightarrow{B}$
$$
\therefore\quad \oint \hbar k\cdot dr=\dfrac{q}{c}\oint r\times B\cdot dr=-\dfrac{qB}{C}\cdot \oint r\times dr=-\dfrac{2q}{C}\phi
$$
$\phi\to$ magnetic flux within the orbit in real space
$$
\oint r\times dr = 2\times \text{ (area enclosed by the orbit)}
$$

\section*{Stokes Theorem}

Now,
\begin{align*}
\frac{q}{c}\oint A\cdot A\cdot dr &= \dfrac{q}{c}\int \nabla\times A\cdot dS \text{ (area integral)}\\
&= \frac{q}{c}\int B\cdot ds=\dfrac{q}{c}\phi\\
\therefore \ \oint p\cdot dr &= -\dfrac{q}{c}\phi = (n+\gamma)2\pi\hbar
\end{align*}
$$
\therefore\quad \fbox{$\phi_{n}=(n+\gamma)\dfrac{2\pi\hbar c}{e}$}\quad q=-e\text{ for electrons.}
$$
$2\pi\hbar c/e\simeq 4.14\times 10^{-7}$ Gauss $m^{2}$ or $Tm^{2}$

Now, we know  $p=\dfrac{q}{c}r\times B\Rightarrow \hbar\overrightarrow{k}=\dfrac{q}{c}\overrightarrow{r}\times \overrightarrow{B}$
$$
\Rightarrow \text{ taking magnitude, } \fbox{$\Delta r=\dfrac{\hbar c}{eB}\cdot \Delta k$}
$$
$\therefore$ Area $A_{n}$ in real space is related are {\em $S_{n}$ in reciprocal space}
\begin{gather*}
a, A_{n}=\left(\dfrac{\hbar C}{eB}\right)^{2}S_{n}\\
\therefore\quad \phi_{n}=\left(\dfrac{\hbar C}{e}\right)^{2}-\dfrac{1}{B}\cdot S_{n}=(n+\gamma)2\pi \hbar C/e\\
\therefore\quad \fbox{$S_{n}=(n+\gamma)\dfrac{2\pi e}{\hbar C}B$}
\end{gather*}
$\to$ Area of an orbit in $k_{n}$, $k_{y}$ space.
\begin{align*}
\therefore\quad \Delta &=\text{ area between successive orbits.}\\
&= S_{n}-S_{n-1}=\dfrac{2\pi eB}{\hbar C}
\end{align*}
If the specimen is a square of side $L$.

This area occupied by a single orbital $=\left(\dfrac{2\pi}{L}\right)^{2}$.

$\therefore$ Degeneracy of a single level $=\dfrac{2neB}{\hbar C}\cdot \left(\dfrac{L}{2\pi}\right)^{2}=\rho B=D_{n}$
$$
\rho=\dfrac{eL^{2}}{2\pi\hbar C}
$$
Magnetic levels are called Landan levels.
\begin{center}
{\bf Figure}
\end{center}

Increase $B$ gradually

Degeneracy increases.

Separation of the levels increases
\begin{gather*}
E_{n}=\left(n-\frac{1}{2}\right)\hbar w_{C}\quad n=1,2,3,\ldots\\
\fbox{$w_{C}=\dfrac{eB}{m^{v}e}$}
\end{gather*}

\begin{tabbing}
Completely filled level \=$\to$ Partially filled\\[3pt]
\>$\to$ Completely filled ... etc.
\end{tabbing}

Critical field at which all the levels are filled
$$
n\rho B_{n}=N\quad N=\text{ total no. of electrons.}
$$
{\bf Energy:} Total Energy for fully filled levels.
$$
=\sum\limits^{n}_{i=1}D_{i}\hbar w_{C}(i-Y_{2})=\frac{1}{2}D_{n}\hbar w_{C}n^{2}
$$

Energy of the partially filled level
$$
=\hbar w_{C}\left(n+\frac{1}{2}\right)(N-nD)
$$
\begin{align*}
\therefore\quad \text{Total energy} &= \frac{1}{2}D\hbar w_{C}n^{2}+\hbar w_{C}\left(n+\frac{1}{2}\right)(N-nD)\\
&= \hbar w_{C}\left[\frac{1}{2}Dn^{2}+\left(n+\frac{1}{2}\right)N-n\left(n+\frac{1}{2}\right)D\right]\\
&= \hbar w_{C}\left[\left(n+\frac{1}{2}\right)N-\frac{1}{2}D(n+1)\right]
\end{align*}
\begin{align*}
D &= \dfrac{eL^{2}}{2\pi \hbar C}\cdot B\\
w_{C} &= \dfrac{eB}{m^{*}C}
\end{align*}
\begin{center}
{\bf Figure}

{\bf Figure}
\end{center}

$\mu=-\dfrac{\partial U}{\partial B}$ Oscillates as a function of $\dfrac{1}{B}$ called $dH_{v}A$ oscillation.

The interval of oscillation $\Delta\left(\dfrac{1}{B}\right)=\dfrac{2\pi e}{\hbar \subset S}$

$S\to$ Extremal area of Fermi surface normal to $\overrightarrow{B}$.
\begin{center}
{\bf Figure}
\end{center}

\section*{Direct method - ARPES}

Discovery of electrons J.J. Thomson - 1897.

Drude (1900) put forward his theory of metallic conduction.

Solid consists of heavy (+ve) charge (immobile) and mobile electrons that makes them neutral solid.

Some fraction of electrons of $z$ electrons ($z=$ atomic no.) are strongly attached to (+ve) charge $\to$ don't move much that rest moves all over the solid and can be treated like classical electron gas, with velocity distance following Maxwell- Boltzmann Distance.
$$
f_{2}=n\left(\dfrac{m}{2\pi k_{B^{T}}}\right)^{\frac{3}{2}}e^{-m\nu^{2}/2k_{B^{T}}}
$$

This worked quite well $\to$ Wiedemann - Franz law.

Specific heat $\dfrac{3}{2}k_{B}$ per electron $\to$ this was not observed.

Later after Quantum theory came, Pauli's exclussion principle needs to be applied and Fermi-Dirac Distribution $\to$ {\em Sommerfeld's model.}

\section*{Free-electron approximation}

\subsection*{Drude's model}

\subsection*{Heat Capacity}
$$
e_{F}\sim k_{B}T_{F}\quad T_{F}\sim 10^{4}K
$$
$\therefore \ \dfrac{T}{T_{F}}$ is very small close to room temperature.

$\therefore$ The fraction of electrons taking part in heat capacity is $\sim \dfrac{T}{T_{F}}$
\begin{gather*}
\therefore \ U_{el}=\left(N\dfrac{T}{T_{F}}\right)K_{B}T\quad N= \text{ total no. of electrons.}\\
\therefore \ C_{el}=\dfrac{\partial U_{el}}{\partial T}\simeq N k_{B}T/T_{F}=\gamma T\\
\gamma \text{ is a const } \to \text{ measure of effective mass.}\\
\therefore \ \fbox{$C=\gamma T+AT^{3}$}\quad T^{3} \text{ is phonon part.}
\end{gather*}
at plot of $C/T$ with $T^{2}$ given $\gamma$ and $A$, and we can measure effective mass that way, $\gamma\to$ measure of effective mass.

A factor $\dfrac{\pi^{2}}{2}$ appears if calculated using Formi-Dirac distribution.

\section*{Force on an electron}
$$
F=-e(\overrightarrow{E}+\dfrac{1}{e}\overrightarrow{\nu}\times \overrightarrow{B})
$$
In an electric field
$$
\delta k=-\dfrac{eEt}{\hbar}\quad \nu=-\dfrac{eE\tau}{m}\Leftarrow m\nu=\hbar k
$$

Due to scattering of electrons, $t\sim \tau\to$ relaxation [The electric field acts till the electron gets scattered] time.
\begin{gather*}
\therefore\quad J=nq\overrightarrow{\nu}=ne^{2}\tau E/m\\
\therefore\quad \sigma =\dfrac{j}{E}=\dfrac{ne^{2}\tau}{m}\quad \rho=\dfrac{1}{r}=\text{ resistivity.}
\end{gather*}

Scattering can be due to phonons $(\tau_{i})$ or impurities $(\tau_{i})$
$$
\dfrac{1}{\widetilde{L}}=\dfrac{1}{\widetilde{L}_{L}}+\dfrac{1}{\widetilde{L}_{i}}\quad \text{Marthiessen's rule.}
$$
$\rho(T)/\rho(0)=$ residual resistivity ratio.

Impurity scattering usually independent of temperature $T$.

Phonon contribution is dependent on no. of phonons at a particular temperature $T$ usually \fbox{$\rho_{L}\alpha T$}.

\section*{Umklapp Scattering}
\begin{center}
{\bf Figure}
\end{center}

Scattering $\to$
$$
k'_{1}=k_{1}+q_{1}
$$
If this scattering involves a reciprocal lattice vector, it is called Umklapp scattering. e.g.,
$$
k'_{2}=k_{2}+q_{2}\pm G
$$
\begin{itemize}
\item a scattering of electron with wave vector $k_{2}$, results to a final wave vector $k'_{2}$ which involves lattice translation $G=AA'$.

\item The angle of scattering is large $\to \sim \pi$ strong scatterers.

\item If the Fermi surface does not intersect zone boundary, one needs minimum value of $q(=q_{0})$ to have Umklapp scattering.

\item At low temperature, Umklapp scattering goes as $e^{-\theta_{U}/T}$

$\theta_{U}=$ Characteristic temperature

For potassium $\theta_{U}=23K$\quad $\theta_{D}=91K$

$\therefore$ at low temperature, Umklapp scattering is almost {\em zero}.
\end{itemize}
$\therefore$ Resistivity is dominated by small angle scattering.

\section{Motion in Magnetic field}

Force on an electron $=F=\dfrac{dp}{dt}=\dfrac{\hbar dk}{dt}\quad P=\hbar k$

However, due to scattering, the fermi sphere will experience a friction at a rate $\dfrac{1}{\tau}$
$$
\therefore \ F=\hbar \left(\dfrac{d}{dt}+\dfrac{1}{\tau}\right)\delta k
$$
change in Momentum + Friction.
$$
\therefore\quad \hbar \dfrac{dk}{dt}+\dfrac{1}{\tau}\hbar \delta k=-e\left[\overrightarrow{E}+\dfrac{1}{C}\overrightarrow{\nu}\times \overrightarrow{B}\right] \text{ (Lorenz force)}
$$
if $\overrightarrow{B}=B\widehat{z}$.
\begin{align*}
m\left(\dfrac{d\nu_{x}}{dt}\right)+\dfrac{v_{n}}{\tau} &= -e\left(E_{x}+\dfrac{B}{C}\nu_{y}\right)\\[3pt]
m\dfrac{d\nu_{y}}{dt}+\dfrac{\nu_{y}}{\tau} &= -e\left(E_{y}-\dfrac{B}{C}\nu_{x}\right)\\
m\dfrac{d\nu_{z}}{dt}+\dfrac{\nu_{z}}{\tau} &= -eE_{z}
\end{align*}
In the steady state in a static electric field, derivatives are zero.
\begin{gather*}
\therefore\quad 
\fbox{
$
\begin{array}{l}
\nu_{x}=-\dfrac{e\tau}{m}E_{x}-w_{C}\tau \nu_{y}\\
\nu_{y}=-\dfrac{e\tau}{m}E_{y}+w_{C}\tau \nu_{x}\\
\nu_{z}=-\dfrac{e\tau}{m}E_{z}
\end{array}
$
}\\
w_{C}=\dfrac{eB}{m_{C}}=\text{ cyclotron frequency}
\end{gather*}
$w_{C}\tau$ is (cyclotron frequency multiplied by time) and an important number.

$w_{C}\tau$ is unity of larger, the effect of magnetic field on electron orbits is significant.

\section*{Hall effect}

Electric field developed across two faces of a conductor in the direction $\overrightarrow{j}\times \overrightarrow{B}$

If $\overrightarrow{J}=J\widehat{x}$; $\overrightarrow{B}=B\widehat{z}$
\begin{center}
{\bf Figure}
\end{center}

Then $\delta \nu_{y}=0$ as current is not flowing out in $y$ direction
$$
\therefore \ E_{y}=\dfrac{w_{C}m}{e}\nu_{x}=-w_{C}\tau E_{x}=-\dfrac{eB\tau}{m_{C}}\cdot E_{x}.
$$

The quantity $\dfrac{E_{y}}{J_{x}B}$ is called Hall co-efficient $R_{H}$.
\begin{gather*}
\therefore \ R_{H} = \left. -\dfrac{eB\tau E_{\lambda}/m_{C}}{ne^{2}\tau E_{x}B/m}=-\dfrac{1}{neC}\right| J=\dfrac{ne^{2}\tau E_{x}}{m}\\[7pt]
\therefore \ \fbox{$R_{H}=-\dfrac{1}{neC}$}\quad\text{or}\quad \fbox{$R_{H}=-\dfrac{1}{ne}$}\quad \text{SI unit.}
\end{gather*}
(-ve) for free electrons. $e$ is magnitude of electrons charge.
\begin{itemize}
\item[(i)] Get carrier concentration

\item[(ii)] Type of carriers.
\end{itemize}
Assumption, all relaxation times are equal.

If both electrons and holes conduct, the scenario is complex.

\section*{Thermal Conductivity of Metals}
$$
J_{Q}=-K\dfrac{dT}{dx}
$$
\begin{align*}
K &= \text{ thermal conductivity}\\
 &= \frac{1}{3}C\nu l
\end{align*}
\begin{quote}
$C$ = heat capacity

$\nu$ = Drift velocity (Avg. velocity)

$l$ = mean free path.
\end{quote}
\begin{align*}
C_{el} &= \frac{\pi^{2}}{2}Nk_{B}T/T_{F}\\
\epsilon_{F} &= k_{B}T_{F}=\frac{1}{2}m\nu^{2}_{F}
\end{align*}
\begin{align*}
\therefore\quad K_{el}=\frac{1}{3}\cdot \frac{\pi^{2}}{2}k_{B}T\cdot \dfrac{k_{B}}{\frac{1}{2}m\nu^{2}_{F}}\cdot \nu_{F}\cdot l &= \frac{\pi^{2}nk^{2}_{B}T}{3m}\tau\\
l &= \nu_{F}\tau
\end{align*}
$\therefore$ The ratio of $K_{el}$ and $\sigma$
$$
\fbox{$\dfrac{K}{\sigma}=\dfrac{\pi^{2}k^{2}_{B}Tn\tau/3m}{ne^{2}\tau/m}=\dfrac{\pi^{2}}{3}\left(\dfrac{k_{B}}{e}\right)^{2}T=LT$}
$$
Wiedemann - Franz law.
\begin{align*}
L = \text{ Lorentz number } =\dfrac{K}{\sigma T} &= \frac{\pi^{2}}{3}\left(\dfrac{k_{B}}{3}\right)^{2}\\
&= 2.72\times 10^{-13}(\text{erg}/\text{esu}-k^{2})\\
&= 2.45\times 10^{-8}\text{ watt-}\sigma hm/k^{2}
\end{align*}
This is remarkable that the ratio does not contain $n$ or $m$ of electrons !!

At high and low temperatures it may be okay.

But at intermediate temperature $\left(\dfrac{K}{\sigma T}\right)$ is temperature dependent !!

\section*{Density of states}
$$
D(E)=\dfrac{dN}{dE}\quad E=\dfrac{\hbar^{2}k^{2}}{2m}
$$
\begin{center}
{\bf Figure}
\end{center}

\section*{One dimension}

Fermi volume $=2k_{F}$
\begin{align*}
\therefore\quad N &= \frac{L}{2\pi}\cdot 2k_{F}=\frac{L}{2\pi}\cdot x\cdot \sqrt{\frac{2m}{\hbar^{2}}}\sqrt{E_{F}}=\sqrt{\dfrac{2m}{\hbar^{2}\pi^{2}}}\cdot \sqrt{E_{F}}\\
\therefore\quad D(E) &= \dfrac{dN}{dE}=\text{ const. } \sum\limits_{i}(E-\epsilon_{i})^{\frac{1}{2}}
\end{align*}
\begin{center}
{\bf Figure}
\end{center}

\section*{Two Dimension}

Formi volume $=\pi k^{2}_{F}=\dfrac{2m E_{F}}{\hbar^{2}}$
$$
\therefore\quad D(E)=\dfrac{dN}{dE}\simeq \text{ Const.}
$$
\begin{center}
{\bf Figure}
\end{center}

\section*{Three dimension}

Fermi volume $=\dfrac{4}{3}\pi k^{3}_F{}=\dfrac{4\pi}{3}\left(\sqrt{\dfrac{2mE_{F}}{\hbar^{2}}}\right)^{3}$
$$
\therefore\quad D(E)\sim \sqrt{E}
$$
\begin{center}
{\bf Figure}
\end{center}

\noindent
{\bf Zero Dimension :- } Atomic levels like.

\noindent
{\bf Absorption :-}

\section{Drude's model}

no. of electron per unit volume
$$
n=\dfrac{N}{V}\sim 6.02\times 10^{23}\dfrac{z\rho m}{A}
$$
\begin{quote}
$Z=$ atomic no.

$\rho_{m}=$ mass density.

$A=$ Atomic mass.
\end{quote}

Conduction electron density is a fraction of $n$ usually $n_{e}\sim 10^{22}/em^{3}$.

If $r_{S}$ is the radius of the volume per conduction electron
\begin{align*}
\therefore\quad \dfrac{V}{N} &=\dfrac{1}{n_{e}}=\frac{4}{3}\pi r^{3}_{S}\\
\therefore\quad r_{S} &= \left(\dfrac{3}{4\pi n_{e}}\right)^{\frac{1}{3}}
\end{align*}
$\dfrac{r_{S}}{a_{0}}\sim 2-3$ in most cases.

$a_{0}=$ Bohr radius.

$\simeq \dfrac{\hbar^{2}}{me^{2}}=0.529\times 10^{-8}$ cm.

In alkali metals $\dfrac{r_{S}}{a_{0}}\Rightarrow 3-6$

In some cases it can go upto $10$.

These are typically 1000 times greater than classical gas at NTP.

There can be strong electron-electron and electron ion interactions $\to$ still, in this model, we assume.
\begin{itemize}
\item[(i)] No. interaction between two collisions
\begin{itemize}
\item[(a)] Neglect of electron - electron interaction

$\to$ Independent electron approximation.

\item[(b)] Neglect of electron - ion interaction

$\to$ free electron approximation.
\end{itemize}
(a) often works well in real system but (b) is no good.

\item[(ii)] Collision of electrons with impenetrable ion core. Drude did not consider electron, electron scattering.

$\to$ One can simply talk about scattering effect without going into the detailed origin of scattering.

\item[(iii)] Probability of collision in a time duration, $dt\sim \dfrac{dt}{\tau}$, $\tau=$ relaxation time.

$\tau$ is independent of electrons position and velocity, it turns out that this approximation works well.

\item[(iv)] electrons achieve thermal equillibrium via collisions $\to$ Thermodynamic equillibrium via collisions.
\end{itemize}

\section*{DC Conductivity}

%page 44




