\chapter{}\label{lec20}

$$
\fbox{$\theta=\dfrac{\hbar w_{D}}{k_{B}}=\dfrac{\hbar \nu_{D}}{k_{B}}\left(\dfrac{6\pi^{2}N}{V}\right)^{\frac{1}{3}}$}
$$
\begin{align*}
\Lt\limits_{x\to 0}\dfrac{x^{4}e^{x}}{(e^{x}-1)^{2}} &= \Lt\limits_{x\to 0}\dfrac{4x^{3}}{2e^{x}(e^{x}-1)}=\Lt\limits_{x\to 0}\dfrac{x^{2}}{e^{x}}\\
\Lt\limits_{x\to 0}\dfrac{x^{2}e^{x}}{(e^{x}-1)^{2}} &= \Lt\limits_{x\to 0}\dfrac{2x}{2e^{x}(e^{x}-1)}=\Lt\limits_{x\to 0}\dfrac{x}{e^{x}-1}=1\\
\therefore \ U &= 9Nk_{B}T\left(\dfrac{T}{\theta}\right)^{3}\int\limits^{x_{D}}_{0}dx\cdot \dfrac{x^{3}}{e^{x}-1}\quad x_{D}=\dfrac{\theta}{T}\\
C_{V} &= \dfrac{3V\hbar}{2\pi^{2}\nu^{3}}\cdot \dfrac{\hbar}{k_{B}T^{2}}\int\limits^{w_{D}}_{0}dw\dfrac{w^{4}e^{\beta\hbar w}}{(e^{\beta\hbar w}-1)^{2}}=9Nk_{B}\left(\dfrac{I}{\theta}\right)^{3}\int\limits^{x_{D}}_{0}dx\dfrac{x^{4}e^{x}}{(e^{x}-1)^{2}}
\end{align*}
for $T\gg \theta$\quad $C_{V}\simeq 3Nk_{B}$
$$
9Nk_{B}\left(\dfrac{T}{\theta}\right)^{3}\int\limits^{x_{D}}_{0}dx\cdot x^{2}=9Nk_{B}\left(\dfrac{T}{\theta}\right)^{3}\cdot \left.\dfrac{x^{3}}{3}\right]^{x_{D}}_{0}=3Nk_{B}.
$$
$C_{V}$ approaches classical value.

at $T$ very small $x_{D}\to \infty$.
\begin{align*}
\therefore\quad \int\limits^{\infty}_{0}dx\cdot \dfrac{x^{3}}{e^{x}-1} &=\int\limits^{\infty}_{0}dx\cdot x^{3}\sum\limits^{\infty}_{s=1}e^{-Sx}=6\sum\limits^{\infty}_{S=1}\dfrac{1}{S^{4}}=\dfrac{\pi^{4}}{15}\\
\Rightarrow \sum\limits^{\infty}_{1}\dfrac{1}{S^{4}} &= \dfrac{\pi^{4}}{90}\text{ found from table.}
\end{align*}
$\therefore \ U\simeq \dfrac{3\pi^{4}Nk_{B}T^{4}}{5\theta^{3}}$
$$
\therefore \ \fbox{$C_{V}=\dfrac{12\pi^{4}}{5}\cdot Nk_{B}\left(\dfrac{T}{\theta}\right)^{3}$} \simeq 234Nk_{B}\left(\dfrac{T}{\theta}\right)^{3}\Leftarrow \text{Debye's $T^{3}$-law.}
$$
\begin{itemize}
\item[$\to$] $T^{3}$ approximation is quite good at very low temperature as only the long wavelength acoustic moes are excited.

\item[$\to$] Short wavelength modes have too high energy to be populated at low temperatures, Hence, the continuum approximation holds good at low $T$.

\item[$\to$] For real systems $T^{3}$ works for $T\lesssim \dfrac{\theta}{50}$
\end{itemize}

\noindent
{\bf Einstein's model (1907):} Introduced the concept of energy quanta for lattice vibrations as phonons in analogy of photons.

\section*{Postulates}
\begin{itemize}
\item[(i)] All atoms in a solid vibrates at a fixed frequency, $\nu$. Energy is not continuous, it is $E=nh\nu$; $\epsilon=h\nu$ is the energy of quasiparticle, phonon.

\item[(ii)] Used statistical physics to calculate energy distribution, specially Boltzmann principle $S=k_{B}ln\Omega$
\begin{quote}
$S=$ entropy, $k_{B}=$ Boltzmann constant

$\Omega=$ No. of available microscopic states
\end{quote}
At temp. $T$, $3N$ fold degenerate level, $\epsilon=h\nu$, Average no. of phonons $\overline{n}$, can be calculated by minimizing the grans potential $J=U-TS-\mu N$ or {\em Landau free energy}, $\mu=$ Chemical potential.

$=F \text{(Helmboltz free energy)} -G \text{(Gibb's free energy)}$ 
$$
G=U+pV-TS=\sum\mu_{1}N_{i}\text{~ for a homogeneous system}
$$
\end{itemize}

$\mu=0$ due to non-conservation of phonon numbers in Bose gas.
\begin{align*}
\therefore\quad 0 &=\dfrac{dJ}{dn}=\dfrac{d}{dn}\left\{n\epsilon-k_{B^{T}}ln \dfrac{(n+3N)!}{n!(3N)!}\right\}/n=\overline{n}\\
&= \epsilon  k_{B^{T}}ln \dfrac{\overline{n}+3N}{\overline{n}}
\end{align*}
$$
\therefore\quad \fbox{$\hbar = 3N \dfrac{1}{e^{\epsilon/k_{B^{T}}}-1}$}=3N \ f_{B.E.}(\epsilon) \text{ (Base-Einstein distance function)}
$$

%page 12

