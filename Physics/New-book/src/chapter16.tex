\chapter[Lecture 16]{}\label{lec16}

\section*{Atomic Radii}

Assume that $R_{C}$ and $R_{A}$ are the radii of cation and anion.

Then the bond length will be $=R_{C}+R_{A}$

But in reality a correction factor comes due to coordination of the site.

\begin{example*}
$BaTiO_{3} \ \to$ Perovskite structure,

$T_{i} \ \to$ Octahedra, $B_{a} \ \to$ Dodica hedra 12 nearest neighbors.
\begin{center}
{\bf Correction terms for diff. Coordination No. \boldmath$N$.}
\medskip

\renewcommand{\arraystretch}{1.2}
\begin{tabular}{cc|cc|cc}
\hline
$N$ & $\Delta_{N}(A^{0})$ & $N$ & $\Delta_{N}(A^{0})$ & $N$ & $\Delta_{N}(A^{0})$\\
1 & $-0.5$ & 5 & $-0.05$ & 9 & $+0.11$\\
2 & $-0.31$ & 6 & 0 & 10 & $+0.14$\\
3 & $-0.19$ & 7 & $+0.04$ & 11 & $+0.17$\\
4 & $-0.11$ & 8 & $+0.08$ & 12 & $+0.19$\\
\hline
\multicolumn{2}{c}{$R_{B^{2+}_{a}}=1.35A^{0}$} & \multicolumn{2}{c}{$R_{O^{2}}=1.4A^{0}$} & \multicolumn{2}{c}{$R_{Ti^{4+}}=0.68A^{0}$}\\
\hline
\end{tabular}
\end{center}

$\therefore \ D_{Ba-O}=R_{Ba^{2+}}+R_{O^{2}}+\Delta_{12}=(1.35+1.4+0.19)A^{0}=2.94A^{0}$

$\therefore \ $ Lattice constant = $2.94\times \sqrt{2}A^{0}=4.16A^{0}$
\begin{gather*}
D_{Ti-O}=R_{Ti^{4+}}+R_{O^{2-}}+\Delta_{6}=(0.68+1.4+0)A^{0}=2.08A^{0}\\
\therefore \ \text{Lattice Constant } =1.08\times 2 A^{0}=4.16A^{0}
\end{gather*}
Experimental lattice constant, $a=4.004A^{0}$, somewhat smaller than the estimated value, as the material is not purely ionic.

In fact, the covalency is quite strong in this material leading to significant reduction in lattice constant.

Covalency $\to$ Overlap of outer orbitals $\to$ Decrease in lattice constant.
\end{example*}

\section*{Elastic Strains}
\begin{itemize}
\item[(i)] Elastic waves have characteristic wavelength, $\lambda \gtrsim 10^{-6}$ cm Frequencies $<10^{11}$ to $10^{12}$ Hz.

\item[(ii)] Therefore, one can consider the solid as a homogeneous continuous media instead of a periodic array of atoms. This is a good approximation.

\item[(iii)] Implement {\em Hooke's law :} In an elastic solid, strain is directly proportional to the stress.

Elastic modules = Stress/Strain

\item[(iv)] This law applies for small strain only. If the strain is large, we consider the system is in {\em non-linear region.}

\item[(v)] Strains are so small, it is not necessary to distinguish differences between isothermal (constant temperature) and adiabatic (constant entropy) deformations. 

\item[(vi)] Such differences are usually small at temperature close to room temperature or lower.

\item[(vii)] Assume that the strain deforms the axis system, so that the definition of the object remains unchanged.
\end{itemize}

Lets assume that the orthogonal unit vectors, $\widehat{x}$, $\widehat{y}$, $\widehat{z}$ have undergone deformation to $\overline{x}'$, $\overline{y}'$, $\overline{z}'$ axes system $\to$ In uniform deformation, each unit cell will deform in the same way.
$$
\therefore\quad \overline{x}^{1}_{i}=\sum\limits_{j}(\delta_{ij}+\epsilon_{ij})\widehat{x}_{j}\quad i_{1}j=1,2,3\Rightarrow \widehat{x}, \widehat{y}, \widehat{z}
$$
$\epsilon_{ij}$ defines the deformation and are $\ll 1$ for small strain.

Note: $x_{i}$ are not unit vectors.
\begin{gather*}
\overline{x}'_{1}\cdot \overline{x}'_{1}=1+2\epsilon_{11}+\epsilon^{2}_{11}+\epsilon^{2}_{12}+\epsilon^{2}_{13}\\
x'_{i}=\sum(\delta_{ij}+\epsilon_{ij})\widehat{x}_{j}\quad x'_{i}-x'_{j}=\sum\limits_{kl}(\delta_{ik}+\epsilon_{ik})\widehat{x}_{k}\cdot (\delta_{jl}+\epsilon_{jl})\widehat{x}_{l}
\end{gather*}
Now, lets assume an atom is located at $\overrightarrow{r}=x\widehat{x}+y\widehat{y}+z\widehat{z}$.

After deformation it's position will be $r'=x\overline{x}'+y\overline{y}'+z\overline{z}'$

$\therefore$ The displacement, $\overrightarrow{R}=r'-r=x(\overline{x}'-\widehat{x})+y(\overline{y}'-\widehat{y})+z(\overline{z}'-\widehat{z})=\sum\limits_{i}x_{i}(\overline{x}'_{i}-\widehat{x}_{i})$

Now, $x'_{i}-\widehat{x}_{i}=\sum\limits_{j}\epsilon_{ij}\widehat{x}_{j}\Leftarrow x'_{i}=\sum\limits_{j}(\delta_{ij}+\epsilon_{ij})\widehat{x}_{j}$
$$
\therefore\quad \overrightarrow{R}=\sum\limits_{ij}x_{i}\epsilon_{ij}\widehat{x}_{j}=\sum\limits_{j}\left(\sum\limits_{i}x_{i}\epsilon_{ij}\right)\widehat{x}_{j}=\sum\limits_{j}u_{j}\widehat{x}_{j}
$$
where
$$
\fbox{$u_{j}=\sum\limits_{i}x_{i}\epsilon_{ij}$}\quad u_{1}=\epsilon_{11}x+\epsilon_{27}y+\epsilon_{31}z
$$
Using Taylor expansion for small strains, one can write \fbox{$\epsilon_{ij}=\dfrac{\partial u_{j}}{\partial x_{i}}$} neglecting higher order terms.

Following usual convention, one can define the deformations as
$$
e_{ii}=\epsilon_{ii}=\dfrac{\partial u_{i}}{\partial x_{i}}\quad e_{ij}=x'_{i}\cdot x'_{j}\simeq \epsilon_{ji}+\epsilon_{y}=\dfrac{\partial u_{i}}{\partial x_{j}}+\dfrac{\partial u_{j}}{\partial x_{i}}
$$
$$
\fbox{$e_{ii}=\dfrac{\partial u_{i}}{\partial x_{i}}$}\quad \fbox{$e_{ij_{i+j}}=\dfrac{\partial u_{i}}{\partial x_{j}}+\dfrac{\partial u_{j}}{\partial x_{i}}$}\quad \text{neglecting $\epsilon^{2}$ terms}
$$
\begin{align*}
x'_{i}\cdot x'_{j} &= \sum\limits_{kl}(\delta_{ik}+\epsilon_{ik})\widehat{x}_{k}(\delta_{jl}+\epsilon_{jl})\widehat{x}_{l}\\
&= \sum\limits_{k}(\delta_{ik}+\epsilon_{ik})(\delta_{jk}+\epsilon_{jk})\\
&= \epsilon_{ij}+\epsilon_{ji}\quad\text{[if $j$ and neglect $\epsilon^{2}$ terms]}
\end{align*}

%page 18
\noindent
{\bf Dilation :} Fractional increase of volume associated with deformation is called {\bf dilation}.
\begin{gather*}
V=\widehat{x}\cdot \widehat{y}\times \widehat{z}\quad V'=x'\cdot y'\times z'\\
= \left(
\begin{array}{ccc}
1+\epsilon_{11} & \epsilon_{12} & \epsilon_{13}\\
\epsilon_{21} & 1+\epsilon_{22} & \epsilon_{23}\\
\epsilon_{31} & \epsilon_{32} & 1+\epsilon_{33}
\end{array}
\right)\simeq 1+e_{11}+e_{22}+e_{33}
\end{gather*}
$\epsilon^{2}$-terms neglected

$\therefore$ Dilation $=\delta=\dfrac{V'-V}{V}=\dfrac{\Delta V}{V}\simeq e_{11}+e_{22}+e_{33}$.

\eject

\noindent
{\bf Stress :} The force acting on a unit area in the solid is called stress.

There are nine stress components $X^{i}_{x_{j}}\Rightarrow X^{i} \ \to$ direction of force, $x_{j} \ \to$ normal to the plane where force is applied.

$X_{Y}\Rightarrow$ force applied in $x$-direction on a plane whose normal is along $y$-direction.

Total no. of terms = $3\times 3=9$

$\to$ In a cube of homogeneous density

Stress is symmetric and reversible
$$
\fbox{$X^{i}_{x_{j}}=X^{j}_{x_{i}}$}\Rightarrow \fbox{$X_{Y}=Y_{x}$}
$$
$\therefore$ Total No. of stress component = Six
$$
\fbox{$X'=X, X^{2}=Y, X^{3}=Z$}
$$

Employing Hooke's law one can write
$$
{\displaystyle{\mathop{e_{ij}}\limits_{i\leq j}}}=\sum\limits_{\substack{k,l=1,3\\ k\leq l}}S_{ijkl}X^{k}_{x_{l}}
$$
$S\to$ {\em Elastic Complience Constant}

Dimension $\Rightarrow$ $\dfrac{\text{Area}}{\text{Force}}=\dfrac{\text{Volume}}{\text{Energy}}$ instead of $X_{z}$, we can write $Z_{x}$
$$
e_{11}=S_{1111}X_{x}+S_{1122}Y_{y}+S_{1133}Z_{2}+S_{1112}X_{y}+S_{1123}Y_{z}+S_{1113}Z_{x}
$$
Since there are six $e_{ij}$'s $\to$ total 36 components are there in $S$

$S \ \to$ Tensor of rank 4.

Similarly, one can write
$$
{\displaystyle{\mathop{X^{i}_{x_{j}}}\limits_{i\leq j; i_{j}=1,3}}}=\sum\limits_{\substack{k,l=1,3\\ k\leq l}}C_{ijkl}e_{kl}
$$
$C=$ Elastic stiffness constant.

\medskip

Dimension $\Rightarrow \ \dfrac{\text{Force}}{\text{Area}}$ or $\dfrac{\text{Energy}}{\text{Volume}}$
$$
X_{x}=C_{1111}e_{11}+C_{1122}e_{22}+C_{1133}e_{33}+C_{1112}e_{12}+C_{1123}e_{23}+C_{1113}e_{31}
$$
$C$ also has 36 components.

\section*{Elastic Energy Density}

Elastic energy density, $U$ can be expressed following the energy expression of a spring $(\frac{1}{2}kx^{2})$
$$
\fbox{$U=\frac{1}{2}\sum\limits_{\substack{ijkl=1,3\\ i\leq j,k\leq l}}$}\quad C\to \text{Tensor of rank 4}
$$
Stress components can be found by taking derivative of $U$ with respect to $e_{ij}$ such as \fbox{$F=-\nabla V$}.
\begin{align*}
\therefore\quad X_{x}=\dfrac{\partial U}{\partial e_{11}} &= \widetilde{C}_{1111}e_{11}+\frac{1}{2}\sum\limits_{\substack{k\leq l\\ k=l\neq 1}}\widetilde{C}_{11kl}e_{kl}+\frac{1}{2}\sum\limits_{\substack{i\leq j\\ i=j\pm i}}\widetilde{C}_{ij11}e_{ij}\\
&= \widetilde{C}_{1111}e_{11}+\frac{1}{2}\left[\widetilde{C}_{1122}e_{22}+\widetilde{C}_{1133}e_{33}\right.\\
&\quad +\widetilde{C}_{1112}e_{12}+\widetilde{C}_{1123}e_{23}+\widetilde{C}_{1131}e_{31}\\
&\quad +\widetilde{C}_{2211}e_{22}+\widetilde{C}_{3311}e_{33}+\widetilde{C}_{1211}e_{12}\\
&\quad \left.+\widetilde{C}_{2311}e_{23}+\widetilde{C}_{3111}e_{31}\right]
\end{align*}
\begin{align*}
U &=\frac{1}{2}\left[C_{11}e_{1}e_{1}+C_{12}e_{1}e_{2}+C_{13}e_{1}e_{3}+C_{14}e_{1}e_{4}+C_{15}e_{1}e_{5}+C_{16}e_{1}e_{6}\right.\\
\dfrac{\partial U}{\partial e_{1}} &= 2C_{11}+C_{12}e_{2}+e_{21}e_{2}+C_{13}e_{3}+C_{31}e_{3}+C_{14}e_{4}+C_{1}e_{4}\\
&\quad +C_{15}e_{5}+C_{51}e_{5}+C_{16}e_{6}+C_{61}c_{6}
\end{align*}
$$
a, \fbox{$X_{x}=\widehat{C}_{1111}e_{11}+\dfrac{1}{2}\sum\limits_{\substack{k\leq l\\ p=l\neq 1}}[C_{11kl}+C_{kl11}]e_{kl}$}
$$
Note : For off-diagonal elements only $\dfrac{1}{2}(\widetilde{C}_{11ke}+\widetilde{C}_{ke11})$ terms enters in Stress-Strain relation $\Rightarrow$ This follows, elastic stiffness constants are symmetrical.
$$
\Rightarrow C_{\alpha\beta}=\frac{1}{2}(\widehat{C}_{\alpha\beta}+\widetilde{C}_{\beta\alpha})=C_{\beta\alpha}
$$
$\therefore$ Among 36 elements of $C\to 6$ diagonal.

Among 30 off-diagonal elements, 15 are distinct.

$\therefore$ Total no. of distinct elements $=15+6=21$

** One can reduce the number further by applying symmetry operations.

\begin{example*}
Take the example of a cube - 
$$
\fbox{$1\to xx; \ 2\to yy; \ 3\to zz; \ 4\to yz; f\to zx; \ 6\to xy$}
$$
Since all axis are equivalent $C_{1111}=C_{2222}=C_{3333}\Rightarrow C_{11}$
$$
C_{1122}=C_{1133}=C_{2233}\Rightarrow C_{12}\ldots
$$
Cube has 4 \ 3-fold rotation symmetry with the body diagonals as rotation axis. Angle of rotation $=\dfrac{2\pi}{3}$
\begin{center}
\begin{tabular}{rl@{\qquad}rl}
(i) & $x^{1}\to y^{2}\to z^{3}\to x^{1}$ & (ii) & $-x\to z\to y\to -x$\\[3pt]
(iii) & $x\to z\to -y\to x$ & (iv) & $-x\to y\to z\to -x$
\end{tabular}
\end{center}
\begin{itemize}
\item[(i)] $\sum\limits^{3}_{i=1}e^{2}_{ii}$ remains unchanged under this rotation.

\item[(ii)] $\sum\limits_{i\neq j}e^{2}_{ij}$ also remains invariant under these rotations.

\item[(iii)] $\sum\limits_{i,j; ij}e_{ii}e_{jj}$ also remains unchanged.
\end{itemize}
\begin{align*}
e_{xy} &=\dfrac{\partial u}{\partial y}+\dfrac{\partial v}{\partial x}\\
e'_{xy} &=\dfrac{\partial u'}{\partial y'}+\dfrac{\partial u'}{\partial x'}=-\dfrac{\partial w}{\partial x}-\dfrac{\partial u}{\partial z}\\
&\quad \text{[under operation 3]}\\
&= -e_{zx}\equiv -e_{xy}
\end{align*}
$\therefore$ The terms $(e_{xx}e_{xy}+\cdots)$; $(e_{yz}e_{zx}+\cdots)$ $(e_{xx}e_{yz}+\cdots)$ are not invariant under these rotations.

$\therefore$ The surviving terms are
\begin{align*}
U &= \frac{1}{2}C_{11}(e^{2}_{xx}+e^{2}_{yy}+e^{2}_{zz})+\frac{1}{2}C_{44}(e^{2}_{yz}+e^{2}_{zx}+e^{2}_{xy})+C_{12}(e_{yy}e_{zz}+e_{zz}e_{xx}+e_{xx}e_{yy})\\
\therefore \ X_{x} &= \dfrac{\partial U}{\partial e_{xx}}=C_{11}e_{xx}+C_{12}(e_{yy}+e_{zz})\\
X_{y} &= \dfrac{\partial U}{\partial e_{xy}}=C_{44}e_{xy}
\end{align*}

\eject

$\therefore$ The elastic stiffness matrix for a cube reduces to
\begin{center}
\begin{tabular}{>{$}c<{$}|>{$}c<{$}>{$}c<{$}>{$}c<{$}>{$}c<{$}>{$}c<{$}>{$}c<{$}}
 & e^{(1)}_{xx} & e^{(2)}_{yy} & e^{(3)}_{zz} & e^{(4)}_{yz} & e^{(5)}_{zx} & e^{(6)}_{xy}\\
\hline
X_{x} & C_{11} & C_{12} & C_{12} & 0 & 0 & 0\\
Y_{y} & C_{12} & C_{11} & C_{12} & 0 & 0 & 0\\
Z_{z} & C_{12} & C_{12} & C_{11} & 0 & 0 & 0\\
Y_{z} & 0 & 0 & 0 & C_{44} & 0 & 0\\
Z_{x} & 0 & 0 & 0 & 0 & C_{44} & 0\\
X_{y} & 0 & 0 & 0 & 0 & 0 & C_{44}
\end{tabular}
\end{center}
The matrix is $(X_{i})=\sum\limits_{j}(C_{ij})(e_{j})$
$$
\therefore \ Y=C_{44}e_{yz}\Rightarrow e_{yz}=\frac{1}{C_{44}}Y_{z}=S_{44}Y_{z}\Rightarrow \fbox{$C_{44}=\dfrac{1}{S_{44}}$}
$$
Similarly one can find.
$$
C_{11}-C_{12}=(S_{11}-S_{12})^{-1}\quad C_{11}+2C_{12}=(S_{11}+2S_{12})^{-1}
$$
$S\to$ Compliance constant\qquad $C \ \to$ Stiffness Constant.
\end{example*}

\section*{Bulk, Modules and Compressibility}

For uniform dilation $e_{xx}=e_{yy}=e_{zz}=\dfrac{\delta}{3}$
$$
\therefore \ U=\dfrac{1}{6}(C_{11}+2C_{12})\delta^{2}
$$
If Bulk modulus is $B$ 
$$
U=\dfrac{1}{2}B\delta^{2}\Rightarrow \fbox{$-V\dfrac{dp}{dV}=B$}
$$
$\therefore \ $ For a cube \fbox{$B=\dfrac{1}{3}(C_{11}+2C_{12})$}

Compressibility $K$ is inverse of Bulk modules $\left(=\dfrac{1}{B}\right)$.
