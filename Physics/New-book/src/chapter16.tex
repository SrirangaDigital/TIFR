\chapter{}\label{lec16}

\section*{Atomic Radii}

Assume that $R_{C}$ and $R_{A}$ are the radii of cation and anion.

Then the bond length will be $=R_{C}+R_{A}$

But in reality a correction factor comes due to coordination of the site.

\begin{example*}
$BaTiO_{3} \ \to$ Perovskite structure,

$T_{i} \ \to$ Octahedra, $B_{a} \ \to$ Dodica hedra 12 nearest neighbors.
\begin{center}
{\bf Correction terms for diff. Coordination No. \boldmath$N$.}
\medskip

\begin{tabular}{cc|cc|cc}
\hline
$N$ & $\Delta_{N}(A^{0})$ & $N$ & $\Delta_{N}(A^{0})$ & $N$ & $\Delta_{N}(A^{0})$\\
1 & $-0.5$ & 5 & $-0.05$ & 9 & $+0.11$\\
2 & $-0.31$ & 6 & 0 & 10 & $+0.14$\\
3 & $-0.19$ & 7 & $+0.04$ & 11 & $+0.17$\\
4 & $-0.11$ & 8 & $+0.08$ & 12 & $+0.19$\\
\hline
\multicolumn{2}{c}{$R_{B^{2+}_{a}}=1.35A^{0}$} & \multicolumn{2}{c}{$R_{O^{2}}=1.4A^{0}$} & \multicolumn{2}{c}{$R_{Ti^{4+}}=0.68A^{0}$}\\
\hline
\end{tabular}
\end{center}

$\therefore \ D_{Ba-O}=R_{Ba^{2+}}+R_{O^{2}}+\Delta_{12}=(1.35+1.4+0.19)A^{0}=2.94A^{0}$

$\therefore \ $ Lattice constant = $2.94\times \sqrt{2}A^{0}=4.16A^{0}$
\begin{gather*}
D_{Ti-O}=R_{Ti^{4+}}+R_{O^{2-}}+\Delta_{6}=(0.68+1.4+0)A^{0}=2.08A^{0}\\
\therefore \ \text{Lattice Constant } =1.08\times 2 A^{0}=4.16A^{0}
\end{gather*}
Experimental lattice constant, $a=4.004A^{0}$, somewhat smaller than the estimated value, as the material is not purely ionic.

In fact, the covalency is quite strong in this material leading to significant reduction in lattice constant.

Covalency $\to$ Overlap of outer orbitals $\to$ Decrease in lattice constant.
\end{example*}

\section*{Elastic Strains}
\begin{itemize}
\item[(i)] Elastic waves have characteristic wavelength, $\lambda \gtrsim 10^{-6}$ cm Frequencies $<10^{11}$ to $10^{12}$ Hz.

\item[(ii)] Therefore, one can consider the solid as a homogeneous continuous media instead of a periodic array of atoms. This is a good approximation.

\item[(iii)] Implement {\em Hooke's law :} In an elastic solid, strain is directly proportional to the stress.

Elastic modules = Stress/Strain

\item[(iv)] This law applies for small strain only. If the strain is large, we consider the system is in {\em non-linear region.}

\item[(v)] Strains are so small, it is not necessary to distinguish differences between isothermal (constant temperature) and adiabatic (constant entropy) deformations. 

\item[(vi)] Such differences are usually small at temperature close to room temperature or lower.

\item[(vii)] Assume that the strain deforms the axis system, so that the definition of the object remains unchanged.
\end{itemize}

Lets assume that the orthogonal unit vectors, $\widehat{x}$, $\widehat{y}$, $\widehat{z}$ have undergone deformation to $\overline{x}'$, $\overline{y}'$, $\overline{z}'$ axes system $\to$ In uniform deformation, each unit cell will deform in the same way.
$$
\therefore\quad \overline{x}^{1}_{i}=\sum\limits_{j}(\delta_{ij}+\epsilon_{ij})\widehat{x}_{j}\quad i_{1}j=1,2,3\Rightarrow \widehat{x}, \widehat{y}, \widehat{z}
$$
$\epsilon_{ij}$ defines the deformation and are $\ll 1$ for small strain.

Note: $x_{i}$ are not unit vectors.
\begin{gather*}
\overline{x}'_{1}\cdot \overline{x}'_{1}=1+2\epsilon_{11}+\epsilon^{2}_{11}+\epsilon^{2}_{12}+\epsilon^{2}_{13}\\
x'_{i}=\sum(\delta_{ij}+\epsilon_{ij})\widehat{x}_{j}\quad x'_{i}-x'_{j}=\sum\limits_{kl}(\delta_{ik}+\epsilon_{ik})\widehat{x}_{k}\cdot (\delta_{jl}+\epsilon_{jl})\widehat{x}_{l}
\end{gather*}
Now, lets assume an atom is located at $\overrightarrow{r}=x\widehat{x}+y\widehat{y}+z\widehat{z}$.

After deformation it's position will be $r'=x\overline{x}'+y\overline{y}'+z\overline{z}'$

$\therefore$ The displacement, $\overrightarrow{R}=r'-r=x(\overline{x}'-\widehat{x})+y(\overline{y}'-\widehat{y})+z(\overline{z}'-\widehat{z})=\sum\limits_{i}x_{i}(\overline{x}'_{i}-\widehat{x}_{i})$

Now, $x'_{i}-\widehat{x}_{i}=\sum\limits_{j}\epsilon_{ij}\widehat{x}_{j}\Leftarrow x'_{i}=\sum\limits_{j}(\delta_{ij}+\epsilon_{ij})\widehat{x}_{j}$
$$
\therefore\quad \overrightarrow{R}=\sum\limits_{ij}x_{i}\epsilon_{ij}\widehat{x}_{j}=\sum\limits_{j}\left(\sum\limits_{i}x_{i}\epsilon_{ij}\right)\widehat{x}_{j}=\sum\limits_{j}u_{j}\widehat{x}_{j}
$$
where
$$
\fbox{$u_{j}=\sum\limits_{i}x_{i}\epsilon_{ij}$}\quad u_{1}=\epsilon_{11}x+\epsilon_{27}y+\epsilon_{31}z
$$
Using Taylor expansion for small strains, one can write \fbox{$\epsilon_{ij}=\dfrac{\partial u_{j}}{\partial x_{i}}$} neglecting higher order terms.

Following usual convention, one can define the deformations as
$$
e_{ii}=\epsilon_{ii}=\dfrac{\partial u_{i}}{\partial x_{i}}\quad e_{ij}=x'_{i}\cdot x'_{j}\simeq \epsilon_{ji}+\epsilon_{y}=\dfrac{\partial u_{i}}{\partial x_{j}}+\dfrac{\partial u_{j}}{\partial x_{i}}
$$
$$
\fbox{$e_{ii}=\dfrac{\partial u_{i}}{\partial x_{i}}$}\quad \fbox{$e_{ij_{i+j}}=\dfrac{\partial u_{i}}{\partial x_{j}}+\dfrac{\partial u_{j}}{\partial x_{i}}$}\quad \text{neglecting $\epsilon^{2}$ terms}
$$
\begin{align*}
x'_{i}\cdot x'_{j} &= \sum\limits_{kl}(\delta_{ik}+\epsilon_{ik})\widehat{x}_{k}(\delta_{jl}+\epsilon_{jl})\widehat{x}_{l}\\
&= \sum\limits_{k}(\delta_{ik}+\epsilon_{ik})(\delta_{jk}+\epsilon_{jk})\\
&= \epsilon_{ij}+\epsilon_{ji}\quad\text{[if $j$ and neglect $\epsilon^{2}$ terms]}
\end{align*}

%page 18
\noindent
{\bf Dilation :} Fractional increase of volume associated with deformation is called {\bf dilation}.

