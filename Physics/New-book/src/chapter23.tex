\chapter{}\label{lec23}

\section*{Equation of motion}

Group velocity, $v_{g}=\dfrac{dw}{dk}$\quad $\epsilon=\hbar w$
$$
\therefore\quad v_{g}=\dfrac{1}{\hbar}\dfrac{d\epsilon}{dk}\quad \text{or}\quad \fbox{$\overrightarrow{V}=\dfrac{1}{\hbar}\overrightarrow{\nabla}_{k}\in (k)$}
$$

If a force act on an electon, one can write equation of motion,
$$
F=\hbar \dfrac{dk}{dt}\quad p=\bhar k.
$$
$F$ can be Lorentz force $F=-eE-e\overrightarrow{v}\times \overrightarrow{B}$
$$
\therefore\quad \fbox{$\dfrac{dk}{dt}=-\dfrac{e}{\hbar^{2}}\overrightarrow{\nabla}_{k}\in \times \overrightarrow{B}$}
$$
in a magnetic field. $v=\dfrac{1}{\hbar}\overrightarrow{\nabla}_{k}\in$.

{\em Evidently, election moves on a surface of constant energy if a magnetic field is applied as $\left(\dfrac{dk}{dt}\right)$ is perpendicular to $(\overrightarrow{\nabla}_{k}\epsilon)$}.

\noindent
{\bf Holes:} Vacant state in an otherwise filled band. The properties of holes are as follows.
\begin{itemize}
\item[(i)] In a filled band $\sum k=0$ sum is over all states in the Brillouin zone.

If an electron of momentum $k_{e}$ is ejected, the momentum (total) becomes $-k_{e}$. This is attributed to the hole momentum, $k_{n}:\fbox{k_{n}=-k_{e}}$.

An electron of $k_{e}$ wave vector excited at $E$. Total system wave-vector is $-k_{e}$.
\begin{center}
{\bf Figure}
\end{center}

But the hole is situated $k_{e}$,

wave vector is graphically situated at $k_{e}$ ($H$ in Fig.) although $k_{n}=-k_{e}$.

\item[(ii)] $\epsilon_{h}(k_{h})=-\epsilon_{e}(k_{e})$

Zero of energy = Fermi energy $\to$ top of valence band. Removal of electron of energy $\epsilon_{e}(k_{e})$ increases the energy of the system by $\epsilon_{e}(k_{e})$ which is the hole energy.

\item[(iii)] Velocity of hole is same as velocity of missing electron.
\begin{gather*}
v_{h}=v_{e}\\
\fbox{$\nabla \epsilon_{h}(k_{n})=\nabla \epsilon_{e}(k_{e})$}\quad\fbox{$v_{h}(k_{h})=v_{e}(k_{e})$}
\end{gather*}

\item[(iv)] $m_{h}=-m_{e}$ effective mass is reversed in sign as hole moves in opposite direction of electron.

\item[(v)] Equation of motion:
$$
\hbar \dfrac{dk_{n}}{dt}=e\left(E+\dfrac{1}{C}v_{h}\times B\right)
$$
Equation of motion of a hole is similar to that of particle of positive charge, `$e$'.
$$
J=(-e)v(G)=(-e)[-v(E)]=ev(E)
$$
Current of a positive charge.
\end{itemize}

\section*{Effective mass}

%page 29


