\chapter{Information on Mathematical Education in Poland}

\begin{center}
{\em By}~ EDWARD MARCZEWSKI
\end{center}
\medskip

\blfootnote{This lecture was given at the South Asian Conference on Mathematical Education held on 22-28 February 1956 at the Tata Institute of Fundamental Research, Bombay.}
\setcounter{pageoriginal}{160}
I\pageoriginale do not intend to present here in detail the system of mathematical education in Poland. I do not even intend to consider the question whether such a system exists or not, and why. I intend only, in a ten minutes' communications, to give some information on mathematical education in Poland.

The scheme of education in Poland is as follows~:
\begin{center}
\begin{tabular}{lcll}
\multirow{2}{*}{General School} & \Bpara{5}{-2}{180}{15} & Fundamental classes : & 7 years\\[2pt]
 & & Lyceum classes : & 4 years\\[2pt]
University : & & & 5 years\\[2pt]
Aspiranture : & & & 3 years
\end{tabular}
\end{center}

The curriculum of the general schools contains elementary mathematics. It does not contain an introduction to calculus, as our experience has been against it. The coordinate method and its applications are included in the programme, but no systematica course on analytic geometry is included.

We have had to face in Polish schools several difficulties but we have had also some success. The number and quality of teachers is not yet adequate, the number of schools increases rapidly and the number of children increases still more rapidly. Some schools, especially in the country, are not sufficiently equipped. And so a fruitful movement is being developed in our schools : the construction of mathematical models and other aids by teachers and pupils for their own schools, and (by exchange) even for others.

Every\pageoriginale year, mathematical competitions, so called ``olympic competitions'', are organized for pupils of 10th and 11th classes by the Polish Mathematical Society under the sponsorship of the Ministry of Education. The tradition of such competitions is older in the Soviet Union and in Hungary. In Poland their scope is very wide and includes all schools having lyceum classes. In each university centre there exists a special regional committee besides a central committee in Warsaw. The competitions are conducted at three levels : 1. Local competitions---exercises to be solved at home. 2. Competitions at regional level. 3. Central competitions. Each year the central committee publishes all problems set during the competitions. (I have presented to the Tata Institute a little book containing the material of our fifth olympic competition.)

A serious effort is made by Polish mathematicians for popularizing mathematics, especially among the pupils of the last lyceum class. Each regional section of the Polish Mathematical Society organizes open lectures. These lectures are published in various forms, for instance, as articles in magazines or as separate papers. The new, much enlarged edition of the well-known book of Steinhaus, {\em Mathematical Snapshots} (under the Polish title {\em Kalejdoskope Matematyczny}) appeared last year.

The journal ``Mathematyka'' for teachers of mathematics and for able pupils is also a result of co-operation between school teachers and university professors. For instance, in its large and interesting section on problems the initials W. S. and H. St. can be found very often : they are Professors Sierpinski and steinhaus who take part systematically in this enterprise.

Let us consider now universities. Our university programme of mathematics is still under discussion. The programme of the first year is the only one which has remained unchanged, since its establishment a long time ago : it contains calculus, analytical geometry, higher algebra and experimental physics.

Apropos the following courses, it is necessary to emphasize that the character of mathematical universities in Poland after the 2nd\pageoriginale World War has been changed essentially to suit the needs of our country and our socialistic economy. First, the liberal system of studies, (which was established after the model of Austrian universities before the 1st World War) was replaced by systematic and compulsory studies, as in Soviet or American universities. And then in order to forge a strong link between mathematics, science in general and applications, various courses on higher analysis and applied mathematics have been introduced. These include also an introduction to set theory, topology, real functions, etc.

A great effort was made and is still being made to give to our university students modern textbooks in Polish on fundamental branches of mathematics. This aim has been almost achieved already.

There are some fundamental traditions of the Polish School of mathematics, which are being closely preserved and carefully developed in our universities : the union of teaching and research, the early beginning of scientific research by gifted students (and the opportunity for such a beginning), and the method of collective research. A great number of seminars common to advanced students, post-graduate fellows and scientific workers is the essential means for the realization of these aims. For instance, there are now in Wroclaw more than 20 seminars organized by the University or by the Mathematical Institute of the Polish Academy of Science. All these seminars are open to all mathematicians, including the students. We strongly think that any barrier between universities and research institutes should be avoided.

In all our educational work, we have had to face several difficulties and problems which have not been mentioned in this communication, because they are common to other countries, and have been discussed in preceding lectures. We have also had some success and the most important one is the emergence of a set of gifted and promising young mathematicians. We hope that they will essentially contribute to the development of our country and to the development of mathematics.

\bigskip
\medskip
{\fontsize{9pt}{11pt}\selectfont 
University of Wroclaw
}\relax
