\chapter{South Asian Conference on Mathematical Education}

\begin{center}
{\bf Bombay, 22-28 February 1956}
\bigskip

{\large\bf APPENDIX}
\medskip

{\bf MATHEMATICAL EDUCATION IN SCHOOLS}
\end{center}

\setcounter{pageoriginal}{172}
{\em Report\pageoriginale of a Committee consisting of Professor T. A. A. Broadbent (Chairman), Professor K. R. Gunjikar (co-Chairman), Professor Aung Hla, Mr. S. D. Manerikar, Dr. R. Naidu, Mr. Poerwadi Poerwadisastro, Mr. Rabil Sitasuwana and Miss H. K. Wong.}

\medskip
\begin{center}
\textsc{Suggested content of primary course}
\end{center}
\medskip

{\em General aim}~: Primary teaching is to be related not to the ideas of the teacher but to the needs and experience of the child.

\medskip

\noindent
{\em Topics}~:
\begin{itemize}
\item[i.] Numbers---counting, measuring; number combinations and simple operations. (The child might be encouraged to construct his own multiplication tables and use them till the number combinations become thoroughly familiar). Normally (i) will be covered in the first three years.

\item[ii.] Simple ideas about fractions and decimal notations.

\item[iii.] Elementary geometrical notions. The child should have familiarity with simple geometrical objects and patterns---tiles, simple models,---giving him {\em informal} ideas about congruence and symmetry. (No {\em formal} geometry is to be included).

These topics (ii) \&\ (iii) should be started not later than the 4th year of the primary course.

\item[iv.] The use of symbols as abbreviations should be developed during the last year of the course.

\item[v.] Pictorial representation of numerical data should be introduced at convenient stages.
\end{itemize}

\medskip
\begin{center}
\textsc{Mathematics\pageoriginale at the transition stage (one year immediately following the primary stage)}
\end{center}
\medskip
\begin{enumerate}
\item Familiarity with arithmetical computation including the use of decimals and fractions must be firmly established early in the first year of the secondary stage, along with their simplest applications related to the experiences of the children.

\item Algebra is to be introduced as generalized arithmetic.

\item Informal geometrical notions acquired in the primary stage should be gathered into a more precise pattern preparatory to formal geometry later in this stage. Graphic representation of familiar data, preferably collected by the child, should be introduced.
\end{enumerate}

\begin{center}
{\textsc{Mathematics at the later secondary stage}}
\medskip

{\textsc{(for all pupils)}}
\end{center}

This general course should be so designed that the special course intended for pupils with a special aptitude for mathematics should grow out of this course.
\begin{enumerate}
\item {\em Arithmetic.} Development of the operations already studied---intro\-duction of the idea of proportion---application to civic arithmetic\----use of algebraic language to express these ideas.

\item {\em Statistics.} Introduction to statistics should be made through the collection of appropriate data by the pupils themselves. Ideas of statistics should be carried far enough to enable the pupil to examine these and to appreciate the significance of the results.

\item {\em Algebra.} Algebra must gradually cease to be generalized arithmetic and become a discipline in its own right, a language which shall enable the pupil to solve more general problems in other fields of mathematical studies. The three laws of algebra and the four rules of operations must be thoroughly appreciated, and their application to elementary problems should be fully worked out.

\item {\em Geometry.} Geometry should begin to crystallise round certain key concepts and theorems. Pupils should be made to appreciate the fundamental logical pattern in geometry. The notion of similarity should\pageoriginale be introduced and linked up with the notion of proportion in arithmetic. Formal geometry must be subordinated to practical needs which will be met at this point by applications to measurement of areas, volumes, and general ideas of mensuration.
\end{enumerate}

\begin{center}
\textsc{Special course in mathematics (9th, 10th and 11th years).}
\end{center}

Besides the general course already mentioned, this should include the following :
\begin{enumerate}
\item {\em Algebra}~: Up to and including the binomial theorem for a positive integral index.

\item {\em Geometry}~: Plane geometry should be made a more systematic study and the logical connection between groups of key concepts and theorems should now be made more explicit. This idea of coordinates should be introduced and made to serve as a link between geometry and trigonometry. An introduction to solid geometry should be made.

\item {\em Calculus}~:  Calculus should not necessarily be attempted in this course, but an approach to the calculus through simple graphical notions or through simple kinematical ideas is desirable. Whether the approach to the calculus is made graphically or kinematically, the link with kinematical ideas should be made at an early stage.

\item {\em Statistics}~: To be worked up to a stage where the idea of standard deviation has been thoroughly grasped and illustrated by examples.
\end{enumerate}
