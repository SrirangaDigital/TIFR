\chapter[A Brief Account of the Present Situation of Mathematical...]{A Brief Account of the Present Situation of Mathematical Education in Chinese Universities}

\begin{center}
{\em By}~ H. F. TUAN
\end{center}
\medskip

\blfootnote{This lecture was given at the South Asian Conference on Mathematical Education held on 22-28 February 1956 at the Tata Institute of Fundamental Research, Bombay.}
\setcounter{pageoriginal}{164}
In\pageoriginale this short speech, I would like to talk exclusively about the situation of Chinese mathematical education at the university level. However, I would like to say at least a few words about the education in elementary schools of six years (beginning from the age of seven) and secondary schools of another six years. Our Government and our people are determined to wipe out completely illiteracy from China in twelve years. By literacy we include also the acquaintance with elementary arithmetic. This is a tremendous task, but we have confidence in carrying it through.

Ever since the founding of the Chinese People's Republic in 1949, especially since 1952, we have been carrying out, on a large scale, what is known as reforms in education.

In 1952, we re-organized our universities and colleges. At present, we have altogether thirteen universities with students in mathematics, of which only Peking University has its department of mathematics and mechanics, with two sub-divisions, mathematics and mechanics. Students in Peking University and a few other universities are to study for five years, while students in the rest will study for four years.

The number of students in mathematics has been greatly increased since 1952. Take for example Peking University. The number of such students in 1952 was only around 100 (being the total number of students in the three mathematical departments which were later incorporated into a single one), while now it is around 400, and next year it will be around 500. The number of post-graduate students\pageoriginale is not yet large, but it will be greatly increased in the not distant future.

The standard of new students has been greatly elevated since 1953. For example, in my department, among around 200 sophomore students, there are around 15 or 20 who are not satisfied with just doing the work of four regular courses of instruction for the first term, i.e. mathematical analysis, higher algebra, theoretical mechanics, differential equations. They seek to direct their excessive energies to read more advanced books (e.g. Natanson's {\em Theory of Functions of Real Variables}, Riesz-Nagy's {\em Lectures on Functional Analysis}) or to do research problems, perhaps, not too elementary (e.g. certain functional equations, infinite products of matrices).

In order to encourage high school students gifted in mathematics to study mathematics after their graduction, we have just started in Peking, what has been done in Hungary, Poland, and the Soviet Union for many years, the Olympic in Mathematics for high school students. Two days before I left Peking, I gave a lecture to around 1,000 of them on ``symmetry''. A few weeks earlier, Professor Hua Loo-keng lectures on ``Yang Hwei Triangles'' (they are usually known to the Western World as ``Pascal Triangles'', but Yang Hwei made the discovery much earlier in 1261).

Since 1952, it has been made quite clear that the aim of mathematical education at the university level is to train future research workers and teachers in secondary schools or higher institutions. The Teachers' Colleges, more in number, are to train exclusively teachers for secondary schools. For Peking University and a few other universities, after graduation, most of the students in mathematics will undertake research work or teach in higher institutions.

Since 1952, we have made several curricula for university education in mathematics and now we have had a more stable one. We are greatly benefited from the experience of other countries, especially from the Soviet Union. Of course, we have to make our own arrangements in order to suit our own needs.

I\pageoriginale would not go into details about courses of instruction. Just for example, I shall mention (theory of) ordinary differential equations, differential equations of mathematical physics, calculus of variations, and integral equations. For freshmen, we have three courses in mathematics, namely, mathematical analysis, analytic geometry and higher algebra. In some universities there will also be given a course on the history of mathematics, which will not only deal with the development of mathematics in the Western World, but will also put due emphasis on the role played by the ancient cultures of Egypt, India and China.

Through the courses in the first three years, we want to lay a solid foundation to the various important branches of mathematics and also neighbouring sciences like physics and mechanics. In this basis, the students are to get more training in a special branch, e.g. the theory of numbers, functional analysis, or else, through special courses and special seminars. Special courses and seminars to be given will differ from university to university, and will be mainly determined by the needs of our national construction, the development of mathematics, and the research work of professors and teachers.

Students will be required to write small papers in the third and fourth years, and to write a thesis in the fifth year. Of course, here again, the level of the thesis will differ from university to university, and even in the same university from student to student. However, we are going to raise the standard from year to year, so that in the not distant future a thesis to be acceptable must contain something new done by the student. In other words, the students are not just copying from textbooks or taking down what the teachers have said to them.

From what I said above, it is clear that we want to have a high standard for mathematical education in the universities. In fact, from the very aim of the university training---to train research workers and teachers, it is clear that teachers and especially professors themselves have to be research workers. We have well-known mathematicians,\pageoriginale but their numbers are far from enough, and the problem facing teachers and especially professors in the universities is to intensify their researches in mathematics.

\eject

We set for ourselves a high standard to strive for. Those universities with better staffs and better students will reach the goal very soon. Others may have some difficulties and may take a longer time in catching up. But all have to come up. We have a huge task, but we shall accomplish it.

\bigskip
\medskip
{\fontsize{9pt}{11pt}\selectfont
Peking University}\relax
