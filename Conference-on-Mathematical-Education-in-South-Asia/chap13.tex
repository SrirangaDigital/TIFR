\chapter{South Asian Conference on Mathematical Education}

\begin{center}
Bombay, 22-28 February 1956
\smallskip

RESOLUTION 1
\end{center}
\medskip

\setcounter{pageoriginal}{168}
\begin{enumerate}
\item The\pageoriginale Conference, having taken into account the existing practices in South Asia, dealt with the problems of mathematical education at three levels, undergraduate, graduate and post-graduate, defined as follows : undergraduate education covers primary and secondary school education as well as the Intermediate stage, irrespective of whether some or all of the Intermediate instruction is given at college; graduate education covers the Bachelor's degree, the Master's degree, and their variants; post-graduate education covers all education beyond the graduate stage; and resolved to adopt the proposals detailed under the headings below.

\item \textsc{Purposes.} The purpose of teaching at the undergraduate level should be utilitarian. The instruction should be related closely and continuously to the needs, the capacity, and the interests of the pupil.

The purpose of teaching at the level of the Bachelor's degree should be to meet the requirements of society in general, and to provide training for teachers of mathematics in secondary schools.

The purpose of teaching at the level of the Master's degree should be to provide training for professional work in the mathematical sciences, including the work of teaching students for the first degree. 

The purpose of teaching at the post-graduate level should be to train students for research, and to fit them for the teaching of advanced mathematics, at least up to the second degree.

\item \textsc{Subjects of study.} By the end of the primary course, the child should understand simple ideas about numbers and spatial relations.


At\pageoriginale the secondary stage, an integrated and simplified course should be taught comprising arithmetic, algebra, and geometry, with the possible addition of simple and essential statistical notions. It is contemplated that special instruction should be offered to those who require it.

Intermediate instruction should include an introduction to analytical geometry, calculus and trigonometry.

Required subjects of study for the Bachelor's degree should be analytical geometry, calculus, algebra, and mechanics. Optional subjects might include numerical methods, principles of statistics, elements of mathematical logic, and higher geometry.

Required subjects of study for the Master's degree should be: real and complex analysis; modern algebra; differential geometry of curves and surfaces; elements of mathematical statistics; methods of mathematical physics; mechanics of continuous media. A variety of optional subjects might be provided in accordance with local conditions.

The subjects of study at the post-graduate stage should include the following : real function theory, including Lebesgue integration, measure theory, probability; complex function theory including Riemann's mapping theorem; modern algebra through Galois theory; theory of topological spaces leading to the study of compact Hausdorff spaces, including Tychonoff's theorem, Urysohn's lemma, and Tietze's extension theorem; affine and projective geometry in connexion with algebra; differential geometry.

The formal course-work at this stage should not exceed two years.

It is not contemplated that this course should culminate in a formal examination.

\item \textsc{Teachers.} Teachers of mathematics in primary schools should have a knowledge of mathematics equivalent to that required for the school-leaving examination, and some special training in teaching. 

Teachers of mathematics in secondary schools should have a degree, with mathematics as a principal subject, and some special training in teaching.

An\pageoriginale appropriate proportion of inspectors of schools, both primary and secondary, should have had experience in the teaching of mathematics.

Teachers of mathematics at the level of the first degree should hold a higher degree with mathematics as a principal subject.

Teachers of mathematics at the level of the second degree should have pursued a course of post-graduate study in mathematics.

Teachers of mathematics at the post-graduate level should be mathematicians with significant experience in research and an adequate background of mathematical knowledge. Teaching duties at the graduate and post-graduate levels should be light enough for the efficient discharge of the teacher's primary obligation to pursue advanced study and research. There is a definite gain in teachers at the post-graduate level participating in graduate instruction.

\item \textsc{Examinations.} The Conference considers that the nature of the examination system has such a powerful influence on the work of the student, and on the character of the teaching, that special attention must be given to its design. The system should give a proper shape and direction to the flow of students through the entire range of the curriculum.

A detailed study of this problem should take into account the following points. Injustice should not be done to the student by compelling him to stake his career on the results of a single examination. The timing of the examinations should be closely related to the different stages of the curriculum. Written examinations should be supplemented by oral examinations, and an evaluation of the student's total performance. At the graduate level, teachers should participate in the examination of their own students.

\item \textsc{Research contracts.} A system of research contracts should be instituted by means of which mathematicians can be supported for limited periods of advanced study or research. The award of such contracts should be based on the scientific recommendations made by qualified referees.

\item \textsc{Summer schools and seminars.}\pageoriginale With the purpose of improving the quality of teaching and research in the field of mathematics, encouragement should be given to the organization of a limited number of summer schools and seminars adapted to the needs of teachers at all levels, students at the post-graduate level, and research workers.

\item \textsc{Scholarships.} Scholarships at the under-graduate and graduate levels should be considered in the context of financial support for students in general. An important aspect of this problem, which deserves special attention, is the need to encourage able students to advance to the post-graduate stage. Post-graduate students should receive generous financial assistance either for advanced study or for research.

\item \textsc{Text-books.} The preparation of suitable text-books is an urgent problem confronting the whole of South-Asia. Governments can help in solving it by creating and financing regional text-book committees. These committees should be composed of mathematicians, and empowered to seek out and induce competent authors to write such books. Publication and prescription of text-books should be the function of independent agencies, separate from the regional text-book committees.
\end{enumerate}

\smallskip
\begin{center}
RESOLUTION 2
\end{center}
\smallskip

The Conference resolved to set up a Committee for Mathematics in South Asia under the chairmanship of the President, Professor K. Chandrasekharan, with the following members : the Secretary of the International Mathematical Union, an expert mathematician to be nominated jointly by the Chairman and the Secretary of the I.M.U., and representatives of the Governments of Burma, Ceylon, India, Indonesia, Malaya-Singapore, Pakistan, and Thailand. The Conference hereby authorizes its President to take all necessary steps for the prompt constitution of the Committee.


