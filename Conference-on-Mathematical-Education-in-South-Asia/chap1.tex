\chapter{South Asian Conference on Mathematical Education}

\begin{center}
{\bf BOMBAY, 22-28 FEBRUARY 1956}
\medskip

{\bf PRESIDENTIAL ADDRESS}
\medskip

{{\em By}~ K. CHANDRASEKHARAN}
\end{center}

\setcounter{pageoriginal}{22}
I\pageoriginale AM keenly sensible of the great honour and responsibility that vest in the presidentship of this Conference. It is hard to find a precedent in the history of Asia for a mathematical gathering of this magnitude and importance. We are particularly grateful to the institutions which have sponsored this Conference---Unesco, the International Mathematical Union, the Ministry of Natural Resources and Scientific Research of the Government of India, the Sir Dorabji Tata Trust, and the Tata Institute of Fundamental Research. I hope that the Conference will prove worthy of their trust.

This conference is in some sense a counterpart of the International Colloquium on Zeta Functions which ended yesterday. In organizing these two meetings in conjunctions, we in the Tata Institute of Fundamental Research wish to affirm our belief in the interdependence of mathematical research and mathematical education. While our primary aim is mathematical research, we wish to do whatever we can to promote the development of a sound system of mathematical education in India.

There have been, in recent years, many conferences and congresses in this part of the world, dealing with educational problems in general, but none wholly devoted to the particular problems of mathematical education. While it is essential that the countries of South Asia should first determine for themselves what educational principles and policies they wish to follow, in the large, it is equally essential that they do not stop with generalities, but translate those policies into action in every branch of education. The purpose of this\pageoriginale Conference is to assist in that task, as far as mathematics is concerned. We here wish to draw on the experience of the mathematically more advanced countries in the West, and to learn from their successes and failures. While we do not wish to imitate them in every respect, we do want to emulate their example in their unremitting and insistent pursuit of mathematical research and the creative assimilation of mathematical knowledge. The tremendous edifice of mathematics, is, after all, our common inheritance. We all wish to do our bit to enlarge it, to enrich it, to make it more beautiful, and thereby to contribute our share to the building up of a civilization. It seems to me that the main problem of mathematical education consists in the transmission to the young of the intellectual excitement implicit in this endeavour, and the inculcation in them of an appreciation of its significance. In this process it is clear that research and education go hand in hand, and that no one who has not experienced the thrill of mathematical discovery in some sense, or gained a perspective of the vast expanse of mathematics at some angle, can possibly inspire the young to become good mathematicians. The problems peculiar to South Asia stem from the fact that research and education have not progressed side by side, but have been tied up in a vicious circle. An antiquated system of education is at work, which has greatly reduced the research potential, so that the motive force for reform never gets to be strong enough. We are at least fifty years behind the times; what is worse, some of us hardly notice that fact. Any enduring improvement of the system requires drastic changes at all levels, from the school to the university. But we have to begin somewhere, and it seems to me that a beginning made at the university stage will accelerate reform all the way down.

We, in India, have heard a great deal said about our universities. We have had many panegyrics on their functions, from all and sundry. To repeat them now would be to get diverted from our main business. But one thing perhaps deserves to be said here, and it is this. The destiny of any people can be fulfilled only by the putting forth of the best that they are capable of in the intellectual sphere.\pageoriginale The creative intellect is the master key to scientific and industrial progress. Human knowledge is not something static, given once for all, but something which grows, and gets transmuted, with every new intellectual achievement. It is the universities of a country that ought to be the centres of its intellectual leadership. If they neglect the function of nurturing creative talent, and of giving it the fullest opportunities for growth and fulfilment, they betray their primary responsibility. The low standards of universities in South Asia are, in my opinion, largely due to a lack of recognition of this basic fact. One hears about special programmes for athletics, for military training, for the civil service, and so on---each of which should of course find its proper place---but one rarely hears of any organized, properly directed, large-scale effort for setting up schools of study and research of an international standard. The appeal exercised on some of our students by the universities of Oxford and Cambridge, or of Paris and G\"ottingen, or of Harvard and Princeton, is a genuine one, not so much because they are places in which every member of the staff is a superior scientist, as because they represent an ideal in action, the ideal of intellectual aspiration and achievement.

While the universities should be the mainspring of all research, it seems to me that specialized research institutes are necessary for concentrated activity in any given science. They are especially important for us in South Asia where the chief difficulty consists in providing a strong initial momentum. Such institutes have, of necessity, to be very limited in number. Although designed to supply national needs, they should be truly international in spirit in order to be effective, for in mathematics, perhaps more than in any other science, international standards are the only acceptable ones. As far as I know, there is but one such permanently established institute in the whole of South Asia at present, devoted to doctoral and post-doctoral mathematical research. I cannot say that this is an ideal state of affairs.

While the cultivation of mathematics, either in the sense of making new discoveries, or of assimilating known theories, is an exciting\pageoriginale form of intellectual activity, it should not be forgotten that mathematics has amply demonstrated its utility and power in its interaction with other sciences, be they physical, social or biological. In fact, the dominant characteristic of modern scientific though may truly be described as mathematical. The nations of South Asia which are bent on industrial and technological progress cannot afford to be mathematically stagnant. We cannot expect first-rate technology to grow up beside third-rate mathematics. The attainment of a respectable level of mathematical culture should therefore be set as an immediate goal by the countries of South Asia.

The distance that separates us from that goal is great, and the obstacles are many. But we can attain it within a reasonable period of time, say thirty years, if we quicken our pace and pursue the right lines of progress. As far as India is concerned, our difficulties stem from an inadequate recognition of the value of creative intellectual activity as a part of the national drive towards prosperity. The creative scientist has not come into his own, though perhaps the administrative scientist has. This can be rectified only by the assumption of scientific leadership by the universities. Since that will take some years to happen, I think that some immediate, though temporary, remedies should be devised.

First, it is necessary to conserve the available research talent like a precious cargo. Gifted research workers in mathematics, of established merit, require a suitable atmosphere and adequate financial support in order to continue their work. Appointments in our mushrooming colleges cannot supply them with either of these needs. Nor can the standard fellowships and studentships meet their requirements, since they involve supervision and bureaucratic control. The proper solution, in the present situation, might be a system of research contracts set up by the Government and administered, for instance, by a National Committee for Mathematics, by means of which individual mathematicians can work for limited periods in surroundings of their choice, on a project offered by them and approved by the administering authority. The approval should be\pageoriginale based on scientific criteria determined by a pool of referees. Such contracts could be entered into not only by individuals and the Government, but by academic institutions and the Government. They can be made to cover not only research work, but also other types of activity like the writing of advanced monographs and treatises. It is only by the opening of such direct channels of assistance from the Government to the research workers that the disintegration of mathematical talent can, at present, be prevented.

Secondly, it is necessary to increase the facilities for the training of students in advanced mathematics. Mathematical research of good quality requires special preparation, particularly when the gap between university courses and active research is as wide as it is in India. The number of advanced mathematical theories which are not studied in any Indian university far out-number those that are touched at all. Modern algebra, and algebraic topology, are, for instance, considered as radical influences or expensive vices. Mathematics as it is taught or learnt in our universities is a ghost of the dead past. It is therefore necessary, as a first step, to set up graduate schools for advanced study, with plenty of studentships, rather than pretentious and anaemic research institutes. These schools, if they are worth the name, will automatically become centres of research. We cannot hope to reap a rich mathematical harvest without having done any sowing.

Thirdly, the courses of study need to be integrated and the demoralizing influence of our system of examinations eliminated. Both the form and the content of these examinations need drastic revision. In India this problem is connected with the fact that university examinations serve as an entrance to the prized civil service examinations, in which mathematics can hardly be distinguished from a certain form of trickery. The mathematical papers set for the civil service examinations should be modernized, so that no practical reason could exist for the continuance of the present system in the universities. Until this is done, it becomes our duty to make a special provision for those students who wish to become professional mathematicians; an alternative system of courses and\pageoriginale examinations should be provided for them, at least from the degree stage. Such an alternative system should ensure that one who has taken the Master's degree has a sound knowledge of at least the fundamentals of analysis, algebra, geometry, and topology, together with some of their applications. It is also essential that the class-work of students receives its just share of credit alongside their performance at a formal examination.

\newpage

Fourthly, the teaching staff in colleges requires rehabilitation and reinforcement if it is properly to discharge its new responsibilities. It is my opinion that the existing staff needs more leisure, and more encouragement, and more study, to cope with the problem of keeping itself up-to-date, alert to the changing aspects of our science. I feel that it has been doing a relatively fine job of teaching the Intermediate students, and criticism really begins at the graduate level. This can mean that teachers require greater opportunities for study, without financial loss,---something which can be met by the system of contracts which I have outlined, or by summer schools. A summer school for mathematics, on a small scale, was organized in Bombay several times in the past. Our colleagues in Ceylon have proposed that this should be enlarged and adapted to the needs of teachers as well as research workers. I know that the authorities of the Tata Institute of Fundamental Research are in favour of such a project, and I hope that in a year or two it will become an established fact. While the needs of the existing staff can thus be met, the principal source for fresh recruitment of staff should be the schools for advanced study and research which I have advocated. Raw graduates, who have not learnt to appreciate the significance of mathematics, should, in any case, be prevented from turning overnight into teachers of advanced students.

Fifthly, we must recognize the urgent need for suitable textbooks. Good textbooks can be a great help to a student, especially if he is not situated in a good atmosphere. They are rather scarce in this country, and, I believe, in all of South Asia. But they have to be written; they cannot be willed into existence. The importing, or\pageoriginale reprinting, of books from abroad can only be a temporary solution. It is my belief that there are quite a few mathematicians in India who can write suitable textbooks at least up to the M.A. standard. But they need an inducement to write the books, they need protection against the financial loss that might be involved in their publication, and they are entitled to a share of the profit, if profit there is, at any rate, so long as we do not have a truly socialistic society ! It is unrealistic to expect commercial publishers to come to our aid, because of the financial hazards that they foresee in such an enterprise. It seems necessary, therefore, to set up a National Textbook Committee, equipped with adequate funds, which will approach competent authors and induce them, on a contractual basis, to write textbooks for a certain remuneration. If the books are published, the authors will, in addition, receive royalties. In no case will they suffer any loss. This Committee need not necessarily be different from the Committee for Research Contracts which I have previously described. But it is important that the authority which has the power to prescribe textbooks is not the agency which launches their publication.

Finally, none of the reforms we think of at the university stage can be fully effective unless the foundations are properly laid at the school stage. Mathematical instruction in schools in South Asia, at any rate in India, has remained unchanged for decades. Srinivasa Ramanujan is perhaps the most famous victim of the inefficient and inelastic system under which we operate. It is our duty to change it. In so doing we must keep in mind the aim of free and compulsory primary education for all children which almost all countries of South Asia have set themselves. We should evolve a system which, on the one hand, does not make too severe a demand on those students who do not intend to proceed to the university, and, on the other, gives every student some basic mathematical knowledge.

It is possible that my remarks are largely inspired by the state of mathematics in India. I assume, however, that many of our problems are by no means peculiar to us. I hope that some of the suggestions\pageoriginale I have made may prove worthy of consideration by the Conference. Profound changes in the system of mathematical education in South Asia cannot suddenly result from a single Conference like this, nor the need for change ever disappear. But this Conference can serve as a good starting point. May it succeed in that purpose. May the countries of South Asia go forward together in their pursuit of mathematics. May it be given to us to function as an effective component of the world community of mathematicians.

\bigskip
\medskip

{\fontsize{9pt}{11pt}\selectfont
Tata Institute of Fundamental Research

Bombay
}\relax

