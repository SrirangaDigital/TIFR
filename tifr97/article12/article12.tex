\title{THE NUMBER OF RATIONAL APPROXIMATIONS TO ALGEBRAIC NUMBERS AND THE NUMBER OF SOLUTIONS OF NORM FORM EQUATIONS}
\markright{The Number of Rational Approximations to Algebraic Numbers and the Number of Solutions of Norm Form Equations}

\author{By~ Wolfgang M. Schmidt}
\markboth{Wolfgang M. Schmidt}{The Number of Rational Approximations to Algebraic Numbers and the Number of Solutions of Norm Form Equations}

\date{}
\maketitle

\setcounter{pageoriginal}{194}
\textsc{Recently, after i}\pageoriginale had lectured on Norm Form equations \cite{art12-key11}, Schinzel said : ``But now can it be, how can it be in number theory, that one could possibly prove the finiteness of a set of natural numbers, but obtain no estimate of its cardinality ?'' The next day, he himself provided the following answer : Suppose we can prove for a set $S$ of natural numbers that for any $x$, $x'$ in $S$ we have $x'\leq 2x$. Then $S$ is finite, but unless we can find a particular $x\in S$, we cannot estimate the cardinality of $S$, and even less can be provide a bound for the size of elements in $S$.

More generally, for $C>1$, call a set $S$ of positive integers a $C$-set, if $x'\leq Cx$ for any $x$, $x'\in S$. What we said above applies more generally to any $C$-set. On the other hand, we define a $\lambda$-set where $\lambda>1$ to be a set $S$ with the following Gap Principle : When $x$, $x'$ are in $S$ with $x'>x$, then $x'\geq \lambda x$. A $(C,\lambda)$-set is both a $C$-set and a $\lambda$-set. Let $x_{0}<x_{1}<\ldots<x_{v}$ be elements of a $(C,\lambda)$-set $S$. Then $x_{v}\leq Cx_{0}$ and $x_{i}\geq \lambda x_{i-1}(i=1,\ldots,v)$, so that $x_{v}\geq \lambda^{v}x_{0}$. Therefore $\lambda^{v}\leq C$, and $v\leq (\log C)/(\log \lambda)$, so that $S$ has cardinality
$$
|S|\leq 1+(\log C)/(\log \lambda).
$$
In this argument, we did not need to assume that $S$ consists of integers.

A situation very much like this occurs in the Thue-Siegel-Roth Theorem. Let me begin with Thue's Theorem. It asserts that when $\alpha$ is algebraic of degree $r\geq 3$, and if $\mu>(r/2)+1$, then there are only finitely many rationals $x/y$ with $|\alpha-(x/y)|<|y|^{-\mu}$. Let $Y$ be the set of positive $y$ such that there is a reduced fraction $x/y$ which satisfies this inequality; then $Y$ is finite according to Thue. It is well known that Thue's Theorem (and subsequent theorems of Siegel, Roth, Schmidt) are ``ineffective'' in the sense that they do not provide an upper bound for the (size of) elements of $Y$. An analysis of Thue's proof shows that it yields an explicit constant $B=B(\alpha,\mu)$, such that if $S_{1}$ is the set of $y\in Y$ with $y<B(\alpha,\mu)$, the so-called ``small solutions'', and if $S_{2}$ is the set of $y\in Y$ with $y\geq B(\alpha,\mu)$, the so-called ``large solutions'', then the following holds. First, it is trivial that the cardinality of $S_{1}$ does not exceed the explicit bound $B(\alpha,\mu)$. Second, there is an explicit $C=C(\alpha,\mu)$ such that $S_{2}$ is an {\em exponential $C$-set} in the sense that
$$
y'\leq y^{C}
$$\pageoriginale
for any $y$, $y'$ in $S_{2}$. This shows that $S_{2}$ (and hence $Y$) is finite, but gives no information on the cardinality. However, it turns out that there is also an exponential Gap Principle. Suppose $y'>y$ lie in $S_{2}$ and $|\alpha-(x/y)|<y^{-\mu}$, $|\alpha-(x'/y')|<{y'}^{-\mu}$. Since $Y$ was defined in terms of reduced fractions, we have $x/y\neq x'/y'$ and
$$
\frac{1}{yy'}\leq \left|\frac{x}{y}-\frac{x'}{y'}\right|\leq \left|\alpha-\frac{x}{y}\right|+\left|\alpha-\frac{x'}{y'}\right|<y^{-\mu}+{y'}^{-\mu}<2y^{-\mu},
$$
so that $y'>\frac{1}{2}y^{\mu-1}$. Now if $1<\lambda<\mu-1$, say if $\lambda=\mu/2$ and if $y\geq B(\alpha,\mu)$ where $B(\alpha,\mu)$ was chosen large enough, then we have
$$
y'>y^{\lambda},
$$
i.e., an exponential Gap Principle. Thus $S_{2}$ is an exponential $(C,\lambda)$-set, and its cardinality may be explicitly bounded by $1+(\log C)/(\log \lambda)$. But note that even though we can estimate the cardinality of a $(C,\lambda)$-set, we cannot estimate the size of its elements, and hence Thue's Theorem remains ineffective.

The situation is similar for Siegel's Theorem, where the condition $\mu>(r/2)+1$ is relaxed to $\mu>2\sqrt{r}$, and for the Dyson-Gelfond Theorem, with the condition $\mu>\sqrt{2r}$.

Roth finally relaxed the condition to $\mu>2$. His Theorem says that for $\alpha$ as above and $\delta>0$, there are only finitely many rationals $x/y$ with $|\alpha-(x/y)|<|y|^{-2-\delta}$. Again we can form the set $Y$ of denominators and distinguish small solutions $y<B(\alpha,\delta)$ and large solutions $y\geq B(\alpha,\delta)$. This time we cannot assert that the large solutions form an exponential $C$-set. But there are explicit $C=C(\alpha,\delta)$ and $m=m(\alpha,\delta)$ where $m$ is an integer, such that the large solutions are the union of $m$ exponential $C$-sets, hence $(C,\lambda)$-sets. This again gives a bound for the cardinality.

Explicit bounds for the number of solutions of $|\alpha-(x/y)|<y^{-2-\delta}$ with $y>0$ were given by Davenport and Roth \cite{art12-key3}. More recently, Bombieri and Van der Poorten \cite{art12-key2} came up with better bounds by using a new theorem of Esnault and Viehweg \cite{art12-key4} in place of ``Roth's Lemma''. They considered the slightly stronger inequality.
$$
\left|\alpha-\frac{x}{y}\right|<\frac{1}{64y^{2+\delta}}
$$
and showed that the number of solutions $x/y$ in reduced form with $y>0$ is 
\begin{equation}
\leq \frac{\log\log 4H}{\log (1+\delta)}+3000\frac{(\log r)^{2}\log(50\delta^{-2}\log r)}{\delta^{5}},\label{art12-eq1}
\end{equation}\pageoriginale 
provided that $0<\delta<\delta_{0}$ with some absolute $\delta_{0}$. Here $H=H(\alpha)$ is the height of $\alpha$ (related to the naive height, which is the maximum modulus of the coefficients of the minimal defining polynomial of $\alpha$ over $\mathbb{Z}$).

The first summand in \eqref{art12-eq1} comes essentially from the small solutions, the second summand from the large solutions. Although initially the small solutions appeared to be more tractable, it turns out that they are responsible for the dependency of $H$ in \eqref{art12-eq1}. In fact, the first summand in \eqref{art12-eq1} is best possible (See, e.g. \cite{art12-key8}.)

Now let us turn to simultaneous approximation. Some years ago, I proved the following \cite{art12-key9} : Suppose $\alpha_{1},\ldots,\alpha_{n}$ are algebraic, with $1$, $\alpha_{1},\ldots,\alpha_{n}$ linearly independent over $\mathbb{O}$. Then there are only finitely many rational points $(x_{1}/y,\ldots,x_{n}/y)$ with $y>0$ and
\begin{equation}
|\alpha_{i}-(x_{i}/y)|<y^{-1-(1/n)-\delta}(i=1,\ldots,n)\label{art12-eq2}
\end{equation}
for given $\delta>0$. Here, when $n>1$, we cannot at present estimate the number of solutions.

Following the method in \cite{art12-key11}, we can try to find explicit $B$, $C$, $m$ depending only on $\alpha_{1},\ldots,\alpha_{n},\delta$, such that the numbers $y>B$ occurring in solutions of \eqref{art12-eq2} (the ``large solutions'') constitute not more than $m$ exponential $C$-sets. The real difficulty is with the Gap Principle.

Write $\alpha=(\alpha_{1},\ldots,\alpha_{n})$, $x=(x_{1}/y,\ldots,x_{n}/y)$, and write \eqref{art12-eq2} as
\begin{equation}
\overbracket{abc} < y^{-1-(1/n)-\delta},\label{art12-eq3}
\end{equation}
where $\overbracket{\qquad}$ denotes the maximum norm. Now let $x_{0},\ldots,x_{n}$ be solutions of \eqref{art12-eq3} with $y_{0}\leq y_{1}\leq \ldots\leq y_{n}$ where $y_{i}=y(x_{i})(i=0,\ldots,n)$. Writing $x_{i}=(x_{i1}/y_{i},\ldots,x_{in}/y_{i})$, and assuming the determinant is not zero, we have
\begin{align*}
\frac{1}{y_{0}y_{1}\ldots y_{n}} &\leq \left|\left| 
\begin{array}{c}
1\frac{x_{01}}{y_{0}}\ldots\frac{x_{0n}}{y_{0}}\\
~\ldots\\
1\frac{x_{n1}}{y_{n}}\ldots\frac{x_{nn}}{y_{n}}
\end{array}
\right|\right|
=
\left|\left|
\begin{array}{c}
1\frac{x_{01}}{y_{0}}-\alpha_{1}\ldots \frac{x_{0n}}{y_{0}}-\alpha_{n}\\
~\ldots\\
1\frac{x_{n1}}{y_{n}}-\alpha_{1}\ldots \frac{x_{nn}}{y_{n}}-\alpha_{n}
\end{array}\right|\right|\\
&\leq (n+1)!(y_{0}y_{1}\ldots y_{n-1})^{-1-(1/n)-\delta}.
\end{align*}
Therefore
$$
y_{n}>\frac{1}{(n+1)!}(y_{0}y_{1}\ldots y_{n-1})^{(1/n)+\delta}\geq \frac{1}{(n+1)!}y_{0}^{1+n\delta}
$$
With\pageoriginale $\lambda=1+(n\delta/2)$ and $y_{0}\geq B$ large, this leads to $y_{n}>y^{\lambda}_{0}$. Although this would give a Gap Principle involving only every $n^{\text{th}}$ term in a sequence $x_{0}$, $x_{1},\ldots$ of approximations, it would be very useful in estimating the number of such approximations.

Where is the catch ? The catch lies in the assumption that the above determinant is not zero. The determinant will be zero precisely when $x_{0},x_{1},\ldots,x_{n}$ (with are in $\mathbb{R}^{n}$) lie in an $(n-1)$-dimensional linear submanifold of $\mathbb{R}^{n}$. Henceforth we shall call such a submanifold a {\em hyperplane}. E.g., when $n=2$, the determinant will be zero when $x_{0},x_{1},x_{2}$ lie on a line.

So we cannot really estimate the number of solutions $x$ of \eqref{art12-eq3}. What we can do is the following (see \cite{art12-key12}). We can give an explicit $t=t(\alpha,\delta)$, such that the solutions $x$ of \eqref{art12-eq3} lie in a collection of $t$ hyperplanes.

The following argument shows why it is unlikely that we will soon be able to estimate the number of solution $x$ of \eqref{art12-eq3} when $n>1$. Given $Q>1$, the inequalities
$$
|u|\leq Q, |\alpha_{i}u-v_{i}|\leq Q^{-1/(n-1)}\qquad (i=1,\ldots,n-1)
$$
define a parallelepiped $\Pi$ of volume $2^{n}$ in the space of vectors $u=(v,u_{1},\ldots u_{n-1})$. By Minkowski, there is a nonzero integer point $u$ in $\Pi$. In fact, for given $\epsilon>0$ and for large $Q$, the $n^{\text{th}}$ minimum $\mu_{n}=\mu_{n}(Q)$ of $\Pi$ has $\mu_{n}<Q^{\epsilon}$ (See \cite{art12-key9}. But we don't know how large $Q$ has to be). For such $Q$ there are $n$ independent integer points $\bu_{1},\ldots,\bu_{n}$ in the blown up parallelepiped $Q^{\epsilon}\Pi$. The linear combinations $c_{1}\bu_{1}+\cdots+c_{n}\bu_{n}$ with $|c_{1}|+\cdots+|c_{n}|\leq Q^{\epsilon}$ lie in $Q^{2\epsilon}\Pi$. Thus for large $Q$, there are $\geq c(n)Q^{n\epsilon}>Q^{\epsilon}$ nonproportional integer points in $Q^{2\epsilon}\Pi$.

Suppose now that
$$
0<\delta <\frac{1}{n(n-1)}
$$
and let $\epsilon$ have $\delta+7\epsilon<1/(n(n+1))$. Suppose we have a rational hyperplane $\mathfrak{H}$ which comes very close to $\alpha=(\alpha_{1},\ldots,\alpha_{n})$. Say $\mathfrak{H}$ is given by
$$
a_{0}+a_{1}X_{1}+\cdots+\cdots+a_{n}X_{n}=0,
$$
with coprime integers $a_{0},a_{1},\ldots,a_{n}$. Suppose $|a_{n}|=\max (|a_{1}|,\ldots,|a_{n}|)=a$, say and $\mathfrak{H}$ is so close to $\alpha$ that
\begin{equation}
|a_{0}+a_{1}\alpha_{1}+\cdots+a_{n}\alpha_{n}|<a^{-2/\epsilon}.\label{art12-eq4}
\end{equation}
Set $Q=a^{1/\epsilon}$, and let $\bu$ be in $Q^{2\epsilon}\Pi(Q)$, so that
\begin{equation}
|u|\leq Q^{1+2\epsilon}, |\alpha_{i}u-v_{i}|\leq Q^{-(1/(n-1))+2\epsilon}\quad (i=1,\ldots,n-1).\label{art12-eq5}
\end{equation}
Then\pageoriginale with $y=a_{n}u$, $x_{i}=a_{n}v_{i}(i=1,\ldots,n-1)$ we have
$$
|\alpha_{i}y-x_{i}|\leq aQ^{-(1/n-1)+2\epsilon}=Q^{-(1/(n-1))+3\epsilon}\quad (i=1,\ldots,n-1).
$$

On the other hand, if we set $x_{n}=-(a_{0}u+a_{1}v_{1}+\cdots+a_{n-1}v_{n-1})$, then \eqref{art12-eq4} yields
\begin{align*}
|\alpha_{n}y-x_{n}| &< naQ^{-(1/(n-1))+2\epsilon}+|u|a^{-2/\epsilon}\leq nQ^{-(1/(n-1))+3\epsilon}+Q^{1+2\epsilon}Q^{-2}\\
&< Q^{-(1/(n-1))+4\epsilon}
\end{align*}
if $a$, and hence $Q$, is large. When $\bu\neq 0$, then $u\neq 0$ by \eqref{art12-eq5}, so let us say that $u>0$. Then $y=au\leq Q^{1+3\epsilon}$; and since
$$
\left(\frac{1}{n-1}-4\epsilon\right)\Big/ (1+3\epsilon)\geq \frac{1}{n-1}-7\epsilon>\frac{1}{n}+\delta,
$$
we have
$$
\left| \alpha_{i}-\frac{x_{i}}{y}\right|<y^{-1-(1/n)-\delta}(i=1,\ldots,n).
$$
This holds for $\bu\neq 0$ in $Q^{2\epsilon}\Pi(Q)$. By what we said above, there will in general be $\geq Q^{\epsilon}$ non-proportional such $u$, and hence there will be $\geq Q^{\epsilon}$ solutions to \eqref{art12-eq2} or \eqref{art12-eq3}.

This will happen if there is a {\em single} hyperplane $\mathfrak{H}$ with \eqref{art12-eq4} and with large $a$. As is well known, the linear form inequality \eqref{art12-eq4} is dual to simultaneous approximations. Thus in order to bound the number of solutions of \eqref{art12-eq2}, we would have to give a bound for the {\em size} $a$ of solutions of the dual inequality \eqref{art12-eq4}. Thus we would have to have an ``effective'' result on the linear forms inequality \eqref{art12-eq4}. But such an effective result is unknown even in the case $n=1$, since the Thue-Siegel-Roth Theorem is ineffetive.

Now let us turn to Thue equations and Norm Form equations. A {\em Thue} equation is an equation
\begin{equation}
F(x,y)=m,\label{art12-eq6}
\end{equation}
where $F$ is a binary form of degree $r\geq 3$ with rational integer coefficients which is irreducible over the rationals. Over $\mathbb{C}$ it factors as $F=\alpha(x-\alpha_{1}y)\ldots(x-\alpha_{r}y)$ where $a\in \mathbb{Z}$, and $\{\alpha_{1},\ldots,\alpha_{r}\}$ is a set of conjugate algebraic numbers. A {\em Norm Form} equation is an equation
\begin{equation}
F(x_{1},\ldots,x_{n})=m\label{art12-eq7}
\end{equation}
where $F$ is a norm form, i.e.
$$
F=a\prod\limits^{r}_{i=1}(\alpha_{1}^{(i)}x_{1}+\cdots+\alpha^{(i)}_{n}x_{n}),
$$
where\pageoriginale $\alpha_{1},\ldots,\alpha_{n}$ lie in an algebraic number field $K$ of degree $r$, and where $\alpha\mapsto \alpha^{(1)},\ldots,\alpha\mapsto \alpha^{(r)}$ are the embeddings of $K$ into $\mathbb{C}$.

Using his result on approximation to algebraic numbers (applied to $\alpha_{1},\ldots,\alpha_{r}$), Thus showed that \eqref{art12-eq6} has only finitely many solutions. Bounds involving $m$, $r$ and the size of the coefficients of $F$ were given by Lewis and Mahler \cite{art12-key6}. Siegel \cite{art12-key14} had conjectured that there were bounds depending only on $m$ and $r$. The first such bounds were derived by Evertse \cite{art12-key5}. Later, Bombieri and Schmidt \cite{art12-key1} established the bound
$$
cr^{1+\omega}
$$
for the number of coprime solutions $x$, $y$, where $c$ is an absolute constant and $\omega=\omega(m)$ the number of distinct prime factors of $m$. Notice the contrast with diophantine approximation, where the dependency of \eqref{art12-eq1} on the height $H$ cannot be eliminated! Siegel also had conjectured that there should be a bound which depends only on the number of nonzero coefficients. It turns out that there is no bound independent of $m$. However, Mueller and Schmidt \cite{art12-key7} proved Siegel's conjecture in the modified form that there is a bound which depends only on $m$ and the number of nonzero coefficients of $F$ (but which is independent of the degree $r$).

Some years age (\cite{art12-key10}), I proved that when $F$ is ``non-degenerate'', then the equation \eqref{art12-eq7} has only finitely many solutions. I now can give an explicit bound for the number of solutions which depends only on $n$, $r$, $m$, but which is independent of the coefficients of $F$. In fact, the proof by induction on the number $n$ of variables would break down if at some stage we had dependency on the coefficients. It will be convenient to formulate the result in terms of {\em primitive} solutions with g.c.d. $(x_{1},\ldots,x_{n})=1$. I can prove (see \cite{art12-key13}) that the number of primitive solutions of a non-degenerate Norm Form equation with coefficients in $\mathbb{Z}$ is
$$
\leq C_{1}C_{2}
$$
where
\begin{align*}
& C_{1}=\min (r^{2^{30n_{r^{2}}}},r^{E})\text{~~ with~~ } E=(2n)^{n^{2^{n+4}}},\\
& C_{2}=\binom{r}{n-1}^{\omega}d_{n-1}(m^{r}),
\end{align*}
with $d_{n-1}(x)$ denoting the number of factorizations $x=x_{1}\ldots x_{n-1}$ with positive factors.

In the proof, I write the equation as $aL_{1}\ldots L_{r}=m$ where $L_{i}=\alpha^{(i)}_{1}x_{1}+\cdots+\alpha^{(i)}_{n}x_{n}(i=1,\ldots,r)$, and I deduce the existence of $i_{1},\ldots,i_{n}$ such that $L_{i_{1}},\ldots,L_{i_{n}}$\pageoriginale are independent and
\begin{equation}
|L_{i_{1}}(\bx)\ldots L_{i_{n}}(\bx)|<|\bx|^{-\delta}|\det (L_{i_{1}},\ldots,L_{i_{n}})|\label{art12-eq8}
\end{equation}
with suitable $\delta>0$, where $\bx=(x_{1},\ldots,x_{n})$, and $|\bx|$ is its norm. Inequalities \eqref{art12-eq8} are derived in various ways, depending on whether $\bx$ is ``small'' or ``large''. Then \eqref{art12-eq8} is dealt with by a semieffective version of the Subspace Theorem \cite{art12-key11}.

It is to be hoped that these results will lead to bounds for the number of solutions of $S$-unit equations, and the multiplicities of linear recursive sequences.

\begin{thebibliography}{99}
\bibitem{art12-key1} \textsc{E. Bombieri} and \textsc{W. M. Schmidt :} On Thue's equaiton, {\em Invent. Math.,} 88 (1987), 69-81.

\bibitem{art12-key2} \textsc{E. Bombieri} and \textsc{A. J. Van der Poorten :} Some quantitative results related to Roth's Theorem, {\em Mac Quarie Univ. Reports} 1987.

\bibitem{art12-key3} \textsc{H. Davenport} and \textsc{K. F. Roth :} Rational approximation to algebraic numbers, {\em Mathematika} 2 (1955), 160-167.

\bibitem{art12-key4} \textsc{H. Esnault} and \textsc{E. Viehweg : } Dyson's Lemma for polynomials in several variables (and the theorem of Roth), {\em Invent. Math.,} 78 (1984), 445-490.

\bibitem{art12-key5} \textsc{H. Evertse :} Upper bounds for the number of solutions of diophantine equations, {\em Math. Centrum Amsterdam} (1983), 1-127.

\bibitem{art12-key6} \textsc{D. Lewis} and \textsc{K. Mahler :} Representation of integers by binary forms, {\em Acta Arith.,} 6 (1961), 333-363.

\bibitem{art12-key7} \textsc{J. Mueller} and \textsc{W. M. Schmidt :} Thue's equation and a conjecture of Siegel, {\em Acta Math.,} 160 (1988), 207-247.

\bibitem{art12-key8} \textsc{J. Mueller} and \textsc{W. M. Schmidt :} The number of good rational approximations to algebraic numbers, {\em Proc. A. M. S.} (to appear).

\bibitem{art12-key9} \textsc{W. M. Schmidt :}\pageoriginale Simultaneous approximation to algebraic number by rationals, {\em Acta Math.,} 125 (1970) 159-201.

\bibitem{art12-key10} \textsc{W. M. Schmidt :} Norm Form equations, {\em Ann. of Math.} 96(1972) 526-551.

\bibitem{art12-key11} \textsc{W. M. Schmidt :} The Subspace Theorem in Diophantine approximations, {\em Compositio Math.,} 69 (1989), 121-173.

\bibitem{art12-key12} \textsc{W. M. Schmidt :} The number of solutions of Norm Form equations (expository paper), {\em Proc. Number Theory Symposium,} Budapest, July 1987, (to appear).

\bibitem{art12-key13} \textsc{W. M. Schmidt :} The number of solutions of Norm Form equations, {\em Trans. A. M. S.} (to appear).

\bibitem{art12-key14} \textsc{C. L. Siegel :} Uber einige Anwendungen diophantischer Approximationen, {\em Abh. Preuss. Akad. Wiss. Phys-Math. Kl.} 1929, Nr. 1.
\end{thebibliography}

\bigskip
\noindent
{\small University of Colorado}

\noindent
{\small Boulder, Colorado}


