\title{International Colloquium on Discrete Subgroups of Lie Groups and Applications to Moduli
\break
{\fontsize{13}{15}\selectfont Bombay, 8-15 January 1973}}

\author{REPORT}
\date{}
\maketitle

\thispagestyle{empty}

AN INTERNATIONAL  COLLOQUIUM on `Discrete Subgroups of Lie Groups and Applications to Moduli' was held at the Tata Institute of Fundamental Research, Bombay, from 8 to 15 January 1973. The purpose of the Colloquium was to discuss recent developments in some aspects of the following topics: (i) Lattices in Lie groups, (ii) Arithmetic groups, automorphic forms and related number-theoretic questions, (iii)  Moduli problems and discrete groups. The  Colloquium was a closed meeting of experts and of others specially interested in the subject.

\medskip
Th Colloquium was jointly sponsored by the International Mathematical Union and the Tata Institute of Fundamental Research, and was financially supported by them and the Sir Dorabji Tata Trust. 

\medskip
An Organizing Committee consisting of Professors A. Borel. M. S. Narasimhan, M. S. Raghunathan, K. G. Ramananthan and E. Vesentini was in charge of the scientific programme. Professors A. Borel and E. Vesentini acted as representatives of the International Mathematical Union on the Organizing Committee.

\medskip
The following mathematicians gave invited addresses at the Colloquium: W. L. Baily, Jr., E. Freitag, H. Garland, P. A. Griffiths, G. Harder, Y. Ihara, G. D. Mostow, D. Mumford, M. S. Raghunathan and W. Schmid.

\medskip

Professor \`E. B. Vinberg, who was unable to attend the Colloquium, sent in a paper.

\medskip
The invited lectures were of fifty minutes' duration. These were followed by discussions. In addition to the programme of invited addresses, there we expository and survey lectures and lectures by some invited speakers giving more details of their work.

\medskip
The social programme during the Colloquium included a Tea Party on 8 January; a Violin recital (Classical Indian Music) on 9 January; a programme of Western Music on 10 January; a performance of Classical Indian Dances (Bharata Natyan) on 12 January; a Film Show (Pather Panchali) on 13 January; and a dinner at the Institute on 14 January.

\lhead[]{}
\markboth{REPORT}{}

\vfill\eject
