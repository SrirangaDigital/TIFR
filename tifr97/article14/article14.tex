\title{SOME EXPONENTIAL DIOPHANTINE EQUATIONS (II)}
\markright{Some Exponential Diophantine Equations (II)}

\author{By~ T. N. Shorey}
\markboth{T. N. Shorey}{Some Exponential Diophantine Equations (II)}

\date{}
\maketitle

\setcounter{pageoriginal}{216}
\section{}\label{art14-sec1}\pageoriginale
Ramanujan \cite{art14-key19} observed in 1913 that
\begin{equation}
\begin{cases}
1^{2}+7=2^{3}, 3^{2}+7=2^{4}, 5^{2}+7=2^{5}\\
11^{2}+7=2^{7}, (181)^{2}+7=2^{15}.
\end{cases}\label{art14-eq1}
\end{equation}
We are looking for the solutions of
\begin{equation}
x^{2}+7=2^{m}\qquad\text{in integers~ } x>0, m>0.\label{art14-eq2}
\end{equation}
Ramanujan \cite{art14-key19} conjectured in 1913 that all the solutions of \eqref{art14-eq2} are given by \eqref{art14-eq1}. Nagell \cite{art14-key17} confirmed this conjecture in 1948. Equation \eqref{art14-eq2} is known as Ramanujan-Nagell equation.

Ratet \cite{art14-key20} observed in 1916 that
\begin{equation}
31=\frac{2^{5}-1}{2-1}=\dfrac{5^{3}-1}{5-1}.\label{art14-eq3}
\end{equation}
Thus 31 has all the digits equal to one with respect to the base 2 as well as the base 5. The next year, Goormaghtigh \cite{art14-key11} found an other integer satisfying a similar property :
\begin{equation}
8191=\frac{2^{13}-1}{2-1}=\dfrac{90^{3}-1}{90-1}.\label{art14-eq4}
\end{equation}
The letter $N$ denotes an integer greater than two and we write $\omega(N-1)$ for the number of distinct prime factors of $N-1$. Let us consider the following equation which gives \eqref{art14-eq3} and \eqref{art14-eq4} :
\begin{equation}
N=\frac{2^{n}-1}{2-1}=\dfrac{y^{3}-1}{y-1}\qquad\text{in integers~ } n>0, y>2.\label{art14-eq5}
\end{equation}
By completing square on the right hand side of \eqref{art14-eq5}, we obtain
$$
4N+4=(2y+1)^{2}+7=2^{n+2}.
$$
Now, we apply the above mentioned result of Nagell to derive that \eqref{art14-eq5} has no solution other than the ones given by \eqref{art14-eq3} and \eqref{art14-eq4}.

More general than equation \eqref{art14-eq5} is 
$$
N=\dfrac{y^{n_{1}}_{1}-1}{y_{1}-1}=\dfrac{y^{n_{2}}_{2}-1}{y_{2}-1}\quad\text{in integers~ } y_{1}>1, y_{2}>1, n_{1}>2,n_{2}>2.
$$
Thus $N$ has all the digits equal to one with respect to the base $y_{1}$ as well as the base $y_{2}$. Let $S(N)$ denote the set of all integers $y$ with $1<y<N-1$ such\pageoriginale that $N$ has all the digits equal to one with respect to the base $y$. Further, we put
$$
s(N)=|S(N)|.
$$
Thus
$$
s(31)=s(8191)=2.
$$
A conjecture, due to Ratat and Goormaghtigh, states that
\begin{equation}
s(N)\leq 1,\quad N\neq 31\text{~~ and~~ } N\neq 8191.
\end{equation}
Goormaghtigh \cite{art14-key11} checked this conjecture for $N<(10)^{4}$. For $y\in S(N)$, we have
\begin{equation}
N=\dfrac{y^{n}-1}{y-1},\quad n\geq 3.\label{art14-eq7}
\end{equation}
We put
\begin{equation}
n=l(N; y).\label{art14-eq8}
\end{equation}
For an integer $v>1$, we denote by $P(v)$ the greatest prime factor of $v$ and $\omega(v)$ the number of distinct prime divisors of $v$. Further, we write $P(1)=1$ and $\omega(1)=0$. Then, the author \cite{art14-key29} proved the following result.

\medskip
\noindent
{\bf Theorem \thnum{1}.\label{art14-thm1}}~{\em Let}
\begin{equation}
N\neq 31,\quad N\neq 8191\text{~~ and~~ } \omega(N-1)\leq 5.
\end{equation}
{\em There is at most one $y\in S(N)$ such that $l(N;y)$ is odd.}

If $N$ is a prime number, then we see from \eqref{art14-eq7} and \eqref{art14-eq8} that $l(N;y)$ is an odd prime and hence, we derive

\medskip
\noindent
{\bf Corollary \thnum{1}.\label{art14-coro1}}~{\em For a prime $N$ satisfying \eqref{art14-eq9}, we have}
$$
s(N)\leq 1.
$$

Thus, the conjecture \eqref{art14-eq6} is valid for all primes $N$ satisfying $\omega(N-1)\leq 5$. By sieve methods, it is known that the number of primes $N\leq Z$ with $\omega(N-1)\leq 5$ is at least constant times $Z(\log Z)^{-2}$. The equation
\begin{equation}
\frac{y^{n_{1}}_{1}-1}{y_{1}-1}=\frac{y^{n_{2}}_{2}-1}{y_{2}-1}\quad\text{in integers~~ } y_{1}>1, y_{2}>1, n_{1}>2, n_{2}>2\label{art14-eq10}
\end{equation}
has been considered by several authors. The first results are due to Makowski and Schinzel \cite{art14-key14}. Further, Davenport, Lewis and Schinzel \cite{art14-key6} applied a theorem of Siegel on integer points on curves to show that equation \eqref{art14-eq10} with fixed $n_{1}$ and $n_{2}$ has only finitely many solutions in integers $y_{1}>1$ and $y_{2}>1$. It follows from Baker's effective version \cite{art14-key2} of Thue's theorem \cite{art14-key36} that equation \eqref{art14-eq10} implies that max $(n_{1},n_{2})$ is bounded by an effectively computable number $C$ depending only on $y_{1}$ and $y_{2}$, The author \cite{art14-key26} applied a theorem of Baker \cite{art14-key1} on the approximations\pageoriginale of certain algebraic numbers by rationals proved by hyper-geometric method that there are at most 17 pairs $(n_{1},n_{2})$ satisfying \eqref{art14-eq10}. Balasubramanian and the author \cite{art14-key4} applied the theory of linear forms in logarithms to show that the number $C$, as above, depends only on the greatest prime factor of $y_{1}y_{2}$. See also \cite{art14-key23} and \cite{art14-key27}. Finally, it follows from a theorem of Schinzel and Tijdeman \cite{art14-key22} that equation \eqref{art14-eq10} implies that mas $(n_{1},y_{2})$ is bounded by an effectively computable number depending only on $y_{1}$ and $n_{2}$. Hence, equation \eqref{art14-eq10} has only finitely many solutions if any two out of the four variables $y_{1},y_{2},n_{1},n_{2}$ are fixed.

A weaker conjecture than \eqref{art14-eq6} states that
$$
s(N)<C_{1},\quad N=3,4,\ldots
$$
where $C_{1}>0$ is an effectively computable absolute constant. See Loxton \cite{art14-key13} where he derived from the theory of linear forms in logarithms that
\begin{equation}
s(N)=O_{\epsilon}((\log N)^{(1/2)+\epsilon}),\epsilon>0.\label{art14-eq11}
\end{equation}
The author \cite{art14-key29} proved that
\begin{equation}
s(N)\leq 
\begin{cases}
\max (2\omega(N-1)-3,0) & \text{if~~ }\omega(N-1)\leq 4\\
2\omega(N-1)-4 & \text{if~~ } \omega(N-1)>4
\end{cases}
\end{equation}
in an elementary way. If $\omega(N-1)$ is small, we observe that \eqref{art14-eq12} is more precise than \eqref{art14-eq11}. Finally, it is easy to observe that $s(N)<\omega(N-1)$ whenever $N$ is prime.

In this paragraph, we suppose that $N$ is a perfect power. Then, we can combine theorem \ref{art14-thm1} with \cite[theorem 5(iv)]{art14-key32} to derive that $s(N)\leq 1$ for every $N$ exceeding certain effectively computable absolute constant $C_{2}$ and satisfying $\omega(N-1)\leq 5$. In fact, it has been conjectured that $s(N)=0$ for $N>C_{2}$ and we refer to \cite{art14-key27} for an account of results proved in this direction.

Suppose that $N-1$ is a perfect power, say a $q$-th perfect power. Then, we subtract one on both the sides of \eqref{art14-eq7} to derive that
$$
N-1=y\frac{y^{n-1}-1}{y-1},\quad y\in S(N).
$$ 
Consequently, we see that both $y$ as well as $(y^{n-1}-1)/(y-1)$ are $q$-th perfect powers. Now, we apply \cite[theorem 3]{art14-key25} to obtain the following result.

\medskip
\noindent
{\bf Theorem \thnum{2}.\label{art14-thm2}}~{\em There\pageoriginale exists an effectively computable absolute constant $C_{3}$ such that $s(N)=0$ whenever $N-1$ is a perfect power and $N>C_{3}$.}

In other words, the equation
$$
z^{q}+1=\dfrac{y^{n}-1}{y-1}\quad\text{in integers~ } z>1, q>1, y>1, n>2.
$$
has only finitely many solutions. Furthermore, this assertion is effective. 

\section{}\label{art14-sec2}
Let us put
$$
u_{0}=1, u_{1}=9, u_{m}=3u_{m-1}-2u_{m-2}(m\geq 2).
$$
It is easy to check that
$$
u_{m}=2^{m+3}-7(m\geq 0).
$$
Ramanujan-Nagell equation \eqref{art14-eq2} asks for squares in this binary recursive sequence. By Nagell, there are only five squares in this sequence. Several authors have worked on finding perfect powers in binary recursive sequences. For example, in the Fibonacci sequence
$$
u_{0}=0, u_{1}=1, u_{m}=u_{m-1}+u_{m-2}(m\geq 2),
$$
Cohn \cite{art14-key5} and Wyler \cite{art14-key37}, independently, proved that
$$
u_{0}=0, u_{1}=1, u_{2}=1, u_{12}=144
$$
are the only squares and London and Finkelstein \cite{art14-key12} showed that
$$
u_{0}=0, u_{1}=1, u_{2}=1, u_{6}=8
$$
are the only cubes. It has been derived from the theory of linear forms in logarithms that there are only finitely many perfect powers in a non-degenerate binary recursive sequence. See Peth\"o \cite{art14-key18} and Shorey and Stewart \cite{art14-key30}; the latter paper and \cite{art14-key31} contain also applications of this and related results to certain Diophantine equations.

Now, we turn to define a non-degenerate binary recursive sequence. Let $r$, $s\in \mathbb{Z}$ with $s\neq 0$ and $r^{2}+4s\neq 0$. Let $u_{0}$, $u_{1}\in \mathbb{Z}$ and
$$
u_{m}=ru_{m-1}+su_{m-2}(m\geq 2).
$$
Let $\alpha$ and $\beta$ be roots of $X^{2}-rX-s$. Observe that $\alpha\beta\neq 0$ and $\alpha\neq \beta$. Further
\begin{equation}
u_{m}=a\alpha^{m}+b\beta^{m}\quad (m\geq 0)\label{art14-eq13}
\end{equation}
where
\begin{equation}
a=\frac{u_{0}\beta-u_{1}}{\beta-\alpha},\quad b=\frac{u_{1}-u_{0}\alpha}{\beta-\alpha}.\label{art14-eq14}
\end{equation}
The sequence $\{u_{m}\}^{\infty}_{m=0}$ is called non-degenerate if $ab\neq 0$ and $\alpha/\beta$ is not a root\pageoriginale of unity. Ramanujan's $\tau$-function satisfies the following recursive relation :
$$
u_{0}=0, u_{1}=1, u_{m}=\tau(p)u_{m-1}-p^{11}u_{m-2}(m\geq 2, p\text{~ prime})
$$
and
$$
u_{m}=\tau(p^{m-1})\quad (m\geq 1).
$$
Let $\alpha_{p}$ and $\beta_{p}$ be roots of $X^{2}-\tau(p)X+p^{11}$. Then, by \eqref{art14-eq13} and \eqref{art14-eq14}, we have
\begin{equation}
u_{m+1}=\tau(p^{m})=\frac{\alpha^{m+1}_{p}-\beta^{m+1}_{p}}{\alpha_{p}-\beta_{p}}\label{art14-eq15}
\end{equation}
This sequence is non-degenerate whenever $\tau(p)\neq 0$. Further, by Deligne,
$$
|\alpha_{p}|=|\beta_{p}|=p^{11/2}.
$$

It is well-known that Thue-Siegel-Roth-Schmidt method and Gel'fond-Baker theory of linear forms in logarithms are powerful tools in studying recursive sequences. In particular, these methods can be applied to obtain some results on Ramanujan's $\tau$-function. For example, an estimate of Baker \cite{art14-key3} applied to \eqref{art14-eq15} gives
$$
|\tau(p^{m})|\geq p^{(11m/2)-C_{4}\log (m+1)}\quad\text{if}\quad \tau(p^{m})\neq 0.
$$
Here $C_{4}>0$ is an effectively computable absolute constant. The author \cite{art14-key28} applied this estimate together with \cite[Corollary 7.1]{art14-key33} to obtain the following result.

\medskip
\noindent
{\bf Theorem \thnum{3}.\label{art14-thm3}}~{\em Let $p$ be a prime number such that $\tau(p)\neq 0$. Then}
$$
\tau(p^{m})=\tau(p^{n})\quad (m\neq n)
$$
{\em implies that}
$$
\max (m,n,p)\leq C_{5}
$$
{\em where $C_{5}>0$ is an effectively computable absolute constant.}

We refer to \cite{art14-key16} and \cite{art14-key28} for an account of applications of the theory of linear forms in logarithms to Ramanujan's $\tau$-function.

\section{}\label{art14-sec3}
This section is a continuation of \S1 of \cite{art14-key27}. Erd\"os \cite{art14-key7} and Rigge \cite{art14-key21}, independently, proved that the product of two or more consecutive positive integers is never a square. Erd\"os and Selfridge \cite{art14-key9}, by developing on an elementary method of Erd\"os \cite{art14-key8}, confirmed an old conjecture by proving that the product of two or more consecutive positive integers is never a power. In this section, we consider the corresponding problem for consecutive members of an arithmetical progression.

First, we introduce some notation. Let $b>0$, $d>0$, $m>0$, $y>0$, $k>2$ and\pageoriginale $l\geq 2$ be integers such that $P(b)\leq k$ and $(m,d)=1$. Let $d_{1}$ be the maximal divisor of $d$ such that all the prime divisors of $d_{1}$ are $\nequiv 1(\mod l)$ and we write $d_{2}=d/d_{1}$. We consider the equation
\begin{equation}
m(m+d)\ldots (m+(k-1)d)=by^{l}.\label{art14-eq16}
\end{equation}
We shall follow this notation, without reference, in this section. As already stated, equation \eqref{art14-eq16} with $d=b=1$ is not possible. Also, due to Fermat and Euler, equation \eqref{art14-eq16} with $b=1$, $l=2$ and $k=4$ is not possible.

Erd\"os conjectured that equation \eqref{art14-eq16} with $b=1$ implies that $k$ is bounded by an effectively computable absolute constant. Under certain restrictions, we wish to confirm this conjecture for equation \eqref{art14-eq16}. For this, it is natural to exclude the case that
\begin{equation}
P(m(m+d)\ldots (m+(k-1)d))\leq k.\label{art14-eq17}
\end{equation}
Then, we refer to \cite{art14-key34} to observe that \label{art14-eq17} implies that either $d=1$ or $m=2$, $d=7$, $k=3$. Therefore, we always suppose, without reference, that
\begin{equation}
P(m(m+d)\ldots (m+(k-1)d))>k \quad\text{if~ } d=1\text{~~ and~~ } b>1.\label{art14-eq18}
\end{equation}
By a well-known theorem of Sylvester, the assumption \eqref{art14-eq18} is certainly satisfied whenever $m>k$.

Marszalek \cite{art14-key15} confirmed the conjecture of Erd\"os for a fixed $d$. More precisely, he proved that equation \eqref{art14-eq16} with $b=1$ implies that 
\begin{equation}
\begin{cases}
k\leq \exp (C_{6}d^{3/2}) & \text{if~ }l=2,\\
k\leq \exp (C_{7}d^{7/3}) & \text{if~ }l=3,\\
k\leq C_{8}d^{5/2} & \text{if~ } l=4,\\
k\leq C_{9}d & \text{if~ } l\geq 5,
\end{cases}\label{art14-eq19}
\end{equation}
where $C_{6}$, $C_{7}$, $C_{8}$ and $C_{9}$ are explicitly given absolute constants.

The author \cite{art14-key27} proved that equation \eqref{art14-eq16} with $l\geq 3$ implies that $k$ is bounded by an effectively computable number depending only on the greatest prime factor of $d$. Also, the author \cite{art14-key27} confirmed the conjecture of Erd\"os whenever $d_{2}=1$ and $l\geq 3$.

Suppose the equation \eqref{art14-eq16} is satisfied. Then, Shorey and Tijdeman \cite{art14-key35}\footnote[1]{We refer to \cite{art14-key35} for more general and more precise versions of the results stated here from \cite{art14-key35}.} proved that $k$ is bounded by an effectively computable number depending only on $l$ and $\omega(d)$. More precisely, they proved that
\begin{equation}
l^{\omega(d)}>C_{10}k/\log k\label{art14-eq20}
\end{equation}\pageoriginale
where $C_{10}$ and the subsequent letters $C_{11},\ldots,C_{28}$ are effectively computable absolute constants. Further, in \cite{art14-key35}, they sharpened the above-mentioned result of the author by proving that
\begin{equation}
(d\geq ) d_{2}>C_{11}k^{l-2}.\label{art14-eq21}
\end{equation}
We combine \eqref{art14-eq20} and \eqref{art14-eq21} to conclude that
$$
d\geq k^{C_{12}(\log \log k)/(\log \log \log k)},\quad k\geq C_{13}.
$$
This improves considerably the estimates \eqref{art14-eq19}.

For $\epsilon>0$, Shorey and Tijdeman \cite{art14-key35} proved that equation \eqref{art14-eq16} with $k\geq C'_{14}=C'_{14}(\epsilon)$ implies that
$$
m\leq d^{2}k^{1+\epsilon}\quad\text{if~ } l=2
$$
and
\begin{equation}
m+(k-1)d\leq C_{15}kd_{2}^{l/(l-2)}\quad\text{if~ } l\geq 3.\label{art14-eq22}
\end{equation}
We show that the estimate \eqref{art14-eq22} is quite precise for sufficiently large $l$.

\medskip
\noindent
{\bf Theorem \thnum{4}.\label{art14-thm4}}~{\em Suppose that equation \eqref{art14-eq16} is satisfied. There exist effectively computable absolute constants $C_{16}>0$ and $C_{17}>0$ such that for $k>C_{16}$, we have}
\begin{equation}
m\geq d^{1-C_{17}\Delta_{l}}\label{art14-eq23}
\end{equation}
{\em where}
\begin{equation}
\Delta_{l}=l^{-1}(\log l)^{2}(\log\log (l+1)).\label{art14-eq24}
\end{equation}

We combine \eqref{art14-eq23} and \eqref{art14-eq21} to derive the following result.

\medskip
\noindent
{\bf Corollary \thnum{2}.\label{art14-coro2}}~{\em There exists an effectively computable absolute constant $C_{18}>0$ such that equation \eqref{art14-eq16} with $l\geq C_{18}$ implies that $k$ is bounded by an effectively computable number depending only on $m$.}

We may combine Corollary \ref{art14-coro2} and \eqref{art14-eq20} to conclude that equation \eqref{art14-eq16} implies that $k$ is bounded by an effectively computable number depending only on $m$ and $\omega(d)$. The proof of theorem \ref{art14-thm4} depends on the estimates of Baker \cite{art14-key3} and the author \cite[lemma 2]{art14-key24} on linear forms in logarithms.

\medskip
\noindent
{\bf Proof of theorem \ref{art14-thm4}.}~We denote by $C_{19},\ldots,C_{28}$ effectively computable absolute positive constants. We may assume that $k\geq C_{19}$ with $C_{19}$ sufficiently large. We may also suppose that $l\geq C_{19}$, otherwise \eqref{art14-eq23} follows immediately. We suppose that
\begin{equation}
m<d^{1-\Delta_{l}}\label{art14-eq25}
\end{equation}
and we shall arrive at a contradiction.

By\pageoriginale \eqref{art14-eq16}, we have
\begin{equation}
m+id=a_{i}x^{l}_{i}\quad (0\leq i<k)\label{art14-eq26}
\end{equation}
where $a_{i}$ and $x_{i}$ are positive integers satisfying
$$
P(a_{i})\leq k,\quad \left(x_{i},\prod\limits_{p\leq k}p\right)=1.
$$
We put
$$
S=\{a_{0},\ldots,a_{k-1}\}.
$$
In view of the theorem of Erd\"os and Selfridge mentioned in the beginning of this section, we may assume that $b>1$ whenever $d=1$. Then, we derive from \eqref{art14-eq18} and \cite{art14-key34} that the left hand side of \eqref{art14-eq16} is divisible by a prime $>k$. Now, it follows from \eqref{art14-eq16} that
$$
m+(k-1)d\geq (k+1)^{l}
$$
which, by \eqref{art14-eq25}, implies that
\begin{equation}
m+d\geq k^{l-1},\quad d\geq k^{l-1}/2.\label{art14-eq27}
\end{equation}

We denote by $S_{1}$ the set of all $a_{i}$ with $1\leq i<k$ such that $x_{i}=1$. Observe, by \eqref{art14-eq26}, that the elements of $S_{1}$ are distinct. Now, we apply an argument of Erd\"os (see \cite{art14-key10}, lemma 2.1) to derive from \eqref{art14-eq27} that
$$
|S_{1}|\leq k/2+\pi (k).
$$
We denote by $T$ the set of all $i$ with $1\leq i<k$ such that $a_{i}\not\in S_{1}$ and we write $S_{2}$ for the set of all $a_{i}\in S$ with $i\in T$. Then
\begin{equation}
|T|\geq k/4.\label{art14-eq28}
\end{equation}

We put
$$
b_{i}=a_{i}/i\quad (i\in T)
$$
and let $S'$ be the set of all $b_{i}$ with $i\in T$. Let $i\in T$, $j\in T$ with $i>j$ and $b_{i}=b_{j}$. Then, by \eqref{art14-eq26},
\begin{equation}
ja_{i}(x^{l}_{i}-x^{l}_{j})=(j-i)m.\label{art14-eq29}
\end{equation}
Now, we observe that the left hand side of \eqref{art14-eq29} exceeds
\begin{equation}
(j-i)(a_{i}x^{l}_{i})^{(l-1)/l}\geq (j-i)(m+d)^{(l-1)/l}.\label{art14-eq30}
\end{equation}
Therefore, by \eqref{art14-eq29} and \eqref{art14-eq30},
$$
d^{(l-1)/l}<(m+d)^{(l-1)/l}<m
$$
which implies \eqref{art14-eq23}. Thus, we may suppose that the elements of $S'$ are distinct.

For every prime $p\leq k$, we choose an $f=f(p)\in T$ such that 
$$
\ord_{p}(a_{f})\geq \max\limits_{i\in T}\ord_{p}(a_{i}).
$$
We\pageoriginale
%page 225
