\title{THE ADJOINT HECKE OPERATOR II}
\markright{The Adoint Hecke Operator II}

\author{By~ R. A. Rankin}
\markboth{R. A. Rankin}{The Adjoint Hecke Operator II}

\date{}
\maketitle

\setcounter{pageoriginal}{160}
\section{Introduction}\label{art10-sec1}
\pageoriginale
Progress on the theory of modular forms and associated Euler products can be divided roughly into three stages. At the first fundamental stage there is the work of Hecke \cite{art10-key3}, who introduced the linear operators $T_{n}$ now associated with his name. The second stage comprises the work of Petersson \cite{art10-key8}, who observed that the space $M$ of cusp forms of given level, weight and character is a finite-dimensional Hilbert space, and showed that the adjoint Hecke operator $T^{*}_{n}$ is a scalar multiple of $T_{n}$, provided that $n$ is a prime to the level $N$ of $M$. The foundations of the third stage were laid by Atkin and Lehner \cite{art10-key1}, who separated off from $M$ the subspace $M^{-}$ consisting essentially of forms of lower level, and concentrated their attention on its orthogonal complement $M^{+}$, showing by delicate methods that $M^{+}$ has an orthogonal basis of forms that are eigenforms for all the operators $T_{n}$ and not only for those with $n$ prime to $N$.

The present paper arose from an effort to simplify the arguments of the third stage, by investigating the properties of the adjoint operator $T^{*}_{n}$ for all $n$, and showing, if possible, that it commutes with $T_{n}$ on the subspace $M^{+}$. We recall that, for any forms $f$ and $g$ in $M$, $T^{*}_{n}$ is defined by
\begin{equation}
(f|T_{n},g)=(f,g|T^{*}_{n}).\label{art10-eq1.1}
\end{equation}
Petersson proved that $T^{*}_{n}=\overline{\chi}(n)T_{n}$ if $(N,n)=1$, where $\chi$ is the associated Dirichlet character. For this he provided two proofs. Of these one \cite{art10-key8} was fairly direct, but had a combinatorial part in which a common left and right transversal of a certain group was shown to exist (Hilfssatz 2), and did not seem applicable to other values of $n$. On the other hand, his other earlier proof \cite{art10-key7} (p. 68), although only valid when the weight $k$ of the space exceeds 2, seemed more promising, although technically somewhat complicated, as it involved the evaluation of $G|T_{n}$ for an arbitrary Poincar\'e series $G$ in $M$.

This is the method developed in my first paper under the same title \cite{art10-key10}, where is yielded an apparently previously unknown explicit definition of $T^{*}_{n}$ for $(n,N)\neq 1$. The case when $N$ is a prime number was then investigated in detail using properties of Poincar\'e series. However, for composite $N$ this method becomes decidedly more complicated, because of increased number of incongruent cusps of verying cusp widths and parameters.\pageoriginale In the present paper the general case is considered in a relatively simple way without the use of Poincar\'e series, and the explicit definition of the adjoint operator, found in \cite{art10-key10}, is proved by a different method.

\section{Groups, matrices and characters}\label{art10-sec2}
As is customary, we write 
\setcounter{equation}{0}
\begin{equation}
\Gamma(1):=\SL(2,\mathbb{Z})\label{art10-eq2.1}
\end{equation}
for the modular group and, for any positive real number $m$, we denote by $\Omega_{m}$ the set of all matrices
\begin{equation}
T:=
\begin{bmatrix} 
a & b\\ 
c & d
\end{bmatrix}\label{art10-eq2.2}
\end{equation}
belonging to $\GL(2,\mathbb{R})$ and having determinant $m$. We shall be particularly concerned with the group
\begin{equation}
\Gamma_{0}(N)=\{T\in \Gamma(1):c\equiv 0(\mod N)\},\label{art10-eq2.3}
\end{equation}
where $N$ is a positive integer, and require the following special matrices in $\Gamma(1)$ :
\begin{equation}
I=
\begin{bmatrix}
1 & 0\\
0 & 1
\end{bmatrix}, \ \ 
U=
\begin{bmatrix}
1 & 1\\
0 & 1
\end{bmatrix}, \ \ 
V=
\begin{bmatrix}
0 & -1\\
1 & 0
\end{bmatrix}, \ \
W=
\begin{bmatrix}
1 & 0\\
1 & 1
\end{bmatrix}.\label{art10-eq2.4}
\end{equation}
Write also
\begin{equation}
\Gamma(N)=\{T\in \Gamma(1): T\equiv I(\mod N)\}.\label{art10-eq2.5}
\end{equation}
For various positive rational values of $m$ we write
\begin{equation}
J_{m}=
\begin{bmatrix}
1 & 0\\
0 & m
\end{bmatrix}.\label{art10-eq2.6}
\end{equation}

Throughout $k$ will be a positive integer and, for typographical reasons we write
\begin{equation}
K=\frac{1}{2} k -1.\label{art10-eq2.7}
\end{equation}
Let
\begin{equation}
\mathbb{H}=\{z\in \mathbb{C} : \Imz>0\},\label{art10-eq2.8}
\end{equation}
and put, as customary,
\begin{equation}
e(z)=\exp (2\pi iz)\quad (z\in \mathbb{C}).\label{art10-eq2.9}
\end{equation}

For $T\in \Omega_{m}$, we define
\begin{equation}
Tz:=\dfrac{az+b}{cz+d}, \ \ T:z=cz+d.\label{art10-eq2.10}
\end{equation}
For any function $f:\mathbb{H}\to \mathbb{C}$ and $T\in \Omega_{m}(m>0)$ we define
\begin{equation}
f(z)|T:=(T:z)^{-k}(\det T)^{k/2}f(Tz).\label{art10-eq2.11}
\end{equation}
This depends of course on $k$, which is fixed. Note that
\begin{equation}
f(z) | J_{m}=m^{-k/2}f(z/m), \ f(z) | J_{m}^{-1}=m^{k/2}f(mz).\label{art10-eq2.12}
\end{equation}

The letters $p$ and $q$ will always denote prime numbers, and we write
\begin{align}
& P=\{0,1,\ldots,p-1\}, \ P^{*}=\{1,2,\ldots,q-1\},\label{art10-eq2.13}\\[2pt]
& Q=\{0,1,\ldots,q-1\}, \ Q^{*}=\{1,2,\ldots,q-1\}.
\end{align}

Throughout,\pageoriginale $\chi$ denotes a character modulo $N$ such that
\begin{equation}
\chi(-1)=(-1)^{k};\label{art10-eq2.15}
\end{equation}
it follows that, when $k$ is odd, $N\geq 3$. We denote by $N(\chi)$ the conductor of $\chi$ and put
\begin{equation}
n(\chi)=N/N(\chi).\label{art10-eq2.16}
\end{equation}
Note that, for any positive integer $r$,
\begin{equation}
r|n(\chi)\Leftrightarrow N(\chi)| (N/r).\label{art10-eq2.17}
\end{equation}

The principal character modulo $m$ is denoted by $\epsilon_{m}$.

Accordingly, $\chi$ may be written as
\begin{equation}
\chi = \chi^{*}\epsilon_{N},\label{art10-eq2.18}
\end{equation}
where $\chi^{*}$ is a primitive character modulo $N(\chi)$. When \eqref{art10-eq2.17} holds 
\begin{equation}
\chi_{r} := \chi^{*}\epsilon_{N/r}\label{art10-eq2.19}
\end{equation}
is a character modulo $N/r$ with conductor $\chi^{*}$.

\section{$M(N,k,\chi)$ and its subspaces}\label{art10-sec3}
A standard notation for the vector space of cusp forms belonging to a group $\Gamma$ and having weight $k$ and multiplier system $v$ is
\setcounter{equation}{0}
\begin{equation}
\{\Gamma, k, v\}_{0}\label{art10-eq3.1}
\end{equation}
and we shall write
\begin{equation}
M=M(N,k,\chi)=\{\Gamma_{0}(N),k,\chi\}_{0}.\label{art10-eq3.2}
\end{equation}
Thus $M$ is a space of level $N$, weight $k$ and character $\chi$.

For any positive integers $r$ and $s$ satisfying
\begin{equation}
r|n(\chi), \ \ s|r\label{art10-eq3.3}
\end{equation}
we define
\begin{equation}
C(r,s,\chi_{r}):= M(N/r,k,\chi_{r})|J^{-1}_{s},\label{art10-eq3.4}
\end{equation}
where $\chi_{r}$ is defined by \eqref{art10-eq2.19}. Note that
\begin{equation}
C(1,1,\chi_{1})=M.\label{art10-eq3.5}
\end{equation}
It is easy to see that
\begin{equation}
C(r,s,\chi_{r})\subseteq M(Ns/r,k,\chi_{r/s})\subseteq M.\label{art10-eq3.6}
\end{equation}

Whenever $r>1$ and $s<r$ the level of $C(r,s,\chi_{r})$ is less than $N$. When $r=s>1$ the level is $N$, but the space is isomorphic to $M(N/r,k,\chi_{r})$ and so can be regarded as a space of essentially lower level. For this reason we define
\begin{equation}
M^{-}=\bigoplus\limits_{r,s}C(r,s,\chi_{r})(r>1, r|n(x),s|r).\label{art10-eq3.7}
\end{equation}
Then $M^{-}$ is a subspace of $M$ of essentially lower level and any member of $M^{-}$ is called an {\em oldform}. Now $M$ is a finite-dimensional Hilbert space, and we define $M^{+}$ to be the orthogonal complement of $M^{-}$ in $M$, so that
\begin{equation}
M=M^{-}\oplus M^{+}.\label{art10-eq3.8}
\end{equation}

The\pageoriginale definition of $M^{-}$ can be simplified, as the following Theorem shows.

\medskip
\noindent
{\bf Theorem \thnum{3.1}.\label{art10-thm3.1}}~{\em We have}
\begin{equation}
M^{-}=\bigoplus\limits_{p|n(\chi)}C(p,\chi_{p}),\label{art10-eq3.9}
\end{equation}
{\em where}
\begin{equation}
C(p,\chi)=C(p,1,\chi_{p})\oplus C(p,p,\chi_{p}).\label{art10-eq3.10}
\end{equation}

\begin{proof}
We observe that
\begin{equation}
C(rs,t,\chi_{rs})\subseteq C(r,t,\chi_{r})(t|r,rs|n(\chi))\label{art10-eq3.11}
\end{equation}
and
\begin{equation}
C(rs,rt,\chi_{rs})\subseteq C(s,t,\chi_{s})(t|s,rs|n(\chi))\label{art10-eq3.12}
\end{equation}

It is clear that \eqref{art10-eq3.11} holds. To prove \eqref{art10-eq3.12} take any $F\in C(rs,rt,\chi_{rs})$, so that we can put
$$
F=g|J^{-1}_{t}=f|J^{-1}_{rt},
$$
where
$$
f\in M(N/rs,k,\chi_{rs}).
$$
Now take any $T\in\Gamma_{0}(N/s)$, so that
$$
T_{1}:=J^{-1}_{r}TJ_{r}\in \Gamma_{0}(N/(rs)).
$$
Then
\begin{gather*}
g|T=f|J^{-1}_{r}T=f|T_{1}J^{-1}_{r}=\chi_{rs}(T_{1})f|J^{-1}_{r}\\
=\chi_{rs}(T_{1})g=\chi_{s}(T)g,
\end{gather*}
so that $g\in M(N/s,k,\chi_{s})$, and this proves \eqref{art10-eq3.12}.

By successive applications of \eqref{art10-eq3.11} and \eqref{art10-eq3.12} we complete the proof of the theorem.
\end{proof}

\section{The Fricke involution $H_{r}$}\label{art10-sec4}
For any $r\in \mathbb{N}$ define
\setcounter{equation}{0}
\begin{equation}
H_{r}=J_{r}V=
\begin{bmatrix}
0 & -1\\
r & 0
\end{bmatrix}\label{art10-eq4.1}
\end{equation}
so that $H^{2}_{r}=-rI$ and $H^{-1}_{r}=-r^{-1}H_{r}$. It is easily verified that
\begin{equation}
H^{-1}_{r}\Gamma_{0}(r)H_{r}=H_{r}\Gamma_{0}(r)H^{-1}_{r}=\Gamma_{0}(r).\label{art10-eq4.2}
\end{equation}

\medskip
\noindent
{\bf Lemma \thnum{4.1}.\label{art10-lem4.1}}~{\em Let $N=rs$ where $r|n(\chi)$. Then}
\begin{equation}
M(N/r,k,\chi_{r})|H_{s}=M(N/r,k,\overline{\chi}_{r}).\label{art10-eq4.3}
\end{equation}
{\em In particular}
\begin{equation}
M(N,k,\chi)| H_{N}=M(N,k,\overline{\chi}).\label{art10-eq4.4}
\end{equation}
{\em Further, if $N=pt$, where $p|n(\chi)$, then}
\begin{equation}
C(p,p,\chi_{p})|H_{N}=C(p,1,\overline{\chi}_{p})\label{art10-eq4.5}
\end{equation}
{\em and}
\begin{equation}
C(p,1,\chi_{p})|H_{N}=C(p,p,\overline{\chi}_{p}).\label{art10-eq4.6}
\end{equation}

\begin{proof}
Let\pageoriginale $f\in M(N/r,k,\chi_{r})|H_{s}$ so that $f=g|H_{s}$, where $g\in M(N/r,k,\chi_{r})$. Take any $T\in \Gamma_{0}(N/r)$ so that
$$
H^{-1}_{s}TH_{s}\in \Gamma_{0}(s).
$$
Then 
\begin{align*}
f|T &= g|TH_{s}=g|H_{s}H_{s}^{-1}TH_{s}=\chi_{r}(H^{-1}_{s}TH_{s})g|H_{s}\\[3pt]
&= \chi_{r}(H^{-1}_{s}TH_{s})f=\overline{\chi}_{r}(T)f.
\end{align*}
Moreover
\begin{align*}
C(p,p,\chi_{p})|H_{N} &= C(p,1,\chi_{p})|J^{-1}_{p}H_{N}=C(p,1,\chi_{p})|H_{t}\\[3pt]
&= C(p,1,\chi_{p}),
\end{align*}
by \eqref{art10-eq4.4} with $N$ replaced by $t$. This gives \eqref{art10-eq4.5} and \eqref{art10-eq4.6} follows by replacing $\chi$ by $\overline{\chi}$ and operating again on the right with $H_{N}$.
\end{proof}

\medskip
\noindent
{\bf Lemma \thnum{4.2}.\label{art10-lem4.2}}~{\em If $N=pt$, then}
\begin{equation}
H_{N}J_{p}U^{u}H^{-1}_{N}=pW^{-ut}J_{p}^{-1}.\label{art10-lem4.7}
\end{equation}

\begin{proof}
Straightforward.
\end{proof}

\section{The Hecke operators $T_{n}$}\label{art10-sec5}
For any $n\in \mathbb{N}$ and any $f\in M(N,k,\chi)$ define the operator $T_{n}(N,\chi)=T_{n}$ by
\setcounter{equation}{0}
\begin{equation}
f|T_{n}=n^{k}\sum\limits_{ad=n}\sum\limits^{d}_{u=1}\chi(a)f|J_{d}U_{u}J^{-1}_{a},\label{art10-eq5.1}
\end{equation}
where $J$ is given by \eqref{art10-eq2.7}, and observe that, for any prime $p$ we have, in particular,
\begin{equation}
f|T_{p}=p^{K}\left(\sum\limits_{u\in P}f|J_{p}U^{u}+\chi(p)f|J^{-1}_{p}\right);\label{art10-eq5.2}
\end{equation}
see \eqref{art10-eq2.12}. It is clear that, in \eqref{art10-eq5.2}, $u$ can run through any complete set of residues modulo $p$.

If
\begin{equation}
N=pt, \ p\nmid t\text{~~ and~~ } f\in C(p,1,\chi_{p}),\label{art10-eq5.3}
\end{equation}
if follows from \eqref{art10-eq5.2} that
\begin{equation}
f|T_{p}(t,\chi_{p})=f|T_{p}(N,\chi)+p^{K}\chi_{p}(p)f|J^{-1}_{p}\label{art10-eq5.4}
\end{equation}
and we note that $\chi_{p}(p)\neq 0$ in this case.

We now summarize some of the known properties of the operators. For any $f\in M$, let 
\begin{equation}
f(z)=\sum\limits^{\infty}_{r=1}a(r)e(rz).\label{art10-eq5.5}
\end{equation}
Then\pageoriginale
\begin{equation}
f(z)|T_{n}=\sum\limits^{\infty}_{r=1}a_{n}(r)e(rz),\label{art10-eq5.6}
\end{equation}
where 
\begin{equation}
a_{n}(r)=\sum\limits_{d|(n,r)}d^{k-1}\chi(d)a(nr/d^{2}).\label{art10-eq5.7}
\end{equation}
Moreover, we have
\begin{equation}
M|T_{n}\subset M\quad (n\in \mathbb{N})\label{art10-eq5.8}
\end{equation}
and
\begin{equation}
(f|T_{m})|T_{n}=\sum\limits_{d|(m,n)}d^{k-1}\chi(d)f|T_{mn/d^{2}}(m\in \mathbb{N},n\in \mathbb{N}).\label{art10-eq5.9}
\end{equation}
It follows that the operators commute and that $T_{n}$ is completely determined when $T_{p}$ is known for each prime $p|n$.

Moreover, as shown in Petersson \cite{art10-key8}, if $f$ and $g$ belong to $M$, then
\begin{equation}
(f|T_{n},g)=\chi(n)(f,g|T_{n})\text{~~ for~~ } (n,N)=1.\label{art10-eq5.10}
\end{equation}
Here the inner product is defined for cusp forms $f$ and $g$ of weight $k$ on a subgroup $\Gamma$ of finite index $h$ in $\Gamma(1)$ by
\begin{equation}
(f,g)=(f,g;\Gamma)=\frac{1}{h}\iint\limits_{\mathscr{F}}f(z)g(z)y^{k-2}dxdy.\label{art10-eq5.11}
\end{equation}
where $x=\rRe z$, $y=\Iim z$ and $\mathscr{F}$ is any fundamental region in $\mathbb{H}$ for $\Gamma$. In \S\ref{art10-sec6} we shall require this definition for various subgroups $\Gamma$ contained in $\Gamma(1)$. It follows from \eqref{art10-eq5.10} that $T^{*}_{n}$, the adjoint operator, is given by 
\begin{equation}
T^{*}_{n}=\overline{\chi}(n)T_{n}\text{~~ for~~ } (n,N)=1.\label{art10-eq5.12}
\end{equation}

\section{The adjoint operator $T^{*}_{p}$ for $p|N$}\label{art10-sec6}
For any prime $p|N$ and $f\in M$ define the operator $T^{*}_{p}=T^{*}_{p}(N,\chi)$ by
\setcounter{equation}{0}
\begin{align}
f|T^{*}_{p}: &= f|H_{N}T_{p}H_{N}^{-1}=f|H_{N}^{-1}T_{p}H_{N}\label{art10-eq6.1}\\
            &= p^{K}\sum\limits_{u\in P}f|H_{N}J_{p}U^{u}H^{-1}_{N}\notag\\
            &= p^{K}\sum\limits_{u\in P}f|W^{-ut}J^{-1}_{p}\label{art10-eq6.2}
\end{align}
by Lemma \ref{art10-lem4.2}.

Since $T_{p}(N,\chi)=T_{p}(N,\overline{\chi})$ it follows from \eqref{art10-eq4.4} that
\begin{equation}
M|T^{*}_{p}\subseteq M.\label{art10-eq6.3}
\end{equation}

\medskip
\noindent
{\bf Theorem \thnum{6.1}.\label{art10-thm6.1}}~{\em For any prime $p|N$, $T^{*}_{p}$ is the adjoint operator to $T_{p}$; i.e.}
\begin{equation}
(f|T^{*}_{p},g)=(f,g|T_{p})\text{~~ for~~ } f\text{~~ and~~ } g\text{~~ in~~ }M.\label{art10-eq6.4}
\end{equation}

For the proof we require the following Lemma, which we quote from Theorem 5.2.1 of \ref{art10-key9}.

\medskip
\begin{description}
\item[{\bf Lemma \thnum{6.2}.\label{art10-lem6.2}} {\rm (i)}] {\em If\pageoriginale $\Gamma_{1}$ and $\Gamma_{2}$ are subgroups of $\Gamma(1)$ of finite index in $\Gamma(1)$ and $\Gamma_{1}\subseteq \Gamma_{2}$, then}
$$
(f,g:\Gamma_{1})=(f,g:\Gamma_{2})
$$
{\em whenever $f$ and $g$ both belong to $\{\Gamma_{2},k,v\}_{0}$.}

\item[{\rm(ii)}] {\em Let $\Gamma$ be a congruence subgroup of $\Gamma(1)$ of finite index and let, for any prime $p$,}
\begin{equation}
\Gamma_{p}=\Gamma\cap\Gamma(p).\label{art10-eq6.5}
\end{equation}
{\em Suppose that $f$ and $g$ belong to $\{\Gamma,k,v\}_{0}$ and let $L\in \Omega_{p}$. Then}
\begin{equation}
(f,g;\Gamma)=(f|L,g|L;L^{-1}\Gamma_{p}L).\label{art10-eq6.6}
\end{equation}
\end{description}

\noindent
{\bf Proof of Theorem.}~Take any $f$ and $g$ in $M$ and write
$$
F=f|T^{*}_{p},
$$
so that $f\in M$ by \eqref{art10-eq6.3}. Note that, if $S\in \Gamma(pN)$, then
$$
f|W^{-ut}J^{-1}_{p}S=f|S'W^{-ut}J^{-1}_{p},
$$
where $S'\in\Gamma(N)$, so that $\chi(S')=1$. Hence, for any $u\in \mathbb{Z}$,
$$
f|W^{-ut}J^{-1}_{p}\in \{\Gamma(pN),k,1\}_{0},
$$
and so, by Lemma \ref{art10-lem6.2}(i) and \eqref{art10-eq6.2},
\begin{align*}
(F,g:\Gamma_{0}(N))=(F,g;\Gamma(pn)) &= p^{K}\sum\limits_{u\in P}(f|W^{-ut}J^{-1}_{P},g;\Gamma(pN))\\[3pt]
&= p^{K}\sum\limits_{u\in P}(f|U^{-u}W^{-ut}J^{-1}_{P},g;\Gamma(pN)).
\end{align*}
Write
$$
A_{u}=J_{p}W^{ut}U^{u}=W^{uN}J_{p}U^{u}\in \Omega_{p}
$$
and note that, when $\Gamma=\Gamma(pN)$,
$$
\Gamma_{p}=\Gamma(pN),
$$
by \eqref{art10-eq6.5}, so that
$$
A_{u}^{-1}\Gamma_{p}A_{u}=J^{-1}_{p}\Gamma(pN)J_{p}\supseteq \Gamma(pN^{2}).
$$
Taking $L=A_{u}$ in \eqref{art10-eq6.6}, we get
\begin{align*}
(F,g;\Gamma_{0}(N)) &= p^{K}\sum\limits_{u\in p}(f,g|A_{u};\Gamma(pN^{2}))\\
&= p^{K}\left(f,\sum\limits_{u\in P}g|W^{uN}J_{p}U^{u};\Gamma(pN^{2})\right)\\
&= p^{K}\left(f,\sum\limits_{u\in P}g|J_{p}U^{u}; \ \Gamma(pN^{2})\right)\\
&= (f,g|T_{p}; \ \Gamma(pN^{2}))\\[3pt]
&= (f,g|T_{p}; \ \Gamma_{0}(N)).
\end{align*}
This completes the proof of the theorem.

\medskip
\noindent
{\bf Theorem \thnum{6.3}.\label{art10-thm6.3}}~{\em Let\pageoriginale $m$ and $n$ be positive integers. Then the following pairs of operators on $M$ commute :}

(i) $T_{m}$, $T_{n}$; (ii) $T^{*}_{m}$, $T^{*}_{n}$; (iii) $T_{m}$, $T^{*}_{n}$ {\em provided that} $(m,n,N)=1$.

\begin{proof}
(i) follows from \eqref{art10-eq5.9} and this yields (ii), since
$$
(f|T^{*}_{m}T^{*}_{n},g)=(f,g|T_{n}T_{m}).
$$
By \eqref{art10-eq5.12} and (i) we need only prove that $T^{*}_{p}$ and $T_{q}$ commute when $p$ and $q$ are different primes dividing $N$.

Write $N=pqs$ and define $S_{u,w}$ by
$$
S_{u,w}J_{q}U^{up}W^{-wqs}J^{-1}_{p}=W^{-wq^{2}s}J^{-1}_{p}J_{q}U^{u}
$$
for $(u,w)\in Q\times P$. Then it is easy to see that $S_{u,w}\in \Gamma_{0}(N)$ and that $\chi(S_{u,w})=1$. Now, if $f\in M$, since $up$ and $wq$ run through complete sets of residues modulo $p$ and modulo $q$ respectively, we have
\begin{align*}
f|T^{*}_{p}T_{q} &= (pq)^{K}\sum\limits_{u\in Q}\sum\limits_{w\in P} f|W^{-wq^{2}s}J^{-1}_{p}J_{q}U^{u}\\[3pt]
&= (pq)^{K}\sum\limits_{w\in P}\sum\limits_{u\in Q}f|J_{q}U^{up}W^{-wqs}J^{-1}_{p}\\[3pt]
&= f|T_{q}T^{*}_{p}.
\end{align*}
\end{proof}

\section{The action of the operators on $M$}\label{art10-sec7}
~

\medskip
\noindent
{\bf Theorem \thnum{7.1}.\label{art10-thm7.1}}~{\em For all $n\in \mathbb{N}$}
\setcounter{equation}{0}
\begin{equation}
M^{-}|T_{n}\subseteq M^{-}\cdot M^{-}|T^{*}_{n}\subset M^{-}.\label{art10-eq7.1}
\end{equation}
{\em In particular, if $p$ is any prime dividing $N$ and $N=pt$, we have~:}
\begin{itemize}
\item[(i)] {\em For $(n,N)=1$ and $p|n(\chi)$}
\begin{equation}
C(p,1,\chi_{p})|T_{n}\subseteq C(p,1,\chi_{p}),C(p,p,\chi_{p})|T_{n}\subseteq C(p,p,\chi_{p}).\label{art10-eq7.2}
\end{equation}

\item[(ii)] {\em If $p|n(\chi)$,}
\begin{equation}
C(p,p,\chi_{p})|T_{p}\subseteq C(p,1,\chi_{p}),C(p,1,\chi_{p})|T^{*}_{p}\subseteq C(p,p,\chi_{p}).\label{art10-eq7.3}
\end{equation}

\item[(iii)] {\em If $p$ and $q$ are different primes dividing $n(\chi)$,}
\begin{align}
& C(q,1,\chi_{q})|T_{p}\subseteq C(q,1,\chi_{q}),C(q,1,\chi_{q})|T^{*}_{p}\subseteq C(q,1,\chi_{q}),\label{art10-eq7.4}\\
& C(q,q,\chi_{q})|T_{p}\subseteq C(q,q,\chi_{q}), C(q,q,\chi_{q})|T^{*}_{p}\subseteq C(q,q,\chi_{q}).\label{art10-eq7.5}
\end{align}

\item[(iv)] {\em If $p|n(\chi)$ and $p^{2}|N$,}
\begin{equation}
C(p,1,\chi_{p})|T_{p}\subseteq C(p,1,\chi_{p}), C(p,p,\chi_{p})|T^{*}_{p}\subseteq C(p,p,\chi_{p})\label{art10-eq7.6}
\end{equation}

\item[(v)] {\em If $p|n(\chi_{p})$ and $p^{2}\nmid N$.}
\begin{equation}
C(p,1,\chi_{p})|T_{p}\subseteq C(p,\chi_{p}),C(p,p,\chi_{p})|T^{*}_{p}\subset C(p,\chi_{p}).\label{art10-eq7.7}
\end{equation}
\end{itemize}

\begin{proof}
In view of Theorem \ref{art10-thm3.1}, \eqref{art10-eq7.1} will follow if we prove parts (i)-(v) of the theorem. For the proof of (i) see pp. 321-322 of \cite{art10-key}. By \eqref{art10-eq6.1}, \eqref{art10-eq4.5} and \eqref{art10-eq4.6} it is only necessary to prove those parts of \eqref{art10-eq7.3}-\eqref{art10-eq7.7} that involve the operator $T_{p}$.

For\pageoriginale \eqref{art10-eq7.3} we note that
$$
C(p,p,\chi_{p})|J_{p}U^{u}=C(p,1,\chi_{p})|U^{u}=C(p,1,\chi_{p}).
$$
For \eqref{art10-eq7.4} note that the operator $T_{p}(N,\chi)$ is the same as the operator $T_{p}(N/q,\chi_{q})$ since $p$ divides $N/q$ and the latter operator maps the space $M(N/q,k,\chi_{q})$ into itself. Also, if $N=pqs$ and $f=g|J^{-1}_{q}\in C(q,q,\chi_{q})$ then $g\in C(q,1,\chi_{q})$ and 
\begin{align*}
f|T_{p} &= p^{K}\sum\limits_{u\in P}g|J^{-1}_{q}J_{p}U^{u}=p^{K}\sum\limits_{u=P}g|J_{p}U^{uq}J^{-1}_{q}\\[3pt]
&= g|T_{p}J^{-1}_{q}\subseteq C(p,1,\chi_{q})|J^{-1}_{q}=C(q,q,\chi_{q}).
\end{align*}
which proves \eqref{art10-eq7.5}.

\eqref{art10-eq7.6} follows for the same reason as \eqref{art10-eq7.4}, since $p$ divides $N/p$ and the operators $T_{p}(N,\chi)$ and $T_{p}(N/p,\chi_{p})$ are identical.

Finally, assume that $p|n(\chi)$ but $p^{2}\nmid N$. We assume \eqref{art10-eq5.3} and deduce that
$$
f|T_{p}(N,\chi)\in C(p,1,\chi_{p})\oplus C(p,p,\chi_{p})=C(p,\chi_{p}),
$$
from \eqref{art10-eq5.4}.
\end{proof}

\medskip
\noindent
{\bf Theorem \thnum{7.2}.\label{art10-thm7.2}}~{\em For all $n\in \mathbb{N}$}
$$
M^{+}|T_{n}\subseteq M^{+}\text{~~ and~~ } M^{+}|T^{*}_{n}\subseteq M^{+}.
$$

\begin{proof}
Take any $f\in M^{+}$ and $g\in M^{-}$. Then
$$
(f|T_{n},g)=(f,g|T^{*}_{n})=0
$$
by Theorem \ref{art10-thm7.1}. Hence $f|T_{n}\in M^{+}$. The proof of the second part is similar.
\end{proof}

\medskip
\noindent
{\bf Theorem \thnum{7.3}.\label{art10-thm7.3}}~{\em Let $M_{H}=M|H_{N}$. Then $(M_{H})^{-}=M^{-}|H_{N}$.}

\begin{proof}
By \eqref{art10-eq4.4}
$$
M_{H}=M(N,k,\overline{\chi}),
$$
so that $(M_{H})^{-}$ is a vector sum of the spaces $C(p,1,\overline{\chi}_{p})$ and $C(p,p,\overline{\chi}_{p})$; the result follows.
\end{proof}

\section{The operators $T^{*}_{p}T_{p}$ and $T_{p}T^{*}_{p}(p|N)$}\label{art10-sec8}
~

\medskip
\noindent
{\bf Lemma \thnum{8.1}.\label{art10-lem8.1}}~{\em Let $\mathfrak{R}$ be a right transversal of $\Gamma_{0}(N)$ in $\Gamma_{0}(t)$, where $N=tp$, and put}
\setcounter{equation}{0}
\begin{equation}
\mathfrak{R}_{0}=\bigcup\limits_{w\in P}W^{-wt}.
\end{equation}
{\em Then we may take}
\begin{itemize}
\item[(i)] $\mathfrak{R}=\mathfrak{R}_{0}$, {\em when $p\nmid t$, and}

\item[(ii)] $\mathfrak{R}=\mathfrak{R}_{0}\cup R^{*}$, {\em when $p\nmid t$, where}
\begin{equation}
R^{*}=p^{-1}J_{p}U^{s}W^{t}J_{p}=
\left(
\begin{matrix}
(1+st)/p & s\\
t & p
\end{matrix}
\right)\label{art10-eq8.2}
\end{equation}
{\em and $s$ is chosen so that $s\in P$ and $st\equiv -1(\mod p)$.}
\end{itemize}

This\pageoriginale is straightforward : note that $R^{*}\in \Gamma(1)$.

\medskip
\noindent
{\bf Lemma \thnum{8.2}.\label{art10-lem8.2}}~{\em Let $f\in M(N,k,\chi)$, where $p|n(\chi)$. Then}
$$
F:=\sum\limits_{R\in \mathfrak{R}}\overline{\chi}(R)f|R\in M(t,k,\chi_{p}),
$$
{\em so that $F\in M^{-}$.}

\begin{proof}
Let the members of $\mathfrak{R}$ be $R_{r}(r=1,2,\ldots,h)$ where $h=[\Gamma_{0}(t):\Gamma_{0}(N)]$, and take any $S\in \Gamma_{0}(t)$. Then $\mathfrak{R}S$ is also a right transversal of $\Gamma_{0}(N)$ in $\Gamma_{0}(t)$ and so
$$
R_{r}S=S_{r}R'_{r},
$$
where $R'_{r}\in \mathfrak{R}$ and $S_{r}\in \Gamma_{0}(N)$. Note that
$$
\chi(R_{r})\chi(S)=\chi(S_{r})\chi(R'_{r}).
$$
Then
\begin{align*}
F|S &= \sum\limits^{h}_{r=1}\overline{\chi}(R_{r})f|R_{r}S=\sum\limits^{h}_{r=1}\overline{\chi}(R_{r})f|S_{r}R'_{r}\\[3pt]
&= \sum\limits^{h}_{r=1}\overline{X}(R_{r})\chi(S_{r})f|R'_{r}=\sum\limits^{h}_{r=1}\chi(S)\overline{\chi}(R'_{r})f|R'_{r}\\[3pt]
&= \chi(S)F.
\end{align*}
It follows that $F\in M(t,k,\chi_{p})$ and this proves the lemma.
\end{proof}

\medskip
\noindent
{\bf Lemma \thnum{8.3}.\label{art10-lem8.3}}~{\em Suppose that $N=pt$, where $p|t$, and that $\chi$ is a character modulo $N$. Then, for some integer $m\in P$,}
\begin{equation}
\chi(1+rt)=e(mr/p)(r\in \mathbb{Z}).\label{art10-eq8.3}
\end{equation}
{\em Moreover}
\begin{equation}
m=0\text{\em~ if and only if~ } p|n(\chi).\label{art10-eq8.4}
\end{equation}

\begin{proof}
Since
$$
(1+t)^{p}\equiv 1(\mod N),
$$
$\chi(1+t)=e(m/p)$ for some $m\in P$ and \eqref{art10-eq8.3} follows since
$$
1+rt\equiv (1+t)^{r}(\mod p).
$$

If $p|n(\chi)$, then $N(\chi)|t$ and so $m=0$, Conversely, if $N(\chi)\nmid t$, then
$$
\chi(n)\neq \chi(n+t)
$$
for some $n$ prime to $N$ and so, taking $rn\equiv 1(\mod N)$.
$$
1\neq \chi(1+rt),
$$
from which it follows that $m\neq 0$, by \eqref{art10-eq8.3}.

We now define
\begin{equation}
\delta(\chi)=0\text{~ if~ } p\nmid n(\chi); \ \delta(\chi)=1\text{~ if~ } p|n(\chi).\label{art10-eq8.5}
\end{equation}

Further, for any prime $p$ dividing $N$ we put $N=pt$, as usual, and define
\begin{equation}
\alpha(p)=
\begin{cases}
0 & \text{if~ } p|t, p|n(\chi),\\
p^{k-2} & \text{if~ } p\nmid t, p|n(\chi),\\
p^{k-1} & \text{if~ } p\nmid n(\chi).
\end{cases}\label{art10-eq8.6}
\end{equation}
Then\pageoriginale we have 
%page 171
\end{proof}
