\title{LINEAR OPERATORS AND AUTOMORPHIC FORMS}
\markright{Linear Operators and Automorphic Forms}

\author{By~ Atle Selberg}
\markboth{Atle Selberg}{Linear Operators and Automorphic Forms}

\date{}
\maketitle

\setcounter{pageoriginal}{202}
\section{}\label{art13-sec1}\pageoriginale
We consider a bounded symmetric complex domain $B$ in the sense of Elie Cartan\footnote{See for instance Siegel \cite{art13-key5}, Chapter XI.} and denote the group of analytic mappings of $B$ onto itself by $G$, points in $B$ by $z$ and the elements of $G$ by $g:z\to gz$. By a multiplier (or automorphy factor) $\rho_{g}(z)$, we understand a function defined on $G\times B$ which is analytic (holomorphic) in $z$ and differentiable in $g$ such that
\begin{equation}
\rho_{g_{1}g_{2}}(z)=\rho_{g_{1}}(g_{2}z)\rho_{g_{2}}(z).\label{art13-eq1.1}
\end{equation}
Any such multiplier defines a kernel function $k_{\rho}(z,\overline{z})$ which transforms in the way
\begin{equation}
k_{\rho}(gz,\overline{gz})=\rho_{g}(z)\overline{\rho_{g}(z)}k_{p}(z,\overline{z}).\label{art13-eq1.2}
\end{equation}
We need only to write, for some fixed $z_{0}$ in $B$,
$$
k_{\rho}(z,\overline{z})=|\rho_{g}(z_{0})|^{2}
$$
where $g$ is a solution of $z=gz_{0}$ and it is clear that this does not depend on the particular $g$ chosen, but only on the point $z$.

From \eqref{art13-eq1.2}, we get that
$$
ds^{2}=\sum\limits_{i,j}\frac{\partial^{2}\log k_{p}(z,\overline{z})}{\partial z_{i}\partial\overline{z}_{j}}\cdot dz_{i}d\overline{z}_{j}
$$
is an invariant metric on $B$ under the action of the group $G$. Thus, if $B$ is irreducible, we get that this metric can differ only by a constant factor from the Bergmann metric. If $B$ is reducible, it must be a linear combination of the Bergmann metrics of the irreducible factors of $B$. For irreducible $B$, one easily derives that, up to a factor of the form $cf(z)\overline{f(z)}$ where $f(z)$ is analytic, $k_{\rho}(z,\overline{z})$ coincides with a real power of the Bergmann kernel function and $\rho_{g}(z)$ is, apart from a `trivial' multiplier of the form $f(gz)/f(z)$, equal to a power of the jacobian $j_{g}(z)$ of the mapping $g$. Similarly, if $B$ is reducible, $\rho_{g}(z)$ is, apart from a trivial factor $f(gz)/f(z)$, equal to a product of powers of the jacobians of the mapping $g$ with respect to the various irreducible factors of $B$. 

We\pageoriginale may mention that essentially the same conclusion could be drawn from the weaker premise that instead of \eqref{art13-eq1.1}, $\rho_{g}(z)$ satisfies the relation 
\begin{equation}
|\rho_{g_{1}g_{2}}(z)|=|\rho_{g_{1}}(g_{2}z)||\rho_{g_{2}}(z)|,
\end{equation}
then, apart from a factor of the form $\epsilon_{g}f(gz)/f(z)$ where $f$ is analytic and $|\epsilon_{g}|=1$, $\rho_{g}(z)$ is equal to a product of powers of the jacobians of the mapping $z\to gz$ with respect to the irreducible factors of $B$.

We shall study linear operators on functions defined on $B$, which have the property of transforming with a multiplier on each side under the mappings of the group $G$. Call the operator $L=L_{z}$ and define $L_{gz}$ through the relation
$$
L_{gz}F(z)=[L_{z}F(g^{-1}z)]_{z\to gz};
$$
then $L$ should transform according to the rule
\begin{equation}
L_{gz}=\rho_{g}(z)L_{2}\sigma^{-1}_{g}(z),\label{art13-eq1.4}
\end{equation}
where $\rho$ and $\sigma$ are two multipliers.\footnote{From now on, we disregard trivial multipliers and consider only products of powers of the jacobians for the irreducible factors of $B$. This is only an apparent restriction.}

We ask the question : for which $B$ and which choices of multipliers $\rho$ and $\sigma$ do such operators exist? And when they exist, determine their form as explicitly as possible.

It is easily seen that we may restrict ourselves to the case that $B$ is irreducible and then derive the results for the general case from those obtained for the irreducible factors of $B$ in case $B$ is irreducible.

As is known\footnote{See Siegel \cite{art13-key5}, Chapter XI for instance.} there are six types of irreducible bounded symmetric domains. If we denote a matrix with $m$ rows and $n$ columns by $Z^{(m,n)}$ and the $(n,n)$ unit matrix $E$ or $E^{(n)}$, there are the four main types :
\begin{itemize}
\item[(I)] $Z=Z^{(m,n)}$, $E-\overline{Z}'Z>0$,

\item[(II)] $Z=Z^{(n,n)}$, $Z'=-Z$, $E-\overline{Z}'Z>0$.

\item[(III)] $Z=Z^{(n,n)}$, $Z'=Z$, $E-\overline{Z}'Z>0$,\quad and

\item[(IV)] $Z=Z^{(n,1)}$, $\overline{Z}'Z<\frac{1}{2}(1+|Z'Z|^{2})<1$.
\end{itemize}

Here $Z'$ denotes the transposed matrix. In addition, there are the types V and VI --- the two exceptional bounded symmetric domains of complex dimension 16 and 27 respectively; we shall not give a definition here.

It\pageoriginale should also be noted that for $n=2$, the domain IV is reducible and that there is also some overlapping between the four types for low dimension; the unit circle $|z|<1$ in one complex variable is, for instance, a special case of all the four types.

\section{}\label{art13-sec2}
The question of linear operators that transform according to the rule \eqref{art13-eq1.4} can be split in two :
\begin{itemize}
\item[({\em a})] Operators that conserve the multiplier, i.e. when \eqref{art13-eq1.4} holds, but with $\rho_{g}(z)\equiv \sigma_{g}(z)$.
\end{itemize}

We shall refer to such operators as invariant (though, strictly speaking, they are so only if $\rho_{g}(z)\equiv 1$ identically).

It is well-known that invariant operators exist for all the bounded symmetric domains $B$ and for all multipliers $\rho_{g}(z)$; for given $\rho$ the differential operators form a finitely generated ring, where the number of independent generators equals the rank of the group $G$ (or of the symmetric space). In this ring, all elements, except the constant, contain differentiations both with respect to $z$ and $\overline{z}$. The form of integral operators is easily given explicitly for $B$ irreducible; if $\rho_{g}(z)=(j_{g}(z))^{-r}$ where $j_{g}(z)$ denotes the jacobian of the mapping $g$, then
\setcounter{equation}{0}
\begin{equation}
Lf=\int\limits_{B} x(z,\zeta)\left(\frac{k(z,\overline{\eta})}{k(\zeta,\overline{\zeta})}\right)^{r}f(\zeta)d\omega_{\zeta}\label{art13-eq2.1}
\end{equation}
where $d\omega_{\zeta}$ is the invariant volume element, $k(z,\overline{\zeta})$ is the Bergmann kernel function and $x(z,\zeta)$ is a ``point pair invariant'' satisfying $x(gz,g\zeta)=x(z,\zeta)$ for all $z$ and $\zeta$ in $B$ and $g$ in $G$.

In particular, for analytic functions $f(z)$, we have the reproducing operator
\begin{equation}
f(z)=c_{r}\int\limits_{B}\left(\frac{k(z,\overline{\zeta})}{k(\zeta,\overline{\zeta})}\right)^{r}f(\zeta)d\omega_{\zeta}\label{art13-eq2.2}
\end{equation}
where $c_{r}$ is a certain polynomial in $r$. \eqref{art13-eq2.2} is valid for a certain hilbertspace of analytic functions if $r>r_{0}$, the largest zero of the polynomial $c_{r}$.\footnote{See Selberg \cite{art13-key4}} 
\begin{itemize}
\item[({\em b})] Operators that change the multiplier, i.e. which transform in the way \eqref{art13-eq1.4} but with $\rho_{g}(z)\nequiv \sigma_{g}(z)$.
\end{itemize}

The question ({\em b}) is more complex than ({\em a}), but it is not difficult to establish that linear operators that change the multiplier do not exist for all the irreducible domains, but only for a certain subclass.

To see this, we may look at the compact subgroup of $G$ which leaves some\pageoriginale point $z_{0}$ in $B$ fixed, the so-called stability group or isotropy group of $z_{0}$. It is simplest to choose the point $O$ where all the coordinates are zero and the compact subgroup $K_{0}$ which keeps $O$ fixed. For all the six types of bounded symmetric domains, the way they are usually defined, the elements of $K_{0}$ are linear transformations, and $K_{0}$ is essentially (sometimes, a slight change of variables being necessary as in type III where we would put a factor $1/\sqrt{2}$ in the elements of the symmetric matrix which are off the main diagonal) a subgroup of the unitary group $U(N)$ where $N$ is the complex dimension of $B$ and $j_{k}(z)=j_{k}(0)$, where $k\in K_{0}$, is a one-dimensional representation of $K_{0}$.

It is clear that if there exists a linear operator satisfying \eqref{art13-eq1.4}, then, in particular, \eqref{art13-eq1.4} must hold for $g$ restricted to $K_{0}$ and we consider the functional $\mathscr{L}$ that $L_{z}$ represents at $z=0$.

On the other hand, it is not hard to show that if we have a linear functional $\mathscr{L}$ which has the required property \eqref{art13-eq1.4} for $g$ in $K_{0}$, then it can be extended to a linear operator $L_{z}$ by means of the relation \eqref{art13-eq1.4} with $z=gO$, but the general form of this operator seems awkward to obtain in this way, particularly if it is a differential operator.

It is easily seen that an integral operator
$$
L_{z}f=\int\limits_{B}h(z,\zeta) f(\zeta)d\omega_{\zeta}
$$
where $h(z,\zeta)$ is a short form for $h(z,\overline{z},\zeta,\overline{\zeta})$, must in order to satisfy \eqref{art13-eq1.4}, have a kernel $h(z,\zeta)$ which satisfies
\begin{equation}
h(gz,g\zeta)=\rho_{g}(z)\sigma^{-1}_{g}(\zeta)h(z,\zeta)\label{art13-eq2.3}
\end{equation}
and, in particular, for $g=k\in K_{0}$, if we put $\zeta=0$, we get
$$
h(kz,0)=\rho_{k}(z)\sigma^{-1}_{k}(0)h(z,0)
$$
or since $\rho_{k}(z)=\rho_{k}(0)$,
\begin{equation}
h(kz,0)=\rho_{k}(0)\sigma^{-1}_{k}(0)h(z,0).\label{art13-eq2.4}
\end{equation}
Since we may assume that $h(z,\zeta)$ is analytic in $z$ and $z$\footnote{If our original $h(z,\zeta)$ is not so, we may form the convolution of $L_{z}$ with a suitable operator of the form \eqref{art13-eq2.1} on the left which preserves the multiplier $\rho_{g}(z)$.}, it is clear that the expansion of $h(z,0)$ in terms of powers of $z$ and $\overline{z}$ for $z$ near $0$, must start with a homogeneous polynomial $\rho(z,\overline{z})$ which also transforms by the factor $\rho_{k}(0)\sigma^{-1}_{k}(0)$ when we replace $z$ by $kz$.

Similarly,\pageoriginale if $D_{z}$ is a differential operator which obeys the transformation rule \eqref{art13-eq1.4}, at $z=0$ it takes the form of a polynomial in $\dfrac{\partial}{\partial z}$ and $\dfrac{\partial}{\partial\overline{z}}$:
$$
D_{0}=P\left(\frac{\partial}{\partial z},\frac{\partial}{\partial\overline{z}}\right).
$$
When $z$ and so $dz$ undergoes a unitary transformation from $K_{0}$, $\dfrac{\partial}{\partial z}$ undergoes the contragredient transformation; so we are again led to a polynomial (which we may assume to be homogeneous, otherwise taking the homogeneous part of lowest degree that is not identically zero) which transforms in the way \eqref{art13-eq2.4} when the variables undergo the contragredient transformation to $k$ (Actually, if we interchange $\dfrac{\partial}{\partial z}$ and $\dfrac{\partial}{\partial\overline{z}}$, the vector $\left(\dfrac{\partial}{\partial\overline{z}},\dfrac{\partial}{\partial z}\right)$ undergoes the same transformation as $(z,\overline{z})$).

It is now easy to see for the various types of $B$ whether such polynomials exist when $\rho_{g}(z)\nequiv \sigma_{g}(z)$.

{\em We find that for type I, they exist only if $m=n$ and are then of the form}
$$
|z|^{r}P(z,\overline{z})\quad\text{or}\quad |\overline{z}|^{r}P(z,\overline{z})
$$
{\em where $|z|$ is the determinant of $z$, $r$ some positive integer and $P(z,\overline{z})$ some homogeneous polynomial which is invariant under $K_{0}$.}

{\em For type II, they exist only if $n$ is even and are then of the form}
$$
P^{r}_{f}(z)P(z,\overline{z}\quad\text{or}\quad P^{r}_{f}(\overline{z})P(z,\overline{z})
$$
{\em where $P_{f}(z)$ is the polynomial called the Pfaffian of $z$ (actually $|z|^{1/2}$, since the determinant is a square in this case), $r$ again is a positive integer and $P$, a homogeneous polynomial which is invariant under $K_{0}$.}

{\em For type III, they exist for all $n$ and are of the form}
$$
|z|^{r}P(z,\overline{z})\quad\text{or}\quad |\overline{z}|^{r}P(z,\overline{z})
$$
{\em where $r$ is a positive integer and $P(z,\overline{z})$ homogeneous and invariant under $K_{0}$.}

{\em For type IV, they again exist and are of the form}
$$
(z' \ z)^{r}P(z,\overline{z})\quad\text{or}\quad (\overline{z}' \ \overline{z})^{r}P(z,\overline{z})
$$
{\em with $r$ a positive integer and $P(z,\text{\st{$z$}})$ again homogeneous and invariant under $K_{0}$.}

For\pageoriginale the types V and VI which (for good reason!) we have not exhibited explicitly, we find they do not exist for type V but, {\em for type} VI, {\em they exist and are given by the form}
$$
p_{3}(z)^{r}P(z,\overline{z})\quad\text{\em or}\quad p_{3}(\overline{z})^{r}P(z,\overline{z})
$$
{\em where $r$ and $P$ are as before and $p_{3}$ is a certain cubic polynomial in $27$ variables.}

\section{}\label{art13-sec3}
In order to derive more explicitly the form of the linear operators that transform according to \eqref{art13-eq1.4} in the cases where we have seen they can exist, we note that the cases we have listed in the previous section are precisely the cases when the bounded domain $B$, by a suitable analytic mapping, becomes a so-called ``positive half-space''\footnote{M. Koecher \cite{art13-key2} writes ``half-space''; I prefer ``positive half-space'' since it indicates the connection with a positivity-domain.}, and when the group $G$ by this mapping, becomes a real group (by which we mean that in this new unbounded version of our domain, we have $\overline{gz}=g\overline{z}$).

By a positive half-space, we understand a domain of $z=x+iy$, where the column vector $x$ is unrestricted, while the vector $y$ is required to lie in a homogeneous positivity-domain $Y$ in the sense of Koecher\footnote{M. Koecher \cite{art13-key1}}. As before, we shall use $N$ to denote the complex dimension.

We recall some of the properties of a homogeneous positivity domain $Y$. It is a cone such that, for any two vectors $y^{(1)}$ and $y^{(2)}$ in $Y$, we have always 
\setcounter{equation}{0}
\footnotetext[8]{Koecher's definition is more general; he has \eqref{art13-eq3.1} in the form $y^{(1)}Sy^{(2)}>0$, where $S$ is a nonsingular symmetric real matrix, but \eqref{art13-eq3.1} covers the cases we consider.}
\begin{equation}
y^{(1)}y^{(2)}>0\footnotemark[8]\label{art13-eq3.1}
\end{equation}
and so that if, for some vector $y^{(1)}$, \eqref{art13-eq3.1} holds for all $y^{(2)}$ in $Y$, then $y^{(1)}$ also lies in $Y$.

There also exists a group $G_{Y}$ of real matrices $A$ such that $y\to Ay$ maps $Y$ onto itself; this group is transitive on $Y$. In particular, for any scalar $\lambda>0$, we have $\lambda y\in Y$ for $y\in Y$, so that $Y$ is a cone. It is seen from \eqref{art13-eq3.1} that if $A$ is in $G_{Y}$, then $y\to {A'}^{-1}y$ also maps $Y$ onto itself; so, we may, without restriction, assume that with $A$, always ${A'}^{-1}$ also lies in $G_{Y}$.

There exists a homogeneous polynomial $Q(y)$, which we choose to be of minimal degree $q>0$ such that $Q(y)$ is positive in $Y$ and 
\footnotetext[9]{Our $Q(y)=(N(y))^{q/N}$, where $N(.)$ is Koecher's ``Norm-function''.}
\begin{equation}
Q(Ay)=|A|^{q/N}Q(y)\footnotemark[9]\label{art13-eq3.2}
\end{equation}\pageoriginale
If we define for $i=1,\ldots,N$,\footnotemark[9]
\begin{equation}
y^{*}_{i}=\dfrac{\partial \log Q(y)}{\partial y_{i}},\label{art13-eq3.3}
\end{equation}
then $y\to y^{*}$ is an involution which carries $Y$ into itself. We have\footnotemark[9]
\begin{equation}
Q(y^{*})Q(y)=\text{constant},\label{art13-eq3.4}
\end{equation}
and, by a suitable choice of $Q$ (which by \eqref{art13-eq3.2} is only determined up to a constant factor) we get
\begin{equation*}
Q(y^{*})Q(y)=1\tag{3.4$'$}\label{art13-eq3.4'}
\end{equation*}

Also, \eqref{art13-eq3.2} gives $y^{*'}y=q$ and $(A \ y)^{*}={A'}^{-1}y^{*}$.

On $Y$, we have an invariant volume element
\begin{equation}
dV_{y}=(Q(y)^{-N/q}dy\label{art13-eq3.5}
\end{equation}
where we have written $dy$ for the euclidean volume element. We also have an invariant metric
\begin{equation}
ds^{2}=-\sum\limits_{1\leq i, j\leq N}\frac{\partial^{2}}{\partial y_{i}\partial y_{j}}\log Q(y) dy_{i}dy_{j}.
\end{equation}
The involution $y\to y^{*}$ (which is actually a symmetry) has a fixed point $e$ and we have $Q(e)=1$.

Now consider the positive half-space of $z=x+iy$ where $x$ is unrestricted and $y$ is in $Y$ and the group generated by translations of the form $z\to z+a$ where $a$ is a real vector, $z\to Az$ for $A$ in $G_{Y}$ and $z\to z^{*}$ where
$$
z_{i}^{*}=-\dfrac{\partial}{\partial z_{i}}\log Q(z),\text{~~ for~~ } i=1,\ldots,N.
$$
We call this group $G$.

If we write
\begin{equation}
D_{z}=Q\left(\frac{\partial}{\partial z}\right),\label{art13-eq3.7}
\end{equation}
we shall show that for $g$ in $G$,
\begin{equation}
D^{r}_{gz}=(j_{g}(z)^{-(1/2)(rq/N+1)}D^{r}_{z}(j_{g}(z))^{-(1/2)(rq/N-1)}\label{art13-eq3.8}
\end{equation}
where $r$ is any positive integer. If $gz=z+a$, \eqref{art13-eq3.8} is obvious and also for $gz=Az$ with $A$ in $G_{Y}$ so we really need to prove \eqref{art13-eq3.8} only for $gz=z^{*}$.

To do this, we first look at the $Y$ space. Actually $Y$ is a symmetric space; for any two points $y^{(1)}$ and $y^{(2)}$ in $Y$, there exists an $A$ in $G_{Y}$ such that $Ay^{(1)}=y^{(2)*}$,\pageoriginale $Ay^{(2)}=y^{(1)*}$. Thus $G_{Y}$ and the $*$ operation satisfy the conditions for $G$ and $\mu$ in Selberg \cite{art13-key3}.\footnote[10]{See Selberg \cite{art13-key10}, p. 51.} 

Also, we see that if $r$ is a positive integer, then
\begin{equation}
L_{y}=Q^{r}(y)Q^{r}\left(\frac{\partial}{\partial y}\right)\label{art13-eq3.9}
\end{equation}
is an operator invariant under the group $G_{Y}$.

It follows from a general result\footnote[11]{See Selberg \cite{art13-key3}, top of p. 53. In the context given there, the proof is obvious.} that under the $*$ operation the operator $L$ given by \eqref{art13-eq3.9} goes into the formal adjoint $L^{*}$ with respect to the invariant measure $dV_{y}$ or otherwise expressed
$$
L_{v^{*}}=L^{*}_{y}
$$
Thus for two suitable functions $f$ and $g$, we have
$$
\int\limits_{Y}f(y)L_{y}g(y)(Q(y))^{-N/q}dy=\int\limits_{Y}g(y)L^{*}_{y}f(y)(Q(y))^{-N/q}dy.
$$
Inserting the expression for $L$, we see that it is easy to find the formal adjoint $L^{*}$, since the formal adjoint of $Q^{r}\left(\dfrac{\partial}{\partial y}\right)$ with respect to the euclidean measure is $Q^{r}\left(-\dfrac{\partial}{\partial y}\right)=(-1)^{rq}Q^{r}\left(\dfrac{\partial}{\partial y}\right)$. We get
\begin{align*}
\int\limits_{Y}f(y)Lg(y)(Q(y))^{-N/q}dy &= \int\limits_{Y}f(y)(Q(y)^{r-N/q}Q^{r}\left(\frac{\partial}{\partial y}\right)g(y)dy\\[3pt]
&= \int\limits_{Y}g(y)Q^{r}\left(-\dfrac{\partial}{\partial y}\right)(Q(y))^{r-N/q}f(y)dy\\[3pt]
&= \int\limits_{Y}g(y)\left(Q(y)^{N/q}Q^{r}\left(-\dfrac{\partial}{\partial y}\right)Q(y)^{r-N/q}f(y)\right)\times{}\\
&\quad \times \frac{dy}{(Q(y))^{N/q}}.
\end{align*}
Thus
$$
L^{*}=Q^{N/q}(y)Q^{r}\left(-\dfrac{\partial}{\partial y}\right)Q^{r-N/q}(y).
$$
Also\pageoriginale
\begin{align*}
L^{*} &= Q^{r}(y^{*})Q^{r}\left(\dfrac{\partial}{\partial y^{*}}\right)\\
&= Q^{-r}(y)Q^{r}\left(\dfrac{\partial}{\partial y^{*}}\right).
\end{align*}
Comparing these two expressions for $L^{*}$, we get
\begin{align}
Q^{r}\left(\frac{\partial}{\partial y^{*}}\right) &= Q^{r+N.q}(y)Q^{r}\left(-\dfrac{\partial}{\partial y}\right)Q^{r-N/q}(y)\label{art13-eq3.10}\\
&= (-1)^{rq}Q^{r+N/q}(y)Q^{r}\left(\dfrac{\partial}{\partial y}\right)Q^{r-N/q}(y).\notag
\end{align}
But, from \eqref{art13-eq3.10}, it follows immediately that
\begin{equation}
Q^{r}\left(\dfrac{\partial}{\partial z^{*}}\right)=Q^{r+N/q}(z)Q^{r}\left(\dfrac{\partial}{\partial z}\right)Q^{r-N/q}(z).\label{art13-eq3.11}
\end{equation}
It remains to determine the jacobian of the mapping $z\to z^{*}$ or $j_{*}(z)$. We have
$$
\frac{\partial z^{*}_{i}}{\partial z_{j}}=-\frac{\partial^{2}}{\partial z_{i}\partial z_{j}}\log Q(z)
$$
so that
$$
j_{*}(z)=\left|-\frac{\partial^{2}}{\partial z_{i} \ \partial z_{j}}\log Q(z)\right|.
$$

If, as before, $dy$ denotes the euclidean volume element, we have, for the invariant volume element in $Y$,
$$
(Q(y^{*}))^{-N/q}dy^{*}=(Q(y))^{-N/q}dy
$$
or using \eqref{art13-eq3.4'},
$$
dy^{*}=(Q(y))^{-2N/q}dy.
$$
Since the symmetry $y\to y^{*}$ preserves orientation or not according as $N$ is even or odd, we get
\begin{align*}
&\left|\frac{\partial y_{i}^{*}}{\partial y_{j}}\right|=(-1)^{N}(Q(y))^{-2N/q},\\
\text{or}\quad & \left|\frac{\partial^{2}\log Q(y)}{\partial y_{i} \ \partial y_{j}}\right|=(-1)^{N}(Q(y))^{-2N/q}.
\end{align*}
It\pageoriginale is therefore obvious that
$$
j_{*}(z)=\left|-\dfrac{\partial^{2}\log Q(z)}{\partial z_{i} \ \partial z_{j}}\right|=(Q(z))^{-2N/q}.
$$
Combining this with \eqref{art13-eq3.11}, we get that \eqref{art13-eq3.8} holds also for $gz=z^{*}$; thus \eqref{art13-eq3.8} holds for all $g$ in $G$.

It is however clear that \eqref{art13-eq3.8}, which is really an algebraic identity, holds in a much larger group than $G$. Let us define $\widetilde{G}_{Y}$ as the group of complex matrices whose entries satisfy the algebraic relations which define $G_{Y}$ and consider the group $\widetilde{G}$ generated by translations $z\to z+a$ where $a$ may now be a complex vector, $z\to Az$ for $A$ in $\widetilde{G}_{Y}$ and $z\to z^{*}$. {\em Clearly} \eqref{art13-eq3.8} {\em as an algebraic identity holds for any transformation $g$ in $G$.}

The transformations of $\widetilde{G}$ do not, in general, map the positive half-space onto itself. $\widetilde{G}$ is actually large enough to map the positive half-space back into a bounded symmetric domain-in most cases, the original one (this being, for instance, true for the first three types listed at the end of \S\ref{art13-sec2}) --- or one may have to add a final unitary transformation which does not lie in $K_{0}$ (this being the case for type IV where the transformation $z_{1}\to z_{1}$, $z_{j}\to iz_{j}$ for $1<j\leq n$ would be needed at the end; for type IV, the positivity domain can be defined as $y_{1}>0$, $y^{2}_{1}-y^{2}_{2}-\cdots-y^{2}_{n}>0$ and we have $Q(y)=\frac{1}{2}(y^{2}_{1}-y^{2}_{2}-\cdots-y^{2}_{n})$; so the last transformation is needed to transform $z^{2}_{1}-z^{2}_{2}-\cdots-z^{2}_{n}$ into $z'z=z^{2}_{1}+\cdots+z^{2}_{n}$). At any rate, we get, in each case, the form of the differential operator and its transformation formula for the original bounded domain.

In the case of type I with $m=n$, we get, writing $\left|\dfrac{\partial}{\partial Z}\right|$ for $\left|\dfrac{\partial}{\partial z_{ij}}\right|$, that {\em if}
$$
gZ=(AZ+B)(CZ+D)^{-1}
$$
{\em where $A$, $B$, $C$, $D$ are complex $(n,n)$ matrices such that $\left|\begin{smallmatrix} A & B\\ C & D\end{smallmatrix}\right|=1$, then}
\begin{equation}
\left|\frac{\partial}{\partial gZ}\right|^{r}=|CZ+D|^{r+n}\left|\frac{\partial}{\partial Z}\right|^{r}|CZ+D|^{r-n}.\label{art13-eq3.12}
\end{equation}

In\pageoriginale the case of type II, for $Z=Z^{(2n,2n)}$ and $Z'=-Z$, {\em let $g$ be the transformation}
$$
gZ=(AZ+B)(CZ+D)^{-1}
$$
{\em where $A$, $B$, $C$ and $D$ are $(2n,2n)$ complex matrices with the property that for $M=\left(\begin{smallmatrix} A & B\\ C & D\end{smallmatrix}\right)$, $J=\left(\begin{smallmatrix} O & E\\ E & O\end{smallmatrix}\right)$, we have}
$$
M'JM=J
$$
and $|M|=1$. {\em Then writing $P_{f}\left(\dfrac{\partial}{\partial Z}\right)$ for $P_{f}\left(\dfrac{\partial}{\partial z_{ij}}\right)$ where $P_{f}$ is the Pfaffian, we have}
\begin{equation}
\left(P_{f}\left(\frac{\partial}{\partial gZ}\right)\right)^{r}=|CZ+D|^{(r+2n-1)/2}\left(P_{f}\left(\frac{\partial}{\partial Z}\right)\right)^{r}|CZ+D|^{(r-2n+1)/2}.\label{art13-eq3.13}
\end{equation}

{\em For type III, if we define}
$$
\left|\dfrac{\partial}{\partial Z}\right|=\left|\frac{1+\delta_{ij}}{2}\frac{\partial}{\partial z_{ij}}\right|
$$
{\em where $\delta_{ij}$ is the Kronecker symbol ($1$ on the main diagonal, $0$ off it) and $gZ=(AZ+B)(CZ+D)^{-1}$ where, for $M=\left(\begin{smallmatrix} A & B\\ C & D\end{smallmatrix}\right)$, $I=\left(\begin{smallmatrix} 0 & -E\\ E & 0\end{smallmatrix}\right)$, we have $M'IM=I$ and $|M|=1$, then again}
\begin{equation}
\left|\frac{\partial}{\partial gZ}\right|^{r}=|CZ+D|^{r+(n+1)/2}\left|\frac{\partial}{\partial Z}\right|^{r}|CZ+D|^{r-(n+1)/2}.\label{art13-eq3.14}
\end{equation}

In the case of type IV, we will confine ourselves to stating the form for the original bounded domain without defining the more general group $\widetilde{G}$ or giving the explicit forms of $g$ or the jacobian $j_{g}(z)$.

{\em If we define $D_{z}=\sum\limits^{n}_{i=1}\dfrac{\partial^{2}}{\partial z_{i}^{2}}$, then}
\begin{equation}
D^{r}_{gz}=(j_{g}(z))^{-(r/n+1/2)}D^{r}_{z}(j_{g}(z))^{-(r/n-1/2)}.\label{art13-eq3.15}
\end{equation}

For the type VI, we do not give explicit formulas.

Since these differential operators only contain differentiations with respect to $z$ and not $\overline{z}$, we see that if, for real $\alpha$, we put in a factor $(k(z,\overline{z}))^{-\alpha}$ on the left side and a factor $(k(z,\overline{z}))^{\alpha}$ on the right ($k$ being again the Bergmann kernel function), we again get an operator which satisfies \eqref{art13-eq1.4} but the two multipliers $\rho_{g}$ and $\sigma_{g}$ have each been multiplied by $(j_{g}(z))^{\alpha}$.

Besides\pageoriginale the operators $D$ so constructed, we may, of course, also consider their complex conjugates $\overline{D}$. These do not satisfy \eqref{art13-eq1.4}, since we required our multipliers to be analytic in $z$. We note that $\overline{D}$ would transform in the way
$$
\overline{D}_{gz}=\overline{(j_{g}(z))}^{-\alpha}\overline{D}_{z}\overline{(j_{g}(z))}^{\beta},
$$
where $\alpha$ and $\beta$ depend on $D$. If we now define
$$
\widetilde{D}_{z}=(k(z,\overline{z}))^{-\alpha}\overline{D}_{z}(k(z,\overline{z}))^{\beta},
$$
we see that
$$
\widetilde{D}_{gz}=(j_{g}(z))^{\alpha}\widetilde{D}_{z}(j_{g}(z))^{-\beta}
$$
and so this operator has the required behaviour. Here, since the differentiations in $\widetilde{D}_{z}$ are with respect to $\overline{z}$ and not $z$, we can clearly replace the pair $(\alpha,\beta)$ by any other pair of real numbers $(\alpha',\beta')$ as long as
$$
\alpha'-\beta'=\alpha-\beta.
$$

It can be shown that all differential operators which satisfy \eqref{art13-eq1.4} can be obtained by combining the operators $D$ or $\widetilde{D}$ with suitable invariant differential operators of the kind mentioned under $(a)$ at the beginning of \S\ref{art13-sec2}.

\section{}\label{art13-sec4}
To find the general form of integral operators that transform in the required way, we may again look at the representation of the domain $B$ as a positive half-space where the analytic mappings $gz$ are real, which is to say : $g\overline{z}=\overline{gz}$. We have, of course, also that $j_{g}(\overline{z})=\overline{j_{g}(z)}$.

Considering the Bergmann kernel function of this half-space, we get thus
\begin{align*}
k(gz,g\overline{\zeta}) &= k(gz,\overline{g\zeta})\\
&= (j_{g}(z)\overline{j_{g}(\zeta)})^{-1}k(z,\overline{\zeta})\\
&= (j_{g}(z)j_{g}(\overline{\zeta}))^{-1}k(z,\overline{\zeta}).
\end{align*}
If we now write $\zeta$ instead of $\overline{\zeta}$, this becomes
$$
k(gz,g\zeta)=(j_{g}(z)j_{g}(\zeta))^{-1}k(z,\overline{\zeta}).
$$

From this, we see that if we put
\setcounter{equation}{0}
\begin{equation}
h_{a,b}(z,\zeta)=\dfrac{(k(z,\overline{\zeta}))^{(a+b)/2}(k(z,\zeta))^{(a-b)/2}}{(k(\zeta,\overline{\zeta}))^{(a+b)/2}},\label{art13-eq4.1}
\end{equation}
where\pageoriginale 

