\title{ON RAMANUJAN'S ELLIPTIC INTEGRALS AND MODULAR IDENTITIES}
\markright{On Ramanujan's Elliptic Integrals and Modular Identities}

\author{By~ S. Raghavan and S. S. Rangachari}
\markboth{S. Raghavan and S. S. Rangachari}{On Ramanujan's Elliptic Integrals and Modular Identities}

\date{}
\maketitle

\setcounter{pageoriginal}{118}
\section*{Introduction}\pageoriginale
It is known from Hecke (\cite{art8-key5}, p. 472) that `special integrals of the third kind' of stufe $N$ (i.e. having logarithmic singularities at most at the cusps of the principal congruence subgroup $\Gamma(N)$ turn out to be of elementary type, namely, logarithms of functions invariant under $\Gamma(N)$. Results of this kind have been discovered much earlier by Ramanujan in \cite{art8-key11} as may be seen in the sequel, especially in \S\ref{art8-sec2}. In this connection, Ramanujan was perhaps, one of the first to have considered the problem of `evaluating' elliptic integrals associated with modular curves of small level, although elliptic integrals, in general, have been investigated in depth by various mathematicians such as Jacobi, Cayley and others. In the literature, transformations of orders 2, 3, 5 have been employed with a view to reduce formidable elliptic integrals to simpler (or more explicit) form \cite{art8-key4}.

In \cite{art8-key11}, Ramanujan has considered elliptic integrals associated with $\Gamma_{0}(N)$ for $N=5,7,10,14,15$ and also a solitary hyperelliptic integral (for $\Gamma_{0}(35)$). Various such elliptic integrals are found in scattered form in \cite{art8-key11}, with endeavours, via quadratic and higher order transformations and interesting modular relations, to simplify them.

The principal objective of this paper is to make a systematic study of all the elliptic integrals and associated formulae recorded by Ramanujan in \cite{art8-key11} and provide complete proofs.

We shall, in addition, uphold various modular identities involving Eisenstein series stated by Ramanujan in different places in \cite{art8-key11} and presumably used by him in computing singular values of modular functions. With the help of such identities, we exhibit a nonlinear differential equation for Eisenstein series denoted in \S\ref{art8-sec3} by $\mathscr{E}_{p}$, for $p=5,7$; for $p=5$, this differential equation is essentially equivalent to the nonlinear differential equation written down by ramanujan \cite{art8-key11} for a function $F(\lambda_{5})$, where $\lambda_{5}$ is a `Hauptmodul' for $\Gamma_{0}(5)$. One has to compare these with nonlinear differential equations obtained by Eichler-Zagier \cite{art8-key2} for divisor values of Weierstrass' elliptic function.

\section{Notation and preliminary results}\label{art8-sec1}\pageoriginale
~

\subsection{}\label{art8-sec1.1}
Let $\mathfrak{H}$ denote the complex upper half-plane and for $z\in \mathfrak{H}$, let $x=e^{2\pi iz}$, so that $|x|<1$. By $\Gamma(1)=\Gamma$, we mean the modular group $\{\left(\begin{smallmatrix} a & b\\ c & d\end{smallmatrix}\right) |a,b,c,d\in \mathbb{Z}$, $ad-bc=1\}$, acting on $\mathfrak{H}$ via the analytic homeomorphisms $z\to (az+b)(cz+d)^{-1}$. As usual, $\Gamma(N):=\{\left(\begin{smallmatrix} a & b\\ c & d\end{smallmatrix}\right)\in \Gamma |c\equiv 0(\mod N)\}$. By $Q$, $R$ we mean the normalized Eisenstein series $E_{4}(z)=1+240\sum\limits^{\infty}_{n=1}\left(\sum\limits_{0<d|n}d^{3}\right)e^{2\pi inz}$, $E_{6}(z)=1-504\sum\limits^{\infty}_{n=1}\left(\sum\limits_{0<d|n}d^{5}\right)\times e^{2\pi inz}$ of weights 4 and 6 respectively. Further, $P$ stands for $E_{2}(z)=1-24\times \sum\limits^{\infty}_{n=1}\left(\sum\limits_{0<d|n}d\right)e^{2\pi inz}$. Connecting $E_{2}$, $E_{4}$ and $E_{6}$, we have the important relations (\cite{art8-key9}, p. 142) :
\begin{equation}
\begin{split}
& E_{4}-E^{2}_{2}=-12\vartheta (E_{2})\\
& E_{6}-E_{2}E_{4}=-3\vartheta (E_{4})\\
& E^{2}_{4}-E_{2}E_{6}=-2\vartheta (E_{6})
\end{split}\label{art8-eq1}
\end{equation}
where $\vartheta := \dfrac{1}{2\pi i}\dfrac{d}{dz}=x\dfrac{d}{dx}$.

\smallskip

Following Ramanujan, we write for $|x|$, $|x'|<1$,
\begin{align*}
f(x,x') :&= 1+\sum\limits^{\infty}_{n=1}(xx')^{n(n-1)/2}(x^{n}+(x')^{n})\\
         &= \prod\limits^{\infty}_{n=0}[1+x(xx')^{n}][1+x'(xx')^{n}][1-(xx')^{n+1})\\
         &\hspace{3cm}\text{(from Gauss and Jacobi).}
\end{align*}
Further, setting $f(-x):=f(-x,-x^{2})$ for $|x|<1$, we know that $\eta(z)=x^{1/24}f(-x)$ is just Dedekind's $\eta$-function; indeed, $\eta(z)=e^{2\pi iz/24}\times \prod\limits^{\infty}_{n=1}(1-e^{2\pi inz})$ and moreover, $\eta^{24}=(1/1728)(E^{3}_{4}-E^{2}_{6})$. For rational $r>0$, let $\eta_{r}$ be defined by $\eta_{r}(z)=\eta(rz)$ for $z\in \mathfrak{H}$. Then the function $x^{-1/8}\times \eta^{2}_{2}(z)/\eta(z)$ is just Ramanujan's function $\psi(z)$, as found in \cite{art8-key10;art8-key11}. Setting 
$$
u=x^{1/5}f(-x,-x^{4})/f(-x^{2},-x^{3})=x^{1/5}\prod\limits^{\infty}_{n=0}\frac{(1-x^{5n+1})(1-x^{5n+4})}{(1-x^{5n+2})(1-x^{5n+3})}
$$
we know that $u$ is a `Hauptmodul' for the group $\Gamma(5)$ of genus $0$. Also, $u$, can be represented as an infinite continued fraction continued fraction $u=\dfrac{x^{1/5}}{1+}\dfrac{x}{1+}\dfrac{x^{2}}{1+}\cdots$ Moreover,\pageoriginale we know from Ramanujan \cite{art8-key10} the modular relations :
\begin{align}
& 1/u-1-u=\eta_{1/5}/\eta_{5},\label{art8-key2}\\
& 1/u^{5}-11-u^{5}=\eta^{6}/\eta^{6}_{5}.\label{art8-key3}
\end{align}
Let us note that, on the right hand side of \eqref{art8-eq3}, we have just $1/\lambda_{5}$, where $\lambda_{5}:=\eta^{6}_{5}/\eta^{6}$ is just a `Hauptmodul' for the group $\Gamma_{0}(5)$ of genus $0$. Likewise, for the congruence subgroup $\Gamma_{0}(7)$ of genus $0$, we have $\lambda_{7} := \eta^{4}_{7}/\eta^{4}$ as a `Hauptmodul'.

Between $u$ and $u_{2}$ defined by $u_{2}(z)=u(2z)$, we have the modular relation (\cite{art8-key10}, p. 326) :
\begin{align}
u_{2}/u^{2} &= (1+uu^{2}_{2})/(1-uu^{2}_{2})\notag\\
          &= (1+k)/(1-k)\label{art8-key4}
\end{align}
where we have written $k$ for $uu^{2}_{2}$ following Ramanujan (\cite{art8-key11}, p. \mycirc{70}/78). From \eqref{art8-eq4}, we see easily that
\begin{equation}
u^{5}=k(1-k)^{2}/(1+k)^{2}, \ u^{5}_{2}=k^{2}(1+k)/(1-k).\label{art8-eq5}
\end{equation}

The group $\Gamma_{0}(p)$ for odd primes $p$ is known to have two cusps $0$ and $\infty$ and Eisenstein series of Nebentypus $(-l, p, \chi_{p})$ corresponding to the two cusps are given. for $l\geq 2$, by
\begin{gather}
E^{0}_{l}(z;\chi_{p}):= \sum\limits^{\infty}_{n=1}\left(\sum\limits_{1\leq d|n}d^{l-1}\chi_{p}(n/d)\right)e^{2\pi inz}\label{art8-eq6}\\
E^{\infty}_{l}(z;\chi_{p}):=\frac{(-1)^{[l/2]}p^{[l-1]/2}}{(2\pi)^{l}}(l-1)!\sum\limits^{\infty}_{n=1}\chi_{p}(n)n^{-l}+\notag\\
\sum\limits^{\infty}_{n=1}\left(\sum\limits_{1<dn}d^{l-1}\chi_{p}(d)\right)e^{2\pi inz}\notag
\end{gather}
where $\chi_{p}(m)$ denotes the Legendre symbol $(\frac{m}{p}),[\quad]$ denotes the integral part and further, $\chi_{p}(-1)=(-1)^{l}$, necessarily (\cite{art8-key5}, p. 818). There exist two linearly independent series $E_{l}(z)$, $E_{l}(pz)$ for even $l>2$, which are of Haupttypus $(-l,p,1)$; however, for $l=2$, there is just one Eisenstein series of Haupttypus $(-2, p, 1)$ and it is given by
\begin{align*}
E_{2}(z;1;\Gamma_{0}(p)) &= E_{2}(z)-pE_{2}(pz)\\
&= 1-p-24\sum\limits^{\infty}_{\substack{n=1\\ p\nmid d}}\left(\sum\limits_{1\leq d|n}\right)e^{2\pi inz}
\end{align*}

For\pageoriginale $p=5$ and $l=2$, we know from \cite{art8-key8} that $E^{0}_{2}(z;\chi_{5})=\eta^{5}_{5}/\eta$ and further, denoting simply by $E_{2}(z;\chi_{5})$ obtained by `normalizing' $E^{\infty}_{2}(z;\chi_{5})$ so as to have constant term $-(1/5)$, we also have $E_{2}(z;\chi_{5})=-(1/5)\eta^{5}/\eta_{5}$.

In the case $p=7$ and $l=3$, $E^{\infty}_{3}(z;\chi_{7})$ is upto a constant factor, just the cube of the Eisenstein series $E_{1}(z;\chi_{7})$ of Nebentypus $(-1,7,\chi_{7})$ given by
$$
E_{1}(z;\chi_{7})=1+2\sum\limits^{\infty}_{n+1}\chi_{7}(n)\dfrac{x^{n}}{1-x^{n}}\quad \text{(see \cite{art8-key11}, \cite{art8-key8})}.
$$

Logarithmic differentiation of $\eta$ with respect to $x$ gives
\begin{equation}
\frac{24}{\eta}x\dfrac{d\eta}{dx}=\dfrac{24}{2\pi i\eta}\dfrac{d\eta}{dz}=E_{2}(z)\label{art8-eq8}
\end{equation}
Similarly, logarithmic differentiation of $u=x^{1/5}\prod\limits^{\infty}_{n=1}(1-x^{n})^{\chi_{5}(n)}$ and $u_{2}=x^{2/5}\prod\limits^{\infty}_{n=1}(1-x^{2n})^{\chi_{5}(n)}$ gives
\begin{align}
\frac{1}{u}\dfrac{du}{dx} &= \frac{1}{5x}-\sum\limits^{\infty}_{n=1}\frac{n\chi_{5}(n)x^{n-1}}{1-x^{n}}\label{art8-eq9}\\
&= -\frac{1}{x}E_{2}(z:\chi_{5}),\notag\\[4pt]
\frac{1}{u_{2}}\frac{du_{2}}{dx} &= -\frac{2}{x}E_{2}(2z;\chi_{5}).\notag
\end{align}
For the Hauptmodul $\lambda_{p}=(\eta_{p}/\eta)^{24/(p-1)}$ for $p=5$, $7$ we have likewise
\begin{equation}
\frac{1-p}{2\pi i\cdot \lambda_{p}}\frac{d\lambda_{p}}{dz}=E_{2}(z;1;\Gamma_{0}(p))\label{art8-eq10}
\end{equation}
in view of \eqref{art8-eq8} and \eqref{art8-eq7}.

Setting $r=\dfrac{k}{1-k^{2}}$, $s=\dfrac{1+k-k^{2}}{1-4k-k^{2}}$ where $k$ is just the function $uu^{2}_{2}$ considered above, we recall from Weber (\cite{art8-key14}, p. 86) the formulae 
\begin{align*}
rs^{5} &= x\prod\limits^{\infty}_{n=1}(1+x^{n})^{24}=(\eta_{2}/\eta)^{24}\\
r^{5}s &= x^{5}\prod\limits^{\infty}_{n=1}(1+x^{5n})^{24}=(\eta_{10}/\eta_{5})^{24}.
\end{align*}
These\pageoriginale
