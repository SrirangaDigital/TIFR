\title{ON RAMANUJAN'S ELLIPTIC INTEGRALS AND MODULAR IDENTITIES}
\markright{On Ramanujan's Elliptic Integrals and Modular Identities}

\author{By~ S. Raghavan and S. S. Rangachari}
\markboth{S. Raghavan and S. S. Rangachari}{On Ramanujan's Elliptic Integrals and Modular Identities}

\date{}
\maketitle

\setcounter{pageoriginal}{118}
\section*{Introduction}\pageoriginale
It is known from Hecke (\cite{art8-key5}, p. 472) that `special integrals of the third kind' of stufe $N$ (i.e. having logarithmic singularities at most at the cusps of the principal congruence subgroup $\Gamma(N)$ turn out to be of elementary type, namely, logarithms of functions invariant under $\Gamma(N)$. Results of this kind have been discovered much earlier by Ramanujan in \cite{art8-key11} as may be seen in the sequel, especially in \S\ref{art8-sec2}. In this connection, Ramanujan was perhaps, one of the first to have considered the problem of `evaluating' elliptic integrals associated with modular curves of small level, although elliptic integrals, in general, have been investigated in depth by various mathematicians such as Jacobi, Cayley and others. In the literature, transformations of orders 2, 3, 5 have been employed with a view to reduce formidable elliptic integrals to simpler (or more explicit) form \cite{art8-key4}.

In \cite{art8-key11}, Ramanujan has considered elliptic integrals associated with $\Gamma_{0}(N)$ for $N=5,7,10,14,15$ and also a solitary hyperelliptic integral (for $\Gamma_{0}(35)$). Various such elliptic integrals are found in scattered form in \cite{art8-key11}, with endeavours, via quadratic and higher order transformations and interesting modular relations, to simplify them.

The principal objective of this paper is to make a systematic study of all the elliptic integrals and associated formulae recorded by Ramanujan in \cite{art8-key11} and provide complete proofs.

We shall, in addition, uphold various modular identities involving Eisenstein series stated by Ramanujan in different places in \cite{art8-key11} and presumably used by him in computing singular values of modular functions. With the help of such identities, we exhibit a nonlinear differential equation for Eisenstein series denoted in \S\ref{art8-sec3} by $\mathscr{E}_{p}$, for $p=5,7$; for $p=5$, this differential equation is essentially equivalent to the nonlinear differential equation written down by ramanujan \cite{art8-key11} for a function $F(\lambda_{5})$, where $\lambda_{5}$ is a `Hauptmodul' for $\Gamma_{0}(5)$. One has to compare these with nonlinear differential equations obtained by Eichler-Zagier \cite{art8-key2} for divisor values of Weierstrass' elliptic function.

\section{Notation and preliminary results}\label{art8-sec1}\pageoriginale
~

\subsection{}\label{art8-sec1.1}
Let $\mathfrak{H}$ denote the complex upper half-plane and for $z\in \mathfrak{H}$, let $x=e^{2\pi iz}$, so that $|x|<1$. By $\Gamma(1)=\Gamma$, we mean the modular group $\{\left(\begin{smallmatrix} a & b\\ c & d\end{smallmatrix}\right) |a,b,c,d\in \mathbb{Z}$, $ad-bc=1\}$, acting on $\mathfrak{H}$ via the analytic homeomorphisms $z\to (az+b)(cz+d)^{-1}$. As usual, $\Gamma(N):=\{\left(\begin{smallmatrix} a & b\\ c & d\end{smallmatrix}\right)\in \Gamma |c\equiv 0(\mod N)\}$. By $Q$, $R$ we mean the normalized Eisenstein series $E_{4}(z)=1+240\sum\limits^{\infty}_{n=1}\left(\sum\limits_{0<d|n}d^{3}\right)e^{2\pi inz}$, $E_{6}(z)=1-504\sum\limits^{\infty}_{n=1}\left(\sum\limits_{0<d|n}d^{5}\right)\times e^{2\pi inz}$ of weights 4 and 6 respectively. Further, $P$ stands for $E_{2}(z)=1-24\times \sum\limits^{\infty}_{n=1}\left(\sum\limits_{0<d|n}d\right)e^{2\pi inz}$. Connecting $E_{2}$, $E_{4}$ and $E_{6}$, we have the important relations (\cite{art8-key9}, p. 142) :
\begin{equation}
\begin{split}
& E_{4}-E^{2}_{2}=-12\vartheta (E_{2})\\
& E_{6}-E_{2}E_{4}=-3\vartheta (E_{4})\\
& E^{2}_{4}-E_{2}E_{6}=-2\vartheta (E_{6})
\end{split}\label{art8-eq1}
\end{equation}
where $\vartheta := \dfrac{1}{2\pi i}\dfrac{d}{dz}=x\dfrac{d}{dx}$.

\smallskip

Following Ramanujan, we write for $|x|$, $|x'|<1$,
\begin{align*}
f(x,x') :&= 1+\sum\limits^{\infty}_{n=1}(xx')^{n(n-1)/2}(x^{n}+(x')^{n})\\
         &= \prod\limits^{\infty}_{n=0}[1+x(xx')^{n}][1+x'(xx')^{n}][1-(xx')^{n+1})\\
         &\hspace{3cm}\text{(from Gauss and Jacobi).}
\end{align*}
Further, setting $f(-x):=f(-x,-x^{2})$ for $|x|<1$, we know that $\eta(z)=x^{1/24}f(-x)$ is just Dedekind's $\eta$-function; indeed, $\eta(z)=e^{2\pi iz/24}\times \prod\limits^{\infty}_{n=1}(1-e^{2\pi inz})$ and moreover, $\eta^{24}=(1/1728)(E^{3}_{4}-E^{2}_{6})$. For rational $r>0$, let $\eta_{r}$ be defined by $\eta_{r}(z)=\eta(rz)$ for $z\in \mathfrak{H}$. Then the function $x^{-1/8}\times \eta^{2}_{2}(z)/\eta(z)$ is just Ramanujan's function $\psi(z)$, as found in \cite{art8-key10;art8-key11}. Setting 
$$
u=x^{1/5}f(-x,-x^{4})/f(-x^{2},-x^{3})=x^{1/5}\prod\limits^{\infty}_{n=0}\frac{(1-x^{5n+1})(1-x^{5n+4})}{(1-x^{5n+2})(1-x^{5n+3})}
$$
we know that $u$ is a `Hauptmodul' for the group $\Gamma(5)$ of genus $0$. Also, $u$, can be represented as an infinite continued fraction continued fraction $u=\dfrac{x^{1/5}}{1+}\dfrac{x}{1+}\dfrac{x^{2}}{1+}\cdots$ Moreover,\pageoriginale we know from Ramanujan \cite{art8-key10} the modular relations :
\begin{align}
& 1/u-1-u=\eta_{1/5}/\eta_{5},\label{art8-eq2}\\
& 1/u^{5}-11-u^{5}=\eta^{6}/\eta^{6}_{5}.\label{art8-eq3}
\end{align}
Let us note that, on the right hand side of \eqref{art8-eq3}, we have just $1/\lambda_{5}$, where $\lambda_{5}:=\eta^{6}_{5}/\eta^{6}$ is just a `Hauptmodul' for the group $\Gamma_{0}(5)$ of genus $0$. Likewise, for the congruence subgroup $\Gamma_{0}(7)$ of genus $0$, we have $\lambda_{7} := \eta^{4}_{7}/\eta^{4}$ as a `Hauptmodul'.

Between $u$ and $u_{2}$ defined by $u_{2}(z)=u(2z)$, we have the modular relation (\cite{art8-key10}, p. 326) :
\begin{align}
u_{2}/u^{2} &= (1+uu^{2}_{2})/(1-uu^{2}_{2})\notag\\
          &= (1+k)/(1-k)\label{art8-eq4}
\end{align}
where we have written $k$ for $uu^{2}_{2}$ following Ramanujan (\cite{art8-key11}, p. \mycirc{70}/78). From \eqref{art8-eq4}, we see easily that
\begin{equation}
u^{5}=k(1-k)^{2}/(1+k)^{2}, \ u^{5}_{2}=k^{2}(1+k)/(1-k).\label{art8-eq5}
\end{equation}

The group $\Gamma_{0}(p)$ for odd primes $p$ is known to have two cusps $0$ and $\infty$ and Eisenstein series of Nebentypus $(-l, p, \chi_{p})$ corresponding to the two cusps are given. for $l\geq 2$, by
\begin{gather}
E^{0}_{l}(z;\chi_{p}):= \sum\limits^{\infty}_{n=1}\left(\sum\limits_{1\leq d|n}d^{l-1}\chi_{p}(n/d)\right)e^{2\pi inz}\label{art8-eq6}\\
E^{\infty}_{l}(z;\chi_{p}):=\frac{(-1)^{[l/2]}p^{[l-1]/2}}{(2\pi)^{l}}(l-1)!\sum\limits^{\infty}_{n=1}\chi_{p}(n)n^{-l}+\notag\\
\sum\limits^{\infty}_{n=1}\left(\sum\limits_{1<dn}d^{l-1}\chi_{p}(d)\right)e^{2\pi inz}\notag
\end{gather}
where $\chi_{p}(m)$ denotes the Legendre symbol $(\frac{m}{p}),[\quad]$ denotes the integral part and further, $\chi_{p}(-1)=(-1)^{l}$, necessarily (\cite{art8-key5}, p. 818). There exist two linearly independent series $E_{l}(z)$, $E_{l}(pz)$ for even $l>2$, which are of Haupttypus $(-l,p,1)$; however, for $l=2$, there is just one Eisenstein series of Haupttypus $(-2, p, 1)$ and it is given by
\begin{align*}
E_{2}(z;1;\Gamma_{0}(p)) &= E_{2}(z)-pE_{2}(pz)\\
&= 1-p-24\sum\limits^{\infty}_{\substack{n=1\\ p\nmid d}}\left(\sum\limits_{1\leq d|n}\right)e^{2\pi inz}
\end{align*}

For\pageoriginale $p=5$ and $l=2$, we know from \cite{art8-key8} that $E^{0}_{2}(z;\chi_{5})=\eta^{5}_{5}/\eta$ and further, denoting simply by $E_{2}(z;\chi_{5})$ obtained by `normalizing' $E^{\infty}_{2}(z;\chi_{5})$ so as to have constant term $-(1/5)$, we also have $E_{2}(z;\chi_{5})=-(1/5)\eta^{5}/\eta_{5}$.

In the case $p=7$ and $l=3$, $E^{\infty}_{3}(z;\chi_{7})$ is upto a constant factor, just the cube of the Eisenstein series $E_{1}(z;\chi_{7})$ of Nebentypus $(-1,7,\chi_{7})$ given by
$$
E_{1}(z;\chi_{7})=1+2\sum\limits^{\infty}_{n+1}\chi_{7}(n)\dfrac{x^{n}}{1-x^{n}}\quad \text{(see \cite{art8-key11}, \cite{art8-key8})}.
$$

Logarithmic differentiation of $\eta$ with respect to $x$ gives
\setcounter{equation}{7}
\begin{equation}
\frac{24}{\eta}x\dfrac{d\eta}{dx}=\dfrac{24}{2\pi i\eta}\dfrac{d\eta}{dz}=E_{2}(z)\label{art8-eq8}
\end{equation}
Similarly, logarithmic differentiation of $u=x^{1/5}\prod\limits^{\infty}_{n=1}(1-x^{n})^{\chi_{5}(n)}$ and $u_{2}=x^{2/5}\prod\limits^{\infty}_{n=1}(1-x^{2n})^{\chi_{5}(n)}$ gives
\begin{align}
\frac{1}{u}\dfrac{du}{dx} &= \frac{1}{5x}-\sum\limits^{\infty}_{n=1}\frac{n\chi_{5}(n)x^{n-1}}{1-x^{n}}\label{art8-eq9}\\
&= -\frac{1}{x}E_{2}(z:\chi_{5}),\notag\\[4pt]
\frac{1}{u_{2}}\frac{du_{2}}{dx} &= -\frac{2}{x}E_{2}(2z;\chi_{5}).\notag
\end{align}
For the Hauptmodul $\lambda_{p}=(\eta_{p}/\eta)^{24/(p-1)}$ for $p=5$, $7$ we have likewise
\begin{equation}
\frac{1-p}{2\pi i\cdot \lambda_{p}}\frac{d\lambda_{p}}{dz}=E_{2}(z;1;\Gamma_{0}(p))\label{art8-eq10}
\end{equation}
in view of \eqref{art8-eq8} and \eqref{art8-eq7}.

Setting $r=\dfrac{k}{1-k^{2}}$, $s=\dfrac{1+k-k^{2}}{1-4k-k^{2}}$ where $k$ is just the function $uu^{2}_{2}$ considered above, we recall from Weber (\cite{art8-key14}, p. 86) the formulae 
\begin{align*}
rs^{5} &= x\prod\limits^{\infty}_{n=1}(1+x^{n})^{24}=(\eta_{2}/\eta)^{24}\\
r^{5}s &= x^{5}\prod\limits^{\infty}_{n=1}(1+x^{5n})^{24}=(\eta_{10}/\eta_{5})^{24}.
\end{align*}
These\pageoriginale imply immediately that $r^{24}=(r^{5}s^{5})/rs^{5}$ and so
\begin{align}
& \frac{1-k^{2}}{k} =\frac{1}{r}=\frac{\eta_{2}\eta^{5}_{5}}{\eta\eta^{5}_{10}}\label{art8-eq11}\\
&=\frac{1}{x\psi^{2}(x^{5})}\frac{\eta_{2}\eta^{3}_{5}}{x^{1/4}\eta\eta_{10}}\notag\\
&=\frac{1}{x\psi^{2}(x^{5})}\cdot f(x,x^{4})f(x^{2},x^{3})\notag\\
&=\frac{\psi^{2}(x)-x\psi^{2}(x^{5})}{x\psi^{2}(x^{5})}\label{art8-eq12}.
\end{align}
The last-mentioned equality is a consequence of Ramanujan's identity
$$
\psi^{2}(x)-x\psi^{2}(x^{5})=f(x,x^{4})f(x^{2},x^{3})
$$
proved by Watson \cite{art8-key13} and called a ``rudimentary'' example of the use of quadratic forms. (See also \cite{art8-key1}, pp. 63-65 which provides a proof ostensibly ``more difficult'' and further ``similar to that of Entry 9(iii), which is obviously an analogue of Entry 10(v), a fact made even more transparent by Entry 10(iv)''). In contrast, the following Proposition 1(ii) gives an independent, refreshingly different and perhaps even elegant proof for Ramanujan's identity above; in fact, one needs merely to substitute $\psi(x)=x^{-1/8}\eta^{2}_{2}(z)/\eta(z)$ therein and note further that $f(x,x^{4})f(x^{2},x^{3})=x^{-1/4}\eta_{2}(z)\eta^{3}_{5}(z)/(n(z)\eta_{10}(z))$. This proof corroborates G.N. Watson's belief that Ramanujan discovered this formula ``not by manipulating quad\-ratic forms but by transforming series of Lambert's type''.

\medskip
\noindent
{\bf Proposition \thnum{1}.\label{art8-prop1}}
\begin{itemize}
\item[(i)] $\eta^{4}_{2}\eta^{2}_{5}-5\eta^{2}\eta^{4}_{10}=\eta^{5}\eta_{5}\eta_{10}/\eta_{2}$

\item[(ii)] $\eta^{4}_{2}\eta^{2}_{5}-\eta^{2}\eta^{4}_{10}=\eta\eta_{2}\eta^{5}_{5}.\eta_{10}$.
\end{itemize}

\begin{proof}
We first rewrite these identities in terms of the Eisenstein series $E^{0}_{2}(z;\chi_{5})$ and $E^{\infty}_{2}(z;\chi_{5})$ obtained by normalizing $E^{\infty}_{2}(z;\chi_{5})$, using the relations $E^{0}_{2}(z;\chi_{5})=\eta^{5}_{5}/\eta$, $E_{2}(z;\chi_{5})=-\frac{1}{5}\eta^{5}/\eta_{5}$ and the `Hauptmodul' $\tau : 10\eta_{2}\eta^{3}_{10}/(\eta^{3}\eta_{5})$ for $\Gamma_{0}(10)$, as follows :
\begin{itemize}
\item[(i)$'$] $E_{2}(z;\chi_{5})-E_{2}(2z;\chi_{5})=-\dfrac{\tau}{2}E_{2}(z;\chi_{5})$

\item[(ii)$'$] $E^{0}_{2}(z;\chi_{5})+E^{0}_{2}(2z;\chi_{5})=\dfrac{\eta^{4}_{2}\eta^{2}_{5}}{\eta^{2}\eta^{4}_{10}}E^{0}_{2}(2z;\chi_{5})$
\end{itemize}
\end{proof}

By\pageoriginale direct checking (see also \cite{art8-key3}, p. 449) we see that the modular function $\tau$ has a simple zero at $i\infty$ and a simple pole at $0$. On the other hand, $E_{2}(z;\chi_{5})$ is regular at $i\infty$ and in view of the relation $E_{2}(-1/z;\chi_{5})=(z^{2};\sqrt{5})E^{0}_{2}(z/5)$ from (\cite{art8-key5}, p. 819), $E_{2}(z;\chi_{5})$ has a zero at $0$. Consequently, $(\tau/2+1)E_{2}(z;\chi_{5})$ which is regular at $i\infty$ and $0$ is indeed an entire modular form of weight 2 (and nebentypus) for $\Gamma_{0}(10)$. Thus the proof of (i)$'$ reduces to showing that $(\tau/2+1)E_{2}(z;\chi_{5})=E_{2}(2z;\chi_{5})$. In view of Hecke's result (\cite{art8-key5}, Satz 2, p. 811 --- see also p. 953), it is enough to compare the first $\dfrac{[\Gamma(1):\Gamma_{0}(10)]\times 2}{12}(=3)$ coefficients in the Fourier expansions of both sides. That the first three Fourier coefficients agree on both sides is immediate from the following expansions :
\begin{align*}
E_{2}(z;\chi_{5}) &= -\frac{1}{5}+e^{2\pi iz}-e^{4\pi iz}-2e^{6\pi iz}+\cdots\\
(\tau/2+1) &= 1+5e^{2\pi iz}+15e^{4\pi iz}+40e^{6\pi iz}+\cdots\\
(\tau/2+1)E_{2}(z;\chi_{5}) &= -\frac{1}{5}+0\cdot e^{2\pi iz}+e^{4\pi iz}+0\cdot e^{6\pi iz}+\cdots\\
E_{2}(2z;\chi_{5}) &= -\frac{1}{5}+0\cdot e^{2\pi iz}+e^{4\pi iz}+0\cdot e^{6\pi iz}+\cdots
\end{align*}
This proves (i)$'$ and therefore (i). Using (i), we may rewrite the factor $\eta^{4}_{2}\eta^{2}_{5}/(\eta^{2}\eta^{4}_{10})$ on the right hand side of (ii)$'$ as $5+10/\tau$. Hence (ii)$'$ will follow, if we establish
\begin{itemize}
\item[(ii)$''$] $\dfrac{\tau}{4\tau+10}E^{0}_{2}(z;\chi_{5})=E^{0}_{2}(2z;\chi_{5})$.
\end{itemize}
Now $\tau/(4\tau+10)$ is seen to be regular at all the cusps of $\Gamma_{0}(10)$, except those equivalent to $\pm 1/5$ where it has a simple pole. Further, $E^{0}_{2}(z;\chi_{5})$ of Nebentypus $(-2,5,\chi_{5})$ has a zero at $\infty$ and hence at $\pm 1/5$ (equivalent to $\infty$ under $\Gamma_{5}(5)$. Consequently $[\tau/(4\tau+10)]E^{0}_{2}(z;\chi_{5})$ is an entire modular form of weight 2 (and Nebentypus) for $\Gamma_{0}(10)$. From the Fourier expansions
\begin{align*}
\frac{\tau}{4\tau+10} &= e^{2\pi iz}-e^{4\pi iz}+0\cdot e^{6\pi iz}+\cdots\\
E^{0}_{2}(z;\chi_{5}) &= e^{2\pi iz}+e^{4\pi iz}+2\cdot e^{6\pi iz}+\cdots,
\end{align*}
it is clear that the first three Fourier coefficients of $[\tau/(4\tau+10)]E^{0}_{2}(z;\chi_{5})$ coincide with the corresponding coefficients of $E^{0}_{2}(2z;\chi_{5})$. This proves (ii)$''$ by Hecke's theorem above and hence (ii) is proved.

\begin{remark*}
As\pageoriginale a further illustration for the utility of Proposition \ref{art8-prop1}, we give an alternative proof for Ramanujan's identity
\begin{gather*}
x\psi^{3}(x)\psi(x^{5})-5x^{2}\psi(x)\psi^{3}(x^{5})\\
=\frac{x}{1-x^{2}}-\frac{2x^{2}}{1-x^{4}}-\frac{3x^{3}}{1-x^{6}}+\frac{4x^{4}}{1-x^{8}}+\frac{6x^{6}}{1-x^{12}}-\cdots
\end{gather*}
(See \cite{art8-key1}, pp. 45-49 for a ``rather difficult'' proof which uses besides ``results from Section 13'', leading nevertheless to no ``circular reasoning'' etc.). The right hand side of this identity is just
\begin{align*}
\sum\limits^{\infty}_{n=1}\frac{nx^{n}\chi_{5}(n)}{1-x^{2n}} &= \sum\limits^{\infty}_{m=0}\sum\limits^{\infty}_{n=1}n\chi_{5}(n)x^{n(2m+1)}\\
&= E^{\infty}_{2}(z;\chi_{5})-E^{0}_{2}(2z,\chi_{5})\\
&= \frac{1}{5}\frac{\eta^{5}}{\eta_{5}}+\frac{1}{5}\frac{\eta^{5}_{2}}{\eta_{10}}\quad (\text{by~ \cite{art8-key8}, p. 227})\\
&= \frac{\eta^{2}}{5\eta^{2}_{5}\eta_{10}}\left(\eta^{4}_{2}\eta^{2}_{5}-\frac{\eta^{5}\eta_{5}\eta_{10}}{\eta_{2}}\right)\\
&= \eta^{2}\eta_{2}\eta^{3}_{10}/\eta^{2}_{5},\text{~~ by Proposition 1(ii),}
\end{align*}
while the left hand side is precisely $\dfrac{\eta^{6}_{2}\eta^{2}_{10}}{\eta^{3}\eta_{5}}-5\dfrac{\eta^{2}_{2}\eta^{6}_{10}}{\eta\eta^{3}_{5}}=\dfrac{\eta^{2}_{2}\eta^{2}_{10}}{\eta^{3}\eta^{3}_{5}}\times (\eta^{4}_{2}\eta^{2}_{5}-5\eta^{2}\eta^{4}_{10})=\dfrac{\eta^{2}\eta_{2}\eta^{3}_{10}}{\eta^{2}_{5}}$, using Proposition 1(i).
\end{remark*}

\subsection{}\label{art8-sec1.2}
We have gathered here, from Chapter XIX of the Notebooks, many of Ramanujan's identities involving the functions $\varphi(q):=1+2\sum\limits^{\infty}_{n=1}q^{n^{2}}$ or $\psi(q):=q^{-1/8}\eta^{2}(2z)/\eta(z)$ introduced by him in Chapter XVI with $q:=\exp (2\pi iz)$ or various transforms of $\varphi$ and $\psi$. We shall rewrite the identities (with a view to elucidate them) in terms of the normalized Eisenstein series $E_{4}(z)$ of weight 4 for $\Gamma_{0}(1)$, its transforms and the following Eisenstein series of Haupttypus or Nebentypus $(-k,N,\epsilon)$ with $\epsilon$ equal to the trivial character modulo $N$ or the real character $\epsilon(n):=(\frac{n}{N})$ modulo $N$ {\em and} $\epsilon(-1)=(-1)^{k}$ namely
\begin{gather*}
E_{k,N,1}(z)=E_{k,1}(z;\Gamma_{0}(N);\epsilon):=\sum\limits^{\infty}_{n=1}\left(\sum\limits_{1\leq d|n}\epsilon(n/d)d^{k-1}\right)e^{2\pi inz}\\
E_{k,N,z}(z)=E_{k,2}(z;\Gamma_{0}(N);\epsilon):=\gamma_{k}(N)+\sum\limits^{\infty}_{n=1}\left(\sum\limits_{1<d|n}\epsilon(d)d^{k-1}\right)e^{2\pi inz}
\end{gather*}
with\pageoriginale a constant $\gamma_{k}(N)$ that can be determined. We note that with $q=\exp(2\pi iz)$, $\varphi^{2}$ is a modular form of weight 1 with multipliers for the congruence subgroup $\Gamma_{0}(4)$. For dealing with products of transforms of $\psi$, we recall from Honda and Miyawaki (J. Math. Soc. Japan, 26(1974), 362-373) that the power product $\prod\limits^{r}_{j=1}\eta^{n_{j}}_{t_{j}}$ with integral $t_{1},\ldots,t_{r}, n_{1},\ldots,n_{r}$ is a modular form of weight $\frac{1}{2}(n_{1}+\cdots+n_{r})$ for the congruence subgroup $\Gamma_{0}(l.\displaystyle{\mathop{c}\limits_{j}}.m.t_{j})$ if 24 divides both $n_{1}t_{1}+\cdots+j_{r}t_{r}$ and $(l.\displaystyle{\mathop{c}\limits_{j}}m. \ t_{j})\times (n_{1}/t_{1}+\cdots+n_{r}/t_{r})$ Ramanujan's identities referred to may now be rewritten as follows.
\begin{enumerate}
\renewcommand{\labelenumi}{(\theenumi)}
\item $\varphi^{2}(q)=E_{1,2}(z;\Gamma_{0}(4);\epsilon)$

\item $\varphi^{4}(q)=8(E_{2,2}(z;\Gamma_{0}(4);1)-4E_{2,2}(4z;\Gamma_{0}(4);1))$

\item $\varphi^{8}(q)=\frac{1}{15}(E_{4}(z)-2E_{4}(2z)+16E_{4}(4z))$

\item $q\psi^{3}(q)\psi(q^{5})-5q^{2}\psi(q)\psi^{3}(q^{5})=E_{2,2}(z;\Gamma_{0}(5);1)-E_{2,2}(2z;\Gamma_{0}(5);1)$

\item $5\varphi(q)\varphi^{3}(q^{5})-\varphi^{3}(q)\varphi(q^{5})=4(E_{2,5,2}(z)-2E_{2,5,2}(2z)-4E_{2,5,2}(4z))$

\item $25\varphi(q)\varphi^{3}(q^{5})-\varphi^{5}(q)/\varphi(q^{5})=40(E_{2,5,2}(z)-4E_{2,5,2}(4z))$

\item $\psi^{5}(q)/\psi(q^{5})-25q^{2}\psi(q)\psi^{3}(q^{5})=5(E_{2,5,2}(z)-2E_{2,5,2}(2z))$

\item $q\psi^{5}(q)\psi(q^{3})-9q^{2}\psi(q)\psi^{5}(q^{3})=E_{3,3,2}(z)-E_{3,3,2}(2z)$

\item $9\varphi(q)\varphi^{5}(q^{3})-\varphi^{5}(q)\varphi(q^{3})=8(E_{3,3,2}(z)-2E_{3,3,2}(2z)-8E_{3,3,2}(4z))$

\item $\psi^{3}(q)/\psi(q^{3})=E_{1,3,2}(z)-E_{1,3,2}(2z)$

\item $\varphi^{3}(q)/\varphi(q^{3})=6(E_{1,3,2}(z)+2E_{1,3,2}(2z)-2E_{1,3,2}(4z))$

\item $q\psi(q^{2})\psi(q^{6})=E_{1,3,2}(z)-E_{1,3,2}(4z)$

\item $\varphi(q)\varphi(q^{3})=2(E_{1,3,2}(z)+2E_{1,3,2}(4z))$

\item $q\psi^{2}(q)\psi^{2}(q^{3})=E_{2,3,2}(z)-E_{2,3,2}(2z)$

\item $\varphi^{2}(q)\varphi^{2}(q^{3})=4(E_{2,3,2}(z)-2E_{2,3,2}(2z)+4E_{2,3,2}(4z))$

\item $q\psi(q)\psi(q^{7})=E_{1,7,2}(z)-E_{1,7,2}(2z)$

\item $\varphi(q)\varphi(q^{7})=E_{1,7,2}(z)-2E_{1,7,2}(2z)+2E_{1,7,2}(4z)$
\end{enumerate}
The identities are easily proved by noting that both sides are modular forms of weight $k$ for $\Gamma_{0}(M)$ for the appropriate value of $M$ and comparing their first $1+\frac{k}{2}(\Gamma_{0}(1):\Gamma_{0}(M)$ corresponding Fourier coefficients in the light of Hecke's theorem (\cite{art8-key5}, p. 811).

\begin{remark*}
Using Proposition \ref{art8-prop1}, we can prove the relation $\gamma=\frac{1}{\mu}\left(\frac{\mu-1}{\mu-5}\right)^{2}$ for the Hauptmoduls $\gamma:=\eta^{6}_{5}/\eta^{6}$ and $\mu:=\psi^{2}(q)/\psi^{2}(q^{5})$ for $\Gamma_{0}(5)$ and $\Gamma_{0}(10)$ respectively. Substituting this relation between $\gamma$ and $\mu$, we obtain
\begin{align*}
1+22\gamma+125\gamma^{2} &=\frac{1}{\mu^{2}(\mu-5)^{4}}\\
&\qquad[\mu^{2}(\mu-5)^{4}+22\mu(\mu-1)^{2}(\mu-5)^{2}+125(\mu-1)^{4}]\\
&= \frac{1}{\mu^{2}(\mu-5)^{4}}(\mu^{2}+2\mu+5)^{2}(\mu^{2}-2\mu+5)
\end{align*}\pageoriginale 
\end{remark*}

As a consequence, we obtain Entry 4(i) of Chapter XXI of the Notebooks in the following form :
$$
(\eta^{5}/\eta_{5})(1+22\gamma+125\gamma^{2})^{1/2}=\frac{\psi^{5}(q^{5})}{\psi(q)}(\mu^{2}+2\mu+5)(\mu^{2}-2\mu+5)^{1/2}
$$

We have, as an analogue of assertion (i) of Proposition \ref{art8-prop1}, the identity $\varphi^{2}(q)-\varphi^{2}(q^{5})=4\eta^{2}_{2}\eta_{5}\eta_{20}/\eta\eta_{4}$ which is a consequence of the remarks above. We note further that Entry 5(i) in Chapter XXI is equivalent to the easily proved relations :
$$
(E_{1,7,2}(z))^{2}=\left(1+2\Sigma\left(\dfrac{n}{7}\right)\frac{q^{n}}{1-q^{n}}\right)^{2}=E_{2,7,2}(z)=E_{2}(z)-7E_{2}(7z)
$$
with the usual definition of $E_{2}$ and
$$
(E_{1,7,2}(z))^{3}=\eta^{7}/\eta_{7}+13\eta^{3}\eta^{3}_{7}+49\eta^{7}_{7}/\eta.
$$
One needs, for these, merely to verify the equality of at most the first 3 corresponding Fourier coefficients on both sides, in view of Hecke's theorem again.

Other relations stated by Ramanujan for $\psi^{2}$, $\psi^{4}$, $\psi^{6}$, $\psi^{8}$, $\varphi^{3}/\varphi_{3}$, $\varphi^{3}_{3}/\varphi$, $\psi^{3}/\psi_{3}$, $\psi^{3}_{3}/\psi$, $\eta^{3}/\eta_{3}$, $\eta^{3}_{9}/\eta_{3}$ may be derived in the same manner.

The proofs of identities \eqref{art8-eq4}--\eqref{art8-eq11} as indicated above may be seen to be simpler than those in \cite{art8-key1} (p. 45-56, 12-16). One may also compare the proofs of identities \eqref{art8-eq12}-\eqref{art8-eq13} with those outlined in Hardy's ``Ramanujan'' (pp. 220-222)

\section{Elliptic integrals considered by Ramanujan}\label{art8-sec2}
~

\subsection{}\label{art8-sec2.1}
Let us being with the simplest example of elliptic integrals for which Ramanujan (\cite{art8-key11}, p. \mycirc{67} + 1) has written explicit primitives :
\begin{equation}
\int\limits^{x}_{e^{-2\pi}}\sqrt{Q}\dfrac{dx}{x}=\log \frac{Q^{3/2}-R}{Q^{3/2}+R}.\label{art8-eq13}
\end{equation}
Applying Ramanujan's ``very useful substitution'' $Z=R^{2}/Q^{3}$, we have $\dfrac{dZ}{Z}=2\dfrac{dR}{R}-3\dfrac{dQ}{Q}$ and hence \eqref{art8-eq1} leads to $\dfrac{1}{Z}x\dfrac{dZ}{dx}=\dfrac{R^{2}-Q^{3}}{QR}$. Now 
$$
x\dfrac{d}{dx}\log \dfrac{Q^{3/2}-R}{Q^{3/2}+R}=x\dfrac{dZ}{dx}\dfrac{d}{dZ}\log\left(\dfrac{1-\sqrt{Z}}{1+\sqrt{Z}}\right)=x\dfrac{dZ}{dx}\cdot \dfrac{1}{\sqrt{Z}(Z-1)}=\sqrt{Q}.
$$\pageoriginale
Since $R{\rm(i)}=0$, \eqref{art8-eq13} is proved.

\subsection{}\label{art8-sec2.2}
We take up next Ramanujan's formula on page \mycirc{70}/78 of \cite{art8-key11} explicitly evaluating an elliptic integral in elementary terms :
\begin{equation}
\frac{8}{5}\int \frac{\psi^{5}(x)}{\psi(x^{5})}\dfrac{dx}{x}=\log u^{2}u^{3}_{2}+\sqrt{5}\log \frac{1+\epsilon^{-3}uu^{2}_{2}}{1-\epsilon^{2}uu^{2}_{2}}\quad \left(\text{with~ } \epsilon =\dfrac{\sqrt{5+1}}{2}\right)\label{art8-eq14}
\end{equation}

From \eqref{art8-eq5}, $\log (u^{2}u^{3}_{2})=\dfrac{1}{5}\log\left(k^{8}\dfrac{(1-k)}{(1+k)}\right)$. Also $\dfrac{u}{u_{2}}=\dfrac{f(x^{2},x^{3})}{x^{1/5}f(x,x^{4})}$ by substituting the infinite product expansion for $f$. Using Ramanujan's identity (\cite{art8-key10} II, p. 234) proved by Berndt (\cite{art8-key1}, p. 57)
$$
\frac{\psi^{5}(x)}{\psi(x^{5})}-25x^{2}\psi(x)\psi^{3}(x^{5})=1-5x\dfrac{d}{dx}\log \frac{f(x^{2},x^{3})}{f(x,x^{4})}
$$
we may thus rewrite \eqref{art8-eq14} as
\begin{align*}
8\int\limits^{x}_{0}x\psi(x)\psi^{3}(x^{5})dx &= \log \frac{1-k}{1+k}+\dfrac{1}{\sqrt{5}}\log \frac{1+\epsilon^{-3}k}{1-\epsilon^{3}k}\\
&= \int\limits^{x}_{0}\dfrac{d}{dx}\left(\log \frac{1-k}{1+k}+\dfrac{1}{\sqrt{5}}\log \frac{1+\epsilon^{-3}k}{1-\epsilon^{3}k}\right)dx\\
&= \int\limits^{x}_{0}\frac{8kk'}{(1-k^{2})(1-4k-k^{2})}dx
\end{align*}
denoting differentiation with respect to $x$ by $'$. Therefore \eqref{art8-eq14} will be established, once we show that
\begin{equation}
\frac{k'}{1-k^{2}}=\left(\frac{1-k^{2}}{k}-4\right)x\psi(x)\psi^{3}(x^{5}).\label{art8-eq15}
\end{equation}
Now we obtain from \eqref{art8-eq4}, by logarithmic differentiation (with respect to $x$) that
\begin{align*}
k'(1-k^{2}) &= \frac{1}{2}u'_{2}/u_{2}-u'/u\\
&= \frac{1}{x}(E_{2}(z;\chi_{5})-E_{2}(2z;\chi_{5})),\quad\text{by \eqref{art8-eq9}}\\
&= \frac{1}{5x}(\eta^{5}_{2}/\eta_{10}-\eta^{5}/\eta_{5})\\
&= \frac{\eta_{2}}{5x\eta^{2}_{5}\eta_{10}}\cdot 5\eta^{2}\eta^{4}_{10},\quad \text{by (i) of Proposition \ref{art8-prop1}.}
\end{align*}\pageoriginale 
Thus using \eqref{art8-eq11}, we see that \eqref{art8-eq15} will be proved, if we show that
\begin{align}
& \frac{\eta^{2}\eta_{2}\eta^{3}_{10}}{x\eta^{2}_{5}}=\dfrac{\eta_{2}\eta^{5}_{5}-4\eta\eta^{5}_{10}}{\eta\eta^{5}_{10}}\cdot x\dfrac{\eta^{2}_{2}\eta^{6}_{10}}{x^{2}\eta\eta^{3}_{5}}\notag\\
\text{i.e.}\quad &\frac{\eta^{5}\eta_{5}\eta_{10}}{\eta_{2}}=\frac{\eta\eta_{2}\eta^{5}_{5}}{\eta_{10}}-4\eta^{2}\eta^{4}_{10}.\label{art8-eq16}
\end{align}
But the right hand side is just $\eta^{4}_{2}\eta^{2}_{5}-5\eta^{2}\eta^{4}_{10}$ by (ii) of proposition \ref{art8-prop1} and so \eqref{art8-eq16} is just identity (i) of the same proposition, proving Ramanujan's assertion \eqref{art8-eq14}.

\section{Elliptic integrals arising from cusp forms}\label{art8-sec3}
~

\subsection{}\label{art8-sec3.1}
On page \mycirc{67} $+1$ of \cite{art8-key11}, Ramanujan writes down the formula 
$$
\int\limits^{x}_{0}\eta\eta_{3}\eta_{5}\eta_{15}\dfrac{dx}{x}=\dfrac{1}{5}\quad \int\limits^{2\tan^{-1}(1/\sqrt{5})}_{2\tan^{-1}(1/\sqrt{5})\left(\sqrt{\frac{1-11v-v^{2}}{1+v-v^{2}}}\right)} \ \ \dfrac{d\phi}{\sqrt{1-\dfrac{9}{25}\sin^{2}\phi}}
$$
where $v:=\dfrac{\eta^{3}(z)\eta^{3}(15z)}{\eta^{3}(3z)\eta^{3}(5z)}$. The integrand on the left hand side is the `unique' holomorphic differential for $\Gamma_{0}(15)$.

For the proof of \eqref{art8-eq17}, we need to note the following from Fricke (\cite{art8-key3}, pp. 438-439). If $\tau : = (\eta_{3}\eta_{5}/\eta\eta_{15})^{3}$ and $\sigma = \dfrac{2\pi i}{4\pi^{2}\eta\eta_{3}\eta_{5}\eta_{15}}\dfrac{d\tau}{dz}$, then $\sigma^{2}=\tau^{4}-10\tau^{3}-13\tau^{2}+10\tau+1$; the two modular functions $\tau$ and $\sigma$ generate the field of modular functions for $\Gamma_{0}(15)$. Note that $\tau=e^{-2\pi iz}+3+\cdots$ has a simple pole at $i\infty$. Now
\setcounter{equation}{17}
\begin{align}
\int\limits^{x}_{0}\eta\eta_{3}\eta_{5}\eta_{15}\dfrac{dx}{x} &=2\pi i \int\limits^{z}_{i\infty}\eta\eta_{3}\eta_{5}\eta_{15} \ dz \quad (z=iy, y>0)\notag\\
&= \int\limits^{\infty}_{\tau}\dfrac{d\tau}{\sqrt{\tau^{4}-10\tau^{3}-13\tau^{2}+10\tau+1}}\label{art8-eq18}
\end{align}\pageoriginale
Further $X^{4}-10X^{3}-13X^{2}+10X+1=(X^{2}-11X-1)(X^{2}+X-1)=(X+\epsilon^{-5})(X-\epsilon^{5})(X-\epsilon^{-1})(X+\epsilon)$ with $\epsilon=(\sqrt{5+1})/2$ and $\epsilon^{5}>\epsilon^{-1}>-\epsilon^{-5}>-\epsilon$. To evaluate the (real-valued) integral \eqref{art8-eq18}, it is enough to consider the range $\epsilon^{5}<\tau<\infty$, since for other values of $\tau$, say $\epsilon^{-1}<\tau<\epsilon^{5}$, the expression inside the radical sign viz. $(\tau+\epsilon^{-5})(\tau-\epsilon^{5})(\tau-\epsilon^{-1})\times (\tau+\epsilon)$ becomes negative and hence the integrand will be purely imaginary. Using formula (78) in \S66 of Greenhill \cite{art8-key4} (with $\alpha=\epsilon^{5}$, $\beta=\epsilon^{-1}$, $\gamma=-\epsilon^{-5}$, $\delta=-\epsilon$) we have, for $\epsilon^{5}<\tau<\infty$.
\begin{align}
& \int\limits^{\tau}_{\epsilon^{5}}\frac{d\tau}{\sqrt{\tau^{4}-10\tau^{3}-13\tau^{2}+10\tau+1}}\notag\\
&= \frac{2}{\sqrt{(\epsilon^{5}+\epsilon^{-5})(\epsilon^{-1}+\epsilon)}}\sn^{-1}\sqrt{\dfrac{(\epsilon^{-1}+\epsilon)(\tau-\epsilon^{5})}{(\epsilon^{5}+\epsilon)(\tau-\epsilon^{-1})}}\notag\\
&= \frac{2}{5}\sn^{-1}\sqrt{\frac{\sqrt{5}}{6+3\sqrt{5}}\dfrac{(\tau-\epsilon^{5})}{(\tau-\epsilon^{-1})}}\notag\\
&= \frac{2}{5}\int\limits^{\phi}_{0}\dfrac{d\phi}{\sqrt{1-\frac{9}{25}\sin^{2}\phi}}\label{art8-eq19}
\end{align}
since $\kappa^{2}=\dfrac{(\epsilon^{-1}+\epsilon^{-5})(\epsilon^{5}+\epsilon)}{(\epsilon^{5}+\epsilon^{-5})(\epsilon^{-1}+\epsilon)}=\dfrac{9}{25}$. The upper limit $\phi$ in \eqref{art8-eq19} is given by $\sin^{2}\phi = \dfrac{\sqrt{5}}{6+3\sqrt{5}}\dfrac{(\tau-\epsilon^{5})}{(\tau-\epsilon^{-1})}$ or $\cos^{2}\phi = \dfrac{6+2\sqrt{5}}{6+3\sqrt{5}}\dfrac{(\tau+\epsilon)}{(\tau-\epsilon^{-1})}$ so that $\phi =\tan^{-1}\left(\dfrac{5^{1/4}}{2\epsilon}\dfrac{\tau-\epsilon^{5}}{\tau+\epsilon}\right)$. From Ramanujan (\cite{art8-key10}, p. 208), we know that, for $\phi_{1}$, $\phi_{2}$ with $\cot \phi_{1}\cdot \tan (\frac{1}{2}\phi_{2})=\sqrt{1-\kappa^{2}\sin^{2}\empty_{1}}$.
\begin{equation}
2\int\limits^{\phi_{1}}_{0}\dfrac{d\phi}{\sqrt{1-\kappa^{2}\sin^{2}\phi}}=\int\limits^{\phi_{2}}_{0}\dfrac{d\phi}{\sqrt{1-\kappa^{2}\sin^{2}\phi}}\label{art8-eq20}
\end{equation}
However, from above, we have
\begin{align*}
(\tan \phi)\sqrt{1-\kappa^{2}\sin^{2}\phi} &=\left(\dfrac{5^{1/4}}{2\epsilon}\sqrt{\frac{\tau-\epsilon^{5}}{\tau+\epsilon}}\right)\sqrt{1-\frac{9}{25}\frac{\sqrt{5}}{3\epsilon^{3}}\frac{(\tau-\epsilon^{5})}{(\tau-\epsilon^{-1})}}\\
&= \frac{5^{1/4}}{2\epsilon}\sqrt{\frac{\tau-\epsilon^{5}}{\tau+\epsilon}}\sqrt{\frac{4\epsilon^{2}}{5^{3/2}}\frac{(\tau+\epsilon^{-5})}{(\tau-\epsilon^{-1})}}\\
&= \frac{1}{\sqrt{5}}\sqrt{\frac{(\tau-\epsilon^{5})(\tau+\epsilon^{-5})}{(\tau+\epsilon)(\tau-\epsilon^{-1})}}\\
&= \frac{1}{\sqrt{5}}\sqrt{\frac{(\tau^{2}-11\tau-1)}{(\tau^{2}+\tau-1)}}
\end{align*}\pageoriginale
Thus from \eqref{art8-eq19} and \eqref{art8-eq20}, we obtain
\begin{align*}
&\int\limits^{\tau}_{\epsilon^{5}}\dfrac{d\tau}{\sqrt{\tau^{4}-10\tau^{3}-13\tau^{2}+10\tau+1}}\\
 &\quad = \frac{1}{5}\int\limits^{2\tan^{-1}\left(\frac{1}{\sqrt{5}}\sqrt{\frac{(\tau^{2}-11\tau-1)}{(\tau^{2}+\tau-1)}}\right)}_{0}\frac{d\phi}{\sqrt{1-\frac{9}{25}\sin^{2}\phi}}\\
&\quad = \frac{1}{5}\int\limits^{2\tan^{-1}\left(\frac{1}{\sqrt{5}}\sqrt{\frac{1-11v-v^{2}}{1+v-v^{2}}}\right)}_{0}\dfrac{d\phi}{\sqrt{1-\frac{9}{25}\sin^{2}\phi}}
\end{align*}
where $v=\dfrac{1}{\tau}=\dfrac{\eta^{3}\eta^{3}_{15}}{\eta^{3}_{3}\eta^{3}_{5}}$. Hence
\begin{align*}
\int\limits^{\infty}_{\tau}\frac{d\tau}{\sqrt{\tau^{4}-\cdots+1}} &= \int\limits^{\infty}_{\epsilon^{5}}\dfrac{d\tau}{\sqrt{\tau^{4}-\cdots+1}}-\int\limits^{\tau}_{\epsilon^{5}}\dfrac{d\tau}{\sqrt{\tau^{4}-\cdots+1}}\\
&= \frac{1}{5}\int\limits^{2\tan^{-1}(1/\sqrt{5})}_{0}\quad \dfrac{d\phi}{\sqrt{1-\frac{9}{25}\sin^{2}\phi}}\\
&\quad {}-\frac{1}{5}\int\limits^{2\tan^{-1}_{0}\left(\begin{smallmatrix} 1 & \surd 1-11v-v^{2}\\ \sqrt{5} & 1+v-v^{2}\end{smallmatrix}\right)}\quad \dfrac{d\phi}{\sqrt{1-\frac{9}{25}\sin^{2}\phi}}\\
&=\frac{1}{5}\int\limits^{2\tan^{-1}\left(\frac{1}{5}\begin{smallmatrix} \surd (1-11v-v^{2})\\ 1+v-v^{2}\end{smallmatrix}\right)}_{2\tan^{-1}(1/\sqrt{5})}\quad \frac{d\phi}{\sqrt{1-\frac{9}{25}\sin^{2}\phi}}
\end{align*}
which together with \eqref{art8-eq18} gives us Ramanujan's formula \eqref{art8-eq17}.

The right hand side of \eqref{art8-eq17} admits of interesting reduction when one resorts\pageoriginale to well-known transformations available for elliptic integrals such as Landen's transformation or Gauss' transformation. On page \mycirc{67} + 1 of \cite{art8-key11}, Ramanujan also notes that
\begin{align}
&= \frac{1}{5}\int\limits^{2\tan^{-1}(1/\sqrt{5})}_{2\tan^{-1}\left(\frac{1}{\sqrt{5}}\sqrt{\frac{(1-11v-v^{2})}{1+v-v^{2}}}\right)}\dfrac{d\phi}{\sqrt{1-\frac{9}{25}\sin^{2}\phi}}\notag\\
&= \frac{1}{9}\int\limits^{\pi/2}_{2\tan^{-1}\left(\frac{(1-v/\epsilon^{3})}{(1+v\epsilon^{3})}\sqrt{\frac{(1+v\epsilon)(1-v\epsilon^{5})}{(1-v/\epsilon)(1+v/\epsilon^{5})}}\right)}\quad \frac{d\psi}{\sqrt{1-\frac{1}{81}\sin^{2}\psi}}\label{art8-eq21}
\end{align}
also equal to
\begin{equation}
\frac{1}{4} \ \ \int\limits^{\tan^{-1}(3-\sqrt{5})}_{\tan^{-1}\left((3-\sqrt{5})\sqrt{\frac{(1-v/\epsilon)(1-v\epsilon^{5})}{(1-v\epsilon)(1+v/\epsilon^{5})}}\right)}\quad \dfrac{d\psi}{\sqrt{1-\frac{15}{16}\sin^{2}\psi}}\label{art8-eq22}
\end{equation}
with $v$ and $\epsilon$ as above.

To prove \eqref{art8-eq21}, let us use Landen's transformation (\cite{art8-key12}, p. 496) :
\begin{align}
\int \frac{d\phi}{\sqrt{1-\kappa^{2}\sin^{2}\phi}} &=\dfrac{1}{1+\sqrt{1-\kappa^{2}}}\int
\frac{d\psi}{\sqrt{1-\left(\frac{1-\sqrt{1-\kappa^{2}}}{1+\sqrt{1-\kappa^{2}}}\right)^{2}\sin^{2}\psi}}\notag\\
&\quad \text{for } \tan (\psi-\phi)-(\sqrt{1-\kappa^{2}})\tan\phi\label{art8-eq23}
\end{align}
with $\kappa=3/5$ and determine the limits of integration on the right hand side corresponding to those on the left hand side of \eqref{art8-eq23}. We have only to verify that the relations 
$$
\tan(\psi-\phi)=\frac{4}{5}\tan\phi, \ 
\tan(\phi/2)=\dfrac{1}{\sqrt{5}}\sqrt{\frac{1-11v-v^{2}}{1+v-v^{2}}}
$$
together imply that 
$$
\tan (\psi/2)=\dfrac{1-v/\epsilon^{3}}{1+v\epsilon^{3}}\sqrt{\frac{(1+v\epsilon)(1-v\epsilon^{5})}{(1-v/\epsilon)(1+v/\epsilon^{5})}},
$$
so that the upper limit $\pi/2$ for $\psi$ in \eqref{art8-eq21} will correspond to $2\tan^{-1}(1/\sqrt{5})$ arising for $v=0$. Setting $t_{1}=\tan(\phi/2)$, $t_{2}=\tan((\psi-\phi)/2)$, we have then
$$
\frac{2t_{2}}{1-t^{2}_{2}}=\tan (\psi-\phi)=\frac{4}{5}\frac{2t_{1}}{1-t^{2}_{1}}
$$\pageoriginale
and so
$$
t_{2}=\frac{5(1-t^{2}_{1})}{4t_{1}}\pm \frac{1}{2}\sqrt{\frac{25}{16}\frac{(1-t^{2}_{1})^{2}}{t^{2}_{1}}+4}.
$$
Since $1-t^{2}_{1}=1-(1-11v-v^{2})/[5(1+v-v^{2})]=(4/5)(1+4v-v^{2})/(1+v-v^{2})$ and $(25/16)(1-t^{2}_{1})^{2}/t^{2}_{1}+4=9(1+v^{2})^{2}/\{(1+v-v^{2})\times (1-11v-v^{2})\}$, we have
$$
t_{2}=\frac{-\sqrt{5}(1+4v-v^{2})+3(1+v^{2})}{2\sqrt{(1+v-v^{2})(1-11v-v^{2})}}
$$
noting that only the positive root has to be taken for $t_{2}$, in view of $t_{2}$ having to be positive for large $v$. Thus
\begin{align*}
t_{2} &= \frac{\epsilon^{2}(\epsilon-v)(\epsilon^{-5}-v)}{\sqrt{(1-v/\epsilon)(1+v\epsilon)(1-v\epsilon^{5})(1+v/\epsilon^{5})}}\\
     &= \epsilon^{-2}\sqrt{\frac{(1-v/\epsilon)(1-v\epsilon^{5})}{(1+v\epsilon)(1+v/\epsilon^{5})}}.
\end{align*}
Finally
\begin{align*}
\tan(\psi/2) &= \frac{t_{1}+t_{2}}{1-t_{1}t_{2}}=\frac{\frac{1}{\sqrt{5}}\sqrt{\frac{1-11v-v^{2}}{1+v-v^{2}}}{1+v-v^{2}}+\epsilon^{-2}\sqrt{\frac{(1-v/\epsilon)(1-v\epsilon^{5})}{(1+v\epsilon)(1+v/\epsilon^{5})}}}{1-\frac{\epsilon^{-2}}{\sqrt{5}}\frac{(1-v\epsilon^{5})}{(1+v\epsilon)}}\\
&= \frac{\sqrt{\frac{(1-v\epsilon^{5})}{(1-v/\epsilon)(1+v\epsilon)(1+v/\epsilon^{5})}}}{1/(1+v\epsilon)}\cdot \frac{(1+v/\epsilon^{5})+\sqrt{5}\epsilon^{-2}(1-v/\epsilon)}{\sqrt{5}(1+v\epsilon)-\epsilon^{-2}(1-v\epsilon^{5})}\\
&= \sqrt{\frac{(1-v\epsilon^{5})(1+v\epsilon)}{(1-v/\epsilon)(1+v/\epsilon^{5})}}\cdot \frac{(1-v\epsilon^{-3})}{(1+v\epsilon^{3})}
\end{align*}
establishing the validity of \eqref{art8-eq21}.

In order to prove \eqref{art8-eq22}, apply Gauss' transformation \cite{art8-key12} :
\begin{align*}
\int \frac{d\phi}{\sqrt{1-\kappa^{2}\sin^{2}\phi}} &=\dfrac{2}{1+\kappa}\int \dfrac{d\psi}{\sqrt{1-[4\kappa/(1+\kappa)^{2}]\sin^{2}\psi}}\\
&\quad \text{for~ }\sin(2\psi-\phi)=\kappa\sin \phi
\end{align*}
with\pageoriginale $\kappa=3/5$ again and determine the limits of integration which correspond to each other on either side. From $\sin(2\psi-\phi)=(3/5)\sin \phi$, we get a quadratic equation for $t:=\tan\psi$, viz.
$$
\frac{5t}{4-t^{2}}=\tan\phi = \frac{2\tan\phi/2}{1-\tan^{2}(\phi/2)}=\frac{\sqrt{5}}{2}\dfrac{\sqrt{(1-11v-v^{2})(1+v-v^{2})}}{(1+4v-v^{2})}
$$
proceeding as in the earlier case. Since by the same calculations,
$$
\frac{20(1+4v-v^{2})^{2}}{(1-11v-v^{2})(1+v-v^{2})}+16=\dfrac{36(1+v^{2})^{2}}{(1-11v-v^{2})(1+v-v^{2})},
$$
we get
$$
\tan \psi=t=\dfrac{-\sqrt{5}(1+4v-v^{2})+3(1+v^{2})}{\sqrt{(1-11v-v^{2})(1+v-v^{2})}}
$$
taking the positive square root as before. Thus 
$$
t=\dfrac{2}{\epsilon^{2}}\sqrt{\dfrac{(1-v/\epsilon)(1-v\epsilon^{5})}{(1+v\epsilon)(1+v/\epsilon^{5})}}
$$
giving the lower limit for $\psi$ in \eqref{art8-eq22}; the upper limit $\tan^{-1}(3-\sqrt{5})$ clearly corresponds to the upper limit $2\tan^{-1}(1/\sqrt{5})$ arising when $v=0$. Consequently Ramanujan's formula \eqref{art8-eq22} is proved.

Using a different `uniformiser' $v_{1}:=\dfrac{\eta^{2}(3z)\eta^{2}(15z)}{\eta^{2}(z)\eta^{2}(5z)}$ related to $\Gamma_{0}(15)$, Ramanujan (\cite{art8-key11}, p. 70/78) also records the formula :
\begin{equation}
\int\limits^{x}_{0}\eta\eta_{3}\eta_{5}\eta_{15}\dfrac{dx}{x}=\dfrac{1}{5} \ \ \int\limits^{2\tan^{-1}(1/\sqrt{5})}_{2\tan^{-1}\dfrac{(1-3v_{1})}{\sqrt{5}(1+3v_{1})}}\sqrt{d\phi}{\sqrt{1-\frac{9}{25}\sin^{2}\phi}}\label{art8-eq24}
\end{equation}

In view of formula \eqref{art8-eq17}, it suffices to show that
\begin{equation}
\sqrt{\frac{1-11v-v^{2}}{1+v-v^{2}}}=\frac{1-3v_{1}}{1+3v_{1}}\label{art8-eq25}
\end{equation}
in order to prove \eqref{art8-eq24}. Substituting for $v$ and $v_{1}$ on both sides of \eqref{art8-eq25} and simplifying further, \eqref{art8-eq25} will follow from
\begin{equation}
(\eta_{3}\eta_{5})^{6}-(\eta\eta_{5})^{5}\eta_{3}\eta_{15}-5(\eta\eta_{3}\eta_{5}\eta_{15})^{3}-9(\eta\eta_{5})(\eta_{3}\eta_{15})^{5}-(\eta\eta_{15})^{6}=0.\label{art8-eq26}
\end{equation}
Formula \eqref{art8-eq26}, however, is a consequence of the following

\medskip
\noindent
{\bf Proposition \thnum{2}.\label{art8-prop2}}~\pageoriginale
\begin{equation}
\frac{\eta^{5}_{3}\eta^{5}_{5}}{\eta\eta_{15}}-\eta^{4}\eta^{4}_{5}-5(\eta\eta_{3}\eta_{5}\eta_{15})^{2}-9(\eta_{3}\eta_{15})^{4}-\dfrac{\eta^{5}\eta^{5}_{15}}{\eta_{3}\eta_{5}}=0.\label{art8-eq27}
\end{equation}

\begin{proof}
Each term on the left hand side of \eqref{art8-eq27} can be shown to be a modular form of weight 4 for $\Gamma_{0}(15)$ and it is not hard to derive the following Fourier expansions (writing $x$ for $e^{2\pi iz}$) :
\begin{align*}
& \frac{\eta^{5}_{3}\eta^{5}_{5}}{\eta\eta_{15}}=x+x^{2}+2x^{3}-2x^{4}-0\cdot x^{5}-8x^{6}-4x^{7}-15x^{8}\\
&\hspace{7.2cm} +7x^{9}-0\cdot x^{10}+\cdots\\
& -\eta^{4}\eta^{4}_{5}=-x+4x^{2}-2x^{3}-8x^{4}+5x^{5}+8x^{6}-6x^{7}+0\cdot x^{8}+23x^{9}\\
&\hspace{8cm} -20\cdot x^{10}+\cdots\\
& -5(\eta\eta_{3}\eta_{5}\eta_{15})^{2}=-5x^{2}+10x^{3}+5x^{4}+0\cdot x^{5}-25x^{6}-10x^{7}+15x^{8}\\
&\hspace{7cm} -10x^{9}+25x^{10}+\cdots\\
& -9(\eta_{3}\eta_{15})^{4}=-9x^{3}+0\cdot x^{4}+0\cdot x^{5}+36x^{6}+0\cdot x^{7}+0\cdot x^{8}-18x^{9}\\
&\hspace{8cm} +0\cdot x^{10}+\cdots\\
& -\frac{(\eta\eta_{15})^{5}}{\eta_{3}\eta_{5}}=-x^{3}+5\cdot x^{4}-5\cdot x^{5}-11x^{6}+20x^{7}+0\cdot x^{8}-2x^{9}\\
&\hspace{8.2cm}-5x^{10}+\cdots
\end{align*}
From these expansions, the left hand side of \eqref{art8-eq27} is a modular form of weight 4 for $\Gamma_{0}(15)$ all of whose Fourier coefficients corresponding to $e^{2\pi inz}$ for $0\leq n\leq 8=4[\Gamma(1) : \Gamma_{0}(15)]/12$ vanish. By Hecke's theorem (\cite{art8-key5}, p. 811-see also p. 953) again, this modular form has to vanish identically.
\end{proof}

\begin{remark*}
We may rewrite \eqref{art8-eq27} in terms of the above modular functions $v$, $v_{1}$ for $\Gamma_{0}(15)$ as 
\begin{equation}
\frac{1}{v}-v-5-9v_{1}-\frac{1}{v_{1}}=0\label{art8-eq28}
\end{equation}
If we show that the left hand side (which is already regular on all of $\mathfrak{H}$ is also regular at all the four cusps of $\Gamma_{0}(15)$, we can obtain an alternative proof for \eqref{art8-eq27}, via \eqref{art8-eq28}.
\end{remark*}

\subsection{}\label{art8-sec3.2}
Ramanujan has considered on the same page \mycirc{67} + 1 of \cite{art8-key11} an elliptic integral coming now from the group $\Gamma_{0}(14)$ of genus $1$ and has noted the formula
\begin{equation}
\int\limits^{x}_{0}\eta\eta_{2}\eta_{7}\eta_{14}\frac{dx}{x}=\ldots\int\limits_{\cos^{-1}\left(\frac{1}{7}\frac{1+v_{2}}{1-v_{2}}\sqrt{13+16\sqrt{2}}\right)}\frac{d\phi}{\sqrt{1-\frac{16\sqrt{2}-13}{32\sqrt{2}}\sin^{2}\phi}}\label{art8-eq29}
\end{equation}
where\pageoriginale $v_{2}:=(\eta\eta_{14}/\eta_{2}\eta_{7})^{4}$.

We know that $\tau:=1/v_{2}$ and $\sigma := \dfrac{2\pi i}{4\pi^{2}\eta\eta_{2}\eta_{7}\eta_{14}}\dfrac{d\tau}{dz}$ are connected by the relation $\sigma^{2}=\tau^{4}-14\tau^{3}+19\tau^{2}-14\tau+1$, from Fricke (\cite{art8-key3}, pp. 451-453) and in fact, they generate the field of modular functions for $\Gamma_{0}(14)$. Proceeding as in \S\ref{art8-sec3.1}, we have
\begin{align}
\int\limits^{x}_{0}\eta\eta_{2}\eta_{7}\eta_{14}\dfrac{dx}{x} &= -\pi i\int\limits^{i\infty}_{z}\eta\eta_{2}\eta_{7}\eta_{14}dz \ (\text{with~ } z=iy, y>0)\notag\\
&= \int\limits^{\infty}_{\tau}\dfrac{d\tau}{\sqrt{\tau^{4}-14\tau^{3}+19\tau^{2}-14\tau+1}}\label{art8-eq30}
\end{align}
Now $X^{4}-14X^{3}+19X^{2}-14X+1=(X-\alpha)(X-\beta)[(X-m)^{2}+n^{2})$ where $\alpha=\frac{1}{2}(7+4\sqrt{2}+\sqrt{7}\sqrt{11+8\sqrt{2}})$, $\beta=\frac{1}{2}(7+4\sqrt{2}-\sqrt{7}\sqrt{11+8\sqrt{2}})$, $m=\frac{1}{2}(7-4\sqrt{2})$, $n^{2}=\frac{7}{4}(8\sqrt{2}-11)$. From Greenhill (\cite{art8-key4}, p. 61 as quoted from page 23 of ``Elliptische Funktionen'' by Enneper), we obtain
\begin{align*}
& \int\limits^{x}_{\alpha}\frac{dX}{\sqrt{(X-\alpha)(X-\beta)[(X-m)^{2}+n^{2}})}\\
&\qquad =\frac{1}{\sqrt{HK}}\cn^{-1}\left\{\frac{H(X-\beta)-K(X-\alpha)}{H(X-\beta)+K(X-\alpha)},\kappa\right\}\\
&\qquad =\frac{1}{\sqrt{HK}}\int\limits^{\phi}_{0}\dfrac{d\phi}{\sqrt{1-\kappa^{2}\sin^{2}\phi}}
\end{align*}
with $\phi=\cos^{-1}\left(\dfrac{H(X-\beta)-K(X-\alpha)}{H(X-\beta)+K(X-\alpha)}\right),H^{2}=(\alpha-m)^{2}+n^{2}=4\sqrt{2}(7+4\sqrt{2}+\sqrt{7}\sqrt{11+8\sqrt{2}})$, $K^{2}=(\beta-m)^{2}+n^{2}=4\sqrt{2}(7+4\sqrt{2}-\sqrt{7}\sqrt{11+8\sqrt{2}})$, and $\kappa^{2}=\frac{1}{2}-\frac{1}{4}\{(\alpha-\beta)^{2}-H^{2}-K^{2}\}/HK$. Also, $HK=8\sqrt{2}$ and $\kappa^{2}=(16\sqrt{2}-13)/(32\sqrt{2})$. Hence,\pageoriginale for the (real-valued) integral \eqref{art8-eq30} wherein $\tau>\alpha$ necessarily, we have the value
\begin{align*}
&\left(\int\limits^{x}_{\alpha}-\int\limits^{\tau}_{\alpha}\right)\dfrac{d\tau}{\sqrt{\tau^{4}-\cdots+1}}\\[8pt]
&\qquad =\dfrac{1}{\sqrt{8\sqrt{2}}}\int\limits^{\cos^{-1}\left(\frac{H-K}{H+K}\right)}_{\cos^{-1}\left(\frac{H-K-v_{2}(H\beta-K\alpha)}{H-K-v_{2}(H\beta+K\alpha)}\right)}\dfrac{d\phi}{\sqrt{1-\dfrac{16\sqrt{2}-13}{32\sqrt{2}}\sin^{2}\phi}}\\[8pt]
&\qquad =\frac{1}{\sqrt{8\sqrt{2}}}\int\limits^{\cos^{-1}\left(\frac{\sqrt{13+16\sqrt{2}}}{7}\right)}_{\cos^{-1}\left(\frac{\sqrt{13+16\sqrt{2}}}{7}\frac{1+v_{2}}{1-v_{2}}\right)}\dfrac{d\phi}{\sqrt{1-\frac{16\sqrt{2}-13}{32\sqrt{2}}\sin^{2}\phi}}
\end{align*}
since $H^{2}\beta=K^{2}\alpha=HK$, $(H\beta-K\alpha)/(K-H)=1=(H\beta+K\alpha)/(H+K)$ and $(H-K)/(H+K)=(H^{2}-K^{2})/(H^{2}+K^{2}+2HK)=\frac{1}{7}\sqrt{13+16\sqrt{2}}$. This proves formula \eqref{art8-eq29}.

\subsection{}\label{art8-sec3.3}
An integral of the same form as above but {\em not} of elliptic type has been mentioned by Ramanujan on page \mycirc{70}/78 of \cite{art8-key11} :
\begin{equation}
\int\limits^{x}_{0}\eta\eta_{5}\eta_{7}\eta_{35}\dfrac{dx}{x}=\frac{1}{2}\int\limits^{v_{3}}_{0}\dfrac{dt}{\sqrt{(1-t+\sqrt{t})(1-t^{3}-\sqrt{t}(5+9t+5t^{2}))}}\label{art8-eq31}
\end{equation}
where $v_{3}=(\eta\eta_{35}\eta_{5}\eta_{7})^{2}$.

If $\tau := 1/\sqrt{v_{3}}$ and $\sigma=\dfrac{4\pi i}{2\pi^{2}\eta^{2}\eta^{2}_{35}}\dfrac{d\tau}{dz}$, then by Fricke (\cite{art8-key3}, pp. 444-445), $\sigma$ and $\tau$ are modular functions for $\Gamma_{0}(35)$ connected by the relation $\sigma^{2}=\tau^{8}-4\tau^{7}-6\tau^{6}-4\tau^{5}-9\tau^{4}+4\tau^{3}-6\tau^{2}+4\tau+1$. [The right hand side factorizes as $(\tau^{2}+\tau-1)(\tau^{6}-5\tau^{5}-9\tau^{3}-5\tau-1)$]. As in earlier examples,
\begin{align*}
\int\limits^{x}_{0}\eta\eta_{5}\eta_{7}\eta_{35}\dfrac{dx}{x} &= \int\limits^{\infty}_{\tau}\dfrac{\tau d\tau}{\sqrt{(\tau^{2}+\tau-1)(\tau^{6}-5\tau^{5}-9\tau^{3}-5\tau-1)}}\\
&= \frac{1}{2}\int\limits^{v_{3}}_{0}\sqrt{dt}{\sqrt{(1-t+\sqrt{t})(1-t^{3}-\sqrt{t}(5+9t+5t^{2}))}}\\
&\qquad\qquad\qquad (\text{setting~ } \tau=1/\sqrt{t})
\end{align*}\pageoriginale

\begin{remarks*}
The further reduction of this {\em hyperelliptic} integral can be carried out by known methods in the theory of elliptic functions (\cite{art8-key4}, pp. 159-160).
\end{remarks*}

It is interesting to note the following relation between $P=\eta/\eta_{7}$ and $Q=\eta_{5}/\eta_{35}$ on page 303 of \cite{art8-key10} :
$$
(PQ)^{2}-5+49/(PQ)^{2}=(Q/P)^{3}-5(Q/P)^{2}-5(P/Q)^{2}-(P/Q)^{3}
$$
which is the same as equation (29) on page 446 of Fricke (3); the latter is itself a consequence of the above relation between $\sigma$ and $\tau$.

\subsection{}\label{art8-sec3.4}
Ramanujan has also considered elliptic integrals wherein the integrand involves (higher) powers of $\eta$. On page \mycirc{45}/54 of \cite{art8-key11}, he has written down the following formulae :
\begin{align}
5^{3/4}\int\limits^{x}_{0}\eta^{2}\eta^{2}_{5}\frac{dx}{x} &= \int\limits^{2\tan^{-1}(5^{3/4}\sqrt{\lambda_{k}})}_{0} \frac{d\phi}{\sqrt{1-\epsilon^{-5}5^{-3/2}\sin^{2}\phi}}\label{art8-eq32}\\[5pt]
&= 2 \int\limits^{\pi/2}_{\cos^{-1}[(\epsilon u)^{5/2}]}\dfrac{d\phi}{\sqrt{1-\epsilon^{-5}5^{-3/2}\sin^{2}\phi}}\label{art8-eq33}\\[5pt]
&= \sqrt{5}\int\limits^{2\tan^{-1}[5^{1/4}\sqrt{x}\psi(x^{5})/\psi(x)]}_{0}\dfrac{d\phi}{\sqrt{1-\epsilon/\sqrt{5})\sin^{2}\phi}}\label{art8-eq34}
\end{align}
recalling that $\lambda_{5}=n^{6}_{5}/\eta^{6}$, $u=\dfrac{x^{1/5}}{1+}\dfrac{x}{1+}\dfrac{x^{2}}{1+}\ldots$, $\psi(x)=x^{-1/8}\times \eta^{2}_{2}(z)/\eta(z)$ and $\epsilon=(\sqrt{5}+1)/2$.

Before proving \eqref{art8-eq32}--\eqref{art8-eq34}, we state

\medskip
\noindent
{\bf Proposition \thnum{3}.\label{art8-prop3}}
\begin{itemize}
\item[(i)] $E_{2}(z)-5E_{2}(5z)=-4(\eta^{5}/\eta_{5})\sqrt{1+22\lambda_{5}+125\lambda^{2}_{5}}$

\item[(ii)] $E_{4}(z)=\eta^{10}/\eta^{2}_{5}+250\eta^{4}\eta^{4}_{5}+3125\eta^{10}_{5}/\eta^{2}$

\item[(iii)] $E_{4}(5z)=\eta^{10}/\eta^{2}_{5}+10\eta^{4}\eta^{4}_{5}+5\eta^{10}_{5}/\eta^{2}$.
\end{itemize}

\begin{proof}
We know that $\eta^{10}/\eta^{2}_{5}$, $\eta^{4}\eta^{4}_{5}$, $\eta^{10}_{5}/\eta^{2}$ form a basis for the space of modular forms of Haupttypus $(-4,5,1)$ and their Fourier expansions are given by
\begin{align*}
\eta^{10}/\eta^{2}_{5} &= 1-10e^{2\pi iz}+35 e^{4\pi iz}+\cdots\\
\eta^{4}\eta^{4}_{5} &= e^{2\pi iz}-4e^{4\pi iz}+\cdots\\
\eta^{10}_{5}/\eta^{2} &= e^{4\pi iz}+\cdots
\end{align*}\pageoriginale
Writing $E_{4}=\alpha \eta^{10}/\eta^{2}_{5}+\beta \eta^{4}\eta^{4}_{5}+\lambda \eta^{10}_{5}/\eta^{2}$ and comparing the first three Fourier coefficients, we have $\alpha=1$, $-10\alpha+\beta=240$, $35\alpha-4\beta+\lambda=2160$ i.e. $\alpha=1$, $\beta=250$, $\lambda=3125$, proving (ii). The proof of (iii) is identical. For proving (i), we have only to argue instead with $[E_{2}(z)-5E_{2}(5z)]^{2}$ of Haupttypus $(-4,5,1)$ and identify it likewise with $16(\eta^{10}/\eta^{2}_{5}+22\eta^{4}\eta^{4}_{5}+125\eta^{10}_{5}/\eta^{2})$. Identity (i) is stated by Ramanujan on page \mycirc{73}/81 of \cite{art8-key11}.
\end{proof}

\begin{coro*}
$x(\frac{d}{dx}\lambda_{5})=\eta^{2}\eta^{2}_{5}\sqrt{\lambda_{5}+22\lambda^{2}_{5}+125\lambda^{3}_{5}}$.
\end{coro*}

\noindent
{\bf Proof} is immediate from \eqref{art8-eq7}, \eqref{art8-eq10} and (i) of Proposition \ref{art8-prop3}.

We now proceed to prove \eqref{art8-eq32}. In fact, from the Corollary, we have
\begin{align}
5^{3/4}\int\limits^{x}_{0}\eta^{2}\eta^{2}_{5}\dfrac{dx}{x} &=\frac{1}{5^{3/4}}\int\limits^{\lambda_{5}}_{0}\dfrac{d\lambda_{5}}{\sqrt{\lambda^{3}_{5}+\frac{22}{125}\lambda^{2}_{5}+\frac{1}{125}\lambda_{5}}}(\text{since~ }\lambda_{5}(i\infty)=0)\notag\\
&= \frac{1}{5^{3/4}}\int\limits^{\lambda}_{0}\dfrac{d\lambda}{\sqrt{\lambda[(\lambda+\frac{11}{125})^{2}+(\frac{2}{125})^{2}]}},\label{art8-eq35}
\end{align}
dropping the suffix 5 from $\lambda_{5}$. Now, using formula (24) on page 40 of Greenhill (\cite{art8-key4}, \S46) with $\alpha=0$, $m=-11/5^{3}$, $n=2/5^{3}$, $H^{2}:=(\alpha-m)^{2}+n^{2}=1/5^{3}$, $\kappa^{2} := \frac{1}{2}[1-(\alpha-m)/H]=(5\sqrt{5}-11)/(2.5^{3/2})=\epsilon^{-5}/5^{3/2}$, we see that \eqref{art8-eq35} is the same as
$$
\frac{1}{5^{3/4}\sqrt{H}}\cn^{-1}\left\{\dfrac{H-\lambda}{H+\lambda},\dfrac{\epsilon^{-5}}{5^{3/2}}\right\}=\int\limits^{\cos^{-1}\left(\frac{5^{-3/2}-\lambda}{5^{-3/2}+\lambda}\right)}_{0}\quad \dfrac{d\phi}{\sqrt{1-\epsilon^{-5}5^{-3/2}\sin^{2}\phi}}
$$
But $\cos\phi=(5^{-3/2}-\lambda)/(5^{-3/2}+\lambda)$ implies that $\tan(\phi/2)=5^{3/4}\sqrt{\lambda}$ and so \eqref{art8-eq32} is proved.

To prove that the right hand side of \eqref{art8-eq32} is the same as \eqref{art8-eq33}, let us first invoke the following transformation formulae from Ramanujan (\cite{art8-key10}, Chapter XVII, 7(vi) and 7(ii), pp. 207-208) :
\begin{align*}
\int\limits^{\beta}_{0}\dfrac{d\phi}{\sqrt{1-\kappa^{2}\sin^{2}\phi}} &=2\int\limits^{\alpha}_{0}\dfrac{d\phi}{\sqrt{1-\kappa^{2}\sin^{2}\phi}}\\
(\text{where~ }\tan(\beta/2) &= (\tan\alpha)|\sqrt{1-\kappa^{2}\sin^{2}\alpha})\\
&=2\int\limits^{\lambda}_{0}\dfrac{d\phi}{\sqrt{1-\kappa^{2}\cos^{2}\phi}}\\
&\quad (\text{where~ }\tan\lambda=(\tan\alpha)\sqrt{1-\kappa^{2}})
\end{align*}\pageoriginale
which together imply that
\begin{equation}
\int\limits^{\beta}_{0}\dfrac{d\phi}{\sqrt{1-\kappa^{2}\sin^{2}\phi}}=2\int\limits^{\pi/2}_{\pi/2-\lambda}\dfrac{d\phi}{\sqrt{1-\kappa^{2}\sin^{2}\phi}}\label{art8-eq36}
\end{equation}
Now taking 
$$
\kappa^{2}=\epsilon^{-5}/5^{3/2}\quad\text{and}\quad\lambda=(\pi/2)-\cos^{-1}((\epsilon u)^{5/2}),
$$ 
we have 
\begin{align*}
& \tan \lambda=(\epsilon u)^{5/2}/\sqrt{1-(\epsilon u)^{5}},\\
& \tan\alpha=(\tan\lambda)/\sqrt{1-\kappa^{2}}=5^{3/4}u^{5/2}/\sqrt{1-(\epsilon u)^{5}},\\
& 1-\kappa^{2}\sin^{2}\alpha=1/(1+\epsilon^{-5}u^{5})
\end{align*}
and 
\begin{align*}
\tan(\beta/2) &=(\tan\alpha)\sqrt{1-\kappa^{2}\sin^{2}\alpha}\\
 &=5^{3/4}u^{5/2}/\sqrt{(1-\epsilon^{5}u^{5})(1+\epsilon^{-5}u^{5})}\\
&=5^{3/4}/\sqrt{u^{-5}-u^{5}-11}=5^{3/4}\sqrt{\lambda_{5}},
\end{align*}
by \eqref{art8-eq3}. Our assertion above concerning \eqref{art8-eq32} and \eqref{art8-eq33} is immediate.

Finally, we show that the right hand side of \eqref{art8-eq32} coincides with \eqref{art8-eq34}. For this, we need to use Ramanujan's transformation formula (\cite{art8-key10}, p. 231 -- see also Smith \cite{art8-key12}, p. 469) :
\begin{equation}
\int\limits^{A}_{0}\dfrac{d\phi}{\sqrt{1-\kappa^{2}\sin^{2}\phi}}=\dfrac{3}{1+2\alpha}\int\limits^{B}_{0}\dfrac{d\phi}{\sqrt{1-\mu^{2}\sin^{2}\phi}}\label{art8-eq37}
\end{equation}
where $\kappa^{2}=\dfrac{\alpha^{3}(2+\alpha)}{1+2\alpha}$, $\mu^{2}=\alpha\left(\dfrac{2+\alpha}{1+2a}\right)^{3}$ and 
\begin{equation}
\tan ((A-B)/2)=(1-\alpha)/(2\alpha+1)\tan B.\label{art8-eq38}
\end{equation}
Taking $\alpha=1/(\epsilon^{2}\sqrt{5})$, we have $1-\alpha=3/(\epsilon\sqrt{5})$, $2\alpha+1=3/\sqrt{5}$, $2+\alpha=3\epsilon/\sqrt{5}$, $\kappa^{2}=\epsilon^{-5}/5^{3/2}$ and $\mu^{2}=\epsilon/\sqrt{5}$. Further, in \eqref{art8-eq37}, if we take $A=2\tan^{-1}(5^{3/4}\sqrt{\lambda_{5}})B=2\tan^{-1}[5^{1/4}\sqrt{x}\psi(x^{5})/\psi(x)]$, we have to verify that \eqref{art8-eq38} holds. But the latter is the same as
$$
\frac{\eta^{3}_{5}}{\eta^{3}}=\dfrac{\eta\eta^{2}_{10}}{\eta^{2}_{2}\eta_{5}}\times \frac{1-(\eta\eta^{2}_{10}/\eta^{2}_{2}\eta_{5})^{2}}{1-5(\eta\eta^{2}_{10}/\eta^{2}_{2}\eta_{5})^{2}}
$$
which, on the other hand, follows at once from Proposition \ref{art8-prop1}. It is now clear that \eqref{art8-eq37} implies our assertion concerning \eqref{art8-eq32} and \eqref{art8-eq34}.

\begin{remark*}
On the same page in \cite{art8-key11} wherein Ramanujan has noted down formulae \eqref{art8-eq32}-\eqref{art8-eq34} as well as the integrals in \eqref{art8-eq39}-\eqref{art8-eq41} considered in \S3.5, one finds also values of $\kappa^{2}$, $\kappa^{2}(1-\kappa^{2})$ corresponding to two successive\pageoriginale ``cubic'' transformations. We should mention here that classically the object of applying such transformations to elliptic integrals repeatedly was the realisation of the complete integral in the limit (\cite{art8-key4}, p. 322).
\end{remark*}

\subsection{}\label{art8-sec3.5}
Formulae \eqref{art8-eq32}-\eqref{art8-eq34} were connected with the modular relation \eqref{art8-eq3} for $u$. The following formulae due to Ramanujan (\cite{art8-key11}, p. \mycirc{45}/54) are analogous and connected with the modular relation \eqref{art8-eq2} instead :
\begin{align}
& 5^{-3/4}\int\limits^{x}_{0}\dfrac{\eta^{5}}{\sqrt{\eta_{1/5}\eta_{5}}}\dfrac{dx}{x}=2\int\limits^{\pi/2}_{\cos^{-1}(\sqrt{su})}\dfrac{d\phi}{\sqrt{1-(\epsilon^{-1}/\sqrt{5})\sin^{2}\phi}}\label{art8-eq39}\\
&= \int\limits^{2\tan^{-1}(5^{1/4\sqrt{\eta_{5}/\eta_{1/5}}})}_{0}\dfrac{d\phi}{\sqrt{1-(\epsilon^{-1}/\sqrt{5})\sin^{2}\phi}}\label{art8-eq40}\\
&=\frac{1}{\sqrt{5}}\int\limits^{2\tan^{-1}(5^{3/4}((\eta_{1/5}+\eta_{5})/(\eta_{1/5}+5\eta_{5}))\sqrt{\eta_{5}/\eta_{1/5}})}_{0} \ \ \dfrac{d\phi}{\sqrt{1-(\epsilon^{5}/5^{3/2})\sin^{2}\phi}}\label{art8-eq41}
\end{align}
Before proving \eqref{art8-eq39}, we note that, by virtue of \eqref{art8-eq9}, $\dfrac{1}{u}\dfrac{du}{dx}=\dfrac{1}{5x}\eta^{5}/\eta_{5}$ and so $u=u(1)\exp \left(-\dfrac{1}{5}\int\limits^{1}_{x}\dfrac{\eta^{5}}{\eta_{5}}\dfrac{dx}{x}\right)$. But $u(1)=\dfrac{1}{1+}\dfrac{1}{1+}\ldots=(\sqrt{5}-1)/2=\epsilon^{-1}$. Moreover, $u<\epsilon^{-1}$ since $x=e^{-2\pi y}<1$ (for $y>0$) and $-\dfrac{1}{5}\int\limits^{1}_{x}\dfrac{\eta^{5}}{\eta_{5}}\dfrac{dx}{x}<0$. Using \eqref{art8-eq2} it is now easily seen that the left hand side of \eqref{art8-eq39} is the same as
\begin{align}
5^{1/4}\int\limits^{u}_{0}\dfrac{du}{\sqrt{u-u^{2}-u^{3}}} &= 5^{1/4}\int\limits^{u}_{0}\dfrac{du}{\sqrt{-u(u+\epsilon)(u-\epsilon^{-1})}}\notag\\
&\qquad\qquad (\text{noting~ } 0<u<\epsilon^{-1})\notag\\
&= 5^{1/4}\left(\int\limits^{\epsilon^{-1}}_{0}-\int\limits^{\epsilon^{-1}}_{u}\right)\dfrac{du}{\sqrt{-u(u+\epsilon)(u-\epsilon^{-1})}}\label{art8-eq42}
\end{align}
But\pageoriginale from Greenhill (\cite{art8-key4}, formula (14), p. 36), we know that
\begin{equation*}
5^{1/4}\int\limits^{\epsilon^{-1}}_{u}\dfrac{du}{\sqrt{-u(u+\epsilon)(u-\epsilon^{-1})}}=2\int\limits^{\cos^{-1}(\sqrt{\epsilon})}_{0}\dfrac{d\phi}{\sqrt{1-(\epsilon^{-1}/\sqrt{5})\sin^{2}\phi}}
\end{equation*}
taking $M=\frac{1}{2}(\epsilon^{-1}+\epsilon)^{1/2}$ and $(\kappa')^{2}=\epsilon^{-1}/(\epsilon+\epsilon^{-1})=\epsilon^{-1}/\sqrt{5}$ therein. Letting $u$ tend to $0$ in \eqref{art8-eq43} and substituting into \eqref{art8-eq42}, we see that formula \eqref{art8-eq39} is true.

In order to verify that \eqref{art8-eq40} is the same as the right hand side of \eqref{art8-eq39}, we have only to use Ramanujan's formula \eqref{art8-eq36} with $\kappa^{2}=\epsilon^{-1}/\sqrt{5}$ and $\lambda=\frac{\pi}{2}-\cos^{-1}(\sqrt{\epsilon u})$; indeed, $\tan \lambda=(\epsilon u/(1-\epsilon u))^{1/2}$, $\tan^{2}\alpha=\sqrt{5} \ u/(1-\epsilon u)$, $1-(\epsilon^{-1}/\sqrt{5})\sin^{2}\alpha=1/(1+\epsilon^{-1}u)$ and consequently $\tan\beta/2=(\epsilon u/(1-\epsilon u))^{1/2}5^{1/4}/(\epsilon+u)^{1/2}=5^{1/4}(\eta_{5}\eta/\eta_{1/5})$ in view of \eqref{art8-eq2}.

For proving that \eqref{art8-eq40} and \eqref{art8-eq41} are the same, we appeal to the Legendre transformation (\cite{art8-key4}, p. 323) :
\begin{align*}
\int\limits^{\infty}_{0}\dfrac{d\phi}{\sqrt{1-\kappa^{2}\sin^{2}\phi}} &=\dfrac{1}{2\alpha+1}\int\limits^{\psi}_{0}\dfrac{d\psi}{\sqrt{1-\mu^{2}\sin^{2}\psi}}\\
&\quad (\text{with } \tan\dfrac{\phi+\psi}{2}=(\alpha+1)\tan\phi)
\end{align*}
where $\kappa^{2}=(\alpha^{4}+2\alpha^{3})/(2\alpha+1)$ and $\mu^{2}=\alpha[(\alpha+2)/(2\alpha+1)]^{3}$; we need only to take $\alpha=\epsilon^{-1}$, $\phi=2  \tan^{-1}(5^{1/4}\sqrt{\eta_{1/5}\eta_{5}})$ and $\psi=2  \tan^{-1} (5^{3/4}\cdot \sqrt{\eta_{5}/\eta_{1/5}}\cdot [(\eta_{1/5}+\eta_{5})/(\eta_{1/5}+5\eta_{5})]$ and verify that $\tan \dfrac{\phi+\psi}{2}=(\alpha+1)\tan \phi$.

\subsection{}\label{art8-sec3.6}
Finally, we take up an interesting formula stated by Ramanujan (\cite{art8-key11}, p. \mycirc{45}/54) concerning $u$, namely
\setcounter{equation}{43}
\begin{equation}
u^{-5}+u^{5}=\dfrac{\eta^{3}}{2\eta^{3}_{5}}\left\{C+\int\limits^{1}_{x}\dfrac{\eta^{8}}{\eta^{4}_{5}}\dfrac{dx}{x}+125\int\limits^{x}_{x}\frac{\eta^{8}_{5}}{\eta^{4}}\dfrac{dx}{x}\right\}\label{art8-eq44}
\end{equation}
where
\begin{align*}
C &= 5^{3/4}\left\{-\pi +4 \int\limits^{\pi/2}_{0}\sqrt{1-\epsilon^{-5}5^{-3/2}\sin^{2}\phi} \ d\phi -\right.\\
  &\qquad \left.\int\limits^{\pi/2}_{0}\sqrt{d\phi}{\sqrt{1-\epsilon^{-5}5^{-3/2}\sin^{2}\phi}}\right\}
\end{align*}\pageoriginale

To prove \eqref{art8-eq44}, we first remark that $G:=2\sqrt{\lambda_{5}}(u^{5}+u^{-5})$ is the same as $2\sqrt{125\lambda_{5}+22+1/\lambda_{5}}$, in view of the modular relation \eqref{art8-eq3}. Hence
\begin{align*}
\dfrac{dG}{dx} &= \frac{125-1/\lambda^{2}_{5}}{\sqrt{125\lambda_{5}+22+1/\lambda_{5}}}\dfrac{d\lambda_{k}}{dx}\\[3pt]
               &= \frac{1}{x}\left(125\frac{\eta^{8}_{5}}{\eta^{4}}-\dfrac{\eta^{8}}{\eta^{4}_{5}}\right),\text{~ by Corollary to Proposition \ref{art8-prop3}.}
\end{align*}
Now, by the fundamental theorem of the integral calculus,
$$
G(x)-G(e^{-2\pi/\theta})=\int\limits^{x}_{e^{-2\pi/\theta}}125\frac{\eta^{8}_{5}}{\eta^{4}}\dfrac{dx}{x}-\int\limits^{x}_{e^{-2\pi/\theta}}\dfrac{\eta^{8}}{\eta^{4}_{5}}\dfrac{dx}{x}(\text{for any } \theta>0)
$$
Consequently, we have
\begin{equation}
u^{-5}+u^{5}=\dfrac{\eta^{3}}{2\eta^{3}_{5}}\left(C'+\int\limits^{1}_{x}\frac{\eta^{8}}{\eta^{4}_{5}}\dfrac{dx}{x}+125\int\limits^{x}_{0}\dfrac{\eta^{8}_{5}}{\eta^{4}}\dfrac{dx}{x}\right)\label{art8-eq45}
\end{equation}
where 
$$
C':=G(e^{-2\pi/\theta})-125\int\limits^{e^{-2\pi /\theta}}_{0}\frac{\eta^{8}_{5}}{\eta^{4}}\dfrac{dx}{x}-\int\limits^{1}_{e^{-2\pi/\theta}}\dfrac{\eta^{8}}{\eta^{4}_{5}}\dfrac{dx}{x}.
$$
It is easy to see that $C'$ is independent of $\theta$ and moreover $C'<G(e^{-2\pi/\theta})$ for all $\theta>0$. From the transformation formula for the $\eta$-function, we find that
\begin{gather}
125\cdot \frac{\eta^{6}(-\frac{1}{z})}{\eta^{6}(-\frac{1}{5z})}=\dfrac{\eta^{6}(z)}{\eta^{6}_{5}(z)}\label{art8-eq46}\\
125\cdot \dfrac{\eta^{8}_{5}}{\eta^{4}}\left(-\dfrac{1}{5z}\right)=5\cdot (z/i)^{2}\cdot \dfrac{\eta^{8}(z)}{\eta^{4}_{5}(z)}\notag
\end{gather}
and therefore
$$
125 \int\limits^{e^{-2\pi /\theta}}_{0}\dfrac{\eta^{8}_{5}}{\eta^{4}}\dfrac{dx}{x}=\int\limits^{1}_{e^{-2\pi \theta/5}}\dfrac{\eta^{8}}{\eta^{4}_{5}}\dfrac{dx}{x}\cdot \left(\text{using~ } z\to -\dfrac{1}{5z}\right).
$$
As a result,
$$
C'=G(e^{-2\pi/\theta})-250\int\limits^{e^{-2\pi/\theta}}_{0}\dfrac{\eta^{8}_{5}}{\eta^{4}}\dfrac{dx}{x}=G(e^{-}2^{\pi/\theta})-2\int\limits^{1}_{e^{-2\pi \theta/5}}\dfrac{\eta^{8}}{\eta^{4}_{5}}\dfrac{dx}{x}.
$$
Formula\pageoriginale \eqref{art8-eq45} may now be rewritten as
$$
u^{-5}+u^{5}=\dfrac{\eta^{3}}{2\eta^{3}_{5}}\left(C+\int\limits^{e^{-2\pi /\theta}}_{x}\dfrac{\eta^{8}}{\eta^{4}_{5}}\dfrac{dx}{x}+125\int\limits^{x}_{e^{-2\pi/\theta}}\dfrac{\eta^{8}_{5}}{\eta^{4}}\dfrac{dx}{x}\right),
$$
denoting
$$
C'+\int\limits^{1}_{e^{-2\pi/\theta}}\dfrac{\eta^{8}}{\eta^{4}_{5}}\dfrac{dx}{x}+125\int\limits^{e^{-2\pi/\theta}}_{0}\dfrac{\eta^{8}_{5}}{\eta^{4}}\dfrac{dx}{x}(=G(e^{-2\pi/\theta})\text{~clearly})\text{~ by } C.
$$
This last stated formula for $u^{-5}+u^{5}$ may be deemed to be obtained from \eqref{art8-eq45}, merely by replacing the limits of integration 0, 1 therein by $e^{-2\pi/\theta}$ and further the constant of integration $C'$ there by $C(=G(e^{-2\pi/\theta}))$. Ramanujan evaluates the constant $C$ for $\theta=1$, $\sqrt{5}$ and $5$. Taking $\theta=\sqrt{5}$ first, we obtain from \eqref{art8-eq46} with $z=i/\sqrt{5}$, that $125\lambda_{5}(-1/(5i/\sqrt{5}))=1/\lambda_{5}(i/\sqrt{5})$, i.e. $\lambda_{5}(i/\sqrt{5})=5^{-3/2}$. Hence
$$
C=G(e^{-2\pi/\sqrt{5}})=2(125\cdot 5^{-3/2}+22+5^{3/2})^{1/2}=4(11+5\sqrt{5})/2)^{1/2},
$$
in agreement with Ramanujan's value for $C$ for this case. Again, using the functional equation for $\lambda_{5}$ implied by \eqref{art8-eq46} with $z=i$, we have $125\lambda_{5}(i)=1/\lambda_{5}(i/5)$ and therefore $G(e^{-2\pi})=G(e^{-2\pi/5})$. From Ramanujan's identity (iii) in \S\ref{art8-sec4}, viz.
$$
E^{2}_{6}=\eta^{24}(\eta^{3}/\eta^{3}_{5}-500\eta^{3}_{5}/\eta^{3}-5^{6}\cdot \eta^{9}_{5}/\eta^{9})^{2}(1+22\lambda_{5}+125\lambda^{2}_{5}),
$$
we infer $\lambda_{5}(i)$ is a root of the polynomial $1-500X-15625X^{2}$ since $E_{6}(i)=0$ and the polynomial $1+22X+125X^{2}$ has no real root. Thus $\lambda_{5}(i)=1/(5\epsilon)^{3}$ and consequently, $G(e^{-2\pi/5})=G(e^{-2\pi})=6\cdot 5^{1/4}(3+\sqrt{5})$; however, this does {\em not} agree with the value for $C$ given by Ramanujan.

\begin{remark*}
It is possible to express $C'$ in terms of complete elliptic integrals of the first and the second kind as Ramanujan has noted in connection with $C$ in \eqref{art8-eq44}.
\end{remark*}

\section{Identities for Eisenstein series}\label{art8-sec4}
We discuss in this section several useful identities for $E_{4}(z)$, $E_{4}(pz)$, $E_{6}(z)$ and $E_{6}(pz)$ for $p=5$, $7$ written down by Ramanujan (\cite{art8-key11}, p. \mycirc{70}/78, also p. \mycirc{67}/75) :
\begin{itemize}
\item[(i)] $E^{3}_{4}(z)=(\eta^{10}/\eta^{2}_{5}+250\eta^{4}\eta^{4}_{5}+5^{5}\cdot \eta^{10}_{5}/\eta^{2})^{3}$

\item[(ii)] $E^{3}_{4}(5z)=(\eta^{10}/\eta^{2}_{5}+10\eta^{4}\eta^{4}_{5}+5\eta^{10}_{5}/\eta^{2})^{3}$

\item[(iii)] $E^{2}_{6}(z)=\eta^{24}(\eta^{3}/\eta^{3}_{5}-500\eta^{3}_{5}/\eta^{3}-5^{6}\eta^{9}_{5}/\eta^{9})^{2}(1+22\eta^{6}_{5}/\eta^{6}+125\eta^{12}_{5}/\eta^{12})$

\item[(iv)] $E^{2}_{6}(5z)=\eta^{24}_{5}(\eta^{15}/\eta^{15}_{5}+4\eta^{9}/\eta^{9}_{5}-\eta^{3}/\eta^{3}_{5})^{2}(1+22\eta^{6}_{5}/\eta^{6}+125\eta^{12}_{5}/\eta^{12})$

\item[(v)]\pageoriginale $E^{3}_{4}(z)=(\eta^{7}/\eta_{7}+5\cdot 7^{2}\eta^{3}\eta^{3}_{7}+7^{4}\eta^{7}_{7}/\eta)^{3}(\eta^{7}/\eta_{7}+13\eta^{3}\eta^{3}_{7}+49\eta^{7}_{7}/\eta)$

\item[(vi)] $E^{3}_{4}(7z)=(\eta^{7}/\eta_{7}+5\eta^{3}\eta^{3}_{7}+\eta^{7}_{7}/\eta)^{3}(\eta^{7}/\eta_{7}+13\eta^{3}\eta^{3}_{7}+49\eta^{7}_{7}/\eta)$

\item[(vii)] $E^{2}_{6}(z)=(\eta^{7}/\eta_{7}-7^{2}(5+2\sqrt{7})\eta^{3}\eta^{3}_{7}-7^{3}(21+8\sqrt{7})\eta^{7}_{7}/\eta)^{2}\times$

\qquad\qquad $(\eta^{7}/\eta_{7}-7^{2}(5-2\sqrt{7})\eta^{3}\eta^{3}_{7}-7^{3}(21-8\sqrt{7})\eta^{7}/\eta)^{2}$

\item[(viii)] $E^{2}_{6}(7z)=(\eta^{7}/\eta_{7}-(7+2\sqrt{7})\eta^{3}\eta^{3}_{7}+(21+8\sqrt{7}\eta^{7}_{7}/\eta)^{2}\times$

\qquad\qquad $(\eta^{7}/\eta_{7}-(7-2\sqrt{7})\eta^{3}\eta^{3}_{7}+(21-8\sqrt{7})\eta^{7}_{7}/\eta)^{2}$
\end{itemize}

Identities (i) and (ii) have already been proved as (ii) and (iii) in Proposition \ref{art8-prop2} in \S\ref{art8-sec3.4}. One can derive the other identities in a similar fashion. We can also deduce these identities from the results of Klein (\cite{art8-key7}, p. 46). If $\tau_{1}:=-\lambda^{-1}_{5}$ and $\tau_{2} := \lambda^{-1}_{7}$, then for elliptic modular function $j(z)$, we have from Klein (see also \cite{art8-key4}, p. 329) the following relations :
\begin{itemize}
\item[(i)] $1728 j(z)=(\tau^{2}_{1}-250\tau_{1}+5^{5})^{3}/(-\tau^{5}_1)$

\item[(ii)] $1728 j(5z)=(\tau^{2}_{1}-10\tau_{1}+5)^{3}/(-\tau_{1})$

\item[(iii)] $1728(j(z)-1)=(\tau^{2}_{1}+500\tau_{1}-5^{6})^{2}(\tau^{2}_{1}-22\tau_{1}+125)/(-(\tau^{5}_{1}))$

\item[(iv)] $1728(j(5z)-1)=(\tau^{2}_{1}-4\tau_{1}-1)^{2}(\tau^{2}_{1}-22\tau_{1}+125)/(-\tau_{1})$

\item[(v)] $1728 j(z)=(\tau^{2}_{2}+5\cdot 7^{2}\tau_{2}+7^{4})^{3}(\tau^{2}+13\tau_{2}+49)/\tau^{7}_{2}$

\item[(vi)] $1728 j(7z)=(\tau^{2}_{2}+5\tau_{2}+1)^{3}(\tau^{2}_{2}+13\tau_{2}+49)/\tau_{2}$

\item[(vii)] $1728 [j(z)-1)=(\tau^{4}_{2}=10\cdot 7^{2}\tau^{3}_{2}-9\cdot 7^{4}\tau^{2}_{2}-2\cdot 7^{6}\tau_{2}-7^{7})^{2}/\tau^{7}_{2}$

\item[(viii)] $1728 (j(7z)-1)=(\tau^{4}_{2}+14\cdot \tau^{3}_{2}+9\cdot 7\tau^{2}_{2}+10\cdot 7\tau_{2}-7)^{2}/\tau_{2}$
\end{itemize}
Substituting for $j(z)=E^{3}_{4}/(E^{3}_{4}-E^{2}_{6})=E^{3}_{4}/(1728 \ \eta^{24})$, we deduce Ramanujan's identities (i)-(viii) immediately from the above identities (i)-(viii).

One finds a few more striking identities involving the Eisenstein series $E_{4}$ and $E_{6}$, stated by Ramanujan (\cite{art8-key11}, p. \mycirc{67} + 1) :
$$
(E^{2}_{4}(z)+94E_{4}(z)E_{4}(5z)+625E^{2}_{4}(5z))^{1/2}
$$
\begin{equation}
=12\sqrt{5}(\eta^{10}/\eta^{2}_{5}+26\eta^{4}\eta^{4}_{5}+125\eta^{10}_{5}/\eta^{2})\label{art8-eq47}
\end{equation}
\vskip -.7cm
\begin{gather}
(5(E_{6}(z)+125E_{6}(5z))^{2}-(126)^{2}E_{6}(z)E_{6}(5z))^{1/2}\label{art8-eq48}\\[4pt]
=252(\eta^{10}/\eta^{2}_{5}+62\eta^{4}\eta^{4}_{5}+125\eta^{10}_{5}/\eta^{2})(\eta^{10}/\eta^{2}_{5}+22\eta^{4}\eta^{4}_{5}+125\eta^{10}_{4}/\eta^{2})^{1/2}\notag
\end{gather}

Identity \eqref{art8-eq47} may be verified by squaring both sides, substituting the expressions for $E_{4}(z)$ and $E_{4}(5z)$ from the identities (i)-(ii) above and checking that the coefficients corresponding to the monomials $\eta^{24}$, $\eta^{18}\eta^{6}_{5}$, $\eta^{12}\eta^{12}_{5}$, $\eta^{6}\eta^{18}_{5}$ and $\eta^{24}_{5}$ are the same on both sides. A similar remark applies to identity \eqref{art8-eq48}, using now identities (iii)-(iv) for $E_{6}$ above.

\section{Differential equations satisfied by `Eisenstein series'}\label{art8-sec5}
We discuss, in this section, a differential equation mentioned by Ramanujan\pageoriginale (\cite{art8-key11}, p. \mycirc{73}/81) for certain `Eisenstein series'. First, let us recall `Hecke summation' (\cite{art8-key5}, p. 469) for Eisenstem series of weight 2 :
\begin{align}
&\Lt\limits_{s\to 0}\sum\limits_{\substack{(c,d)=1\\ c\geq 0}}(cz+d)^{-2}|cz+d|^{-s}\notag\\
&\qquad =\frac{-6i}{\pi(z-\overline{z})}+1-24\sum\limits^{\infty}_{n=1}\left(\sum\limits_{0<d|n}d\right)e^{2\pi inz}\notag\\
&\qquad\qquad =\dfrac{-6i}{\pi(z-\overline{z})}+E_{2}(z)\label{art8-eq49}
\end{align}
The left hand side of \eqref{art8-eq49} picks up a factor of automorphy of weight 2 under modular substitutions but is not holomorphic in $z$, due to the presence of $-6i/(\pi(z-\overline{z}))$ on the right hand side; on the other hand, $E_{2}$ is holomorphic but it does not have a nice behaviour under modular transformations. Now associated with $E_{2}(z)$ and $E_{2}(5z)$, Ramanujan has introduced on page \mycirc{73}/81 of \cite{art8-key11} a function $F$ through the equations
\begin{align}
E_{2}(z) &= (\eta^{5}/\eta_{5})[(1+22\lambda_{5}+125\lambda^{2}_{5})^{1/2}-30F(\lambda_{5})]\notag\\
E_{2}(5z) &= (\eta^{5}/\eta_{5})[(1+22\lambda_{5}+125\lambda^{2}_{5})^{1/2}-6F(\lambda_{5})]\label{art8-eq50}
\end{align}
These two equations for $F$ are certainly consistent by virtue of (i) in Proposition \ref{art8-prop3} above; further, they clearly imply that
\begin{gather}
F(\lambda_{5})=-(1/24)(\eta_{5}/\eta^{5})(E_{2}(z)-E_{2}(5z))\notag\\
=-(\eta_{5}/\eta^{5})\mathscr{E}_{5}(z)\label{art8-eq51}
\end{gather}
on defining $\mathscr{E}_{5}(z)=(E_{2}(z)-E_{2}(5z))/24$. For $F(\lambda_{5})$, Ramanujan (\cite{art8-key11}, p. \mycirc{73}) has recorded the non-linear differential equation.
\begin{equation}
(1+22\lambda_{5}+125\lambda^{2}_{5})^{1/2}\left(\frac{d}{d\lambda_{5}}F\right)=1+(25/2)\lambda_{5}+[5/(2\lambda_{5})]F^{2}(\lambda_{5}).\label{art8-eq52}
\end{equation}
This differential equation may be verified by using the relation $E_{4}=E^{2}_{2}-12\vartheta(E_{2})$ from (1) and the identity (i) above, namely
$$
E_{4}=g^{2}_{5}(1+250\lambda_{5}+3125\lambda^{2}_{5})=g^{2}_{5}(h_{5}(\lambda_{5})+228\lambda_{5}+3000\lambda_{5}^{2})
$$
with
$$
g_{5}:=\eta^{5}/\eta_{5}\quad\text{and}\quad h_{5}(X):=1+22X+125X^{2}.
$$
In fact, by \eqref{art8-eq50}, $E^{2}_{2}=g^{2}_{5}(h_{5}(\lambda_{5})-60\sqrt{h_{5}(\lambda_{5})}F+900F^{2})$ and further $\vartheta(E_{2})=g^{2}_{5}(-5\sqrt{h_{5}(\lambda_{5})}F+150F^{2}+11\lambda_{5}+125\lambda^{2}_{5}-30\lambda_{5}\sqrt{h_{5}(\lambda_{5})}\frac{d}{d\lambda_{5}}F)$ on noting that $\vartheta\lambda_{5}=g_{5}\lambda_{5}\sqrt{h_{5}(\lambda_{5})}$ in view of the Corollary to Proposition \ref{art8-prop3}. Now $F=-\mathscr{E}_{5}/g_{5}$ and logarithmic differentiation with respect to $z$ gives $(\vartheta F)/F=(\vartheta\mathscr{E}_{5})/\mathscr{E}_{5}-(\vartheta g_{5})/g_{5}=(\vartheta\mathscr{E}_{5})/\mathscr{E}_{5}+5g_{5}F$, by \eqref{art8-eq8} and \eqref{art8-eq50}. This enables us to rewrite \eqref{art8-eq52} as a differential equation for $\mathscr{E}_{5}$~:
\begin{equation}
\vartheta\mathscr{E}_{5}-\frac{5}{2}\mathscr{E}^{2}_{5}=-\eta^{4}\eta^{4}_{5}-\frac{25}{2}\eta^{10}_{5}/\eta^{2}.\label{art8-eq53}
\end{equation}
The\pageoriginale right hand side can be expressed as a linear combination of $E_{4}(z)$, $E_{4}(5z)$ and $\eta^{4}\eta^{4}_{5}$ which form a basis for the space of modular forms of Haupttypus $(-4,5,1)$.

\begin{remark*}
The differential equation \eqref{art8-eq53} for $\mathscr{E}_{5}$ is perhaps the right analogue of the first relation in (1), for the case of $\Gamma_{0}(5)$. We derive next an analogue for the case of $\Gamma_{0}(7)$ as follows.
\end{remark*}

Let us start from Ramanujan's identity (v) above for $E_{4}$, namely
\begin{equation}
E_{4}=g_{7}(1+5\cdot 7^{2}\lambda_{7}+7^{4}\lambda^{2}_{7})(g_{7}\cdot h_{7}(\lambda_{7}))^{1/3}\label{art8-eq54}
\end{equation}
where $g_{7}:=\eta^{7}/\eta_{7}$ and $h_{7}(X):=1+13X+49X^{2}$. We might mention here that $(g_{7}\cdot h_{7}(\lambda_{7}))^{1/3}$ is just the Eisenstein series $E_{1}(z;\chi_{7})=1+2\sum\limits^{\infty}_{n=1}\chi_{7}(n)\times x^{n}/(1-x^{n})$ of weight $1$ and (real) Nebentypus $(-1,7,\chi_{7})$; further $E^{2}_{1}(z,\chi_{7})$ happens to be precisely $-(1/6)E_{2}(z;1;\Gamma_{0}(7))$. By analogy with Ramanujan's function $F(\lambda_{5})$ above, let us introduce $F_{1}=F_{1}(\lambda_{7})$ by the two equations
\begin{align}
E_{2}(z) &= g^{2/3}_{7}((7/6)F_{1}(\lambda_{7})+h^{2/3}_{7}(\lambda_{7}))\notag\\
E_{2}(7z) &= g^{2/3}_{7}((1/6)F_{1}(\lambda_{7})+h^{2/3}_{7}(\lambda_{7}))\label{art8-eq55}
\end{align}
The consistency of \eqref{art8-eq55} follows from the relation 
$$
E_{2}(z;1;\Gamma_{0}(7))+6E^{2}_{1}(z;\chi_{7})=0.
$$ 
In view of \eqref{art8-eq8}, we have 
$$
F_{1}(\lambda_{7})=g^{-2/3}_{7}(E_{2}(z)-E_{2}(7z))=(12/7\pi i)g^{-2/3}_{7}\frac{d}{dz}(\log g_{7}).
$$ 
Moreover, by \eqref{art8-eq10}, \eqref{art8-eq7} and \eqref{art8-eq55}.
$$
(1/\lambda_{7})\left(\frac{d}{dz}{\lambda_{7}}\right)=(2\pi i/6)(7E_{2}(7z)-E_{2}(z))=2\pi ig^{2/3}_{7}h^{2/3}_{7}(\lambda_{7}).
$$
Now \eqref{art8-eq1}, \eqref{art8-eq54} and \eqref{art8-eq55} together imply that
\begin{align*}
& g^{4/3}_{7}h^{1/3}_{7}(\lambda_{7})(1+5\cdot 7^{2}\lambda_{7}+7^{4}\cdot \lambda^{2}_{7})\\
&\qquad =g^{4/3}_{7}((49/36)F^{2}_{1}+(7/3)F_{1}\cdot h^{2/3}_{7}(\lambda_{7})+\\
& h^{4/3}_{7}(\lambda_{7}))-12\vartheta E_{2}\text{~~ and~~ } \left(\frac{d}{dz}E_{2}\right)\\
&\qquad =(2/3)g^{-1/3}_{7}\left(\frac{d}{dz}g_{7}\right)\left(\frac{7}{6}F_{1}+h^{2/3}_{7}(\lambda_{7})\right)+\\
& g^{2/3}_{7}\left(\frac{d}{dz}\lambda_{7}\right)\left(\frac{7}{6}F'_{1}(\lambda_{7})+\frac{2}{3}h^{-1/3}_{7}(\lambda_{7})(13+98\lambda_{7})\right).
\end{align*}
Consequently, we obtain the differential equation
$$
F'_{1}(\lambda_{7})h^{1/3}_{7}(\lambda_{7})+[7/(72\lambda_{7})]h^{-1/3}_{7}F^{2}_{1}(\lambda_{7})+224\lambda_{7}+24=0.
$$
Defining $\mathscr{E}_{7}$ by $\mathscr{E}_{7}=(1/24)g^{2/3}_{7}F_{1}$, we see that
\begin{align*}
(1/\mathscr{E}_{7})\dfrac{d\mathscr{E}_{7}}{dz} &= [F_{1}(\lambda_{7})/F_{1}(\lambda_{7})]\dfrac{d\lambda_{7}}{dz}+(2/3)\dfrac{d}{dz}(\log g_{7})\\
&=-2\pi i\lambda_{7} \ g^{2/3}_{7}\times h^{1/3}_{7}(\lambda_{7})\times [(7/72\lambda_{7})]h^{-1/3}_{7}(\lambda_{7})\\
&\qquad\qquad F_{1}+(224\lambda_{7}+24)/F_{1})+(28\pi i/3)\mathscr{E}_{7}\\
&= (14\pi i/3)\mathscr{E}_{7}-(2\pi i/3)(3+28\lambda_{7})\lambda_{7}g^{4/3}_{7}h^{1/3}_{7}(\lambda_{7})/\mathscr{E}_{7}.
\end{align*}
This leads at once to a differential equation for $\mathscr{E}_{7}$ (analogous to \eqref{art8-eq53}) :
\begin{align*}
\vartheta\mathscr{E}_{7}-\frac{7}{3}\mathscr{E}\frac{2}{7} &=-\frac{1}{3}(3\lambda_{7}+28\lambda^{2}_{7})g^{4/3}_{7}h^{1/3}_{7}(\lambda_{7})\\
&= -\frac{1}{3}(3\lambda_{7}g_{7}+28\lambda^{2}_{7}g_{7})E_{1}(z;\chi_{7})\\
&= -\eta^{3}\eta^{3}_{7}E_{1}(z;\chi_{7})-(28/3)(\eta^{7}_{7}/\eta)E_{1}(z;\chi_{7}).
\end{align*}\pageoriginale
The right hand side, being a modular form of weight 4 for $\Gamma_{0}(7)$ is expressible as a linear combination of $E_{4}(z)$, $E_{4}(7z)$ and $(\eta\eta_{7})^{3}E_{1}(z;\chi_{7})$; note that by Ramanujan's identities (v), (vi) above, $E_{1}(z;\chi_{7})$ divides $E_{4}(z)$ and $E_{4}(7z)$ and hence every modular form of Haupttypus $(-4,7,1)$.

\begin{remark*}
Eichler and Zagier \cite{art8-key2} have obtained for the $N$-divisor values of Weierstrass' $\wp$-function a non-linear differential equation of degree 2 with the coefficients independent of $N$. Eisenstein series of weight 2 for $\Gamma(N)$ are known to be expressible linearly in terms of these Weierstrass' $N$-divisor values. For the `Eisenstein series' $\mathscr{E}_{5}$, $\mathscr{E}_{7}$ which are analogues of $E_{2}$ for $\Gamma_{0}(5)$, $\Gamma_{0}(7)$ respectively, we have again a non-linear differential equation whose coefficients however {\em depend on the stufe}. In any case, it is remarkable that Ramanujan has recorded in \cite{art8-key11} the interesting differential equation \eqref{art8-eq52} for $F(\lambda_{5})$ which is the same as \eqref{art8-eq53} and which does not seem to have been observed prior to Ramanujan.
\end{remark*}

\begin{thebibliography}{99}
\bibitem{art8-key1} \textsc{B. C. Berndt :} Chapter 19 of Ramanujan's second notebook (Preprint).

\bibitem{art8-key2} \textsc{M. Eichler} and \textsc{D. Zagier :} On the zeros of the Weierstrass $\wp$-function, {\em Math. Annalen} 258 (1982), 399-407.

\bibitem{art8-key3} \textsc{R. Fricke :} {\em Die Elliptische Funktionen und ihre Anwendungen}, Bd. II, B. G. Teubner Verlag, 1922.

\bibitem{art8-key4} \textsc{A. G. Greenhill :} {\em The Applications of Elliptic Functions}, Dover Publications Inc., New York.

\bibitem{art8-key5} \textsc{E. Hecke :} {\em Gesammelte Abhandlungen,} Vandenhoeck und Ruprecht, G\"ottingen.

\bibitem{art8-key6} \textsc{G. H. Hardy :} {\em Ramanujan,} Chelsea Publishing Company, New York.

\bibitem{art8-key7} \textsc{F. Klein :}\pageoriginale {\em Gesammelte Abhandlungen,} Vol. III, Springer-Verlag.

\bibitem{art8-key8} \textsc{S. Raghavan :} On certain identities due to Ramanujan, Quart. J. Math. Oxford (2), 37(1986) 221-229.

\bibitem{art8-key9} \textsc{S. Ramanujan :} {\em Collected Papers}, Cambridge University Press.

\bibitem{art8-key10} \textsc{S. Ramanujan :} {\em Notebooks of Srinivasa Ramanujan}, Vol. II, Tata Institute of Fundamental Research, Bombay.

\bibitem{art8-key11} \textsc{S. Ramanujan :} {\em ``Lost Notebook''} (Manuscript with Trinity College Library, Cambridge), {\em The Lost Notebook and Other Unpublished Papers,} Narosa Publishing House, New Delhi.

\bibitem{art8-key12} \textsc{H. J. S. Smith :} {\em Collected Papers,} Vol. II, Chelsea Publishing Company, New York.

\bibitem{art8-key13} \textsc{G. N. Watson :} Proof of certain identities in combinatory analysis, {\em Jour. Indian Math. Soc.} 20 (1933), 57-69.

\bibitem{art8-key14} \textsc{H. Weber :} {\em lehrbuch der Algebra,} Bd. III, Chelsea Publishing Company, New York.

\end{thebibliography}

\bigskip
\noindent
{\small School of Mathematics}

\noindent
{\small Tata Institute of Fundamental Research}

\noindent
{\small Homi Bhabha Road}

\noindent
{\small Bombay 400 005 (India)}
