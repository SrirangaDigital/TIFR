\title{ON RAMANUJAN'S ELLIPTIC INTEGRALS AND MODULAR IDENTITIES}
\markright{On Ramanujan's Elliptic Integrals and Modular Identities}

\author{By~ S. Raghavan and S. S. Rangachari}
\markboth{S. Raghavan and S. S. Rangachari}{On Ramanujan's Elliptic Integrals and Modular Identities}

\date{}
\maketitle

\setcounter{pageoriginal}{118}
\section*{Introduction}\pageoriginale
It is known from Hecke (\cite{art8-key5}, p. 472) that `special integrals of the third kind' of stufe $N$ (i.e. having logarithmic singularities at most at the cusps of the principal congruence subgroup $\Gamma(N)$ turn out to be of elementary type, namely, logarithms of functions invariant under $\Gamma(N)$. Results of this kind have been discovered much earlier by Ramanujan in \cite{art8-key11} as may be seen in the sequel, especially in \S\ref{art8-sec2}. In this connection, Ramanujan was perhaps, one of the first to have considered the problem of `evaluating' elliptic integrals associated with modular curves of small level, although elliptic integrals, in general, have been investigated in depth by various mathematicians such as Jacobi, Cayley and others. In the literature, transformations of orders 2, 3, 5 have been employed with a view to reduce formidable elliptic integrals to simpler (or more explicit) form \cite{art8-key4}.

In \cite{art8-key11}, Ramanujan has considered elliptic integrals associated with $\Gamma_{0}(N)$ for $N=5,7,10,14,15$ and also a solitary hyperelliptic integral (for $\Gamma_{0}(35)$). Various such elliptic integrals are found in scattered form in \cite{art8-key11}, with endeavours, via quadratic and higher order transformations and interesting modular relations, to simplify them.

The principal objective of this paper is to make a systematic study of all the elliptic integrals and associated formulae recorded by Ramanujan in \cite{art8-key11} and provide complete proofs.

We shall, in addition, uphold various modular identities involving Eisenstein series stated by Ramanujan in different places in \cite{art8-key11} and presumably used by him in computing singular values of modular functions. With the help of such identities, we exhibit a nonlinear differential equation for Eisenstein series denoted in \S\ref{art8-sec3} by $\mathscr{E}_{p}$, for $p=5,7$; for $p=5$, this differential equation is essentially equivalent to the nonlinear differential equation written down by ramanujan \cite{art8-key11} for a function $F(\lambda_{5})$, where $\lambda_{5}$ is a `Hauptmodul' for $\Gamma_{0}(5)$. One has to compare these with nonlinear differential equations obtained by Eichler-Zagier \cite{art8-key2} for divisor values of Weierstrass' elliptic function.

\section{Notation and preliminary results}\label{art8-sec1}\pageoriginale
~

\subsection{}\label{art8-sec1.1}
Let $\mathfrak{H}$ denote the complex upper half-plane and for $z\in \mathfrak{H}$, let $x=e^{2\pi iz}$, so that $|x|<1$. By $\Gamma(1)=\Gamma$, we mean the modular group $\{\left(\begin{smallmatrix} a & b\\ c & d\end{smallmatrix}\right) |a,b,c,d\in \mathbb{Z}$, $ad-bc=1\}$, acting on $\mathfrak{H}$ via the analytic homeomorphisms $z\to (az+b)(cz+d)^{-1}$. As usual, $\Gamma(N):=\{\left(\begin{smallmatrix} a & b\\ c & d\end{smallmatrix}\right)\in \Gamma |c\equiv 0(\mod N)\}$. By $Q$, $R$ we mean the normalized Eisenstein series $E_{4}(z)=1+240\sum\limits^{\infty}_{n=1}\left(\sum\limits_{0<d|n}d^{3}\right)e^{2\pi inz}$, $E_{6}(z)=1-504\sum\limits^{\infty}_{n=1}\left(\sum\limits_{0<d|n}d^{5}\right)\times e^{2\pi inz}$ of weights 4 and 6 respectively. Further, $P$ stands for $E_{2}(z)=1-24\times \sum\limits^{\infty}_{n=1}\left(\sum\limits_{0<d|n}d\right)e^{2\pi inz}$. Connecting $E_{2}$, $E_{4}$ and $E_{6}$, we have the important relations (\cite{art8-key9}, p. 142) :
\begin{equation}
\begin{split}
& E_{4}-E^{2}_{2}=-12\vartheta (E_{2})\\
& E_{6}-E_{2}E_{4}=-3\vartheta (E_{4})\\
& E^{2}_{4}-E_{2}E_{6}=-2\vartheta (E_{6})
\end{split}\label{art8-eq1}
\end{equation}
where $\vartheta := \dfrac{1}{2\pi i}\dfrac{d}{dz}=x\dfrac{d}{dx}$.

\smallskip

Following Ramanujan, we write for $|x|$, $|x'|<1$,
\begin{align*}
f(x,x') :&= 1+\sum\limits^{\infty}_{n=1}(xx')^{n(n-1)/2}(x^{n}+(x')^{n})\\
         &= \prod\limits^{\infty}_{n=0}[1+x(xx')^{n}][1+x'(xx')^{n}][1-(xx')^{n+1})\\
         &\hspace{3cm}\text{(from Gauss and Jacobi).}
\end{align*}
Further, setting $f(-x):=f(-x,-x^{2})$ for $|x|<1$, we know that $\eta(z)=x^{1/24}f(-x)$ is just Dedekind's $\eta$-function; indeed, $\eta(z)=e^{2\pi iz/24}\times \prod\limits^{\infty}_{n=1}(1-e^{2\pi inz})$ and moreover, $\eta^{24}=(1/1728)(E^{3}_{4}-E^{2}_{6})$. For rational $r>0$, let $\eta_{r}$ be defined by $\eta_{r}(z)=\eta(rz)$ for $z\in \mathfrak{H}$. Then the function $x^{-1/8}\times \eta^{2}_{2}(z)/\eta(z)$ is just Ramanujan's function $\psi(z)$, as found in \cite{art8-key10;art8-key11}. Setting 
$$
u=x^{1/5}f(-x,-x^{4})/f(-x^{2},-x^{3})=x^{1/5}\prod\limits^{\infty}_{n=0}\frac{(1-x^{5n+1})(1-x^{5n+4})}{(1-x^{5n+2})(1-x^{5n+3})}
$$
we know that $u$ is a `Hauptmodul' for the group $\Gamma(5)$ of genus $0$. Also, $u$, can be represented as an infinite continued fraction continued fraction $u=\dfrac{x^{1/5}}{1+}\dfrac{x}{1+}\dfrac{x^{2}}{1+}\cdots$ Moreover,\pageoriginale we know from Ramanujan \cite{art8-key10} the modular relations :
\begin{align}
& 1/u-1-u=\eta_{1/5}/\eta_{5},\label{art8-key2}\\
& 1/u^{5}-11-u^{5}=\eta^{6}/\eta^{6}_{5}.\label{art8-key3}
\end{align}
Let us note that, on the right hand side of \eqref{art8-eq3}, we have just $1/\lambda_{5}$, where $\lambda_{5}:=\eta^{6}_{5}/\eta^{6}$ is just a `Hauptmodul' for the group $\Gamma_{0}(5)$ of genus $0$. Likewise, for the congruence subgroup $\Gamma_{0}(7)$ of genus $0$, we have $\lambda_{7} := \eta^{4}_{7}/\eta^{4}$ as a `Hauptmodul'.

Between $u$ and $u_{2}$ defined by $u_{2}(z)=u(2z)$, we have the modular relation (\cite{art8-key10}, p. 326) :
\begin{align}
u_{2}/u^{2} &= (1+uu^{2}_{2})/(1-uu^{2}_{2})\notag\\
          &= (1+k)/(1-k)\label{art8-key4}
\end{align}
where we have written $k$ for $uu^{2}_{2}$ following Ramanujan (\cite{art8-key11}, p. \mycirc{70}/78). From \eqref{art8-eq4}, we see easily that
\begin{equation}
u^{5}=k(1-k)^{2}/(1+k)^{2}, \ u^{5}_{2}=k^{2}(1+k)/(1-k).\label{art8-eq5}
\end{equation}

The group $\Gamma_{0}(p)$ for odd primes $p$ is known to have two cusps $0$ and $\infty$ and Eisenstein series of Nebentypus $(-l, p, \chi_{p})$ corresponding to the two cusps are given. for $l\geq 2$, by
\begin{gather}
E^{0}_{l}(z;\chi_{p}):= \sum\limits^{\infty}_{n=1}\left(\sum\limits_{1\leq d|n}d^{l-1}\chi_{p}(n/d)\right)e^{2\pi inz}\label{art8-eq6}\\
E^{\infty}_{l}(z;\chi_{p}):=\frac{(-1)^{[l/2]}p^{[l-1]/2}}{(2\pi)^{l}}(l-1)!\sum\limits^{\infty}_{n=1}\chi_{p}(n)n^{-l}+\notag\\
\sum\limits^{\infty}_{n=1}\left(\sum\limits_{1<dn}d^{l-1}\chi_{p}(d)\right)e^{2\pi inz}\notag
\end{gather}
where $\chi_{p}(m)$ denotes the Legendre symbol $(\frac{m}{p}),[\quad]$ denotes the integral part and further, $\chi_{p}(-1)=(-1)^{l}$, necessarily (\cite{art8-key5}, p. 818). There exist two linearly independent series $E_{l}(z)$, $E_{l}(pz)$ for even $l>2$, which are of Haupttypus $(-l,p,1)$; however, for $l=2$, there is just one Eisenstein series of Haupttypus $(-2, p, 1)$ and it is given by
\begin{align*}
E_{2}(z;1;\Gamma_{0}(p)) &= E_{2}(z)-pE_{2}(pz)\\
&= 1-p-24\sum\limits^{\infty}_{\substack{n=1\\ p\nmid d}}\left(\sum\limits_{1\leq d|n}\right)e^{2\pi inz}
\end{align*}

For\pageoriginale $p=5$ and $l=2$, we know from \cite{art8-key8} that $E^{0}_{2}(z;\chi_{5})=\eta^{5}_{5}/\eta$ and further, denoting simply by $E_{2}(z;\chi_{5})$ obtained by `normalizing' $E^{\infty}_{2}(z;\chi_{5})$ so as to have constant term $-(1/5)$, we also have $E_{2}(z;\chi_{5})=-(1/5)\eta^{5}/\eta_{5}$.

In the case $p=7$ and $l=3$, $E^{\infty}_{3}(z;\chi_{7})$ is upto a constant factor, just the cube of the Eisenstein series $E_{1}(z;\chi_{7})$ of Nebentypus $(-1,7,\chi_{7})$ given by
$$
E_{1}(z;\chi_{7})=1+2\sum\limits^{\infty}_{n+1}\chi_{7}(n)\dfrac{x^{n}}{1-x^{n}}\quad \text{(see \cite{art8-key11}, \cite{art8-key8})}.
$$

Logarithmic differentiation of $\eta$ with respect to $x$ gives
\begin{equation}
\frac{24}{\eta}x\dfrac{d\eta}{dx}=\dfrac{24}{2\pi i\eta}\dfrac{d\eta}{dz}=E_{2}(z)\label{art8-eq8}
\end{equation}
Similarly, logarithmic differentiation of $u=x^{1/5}\prod\limits^{\infty}_{n=1}(1-x^{n})^{\chi_{5}(n)}$ and $u_{2}=x^{2/5}\prod\limits^{\infty}_{n=1}(1-x^{2n})^{\chi_{5}(n)}$ gives
\begin{align}
\frac{1}{u}\dfrac{du}{dx} &= \frac{1}{5x}-\sum\limits^{\infty}_{n=1}\frac{n\chi_{5}(n)x^{n-1}}{1-x^{n}}\label{art8-eq9}\\
&= -\frac{1}{x}E_{2}(z:\chi_{5}),\notag\\[4pt]
\frac{1}{u_{2}}\frac{du_{2}}{dx} &= -\frac{2}{x}E_{2}(2z;\chi_{5}).\notag
\end{align}
For the Hauptmodul $\lambda_{p}=(\eta_{p}/\eta)^{24/(p-1)}$ for $p=5$, $7$ we have likewise
\begin{equation}
\frac{1-p}{2\pi i\cdot \lambda_{p}}\frac{d\lambda_{p}}{dz}=E_{2}(z;1;\Gamma_{0}(p))\label{art8-eq10}
\end{equation}
in view of \eqref{art8-eq8} and \eqref{art8-eq7}.

Setting $r=\dfrac{k}{1-k^{2}}$, $s=\dfrac{1+k-k^{2}}{1-4k-k^{2}}$ where $k$ is just the function $uu^{2}_{2}$ considered above, we recall from Weber (\cite{art8-key14}, p. 86) the formulae 
\begin{align*}
rs^{5} &= x\prod\limits^{\infty}_{n=1}(1+x^{n})^{24}=(\eta_{2}/\eta)^{24}\\
r^{5}s &= x^{5}\prod\limits^{\infty}_{n=1}(1+x^{5n})^{24}=(\eta_{10}/\eta_{5})^{24}.
\end{align*}
These\pageoriginale imply immediately that $r^{24}=(r^{5}s^{5})/rs^{5}$ and so
\begin{align}
& \frac{1-k^{2}}{k} =\frac{1}{r}=\frac{\eta_{2}\eta^{5}_{5}}{\eta\eta^{5}_{10}}\label{art8-eq11}\\
&=\frac{1}{x\psi^{2}(x^{5})}\frac{\eta_{2}\eta^{3}_{5}}{x^{1/4}\eta\eta_{10}}\notag\\
&=\frac{1}{x\psi^{2}(x^{5})}\cdot f(x,x^{4})f(x^{2},x^{3})\notag\\
&=\frac{\psi^{2}(x)-x\psi^{2}(x^{5})}{x\psi^{2}(x^{5})}\label{art8-eq12}.
\end{align}
The last-mentioned equality is a consequence of Ramanujan's identity
$$
\psi^{2}(x)-x\psi^{2}(x^{5})=f(x,x^{4})f(x^{2},x^{3})
$$
proved by Watson \cite{art8-key13} and called a ``rudimentary'' example of the use of quadratic forms. (See also \cite{art8-key1}, pp. 63-65 which provides a proof ostensibly ``more difficult'' and further ``similar to that of Entry 9(iii), which is obviously an analogue of Entry 10(v), a fact made even more transparent by Entry 10(iv)''). In contrast, the following Proposition 1(ii) gives an independent, refreshingly different and perhaps even elegant proof for Ramanujan's identity above; in fact, one needs merely to substitute $\psi(x)=x^{-1/8}\eta^{2}_{2}(z)/\eta(z)$ therein and note further that $f(x,x^{4})f(x^{2},x^{3})=x^{-1/4}\eta_{2}(z)\eta^{3}_{5}(z)/(n(z)\eta_{10}(z))$. This proof corroborates G.N. Watson's belief that Ramanujan discovered this formula ``not by manipulating quadratic forms but by transforming series of Lambert's type''.

\medskip
\noindent
{\bf Proposition \thnum{1}.\label{art8-prop1}}
\begin{itemize}
\item[(i)] $\eta^{4}_{2}\eta^{2}_{5}-5\eta^{2}\eta^{4}_{10}=\eta^{5}\eta_{5}\eta_{10}/\eta_{2}$

\item[(ii)] $\eta^{4}_{2}\eta^{2}_{5}-\eta^{2}\eta^{4}_{10}=\eta\eta_{2}\eta^{5}_{5}.\eta_{10}$.
\end{itemize}

\begin{proof}
We first rewrite these identities in terms of the Eisenstein series $E^{0}_{2}(z;\chi_{5})$ and $E^{\infty}_{2}(z;\chi_{5})$ obtained by normalizing $E^{\infty}_{2}(z;\chi_{5})$, using the relations $E^{0}_{2}(z;\chi_{5})=\eta^{5}_{5}/\eta$, $E_{2}(z;\chi_{5})=-\frac{1}{5}\eta^{5}/\eta_{5}$ and the `Hauptmodul' $\tau : 10\eta_{2}\eta^{3}_{10}/(\eta^{3}\eta_{5})$ for $\Gamma_{0}(10)$, as follows :
\begin{itemize}
\item[(i)$'$] $E_{2}(z;\chi_{5})-E_{2}(2z;\chi_{5})=-\dfrac{\tau}{2}E_{2}(z;\chi_{5})$

\item[(ii)$'$] $E^{0}_{2}(z;\chi_{5})+E^{0}_{2}(2z;\chi_{5})=\dfrac{\eta^{4}_{2}\eta^{2}_{5}}{\eta^{2}\eta^{4}_{10}}E^{0}_{2}(2z;\chi_{5})$
\end{itemize}
\end{proof}

By\pageoriginale direct checking (see also \cite{art8-key3}, p. 449) we see that the modular function $\tau$ has a simple zero at $i\infty$ and a simple pole at $0$. On the other hand, $E_{2}(z;\chi_{5})$ is regular at $i\infty$ and in view of the relation $E_{2}(-1/z;\chi_{5})=(z^{2};\sqrt{5})E^{0}_{2}(z/5)$ from (\cite{art8-key5}, p. 819), $E_{2}(z;\chi_{5})$ has a zero at $0$. Consequently, $(\tau/2+1)E_{2}(z;\chi_{5})$ which is regular at $i\infty$ and $0$ is indeed an entire modular form of weight 2 (and nebentypus) for $\Gamma_{0}(10)$. Thus the proof of (i)$'$ reduces to showing that $(\tau/2+1)E_{2}(z;\chi_{5})=E_{2}(2z;\chi_{5})$. In view of Hecke's result (\cite{art8-key5}, Satz 2, p. 811 --- see also p. 953), it is enough to compare the first $\dfrac{[\Gamma(1):\Gamma_{0}(10)]\times 2}{12}(=3)$ coefficients in the Fourier expansions of both sides. That the first three Fourier coefficients agree on both sides is immediate from the following expansions :
\begin{align*}
E_{2}(z;\chi_{5}) &= -\frac{1}{5}+e^{2\pi iz}-e^{4\pi iz}-2e^{6\pi iz}+\cdots\\
(\tau/2+1) &= 1+5e^{2\pi iz}+15e^{4\pi iz}+40e^{6\pi iz}+\cdots\\
(\tau/2+1)E_{2}(z;\chi_{5}) &= -\frac{1}{5}+0\cdot e^{2\pi iz}+e^{4\pi iz}+0\cdot e^{6\pi iz}+\cdots\\
E_{2}(2z;\chi_{5}) &= -\frac{1}{5}+0\cdot e^{2\pi iz}+e^{4\pi iz}+0\cdot e^{6\pi iz}+\cdots
\end{align*}
This proves (i)$'$ and therefore (i). Using (i), we may rewrite the factor $\eta^{4}_{2}\eta^{2}_{5}/(\eta^{2}\eta^{4}_{10})$ on the right hand side of (ii)$'$ as $5+10/\tau$. Hence (ii)$'$ will follow, if we establish
\begin{itemize}
\item[(ii)$''$] $\dfrac{\tau}{4\tau+10}E^{0}_{2}(z;\chi_{5})=E^{0}_{2}(2z;\chi_{5})$.
\end{itemize}
Now $\tau/(4\tau+10)$ is seen to be regular at all the cusps of $\Gamma_{0}(10)$, except those equivalent to $\pm 1/5$ where it has a simple pole. Further, $E^{0}_{2}(z;\chi_{5})$ of Nebentypus $(-2,5,\chi_{5})$ has a zero at $\infty$ and hence at $\pm 1/5$ (equivalent to $\infty$ under $\Gamma_{5}(5)$. Consequently $[\tau/(4\tau+10)]E^{0}_{2}(z;\chi_{5})$ is an entire modular form of weight 2 (and Nebentypus) for $\Gamma_{0}(10)$. From the Fourier expansions
\begin{align*}
\frac{\tau}{4\tau+10} &= e^{2\pi iz}-e^{4\pi iz}+0\cdot e^{6\pi iz}+\cdots\\
E^{0}_{2}(z;\chi_{5}) &= e^{2\pi iz}+e^{4\pi iz}+2\cdot e^{6\pi iz}+\cdots,
\end{align*}
it is clear that the first three Fourier coefficients of $[\tau/(4\tau+10)]E^{0}_{2}(z;\chi_{5})$ coincide with the corresponding coefficients of $E^{0}_{2}(2z;\chi_{5})$. This proves (ii)$''$ by Hecke's theorem above and hence (ii) is proved.

\begin{remark*}
As\pageoriginale a further illustration for the utility of Proposition \ref{art8-prop1}, we give an alternative proof for Ramanujan's identity
\begin{gather*}
x\psi^{3}(x)\psi(x^{5})-5x^{2}\psi(x)\psi^{3}(x^{5})\\
=\frac{x}{1-x^{2}}-\frac{2x^{2}}{1-x^{4}}-\frac{3x^{3}}{1-x^{6}}+\frac{4x^{4}}{1-x^{8}}+\frac{6x^{6}}{1-x^{12}}-\cdots
\end{gather*}
(See \cite{art8-key1}, pp. 45-49 for a ``rather difficult'' proof which uses besides ``results from Section 13'', leading nevertheless to no ``circular reasoning'' etc.). The right hand side of this identity is just
\begin{align*}
\sum\limits^{\infty}_{n=1}\frac{nx^{n}\chi_{5}(n)}{1-x^{2n}} &= \sum\limits^{\infty}_{m=0}\sum\limits^{\infty}_{n=1}n\chi_{5}(n)x^{n(2m+1)}\\
&= E^{\infty}_{2}(z;\chi_{5})-E^{0}_{2}(2z,\chi_{5})\\
&= \frac{1}{5}\frac{\eta^{5}}{\eta_{5}}+\frac{1}{5}\frac{\eta^{5}_{2}}{\eta_{10}}\quad (\text{by~ \cite{art8-key8}, p. 227})\\
&= \frac{\eta^{2}}{5\eta^{2}_{5}\eta_{10}}\left(\eta^{4}_{2}\eta^{2}_{5}-\frac{\eta^{5}\eta_{5}\eta_{10}}{\eta_{2}}\right)\\
&= \eta^{2}\eta_{2}\eta^{3}_{10}/\eta^{2}_{5},\text{~~ by Proposition 1(ii),}
\end{align*}
while the left hand side is precisely $\dfrac{\eta^{6}_{2}\eta^{2}_{10}}{\eta^{3}\eta_{5}}-5\dfrac{\eta^{2}_{2}\eta^{6}_{10}}{\eta\eta^{3}_{5}}=\dfrac{\eta^{2}_{2}\eta^{2}_{10}}{\eta^{3}\eta^{3}_{5}}\times (\eta^{4}_{2}\eta^{2}_{5}-5\eta^{2}\eta^{4}_{10})=\dfrac{\eta^{2}\eta_{2}\eta^{3}_{10}}{\eta^{2}_{5}}$, using Proposition 1(i).
\end{remark*}

\subsection{}\label{art8-sec1.2}
We have gathered here, from Chapter XIX of the Notebooks, many of Ramanujan's identities involving the functions $\varphi(q):=1+2\sum\limits^{\infty}_{n=1}q^{n^{2}}$ or $\psi(q):=q^{-1/8}\eta^{2}(2z)/\eta(z)$ introduced by him in Chapter XVI with $q:=\exp (2\pi iz)$ or various transforms of $\varphi$ and $\psi$. We shall rewrite the identities (with a view to elucidate them) in terms of the normalized Eisenstein series $E_{4}(z)$ of weight 4 for $\Gamma_{0}(1)$, its transforms and the following Eisenstein series of Haupttypus or Nebentypus $(-k,N,\epsilon)$ with $\epsilon$ equal to the trivial character modulo $N$ or the real character $\epsilon(n):=(\frac{n}{N})$ modulo $N$ {\em and} $\epsilon(-1)=(-1)^{k}$ namely
\begin{gather*}
E_{k,N,1}(z)=E_{k,1}(z;\Gamma_{0}(N);\epsilon):=\sum\limits^{\infty}_{n=1}\left(\sum\limits_{1\leq d|n}\epsilon(n/d)d^{k-1}\right)e^{2\pi inz}\\
E_{k,N,z}(z)=E_{k,2}(z;\Gamma_{0}(N);\epsilon):=\gamma_{k}(N)+\sum\limits^{\infty}_{n=1}\left(\sum\limits_{1<d|n}\epsilon(d)d^{k-1}\right)e^{2\pi inz}
\end{gather*}
with\pageoriginale a constant $\gamma_{k}(N)$ that can be determined. We note that with $q=\exp(2\pi iz)$, $\varphi^{2}$ is a modular form of weight 1 with multipliers for the congruence subgroup $\Gamma_{0}(4)$. For dealing with products of transforms of $\psi$, we recall from Honda and Miyawaki (J. Math. Soc. Japan, 26(1974), 362-373) that the power product $\prod\limits^{r}_{j=1}\eta^{n_{j}}_{t_{j}}$ with integral $t_{1},\ldots,t_{r}, n_{1},\ldots,n_{r}$ is a modular form of weight $\frac{1}{2}(n_{1}+\cdots+n_{r})$ for the congruence subgroup $\Gamma_{0}(l.\displaystyle{\mathop{c}\limits_{j}}.m.t_{j})$ if 24 divides both $n_{1}t_{1}+\cdots+j_{r}t_{r}$ and $(l.\displaystyle{\mathop{c}\limits_{j}}m. \ t_{j})\times (n_{1}/t_{1}+\cdots+n_{r}/t_{r})$ Ramanujan's identities referred to may now be rewritten as follows.
\begin{enumerate}
\renewcommand{\labelenumi}{(\theenumi)}
\item $\varphi^{2}(q)=E_{1,2}(z;\Gamma_{0}(4);\epsilon)$

\item $\varphi^{4}(q)=8(E_{2,2}(z;\Gamma_{0}(4);1)-4E_{2,2}(4z;\Gamma_{0}(4);1))$

\item $\varphi^{8}(q)=\frac{1}{15}(E_{4}(z)-2E_{4}(2z)+16E_{4}(4z))$

\item $q\psi^{3}(q)\psi(q^{5})-5q^{2}\psi(q)\psi^{3}(q^{5})=E_{2,2}(z;\Gamma_{0}(5);1)-E_{2,2}(2z;\Gamma_{0}(5);1)$

\item $5\varphi(q)\varphi^{3}(q^{5})-\varphi^{3}(q)\varphi(q^{5})=4(E_{2,5,2}(z)-2E_{2,5,2}(2z)-4E_{2,5,2}(4z))$

\item $25\varphi(q)\varphi^{3}(q^{5})-\varphi^{5}(q)/\varphi(q^{5})=40(E_{2,5,2}(z)-4E_{2,5,2}(4z))$

\item $\psi^{5}(q)/\psi(q^{5})-25q^{2}\psi(q)\psi^{3}(q^{5})=5(E_{2,5,2}(z)-2E_{2,5,2}(2z))$

\item $q\psi^{5}(q)\psi(q^{3})-9q^{2}\psi(q)\psi^{5}(q^{3})=E_{3,3,2}(z)-E_{3,3,2}(2z)$

\item $9\varphi(q)\varphi^{5}(q^{3})-\varphi^{5}(q)\varphi(q^{3})=8(E_{3,3,2}(z)-2E_{3,3,2}(2z)-8E_{3,3,2}(4z))$

\item $\psi^{3}(q)/\psi(q^{3})=E_{1,3,2}(z)-E_{1,3,2}(2z)$

\item $\varphi^{3}(q)/\varphi(q^{3})=6(E_{1,3,2}(z)+2E_{1,3,2}(2z)-2E_{1,3,2}(4z))$

\item $q\psi(q^{2})\psi(q^{6})=E_{1,3,2}(z)-E_{1,3,2}(4z)$

\item $\varphi(q)\varphi(q^{3})=2(E_{1,3,2}(z)+2E_{1,3,2}(4z))$

\item $q\psi^{2}(q)\psi^{2}(q^{3})=E_{2,3,2}(z)-E_{2,3,2}(2z)$

\item $\varphi^{2}(q)\varphi^{2}(q^{3})=4(E_{2,3,2}(z)-2E_{2,3,2}(2z)+4E_{2,3,2}(4z))$

\item $q\psi(q)\psi(q^{7})=E_{1,7,2}(z)-E_{1,7,2}(2z)$

\item $\varphi(q)\varphi(q^{7})=E_{1,7,2}(z)-2E_{1,7,2}(2z)+2E_{1,7,2}(4z)$
\end{enumerate}
The identities are easily proved by noting that both sides are modular forms of weight $k$ for $\Gamma_{0}(M)$ for the appropriate value of $M$ and comparing their first $1+\frac{k}{2}(\Gamma_{0}(1):\Gamma_{0}(M)$ corresponding Fourier coefficients in the light of Hecke's theorem (\cite{art8-key5}, p. 811).

\begin{remark*}
Using Proposition \ref{art8-prop1}, we can prove the relation $\gamma=\frac{1}{\mu}\left(\frac{\mu-1}{\mu-5}\right)^{2}$ for the Hauptmoduls $\gamma:=\eta^{6}_{5}/\eta^{6}$ and $\mu:=\psi^{2}(q)/\psi^{2}(q^{5})$ for $\Gamma_{0}(5)$ and $\Gamma_{0}(10)$ respectively. Substituting this relation between $\gamma$ and $\mu$, we obtain
\begin{align*}
1+22\gamma+125\gamma^{2} &=\frac{1}{\mu^{2}(\mu-5)^{4}}[\mu^{2}(\mu-5)^{4}+22\mu(\mu-1)^{2}(\mu-5)^{2}+125(\mu-1)^{4}]\\
&= \frac{1}{\mu^{2}(\mu-5)^{4}}(\mu^{2}+2\mu+5)^{2}(\mu^{2}-2\mu+5)
\end{align*}\pageoriginale 
\end{remark*}

As a consequence, we obtain Entry 4(i) of Chapter XXI of the Notebooks in the following form :
$$
(\eta^{5}/\eta_{5})(1+22\gamma+125\gamma^{2})^{1/2}=\frac{\psi^{5}(q^{5})}{\psi(q)}(\mu^{2}+2\mu+5)(\mu^{2}-2\mu+5)^{1/2}
$$

We have, as an analogue of assertion (i) of Proposition \ref{art8-prop1}, the identity $\varphi^{2}(q)-\varphi^{2}(q^{5})=4\eta^{2}_{2}\eta_{5}\eta_{20}/\eta\eta_{4}$ which is a consequence of the remarks above. We note further that Entry 5(i) in Chapter XXI is equivalent to the easily proved relations :
$$
(E_{1,7,2}(z))^{2}=\left(1+2\Sigma\left(\dfrac{n}{7}\right)\frac{q^{n}}{1-q^{n}}\right)^{2}=E_{2,7,2}(z)=E_{2}(z)-7E_{2}(7z)
$$
with the usual definition of $E_{2}$ and
$$
(E_{1,7,2}(z))^{3}=\eta^{7}/\eta_{7}+13\eta^{3}\eta^{3}_{7}+49\eta^{7}_{7}/\eta.
$$
One needs, for these, merely to verify the equality of at most the first 3 corresponding Fourier coefficients on both sides, in view of Hecke's theorem again.

Other relations stated by Ramanujan for $\psi^{2}$, $\psi^{4}$, $\psi^{6}$, $\psi^{8}$, $\varphi^{3}/\varphi_{3}$, $\varphi^{3}_{3}/\varphi$, $\psi^{3}/\psi_{3}$, $\psi^{3}_{3}/\psi$, $\eta^{3}/\eta_{3}$, $\eta^{3}_{9}/\eta_{3}$ may be derived in the same manner.

The proofs of identities \eqref{art8-eq4}--\eqref{art8-eq11} as indicated above may be seen to be simpler than those in \cite{art8-key1} (p. 45-56, 12-16). One may also compare the proofs of identities \eqref{art8-eq12}-\eqref{art8-eq13} with those outlined in Hardy's ``Ramanujan'' (pp. 220-222)

\section{Elliptic integrals considered by Ramanujan}
~

\subsection{}\label{art8-sec2.1}
Let us being with the simplest example of elliptic integrals for which Ramanujan (\cite{art8-key11}, p. \mycirc{67} + 1) has written explicit primitives :
\begin{equation}
\int\limits^{x}_{e^{-2\pi}}\sqrt{Q}\dfrac{dx}{x}=\log \frac{Q^{3/2}-R}{Q^{3/2}+R}.\label{art8-eq13}
\end{equation}
Applying Ramanujan's ``very useful substitution'' $Z=R^{2}/Q^{3}$, we have $\dfrac{dZ}{Z}=2\dfrac{dR}{R}-3\dfrac{dQ}{Q}$ and hence \eqref{art8-eq1} leads to $\dfrac{1}{Z}x\dfrac{dZ}{dx}=\dfrac{R^{2}-Q^{3}}{QR}$. Now 
%page 128
