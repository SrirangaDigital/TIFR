
\title{ON THE PROOF OF ANDREWS' $q$-DYSON CONJECTURE}
\markright{ON THE PROOF OF ANDREWS' $q$-DYSON CONJECTURE}    

\author{By~  D. M. Bressoud\footnote{Partially supported by N.S.F. grant no. DMS.-8521580}}

\date{}
\maketitle

\setcounter{pageoriginal}{30} 


\textsc{This is a}\pageoriginale breif sketch of work done by Doron Zeilberger, Ian Goulden and myself in late 1983 and early 1984 which settled in the affirmative a conjecture made by George Andrews \cite{art4-key1} as well as more detailed conjectures made by Kevin Kadell \cite{art4-key12}.

The problem has its origins in the evaluation of a definite integral which arose in a physical problem \cite{art4-key5}, its solution has given evaluations for other definite integrals arising in physics \cite{art4-key4}. The original integral was discovered by Freeman Dyson \cite{art4-key5}:
\begin{equation}
I (n,z) = (2\pi)^{-n} \int\limits^{2\pi}_0 \ldots \int\limits^{2\pi}_0 |\Delta_n(e^{i\theta})|^{2z} d \theta_1 \ldots d \theta_n, \label{art4-eq1}
\end{equation}
where 
$$
\Delta_n(e^{i\theta}) = \Pi (e^{i\theta_j} - e^{i\theta_k}), \; 1 \leqslant j < k \leqslant n.
$$
Dyson conjectured that 
\begin{equation}
I (n,z) = \frac{\Gamma (nz+1)}{\Gamma^n (z+1)}, \label{art4-eq2}
\end{equation}
a conjecture which was simultaneously and independently proved by Gunson \cite{art4-key7} and Wilson \cite{art4-key23}.

It is sufficient to prove that conjecture for positive integral integral $z$. In this case, we can use the following equality:
\begin{gather}
|\Delta_n (e^{i\theta})|^2 = \Pi (e^{i\theta_j} - e^{i\theta_k}) (e^{-i\theta_j} - e^{-i\theta_k}).\label{art4-eq3}\\
= \Pi (1-e^{i(\theta_j - \theta_k)}) (1-e^{i(\theta_k -\theta_j)}).\notag
\end{gather}

If we set $x_j =e^{i\theta_j}$, then the integral picks out the constant term in a polynomial in $x_1, x^{-1}_1, \ldots, x_n, x^{-1}_n$. Given a monomial, $M$, in the $x_i$'s, let [M] denote the coefficient of $M$ in the succeeding polynomial. Let $x^0$ denote the monomial in which each $x_i$ appears to the power 0. Equation \eqref{art4-eq2} for $z \in \bbN$ can be restated as
\begin{equation}
[x^0] \Pi (1-x_j/x_k)^z (1-x_k/x_j)^z = \frac{(nz)!}{(z!)^n}, \; 1 \leqslant j < k \leqslant n. \label{art4-eq4}
\end{equation}\pageoriginale 

Dyson discovered that more was probably true, and actually stated his conjecture in the following form:
\begin{gather}
[x^0] \Pi (1-x_j/x_k)^{a_k} (1-x_k/x_j)^{a_j} \label{art4-eq5}\\
= \frac{(a_1+ \ldots + a_n)}{a_1 !\ldots a_n!} \notag
\end{gather}

In 1975, Andrews \cite{art4-key1} noted that equation \eqref{art4-eq5} seemed to have a nice generalization in which the product $(1-x)^a$ could be replaced by 
$$
(x)_a = (1-x) (1-xq) (1-xq^2) \ldots (1-xq^{a-1})
$$
Specifically, Andrews conjectured the following:
\begin{gather}
[x^0] \Pi (x_j/x_k)_{a_k} (qx_k/x_j)_{a_j}, \; 1 \leqslant j < k \leqslant n,  \label{art4-eq6}\\
=\frac{(q)_{a_1+ \ldots + a_n}}{(q)_{a_1} \ldots (q)_{q_n}} . \notag
\end{gather}

On reason for the interest in equations \eqref{art4-eq5} and \eqref{art4-eq6} is the intractability of the blunt approach. If one expands the binomials in equations \eqref{art4-eq5}, the constant term is a simple summation when $n=3$, and Dyson's conjecture is the classical identity:
\begin{gather}
\sum\limits_i (-1)^i \begin{pmatrix}
a_1 +a_2\\i
\end{pmatrix} \begin{pmatrix}
a_2+ a_3\\
i - a_2 + a_3
\end{pmatrix} 
\begin{pmatrix}
a_1 + a_3\\
i+a_1 -a_2
\end{pmatrix} \label{art4-eq7}\\
= (-1)^{a_2} 
\begin{pmatrix}
a_1 + a_2 + a_3\\
a_1, a_2, a_3
\end{pmatrix}. \notag
\end{gather}

For larger $n$, however, the constant term is an $\left(\begin{smallmatrix}
n-1\\2\end{smallmatrix} \right)$-fold summation, and virtually nothing is known about such non-trivial multiple summations.

The same situation applies to Andrews' conjecture, except that instead of multiple hypergeometric series we get multiple basic hypergeometric series. 

To understand how equation \eqref{art4-eq6} was first proved, one must understand an ingenious combinatorial  proof of Dyson's equation \eqref{art4-eq5} which was found by Zeilberger \cite{art4-key24} a few years earlier. Equation \eqref{art4-eq5} is equivalent to 
\begin{gather}
[(x^{a_1}_1 \ldots x^{a_n}_n)^{n-1}] \Pi (x_j - x_k)^{a_j+a_k} \label{art4-eq8} \\
= (-1)^{a_2 + 2a_3 + \ldots  + (n-1) a_n} \frac{(a_1+ \ldots + a_n)!}{a_1 !\ldots a_n!}. \notag
\end{gather}\pageoriginale 
We can formally expand the product of binomials in equation \eqref{art4-eq8}:
\begin{equation}
\Pi (x_j - x_k)^{a_j+ a_k} = \sum\limits_{T \in \sT^\ast} (-1)^{u(T)} x_1^{w_1 (T)} \ldots x_n^{w_n(T)} \label{art4-eq9}
\end{equation}
where $\sT^\ast$ is the set of ``multi-tournaments'' in which each pair of players, say $j$ and $k$, meet a total of $a_j + a_k $ times and the ``winner'' of each game is recorded. The exponent $w_i (T)$ is the number of games won by player $i$, and $u(T)$ records the number of ``upsets'' $: k > j$ and $k$ beats $j$.

If we let $\sT \subseteq \sT^\ast$ be the subset of multi-tournaments in which each player $j$ wins $(n-1)a_j$ games,  then equation \eqref{art4-eq8} can be restated as:
\begin{equation}
\sum\limits_{T \in \sT} (-1)^{u(T)} = (-1)^{a_2 + \ldots + (n-1) a_n} 
\begin{pmatrix}
a_1 + \ldots + a_n\\
a_1, \ldots, a_n \end{pmatrix}. \label{art4-eq10}
\end{equation}

The right side of equation \eqref{art4-eq10} involves the multinomial coefficient which counts the number of ``words'' which can be constructed with $a_1 1's$, $a_2 2$' $s, \ldots, a_n n's$. Each such word corresponds to a multi-tourna\-ment in a natural way. Given $j$ and $k$, remove the subword of length $a_j + a_k$ in the letters $j$ and $k$. The winners in order are read off left to right. 

As an example, if $n = 4$, $a_1 =a_2 = a_3 =a_4 =2$, the word 32114243 corresponds to the multi-tournament:
\begin{align*}
2112\\
3113\\
1144\\
3223\\
2424\\
3443
\end{align*}
We observe that the number of upsets is always $a_2 +2a_3 + \ldots + (n-1)a_n$.

If we let $\sT' \subseteq \sF$ be the subset of multi-tournaments which do not correspond to a word, then equation \eqref{art4-eq10} can be further simplified to
\begin{equation}
\sum\limits_{T \in \sT'} (-1)^{\mu(T)} = 0. \label{art4-eq11}
\end{equation}

Zeilberger showed how to prove this by establishing a bijection between the set of $T \in \sT'$ for which $u(T)$ is even and the set of $T \in \sT'$ for which $u(T)$ is odd.\pageoriginale We shall demonstrate the bijection with an example. Let $T$ be 
\begin{align*}
2111\\
3133\\
1144\\
2232\\
2424\\
3443
\end{align*}

Inspection immediately shows us that while this element is in $\sT$, it cannot be the two letter subwords of a single word. Nevertheless, we shall attempt to construct a word to which this multi-tournament corresponds.

The leading entries of each row define a tournement:

2 beats 1, 3 beats 1, 1 beats 4, etc. Schematically, this tournament is given by:
$$
\xymatrix{
& 1\ar[d] & \\
& 4 & \\
2\ar[uur] \ar[ur]\ar[rr] & & 3 \ar[uul]\ar[ul]
}
$$
We call a tournament ``transitive'' if it contains no cycles, ``non-transi\-tive'' otherwise. If our multi-tournament arose from a single word, then this tournament is transitive and the player beating everyone else is the first letter of the word. Since our tournament is transitive, it is possible at this stage that it comes from a single word. We record the first letter : 2, and modify the tournement by looking at the next outcome of the games of player 2: 1 beats 2, 2 beats 3, 4 beats 2.
\begin{gather*}
\text{ The tournament becomes :}\\
\xymatrix{
& 1\ar[d] \ar[ddl] & \\
& 4\ar[dl] & \\
2\ar[rr] & & 3 \ar[uul]\ar[ul]
}
\end{gather*}

Our tournament is now non-transitive which will eventually happen if and only if $T$ is in $\sT'$.

Every non-transitive tournament contains a 3-cycle and reversing the arrows in a 3-cycle will change the parity of the number of upsets in the tournament. We have two 3-cycles in this tournament. which one we choose\pageoriginale to reverse is significant. 

If we reverse $2 \to 3 \to 4 \to 2$ and then restore the first letter, 2, we get the multi-tournament
\begin{align*}
&2111\\
&3133\\
&1144\\
&2332\\
&2224\\
&4443
\end{align*}
But the leading entries of this multi-tournament give us a non-transitive tournament:
$$
\xymatrix{
& 1\ar[d] & \\
& 4 \ar[dr] & \\
2\ar[uur] \ar[ur]\ar[rr] & & 3 \ar[uul]
}
$$
An iteration of our procedure would not take us back to the original multitournament.

If no letters of the word have been recorded, then it doesn't matter which 3-cycle we reverse as long as we are consistent. If at least one letter has been recorded, then we are in a peculiar situation. Let $v_1$ be the last letter recorded. Since we have only changed the arrows connected to $v_1$, all cycles of the non-transitive tournament include vertex $v_1$.

Let the remaining vertices be labelled $v_2, v_3, \ldots, v_n$ where $v_2$ beats $v_3$ beats $\ldots$ beats $v_n$, and choose the smallest $i$ for which $v_1$ beats $v_i$ and $v_{i+1}$ beats $v_1$. It is the 3-cycle $v_1 \to v_i \to v_{i+1} \to v_1$ that we reverse.

it is exactly this procedure that was used to prove Andrews' conjecture, except that the details are more complicated because the parameter $q$ introduces an additional weight on the multi-tournaments. 

The proof first demonstrates that
\begin{gather}
[x^0] \Pi (x_j/x_k)_{a_k} (qx_k/x_j)_{a_j}\\
= (-1)^{a_2 + \ldots + (n-1) a_n} \sum\limits_{t \in \sF} (-1)^{\mu(T)} q^{wt (T)},
\label{art4-eq12}
\end{gather}
where $wt (T)$ is the sum of the ``Major Indices'' of all the two letter words in the multi-tournament. The Major Index of a word is the sum of the number\pageoriginale of letters to the left of each ``descent'' in the word. Thus
$$
32114243
$$
has four descents : (32, 21, 42, 43) , and its major index is $1+2+5+7=15$. 

On the other hand, if we sum the major indices of the two letter subwords of 32114243, we get $1+1+0+1+2+3=8$. This sum of Major Indices is called the $Z$-statistic, denoted $Z(T)$. The second part of the proof involves showing that the sum of $q^{Z(T)}$ over all multi-tournaments corresponding to a single word is equal to 
$$
\frac{(q)_{a_1+\ldots + a_n}}{(q)_{a_1}\ldots (q)_{a_n}}
$$

Equation \eqref{art4-eq6} now reduces to verifying that 
\begin{equation}
\sum\limits_{T \in \sT'} (-1)^{u(T)} q^{wt(T)} =0. \label{art4-eq13}
\end{equation}
The bijection given above does not preserve weights. The last and most elaborate part of the proof involves finding and verifying a bijection which does.

It is curious that this combinatorial approach is still the only known proof of equation \eqref{art4-eq6}.

Goulden and I\cite{art4-key3} generalized this proof of yield a more useful identity. In the following we let $A$ be an arbitrary set  of unordered pairs $(j,k)$, $1 \leqslant j \neq k \leqslant n$, $\chi (S)$ is 1 if $S$ is true, 0 otherwise, $\fS_A$ is the set of permutations of $\{1,\ldots, n\}$ for which $j>i$ and $\sigma^{-1} (i)$ implies $(i,j) \not\in A$, and $wt (\sigma)$ is the sum over all $j$ of $a_j$ times the number of $k < j$ for which $\sigma^{-1}(j)< \sigma^{-1}(k)$.
\begin{gather}
[x^0] \Pi (x_j/x_k)_{a_j} (qx_k / x_j)_{a_k - \chi ((j,k) \in A)}\\
= \frac{(q)_{a_1+ \ldots + a_n}}{(q)_{a_1} \ldots (q)_{a_n}} \sum\limits_{\sigma \in \fS_A} q^{wt(\sigma)} \Pi_j \frac{1-q^{a_{\sigma(j)}}}{1-q^{a_{\sigma(1)}+ \ldots + a_{\sigma(j)}}}
\label{art4-eq14}
\end{gather}

This identity implies several conjectures of Kadell \cite{art4-key12} and has had applications in studying the characters of $SL (n,\bbC)\cite{art4-key21}$ and in evaluating definite integrals arising in statistical mechanics \cite{art4-key4}.

The theorems first conjectured by Dyson and Andrews are only the tip of the iceberg of a very extensive theory. These identities are related to the Vandermonde determinant formula which is Weyl's denominator formula for the root system $A_n$. Macdonald \cite{art4-key15} conjectured the appropriate generalizations\pageoriginale to arbitrary root systems and he and W.G. Morris \cite{art4-key16} gave conjectures and some proofs for the basic analogs.

Macdonald's conjecture for the root system $BC_n$ was discovered to be equivalent to a multi-dimensional beta integral evaluation of Selberg \cite{art4-key18, art4-key19}. A basic analog of this was conjectured by Askey \cite{art4-key2}. Habsieger \cite{art4-key8} and Kadell \cite{art4-key13} independently proved Askey's conjecture and then Habsieger \cite{art4-key9} and Zeilberger \cite{art4-key25} showed that this integral evaluation implied some of Morris' conjectures.

Most recently, Kadell \cite{art4-key14} has proved Macdonal's conjecture for the basic analog of the $BC_n$ conjecture, Garvan \cite{art4-key6} has done the same for $F_4$, and E.M. Opdam \cite{art4-key17} has proved the original Macdonald conjecture for arbitrary root systems. Only the basic analogs for the special root systems $E_6$, $E_7$ and $E_8$ are unproven at the moment.

Stembridge \cite{art4-key22} has found a strikingly simple proof of Andrews' conjecture in the case where the parameters are equal. He has also found formulas for some of the non-constant terms \cite{art4-key21}. Connections with representation theory can be found in an article by Stanley \cite{art4-key20}. Hanlon has pursued the connections between these identities and cyclic homology \cite{art4-key10, art4-key11}.


\begin{thebibliography}{99}
\bibitem{art4-key1} \textsc{G. E. Andrews} : Problems and prospects for basic hypergeometric functions, in \textit{Theory and Application of Special Functions,} ed. R. Askey, Academic Press, New York, 1975, 191-224.

\bibitem{art4-key2} \textsc{R. A. Askey} : Some basic hypergeometric extensions of integrals of Selberg and Andrews, \textit{SIAM J. Math. Anal.} 11 (1980), 938-951.

\bibitem{art4-key3} \textsc{D. M. Bressoud} and \textsc{I. P. Goulden} : Constant term identities extending the $q$-Dyson Theorem, \textit{Trans. Amer. Math. Soc.} 291 (1985), 203-228.

\bibitem{art4-key4} \textsc{D. M. Bressoud}\pageoriginale and \textsc{I. P. Goulden} : The generalized plasma in one dimension : evaluation of a partition function, \textit{Commun. Math. Phys.} 110(1987), 287-291.

\bibitem{art4-key5} \textsc{F. J. Dyson} : Statistical theory of the energy levels of complex systems, \textit{J. Math. Physics} 3(1962), 140-156.

\bibitem{art4-key6} \textsc{F. Garvan} : Personal communication.

\bibitem{art4-key7} \textsc{J. Gunson} : Proof of a conjecture by Dyson in the statistical theory of energy levels, \textit{J. Math. Physics} 3 (1962) , 752-753.

\bibitem{art4-key8} \textsc{L. Habsieger} : Une $q$-integrale de Selberg-Askey, \textit{SIAM J. Math, Anal.,} to appear.

\bibitem{art4-key9} \textsc{L. Habsieger} : La $q$-conjecture de Macdonald-Morris pour $G_2$, \textit{C. R. Acad. Sc. Paris} 302 (1986), 615-618.

\bibitem{art4-key10} \textsc{P. Hanlon} : The proof of a limiting case of Macdonald's root system conjecture, \textit{Proc. London Math. Soc.} 49 (1984), 170-182.

\bibitem{art4-key11} \textsc{P. Hanlon} : Cyclic homology and the Macdonald conjectures, \textit{Invent. Math.} 86 (1986), 131-159.

\bibitem{art4-key12} \textsc{K. Kadell} : Andrews' $q$-Dyson conjecture : $n=4$, \textit{Trans. Amer. Math. Soc.} 290 (1985), 127-144.

\bibitem{art4-key13} \textsc{K. Kadell} : A proof of Askey's conjectured $q$-analog of Selberg's integral and a conjecture of Morris, \textit{SIAM J. Math. Anal.,}  to appear. 

\bibitem{art4-key14} \textsc{K. Kadell} : Personal communication.

\bibitem{art4-key15} \textsc{I. G. Macdonald} :  Some conjectures for root systems. \textit{SIAM J.  Math. Anal.} 13 (1982), 988-1007.

\bibitem{art4-key16} \textsc{W. G. Morris} : \textit{Constant term identities for finite and infinite root systems,} Ph. D. thesis, University of Wisconsin, Madison, 1982.

\bibitem{art4-key17} \textsc{E. M. Opdam} : Doctoral thesis, University of Leiden, Netherlands.

\bibitem{art4-key18} \textsc{A. Selberg} : Uber einen Satz von A. Gelfond, \textit{Arch. Math. Naturvid.} 44 (1941), 159-170.

\bibitem{art4-key19} \textsc{A. Selberg} : Bemerkninger om et multiplelt integral, \textit{Norsk Mat. Tidsskr.} 26, (1944), 71-78.

\bibitem{art4-key20} \textsc{R. Stanley} :\pageoriginale The $q$-Dyson conjecture, generalized exponents and the internal product of Schur functions, \textit{in ``Combinatorics and Algebra''} ed. Curtis Greene, Amer. Math. Soc., Providence, 1984, 81-94.

\bibitem{art4-key21} \textsc{J. Stembridge} : First layer formula for the characters of $SL (n, C)$, \textit{Trans, Amer. Math. Soc.} 299 (1987), 319-350.

\bibitem{art4-key22} \textsc{J. Stembridge} : A short proof of Macdonald's conjecture for the root systems of type A. Preprint.

\bibitem{art4-key23} \textsc{K. Wilson} : Proof of a conjecture by Dyson, \textit{J. Math. Physics} 3 (1962) 1040-1043.

\bibitem{art4-key24} \textsc{D. Zeilberger} : A combinatorial proof of Dyson's conjecture, \textit{Discrete Math.} 41 (1982), 988-1007.

\bibitem{art4-key25} \textsc{D. Zeilberger} : A proof of the $G_2$ case of Macdonald's root system -- Dyson conjecture, \textit{SIAM J. Math Anal.} 18 (1987), 880-883.

\bibitem{art4-key26} \textsc{D. Zeilberger} and \textsc{D. Bressoud} : A proof of Andrews' $q$-Dyson conjecture, \textit{Discrete Math.} 54 (1985), 201-224.
\end{thebibliography}

\medskip
\noindent
{\small Penn State University,}


\noindent
{\small University Park, PA 16802}

