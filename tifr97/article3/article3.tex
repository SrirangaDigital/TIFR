
\title{RAMANUJAN'S FORMULAS FOR EISENSTEIN SERIES}
\markright{RAMANUJAN'S FORMULAS FOR EISENSTEIN SERIES}

\author{By~  Bruce C. Berndt\footnote{Research partially supported by a grant from the Vaughn Foundation.}}

\date{}
\maketitle

\setcounter{pageoriginal}{22} 

\noindent
\textsc{As is customary},\pageoriginale $\bbN$ denotes the set of positive integers, $\bbZ$ denotes the ring of rational integers, $\sH= \left\{\tau : \Iim \tau > 0 right\}$, and 
$$
\Gamma\_0 (n) = \left\{
\begin{pmatrix}
a & b\\
c & d
\end{pmatrix}: a, b, c, d \in \bbZ, \; ad - bc = 1, \; c \equiv 0 (\mod n)
\right\},
$$
where $n \in \bbN$. If $n=1$, $\Gamma_0 (1)$ is the full modular group $\Gamma (1)$. 

Let
$$
E_2 (\tau ) =1 -24 \sum\limits^\infty_{k=1} \frac{kq^{2k}}{1-q^{2k}}
$$
and 
$$
F_n(\tau) = E_2 (\tau)  -n E_2 (n \tau),
$$
where $q =e^{\pi i \tau}$, $\tau \in \sH$, and $n \in \bbN$. Although $E_2 (\tau)$ is not a modular form, it can be easily shown that $F_n (\tau)$ is a modular form of weight 2 and trivial multiplier system on $\Gamma_0(n)$.

In a very famous paper \cite{art3-key8}, [10, pp. 23-39], Ramanujan gave formulas for $F_n$ when $n = 2, 3,4,5,7,11,15,17,19,23,31,35$. However, no proofs are indicated. Furthermore, in Chapter 21 of his second notebook \cite{art3-key9}, Ramanujan offers, without proofs, formulas for $F_n$ when $n = 3,5,7,9,11, 15, 17, 19, 23, 25, 31, 35$. In contrast to \cite{art3-key8} where only one formula is given for each value of $n$, in \cite{art3-key9} several formulas are stated for most values of $n$.

Part of Ramanujan's motivation in calculating $F_n$ arose from its appearance in certain approximations to $\pi$ found by Ramanujan \cite{art3-key8}. J.M. and P.B. Borwein

%%%% 31 page 

\begin{thebibliography}{99}
\bibitem{art3-key1}
\end{thebibliography}
