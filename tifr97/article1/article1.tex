
\title{VARIANTS OF CLAUSEN'S FORMULA FOR THE SQUARE OF A SPECIAL$_2$F$_1$}
\markright{VARIANTS OF CLAUSEN'S FORMULA FOR THE SQUARE OF A SPECIAL$_2$F$_1$}

\author{By~ RICHARD ASKEY\footnote{Supported in part by an NSF grant, in part by a sabbatical leave from the University of Wisconsin, and in part by funds the Graduate School of the University of Wisconsin.}}
\markboth{RICHARD ASKEY}{VARIANTS OF CLAUSEN'S FORMULA FOR THE SQUARE OF A SPECIAL$_2$F$_1$}

\date{}
\maketitle

\section{Introduction}\label{art9-sec1}
One\pageoriginale of the most striking series Ramanujan \cite{art1-key10} found is 
\begin{equation}
\frac{9801}{2\pi\sqrt{2}} = \sum\limits^\infty_{n=0} [1103+26390n] \frac{(4n)!}{[n!]^4 (4.99)^{4n}}.\label{art1-eq1.1}
\end{equation}
The first proofs of \ref{art1-eq1.1} have been given recently by Jonathan and Peter Borwein \cite{art1-key3} and by David and Gregory Chudnovsky \cite{art1-key5}. They have also found other identities of a similar nature, \cite{art1-key4}, \cite{art1-key5}. As they remark, Clausen's identity \cite{art1-key6}
\begin{equation}
\left[{}_2 F_1 \left(
\begin{aligned}
& a, b\\
& a + b + \frac{1}{2}\end{aligned};x
\right) \right]^2 = {}_3 F_2 \left(
\begin{aligned}
& 2a, 2b, a + b \\
&  a+ b+ \frac{1}{2}, \; 2a+ 2 b 
\end{aligned} ;x
 \right)
\label{art1-eq1.2}
\end{equation}
plays a central role in the derivation of \eqref{art1-eq1.1}. Here 
\begin{equation}
{}_p F_p \left(
\begin{aligned}
a_1, \ldots, a_p\\
b_1, \ldots, b_q
\end{aligned}; x
\right) = \sum\limits^\infty_{n=0} \frac{(a_1)_n \ldots (a_p)_n x^n}{(b_1)_n \ldots (b_q)_n n!}
\label{art1-eq1.3}
\end{equation}
with 
\begin{equation}
(a)_n = \Gamma (n+a) / \Gamma (a). \label{art1-eq1.4}
\end{equation}
Ramanujan \cite{art1-key11} stated an extension of Clausen's formula
\begin{gather}
{}_2 F_1 \left(
\begin{aligned}
& a, b\\
& ~ c
\end{aligned};
\frac{1 - \sqrt{1-x}}{2} \right) {}_2 F_1 
\left(
\begin{aligned}
& a, b\\
& ~ d
\end{aligned} ; 
\frac{1-\sqrt{1-x}}{2}
 \right) \label{art1-eq1.5}\\
= {}_4 F_3 
\left(
\begin{aligned}
&a, b, (a+b) / 2, (c+ d) / 2\\
& \qquad c, d, a + b
\end{aligned};x \right)
\notag
\end{gather}
when $c+ d = a+ b +1$. When $c =d$ and the quadratic transformation
$$
{}_2 F_1 \left(
\begin{aligned}
& \quad a, b\\
& (a+b+1)/2 
\end{aligned}; \frac{1-\sqrt{1-x}}{2}
 \right) = {}_2 F_1 
\left(
\begin{aligned}
& a/2, b/2\\
& (a+b+1)/2
\end{aligned}; x
\right)
$$
is used, the result is \eqref{art9-eq1.2}. The first published proof of \eqref{art9-eq1.5} is due to Bailey \cite{art1-key1}.

David\pageoriginale and Gregory Chudnovsky have been asking me if there are other results like Clausen's formula, where the square of a ${}_2 F_1$ is represented as a generalized hypergeometric series. There are other instances, ans one will be given explicitly. The method of deriving it is probably similar to Ramanujan's method of deriving Clausen's formula. As a warm up, here is how I think Ramanujan derived \eqref{art9-eq1.2}.

There are two chapters in Ramanujan's Second Notebook devoted to hypergeometric series. The first formula in this first of these two chapters is the sum of the 2-balanced very well posited ${}_7 F_6$. This is a fundamental formula, as Ramanujan knew, since he started with it. This sum is
\begin{align}
& {}_7 F_6 \left(
\begin{aligned}
& a, 1 + (a/2), b, c, d, e, -n\\
& a/2, \;a+1-b, \;a+1-c, \;a+1-d, \;a+1-e, \;a + 1 -n
\end{aligned};1
\right) \label{art1-eq1.6}\\
& =
\frac{(a+1)_n (a+1-b-c)_n (a+1-b-d)_n(a+1-c-d)_n}{(a+1-b)_n (a+1-c)_n (a+1-d)_n (a+1-b-c-d)_n} \notag
\end{align}
and
\begin{equation}
e = 2 a + 1 + n - b - c -d ,  \label{art1-eq1.7}
\end{equation}

The phrases \textit{very well poised} and 2-\textit{balanced} are defined as follows. A series
\begin{equation}
{}_{p+1} F_p \left(
\begin{aligned}
&a_0, a_1, \ldots, a_p\\
& b_1, \ldots, b_p
\end{aligned} ;x
\right)
\label{art1-eq1.8}
\end{equation}
is said to be $k$-\textit{balanced} if $x =1$, if one of the numerator parameters is a negative integer, and if 
$$
k + \sum\limits^{p}_{j=0} a_j = \sum\limits^{p}_{j=1} b_j.
$$
The series \ref{art1-eq1.8} is said to be \textit{well poised} if $a_0 +1 = a_1 + b_1 = \ldots = a_{p} + b_p$. It is \textit{very well poised} if it is well poised and if $a_1 = b_1 + 1$. Observe that the condition \eqref{art1-eq1.7} comes from the series being 2-balanced.

Dougall \cite{art1-key7} published the first derivation of \eqref{art1-eq1.6}. Ramanujan's discovery was probably later, but not much later. 

To derive Clausen's formula, first consider
\begin{align}
&\left[{}_2 F_1 \left(
\begin{aligned}
& a, b\\
& ~ c
\end{aligned}; x\right) \right]^2 = \sum\limits^\infty_{n=0} x^n \sum\limits^n_{k=0} \frac{(a)_k (b)_k (a)_{n-k} (b)_{n-k}}{(c)_k k! (c)_{n-k} (n-k)!}
\label{art1-eq1.9}\\
& =\sum\limits^\infty_{n-0} \frac{(a)_n (b)_n}{(c)_n n!} {}_4 F_3
\left(\begin{aligned}
&-n, a, b, 1 - n - c \\
& 1 - n - a \; 1 - n - b , \; c
\end{aligned}; 1
 \right) x^n. \notag
\end{align}\pageoriginale 
The ${}_4 F_3$ series that multiplies $x^n$ in the expression in \eqref{art1-eq1.9} is well poised. While a well poised ${}_3 F_2$ at $x =1$ can be summed, and a very well poised ${}_5 F_4$ can be summed when $x =1$, a general well poised ${}_4 F_3$ at $x =1$ cannot be summed. However when the series is 2-balanced it can be summed. To see this, first reduce the very well poised ${}_7 F_6$ to a well poised ${}_4 F_3$. This is done by setting $d = a /2$, $c= (a+1)/2$. Then \eqref{art1-eq1.6} becomes 
\begin{align}
&{}_4 F_3 \left(
\begin{aligned}
&a, b, e, -n\\
& a+1 - b, a+1-e, a + 1 + n
\end{aligned};1
 \right)
\label{art1-eq1.10}\\
& = \frac{(a+1)_n ((a+1-2b)/2)_n ((a+2-2b)/2)_n (1/2)_n}{(a+1-b)_n ((a+1)/2)_n ((a+2)/2)_n ((1-2b)/2)_n}\notag\\
& = \frac{(a+1)_n (a+1-2b)_{2n} (1/2)_n}{(a+1-b)_n (a+1)_{2n} ((1-2b)/2)_n}\notag\\
& = \frac{(a+1-2b)_{2n} (1/2)_n}{(a+1-b)_n (a+n+1)_n ((1-2b)/2)_n}\notag\\
& = \frac{\Gamma (a+1-2b+2n)\Gamma (n+1/2) \Gamma (a+1-b) \Gamma (a+n+1) \Gamma( 1/2-b)}{\Gamma (a+1-2b)\Gamma (1/2) \Gamma (a+1-b + n) \Gamma (a+2 n + 1) \Gamma ((1/2) - b + n)} \notag
\end{align}
This last expression can be used when $a= -k$. Then
\begin{gather*}
{}_4 F_3 \left(\begin{aligned}& -k, b, e, -n\\ &1- b-k, 1 -e-k, 1 + n - k \end{aligned} ; 1\right)\\
 = \frac{\Gamma (1-k - 2 b + 2 n ) \Gamma (1/2 + n)  \Gamma ((1/2)-b) \Gamma (1-b-k) \Gamma (1+ n -k)}{\Gamma (1-k - 2 b) \Gamma (1-k + 2n) \Gamma (1/2) \Gamma (1-k-b + n) \Gamma ((1/2) - b + n)} .
\end{gather*}
holds for $n = k$, $k+1, \ldots$, and is a rational function of $n$, so it holds when $n$ is replaced by continuous parameter $-a$. The result is 
\begin{equation}
{}_4 F_3 \left(\begin{aligned}& -k, a, b, e \\ & 1 - a - k, \; 1 -b-k, \; 1 - e -k
\end{aligned}; 1\right)= \frac{(2a)_k (2b)_k (a+b)_k}{(a)_k (b)_k (2a+2b)_k} \label{art1-eq1.11}
\end{equation}
after simplification. Recall that this series is 2-balanced, so $e = -a -b -k + (1/2)$.

One can take $a= -k$ in \eqref{art1-eq1.6} and then remove the restriction that one of the other parameters is a negative integer. However setting $c = (1-k)/2$, $d = -k/2$ to obtain the ${}_4 F_3$ leads to an indeterminate form, so it is better to reduce\pageoriginale to a ${}_4 F_3$ initially before letting $a \to -k$.

Both \eqref{art1-eq1.10} and \eqref{art1-eq1.11} are 2-balanced well poised series, but they are different in that different parameters are used to terminate the series. When \eqref{art1-eq1.11} is used in \eqref{art1-eq1.9}, he result is Clausen's formula \eqref{art1-eq1.2}.

\section{The four balanced very well poised ${}_7$F$_6$}\label{art1-sec2}
To find another formula like Clausen's identity, we can look for another well poised series that can be summed. The obvious candidate is the 4-balanced very well poised ${}_7$F$_6$. There are two natural ways to sum this series. One is an easy consequence of \eqref{art1-eq1.6}, so it is a derivation Ramanujan could have easily given. We start with it. Set
\setcounter{equation}{0}
\begin{equation}
f_k (b) = \frac{(b)_k(e)_k}{(a+1-b)_k(a+1-e)_k}
\label{art1-eq2.1}
\end{equation}
and use the 2-balanced condition
\begin{equation}
e= 2 a + 1 + n -b - c - d. 
\label{art1-eq2.2}
\end{equation}
A routine calculation gives 
\begin{gather*}
b (a-b) f_k (b+1) - (e-1) (a+1-e)f_k (b)\\
= \frac{(b)_k(e-1)_k}{(a+1-b)_k (a+2 -e)_k} [b(a-b) -(e-1) (a+1-e)].
\end{gather*}
Observe that the last factor is 
\begin{align*}
& \qquad b (a-b) - (2 a+ n - b - c -d) (b+c + d- n -a)\\
& = (n+ \frac{3a}{2} - b - c - d + \frac{a}{2}) (n + \frac{3a}{2} - b - c -d -\frac{a}{2}) - (b - \frac{a}{2}- \frac{a}{2}) (b- \frac{a}{2} + \frac{a}{2})\\
& = (n+ \frac{3a}{2} - b - c - d)^2 - (b- \frac{a}{2})^2 = (n+2 a -  2 b - c -d) (n+a - c -d);
\end{align*}
so 
\begin{align*}
& (n+2 a - 2 b - c -d) (n+a - c -d) \times \\
& \quad \times  {}_7 F_6 \left(\begin{aligned} 
&a, \frac{a}{2} + 1, b, c, d, e -1, -n\\
&\frac{a}{2}, a+1 - b, a + 1-c, a+1 -d, a+2 - e, a +1 + n
\end{aligned};1 \right)\\
& = b (a-b) {}_7 F_6
\left(\begin{aligned}
& a, \frac{a}{2} + 1, b+1, c, d, e-1, -n\\
& \frac{a}{2}, a-b, a+1 - c, a + 1-d, a+2 -e, a+ 1 + n
\end{aligned} ; 1\right)\\
& - (e-1) (a+1-e) \times \\
& \qquad \times {}_7 F_6 
\left(\begin{aligned}
& a, \frac{a}{2} + 1, b, c, d, e, -n\\
& \frac{a}{2}, a+ 1- b, a+1 - c, a +1 - d, a+1 - e, a + 1 + n
\end{aligned} ;1\right)\\
& = \frac{b(a-b+n) (a+1)_n (a-b-c)_n (a-b-d)_n (a+1-c-d)_n}{(a+1-b)_n (a+1-c)_n (a+1-d)_n (a-b-c-d)_n}\\
& - (2a + n - b - c -d) (b+c + d -a) \times \\
& \times \frac{(a+1)_n (a+ 1 - b - c)_n (a+ 1 - b -d)_n (a+ 1 - c -d)_n}{(a+1-b)_n (a+1-c)_n (a+1-d)_n(a-b-c-d)_n}
\end{align*}\pageoriginale
or shifting $e$ up by 1 and doing some algebra:
\begin{align}
& {}_7F_6 \left(\begin{aligned}
& a, \frac{a}{2} + 1, b, c, d, e, -n\\
& \frac{a}{2}, a+ 1 - b, \; a + 1 -c, \; a+ 1 - d,\; a+ 1 - e, a+ 1+ n
\end{aligned} ; 1\right)\label{art1-eq2.3}\\
& = \frac{(a+1)_n (a-b-c)_n (a-b-d)_n(a-c-d)_n}{(a+1-b)_n (a+1-c)_n(a+1-d)_n (a-b-c-d)_n} \times \notag\\
& \times \qquad \qquad \left[1+ \frac{n (n+2a-b-c-d) (a-b-c-d)}{(a-b-c) (a-b-d)(a-c-d)} \right] \notag
\end{align}
when the series is 4-balanced, or equivalently when
\begin{equation}
e = 2 a + n - b - c -d.  \label{art1-eq2.4}
\end{equation}

The second natural way to derive \eqref{art1-eq2.3} uses a more complicated formula than \eqref{art1-eq1.6}, but the calculations from the starting formula are easier, and one can see how to extend the sum to the very well poised $2k$-balanced series. The starting formula is Whipple's transformation \cite{art1-key14} between a very well poised ${}_7F_6$ and a balanced $_4F_3$:
\begin{align}
& {}_7 F_6 
\left(\begin{aligned}
& a, \frac{a}{2} + 1, b, c, d, e, -n\\
& \frac{a}{2}, a + 1 - b, a + 1 - c, a + 1 - d, a+1-e, a + 1 + n 
\end{aligned}; 1
\right) \label{art1-eq2.5}\\
& = \frac{(a+1)_n (a+1-b-c)_n}{(a+1-b)_n (a+1-c)_n} {}_4 F_3
\left(\begin{aligned}
& -n, a + 1 -d -e, b, c\\
& b + c -n -a, a+1-d, a+1 -e
\end{aligned} ;1
\right). \notag
\end{align}
When $e = 2 a + n - b - c -d$, the ${}_4 F_3$ on the right is
\begin{gather*}
 {}_4 F_3 
\left(
\begin{aligned}
-n, b+ c + 1 - n -a, b,c\\
 b+ c -n -a, a+1 - d , \; b + c + d +1 - n -a
\end{aligned} ;1\right)\\
 = \sum\limits^n_{k=0} 
\frac{(-n)_k (b)_k (c)_k}{(a+1-d)_k (b+c + d +1-n -a)_k k!} \cdot \frac{(k+b+c-n-a)}{(b+c-n -a)}\\
= {}_3 F_2 
\left(
\begin{aligned}
& -n, b, c\\
& a+1-d, b + c +d +1 -n -a
\end{aligned} ; 1
\right) +\\
+ \frac{(-n) bc}{(a+1-d) (b+c -n -a) (b+c+d +1 -n -a)} \times \\
\times {}_3 F_2 
\left(
\begin{aligned}
1-n, b+1, c+1\\
a+2 -d, b+ c + d +2 -n -a
\end{aligned}
\right) ;1
\end{gather*}\pageoriginale
The second ${}_3 F_2$ is balanced, and so can be summed using the Pfaff-Saalsch\"utz sum
\begin{equation}
{}_3 F_2 
\left(\begin{aligned}
& -n, b, c\\
& d, 1+b + c - n -d 
\end{aligned}
\right);1 = \frac{(d-b)_n(d-c)_n}{(d)_n (d-b-c)_n} .
\label{art1-eq2.6}  
\end{equation}
The first ${}_3 F_2$ is two balanced, and so can be written as the sum of 2 terms by use of the transformation formula: 
\begin{align}
&{}_3 F_2 
\left( 
\begin{aligned}
& -n , a, b\\
& c, d
\end{aligned} ; 1
\right) = \frac{(c-a)_n (c-b)_n}{(c)_n (c-a-b)_n} \; \times\label{art1-eq2.7} \\
\times\quad  & {}_3 F_2 
\left(\begin{aligned}
& -n, a , a + b+1-n - c -d\\
& a+1-n-c, a+1-n-d
\end{aligned};1
\right) . \notag
\end{align}
For, when the series on the left of \eqref{art1-eq2.7} is $k$-balanced, the third numerator parameter in the series on the right is $1-k$; so the series can be written as the sum of $k$ terms when $k=1,2,\ldots$

For those unacquainted with \eqref{art1-eq2.7}, an argument giving a $q$-extension is in the last section.

These series combine to give another derivation of \eqref{art1-eq2.3} when \eqref{art1-eq2.4} has been assumed. This method clearly extends to give the sum of the $2k$-balanced very well poised ${}_7 F_6$, but the resulting identity is too messy to be worth stating until it is needed.

\section{Another Clausen type identity.}\label{art1-sec3}
To obtain the next Clausen type identity take the ${}_4 F_3$ in \eqref{art1-eq1.9} to be 4-balanced, or take $c = a + b + 3/2$. As before, specialize \eqref{art1-eq2.3} by taking $c = a/2$, $d = (a+1)/2$ and make the series on the left 4-balanced. The resulting series is 
\setcounter{equation}{0}
\begin{gather}
 \quad {}_4 F_3 
\left(\begin{aligned}
& -n , a, b, e\\
& a+ 1+ n, a + 1 - b, a+ 1 -e 
\end{aligned} ;1
\right)\label{art1-eq3.1}\\
 =  \frac{(a+1)_n ((a-2b)/2)_n ((a-2b-1)/2)_n(-\frac{1}{2})_n}{(a+1-b)_n ((a+2)/2)_n ((a+2)/2)_n (-\frac{1}{2} -b)_n}
\times \notag\\
 \times
\left(1+ \frac{n (n+a-b-\frac{1}{2})}{[(a-2b)/2] [(a-2b-1)/2] (-\frac{1}{2})} \right)  \notag\\
= \frac{(a-2b-1)_{2n} (-\frac{1}{2})_n}{(a+1-b)_n (a+n+1)_n (-\frac{1}{2}-b)_n} 
\left[1+ \frac{4n (n+a-b-\frac{1}{2})(2b+1)}{(a+2b) (a-2b-1)} \right]. \notag
\end{gather}\pageoriginale 
The replace $a$ by $-k$ and after the same argument given above, replace $-n$ by $a$. The result is 
\begin{equation}
{}_4 F_3 
\left(
\begin{aligned}
&-k, a, b , e\\
& 1- a -k, 1-b-k, 1-e-k
\end{aligned};1
 \right) = \frac{(2a)_k (2b)_k (a+b)_k}{(a)_k (b)_k (2a+2b+2)_k} \times A
\label{art1-eq3.2}
\end{equation}
with $A$ given by 
\begin{equation}
A = 1 + \frac{(2+ 4a + 4 b + 8ab)k +k^2 -k}{2 (a+b) (2a+1) (2b+1)}
\label{art1-eq3.3}
\end{equation}
or by 
\begin{equation}
A = \frac{k^2 + (8ab + 4 a + 4b+1) k + 2 (a+b) (2a+1) (2b+1)}{2(a+b)(2a+1)(2b+1)}
\label{art1-eq3.4}
\end{equation}
and 
\begin{equation}
e =-k -a -b -\frac{1}{2}. \label{art1-eq3.5}
\end{equation}

Using \eqref{art1-eq3.2} with $A$ given by \eqref{art1-eq3.3} in \eqref{art1-eq1.9}, we obtain
\begin{align}
& \left[{}_2 F_2 \left(
\begin{aligned}
& a, b\\
& a+ b + \frac{3}{2}
\end{aligned};x
 \right) \right]^2 = {}_3 F_2 
\left(
\begin{aligned}
& \quad 2a, 2b, a+b\\
& a+b+\frac{3}{2}, 2a + 2b + 2
\end{aligned} ;x
\right)\label{art1-eq3.6}\\
& + \frac{2ab \; x}{(a+b+1) (a+b+3/2)} {}_3 F_2 
\left(
\begin{aligned}
& 2a + 1, 2 b+1, a+b+1\\
& a+b+\frac{5}{2}, 2a + 2b+3
\end{aligned};x
 \right)\notag\\
& +  \frac{ab x^2}{2 (a+b+3/2)^2 (a+b+5/2)} {}_3 F_2 
\left(
\begin{aligned}
& 2a +2, 2b+2, a+ b+2\\
& a+b +\frac{7}{2}, 2a+ 2b+4 
\end{aligned};x
\right) \notag
\end{align}

Using \eqref{art1-eq3.2} with $A$ given by \eqref{art1-eq3.4} gives
\begin{equation}
\left[{}_2 F_1 
\left(
\begin{aligned}
& a, b\\
& a + b+ \frac{3}{2}
\end{aligned} ;x  
\right) \right]^2 = {}_5 F_4 
\left(
\begin{aligned}
& 2a, 2b, a + b, c+ 1, d+1\\
& a+b+\frac{3}{2}, 2a+ 2b + 2, c, d
\end{aligned};x
\right) \label{art1-eq3.7}
\end{equation}
where $c$ and $d$ are determined by 
\begin{equation}
x^2 + (8ab +4a+4b+1) x + 2  (a+b) (2a+1) (2b+1) = (x+c) (x+d). 
\label{art1-eq3.8}
\end{equation}

\section{Comments.}\label{art1-sec4}
After\pageoriginale working out the above results, I went to a library to see if they were new. The fact that
\setcounter{equation}{0}
\begin{equation}
\left[{}_2 F_1 
\left(\begin{aligned}
& a, b\\
&  a+b +n + \frac{1}{2}
\end{aligned} ; x
\right)
\right]^2 , \; n = 0, 1, \ldots,
\label{art1-eq4.1}
\end{equation}
is a generalized hypergeomatric series was proved by Goursat \cite{art1-key8}. He also showed that
$$
\left[{}_2 F_1 \left(\begin{aligned}
& a, b\\
& ~ c
\end{aligned} ;x
\right) \right]^2
$$
is a generalized hypergeometric series only when $c = a + b+ n + \frac{1}{2}$, $n = 0$, $1, \ldots$. His proof that \eqref{art1-eq4.1} is a generalized hypergeometric series uses Clausen's formula \eqref{art1-eq1.2}, its derivative 
\begin{align}
& {}_2 F_1 
\left(
\begin{aligned}
&a, b\\
&  a+ b + \frac{1}{2}
\end{aligned} ;x
\right) {}_2 F_1 
\left(
\begin{aligned}
& a+ 1, b +1\\
& a + b + \frac{3}{2}
\end{aligned};x\right) \\
&= {}_3 F_2 
\left(
\begin{aligned}
2a + 1, 2b + 1, a+b +1\\
a+b + \frac{3}{2}, 2a + 2b +1 
\end{aligned};x
\right)
 \notag
\end{align}
and the transformation
$$
{}_2 F_1 
\left(
\begin{aligned}
& a, b\\
& a+ b + \frac{1}{2}
\end{aligned} ; x
 \right) = (1-x)^{1/2}
{}_2 F_1 
\left(
\begin{aligned}
a+ \frac{1}{2}, b + \frac{1}{2}\\
a+b+\frac{1}{2}
\end{aligned};x
\right).
$$
Of course Ramanujan knew all of those facts. Goursat also used some recurrence relations. Ramanujan knew about some of the recurrence relations hypergeometric series satisfy, and almost surely derived some of his continued fractions from these recurrence relations. However Ramanujan did not use recurrence relations as much as he could have, or as often as he used other properties of hypergeometric series. While  Ramanujan almost surely could have rediscovered Goursat's result if he had needed it, it is more likely he would have used an argument like the one given above. Ramanujan does not seem to have found Whipple's transformation formula \eqref{art1-eq2.5}. He did find a limiting case with one parameter missing, but we have not found \eqref{art1-eq2.5} in any of the sheets of his. If there is another treasure like the sheets in Trinity College, I would not be surprised in \eqref{art1-eq2.5} is there.

Actually, I would be surprised if Ramanujan was very interested in Goursat's result. What he really loved was not general results that could not be made very explicit, but beautiful formulas. I could imagine Ramanujan\pageoriginale working out the details in section 3, but the resulting formulas are already starting to be messier than those he loved.
 
I sent an outline of the results in sections 2 and 3 to a couple of people, and George Andrews wrote back that the 4-balanced very well poised ${}_7 F_6$ sum was found by Lakin \cite{art1-key9}. The two proofs given in section 2 are easier than the two Lakin gave, so it is worth including them above. Lakin also found a basic hypergeometric extension of this sum. The derivation of his result from the $q$-extension of Whipple's formula is the most natural one, so it will be given in the next section.

\section{The 3-balanced very well poised ${}_8 \varphi_7$.}\label{art1-sec5}

%%%% 9 page




\begin{thebibliography}{99}
\bibitem{art1-key1} 
\end{thebibliography}

