\title{ON SOME THEOREMS STATED BY RAMANUJAN}
\markright{On Some Theorems Stated by Ramanujan}

\author{By~ K. G. Ramanathan}
\markboth{K. G. Ramanathan}{On Some Theorems Stated by Ramanujan}

\date{}
\maketitle

\setcounter{pageoriginal}{150}
\section{}\label{art9-sec1}
Ramanujan\pageoriginale seems to have been fascinated by the continued fractions 
\begin{equation}
R(\tau)=\frac{e^{2\pi i\tau/5}}{1+}\frac{e^{2\pi i\tau}}{1+}\frac{e^{4\pi i\tau}}{1+}\label{art09-eq1}
\end{equation}
and
\begin{equation}
S(\tau)=\dfrac{e^{\pi i\tau/5}}{1-}\dfrac{e^{\pi i\tau}}{1+}\dfrac{e^{2\pi i\tau}}{1-}
\end{equation}
where $\tau=x+iy$, $y>0$, $i=\sqrt{-1}$. We discusses them in many places in the Notebooks and more importantly in the `Lost' Notebook. In particular, he evaluated $R(\tau)$ and $S(\tau)$ for $\tau=i\sqrt{n}$ for many rational values of $n>0$. Some of these evaluations were sent by him to Hardy in his early letters from India. A number of evaluations of $R(\tau)$ and $S(\tau)$ contained in the `Lost' Notebook were discussed and upheld by us \cite{art09-key4} using the Kronecker limit formula which seems to be well adapted for these problems. We do not, of course, know Ramanujan's methods. They could not be the method using the limit formula. There are two evaluations \cite[p. 46]{art09-key7} which are particularly intriguing. The are 
\begin{align*}
S(i\sqrt{3}) &= \frac{(-3+\sqrt{5})+\sqrt{6(5+\sqrt{5})}}{4}\tag*{(3)$_{\text{R}}$}\label{art08-eq3R}\\[3pt]
S(i/\sqrt{3}) &=\frac{-3+\sqrt{5}+\sqrt{6(5-\sqrt{5})}}{4}\tag*{(4)$_{\text{R}}$}\label{art08-eq4R}
\end{align*}
As far as we know, these two results have not been proved until now. In attempting to prove these, we encountered another of Ramanujan's evaluations. If $\lambda_{n}$ for integers $n\geq 1$ is defined by
$$
\lambda_{n}=\dfrac{e^{(\pi/2)\sqrt{n/3}}}{3\sqrt{3}}\{(1+e^{-\pi\sqrt{n/3}})(1-e^{-2\pi\sqrt{n/3}})(1-e^{-4\pi\sqrt{n/3}})\ldots\}^{6}
$$
the Ramanujan states
$$
\lambda_{1}=1, \lambda_{9}=3, \lambda_{17}=4+\sqrt{17}, \lambda_{25}=(2+\sqrt{5})^{2},
$$
%raghu, page 152

