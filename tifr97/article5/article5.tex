
\title{WEYL'S INEQUALITY, WARING'S PROBLEM AND DIOPHANTINE APPROXIMATION}
\markright{WEYL'S INEQUALITY, WARING'S PROBLEM AND DIOPHANTINE APPROXIMATION}

\author{By~  D. R. Heath-Brown}

\date{}
\maketitle

\setcounter{pageoriginal}{40} 


\textsc{For fixed positive}\pageoriginale integers $s$ and $k$, we define
$$
R_{s,k} (N) = \# \{(n_1, \ldots, n_2) \in \bbN^s : \sum\limits^s_1 n^k_j = N\}
$$

One central question in Waring's problem is to prove the Hardy-Littlewood asymptotic formula
\begin{equation}
R_{s,k} (N) = \frac{\Gamma (1+1/k)^s}{\Gamma (s/k)} \fS (N) N^{(s/k) -1} + O (N^{(s/k) -1 - \delta}) 
\label{art5-eq1}
\end{equation}
for as large a range of $s$ as possible. To tackle this, one uses an exponential sum
$$
S (\alpha) = \sum\limits^P_{n=1} e(\alpha n^k),
$$
where $P= [N^{1/k}]$. One then has
\begin{equation}
R_{s,k} (N) = \int\limits^1_0 S (\alpha)^s e(-\alpha N) dx.  \label{art5-eq2}
\end{equation}

The trivial bound for $S(\alpha)$ is $|S(\alpha)| \leqslant P$. However, one can improve on this for suitable $\alpha$, by using the following estimate.

\noindent
\textsc{Weyl's inequality}. \textit{Let $|\alpha - a/q| \leqslant q^{-2}$, with $(a, q) =1$. Then }
$$
S(\alpha) \ll_\epsilon P^{1+\epsilon} (q^{-1} + P^{-1} + qP^{-k})^{2^{1-k}},
$$
\textit{for any  } $\epsilon >0$.

Thus if $\alpha$ can be approximated with $P\leqslant q \leqslant P^{k-1}$ one has 
\begin{equation}
S (\alpha) \ll P^{1-2^{1-k} + \epsilon}, \label{art5-eq3}
\end{equation}
and the corresponding contribution to (2) is
$$
\ll P^{s(1-2^{1-k} +\epsilon)} \ll N^{s/k -1-\delta}, 
$$
provided that $s > 2^{k-1} k$. Those $\alpha$ which have no useable approximation produce the main term of \eqref{art5-eq1}. Thus one obtains \eqref{art5-eq1} for $s \geqslant 1+ 2^{k-1} k$.

One\pageoriginale can improve on this agrument by using an average bound. 

\textsc{Hua's inequality.} \textit{For any $\epsilon >0$, one has}
$$
\int\limits^1_0 |S(\alpha)|^{2^k} d\alpha \ll_\epsilon P^{2^k -k + \epsilon}.
$$

This leads to \eqref{art5-eq1} for $s \geqslant 1+ 2^k$. Until recently, this was the best known range for \eqref{art5-eq1}, for small $k \geqslant 3$.

The sum $S(\alpha)$ may also be used in Diophantine approximation problems. It was shown by Danicic \cite{art5-key2} that if $\epsilon >0$ and $k \in N$ are given, then there exists $P(\epsilon, k)$ as follows. For any $P \geqslant P(\epsilon, k)$ and any $\alpha \in \bbR$, one can find $n \leqslant P$ with 
\begin{equation}
||\alpha n^k || \leqslant P^{\epsilon -2^{1-k}}. \label{art5-eq4}
\end{equation}

This generalizes Dirichlet's approximation Theorem, when $k=1$, and a result of Heilbronn \eqref{art5-eq4}, for $k=2$. To prove Danicic's theorem one can use a result of Montgomery (see Baker \cite[Theorem 2.2]{art5-key1}): If $||a_n|| >\Delta$ for $1\leqslant n \leqslant P$, then
$$
\sum\limits_{1 \leqslant h \leqslant \Delta^{-1}} |\sum\limits_{n\leqslant P} e(ha_n)| \geqslant  P/6.
$$
We therefore wish to estimate
\begin{equation}
\sum\limits_{h \leqslant \Delta^{-1}} |S(\alpha h)|. \label{art5-eq5}
\end{equation}

As with Weyl's inequality, this can be done satisfactorily, with a relative saving of $P^{-2^{1-k} + \epsilon}$, unless $\alpha$ has an approximation
\begin{equation}
|\alpha - a/q| \leqslant \frac{\Delta}{q P^{k-1}}, q \leqslant P.  \label{art5-eq6}
\end{equation}

Thus Montgomery's result allows us to take $\Delta \approx P^{\epsilon -2^{1-k}}$. Of course, if \eqref{art5-eq6} holds then $||\alpha q^k|| \leqslant \Delta$, and $q \leqslant P$.

A sharpening of Weyl's inequality has recently been obtained (Heath-Brown \cite{art5-key3}).

\begin{theorem}\label{art5-thm1}
Let $|\alpha - a/q|\leqslant q^{-2}$ with $(a, q) =1$, and suppose that $k \geqslant 6$. Then
$$
S(\alpha) \ll_\epsilon P^{1+\epsilon} (Pq^{-1} + P^{-2} + qP^{1-k})^{(4+3)2^{-k}}
$$
for any $\epsilon >0$.
\end{theorem}

Thus\pageoriginale 
$$
S(\alpha) \ll P^{1-(4/3)2^{1-k} + \epsilon},
$$
if $P^3 \leqslant q \leqslant P^{k-3}$. One therefore has a sharper bound than \eqref{art5-eq3}, but for a shorter range of $q$, (and only for $k \geqslant 6$). Closely related to Theorem 1 is an improvement on Hua's Inequality (Heath-Brown \cite{art5-key3}).

\begin{theorem}\label{art5-thm2}
Let $k \geqslant 6$ and $\epsilon > 0$. Then 
$$
\int\limits^1_0 |S(\alpha)|^{(7/8)2^k} \; d \alpha \ll P^{(7/8)2^k -k+\epsilon}.
$$
\end{theorem}

As before one may deduce :

\begin{coro*}
The Hardy-Littlewood asymptotic formula \cite{art5-key1} holds for $k \geqslant 6$ and $s \geqslant 1 + \dfrac{7}{8} 2^k$.
\end{coro*}

One may also try to sharpen Danicic's result.  One obtains a saving in \eqref{art5-eq5} of
$$
P^{-(4/3)2^{1-k} + \epsilon}
$$
relative to the trivial estimate, unless
$$
|\alpha -a/q| \leqslant \frac{\Delta}{qP^{k-3}}, \; q  \leqslant P^3.
$$

Unfortunately in this latter case, one gets no useable bound for $||aq^k||$. The attempt to improve on \eqref{art5-eq4} therefore fails. However, if one starts with an approximation $|\alpha -a/q| \leqslant q^{-2}$ and fixes $P =[q^{(1/3)}]$, for example, one is led to an ``unlocalized''  result (Heath-Brown \cite{art5-key3}).

\begin{theorem}\label{art5-thm3}
Let $\alpha \in \bbR$ and $\epsilon > 0$ be given. For any integer $k \geqslant 6$, there are infinitely many $n \in \bbN$ with 
$$
||\alpha n^k || \leqslant n^{\epsilon - (4/3)2^{1-k}.}
$$
\end{theorem}

Let us now look at the proof of Theorem 1. One uses Weyl's ``square and difference'' trick, but with the symmetric difference 
$$
(\nabla_h \empty) (x) =\empty (x+h) - \empty (x-h)
$$
in place of the forward difference. After $j$ steps,  one has
\begin{equation}
|S(\alpha)|^{2^j} \ll P^{2^j - j -1} \sum\limits_{h_1, \ldots, h_j} |R(\alpha)|,
\label{art5-eq7}
\end{equation}
where\pageoriginale $|h_i| < P/2$ and
$$
R(\alpha) = R (\alpha; h_1, \ldots, h_j) = \sum\limits_{n\in I} e (\alpha \nabla_{h_1} \ldots \nabla_{h_j} (n^k)).
$$

Here $I$ is a subinterval of [1, P], depending on $h_1, \ldots, h_j$. As a function of $n$, the polynomial
$$
\nabla_{h_1}\ldots \nabla_{h_j} (n^k)
$$
has degree $k-j$. An appropriate version of Weyl's Inequality would therefore give
\begin{equation}
R (\alpha) \ll P^{1-2^{1-(k-j)} +\epsilon}, \label{art5-eq8}
\end{equation}
for suitable $\alpha$. in conjunction with \eqref{art5-eq7} we would then obtain
$$
|S(\alpha)|^{2^j} \ll P^{2^j -j -1}\cdot P^j \cdot P^{1-2^{1-(k-j)+\epsilon}},
$$
whence
$$
S(\alpha) \ll P^{1-2^{1-k} + \epsilon},
$$
for suitable $\alpha$. One thus merely recovers Weyl's Inequality again.

To improve on this, we replace \eqref{art5-eq8} by a mean-value bound, where one averages over the parameters $h_i$. If one takes $j=k-1$ or $k-2$ then $R(\alpha)$ is a linear or quadratic sum, and the bound \eqref{art5-eq8} is essentially best possible. Thus nothing can be gained by averaging. One therefore chooses $j=k-3$, in which case $R(\alpha)$ is a cubic sum of the form
$$
R(\alpha) = \sum\limits_{n\in 1} e (An^3 + Bn).
$$

Here the interval $I$ and the coefficients $A$ and $B$ depend on the $h_i$. In fact, $A$ takes the form 
$$
A = \frac{k!}{6} 2^{k-3} h_1 \ldots h_{k-3}.
$$

Had one used forward differences in deriving \eqref{art5-eq7} rather than symmetric differences, there would have been a term in $n^2$ appearing in $R(\alpha)$, and so one would have to average over three coefficients, rather than two. With  $j=k-3$, the Weyl bound now takes the form
\begin{equation}
|R(\alpha)| \ll P^{3/4+\epsilon} , \label{art5-eq9}
\end{equation}
for suitable $\alpha$, whereas one would conjecture that
$$
|R(\alpha)| \ll P^{1/2+\epsilon},
$$
in general.\pageoriginale In fact, one can easily prove that
\begin{equation}
\int\limits^1_0 \int\limits^1_0 |\sum\limits_{n \leqslant P} e(An^3+ Bn)|^6 d A d B \ll P^{3+\epsilon}, 
\label{art5-eq10}
\end{equation}
by counting the number of solutions of the simultaneous equations
\begin{equation*}
\hspace{2cm} 
\begin{matrix}
n^3_1 + n^3_2 + n^3_3 = n^3_4 + n^3_4 + n^3_6\\
~n_1 + n_2 + n_3 = n_4 + n_5 + n_6. 
\end{matrix} \hspace{2cm} (1\leqslant n_i \leqslant P)
\end{equation*}

To pass from the sum on the right hand side of \eqref{art5-eq7} to the mean value \eqref{art5-eq10}, one uses the bound
$$
\sum\limits'_{h_i \ldots, h_{k-3}} |R(\alpha)|^6 \ll P^{4 +\epsilon} \sN \int\limits^1_0 \int\limits^1_0 |\sum\limits_{n\leqslant P} e (An^3 + Bn)|^6 d A d B,
$$
where 
$$
\sN = \max\limits_{A \in [0,1]} \# \{(h_1, \ldots, h_{k-3}) : ||\frac{k!}{6} 2^{k-3} h_1\ldots h_{k-3} \alpha -A|| \leqslant P^{-3}\} .
$$

Here we exclude the possibility that any $h_i$ vanishes, both in the sum $\sum'$ and in the maximum occurring in the definition of $\sN$. It is apparent that there is a loss of a factor $P$ in passing from the discrete average of $R(\alpha)$ over the $h_i$ to the mean-value \eqref{art5-eq10}. Nonetheless, one finds that $R(\alpha)$ is $O(P^{2/3 + \epsilon})$ on average, and this is a sufficient improvement on \eqref{art5-eq9} for the proof of Theorem \ref{art5-thm1}.

\begin{thebibliography}{99}
\bibitem{art5-key1} \textsc{R. C. Baker} : \textit{Diophantine Inequalities} (Oxford Science Publications, 1986).

\bibitem{art5-key2} \textsc{I. Danicic} : \textit{Contributions to Number Theory} (Ph.D. Thesis, London, 1957).

\bibitem{art5-key3} \textsc{D. R. Heath-Brown}  : Weyl's inequality, Hua's inequality, and Waring's problem, \textit{J. London Math. Soc.} (2), to appear.

\bibitem{art5-key4} \textsc{D. R. Heath-Brown}\pageoriginale : The fractional part of $\alpha n^k$, \textit{Mathematika,} to appear. 

\bibitem{art5-key5} \textsc{H. Heilbronn} : On the distribution of the sequence $n^2 \theta (\mod 1)$, \textit{Quart. J. Math. Oxford Ser.,} 19 (1948), 249-256.
\end{thebibliography}

\medskip
\noindent
{\small Magdalene College,}

\noindent
{\small Oxford OX1 4AU,}

\noindent
{\small United Kingdom.}


