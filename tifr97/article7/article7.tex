
\title{SUMS OF KLOOSTERMAN SUMS AND THE EIGHTH POWER MOMENT OF THE RIEMANN ZETA-FUNCTION}
\markright{SUMS OF KLOOSTERMAN SUMS AND THE EIGHTH POWER MOMENT OF THE RIEMANN ZETA-FUNCTION}

\author{By~ N. V. Kuznetsov}

\date{}
\maketitle

\setcounter{pageoriginal}{56} 

\hfill \textit{Dedicated to Atle Selberg}

\setcounter{section}{-1}
\section{Introduction.} The\pageoriginale domain of mathematics which will be discussed here was called ``Kloostermania'' by M. Huxley. This name outlined (but not too sharply) the boundary between number theory, the theory of the modular and automorphic functions, spectral theory and geometry.

The beginning was due to Poincar\'e. The contribution which defined the base of this theory was given by Petersson, Hecke, Rankin and Maass. In the last few decades, its development was stimulated by the famous talk of Atle Selberg at Tata Institute and by  L. Faddeev's work which clarified the spectral expansion.

It was ten years ago when I found the explicit form of the connection between sums of Kloosterman sums (``known quantities'') and Fourier coefficients of cusp forms (unknown quantities which are very mysterious up to this day). In the next year, R. Bruggeman rediscovered (independently) a part of these results. From that time, the number of publications is increasing rapidly in this domain. 

So it happened that the Kloosterman sums (which will be defined below) arose firstly for improving the Hardy-Littlewood ``circle method''. 

But these sums would have arisen earlier, if Poincare had wished to calculated the Fourier coefficients of the series which are called today ``the Poincar\'e series''.

The goal of this paper is to make more popular this dynamic branch of Mathematics and to demonstrate new possibilities for the Riemann zeta-functions.

The first part of the paper (the short subsection 1.1-1.13) contains known results. The second part gives, as a new consequence of the ``Kuznetsov trace formula'', the exact order for the eighth power moment of the Riemann zeta-function. Namely. we have, for some absolute constant $B > 4$.
$$
\int\limits^T_0 |\zeta (\frac{1}{2} + it)^8 dt \ll T (\log T)^{16+B} , T \to + \infty.
$$

From\pageoriginale various consequences of this estimate, one can derive the conclusion: there is a fixed constant $B$ so that 
$$
|\zeta (\frac{1}{2} + it)| \ll |t|^{1/8} \; (\log |t|)^B, |t| \to \infty, t \text{ is real.}
$$

\bigskip
\begin{center}
{\Large \textbf{Part \thnum{I}.\label{art7-partI} Sums of Kloosterman sums}}
\end{center}

\setcounter{section}{1}
\subsection{The Lobatcevskii plane.}\label{art7-subsec1.1}
This plane will be considered as the upper half plane $\bbH$ of the complex variable $z = x + iy$, $x, y \in \bbR$, $y >0$, with the metric
\begin{equation}
ds^2 = y^{-2} (dx^2 + dy^2), \label{art7-eq1.1}
\end{equation}
measure
\begin{equation}
d\mu(z) = y^{-2} dx \; dy \label{art7-eq1.2}
\end{equation}
and with the corresponding Laplace operator 
\begin{equation}
L = - y^2 \left(\frac{\partial^2}{\partial x^2} + \frac{\partial}{\partial y^2} \right).\label{art7-eq1.3}
\end{equation}

The full modular group acts on this plane in the natural way:
\begin{equation}
z \mapsto \gamma z = \frac{ax+b}{cz+d}, a, b, c, d \in \bbZ, \; ad - bc = 1. \label{art7-eq1.4}
\end{equation}

Most of the results may be developed for certain Fuchsian groups but there are no essentially new ideas for these cases; so, I restrict myself to the full modular group $\Gamma$ only. 

\subsection{The appearance of Kloosterman sums.}\label{art7-subsec1.2}

Their appearance is inescapable, if one calculates the Fourier coefficients of an automorphic function which is defined as a sum over a gorup.

For example, let us define the classical Poincar\'e series by 
\begin{equation}
P_n (z;k) = \frac{(4\pi n)^{k-1}}{\Gamma (k-1)} \sum\limits_{\gamma \in \Gamma_\infty / \Gamma} j^{-k} (\gamma, z) e (n\gamma z), n \geqslant 1 \label{art7-eq1.5}
\end{equation}
(where $\Gamma_\infty$ is generated by the translation $z \mapsto z + 1$, $j(\gamma, z) = cz + d$ if the transformation $\gamma$ is defined by a matrix $\left(\begin{smallmatrix}
\ast & \ast\\
c & d
\end{smallmatrix} \right)$ and we assume that $k$ is an even integer and $k \geqslant 4$). Then, for the $m$-th Fourier coefficient fo this series we have an almost obvious formula (the so-called ``Petersson formula''):
\begin{align}
p_{n,m} (k) : = \int\limits^1_0 P_n (z,k) e (-mx) dx \; e^{2\pi my}\label{art7-eq1.6}\\
 = \frac{(4\pi \sqrt{nm})^{k-1}}{\Gamma (k-1)} (\delta_{n,m} + 2 \pi i^{-k} \sum\limits_{c \geqslant 1} \frac{1}{c} S (n, m; c) J_{k-1} \left( \frac{4\pi \sqrt{nm}}{c}\right)), n, m \geqslant 1,
\end{align}
where\pageoriginale $J_{k-1}$ is the Bessel function of the order $k-1$ and $S$ is the Kloosterman sum
\begin{equation}
S(n, m;c): =  \sum\limits_{\substack{1\leqslant d \leqslant |c|, (d,c) =1\\ dd' \equiv 1 (\mod c)}} e \left(\frac{nd+md'}{c} \right).
\label{art7-eq1.7}
\end{equation}

We have a similar (but more complicated) representation, for the Fourier coefficients of the non-holomorphic Poincar\'e-Selberg series which, for $\re s>1$, is defined by 
\begin{equation}
U_n(z,s) : = \sum\limits_{\gamma \in \Gamma_\infty / \gamma} (\Iim \gamma z)^s e (n \gamma z), \; n \geqslant 1, \label{art7-eq1.8}
\end{equation}
(For $n=0$, it is the Eisenstein series.)

For the Kloosterman sums, we have the famous estimate due to A. Weil:
\begin{equation}
|S(n, m; c)| \leqslant (2n, 2m, c) d^2 (c) c^{1/2}. \label{art7-eq1.9}
\end{equation}

But, for the applications, we need estimates for the averages of these sums. Yu V. Linnik was the first to conjecture that Kloosterman sums oscillate regularly; his conjecture is that
\begin{equation}
\left|\sum\limits_{c \leqslant X} \frac{1}{c} S (n, m; c) \right| \ll_{n,m} X^\epsilon \label{art7-eq1.10}
\end{equation}
for every $\epsilon > 0$ as $X  \to + \infty$.

It is obvious that $A$-Weil's estimate give only $O(X^{(1/2)+\epsilon})$ on the right side and A. Selberg destroyed the hopes of any near progress in this conjecture when he constructed the counterexamples of groups for which Linnik conjecture is not valid (1963).

At this point, there was a nice result from my first paper on this subject (1977): for every fixed $\epsilon >0$, we have
\begin{equation}
\left|\sum\limits_{1 \leqslant c \leqslant X } \frac{1}{c} S (n, m;c) \right| \ll_{n,m} X^{(1/6)+\epsilon} \label{art7-eq1.11}
\end{equation}

At the same time, for the ``smoothing'' average, we have a stronger estimate : if $\varphi \in C^\infty (0,\infty)$, $\varphi =0$  outside the interval $(a, 2a)$ and if, for every fixed integer $r \geqslant 0$, we have $\left(\dfrac{\partial}{\partial x} \right)^r \varphi (x) \ll a^{-r}$, then, for every fixed $A>0$, the following estimate is valid:
\begin{equation}
\left|\sum\limits_{a \leqslant c \leqslant 2a} \varphi (c) S (n, m; c) \right| \ll a^{-A}, a \to + \infty.
\label{art7-eq1.12}
\end{equation}

Thus it is a confirmation of the Linnik conjecture.

\subsection{The eigenfunctions of the automorphic Laplacian.}\label{art7-subsec1.3}

As a\pageoriginale generalization of the classical cusp forms of even integral weight $k$ (which are regular functions on the upper half plane such that $f(\gamma z) = j^k (\gamma, z) f(z)$ for any $\gamma \in\Gamma$ and $y^{k/2} | f(z)|$ is bounded for $y >0$, the Poincar\'e series $P_n (z, k), n\geqslant 1$, being an example of a cusp form of weight $k$ with respect to full modular group), Maass introduced the non-holomorphic cusp forms (the so called Maass waves).

The Laplace operator $L$ in $L^2 (\Gamma / \bbH)$  has a continuous spectrum on the half axis $\lambda \geqslant \frac{1}{4}$ and a discrete spectrum $\lambda_0=0$, $0< \lambda_1 < \lambda_2 \leqslant \ldots$ with limit point at $\infty$ (note that $\lambda_1 \simeq 91.07\ldots$). For the case of the full modular group, there are no exceptional eigenvalues in the interval $(0, \frac{1}{4})$; (Huxley proved that the same is true for any congruence subgroup with the level $\leqslant 19$). So $L^2 (\Gamma / \bbH)$ decomposes as $L^2_{\text{eis}} (\Gamma /\bbH) \oplus L^2_{\text{cusp}} (\Gamma / \bbH)$ where $L^2_{\text{eis}}$ is the continuous  direct sum of $E(z, \frac{1}{2}+ it)$, $t \in \bbR$ ($E$ being the Eisenstein series) and $L^2_{\text{cusp}}$ is spanned by the eigenfunctions $u_j(z)$ given by 
\begin{gather*}
Lu_j \equiv - y^2 \left(\frac{\partial^2 u_j}{\partial x^2} + \frac{\partial^2 u_j}{\partial y^2} \right) = \lambda_j u_j, \label{art7-eq1.13}\\
u_j (\gamma z) = u_j  (z), \; \gamma \in \Gamma ; (u_j, u_j) = \int\limits_{\Gamma / \bbH} |u_j|^2 d\mu (z) < \infty.\notag
\end{gather*}

Any $f \in L^2 (\Gamma / \bbH)$ which is smooth enough can be expanded into eigenfunctions of $L$ and we have 
\begin{equation}
f(z) = \sum\limits_{j \geqslant 0} (f, u_j) u_j (z)  + \frac{1}{4\pi} \int\limits^\infty_{-\infty} (f, E (., \frac{1}{2} + ir)) E (z, \frac{1}{2} + ir) dr \label{art7-eq1.14}
\end{equation}
if we choose $u_j$ so that we have an orthonormal basis $\{u_j\}_{j\geqslant 0}$.

Note that the Eisenstein series has the Fourier expansion
\begin{align}
E (z,s) = y^2 + \frac{\xi (1-s)}{\xi(s)} y^{1-s} + \label{art7-eq1.15} \\
+ \frac{2\sqrt{y}}{\xi (x)} \sum\limits_{n\neq 0} \tau_s (n) e (nx) K_{s-1/2} (2\pi |n| y), \xi (s) : = \pi^{-s} \Gamma (s) \zeta (2s), \notag
\end{align}
where $K_{s-1/2} (.)$ is the modified Bessel function of order $s - \frac{1}{2}$ and 
\begin{equation}
\tau_s (n) = |n|^{s-1/2} \sigma_{1-2s(n)} = \sum\limits_{\substack{d|n\\d>0}} \left(\frac{|n|}{d^2} \right)^{s-1/2} \label{art7-eq1.16}
\end{equation} 

The\pageoriginale eigenfunctions of a point $\lambda_j$ of the discrete spectrum have a similar Fourier expansion
\begin{equation}
u_j = \sum\limits_{n \neq 0} \rho_j (n) e (nx) \sqrt{y} K_{i\chi_j} (2\pi|n| y) \label{art7-eq1.17}
\end{equation}
with $\chi_j = \sqrt{\lambda_j -\frac{1}{4}}, \lambda_j > \frac{1}{4}$.

\subsection{The Hecke operators.}\label{art7-subsec1.4}
The ideas behind Hecke operators go back to Poincare and Mordell used them to prove that Ramanujan's $\tau$-function was multiplicative.

The main observation is a simple fact that if $H$ is a subgroup if $\Gamma$ with finite index, so that $\Gamma$ is a finite coset union $\bigcup\limits_j H_{\gamma_j}$, and $f$ is automorphic on $H$, then $\sum\limits_j f(\gamma_jz)$ is automorphic on $\Gamma$.

By appropriately choosing the set of representatives, we can define the $n$-th Hecke operator as the average
\begin{equation}
(T_n f) (z) = \frac{1}{\sqrt{n}} \sum\limits_{\substack{ad =n\\d>0}} \sum\limits_{b(\mod d)} f \left(\frac{az+b}{d} \right)
\label{art7-eq1.18}
\end{equation}

For this normalization, we have
\begin{equation}
T_n T_m = \sum\limits_{d|(n, m)} T_{nm/d^2} \label{art7-eq1.19}
\end{equation}
and all these operators commute.

Now we have a set of commuting Hermitian operators, with the same set of eigenfunctions that arose for the Laplace operator. Thus we can choose the eigenfunctions of the Laplace operator, so that in the basis which was constructed from these, each Hecke operator has a diagonal form. Then these eigenfunctions will be called ``Maass waves''. For that, choose 
\begin{gather}
T_n u_j  = t_j (n) u_j, \; n \geqslant 1, j\geqslant 0,\label{art7-eq1.20}\\
T_n E (.,s) = \tau_s (n) E(., s).\label{art7-eq1.21}
\end{gather}

The eigenvalues $t_j(n)$ of the discrete spectrum of the $n$-th Hecke operator $T_n$ are connected with the Fourier coefficients of $u_j$ by the equalities 
\begin{equation}
\rho_j (1) t_j (n) = \rho_j(n), \; n \geqslant 1, j \geqslant 1. \label{art7-eq1.22}
\end{equation}

It is convenient to choose the eigenfunctions so that they will be eigenfunctions of the operator $T_{-1}$:
$$
(T_{-1} f) (z) = f(-\bar{z}).
$$

Then\pageoriginale $T_{-1} u_j =\epsilon_j u_j$ with $\epsilon_j = \pm 1$ and we have
\begin{equation}
\rho_j (-n) = \epsilon_j \rho_j (1) t_j (n), \; n \geqslant 1, j \geqslant 1.  \label{art7-eq1.23}
\end{equation}

\subsection{The sum formulae for Kloosterman sums.}\label{art7-subsec1.5}
The natural generalization of the classical Petersson formula
\begin{align}
(f,P_n) & = \int\limits_{\Gamma/\bbH} f (z) \overline{P_n (z,k)} y^k d\mu(z) \label{art7-eq1.24} \\
& = a_f (n) (= n-\text{th Fourier coefficient}) \notag
\end{align}
for any $f$ from the space $M_k$ of the regular cusp forms of weight $k$ is the same formula for the inner product $(f, U_n(., \bar{s}))$ for an automorphic $f$ from the space of cusp forms $M_0$ of weight zero.

It is not hard to show that the non-holomorphic Poincar\'e series $U_n(z,s)$ may be continued analytically (with its Fourier expansion) in the half plane $\re s > \frac{3}{4}$ (this being based on A. Weil's estimate for the Kloosterman sum). So, for $\re s_1$, $\Re s_2 > \frac{3}{4}$, the inner product $(U_n (., s_1), U_m (., \overline{s_2}))$ is well-defined. Since $U_n$ may be expressed as a sum over a group, this inner product is a sum of Kloosterman sums. On the other hand, the inner product $(u_j, U_n)$ may be calculated explicitly in terms of $\Gamma$-functions and the $n$-th Fourier coefficient of $j$-th eigenfunction $u_j$ of the automorphic Laplacian, Hence the bilinear form of $n$-th Fourier coefficients of the eigenfunctions 
$$
\sum\limits_{j \geqslant 0} \rho_j (n) \overline{\rho_j(m)} h (\chi_j)
$$
for a certain test function $h$, may be expressed as a sum of Kloosterman sums. 

Of course, we have, for the case $(U_n (., s_1), U_m (., \bar{s}_2))$, two free variables $s_1$, $s_2$ and we can try to construct an arbitrary test function in our bilinear form by integration with respect to these variables.

This plan was fulfilled in my first paper and in this way, we have following sum formula (referred to by some authors as the ``Kuznetsov trace formula'').

\medskip
\noindent
{\bfseries Theorem \thnum{1}.\label{art7-thm1}} \textit{Let us assume that the function $h(r)$ of the complex variable $r$ is regular in the strip $|\Iim r| \leqslant \delta$ with some $\delta > \frac{1}{2}$ and $|h(r)| \ll |r|^{-B}$ for some $B > 2$ when $r \to\infty$ in this strip.}


\textit{Then,\pageoriginale for any integers $n, m \geqslant 1$, we have}
\begin{gather}
\sum\limits^\infty_{j=1} \alpha_j t_j (n) t_j (m) h (\chi_j) + \frac{1}{4\pi} \int\limits^\infty_{-\infty} \tau_{(1/2) + ir} (n) \tau_{(1/2) + ir} (m) \times 
\label{art7-eq1.25} \\
\times \frac{h(r)dr}{|\zeta (1+2ir)|^2} = \notag\\
= \frac{\delta_{n,m}}{\pi^2} \int\limits^\infty_{-\infty} r \text{ th}(\pi r) h (r) dr + \sum\limits_{c \geqslant 1} \frac{1}{c} S (n, m; c) \varphi \left(\frac{4\pi \sqrt{nm}}{c} \right) , \notag
\end{gather}
\textit{where }
\begin{equation}
\alpha_j = (\text{ch} (\pi \chi_j))^{-1} |\rho_j (1)|^2,  \label{art7-eq1.26}
\end{equation}
\textit{$\zeta$ is the Riemann zeta function and for $x > 0$, the function $\varphi(x)$ is defined in terms of $h$ by the integral transform}
\begin{equation}
\varphi (x) = \frac{2i}{pi} \int\limits^{\infty}_{-\infty} J_{2 ir} (x) \frac{rh(r)}{\text{ch} (\pi r)} dr. \label{art7-eq1.27}
\end{equation}

Identity \eqref{art7-eq1.25} is modified in the following manner, if the integers, $n$, $m$ on the right-hand side have different signs.

\medskip
\noindent
{\bfseries Theorem \thnum{2}.\label{art7-thm2}} \textit{Assume that the function $h$ satisfies the conditions of the preceding theorem. Then, for any integers $n$, $m \geqslant 1$, we have}
\begin{gather}
\sum\limits_{j \geqslant 1} \epsilon_j \alpha_j t_j (n) t_j (m) h (\chi_j) + \frac{1}{4\pi} \int\limits^\infty_{-\infty} \tau_{(1/2)+ ir} (n) \tau_{(1/2) + ir} (m) \frac{h(r)dr}{|\zeta (1+2ir)|^2} = \label{art7-eq1.28}\\
= \sum\limits_{c \geqslant 1} \frac{1}{c} S (n, -m; c) \psi \left(\frac{4\pi \sqrt{nm}}{c} \right) \notag
\end{gather}
\textit{where $\psi (x)$, for $x>0$, is defined in terms of $h$ by the integral}
\begin{equation}
\psi(x) = \frac{4}{\pi^2} \int\limits^\infty_{-\infty} K_{2ir} (x) h (r), sh (\pi r) dr.  \label{art7-eq1.29}
\end{equation}

We can invert identities \eqref{art7-eq1.25} and \eqref{art7-eq1.28} and we shall assume that the sum of Kloosterman sums is given rather than the bilinear form in the Fourier coefficients.

\medskip
\noindent
{\bfseries Theorem \thnum{3}.\label{art7-thm3}} \textit{Assume\pageoriginale that to a function $\psi : [0, \infty) \to\bbC$, the integral transform}
\begin{equation}
h(r) = 2\text{ch} (\pi r) \int\limits^\infty_0 K_{2ir} (x) \psi (x)  \frac{dx}{x} \label{art7-eq1.30}
\end{equation}
\textit{associates the functions $h(r)$ satisfying the conditions of Theorem \ref{art7-thm1}. Then, for this $\psi$ and for integers $n$, $m \geqslant 1$, identity \eqref{art7-eq1.28} is satisfied, where $h$ is defined by the integral \eqref{art7-eq1.30}.}

\medskip
\noindent
{\bfseries Theorem \thnum{4}.\label{art7-thm4}}
Let $\varphi \in C^3 (0, \infty)$, $\varphi (0) = \varphi'(0) = 0$ and assume that $\varphi (x)$, together with its derivatives up to the third order, is $O(x^{-B})$ for some $B > 2$, as $x \to \infty$. Then, for any integers $n, m \geqslant 1$, we have
\begin{gather}
\sum\limits_{c \geqslant 1} \frac{1}{c} S (n, m; c) \varphi_H \left(\frac{4\pi \sqrt{nm}}{c} \right) = - \frac{\delta_{n,m}}{2\pi} \int\limits^\infty_0 J_0 (x) \varphi (x) dx +\label{art7-eq1.31}\\
+ \sum\limits_{j \geqslant 1} \alpha_j t_j (n) t_j (m) h (\chi_j) + \frac{1}{4\pi} \int\limits^\infty_{-\infty} \tau_{(1/2) + ir } (n) \tau_{(1/2) + ir} (m) \frac{h(r)dr}{|\zeta (1+2ir)|^2},\notag
\end{gather}
\textit{where the functions $\varphi_H(x)$ and $h(r)$ are defined in terms of $\varphi$ by the integral transforms}
\begin{align}
\varphi_H(x) & = \varphi(x) - 2 \sum\limits^\infty_{k=1} (2k -1) J_{2k -1} (x) \int\limits^\infty_0 J_{2k-1} (y) \varphi (y) \frac{dy}{y}. \label{art7-eq1.32}\\
h(r) & = \frac{i \pi}{2sh (\pi r)} \int\limits^\infty_0 (J_{2 ir} (x) - J_{-2ir} (X)) \varphi (x) \frac{dx}{x}. \label{art7-eq1.33}
\end{align}

It should be useful to note that the transformation $\varphi \to \varphi_H$ in \eqref{art7-eq1.32} is a projection by which, to a given $\varphi$, one associates its component orthogonal on the semiaxis $x \geqslant 0$ (with respect to the measure $x^{-1}dx$) to all the Bessel functions of odd integral order.

Together with \eqref{art7-eq1.32}, this projection can be defined by the equality
\begin{align}
\varphi_H (x) & = \varphi (x) - x \int\limits^\infty_0 \varphi (u) (\int\limits^1_0 \xi J_0 (x\xi) J_0 (u \xi) d \xi) du \label{art7-eq1.34}\\
& = \varphi (x) - x \int\limits^\infty_0 \varphi (u) \frac{x J_0 (u) J_1 (x) - u J_0 (x) J_1 (u)}{x^2 - u^2} du \notag
\end{align}
and any\pageoriginale sufficiently smooth $\varphi$ admits a decomposition
\begin{equation}
\varphi =\varphi_H + (\varphi - \varphi_H)\label{art7-eq1.35} 
\end{equation}
where $\varphi - \varphi_H$ is a combination of the Bessel functions defined by \eqref{art7-eq1.32} while $\varphi_H$ is equal to integral \eqref{art7-eq1.27}, in which by $h$ one means the integral transform \eqref{art7-eq1.33} of the function $\varphi$.

The classical Petersson formula
\begin{gather}
\sum\limits^{v_k}_{j=1} ||f_{j,k}||^{-2} t_{j,k} (n) t_{j,k} (m) = \label{art7-eq1.36}\\
= i^k \delta_{n,m} + 2 \pi \sum\limits^\infty_{c =1} \frac{1}{c} S (n, m, c) J_{k-1} \left(\frac{4\pi \sqrt{nm}}{c} \right) \notag
\end{gather}
(where $f_{j,k}$ form an orthogonal basis in the space $\sM_k$ of cusp forms of weight $k, ||f_{j,k}||^2 = (f_{j,k,} f_{j,k})$ and $v_k = \dim \sM_k$) allows us to represent the sum
\begin{equation}
\sum\limits_{c \geqslant 1} \frac{1}{c} S ((n, m; c)) \varphi \left(\frac{4\pi \sqrt{nm}}{c} \right)\label{art7-eq1.37} 
\end{equation}
as a bilinear form in the eigenvalues of the Hecke operators for the case when $\varphi$ may be represented by the Neumann series of the Bessel functions  of odd order. Together with \eqref{art7-eq1.31} this gives a representation of the sum \eqref{art7-eq1.37} as a bilinear form of the eigenvalues of the Hecke operators (in all $\sM_k$ with even $k$ and $\sM_0$) for an arbitrary ``good'' function $\varphi$.


\subsection{Some relations with Bessel functions.}\label{art7-subsec1.6}
The special case of the following expansion in terms of Bessel functions is the crucial key to prove the identities of the preceding theorems (Really our identities are consequences of a  suitable averaging of the initial identity which results from a comparison of two different expressions for the inner product $(U_n (., 1+ it), U_m (., 1 - it))$, $t \in \bbR$).

\medskip
\noindent
{\bfseries Theorem \thnum{5}.\label{art7-thm5}}%\noindent
\textit{Let $f\in C^2 (0, \infty), f (0) = 0$ and $\sum\limits^2_{r=0} |f^{(t)} (x)| \ll x^{-B}$ for some $B>2$, as $x \to + \infty$. Let $\alpha \in \bbR$ and $F(x,t;\alpha)$ be defined by the equality} 
\begin{equation}
F(x, t; \alpha) = J_{it} (x) \cos \frac{\pi}{2} (\alpha - it) - J_{-it}  (x) \cos \frac{\pi}{2} (\alpha + it) . \label{art7-eq1.38}
\end{equation}

\textit{Then we have the representation}
\begin{gather}
f(x) = - \int\limits^\infty_0 F(x, t; \alpha) \hat{f} (t;\alpha) \frac{t \; dt}{ sh (\pi t) (\text{ch}(\pi t) + \cos (\pi \alpha))} +\label{art7-eq1.39}\\
+ \sum\limits_{n>(\alpha -1) /2} J_{2n + 1-\alpha} (x) h_n (f) \notag
\end{gather}
\textit{where }
\begin{gather}
\hat{f} (t;\alpha) = \int\limits^\infty_0 F(x, t; \alpha) f(x) \frac{dx}{x}, \label{art7-eq1.40}\\
h_n(f) = 2 (2n+1-\alpha) \int\limits^\infty_0 J_{2n +1 -\alpha}  (x) f (x) \frac{dx}{x} .
\label{art7-eq1.41}
\end{gather}

\subsection{Some consequences.}\label{art7-subsec1.7}
We have an explicit from of the connection between $\rho_j(n)$ and the sum of Kloosterman sums. For this reason, we can transform the information about the Kloosterman sums into information about the Fourier coefficients of the eigenfunctions and vice versa.

The first example is the confirmation of the Linnik conjecture. The second is 

\noindent
{\bfseries Theorem \thnum{6}.\label{art7-thm6}} \textit{For any $n\geqslant 1$, as $T \to + \infty$, we have}
\begin{equation}
\sum\limits_{\chi_j \leqslant T} \alpha_j t \frac{2}{j} (n) = \frac{T^2}{\pi^2} + O(T (\log T + d^2 (n))) + O(\sqrt{nd_3 (n)} \log^2 n)  \label{art7-eq1.42}
\end{equation}
\textit{where} $\alpha_j = (\text{ch}(\pi \chi_j))^{-1} |\rho_j(1)|^2$, $d_3 (n) = \sum\limits_{d_1 d_2 d_3 = n} = \sum\limits_{d|n} d \left(\dfrac{n}{d} \right)$

The following (indirect) consequence is due to V. Bykovskij:
\begin{equation}
\sum\limits_{n \leqslant T} d (n^2 -D) = T (e_1 (D) \log T + c_0 (D)) + O_D ((T \log T)^{2/3}) \label{art7-eq1.43}
\end{equation}
where $D$ is a fixed non-square and $c_1$, $c_0$ are constants.

H. Iwaniec proved the excellent estimate for the number $\pi_\Gamma (X)$ of the conjugate primitive hyperbolic classes $\{P_0\}$ with $NP_0 < X$:
\begin{equation}
\pi_\Gamma (X) = liX + O(X^{(35/48) +\epsilon}) \text{ for any } \epsilon > 0. \label{art7-eq1.44}
\end{equation}

The proof is based essentially on the sum formulae for Kloosterman sums. 

We have some progress in the additive divisor problem (H. Iwaniec and J.-M. Deshoouillers and myself):
\begin{equation}
\sum\limits_{n \leqslant T}  d(n) d (n+N) = T P_2 (\log T, N) + O_N(T \log T)^{2/3}) \label{art7-eq1.45}
\end{equation}
where $P_2 (z, N)$ is a polynomial in $z$ of degree 2. 

\subsection{The Hecke series.}\label{art7-subsec1.8}

To each eigenfunction of the ring of Hecke operators (regular in the case of $\sM_k$\pageoriginale with $k >0$ and real analytic in the case of $\sM_0$), we associate the Dirichlet series whose $n$-th coefficient is the eigenvalue of the $n$-th Hecke operator.

As we have relations connecting the spectra of the Hecke operators with the Fourier coefficients of the eigenfunctions, these series differ, only upto normalization, from the series associated by Hecke to regular parabolic cusps by means of the Mellin transform. 

We set 
\begin{align}
\sH_{j,k} (s;x) & = \sum\limits^\infty_{n =1} e(nx) n^{-s} t_{j,k} (n), \label{art7-eq1.46}\\
\sH_{j} (s;x) & = \sum\limits^\infty_{n=1} e(nx) n^{-s} t_j(n), \notag
\end{align}
and we denote by $\sL_v(s;x)$ the Hecke series associated with the Eisenstein-Maass series $E(z,v)$,
\begin{equation}
\sL_v (s;x) = \sum\limits_{n \geqslant 1} e (nx) n^{-s} \tau_v (n). \label{art7-eq1.47}
\end{equation}

For $x =0$, these series are denoted by $\sH_{j,k} (s)$, $\sH_j(s)$, $\sL_v(s)$ respectively.

\medskip
\noindent
{\bfseries Theorem \thnum{7}.\label{art7-thm7}}
\textit{Let $x$ be rational, $x =\frac{d}{c}$ with $(d,c) =1$, $c \geqslant 1$. Then }
\begin{enumerate}
\item[(1)] $\sH_{j,k} (s, d/c)$, $\sH_j(s, d/c)$ \textit{ are entire functions of $s$,}

\item[(2)] \textit{for $v \neq \frac{1}{2}$, the only singularities of $\sL_v (s, d/c)$ are simple poles at the point $s_1 = v+\frac{1}{2}$ and $s_2 = \frac{3}{2}-v$ with residues $c^{-2v} \zeta(2v)$ and $c^{2v-2} \zeta(2-2v);$ the function $((S-1)^2 -(v-\frac{1}{2})^2) \sL_v(s, d/c)$ is an entire function of $s$.}       
\end{enumerate}
 
For what follows, it is convenient to set
\begin{equation}
\gamma (u, v) = \frac{2^{2\mu-1}}{\pi} \Gamma (u+ v - \frac{1}{2}) \Gamma (u -v + \frac{1}{2}); 
\label{art7-eq1.48}
\end{equation}
as a consequence of the functional equation for the gamma function, this function for any $u$, $v \in C$ satisfies the relation
\begin{equation}
\gamma (u,v) \gamma (1-u, v) = - (\cos^2 \pi u - \sin^2 \pi v)^{-1} . \label{art7-eq1.49}
\end{equation}

\medskip
\noindent
{\bfseries Theorem \thnum{8}.\label{art7-thm8}} The Hecke series have functional equations of the Riemann type; moreover
\begin{enumerate}
\item[1)] \textit{for even integers $k \geqslant 12$ and for $(d,c) =1$, $c \geqslant 1$, we have}
\begin{equation}
\sH_{j,k} (s, d/c) = - (4\pi /c)^{2s-1} \gamma (1-s, k/2) \cos (\pi s) \sH_{j,k} (1-s, -d'/c)
\label{art7-eq1.50}
\end{equation}
\textit{where $d'$ is defined by the congruence $dd' \equiv 1 (\mod c)$}

\item[2)] \textit{with the same $d'$,}
\begin{gather}
\sL_v (s, d/c) = (4\pi /c)^{2s-1} \gamma (1-s, v) \{-\cos (\pi s) \sL_v(1-s, -d/c) +\label{art7-eq1.51} \\
+ \sin (\pi v) \sL_v (1-s, d'/c) \}, \notag.\\
\end{gather}
\end{enumerate}


%%%%% 68 page






%{thebibliography}{99}
%\bibitem{art6-key1}
%\end{thebibliography}


%\noindent
%{\bfseries Theorem \thnum{2}.\label{art8-thm2}}
