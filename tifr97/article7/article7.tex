
\title{SUMS OF KLOOSTERMAN SUMS AND THE EIGHTH POWER MOMENT OF THE RIEMANN ZETA-FUNCTION}
\markright{SUMS OF KLOOSTERMAN SUMS AND THE EIGHTH POWER MOMENT OF THE RIEMANN ZETA-FUNCTION}

\author{By~ N. V. Kuznetsov}

\date{}
\maketitle

\setcounter{pageoriginal}{56} 

\hfill \textit{Dedicated to Atle Selberg}

\setcounter{section}{-1}
\section{Introduction.} The\pageoriginale domain of mathematics which will be discussed here was called ``Kloostermania'' by M. Huxley. This name outlined (but not too sharply) the boundary between number theory, the theory of the modular and automorphic functions, spectral theory and geometry.

The beginning was due to Poincar\'e. The contribution which defined the base of this theory was given by Petersson, Hecke, Rankin and Maass. In the last few decades, its development was stimulated by the famous talk of Atle Selberg at Tata Institute and by  L. Faddeev's work which clarified the spectral expansion.

It was ten years ago when I found the explicit form of the connection between sums of Kloosterman sums (``known quantities'') and Fourier coefficients of cusp forms (unknown quantities which are very mysterious up to this day). In the next year, R. Bruggeman rediscovered (independently) a part of these results. From that time, the number of publications is increasing rapidly in this domain. 

So it happened that the Kloosterman sums (which will be defined below) arose firstly for improving the Hardy-Littlewood ``circle method''. 

But these sums would have arisen earlier, if Poincare had wished to calculated the Fourier coefficients of the series which are called today ``the Poincar\'e series''.

The goal of this paper is to make more popular this dynamic branch of Mathematics and to demonstrate new possibilities for the Riemann zeta-functions.

The first part of the paper (the short subsection 1.1-1.13) contains known results. The second part gives, as a new consequence of the ``Kuznetsov trace formula'', the exact order for the eighth power moment of the Riemann zeta-function. Namely. we have, for some absolute constant $B > 4$.
$$
\int\limits^T_0 |\zeta (\frac{1}{2} + it)^8 dt \ll T (\log T)^{16+B} , T \to + \infty.
$$

From\pageoriginale various consequences of this estimate, one can derive the conclusion: there is a fixed constant $B$ so that 
$$
|\zeta (\frac{1}{2} + it)| \ll |t|^{1/8} \; (\log |t|)^B, |t| \to \infty, t \text{ is real.}
$$

\bigskip
\begin{center}
{\Large \textbf{Part \thnum{I}.\label{art7-partI} Sums of Kloosterman sums}}
\end{center}

\setcounter{section}{1}
\subsection{The Lobatcevskii plane.}\label{art7-subsec1.1}
This plane will be considered as the upper half plane $\bbH$ of the complex variable $z = x + iy$, $x, y \in \bbR$, $y >0$, with the metric
\begin{equation}
ds^2 = y^{-2} (dx^2 + dy^2), \label{art7-eq1.1}
\end{equation}
measure
\begin{equation}
d\mu(z) = y^{-2} dx \; dy \label{art7-eq1.2}
\end{equation}
and with the corresponding Laplace operator 
\begin{equation}
L = - y^2 \left(\frac{\partial^2}{\partial x^2} + \frac{\partial}{\partial y^2} \right).\label{art7-eq1.3}
\end{equation}

The full modular group acts on this plane in the natural way:
\begin{equation}
z \mapsto \gamma z = \frac{ax+b}{cz+d}, a, b, c, d \in \bbZ, \; ad - bc = 1. \label{art7-eq1.4}
\end{equation}

Most of the results may be developed for certain Fuchsian groups but there are no essentially new ideas for these cases; so, I restrict myself to the full modular group $\Gamma$ only. 

\subsection{The appearance of Kloosterman sums.}\label{art7-subsec1.2}

Their appearance is inescapable, if one calculates the Fourier coefficients of an automorphic function which is defined as a sum over a gorup.

For example, let us define the classical Poincar\'e series by 
\begin{equation}
P_n (z;k) = \frac{(4\pi n)^{k-1}}{\Gamma (k-1)} \sum\limits_{\gamma \in \Gamma_\infty / \Gamma} j^{-k} (\gamma, z) e (n\gamma z), n \geqslant 1 \label{art7-eq1.5}
\end{equation}
(where $\Gamma_\infty$ is generated by the translation $z \mapsto z + 1$, $j(\gamma, z) = cz + d$ if the transformation $\gamma$ is defined by a matrix $\left(\begin{smallmatrix}
\ast & \ast\\
c & d
\end{smallmatrix} \right)$ and we assume that $k$ is an even integer and $k \geqslant 4$). Then, for the $m$-th Fourier coefficient fo this series we have an almost obvious formula (the so-called ``Petersson formula''):
\begin{align}
p_{n,m} (k) : = \int\limits^1_0 P_n (z,k) e (-mx) dx \; e^{2\pi my}\label{art7-eq1.6}\\
 = \frac{(4\pi \sqrt{nm})^{k-1}}{\Gamma (k-1)} (\delta_{n,m} + 2 \pi i^{-k} \sum\limits_{c \geqslant 1} \frac{1}{c} S (n, m; c) J_{k-1} \left( \frac{4\pi \sqrt{nm}}{c}\right)), n, m \geqslant 1,
\end{align}
where\pageoriginale $J_{k-1}$ is the Bessel function of the order $k-1$ and $S$ is the Kloosterman sum
\begin{equation}
S(n, m;c): =  \sum\limits_{\substack{1\leqslant d \leqslant |c|, (d,c) =1\\ dd' \equiv 1 (\mod c)}} e \left(\frac{nd+md'}{c} \right).
\label{art7-eq1.7}
\end{equation}

We have a similar (but more complicated) representation, for the Fourier coefficients of the non-holomorphic Poincar\'e-Selberg series which, for $\re s>1$, is defined by 
\begin{equation}
U_n(z,s) : = \sum\limits_{\gamma \in \Gamma_\infty / \gamma} (\Iim \gamma z)^s e (n \gamma z), \; n \geqslant 1, \label{art7-eq1.8}
\end{equation}
(For $n=0$, it is the Eisenstein series.)

For the Kloosterman sums, we have the famous estimate due to A. Weil:
\begin{equation}
|S(n, m; c)| \leqslant (2n, 2m, c) d^2 (c) c^{1/2}. \label{art7-eq1.9}
\end{equation}

But, for the applications, we need estimates for the averages of these sums. Yu V. Linnik was the first to conjecture that Kloosterman sums oscillate regularly; his conjecture is that
\begin{equation}
\left|\sum\limits_{c \leqslant X} \frac{1}{c} S (n, m; c) \right| \ll_{n,m} X^\epsilon \label{art7-eq1.10}
\end{equation}
for every $\epsilon > 0$ as $X  \to + \infty$.

It is obvious that $A$-Weil's estimate give only $O(X^{(1/2)+\epsilon})$ on the right side and A. Selberg destroyed the hopes of any near progress in this conjecture when he constructed the counterexamples of groups for which Linnik conjecture is not valid (1963).

At this point, there was a nice result from my first paper on this subject (1977): for every fixed $\epsilon >0$, we have
\begin{equation}
\left|\sum\limits_{1 \leqslant c \leqslant X } \frac{1}{c} S (n, m;c) \right| \ll_{n,m} X^{(1/6)+\epsilon} \label{art7-eq1.11}
\end{equation}

At the same time, for the ``smoothing'' average, we have a stronger estimate : if $\varphi \in C^\infty (0,\infty)$, $\varphi =0$  outside the interval $(a, 2a)$ and if, for every fixed integer $r \geqslant 0$, we have $\left(\dfrac{\partial}{\partial x} \right)^r \varphi (x) \ll a^{-r}$, then, for every fixed $A>0$, the following estimate is valid:
\begin{equation}
\left|\sum\limits_{a \leqslant c \leqslant 2a} \varphi (c) S (n, m; c) \right| \ll a^{-A}, a \to + \infty.
\label{art7-eq1.12}
\end{equation}

Thus it is a confirmation of the Linnik conjecture.

\subsection{The eigenfunctions of the automorphic Laplacian.}\label{art7-subsec1.3}

As a\pageoriginale generalization of the classical cusp forms of even integral weight $k$ (which are regular functions on the upper half plane such that $f(\gamma z) = j^k (\gamma, z) f(z)$ for any $\gamma \in\Gamma$ and $y^{k/2} | f(z)|$ is bounded for $y >0$, the Poincar\'e series $P_n (z, k), n\geqslant 1$, being an example of a cusp form of weight $k$ with respect to full modular group), Maass introduced the non-holomorphic cusp forms (the so called Maass waves).

The Laplace operator $L$ in $L^2 (\Gamma / \bbH)$  has a continuous spectrum on the half axis $\lambda \geqslant \frac{1}{4}$ and a discrete spectrum $\lambda_0=0$, $0< \lambda_1 < \lambda_2 \leqslant \ldots$ with limit point at $\infty$ (note that $\lambda_1 \simeq 91.07\ldots$). For the case of the full modular group, there are no exceptional eigenvalues in the interval $(0, \frac{1}{4})$; (Huxley proved that the same is true for any congruence subgroup with the level $\leqslant 19$). So $L^2 (\Gamma / \bbH)$ decomposes as $L^2_{\text{eis}} (\Gamma /\bbH) \oplus L^2_{\text{cusp}} (\Gamma / \bbH)$ where $L^2_{\text{eis}}$ is the continuous  direct sum of $E(z, \frac{1}{2}+ it)$, $t \in \bbR$ ($E$ being the Eisenstein series) and $L^2_{\text{cusp}}$ is spanned by the eigenfunctions $u_j(z)$ given by 
\begin{gather*}
Lu_j \equiv - y^2 \left(\frac{\partial^2 u_j}{\partial x^2} + \frac{\partial^2 u_j}{\partial y^2} \right) = \lambda_j u_j, \label{art7-eq1.13}\\
u_j (\gamma z) = u_j  (z), \; \gamma \in \Gamma ; (u_j, u_j) = \int\limits_{\Gamma / \bbH} |u_j|^2 d\mu (z) < \infty.\notag
\end{gather*}

Any $f \in L^2 (\Gamma / \bbH)$ which is smooth enough can be expanded into eigenfunctions of $L$ and we have 
\begin{equation}
f(z) = \sum\limits_{j \geqslant 0} (f, u_j) u_j (z)  + \frac{1}{4\pi} \int\limits^\infty_{-\infty} (f, E (., \frac{1}{2} + ir)) E (z, \frac{1}{2} + ir) dr \label{art7-eq1.14}
\end{equation}
if we choose $u_j$ so that we have an orthonormal basis $\{u_j\}_{j\geqslant 0}$.

Note that the Eisenstein series has the Fourier expansion
\begin{align}
E (z,s) = y^2 + \frac{\xi (1-s)}{\xi(s)} y^{1-s} + \label{art7-eq1.15} \\
+ \frac{2\sqrt{y}}{\xi (x)} \sum\limits_{n\neq 0} \tau_s (n) e (nx) K_{s-1/2} (2\pi |n| y), \xi (s) : = \pi^{-s} \Gamma (s) \zeta (2s), \notag
\end{align}
where $K_{s-1/2} (.)$ is the modified Bessel function of order $s - \frac{1}{2}$ and 
\begin{equation}
\tau_s (n) = |n|^{s-1/2} \sigma_{1-2s(n)} = \sum\limits_{\substack{d|n\\d>0}} \left(\frac{|n|}{d^2} \right)^{s-1/2} \label{art7-eq1.16}
\end{equation} 

The\pageoriginale eigenfunctions of a point $\lambda_j$ of the discrete spectrum have a similar Fourier expansion
\begin{equation}
u_j = \sum\limits_{n \neq 0} \rho_j (n) e (nx) \sqrt{y} K_{i\chi_j} (2\pi|n| y) \label{art7-eq1.17}
\end{equation}
with $\chi_j = \sqrt{\lambda_j -\frac{1}{4}}, \lambda_j > \frac{1}{4}$.

\subsection{The Hecke operators.}\label{art7-subsec1.4}
The ideas behind Hecke operators go back to Poincare and Mordell used them to prove that Ramanujan's $\tau$-function was multiplicative.

The main observation is a simple fact that if $H$ is a subgroup if $\Gamma$ with finite index, so that $\Gamma$ is a finite coset union $\bigcup\limits_j H_{\gamma_j}$, and $f$ is automorphic on $H$, then $\sum\limits_j f(\gamma_jz)$ is automorphic on $\Gamma$.

By appropriately choosing the set of representatives, we can define the $n$-th Hecke operator as the average
\begin{equation}
(T_n f) (z) = \frac{1}{\sqrt{n}} \sum\limits_{\substack{ad =n\\d>0}} \sum\limits_{b(\mod d)} f \left(\frac{az+b}{d} \right)
\label{art7-eq1.18}
\end{equation}

For this normalization, we have
\begin{equation}
T_n T_m = \sum\limits_{d|(n, m)} T_{nm/d^2} \label{art7-eq1.19}
\end{equation}
and all these operators commute.

Now we have a set of commuting Hermitian operators, with the same set of eigenfunctions that arose for the Laplace operator. Thus we can choose the eigenfunctions of the Laplace operator, so that in the basis which was constructed from these, each Hecke operator has a diagonal form. Then these eigenfunctions will be called ``Maass waves''. For that, choose 
\begin{gather}
T_n u_j  = t_j (n) u_j, \; n \geqslant 1, j\geqslant 0,\label{art7-eq1.20}\\
T_n E (.,s) = \tau_s (n) E(., s).\label{art7-eq1.21}
\end{gather}

The eigenvalues $t_j(n)$ of the discrete spectrum of the $n$-th Hecke operator $T_n$ are connected with the Fourier coefficients of $u_j$ by the equalities 
\begin{equation}
\rho_j (1) t_j (n) = \rho_j(n), \; n \geqslant 1, j \geqslant 1. \label{art7-eq1.22}
\end{equation}

It is convenient to choose the eigenfunctions so that they will be eigenfunctions of the operator $T_{-1}$:
$$
(T_{-1} f) (z) = f(-\bar{z}).
$$

Then\pageoriginale $T_{-1} u_j =\epsilon_j u_j$ with $\epsilon_j = \pm 1$ and we have
\begin{equation}
\rho_j (-n) = \epsilon_j \rho_j (1) t_j (n), \; n \geqslant 1, j \geqslant 1.  \label{art7-eq1.23}
\end{equation}

\subsection{The sum formulae for Kloosterman sums.}\label{art7-subsec1.5}
The natural generalization of the classical Petersson formula
\begin{align}
(f,P_n) & = \int\limits_{\Gamma/\bbH} f (z) \overline{P_n (z,k)} y^k d\mu(z) \label{art7-eq1.24} \\
& = a_f (n) (= n-\text{th Fourier coefficient}) \notag
\end{align}
for any $f$ from the space $M_k$ of the regular cusp forms of weight $k$ is the same formula for the inner product $(f, U_n(., \bar{s}))$ for an automorphic $f$ from the space of cusp forms $M_0$ of weight zero.

It is not hard to show that the non-holomorphic Poincar\'e series $U_n(z,s)$ may be continued analytically (with its Fourier expansion) in the half plane $\re s > \frac{3}{4}$ (this being based on A. Weil's estimate for the Kloosterman sum). So, for $\re s_1$, $\Re s_2 > \frac{3}{4}$, the inner product $(U_n (., s_1), U_m (., \overline{s_2}))$ is well-defined. Since $U_n$ may be expressed as a sum over a group, this inner product is a sum of Kloosterman sums. On the other hand, the inner product $(u_j, U_n)$ may be calculated explicitly in terms of $\Gamma$-functions and the $n$-th Fourier coefficient of $j$-th eigenfunction $u_j$ of the automorphic Laplacian, Hence the bilinear form of $n$-th Fourier coefficients of the eigenfunctions 
$$
\sum\limits_{j \geqslant 0} \rho_j (n) \overline{\rho_j(m)} h (\chi_j)
$$
for a certain test function $h$, may be expressed as a sum of Kloosterman sums. 

Of course, we have, for the case $(U_n (., s_1), U_m (., \bar{s}_2))$, two free variables $s_1$, $s_2$ and we can try to construct an arbitrary test function in our bilinear form by integration with respect to these variables.

This plan was fulfilled in my first paper and in this way, we have following sum formula (referred to by some authors as the ``Kuznetsov trace formula'').

\medskip
\noindent
{\bfseries Theorem \thnum{1}.\label{art7-thm1}} \textit{Let us assume that the function $h(r)$ of the complex variable $r$ is regular in the strip $|\Iim r| \leqslant \delta$ with some $\delta > \frac{1}{2}$ and $|h(r)| \ll |r|^{-B}$ for some $B > 2$ when $r \to\infty$ in this strip.}


\textit{Then,\pageoriginale for any integers $n, m \geqslant 1$, we have}
\begin{gather}
\sum\limits^\infty_{j=1} \alpha_j t_j (n) t_j (m) h (\chi_j) + \frac{1}{4\pi} \int\limits^\infty_{-\infty} \tau_{(1/2) + ir} (n) \tau_{(1/2) + ir} (m) \times 
\label{art7-eq1.25} \\
\times \frac{h(r)dr}{|\zeta (1+2ir)|^2} = \notag\\
= \frac{\delta_{n,m}}{\pi^2} \int\limits^\infty_{-\infty} r \text{ th}(\pi r) h (r) dr + \sum\limits_{c \geqslant 1} \frac{1}{c} S (n, m; c) \varphi \left(\frac{4\pi \sqrt{nm}}{c} \right) , \notag
\end{gather}
\textit{where }
\begin{equation}
\alpha_j = (\text{ch} (\pi \chi_j))^{-1} |\rho_j (1)|^2,  \label{art7-eq1.26}
\end{equation}
\textit{$\zeta$ is the Riemann zeta function and for $x > 0$, the function $\varphi(x)$ is defined in terms of $h$ by the integral transform}
\begin{equation}
\varphi (x) = \frac{2i}{pi} \int\limits^{\infty}_{-\infty} J_{2 ir} (x) \frac{rh(r)}{\text{ch} (\pi r)} dr. \label{art7-eq1.27}
\end{equation}

Identity \eqref{art7-eq1.25} is modified in the following manner, if the integers, $n$, $m$ on the right-hand side have different signs.

\medskip
\noindent
{\bfseries Theorem \thnum{2}.\label{art7-thm2}} \textit{Assume that the function $h$ satisfies the conditions of the preceding theorem. Then, for any integers $n$, $m \geqslant 1$, we have}
\begin{gather}
\sum\limits_{j \geqslant 1} \epsilon_j \alpha_j t_j (n) t_j (m) h (\chi_j) + \frac{1}{4\pi} \int\limits^\infty_{-\infty} \tau_{(1/2)+ ir} (n) \tau_{(1/2) + ir} (m) \frac{h(r)dr}{|\zeta (1+2ir)|^2} = \label{art7-eq1.28}\\
= \sum\limits_{c \geqslant 1} \frac{1}{c} S (n, -m; c) \psi \left(\frac{4\pi \sqrt{nm}}{c} \right) \notag
\end{gather}
\textit{where $\psi (x)$, for $x>0$, is defined in terms of $h$ by the integral}
\begin{equation}
\psi(x) = \frac{4}{\pi^2} \int\limits^\infty_{-\infty} K_{2ir} (x) h (r), sh (\pi r) dr.  \label{art7-eq1.29}
\end{equation}

We can invert identities \eqref{art7-eq1.25} and \eqref{art7-eq1.28} and we shall assume that the sum of Kloosterman sums is given rather than the bilinear form in the Fourier coefficients.

\medskip
\noindent
{\bfseries Theorem \thnum{3}.\label{art7-thm3}} \textit{Assume\pageoriginale that to a function $\psi : [0, \infty) \to\bbC$, the integral transform}
\begin{equation}
h(r) = 2\text{ch} (\pi r) \int\limits^\infty_0 K_{2ir} (x) \psi (x)  \frac{dx}{x} \label{art7-eq1.30}
\end{equation}
\textit{associates the functions $h(r)$ satisfying the conditions of Theorem \ref{art7-thm1}. Then, for this $\psi$ and for integers $n$, $m \geqslant 1$, identity \eqref{art7-eq1.28} is satisfied, where $h$ is defined by the integral \eqref{art7-eq1.30}.}

\medskip
\noindent
{\bfseries Theorem \thnum{4}.\label{art7-thm4}}
Let $\varphi \in C^3 (0, \infty)$, $\varphi (0) = \varphi'(0) = 0$ and assume that $\varphi (x)$, together with its derivatives up to the third order, is $O(x^{-B})$ for some $B > 2$, as $x \to \infty$. Then, for any integers $n, m \geqslant 1$, we have
\begin{gather}
\sum\limits_{c \geqslant 1} \frac{1}{c} S (n, m; c) \varphi_H \left(\frac{4\pi \sqrt{nm}}{c} \right) = - \frac{\delta_{n,m}}{2\pi} \int\limits^\infty_0 J_0 (x) \varphi (x) dx +\label{art7-eq1.31}\\
+ \sum\limits_{j \geqslant 1} \alpha_j t_j (n) t_j (m) h (\chi_j) + \frac{1}{4\pi} \int\limits^\infty_{-\infty} \tau_{(1/2) + ir } (n) \tau_{(1/2) + ir} (m) \frac{h(r)dr}{|\zeta (1+2ir)|^2},\notag
\end{gather}
\textit{where the functions $\varphi_H(x)$ and $h(r)$ are defined in terms of $\varphi$ by the integral transforms}
\begin{align}
\varphi_H(x) & = \varphi(x) - 2 \sum\limits^\infty_{k=1} (2k -1) J_{2k -1} (x) \int\limits^\infty_0 J_{2k-1} (y) \varphi (y) \frac{dy}{y}. \label{art7-eq1.32}\\
h(r) & = \frac{i \pi}{2sh (\pi r)} \int\limits^\infty_0 (J_{2 ir} (x) - J_{-2ir} (X)) \varphi (x) \frac{dx}{x}. \label{art7-eq1.33}
\end{align}

It should be useful to note that the transformation $\varphi \to \varphi_H$ in \eqref{art7-eq1.32} is a projection by which, to a given $\varphi$, one associates its component orthogonal on the semiaxis $x \geqslant 0$ (with respect to the measure $x^{-1}dx$) to all the Bessel functions of odd integral order.

Together with \eqref{art7-eq1.32}, this projection can be defined by the equality
\begin{align}
\varphi_H (x) & = \varphi (x) - x \int\limits^\infty_0 \varphi (u) (\int\limits^1_0 \xi J_0 (x\xi) J_0 (u \xi) d \xi) du \label{art7-eq1.34}\\
& = \varphi (x) - x \int\limits^\infty_0 \varphi (u) \frac{x J_0 (u) J_1 (x) - u J_0 (x) J_1 (u)}{x^2 - u^2} du \notag
\end{align}
and any\pageoriginale sufficiently smooth $\varphi$ admits a decomposition
\begin{equation}
\varphi =\varphi_H + (\varphi - \varphi_H)\label{art7-eq1.35} 
\end{equation}
where $\varphi - \varphi_H$ is a combination of the Bessel functions defined by \eqref{art7-eq1.32} while $\varphi_H$ is equal to integral \eqref{art7-eq1.27}, in which by $h$ one means the integral transform \eqref{art7-eq1.33} of the function $\varphi$.

The classical Petersson formula
\begin{gather}
\sum\limits^{v_k}_{j=1} ||f_{j,k}||^{-2} t_{j,k} (n) t_{j,k} (m) = \label{art7-eq1.36}\\
= i^k \delta_{n,m} + 2 \pi \sum\limits^\infty_{c =1} \frac{1}{c} S (n, m, c) J_{k-1} \left(\frac{4\pi \sqrt{nm}}{c} \right) \notag
\end{gather}
(where $f_{j,k}$ form an orthogonal basis in the space $\sM_k$ of cusp forms of weight $k, ||f_{j,k}||^2 = (f_{j,k,} f_{j,k})$ and $v_k = \dim \sM_k$) allows us to represent the sum
\begin{equation}
\sum\limits_{c \geqslant 1} \frac{1}{c} S ((n, m; c)) \varphi \left(\frac{4\pi \sqrt{nm}}{c} \right)\label{art7-eq1.37} 
\end{equation}
as a bilinear form in the eigenvalues of the Hecke operators for the case when $\varphi$ may be represented by the Neumann series of the Bessel functions  of odd order. Together with \eqref{art7-eq1.31} this gives a representation of the sum \eqref{art7-eq1.37} as a bilinear form of the eigenvalues of the Hecke operators (in all $\sM_k$ with even $k$ and $\sM_0$) for an arbitrary ``good'' function $\varphi$.


\subsection{Some relations with Bessel functions.}\label{art7-subsec1.6}
The special case of the following expansion in terms of Bessel functions is the crucial key to prove the identities of the preceding theorems (Really our identities are consequences of a  suitable averaging of the initial identity which results from a comparison of two different expressions for the inner product $(U_n (., 1+ it), U_m (., 1 - it))$, $t \in \bbR$).

\medskip
\noindent
{\bfseries Theorem \thnum{5}.\label{art7-thm5}}%\noindent
\textit{Let $f\in C^2 (0, \infty), f (0) = 0$ and $\sum\limits^2_{r=0} |f^{(t)} (x)| \ll x^{-B}$ for some $B>2$, as $x \to + \infty$. Let $\alpha \in \bbR$ and $F(x,t;\alpha)$ be defined by the equality} 
\begin{equation}
F(x, t; \alpha) = J_{it} (x) \cos \frac{\pi}{2} (\alpha - it) - J_{-it}  (x) \cos \frac{\pi}{2} (\alpha + it) . \label{art7-eq1.38}
\end{equation}

\textit{Then we have the representation}
\begin{gather}
f(x) = - \int\limits^\infty_0 F(x, t; \alpha) \hat{f} (t;\alpha) \frac{t \; dt}{ sh (\pi t) (\text{ch}(\pi t) + \cos (\pi \alpha))} +\label{art7-eq1.39}\\
+ \sum\limits_{n>(\alpha -1) /2} J_{2n + 1-\alpha} (x) h_n (f) \notag
\end{gather}
\textit{where }
\begin{gather}
\hat{f} (t;\alpha) = \int\limits^\infty_0 F(x, t; \alpha) f(x) \frac{dx}{x}, \label{art7-eq1.40}\\
h_n(f) = 2 (2n+1-\alpha) \int\limits^\infty_0 J_{2n +1 -\alpha}  (x) f (x) \frac{dx}{x} .
\label{art7-eq1.41}
\end{gather}

\subsection{Some consequences.}\label{art7-subsec1.7}
We have an explicit from of the connection between $\rho_j(n)$ and the sum of Kloosterman sums. For this reason, we can transform the information about the Kloosterman sums into information about the Fourier coefficients of the eigenfunctions and vice versa.

The first example is the confirmation of the Linnik conjecture. The second is 

\noindent
{\bfseries Theorem \thnum{6}.\label{art7-thm6}} \textit{For any $n\geqslant 1$, as $T \to + \infty$, we have}
\begin{equation}
\sum\limits_{\chi_j \leqslant T} \alpha_j t \frac{2}{j} (n) = \frac{T^2}{\pi^2} + O(T (\log T + d^2 (n))) + O(\sqrt{nd_3 (n)} \log^2 n)  \label{art7-eq1.42}
\end{equation}
\textit{where} $\alpha_j = (\text{ch}(\pi \chi_j))^{-1} |\rho_j(1)|^2$, $d_3 (n) = \sum\limits_{d_1 d_2 d_3 = n} = \sum\limits_{d|n} d \left(\dfrac{n}{d} \right)$

The following (indirect) consequence is due to V. Bykovskij:
\begin{equation}
\sum\limits_{n \leqslant T} d (n^2 -D) = T (e_1 (D) \log T + c_0 (D)) + O_D ((T \log T)^{2/3}) \label{art7-eq1.43}
\end{equation}
where $D$ is a fixed non-square and $c_1$, $c_0$ are constants.

H. Iwaniec proved the excellent estimate for the number $\pi_\Gamma (X)$ of the conjugate primitive hyperbolic classes $\{P_0\}$ with $NP_0 < X$:
\begin{equation}
\pi_\Gamma (X) = liX + O(X^{(35/48) +\epsilon}) \text{ for any } \epsilon > 0. \label{art7-eq1.44}
\end{equation}

The proof is based essentially on the sum formulae for Kloosterman sums. 

We have some progress in the additive divisor problem (H. Iwaniec and J.-M. Deshoouillers and myself):
\begin{equation}
\sum\limits_{n \leqslant T}  d(n) d (n+N) = T P_2 (\log T, N) + O_N(T \log T)^{2/3}) \label{art7-eq1.45}
\end{equation}
where $P_2 (z, N)$ is a polynomial in $z$ of degree 2. 

\subsection{The Hecke series.}\label{art7-subsec1.8}

To each eigenfunction of the ring of Hecke operators (regular in the case of $\sM_k$\pageoriginale with $k >0$ and real analytic in the case of $\sM_0$), we associate the Dirichlet series whose $n$-th coefficient is the eigenvalue of the $n$-th Hecke operator.

As we have relations connecting the spectra of the Hecke operators with the Fourier coefficients of the eigenfunctions, these series differ, only upto normalization, from the series associated by Hecke to regular parabolic cusps by means of the Mellin transform. 

We set 
\begin{align}
\sH_{j,k} (s;x) & = \sum\limits^\infty_{n =1} e(nx) n^{-s} t_{j,k} (n), \label{art7-eq1.46}\\
\sH_{j} (s;x) & = \sum\limits^\infty_{n=1} e(nx) n^{-s} t_j(n), \notag
\end{align}
and we denote by $\sL_v(s;x)$ the Hecke series associated with the Eisenstein-Maass series $E(z,v)$,
\begin{equation}
\sL_v (s;x) = \sum\limits_{n \geqslant 1} e (nx) n^{-s} \tau_v (n). \label{art7-eq1.47}
\end{equation}

For $x =0$, these series are denoted by $\sH_{j,k} (s)$, $\sH_j(s)$, $\sL_v(s)$ respectively.

\medskip
\noindent
{\bfseries Theorem \thnum{7}.\label{art7-thm7}}
\textit{Let $x$ be rational, $x =\frac{d}{c}$ with $(d,c) =1$, $c \geqslant 1$. Then }
\begin{enumerate}
\item[(1)] $\sH_{j,k} (s, d/c)$, $\sH_j(s, d/c)$ \textit{ are entire functions of $s$,}

\item[(2)] \textit{for $v \neq \frac{1}{2}$, the only singularities of $\sL_v (s, d/c)$ are simple poles at the point $s_1 = v+\frac{1}{2}$ and $s_2 = \frac{3}{2}-v$ with residues $c^{-2v} \zeta(2v)$ and $c^{2v-2} \zeta(2-2v);$ the function $((S-1)^2 -(v-\frac{1}{2})^2) \sL_v(s, d/c)$ is an entire function of $s$.}       
\end{enumerate}
 
For what follows, it is convenient to set
\begin{equation}
\gamma (u, v) = \frac{2^{2\mu-1}}{\pi} \Gamma (u+ v - \frac{1}{2}) \Gamma (u -v + \frac{1}{2}); 
\label{art7-eq1.48}
\end{equation}
as a consequence of the functional equation for the gamma function, this function for any $u$, $v \in C$ satisfies the relation
\begin{equation}
\gamma (u,v) \gamma (1-u, v) = - (\cos^2 \pi u - \sin^2 \pi v)^{-1} . \label{art7-eq1.49}
\end{equation}

\medskip
\noindent
{\bfseries Theorem \thnum{8}.\label{art7-thm8}} The Hecke series have functional equations of the Riemann type; moreover
\begin{enumerate}
\item[1)] \textit{for even integers $k \geqslant 12$ and for $(d,c) =1$, $c \geqslant 1$, we have}
\begin{equation}
\sH_{j,k} (s, d/c) = - (4\pi /c)^{2s-1} \gamma (1-s, k/2) \cos (\pi s) \sH_{j,k} (1-s, -d'/c)
\label{art7-eq1.50}
\end{equation}
\textit{where $d'$ is defined by the congruence $dd' \equiv 1 (\mod c)$}

\item[2)] \textit{with the same $d'$,}
\begin{gather}
\sL_v (s, d/c) = (4\pi /c)^{2s-1} \gamma (1-s, v) \{-\cos (\pi s) \sL_v(1-s, -d/c) +\label{art7-eq1.51} \\
+ \sin (\pi v) \sL_v (1-s, d'/c) \}, \notag\\
\sH_j(s, d/c) = (4\pi/c)^{2c-1} \gamma (1-s,\frac{1}{2} + i\chi_j)  \label{art7-eq1.52}\\
\{-\cos (\pi s) \sH_j(1-s, d'/c) + \epsilon_j \text{ch} (\pi \chi_j) \sH_j (1-s, d/c)\}. \notag
\end{gather}\pageoriginale
\end{enumerate}

We conclude with the simple but important consequence of the multiplicative relations \eqref{art7-eq1.19} for Hecke operators : for $\re s>1 + |\re  v- \frac{1}{2}|$, we have
\begin{equation}
\sum\limits^\infty_{n=1} \frac{\tau_v (n) t_j (n)}{n^s} = \frac{1}{\zeta(2s)} \sH_j(s+v-\frac{1}{2}) \sH_j(s - v + \frac{1}{2}).
\label{art7-eq1.53}
\end{equation}

If we replace $t_j(n)$ by $\tau_\mu(n)$ (that corresponds to the continuous spectrum of the Hecke operators) then the well-known Ramanujan identity will arise instead of \eqref{art7-eq1.53}:
\begin{gather}
\sum\limits^\infty_{n=1} \frac{\tau_v(n) \tau_\mu(n)}{n^s} =  \label{art7-eq1.54}\\
=\frac{1}{\zeta(2s)} \zeta(s+v -\mu) \zeta(s+\mu-v) \zeta(s-v-\mu +1) \zeta(s+v+\mu -1). \notag
\end{gather}

For this reason, equality \eqref{art7-eq1.53} is a direct generalization of the Ramanujan identity; both will be essential for the estimate of the eighth moment of the Riemann zeta-function.

\subsection{The spectral mean of Hecke series.}\label{art7-subsec1.9}
Let  $N\geqslant 1$ be an integer and let $s$, $v$ be complex variables. We set 
\begin{align}
Z^{(d)}_N (s, v; h) &= \sum\limits_{j \geqslant 1}  \alpha_j t_j (N) \sH_j (s+v -\frac{1}{2}) \sH_j(s-v+\frac{1}{2}) h (\chi_j) \label{art7-eq1.55}\\
Z^{(d)}_N (s, v; h) & = \sum\limits_{j \geqslant 1} \epsilon_j \alpha_j t_j (N) \sH_j (s+v -\frac{1}{2}) \sH_j (s-v+\frac{1}{2}) h (\chi_j) \label{art7-eq1.56}
\end{align}
(with $\alpha_j = (\text{ch}(\pi\chi_j))^{-1} |\rho_j(1)|^2$). Here the summation is over the positive discrete spectrum of the automorphic Laplacian and one assumes that its eigenfunctions have been selected in such a manner that they are at the same time eigenfunctions of the ring of Hecke operators and of the reflection operator $T_{-1} (\epsilon_j = \pm 1$ are the eigenvalues of $T_{-1}$).

Further, we define the square mean of the Hecke series over the continuous spectrum by the equality
{\fontsize{10}{12}\selectfont
\begin{align}
&Z^{(c)}_N(s,v;h) = \label{art7-eq1.57} \\
&=\frac{1}{\pi} \int\limits^\infty_{-\infty} \frac{\zeta (s+v-\frac{1}{2} +ir) \zeta(s+v-\frac{1}{2}-ir) \zeta (s-v+\frac{1}{2} + ir) \zeta(s-v+\frac{1}{2} -ir)}{\zeta (1+2ir) \zeta(1-2ir)} \notag\\
& \times \tau_{(1/2) + ir} (N) h(r) dr \notag
\end{align}}\relax\pageoriginale
with the stipulation that, by means of integral \eqref{art7-eq1.57}, the function $Z^{(c)}_N$ is defined under the conditions
$$
\re(s+v-\frac{1}{2}) < 1, \re (s-v + \frac{1}{2}) < 1.
$$

If any one of the points $s \pm (v - \frac{1}{2})$ lies to the right hand side of the unit line, then the integral \eqref{art7-eq1.57} defines another function, connected with $Z^{(c)}_N$ by the Sokhotskii formulae. Fro example, if by $\tilde{Z}^{(c)}_N$, we denote the function which is defined by \eqref{art7-eq1.57} with $\re s >1$, $\re v = \frac{1}{2}$, then a simple computation gives
\begin{gather*}
\tilde{Z}_N^{(c)} (s,v;h) = Z^{(c)}_N (s, v; h) + 4 \zeta_N(s,v ) h (i(s-v-\frac{1}{2})) + \\
+ 4\zeta_N (s, 1-v) h(i(s+v-\frac{3}{2}))
\end{gather*}
where we have introduced the notation
$$
\xi_N (s,v) = \frac{\zeta (2s-1) \zeta(2v)}{\zeta(2-2s+2v)} \tau_{s-v} (N)
$$
and the regularity strip of $h$ is assumed to be sufficiently wide for the right hand side to make sense.

Now we need the mean with respect to the weights of the Hecke series associated with regular cusp forms. For an integer $k \geqslant 1$, we set
\begin{gather}
Z_{N,k} (s,v) = 2 (-1)^k\frac{\Gamma (2k-1)}{(4\pi)^{2k}} \sum\limits^{v_{2k}}_{j=1} |\alpha_{j,2k} (1)|^2 t_{j,2k} (N) \label{art7-eq1.58}\\
\times \sH_{j, 2k} (s+v-\frac{1}{2}) \sH_{j, 2k} (s-v+\frac{1}{2})\notag
\end{gather}
where $t_{j, 2k} (N)$ is the eigenvalue of the N-th Hecke operator in the space $\sM_{2k}$ of regular cusp forms of weight $2k$, $v_{2k} = \dim \sM_{2k}$; the empty sum for $1 \leqslant k \leqslant 5$ and $k=7$ is assumed to be equal to zero.

Assume now that $h^\ast =\{h_{2k-1}\}^\infty_{k=1}$ is a sequence of sufficiently fast decreasing numbers; we define the mean of the Hecke series with respect to weights by the equality
\begin{equation}
Z^{(p)}_N (s, v; h^\ast) = \sum\limits_{k \geqslant 6} h_{2k-1} Z_{N,k} (s,v)\label{art7-eq1.59}
\end{equation}

\subsection{The convolution formula.}\label{art7-subsec1.10}
Some of\pageoriginale the consequences of the algebra of modular forms are the so called ``exact formulae'', an example of which is the identity
\begin{equation}
\sum\limits^{N-1}_{n=1} \sigma_3 (n) \sigma_3 (N-n) = \frac{1}{120} (\sigma_7(N)-\sigma_3(N)), \sigma_a (n) = \sum\limits_{d|n} d^^a \label{art7-eq1.60}
\end{equation}
A source of similar identities is the obvious assertion that the product of modular forms of weight $k$ and $l$ is a modular form of weight $k+l$.

There are analogues of these identities for the real analytic Eisenstein series of weight zero. For an integer $N\geqslant 1$, we associate to a pair of series $E(z,s)$ and $E(z,v)$ the expression of convolution type 
\begin{gather}
W_N(s,v; w_0, w_1) = N^{s-1} \sum\limits^\infty_{n=1} \tau_v (n) (\sigma_{1-2s} (n-N)w_0(\sqrt{n/N})\\
+ \sigma_{1-2s} (n+N) w_1 (\sqrt{n/N})) \notag
\end{gather}
where $\sigma_{1-2s} (0)$ means $\zeta(2s-1)$ and $w_0, w_1$ are assumed to be sufficiently smooth and sufficiently fast decreasing for $x \to + \infty$.

\medskip
\noindent
{\bfseries Theorem \thnum{9}.\label{art7-thm9}} 
\textit{Assume that the functions $w_0$, $w_1$ are continuous on the semiaxis $x \geqslant 0$ together with derivatives up to the fourth order, $w_j (0) = w'_j(0) = 0$ for $j=0,1$ and that, for $x \to + \infty$, the functions $w_j(x)$ as well as their derivatives up to the third order are $O(x^{-B})$ for some $B>4$.}

\textit{Then, for any integer $N \geqslant 1$ and $s$, $v \in \bbC$ satisfying $\re v = \frac{1}{2},\frac{1}{2} < \re S<1$, we have}
\begin{align}
& W_N(s, v; w_0; w_1) = Z^{(d)}_N (s, v; h_0) + Z^{(d)}_N (s,v;h_1) + \label{art7-eq1.62} \\
& Z^{(c)}_N (s,v; h_0 +h_1) + Z^{(p)}_N (s, v; h^\ast) + \zeta_N (s,v)V (\frac{1}{2},v) + \notag\\
& \zeta_N (s, 1-v) V (\frac{1}{2}, 1-v) + \zeta_N (1-s,v) V(s,v) + \zeta_N(1-s, 1-v) |\times \notag\\
& V (s, 1-v) \notag
\end{align}
\textit{where}
\begin{gather}
\zeta_N(s,v) = \frac{\zeta(2s) \zeta(2v)}{\zeta(2s+2v)} \tau_{s+v} (N), \label{art7-eq1.63}\\
V(s,v) = 2 \int\limits^\infty_0 (|1-x^2|^{1-2s}) w_0 (x) \to (1+x^2)^{1-2s} w_1(x)) x^{2v} dx \label{art7-eq1.64}
\end{gather}
\textit{and the column vector $h(r;s, v) = \left(\begin{matrix}
h_0 \\h_1\end{matrix}\right)$ is defined in terms of $w = \left(\begin{matrix}
w_0 \\ w_1 \end{matrix}\right)$ by the integral transform}
\begin{gather}
h(r; s, v) = \pi \int\limits^\infty_0 
\begin{pmatrix} 
k_0 (x, \frac{1}{2} + ir) & 0\\
0 & k_1 (x, \frac{1}{2} + ir)
\end{pmatrix}
\begin{pmatrix}
\frac{x}{4\pi}
\end{pmatrix}^{2s-1} \times \label{art7-eq1.65}\\
 \int\limits^\infty_0 
\begin{pmatrix}
k_0 (xy, v) & k_1 (xy, v) \\
k_1 (xy, v) & k_0 (xy, v)
\end{pmatrix} w(y) y dy dx \notag
\end{gather}\pageoriginale
\textit{with the kernels}
\begin{align}
k_0 (x,v) & = \frac{J_{2v-1} (x) - J_{1-2v} (x)}{2 \cos (\pi v)},\label{art7-eq1.66}\\
k_1 (x,v) & = \frac{2}{\pi} \sin (\pi v) K_{2v -1} (x). \notag
\end{align}

\textit{Finally, the coefficients of the mean of the regular forms $Z^{(\rho)}_N$ are given by the relations}
\begin{gather*}
h_{2k-1} = 2 (2k-1) \times \\
\times \int\limits^\infty_0 J_{2k-1} (x) 
\left(\frac{x}{4\pi} \right)^{2s-1} \int\limits^{\infty}_0 (k_0 (xy, v) w_0 (y)+ k_1 (xy, v) w_1 (y) ) y dy \; dx. 
\end{gather*}

\subsection{Some consequences of the convolution formula.}\label{art7-subsec1.11}
The first example of the use of \eqref{art7-eq1.62} is the additive divisor problem; if we choose $s = v =1/2$, $w_1 =0$ and $w_0$ so that it is close to 1 in the interval $(0, \sqrt{T/N})$ (so that $w_0 (\sqrt{n/N})$ will be close to 1 for $n \leqslant T$), then the left hand side of \eqref{art7-eq1.62} gives the sum on the left side of \eqref{art7-eq1.45}. Terms with the integral \eqref{art7-eq1.64} are leading terms and all other terms give the remainder term. 

Of course, the asymptotic formula for the additive divisor problem is crucial for the investigation of the fourth power moment of the Riemann zeta-function. A consequence of \eqref{art7-eq1.62} in this direction is the following 

\begin{theorem*}
(N. Zavorotnyi, 1987). Let $T \to + \infty$; then, for any $\epsilon > 0$, we have
\begin{equation}
\int\limits^T_0 |\zeta(\frac{1}{2} + it)|^4 dt = TP_4 (\log T) + O(T^{2/3 + \epsilon}) \label{art7-eq1.67}
\end{equation}
\textit{where $P_4 (z)$ is a polynomial in $z$ of the fourth degree with constant coefficients.}
\end{theorem*}

We\pageoriginale can consider the functions $h_0$ and $h_1$ in \eqref{art7-eq1.62} as given; the following unusual integral transform is useful to invert \eqref{art7-eq1.62}.

Let us define the matrix kernel $\bbK (x,v)$ by the equality
\begin{equation}
\bbK (x,v) = 
\begin{pmatrix}
k_0 (x,v) & k_1 (x,v) \\
k_1(x,v) & k_0 (x,v)
\end{pmatrix} \label{art7-eq1.68}
\end{equation}
with $k_0$, $k_1$ from \eqref{art7-eq1.66}. Now we shall consider the matrix equation
\begin{equation}
w(x) = \int\limits^\infty_0 \bbK (xy, v) u (y) \sqrt{xy} dy \label{art7-eq1.69}
\end{equation}
where $w = \begin{pmatrix}
w_0 \\
w_1
\end{pmatrix} (x)$, $u = 
\begin{pmatrix}
u_0 \\
u_1
\end{pmatrix} (x)$. 

\medskip
\noindent
{\bfseries Theorem \thnum{10}.\label{art7-thm10}}
\textit{Let $\re v =\frac{1}{2}$ and $w \in L^2 (0,\infty)$ in the sense that $w_0, w_1'\in L^2 (0,\infty)$. Then there exists a unique solution $u$ in $L^2 (0,\infty)$ of the equation \eqref{art7-eq1.69} and this solution is given by the formula}
\begin{equation}
u(x) = \int\limits^\infty_0 \bbK (xy, u) w(y) \sqrt{xy} dy \label{art7-1.70}
\end{equation}
\textit{where the integral is understood in the mean-square sense.}

Now, as a special case of the convolution formula \eqref{art7-eq1.62}, we have the following asymptotic formulae.

\medskip
\noindent
{\bfseries Theorem \thnum{11}.\label{art7-thm11}}
\textit{Let $T \to + \infty$. Then for a fixed $\sigma$ and $t \in \bbR$ with $\frac{1}{2} < \sigma < 1$, we have}
\begin{equation}
\sum\limits_{\chi_j \leqslant T} \alpha_j |\sH_j (\sigma +it)|^2 =\frac{T^2}{\pi^2} (\zeta (2\sigma) + \frac{\zeta(2-2\sigma)}{2(1-\sigma)} \left(\frac{T^2}{2\pi} \right)^{1-2\sigma} ) + O(T\log T)  \label{art7-eq1.71}
\end{equation}
\textit{while, for $\sigma =\frac{1}{2}$ the right-hand side has to be replaced by}
\begin{equation}
\frac{2T^2}{\pi^2} (\log T +2 \gamma - 1 +  2 log (2\pi)) + O(T \log T). \label{art7-eq1.72}
\end{equation}

\subsection{The explicit formulae for the transformation \eqref{art7-eq1.65}}\label{art7-eq1.12}

We rewrite equality \eqref{art7-eq1.65} in the form
\begin{equation}
h = \int\limits^\infty_0 A(r, y; s, v)
\begin{pmatrix}
w_0 \\
w_1
\end{pmatrix} 
(y) dy, A = 
\begin{pmatrix}
A_{00} & A_{01}\\
A_{10} & A_{11}
\end{pmatrix},  \label{art7-eq1.73}
\end{equation}
where the matrix kernel $A$, under the conditions $\re v = \frac{1}{2}$, $\frac{1}{2} < \re s < 1$, $|Iim r| < \re s$,\pageoriginale is determined by the integrals that appear in the term-by-term integration in \eqref{art7-eq1.65}. All these integrals are given in tables (these are the Weber-Schafheitlin integrals); we will be needing the following explicit form for the kernels $A_{j,l}, j$, $l=0,1$.

\medskip
\noindent
{\bfseries Proposition \thnum{1}.\label{art7-prop1}}
\textit{Let us denote, for $\re v = \frac{1}{2},\frac{1}{2} < \re s < 1$, $|\Iim r| < \re s$}
\begin{align}
a & = s+v - \frac{1}{2} + ir, \; b = s - v + \frac{1}{2} + ir, \; c = 1 + 2 ir, \label{art7-eq1.74}\\
a' & = s + v =-\frac{1}{2} - ir, \; b' = s-v +\frac{1}{2} - ir, \; c' = 1-2ir. \label{art7-eq1.75}
\end{align}
\textit{Then, for $0< y < 1$, we have}
\begin{gather}
(2\pi)^{2s-1} A_{00}(r,y;s,v) : = \pi y \int\limits^\infty_0 k_0 (x, \frac{1}{2} + ir) k_0 (xy, v) (\frac{x}{2})^{2s-1} dx\label{art7-eq1.76}\\
= \frac{1}{2\cos (\pi v)} \left\{\frac{\Gamma (a) \Gamma (a')}{\Gamma (2v)} y^{2v} F(a, a'; 2v; y^2) \sin \pi (s+v) + \right. \hspace{1.5cm}\notag\\
\hspace{1.5cm} \left. +\frac{\Gamma (b) \Gamma (b')}{\Gamma (2-2v)} y^{2-2v} F (b, b'; 2-2v; y^2) \sin \pi (s-v)  \right\} \notag
\end{gather}
\textit{and for $y>1$,}
\begin{align}
& (2\pi)^{2s-1} A_{00} (r, y; s, v)= \label{art7-eq1.77}\\
& \; = \frac{iy^{1-2s}}{2sh (\pi r)} \left\{\frac{\Gamma (a) \Gamma (b)}{y^{c-1} \Gamma (c)} F \left(a, b; c; \frac{1}{y^2} \right) \cos \pi (s+ir) - \right. \notag\\
& \qquad \left. - \frac{\Gamma (a') \Gamma (b')}{y^{c'-1} \Gamma (c')}  F \left(a', b'; c'; \frac{1}{y^2} \right)  \cos \pi (s-ir)\right\} \notag
\end{align}

\textit{At the same time, for all $y>0$, we have}
\begin{gather}
(2\pi)^{2s-1} A_{00} (r, y; s, v) = \sin (\pi s) \Gamma (2s-1) y^{2v} |1-y^2|^{1-2s} \times \label{art7-eq1.78}\\
F(1-b, 1-b'; 2 - 2s; 1-y^2) + \frac{\Gamma (a) \Gamma (a') \Gamma (b) \Gamma (b')}{2\pi \Gamma (2s) \cos (\pi s)} \times \notag\\
(\text{ch}^2 \pi r + \sin^2 \pi v - \sin^2 \pi s) y^{2v} F (a, a'; 2s; 1-y^2) \notag 
\end{gather}
\textit{where, in the first term, the absolute value sign combines the two cases $y<1$ and $y>1$.}

\medskip
\noindent
{\bfseries Proposition \thnum{2}.\label{art7-prop2}}
With\pageoriginale the same parameters, we have
\begin{gather}
(2\pi)^{2s-1} A_{01} (r, y; s, v) : = \pi y \int\limits^\infty_0 k_0 (x, \frac{1}{2} + ir)  k_1 (xy, v) (\frac{x}{2})^{2s-1} dx \label{art7-eq1.79}\\
= \frac{iy^{1-2s} \sin (\pi v)}{2\text{sh} (\pi r)} \left\{\frac{\Gamma (a) \Gamma (b)}{y^{c-1} \Gamma (c)} F \left(a, b; c; -\frac{1}{y^2} \right) \right.\notag\\
\left. -\frac{\Gamma (a') \Gamma (b')}{y^{c'-1} \Gamma (c')} F \left(a',b '; c'; -\frac{1}{y^2} \right) \right\} \notag
\end{gather}

\medskip
\noindent
{\bfseries Proposition \thnum{3}.\label{art7-prop3}}
\textit{The kernel $A_{10}$ is defined by the relation}
\begin{gather}
(2\pi)^{2s-1} A_{10} (r,y;s,v) : = \pi y \int\limits^\infty_0 k_1 (x, \frac{1}{2} + ir) k_1 (xy, v) (\frac{x}{2})^{2s-1} dx\label{art7-eq1.80}\\
=\frac{\Gamma (a) \Gamma (a') \Gamma (b) \Gamma (b')}{\pi \Gamma (2s)} \text{ ch} (\pi r) \sin (\pi v) y^{2v} F (a, a'; 2s; 1-y^2).  \notag
\end{gather}

\medskip
\noindent
{\bfseries Proposition \thnum{4}.\label{art7-prop4}} 
\textit{With the parameters \eqref{art7-eq1.74}-\eqref{art7-eq1.75}, we have}
\begin{gather}
(2\pi)^{2s-1} A_{11} (r,y;s,v) : = \pi y \int\limits^\infty_0 k_1 (x, \frac{1}{2} + ir) k_0 (xy,v) (\frac{x}{2})^{2s-1} dx =\notag\\
= \frac{\text{ch} (\pi r)}{2\cos (\pi v)} \left\{\frac{\Gamma (a) \Gamma (a')}{\Gamma (2v)} y^{2v} F(a, a'; 2v; -y^2) - \right. \notag\\
\left. -\frac{\Gamma (b) \Gamma (b')}{\Gamma (2-2v)} y^{2-2v} F(b,b'; 2 -2 v; -y^2)  \right\} \label{art7-eq1.81}
\end{gather}
\textit{and, at the same time,}
\begin{align}
& (2\pi)^{2s-1} A_{11} (r, y; s, v) = \label{art7-eq1.82}\\
& = \frac{iy^{1-2s}}{2\text{sh}(\pi r)} \left\{\frac{\Gamma (a) \Gamma (b)}{y^{c-1} \Gamma (c)} F \left(a, b; c; - \frac{1}{y^2} \right) \cos \pi(s+ir) - \right. \notag\\
& \left. -\frac{\Gamma (a') \Gamma (b')}{ y^{c'-1} \Gamma (c')} F \left(a'; b'; c'; - \frac{1}{y^2} \right) \cos \pi (s-ir)  \right\}. \notag
\end{align}


\medskip
\noindent
{\bfseries Proposition \thnum{5}.\label{art7-prop5}} 
\textit{Let us write the quantities $h_{2k-1}$ in \eqref{art7-eq1.62} as}
\begin{equation}
h_{2k-1} (s,v) = \int\limits^\infty_0 (A^0_k (y;s, v) w_0 (y) + A^1_k (y;s,v) w_1 (y)) dy. \label{art7-eq1.83}
\end{equation}
\textit{Then, for $0<y<1$,}
\begin{align}
&(2\pi)^{2s-1} A^0_k (y; s, v) =\label{art7-eq1.84}\\
&\frac{2k-1}{\cos (\pi v)} \left\{\frac{\Gamma (s+ v-1+k)}{\Gamma (2v) \Gamma (1-s-v+k)} y^{2v} F (k+s+v -1, s+v-k; 2 v ; y^2) - \right.\notag \\
& \left. -\frac{\Gamma (k+s-v)}{\Gamma (2-2v)\Gamma (v-s+k)} y^{2-2v} F(k+s-v, s+1-v-k; 2-2v; y^2) \right\} \notag
\end{align}
\textit{and, for $y>1$,}
\begin{align}
& (2\pi)^{2s-1} A^0_k (y;s,n) = \hspace{2cm} \label{art7-eq1.85} \\
& \hspace{2cm} \frac{(-1)^{k-1} \sin (\pi s) }{\pi y^{2k+2s-2}} \frac{\Gamma (k+s+v-1) \Gamma(k+s-v)}{\Gamma (2k-1)} \times \notag\\
& \hspace{3cm} F \left(k+s + v -1 , k + s-v; 2k; \frac{1}{y^2} \right). \notag
\end{align}

\textit{For the second kernel in \eqref{art7-eq1.83}, we have}
\begin{align}
&(2\pi)^{2s-1} A^1_k (y; s, v) = \hspace{2cm}\\
&\hspace{2cm} = - \frac{\sin (\pi v)}{\pi y^{2k+ 2s -2}} \frac{\Gamma (k+ s+v -1) \Gamma (k+s-v)}{ \Gamma (2k-1)} \times  \notag\\
&\hspace{3cm} F\left(k + s + v -1, \; k + s- v; 2k; - \frac{1}{y^2} \right) \notag
\end{align}

\bigskip
\begin{center}
{\Large \textbf{Part \thnum{II}.\label{art7-partII} The eighth moment of the Riemann zeta-function.}}
\end{center}

\setcounter{section}{2}
\setcounter{subsection}{0}
\subsection{The result and a rough sketch of the proof.}\label{art7-sec2.1}
Since the question about the true order of zeta-function on the critical line is open even today - and it will be so in the foreseeable future-, a sizeable part of the theory of the Riemann zeta-function is an attempt to present the asymptotic mean value
\setcounter{equation}{0}
\begin{equation}
\frac{1}{T} \int\limits^T_0 |\zeta (\frac{1}{2} + it)|^{2k} dt, \; k = 1, 2, \ldots \label{art7-eq2.1}
\end{equation}

The case $k=4$ will be investigated here; for this case, the following new estimate will be given. 

\medskip
\noindent
{\bfseries Theorem \thnum{12}.\label{art7-thm12}}
\textit{There\pageoriginale is an absolute constant $B$ such that}
\begin{equation}
\int\limits^T_0 |\zeta (\frac{1}{2} + it)|^8 dt \ll T (\log T)^B \label{art7-eq2.2}
\end{equation}
\textit{when $T \to + \infty$.}

Furthermore, the same estimate is valid for the fourth power moment of the Hecke series of the discrete spectrum of the automorphic Laplacian. Namely, we have

\medskip
\noindent
{\bfseries Theorem  \thnum{13}.\label{art7-thm13}}
\textit{For every fixed $j\geqslant 1$ with the same $B$ as in \eqref{art7-eq2.2}, we have }
\begin{equation}
\alpha_j \int\limits^T_0 |\sH_j (\frac{1}{2} + it)|^4 dt  \ll T (\log T)^{B-6},T \to + \infty.  \label{art7-eq2.3}
\end{equation}

To give these estimates we shall consider the fourth spectral moment of the Hecke series over the discrete and continuous spectrum. The one over the discrete spectrum is defined by 
\begin{gather}
Z^{\text{dis}} 
\left(
\left.
\begin{matrix}
s, v\\
\rho, \mu
\end{matrix}
\right| h
\right) = 
\label{art7-eq2.4}\\
\sum\limits_{j \geqslant 1} \alpha_j \frac{\sH_j (s+v -\frac{1}{2}) \sH_j (s- v + \frac{1}{2}) \sH_j (\rho + \mu - \frac{1}{2}) \sH_j(\rho -\mu +\frac{1}{2}) }{\zeta(2s) \zeta(2\rho)} h(\chi_j) \notag
\end{gather}
with
\begin{equation}
\alpha_j = (\text{ch} (\pi \chi_j))^{-1} |\rho_j(1)|^2. \label{art7-eq2.5}
\end{equation}

Of course, this function results from the following summation (see the generalized Ramanujan identity \eqref{art7-eq1.53}):
\begin{equation}
Z^{\text{dis}} 
\left(
\left.
\begin{matrix}
s, v\\
\rho, \mu
\end{matrix}
\right| h
\right)  = \sum\limits_{n, m \geqslant 1} \frac{\tau_v (n)}{n^s} \frac{\tau_\mu(m)}{m^\rho} \left(\sum\limits_{j \geqslant 1} \alpha_j t_j (n) t_j (m) h(\chi_j) \right)
\label{art7-eq2.6}
\end{equation}
when $\re s > \re v -\frac{1}{2}| +1$, $\re \rho > |\re \mu - \frac{1}{2}| + 1$. The function 
$$
\zeta(2s) \zeta(2 \rho) Z^{\text{dis}}
\left(
\left.
\begin{matrix}
s, v\\
\rho, \mu
\end{matrix}
\right| h
\right) 
$$
is regular in some domain of the kind $\re s > s_0$' $\re \rho > \rho_0$, where $s_0' \rho_0$ depend on the order of decay of the function $|h(r)|$ for $|r| \to + \infty$. 

We shall denote by $\tilde{Z}^{\text{dis}} \left(
\left.
\begin{matrix}
s, v\\
\rho, \mu
\end{matrix}
\right| h
\right) $ the expression obtained on replacing $\alpha_j$ by $\epsilon_j \alpha_j$ in \eqref{art7-eq2.6}:
\begin{equation}
\tilde{Z}^{\text{dis}} \left(
\left.
\begin{matrix}
s, v\\
\rho, \mu
\end{matrix}
\right| h
\right)  = \sum\limits_{n,m \geqslant 1} \frac{\tau_v(n)}{n^2} \frac{\tau_\mu(m)}{m^\rho} \left(\sum\limits_{j \geqslant 1} \epsilon_j \alpha_j t_j (n) t_j (m) h(\chi_j) \right) \label{art7-eq2.7}
\end{equation}

In the same manner as in \eqref{art7-eq2.6}, we define the fourth spectral moment over the continuous spectrum by
\begin{equation}
Z^{\text{con}}_0 
\left(
\left.
\begin{matrix}
s, v\\
\rho, \mu
\end{matrix}
\right| h
\right)  = \frac{1}{4\pi} \int\limits^\infty_{-\infty} Z (s; v,\frac{1}{2} + ir) Z (\rho ; \mu, \frac{1}{2} + ir) \frac{h(r) dr}{\zeta(1+ 2 ir) \zeta (1-2ir)} \label{art7-eq2.8}
\end{equation}
where we assume $\re (s \pm (v-\frac{1}{2})) >1$, $\re (\rho \pm (\mu - \frac{1}{2})) >1$ and the notation $z$ is introduced for the right side of the well-known Ramanujan identity:
\begin{align}
z(s; v, \mu) & = \sum\limits^\infty_{n=1} \frac{\tau_v (n) \tau_\mu (n)}{n^s}\label{art7-eq2.9}\\
& = \frac{1}{\zeta(2s)} \zeta(s+v - \mu) \zeta(s-v+\mu) \zeta (s+v+ \mu-1) \times \notag
\end{align}
$\zeta(s-v-\mu+1)$.

Finally, for the given sequence $h^\ast =\{h_{2k-1}\}^\infty_{k=1}$ we define the fourth moment of the Hecke series over regular cusp forms 
\begin{gather}
Z^{\text{cusp}} \left(
\left.
\begin{matrix}
s, v\\
\rho, \mu
\end{matrix}
\right| h^\ast
\right)  =  2 \sum\limits^\infty_{\substack{k=2\\k \equiv 0 (\mod 2)}} (-1)^{k/2} h_{k-1} \label{art7-eq2.10}\\
\times \sum\limits^{v_k}_{j=1} \alpha_{j,k} \sum\limits_{n, m\geqslant 1} \frac{\tau_v (n) \tau_\mu (m)}{n^s m^\rho} t_{j,k} (n) t_{j,k} (m), \notag
\end{gather}
where $v_k =\dim \sM_k$ is the dimension of the space $\sM_k$ of the regular cusp forms of weight $k$ and $t_{j,k} (n)$ are the eigenvalues of the $n$-th Hecke operators in this space $\sM_k$. The quantities $\alpha_{j,k}$ are the normalized coefficients; if the functions $f_{j,k}$ form an orthonormal basis in $\sM_k$, then 
\begin{equation}
\alpha_{j,k} = \frac{\Gamma (K-1)}{(4\pi)^k} |a_{j,k} (1)|^2, \label{art7-eq2.11}
\end{equation}
where $a_{j, k} (1)$ is the first Fourier coefficient of $f_{j,k}$. Together with \eqref{art7-eq2.10}, we have 
\begin{gather}
Z^{\text{cusp}} 
\left(\left.
\begin{matrix}
s, & v \\
\rho,& v
\end{matrix}
\right|h^\ast\right) = 2 \sum\limits_{\substack{k\geqslant 2\\ k \equiv 0 (\mod 2)}} (-1)^{k/2} h_{k-1} \sum\limits^{v_k}_{j=1} \alpha_{j, k} \times \label{art7-eq2.12}\\
\frac{\sH_{j,k} (s+v -\frac{1}{2}) \sH_{j,k} (s-v+\frac{1}{2}) \sH_{j,k} (\rho + \mu - \frac{1}{2}) \sH_{j,k} (\rho-\mu +\frac{1}{2})}{\zeta(2s) \zeta (2\rho)} ; \notag
\end{gather}
this\pageoriginale equality is quite similar to \eqref{art7-eq2.4}.

Now we shall describe the main idea. We shall consider the double Dirichlet series 
\begin{equation}
L^{(\pm)} 
\left(\left.
\begin{matrix}
s, & v \\
\rho,& v
\end{matrix}
\right|\varphi\right)  = \sum\limits^{\infty}_{n,m =1} \frac{\tau_v (n) \tau_\mu (m)}{n^s m^\rho} K^{(\pm)}_{m,n} (\varphi), \label{art7-eq2.13}
\end{equation}
where the coefficients are sums of Kloosterman sums with the smooth ``test'' function $\varphi$:
\begin{equation}
K^{(\pm)}_{m,n} (\varphi) = \sum\limits_{c \geqslant 1 } \frac{1}{c} S (n, \pm m; c) \varphi \left(\frac{4\pi \sqrt{nm}}{c} \right) . 
\label{art7-eq2.14}
\end{equation}


Since there is a functional equation of the Riemann type for the Dirichlet series 
$$
\sL_v \left(S, \frac{a}{c} \right) = \sum\limits^\infty_{n=1} n^{-s} \tau_v (n) e (n\frac{a}{c}), e(x) : = e^{2\pi i x}, 
$$
when $a$, $c \in \bbZ$ are coprime and this equation connects $\sL_v (S, \frac{a}{c})$ with $\sL_v (1-s, \pm \frac{a'}{c'})$, $aa'\equiv 1 (\mod c)$, a functional equation has to exist for the functions $L^{(\pm)} \left(\left.
\begin{matrix}
s, & v \\
\rho,& \mu
\end{matrix}
\right| \varphi \right)$; it will connect this function with the functions of the same kind $L^{(\pm)} \left(\left.\begin{matrix}
\rho ,& v\\
s, & \mu
\end{matrix}
\right| \varphi \right) $  for an appropriate $\varphi$. As a consequence of the sum formula for Kloosterman sums, it means that there is a functional equation for the function
\begin{equation}
Z\left(\left.\begin{matrix}
s, & v\\
\rho, & \mu
\end{matrix}
\right|  h \right)   = Z^{\text{dis}}
\left(\left.\begin{matrix}
s, & v\\
\rho, & \mu
\end{matrix}
\right| h \right)  + Z^{\text{con}}
\left( 
\left. 
\begin{matrix}
s, & v\\
\rho, & \mu
\end{matrix}
\right| h
\right); \label{art7-eq2.15}
\end{equation}
roughly speaking, this equation (which will be written below in detail) is the result of the exchange of $s$ and $\rho$ and the replacement of $h$ by some integral transform.

Now, on the left side of this functional equation, for the special case
\begin{equation}
s = \mu =\frac{1}{2}, \rho = v = \frac{1}{2} + it, t \in \bbR \text{ is large and positive, }\label{art7-eq2.16}
\end{equation}
we have in the continuous spectrum the product
\begin{equation}
\zeta^3 (\frac{1}{2} + it + ir) \zeta^3 (\frac{1}{2} + it - ir) \zeta (\frac{1}{2} - it + ir) \zeta(\frac{1}{2} - it - ir) \label{art7-eq2.17}
\end{equation}
and the other product will be on the right side; namely, we have therein 
\begin{equation}
\zeta^3 (\frac{1}{2} + ir) \zeta^3 (\frac{1}{2} - ir) \zeta (\frac{1}{2} + 2 it + 2 ir) \zeta (\frac{1}{2} + 2 it - 2 ir). 
\label{art7-eq2.18}
\end{equation}
After this specialization, we shall choose the special function $h$ essentially as $\exp (-\alpha r^2)$ with a fixed positive $\alpha$. Then the essential part of the interval of the integration is $|r| \ll (\log t)^{(1/2)}$. The length of this interval is small in comparison to the large $t$ and for this reason, (using the Riemann functional equation), we can reduce the main term of our product on the left side to the form 
$$
|\zeta^4 (\frac{1}{2} + it + ir) \zeta^4 (\frac{1}{2} + it - ir)|. 
$$\pageoriginale
It means that we may hope to estimate the integral 
\begin{equation}
\int\limits^\epsilon_{-\epsilon} \int\limits^\infty_{0} \omega_T (t) |\zeta^4 (\frac{1}{2} + it+ ir) \zeta^4 (\frac{1}{2} + it - ir)| dt r^2 dr  \label{art7-eq2.19}
\end{equation}
for arbitrary small positive $\epsilon$, if $\omega_T (t)$ is the smooth function which is not zero for $t \in (T, 2 T)$ only and close to 1 when $\frac{5}{4} T \leqslant t \leqslant \frac{7}{4} T$ (see picture).
\begin{center}
{\bf figure}
\end{center}
The main term will be close to the integral
$$
\epsilon^3 \int\limits^\infty_0 \omega_T (t) |\zeta (\frac{1}{2} + it)|^\infty dt 
$$
if $\epsilon \ll (\log T)^{-2}$ (see subsection \eqref{art7-eq2.4}); so we have the eighth moment of the Riemann zeta-function here. At the same time, the contribution of the discrete spectrum is positive too. Hence the desired conclusion follows if the integrals on the right side can be estimated with sufficient accuracy. 

But the integrand for these integrals contains one Hecke series only; so the integration may be done asymptotically. As a result, we shall reduce the problem of the estimate of the eighth power moment to the problem of the estimate for the fourth spectral moment. It is sufficient to prove  our theorem for the latter. 

In the conclusion of the introduction of the second part, we shall note that the estimate 
$$
|\zeta(\frac{1}{2} + it)| \ll  |t|^{1/8} (\log |t|)^{B_1} ,, t \to \pm \infty, 
$$
follows from \eqref{art7-eq2.2}, $B_1 = \frac{1}{8} B + \frac{1}{2}$.

It is preceptibly better than the last achievement in the long chain of the results of the kind $|\zeta(\frac{1}{2} + it) \ll |t|^\gamma$.


\subsection{The first functional equation}\label{art7-subsec2.2}
Since the sum $Z^{\text{dis}} + Z^{\text{con}}$ is connected with the sum of Kloosterman sums, we shall consider the triple sum 
\setcounter{section}{3}
\setcounter{equation}{0}
\begin{equation}
L^{(\pm)} 
\left( 
\left. 
\begin{matrix}
s, & v \\
\rho, & \mu 
\end{matrix}
\right| \varphi 
\right) = \sum\limits^\infty_{n, m=1} 
\frac{\tau_v (n) \tau_\mu (m)}{n^s m^\rho} K^{(\pm)}_{n,m} (\varphi).
\label{art7-eq3.1}
\end{equation}

Here the notation \eqref{art7-eq2.14} is used and we assume that a function $\varphi$ is ``good'' namely, the Mellin transform $\hat{\varphi} (u)$
\begin{equation}
\hat{\varphi} (u) = \int\limits^\infty_0 \varphi (x) x^{u-1} d x, \label{art7-eq3.2}
\end{equation}
is regular in the strip $-\alpha_0 \leqslant \re u \leqslant \alpha_1$ with positive $\alpha_0, \alpha_1$ and $|\hat{\varphi} (u)|$ decreases sufficiently rapidly in this strip. For this reason, we can write 
\begin{equation}
\varphi (x) = \frac{1}{i\pi} \int\limits^\infty_{i\pi} \int\limits^\infty_{(\alpha)} \hat{\varphi} (2u) x^{-2u} du, x >0, \label{art7-eq3.3}
\end{equation}
where $\int\limits^\infty_{(\alpha)}$ stands for the integral over the line $\re u = \alpha$. As we have 
$$
|S(n,m;c)| \ll c^{1/2} (n,m,c) d (c),
$$
the triple sum \eqref{art7-eq3.1} converges absolutely if $\alpha < - \frac{1}{4}$ and both $\re (s+\alpha)$, $\re (\rho + \alpha)$ are larger than 1. For this case, we have, for the sum \eqref{art7-eq3.1}, the following expression
\begin{gather}
L^{(\pm)}
\left( 
\left. 
\begin{matrix}
s, & v \\
\rho, & \mu 
\end{matrix}
\right| \varphi 
\right)=
\frac{1}{i\pi} \sum\limits_{c \geqslant 1} \frac{1}{c} \times \label{art7-eq3.4} \\
\sum\limits_{ad \equiv 1 (\mod c)} \int\limits_{(\alpha)} \sL_v (s+ u, \frac{a}{c}) \sL_\mu (\rho + \mu, \pm \frac{d}{c}) (c/4\pi)^{2u} \hat{\varphi} (2\mu) du. \notag
\end{gather}
Now we shall integrate over the line $\re u = \alpha_0$ where $\alpha_0$ will be chosen so that both $\re (s+\alpha_0)$, $\re(\rho+ \alpha_0)$ are negative. Taking into account the contribution from the poles, we have 
{\fontsize{10}{12}\selectfont
\begin{align}
& L^{(\pm)} 
\left( 
\left. 
\begin{matrix}
s, & v \\
\rho, & \mu 
\end{matrix}
\right| \varphi 
\right)\\
& = 2 \sum\limits_{c \geqslant 1} \sum\limits_{ad \equiv 1 (\mod c)} \left\{\frac{(4\pi)^{2s-1}}{c^{2s}} \left(\frac{\zeta (2v)}{(4\pi)^{2v}} \sL_\mu (\rho - s+ v + \frac{1}{2}, \pm d/c) \hat{\varphi} (2v+ 1-2s) + \right. \right. \notag\\
& \left. + \frac{\zeta (2-2v)}{(4\pi)^{2-2v}} \sL_\mu (\rho -s + \frac{3}{2} -v, \pm d/c) \hat{\varphi} (3-2v-2s)  \right) +\notag\\
& + \frac{(4\pi)^{2\rho-1}}{c^{2\rho}} \left(\left(\frac{\zeta (2\mu)}{(4\pi)^{2\mu}} \sL_v  (s-\rho + \mu + \frac{1}{2}, a/c) \right. \right) \times 
\notag\\
&\left.\left. \hat{\varphi} (2\mu+ 1-2\rho) + \frac{\zeta(2-2\mu)}{(4\pi)^{2-2\mu}} \sL_v (s-\rho + \frac{3}{2} -\mu, a/c) \hat{\varphi} (3-2\mu-2\rho) \right)\right\}  + \notag\\
&+  \frac{1}{i\pi} \sum\limits_{c \geqslant 1} \frac{1}{c} \sum\limits_{ad \equiv 1 (\mod c)} \int\limits_{\alpha_0} \sL_v (s+u, a/c) \sL_\mu (\rho + u, \pm d/c) (\frac{c}{4\pi})^{2u} \hat{\varphi} (2u) du. \notag
\end{align}}\pageoriginale
In the last term, we shall use the functional equation \eqref{art7-eq1.51}. If the sign ``plus'' is taken, then it gives for our sum the expression
\begin{gather}
\sum\limits_{c \geqslant 1} \frac{1}{c} \sum\limits_{n, m \geqslant 1} \frac{\tau_v (n) \tau_\mu (m)}{n^\rho m^s} (S(n, m; c) \Phi_0 \left(\frac{4\pi \sqrt{nm}}{c} \right) + \label{art7-eq3.6}\\
\left. + S( n, -m; c) \Phi_1 \left(\frac{4\pi \sqrt{nm}}{c} \right) \right) \notag
\end{gather}
where 
\begin{gather}
\Phi_0 = \Phi_0 (x; s, v; \rho, \mu ) = \label{art7-eq3.7}\\
=\frac{1}{i\pi} x^{2s+ 2\rho-2} \int\limits_{(\alpha_0)} \gamma (1-s-u, v) \gamma (1-\rho - u, \mu) \times \notag\\
(\cos \pi (s+ \mu) \cos \pi (\rho + u) + \sin (\pi v) \sin (\pi \mu)) x^{2u } \hat{\varphi} (2u) du, \notag\\
\Phi_1 =\Phi_1 (x; s, v; \rho, \mu) = \label{art7-eq3.8}\\
= -\frac{1}{i\pi} x^{2s+ 2 \rho -2} \int\limits_{(\alpha_0)} \gamma (1-s-u, v) \gamma (1-\rho-u, \mu) (\sin (\pi \mu) \times \notag\\
\cos \pi( s+ u) + \sin (\pi v) \cos \pi (\rho + \mu) x^{2u} \hat{\varphi} (2u)   du, \notag
\end{gather}

Of course, when the sign ``minus'' is taken in \eqref{art7-eq3.5}, then the same function s $\Phi_0$ and $\Phi_1$ are the coefficients, but $\Phi_0$ will occur with $S(n,-m;c)$ and $\Phi_1$ with $S (n, m;c)$.

We have $\Phi_j(x) = O (x^{2 \min (\re s, \re \rho)})$ as $x \to 0 + $ and these functions are bounded when $x$ is large. For this reason, the triple sums in \eqref{art7-eq3.6} converge absolutely and we can again interchange the order of the summations. Hence we have, in \eqref{art7-eq3.7}, the sum 
\begin{equation}
L^{(+)}
\left( 
\left. 
\begin{matrix}
\rho, & v\\
s, & \mu
\end{matrix}
\right| \Phi_0 
\right) + L^{(-)}
\left( 
\left. 
\begin{matrix}
\rho, & v\\
s, & \mu
\end{matrix}
\right| \Phi_1 
\right)\label{art7-eq3.9}
\end{equation}
for the case ``+'' on the left side \eqref{art7-eq3.5} and 
\begin{equation}
L^{(+)}
\left( 
\left. 
\begin{matrix}
\rho, & v\\
s, & \mu
\end{matrix}
\right| \Phi_1 
\right) + L^{(-)}
\left( 
\left. 
\begin{matrix}
\rho, & v\\
s, & \mu
\end{matrix}
\right| \Phi_0 
\right)
\label{art7-eq3.10}
\end{equation}
for the other case.

Now we are ready to give the first functional equation.

\medskip
\noindent
{\bfseries Theorem \thnum{14}.\label{art7-thm14}}
\textit{Let $\re v = \re \mu =\frac{1}{2}$ and let, for some positive $\delta < \frac{1}{4}$, the variables $s$, $\rho$ satisfy $\frac{5}{4} < \re s$, $\re \rho < \frac{5}{4} + \delta$. Let $\varphi : [0, \infty) \to \bbC$ have the Mellin transform $\hat{\varphi} (u)$ such that $\hat{\varphi} (u)$ is regular for $-\frac{3}{2} - 2 \delta \leqslant \re u \leqslant 2$. Then we have}
\begin{align}
&L^{(+)}
\left( 
\left. 
\begin{matrix}
s, & v\\
\rho , & \mu
\end{matrix}
\right| \varphi 
\right) = L^{(+)}
\left( 
\left. 
\begin{matrix}
\rho, & v\\
s, & \mu
\end{matrix}
\right| \varphi_0
\right) L^{(-)}
\left( 
\left. 
\begin{matrix}
\rho, & v\\
s, & \mu
\end{matrix}
\right| \varphi_1
\right) +\label{art7-eq3.11}\\
&+ \frac{2(4\pi)^{2s-1}}{\zeta(2s)} \left(\frac{\zeta (2v)}{(4\pi)^{2v}} z(\rho + v; s, \mu) \hat{\varphi} (2v+ 1 - 2 s) +  \right. \notag\\
& \qquad +\left. \frac{\zeta(2-2v)}{(4\pi)^{2-2v}} z (\rho + 1 - v ; s, \mu) \hat{\varphi} (3-2v-2s) \right)+ \notag\\
& + \frac{2(4\pi)^{2\rho-1}}{\zeta(2\rho)} \left(\frac{\zeta (2(\mu)}{(4\pi)^{2\mu}} z (s+\mu; \rho, v) \hat{\varphi}  (2\mu + 1 - 2 \rho) + \right. \notag\\
& \qquad + \left. \frac{\zeta (2-2\mu)}{(4\pi)^{2-2\mu}} z (s+1-\mu; \rho, v) \hat{\varphi} (3-2\mu -2 rho)\right), \notag
\end{align}
\textit{where $\Phi_0$, $\Phi_1$ are defined by the following integral transformations}
\begin{align}
&\Phi_0 (x) \equiv \Phi_0 (x; s, v; \rho, \mu) = x^{2s+2\rho-2} \iint\limits^\infty_0 (k_0 (\xi, v) k_0 (\eta, \mu) + \label{art7-eq3.12}\\
& \qquad k_1 (\xi, v) k_1 (\eta, \mu)) \varphi (\xi \eta / x) \xi^{1-2s} \eta^{1-2\rho} d \xi d \eta \notag\\
& \Phi_1 (x) \equiv \Phi_1 (x; s , v; \rho, \mu) = x^{2s+ 2\rho -2} \iint\limits^{\infty}_0 (k_0 (\xi, v)k_0 (\eta, \mu) + \label{art7-eq3.13}\\
& \qquad k_1 (\xi, v) k_0 (\eta, \mu)) \varphi  (\xi \eta / x) \xi^{1-2s} \eta^{1-2\rho} d \xi d \eta.  \notag
\end{align}

Of course, it is the same as what we have in \eqref{art7-eq3.5}. If $\re \rho > \re s$, then in the first term on the right side of \eqref{art7-eq3.5}, one has the sum 
\begin{align}
\sum\limits_{c \geqslant 1} \frac{1}{c^{2s}} \sum\limits^\infty_{n=1} \frac{\tau_\mu(n)}{n^{\rho-s+v+(1/2)}} S(0, n ; c) = \label{art7-eq3.14}\\
= \frac{1}{\zeta(2s)} \sum\limits^\infty_{n=1} \frac{\tau_\mu (n) \sigma_{1-2s } (n)}{n^{\rho -s + v + (1/2)}} = \frac{z(\rho + v; s, \mu)}{ \zeta(2s)} \notag
\end{align}
On the\pageoriginale right side, we have a meromorphic function of $\rho$ in the half-plane $\re \rho > \frac{1}{2}$; so this equality holds for the analytic continuation of the initial sum 
\begin{equation}
\sum\limits_{c \geqslant 1} \frac{1}{c^{2s}} \sum\limits_{(d,c) =1} \sL_\mu \left(\rho - s+ v + \frac{1}{2}, \frac{d}{c} \right) \label{art7-eq3.15}
\end{equation}
if we can be sure that this function is meromorphic not only for $\re \rho > \re s$. It is sufficient for this to know that $\sL_\mu(w,\frac{d}{c})$ as a function of $c$ is bounded in the mean when $\re w > \frac{1}{2}$ (except at the poles). But this fact is a consequence of the Bombieri-Vinogradov inequality which asserts that
\begin{equation}
\sum\limits_{1 \leqslant  c \leqslant M} \sum\limits_{(d, c) =1}
\left|\sum\limits^{P+Q})_{n=P} b(n) e (\frac{nd}{c}) \right|^2 \ll \max (Q, M^2) \sum\limits^{P+Q}_{n=P} |b(n)|^2 \label{art7-eq3.16}
\end{equation}
for an arbitrary sequence of complex numbers $b(n)$.

Now one can check that relations \eqref{art7-eq3.12} - \eqref{art7-eq3.13} and \eqref{art7-eq3.7} - \eqref{art7-eq3.8} are idential. We have the tabular integrals 
\begin{align}
& \int\limits^\infty_0 k_0(x, v) x^{w-1} dx = \gamma \left(\frac{w}{2}, v \right)  \cos \left(\frac{\pi w}{2} \right), \; 0 < \re w < \frac{3}{2}, \label{art7-eq3.17}\\ 
& \int\limits^\infty_0 k_1 (x,v) x^{w-1} dx = \gamma \left(\frac{w}{2}, v \right) \sin (\pi v), \re w > 0, \label{art7-eq3.18}
\end{align}

After writing $\varphi$ in \eqref{art7-eq3.12}-\eqref{art7-eq3.13} as the Mellin integral, 
$$
\varphi \left(\frac{\xi \eta}{x} \right) = \frac{1}{i\pi} \int \hat{\varphi} (2 u) \left(\frac{x}{\xi \eta} \right)^{2u} du, 
$$
we shall come to an absolutely convergent triple integral if 
\begin{equation*}
\max \left(\frac{3}{4} - \re s, \frac{3}{4} - \re \rho \right) <\re u < \min (1-\re s, 1 -\re \rho). 
\end{equation*}
Hence there is a non-empty strip where we can integrate in any order; this gives our relations for $\Phi_0$ and $\Phi_1$.

\setcounter{section}{2}
\setcounter{equation}{0}
\subsection{The main functional equation: the preparations.}\label{art7-subsec2.3}

For the given function $h(r)$ and the sequence $h^\ast : = \{h_{2k-1}\}^\infty_{k-1}$, we shall consider the function
\setcounter{section}{4}
\setcounter{equation}{0}
\begin{align}
Z 
\left( 
\left. 
\begin{matrix}
s, & v\\
\rho, &\mu
\end{matrix}
\right| h
\right) = Z^{\text{dis}}
\left( 
\left. 
\begin{matrix}
s, & v\\
\rho, &\mu
\end{matrix}
\right| h
\right) + Z^{\text{con}}_0
\left( 
\left. 
\begin{matrix}
s, & v\\
\rho, &\mu
\end{matrix}
\right| h
\right) +Z^{\text{cusp}}
\left( 
\left. 
\begin{matrix}
s, & v\\
\rho, &\mu
\end{matrix}
\right| h^\ast
\right).
\label{art7-eq4.1}
\end{align}

When\pageoriginale $\re s$, $\re \rho>1$, this function is equal to 
\begin{align}
& \sum\limits^\infty_{n, m=1} \frac{\tau_v (n) \tau_\mu (m)}{n^s m^\rho} \{K^{(+)}_{n,m} (\varphi) + \frac{\delta_{n,m}}{\pi^2} \int\limits^\infty_{-\infty} r \text{th} (\pi r) h(r) dr \} = \label{art7-eq4.2}\\
&  \qquad =  L^{(+)} 
\left( 
\left. 
\begin{matrix}
s, & v\\
\rho, &\mu
\end{matrix}
\right| \varphi
\right) + \frac{1}{\pi^2} z (s+ \rho; v, \mu) \int\limits^\infty_{-\infty} r \text{th} (\pi r) h(r) dr \notag
\end{align}
where $\varphi$ corresponds to $h$ and $h^\ast$ in the sense of Theorems \ref{art7-thm1} and \ref{art7-thm4}.

Our intention must be clear now; we shall use the first functional equation for $L^{(+)}$ and after this, the analytic  continuation of both sides will be carried out. 

Firstly, it is convenient to write the analytic continuation for the function $Z^{\text{con}}_0$. Let us denote by $Z^{\text{con}}$ the integral in which (under the usual conditions $\re v = \re \mu =\frac{1}{2}$) we have $\re s < 1$, $\re \rho <1$. Then $Z^{\text{con}}_0$ and $Z^{\text{con}}$ are connected by the following relation.

\medskip
\noindent
{\bfseries Proposition \thnum{6}.\label{art7-prop6}}
\textit{Let $h$ be a regular function on the sufficiently wide strip $|\Iim r | \leqslant \Delta$, $\Delta > \frac{1}{2}$. Then for $\re s > 1$, $\re \rho < 1$, the meromorphic continuation of $Z^{\text{con}}_0 
\left( 
\left. 
\begin{matrix}
s, & v\\
\rho, &\mu
\end{matrix}
\right| h
\right)$ is given by the equality.} 
\begin{align}
&Z^{\text{con}}_0 
\left( 
\left. 
\begin{matrix}
s, & v\\
\rho, &\mu
\end{matrix}
\right| h
\right) - Z^{\text{con}}
\left( 
\left. 
\begin{matrix}
s, & v\\
\rho, &\mu
\end{matrix}
\right| h
\right) + \label{art7-eq4.3}\\
& ~~~+ \frac{\zeta(2s-1)}{\zeta(2s)} \left\{\frac{\zeta(2v) z (\rho; \mu, 1 -s + v)}{\zeta(2+ 2 v - 2s)} h (i(s-v-\frac{1}{2})) \right\}  +
\notag \\
& \qquad + \frac{\zeta (2-2v) z (\rho ;\mu, 2 - s -v)}{\zeta(4-2v-2s)} h(i (s+ v - \frac{3}{2})) + \notag \\
& ~~~  +\frac{\zeta (2 \rho -1)}{\zeta (2\rho)}  \left\{ \frac{\zeta (2\mu) z (s; v , 1 - \rho + \mu)}{\zeta (z+ 2 \mu - 2 \rho)} h (i(\rho - \mu -\frac{1}{2})) + \right.\notag\\
& \qquad \left. + \frac{\zeta(2-2\mu) z (s; v, 2 - \rho -\mu)}{\zeta(4-2\mu - 2 \rho)} h (i (\rho + \mu - \frac{3}{2})) \right\}. \notag
\end{align}

Really $Z^{\text{con}}_0$ is a Cauchy integral, because $\zeta$ has only a simple pole. so the poles of $z(s; v, \frac{1}{2} + ir)$ are the points $r_j$, $1\leqslant j \leqslant 4$, with
$$
ir_1 = \frac{1}{2} - s+ v, \;  ir_2 =\frac{3}{2} - s - v, \; r_3 = - r_1, \; r_4 = -r_2.
$$
When $\re s > 1$ the points $r_1$, $r_2$ are lying above the real axis and if $\re s < 1$, they are below the same. Now one can deform the path of integration (see picture; the deformation must be so small that the functions\pageoriginale  $\zeta( 1+ \pm 2 ir)$ have no zeros inside the lines; it is possible, since the Riemann zeta-function has no zeros on the line $\re s =1$) and the desired conclusion is the result of the direct calculation of the residues. 
\begin{center}
\textbf{\bf figure}
\end{center}

The next step is the representation of the functions $L^{(\pm)} \left( 
\left. 
\begin{matrix}
\rho, & v\\
s, & \mu
\end{matrix}
\right| \Phi_j
\right)$ as a bilinear form in the eigenvalues of the Hecke operators. For this, we need to consider the integral transforms of Theorems \eqref{art7-thm4} and \eqref{art7-thm1}. The situation now is the following : for a given $h$, we define $\varphi$ by the transformation \eqref{art7-eq1.27}, or what is the same, by the equality 
\begin{equation}
\varphi(x) = \frac{1}{\pi} \int\limits^\infty_{-\infty} k_0 (x, \frac{1}{2} + iu) u \text{th} (\pi u) h (u) du,  \label{art7-eq4.4}
\end{equation}
and thereafter, we should calculate the integrals $\Phi_0$ and $\Phi_1$ in \eqref{art7-eq3.12} and \eqref{art7-eq3.13}  with this $\varphi$ and finally the integral transformations
\begin{align}
& h_0(r) \equiv h_0 (r; s , v; \rho, \mu)  = \pi \int\limits^\infty_0 k_0 (x,\frac{1}{2} + ir) \Phi_0(x) \frac{dx}{x}, \label{art7-eq4.5}\\
& h_1 (r) \equiv h_1 (r; s, v; \rho, \mu) = \pi \int\limits^\infty_0 k_1 (x, \frac{1}{2} + ir) \Phi_1 (x) \frac{dx}{x}. \label{art7-eq4.6}
\end{align}

In order to obtain an asymptotic estimate, it is preferable to diminish the length of the sequence of these integral transformations; we give the results in the following 

\medskip
\noindent
{\bfseries Proposition \thnum{7}.\label{art7-prop7}}
\textit{Assume that the function $h(u)$ is even and regular in the strip $|\Iim u| \leqslant \frac{3}{2}$ and $h$ has zeros at $u = \pm \frac{i}{2}$. Let $|h|$ decrease as $O(|u|^{-B})$ for some $B >4$ when $|u| \to \infty$ with $|\Iim u| \leqslant \frac{3}{2}$. Then the function $h_0$ is given by the integral transform}
\begin{equation}
h_0(r) = \frac{2}{\pi^2} \int\limits^\infty_{-\infty} B_0 (r; u; \rho, v, \mu; s) u \text{ th} (\pi u) h(u) du \label{art7-eq4.7}
\end{equation}
\textit{where, with the notation from \eqref{art7-eq1.78}, \eqref{art7-eq1.79}, we have, for $\re v = \re \mu = \frac{1}{2}$, $\frac{1}{2} \leqslant \re s$, $\re \rho < 1 $}
\begin{gather}
B_0 (r, u; \rho, v, \mu; s) = \int\limits^\infty_0 (A_{00} (r, \xi, \rho, v) A_{00} (u, \frac{1}{\xi}; 1 - \rho, \mu) +\label{art7-eq4.8} \\
A_{01} (r, \xi; \rho, v) A_{01} (u, \frac{1}{\xi} ; 1 - \rho, \mu)) \xi^{2\rho - 2 s-1} d \xi \notag
\end{gather}\pageoriginale 
and here 
\begin{equation}
B_0 (r, u; \rho, v, \mu; s) = B_0 (r, u; s, \mu, v; \rho). \label{art7-eq4.9}
\end{equation}

\medskip
\noindent
{\bfseries Proposition \thnum{8}.\label{art7-prop8}}
\textit{Under the same conditions}
\begin{equation}
h_1(r) = \frac{2}{\pi^2} \int\limits^\infty_{-\infty} B_1 (r, u; s, \mu, v; \rho) u \text{ th}  (\pi u) h(u) du \label{art7-eq4.10}
\end{equation}
\textit{where, with the notation \eqref{art7-eq1.78}, \eqref{art7-eq1.80},}
\begin{align}
& B_1 (r, u; s , \mu, v; \rho) = B_1(r,u; \rho, v, \mu; s) = \label{art7-eq4.11}\\
& = \int\limits^\infty_0 (A_{10} (r, \xi, s, \mu) A_{00} (u, \frac{1}{\xi}; 1- s, v) + \notag\\
& + A_{11} (r, \xi, s, \mu) A_{01} (u, \frac{1}{\xi}; 1 - s , v) ) \xi^{2s-2\rho -1} d\xi. \notag
\end{align}

Both the propositions result from term-by-term integration in the corresponding multiple integrals; it is sufficient to consider the first relation \eqref{art7-eq4.7}.

First of all, the function $\varphi$ in \eqref{art7-eq4.4}, for our case, is $O(x^3)$ when $x \to 0$ and $O(x^{-(1/2)})$ for $x \to + \infty$. Furthermore, the Mellin transform of this function, which is defined by the integral
\begin{align}
\hat{\varphi} (w) : & =  \int\limits^\infty_0 \varphi (x) x^{w-1}   dx = \label{art7-eq4.12}\\
& = \frac{2}{\pi} \cos \left(\frac{\pi w}{2} \right) \int\limits^\infty_{-\infty} \gamma \left(\frac{w}{2}, \frac{1}{2} + iu \right) u \text{ th} (\pi u) h(u) du \notag
\end{align}
is regular for $\re w > -3$ and $|\hat{\varphi} (w)|$ may be estimated as $O(|w|^{\re w-1})$ when $|w | \to \infty$ and $\re w$ is fixed. 

For this reason, the integrals \eqref{art7-eq3.7} - \eqref{art7-eq3.8} are absolutely convergent if $\alpha_0 < \re (s + \rho) -1$. At the same time, both the integrals with $k_j (x, \frac{1}{2} + ir) \times x^{2s+ 2 \rho  + 2 u -3}$ for $j =0,1$ are absolutely convergent for $1-\re (s+ \rho ) < \re u < \frac{5}{4} - \re (s+ \rho)$. If $\re (s+ \rho) > \frac{1}{2}$, we can choose $\alpha_0$ in such a manner that the term-by-term integration would be valid in the integrals which will arise on replacing $\Phi_j$ in \eqref{art7-eq4.5} and \eqref{art7-eq4.6} by the representations \eqref{art7-eq3.7} and \eqref{art7-eq3.8}. In this way, we have 
\begin{align}
& h_0(r) = \frac{2i}{\pi^2} \int\limits^\infty_{-\infty} u \text{th }(\pi u) h (u) \times \\
& \times \infty_{(\alpha_0)} \gamma (s+ \rho+ w -1,\frac{1}{2} + ir) \gamma (1-s-w, v) \gamma (1-\rho - w, \mu) \gamma (w, \frac{1}{2} 
+ iu) \times \\
& \quad \times \cos (\pi w) \sin \pi (s+ \rho + w) (\cos\pi (s+w) \sin (\pi v) +\\
& \qquad \qquad + \cos \pi (\rho + w) \sin (\pi \mu) ) dw du. 
\end{align}
After this, it is sufficient to check that two representations are identical for $\re s$, $\re \rho <1$; but this results immediately from the explicit formulae for the Mellin transforms of the kernels $k_j$ and the definitions of the kernels $A_{k,l}$.

To finish the preparations, it remains to write the coefficients in the sum over the regular cusps for the sum of Kloosterman sums with weight function $\Phi_0$ and, finally, to consider the analytic continuation of the function $Z^{\text{con}}_0 
\left(
\left.
\begin{matrix}
\rho, & v \\
s, & \mu 
\end{matrix}
\right| h_0 + h_1
\right)$.

The first is not difficult; it is sufficient to do the formal substitution $r = i (k -\frac{1}{2})$ in the expression for $h_0(r)$ and to note the well-known limiting case
\begin{align}
\lim\limits_{c \to - 2k } (\Gamma (c))^{-1} F (a, b; c; z) = \frac{\Gamma (a+ 2 k + 1) \Gamma (b+ 2 k + 1)}{\Gamma (2k + 2)} \times \label{art7-eq4.14}\\
\times z^{2k + 1} F (a+ 2k + 1, b + 2 k + 1 ; 2 k + 2; z) \notag
\end{align}
which holds for a positive integer $k$.

The analytic continuation is given by the same kind of relation as in \eqref{art7-eq4.3}; so it is sufficient to calculate the values $(h_0 + h_1)(i (s-1) \pm (\mu - \frac{1}{2} ))$ and  $(h_0 + h_1) (i(\rho -1) \pm (v - \frac{1}{2}))$.

\medskip
\noindent
{\bfseries Proposition \thnum{9}.\label{art7-prop9}}
\textit{Let $h$ be the same function as in Proposition \eqref{art7-prop7}; then, for $\frac{1}{2} < \re s$, $\re \rho < 1$, $\re v = \re \mu = \frac{1}{2}$, we have}
\begin{equation}
(h_0 + h_1) (i(\rho - v - \frac{1}{2})) = \frac{2}{\pi^2} \int\limits^\infty_{-\infty} u \text{ th } (\pi u) h(u) \tilde{B}_0 (u; \rho, \mu, s -v) du \label{art7-eq4.15}
\end{equation}
\textit{where}
\begin{align}
& \tilde{B}_0 (u; \rho, \mu, w) = (2\pi)^{1-2\rho} \sin (\pi \rho) \Gamma  (2 \rho - 1) \times \label{art7-eq4.16}\\
& \qquad \times  \int\limits^\infty_{0} (|1-\xi^{2}|^{1-2 \rho} A_{00} (u, 1 / \xi; 1 -  \rho, \mu) + \notag\\
& + (1+ \xi^2)^{1-2\rho} A_{01} (u. 1/\xi; 1- \rho, \mu)) \xi^{2\rho - 2 w-1} d \xi. \notag
\end{align}\pageoriginale

This relation is a consequence of the explicit formulae for the kernels $A_{k,l}$. If $r = i (\rho - v - \frac{1}{2})$, then, in these formulae, we have
$$
a = 2 v , b =1, c = 2-2\rho + 2 v; a'=2\rho -1, b' =2 \rho - 2 v , c' = 2 \rho - 2 v. 
$$
Now, we have, for the special case of the hypergeometric functions, $F(0, b; c; z) \equiv 1$ and $F(a, b; b; z) = (1-z)^{-a}$ and as a result, we have the following equalities 
\begin{gather}
A_{00} (i(\rho - v - \frac{1}{2}), \xi; \rho, v) + A_{10}  (i( \rho - v - \frac{1}{2}) , \xi ; \rho, v)\label{art7-eq4.17}\\
= (2\pi)^{1-2\rho} \sin (\pi \rho) \Gamma (2 \rho -1) |\xi^{2} -1|^{1-2\rho} \xi^{2v} \notag
\end{gather}
and 
\begin{gather*}
A_{01} (i( \rho - v - \frac{1}{2}), \xi ; \rho, v) + A_{11} (i(\rho - v -\frac{1}{2}), \xi; \rho, v)\\
= (2\pi)^{1-2\rho} \sin (\pi \rho) \Gamma (2\rho -1) |\xi^2+1|^{1-2p} \xi^{2v};  
\end{gather*}
our proposition follows from these expressions.

\setcounter{section}{2}
\setcounter{subsection}{3}
\subsection{The main functional equation and the specialization.}\label{art7-subsec2.4}

\medskip
\noindent
{\bfseries Theorem \thnum{15}.\label{art7-thm15}}
\textit{Assume that the even function $h(r)$ is regular in the strip $|\Iim r| \leqslant \frac{3}{2}$, decreases as $O(|r|^{-B})$, $B>4$, when $r \to \infty$ in this strip and has zeros at $r =  \pm \frac{i}{2}$. Then we have, for $ \re v = \re \mu = \frac{1}{2}$, $\frac{1}{2} < \re s$, $\re \rho < 1$, the following functional equation}
\setcounter{section}{5}
\setcounter{equation}{0}
\begin{gather}
Z^{\text{dix}}
\left(
\left.
\begin{matrix}
s, & v \\
\rho, &  \mu
\end{matrix}
\right| h
\right) + Z^{\text{con}}
\left(
\left.
\begin{matrix}
s, & v \\
\rho, &  \mu
\end{matrix}
\right| h
\right) = \label{art7-eq5.1}\\
= \frac{z(s + \rho; v, \mu)}{\pi^2} \int\limits^\infty_{-\infty} r \text{ th} (\pi r) (h(r) -h_0 (r; s , v; \rho, \mu)) dr + \notag\\
+ Z^{\text{dis}}
\left(
\left.
\begin{matrix}
\rho, & v \\
s, & \mu
\end{matrix}
\right| h_0
\right) + \tilde{Z}^{\text{dis}}
\left(
\left.
\begin{matrix}
\rho, &v\\
s, & \mu
\end{matrix}
\right| h_1
\right) + \notag \\
+ Z^{\text{con}}  
\left(
\left.
\begin{matrix}
\rho, &v\\
s, & \mu
\end{matrix}
\right| h_0 + h_1
\right) + Z^{\text{cusp}}
\left(
\left.
\begin{matrix}
\rho, &v\\
s, & \mu
\end{matrix}
\right| h^\ast
\right). \notag\\
+ \Phi_\varphi (s, v; \rho, \mu) + \Phi_\varphi (s,  1-v; \rho, \mu) + \Phi_\varphi (\rho,\mu; s, v) + \notag\\
 \Phi_{\varphi} ( \rho , 1 - \mu; s, v) + \vartheta_h(s,v ; \rho, \mu) + \vartheta_h(s, 1- v ; \rho, \mu) + \notag\\
+ \vartheta_h(\rho, \mu; s, v) + \vartheta_h (\rho, 1 -\mu; s, v) + \vartheta_{h_0 + h_1} (s, v; \rho, \mu) +  \notag\\
+ \vartheta_{h_0 + h_1} (s, 1 - v; \rho, \mu) + \vartheta_{h_0 + h_1} (\rho, \mu; s, v) + \varphi_{h_0+ h_1} (\rho, 1- \mu; s, v) \notag
\end{gather}
\textit{where $h_0$ and $h_1$ are defined in terms of $h$ by \eqref{art7-eq4.7} and \eqref{art7-eq4.10}, the sequence $h^\ast$ is the result of the formal substitution of $i(k-\frac{1}{2})$ in place of $r$ in the expression for $h_0(r)$ and}
\begin{align}
\vartheta_n(s, v; \rho, \mu) & = \frac{\zeta(2s-1) \zeta(2v)}{\zeta(2\rho) \zeta (2+2v-2s)} z (s;\mu, 1-\rho + v) h(i(\rho - v - \frac{1}{2}))
\label{art7-eq5.2}\\
\Phi_\varphi (s, v; \rho, \mu) & = 2 \frac{(4\pi)^{2s-2v-1} \zeta (2v)}{\zeta(2s)} z (\rho + v; s , \mu) \hat{\varphi } (2 v+ 1 - 2s)  
\label{art7-eq5.3}
\end{align}
\textit{with $\varphi$ from \eqref{art7-eq4.4}.}

This functional equation follows simply on putting together the preceding considerations. 

Of particular interest is the special case when $s, v, \rho, \mu$ are chosen as in \eqref{art7-eq2.16} and the function $h$ is positive for real $r$ and decreases very rapidly; namely, we choose 
\begin{equation}
h(r) = (r^2+ \frac{1}{4})^2 (r^2 + \frac{9}{4}) (r^2 + \frac{25}{4}) e^{-\alpha r^2}, \; \alpha > 0.
\label{art7-eq5.4}
\end{equation}

Now we have only one variable $t$ and, for brevity, we shall introduce new notation. Let, for $s = \mu = \frac{1}{2}$, $\rho= \mu = \frac{1}{2} +it$ and further, for the function $h$ from \eqref{art7-eq5.4}, let
\begin{equation}
Z_c (t) = \zeta (2s) \zeta(2\rho) Z^{\text{con}} 
\left(
\left.
\begin{matrix}
s, & v \\
\rho, & \mu
\end{matrix}
\right| h
\right)
\label{art7-eq5.5}
\end{equation}

Then we have 
\begin{align}
&Z_c (t) = \label{art7-eq5.6}\\
& \frac{1}{4\pi} \int\limits^\infty_{-\infty} \frac{\zeta^3 (\frac{1}{2} + it - ir) \zeta^3 (\frac{1}{2} + it- ir) \zeta(\frac{1}{2} - it + ir) \zeta (\frac{1}{2} - it - ir)}{|\zeta(1+2ir)|^2} h (r) dr, \notag
\end{align}
where the main contribution is determined by the interval $|r| \ll (\log t)^{1/2}$ (we assume that $t$ is a positive large number). For this reason, we can write 
\begin{equation}
\zeta(\frac{1}{2} + it + ir)  \zeta (\frac{1}{2} + it - ir) = \zeta(\frac{1}{2} - it + ir) \zeta (\frac{1}{2} - it - ir) \chi(t, r), \label{art7-eq5.7}
\end{equation}
where
\begin{align}
\chi(t,r) & = \pi^{2 it} \frac{\Gamma (\frac{1}{4} - \frac{it}{2} + \frac{ir}{2}) \Gamma (\frac{1}{4} - \frac{it}{2} - \frac{ir}{2})}{\Gamma (\frac{1}{4} + \frac{it}{2} + \frac{ir}{2}) \Gamma (\frac{1}{4} + \frac{it}{2} - \frac{ir}{2})}\label{art7-eq5.8}\\
&  = i \left(\frac{2\pi}{t} \right)^{2 it} e^{2 it} \left(1+ O \left(\frac{r^2+1}{t} \right) \right). \notag
\end{align}\pageoriginale
Now we have 
\begin{align} 
& \overline{\chi (t,0)} Z_c (t) = \label{art7-eq5.9}\\
& \frac{1}{4\pi} \int\limits^\infty_{-\infty} \frac{|\zeta^4 (\frac{1}{2} + it + ir) \zeta^4(\frac{1}{2} + it - ir )|}{|\zeta(1+2 ir)|^2} (1+O\left(\frac{1+r^2}{t} \right)) h (r) dt \notag
\end{align}
and the main term in the integrand is positive. If we estimate the integral 
\begin{equation}
\int\limits^\infty_0 \omega_T (t) \overline{\chi} (t, 0) Z_c (t) dt, T \to + \infty,  \label{art7-eq5.10}
\end{equation}
then the desired estimate for the eighth moment will be a consequence of the following simple statement.

\medskip
\noindent
{\bfseries Proposition \thnum{10}.\label{art7-prop10}}
\textit{For $T \to+ \infty$, we have, with a fixed positive integer $k \geqslant 1$ and for every fixed $\delta > 0$,}
\begin{equation}
\int\limits^{2T}_T  |\zeta' (\frac{1}{2} + it) |^{2k} dt \ll (\log T)^{4k} \int\limits^{2T (1+\delta)}_{T(1-\delta)} |\zeta (\frac{1}{2} + it)|^{2k} dt\label{art7-eq5.11} 
\end{equation}
To prove this inequality, one can see firstly that 
\begin{equation}
|\zeta' (\frac{1}{2} + ut)| \ll \log T \max\limits_{x \geqslant 1} \left|\sum\limits_{n \leqslant t} \frac{1}{n^s} \right|, \; s = \frac{1}{2} + it 
\label{art7-5.12}
\end{equation}
and we have with $4M = T^{2/3}$ and $\epsilon = (\log T)^{-1}$
\begin{equation}
\sum\limits_{n \leqslant x} \frac{1}{n^2} = \frac{1}{2\pi  i} \int\limits^{\epsilon + i M}_{\epsilon - i M} \zeta(s+ w)  x^w \frac{dw}{w} + O(1), x \leqslant t.\label{art7-eq5.13}
\end{equation}
As a consequence of \eqref{art7-eq5.12}, \eqref{art7-eq5.13} and Holder's inequality, we have
\begin{gather}
\int\limits^{2T}_T |\zeta' (\frac{1}{2} + it)|^{2k} dt \ll (\log T)^{2k} \int\limits^{2T+ M}_{T-M} |\zeta(\frac{1}{2} + \epsilon + it)|^{2k} dt \left( \int\limits^M_{-M} \frac{d\eta}{|\epsilon + i\eta|}\right)^{2k} \label{art7-eq5.14}\\
\ll( \log T)^{4k} \int\limits^{2T+M}_{T-M} |\zeta(\frac{1}{2} + \epsilon + it)|^{2k} dt\\
\ll (\log T)^{4k} \int\limits^{2T+ M}_{T-M} |\zeta(\frac{1}{2} + it) |^{2k} dt, M = T^{2/3},
\end{gather}\pageoriginale 
since the last integral is non-increasing as a function of $\epsilon$.

Now, for every positive $\epsilon \in (0,1)$ we have 
\begin{gather}
|\overline{\chi (t,0)} Z_c (t)| \gg \int\limits^\epsilon_{-\epsilon} |\zeta^4 (\frac{1}{2} +it + ir) \zeta^4 (\frac{1}{2} + it - ir) | r^2 h (r) dr \notag\\
\gg \epsilon^3 |\zeta^8 (\frac{1}{2} + it)| - \int\limits^{\epsilon} _{-\epsilon} (\epsilon -r) r^2 h(r) \frac{\partial}{\partial r} |\zeta^4(\frac{1}{2} + it + ir) \zeta^4 (\frac{1}{2} + it - ir)| dr, \label{art7-eq5.15}
\end{gather}
so that 
\begin{gather}
\int \omega_T (t) \overline{\chi (t, 0)} Z_c (t) dt \gg \epsilon^3 \int |\zeta^8 (\frac{1}{2} + it)| \omega_T (t) dt - \label{art7-5.16}\\
-\epsilon^4(\int \omega_t (t) |\zeta^8 (\frac{1}{2} + it)| dt)^{7/8} (\int (\omega_T (t)| \zeta' (\frac{1}{2} + it)|^8 dt)^{1/8} \notag
\end{gather}

If $\epsilon = A(\log T)^{-2}$ with some sufficiently small constant $A$, then the last term of the right side of \eqref{art7-eq5.16} is of a lower order than the first term and so we have the inequality
\begin{equation}
\int \omega_T (t) |\zeta^\infty(\frac{1}{2} + it)| dt \ll (\log T)^6 |\int \omega_T (t) \overline{\chi (t, 0)} Z_c (t) dt|. \label{art7-eq5.17}
\end{equation}
For this reason, an estimate for the integral in \eqref{art7-eq5.10} is sufficient for our purpose.

\setcounter{section}{2}
\subsection{The functions $h_j$: the non-essential terms.}\label{art7-subsec2.5}
To carry out the non-trivial integration over $t$, we must know the asymptotic behaviour of the functions $h_0$ and $h_1$ in the special case \eqref{art7-eq2.16}, where $\rho = v = \frac{1}{2} + it$ with some large positive $t$. The plan is simple: instead of the hypergeometric functions we shall use the corresponding asymptotic formulae (these will be written by the asymptotic integration of the differential equation with a large parameter) and after this, using the saddle-point method, we shal integrate over $\xi$.

\subsubsection{The integral with $A_{01}$ in \eqref{art7-eq4.8}}\label{art7-subsubsec2.5.1}

\setcounter{section}{6}
\setcounter{equation}{0}
We have (see \eqref{art7-eq1.79})
\begin{gather}
A_{01} \left(u, \frac{1}{\xi}; 1 - \rho, \frac{1}{2} \right) = \frac{i(\xi/ 2 \pi)^{1-2} \rho}{2 \text{sh} (\pi u)} \xi^{2 i u} \times \label{art7-eq6.1}\\
\frac{\Gamma^2 (1-\rho + iu)}{\Gamma (1+ 2 iu)} F (1- \rho + iu, 1 - \rho + i u ; 1 + 2 i u ; - \xi^2) + \notag\\
+ \left\{\text{the same with } u \to -u \right\}, \notag\\
A_{01} (r, \xi, \rho, \rho) = \frac{i(2 \pi \xi)^{1-2\rho} \sin (\pi \rho) \Gamma (2 \rho - \frac{1}{2} + ir) \Gamma (\frac{1}{2} + ir)}{2 \xi^{2ir} \text{sh} (\pi \xi) \Gamma (1+ 2 ir)}  \times \label{art7-eq6.2}\\
F \left(2 \rho - \frac{1}{2} + ir , \frac{1}{2} + ir; 1 + 2 ir; -\frac{1}{\xi^2} \right) + \left\{\text{the same with } r \to - r \right\}, \notag
\end{gather}

Later, we shall use the following method of considering our integrals. It is well-known that the function
$$
w = z^{c/2} (1 \mp z)^{(a+ b + 1 - c)/2} F (a, b; c; \pm z)
$$
is a solution of the differential euqation 
\begin{equation}
w'' + \left(\frac{c(2-c)}{4z^2} + \frac{1-(a+b-c)^2}{4(1\mp z)^2} \pm \frac{\frac{1}{2}c (a+ b + 1 - c) - ab}{z(1 \mp z)} w = 0. \right)\label{art7-eq6.3} 
\end{equation}

As a consequence (using an appropriate transformation of the varialbe), we see that the function 
\begin{gather}
W(\eta) = (tg \eta/2)^{(1/2) + 2 iu} (\cos \eta/ 2)^{2 \rho -2 } \times \label{art7-eq6.4}\\
F (1- \rho + iu, 1 - \rho + iu; 1 + 2 iu; - tg^2 \eta/2) \notag
\end{gather}
satisfies the differential equation (for $\rho = \frac{1}{2} + it$):
\begin{equation}
\frac{d^2w}{d\eta^2} + \left(-t^2 + \frac{u^2}{\sin^2 \eta/2} + \frac{1}{4 \sin^2 \eta}\right) W = 0. \label{art7-eq6.5}
\end{equation}

Hence, for large $t$ and $0< \eta < \pi - \delta$ for any fixed $\delta>0$, we have 
\begin{equation}
w = \sqrt{\eta/ 2} I_{2 i u} (t \eta) \frac{\Gamma (1+ 2 iu )}{t^{2 i u}} (1+ O(\frac{1}{t})). \label{art7-eq6.6}
\end{equation}

This consequence is the distinctive feature of our method of considereing the asymptotic behaviour for all hypergeometric functions here. This method is based on the principle: ``neighbouring equations have neighbouring\pageoriginale solutions''; the method of estimation for the corresponding closeness is routine today (see, for example, \cite{art7-key5}, where the estimates are written for similar equations).

Now the following statment would be obvious for the reader: the contribution of the term with the kernels $A_{01}$ from \eqref{art7-eq4.8} is negligible for large values of $t$. Indeed,
$$
|t^{-2iu} \Gamma^2 (1- \rho + it)| \ll e^{-\pi t}
$$
and the part of the integral with $\xi \leqslant A t(\log t)^{-1}$ with some (small) fixed $A$ is small. But, for large $\xi$, we have an additional resource.  We shall assume that parameter $\alpha$ in the definition of the initial function $h(r)$ will be small; then we can move the path of the integration over $u$ in \eqref{art7-eq4.7}  and \eqref{art7-eq4.10} and the factor of the type $\xi^{2i u}$ for $\xi \gg t (\log t)^{-1}$ and $\Iim u \geqslant  + \Delta$ will give $O(t^{2\Delta} (\log t)^{2\Delta})$. Here $\Delta$ is defined by the width of the strip where $(\text{ch}(\pi u))^{-1} h(u)$ is regular; for the function \eqref{art7-eq5.4}, one can choose $\Delta = \frac{7}{2}$. 

For the same reason, one can reject the term with $A_{01} (u, \frac{1}{\xi}; 1 - \rho. \frac{1}{2})$ in the expression \eqref{art7-eq4.10}, \eqref{art7-eq4.11} for the function $h_1$.

Furthermore, we have 
\begin{gather}
A_{10} (r, \xi, \rho, \rho) = \frac{\Gamma (2 \rho - \frac{1}{2} + ir) \Gamma (2 \rho - \frac{1}{2} - ir)}{\Gamma (2 \rho)} \times \label{art7-eq6.7}\\
\sin (\pi \rho) \xi^{2\rho} F (2 \rho - \frac{1}{2} + ir, 2 \rho - \frac{1}{2} - ir; 2 \rho ; 1 - \xi^2)  \notag
\end{gather}
and the hypergeometric function with these parameters is a solution of the differential equation
\begin{equation}
w''  + \left( \frac{\rho(1-\rho)}{z^2 (1\pm z)^2} \pm \frac{r^2 + \frac{1}{4}}{z(1\pm z)}\right) w =0\label{art7-eq6.8}
\end{equation}
if $w = z^\rho (1\pm z)^\rho F (2 \rho - \frac{1}{2} + ir, 2 \rho - \frac{1}{2}  -ir; 2 \rho; \mp z)$. For the upper sign (which corresponds to $\xi > 1$), all solutions are oscillating; at the same time, we have 
\begin{align}
& \left|\frac{\Gamma (2 \rho - \frac{1}{2} + ir) \Gamma (2 \rho - \frac{1}{2}  - ir)}{\Gamma (2 \rho)}  \sin (\pi \rho)\right| \ll \label{art7-eq6.9}\\
& \qquad \ll \exp (-\frac{\pi}{2} (|2t+ r |+| 2t - r| - 3 t) \notag\\
& \qquad \ll \exp (-\frac{\pi}{2} (\max (t, 2t - 3t)) \notag
\end{align}

So,  if $\xi \geqslant 1$, the kernel \eqref{art7-eq6.7} is exponentially small. For the case $\xi < 1$ (which corresponds in \eqref{art7-eq6.8} to the case $z \in (0,1)$ and the sign ``minus''), the solution \eqref{art7-eq6.8} does not exceed $\exp (r \text{arcsin} \sqrt{1-\xi^2})$ and so we have the factor $e^{\pi r/2}$ for $\xi =0$ only. But the contribution of the interval with small\pageoriginale $\xi$, $\xi \ll t^{-1} \log t$, is small (for the same reason - one can move the path of the integration over $u$ and to render the factor $\xi^{2i u}$ small).

\setcounter{section}{2}
\subsection{The integral with $A_{00}$.}\label{art7-subsec2.6}
\subsubsection{The explicit form.}\label{art7-subsubsec2.6.1}
The unique essential term is the first integral in \eqref{art7-eq4.8} and we shall consider this term in greater detail; in passing, we shall give some examples of the asymptotic integration of the differential equations with a large parameter.

First of all, we shall write the result of substituting the special values for our parameters. Let us introduce the notation 
\setcounter{section}{7}
\setcounter{equation}{0}
\begin{equation}
v \equiv v (z; \rho, r) = |z|^{1-\rho} (1+z)^\rho F (\frac{1}{2} + ir, \frac{1}{2} - ir; 2 - 2 \rho ; - z)  \label{art7-eq7.1}
\end{equation}
and 
\begin{equation}
 w=w (z; \rho, u) =  |z|^\rho (1+z)^{(1/2) + iu} F (\rho + iu, \rho + iu; 2 \rho; - z) \label{art7-eq7.2}
\end{equation}
(here $z$ is a real variable and $z \geqslant -1$). Then we have, for all $\xi > 0$,
\begin{gather}
(2\pi)^{2 \rho-1} A_{00} (r, \xi; \rho, \rho) = \label{art7-eq7.3}\\
= \sin (\pi \rho) \Gamma (2\rho -1) (v (\xi^2 -1; \rho, r) + A v (\xi^2 -1; 1-\rho, r))\notag
\end{gather}
where
\begin{equation}
A = \frac{\Gamma (2 \rho - \frac{1}{2} + ir) \Gamma (2 \rho -\frac{1}{2} - ir)}{\Gamma (2\rho)\Gamma (2 \rho b-1)} \frac{\text{ch} (\pi r)}{\sin (2 \pi \rho)}. 
\label{art7-eq7.4}
\end{equation}

This relation is a consequence of \eqref{art7-eq1.78} and the simple relationship $F(a, b; c; z) = (1-z)^{c-a-b} F (c-a, c-b; c;z)$. The representation \eqref{art7-eq7.3} is very convenient for $r < 2 t$, because,  in this case we have $|A| \ll e^{-\pi (2t-r)}$. Hence with exponential accuracy, we can retain just the first term on the right side \eqref{art7-eq7.3} for $r \leqslant 2t (1-\delta)$ with some fixed (small) $\delta >0$.

The representation \eqref{art7-eq1.78} will be used for the other kernel too; here, we shall use the relation $F(a, b; c,z) = z^{-a} F(a, c-b; c; \frac{z}{z-1})$ and as a consequence we shall obtain the equality
\begin{gather}
(2\pi)^{1-2\rho} A_{00} (u , \frac{1}{\xi}; 1 -\rho, \frac{1}{2}) = \label{art7-eq7.5}\\
= \sin (\pi \rho) \Gamma (1 -2 \rho) (w (\xi^2 -1; \rho, u) - Bw(\xi^2 - 1; 1 -\rho,u)) \notag
\end{gather}
where
\begin{equation}
B = \frac{\text{ch}^2 \pi u+ \cos^2 \pi \rho}{\pi \sin (2 \pi \rho)} \cdot \frac{\Gamma^2 (1-\rho+ iu) \Gamma^2 (1-\rho - iu)}{\Gamma (2-2\rho) \Gamma (1-2 \rho)}
\label{art7-eq7.6}
\end{equation}

Now, after the change of the variable of integration $\xi^2 -1 \mapsto z$, we have a representation for the essential part of the function $(h_0 + h_1)$:
\begin{gather}
h^{(0)} (r, u; t) = \int\limits^\infty_0 A_{00} (r, \xi; \rho, \rho) A_{00} (u, \frac{1}{\xi}; 1 - \rho, \frac{1}{2}) \xi^{2 \rho -2} d \xi  \label{art7-eq7.7}\\
= \frac{\pi}{8} \frac{tg (\pi \rho)}{2 \rho-1} \int\limits^\infty_{-1} (v (z; \rho, r) + Av (z ; 1 - \rho, r)) (w (z; \rho, u)-\notag\\
- B w (z; 1 - \rho, u)) \frac{dz}{|z| (1+z)^{3/2}} \notag
\end{gather}
and for $r \leqslant 2 (1-\delta)t$  with fixed $\delta>0$, we can reject the term with $A$. 

If $r > 2 t$, then the representation \eqref{art7-eq1.78} is not convenient: the bounded function is expressed here as a linear combination of exponentially large terms. The relations \eqref{art7-eq1.76} and \eqref{art7-eq1.77} are more suitable in this case (One can see that \eqref{art7-eq1.78} is the consequence of the preceding equalities and the Kummer relations, which connect the hypergeometric function in $z$ with the functions of the argument $1-z$).  

To write the explicit form of the obtained equality for $h^{(0)}$, we shall introduce the additional notation
\begin{align}
& \tilde{v} (z; \rho, r) = |z|^{(1/2)+ir} (1-z)^{\rho} F (2 \rho -\frac{1}{2} + ir, \frac{1}{2} + ir; 1 + 2 ir; z), 
\label{art7-eq7.8}\\
&  \tilde{w} (z; \rho, u) = z^{\rho} (1-z)^{1/2} F(\rho + iu, \rho- iu; 2 \rho; z).\label{art-eq7.9}
\end{align}

Then, for the function $h^{(0)}(r, u;t)$, we have the representation 
\begin{gather}
h^{(0)} (r, u; t) = C_1 (r, \rho) \int\limits^1_0 v (-z ; 1 -\rho,r) (w (z-1; \rho, u)- \label{art7-eq7.10}\\
- B w (z-1; 1-\rho, u)) \frac{dz}{z^{3/2} (1-z)} + \int\limits^1_0 (C_2 (r,\rho) \tilde{v} (z; \rho, r)+\notag\\
+ C_2 (-r, \rho) \tilde{v} (z; \rho, -r)) (\tilde{w}(1-z; \rho, u) - B \tilde{w} (1-z; 1-\rho, u)) \frac{dz}{z^{3/2} (1-z)} \notag
\end{gather}
where 
%\begin{gather}
%C_1 = \frac{1}{2 \sin^2 (\pi \rho) \frac{\Gamma (1-2 \rho)}{\Gamma (2 \rho)} \Gamma (2 \rho -\frac{1}{2} + ir) \Gamma (2 \rho -\frac{1}{2} -ir),\label{art7-eq7.11}\\
%C_2 = \frac{i}{4} \frac{\cos \pi (\rho + ir) \sin (\pi \rho)}{\text{sh} (\pi r) \Gamma (1+2ir)} \Gamma (1-2\rho) \Gamma (2 \rho -\frac{1}{2} + ir) \Gamma (\frac{1}{2} %+ ir) \label{art7-eq7.12}
%\end{gather}

Now we have 10 integrals $V_j$, $0 \leqslant j \leqslant 9$; we shall enumerate these integrals so that $h^{(0)}$ is equal to the sum
\begin{gather*}
V_0 + A V_1 - B V_2 - ABV_3 \text{ or } \\
C_1 V_4 -C_1 B V_5 + C_2 (r, \rho) V_6 + C_2 (-r, \rho) V_7 -B C_2 (r, \rho) V_8 - B C_2 (-r, \rho) V_9.
\end{gather*}

Later, \pageoriginale we shall see that the essential contribution will arise only from the integrals $V_0$ and $V_4$.

\setcounter{section}{2}
\subsubsection{The Liouville-Green transformation}\label{art7-subsubsec2.6.2}
There is a clear method worked out for the asymptotic integration of the differential equations of the second order with a large parameter. This method is based on the Liouvill-Green transformation. Assume we have a differential equation of the kind 
\setcounter{section}{7}
\setcounter{equation}{12}
\begin{equation}
v + (t^2 p_0 (z)+ p_1 (z)) v =0, \cdot : = \frac{d}{dz}, \label{art7-eq7.13}
\end{equation}
where $t$ is a large positive parameter and $p_0$, $p_1$ are real functions. then we can transform the independent variable and the unknown function by the relation
\begin{align}
v = \dot{\xi}^{-(1/2)} (z) W(\xi(z)), \dot{\xi} : = \frac{d\xi}{dz}. \label{art7-eq7.14}
\end{align}
The formal differentiation gives, for the function $W$ the equation
\begin{equation}
\frac{d^2 W}{d\xi^2} + \dot{\xi}^{-2} (t^2 p_0 + p_1 - \frac{1}{2} \{\xi, z\})  W = 0.\label{art7-eq7.15} 
\end{equation}
where $\{\xi, z\}$ denotes the Schwarzian derivative, 
$$
\{\xi,z\} = \frac{\dddot{\xi}}{\dot{\xi}}- \frac{3}{2} \frac{\ddot{\xi}^2}{\dot{\xi}^2}.
$$

If one can choose the function $\xi$ so that the new equation is close to the equation with the known solution, then we shall be successful in finding the desired asymptotic approximation. The possibility of getting the known functions is explained by the vast set of the investigated equations for the special functions.

The simplest case is one when $p_0$ has no zeros and $p_1$ is smooth and bounded. Then we can choose $\xi$ so that $\dot{\xi}^{-2}p_0 = \pm 1$. If $p_0$ has a zero of the first order, then we can transform our equation, choosing $\dot{\xi}^{-2} p_0 = \xi$ (so that $\dot{\xi}^{-1}$ will be smooth); the Airy function will arise as the main term of the asymptotic formula. For the case when $p_0$ has two simple zeros nearby, one transforms the initial equation to the Weber equation; if $p_0$ has a zero and a pole (both simple), then the transformation to the Whittaker equation will be useful and so on. 


For the purpose of giving asymptotic forumlae for the four functions $v$, $w$, $\tilde{v}$, $\tilde{w}$ in the integrals $V_j$, it is sufficient to use the inequalities from \cite{art7-key56}.

The\pageoriginale  initial differential equations for these functions have the form \eqref{art7-eq7.13}; the coefficients $p_0, p_1$ are given in the following tables where the parameter $\alpha$ is equal to $t^{-1} r$ and $q (z) = z (1+z)$:
\begin{center}
{\fontsize{10}{12}\selectfont
\renewcommand{\arraystretch}{1.2}
\tabcolsep=5pt
\begin{tabular}{|l|l|l|}
\hline
Function & \multicolumn{2}{c|}{Coefficients:}\\\cline{2-3}
& $p_0$ & $p_1$ \\\hline
$v$ & $q^{-2}(z) (1+\alpha^2 q (z))$ & $(2q (z))^{-2} (1+ q(z))$\\\hline
$w$ & $(z q(z))^{-1}$ & $-u^2 ((1+ z) q (z))^{-1} +$\\
& & $+ (2 q (z))^{-2} (1+ q(z))$\\\hline
$\tilde{v}$ & $(q(-z))^{-2} (\alpha^2 - \alpha^2 z + z^2)$ & $(2q (-z))^{-2} (1+ q (-z))$ \\\hline
$\tilde{w}$ & $(-zq (-z))^{-1}$ & $u^2 (q (-z))^{-1} + (2q (-z))^{-2} (1+ q (-z))$\\\hline
\end{tabular}}\relax
\end{center}

\setcounter{section}{2}
\subsubsection{The function $v$, the case $z>0$ or the case $z \in (-1,0)$ and $\alpha <2 $.}\label{art7-subsubsec2.6.3}
The function $v$, the case $z > 0$ or the case $z \in (-1,0)$ and $\alpha < 2$. For these cases, we shall use the transformation \eqref{art7-eq7.14} by choosing 
\setcounter{section}{7}
\setcounter{equation}{15}
\begin{equation}
\dot{\xi}^2 = \frac{1+ \alpha^2 q}{q^2} , \; q = z(1+z), \label{art7-eq7.16}
\end{equation}
and therefore we can assume
\begin{gather}
\xi(z) = \log  |q| + \alpha \log \frac{2\sqrt{+ \alpha^2 1} + \alpha \sqrt{4 q +1}}{2+\alpha} \label{art7-eq7.17}\\
- 2 \log \frac{\sqrt{1+ \alpha^2 q + \sqrt{4q+1}}}{2}; \notag
\end{gather}
so $\xi = \log |z| + O(z)$ when $z \to 0$.

Now equation \eqref{art7-eq7.15}, for this case, has the form 
\begin{equation}
\frac{d^2 W}{d\xi^2} + t^2 W = Q_1 (\xi. \alpha) W \label{art7-eq7.18}
\end{equation}
where, with $q =q (z (\xi))$, we have
\begin{equation}
Q_1 = - \frac{1}{16} q (1+ \alpha^2q)^{-3} (\alpha^2 (\alpha^2 -16)q-4 (\alpha^2 + 2)). \label{art7-eq7.19}
\end{equation}

It is essential that this function tends to zero both when $q \to 0$ and $q \to \infty$. Taking into account the fact $v = |z|^{-\textit{it}} (1+ O(z)) = e^{-\textit{it} \xi} (1+ O(e^\xi))$ when $q \to 0$ (it corresponds to $\xi \to - \infty$), we conclude that, for all $z \geqslant 0$, $v$ has the asymptotic expansion
\begin{equation}
\dot{\xi}^{1/2} (z) v (z; \rho, r) = e^{-\textit{it} \xi}  \sum\limits_{n \geqslant 0} \frac{a_n (\xi; Q_1)}{(-2 it)^n}
\label{art7-eq7.20}
\end{equation}
where\pageoriginale
\begin{equation}
a_0 = 1, a_1 = \int\limits^\xi_{-\infty} Q_1 (\eta) d \eta, \ldots, a_{n+1} = a'_n + \int\limits^\xi_{-\infty} Q_1 (\eta) a_n (\eta) d \eta. 
\label{art7-eq7.21}
\end{equation}

The polynomial $1+ \alpha^2 q (-2) =  1- \alpha^2 z(1-z)$ has no zeros in the interval $z \in (0, 1)$ if $\alpha^2 < 4$; for this reason, we can use the same transformation and we have the same expansion \eqref{art7-eq7.20} for $-z \in (0,1)$ if $\alpha^2 \leqslant 4 (1-\delta)$.

\setcounter{section}{2}
\subsubsection{The function $w:z$ positive.}\label{art7-subsubsec2.6.4}
For the case $z>0$, we suppose $\dot{\xi}^2 = z^{-2} (1+z)^{-1}$, so that $z = sh^{-2} \frac{\xi}{2}$ and $\xi = 2 \log ((1/\sqrt{z}) + (\sqrt{1+ (1/z)}))$. The transformed equation has the form 
\setcounter{section}{7}
\setcounter{equation}{21}
\begin{equation}
\frac{d^2 W}{d\xi^2} + \left(t^2 + \frac{1}{4\xi^2} \right) W = Q_2 (\xi) W \label{art7-eq7.22}
\end{equation}
with 
\begin{equation}
Q_2 = \frac{u^2}{\text{ch}^2 \xi/2}  + \frac{1}{4} \left(\frac{1}{\xi^2} - \frac{1}{\text{sh}^2 \xi}  \right).\label{art7-eq7.23}
\end{equation}
when $\xi \to 0$ (which corresponds to $z \to \infty$), we have 
\begin{gather}
\dot{\xi}^{1/2} w (z; \rho, u) = \frac{\Gamma (2 \rho) z^{-(1/4)}}{\Gamma (\rho + iu) \Gamma (\rho - iu)} \times \label{art7-eq7.24}\\
\left(\log z+2 \frac{\Gamma'}{\Gamma} (1) - \frac{\Gamma'}{\Gamma} (\rho + iu) - \frac{\Gamma'}{\Gamma} (\rho - iu) + O \left( \frac{\log z}{z} \right) \right). \notag
\end{gather}
(Here the analytic continuation of the hypergeometric function is used in the logarithmic case). If $z \to 0$, then $\dot{\xi}^{1/2} w = z^{it} (1+ O(z)) =2^{2 it} e^{-it \xi} \times (1+ O (e^{-\xi}))$  and for this reason, our solution must be proportional to $\sqrt{\xi} \times H^{(2)}_0 (t \xi)$  (it is the Hankel function). Finally, have the uniform asymptotic expansion
\begin{gather}
\dot{\xi}^{1/2} w (z; \rho u) = - \frac{i\pi \Gamma (2 \rho)}{\sqrt{2} \Gamma (\rho + iu) \Gamma (\rho - iu)} \cdot \left(\sqrt{\xi} H^{(2)}_0 (t \xi) \sum\limits_{n \geqslant 0} \frac{b_n (\xi)}{t^{2n}} + \right.\label{art7-eq7.25}\\
 \quad  \left.  + (\sqrt{\xi} H^{(2)}_0 (t\xi))' \sum\limits_{n >1} \frac{C_n (\xi)}{t^{2n}} \right)\notag
\end{gather}
where $b_0 =1$, $c_1 = \dfrac{1}{2} \int\limits^\xi_0 Q_2 d \eta$ and for $n \geqslant 1$,
\begin{gather}
b_n(\xi) = -\frac{1}{2} c_n(\xi) - \frac{1}{2} \int\limits^\xi Q_2 (x) c_n (x) dx, \label{art7-eq7.26}\\
c_{n+1} (\xi) = \frac{1}{2} b'_n (\xi) + \frac{1}{2} \int\limits^\xi_0 Q_2 (x) b_n (x) dx - \frac{1}{4} \int\limits^\xi_0 (x^{-1} c_n(x)) \frac{dx}{x}. \label{art7-eq7.27}
\end{gather}\pageoriginale

The same solution may be expanded again when $\xi \geqslant \xi_0$ with some fixed $\xi_0>0$; then we have 
\begin{equation}
\dot{\xi}^{1/2} w(z; \rho , u) = 2^{2 it} e^{-it \xi} \sum\limits_{n \geqslant 0} \frac{a_n(\xi, \tilde{Q}_2)}{(-2 it)^n} \label{art7-eq7.28}
\end{equation}
where $a_0 =1$ and the other coefficients are given by the recurrence relations \eqref{art7-eq7.21} with the replacement of $Q_1$ by $\tilde{Q}_2 =Q_2 - (\frac{1}{4}) \xi^{-2}$.


\setcounter{section}{2}
\subsubsection{The function $w:z$ negative.}\label{art7-subsub2.6.5}
In essence, there is no difference from the previous case. To get an asymptotic formula for $w (-z; \rho, u)$ with $z \in (0,1)$, we choose the new variable $\xi(z) =2 \log (1/\sqrt{z} + (\sqrt{1/z})-1)$, so that $\dot{\xi}^2 = z^{-2} (1-z)^{-1}$ and $z = (\text{ch}\frac{\xi}{2})^{-2}$; $z =0$ corresponds to $\xi = + \infty$. The transformed equation for the function $W = \dot{\xi}^{1/2} W(-z; \rho, u)$ has the form 
\setcounter{section}{7}
\setcounter{equation}{29}
\begin{equation}
\frac{d^2 W}{d\xi^2} + \left( t^2 + \frac{u^2}{\text{sh}^2 \xi/2} + \frac{1}{4 \text{sh}^2 \xi}\right) W = 0.  \label{art7-eq7.30}
\end{equation}

As the initial condition at $z =0$ is 
$$
w(-z; \rho, u) = z^\rho (1+ O(z)),
$$
we have, for $\xi \geqslant \xi_0$ with fixed $\xi_0 > 0$, the expansion
\begin{equation}
\dot{\xi}^{1/2} w (-z; \rho, u) = 2^{2 it} e^{-it \xi} \sum\limits_{n \geqslant 0} \frac{a_n(\xi, Q_3)}{(-2 it)^n}, \label{art7-eq7.31}
\end{equation}
where $a_0 =1$ and $a_n$, $n \geqslant 1$, are defined by the relations \eqref{art7-eq7.21} with $Q_3 = - u^2 (\text{sh} \xi/2)^{-2} -(2 \text{sh} \xi)^{-2}$ instead of $Q_1$.

If $\xi$ were small (which corresponds to a neighbourhood of $z =1$), then we rewrite equation \eqref{art7-eq7.30} as
\begin{equation}
\begin{split}
& \frac{d^2 W}{d \xi^2} + \left(t^2+ \left(4u^2 + \frac{1}{4} \right) \frac{1}{\xi^2} \right) W = Q_4 W,\\
& Q_4 = u^2 \left(\frac{4}{\xi^2} -\frac{1}{\text{sh}^2 \xi/2}  \right) + \frac{1}{4} \left(\frac{1}{\xi^2} - \frac{1}{\text{sh}^2} \right) .
\end{split}\label{art7-eq7.32}
\end{equation}

When $z \to 1$, we have, as a consequence of the Kummer relation between the hypergeometric function in $z$ and in $(1-z)$.
\begin{equation}
\begin{split}
& F(\rho + iu, \rho + iu; 2 \rho ; z) = \frac{\Gamma (2 \rho) \Gamma (-2 iu)}{\Gamma^2 (\rho - iu)} F (\rho + iu, \rho + iu; 1+ 2 iu ; 1-z) +\\
& + \frac{\Gamma (2\rho) \Gamma (2 iu)}{\Gamma^2 (\rho + iu)} (1-z)^{-2iu} F (\rho - iu, \rho - iu; 1-2 iu ; 1-z). 
\end{split}
\label{art7-eq7.33}
\end{equation}\pageoriginale 

It gives the initial condition at $\xi = 0$ for our function $\dot{\xi}^{-(1/2)} W (-z ; \rho , u)$:
\begin{gather}
W = \Gamma (2 \rho) (\frac{\xi}{2})^{1/2}  \left(\frac{\Gamma (-2 iu)}{\Gamma^2(\rho - iu)} (\frac{\xi}{2})^{2iu} (1+O(\xi^2) \right) +\label{art7-eq7.34}\\
+ \frac{\Gamma (2 i u)}{\Gamma^2 (\rho + iu)} (\frac{\xi}{2})^{-2 iu} (1+ O (\xi^2))). \notag
\end{gather}

It means that this solution is a linear combination of solutions which are close to $A^{(\pm)} (\rho, u) \sqrt{\xi} J_{\pm 2 iu } (t \xi)$ and we have
\begin{gather}
\dot{\xi}^{1/2} w (-z; \rho,u) = \frac{i\pi}{2} \frac{\Gamma (2 \rho) t^{-2 i u}}{\text{sh} (2 \pi u) \Gamma^2 (\rho - iu)} \times \label{art7-eq7.35}\\
\left\{\sqrt{\xi} J_{2 i u} (t \xi) \sum\limits_{n \geqslant 0} \frac{\tilde{b}_n (\xi)}{t^{2n}} + (\sqrt{\xi} J_{ 2i u} (t \xi))' \sum\limits_{n \geqslant 1} \frac{\tilde{c}_n (\xi)}{t^{2n}}\right\}  + \notag\\
+ \left\{\text{the same with } u \mapsto - u \right\}\notag
\end{gather}
where $\tilde{b}_0 \equiv 1$ and the coefficients are defined by relations which are similar to \eqref{art7-eq7.26} and \eqref{art7-eq7.27}.

\setcounter{section}{2}
\subsubsection{The function $h^{(0)}$ for $\alpha^2 \leqslant 4 (1-\delta)$.}\label{art7-subsubsec2.6.6}
We shall use the standard formulae for the method of the stationary phase from \cite{art7-key6}. The main principle (not an all-embracing one and nevertheless true for our integrals with hypergeometric functions) is the following statement: if one has an integral without points of the stationary phase, then this integral will be small in a suitable sense. 

It is easy to check that there is no point of the stationary phase in the integral with $v (z; \rho, r) w (z; 1 - \rho, u)$. Furthermore, the coefficient $A$ in \eqref{art7-eq7.7} is exponentially small for $\alpha^2 \leqslant 4 (1 - \delta)$. For these reasons, the function $h^{(0)}$ is defined by the integral $V_0$ only. 

To distinguish the functions ``$\xi$'' in the asymptotic formulae for $v$ and $w$ we shall write $\xi_v$ and $\xi_w$. With this agreement, the integral $V_0$ is equal asymptotically to 
\setcounter{section}{7}
\begin{gather}
t^{-1} 2^{2it} \int\limits^\infty_{-1} \frac{\exp (-it (\xi_v (z) + \xi_2 (z)))}{(1+ \alpha^2 q)^{1/4} (1+z)^{3/4}} \sE (z, \alpha) d z \label{art7-eq7.36}
\end{gather}
where\pageoriginale $\sE$ is an asymptotic series in $t^{-1}$ with smooth and bounded coefficients; the main term in $\sE$ is equal to $\pi/16$. Now
$$
\dot{\xi}_v + \xi_w = \frac{\sqrt{1+\alpha^2 q}}{q} - \frac{1}{z\sqrt{1+z}} 
$$
and the point of the stationary phase is equal to 
\begin{equation}
z_0 = \alpha^{-2} -1 \label{art7-eq7.37}
\end{equation}

At this point, we have 
\begin{equation}
2 t \log 2 - t \xi_v (z_0) - t \xi_w (z_0) = (2 t - r) \log (2t - r) - 2 t \log t + r \log r \label{art7-eq7.38}
\end{equation}
and 
\begin{equation}
t \ddot{\xi}_v (z_0) + t \ddot{\xi}_w (z_0) = - \frac{1}{2} t \alpha^5 . \label{art7-eq7.39}
\end{equation}

The other details may be omitted here; the methods explained in \cite{art7-key6} give us the following 

\medskip
\noindent
{\bfseries Proposition \thnum{11}.\label{art7-prop11}}
\textit{Let $r \leqslant 2t (1-\delta)$ with some fixed small $\delta > 0$ and $t$ be large. Then the function $h^{(0)}$ can be written as}
\begin{equation}
h^{(0)} (r, u, t) = \frac{1}{t\sqrt{r}} e^{i\psi (t, r)} \sE (t, r, u) \label{art7-eq7.40}
\end{equation}
\textit{where}
\begin{equation}
\psi (t, r) : = (2t -r) \log (2t -r) - 2t \log t + r \log r - \frac{\pi}{4} \label{art7-eq7.41}
\end{equation}
\textit{and $\sE$ is a smooth non-oscillating function, $|\sE| \ll 1$ and for any fixed integer $n \geqslant 1$, $|(\partial / \partial t)^n \sE| \ll t^{-n}$.} 

\setcounter{section}{2}
\subsubsection{The case {\boldmath$r \geqslant 2 (1-\delta)t$}.}\label{art7-subsubsec2.6.7}
Now we shall use the representation \eqref{art7-eq7.10}. Here $C_2 (\pm r, \rho)$ is exponentially small for $2t - |r| \gg 1$. At the same time, for all $\alpha^2$, there are not turning points in the equation for $\tilde{v}$ and this function has an oscillating nature. For the points of the stationary phase in the integrals with $\tilde{v}$ and $\tilde{w}$, we have the equation 
\setcounter{section}{7}
\begin{equation}
(z(1-z))^{-1} \sqrt{\alpha^2 (1-z) + z^2} = ((1-z) \sqrt{z})^{-1}, \label{art7-eq7.42}
\end{equation}
or, what is the same, $z_0 = \alpha^2$. So, there are no such points in the interval (0, 1); for this reason, the last integral on the right side of \eqref{art7-eq7.10} can be omitted.

When $\alpha$ is close to 2, the full asymptotic investigation of the function $v(-z; 1 - \rho , r)$ is very complicated. But due to a fortunate coincidence, the simplest case is sufficient for our purposes.

The fact of the matter is given by (1) the exponentially small nature of the\pageoriginale coefficient $C_1 (r, \rho)$ for $r- 2 t \gg 1$ and 2) the absence of points of the stationary phase in the interval $z > \frac{1}{2}$ for $ r \geqslant 2(1 -\delta)$. Really, the equation for these points is
\begin{equation}
(z (1-z))^{-1} \sqrt{1 - \alpha^2 z (1 -z)} = ((1 -z)\sqrt{z})^{-1}, 1 -\alpha^2 z(1-z) > 0, \label{art7-eq7.43}
\end{equation}
and $z_0 = \alpha^{-2}$ is the unique possible solution. For this reason, it is sufficient to know the exact asymptotic formulae for the function $v(-z; 1- \rho, r)$  in the interval $z \leqslant \alpha^{-2} (1+\delta)$ only. But the turning points of our equation (the zeros of the polynomial $1-\alpha^2 z (1-z)$) are $z^{(\pm)} = 2 (\alpha (\alpha \pm \sqrt{\alpha^2 -4}))^{-1}$; these points are close to $\frac{1}{2}$ when $\alpha$ is close to 2. So we have the interval ($(\frac{1}{8}, \frac{3}{8})$, for example) where the stationary point lies and the positive polynomial $1+ \alpha^2  q(-z)$ is strongly separated from zero. For the last reason, we have, in this interval, an asymptotic expansion of the same kind as in \eqref{art7-eq7.20}. The unique natural difference is the exchange of the signs, because $\rho$ is replaced by $1-\rho$ here:
\setcounter{equation}{42}
\begin{equation}
\dot{\xi}^{1/2}_v v(-z; 1 - \rho, r) = e^{it \xi_v} \sum\limits_{n \geqslant 0} \frac{a_n (\xi_v, Q_1)}{(2 it)^n}.  \label{art7-add-eq7.43}
\end{equation}

Now one can see that at the stationary point $z_0= \alpha^{-2}$, we have 
\begin{equation}
t (\xi_v (z_0) - \xi_w(z_0))  = - (2t +r) \log (2t+r) + 2 t \log (2t) + r \log r \label{art7-eq7.44}
\end{equation}
and $\ddot{\xi}_v - \ddot{\xi}_w = - \frac{1}{2} \alpha^5$ at this point.

As a consequence of the Stirling expansion, in the case $2t - r \gg 1$,
\begin{gather}
C_1 (r, \rho) = \frac{\pi}{8t} \exp \left(i (2t + r)\log (2t+t) + \right.\label{art7-eq7.45}\\
\left.\left. + (2t -r) \log (2t - r) - 4 t \log (2t) + O \left(\frac{1}{2t -r} \right) \right) \right)\notag
\end{gather}
so that for $r \leqslant 2 t (1-\delta)$ with $\delta >0$ at the point $z_0 = \alpha^{-2}$
\begin{equation}
C_1 (r, \rho) e^{it(\xi_v - \xi_w)-i\pi/4} = e^{i\psi (t,r)} \cdot \frac{\pi}{8t} \left(1+ O \left(\frac{1}{t} \right) \right),
\label{art7-eq7.46}
\end{equation}
where $\psi$ is the same phase as in \eqref{art7-eq7.40}.

To estimate the contribution of the integration over the complement of the interval $(\alpha^{-2} - \delta, \alpha^{-2} + \delta)$, especially in the transition region $|\alpha^2 - 4|\leqslant \delta$, we shall use approximation by the Weber functions.

Let, for definiteness, $\alpha >2$ and the quantity $\epsilon^2 = (\frac{1}{4} - \alpha^{-2})$ be small. Then the differential equation for $v(-z; \rho, r)$ may be written in the form 
\begin{equation}
v'' + \left(r^2 \frac{16(z^2 - \epsilon^2)}{(1-4z^2)^2} + \frac{3+4z^2}{(1-4z^2)^2} \right) v = 0, -\frac{1}{2} < z < \frac{1}{2} \label{art7-eq7.47}
\end{equation}
(here\pageoriginale $z$ is written instead of $z -\frac{1}{2}$ in the initial equation).

The corresponding Liouville-Green transformation is taken by choosing
\begin{equation}
\dot{\xi}^2  (\xi^2 - \gamma^2) = \frac{16(z^2 - \epsilon^2)}{(1-4z^2)^2} \label{art7-eq7.48}
\end{equation}
with the conditions $\xi(-\epsilon) = - \gamma$, $\dot{\xi} >0$. The new parameter $\gamma$ is chosen so that the equality $\xi (+\epsilon) = \gamma$ is fulfilled. This last condition gives 
\begin{equation}
\gamma^2 = 2 (1-\sqrt{1-4\epsilon^2}) = 4 \epsilon^2 (1+ \epsilon^2 + \ldots) \label{art7-eq7.49}
\end{equation}
If we denote $\epsilon^{-1} z$ and $\gamma^{-1} \xi$ as $x$ and $y (x, \epsilon)$, then for the Schwarzian derivative $\{\xi, z\}$, we have the expression $\epsilon^{-2} \{y, x\}$ and the function $y(x, \epsilon)$ is defined by the equation 
\begin{equation}
(y^2 -1) \left(\frac{dy}{dx} \right)^2 = \frac{16\epsilon^4}{\gamma^4} \frac{x^2 -1}{(1-4 \epsilon^2 x^2)^2} . \label{art7-eq7.50}
\end{equation}
Here the function on the right hand side is a power series in $\epsilon^2$ with the leading term $(x^2 -1)$. For this reason, we have a solution of the form 
\begin{equation}
y (x, \epsilon) = x + \epsilon^2 y_1 (x) + \epsilon^4 y_2 (x) + \ldots \label{art7-eq7.51}
\end{equation}
Hence the Schwarzian derivative $\{y, x\}$ is of order $O(\epsilon^2)$ (it being obvious that $\{x, x\} =0$) and as a result, we have the boundedness of $\{\xi, z\}$ in a certain interval $\epsilon^2 \leqslant \epsilon^2_0$. Now we have the transformed equation for the function $W = \dot{\xi}^{1/2} v$:
\begin{equation}
\frac{d^2 W}{d\xi^2} + r^2 (\xi^2 - \gamma^2) W = Q_4 (\xi, \epsilon) W \label{art7-eq7.52}
\end{equation}
\text{where} $Q_4$ is bounded uniformly (in $\epsilon$) for all $\xi \in (-\infty, \infty)$ and at the same time, this function tends to zero, for $\xi \to \pm \infty$, as $O(\xi^{-2})$.

An estimate of the closeness of the solutions of this equation to the solutions of the equation with $Q_4 \equiv 0$ (the Weber functions is given  in \cite{art-key7}). We have useful inequalities for the Weber functions and the full asymptotic expansions due to F. Olver \cite{art7-key}. They allow us to given the asymptotic formulae for $v$ in the transition region $\alpha^2 \tilde 4$. After that, everyone who is a past master in integration by parts will be also to prove the smallness for all integrals, except in the case considered. As a result we have 

\medskip
\noindent
{\bfseries Proposition \thnum{12}.\label{art7-prop12}}
\textit{For any $r$ with the condition $1 \ll r \leqslant 2t  + B_0 \log t$, for fixed $B_0 \geqslant 1$, we have}
\begin{equation}
h^{(0)} (r,t) = \frac{1}{\sqrt{r}} C_1 \left(r, \frac{1}{2} + it \right) e^{i\psi_0 (t,r)} \sE (t, r) \label{art7-eq7.53}
\end{equation}
\textit{where}
\begin{equation}
\psi_0 (t,r) = - (2t +r) \log (2t+r) + 2t \log (2t) + r \log r -\frac{\pi}{4} \label{art7-eq7.54}
\end{equation}\pageoriginale
\textit{and $\sE$ is a smooth non-oscillating function,}
\begin{equation}
|\sE (t,r )| \ll 1, \left|\left(\frac{\partial}{\partial t} \right)^n \sE (t, r) \right| \ll t^{-n}, n = 0, 1, \ldots \label{art7-eq7.55}
\end{equation}
\textit{If $r \geqslant 2t + B_0 \log t$ with fixed $B_0 \geqslant 1$, then}
\begin{equation}
|h^{(0)} (r,t)|^1 r^{-3B_0}. \label{art7-eq7.56}
\end{equation}

\setcounter{section}{2}
\subsection{The integration over $t$}\label{art7-sec2.7}
\subsubsection{The summation formulae}\label{art7-subsubsec2.7.1}

The next step is the calculation of the integrals over $t$, where the integrand contains the Hecke series (associated with the continuous or discrete spectrum) and the function $h^{(0)} (r,t)$. To do this, we need to approximate the corresponding Hecke series by a finite sum; it will be achieved by using the following summation formulae (using other forms of the functional equations for the Hecke series).

\medskip
\noindent
{\bfseries Proposition \thnum{13}.\label{art7-prop13}}
\textit{Assume that $\varphi:[0,\infty) \to \bbC$ and its Mellin transform $\hat{\varphi} (s)$ satisfies the conditions:}
\begin{itemize}
\item[i)] $\hat{\varphi}(2s)$ \textit{is regular in the strip $\alpha_0 \leqslant \re s \leqslant \alpha_1$ with some $\alpha_0 <0$ and $\alpha_1 >1$;}

\item[ii)] \textit{for $\sigma \in [\alpha_0 , \alpha_1]$, the function}
$$
((1+|t|)^{-1-2\sigma} +1)^{-1} |\hat{\varphi} (2 \sigma + 2 it)|
$$
\end{itemize}
\textit{is integrable on the axis $(-\infty, + \infty)$. Then, for any $v$ with $\re v = \frac{1}{2}$ and for any relatively prime integers $c$, $d$ with $c \geqslant 1$, one has the identity}
\setcounter{section}{8}
\setcounter{equation}{0}
\begin{gather}
\frac{4\pi}{c} \sum\limits_{n \geqslant 1} e \left(\frac{nd}{c} \right) \tau_v (n) \varphi \left(\frac{4\pi \sqrt{n}}{c} \right) = 2 \frac{\zeta (2v)}{(4 \pi)^{2v}} \hat{\varphi} (2v+1) + \\\label{art7-eq8.1}
+ 2 \frac{\zeta (2 - 2 v)}{(4\pi)^{2-2v}} \hat{\varphi} (3 - 2 v) + \notag\\
+ \sum\limits_{n \geqslant 1} \tau_v (n) \int\limits^\infty_0 (e (-nd'/c) k_0 (x \sqrt{n}, v) + e(nd'/ c) k_1 (x \sqrt{n}, v) ) \varphi  (x) x d x\notag
\end{gather}
\textit{where $d'$ is defined by the congruence $dd' \equiv 1 (\mod c)$ and the kernels $k_0, k_1$ are defined by the relations} \eqref{art7-eq1.66}.

\medskip
\noindent
{\bfseries Proposition \thnum{14}.\label{art7-prop14}}
\textit{Let $\varphi$ have the same properties as in \eqref{art7-eq8.1} and let $t_j(n)$, $n = 1, 2, \ldots$, be the eigenvalues of $n$-th Hecke operator. Let $\lambda_j > \frac{1}{4}$ and let the $j$-th eigenfunction  of the automorphic Laplacian be even. Then, for any coprime integers $c$, $d$ with $c \geqslant 1$, we have}
\begin{gather}
\frac{4\pi}{c} \sum\limits_{n \geqslant 1} e \left(\frac{nd}{c} \right) t _j(n) \varphi \left(\frac{4\pi \sqrt{n}}{c} \right) = \label{art7-eq8.2}\\
= \sum\limits_{n \geqslant 1} t_j(n) \inf\limits^\infty_0 \left(e (-nd'/c) k_0 (x\sqrt{n},\frac{1}{2} + i_{\chi_j}) + \right. \notag \\
\left. + e \left(\frac{nd'}{c} \right) k_1 \left(x \sqrt{n}, \frac{1}{2} + i \chi_j \right) \right) \varphi (x)x \;d x.\notag
\end{gather}\pageoriginale 

\subsubsection{The integration over $t$.}\label{art7-subsubsec2.7.2}
Our next problem is the asymptotic calculation of the integral
\begin{equation}
\sJ (T) = \int \omega_T (t) \sH_j (\frac{1}{2} + 2 it) h^{(0)} (\chi_j, t) \overline{\chi (t, 0)} dt \label{art7-eq8.3}
\end{equation}
(where $\chi$ is defined by the equality \eqref{art7-eq5.8}) and the similar integral 
\begin{equation}
\sJ (T, r) = \int \omega_T (t) \zeta (\frac{1}{2} + 2 it + ir) \zeta (\frac{1}{2} + 2 i t - ir) h^{(0)} (r,t) \overline{\chi (t,0)} dt. 
\label{art7-eq8.4}
\end{equation}

We shall consider the second integral; the first one may be considered in the same manner. 

Let $\beta : [0, \infty) \to [0,1]$ be the infinitely smooth monotone function with the conditions
\begin{equation}
\beta(x) + \beta (1/x) \equiv 1, \; \beta (x) \equiv 0 \textit{ for  } 0 \leqslant x \leqslant \frac{1}{2}
\label{art7-eq8.5}
\end{equation}
(and for this reason $\beta (x) \equiv 1$ if $x \geqslant 2$).

If $\re s >1$, writing $v$ instead of $\frac{1}{2} + ir $ for brevity, we have, for any positive $\delta$, 
\begin{gather}
\zeta (s + v - 1/2) \zeta (s- v+ 1/2) = \sum\limits^\infty_{n=1} \frac{\tau_v (n)}{n^s} = \label{art7-eq8.6}\\
= \sum\limits^\infty_{n=1} n^{-s} \beta (\delta n) \tau_v (n) + \sum\limits^\infty_{n=1} n^{-s} \beta (1/\delta n) \tau_v (n). \notag
\end{gather}
 
Applying, to the second sum, the summation formula \eqref{art7-eq8.1} with $c = d =1$, we shall obtain the representation 

%%%%% 6 page 

%{the bibliography}{99}
%\bibitem{art6-key1}
%\end{thebibliography}

%\medskip
%\noindent
%{\bfseries Proposition \thnum{5}.\label{art7-prop2}}
