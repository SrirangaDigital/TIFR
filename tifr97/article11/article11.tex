\title{ON ZETA FUNCTIONS ASSOCIATED WITH SELF-DUAL HOMOGENEOUS CONES}
\markright{On Zeta Functions Associated with Self-Dual Homogeneous Cones}

\author{By~ Ichiro Satake}
\markboth{Ichiro Satake}{On Zeta Functions Associated with Self-Dual Homogeneous Cones}

\date{}
\maketitle

\setcounter{pageoriginal}{176}
\textsc{Let}\pageoriginale $\mathscr{C}$ \text{be} and (irreducible) self-dual homogeneous cone is a vector space $V$ with a $\mathbb{Q}$-structure such that the automorphism group $G=\Aut(V,\mathscr{C})^{\circ}$ is defined over $\mathbb{O}$. Let $M$ be a lattice in $V$ and let $\Gamma=\{g\in G|gM=M\}$. Then by definition the zeta function associated with $\mathscr{C}$ is given by
\begin{equation}
Z_{\mathscr{C}}(M;s)=\sum\limits_{x:\Gamma\backslash\mathscr{C}\cap M}|\Gamma_{x}|^{-1}N(x)^{-s}\quad (s\in\mathbb{C}),\label{art11-eq1}
\end{equation}
where $\Gamma_{x}=\{\lambda\in\Gamma|\lambda x=x\}$ and $N(x)$ is the ``norm'' of $x$ (see \ref{art11-sec1}).

The purpose of this note is to supplement our previous report \cite{art11-keySO} in the following points. First, in \ref{art11-sec2}, we will show that, except for the case $\mathscr{C}=\mathscr{P}_{r}(\mathbb{R})$ and $G$ is $\mathbb{O}$-split (treated in \cite{art11-keySh}), the fundamental assumption (2.6) in \cite{art11-keySS} is satisfied, so that we can apply the general results of Sato-Shintani on the zeta functions of prehomogeneous vector spaces to our case. In \S\ref{art11-sec5}, we will determine that poles and the residues of the zeta functions, and, in \ref{art11-sec6}, the functional equations (cf. \cite{art11-keySO}, Th. 2.3.1, 2.3.3). These will be done including the case $d$ is odd, which was excluded in \cite{art11-keySF} and \cite{art11-keySO}. In particular, we will show that the matrix $U^{(r)}(x)$ giving the functional equations is always diagonalizable.

\section{}\label{art11-sec1}
Let $V$ be a real vector space of dimension $n$ endowed with a positive definite inner product $\langle \ \rangle$. Let $\mathscr{C}$ be a self-dual homogeneous cone in $V$, i.e. an open convex cone with vertex at 0 satisfying the following two conditions : 
\begin{itemize}
\item[(i)] $\mathscr{C}$ is ``self-dual'', {\em i.e.} one has
$$
\mathscr{C}=\mathscr{C}^{*}=\{x\in V|\langle x,y\rangle \ \rangle 0\text{~ for all~ }y\in \overline{\mathscr{C}}-\{0\}\}.
$$

\item[(ii)] The automorphism group of $\mathscr{C}$,
$$
G=\Aut(V,\mathscr{C})^{\circ}=\{g\in \GL(V)|g\mathscr{C}=\mathscr{C}\}^{\circ},
$$
is transitive on $\mathscr{C}$ ($^{\circ}$ denotes the connected component of the identity.)
\end{itemize}
In what follows, we assume for simplicity that $\mathscr{C}$ is irreducible and exclude the trivial case $\mathscr{C}=\mathbb{R}_{+}$ (the half-line of positive numbers). Then $G$ is (the identity connected component of) a reductive algebraic group defined over $\mathbb{R}$ with $\mathbb{R}$-rank $r\geq 2$ and one has $G=G^{s}\times \mathbb{R}_{+}$, where $G^{s}$ is $\mathbb{R}$-simple. (For the treatment of the reducible case, see \cite{art11-keySO}.) For any $c_{0}\in \mathscr{C}$, the stabilizer $K=G_{c_{0}}$\pageoriginale is a maximal compact subgroup of $G$ and one has $G/K\cong \mathscr{C}$. We can (hence will) assume that the base point $c_{0}$ and the inner product $\langle \ \rangle$ are so chosen that for $g\in G$ one has $gc_{0}=c_{0}$ if and only if ${}^{t}g^{-1}=g$. We further normalize $\langle \ \rangle$ by $\langle c_{0},c_{0}\rangle=r$.

We set $\bg=\Lie G$, $\bk=\Lie K$, $\bk=\Lie K$ and let $\bg=\bk+\bp$ be the corresponding Cartan decomposition. As is well-known (see {\em e.g.} \cite{art11-keyS1}), there exists a unique structure of Jordan algebra on $V$ with the unit element $c_{0}$ such that, denoting by $T_{x}(x\in V)$ the Jordan multiplication $y\mapsto xy(y\in V)$, one has $\bp=\{T_{x}(x\in V)\}$. We denote by $N(x)$ the reduced norm of this Jordan algebra. Then $N:V\to \mathbb{R}$ is a polynomial function of degree $r$ defined over $\mathbb{R}$ satisfying the following conditions :
\begin{equation}
N(c_{0})=1, N(gx)=\det(g)^{r/n}N(x)\quad (g\in G, x\in V).\label{art11-eq2}
\end{equation}
It is then clear that $\chi(g)=\det(g)^{r/n}$ is a rational character of the algebraic group $G$.

One can find a system of mutually orthogonal primitive idempotents $\{e_{i}(1\leq i\leq r)\}$ such that
$$
c_{0}=\sum\limits^{r}_{i=1}e_{i},\qquad e_{i}e_{j}=\delta_{ij}e_{i}, 
$$
which we call a ``primitive decomposition'' of $c_{0}$. Then $\ba=\{T_{e_{i}}(1\leq i\leq r)\}_{\mathbb{R}}$ is a maximal (abelian) subalgebra in $\bp$. It is known that the system of $\mathbb{R}$-roots (relative to $\ba$) is of type $(A_{r})$ and all the $\mathbb{R}$-roots have the same multiplicity $d$. One has a direct sum decomposition
$$
V=\bigoplus\limits_{k\leq l}V_{kl},
$$
where
$$
V_{kl}=
\begin{cases}
\{x\in V|e_{k}x=x\} & (k=l),\\
\{x\in V|e_{k}x=e_{l}x=\frac{1}{2}x\} & (k<l),
\end{cases}
$$
and one has $\dim V_{kk}=1$ and $\dim V_{kl}=d(k<l)$. Hence one has the relation
\begin{equation}
\frac{n}{r}=1+\frac{d}{2}(r-1).\label{art11-eq3}
\end{equation}

We assume that there is given a $\mathbb{O}$-structure on $V$ ({\em i.e.} a $\mathbb{O}$-vector space $V_{\mathbb{O}}$ in $V$ with $V=V_{\mathbb{O}}\otimes_{\mathbb{O}}\mathbb{R}$) such that $G$ is defined over $\mathbb{O}$ and $c_{0}\in V_{\mathbb{O}}$. Then, clearly, $K$, $\langle \ \rangle$, $N$, $\chi$ are all defined over $\mathbb{O}$. We denote by $r_{0}$ the $\mathbb{O}$-rank of $G$. Then it can be shown that $r_{0}$ is a divisor of $r$. So we set $\delta=r/r_{0}$. The possible values of $\delta$ are as listed below.
\begin{center}
\tabcolsep=10pt
\begin{tabular}{lrrc}
\multicolumn{1}{c}{$\mathscr{C}$} & \multicolumn{1}{c}{$r$} & \multicolumn{1}{c}{$d$} & \multicolumn{1}{c}{$\delta$}\\ \hline
$\mathscr{P}_{r}(\mathbb{R})$ & $\geq 2$ & 1 & 1 or 2 ($r$ even)\\
$\mathscr{P}_{r}(\mathbb{C})$ & $\geq 2$ & 2 & $\delta | r$\\
$\mathscr{P}_{r}(\mathbb{H})$ & $\geq 3$ & 4 & 1 \\
$\mathscr{P}_{3}(\mathbb{O})$ & 3 & 8 & 1 \\
$\mathscr{P}(1, n-1)$ & 2 & $\geq 3$ & 1
\end{tabular}\pageoriginale
\end{center}

\section{}\label{art11-sec2}
To define the zeta function, we fix a lattice $M$ in $V$ compatible with the given $\mathbb{O}$-structure and let $\Gamma$ be the stabilizer of $M$ in $G$. Then $\Gamma$ is an arithmetic subgroup of $G$ acting properly discontinuously on $\mathscr{C}$. For $x\in V$, we denote by $G_{x}$ and $\Gamma_{x}$ the stabilizers of $x$ in $G$ and $\Gamma$.

Let $S$ denote the singular set $\{x\in V|N(x)=0\}$ and put $V^{\times}=V-S$. Let $V^{\times}_{i}$ denote the set of all $x\in V^{\times}$ with ``signature'' $(r-i,i)$ (see \S\ref{art11-sec3}). Then one has an (open $G$-orbit decomposition
\begin{equation}
V^{\times}=\coprod\limits^{r}_{i=0}V^{\times}_{i}.\label{art11-eq4}
\end{equation}
Clearly one has $V^{\times}_{i}=-V^{\times}_{r-i}$, $V^{\times}_{0}=\mathscr{C}$.

For $x\in V^{\times}_{\mathbb{O}}$, $G_{x}$ is a reductive subgroup defined over $\mathbb{O}$ and $\Gamma_{x}$ is an arithmetic subgroup. We denote by $\mu(x)$ the volume of $\Gamma_{x}\backslash G_{x}$ with respect to a suitably normalized Haar measure on $G_{x}$. In particular, if $x\in \mathscr{C}$, then $G_{x}$ is compact, $\Gamma_{x}$ is finite, and one has $\mu(x)=|\Gamma_{x}|^{-1}<\infty$. For all $x\in V^{\times}_{i\mathbb{Q}}(1\leq i\leq r-1)$, one has $\mu(x)<\infty$ except for the case $r=2$, $d=\delta=1$. In what follows, we exclude this case, which is treated in \cite{art11-keySi} and \cite{art11-keySh}. For $0\leq i\leq r$ we define a zeta function associated with the $G$-orbit $V^{\times}_{i}$ by
\begin{equation}
\xi_{i}(M;s)=\sum\limits_{x\in \Gamma\backslash M\cap V^{\times}_{i}}\mu(x)|N(x)|^{-s},\label{art11-eq5}
\end{equation}
where the summation is taken over a complete set of representatives of the $\Gamma$-orbits in $M\cap V^{\times}_{i}$. Clearly one has $\xi_{i}=\xi_{r-i}$ and $\xi_{0}(M;s)$ is the zeta function $Z_{\mathscr{C}}(M;s)$ associated with the self-dual homogeneous cone $\mathscr{C}$.

To discuss the convergence of these zeta functions, we need

\medskip
\noindent
{\bf Lemma \thnum{1}.\label{art11-lem1}}~{\em Let $G^{1}=\{g\in G|\det(g)=1\}$ and for $f\in \mathscr{S}(V)$ (the Schwartz space) set}
$$
I(f,M)=\int\limits_{G^{1}/\Gamma\cap G^{1}}\left(\sum\limits_{x\in M}f(gx)\right)\rmd^{1}g.
$$
{\em where\pageoriginale $\rmd^{1}g$ is a (suitably normalized) Haar measure on $G^{1}$. Then, if $d\delta\geq 2$, the integral on the right hand side is absolutely convergent and the map $f\mapsto I(f,M)$ is a tempered distribution on $V$.}

This is proved by applying Weil's criterion (\cite{art11-keyW}, p. 90, Lem. 5). For $c>0$ put
\begin{align*}
A_{c}=\{\diag(t_{1},\ldots,t_{r_{0}})|t_{i}\in \mathbb{R}_{+}, &\prod\limits^{r_{0}}_{i=1}t_{i}=1, t_{i}/t_{i+1}\geq c\\
&(1\leq i\leq r_{0}-1)\}.
\end{align*}
Then, since every $\mathbb{O}$-root has the multiplicity $d\delta^{2}$, it is enough to show that
\begin{multline*}
\int\limits_{A_{c}}\left\{\prod\limits^{r_{0}}_{i=1}\Sup (1,t^{-2}_{i})^{\delta(1+(d/2)(\delta-1))}\times{}\right.\\
\left.{}\times \prod\limits_{1\leq i<j\leq r_{0}}\Sup (1,t^{-1}_{i}t^{-1}_{j})^{d\delta^{2}}(t_{i}t_{j}^{-1})^{-d\delta^{2}}\right\}^{1/2}\prod\limits^{r_{0}-1}_{i=1}t^{-1}_{i}\rmd t_{i}<\infty.
\end{multline*}
(See \cite{art11-keySS}, p. 166, Lem. 4.3.) Putting $\tau_{i}=(t_{i}t_{i+1}^{-1})^{1/r_{0}}$, one has for some $c_{1}>0$
\begin{gather*}
\Sup (1,t^{-2}_{i})\leq c_{1}\prod\limits^{i-1}_{k=1}\tau^{2k}_{k},\\
\Sup (1,t^{-1}_{t}t^{-1}_{j})\leq c_{1}\prod\limits^{i-1}_{k=1}\tau^{2k}_{k}\prod\limits_{\substack{i\leq k<j\\ k\geq r_{0}/2}}\tau^{2k-r_{0}}_{k}(i<j),\\
\prod\limits_{i<j}(t_{i}t^{-1}_{j})^{-1}=\prod\limits^{r_{0}-1}_{i=1}\tau^{-i(r_{0}-i)r_{0}}_{i}.
\end{gather*}
In view of these estimates, one sees that the above integral is
$$
\leq c_{2}\int\limits_{A_{c}}\left(\prod\limits^{r_{0}-1}_{i=1}\tau^{i(r_{0}-i)\delta v_{i}}_{i}\right)^{1/2}\prod\limits^{r_{0}-1}_{i=1}t^{-1}_{i}\rmd t_{i}
$$
for some $c_{2}>0$, where
$$
v_{i}=
\begin{cases}
2-d(\delta i+1) & \text{for~ } 1\leq i\leq [r_{0}/2],\\
2-d(\delta(r_{0}-i)+1) & \text{for~ } [r_{0}/2]+1\leq i\leq r_{0}-1.
\end{cases}
$$
If $d\delta\geq 2$, one has $v_{i}<0$ for all $1\leq i\leq r_{0}-1$, which proves our assertion.

In what follows, we assume that $d\delta \geq 2$. Then Lemma \ref{art11-lem1} assures that the fundamental\pageoriginale assumption (2.6) in \cite{art11-keySS} is satisfies, so that we can apply the general results obtained there. (As we shall see in \S\ref{art11-sec3}, the condition (2.13) in \cite{art11-keySS} is also satisfied). In particular, by \cite{art11-keySS}, Theorem 2, ({\em i}), the Dirichlet series on the right hand side of \eqref{art11-eq5} converges absolutely for $\rRe s>n/r$ and the function $\xi_{i}(M;s)$ thus defined can be continued to a meromorphic function on the whole complex plane. It is known that, even in the case $d=\delta=1$, the Dirichlet series defining $Z_{\mathscr{C}}(M;s)$ has the same property (\cite{art11-keySh}).

\section{}\label{art11-sec3}
We now consider the $G$-orbit decomposition of the singular set $S=V-V^{\times}$. Every element $x$ in $V$ can be expressed in the form
\begin{equation}
x=k\left(\sum\limits^{r}_{v=1}\alpha_{v}e_{v}\right)\quad\text{with}\quad k\in K, \alpha_{v}\in \mathbb{R},\label{art11-eq6}
\end{equation}
where $(\alpha_{1},\ldots,\alpha_{r})$ is uniquely determined up to the order (independently of the choice of the primitive decomposition $\{e_{v}\}$) (\cite{art11-keyS3}, Prop. 3). We say that $x$ is of {\em rank} $\rho$ and of {\em signature} $(\rho-i,i)$ if, in a suitable order of $(\alpha_{v})$, one has $\alpha_{1},\ldots,\alpha_{i}<0$, $\alpha_{i+1},\ldots,\alpha_{p}>0$, $\alpha_{\rho+1}=\ldots=\alpha_{r}=0$. For $0\leq\rho \leq r-1$ and $0\leq i\leq \rho$, we set
\begin{align*}
& S^{(\rho)}=\{x\in V|\rank x=\rho\},\\
& S^{(\rho)}_{i}=\{x\in V|\sign x=(\rho-i,i)\}.
\end{align*}
Then it is easy to see that the $G$-orbit decomposition of $S$ is given by 
\begin{equation}
S=\coprod\limits_{\substack{0\leq \rho \leq r-1\\ 0\leq i\leq \rho}}S^{(\rho)}_{i}.\label{art11-eq7}
\end{equation}
Since $G^{1}$ is transitive on each $S^{(\rho)}_{i}$, \eqref{art11-eq7} is also the $G^{1}$-orbit decomposition of $S$. Thus the condition (2.13) in \cite{art11-keySS} is certainly satisfied.

By \cite{art11-keySS}, Lemmas 2.7 and 2.8, ({\em i}), there exists a $G^{1}$-invariant measure $\rmd v^{(\rho)}(v)$ on $S^{(\rho)}$ satisfying the relation
\begin{equation}
\rmd v^{(\rho)}(gv)=\chi(g)^{s_{\rho},i}\rmd v^{(\rho)}(v)\qquad (g\in G, v\in S^{(\rho)}_{i})\label{art11-eq8}
\end{equation}
for some $s_{\rho,i}\in \mathbb{R}$. To describe the measure $\rmd v^{(\rho)}$ explicitly, we use the following parametrization of $S^{(\rho)}$.

Set
$$
e=\sum\limits^{r-\rho}_{v=1}e_{v},\quad e'=c_{0}-e=\sum\limits^{r}_{v=r-\rho+1}e_{v}
$$
and $V_{\lambda}=V_{\lambda}(e)=\{x\in V|ex=\lambda x\}$. Writing $v\in V$ in the form
$$
v=v_{1}+v_{1/2}+v_{0}, \ \ v_{\lambda}\in V_{\lambda}(e),
$$
we\pageoriginale set
\begin{equation}
S^{(\rho)}(e')=\{v\in S^{(\rho)}|N_{0}(v_{0})\neq 0\},\label{art11-eq9}
\end{equation}
where $N_{0}$ denotes the norm of the Jordan subalgebra $V_{0}(e)$. Then $S^{(\rho)}(e')$ is a Zariski open set in $S^{(\rho)}$ and by \cite{art11-keyS3}, Lemma 1 every element $v$ in $S^{(\rho)}(e')$ can be written uniquely in the form
\begin{equation}
v=\exp(e\Box y)v_{0}\quad\text{with}\quad y\in V_{1/2}, v_{0}\in V_{0},\label{art11-eq10}
\end{equation}
where in general $x\Box y=T_{xy}+[T_{x},T_{y}]$ (Koecher's notation).

By a well-known identity in the Jordan algebra one has for any $y$, $y'$ in $V_{1/2}(e)$
$$
[[T_{y},T_{y'}],T_{e}]=T_{y(y'e)-y'(ye)}=0.
$$
Hence one has $[e\Box y, e\Box y']=0$ (by \cite{art11-key11S3}, (4)) and
$$
\exp(e\Box y)\cdot \exp (e\Box y')=\exp(e\Box (y+y')).
$$
Therefore $S^{(\rho)}(e')$ can be viewed as a principal bundle of the additive group $V_{1/2}(e)$ with base space $V_{0}=V_{0}(e)$ by the action $v\mapsto \exp(e\Box y)v(y\in V_{1/2}(e),v\in S^{(\rho)}(e'))$. It follows, that, if one puts
$$
\rmd \mu(v)=\rmd y\cdot \rmd v_{0}\quad\text{for}\quad v=\exp(e\Box y)v_{0}, y\in V_{1/2}, v_{0}\in V_{0},
$$
then there exists a continuous function $\varphi=\varphi_{\rho,i}:V_{0}(e)\to \mathbb{R}$ such that $\rmd v^{(\rho)}(v)=\varphi(v_{0})^{-1}\rmd \mu(v)$. Putting $g=\lambda 1$ in \eqref{artt1-eq8}, one sees that $\varphi$ is homogeneous of degree $\rho(1+\frac{d}{2}(\rho-1))-rs_{\rho,i}$.

Not let $G_{0}$ be the subgroup of $G$ generated by $\exp T_{x}(x\in V_{0}(e))$. Then the $V_{\lambda}(e)'$s are $G_{0}$-invariant. For $g_{0}\in G_{0}$, one has $g_{0}|V_{1}(e)=\text{id.}$ and $N_{0}(g_{0}v_{0})=\chi_{0}(g_{0})N_{0}(v_{0})$ for $v_{0}\in V_{0}(e)$, where $\chi_{0}$ is a rational character of $G_{0}$ satisfying the relation
\begin{equation}
\det (g_{0}|V_{0}(e))=\chi_{0}(g_{0})^{1+(d/2)(\rho-1)}.\label{art11-eq11}
\end{equation}

\medskip
\noindent
{\bf Lemma \thnum{2}.\label{art11-lem2}}~{\em For $g_{0}\in G_{0}$, one has}
\begin{align}
\det (g_{0}|V_{1/2}) &= \chi_{0}(g_{0})^{(d/2)(r-\rho)},\label{art11-eq12}\\
\chi(g_{0}) &=\chi_{0}(g_{0}).\label{art11-eq13}
\end{align}

\begin{proof}
Since $g_{0}e=e$, one has for $y\in V_{1/2}$
$$
g_{0}(e\Box y)v_{0}=(e\Box {}^{t}g^{-1}_{0}y)g_{0}v_{0}.
$$
Hence $g_{0}(yv_{0})=({}^{t}g^{-1}_{0}y)\cdot (g_{0}v_{0})$, or
\begin{equation*}
g_{0}T_{v_{0}}{}^{t}g_{0}=T_{g_{0}v_{0}}\quad\text{on}\quad V_{1/2}(e).\tag{*}
\end{equation*}
By \cite{art11-keySF}, Lemma 2, ({\em i}), one has $\det(2T_{v_{0}}|V_{1/2})=N_{0}(v_{0})^{d(r-\rho)}$. Taking the determinant of both sides of (*), one has
$$
\det (g_{0}|V_{1/2})^{2}N_{0}(v_{0})^{d(r-\rho)}=N_{0}(g_{0}v_{0})^{d(r-\rho)},
$$
whence\pageoriginale follows \eqref{art11-eq12}. Since
$$
\chi(g_{0})^{1+(d/2)(r-1)}=\det(g_{0})=\det(g_{0}|V_{1/2})\cdot \det(g_{0}|V_{0}),
$$
\eqref{art11-eq13} follows from \eqref{art11-eq11} and \eqref{art11-eq12}.
\end{proof}

By \eqref{art11-eq11} and \eqref{art11-eq12} one has
\begin{align*}
\rmd \mu(g_{0}v)&= d({}^{t}g_{0}^{-1}y)\cdot \rmd(g_{0}v_{0})\\[3pt]
&= \chi_{0}(g_{0})^{-(d/2)(r-\rho)+1+(d/2)(\rho-1)}\rmd \mu(v).
\end{align*}
Hence by \eqref{art11-eq8} and \eqref{art11-eq13} one has
$$
\varphi(g_{0}v_{0})=\chi_{0}(g_{0})^{1-(d/2)(r-2\rho+1)-s_{\rho,i}}\varphi(v_{0}).
$$
In particular, $\varphi$ is homogeneous of degree $\rho(1-\frac{d}{2}(r-2\rho+1)-s_{\rho,i})$. Combining this with what we mentioned above, we see that $s_{\rho,i}=\frac{d}{2}\rho$, which is independent of $0\leq i\leq \rho$, and that $\varphi(v_{0})$ is given by $cN(v_{0})^{1-(d/2)(r-\rho+1)}$ for some $c>0$. We normalize $\rmd v^{(\rho)}(v)$ by putting $c=2^{-d\rho(r-\rho)}$.

Summing up, we have

\medskip
\noindent
{\bf Lemma \thnum{3}.\label{art11-thm3}}~{\em In the expression \eqref{art11-eq10}, $aG^{1}$-invariant measure on $S^{(\rho)}$ is given by}
\begin{equation}
\rmd v^{(\rho)}(v)=2^{-d\rho(r-\rho)}N_{0}(v_{0})^{(d/2)(r-\rho+1)-1}\rmd y \ \rmd v_{0}.\label{art11-eq14}
\end{equation}
{\em and one has}
\begin{equation}
\rmd v^{(\rho)}(gv)=\chi(g)^{(d/2)\rho}\rmd v^{(\rho)}(v)\qquad (g\in G, v\in S^{(\rho)}).\label{art11-eq15}
\end{equation}

\section{}\label{art11-sec4}
We set $\mathscr{V}_{k}=\oplus_{j<k}V_{jk}(2\leq k\leq r)$. Then every element $v$ in $S^{(\rho)}(e')$ can be written uniquely in the form
\begin{align}
& v=\sum\limits^{r}_{k=r-\rho+1}\epsilon_{k}t_{k}\left(e_{k}+\frac{1}{2}x'_{k}+\frac{1}{4}{x'}^{2}_{k}(1-e_{k})\right),\label{art11-eq16}\\
& x'_{k}\in \mathscr{V}_{k}, t_{k}\in \mathbb{R}_{+},\epsilon_{k}=\pm 1(r-\rho+1\leq k\leq r)\notag
\end{align}
(see \cite{art11-keyS3}, Prop. 1). In this expression, it is easy to see that
\begin{equation}
\rmd v^{(\rho)}(v)=2^{-(d/2)\rho(2r-\rho-1)}\prod\limits^{r}_{k=r-\rho+1}(t_{k}^{(d/2)(2k-r+\rho-1)-1}\rmd t_{k})\rmd x',\label{art11-eq17}
\end{equation}
where $\rmd x'=\Pi \rmd x'_{k}$ is the Euclidean measure on $\oplus^{r}_{k=r-\rho+1}\mathscr{V}_{k}$.

For $0\leq \rho\leq r-1$ and $0\leq i\leq \rho$, let $\mathscr{E}^{(r)}_{\rho,i}$ denote the set of all $r$-tuples $\epsilon=(\epsilon_{v})\in \{0,1,-1\}^{r}$ such that
\begin{gather*}
\epsilon_{v}=0(0\leq v\leq r-\rho), =\pm 1(r-\rho+1\leq v\leq r),\\
\text{and~~} \sharp \{v|\epsilon_{v}=-1\}=i.
\end{gather*}
For\pageoriginale $\epsilon=(\epsilon_{v})\in \mathscr{E}^{(r)}_{\rho,i}$, let $S^{(\rho)}_{\epsilon}$ denote the set of all $v$ in $S^{(\rho)}(e')$ of the form \eqref{art11-eq16} with the given $(\epsilon_{v})$. Then clearly one has
$$
S^{(\rho)}(e')\cap S^{(\rho)}_{i}=\coprod\limits_{\epsilon\in \mathscr{E}^{(r)}_{\rho,i}}S^{(\rho)}_{\epsilon}.
$$
Let $\mathscr{E}^{(r)}=\{\pm 1\}^{r}$, and for $\eta=(\eta_{k})\in \mathscr{E}^{(r)}$ let $V^{\times}_{\eta}$ denote the $AN$-orbit of $\sum^{r}_{k=1}\eta_{k}e_{k}$, where $A=\exp \ba$ and $N$ is the unipotent subgroup of $G$ generated by $\exp (\sum^{r}_{k=2}e_{k}\Box x_{k})(x_{k}\in \mathscr{V}_{k})$. Then $\coprod\limits_{\eta\in \mathscr{E}^{(r)}}V_{\eta}^{\times}$ is a Zariski open subset of $V^{\times}$.

\medskip
\noindent
{\bf Proposition \thnum{1}.\label{art11-prop1}}~{\em Let $0\leq \rho\leq r-1$ and $0\leq i\leq \rho$. Then, for $f\in C_{0}^{\infty}(\coprod V^{\times}_{\eta})$, one has}
\begin{gather*}
\int\limits_{S_{i}^{(\rho)}}\widehat{f}(v)\rmd v^{(\rho)}(v)=\prod\limits^{\rho}_{k=1}[(2\pi)^{-(d/2)k}\Gamma\left(\dfrac{d}{2}k\right)\times{}\\
{}\times \sum\limits_{\epsilon,\eta}\be \left(\frac{d}{8}N^{(\rho)}_{\epsilon\eta}\right)\int\limits_{V^{\times}_{\eta}}f(u) | N(u)|^{-(d/2)_{\rho}}\rmd u,
\end{gather*}
{\em where the summation is taken over all $\epsilon\in \mathscr{E}^{(r)}_{\rho,i}\eta\in \mathscr{E}^{(r)}$, $\be(\cdot)$ stands for $\exp(2\pi \sqrt{-1}\cdot)$ and}
$$
N^{(\rho)}_{\epsilon\eta}=\sum\limits^{r}_{k=r-\rho+1}\epsilon_{k}\left(\sum\limits^{k-1}_{l=1}\eta_{1}+\eta_{k}(k-r+\rho)\right).
$$

The proof is similar to that of \cite{art11-keySF}, (21), and we use basically the same notation as in \cite{art11-keySF}. Write $u\in V$ in the form
$$
u=\sum\limits^{r}_{k=1}\xi_{k}e_{k}+\sum\limits^{r}_{k=2}u_{k},
$$
with
$$
\xi_{k}\in \mathbb{R}^{\times}, \ u_{k}\in \mathscr{V}_{k}
$$
and for $1\leq k\leq r$ set
$$
u^{(k)}=\sum\limits^{k}_{j=1}\xi_{j}e_{j}+\sum\limits^{k}_{j=2}u_{j}.
$$
For a self-adjoint linear operator $T$ on $\mathscr{V}_{k}$ and $x_{k}\in \mathscr{V}_{k}$, we write
$$
T[x_{k}]=\langle x_{k}, Tx_{k}\rangle.
$$
In\pageoriginale particular,
$$
u^{(k-1)}[x_{k}]=T_{u^{(k-1)}}[x_{k}]=\langle x_{k},u^{(k-1)}x_{k}\rangle.
$$

Now, writing $v\in S^{(\rho)}_{\epsilon}$ in the form \eqref{art11-eq16}, one has
$$
\langle u,v\rangle = \sum\limits^{r}_{k=r-\rho+1}\epsilon_{k}t_{k}\left(\xi_{k}+\frac{1}{2}\langle u_{k},x'_{k}\rangle+\frac{1}{4}u^{(k-1)}[x'_{k}]\right)
$$
(see \cite{art11-keySF}, p. 476). For $\lambda > 0$ and $r-\rho+1\leq k\leq r$, put
$$
Q_{k}=Q_{k}(\lambda,\epsilon_{k},u^{(k-1)})=\lambda 1_{\mathscr{V}_{k}}-\dfrac{\sqrt{-1}}{2}\epsilon_{k}T_{u^{(k-1)}}|\mathscr{V}_{k}.
$$
Then, for $u\in \coprod V^{\times}_{\eta}$, one has 
\begin{multline*}
\langle u,v\rangle =\dfrac{\sqrt{-1}}{2}\lim\limits_{\lambda\to 0}\sum\limits^{r}_{k=r-\rho+1}t_{k}(\lambda-2\sqrt{-1}\epsilon_{k}\xi_{k}+{}\\
{}+ Q_{k}\left[x'_{k}-\dfrac{\sqrt{-1}}{2}\epsilon_{k}Q_{k}^{-1}u_{k}\right]+\frac{1}{4}Q_{k}^{-1}[u_{k}])
\end{multline*}
Hence, putting $q_{k}(\lambda)=\lambda-2\sqrt{-1}\epsilon_{k}\xi_{k}+\dfrac{1}{4}Q_{k}^{-1}[u_{k}]$, one has
\begin{align*}
I_{\epsilon} &= \int\limits_{S_{\epsilon}}^{(\rho)}\widehat{f}(v)\rmd v^{(\rho)}(v)\\
&= \int\limits_{S_{\epsilon}^{(\rho)}}\left(\int\limits_{V}f(u)\be(\langle u, v\rangle)\rmd u\right)\rmd v^{(\rho)}(v)\\
&= 2^{-(d/2)\rho(2r-\rho-1)}\sum\limits_{\eta\in \mathscr{E}^{(r)}}\int\limits_{V^{\times}_{\eta}}f(u)\rmd u\times {}\\
&\quad +\lim\limits_{\lambda\to 0}\sum\limits^{r}_{k=r-\rho+1}\int\limits^{\infty}_{0}t_{k}^{(d/2)(2k-r+\rho-1)-1}\be\left(\frac{\sqrt{-1}}{2}t_{k}q_{k}(\lambda)\right)\rmd t_{k}\times{}\\
&\quad {}\times \int\limits_{???????????}\be\left(\dfrac{\sqrt{-1}}{2}t_{k}Q_{k}\left[x'_{k}-\dfrac{\sqrt{-1}}{2}\epsilon_{k}Q_{k}^{-1}u_{k}\right]\right)\rmd x'_{k}.
\end{align*}
For\pageoriginale $u\in V^{\times}_{\eta}$, one has, in the notation of \cite{art11-keySF}, sign $\chi_{k}(u)=\eta_{k}(1\leq k\leq r)$. Hence
\begin{align*}
& \int\limits_{\mathscr{V}_{k}}\be \left(\dfrac{\sqrt{-1}}{2}t_{k}Q_{k}\left[x'_{k}-\dfrac{\sqrt{-1}}{2}\epsilon_{k}Q_{k}^{-1}u_{k}\right]\right)\rmd x'_{k}=t_{k}^{-(d/2)(k-1)}\det (Q_{k})^{-1/2}\\
&\to 2^{d(k-1)}t_{k}^{-(d/2)(k-1)}\be \left(\dfrac{d}{8}\epsilon_{k}\sum\limits^{k-1}_{l=1}\eta_{l}\right) | N^{(k-1)}(u^{(k-1)}) |^{-d/2}\\
&\tag*{$(\lambda\to 0)$,}\\
&\int\limits^{\infty}_{0}t^{(d/2)(k-r+\rho)-1}\be\left(\dfrac{\sqrt{-1}}{2}t_{k}q_{k}(\lambda)\right)\rmd t_{k}\\
&=\Gamma\left(\dfrac{d}{2}(k-r+\rho)\right)(\pi q_{k}(\lambda))^{-(d/2)(k-r+\rho)}\\
&\to \Gamma\left(\dfrac{d}{2}(k-r+\rho)\right)\be\left(\dfrac{d}{8}\epsilon_{k}\eta_{k}(k-r-\rho)\right)(2|\chi_{k}(u)|)^{-(d/2)(k-r+\rho)}\\
&\tag*{$(\lambda\to 0).$}
\end{align*}
Moreover, one has
$$
\prod\limits^{r}_{k=r-\rho+1}N^{(k-1)}(u^{(k-1)})|\chi_{k}(u)|^{k-r+\rho}=N(u)^{\rho}.
$$
Therefore, putting $k'=k-r+\rho(r-\rho+1\leq k\leq r)$, one has 
\begin{align*}
I_{\epsilon} &= 2^{(d/2)\rho(2r-\rho-1)}\sum\limits_{\eta\in \mathscr{E}^{(r)}}\int\limits_{V^{\times}_{\eta}}f(u)\rmd u\times {}\\
&\quad {}\times \lim\limits_{\lambda\to 0}\left(\sum\limits^{r}_{k=r-\rho+1}\det (Q_{k})^{-1/2}\Gamma\left(\dfrac{d}{2}k'\right)(\pi q_{k}(\lambda))^{-(d/2)k'}\right)\\
&=\prod\limits^{\rho}_{k'=1}\Big(2\pi)^{-(d/2)k'}\Gamma((d/2)k')\Big]\\
&{}\times \sum\limits_{\eta}\be \left(\dfrac{d}{8}\sum\limits^{r}_{k=r-\rho+1}\epsilon_{k}\left(\left(\sum\limits_{l<k}\eta_{l}\right)-\eta_{k}(k-r+\rho)\right)\right)\int\limits_{V^{\times}_{\eta}}f(u)|N(u)|^{-(d/2)\rho}\rmd u,
\end{align*}
which proves the Proposition.

\section{}\label{art11-sec5}
We put
$$
\lambda_{\rho}=\prod\limits^{\rho}_{k=1}\left[(2\pi)^{-(d/2)k}\Gamma\left(\frac{d}{2}k\right)\right]
$$
Also,\pageoriginale putting $n(\eta)=\sharp \{k|1\leq k\leq r, \eta_{k}=-1\}$ for $\eta\in\mathscr{E}^{(r)}$, we set $\mathscr{E}^{(r)}_{j}=\{\eta\in \mathscr{E}^{(r)}|n(\eta)=j\}$. Then for $\eta\in \mathscr{E}^{(r)}_{j}$ an easy computation shows that
\begin{align*}
N^{(\rho)}_{\epsilon \eta} &= (r-2j)\rho-2i(r-\rho-2p-1)+{}\\
&\quad {} + \sum\limits^{r}_{k=r-\rho+1}(1-\epsilon_{k})\left\{\sum\limits^{k-1}_{l=r-\rho+1}(1-\eta_{l})-(1+\eta_{k})(k-r+\rho)\right\},
\end{align*}
where $p=\sharp \{k | 1\leq k\leq r-\rho, \eta_{k}=-1\}$ and $\Sigma'$ indicates that the summation is taken over $k$ (or $l$) $\geq r-\rho+1$.

We claim that $\sum\limits_{\epsilon\in \mathscr{E}^{(r)}_{\rho,i}}\be \left(\dfrac{d}{8}N^{(\rho)}_{\epsilon\eta}\right)$ depends only on $i$ and $j=n(\eta)$ and is independent of the choice of $\eta\in \mathscr{E}^{(r)}_{j}$. Then the formula in Proposition \ref{art11-prop1} can be written as
\setcounter{equation}{18}
\begin{equation}
\int\limits_{S_{i}^{(\rho)}}f(v)\rmd v^{(\rho)}(v)=\sum\limits^{r}_{j=0}r_{ij}\int\limits_{V^{\times}_{j}}f(u) | N(u)|^{-(d/2)\rho}\rmd u,\label{art11-eq19}
\end{equation}
where
\begin{equation}
r_{ij}=\lambda_{\rho}\sum\limits_{\epsilon\in \mathscr{E}^{(r)}_{\rho,i}}\be \left(\dfrac{d}{8}N^{(\rho)}_{\epsilon\eta}\right)\quad (\eta\in \mathscr{E}^{(r)}_{j}).\label{art11-eq20}
\end{equation}

When $d$ is even, the above claim is obvious, since
$$
\be \left(\dfrac{d}{8}N^{(\rho)}_{\epsilon\eta}\right)=\sqrt{-1}^{(d/2)r\rho}(-1)^{(d/2)((r-\rho-1)i+\rho j)}.
$$
In this case, one has
\begin{equation}
r_{ij}=\sqrt{-1}^{(d/2)r\rho}\lambda_{\rho}\cdot (-1)^{(d/2)(r-\rho-1)i}\binom{\rho}{i}\cdot (-1)^{(d/2)\rho j}.\label{art11-eq21}
\end{equation}
Introducing a new variable $y$, one has the relation
\begin{equation}
\sum\limits^{\rho}_{i=0}r_{ij}y^{i}=\sqrt{-1}^{(d/2)r\rho}\lambda_{\rho}\cdot (-1)^{(d/2)\rho j}(1+(-1)^{(d/2)(r-\rho-1)}y)^{\rho}.\label{art11-eq22}
\end{equation}

When $d$ is odd, we set
$$
\omega=(-1)^{\rho}(-\sqrt{-1})^{d(r-\rho-1)}
$$
Then for $\eta\in \mathscr{E}^{(r)}_{j}$ one has
\begin{align*}
& \sum\limits^{\rho}_{i=0}\left(\sum\limits_{\epsilon\in \mathscr{E}^{(r)}_{\rho,i}}\be\left(\dfrac{d}{8}N^{(\rho)}_{\epsilon\eta}\right)\right)y^{i}={\zeta'}_{8}^{dr\rho}(-\sqrt{-1})^{d\rho j}\times {}\\
&\quad {}\times \sum\limits_{i,\epsilon}(-1)^{\sum\limits_{k}(1-\epsilon_{k})/2\{\sum\limits_{l<k}(1-\eta_{l})/2+(1+\eta_{k})(k-r+\rho)/2\}}(\omega y)^{i}\\
&= \zeta^{dr\rho}_{8}(-\sqrt{-1})^{d\rho j}\prod\limits^{\rho}_{k'=1}(1+(-1)^{{\mathop{\sum'}\limits_{l<k}} (1-\eta_{l})/2+(1+\eta_{k})k/2}\omega y),
\end{align*}\pageoriginale
where the summation $\sum\limits_{i,\epsilon}$ is taken over all $0\leq i\leq \rho$, $\epsilon\in \mathscr{E}^{(r)}_{\rho,i}$. It can be shown that the last expression depends only on $i$ and $j$. Actually, defining $r_{ij}$ by \eqref{art11-eq20}, one obtains the relation
\begin{align*}
&\sum\limits^{\rho}_{i=0}r_{ij}y^{i} = \zeta_{8}^{dr\rho}\lambda_{\rho}(-\sqrt{-1})^{d\rho j}\tag*{$(22')$}\label{art11-eq22'}\\[3pt]
&\quad {}\times
\begin{cases}
(1-\omega^{2}y^{2})^{\rho/2} & (\rho \text{~ even}),\\
(1-\omega^{2}y^{2})^{(\rho-1)/2} & (1-(-1)^{j}(-\sqrt{-1})^{d(r-\rho-1)}y)(\rho \text{~odd}).
\end{cases}
\end{align*}

Now, for the given $\mathbb{Q}$-structure one has $S^{(\rho)}_{i}\cap V_{\mathbb{Q}}\neq \emptyset$ if and only if $\rho$ is a multiple of $\delta=r/r_{0}$. By the general theory of Sato-Shintani, we know that for $x\in S^{(\rho)}_{i}\cap V_{\mathbb{O}}$ the stabilizer $G^{1}_{x}$ of $x$ in $G^{1}$ is unimodular and for a normalized Haar measure $\rmd v_{x}$ on $G^{1}_{x}$ one has
$$
\mu^{1}(x)=\int\limits_{G^{1}_{x}/\Gamma_{x}}\rmd v_{x}<\infty,
$$
the normalization being made by the relation
$$
\int\limits_{G^{1}/\Gamma_{x}}f(gx)\rmd^{1}g=\mu^{1}(x)\int\limits_{S_{i}^{(\rho)}}f(v)\rmd v^{(\rho)}(v)\qquad (f\in C^{\infty}_{0}(S^{(\rho)}_{i})).
$$
It is known furthermore that one has
\begin{equation}
\kappa_{i}^{(\rho)}(M)=\sum\limits_{x:\Gamma\backslash M\cap S_{i}^{(\rho)}}\mu^{1}(x)<\infty
\end{equation}
(\cite{art11-keySS}, Proof of Lemma 2.7). It can be shown that $\kappa^{(\rho)}_{i}(M)$ is a finite sum of certain special values of the zeta functions associated with the ``rational boundary components'' in $S^{(\rho)}_{i}$.

By \cite{art11-keySS}, Theorem 2, ({\em ii}), one obtains the follows

\medskip
\noindent
{\bf Proposition \thnum{2}.\label{art11-prop2}}~{\em Assume that $d\delta \geq 2$. Then the zeta functions $\xi_{j}(M;s)(0\leq j\leq r)$ are holomorphic except for possible simple poles at $s=\frac{n}{r}-\frac{d}{2}\rho$ with $0\leq \rho\leq r-1$, $\delta |\rho$ and}
\begin{equation}
\Res_{S=\frac{n}{r}-\frac{d}{2}\rho}\xi_{j}=\text{vol} (V/M^{*})\sum\limits^{\rho}_{i=0}\kappa^{(\rho)}_{i}(M^{*})r_{ij},\label{art11-eq24}
\end{equation}
{\em where\pageoriginale $M^{*}$ is the dual lattice of $M$ and $\kappa^{(\rho)}_{i}(M^{*})$ and $r_{ij}$ are given by \eqref{art11-eq23} (for $M^{*}$) and \eqref{art11-eq22}, \eqref{art11-eq22'}.}

In the case $d$ is even, we set
\begin{equation}
R^{(\rho)}=\text{vol}(V/M^{*})\lambda_{\rho}\sqrt{-1}^{(d/2)r\rho}\cdot \sum\limits^{\rho}_{i=0}(-1)^{(d/2)(r-\rho+1)i}\binom{\rho}{i}\kappa^{(\rho)}_{i}(M^{*}).\label{art11-eq25}
\end{equation}
In particular, for $\rho=0$, one has $R^{(0)}=\text{vol}(V/M^{*})\cdot \mu^{1}(0)>0$.

\medskip
\noindent
{\bf Lemma \thnum{4}.\label{art11-lem4}}~{\em When $d\equiv 0(4)$ or $\rho\equiv r-1(2)$, one has $R^{(\rho)}\neq 0$. When $d\equiv 2$ (4) and $\rho\equiv r\equiv 1(2)$, one has $R^{(\rho)}=0$.}

This follows immediately from the fact that $\kappa^{(\rho)}_{i}(M^{*})>0$ and $\kappa^{(\rho)}_{i}(M^{*})=\kappa^{(\rho)}_{\rho-i}(M^{*})$.

By Proposition \ref{art11-prop2}, the residue of $\xi_{j}$ at $s=\frac{n}{r}-\frac{d}{2}\rho(0\leq \rho\leq r-1,\delta|\rho)$ is given by $(-1)^{(d/2)\rho j}R^{(\rho)}$. Hence, in particular, every $\xi_{j}(0\leq J\leq r)$ has a pole at $s=n/r$. It follows from Lemma \ref{art11-lem4} that, when $d\equiv 0(4)$, every $\xi_{j}$ has exactly $r_{0}$ poles at $s=\frac{n}{r}-\frac{d}{2}\delta k(0\leq k\leq r_{0}-1)$. When $d\equiv 2(4)$ and $r$ is odd, every $\xi_{j}$ has exactly $(r_{0}+1)/2$ poles at the above $s$ with even $k$. When $d\equiv 2(4)$ and $r$ is even, one can only say that, if $\delta$ is odd, every $\xi_{j}$ has at least $[(r_{0}+1)/2]$ poles.

\section{}\label{art11-sec6}
By the general theory of Sato-Shintani we know that the zeta functions satisfy functional equations of the following form
\begin{gather}
\xi_{j}\left(M^{*};\frac{n}{r}-s\right)=\text{vol}(V/M)\prod\limits^{r}_{k=1}[(2\pi)^{-s+(d/2)(k-1)}\Gamma\left(s-\dfrac{d}{2}(k-1))\right]\times{}\label{art11-eq26}\\
{}\times \be \left(\dfrac{r}{4}s\right)\sum\limits^{r}_{i=0}\xi_{i}(M;s)u_{ij}(s)\notag\\
(0\leq j=\leq r),\notag
\end{gather}
where $u_{ij}(s)$ is a polynomial in $\be(-\frac{s}{2})$ determined by the following relation 
\begin{gather}
\int\limits_{V^{\times}_{i}}\widehat{f}(u)|N(u)|^{s-(n/r)}\rmd u=\sum\limits^{r}_{k=1}[(2\pi)^{-s+(d/2)(k-1)}\Gamma\left(s-\dfrac{d}{2}(k-1))\right]\times{}\label{art11-eq27}\\
{}\times \be \left(\dfrac{r}{4}s\right)\sum\limits^{r}_{j=0}u_{ij}(s)\int\limits_{V^{\times}_{j}}f(u)|N(u)|^{-s}\rmd u\notag\\
(0\leq i\leq r, f\in \mathscr{S}(V)).\notag
\end{gather}

By\pageoriginale computing the left hand side of \eqref{art11-eq27}, it was shown in \cite{art11-keySF} that
\begin{equation*}
u_{ij}(s)=\sum\limits_{\epsilon\in \mathscr{E}^{(r)}_{i}}u_{\epsilon\eta}(s)\qquad (n\in \mathscr{E}^{(r)}_{j}),\tag{$\sharp$}\label{art11-sharp}
\end{equation*}
where
$$
u_{\epsilon\eta}(s)=\be\left(\frac{d}{8}N^{(r)}_{\epsilon\eta}+\frac{1}{4}\sum\limits^{r}_{k=1}\left(\epsilon_{k}\eta_{k}\left(s-\dfrac{d}{2}r\right)-s\right)\right..
$$
(The fact that the right hand side of \eqref{art11-sharp} depends only on $j=n(\eta)$ is shown in the following lines. This is not {\em a priori} clear as stated in \cite{art11-keySF}, p. 477.)

First, one can write
\begin{gather*}
u_{\epsilon\eta}(s)=\be\left(\dfrac{d}{2}\sum\limits^{r}_{k=1}\frac{1-\epsilon_{k}}{2}\left(\sum\limits^{k-1}_{l=1}\frac{1-\eta_{l}}{2}-\frac{1+\eta_{k}}{2}k\right)\right.+\frac{d}{4}(i-rj)-\\
-\frac{1}{2}\sum\limits^{r}_{k=1}\left(\frac{1+\epsilon_{k}}{2}\frac{1-\eta_{k}}{2}+\frac{1-\epsilon_{k}}{2}\frac{1+\eta_{k}}{2}\right)\left(s-\frac{d}{2}r\right)
\end{gather*}
Put
\begin{gather*}
x=\be(-\frac{d}{2}),\quad \zeta=\sqrt{-1}^{d(r+1)},\\
\beta_{k}=\be\left(\dfrac{d}{2}\sum\limits^{k-1}_{l=1}\frac{1-\eta_{l}}{2}\right)=
\begin{cases}
1 & (d\text{~ even}),\\
(-1)^{\# \{l|1\leq l\leq k-1,\eta_{l}=-1\}} & (d\text{~ odd}),
\end{cases}\\[4pt]
u_{\eta}(x,y)=\sum\limits_{i,\epsilon}u_{\epsilon\eta}(s)y^{i}, 
\end{gather*}
where the summation $\Sigma_{\epsilon}$ is taken over all $0\leq i\leq r$, $\epsilon\in \mathscr{E}^{(r)}_{i}$. Then by an easy computation one obtains
\begin{align*}
& u_{\eta}(x,y)=(-\sqrt{-1})^{drj}\prod\limits^{r}_{k=1}\left(\be\left(-\frac{1}{2}\frac{1-\eta_{k}}{2}\left(s-\dfrac{d}{2}r\right)\right)\right.+{}\\
&{}+\be \left(-\frac{1}{2}\left(\frac{1+\eta_{k}}{2}\left(s-\frac{d}{2}r+dk\right)-d\sum\limits^{k-1}_{l=1}\frac{1-\eta_{l}}{2}\right)\sqrt{-1}^{d}y\right)\\
&= \prod\limits_{\eta_{k}=1}(1+(-1)^{dk}\zeta\beta_{k}xy)\prod\limits_{\eta_{k}=-1}(x+(-1)^{d}\zeta^{-1}\beta_{k}y).
\end{align*}
When $d$ is even, one has $\zeta=(-1)^{(d/2)(r+1)},\beta_{k}=1$ and
\begin{equation}
u_{\eta}(x,y)=(1+\zeta xy)^{r-j}(x+\zeta y)^{j}.
\end{equation}
When $d$ is odd, one has $\zeta^{2}=(-1)^{r+1}$ and 
\begin{gather}
u_{\eta}(x,y)=(1+\zeta xy)^{[(r-j)/2]}(1-\zeta xy)^{r-j-[(r-j)/1]}\times{}\label{art11-eq29}\\
{}\times (x+\zeta^{-1}y)^{[j/2]}(x-\zeta^{-1}y)^{j-[j/2]}.\notag
\end{gather}\pageoriginale
Thus, in either case, one sees that $u_{\eta}(x,y)$ depends only on $j=n(\eta)$. Hence one writes $u_{j}(x,y)$ for it. [\eqref{art11-eq28} and \eqref{art11-eq29} are the same as \eqref{art11-25'} in \cite{art11-keySF}.]

The semisimplicity of the matrix $U^{(r)}(x)=(u_{ij}(x))$ was shown in \cite{art11-keySF} except for the case $d=1$. Hence, in the rest of the paper, we assume that $d=1$. By our assumption $d\delta\geq 2$, one then has $\delta=2$ and so $r$ is even. However, we include also the case $r$ is odd.

First, suppose that $r$ is odd. Then $\zeta=(-1)^{(r+1)/2}$ and one has
\begin{equation}
u_{j}(x,y)=
\begin{cases}
(1-x^{2}y^{2})^{(r-j-1)/2}(x^{2}-y^{2})^{j/2}(1-\zeta xy) & (j\text{~even}),\\
(1-x^{2}y^{2})^{(r-j)/2}(x^{2}-y^{2})^{(j-1)/2}(x-\zeta y) & (j\text{~odd}).
\end{cases}\label{art11-eq30}
\end{equation}
Dividing the set of indices in two blocks by their parity, we write
$$
U^{(r)}(x)=
\left(
\begin{matrix}
U_{++} & U_{+-}\\
U_{-+} & U_{--}
\end{matrix}
\right),
$$
where $U_{++,\ldots}$ are square matrices of size $\frac{r+1}{2}$ consisting of $u_{ij}(x)$ with $(i,j)$ of the given parity ({\em e.g.} $U_{+-}$ consisting of $u_{ij}(x)$ with $i$ even and $j$ odd). Then \eqref{art11-eq30} gives
\begin{align*}
&U_{++}=\rho_{(r-1)/2}\left(\left(
\begin{matrix}
1 & x^{2}\\
-x^{2} & -1
\end{matrix}
\right)\right), \ \ U_{+-}=xU_{++},\\
&U_{-+}=-\zeta xU_{++}, \ \ U_{--}=-\zeta U_{++},
\end{align*}
or more symbolically
\begin{equation}
U^{(r)}(x)=\rho_{(r-1)/2}\left(\left(
\begin{matrix}
1 & x^{2}\\
-x^{2} & -1
\end{matrix}
\right)\right)
\otimes 
\left(
\begin{matrix}
1 & x\\
-\zeta x & -\zeta
\end{matrix}
\right),\label{art11-eq31}
\end{equation}
where $\rho_{(r-1)/2}$ denotes the symmetric tensor representation of degree $\frac{r-1}{2}$. Thus we see the $U^{(r)}(x)$ is diagonalizable and similar to
$$
\rho_{(r-1)/2}\left(\left(
\begin{matrix}
0 & 1-x^{2}\\
1+x^{2} & 0
\end{matrix}
\right)\right)\otimes
\left(
\begin{matrix}
0 & 1-x\\
1+x & 0
\end{matrix}
\right)(\zeta=1),
$$
or
$$
\rho_{(r-1)/2}\left(\left(
\begin{matrix}
0 & 1-x^{2}\\
1+x^{2} & 0
\end{matrix}
\right)\right)\otimes
\left(
\begin{matrix}
1+x & 0\\
0 & 1-x
\end{matrix}
\right)(\zeta=-1).
$$

Next, suppose that $r$ is even. Then $\zeta=(-1)^{r/2}\sqrt{-1}$ and 
\begin{equation}
u_{j}(x,y)=
\begin{cases}
(1+x^{2}y^{2})^{(r-j)/2}(x^{2}+y^{2})^{j/2} & (j\text{~even}),\\
(1+x^{2}y^{2})^{(r-j-1)/2}(x^{2}+y^{2})^{(j-1)/2}(1-\zeta xy)(x+\zeta y) & (j\text{~odd}).
\end{cases}\label{art11-eq32}
\end{equation}\pageoriginale
Hence, in the notation similar to the above, one has
\begin{align*}
U_{++} &= \rho_{r/2}\left(\left(
\begin{matrix}
1 & x^{2}\\
x^{2} & 1
\end{matrix}
\right)\right),\\
U_{+-} &= x(\delta_{ij}+\delta_{i,j+1})\rho_{r/2-1}\left(\left(
\begin{matrix}
1 & x^{2}\\
x^{2} & 1
\end{matrix}
\right)\right),\\
U_{-+} &= 0,\\
U_{--} &= \zeta(1-x^{2})\rho_{r/2-1}
\left(\left(
\begin{matrix}
1 & x^{2}\\
x^{2} & 1
\end{matrix}
\right)\right).
\end{align*}
Thus $U^{(r)}(x)$ is again diagonalizable and similar to
$$
\rho_{r/2}\left(\left(
\begin{matrix}
1+x^{2} & 0\\
0 & 1-x^{2}
\end{matrix}
\right)\right)
\oplus
\zeta(1-x^{2})\rho_{r/2-1}\left(\left(
\begin{matrix}
1+x^{2} & 0\\
0 & 1-x^{2}
\end{matrix}
\right)\right).
$$

These results imply that one can simplify the functional equations, introducing certain $L$-functions, and obtain some information about the special values of the zeta functions, as was done in \cite{art11-keySO} in the case $d$ is even.

\begin{thebibliography}{99}
\bibitem[A]{art11-keyA} \textsc{T. Arakawa :} The dimension of the space of cusp forms on the Siegel upper half plane of degree two related to a quaternion unitary group, {\em J. Math. Soc. Japan} 33 (1981), 125-145.

\bibitem[M]{art11-keyM} \textsc{M. Muro :} Microlocal analysis and calculations on some relatively invariant hyperfunctions related to zeta functions associated with the vector spaces of quadratic forms, {\em Publ. RIMS Kyoto Univ.} 22 (1986), 395-463.

\bibitem[O]{art11-keyO} \textsc{S. Ogata :} Special values of zeta functions associated to cusp singularities, {\em Tohoku Math. J.} 37 (1985), 367-384.

\bibitem[S1]{art11-keyS1} \textsc{I. Satake :}\pageoriginale {\em Algebraic Structures of Symmetric Domains,} Iwanami-Shoten and Princeton Univ. Press, 1980.

\bibitem[S2]{art11-keyS2} \textsc{I. Satake :} On numerical invariants of arithmetic varieties on $\mathbb{O}$-rank one, {\em Automophic Forms of Several Variables} (Taniguchi Symposium, Katata, 1983), Progress in Math. 46, Birkh\"auser, 1984, 353-369.

\bibitem[S3]{art11-keyS3} \textsc{I. Satake :} A formula in simple Jodan algebras, {\em Tohoku Math. J.} 36 (1984), 611-622.

\bibitem[S4]{art11-keyS4} \textsc{T. Shintani :} On zeta-functions associated with the {\em v}ector space of quadratic forms, {\em J. Fac. Sci. Univ. Tokyo} 22 (1975), 25-65.

\bibitem[S5]{art11-keyS5} \textsc{C. L. Siegel :} \"Uber die Zetafunktionen indefiniter quadratischer Formen, {\em Math. Z.} 43 (1938), 393-417.

\bibitem[Si]{art11-keySi} \textsc{I. Satake} and \textsc{J. Faraut :} The functional equation of zeta distributions associated with formally real Jordan algebras, {\em Tohoku Math. J.} 36 (1984), 469-482.

\bibitem[SF]{art11-keySF} \textsc{I. Satake} and \textsc{S. Ogata :} Zeta functions associated to cones and their special values, to appear in {\em Adv. St. in P. Math.} 15.

\bibitem[SO]{art11-keySO} \textsc{I. Satake} and \textsc{S. Ogata :} Zeta functions associated to cones and their special values, to appear in {\em Adv. St. in P. Math.} 15.

\bibitem[SS]{art11-keySS} \textsc{M. Sato} and \textsc{T. Shintani :} On zeta functions associated with prehomogeneous vector spaces, {\em Ann. of Math.} 100 (1974), 131-170.

\bibitem[V]{art11-keyV} \textsc{\`E. B. Vinberg :} The theory of convex homogeneous cones, {\em Trudy Moskov. Mat. obsc.} 12 (1963), 303-358; = {\em Trans. Moscow Math. Soc.} 1963, 340-403.

\bibitem[W]{art11-keyW} \textsc{A. Weil :} Sur la formule de Siegel dans la th\'eorie des groupes classiques, {\em Acta Math.} 113 (1965), 1-87.
\end{thebibliography}

\bigskip
\noindent
{\small Mathematical Institute}

\noindent
{\small Tohoku University}

\noindent
{\small Sendai 980, Japan}
